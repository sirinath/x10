
\begin{abstract}

Solving  large, irregular graph problems efficiently is challenging. Particularly current software systems and commodity multiprocessors do not support fine-grained, irregular parallelism well. We present \XWS{}, the \Xten{} Work Stealing framework, intended as an open-source runtime for the parallel programming
language \Xten{} and also as a library to be used directly by application writers. \XWS{} extends the Cilk work-stealing framework with several features
necessary to efficiently implement graph algorithms, viz., support for
improperly nested procedures, worker-specific data-structures, global
termination detection, and phased computation. We also present a strategy to adaptively control the granularity of parallel tasks in the work-stealing scheme, which is sensitive to the instantaneous size of the work queue. We compare the performance of the \XWS{} implementations of spanning tree algorithms with that of the hand-written C and Cilk implementations using various graph inputs. We show that the \XWS{} programs (written in Java) scale and exhibit comparable or better performance.

\end{abstract}

