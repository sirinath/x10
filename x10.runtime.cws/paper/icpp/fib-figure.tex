\begin{figure*}
\begin{minipage}{0.25\textwidth}
\scriptsize
\begin{verbatim}
int fib(int n) {
  int x, y;
  if(n<2) return n;
  finish {
    async x=fib(n-1);
    async y=fib(n-2);
 }
  return x+y;
}
    (a)
\end{verbatim}
\end{minipage}%
\begin{minipage}{0.4\textwidth}
\scriptsize
\begin{verbatim}
int fib(Worker w, int n) 
 throws StealAbort { 
  int x, y;
  if (n < 2) return n;

  FibFrame frame = new FibFrame(n);
  frame.PC=LABEL_1;
  w.pushFrame(frame);

  x = fib(w, n-1);
  w.abortOnSteal(x);

  frame.x=x;
  frame.PC=LABEL_2;

  y=fib(w, n-2);
  w.abortOnSteal(y);

  w.popFrame();
  return frame.x+y;
}

      (b)
\end{verbatim}
\end{minipage}%
\begin{minipage}{0.45\textwidth}
\scriptsize
\begin{verbatim}
void compute(Worker w, 
 Frame frm) throws StealAbort {
  int x, y;
  FibFrame f=(FibFrame)frm;
  int n = f.n;
  switch (f.PC) {
  case ENTRY: 
    if (n < 2) {
      result = n;
      setupReturn();
      return;
   }
    f.PC=LABEL_1;
    x = fib(w, n-1);
    w.abortOnSteal(x);
    f.x=x;
  case LABEL_1: 
    f.PC=LABEL_2;
    int y=fib(w,n-2);
    w.abortOnSteal(y);
    f.y=y;
  case LABEL_2: 
    f.PC=LABEL_3;
    if (sync(w)) return;
  case LABEL_3:
    result=f.x+f.y;
    setupReturn();
 }
}
       (c)
\end{verbatim}
\end{minipage}%
\caption{(a) \Xten{} program for Fibonacci. (b) Fast version. (c) Slow version}%
\label{fig:fib-ill}
\end{figure*}
