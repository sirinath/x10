
\section{Designing Parallel Graph Algorithms in X10}
\label{s:design}

 Invariably algorithms are designed, although sometimes implicitly, with an abstract algorithmic model. Programming languages play an important role in the translation of an algorithm to an executable program on the target system. When there is a large gap between model and architecture, programming languages and efficient runtime support can be immensely helpful in achieving high performance with relatively ease of programming. 

 Large scale graph problems with irregular instances are challenging to solve on current and emerging parallel systems \cite{BC07}. Despite the large existing body of theoretically-fast PRAM algorithms and communication-optimal BSP algorithms, there are few implementations that achieve good parallel performance. With the PRAM model, synchronous processors are available in an unlimited amount, and accesses to memory locations are of uniform distance. PRAM is highly idealistic and conducive to exploring the inherent parallelism of a problem. For practical purposes, PRAM does not reflect features of main-stream architectures that are critical to performance. Significant effort is necessary to produce efficient implementations on even shared-memory systems, for example, SMPs. Few PRAM algorithms have been adapted to the distributed-memory environment with significant overhaul of major steps. More practical models such as BSP and LogP, on the other hand, parameterize communication, synchronization, and even memory access costs. As a result, algorithms may be mapped onto real machines with reasonable ease and performance, yet design choices are severely limited as the dimension of fine-grained parallelism is virtually excluded from the design. Designing and implementing parallel graph algorithms for large, irregular inputs on either class of models is challenging. 

 One of the biggest challenges of efficiently simulating a PRAM algorithm on modern architectures is load balancing. Load-balancing is not an issue with PRAM as there are always enough processors. Consider a graph algorithm for sparse, irregular instances with adjacency list as the input. A PRAM algorithm may assign for each edge (in this case each neighbor $v$ in the adjacency list of vertex $u$) a processor, and takes $O(n)$ processors. When simulating on $p$ ($p\ll n$) processors, $\frac{n}{p}$ vertices are usually assigned to one processor. As the graph is irregular with potentially huge difference among the number of neighbors for each processor, load-imbalance causes serious performance problems. Keeping track of workload distribution becomes even more cumbersome when the algorithms compact the input graphs. The challenge with popular distributed-memory models such as BSP and LogP is graph partitioning. For irregular instances there is no known practical algorithm that partitions a graph evenly.  

 X10 provides a programming model that allows for expression of data and task parallelism,  and at the same time good efficiency guarantee. With expressive programming constructs and efficient runtime support, X10 greatly reduces for a programmer the gap between algorithm and target systems. The programming model enables an algorithm designer to focus on expressing the appropriate parallelism of a problem, and leaves to the runtime system mapping and scheduling activities on to the processors. Of course X10 also provides the mechanism for an algorithm to take advantage of architecture features, for example, memory affinity, that impact performance. X10 makes a big step in solving large-scale, irregular graph problems on current supercomputers.

%%  X10 is designed for applications that run on a cluster of SMPs. First note that similar to PGAS languages, a global virtual space is provided and this is critical in solving sparse, irregular graph instances as no graph partitioning techniques are available. Also X10 provides a notion of affinity that helps. X10 also provides various task constructs that is critical in expressing various kinds of parallelism. 

 We next present two algorithms that are programmed in X10 and showcase its expressiveness in designing graph algorithms. Among the many productivity-improving features, we here choose to show those that are related to activity scheduling. 
As an example, we consider the spanning tree problem. Finding a spanning tree of a graph is an important building
block for many graph algorithms, for example, biconnected components
and ear decomposition \cite{MR86}, and can be used in graph planarity
testing \cite{KR88}.Section~\ref{s:trav} describes a new algorithm on SMP, and section~\ref{s:trav-dist} is the algorithm
for distributed environment.


 
%% The Shiloach-Vishkin algorithm, representative of the ``graft-and-shortcut'' parallel approach,
%%  runs in \bigO{\log n} & \bigO{(m+n) \log n} on priority CRCW PRAM.

\subsection{A Parallel Spanning Tree Algorithm based on graph traversal in X10}
\label{s:trav}

Many fast PRAM algorithms have been proposed for the spanning tree problem, and they 
are drastically different from the sequential depth-first search (DFS) or breadth-first search (BFS) approaches.
DFS and BFS are efficient with very low overhead, and for quite a long time no parallel implementations beat the
best sequential implementation for irregular inputs \cite{BC04a}. 

Bader and Cong \cite{BC04a} presented the first fast parallel spanning tree algorithm that achieved good 
speedups on SMPs. Their algorithm is based on a graph traversal approach, and is similar to DFS or BFS. 
There are two steps to the algorithm. First a small stub tree of size $O(p)$ is generated by having one processor randomly walking the graph, and the vertices are evenly distributed to each processor.
The processors then traverse the graph in a manner similar to sequential DFS or BFS. 
To achieve good load-balancing, a processor checks for non-empty stack/queue on some other processor when it runs out of work, and takes a portion of the stack/queue as its own.

X10 allows expressing the essential parallelism in the algorithm in a very concise and elegant way. Alg.~\ref{alg:st-x10} is a recursive algorithm that traverses the graph and computes a spanning tree. It looks very much like the recursive sequential DFS. The few important differences include creating parallel activities and lock-free synchronization.  
\begin{algorithm}
\begin{verbatim} 
   void traverse(int u) {
        int k,v;
        finish for(k=0;k<G[u].degree;k++)
        {
            v=G[u].neighbors[k];
            if(color[v].compareAndSet(0,1)) {
                G[v].parent=u;
                final int V=v;
                async traverse(V);
            }
        }
   }
\end{verbatim}
\caption{A spanning tree algorithm on an SMP node in X10}
\label{alg:st-x10}
\end{algorithm}

The \emph{async} keyword creates a logical parallel activity. While visiting each neighbor $v$ of vertex $u$, the algorithm spawns a new traversal activity. Since the algorithm is recursive, for a graph with millions of vertices, massive amount of parallelism is available. It puts great pressure on the runtime system for task management, scheduling, and load-balancing. In Alg.~\ref{alg:st-x10} \emph{color} is an array of atomic integers. An asynchronous activity is only created when the color of the vertex being visited is 0. The correctness of Alg~\ref{alg:st-x10} can be similarly reasoned as in Theorem in \cite{BC04a}. 

A similar algorithm can also be expressed using Cilk. However, the X10 runtime support is very different from that of Cilk, and is much more friendly to graph algorithms. We leave the discussion of the X10 runtime to Section~\ref{s:runtime}. For a very simple comparison, Bader and Cong's spanning tree algorithm contains over 300 lines of C code.


\subsection{A spanning tree algorithm in the distributed-memory setting}
\label{s:trav-dist}

 When the target system is of NUMA architecture, or even a cluster of SMPs, X10 provides constructs that can easily transform Alg.~\ref{alg:st-x10} to take advantage of memory affinity . Alg.~\ref{alg:st-dist-x10} computes a spanning tree for a graph distributed on multiple nodes. In Alg.~\ref{alg:st-dist-x10}, again \emph{async} creates parallel activities. Compared with Alg.~\ref{alg:st-x10}, here \emph{async} takes the parameter of the distribution of a vertex $v$, and creates a traversal activity on a node that owns vertex $v$ so that the activity accesses data mostly on that node. Note that the activity can be created on a remote node. Obviously complex scheduling and synchronization support are necessary for fast execution of Alg.~\ref{alg:st-dist-x10}.
 
\begin{algorithm}
\begin{verbatim} 
        void traverse(int u) {
            int k,v;
            finish for(k=0;k<G[u].degree;k++)
            {
                    v=G[u].neighbors[k];
                    if(color[v]==0) {
                            color[v]=1;
                            final int U=u, V=v;
                            async (G.distribution[v]){
                                    G[V].parent=U;
                                    traverse(V);
                            }
                    }
            }
        }
\end{verbatim}
\caption{A spanning tree algorithm on a cluster of SMPs in X10}
\label{alg:st-dist-x10}
\end{algorithm}


