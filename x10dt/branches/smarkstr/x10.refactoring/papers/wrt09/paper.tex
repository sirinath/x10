\documentclass[natbib]{sigplanconf}
%\documentclass{llncs}
\usepackage{times}
\usepackage{ifthen}
\usepackage{bcprules}
\usepackage{listings}
\usepackage{url}

\newenvironment{code}{\begin{center}\begin{small}\begin{tt}\begin{tabbing}}
{\end{tabbing}\end{tt}\end{small}\end{center}}

\newcommand{\subf}[2]{\begin{minipage}[t]{3in} #2 \end{minipage}\vspace{-.1in}\\\begin{minipage}[t]{3in}\center (#1) \end{minipage}\\}

\newcommand{\twocolfig}[2]{\begin{minipage}[t]{3in} #1 \end{minipage}\hspace{.5in}\begin{minipage}[t]{3in} #2 \end{minipage}}

\newcommand{\bug}[1]{\typeout{BUG: #1}$\spadesuit$ {\bf #1}}

\newcommand{\todo}[1]{\textbf{[[{#1}]]}}

\begin{document}

\conferenceinfo{WRT'09,} {October 25, 2009, Orlando, Florida, USA.}
\CopyrightYear{2009}
\title{Extracting Concurrency via Refactoring in X10}

\authorinfo{Shane A. Markstrum}{Bucknell University\\Computer Science Department\\Lewisburg, PA USA}{shane.markstrum@bucknell.edu} \authorinfo{Robert M. Fuhrer}{IBM T.J. Watson Research Center\\Hawthorne, NY USA}{rfuhrer@us.ibm.com}

%\author{Shane Markstrum}

%\institute{IBM Research \\ \email{smarkst@us.ibm.com}}

\date{7/30/08}

\maketitle

\begin{abstract}
In this paper, we present our vision of refactoring 
support for languages with a partitioned global address space memory model as 
embodied in the X10 programming language. We examine a novel refactoring, 
\emph{extract concurrent}, that introduces additional concurrency 
into a program by arranging for selected code in a loop body to run in parallel
with future iterations of the loop. We discuss the mechanisms and algorithms required
to implement the refactoring; provide insights on how development of this
refactoring aids designing future concurrency refactorings; and outline new
strategies for determining the effectiveness of concurrent refactorings.
\end{abstract}

\category{D.1.3}{Programming Techniques}{Concurrent Programming}
\category{D.2.6}{Software Engineering}{Programming Environments}[Integrated
environments]
\category{F.3.2}{Logics and Meanings of Programs}{Semantics of Programming
Languages}[Program analysis]

\terms
Algorithms, Languages



%
%I. Introduction and overview (3 pages)

%Design of X10, concurrency, high productivity, high performance,
%practical language.
%
%Dependent types arise naturally. arrays, regions, distributions, place
%types.
%
%Indeed you can look around and recognize many OO type systems proposed
%in the last decade or so as specific kinds of applied dependent type
%systems.
%
%Our goal is to develop a general framework for dependent types for
%statically typed OO languages ("Java-like languages"). 

\Xten{} is a modern statically typed object-oriented
language designed for high productivity in the high performance
computing (HPC) domain~\cite{X10}. Built essentially on
sequential imperative object-oriented core
similar to
Scala or
$\mbox{Java}^{\mbox{\scriptsize\sc tm}}$,
\Xten{} introduces constructs for distribution and
fine-grained concurrency (asynchrony, atomicity, ordering).

The design of \Xten{} requires a rich type system to permit a
large variety of errors to be ruled out at compile time and to 
generate efficient code.  
\Xten{}, like most object-oriented languages supports classes;
however, it places
equal emphasis on {\em arrays}, a central data structure in high
performance computing.
In particular, \Xten{} supports dense,
distributed multi-dimensional arrays of value and reference types,
built over index sets known as {\em regions}.%, and mappings from index
%sets to places, known as {\em distributions}.  \Xten{} supports a rich
%algebra of operations over regions, distributions and arrays.

A key goal of \Xten{} is to rule out large classes of error
by design. For instance, the possibility of indexing a 2-d array with 3-d
points should simply be ruled out at compile-time. This means that one
must permit the programmer to express types such as \xcd{Region(2)},
the type of all two-dimensional regions;
\xcd{Array[int](5)}, the
type of all arrays of \xcd{int} of length \xcd{5};
\xcd{Array[int](Region(2))}, the type of all \xcd{int} arrays
over two-dimensional regions; and
\xcd{Tree\{loc==here\}}, the type of all \xcd{Tree} objects located on the
current node. For concurrent computations, one needs the ability to
statically check that a method is being invoked by an activity that is
registered with a given clock (i.e., dynamic barrier)~\cite{X10}.

For performance, it is necessary that array index accesses are
bounds-checked statically.  Further, certain regions (e.g.,
rectangular regions) may be represented particularly
efficiently.  Hence, if a variable is to range only over
rectangular regions, it is important that this information is
conveyed through the type system to the code generator.

In this paper we describe {\Xten}'s support for {\em
constrained types},
a form of {\em dependent
type}~\cite{dependent-types,xi99dependent,ocrz-ecoop03,aspinall-attapl,cayenne,epigram-matter,calc-constructions}---types parametrized by values---defined 
on predicates over the {\em immutable}
state of objects. Constrained types statically capture many common invariants
that naturally arise in code. For instance, typically the shape of an
array (the number of dimensions (the rank) and the size of each dimension)
is determined at
run time, but is fixed once the array is constructed. Thus, the shape of an
array is part of its immutable state.
Both mutable and immutable variables may have a constrained
type: the constraint specifies invariants on the immutable state
of the object stored in the variable. 

\Xten{} provides a framework for specifying and checking constrained types
that achieves certain desirable properties:
\begin{itemize}
\item 
{\bf Ease of use.}  
The syntax of constrained types is a simple and
natural extension of nominal class types.
% Constrained types in
% \Xten{} interoperate smoothly with Java libraries.

\item
{\bf Flexibility.}
The framework
permits the development of concrete,
specific type systems tailored to the application area at
hand.  \Xten{}'s compiler permits extension with different constraint systems
via compiler plugins, enabling a kind of pluggable type system~\cite{bracha04-pluggable}.
The framework is parametric in the kinds of
expressions used in the type system, permitting the installed constraint
system to interpret the constraints.

\item
{\bf Modularity.}
The rules for type-checking
are specified once in a way that is independent of the
particular vocabulary of operations used in the dependent type
system.
The type system supports separate compilation.

\item
{\bf Static checking.}  The framework permits mostly static
type-checking. The user is able to escape the confines of
static type-checking using dynamic casts.
%, as is common for Java-like
%languages.
\end{itemize}

\subsection{Constrained types}

\begin{figure}[t]
{
\footnotesize
\begin{xtenlines}
class List(n: int{self >= 0}) {
  var head: Object = null;
  var tail: List(n-1) = null;

  def this(): List(0) { property(0); }

  def this(head: Object, tail: List): List(tail.n+1) {
    property(tail.n+1);
    this.head = head;
    this.tail = tail;
  }

  def append(arg: List): List(n+arg.n) {
    return n==0
      ? arg : new List(head, tail.append(arg));
  }

  def reverse(): List(n) = rev(new List());
  def rev(acc: List): List(n+acc.n) {
    return n==0
      ? acc : tail.rev(new List(head, acc));
  }

  def filter(f: Predicate): List{self.n <= this.n} {
    if (n==0) return this;
    val l: List{self.n <= this.n-1} = tail.filter(f);
    return (f.isTrue(head)) ? new List(head,l) : l;
  }
}
\end{xtenlines}
}
\caption{
This program implements a mutable list of Objects. The size of a list
does not change through its lifetime, even though at different points
in time its head and tail might point to different structures.}
\label{fig:list-example}
\end{figure}

X10's sequential syntax is similar to Scala's.
We permit the definition of a class \xcd{C} to specify
a list of typed parameters or {\em properties},
${\tt f}_1: {\tt T}_1, \dots, {\tt f}_k: {\tt T}_k$,
similar in syntactic structure to a method formal parameter list.
%
Each property in this list is treated as a public final instance field.
%
We also permit the
specification of a {\em class invariant}
%, a {\em where clause}~\cite{where-clauses}
in the class definition. A class invariant
is a boolean expression on the properties of the class.
The compiler ensures that all
instances of the class created at run time satisfy the invariant.
%
Syntactically, the class invariant follows the property list.
%
For instance, we may specify a class \xcd{List} with an
\xcd{int} \xcd{length} property as follows:
\begin{xtennoindent}
  class List(length: int){length >= 0} {...}
\end{xtennoindent}
Given such a definition for a class \xcd{C}, types can be
constructed by {\em constraining} the properties of \xcd{C}.  In
principle, {\em any} boolean expression over the properties
specifies a type: the type of all instances of the class
satisfying the boolean expression. Thus, \xcd{List\{length == 3\}}
is a permissible type, as are \xcd{List\{length <= 42\}} and
even \xcd{List\{length * f() >= g()\}} where \xcd{f} and \xcd{g}
are functions on the immutable state of the \xcd{List} object
and the variables in scope where the type appears.
In practice, the constraint expression is restricted by the
particular constraint system in use.

Our basic approach to introducing constrained types into \Xten{}
is to follow the spirit of generic types, but to use values
instead of types.

In general, a {\em constrained type} is of the form \xcd{C\{e\}},
the name of a class or interface\footnote{In \Xten{}, primitive
types such as \xcd{int} and \xcd{double} are object types; thus,
for example, \xcd{int\{self==0\}} is a legal constrained type.}
\xcd{C}, called the {\em base class}, followed
by a {\em condition} \xcd{e},
a boolean expression on the properties of the
base class and the final variables in scope at the type.
Such a type represents a refinement of \xcd{C}: the set of all
instances of \xcd{C} whose immutable state satisfies the
condition \xcd{e}.
%
We write \xcd{C} for 
the vacuously constrained type \xcd{C\{true\}}, and
write
\tcd{C(${\tt e}_1,\ldots,{\tt e}_k$)} for
the type
\tcd{C\{${\tt f}_1$==${\tt e}_1,\ldots,{\tt f}_k$==${\tt e}_k$\}}
where \xcd{C} declares the $k$ properties
${\tt f}_1,\ldots,{\tt f}_k$.

Constrained types may occur wherever normal types occur. In
particular, they may be used to specify the types of properties,
(possibly mutable) local variables or fields,
arguments to methods, return types of methods; they may also be
used in casts, etc.

Using the definitions above, \xcd{List(n)}, shown in
Figure~\ref{fig:list-example}, is the type of all lists of
length \xcd{n}.
%
Intuitively, this definition states that a \xcd{List} has an \xcd{int}
property \xcd{n}, which must be non-negative.
The properties
of the
class are set through the invocation of \xcd{property}\tcd{(\ldots)}
(analogously to \xcd{super}\tcd{(\ldots)}) in the constructors
of the class.

In a constraint, the name \xcd{self} is bound and refers to the type being
constrained.  The name \xcd{this}, by contrast, is a free
variable in the
constraint and refers to the receiver parameter of the current
method or constructor.  Use of \xcd{this} is not permitted in static
methods.

The \xcd{List} class has two
fields that hold the head and tail of the list.  The fields are
declared with the \xcd{var} keyword, indicating that they are
not final.  Variables declared with the \xcd{val} keyword, or
without a keyword are final.

Constructors have ``return
types'' that can specify an invariant satisfied by the object being
constructed.  The compiler verifies that the
constructor return type and the class invariant are implied by the
\xcd{property} statement and any \xcd{super} calls in the constructor
body.
A constructor must either invoke another constructor of the same
class via a
\xcd{this} call
or must have a \xcd{property} statement on every
non-exceptional path
to ensure the properties are initialized.
The \xcd{List} class has two constructors: the first
constructor returns an empty list;
the second
returns a list of length \xcd{m+1}, where \xcd{m} is the length
of the second argument. 

In the second constructor (lines 7--11), as well as 
the \xcd{append} (line 13) and \xcd{rev} (line 20) methods,
the return type
depends on properties of the formal parameters. 
If an argument appears in a
return type then the parameter must be final,
ensuring the
argument points to the same object throughout the evaluation of
the method or constructor body.  A parameter may also depend on
another parameter in the argument list.

The use of constraints makes existential types very natural.
Consider the return type of \xcd{filter} (line 24): it specifies
that the list returned is of some unknown length. The only thing
known about it is that its size is bounded by \tcd{n}.
Thus,
constrained types naturally subsume existential dependent types.
Indeed, every base type \xcd{C} is an ``existential''
constrained type since it does not specify any constraint on its
properties. Thus, code written with constrained types can
interact seamlessly with legacy library code---using just base
types wherever appropriate.

The return type of \xcd{filter} also illustrates the difference
between \xcd{self} and \xcd{this}.  Here, \xcd{self} refers to
the \xcd{List} being returned by the method; \xcd{this} refers
to the method's receiver.

\subsection{Constraint system plugins}

The \Xten{} compiler allows  
programmers to extend the semantics of the language with
compiler plugins.  Plugins may be used to support different constraint
systems to be used in constrained types.
Constraint systems provide code for checking consistency and
entailment.

The condition of a constrained type is parsed and type-checked
as a normal boolean expression over properties and
the \xcd{final} variables in scope at the type.  Installed
constraint systems translate the expression into an internal
form, rejecting expressions that cannot be represented.
%
A given condition may be a conjunction of constraints from
multiple constraint systems.
A Nelson--Oppen procedure~\cite{nelson-oppen} is used to check
consistency of the constraints.

The \Xten{} compiler
implements a simple
equality-based constraint system.  Constraint solver plugins
have been implemented for inequality constraints, for
Presburger constraints using
the CVC3 theorem prover~\cite{cvc}, and for
set-based constraints also using CVC3.
These constraint systems are described in
Section~\ref{sec:examples} and the implementation is
discussed in Section~\ref{sec:impl}.

\subsection{Claims}

The paper presents constrained types in the \Xten{} programming
language.
We claim that the design is natural, easy to use, and useful. Many
example programs have been written using constrained types and are
available at {\tt x10.sf.net/\allowbreak applications/\allowbreak examples}.

As in staged languages~\cite{nielson-multistage,ts97-multistage}, the
design distinguishes between compile-time and run-time
evaluation. Constrained types are checked (mostly) at compile-time.
The compiler uses a constraint solver to perform universal reasoning
(e.g., ``for all possible values of method parameters'') for dependent
type-checking.  There is no run-time constraint-solving.  However,
run-time casts and \xcd{instanceof} checks involving dependent types
are permitted; these tests involve
arithmetic, not algebra---the values of all parameters are known.

The design supports separate compilation: a class needs to be
recompiled only when it is modified or when the method
and field signatures or invariants of classes on which it
depends are modified.

We claim that the design is flexible. The language design is
parametric on the constraint system being used.
%We are planning on extending the current
%implementation to support multiple user-defined constraint systems,
%thereby supporting pluggable types.
The compiler supports
integration of
different constraint solvers into the language.
Dependent clauses  also form
the basis of a general user-definable annotation framework we have
implemented separately~\cite{ns07-x10anno}. 

We claim the design is clean and modular. We present a simple core
language \CFJ, extending \FJ{}~\cite{FJ} with constrained types on top
of an arbitrary constraint system. We present rules for type-checking
\CFJ{} programs that are parametric in the constraint system
and establish subject reduction and progress theorems. 

%
% XXX contrast with hybrid type checking.

\paragraph{Rest of this paper.}

Section~\ref{sec:lang} describes the syntax and semantics of
constrained types.
Section~\ref{sec:examples} works through a number of
examples using a variety of constraint systems.
The compiler implementation, including support for constraint
system plugins, is described Section~\ref{sec:impl}.
A formal semantics for a core language with constrained types 
is presented in Section~\ref{sec:semantics}, and a soundness
proof is presented in the appendix.
Section~\ref{sec:related} reviews related work.
The paper concludes in Section~\ref{sec:future}
with a discussion of future work.



%\section{Proof of Concept: X10 Refactoring Engine}
\section{Extract Concurrent}

As a first step toward our vision, we are developing a refactoring
called {\em extract
concurrent} for the X10 language within the X10DT, our Eclipse-based IDE. 
The transformation introduces concurrency within a loop by arranging
for some user-selected code in the loop body 
to run in parallel with other iterations of the 
loop.\footnote{The transformation can be thought of as
a fine-grained variant of loop distribution~\cite{loopdist} where only
a portion of the loop body is executed in parallel with other iterations.}

As an example, consider the X10 code in Figure~\ref{fig:CHM-X10}, an excerpt from
an X10 implementation of the {\tt ConcurrentHashMap} class from the Java 
standard utilities. In this snippet from the {\tt containsValue} method,
a snapshot of the modification status of each map segment is taken before
determining whether a particular segment contains the desired value. This
approach allows value lookup to occur without locking the entire map. Given
the PGAS model for data distribution, the individual elements of the array 
{\tt segments} could reside anywhere in the global address space, increasing
their access cost. Further, each call to {\tt modCount()} must block until the
call completes, per X10 semantics. Thus it is possible that asynchronously
executing the {\tt modCount} method for each array element will speed up the
overall execution of the loop.

\begin{figure}[tp]
  \begin{code}
int mcsum=0; \\
fo\=r ( point p : segments ) \{ \\
\>  mc[p] = segments[p].modCount(); \\
\>  mcsum += mc[p]; \\
\>  if\=(segments[p].containsValue(value)) \\
\>\>    return true; \\
\}
  \end{code}
\caption{\label{fig:CHM-X10} An excerpt from the X10 version of the Java 
library {\tt java.util.concurrent.ConcurrentHashMap}.  % It is 
% presented in a three address code form in order to simplify explanation.
}
\end{figure}

One way to introduce this concurrency, shown in Figure~\ref{fig:CHM-X10-future},
is to execute each of the {\tt modCount} invocations as {\tt future}s in a new loop
and only synchronize with those executions via a {\tt force}
operation when their results are actually needed.
This is in fact the transformation that was manually applied to
the code in our example after the initial translation
from Java to X10.  Our
{\em extract concurrent} refactoring automates this transformation and
ensures that the transformation preserves program behavior.  Our
refactoring also supports a generalization of this transformation that
allows a block of statements to be safely executed asynchronously,
but we do not discuss it here for brevity's sake.

% There are two ways in which this can be done.
%The first is to create a
%finished asynchronous {\tt foreach} loop that executes the method, but
%this causes a non-ideal situation where the rest of the loop must wait until
%every method invocation is cached before execution. 

% We chose to implement our refactoring scheme in the X10 language. X10 not only
% has a PGAS data consistency model, but also includes explicit higher-level
% concurrency constructs such as asynchronous blocks and future
% expressions. These concurrency constructs further simplify the analysis of X10
% code by making all asynchronous and atomic code apparent to both the analysis
% engine and the programmer.

%%\begin{figure}[tp]
%%  \begin{code}
%%    int mcsum=0; \\
%%    fo\=r (i=0; i<segments.length; i++)\{ \\
%%    \>  mcsum += mc[i] = segments[i].modCount(); \\
%%    \>  if\=(segments[i].containsValue(value)) \\
%%    \>\>    return true; \\
%%    \} \\
%%  \end{code}
%%\caption{\label{fig:CHM} A Java code excerpt from the library class {\tt
%%java.util.concurrent.ConcurrentHashMap} which illustrates a loop that
%%would not be parallelizable via traditional automatic loop
%%parallelization methods.}
%%\end{figure}

%%Because languages designed for the PGAS model do not focus on
%%providing support for the parallel execution of the same code over
% different sets of data, they can provide a more flexible alternative
% to whole loop parallelization: the loops can be executed sequentially
% and updates to distributed data structures can occur in parallel. 

%// segments.region $\rightarrow$ all future objects are {\em here}\\

\begin{figure}[tp]
  \begin{code}
int mcsum=0; \\
future<int>[.] f\_\=segments = \\
\>new future<int>[segments.region]; \\
fo\=r ( point p : segments ) \{ \\
\>  f\_\=segments[p] = \\
\>\>future(segments[p])\{segments[p].modCount()\}; \\
\} \\ \\
for ( point p : segments ) \{ \\
\>  mc[p] = f\_segments[p].force(); \\
\>  mcsum += mc[p]; \\
\>  if\=(segments[p].containsValue(value)) \\
\>\>    return true; \\
\} 
  \end{code}
\caption{\label{fig:CHM-X10-future} A transformation of the program excerpt 
from Figure~\ref{fig:CHM-X10} introducing additional concurrency via the 
X10 {\tt future} construct.}
\end{figure}

%\subsection{Extract Concurrent Refactoring}

%To further demonstrate our proof of concept, we are developing a refactoring
%which we call {\em Extract Concurrent}.


% We present in this paper the source code transformation {\em extract concurrent} for
% the X10 language which allows a programmer to transform loop code
% like that in Figure~\ref{fig:CHM-X10} to one that takes maximum advantage of
% asynchronous execution. 

Our refactoring involves two main components:
\begin{enumerate}
\item {\em Loop dependence analysis.} Since introducing parallelism in
the middle of a loop might affect the ability of other statements in a
loop to evaluate properly, it is important that loops do not depend
on the results of any asynchronously executed statements. We
have developed a set of analyses to determine whether {\em extract
concurrent} will adversely affect the execution of the code and
violate its perceived sequential consistency.

\item {\em Transformation pattern.} We have developed a general
pattern for the {\em extract concurrent} transformation on viable
sequential loops.
%finished asynchronous loops or future-caching loops.
The pattern splits the loop in two, as shown in Figure~\ref{fig:CHM-X10-future}: the
first loop introduces the desired statement- and/or expression-level
parallelism, while the second loop synchronizes with and utilizes the
results of the asynchronous execution.
% Because this split requires some code duplication, we
% present the results of a formal analysis on how the transformation
% affects the runtime of the loop and define the conditions under which
% the transformation provides the potential for better runtime. 
In practice, this
 transformation is more widely applicable to multiple statements or expressions.
We present here only the single statement or expression case.
\end{enumerate}

%%The primary goal of the PGAS model is to allow parallel asynchronous activity to
%%occur among multiple concurrent platforms, each having its own local memory and
%%potentially lacking a joint shared memory, without losing the ability to read
%%and write to global data. Global 
% shared data is actually owned by locally by processing units but is globally
% addressable. Other processors which would like
% to access or update global data then communicate directly with the owning
% location to perform any necessary actions. Thus, programmers take an active role
% in defining how data in arrays or data structures is distributed
% over the address space so as to maximize locality, thus taking maximum advantage
% of their concurrent environments. This removes the programmer's (or analysis')
% burden of determining
% how complicated structures should be partitioned among execution sites at
% points in the code where parallelism is desired (e.g., inside loops). 

The concurrency
constructs and the PGAS model used by X10 simplify the analysis required to determine
when program transformations are safe. For example, the static analysis required to
determine whether a statement may be executed asynchronously is
reduced to determining local and loop-carried dependencies that prevent a
statement from being asynchronously executed. With a traditional model, this
same transformation might also require an analysis to determine how much
dependent data would need to be copied for asynchronous execution and
where that execution should take place. The example highlights a perceived benefit of
refactoring X10: data locality and asynchronous execution are separable, or
{\em lateral}, concerns. Thus, a programmer may manipulate the amount of program
concurrency while keeping the data distribution fixed, or manipulate the
distribution while keeping perceived program concurrency relatively unchanged.

% \bug{This statement really applies
% to this particular transformation. I.e., the programmer does have to determine the partitioning at some point,
% but doesn't need to worry about it while applying this transformation. So
% perhaps we should say that we envision different kinds of {\em lateral} moves,
% e.g.: manipulate concurrency while keeping the distribution fixed, and
% manipulating the distribution while keeping concurrency relatively unchanged.}

We have built a prototype of the {\em extract concurrent} 
refactoring and the supporting analysis
in the X10DT, and we are in the process of refining the
implementation and experimenting with it on some X10 applications. 
It is our hope that 
the analysis itself will prove useful in implementing future refactorings,
and although the present implementation targets X10, we believe that the
transformation is readily applicable to other languages with a PGAS
programming model.
% \bug{We will insert more about the implementation when it's actually been done.}

\section{Algorithmic Details}

The goal of the {\it extract concurrent} refactoring is to allow
users to (somewhat) safely introduce parallelism/concurrency to
array accesses and updates that are done within a {\tt for}
loop.
%Since so much work has been done attempting to automate the
%parallel transformation of such loops, and
Since loops tend to be execution hotspots, and X10 supports explicit
asynchronous event constructs, this
refactoring is the type that X10 users should expect to find in an IDE
like Eclipse.

The general schema for the transformation is illustrated by the two
code skeletons shown in Figure~\ref{fig:transformSchema}, which
depict the original and the transformed code.

\begin{figure}[tp]
 Before:
\begin{code}
for(\=$\ldots$) \{\\
\> $\ldots$ target $\ldots$\\
\}
\end{code}

After:
\begin{code}
for(\=$\ldots$)\= \{\\
\> $\ldots$\\
\> targetFuture = future \{ target \};\\
\> $\ldots$\\
\}\\\\
for($\ldots$) \{\\
\> $\ldots$ targetFuture.force() $\ldots$\\
\}
\end{code}
\caption{\label{fig:transformSchema}The transformation schema for futurizing an expression.}
\end{figure}

A basic assumption of this transformation is that it will introduce
no activities that remain unfinished outside the scope of the parent
construct containing the target distributed array.
As a result, the developer need not expend mental effort to determine
when the results of the activity will be available.
%At the same time, in the {\tt async} case, the code generated will not
%allow overlapped execution between the two loops.
%This is conservatively safe, but comes at the cost of potentially
%useful concurrency.
In this case, overlap between the two loops can be
easily permitted, since the {\tt force()} calls in the second loop
perform all of the necessary synchronization.
%The synchronization required to in{\tt async} case is
%significantly more complex, due to the difference in nature between the
%data-centric {\tt future} construct and the operation-centric {\tt async}.

\subsection{Pre-conditions}

\begin{figure}[tp]
\begin{code}
Lo\=op (InductionVars $\ldots$)\{ \\
\> Block1 \\
\> Result = (Exp1 op f(Target[])) op Exp2; \\
\> Block2 \\
\}
\end{code}
\caption{\label{fig:basic_loop}The generic loop structure for the {\it
extract concurrent} refactoring.}
\end{figure}

In general, the code for a candidate {\tt for} loop will look as in
Figure~\ref{fig:basic_loop}. In this case, the user would be
interested in potentially making the function {\tt f(Target[])} representing
the targeted expression -- which is not
necessarily a method call -- into a futurized expression. The
candidate expression or statement will be called the {\em target}. Here,
{\tt Block1} and {\tt Block2} are both arbitrary blocks of
code%, although they must not define an {\tt atomic} block that starts
%in {\tt Block1} and ends in {\tt Block2}; concurrent activities cannot be
%spawned from within an {\tt atomic} block. There may also be no clock
%manipulations present in the body of the loop -- the sole exception
%here is {\tt resume}, since it is a non-blocking update to the clock
%which cannot have loop-carried dependencies.

As with all parallelizing algorithms, the most important information to
track is loop-carried dependencies. In this case, all loop-carried dependencies
must be reproducible on all executions of the loop. As a result, the following
pre-conditions must be met in order to refactor the code:
\begin{itemize}
\item No loop-carried dependencies are allowed on the target distributed array.
%\item The pivot must not have side-effects that affect any loop-carried
%      dependencies.
\item The target must not produce side-effects.
\item There must be no loop-carried dependencies on user input or the
      evaluation of any chaotic functions (e.g., random number generators).
%\item {\tt Exp2} and {\tt Block2} must not have side-effects on the target
\item Loop-carried dependent statements that occur semantically after the
target may not have side-effects that directly affect the target.
\end{itemize}

\subsection{Interactions with X10 Constructs}

In this section, the interaction of the transformation with various
X10 constructs will be discussed. Because the transformation is
aimed at targets residing within loops, any errors that are a
result of poorly formed code outside of the loop will be omitted from
this discussion. It is noted, however, that any asynchronous activity
that remains unfinished or unforced before the loop is entered may
create data races within the program.

{\bf Asyncs} -- %If the target is contained within an unfinished async
%block, 
%it is not clear that the transformation will provide improved
%parallelism unless the activity occurs in a different place than the
%target. Such a situation should be 
%the decision to apply the refactoring is left to the programmer
%to
%determine, although support could be provided to warn a programmer who
%ttempts to refactor under these circumstances. 
Unfinished
asynchronous activities that have (possibly indirect) side-effects on
the target and occur semantically before the target in the loop should be
must stop the transformation from occurring.
%This points out a potential race condition in the original 
%code, so we abort the transformation to avoid exacerbating the problem.
All other finished
asynchronous activity is allowed if it conforms with the X10 language
rules and meets the pre-conditions stated
in the previous subsection. Note that choosing to extract concurrent on
a target within a finished async block effectively precludes
introduction of new parallelism.

{\bf Futures} -- Futures, in general, are treated the same as asyncs.
However, code inserted between the two
constructed loops must not have side-effects on any futurized expression that
is used in the second loop

 {\bf Atomics} -- As long as the target does not fall
within an atomic block, there are no restrictions on the use of atomic in the
loop. The target may contain an atomic block itself.

{\bf Clocks} -- Because the transformation creates new paths in the program,
%many clock-associated operations are disallowed within the loop. In
%particular, the {\tt registered} and {\tt resume} methods may be called at any
%time within the loop. But 
{\tt next} and {\tt drop} may not be called on a clock in the
first half of the loop or in any statement which has loop-carried
dependencies. %Since clock operations that exist in the loop probably
%signal that a full iteration of a loop has occurred, executing one of
%these operations during evaluation of the first half of the transformed loop
%would break this assumption. However, {\tt registered} and {\tt resume} are
%harmless operations in that multiple calls to either will not cause any
%effects when {\tt next} and {\tt drop} cannot be called. Since {\tt resume}
%does not return a value, it is also safe to completely remove it from the
%first loop and place it in the second with a slicing algorithm. 
After execution of the target can be guaranteed to have completed (i.e., at any
non-loop-carried dependent statement after the target), it is safe to perform
any of the clock operations.

{\bf Exceptions} -- If the target is within a try block, then as a
conservative measure, we force the transformation to fail.
This is based on the semantics of X10: uncaught exceptions thrown by
asynchronous activities are accumulated by the collecting {\tt finish}.
Thus, exceptions thrown by activities spawned inside a try block are not
propagated to the try block's catch clauses.
%In particular, if the exception's purpose was to short-circuit the loop,
%delaying its delivery would result in a change to the loop's control
%flow (i.e., in improper execution of code that would not otherwise
%have executed).
%Note that forcing the transformation to fail is a bit overly conservative in
%that the target may not throw any exception named in the {\tt catch} blocks,
%so that moving the target would not change the exception behavior of
%the code.
\section{Evaluation}

\subsection{Methods of Evaluating Transformed Code}

There are two vectors through which we can measure the effectiveness
of our extract concurrent transformation: running time and communication
overhead. If the running time of the transformed code is reduced and the amount
of additional communication is not made substantially worse, then the
refactoring would be considered a success.

In particular, comparing the performance of the transformation to a
program which only uses Java arrays -- which exist solely in one place
-- would not be a perfectly valid comparison in terms of running time
or interprocess communication. Distributed arrays inherently involve
at least some interprocess communication and, as a result, may require
more time to access or update elements of the array. As a result,
transformed code should be measured against loops similar to the one
seen in Figure~\ref{fig:async_loop}.

\begin{figure}
  \begin{code}
    Lo\=op (InductionVars $\ldots$)\{ \\
    \>  Block1 \\
    \>  Re\=sult = Exp1 op \\
    \>\> future(Target.dist[])\{f(Target[])\}.force() op\\
    \>\> Exp2; \\
    \>  Block2 \\
    \}
  \end{code}
\caption{\label{fig:async_loop} The basic candidate transformation
loop with explicit support for data distribution.}
\end{figure}

Most current X10 implementations do not have compiler optimizations that
significantly speed up concurrent data access, although some do support true
multiple processor data distribution and communication. Due to this, the time
measurements on both the basic asynchronous loop and the transformed
asynchronous loops will most likely exhibit much worse behavior than standard
Java. It is also possible that clock time between the basic and transformed
loops will not be accurate and comparisons on these results might be skewed
(e.g., the transformed loops might exhibit worse clock time because of
unoptimized thread spawning and thread communication). However, this is
simply a result of the current state of X10 implementation and not an
inherent shortcoming of the X10 language or other PGAS model
languages. At least one X10 implementation supports measurements that
separate out overheads associated with the current implementations from
those of the runtimes of X10 programs. It is with these tools that we hope
to effectively measure the costs and benefits associated with applying this
refactoring.

\subsection{A Formal Cost-Benefit Analysis of the Transformation}

In lieu of testing numbers, we present a more formal analysis of the
running time of the algorithm for best, worst, and average case scenarios. To
perform the analysis, we must first define a few variables. Let $m$ be the running time of
{\tt f(Target[])}, $n$ be the size of {\tt Target[]}, $C_{fut}$ be the overhead associated with starting a {\tt future}, $C_{for}$ be the overhead associated with calling {\tt force}, $p$ by the amount of available concurrency (i.e., number of
places or threads) and $p \leq n$, and $b$ be $O({\tt Block1}+{\tt
Exp1}+{\tt Exp2})+{\tt Block2})$.

For the code in Figure~\ref{fig:async_loop}, the running time would be

\vspace{-.1in}
\[O(n \cdot (b + C_{fut} + C_{for} + m))\].
\vspace{-.2in}

To analyze the {\tt future} transformation, we will break it down into
best case, worst case, and average case and ignore any thread overhead.

{\bf Best case} -- In the best case, the running time of the targeted
expression completely disappears. Thus, the running time in the best case is 

\vspace{-.1in}
\[\begin{array}{l}
O(n \cdot (b+C_{fut}) + n \cdot (b+C_{for})) \\ 
\vspace{.1in}
= O(n \cdot (2b + C_{fut} + C_{for})
\end{array}\]
\vspace{-.25in}

Thus the transformation is at least as good as the standard loop if $m
\leq b$.

{\bf Worst case} -- In the worst case, the full running time of the targeted
expression is observed every time at every {\tt force} expression. The running
time in the worst case is

\vspace{-.1in}
\[\begin{array}{l}
O(n \cdot (b+C_{fut}) + n \cdot (b+{C_for}+m)) \\
= O(n \cdot (2b + C_{fut} + C_{for} + m))
\end{array}\]
\vspace{-.15in}

In this case, there is no way for the transformed code
to match the performance of the standard loop unless $b = 0$. This
case is also equivalent to the case of having distributed data but
having no apparent concurrency.

{\bf Average case} -- In the average case analysis, $2 \leq p$, or the
worst case behavior is exhibited. We shall consider two subcases for
the average case analysis: with and without consideration of time
spent executing each loop when calculating the amount time of
evaluating {\tt f(Target[])}.

In the case of loop execution time not being factored, assume for the
sake of simplicity, but without loss of generality, that for every $p$
processes encountered, $m$ is accrued. Then the running time is 

\vspace{-.1in}
\[O(n \cdot (2b + C_{fut} + C_{for} + \frac{m}{p}))\]
\vspace{-.2in}

In this case, the transformation provides a performance boost if $m >
\frac{bp}{p-1}$. In the limit on $p$, this case is equivalent to the
best case.

If loop execution time is factored into the concurrent running time,
the analysis is slightly more complicated. First, the total running
time for the loop code that executes simultaneously with execution of
{\tt f(Target[])} is 

\[\begin{array}{l}
O(n \cdot (b + C_{fut}) - \frac{1}{2}b + n \cdot (b + C_{for}) -
\frac{1}{2}b) \\ 
= O(n \cdot ((2 - \frac{1}{n})b + C_{fut} + C_{for}))
\end{array}\]
\vspace{-.1in}

The total running time for the concurrent execution of {\tt f(Target[])},
assuming a similar perfect concurrent pipeline as in the previous
case, is $O(\frac{nm}{p} + p(b + C_{fut}))$. This is equivalent to the
best case when

\[m \leq ((2-\frac{p+1}{n})b + (1-\frac{p+1}{n})C_{fut}+C_{for})\]
\vspace{-.2in}

If the above inequality does not hold, running time is

\vspace{-.05in}
\[O(\frac{n}{p}((p-1)(2b+C_{fut}+C_{for})+2m)+(1+\frac{1}{p})(b+C_{fut}))\]
\vspace{-.1in}

This is at least as good as the basic loop if 

\vspace{-.05in}
\[m \geq (1 + \frac{p+1}{n(p-2)})b + \frac{n-p-1}{n(p-2)}C_{fut} -
\frac{1}{p-2}C_{for}\]
\vspace{-.1in}

In the limit of $p$, this is equivalent to the best case.

Except in the worst case scenario, this formal analysis indicates that the
extract concurrent refactoring \emph{should} provide beneficial results as long
as there are a large number of places and the data associated with the targeted
expression is relatively well distributed among the places. It will be
interesting to see how often this will be the case when applying the
refactoring to real X10 code.

\section{Related Work}
\label{sec:related}

To our knowledge, ours is the first work to consider automated code
refactorings in the context of the PGAS model.  However,
a number of IDEs and
tools have been developed to aid parallel language users. 
We elide discussion here of automatic parallelization
techniques and parallel program analysis to focus on
related work in tooling support. Such tooling support is crucial since complete
automatic parallelization of programs is now generally regarded as infeasible.
Thus, refactoring and transformation tools allow programmers to remain aware
of, and control to a certain degree, the level of concurrency in their
code.

The SUIF Explorer~\cite{Liao99}, for the SUIF compiler
system~\cite{SUIF}, combines automatic parallelization techniques with a
dynamic dependence analyzer to assist programmers in parallelizing their
code, but does not feature integrated refactoring support.
The ParaScope Editor~\cite{Kennedy91,Hall93} is an IDE that
enables exploration and manipulation of
loop-level parallelism in a Fortran-like language. It makes
analysis results and a number of program transformations, including {\em loop
interchange} and {\em loop distribution}, available to the user.

Photran~\cite{Overbey05} is an IDE that intends to, but does not yet, 
provide concurrency refactoring support for HPC applications in Fortran.
TSF~\cite{TSF} is an IDE tool for writing
transformation scripts and transforming parallel Fortran programs. Some of
the transformations it provides have preconditions for verifying soundness,
which is a feature we also integrate into the extract concurrent
refactoring. Another difference between this work and SUIF, ParaScope, Photran,
and TSF, is our focus on PGAS model languages as opposed to the more traditional
memory models of the procedural C and Fortran languages.
%
%A fair amount of work on transformations of skeletons, algorithmic patterns
%for defining stream-like programs, for parallel
%programming has been done in the FAN and Meta development
%frameworks~\cite{Bacci99, Aldinucci00}. Both FAN and Meta provide support for
%applying previously defined pattern transformations on the skeleton
%programs.

% \subsection{Concurrent Slicing}
% 
% Zhao~\cite{Zhao99} describes a method for determining a whole program slice for
% multi-threaded, concurrent Java programs using monitors for
% synchronization of data. To this end, an algorithm for creating a
% multithreaded dependence graph (MDG) is presented as an extension to
% the standard Object Oriented system dependence graph. By using a
% simple two-phase marking algorithm on the MDG, a program slice can be
% obtained. The analysis appears to be tailored to work on a
% multi-threaded shared memory program model, but may be modifiable to
% fit other types of concurrency architectures.
% 
% Chen and Xu\cite{Chen01} introduces the notion of object and class slicing as better
% ways to do program slicing for OO programs. To implement these new
% notions of slicing, a variant of program dependence graphs (PDGs) for
% methods is introduced which uses tagged dependence edges. A PDG for
% classes is then constructed by taking the union of the PDGs of the
% methods. While taking into account OO principles, this paper
% explicitly states that it is not concerned with concurrent behavior.
% 
% Chen, et al.,~\cite{Chen02} expands the work of \cite{Chen01, Zhao99}
% to define a dependence analysis over Ada 95's concurrent programming
% model. To do the analysis, they create two new types of CFG for
% concurrency and synchronization. They also define two Ada-specific
% types of dependence: task data dependence and synchronal
% dependence. Fundamental difference between Ada tasks and rendezvouz
% prevent the analysis from applying to X10 programs, though some of the
% concurrency analysis might be relevant.
% 
% Krinke~\cite{Krinke03} presents a context-sensitive concurrent slicing
% algorithm. Utilizing virtual inlining and a notion of interference
% dependence between threads, the slices can reach over thread
% boundaries and present a data dependency slice over the whole
% concurrent program. However, this approach elides certain concurrent
% programming constructs like synchronized blocks or dynamic activity
% creation.

% \subsection{To Be Categorized}

% The Illinois Concert System~\cite{Chien98} defined ICC++, an
% OO language with constructs for concurrency and distributed data, and a
% system of tools for its effective analysis and optimization of
% programs. However, very little support for
% code refactoring or transformation was developed for the Concert
% System.
% Closely related to concerns of X10, dynamic pointer
% alignment optimizations~\cite{Zhang97} attempt to optimize the
% parallelization of loops by arranging the tiling of loop iterations by
% location and aggregating data access. 

% The StreamIt Development Tool (SDT)~\cite{Kuo05} is an Eclipse IDE for
% advanced visual debugging support for the StreamIt
% language~\cite{StreamIt}. The language supports explicit parallel data
% streams via the {\em splitjoin} construct. SDT does a data flow
% analysis of the code to track data items as they enter and exit
% different parallel streams. The domain of streaming applications is
% particularly limited and this work does not address concerns of
% program transformation.
% 
% Solar-Lezama, et al,~\cite{Solar05, Solar06} discuss programming by
% sketching for bit- or integer-manipulation programs. To implement
% programs in this manner, a non-optimal program is created that has the
% desired output. A closer-to-optimal sketch of this program with
% explicit numeric holes is then written with the holes to be filled by
% the sketching synthesizer. With this technique, they have developed
% near-optimal programs that take advantage of bit-word structure and
% parallel bit manipulation. Sketching techinques might be advantageous
% for X10 code development to make optimal use of places -- especially
% with arrays -- but the work is too low level at this point to be
% realizable on an OO language.
% 
% Once transformations have been performed, skeletons can
% be implemented via optimized code from another language. Because this
% type of meta-language system does not allow explicit usage of
% synchronization or side-effects, transformations do not have to take
% such behavior into account. Furthermore, it is not clear how well
% these skeleton systems handle fine-grained control of concurrency.

% Allen, et al.,~\cite{Allen87} developed an automatic parallelization
% tool for {\tt DO} loops in Fortran. They determined that if all arrays
% in used in the loop could be privatized and there were no loop-carried
% dependencies between loop iterations, then each iteration could be
% performed on a separate processor. In researching the privatizability
% of arrays, they investigated code alignment and replication to handle
% some of the synchronization needed to deal with loop-carried
% dependencies between one iteration and the next. This work limited
% itself to attempting to parallelize whole loop bodies on individual
% processor instead of allowing for fine-grained parallel control, and
% while some local memory exists for each processor, depends on access
% to a shared global memory.
% 
% Hall, et al.,~\cite{Hall95} introduces automatic parallelization
% techniques for shared memory multiprocessor systems incorporating
% interprocedural analysis and granularity of loop bodies and array
% reductions for the SUIF compiler system~\cite{SUIF}. Using a selective
% procedure cloning, they avoid having to do loop unrolling and function
% inlining by doing a flow-sensitive analysis of procedure bodies and
% memoizing data-flow path information. To ensure highest loop
% granularity -- or lowest synchronization overhead -- they only
% parallelize the outermost loop for a set of parallelizable nested
% loops and suppress parallelization of array reductions if the cost of
% such parallelization outweigh sequential array reduction. They
% evaluate the success of these parallelizations by measuring
% parallelism coverage, granularity, and speedup on four processors.

% They include patterns for sequential, task-parallel, and
% data-parallel code. 
% Example transformations include {\em
% {\tt copy-scan} to {\tt copy-map}} and {\em {\tt map} function
% composition}. 

% Using both the static and dynamic dependence
% information, an interprocedural program slice of the statements
% affecting an examined loop is presented to the programmer. The
% programmer can then determine whether or not to parallelize the loop.
% They use a fairly standard interprocedural slicing algorithm to
% calculate the slice based on the determined dependencies. Because the
% SUIF system does not introduce parallel constructs to Fortran,
% concurrent thread execution is ignored by the static and dynamic
% analysis.

% Jrpm~\cite{Chen03} uses dynamic thread-level speculation to
% automatically parallelize Java loops. Building on top of the Hydra
% chip multiprocessor, which has hardware support for thread-level and
% data speculation, Jrpm profiles loops for dependency timing and buffer
% usage to determine if loop iterations can be parallelized. If the
% compiled statistics meet certain requirements, the Jrpm microJIT will
% recompile to create a speculative thread loop which spawns a new
% thread for every loop iteration. While some of the data dependency
% information generated could potentially be presented back to the user,
% this is not a practical approach to explicitly introduce parallelism
% in the code. The performance of this system is also tied to a
% particular multiprocessor architecture, and would not apply to systems
% that do not support hardware thread-level speculation.
% 
% Kamil and Yelick~\cite{Kamil05} developed a currency analysis for
% Titanium, a dialect of Java with explicit parallel {\em single
% program, multiple data} (SPMD) control. Titanium uses textually
% aligned barrier statements to control synchronization of parallel
% threads. Using these barriers, an analysis of the code is performed
% and a set of code phases is determined. An analysis to determine
% feasible call paths to and from procedures is also conducted. It is
% then determined that two statements may execute concurrently if they
% occur within the same code phase on a feasible path. The results of
% this analysis can then be used to find race conditions. While the
% analysis would be useful for determining loop carried dependencies in
% X10, the SPMD model is not compatible with X10's asynchronous activity
% model because of the lack of textually aligned barriers. Also, though
% this work has the capability to detect concurrency, the results do not
% provide insight where concurrency might be introduced.

% While this work is different from {\tt extract async/future} due to its
% loop-level parallelism, it was the first IDE which included parallel
% programming support with transformations. Also, it does not support {\em multi
% processor, distributed data} analysis.




\bibliographystyle{plainnat}
\begin{thebibliography}{99}

\bibitem{TSF}
F.~Bodin, Y.~M{\'e}vel, and R.~Quiniou.
\newblock A user level program transformation tool.
\newblock In {\em International Conference on Supercomputing}, pages 180--187,
  1998.

\bibitem{Charles05}
P.~Charles, C.~Grothoff, V.~Saraswat, C.~Donawa, A.~Kielstra, K.~Ebcioglu,
  C.~von Praun, and V.~Sarkar.
\newblock X10: an object-oriented approach to non-uniform cluster computing.
\newblock In {\em OOPSLA '05: Proceedings of the 20th annual ACM SIGPLAN
  conference on Object oriented programming, systems, languages, and
  applications}, pages 519--538, New York, NY, USA, 2005. ACM Press.

\bibitem{eclipse}
{Eclipse} home page.
\newblock \url{http://www.eclipse.org}.

\bibitem{ElGhazawi03}
T.~A. El-Ghazawi, W.~W. Carlson, and J.~M. Draper.
\newblock {U}{P}{C} language specifications v1.1.1, October 2003.

\bibitem[Hall et~al.(1993)Hall, Harvey, Kennedy, McIntosh, McKinley, Oldham,
  Paleczny, and Roth]{Hall93}
M.~W. Hall, T.~Harvey, K.~Kennedy, N.~McIntosh, K.~S. M\raisebox{.2em}{c}Kinley,
J.~D. Oldham, M.~Paleczny, and G.~Roth.
\newblock Experiences using the {Para{Scope} {Editor}}: an interactive parallel
  programming tool.
\newblock In \emph{Proceedings of the Fourth {ACM} {SIGPLAN} Symposium on
  Principles and Practice of Parallel Programming}, San Diego, CA, 1993.
\newblock URL \url{citeseer.ist.psu.edu/hall93experiences.html}.

\bibitem{Kennedy91}
K.~Kennedy, K.~S. M\raisebox{.2em}{c}Kinley, and C.-W. Tseng.
\newblock Analysis and transformation in the {Para{Scope} {Editor}}.
\newblock In {\em Proceedings of the 1991 {ACM} International Conference on
  Supercomputing}, Cologne, Germany, 1991.

\bibitem{Liao99}
S.-W. Liao, A.~Diwan, J.~Robert P.~Bosch, A.~Ghuloum, and M.~S. Lam.
\newblock {S}{U}{I}{F} {Explorer}: An interactive and interprocedural
  parallelizer.
\newblock In {\em PPoPP '99: Proceedings of the Seventh ACM SIGPLAN Symposium
  on Principles and Practice of Parallel Programming}, pages 37--48, New York,
  NY, USA, 1999. ACM Press.

\bibitem[Muraoka(1971)]{loopdist}
Yoichi Muraoka.
\newblock \emph{Parallelism exposure and exploitation in programs}.
\newblock PhD thesis, University of Illinois at Urbana-Champaign, Champaign,
  IL, USA, 1971.

\bibitem{Overbey05}
J.~Overbey, S.~Xanthos, R.~Johnson, and B.~Foote.
\newblock Refactorings for {Fortran} and high-performance computing.
\newblock In {\em SE-HPCS '05: Proceedings of the 2nd International Workshop on
  Software Engineering for High Performance Computing System Applications},
  pages 37--39, New York, NY, USA, 2005. ACM Press.

\bibitem[{S}{U}{I}{F} compiler system()]{SUIF}
{S}{U}{I}{F} compiler system.
\newblock The {S}{U}{I}{F} compiler system.
\newblock http://suif.stanford.edu/suif/suif2/.

\bibitem{X10}
V.~Saraswat.
\newblock Report on the experimental language {X10} v0.41.
\newblock http://www.research.ibm.com/x10/.

\bibitem{Yelick98}
K.~Yelick, L.~Semenzato, G.~Pike, C.~Miyamoto, B.~Liblit, A.~Krishnamurthy,
  P.~Hilfinger, S.~Graham, D.~Gay, P.~Colella, and A.~Aiken.
\newblock {Titanium}: {A} high-performance {Java} dialect.
\newblock In {ACM}, editor, {\em {ACM} 1998 Workshop on Java for
  High-Performance Network Computing}, New York, NY 10036, USA, 1998. ACM
  Press.

\end{thebibliography}

%\small{
%\bibliography{references}
%\bibliographystyle{plainnat}
%}

\end{document}
