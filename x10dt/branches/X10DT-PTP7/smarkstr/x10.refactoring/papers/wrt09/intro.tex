\section{Introduction}

The industry's shift to multicore processors has sparked a trend
toward pushing parallel programming into the mainstream.  This trend
poses a significant challenge, since creating and maintaining parallel
programs that are both efficient and reliable is notoriously
difficult.  One response to this challenge
by the programming languages community has been to create new (and
revisit old) language
abstractions and programming models 
for parallel programming and to
develop new languages based on these abstractions.  % For example, there
% has been lots of recent work on support for software transactional
% memory~\cite{}, and the DARPA PERCS project is funding several new
% parallel programming languages~\cite{}.

While new languages can greatly aid programmers in developing 
parallel programs, we believe that new languages cannot
achieve mainstream success without associated tooling support.
Modern integrated development environments
(IDEs) such as Eclipse~\cite{eclipse} provide many benefits that programmers
have come to rely upon, helping
them to more easily navigate through a program, understand
dependencies among parts of a program, and safely evolve a program to
improve its quality along some dimension.  The latter benefit is typically
provided through code {\em refactorings}.  
% Part of the difficulty in programming for concurrent systems is
% that they have often lagged behind in tooling support. Practitioners of most
% modern sequential languages have the advantage of employing a wide variety of
% analyses and refactorings in order to improve the quality of their code. While
% analysis research has continued to show forward momentum in the parallel
% language context, we believe that similar tooling support for refactorig in
% parallel languages will help users improve the quality of their parallel code;
% a belief that a number of others also share~\cite{Kennedy91, Liao99, Overbey05}.

We believe that
specialized refactoring support will be a critical tool to help programmers improve the
quality of parallel code.  Such refactorings could be used
to improve efficiency while preserving program behavior and key
concurrency invariants (e.g., atomicity, deadlock-freedom).
%Such
%refactorings could also be used to improve a parallel program's readability and
%extensibility.

Any set of refactorings will of necessity be tailored to the needs of a particular
parallel programming model and language.
We focus on providing refactoring support for the
\emph{partitioned global address space} (PGAS) memory model as
embodied in the X10~\cite{X10,Charles05}, UPC~\cite{ElGhazawi03} and
Titanium~\cite{Yelick98} programming languages.  In this model, the
programmer sees a uniform 
representation of data and data structures over distributed nodes, regardless
of the physical location of the data.  However, each piece of data is
assigned to a fixed partition (or {\em place} in X10 terminology)
and can only be accessed by {\em activities}
that run at that place.
% That is, a reference to a piece of
% data is treated as a first-class entity in this system and can be treated
% similarly to a local reference even if the owner of said data is at a
% different location. In such cases, the language's compiler or virtual machine
% will handle the messy details of handling message and data passing. The PGAS
% model attempts to minimize background communication overhead by
% locating data at a place where it has good data or processing affinity.

In this presentation, we focus on the X10 language, an object-oriented language
providing first-class, high-level constructs for asynchronous activities, synchronization,
phased computations, data distribution, and atomicity.
%Further, as stated in the introduction, the X10 language is in the family of
%concurrent languages with a {\em partitioned global address space} (PGAS)
%memory model. In the case of X10, the set of partitions in the global address
%space is fixed at the start of the program.
By incorporating such abstractions as first-class constructs in the language,
the burden of reasoning about various program properties is often reduced
from global reasoning involving complex control flow to simple and modular
reasoning about lexical containment.
In fact, many interesting properties (e.g. deadlock freedom) can often
be ensured statically.
In this way, X10's constructs simplify both the programmer's task of
understanding and tools' tasks of static program analysis.
% make all asynchronous and atomic code apparent
%to both the analysis engine and the programmer.

The PGAS model provides particular opportunities and challenges for
automated refactorings.  On the one hand, code transformations are
simplified since the code need not explicitly handle inter-processor
communication.  
On the other hand, transformations must properly handle the
asynchronicity that arises among activities and must respect
the synchronization constraints imposed on these activities by the
semantics of the various language constructs. 
% Transformations must also
% ensure that an activity only accesses data from the current place.
In the rest of this paper we describe an initial concurrency
refactoring for X10 that we
have been developing inside the X10DT plugin for Eclipse.

%%%%%%%%%%%%%%%%%%%%%%%%%%%%%%%%%%%%%%%%%%%%%%%%%%%%%%%%%%%%%%%%%%%%
% 
% Previous Version: for comparison
% 
% \section{Introduction}
% 
% Many modern parallel languages, particularly high-performance computing (HPC)
% languages, have trended towards a \emph{global address
% space} (GAS) memory model. In fact, all three of the remaining DARPA HPC
% Challenge programming languages have built their languages around a GAS
% model. A GAS model simplifies the development of code since the programmer
% does not have to deal with the explicit passing of data from one
% place of execution to another via interfaces such as MPI or OpenMP. Instead, a
% programmer can directly reference any desired data and the language's compiler
% or virtual machine will handle the messy details. This surface simplicity comes
% with a small caveat, though: a lot of extra communication may be incurred in
% order to actually determine which place actually owns some piece of data, all
% of which is hidden at the source level. One solution to this issue is to
% use a \emph{partitioned global address space} (PGAS) model in which data is
% created and kept in a single place for its lifetime, thus minimizing such
% communication overhead. The PGAS model itself is employed in a number of
% languages, such as UPC, Titanium, and X10~\cite{ElGhazawi03, Yelick98, Charles05}.
% 
% Even in languages with an abstracted memory model like the GAS model, creating
% efficient code poses a challenge. Programmers still struggle with such tasks
% as introducing maximal concurrency, avoiding deadlock, minimizing transaction
% conflicts, and numerous other plagues of the parallel world. Making
% such code readable and extensible is
% even more daunting. Most modern sequential languages often employ
% refactoring and other transformation engines to make code more efficient
% without sacrificing readability or introducing errors. We believe that
% similar tooling support for parallel languages can provide these benefits to the
% parallel programming community; a belief that a number of others also
% share~\cite{Kennedy91, Liao99, Overbey05}. 
% 
% We propose a framework in which a refactoring engine, standing alone or coupled
% with an IDE, is available to aid in the creation of parallel programs.
% The refactoring engine should be capable of performing its own program analysis
% to guarantee that transforming the code will not introduce new and subtle
% errors. Its associated transformation can also use the results of the analysis
% to transform the code appropriately. We feel that such sanity checks are vitally
% important to ensuring that new bugs are not introduced, in contrast to current
% commonly-used refactorings which are mostly unchecked but relatively
% cosmetic such as renaming variables or abstracting blocks as methods. Having
% such a refactoring engine allows a developer to write simple and readable
% code with the idea that the refactoring engine will improve the performance of
% the areas in the code that she has identified as good candidates for increased
% efficiency.

%%%%%%%%%%%%%%%%%%%%%%%%%%%%%%%%%%%%%%%%%%%%%%%%%%%%%%%%%%%%%%%%%%%%%


%%Since the advent of inter-processor communication, the desire to
%%create programs which take advantage of parallel execution has
%%existed. 
% The desire to take advantage of parallel execution has
% existed since the advent of the computer processor, and this desire is only
% more fervent in this age of processor ubiquity. However, the creation and
% maintenance of parallel programs involves the difficult challenges of
% enforcing proper program and memory consistency and synchronization.
%%To compound the issue,
%%A. J. Bernstein has already shown that, in the general case,
%%automatically determining parallelism of sequential code is an
%%undecidable problem~\cite{Bernstein66}.
% In the programming language community, two main schools of thought exist on
% solving these problems. One school of thought is to simply develop languages
% that included explicit constructs or compiler optimizations for parallelism.
%% These include variants of Fortran and Lisp and languages built with parallel
%% execution or data distribution support like Ada, Linda, and
%% Emerald~\cite{Allen87, Hutchinson87, Griss82, Ada, Gelernter85}. 
% Even with language level support, though, it is still difficult for
% programmers to quickly create error-free code which efficiently uses
% parallelism. Another approach is to automate the parallelization of
% sequential code via extensive program analysis at points in a program, such as
% loops, that seem to provide natural boundaries for parallel execution. However, this
% is complicated by the difficulty of
%%Fortunately, certain sequential programming language constructs do
%%seem to provide natural boundaries for automatic parallel compiler
%%optimizations: loops~\cite{Lamport74, Allen84, Allen87}. By
%%recognizing that the majority of time in running programs is spent
%%executing the same set of calculations over a different set of data,
%%it is natural to establish parallel execution of loop bodies on
%%different processors or using vector supercomputers. One drawback to
%%this approach, though, is that 
% precise static analysis of the parallelism targets.
% %% is a very difficult task. 
% Dependence on structured memory usage, such as
% arrays, which benefit the most from parallel execution, complicates
% loop analysis further.

% We believe that 
% %%Many research projects found that 
% the key for better parallelism detection is to take advantage of user
% interaction, 


%%As a result, user interactivity in the form of 
% In particular, we hold that
% source code refactoring support in IDEs such as Eclipse can be a
% powerful ally in aiding parallelization.
%%in the compiler optimization and program transformation process
%%A typical way of involving the user in this process
%%solution for these systems 
%%is to build support for some parallel transformations into an IDE and display
%%the
%%collected memory dependence information for a given line of code to the user. If
%%the user determines that the dependences would not prevent parallel
%%execution, then the user can choose to transform the code. For a whole
%%program analysis, such a presentation might be overwhelming and difficult to
%%decipher, and is probably not the best way to involve the user. Our approach
%%is to 

% We think that programmers often know which statements in their code could
% benefit the most from
% added efficiency, even if they do not know how best to introduce this
% efficiency on their own.
% %  and where they would like to insert more concurrency.
% Static and dynamic analysis tools can also help users identify hotspots
% in their programs.
% %In either case, the
% The tasks of identification of suitable parallelism candidates and of
% transforming the code to introduce/manipulate parallelism are separable.
%%Thus, programmers can specify which lines of code
%%would like to refactor into a concurrent form and 


% provides the link
% between these two vectors of concurrency introduction. The refactoring
% engine should determine if refactoring at these points is safe
% and how the code should be transformed to introduce more concurrency. A developer can then

% of which
% statements could benefit from added parallelism or use a tool to identify possible target sites
% %%, such as the code in Figure~\ref{fig:CHM-X10}, 
% and then use 


% Moreover,
% we believe that this approach can be applied to other application concerns of
% parallel programmers, such as introducing new synchronization points or
% refitting blocks as atomic.
%%, such as the code in
%%Figures~\ref{fig:CHM-X10-future} and~\ref{fig:CHM-X10-async}.
%%information is often difficult for a user to decipher for a whole program, though,
%%and is not Because following the flow of data for parallel programs is
%%difficult in these IDEs, advanced concurrency slicing techniques were
%%developed~\cite{Zhao99,Chen01,Chen02,Krinke03}. We believe that
%%including user input is an essential part of making parallelism
%%apparent to the programmer

%%\bug{Now introduce we will use X10 and why we like it: explicit higher-level
%%constructs and PGAS model} 

%%The target of our research is to develop practical refactoring schemes for
%%introducing increased parallelism in the X10 language. X10 includes explicit
%higher-level concurrency constructs such as asynchronous blocks and future
%%expressions. It also incorporates a {\em partitioned global address space}
%%(PGAS) model of data consistency. The combination of the explicit constructs
%%and the PGAS model in X10 allows data dependency analysis to be simplified.

%%Many recent languages have been developed with a focus on
%%the {\em partitioned global address space} (PGAS) model of data consistency
%%including UPC, Titanium, Co-Array Fortran, and X10~\cite{ElGhazawi03,
%%Yelick98, Numrich98, Charles05}. The target of our research is to develop
%%a practical refactoring scheme for X10, although this scheme should be
%%adaptable to other languages based on the PGAS model. 

% To deal with some of the static analysis issues associated with parallelization,
% we focus focus our refactorings on languages with a {\em partitioned global 
% address space} (PGAS) model of data consistency. In the PGAS model,
% %%The primary goal of the PGAS model is to allow parallel asynchronous activity to
% %%occur among multiple concurrent platforms, each having its own local memory and
% %%potentially lacking a joint shared memory, without losing the ability to read
% %%and write to global data. Global 
% shared data is actually owned by local processors but is globally addressable.
% Other processors which would like
% to access or update global data then communicate directly with the owning
% location to perform any necessary actions. Thus, programmers take an active role
% in defining how data in arrays or data structures is distributed
% over the address space so as to maximize locality, thus taking maximum advantage
% of their concurrent environments. This removes the programmer's (or analysis')
% burden of determining
% how complicated structures should be partitioned to the various execution sites at
% refactoring points in the code where parallelism is desired (e.g., inside 
% loops). As a result, the required static analysis is reduced to determining
% local and loop-carried dependencies that prevent a statement from being
% asynchronously executed.\bug{This statement really applies to this particular
% transformation. I.e., the programmer does have to determine the partitioning at
% some point, but doesn't need to worry about it while applying this
% transformation. So perhaps we should say that we envision different kinds of
% {\em lateral} moves, e.g.: manipulate concurrency while keeping the distribution
% fixed, and manipulating the distribution while keeping concurrency relatively
% unchanged.}
% 
% We chose to implement our refactoring scheme in the X10 language. X10 not only
% has a PGAS data consistency model, but also includes explicit higher-level
% concurrency constructs such as asynchronous blocks and future
% expressions. These concurrency constructs further simplify the analysis of X10
% code by making all asynchronous and atomic code apparent to both the analysis
% engine and the programmer.
% 
% %%\begin{figure}[tp]
% %%  \begin{code}
% %%    int mcsum=0; \\
% %%    fo\=r (i=0; i<segments.length; i++)\{ \\
% %%    \>  mcsum += mc[i] = segments[i].modCount(); \\
% %%    \>  if\=(segments[i].containsValue(value)) \\
% %%    \>\>    return true; \\
% %%    \} \\
% %%  \end{code}
% %%\caption{\label{fig:CHM} A Java code excerpt from the library class {\tt
% %%java.util.concurrent.ConcurrentHashMap} which illustrates a loop that
% %%would not be parallelizable via traditional automatic loop
% %%parallelization methods.}
% %%\end{figure}
% 
% %%Because languages designed for the PGAS model do not focus on
% %%providing support for the parallel execution of the same code over
% % different sets of data, they can provide a more flexible alternative
% % to whole loop parallelization: the loops can be executed sequentially
% % and updates to distributed data structures can occur in parallel. 
% As an example, consider the X10 code in Figure~\ref{fig:CHM-X10}, an excerpt 
% from an X10 implementation of the {\tt java.util.concurrent.ConcurrentHashMap} 
% class. In this example, it is possible the {\tt modCount} and {\tt 
% containsValue} method invocations are a both very expensive operations. Because 
% of the PGAS model, the individual elements of the array {\tt segments} could 
% exist anywhere in the global address space. However, automatic whole loop 
% parallelization techniques developed previously would fail on this code because 
% of the conditional return statement and the loop carried dependency summing
% the results of the calls to {\tt modCount}. However, the programmer might still
% be able to take advantage of parallel execution by making the invocations of
% {\tt modCount} asynchronous and caching the results.
% Figures~\ref{fig:CHM-X10-async} and~\ref{fig:CHM-X10-future} show examples of
% adding concurrency in this fashion.
% 
% We present in this paper the source code transformation {\em extract concurrent} for
% the X10 language which allows a programmer to transform loop code
% like that in Figure~\ref{fig:CHM-X10} to one that takes maximum advantage of
% asynchronous execution. We acheive this refactoring through two means:
% 
% \begin{enumerate}
% \item {\em Loop dependence analysis.} Since introducing parallelism in
% the middle of a loop might affect the ability of other statements in a
% loop to properly evaluate, it is important that loops do not carry
% dependence on the results of any asynchronously evaluated statements. We
% have developed a series of analyses to determine whether {\em extract
% concurrent} will adversely affect the execution of the code and
% violate its sequential consistency.
% 
% \item {\em Transformation pattern.} We have developed a general
% pattern for the {\em extract concurrent} transformation on viable
% sequential loops. This pattern uses program slicing
% techniques to split a loop in two. The first loop will allow introduction of
% parallelism while the second loop utilizes the results of the asynchronous
% execution. Because this split requires some code duplication, we
% present the results of a formal analysis on how the transformation
% affects the runtime of the loop and define the conditions under which
% the transformation provides the potential for better runtime. In practice, this
% transformation is more widely applicable to multiple statements or expressions.
% We present here only the single statement or expression case.
% \end{enumerate}
% 
% We have implemented the transformation as refactoring in the X10DT, a
% development framework for the X10 language in Eclipse, however we believe that
% the transformation is adapatable to all languages with a PGAS consistency model.
% \bug{We will insert more about the implementation when it's actually been done.}

% The rest of the paper is organized as follows. Section 2 presents an
% overview of the X10 programming language. Section 3 contains a
% description and analysis of the the algorithm for determining loop
% candidacy for the transformation. Section 4 details the {\em extract
% concurrent} transformation and discusses the impact of the
% transformation on running time. Section 5 presents the implementation
% and evaluation of the transformation in X10DT. Section 6 describes the
% related work and Section 7 contains the conclusion and a summary of
% the paper.
