%-----------------------------------------------------------------------------
%
%               Template for sigplanconf LaTeX Class
%
% Name:         sigplanconf-template.tex
%
% Purpose:      A template for sigplanconf.cls, which is a LaTeX 2e class
%               file for SIGPLAN conference proceedings.
%
% Author:       Paul C. Anagnostopoulos
%               Windfall Software
%               978 371-2316
%               paul@windfall.com
%
% Created:      15 February 2005
%
%-----------------------------------------------------------------------------


\documentclass[preprint]{sigplanconf}

% The following \documentclass options may be useful:
%
% 10pt          To set in 10-point type instead of 9-point.
% 11pt          To set in 11-point type instead of 9-point.
% authoryear    To obtain author/year citation style instead of numeric.

\usepackage{amsmath}

\begin{document}

\conferenceinfo{WXYZ '05}{date, City.} 
\copyrightyear{2005} 
\copyrightdata{[to be supplied]} 

%\titlebanner{banner above paper title}        % These are ignored unless
%\preprintfooter{short description of paper}   % 'preprint' option specified.

\title{Static Analysis and Optimization of Communication Patterns in X10}
%\subtitle{Subtitle Text, if any}

\authorinfo{Baolin Shao}
           {Columbia University}
           {bshao@cs.columbia.edu}
\authorinfo{Olivier Tarduer}
           {IBM T.J Watson Research Center}
           {tardieu@us.ibm.com}
\authorinfo{Stephen A. Edwards}
           {Columbia University}
           {sedwards@cs.columbia.edu}
\
\maketitle

\begin{abstract}
Dynamic parallelism is a model of computation, which allows arbitrary number of 
asynchronous activities spawned on the fly. These activities operate in parallel, 
and are sequenced by synchronization primitives when necessary. 
The synchronization problem, a form of Distributed Termination Detection (DTD), 
is a classical problem in distributed computing. It is difficult to achieve efficiently, 
because solving this problem requires the knowledge of a system's global state. 

We tackle the termination detection problem in X10, an object-oriented 
parallel programming language. It supports dynamic parallelism and global 
barriers. For example, the statement \emph{finish s} continues its execution 
when \emph{s} terminates, and is implemented in X10's standard library 
by a classic DTD algorithm. 
Therefore, in the paper we identify four special patterns for \emph{finish} 
together with four optimized algorithms. 
We also present a static analysis technique that will find out which pattern 
a \emph{finish} belongs to. Finally we report our findings for a small
set of benchmarks on IBM's supercomputer, BlueGens. 

\end{abstract}

\category{CR-number}{subcategory}{third-level}

\terms
term1, term2

\keywords
static analysis, communication pattern, scalability, x10 programming language
\section{Introduction}

\section{The X10 Programming Language}
\subsection{\emph{async}}
\subsection{\emph{finish}}
\subsection{\emph{ateach}}
\section{Communication Graph Construction and Analysis}
\section{Code Optimizer}
\section{Results}
\section{Related Work}
\section{Conclusion}
%\appendix
%\section{Appendix Title}

%This is the text of the appendix, if you need one.

%\acks

%Acknowledgments, if needed.

% We recommend abbrvnat bibliography style.

\bibliographystyle{abbrvnat}

% The bibliography should be embedded for final submission.

\begin{thebibliography}{}
\softraggedright

\bibitem[Smith et~al.(2009)Smith, Jones]{smith02}
P. Q. Smith, and X. Y. Jones. ...reference text...

\end{thebibliography}

\end{document}
