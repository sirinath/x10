%\documentclass[onecolumn,12pt]{ieee}
\documentclass[10pt]{article}
%------------------------------------------------------------------------- 
% 
% Use \documentclass[pagenumbers]{ieee}
%
% to produce page numbers, bottom centered, in the output. Default is 
% no page numbers for camera-ready copy.
%
%------------------------------------------------------------------------- 
\usepackage[dvips,letterpaper,margin=1in]{geometry}
\usepackage[pdftex]{graphicx}
\usepackage{times}
\usepackage{epsfig}
\usepackage{enumerate, amscd,  amsmath}  
\usepackage{amssymb}
\usepackage{algorithm2e}
\usepackage{url}
\numberwithin{equation}{section}

%\newenvironment{javaen}{\begin{small}\begin{verbatim}}{\end{verbatim}\end{small}}
\newcommand{\java}{\tt}
\def\Xten{{\sf X10}}
\def\XWS{{\sf XWS}}

\renewcommand{\baselinestretch}{1.0}

\begin{document}

\DeclareGraphicsExtensions{.jpg, .pdf, .mps, .gif, .png}
\title{Solving Large, Irregular Graph Problems in \Xten}

\author{
Guojing Cong,  Vijay Saraswat, and Tong Wen\\
IBM T. J. Watson Research Center\\
 \{ gcong, vsaraswa, tongwen \}@us.ibm.com\\ 
\vspace*{-2ex} \\
Sreedhar Kodali\\
IBM Systems and Technology Group\\
srkodali@in.ibm.com\\
\vspace*{-2ex} \\
Sriram Krishnamoorthy \\
Ohio State University\\
krishnsr@cse.ohio-state.edu
\vspace*{-3ex} \\
}

\date{}

\maketitle
\thispagestyle{empty}

\input abstract
\input introduction

%------------------------------------------------------------------------- 
\input introduction2X10
%------------------------------------------------------------------------- 

%\input x10graphdesign
\input runtime
\input perf-eval
\input conclusion
{\footnotesize
\bibliographystyle{ieee}
\bibliography{paper,parallel}
}
\end{document}

