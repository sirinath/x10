\documentclass[10pt]{sigplanconf}

\newif\iflncs
\lncsfalse

\usepackage{times-lite}
\usepackage{mathptm}
\usepackage{txtt}
\usepackage{stmaryrd}
\usepackage{code}
\usepackage{bcprules}
%\usepackage{ttquot}
\usepackage{amsmath}
\usepackage{amssymb}
\usepackage{afterpage}
\usepackage{balance}
\usepackage{floatflt}
\usepackage{defs}
\usepackage{pldefs}
\usepackage{utils}
\usepackage[pdftex]{graphicx}
\usepackage{ifpdf}
\newif\iflstworks
\lstworkstrue
\iflstworks
\usepackage{listings}
\fi

\hfuzz=1pt

\iflstworks
\lstdefinelanguage{X10}%
  {morekeywords={abstract,break,case,catch,class,%
      const,continue,default,do,else,extends,false,final,%
      finally,for,goto,if,implements,import,instanceof,%
      interface,label,native,new,null,package,private,protected,%
      public,return,static,super,switch,synchronized,this,throw,%
      throws,transient,true,try,void,volatile,while,%
      async,atomic,when,foreach,ateach,finish,clock,clocked,%
      here,%
      self,property,nullable,future},%
   sensitive,%
   morecomment=[l]//,%
   morecomment=[s]{/*}{*/},%
   morestring=[b]",%
   morestring=[b]',%
  }
\fi

\newcommand\codestyle\tt

\iflstworks
\lstnewenvironment{displayxten}
  {\lstset{language=X10,basicstyle=\codestyle,tabsize=4,columns=fullflexible,captionpos=b,xleftmargin=1em,xrightmargin=1em,mathescape=true,keepspaces=true,keywordstyle=}}
  {}
\lstnewenvironment{xten}
  {\lstset{language=X10,basicstyle=\codestyle\footnotesize,tabsize=4,columns=fullflexible,captionpos=b,keywordstyle=,numbers=left,numberstyle=\tiny}}
  {}
\newcommand\inputxten[1]{\lstinputlisting[language=X10,basicstyle=\codestyle\footnotesize,columns=fullflexible,tabsize=4,captionpos=b,keepspaces,keywordstyle=,numbers=left,numberstyle=\tiny]{#1}}

%\newenvironment{displayxten}{\begin{xten}}{\end{xten}}


\newcommand{\xcd}[1]{{\lstinline[language=X10,basicstyle=\codestyle,mathescape=true,columns=flexible,breaklines,keywordstyle=]{#1}}}
\newcommand{\tcd}[1]{{\codestyle{#1}}}

\else
\newcommand{\xcd}[1]{{\tt #1}}
\newcommand{\tcd}[1]{{\tt #1}}
\fi

\pagestyle{plain}


\ifpdf
\setlength{\pdfpagewidth}{8.5in}
\setlength{\pdfpageheight}{11in}
\fi


% \input{../../../../vj/res/pagesizes}
% \input{../../../../vj/res/vijay-macros}
\newcommand\alt{\bnf}

\newcommand\Implies{\Rightarrow}

\iflncs
\else
\newtheorem{example}{Example}[section]
\newtheorem{theorem}{Theorem}[section]
\newtheorem{lemma}[theorem]{Lemma}
\newenvironment{proof}{
\trivlist
\item[\hskip \labelsep \textsc{Proof.}]
\selectfont
\ignorespaces}{$\Box$}

%\newtheorem{proof}[theorem]{Proof}
\fi

\newcommand\Xten{{\sf X10}}
\newcommand\DML{{DML}}
\newcommand\FJ{{FJ}}
\newcommand\CFJ{{CFJ}}
\newcommand\java{{Java}}
\newcommand\Java{{\java}}
\newcommand\csharp{{C$\sharp$}}
\newcommand\FXten{{FX10}}

\begin{document}

\iflncs
\title{Constrained Types for Object-Oriented Languages}
\titlerunning{Constrained Types}

\author{Vijay Saraswat\inst{1} \and Nathaniel Nystrom\inst{1}
\and Jens Palsberg\inst{2} \and Christian Grothoff\inst{3}}
\authorrunning{Vijay Saraswat et al.}

\institute{IBM T.~J. Watson Research~Center, P.O.~Box~704, Yorktown~Heights NY 10598 USA,
\email{\{vsaraswa,nystrom\}@us.ibm.com}
\and
UCLA~Computer~Science~Department,
Boelter~Hall, Los~Angeles CA 90095 USA,
\email{palsberg@cs.ucla.edu}
\and
Department~of~Computer~Science, University~of~Denver,
2360~S.~Gaylord~Street, John~Green~Hall, Room~214, Denver~CO, 80208 USA,
\email{christian@grothoff.org}}

\else

\title{Constrained Types for Object-Oriented Languages}
\authorinfo{Vijay Saraswat\titlenote{IBM T.~J. Watson Research
Center, P.O. Box 704, Yorktown Heights NY 10598 USA}}{}
  {vsaraswa@us.ibm.com}
\authorinfo{Nathaniel Nystrom$^{\;*}$}{}
  {nystrom@us.ibm.com}
%\authorinfo{Radha Jagadeesan\titlenote{School of CTI, DePaul
%University, 243 S. Wabash Avenue, Chicago IL 60604 USA}}{}
  %{rjagadeesan@cs.depaul.edu}
\authorinfo{Jens Palsberg\titlenote{UCLA Computer Science
Department, Boelter Hall, Los Angeles CA 90095 USA}}{}
  {palsberg@cs.ucla.edu}
\authorinfo{Christian Grothoff\titlenote{
Department of Computer Science, University of Denver,
2360 S. Gaylord Street, John
Green Hall, Room 214, Denver CO, 80208 USA}}{}
  {christian@grothoff.org}

% \conferenceinfo{POPL'08}{XXX}
% \copyrightyear{2008}
% \copyrightdata{[to be supplied]}

\fi

\maketitle

\begin{abstract}
Modern object-oriented languages such as \Xten require a rich framework
for types capable of expressing value-dependency, type-dependency
and supporting pluggable, application-specific extensions.

In earlier work, we presented the framework of \emph{constrained
types} for concurrent, object-oriented languages, parametrized by
an underlying constraint system $\cal X$. Constraint systems are a
very expressive framework for partial information. Types are viewed
as formulas \Xcd{C\{c\}} where \Xcd{C} is the name of a class
or an interface and \Xcd{c} is a constraint in $\cal X$ on the
immutable instance state of \Xcd{C} (the \emph{properties}).
Many (value-)dependent type systems for object-oriented languages
can be viewed as constrained types.

This paper extends the constrained types approach to handle
\emph{type-dependency} (``genericity''). The key idea is to extend
the constraint system to include predicates over types such as
\Xcd{X} is a subtype of \Xcd{T}.  Generic types are supported
by introducing type parameters and permitting programs to impose
constraints on such parameters.

To illustrate the underlying theory, we develop a formal family of
programming languages with a common set of sound type-checking rules
parameterized on a constraint system $\cal X$.  By varying $\cal X$
and extending the type system, we obtain languages with the power
of \FJ, \FGJ, and languages that provide dependent types, structural
subtyping, and constraints that relate values and types.  The core
of the \Xten language is a concrete instantiation of the framework.  We
describe the design and implementation of \Xten, which is available
for download at \texttt{x10-lang.org}.


\end{abstract}

\iflstworks
\section{Introduction}\label{sec:intro}

%
%I. Introduction and overview (3 pages)

%Design of X10, concurrency, high productivity, high performance,
%practical language.
%
%Dependent types arise naturally. arrays, regions, distributions, place
%types.
%
%Indeed you can look around and recognize many OO type systems proposed
%in the last decade or so as specific kinds of applied dependent type
%systems.
%
%Our goal is to develop a general framework for dependent types for
%statically typed OO languages ("Java-like languages"). 

\Xten{} is a modern statically typed object-oriented
language designed for high productivity in the high performance
computing (HPC) domain~\cite{X10}. Built essentially on
sequential imperative object-oriented core
similar to
Scala or
$\mbox{Java}^{\mbox{\scriptsize\sc tm}}$,
\Xten{} introduces constructs for distribution and
fine-grained concurrency (asynchrony, atomicity, ordering).

The design of \Xten{} requires a rich type system to permit a
large variety of errors to be ruled out at compile time and to 
generate efficient code.  
\Xten{}, like most object-oriented languages supports classes;
however, it places
equal emphasis on {\em arrays}, a central data structure in high
performance computing.
In particular, \Xten{} supports dense,
distributed multi-dimensional arrays of value and reference types,
built over index sets known as {\em regions}.%, and mappings from index
%sets to places, known as {\em distributions}.  \Xten{} supports a rich
%algebra of operations over regions, distributions and arrays.

A key goal of \Xten{} is to rule out large classes of error
by design. For instance, the possibility of indexing a 2-d array with 3-d
points should simply be ruled out at compile-time. This means that one
must permit the programmer to express types such as \xcd{Region(2)},
the type of all two-dimensional regions;
\xcd{Array[int](5)}, the
type of all arrays of \xcd{int} of length \xcd{5};
\xcd{Array[int](Region(2))}, the type of all \xcd{int} arrays
over two-dimensional regions; and
\xcd{Tree\{loc==here\}}, the type of all \xcd{Tree} objects located on the
current node. For concurrent computations, one needs the ability to
statically check that a method is being invoked by an activity that is
registered with a given clock (i.e., dynamic barrier)~\cite{X10}.

For performance, it is necessary that array index accesses are
bounds-checked statically.  Further, certain regions (e.g.,
rectangular regions) may be represented particularly
efficiently.  Hence, if a variable is to range only over
rectangular regions, it is important that this information is
conveyed through the type system to the code generator.

In this paper we describe {\Xten}'s support for {\em
constrained types},
a form of {\em dependent
type}~\cite{dependent-types,xi99dependent,ocrz-ecoop03,aspinall-attapl,cayenne,epigram-matter,calc-constructions}---types parametrized by values---defined 
on predicates over the {\em immutable}
state of objects. Constrained types statically capture many common invariants
that naturally arise in code. For instance, typically the shape of an
array (the number of dimensions (the rank) and the size of each dimension)
is determined at
run time, but is fixed once the array is constructed. Thus, the shape of an
array is part of its immutable state.
Both mutable and immutable variables may have a constrained
type: the constraint specifies invariants on the immutable state
of the object stored in the variable. 

\Xten{} provides a framework for specifying and checking constrained types
that achieves certain desirable properties:
\begin{itemize}
\item 
{\bf Ease of use.}  
The syntax of constrained types is a simple and
natural extension of nominal class types.
% Constrained types in
% \Xten{} interoperate smoothly with Java libraries.

\item
{\bf Flexibility.}
The framework
permits the development of concrete,
specific type systems tailored to the application area at
hand.  \Xten{}'s compiler permits extension with different constraint systems
via compiler plugins, enabling a kind of pluggable type system~\cite{bracha04-pluggable}.
The framework is parametric in the kinds of
expressions used in the type system, permitting the installed constraint
system to interpret the constraints.

\item
{\bf Modularity.}
The rules for type-checking
are specified once in a way that is independent of the
particular vocabulary of operations used in the dependent type
system.
The type system supports separate compilation.

\item
{\bf Static checking.}  The framework permits mostly static
type-checking. The user is able to escape the confines of
static type-checking using dynamic casts.
%, as is common for Java-like
%languages.
\end{itemize}

\subsection{Constrained types}

\begin{figure}[t]
{
\footnotesize
\begin{xtenlines}
class List(n: int{self >= 0}) {
  var head: Object = null;
  var tail: List(n-1) = null;

  def this(): List(0) { property(0); }

  def this(head: Object, tail: List): List(tail.n+1) {
    property(tail.n+1);
    this.head = head;
    this.tail = tail;
  }

  def append(arg: List): List(n+arg.n) {
    return n==0
      ? arg : new List(head, tail.append(arg));
  }

  def reverse(): List(n) = rev(new List());
  def rev(acc: List): List(n+acc.n) {
    return n==0
      ? acc : tail.rev(new List(head, acc));
  }

  def filter(f: Predicate): List{self.n <= this.n} {
    if (n==0) return this;
    val l: List{self.n <= this.n-1} = tail.filter(f);
    return (f.isTrue(head)) ? new List(head,l) : l;
  }
}
\end{xtenlines}
}
\caption{
This program implements a mutable list of Objects. The size of a list
does not change through its lifetime, even though at different points
in time its head and tail might point to different structures.}
\label{fig:list-example}
\end{figure}

X10's sequential syntax is similar to Scala's.
We permit the definition of a class \xcd{C} to specify
a list of typed parameters or {\em properties},
${\tt f}_1: {\tt T}_1, \dots, {\tt f}_k: {\tt T}_k$,
similar in syntactic structure to a method formal parameter list.
%
Each property in this list is treated as a public final instance field.
%
We also permit the
specification of a {\em class invariant}
%, a {\em where clause}~\cite{where-clauses}
in the class definition. A class invariant
is a boolean expression on the properties of the class.
The compiler ensures that all
instances of the class created at run time satisfy the invariant.
%
Syntactically, the class invariant follows the property list.
%
For instance, we may specify a class \xcd{List} with an
\xcd{int} \xcd{length} property as follows:
\begin{xtennoindent}
  class List(length: int){length >= 0} {...}
\end{xtennoindent}
Given such a definition for a class \xcd{C}, types can be
constructed by {\em constraining} the properties of \xcd{C}.  In
principle, {\em any} boolean expression over the properties
specifies a type: the type of all instances of the class
satisfying the boolean expression. Thus, \xcd{List\{length == 3\}}
is a permissible type, as are \xcd{List\{length <= 42\}} and
even \xcd{List\{length * f() >= g()\}} where \xcd{f} and \xcd{g}
are functions on the immutable state of the \xcd{List} object
and the variables in scope where the type appears.
In practice, the constraint expression is restricted by the
particular constraint system in use.

Our basic approach to introducing constrained types into \Xten{}
is to follow the spirit of generic types, but to use values
instead of types.

In general, a {\em constrained type} is of the form \xcd{C\{e\}},
the name of a class or interface\footnote{In \Xten{}, primitive
types such as \xcd{int} and \xcd{double} are object types; thus,
for example, \xcd{int\{self==0\}} is a legal constrained type.}
\xcd{C}, called the {\em base class}, followed
by a {\em condition} \xcd{e},
a boolean expression on the properties of the
base class and the final variables in scope at the type.
Such a type represents a refinement of \xcd{C}: the set of all
instances of \xcd{C} whose immutable state satisfies the
condition \xcd{e}.
%
We write \xcd{C} for 
the vacuously constrained type \xcd{C\{true\}}, and
write
\tcd{C(${\tt e}_1,\ldots,{\tt e}_k$)} for
the type
\tcd{C\{${\tt f}_1$==${\tt e}_1,\ldots,{\tt f}_k$==${\tt e}_k$\}}
where \xcd{C} declares the $k$ properties
${\tt f}_1,\ldots,{\tt f}_k$.

Constrained types may occur wherever normal types occur. In
particular, they may be used to specify the types of properties,
(possibly mutable) local variables or fields,
arguments to methods, return types of methods; they may also be
used in casts, etc.

Using the definitions above, \xcd{List(n)}, shown in
Figure~\ref{fig:list-example}, is the type of all lists of
length \xcd{n}.
%
Intuitively, this definition states that a \xcd{List} has an \xcd{int}
property \xcd{n}, which must be non-negative.
The properties
of the
class are set through the invocation of \xcd{property}\tcd{(\ldots)}
(analogously to \xcd{super}\tcd{(\ldots)}) in the constructors
of the class.

In a constraint, the name \xcd{self} is bound and refers to the type being
constrained.  The name \xcd{this}, by contrast, is a free
variable in the
constraint and refers to the receiver parameter of the current
method or constructor.  Use of \xcd{this} is not permitted in static
methods.

The \xcd{List} class has two
fields that hold the head and tail of the list.  The fields are
declared with the \xcd{var} keyword, indicating that they are
not final.  Variables declared with the \xcd{val} keyword, or
without a keyword are final.

Constructors have ``return
types'' that can specify an invariant satisfied by the object being
constructed.  The compiler verifies that the
constructor return type and the class invariant are implied by the
\xcd{property} statement and any \xcd{super} calls in the constructor
body.
A constructor must either invoke another constructor of the same
class via a
\xcd{this} call
or must have a \xcd{property} statement on every
non-exceptional path
to ensure the properties are initialized.
The \xcd{List} class has two constructors: the first
constructor returns an empty list;
the second
returns a list of length \xcd{m+1}, where \xcd{m} is the length
of the second argument. 

In the second constructor (lines 7--11), as well as 
the \xcd{append} (line 13) and \xcd{rev} (line 20) methods,
the return type
depends on properties of the formal parameters. 
If an argument appears in a
return type then the parameter must be final,
ensuring the
argument points to the same object throughout the evaluation of
the method or constructor body.  A parameter may also depend on
another parameter in the argument list.

The use of constraints makes existential types very natural.
Consider the return type of \xcd{filter} (line 24): it specifies
that the list returned is of some unknown length. The only thing
known about it is that its size is bounded by \tcd{n}.
Thus,
constrained types naturally subsume existential dependent types.
Indeed, every base type \xcd{C} is an ``existential''
constrained type since it does not specify any constraint on its
properties. Thus, code written with constrained types can
interact seamlessly with legacy library code---using just base
types wherever appropriate.

The return type of \xcd{filter} also illustrates the difference
between \xcd{self} and \xcd{this}.  Here, \xcd{self} refers to
the \xcd{List} being returned by the method; \xcd{this} refers
to the method's receiver.

\subsection{Constraint system plugins}

The \Xten{} compiler allows  
programmers to extend the semantics of the language with
compiler plugins.  Plugins may be used to support different constraint
systems to be used in constrained types.
Constraint systems provide code for checking consistency and
entailment.

The condition of a constrained type is parsed and type-checked
as a normal boolean expression over properties and
the \xcd{final} variables in scope at the type.  Installed
constraint systems translate the expression into an internal
form, rejecting expressions that cannot be represented.
%
A given condition may be a conjunction of constraints from
multiple constraint systems.
A Nelson--Oppen procedure~\cite{nelson-oppen} is used to check
consistency of the constraints.

The \Xten{} compiler
implements a simple
equality-based constraint system.  Constraint solver plugins
have been implemented for inequality constraints, for
Presburger constraints using
the CVC3 theorem prover~\cite{cvc}, and for
set-based constraints also using CVC3.
These constraint systems are described in
Section~\ref{sec:examples} and the implementation is
discussed in Section~\ref{sec:impl}.

\subsection{Claims}

The paper presents constrained types in the \Xten{} programming
language.
We claim that the design is natural, easy to use, and useful. Many
example programs have been written using constrained types and are
available at {\tt x10.sf.net/\allowbreak applications/\allowbreak examples}.

As in staged languages~\cite{nielson-multistage,ts97-multistage}, the
design distinguishes between compile-time and run-time
evaluation. Constrained types are checked (mostly) at compile-time.
The compiler uses a constraint solver to perform universal reasoning
(e.g., ``for all possible values of method parameters'') for dependent
type-checking.  There is no run-time constraint-solving.  However,
run-time casts and \xcd{instanceof} checks involving dependent types
are permitted; these tests involve
arithmetic, not algebra---the values of all parameters are known.

The design supports separate compilation: a class needs to be
recompiled only when it is modified or when the method
and field signatures or invariants of classes on which it
depends are modified.

We claim that the design is flexible. The language design is
parametric on the constraint system being used.
%We are planning on extending the current
%implementation to support multiple user-defined constraint systems,
%thereby supporting pluggable types.
The compiler supports
integration of
different constraint solvers into the language.
Dependent clauses  also form
the basis of a general user-definable annotation framework we have
implemented separately~\cite{ns07-x10anno}. 

We claim the design is clean and modular. We present a simple core
language \CFJ, extending \FJ{}~\cite{FJ} with constrained types on top
of an arbitrary constraint system. We present rules for type-checking
\CFJ{} programs that are parametric in the constraint system
and establish subject reduction and progress theorems. 

%
% XXX contrast with hybrid type checking.

\paragraph{Rest of this paper.}

Section~\ref{sec:lang} describes the syntax and semantics of
constrained types.
Section~\ref{sec:examples} works through a number of
examples using a variety of constraint systems.
The compiler implementation, including support for constraint
system plugins, is described Section~\ref{sec:impl}.
A formal semantics for a core language with constrained types 
is presented in Section~\ref{sec:semantics}, and a soundness
proof is presented in the appendix.
Section~\ref{sec:related} reviews related work.
The paper concludes in Section~\ref{sec:future}
with a discussion of future work.




\section{Constrained types}\label{sec:lang}
%II. Language design and rules (CFJ) (3.5 pages)

\label{method-sec}
This section describes constrained types in \Xten{}.

\subsection{Properties}

A property is a \xcd{public} \xcd{final} instance field of the
class that cannot be overridden by subclassing. Like any other field,
a property is typed, and its type need not necessarily be
primitive. Thus, properties capture the immutable public state of an
object, initialized when the object is created, that can be
classified by constrained types. Syntactically, properties are
specified in a parameter list right after the name of the class in a
class definition. The class body may contain specifications of other
fields; these fields are considered mutable.

Properties may be of arbitrary type. For instance, the class
\xcd{
region} has an \xcd{int} property called \xcd{rank}. In turn, the
class \xcd{dist} has a \xcd{region} property, called \xcd{region}, and
also an \xcd{int} property \xcd{rank}.  The invariant for \xcd{dist}
ensures that \xcd{rank == region.rank}. Similarly, an array has
properties \xcd{dist}, \xcd{region}, and \xcd{rank}
and appropriate constraints ensuring that the statically available
information about them is consistent.\footnote{All constraint
languages used in constrained types permit object references, field
selection and equality. Such constraint systems have been studied
extensively under the name of ``feature structures''
\cite{feature-structures}.}
%
In this way, rich  
constraints on the immutable portion of
the object reference graph, rooted at the current object and utilizing
objects at user-defined types, may be specified.

\subsection{Constraints}
A constrained type is of the form \xcd{C(:e)}, consisting of a
{\em base class} \xcd{C} and a {\em condition} \xcd{e}, a
boolean expresion on the properties of the base class and the
\xcd{final} variables in scope at the type.  Constraints specify
(possibly) partial information about the variables of interest.
Such a type represents a refinement of \xcd{C}---the set of all
instances of \xcd{C} whose immutable state satisfies the
condition \xcd{c}.

Constraints may use the special variable \xcd{self} to stand for
the object whose type is being defined. Thus, \xcd{int(:self >= 0)} is
the set of natural numbers, and \xcd{point(:x*x + y*y <= 1.0)}
represents the interior of a circle (for a class \xcd{point} with two
\xcd{float} properties \xcd{x} and \xcd{y}). The type \xcd{C(:self !=
null)} represents all instances of \xcd{C}. When there is no
ambiguity, a property reference \xcd{self.x} may be abbreviated
to \xcd{x}. The type \xcd{int(:self==v)} represents a ``singleton'' type, an
\xcd{int} is of this type only if it has the same value as \xcd{v}.

To be clear, \xcd{self} is not the same as \xcd{this}.  In the
\xcd{List} example of Figure~\ref{fig:list-example}, a list
a list with type
\xcd{List(:self.n <= this.n)} is returned by the 
\xcd{filter} method:  \xcd{self.n} is the length of
the returned \xcd{List}; \xcd{this.n} is the length of the 
receiver of the call to \xcd{filter}.

Constraints are specified in terms of an underlying
constraint system~\cite{CCCC}---a pre-defined logical vocabulary of
functions and predicates with algorithms for consistency and
entailment.  The \Xten{} compiler permits different constraint
systems to be installed using compiler plugins.~\cite{??}.
%
Constraint system plugins define a language of constraints by
symbolically interpreting
the boolean expression specifying a type's condition; plugins
may report an error if the condition cannot be interpreted.

In principle, types may be constrained by any boolean
expression over the properties.  For practical reasons,
restrictions need to be imposed to ensure constraint checking is
decidable.

The condition of a constrained type must be a pure
function only of the properties of the base class.
Because properties are
\xcd{final} instance fields of the object,
this requirement
ensures that whether or not an object belongs to a constrained type does
not depend on the {\em mutable} state of the object.
That is, the status of the
predicate ``this object belongs to this type'' does not
change over the lifetime of the object.  Second, by insisting that each
property be a {\em field} of the object, the question of
whether an object is of a given type can be
determined merely by examining the state of the object and evaluating
a boolean expression. Of course, an implementation is free to not {\em
explicitly} allocate memory in the object for such fields. For
instance, it may use some scheme of tagged pointers to implicitly
encode the values of these fields.

Further, by requiring that the programmer distinguish certain
\xcd{
final} fields of a class as properties, we ensure that the programmer
consciously controls {\em which} \xcd{final} fields should be available for
constructing constrained types. A field that is ``accidentally''
\xcd{
final} may not be used in the construction of a constrained type. It must be
declared as a property.


\subsection{Subtyping}
Constrained types naturally come equipped with a subtype relation that
combines the nominal subtyping relation of classes and interfaces with
the logical entailment relation of the constraint system. Namely, a
constraint \xcd{C(:c)} is a subtype of \xcd{D(:d)} if \xcd{C} is a
subtype of \xcd{D} and every value in \xcd{C} that satisfies \xcd{c}
also satisfies \xcd{d}.

% Thus, the set of constrained types on a base
% type \xcd{C} inherits a lattice structure from the underlying
% constraint system. The maximal element is \xcd{C(:true)}, which is
% just \xcd{C}, and the minimal element is the unsatisfiable constraint.

% Constrained types naturally come equipped with a {\em subtyping
% structure}: type $t_1$ is a subtype of $t_2$ if the denotation of
% $t_1$ is a subset of $t_2$.
This definition 
% satisfies Liskov's Substitution Principle~\cite{liskov-behaviors}) and
implies that
\xcd{C(:e1)} is a subtype of \xcd{C(:e2)} if \xcd{e1} implies \xcd{e2}.
In particular, for all conditions \xcd{e},
\xcd{C(:e)} is a subtype of \xcd{C}.
\xcd{C(:e)} is empty exactly
when \xcd{e} conjoined with \xcd{C}'s class invariant is inconsistent.

Two constrained types \xcd{C1(:e1)} and \xcd{C2(:e2)} are considered
equivalent if \xcd{C1} and \xcd{C2} are the same base type and
\xcd{
e1} and \xcd{e2} are equivalent when considered as logical
expressions. Thus, \xcd{C(:x*x==4)} and \xcd{C(:x==2 || x==-2)} are
equivalent types.

\subsection{Final variables}

The use of \xcd{final} local variables, formal parameters, and
fields in constrained
types has proven to be particularly valuable in practice. The same
variable that is being used in computation can also be used to specify
types. There is no need to introduce separate, universally and
existentially quantified ``index'' variables.
%
During type-checking, \xcd{final} variables are turned into symbolic
variables---some fixed but unknown value---of the same type.
Computation is performed in a constraint-based fashion on such
variables.

\subsection{Method and constructor preconditions}

Methods and constructors may specify preconditions on their parameters
as where clauses.  For an invocation of a method (or constructor) to
be type-correct, the associated where clause must be statically known
to be satisfied. Note that the where clause may contain constraints on
the properties of \xcd{this}. Thus the where clause may be used to
specify that a method is {\em conditionally} available on some objects
of the class and not others.

The
return type of a method may also contain expressions involving the
arguments to the method.  Any argument used in this way must
be declared \xcd{final},
ensuring it is not mutated by the method body.  For instance:
\begin{displayxten}
  List(arg.length-1) tail(final List arg : arg.length > 0) {...}
\end{displayxten}
\noindent is a valid method declaration. It says that
\xcd{tail} must be passed a non-empty list, and it returns a list
whose length is one less than its argument.

\subsection{Method overloading and overriding}

The definitions of method overloading, overriding, hiding,
shadowing and obscuring in \Xten{} are the same as in
\Java~\cite{Java3}, modulo the following considerations
motivated by dependent types.

Our current implementation
erases dependent type information when compiling to Java. Therefore it
must be the case that a class does not have two different method
definitions that conflict with each other when the constrained
clauses in their types are erased.

A class \xcd{C} inherits from its direct superclass and
superinterfaces all their methods that are visible according to the access
modifiers and that are not hidden or overridden. A method
\xcd{m1} in a class \xcd{C} overrides a method \xcd{m2} in a
superclass \xcd{D} if \xcd{m1} and \xcd{m2} have signatures
with equivalent (unerased) formal parameter types. 

Dynamic method lookup does not take dependent type information into
account, only the class hierarchy. This design decision ensures that
serious errors such as method invocation errors are captured at
compile-time. (Such errors can arise because multiple incomparable
methods with the same name and acceptable argument lists might be
available at the dynamic dependent type of the subject. Examples are
not difficult to construct.)

\eat{
The current \Xten{} compiler produces \Java{} code. It further
implements the restriction that no two methods for the same class can
have the same signature after their constraints are erased. This
simplifies implementation---no name mangling is needed to preserve
the dependent type distinction in the generated \Java{} code.
However, this approach does cut down on the usefulness of constrained
clauses for method dispatch.
}

\subsection{Constructors for dependent classes}

\eat{
Like a method definition,
a constructor may
specify preconditions on its arguments
and a postcondition on the value produced by the constructor.
%
Postconditions may be specified in a constructor declaration between
the name of the class and the argument list of the constructor using a
where clause. The where clause can reference only the properties of
the class.
}

Constructors must ensure that the class invariants of the given
class and its superclasses and superinterfaces hold.
For instance, the
nullary constructor for \xcd{List} ensures that the property
\xcd{length} has the value \xcd{0}:
\begin{displayxten}
    public List(0)() { property(0); }
\end{displayxten}
The \xcd{property} statement is used to set all the properties
of the new object simultaneously.  Capturing this assignment in
a single statement simplifies checking that the constructor
postcondition and class invariant are established.  If a class
has properties, every path through the constructor must contain
exactly one \xcd{property} statement.

\java{}-like languages permit constructors to throw exceptions. This
is necessary to deal with the situation in which the arguments to a
constructor for a class \xcd{C} are such that no object can be
constructed that satisfies the invariants for \xcd{C}. Dependent
types make it possible to perform some of these checks at
compile-time. The class invariant of a class explicitly captures
conditions on the properties of the class that must be satisfied by
any instance of the class.  Constructor preconditions capture
conditions on the constructor arguments.
The compiler's static check for
non-emptiness of the type of any variable captures these invariant
violations at compile-time.

%% Cannot throw an exception.

%% Figure out the real condition. Not sure this is important.

\subsection{Extending dependent classes}

A class may extend a constrained class, e.g.,
\xcd{class C(}\dots\xcd{) extends D(:d)}.
This documents the programmer's intention that
every call to \xcd{super} in a constructor for \xcd{C} must ensure
that the invariant \xcd{d} is established on the state of the class
\xcd{D}. The expressions in the actual parameter list for the super
class may involve only the properties of the class being defined.

%{\em MetaNote: This should be standard. A class definition may extend
%a dependent super class, e.g. class Foo(int i) extends Fum(i*i) \{
%\ldots \}. The expressions in the actual parameter list for the super
%class may involve only the properties of the class being defined. The
%intuition is that these parameters are analogous to explicit arguments
%that must be passed in every super-constructor invocation.}

\subsection{Dependent interfaces}

\java{} does not allow interfaces to specify instance fields. Rather all
fields in an interface are final static fields (constants).
However,
since properties play a central role in the specification of
refinements of a type, it makes sense to permit interfaces to specify
properties.
Similarly, an interface
definition may specify an invariant on its properties.  Methods
in the body of an interface may have where clauses
as well.

All classes implementing an interface must have a property
with the same name and
type (either declared in the class or inherited from the superclass)
for each property in the interface. If a class implements
multiple interfaces and more than one of them specify a property
with the same name, then they must all agree on the type of the
property. The class must have a single property with the given name
and type.

The general form of a class declaration is now:
\begin{displayxten}
  class C(T1 x1, ..., Tk xk)
        extends B(:c)
        implements I1(:c1), ..., In(:cn) {...}
\end{displayxten}
\noindent
For such a declaration to type-check, it must be that the class
invariant of \xcd{C} implies \inv(\xcd{Ii})\xcd{\ & ci}, where
\inv(\xcd{Ii}) is the
invariant associated with interface \xcd{Ii}.  Again, a constrained
class or interface \xcd{I} is taken as shorthand for \xcd{I(:true)}.
Further, every method specified in the interface must have a
corresponding method in the class with the same signature whose
precondition, if any, is implied by the precondition of the method in
the interface.


\subsection{Separation between compile-time and run-time computation}

Our design distinguishes between compile-time execution (performed
during type-checking) and run-time execution. At compile-time, the
compiler processes the abstract syntax tree of the program generating
queries to the constraint solver. The only computation engine running
is the constraint solver, which operates on its own vocabulary of
predicates and functions. Program variables (such as local variables)
that occur in types are dealt with symbolically. They are replaced
with logical variables---some fixed, but unknown value---of the same
type. The constraint solver must know how to process pieces of partial
information about these logical variables in order to determine
whether some constraint is entailed. At run time, the same program
variable will have a concrete value and will perform ``arithmetic''
(calculations) where the compiler performed ``algebra'' (symbolic
analysis).

Constrained types may occur in a class cast \xcd{(T)}\;\xcd{e}.  Code is
generated to check at run time that the expression \xcd{e}
satisfies any constraints in \xcd{T}.

\subsection{Query evaluation}

Because object-oriented languages permit arbitrary mutual recursion between
classes: classes \xcd{A} and \xcd{B} may have fields of type \xcd{B} and
\xcd{A} respectively---the type/property graph may have loops. The nodes
in this graph are base types (class and interface names). There is an
edge from node \xcd{A} to node \xcd{B} if \xcd{A} has a property whose
base type is \xcd{B}.

Let us define the {\em real clause} of a constrained type \xcd{C(:c)} to be
the set of constraints that must be satisfied by any instance of
\xcd{C(:c)}. This includes not only the condition \xcd{c} but also
constraints that hold for all instances of \xcd{C}, as
determined by \xcd{C}'s class invariant. Let 
$\rc({\tt C(:c)})$ denote the {\em real clause} of \xcd{C(:c)}.
For simplicity, we consider only top-level classes; thus, the
only free variable
in $\rc({\tt C(:c)})$ is \xcd{self}.  We draw out \xcd{self} as a
formal parameter and write $\rc({\tt C(:c)}, {\tt X})$
for $\rc({\tt C(:c}[{\tt X}/\mbox{\xcd{self}}]))$.

Consider a general class definition:
\begin{displayxten}
class C(C$_k$(:c$_1$) x$_1$, ..., C$_k$(:c$_k$) x$_k$: c) extends D(:d) { ... }
\end{displayxten}

\noindent From this, we get:

$$\rc({\tt C},{\tt X}) ~~~ \iff
\begin{array}{l}
\quad  ({\tt c} \wedge {\tt d})[{\tt X}/\this]
\;\wedge\; \rc({\tt D},{\tt X}) \;\wedge \\
\quad  \rc(\mbox{\tcd{C$_1$(:c$_1$)}}, {\tt X.x}_1) \;\wedge\; \cdots
\;\wedge\; \rc(\mbox{\tcd{C$_k$(:c$_k$)}}, {\tt X.x}_k)
\end{array}
$$

That is, given a program $P$ with classes ${\tt C}_1,\ldots, {\tt
C}_k$, the set of real clauses for ${\tt C}_1,\ldots, {\tt C}_k$ are
defined in a mutually recursive fashion through the Clark completion
of a Horn clause theory (over an underlying constraint system).

The central algorithmic question now becomes whether given a
constrained clause {\tt d}, does $\rc(\mbox{\tt C(:c)},{\tt X})$ entail {\tt d}? 
%
From the above formulation the question is clearly semi-decidable. It
is not clear however whether it is decidable. This is a direction for
further work.

In practice, many data structures have non-cyclic dependency
graphs. For such programs the real clause can be computed quickly and
only a bounded number of questions to the constraint-solver are
generated during type-checking.

\eat{
\subsection{Existential types}

The use of constraints makes existential types very natural.  Consider
the return type of \xcd{filter} in Figure~\ref{fig:list-example}: it specifies that the list
returned is of some unknown length, which is known
to be bounded by \xcd{n}. Thus, constrained types naturally
subsume existential dependent types. Indeed, every base type \xcd{C}
is an ``existential'' constrained type since it does not specify any
constraint on its properties. Thus, code written with constrained types
can interact seamlessly with legacy library code---using just base
types wherever appropriate.
}

\subsection{Parametric consistency}

Consider the set of final variables that are referenced in a
type \xcd{T} = \xcd{C(:c)}. These are the {\em parameters} of
the type. A type is said to be {\em parametrically consistent}
if its where clause \xcd{c} is solvable for each possible assignment of
values to parameters.  A parametrically consistent type has the
property that its extension will never be empty. 
Types are required to be parametrically consistent.

Consider  a variation of List from Figure~\ref{fig:list-example}:
\begin{displayxten}
class List(int(:self >= 0) n) {
  Object head;
  List(:self.n == this.n-1 & self != null) tail;
  ...
}
\end{displayxten}
The type of the field \xcd{tail} is not parametrically
consistent. There exists a value for its parameter \xcd{n}, namely
\xcd{0}, for which the real clause \xcd{self}~\xcd{!=}~\xcd{null} \xcd{&}
\xcd{self.n}~\xcd{==}~\xcd{this.n-1} \xcd{&}
\xcd{self.n}~\xcd{>=}~\xcd{0} is not satisfiable.

The compiler will throw a type error when it encounters the
initializer for this field in a constructor since it will not be able
to prove that the initial value is of the given type.




\section{Examples}\label{sec:examples}
%III. Applied constrained calculi. (3 pages)
%
%For each example below, formal static and dynamic semantics rules for
%new constructs extension over the core CFJ. Subject-reduction and
%type-soundness theorems. Proofs to be found in fuller version of
%paper.
%
%(a) arrays, region, distributions -- type safe implies no arrayoutofbounds
%exceptions, only ClassCastExceptions (when dynamic checks fail).
%
%Use Satish's conditional constraints example.
%-- emphasize what is new over DML. 
%
%(b) places, concurrency -- place types.
%
%(c) ownership types, alias control.
%
The following section presents example uses of constrained types
using several different
constraint systems.
%
\eat{
Many common object-oriented
idioms and
object-oriented type systems can be captured naturally using
constrained types: specifically we discuss types for places,
aliases,
ownership, arrays and clocks.  \ref{TODO}
}

\eat{
\ref{TODO}
Many of these constraint systems are more
expressive than the constraint system implemented in the current
\Xten{} compiler and have not (yet) been implemented.
}

\eat{
\ref{TODO}
In the following we will use the shorthand $\tt C(\bar{t}:c)$ for the
type $\tt C(:\bar{f}=\bar{t},c)$ where the declaration of the class
{\tt C} is $\tt \class\ C(\bar{\tt T}\ \bar{\tt f}:c)\ldots$  Also,
we abbreviate $\tt C(\bar{t}:\true)$ as $\tt C(\bar{t})$.
Finally, we use the shorthand $\tt T\;x=t;~c$ for the constraint
$\tt T\;x;~x=t;~c$.
}

\eat{
Finally, we
will also have need to use the shorthand
${\tt C}_1(\bar{t}_1:{\tt c}_1)\& \ldots {\tt C}_k(\bar{\tt t}_k:{\tt c}_k)$
for the type
${\tt C}_1(:\bar{\tt f}_1=\bar{\tt t}_1, \ldots,
            \bar{\tt f}_k=\bar{\tt t}_k,{\tt c}_1,\ldots,{\tt c}_k)$ 
provided that the ${\tt C}_i$ form a subtype chain
and the declared fields of ${\tt C}_i$ are ${\tt f}_i$.

Constraints naturally allow for partial specification
(e.g., inequalities) or incomplete specification (no constraint on a
variable) with the same simple syntax. In the example below,
the type of {\tt a} does not place any constraint on the second
dimension of {\tt a}, but this dimension can be used in other
types (e.g., the return type).
\begin{xten}
  class Matrix(int m, int n) {
    Matrix(m,a.n) mul(Matrix(:m==this.n) a) {...}
    ...
  }
\end{xten}

Constraints also naturally permit the expression of existential types:
\begin{xten}
  class List(int length) { 
    List(:length <= this.length) filter(Comparator k) {...} 
    ...
  }
\end{xten}
\noindent
Here, the length of the list returned by the \xcd{filter} method is 
unknown, but is bound by the length of the original list.
}

\if 0
\subsection{Presburger constraints: array bounds}

Xi and Pfenning proposed using dependent types for eliminating
array bounds checks~\cite{xi98array}.
\Xten{} does not (yet) support generic types, however XXX
%
In CFJ, an array of type \xcd{T[]} indexed by (signed) integers
can be modeled as a class with the following
signature:\footnote{For this example, we assume generics
are supported.}
\begin{xten}
interface Array<T>(int(:self >= 0) length) {
  T get(int(:0 <= self, self < this.length) i);
  void set(int(:0 <= self, self < this.length) i, T v);
}
\end{xten}

Bounds can be checked using a constraint system based on
Presburger arithmetic~\cite{omega}.  Constraint terms include
integer constraints, scalar multiplication, and addition;
constraints include inequalities:
\fi


\eat{
Some code that iterates over an array (sugaring {\tt get} and {\tt set}):
\begin{xten}
double dot(double[] x, double[] y
         : x.length == y.length) {
  double r = 0.; 
  for (int(:self >= 0, self < x.length)
       i = 0; i < x.length; i++) {
    r += x[i] * y[i];
  }
  return r;
}
\end{xten}
}

\eat{
Another one:
\begin{xten}
double[](:length = x.length) saxpy(double a, double[] x, double[] y : x.length = y.length) {
    double[](:length = x.length) result = new double[x.length];
    for (int(:self >= 0, self < x.length) i = 0; i < x.length; i++) {
        result[i] = a * x[i] + y[i];
    }
    return result;
}
\end{xten}
}

% \subsection{Presburger constraints: blocked LU factorization}

\subsection{Equality constraints}

The \Xten{} compiler includes a simple equality-based constraint
system, described in Section~\ref{sec:lang}.
Equalities constraints
are used throughout \Xten{} programs.  For example, to ensure
$n$-dimensional arrays are indexed only be $n$-dimensional
index points, the array access operation requires that the
array's \xcd{rank} property be equal to the index's \xcd{rank}.

Equality constraints specified in the X10 run-time library are used by the
compiler to generate efficient code.  For instance, an iteration over
the points in a region can be optimized to a set of nested loops
if the constraint on the region's type specifies that the region
is rectangular and of constant rank.


\eat{
\subsection{Equality constraints with disjunction: place types}

This example is due to Satish Chandra. We wish to specify a balanced
distributed tree with the property that its right child is always at
the same place as its parent, and once the left child is at the same
place then the entire subtree is at that place.  In
\Xten{}, every object has a field {\tt location} of type
{\tt place} that specifies the location at which the object is located.
%
The desired property may be specified thus:
\begin{xten}
class Tree(boolean localLeft) extends Object {
  Tree(:!this.localLeft || (location==here && self.localLeft)) left; 
  Tree(:location==here) right);
  ...
}
\end{xten}
The constraint on \xcd{left} states that if the \xcd{localLeft} property is
true for the current node, then the location of the \xcd{left} child must be
\xcd{here} and its \xcd{localLeft} property must be set.  This ensures,
recursively, that the entire left subtree will be located at the same place.
}

\subsection{Presburger constraints}

Presburger constraints are linear integer inequalities.
%A constraint solver plugin was implemented using a port to Java of the
%Omega library.~\cite{omega,scale}
%A separate implementation
%of a Presburger constraint solver was implemented using
%CVC3~\cite{cvc}. 
A Presburger constraint solver plugin was implemented using
CVC3~\cite{cvclite,cvc}.  The list example in
Figure~\ref{fig:list-example} type-checks using this constraint system.

Presburger constraints are particularly useful in a
high-performance computing setting where array operations are
pervasive.
Xi and Pfenning proposed using dependent types for eliminating
array bounds checks~\cite{xi98array}.  A Presburger constraint
system can be used to keep track of array dimensions and array
indices to ensure bounds violations do not occur.

\subsection{Set constraints: region-based arrays}

Rather than using Presburger constraints, 
\Xten{} takes another approach:
following ZPL~\cite{ZPL}, arrays in \Xten{}
are defined over
{\em regions},
sets of $n$-dimensional {\em index points}~\cite{gps06-arrays}.
For instance, the region \xcd{[0:200,}\xcd{1:100]} specifies a
collection of two-dimensional points \xcd{(i,j)} with \xcd{i}
ranging from \xcd{0} to \xcd{200} and \xcd{j} ranging from
\xcd{1} to \xcd{100}.

Regions and points were modeled in CVC3~\cite{cvc} to create a
constraint solver that ensures array bounds
violations do not occur:
an array access type-checks if the index point can be statically
determined to be in the region over which the array is defined.

Region constraints are subset constraints
written as calls to the \xcd{contains}
method of the \xcd{region} class.
The constraint solver does not actually evaluate the calls to
the \xcd{contains} method, rather it interprets these calls
symbolically
as subset constraints at compile time.

Constraints have the following syntax:

{
\small
\begin{tabular}{r@{\quad}rcl}
\\
  (Constraint)   &\xcd{c} &::=& \xcd{r.contains(r)} \bnf \dots \\
  (Region) &\xcd{r} &::=& \xcd{t} \bnf [${\tt b}_1$:${\tt d}_1$,\ldots,${\tt b}_k$:${\tt d}_k$]
           \\
           &        &  \bnf &
           \xcd{r | r} \bnf \xcd{r & r} \bnf \xcd{r - r}
           \\
           &        &  \bnf &
           \xcd{r + p} \bnf \xcd{r - p} \\
  (Point)  &\xcd{p} &::=& \xcd{t} \bnf $[{\tt b}_1,\ldots,{\tt b}_k]$ \\
(Integer)&\xcd{b},\xcd{d} &::=& \xcd{t} \bnf \xcd{n} \\
\\
\end{tabular}
}

\noindent
where \xcd{t} are constraint terms (properties and final variables)
and \xcd{n} are integer literals.

Regions used in constraints are either constraint terms \xcd{t},
region constants, unions (\xcd{|}), intersections (\xcd{&}),
or differences (\xcd{-}), or regions where each point is
offset by another point \xcd{p} using \xcd{+} or \xcd{-}.

% $\xcd{r}_1$\xcd{.contains(}$\xcd{r}_2$\xcd{)}.

\begin{figure}[t]
\footnotesize

\inputxten{sor.x10}

\caption{Successive over-relaxation with regions}
\label{fig:sor}
\end{figure}

For example, the code in Figure~\ref{fig:sor} performs a successive
over-relaxation~\cite{sor} of a matrix \tcd{G} with rank 2.
The function declares a region variable \tcd{outer} as an alias for
\tcd{G}'s region and a region variable \tcd{inner} to be 
the subset of \tcd{outer} that excludes the boundary points,
formed by intersecting the \tcd{outer} region with itself shifted up, down,
left, and right by one.
The function then declares two more regions \tcd{d0} and \tcd{d1},
where ${\tt d}_i$ is set of points ${\tt x}_i$ where
$({\tt x}_0, {\tt x}_1)$ is in \tcd{inner}.  The function
iterates multiple times over points \tcd{i} in \tcd{d0}.
The syntax \tcd{finish} \tcd{foreach} (line 22) tells the
compiler to execute each loop iteration in parallel and to wait
for all concurrent activities to terminate.
The inner loop (lines 24--28) iterates over a subregion of
\tcd{inner}.

The type checker establishes that the \tcd{region} property of
the point \tcd{ij} (line 24) is \tcd{inner}
\xcd{&}~\xcd{[i..i,d1min..d1max]}, and that this region is a
subset of \tcd{inner}, which is in turn a subset of \tcd{outer},
the region of the array \tcd{G}.
Thus, the accesses to the array in the loop body
do not violate the bounds of the array.

A key to making the program type-check is that the region
intersection that defines \tcd{inner} (lines 10--11)
is explicitly intersected with \tcd{outer} so that the 
constraint solver can determine that
the result is a subset of \tcd{outer}.


\eat{
\subsection{AVL trees and red--black trees}

AVL trees and red-black trees can be modeled so that the
data structure invariant is enforced statically.

\begin{xten}
class AVLTree(int(:self >= 0) height) {...}
class Leaf(Object key) extends AVLTree(0) {...}
class Node(Object key, AVLTree l, AVLTree r
         : int d=l.height-r.height; -1 <= d, d <= 1) 
    extends AVLTree(max(l.height,r.height)+1) {...}
\end{xten}

Red--black trees may be modeled similarly. Such trees have the
invariant that (a) all leaves are black, (b) each non-leaf node has
the same number of black nodes on every path to a leaf (the black
height), (c) the immediate children of every red node are black.
\begin{xten}
class RBTree(int blackHeight) {...}
class Leaf extends RBTree(0) { int value; ... }
class Node(boolean isBlack, 
           RBTree(:this.isBlack || isBlack) l, 
           RBTree(:this.isBlack || isBlack,
                   blackHeight=l.blackHeight) r)
    extends RBTree(l.blackHeight+1) { int value; ... }
\end{xten}
}

\eat{
\subsection{Self types and binary methods}

Self types~\cite{bsg95,bfp-ecoop97-match} can be implemented
using a {\tt klass} property on objects.  The {\tt klass}
property represents the run-time class of the object.
Self types can be used to solve the binary method problem \cite{bruce-binary}.

In the example below, the {\tt Set} interface has a {\tt union} method
whose argument must be of the same class as {\tt this}.
\noindent This enables the {\tt IntSet} class's {\tt union}
method to access the {\tt bits} field of its argument {\tt s}.
\begin{xten}
  interface Set(:Class klass) {
    Set(this.klass) union(Set(this.klass) s);
  }
  class IntSet(:Class klass) implements Set(klass) {
    long bits;

    IntSet(IntSet.class)() { property(IntSet.class); }

    IntSet(IntSet.class)(int(:0 <= self, self <= 63) i) {
      property(IntSet.class);
      bits = 1 << i; }

    Set(this.klass) union(Set(this.klass) s) {
      IntSet(this.klass) r = new IntSet(this.klass);
      r.bits = this.bits | s.bits;
      return r; }
  }
\end{xten}
\noindent
The key to ensuring that this code type-checks is the
\rn{T-constr}
rule.
With a constraint system ${\cal C}_{\mathsf{klass}}$ aware of
the {\tt klass} property, the rule 
\rn{T-var} is used to subsume an expression of type
${\tt Set(this.class)}$ to type ${\tt IntSet(this.class)}$
when {\tt this} is known to be an {\tt IntSet}:
{\footnotesize
\[
\from{\begin{array}{c}
{\tt IntSet}~{\tt this}, {\tt Set}({\tt this.klass})~{\tt s}
        \vdash {\tt Set}({\tt this.klass})~{\tt s} \\
{\tt IntSet}~{\tt this}, {\tt Set}({\tt this.klass})~{\tt s}
        \vdash_{{\cal C}_{\mathsf{klass}}} {\tt IntSet}({\tt this.klass})~{\tt s} \\
\end{array}}
\infer{
{\tt IntSet}~{\tt this}, {\tt Set}({\tt this.klass})~{\tt s}
        \vdash {\tt IntSet}({\tt this.klass})~{\tt s}}
\]}
}


\eat{
\subsection{Binary search}

An informal study by Jon Bentley~\cite{programming-pearls}
found that x\% of professional programmers attending in a course
could not correctly implement binary search.

Dependent types can help here by adding the invariants to the
index types.

\subsection{Quicksort}

\begin{xten}
int(:left <= self & self <= right)
partition(T[] array, int left, int right, int pivotIndex : left <= pivotIndex & pivotIndex <= right) {
     T pivotValue = array[pivotIndex];

     // Move pivot to end
     swap(array, pivotIndex, right);

     int(:left <= self & self <= right) storeIndex;
     storeIndex = left;
     for (int(:left <= self & self <= right-1) i = left; i < right; i++) {
         if (array[i] <= pivotValue) {
             swap(array, storeIndex, i);
             storeIndex++;
         }
     }

     // Move pivot to its final place
     swap(array, right, storeIndex)
     return storeIndex;
}

void swap(T[] array,
          int(:0 <= self & self < array.length i,
          int(:0 <= self & self < array.length j) {
    T tmp = array[i];
    array[i] = array[j];
    array[j] = tmp;
}

void quicksort(T[] array, int left, int right : left <= right) {
    if (left < right) {
         // select a pivot index
         int(:left <= self & self <= right) pivotIndex = (left + right) / 2;
         pivotNewIndex = partition(array, left, right, pivotIndex)
         quicksort(array, left, pivotNewIndex-1)
         quicksort(array, pivotNewIndex+1, right)
    }
}
\end{xten}
}


\newif\ifowner
\ownerfalse

\ifowner

\subsection{Ownership constraints}

\begin{figure}[t]
\inputxten{LO.x10}
\caption{Ownership types}
\label{fig:ownership}
\end{figure}

Using a simple extension of \Xten{}'s built-in equality
constraint system,
constrained types can also be used to encode a form of ownership
types~\cite{ownership-types,liskov-popl2003}.

Figure~\ref{fig:ownership} shows a fragment of a \xcd{List}
class with ownership types.
Each \xcd{Owned} object has an \xcd{owner} property.  Objects
also have properties that are used as owner parameters.
%
The \xcd{List} class has an \xcd{owner} property inherited from
\xcd{Owned} and also declares a \xcd{valOwner} property that is
instantiated with the owner of the values in the list, stored in
the \xcd{head} field of each element.  The \xcd{tail} of the
list is owned by the list object itself.

\Xten{}'s equality-based constraint system is sufficient for
tracking object ownership, however is does not capture all
properties of ownership type systems.
Ownership type systems enforce an ``owners as dominators''
property: the ownership relation forms a tree within the object
graph; a reference is not permitted to point directly to objects
with a different owner.
%
To encode this property, the owner of
the values \xcd{valOwner} must be contained within the owner
of the list itself; that is, \xcd{valOwner} must be \xcd{owner}
or \xcd{valOwner}'s owner must be contained in \xcd{owner}.
This is captured by the constraint \xcd{self.owns(owner)} on
\xcd{valOwner}.  Calls to the \xcd{owns} method in constraints
are interpreted by the ownership constraint solver as the
disjunction of conditions shown in the body of \xcd{owns}.
The object \xcd{world} is the root of the ownership tree; 
all objects are transitively owned by \xcd{world}.

For example, the type \xcd{List(:owner==world & valOwner == this)}
is invalid, because
its constraint is interpreted as
\xcd{owner == world & valOwner == this & this.owns(world)},
which is satisfiable only when \xcd{this == world} (which it is not).

An additional check is needed to ensure objects owned by
\xcd{this} are encapsulated.
The \xcd{tail()} method for instance, incorrectly leaks the
list's \xcd{tail} field.  To eliminate this case, the ownership
constraint system must additionally check that owner parameters
are bound only to 
\xcd{this}, \xcd{world}, or an owner property of \xcd{this}.
These conditions ensure that \xcd{tail()} can be called only on
\xcd{this} because its return type is otherwise not valid.
For instance, in the following code, the type of \xcd{ys} is
not valid because the \xcd{owner} property is bound to \xcd{xs}:
\begin{xten}
    final Owned o = ...;
    final List(:owner==o & valOwner==o) xs;
    List(:owner==xs & valOwner==o) ys = xs.tail();
\end{xten}

\fi

\if 0
\subsection{Disequalities: non-null types}

A constraint system that supports disequalities can be used to
enforce a non-null invariant on reference types.
A non-null type \xcd{C} can be written simply as \xcd{C(:self != null)}.
\fi

\eat{
\subsection{Clocked types}

Clocks are barriers that are adapted to a context where activities may be
dynamically created, and are designed so that all clock operations are
determinate.

For each arity $n$, we introduce a {\em Gentzen predicate}
${\tt clocked(\bar{t})}$. A $k$-ary Gentzen predicate $a$ satisfies the
property that $a(t_1,\ldots, t_k) \vdash a(s_1,\ldots,s_n)$ iff $k=n$
and $t_i=s_i$ for $i\leq k$.

Such a \xcd{clocked} atom is added to the context by an \xcd{clocked async}:
$$
\from{\Gamma, {\tt clocked(\bar{\tt v})} \vdash {\tt T}\ {\tt e}}
\infer{\Gamma \vdash {\tt T}\ {\tt async}\ {\tt clocked}(\bar{\tt v}) {\tt e}}
$$

A programmer can require that a method may be invoked only if the
invoking activity is registered on the clocks $\bar{\tt k}$ by adding
a \xcd{clocked} clause. The rule for method elaboration and method invocation then change:
$$
\begin{array}{l}
\from{ \bar{\tt T}\ \bar{\tt x}, {\tt C}\ \this, {\tt c},\clocked(\bar{\tt k}) \vdash {\tt S}\ {\tt e}, {\tt S} \subtype {\tt T} }   
\infer{\tt T\ m(\bar{\tt T}\,\bar{\tt x} : c) \clocked(\bar{\tt  k})\{\return\ e;\}\ \mbox{OK in}\ C} 
\\ \quad\\ 
\rname{T-Invk}%
\from{\begin{array}{l}
\Gamma \vdash {\tt T}_{0:n} \ {\tt e}_{0:n}  \\
\mtype({\tt  T}_0,{\tt  m},{\tt  z}_0)= \tt {\tt  Z}_{1:n}\ {\tt  z}_{1:n}:c,clocked(\bar{\tt  k}) \rightarrow {\tt  S} \\
\Gamma, {\tt  T}_{0:n}\ {\tt  z}_{0:n} \vdash {\tt  T}_{1:n} \subtype {\tt  Z}_{1:n}\\
\sigma(\Gamma, {\tt  T}_{0:n}\ {\tt  z}_{0:n}) \vdash_{\cal C} {\tt  c} \ \ \ 
\mbox {(${\tt  z}_{0:n}$ fresh)} \\
\Gamma \vdash \clocked(\bar{\tt  k})\\
\end{array}}
\infer{\Gamma \vdash ({\tt  T}_{0:n}\ {\tt  z}_{0:n}; S)\ {\tt  e}_0.{\tt  m({\tt  e}_{1:n})}}
\end{array}
$$
}

\eat{
\subsection{Capabilities}

Capabilities (from Radha and Vijay's paper on neighborhoods)
}

\eat{
\subsection{Activity-local objects}

Parallelism in \Xten{} is supported through lightweight asynchronous {\em
activities}, created by {\tt  async} statements.
It is often useful to restrict objects so that they are {\em local} to a
particular activity.
A local object may be accessed only by
the activity that created it or by an ancestor of that activity.
% it may be written only by the activity that created
% it or by a descendant of that activity.
Local objects are declared and created by qualifying their type
with {\tt  local}:
\begin{xten}
  local C o = new local C();
\end{xten}

To encode local objects in \Xten{}, we add an {\tt  activity}
property to objects:
\begin{xten}
  class Object(Activity activity) { ... }
\end{xten}
\noindent
where {\tt  Activity} has a possibly null {\tt  parent} property:
\begin{xten}
  class Activity(Activity parent) { ... }
\end{xten}
\noindent

To track the current activity ({\tt  z}), we augment typing judgments
as follows:
\[
  {\tt  z};~\Gamma \vdash {\tt  T}\ {\tt  e}
\]
\noindent where ${\tt  Activity}({{\tt  z}'})~{\tt  z} \in \Gamma$.
When the current activity is {\tt  z},
we encode the type {\tt  local C} as ${\tt  C}({\tt  z})$.

Spawning a new activity with an {\tt  async} statement
introduces a fresh activity ${\tt  z}'$:
\[
\from{
{\tt  z}';~\Gamma,~{\tt  Activity}({\tt  z})~{\tt  z'} \vdash {\tt  T}\ {\tt  e}\ \ \ 
\mbox{(${\tt  z}'$ fresh)}
}
\infer{
{\tt  z};~\Gamma \vdash {\tt  T}\ ({\tt  async}\ {\tt  e})
}
\]
The rule \rn{T-Field} is strengthened to require that reads 
only be performed on objects whose {\tt  activity} property is a
descendant of the current activity.
%\rname{T-Field-Local}%
\[
\from{
\begin{array}{ll}
{\tt  z};~\Gamma \vdash {\tt  T}_0\ {\tt  e} \\
\mathit{fields}({\tt  T}_0,{\tt  z}_0)= \bar{\tt  U}\ \bar{\tt  f}_i &
\mbox{(${\tt  z}_0$ fresh)} \\
{\tt  z};~\Gamma \vdash {\tt  T}_0 \subtype {\tt  C}(:{\tt  activity} = {\tt  z}') &
\Gamma \vdash {\tt  z}~\mathsf{spawns}~{\tt  z}'
\end{array}
}
\infer{{\tt  z};~\Gamma \vdash ({\tt  T}_0\ {\tt  z}_0; {\tt  z}_0.{\tt  f}_i=\self;{\tt  U}_i)\ {\tt  e.f}_i}
\]

%\Gamma \vdash {\tt  z}_0.{\tt  activity} = {\tt  z}' &

\noindent
where the $\mathsf{spawns}$ relation is defined as follows:
\[
\Gamma \vdash {\tt  z}~\mathsf{spawns}~{\tt  z}
\]
\[
\from{
\Gamma \vdash {\tt  z_1}~\mathsf{spawns}~{\tt  z_2} \ \ \ 
\Gamma \vdash {\tt  z_2}~\mathsf{spawns}~{\tt  z_3}}
\infer{\Gamma \vdash {\tt  z_1}~\mathsf{spawns}~{\tt  z_3}}
\]
\[
\from{\Gamma \vdash {\tt  z_2}.{\tt  parent} = {\tt  z_1}}
\infer{\Gamma \vdash {\tt  z_1}~\mathsf{spawns}~{\tt  z_2}}
\]

\eat{
local objects owned by activity that created it.

locals cannot be read by contained asyncs.

locals can be written by contained asyncs.

locals created by an activity are inherited by the parent when
the activity terminates.

\begin{xten}
C(:thread = current) x = ...;
finish foreach (...) {
  C(:thread = current) y = x; // no!
  x = y;
}
\end{xten}

// can read if thread prop is current, or an ancestor of current
// can write if thread prop is current or a child of current

e : C(:thread = x)
current owns x
fields(...) = Ti fi
-----------------------
e.fi : Ti

extensions:

1. add thread to context
2. strengthen T-field rule
}
}

\eat{
\subsection{Discussion}

\paragraph{Control-flow.}
Tricky to encode.  Need something like {\tt pc} label~\cite{jif}.

\paragraph{Type state.}
Type state depends on the mutable state of the 
objects.  Cannot do in this framework.

Dependent types are of use in annotations~\cite{ns07-x10anno}.
}


\section{Implementation}\label{sec:implementation}
\label{sec:impl}
%IV. Implementation (0.5 page)
%
%Specify what has been implemented and how. What is interesting about
%the implementation.

The \Xten{} compiler provides a framework for writing and
checking constrained types.  The \Xten{} 
language, 
constraints in \Xten{} are conjunctions of equalities over immutable
side-effect-free expressions.  Compiler plugins may be installed
that support other constraint languages, Presburger
constraints (linear inequalities)~\cite{??}.

The \Xten{} compiler is implemented as an extension of
Java using the Polyglot compiler framework~\cite{ncm03}.
Expressions used in constrained
types are type-checked as normal non-dependent \Xten{} expressions;
no constraint solving is performed on these expressions.
During type-checking, constraints are generated and solved using
the built-in constraint solver or using solvers provided by
plugins.  The system allows types to constrained by conjunctions
of constraints in different constraint languages.
If constraints cannot be solved, an error is reported.

\subsection{Constraint checking}

After type-checking a constraint as a boolean expression {\tt e},
the abstract syntax tree for
the boolean expression is transformed into a list of conjuncts.
{\tt e1 \&\& ... \&\& ek}.  Each conjunct {\tt ei} is given to 
the installed constraint system plugins, which symbolically
evaluate the expression to create an internal representation of
the conjunct.  If no
constraint system can handle the conjunct, an error is reported.

To interoperate, the constraint solvers must share a common
vocabulary: constraint terms {\tt t} range over the properties of the
base type, the final variables in scope at the type (including
{\tt this}), the special variable {\tt self} representing
the a value of the type, and field selections {\tt t.f}.
All
constraint systems are required to support the trivial
constraint
\true, conjunction, existential quantification and equality on
constraint terms.


In this form, the constraint is represented as a 
conjunction of constraints from different theories.  Constraints
are checked for consistency using a Nelson--Oppen
procedure~\cite{nelson-oppen}.
After constructing a constraint-system specific 
representation of a conjunct, each plugin computes the set of
term equalities entailed by the conjunct.  These equalities are
propagated to the other conjuncts.  If an inconsistency is
found, an error is reported. 

\if 0
Individual constraint systems may interpret a conjunct as a
formula containing constraint-system-specific atomic functions
{\tt g} and predicates {\tt p}.  

For example, given the constrained type
{\tt Node(:self.x >= 0 \&\& reachable(self).contains(y))},
Presburger constraint system interprets the formula as:

a Presburger constraint {\tt self.x >= 0}
and
an uninterpreted constraint {\tt `reachable/contains'(self,y)}.

A reachability constraint system interprets the formula with the
inequality uninterpreted and the reachability constraint
interpreted.

The base system sees {\tt '>='(self, 0) \&\& 'reachable-contains'(self,y)} 

Neet to propagate implied bindings between terms.  For instance:

{\tt int(:self >= x \&\& self <= x)}
should be equivalent to the type
{\tt int(:self == x)}.

By ensuring constraint systems share the same vocabulary and all
support equality on constraint terms, we can propagate
equalities between constraint systems.
cf. Nelson-Oppen.

Constraint systems must not have conflicting or overlaping
constraints ..XXX

Above, ``,'' binds tighter than ``;''. We use the syntax {\tt
{\tt T\;x};\;c} for the constraint obtained by existentially
quantifying the
variable {\tt x} of type {\tt T} in {\tt c}. {\tt p} ranges over
the collection of predicates supplied by the underlying
constraint
system, and {\tt g} over the collection of functions.




After this translation step, constraints are represented as
lists of constraint-system-specific conjuncts.

The constraints are checked for consistency as follows.
\ref{TODO}
\fi

During type-checking, the type checker needs to determine if the
type
{\tt C(:c)} is a subtype of {\tt D(:d)}.  This is true if 
the base type {\tt C} is a subtype of {\tt D} and if the
constraint {\tt c} entails {\tt d}.

To check entailment, each constraint solver is asked if
a given conjunct of {\tt d} is entailed by {\tt c}.
If any report false, the entailment does not hold and the
subtyping check fails.

% what if a conjunct of d that makes c unsat is ignored?

% propagation of bindings.



\subsection{Translation}

After constraint-checking, the \Xten{} code is translated to
Java in a straightforward manner.  Each dependent class
is translated into a single class of the same name without dependent
types. The explicit properties of the dependent class are translated
into {\tt public final} (instance) fields of the target class.
A {\tt property} statement in a constructor is translated to a
sequence of assignments to initialize the property fields.

For each property, a getter method is also generated in the
target Java class.
Properties declared in interfaces are translated into getter
method signatures.  Subclasses implementing these interfaces
thus provide the required properties by implementing the
generated interfaces.

Usually, constrained types are simply translated to
non-constrained types by erasure; constraints are checked
statically and need no run-time representation.
However, dependent types may be used in casts
and {\tt instanceof} expressions.  Because the constraint syntax
in \Xten{} is a subset of the \Xten{} expression syntax, run-time tests
of constrained types are translated to Java
by evaluating the constraint with
{\tt self} bound to the expression being tested.
For examples, casts are translated as:
\eat{
\begin{code}
  $\Lb$e instanceof C(:c)$\Rb$ = 
    new Object() \{
      boolean check(Object o) \{
        if (o instanceof C) \{
          C self = (C) o;
          return $\Lb$c$\Rb$;
        \}
        return false;
      \}
    \}.check($\Lb$e$\Rb$)
\end{code}
}
\begin{code}
  $\Lb$(C(:c) e$\Rb$ = 
    {\bf new} Object() \{
      C cast(C self) \{
        if ($\Lb$c$\Rb$) return self;
        throw new ClassCastException(); \}
    \}.cast((C) $\Lb$e$\Rb$)
\end{code}
\noindent Wrapping the evaluation of {\tt c} in an anonymous class
ensures the expression {\tt e} is evaluated only once.



\section{Formal semantics}
\label{sec:semantics}
Featherweight X10 (FX10) is a formal calculus for X10 intended to  complement Featherweight Java
(FJ).  It models imperative aspects of X10 including the concurrency
constructs \hfinish{} and \hasync{}.


\paragraph{Overview of formalism}
\Subsection{Syntax}

\begin{figure}[htpb!]
\begin{center}
\begin{tabular}{|l|l|}
\hline

$\hP ::= \ol{\hL},\hS$ & Program. \\

$\hL ::= \hclass ~ \hC~\hextends~\hD~\lb~\ol{\hF};~\ol{\hM}~\rb$
& cLass declaration. \\

$\hF ::= \hvar\,\hf:\hC$
& Field declaration. \\

$\hM ::= \hdef\ \hm(\ol{\hx}:\ol{\hC}):\hC\{\hS\}$
& Method declaration. \\

$\hp ::= \hl ~~|~~ \hx$
& Path. \\ %(location or parameter)

$\he ::=  \hp.\hf  ~|~ \hnew{\hC} ~|~ \hnew{\hAcc(\hr,\hz)}$
& Expressions. \\ %: locations, parameters, field access\&assignment,  %invocation, \code{new}

$\hS ::=  \hp.\hf = \hp; ~|~ \hp.\hm(\ol{\hp});  ~|~ \hval{\hx}{\he}{\hS}$ &\\
$~~~~|~\acc{\hx}{\hnew{\hAcc(\hr,\hz)}}{\hS} ~|~ \ha \leftarrow \hp$ &\\
$~~~~|~ \pclocked~\finish{\hS}$&\\
$~~~~|~ \pclocked~\async{\hS} ~|~ \hS~\hS$
& Statements. \\ %: locations, parameters, field access\&assignment, invocation, \code{new}

\hline
\end{tabular}
\end{center}
\caption{FX10 Syntax.
    The terminals are locations (\hl), parameters and \hthis (\hx), field name (\hf), method name (\hm), class name (\hB,\hC,\hD,\hObject),
        and keywords (\hhnew, \hfinish, \hasync, \code{val}).
    The program source code cannot contain locations (\hl), because locations are only created during execution/reduction in \RULE{R-New} of \Ref{Figure}{reduction}.
    }
\label{Figure:syntax}
\end{figure}

\Ref{Figure}{syntax} shows the syntax of FX10.
%(\Ref{Section}{val} will later add the \hval and \hvar field modifiers.)
Expression~$\hval{\hx}{\he}{\hS}$ evaluates $\he$, assigns it to a
new variable $\hx$, and then evaluates \hS. The scope of \hx{} is \hS.

The syntax is similar to the real X10 syntax with the following difference:
%Non-escaping methods are marked with \code{@NonEscaping}, such methods
%can be invoked on raw objects (and can be used to initialize them).
FX10 does not have constructors; instead, an object is initialized by assigning to its fields or
    by calling
    non-escaping methods.

\Subsection{Reduction}
A {\em heap}~$H$ is a mapping from a given set of locations to {\em
  objects}. An object is a pair $C(F)$ where $C$ is a class (the exact
class of the object), and $F$ is a partial map from the fields of $C$
to locations.
%We say the object~\hl is {\em total/cooked} (written~$\cooked_H(\hl)$)
%    if its map is total, i.e.,~$H(\hl)=\hC(F) \gap \dom(F)=\fields(\hC)$.

%We say that a heap~$H$ {\em satisfies} $\phi$ (written~$H \vdash \phi$)
%    if the plus assertions in~$\phi$ (ignoring the minus assertions) are true in~$H$,
%    i.e., if~$\phi \vdash +\hl$ then~$\hl$ is cooked in~$H$
%    and if~$\phi \vdash +\hl.\hf$ then~$H(\hl)=\hC(F)$ and~$F(\hf)$ is cooked in~$H$.


%An {\em annotation} $N$ for a heap $H$ maps each $l \in \dom(H)$ to a
%possibly empty set of fields $a(H(l))$ of the class of $H(l)$ disjoint
%from $\dom(H(l))$. (These are the fields currently being
%asynchronously initialized.) The logic of initialization described
%above is clearly sound for the obvious interpretation of formulas over
%annotated heaps. For future reference, we say that that a heap $H$
%{\em satisfies} $\phi$ if there is some annotation $N$ (and some
%valuation $v$ assigning locations in $\dom(H)$ to free variables of
%$\phi$) such that $\phi$ evaluates to true.

The reduction relation is described in
Figure~\ref{Figure:reduction}. An S-configuration is of the form
$\hS,H$ where \hS{} is a statement and $H$ is a heap (representing a
computation which is to execute $\hS$ in the heap $H$), or $H$
(representing terminated computation). An
E-configuration is of the form $\he,H$ and represents the
computation which is to evaluate $\he$ in the configuration $H$. The
set of {\em values} is the set of locations; hence E-configurations of
the form $\hl,H$ are terminal.

Two transition relations $\preduce{}$ are defined, one over
S-configurations and the other over E-configurations. Here $\pi$ is a
set of locations which can currently be asynchronously accessed.  Thus
each transition is performed in a context that knows about the current
set of capabilities.  For $X$ a partial function, we use the notation
$X[v \mapsto e]$ to represent the partial function which is the same
as $X$ except that it maps $v$ to $e$.

The rules for termination, step, val, new, invoke, access and assign
are standard.  The only minor novelty is in how \hasync{} is
defined. The critical rule is the last rule in~\RULE{(R-Step)} -- it
specifies the ``asynchronous'' nature of \hasync{} by permitting \hS{}
to make a step even if it is preceded by $\async{\hS_1}$. Further,
each async records its set of synchronous capabilities. When
descending into an async the async's own capabiliites are added to
those obtained from the environment.

%
The rule~\RULE{(R-New)} returns a new location that is bound to a new
object that is an instance of \hC{} with none of its fields initialized.
%
The rule~\RULE{(R-Access)} ensures that the field is initialized before it is
read ($\hf_i$ is contained in $\ol{\hf}$).
%
The rule~\RULE{(R-Acc-N)} adds the new location bound to the
accumulator as a synchronous access capability to the current async.
%

The rule~\RULE{(R-Acc-A-R)} permits the accumulator to be read by an
unclocked async provided that the current set of capabilities permit
it, and provided that there is no nested async prior to the read
(i.e.{} if there were any, they have terminated).

The rule~\RULE{(R-Acc-CA-R)} permits the accumulator to be read by 
clocked async provided that the current set of capabilities permit
it, and provided that the only statements prior to the read are
clocked asyncs that are stuck at an \hadvance. By convention we regard
the empty statement as stuck, hence this rule can be applied even if
there is no statement preceding the read. 

The rule~\RULE{(R-Acc-W)} updates the current contents of the
accumulator provided that the current set of capabilities permit
asynchronous access to the accumulator.

The rule~\RULE{(R-Advance)} permits a \code{clocked finish} to advance
only if all the top-level \code{clocked asyncs} in the scope of the
\code{clocked finish} and before an \code{advance} are stuck at an
\code{advance}. A \code{clocked finish} can also advance if all the
the top-level \code{clocked asyncs} in its scope are stuck. In this
case, the statement in the body of hte \code{clocked finish} has
terminated, and left behind only possibly clocked \code{async}. This
rule corresponds to the notion that the body of a \code{clocked
  finish} deregisters itself from the clock on local termination.
Note that in both these rules unlocked \code{aync} may exist in the
scope of the \code{clocked finish}; they do not come in the way of the
\code{clocked finish} advancing.

\begin{figure*}[t]
\begin{center}
\begin{tabular}{|c|}
\hline
$\typerule{
 \hS,H \preduce H'
}{
  \begin{array}{l}
    \pclocked~\finish{\hS},H \preduce H'\\
    \pclocked~\asynct{\pi_1}{\hS},H \ptreduce H'\ \ \ \pi=\pi_1,\pi_2\\
    \hS~\hS', H \preduce \hS',H'\\
  \end{array}
}$~\RULE{(R-Term)}
\\\\
$\typerule{
  \hS,H \preduce \hS', H'
}{
  \begin{array}{l}
    \pclocked~\finish{\hS},H \preduce \pclocked~\finish{\hS'},H'\\
    \pclocked~\asynct{\pi_1}{\hS},H \ptreduce  \pclocked~\asynct{\pi_1}{\hS'}, H'\ \ \ \pi=\pi_1,\pi_2\\
    \hS~\hS_1, H \preduce \hS'~\hS_1,H'\\
    \asynct{\pi_1}{\hS_1}~\hS, H \preduce \asynct{\pi_1}{\hS_1}~\hS', H'\\
  \end{array}
}$~\RULE{(R-Step)}
\\\\

$\typerule{
  \he,H \reduce \hl,H'
}{
  \hval{\hx}{\he}{\hS},H \preduce \hS[\hl/\hx], H'
}$~\RULE{(R-Val)}
~
$\typerule{
    \hl' \not \in \dom(H)
}{
  \hnew{\hC},H \preduce \hl',H[ \hl' \mapsto \hC()]
}$~\RULE{(R-New)}
~
$\typerule{
    H(\hl')=\hC(\ldots)
        \gap
    \mbody{}(\hm,\hC)=\ol{\hx}.\hS
}{
  \hl'.\hm(\ol{\hl}),H \preduce \hS[\ol{\hl}/\ol{\hx},\hl'/\hthis],H
}$~\RULE{(R-Invoke)}
\\\\
$\typerule{
    H(\hl)=\hC(\ol{\hf}\mapsto\ol{\hl'})
}{
  \hl.\hf_i,H \preduce \hl_i',H
}$~\RULE{(R-Access)}
\quad
$\typerule{
    H(\hl)=\hC(F) 
}{
  \hl.\hf=\hl',H \preduce H[ \hl \mapsto \hC(F[ \hf \mapsto \hl'])]
}$~\RULE{(R-Assign)}
\\\\
$\typerule{
    \hl \not \in \dom(H)
}{
 \asynct{\pi_1}{\acc{\hx}{\hnew{\hAcc(\hr,\hz)}}~\hS},H \preduce 
   \asynct{\pi_1,l}{\hS[\hl/\hx]},H[ \hl \mapsto \hAcc(\hr,\hz)]
}$~\RULE{(R-Acc-N)}
\\\\
$\typerule{
   \ha\in \pi \gap H(a)=\hAcc(\hr,\hv)\gap \hx \in \pi_1
}{
\asynct{\pi_1}{\hval{\hx}{\ha()}{\hS}},H \preduce 
\asynct{\pi_1}{\hS[\hv/\hx]}, H'
}$~\RULE{(R-Acc-A-R)}
\\\\
$\typerule{}{
    \hclocked~\asynct{\pi}{\xadvance~\hS} \stuck
}$~\RULE{(R-Stuck-CA)}
~
$\typerule{
  \hS_1 \stuck \gap \hS_2 \stuck
}{
   \hS_1~\hS_2 \stuck
}$~\RULE{(R-Stuck-S)}
\\\\
$\typerule{
   \ha\in \pi \gap H(a)=\hAcc(\hr,\hv)\gap \hx \in \pi_1 \gap S\stuck
}{
\hclocked~\asynct{\pi_1}{\hS~\hval{\hx}{\ha()}{\hS_1}},H \preduce 
\hclocked~\asynct{\pi_1}{\hS~\hS_1[\hv/\hx]}, H'
}$~\RULE{(R-Acc-CA-R)}
\\\\
$\typerule{
  \ha\in \pi\gap H(\ha)=\hAcc(\hr,\hv)\gap \hw=\hr(\hv,\hp)  
}{
  \ha \leftarrow \hp,H \preduce H[\ha \mapsto \hAcc(\hr,\hw)]
}$~\RULE{(R-Acc-W)}
\\\\

$\typerule{}{
    \asynct{\pi}{\hS}\ureduce \asynct{\pi}{\hS}\\
    \hclocked~\asynct{\pi}{\xadvance~\hS} \ureduce
    \hclocked~\asynct{\pi}{\hS} \\
    \hclocked~\finish{\hS}\ureduce \hclocked~\finish{\hS}\\
}$~\RULE{(R-Advance-A,CA,CF)}
\quad
$\typerule{
    \hS_1 \ureduce \hS_1'\ \     \hS_2 \ureduce \hS_2'
}{
  \hS_1~\hS_2\ureduce \hS_1'~\hS_2'
}$~\RULE{(R-Advance-S)}
\\\\
$\typerule{
    \hS \ureduce \hS'
}{
  \begin{array}{l}
    \hclocked~\finish{\hS~\xadvance~\hS_1},H\preduce \hclocked~\finish{\hS'~\hS_1},H\\
    \hclocked~\finish{\hS},H\preduce  \hclocked~\finish{\hS'},H
  \end{array}
}$~\RULE{(R-Advance)}
\\
\hline
\end{tabular}
\end{center}
\caption{FX10 Reduction Rules ($\hS,H \preduce \hS',H' ~|~H'$ and~$\he,H \preduce \hl,H'$).}
\label{Figure:reduction}
\end{figure*}


\Subsection{Results}


\section{Related work}\label{sec:related}
Constraint-based type systems, dependent types, and generic types
have been well-studied in the literature.

\paragraph{Constraint-based type systems.}

The use of constraints for type inference and subtyping has a history
going back to Mitchell~\cite{mitchell84} and by
Reynolds~\cite{reynolds85}.  These and subsequent systems are based on
constraints over types, but not over values.  Trifonov and
Smith~\cite{trifonov96} proposed a type system in which types are
refined using subtyping constraints.
Pottier~\cite{pottier96simplifying} presents a constraint-based type
system for an ML-like language with subtyping.  These developments
lead to \hmx~\cite{sulzmann97type}, a constraint-based framework for
Hindley--Milner-style type systems.  The framework is parametrized on
the specific constraint system $X$; instantiating $X$ yields
extensions of the HM type system.  Constraints in \hmx{} are over
types, not values. The \hmx{} approach is an important precursor to
our constrained types approach. The principal difference is that
\hmx{} applies to functional languages and does not integrate
dependent types.

%
Sulzmann and Stuckey~\cite{sulzmann-hmx-clpx} showed that the
type inference algorithm for \hmx can be encoded as a
constraint logic program parametrized by the constraint system
$X$. This is very much in spirit with our approach.
Constrained types open the door to {\em user-defined}
predicates and functions, effectively permitting the user to enrich
$\cal C$ (hence the power of the compile-time type-checker) by
developing application-specific constraints using a constraint
programming language such as CLP($\cal C$) \cite{clp} or the richer
RCC($\cal C$) \cite{DBLP:conf/fsttcs/JagadeesanNS05}.

\paragraph{Dependent types.}

Dependent type
systems~\cite{xi99dependent,calc-constructions,epigram,cayenne}
parametrize types on values.  Refinement type
systems~\cite{refinement-types,conditional-types,jones94,sized-types,flanagan-popl06,flanagan-fool06,liquid-types},
introduced by Freeman and Pfenning~\cite{refinement-types}, are dependent type
systems that extend a base type system through constraints on values.  These
systems do not treat value and type constraints uniformly.

Our work is closely related to DML, \cite{xi99dependent}, an
extension of ML with dependent types. DML is also built
parametrically on a constraint solver. Types are refinement types;
they do not affect the operational semantics and erasing the
constraints yields a legal DML program.  This differs from generic constrained
types, where erasure of subtyping constraints can prevent the program from
type-checking.
DML does not permit any run-time checking of constraints
(dynamic casts).

The most obvious distinction between DML and constrained types
lies in the target
domain: DML is designed for functional programming
whereas constrained types are designed for imperative, concurrent
object-oriented languages. 
But there are several other
crucial differences as well.

DML achieves its separation between compile-time and run-time processing
by not permitting program
variables to be used in types. Instead, a parallel set of (universally
or existentially quantified) ``index'' variables are
introduced.
%
Second, DML permits only variables of basic index sorts known to
the constraint solver (e.g., \Xcd{bool}, \Xcd{int}, \Xcd{nat}) to
occur in types. In contrast, constrained types permit program
variables at any type to occur in constrained types. As with DML
only operations specified by the constraint system are permitted in
types. However, these operations always include field selection and
equality on object references.  Note that DML-style constraints are easily
encoded in constrained types.

% {\em Conditional
% types}~\cite{conditional-types} extend refinement types to
% encode control-flow information in the types.
% %
% Jones introduced {\em qualified types}, which permit
% types to be constrained by a finite set of
% predicates~\cite{jones94}.
% %
% {\em Sized types}~\cite{sized-types}
% annotate types with the sizes of recursive data structures.
% Sizes are linear functions of size variables.
% Size inference is performed using a constraint solver for
% Presburger arithmetic~\cite{omega}.
% % constraints on types, support primitive recursion only

% Index objects must be pure.
% Singleton types int(n).
% ML$^{\Pi}_0$:
% Refinement of the ML type system: does not affect the
% operational semantics.  Can erase to ML$_0$.

% Jay and Sekanina 1996: array bounds checking based on shape
% types.

% Ada dependent types.
% Ada has constrained array definitions.  A constraint
% \cite{ada-ref-man}.  Not clear if they're dependent.  Are
% there other dependent types?  Generics are dependent?

        % Used for array bounds by Morrisett et al (I think--need
        % to find paper)

% Singleton types~\cite{aspinall-singletons}.

Logically qualified types, or liquid types~\cite{liquid-types},
permit types in a base Hindley--Milner-style type system to be refined with
conjunctions of logical qualifiers.  The subtyping relation is similar to
\Xten{}'s, that is, two liquid types are in the subtyping relation if their base
types are and if one type's qualifier implies the other's.
The Hindley--Milner type
inference algorithm is used to infer base types; these types are used as templates for inference of the liquid types.
The types of certain expressions are over-approximated to ensure inference
is decidable.
To improve precision of the inference algorithm, and hence
to reduce the annotation burden on the programmer, 
the type system is path sensitive.

Hybrid type-checking~\cite{flanagan-popl06,flanagan-fool06}
introduced another refinement type system.
While typing is undecidable, dynamic checks are inserted into
the program when necessary if the type-checker (which
includes a constraint solver) cannot determine
type safety statically.
In \FXG{}, dynamic type checks, including tests of dependent
constraints, are inserted only at explicit casts or
\Xcd{instanceof} expressions; constraint solving is performed at compile time.

% Where clauses for F-bounded polymorphism~\cite{where-clauses}
% Bounded quantification: Cardelli and Wegner.  Bound T with T'
% In F-bounded polymorphism~\cite{f-bounds}, type variables are bounded by a function of 
% the type variable. 
% Not dependent types.

Concoqtion~\cite{concoqtion} extends types in OCaml~\cite{ocaml}
with constraints written as Coq~\cite{coq} rules.
While the types are expressive, supporting the full generality
of the Coq language, proofs must be
provided to satisfy the type checker.
\Xten{} supports only constraints that can be checked by a
constraint solver during compilation.
Concoqtion encodes OCaml types and value to allow reasoning in
the Coq formulae; however, there is an impedance mismatch
caused by the differing syntax, representation, and behavior
of OCaml versus Coq.

\eat{
Cayenne~\cite{cayenne} is a Haskell-like language with fully dependent types.
There is no distinction between static and dynamic types.
Type-checking is undecidable.
There is no notion of datatype refinement as in DML.

Epigram~\cite{epigram,epigram-matter}
is a dependently typed functional programming language based on
a type theory with inductive families.
Epigram does not have a phase distinction between values and
types.
}

\eat{
$\lambda^{\sf Con}$ is a lambda calculus with assertions.
Findler, Felleisen, Contracts for higher-order functions (ICFP02)

  example: int[> 9]

contracts are either simple predicates or function contracts.
defined by (define/contract ...)

enforced at run-time.
}

% Jif~\cite{jif,jflow} is an extension of Java in which
% types are labeled with security policies enforced by the
% compiler.

\eat{
ESC/Java~\cite{esc-java}
allow programmers to write object invariants and pre- and
post-conditions that are enforced statically
by the compiler using an automated theorem prover.
Static checking is undecidable and, in the presence of loops,
is unsound (but still useful) unless the programmer supplies loop invariants.
ESC/Java can enforce invariants on mutable state.
}

% and Spec$\sharp$~\cite{specsharp}

\eat{
Pluggable and optional type systems were proposed by
Bracha~\cite{bracha04-pluggable} and provide another means of
specifying refinement types.
Type annotations, implemented in compiler plugins, serve only to
reject programs statically that might otherwise have dynamic
type errors.
CQual~\cite{foster-popl02} extends C with user-defined type
qualifiers.  These
qualifiers may be flow-sensitive and may be inferred. 
CQual supports only a fixed set of typing rules
for all qualifiers.
In contrast, the {\em semantic type qualifiers} of
Chin, Markstrum, and Millstein~\cite{chin05-qualifiers}
allow programmers to define typing rules for qualifiers
in a meta language that allows type-checking rules to be
specified declaratively.
JavaCOP~\cite{javacop-oopsla06} is a pluggable type system
framework for Java.  Annotations are defined in a meta language
that allows type-checking rules to be specified declaratively.
JSR 308~\cite{jsr308} is a proposal for adding user-defined type qualifiers
to Java.
}

% Holt, Cordy, the Turing programming language

% Ou, Tan, Mandelbaum, Walker, Dynamic typing with dependent types
% Separate dependent and simple parts of the program.
% Statically type the dependent parts.
% Dynamic checks when passing values into dependent part.

\paragraph{Genericity.}

Genericity in object-oriented languages is usually
supported through
type parametrization.

A number of proposals 
for adding genericity to Java quickly followed
the initial release of
the language~\cite{GJ,Pizza,java-popl97,thorup97,allen03}.
GJ~\cite{GJ} implements invariant type
parameters via type erasure.
PolyJ~\cite{java-popl97} supports run-time representation of types
via adapter objects, and also permits instantiation of
parameters on primitive types and structural parameter bounds.
Viroli and Natali~\cite{reflective-generics,type-passing-generics}
also support
a run-time representation of types, using Java's reflection API.
NextGen~\cite{nextgen,allen03} was implemented using run-time 
instantiation.
\Xten{}'s generics have a hybrid implementation, adopting PolyJ's
adapter object approach for dependent types and for 
type introspection and using NextGen's run-time
instantiation approach for greater efficiency.
% MixGen~\cite{allen04} extends NextGen with mixins.

\csharp also supports generic types via run-time instantiation in the
CLR~\cite{csharp-generics}.  Type parameters may be declared
with definition-site variance tags.
Generalized type constraints were proposed for
\csharp~\cite{emir06}.  Methods can be annotated with subtyping
constraints that must be satisfied to invoke the method.
Generic \Xten{} supports these constraints, as well as constraints
on values, with method and constructor where clauses.

\eat{
\FXG{} does not support \emph{bivariance}~\cite{variant-parametric-types}; a
class \xcd"C" is bivariant in a type property \xcd"X" if \xcd"C{self.X==S}" is
a subtype of \xcd"C{self.X==T}" for any \xcd"S" and \xcd"T".  Bivariance is
useful for writing code in which the property \xcd"X" is ignored.  One can
achieve  this effect in \FXG{} simply by leaving \xcd"X" unconstrained.
}

\eat{
Parametric types with use-site variance are related to existential types:
\xcd"C<+T>" corresponds to the bounded existential $\exists\tcd{X<:T}.C<X>$;
\xcd"C<-T>" corresponds to the bounded existential $\exists\tcd{X:>T}.C<X>$;
\xcd"C<*>" corresponds to the unbounded existential $\exists\tcd{X}.C<X>$.
\FXG{} has a similar correspondence:
\xcd"C{X<:T}" corresponds to the bounded existential \xcdmath"C\{\exists\tcd{self}:C.self.X<:T\}";
\xcd"C{X:>T}" corresponds to the bounded existential \xcdmath"C\{\exists\tcd{self}.C<X\}";
\xcd"C" corresponds to the unbounded existential \xcdmath"C\{\exists\tcd{self}.C<X\}".
}



\section{Conclusion and future work}\label{sec:future}\label{sec:conclusions}
%%V. Conclusion and future work. (0.5 page)
%
%state-dependent constrained types.
%
%use of dependent types for optimization. 
%
%type-inference.
%
%Bibliography (1.5 page)

We are actively writing programs with constrained types in
\Xten{} and 
working to extend the expressive power of the type system.
In particular, we are focusing on the following extensions:

\paragraph{Driving optimization.}
\paragraph{Clocks.}
\paragraph{Type constraints.}
By allowing classes to have type properties as well as value properties,
and by incorporating subtyping constraints into the language, we believe we can achieve the expressive power of generics.

\paragraph{Type inference.}  Constrainted types can be burdensome for
programmer to write down.  A type inference algorithm for constrained types.

\paragraph{Flow sensitivity.}  The following code should type-check:
\begin{xten}
int x = ...;
int(:self >= 0) y;
if (x >= 0) {
    y = x;
}
\end{xten}

\paragraph{State-dependent constrained types.}


\section{Conclusion}\label{s:concl}

\paragraph{Acknowledgements.} \XWS{} is being designed and implemented in collabration with Doug Lea. We thank Raj Barik for his contributions to the implementation of the C++ version of \XWS. We thank the rest of the X10 team for many discussions of these issues.
\fi

\section*{Acknowledgments}

The authors thank Radha Jagadeesan,
Norman Cohen, Pradeep Varma,
Satish Chandra, Martin Hirzel, Igor Peshansky,
Lex Spoon, Vincent Cave, Vivek Sarkar,
and the X10 team for fruitful discussions and implementation of
the X10 compiler and examples.
This material is based upon work supported by the Defense
Advanced Research Projects Agency under its Agreement No.
HR0011-07-9-0002.


\bibliographystyle{plain}
\bibliography{master}

\appendix

\section{Soundness}
\label{sec:proof}
Here we prove a soundness theorem for \CFJ{}.  

\begin{lemma}[Substitution Lemma]
\label{substitution}
If
$\Gamma \vdash \tbar{d}:\tbar{U}$,
$\Gamma, \bar{\tt x}:\tbar{U} \vdash \tbar{U}\subtype\tbar{V}$,
and
$\Gamma, \tbar{x}:\tbar{V} \vdash {\tt e:T}$,
then for some type {\tt S},
$\Gamma \vdash {\tt e[\tbar{d}/\tbar{x}]:S}, {\tt S \subtype \tbar{x}:\tbar{V};T}.$
\end{lemma}

\begin{proof}
Straightforward.
\end{proof}

\eat{
\begin{proof}
Assume the premises.  The proof is by structural induction on
{\tt e}.

\begin{itemize}
\item \xcd{x}.
Then $\Gamma, \xcd{x}_i \ty \xcd{V}_i \vdash {\tt T}$.
There are two subcases.

\begin{itemize}
\item
If \xcd{x} is \xcd{x$_i$}, then 
${\tt e[\tbar{d}/\tbar{x}]}$ = \xcd{d$_i$}.
$\Gamma \vdash {\tt d}_i \ty {\tt U}_i$.

$\Gamma \vdash {\tt U}_i \subtype {\tt x}_i \ty {\tt V}_i; {\tt T}$
follows from \rn{S-Exists-R}.

and
$\Gamma \vdash {\tt U}_i \subtype {\tt T}[{\tt d}_i/{\tt x}_i]$

Since,
$\Gamma \vdash {\tt d}_i \ty {\tt U}_i$
by \rn{T-sub} we have
$\Gamma \vdash {\tt d}_i \ty {\tt V}_i$.

\item
Otherwise, 
${\tt e[\tbar{d}/\tbar{x}] = e}$
and the case
follows from \rn{S-Exists-R}.
\end{itemize}

\item $\xcd{e}.\xcd{f}$
\item $\xcd{e}.\xcd{m}(\tbar{e})$
\item $\xcd{new}~\xcd{C}(\tbar{e})$
\item $\xcd{e}~\xcd{as}~\xcd{T}$
\end{itemize}
\end{proof}
}

% Unchanged from FJ
\begin{lemma}[Weakening]
\label{weakening}
If $\Gamma \vdash {\tt e} \ty {\tt T}$, then
$\Gamma, {\tt x} \ty {\tt S} \vdash {\tt e} \ty {\tt T}$.
\end{lemma}

\begin{proof}
Straightforward.
\end{proof}

\eat{
% Unchanged from FJ
\begin{lemma}[Method body type]
\label{body-type}
If
$$\Gamma, {\tt z}:{\tt T} \vdash {\tt z}\ \has\ {\tt m}(\bar{\tt z} \ty \bar{\tt U})\{{\tt c}\}: {\tt S} = {\tt e},$$
and 
$\Gamma, {\tt z}:{\tt T},\bar{\tt z}: \bar{\tt T} \vdash \bar{\tt T} \subtype \bar{\tt U}$,
then for some type ${\tt S}'$ it is the case that 
$\Gamma, {\tt z}:{\tt T},\bar{\tt z}: \bar{\tt T} \vdash {\tt e}: {\tt S}',{\tt S}'\ty {\tt S}$.
\end{lemma}

%%vj. Updated above formulation to use bars. Proof below needs to be fixed.
\begin{proof}
\eat{
From
$\Gamma, {\tt z}: {\tt T} \vdash {\tt z}\ \has\ {\tt m}(\bar{\tt z} \ty \bar{\tt U})\{{\tt c}\}: {\tt S} = {\tt e}$,
it is easy to show that for some {\tt C} such that $\Gamma \vdash {\tt T} \subtype {\tt C}$,
${\tt def}\ {\tt m}(\tbar{z}:\tbar{U}[\this/{\tt z}])\{{\tt c[\this/{\tt z}]}\}:{\tt S}[\this/{\tt z}] = {\tt e} \in P$.

By \rn{Method OK},
$\this:{\tt C}\vdash {\tt c}:{\tt o}$
and
$\this:{\tt C},\tbar{z}:\tbar{U}[\this/{\tt z}],{\tt c}[\this/{\tt z}] \vdash {\tt S}[\this/{\tt z}] \ \type, \tbar{U}[\this/{\tt z}] \ \type, {\tt e}:{\tt V}, {\tt V} \subtype {\tt S}[\this/{\tt z}]$.
}

Straightforward.
\end{proof}
}
\begin{lemma}
\label{existential-subtyping}
If   $\Gamma \vdash {\tt S} \subtype {\tt T}$,
then $$({\tt z}:{\tt S};~{\tt c}_0)[{\tt x}/\self] \vdashO ({\tt z}:{\tt T};~{\tt c}_0)[{\tt x}/\self]$$
where {\tt x} is fresh.
\end{lemma}

\begin{proof}
Straightforward.
\end{proof}

\eat{
\begin{lemma}
\label{erased-subtyping}
If   $\Gamma \vdash \xcd{C\{c\}} \subtype \xcd{D\{d\}}$,
then $\Gamma \vdash \xcd{C} \subtype \xcd{D}$.
\end{lemma}

\begin{proof}
From \rn{S-Id} and \rn{S-Const-L}, we have
$\Gamma \vdash \xcd{D\{d\}} \subtype \xcd{D}$.
From \rn{S-Const-R}
$\Gamma \vdash \xcd{C\{c\}} \subtype \xcd{D}$.
\end{proof}
}

\begin{lemma}
\label{subst-in-assumption}
If   $\Gamma, {\tt x}: {\tt U}, {\tt c} \vdashO {\tt c}_0$
and  $\Gamma \vdash {\tt t}: {\tt U}$,
then $$\Gamma, {\tt c}[{\tt t}/{\tt x}] \vdashO {\tt c}_0[{\tt t}/{\tt x}].$$
\end{lemma}

\begin{proof}
Straightforward.
\end{proof}

\begin{lemma}
\label{constraint-lemma} % 4. 
If   $\Gamma \vdash {\tt S} \subtype {\tt T}$,
and  $\sigma(\Gamma,{\tt f}: {\tt T}) \vdashO {\tt c}_0$,
then $\sigma(\Gamma,{\tt f}: {\tt S}) \vdashO {\tt c}_0$.
\end{lemma}

\begin{proof}
By induction on the derivation of 
     $\Gamma \vdash {\tt S} \subtype {\tt T}$.
We proceed by cases for the last judgment in the derivation.
\begin{itemize}
\item \textsc{S-Id}.
Trivial.
\item \textsc{S-Trans}.
Straightforward from the induction hypothesis.
\item \textsc{S-Extends}.
We have
${\tt S} = {\tt C}$ and ${\tt T} = {\tt D}$.
From the definition of $\sigma(\cdot)$ we have
    $$\sigma(\Gamma,{\tt f}: {\tt C}) =
    \sigma(\Gamma,{\tt f}: {\tt D}) = \sigma(\Gamma).$$
The conclusion follows easily.

\item \textsc{S-Const-L}.
We have
${\tt S} = {\tt T\{c\}}$.
Assume
$\sigma(\Gamma,{\tt f}: {\tt T}) \vdashO {\tt c}_0$.
From the definition of $\sigma(\cdot)$ we have
    $$\sigma(\Gamma,{\tt f}: {\tt T\{c\}}) = 
      \sigma(\Gamma, {\tt f}: {\tt T}), {\tt c}[{\tt f}/\self].$$
Since
$\sigma(\Gamma,{\tt f}: {\tt T}) \vdashO {\tt c}_0$, it follows immediately that
$\sigma(\Gamma,{\tt f}: {\tt T}), {\tt c}[{\tt f}/\self] \vdashO {\tt c}_0$.

\item \textsc{S-Const-R}.
We have
${\tt T} = {\tt U\{c\}}$ and $\Gamma \vdash {\tt S} \subtype {\tt U}$
and $\Gamma, \self: {\tt S} \vdash {\tt c}$.

Assume
$\sigma(\Gamma,{\tt f}: {\tt U\{c\}}) \vdashO {\tt c}_0$.
From the definition of $\sigma(\cdot)$ we have
    $$\sigma(\Gamma,{\tt f}: {\tt U\{c\}}) = 
      \sigma(\Gamma, {\tt f}: {\tt U}), {\tt c}[{\tt f}/\self].$$
Thus,
    $\sigma(\Gamma, {\tt f}: {\tt U}), {\tt c}[{\tt f}/\self] \vdashO {\tt c}_0$.

Since
$\Gamma, \self: {\tt S} \vdash {\tt c}$,
we have
$\Gamma, {\tt f}: {\tt S} \vdash {\tt c}[{\tt f}/\self]$,
and hence
$\sigma(\Gamma, {\tt f}: {\tt S}) \vdashO {\tt c}[{\tt f}/\self]$.

Since 
$\Gamma \vdash {\tt S} \subtype {\tt U}$,
applying the induction hypothesis to ${\tt c}[{\tt f}/\self]$, we have
$\sigma(\Gamma, {\tt f}: {\tt U}) \vdashO {\tt c}[{\tt f}/\self]$.
%
Therefore, in the judgment
$\sigma(\Gamma, {\tt f}: {\tt U}), {\tt c}[{\tt f}/\self] \vdashO {\tt c}_0$,
${\tt c}[{\tt f}/\self]$ is redundant and we can conclude
    $\sigma(\Gamma, {\tt f}: {\tt U}) \vdashO {\tt c}_0$.

Finally,
applying the induction hypothesis to ${\tt c}_0$, we have
$\sigma(\Gamma,{\tt f}: {\tt S}) \vdashO {\tt c}_0$.

\item \textsc{S-Exists-L}.
We have
${\tt S} = {\tt x}: {\tt U};~{\tt V}$
    and
    $\Gamma \vdash {\tt U}~\type$
    and
    $\Gamma \vdashO {\tt V} \subtype {\tt T}$
    where {\tt x} is fresh.

Assume $\sigma(\Gamma,{\tt f}: {\tt T}) \vdashO {\tt c}_0$.
%
By the induction hypothesis,
$\sigma(\Gamma,{\tt f}: {\tt V}) \vdashO {\tt c}_0$.
%
Adding ${\tt x}: {\tt U}$ to the assumptions,
we can conclude
$\sigma(\Gamma, {\tt x}: {\tt U}, {\tt f}: {\tt V}) \vdashO {\tt c}_0$.

From the definition of $\sigma(\cdot)$ we have
    $$\sigma(\Gamma,{\tt f}: ({\tt x}: {\tt U};~{\tt V})) =
      \sigma(\Gamma, {\tt x}: {\tt U}, {\tt f}: {\tt V}).$$
%
Thus,
$\sigma(\Gamma,{\tt f}: {\tt S}) \vdashO {\tt c}_0$.

\item \textsc{S-Exists-R}.
We have
${\tt T} = {\tt x}: {\tt U}; {\tt V}$
    and
    $\Gamma \vdash {\tt t}: {\tt U}$
    and
    $\Gamma \vdashO {\tt S} \subtype {\tt V}[{\tt t}/{\tt x}]$.

Assume
$\sigma(\Gamma,{\tt f}: {\tt T}) \vdashO {\tt c}_0$.
%
From the definition of $\sigma(\cdot)$ we have
    $$\sigma(\Gamma,{\tt f}: ({\tt x}: {\tt U};~{\tt V})) =
      \sigma(\Gamma, {\tt x}: {\tt U}, {\tt f}: {\tt V}).$$
%
Thus,
$\sigma(\Gamma, {\tt x}: {\tt U}, {\tt f}: {\tt V}) \vdashO {\tt c}_0$.
Since
$\Gamma \vdash {\tt x}: {\tt U}$, we can show via Lemma~\ref{subst-in-assumption} that
$\sigma(\Gamma, {\tt f}: {\tt V[{\tt t}/{\tt x}]}) \vdashO {\tt c}_0[{\tt t}/{\tt x}]$.

% XXX is this right?
Since $\Gamma \vdash {\tt c}_0 : {\tt o}$, {\tt x} is not free in ${\tt c}_0$.
Thus ${\tt c}_0[{\tt t}/{\tt x}] = {\tt c}_0$
and
$\sigma(\Gamma, {\tt f}: {\tt V[{\tt t}/{\tt x}]}) \vdashO {\tt c}_0$.
%
Since,
$\Gamma \vdashO {\tt S} \subtype {\tt V}[{\tt t}/{\tt x}$
and
$\sigma(\Gamma, {\tt f}: {\tt V[{\tt t}/{\tt x}}) \vdashO {\tt c}_0$,
by the induction hypothesis, we have
$\sigma(\Gamma, {\tt f}: {\tt S}) \vdashO {\tt c}_0$.
\end{itemize}
\end{proof}

\begin{lemma}
\label{subtype-lemma} % 3. 
if   $\Gamma, {\tt f}: {\tt T} \vdash {\tt U} \subtype {\tt U}'$,
and  $\Gamma \vdash {\tt S} \subtype {\tt T}$, then $\Gamma, {\tt f}: {\tt S} \vdash {\tt U} \subtype {\tt U}'$.
\end{lemma}

\begin{proof}
Follows From Lemma~\ref{constraint-lemma}.
\end{proof}

\begin{lemma}
\label{subtyping-in-existential-lemma} % 5. 
if   $\Gamma \vdash {\tt S} \subtype {\tt T}$,
then $\Gamma \vdash {\tt E}\{{\tt z}: {\tt S}; {\tt c}_0\} \subtype {\tt E}\{{\tt z}: {\tt T}; {\tt c}_0\}$.
\end{lemma}

\begin{proof}
To prove the desired conclusion 
$$\Gamma \vdash {\tt E}\{{\tt z}: {\tt S}; {\tt c}_0\} \subtype {\tt E}\{{\tt z}: {\tt T}; {\tt c}_0\},$$
we need to show that
$$\sigma(\Gamma, {\tt x}: {\tt E}\{{\tt z}: {\tt S}; {\tt c}_0\}) \vdashO ({\tt z}: {\tt T}; {\tt c}_0)[{\tt x}/\self].$$
%
We have 
$$
\sigma(\Gamma, {\tt x}: {\tt E}\{{\tt z}: {\tt S}; {\tt c}_0\})
                                                        = \sigma(\Gamma, ({\tt z}: {\tt S}; {\tt c}_0)[{\tt x}/\self])
$$
From Lemma~\ref{existential-subtyping} and
$\Gamma \vdash {\tt S} \subtype {\tt T}$, 
we have
$$({\tt z}: {\tt S}; {\tt c}_0)[{\tt x}/\self] \vdashO ({\tt z}: {\tt T}; {\tt c}_0)[{\tt x}/\self].$$
From $$
\sigma(\Gamma, {\tt x}: {\tt E}\{{\tt z}: {\tt S}; {\tt c}_0\})
                                                        = \sigma(\Gamma, ({\tt z}: {\tt S}; {\tt c}_0)[{\tt x}/\self])$$
and
$$({\tt z}: {\tt S}; {\tt c}_0)[{\tt x}/\self] \vdashO ({\tt z}: {\tt T}; {\tt c}_0)[{\tt x}/\self],$$
we conclude 
$$\sigma(\Gamma, {\tt x}: {\tt E}\{{\tt z}: {\tt S}; {\tt c}_0\}) \vdashO ({\tt z}: {\tt T}; {\tt c}_0)[{\tt x}/\self].$$
\end{proof}

% \begin{lemma}
% \label{lemmasix} % 6. 
% If $\Gamma \vdash \tbar{U}\ \tbar{d}$,
% $\theta = [\tbar{f} / \this.\tbar{f}]$,
% $\fields(C,\theta) = \tbar{Z} \tbar{f}$,
% $\Gamma, \tbar{U}\ \tbar{f} \vdash \tbar{U} \subtype \tbar{Z}$,
% $\sigma(\Gamma, \tbar{U}\ \tbar{f}) \vdashO inv(C,\theta)$,
% $\vdash C \subtype T[new C(\tbar{d}) / \self]$,
% then $\Gamma \vdash C(: \tbar{U}\ \tbar{f}; \self.\tbar{f} = 
%      \tbar{f}) \subtype T$.
% \end{lemma}
% 
% \begin{proof}
% (Notes)
% It is reasonable to assume that the constraint system {\cal C} satisfies the
% property that if $\tbar{d} = \tbar{e}$ (where $\tbar{d}$ and $\tbar{e}$ 
% are sequence of values) 
% then
% for any sequence of constraints $\tbar{c}$, $\tbar{c}[\tbar{d}/\self]$ and 
% $\tbar{c} [\tbar{e}/\self]$
% are equi-satisfiable, i.e., one holds iff the other holds. 
% (Here by $\tbar{c} [\tbar{d}/\self]$ we mean $c_1[d1/\self], \ldots, c_n[dn/\self]$.)
% 
% Now when proving subject reduction for this case we take the type $S$ to be 
% $C(: \tbar{U}\ \tbar{f}; \self.\tbar{f} = \tbar{f}, \self=o)$.
% Note the addition of the $\self=o$ clause. 
% Note also that from
% 
% $\Gamma \vdash C(: \tbar{U}\ \tbar{f}; \self.\tbar{f} = \tbar{f}) o$
% 
% we can derive
% 
% $\Gamma \vdash C(: \tbar{U}\ \tbar{f}; \self.\tbar{f} = \tbar{f}, \self=o) o$
% 
% This of course makes complete sense ... all we need to show is that {\em this
% particular object} $o$ is of the type that it has been cast to.
% 
% But we have just checked this condition, i.e. 
% $type(o) \subtype D and \vdashO d[o/\self]$. 
% So we are done.
% 
% Note about the proof:
% the trick of adding $\self=o$ is critical. There is no hope of showing
% that $C(: \tbar{U}\ \tbar{f}; \self.\tbar{f} = \tbar{f}) \subtype D(:d)$. 
% All we have checked is the one case that {\em this one object\/} $o$ 
% satisfies the condition $d$, not that *all* objects 
% that satisfy $C(: \tbar{U}\ \tbar{f}; \self.\tbar{f} = \tbar{f})$ 
% also satisfy $D(:d)$. 
% So we take advantage of our ability to choose the type $S$ for $o$ 
% to get this done.
%\end{proof}

% \begin{lemma}
% \label{lemmaseven} % 7. 
% If   $\Gamma, \tbar{T} \tbar{f} \vdash \tbar{T} \subtype \tbar{Z}$
% and  $\theta = [\tbar{f} / \this.\tbar{f}]$,
% and  $\fields(C,\theta) = \tbar{Z} \tbar{f}$,
% and  $\fields(T_0,z_0) = \tbar{U}\ \tbar{f}$,
% and  $T_0 equiv C(\tbar{T} \tbar{f}; \self.\tbar{f} = \tbar{f})$,
% then $\Gamma \vdash T_i \subtype (T_0 z_0; z_0.f_i = \self; U_i)$
% \end{lemma}
% 
% \begin{proof}
% From these we can conclude
%    $\Gamma \vdash T_0 new C(bar e)$
% where $T_0$ is $C(: \tbar{T} \tbar{p}; \self.\tbar{f} = \tbar{p})$.
% 
% Now the case we are concerned about is $new C(\tbar{e}).f_i --> e_i$.
% 
% We want to show that the static type of $e_i$, namely $T_i$, 
% is a subtype of the static
% type of $new C(\tbar{e}).f_i$, which is
% $(T_0 z_0; z_0.f_i=\self, V_i[z_0.\tbar{f}/\this.\tbar{f}])$
% 
% Let $x$ be an arbitrary element of $T_i$. 
% We wish to show that $x$ is an element of
% the type $(T_0 z_0; z_0.f_i=\self, V_i[z_0.\tbar{f}/\this.\tbar{f}])$.
% To do this we have
% to show that it is possible to construct from $x$ an object $z_0$ of type 
% $T_0$ such
% that $x$ is the $f_i$'th field of $z_0$. But this is given to us by (4).
% Let $\tbar{t}$ be
% a set of values of type $bar T$, with $x$ chosen at index $i$.
% Then per (4), $T_i$ is a
% subtype of $V_i[\tbar{t}/\this.\tbar{f}]$. Since $x$ is a value of the type $T_i$, 
% it is a value of the type $V_i[\tbar{t}/\this.\tbar{f}]$. 
% Therefore we have constructed the object $z_0$ 
% which is required to show that $x$ is an element of 
% $(T_0 z_0; z_0.f_i=\self, V_i[z_0.\tbar{f}/\this.\tbar{f}])$.
% 
% I have implicitly used here the following genericity property of constraint
% systems (see Vijay Saraswat's LICS 91 paper):
% If $\Gamma \vdash A$ then $\Gamma[\tbar{t}/\tbar{y}] \vdash A [\tbar{t}/\tbar{y}]$.
% That is the set of axioms of the constraint system may contain free
% variables but are assumed to be closed under instantiation.
% \end{proof}

\begin{lemma}
\label{fields-lemma} % 1. 
If   $\Gamma \vdash {\tt S} \subtype {\tt T}$,
and  $\Gamma, {\tt z}: {\tt S} \vdash \fields({\tt z}) = \tbar{F}_1$,
and  $\Gamma, {\tt z}: {\tt T} \vdash \fields({\tt z}) = \tbar{F}_2$,
then $\tbar{F}_2$ is a prefix of $\tbar{F}_1$.
\end{lemma}

\begin{proof}
Follows from Lemma~\ref{constraint-lemma} and rules for $\fields$  in ${\cal O}$.
\end{proof}

\begin{lemma}
\label{has-lemma} % 2. 
If   $\Gamma \vdash {\tt S} \subtype {\tt T}$,
and  $\Gamma, {\tt z}: {\tt T} \vdash {\tt z}\ \has\ {\tt I}$,
then $\Gamma, {\tt z}: {\tt S} \vdash {\tt z}\ \has\ {\tt I}$.
\end{lemma}

\begin{proof}
Follows from Lemma~\ref{constraint-lemma}.
\end{proof}

\begin{lemma}\label{existslemma}
If 
$\Gamma \vdash ({\tt x}: {\tt S}; \xcd{T\{c\}})\ \type$, then
$\Gamma \vdash ({\tt x}: {\tt S}; \xcd{T\{c\}}) \subtype \xcd{T\{x: S; c\}})$.
\end{lemma}

\begin{proof}
% From S-Const-L
% G, x:S |- T{c} <: T
% Then from S-Exists-L
% G |- x:S; T{c} <: T
% By definition of \sigma and c |- c,
% G,self:(x:S; T{c}) |-O {x: S; c},
% From
% G |- x:S; T{c} <: T
% and
% G,self:(x:S; T{c}) |-O {x: S; c},
% by S-Const-R we have
% G |- x: S; T{c} <: T{x: S; c}
Straightforward.
\end{proof}


\begin{theorem}[Subject Reduction] 
\label{preservation}
If $\Gamma \vdash {\tt e}: {\tt V}$ and ${\tt e} \derives {\tt e}'$, then for some type ${\tt V}'$,
$\Gamma \vdash {\tt e}': {\tt V}'$ and $\Gamma \vdash {\tt V}' \subtype {\tt V}$.
\end{theorem}

\begin{proof}
We proceed by induction on the
structure of the derivation of $\Gamma \vdash {\tt e}: {\tt T}$.  We now have five
cases depending on the last rule used in the derivation
of $\Gamma \vdash {\tt e}: {\tt T}$.
\begin{itemize}
\item
\TVar: The expression cannot take a step, so the conclusion is immediate.
\item
\TCast: We have two subcases.
   \begin{itemize}
   \item
   \RCast:  For the expression {\tt o~as~V}, where 
            ${\tt o} = {\tt \new\ {\tt C(\tbar{d})}}$,
            we have from \TNew\ that 
            $$\Gamma \vdash {\tt o: {\tt C}\{\tbar{z}: \tbar{T}; \new\ {\tt C}(\tbar{z}) = \self; \inv({\tt C},\self)\}}.$$
            Additionally, we have from \RCast\ that 
            $$\vdash {\tt C}\{\new\ {\tt C}(\tbar{d}) = \self\} \subtype {\tt V}.$$
            We now choose 
            $${\tt V}' = {{\tt C}\{\tbar{z}: \tbar{T}; \new\ {\tt C}(\tbar{z}) = \self; \inv({\tt C},\self)\}}.$$

            From \rn{S-Exists-L},
            $$
            \begin{array}{r}
            \Gamma \vdash \tbar{z}: \tbar{T}; {{\tt C}\{\new\ {\tt C}(\tbar{z}) = \self; \inv({\tt C},\self)\}} \\
            \subtype {\tt C}\{\new\ {\tt C}(\tbar{d}) = \self\}.
            \end{array}
            $$

            From Lemma~\ref{existslemma},
            $$
            \begin{array}{r}
            \Gamma \vdash
            \tbar{z}: \tbar{T}; {{\tt C}\{\new\ {\tt C}(\tbar{z}) = \self; \inv({\tt C},\self)\}} \\
            \subtype
                {{\tt C}\{\tbar{z}: \tbar{T}; \new\ {\tt C}(\tbar{z}) = \self; \inv({\tt C},\self)\}}
            \end{array}
            $$

        Thus,
            from \rn{S-Trans}, $\Gamma \vdash {\tt V}' \subtype {\tt V}$.
   \item
   \RCCast: For the expression {\tt o~as~V}, we have from \TCast\ that
            $\Gamma \vdash {\tt o}: {\tt U}$.
            Additionally, we have from \RCCast\ that
            ${\tt o} \derives {{\tt o}}'$.
            From the induction hypothesis, we have ${\tt U}'$ such that
            $\Gamma \vdash {\tt o}': {\tt U}'$ and
            $\Gamma \vdash {\tt U}' \subtype {\tt U}$.
            We now choose ${\tt V}'={\tt V}$.
            From $\Gamma \vdash {\tt o}': {\tt U}'$ and \TCast\ we derive
            $\Gamma \vdash {\tt o}'{\tt ~as~V} : {\tt V}$.
            From ${\tt V}'={\tt V}$ and \rn{S-Id}
            we have $\Gamma \vdash {\tt V}' \subtype {\tt V}$.
   \end{itemize}
\item
\TNew: We have a single case.
   \begin{itemize}
   \item
   \RCNewArg: For the expression ${\new\ {\tt C}(\tbar{e})}$,
            we have from \TNew\ that
            $$
            \begin{array}{l}
            \Gamma \vdash \tbar{e}: \tbar{T}, \\
            \vdash \class({\tt C}), \\
            \Gamma \vdash {\tt z}: {\tt C} \vdash \fields({\tt z}) = \tbar{f}: \tbar{S}, \\
            \Gamma \vdash {\tt z}: {\tt C}, \tbar{z}: \tbar{T}, {\tt z}.\tbar{f} = \tbar{z} \vdash \tbar{T} \subtype \tbar{S}, \inv({\tt C}, {\tt z}).
            \end{array}$$

            We choose
            ${\tt V} = {\tt C}\{ \tbar{z}: \tbar{T}; \new\ {\tt C}(\tbar{z}) = \self, \inv({\tt C}, \self)\}$.

            Additionally, we have from \RCNewArg\ that
            ${\tt e}_i \derives {{\tt e}}_i'$.
            From the induction hypothesis, we have ${\tt U}_i$ such that
            $\Gamma \vdash {\tt e}_i': {{\tt U}}_i$ and 
            $\Gamma \vdash U_i \subtype T_i$.

            For all $j$ except $i$, define ${\tt U}_j = {\tt T}_j$ and ${\tt e}_j' = {\tt e}_j$.
            We have 
            $\Gamma \vdash \tbar{e}': \tbar{U}$ and
            $\Gamma \vdash \tbar{U} \subtype \tbar{T}$.

            From Lemma~\ref{weakening}, we have
            $$\Gamma \vdash {\tt z}: {\tt C}, \tbar{z}: \tbar{U}, {\tt z}.\tbar{f} = \tbar{z} \vdash \tbar{U} \subtype \tbar{T}.$$

            % From Lemma~\ref{subtype-lemma},
            % $\Gamma, \tbar{f}: \tbar{T} \vdash \tbar{T} \subtype \tbar{Z}$,
            % and $\Gamma \vdash \tbar{S} \subtype \tbar{T}$, we have $\Gamma, \tbar{f}: \tbar{S} \vdash \tbar{T} \subtype \tbar{Z}.$

            % From Lemma~\ref{weakening}, we have
            % $\Gamma \vdash {\tt z}: {\tt C}, \tbar{z}: \tbar{U}, {\tt z}.\tbar{f} = \tbar{z} \vdash \tbar{U} \subtype \tbar{S}$.

            From \TNew,
            we have
            $$\Gamma \vdash {\tt z}: {\tt C}, \tbar{z}: \tbar{T}, {\tt z}.\tbar{f} = \tbar{z} \vdash \tbar{T} \subtype \tbar{S}.$$
            From Lemma~\ref{subtype-lemma},
            $$\Gamma \vdash {\tt z}: {\tt C}, \tbar{z}: \tbar{U}, {\tt z}.\tbar{f} = \tbar{z} \vdash \tbar{T} \subtype \tbar{S}.$$

            From \rn{S-Trans},
            we have
            $$\Gamma \vdash {\tt z}: {\tt C}, \tbar{z}: \tbar{U}, {\tt z}.\tbar{f} = \tbar{z} \vdash \tbar{U} \subtype \tbar{S}.$$

            From \TNew,
            $$\Gamma \vdash {\tt z}: {\tt C}, \tbar{z}: \tbar{T}, {\tt z}.\tbar{f} = \tbar{z} \vdash \inv({\tt C}, {\tt z}).$$
            From Lemma~\ref{subtype-lemma},
            $$\Gamma \vdash {\tt z}: {\tt C}, \tbar{z}: \tbar{U}, {\tt z}.\tbar{f} = \tbar{z} \vdash \inv({\tt C}, {\tt z}).$$

            Thus, by \TNew,
            $$\Gamma \vdash {\new\ {\tt C}(\tbar{e}')} : {\tt C}\{ \tbar{z}: \tbar{U}; \new\ {\tt C}(\tbar{z}) = \self, \inv({\tt C}, \self)\}$$
            and we choose
            $${\tt V}' = {\tt C}\{ \tbar{z}: \tbar{U}; \new\ {\tt C}(\tbar{z}) = \self, \inv({\tt C}, \self)\}.$$

            From Lemma~\ref{existential-subtyping}, we have
            $\Gamma \vdash {\tt V}' \subtype {\tt V}$.
   \end{itemize}
\item
\TField: We have two subcases.
   \begin{itemize}
   \item
   \RField:  For the expression ${\tt (\new\ C(\tbar{e})).f_i}$, 
             we have from \TField\ that
             $$
             \begin{array}{l}
             \Gamma \vdash {\tt e}: {\tt S}, \\
             \Gamma, {\tt z}: {\tt S} \vdash {\tt z}\ \has\ {\tt f}_i: {\tt U}_i. \\
             \end{array}$$
             Let ${\tt V} = ({\tt z}: {\tt S}; {\tt U}_i\{\self{\tt ==}{\tt z}.{\tt f}_i\})$.
             ${\tt z}$ is fresh.

             % Additionally, we have from \RField\ that
             % ${\tt z}: {\tt C} \vdash \fields({\tt z}) = \tbar{f}: \tbar{T}$.

             We have 
             $${\tt S} = {\tt C}\{\tbar{z}: \tbar{T}{\tt ; \new\ {\tt C}(\tbar{z}) = \self, \inv({\tt C},\self)}\}.$$

             From \TNew, we have
             $\Gamma \vdash \tbar{e}: \tbar{T}$
             %,
             %$\vdash \class(C)$,
             %$\Gamma \vdash {\tt z}: {\tt C} \vdash \fields({\tt z}) = \tbar{f}: \tcd{U}_i$,
             and
             $$\Gamma \vdash {\tt z}: {\tt C}, \tbar{z}: \tbar{T}, {\tt z}.\tbar{f} = \tbar{z} \vdash \tcd{T}_i \subtype \tcd{U}_i.$$

             From $\Gamma \vdash \tbar{e}: \tbar{T}$, we have
             $\Gamma \vdash {\tt e}_i: {\tt T}_i$.
             We now choose ${\tt V}' = {\tt T}_i$.

             By \TNew,
             $$\Gamma, {\tt z}: {\tt C}, \tbar{z}: \tbar{T}, {\tt z}.\tbar{f} = \tbar{z} \vdash \tcd{T}_i \subtype \tcd{U}_i.$$

             By \rn{S-Const-R},
             $$\Gamma, {\tt z}: {\tt C}, \tbar{z}: \tbar{T}, {\tt z}.\tbar{f} = \tbar{z} \vdash \tcd{T}_i \subtype \tcd{U}_i\{\self = {\tt z}_i\}.$$

             Since ${\tt z}.{\tt f}_i = {\tt z}_i$, by application of \rn{S-Id}, \rn{S-Const-L}, and \rn{S-Const-R}, we have
             $$\Gamma, {\tt z}: {\tt C}, \tbar{z}: \tbar{T}, {\tt z}.\tbar{f} = \tbar{z} \vdash \tcd{T}_i \subtype \tcd{U}_i\{\self = {\tt z}.{\tt f}_i\}.$$

             % XXX leap of logic: need to simplify the environment; it should be fine since the zs are fresh and not used
             We can then show via \rn{S-Exists-R} that
             $$\Gamma \vdash {\tt T}_i \subtype ({\tt z}: {\tt S}; {\tt U}_i\{\self = {\tt z}.{\tt f}_i\}),$$
             or more simply
             $\Gamma \vdash {\tt V}' \subtype {\tt V}$.

   \item
   \RCField:
             Follows from the induction hypothesis and application of Lemma~\ref{subtyping-in-existential-lemma}.
        \eat{   
             For the expression ${\tt e.f}_i$, we have from \TField\ that
             $\Gamma \vdash {\tt e}: {\tt S}$ and
             $\Gamma, {\tt z}: {\tt S} \vdash {\tt z}\ \has\ {\tt f}: {\tt T}$
             where ${\tt z}$ is fresh.
             Let ${\tt V} = ({\tt z}: {\tt S}; {\tt T}\{\self = {\tt z}.{\tt f}\})$.
             Additionally, we have from \RCField\ that  
             ${\tt e} \derives {{\tt e}}'$.
             From the induction hypothesis, we have ${\tt S}'$ such that 
             $\Gamma \vdash {\tt e}': {\tt S}'$ and $\Gamma \vdash {\tt S}' \subtype {\tt S}$.

             We now choose 
             ${\tt V'} = ({\tt z}: {\tt S}'; {\tt z}.{\tt f}_i=\self;{\tt U}_i)$.
             From $\Gamma \vdash {\tt e}': {\tt S}'$, and
             Lemma~\ref{fields-lemma},
             $\Gamma \vdash {\tt e}' : {\tt V}'$.

             From $\Gamma \vdash {\tt S} \subtype {\tt S}'$ and 
             Lemma~\ref{subtyping-in-existential-lemma}, we have $\Gamma \vdash {\tt V} \subtype {\tt V}'$.
             }
   \end{itemize}
\item
\TInvk: We have three subcases.
   \begin{itemize}
   \item
   \RInvk:  
            For simplicity, define ${\tt d}_0 = \new\ {\tt C}(\tbar{e})$.
            For the expression 
            ${\tt d}_0.{\tt m}(\tbar{d})$
            we have from $\TInvk$ that
            $$
            \begin{array}{l}
            \Gamma \vdash {\tt d}_0: {\tt T}_0 \\
            \Gamma \vdash {\tt d}_{1:n}: {\tt T}_{1:n} \\
            \Gamma, {\tt z}_{0:n}: {\tt T}_{0:n} \vdash {\tt z}_0\ \has\ {\tt m}({\tt z}_{1:n} \ty {\tt U}_{1:n})\{c\}: {\tt S} = {\tt e} \\
            \Gamma, {\tt z}_{0:n}: {\tt T}_{0:n} \vdash {\tt T}_{1:n} \subtype {\tt U}_{1:n} \\
            \Gamma, {\tt z}_{0:n}: {\tt T}_{0:n} \vdash {\tt c} \\
            \end{array}
            $$
            \noindent
            where ${\tt z}_{0:n}$ is fresh.  By \TNew, we have
            $\Gamma \vdash \tbar{e}: \tbar{A}$
            and
            $${\tt T}_0 = {\tt C}\{\tbar{z}: \tbar{A}; \self = \new\ {\tt C}(\tbar{z}), \inv({\tt C},\self)\}.$$

            Since
            $$\Gamma, {\tt z}_{0:n}: {\tt T}_{0:n} \vdash {\tt z}_0\ \has\ {\tt m}({\tt z}_{1:n} \ty {\tt U}_{1:n})\{c\}: {\tt S} = {\tt e},$$
            and 
            $\Gamma, {\tt z}_{0:n}: {\tt T}_{0:n} \vdash {\tt T}_{1:n} \subtype {\tt U}_{1:n}$,
            by Lemma~\ref{body-type}, we have
            $\Gamma {\tt z}_{0:n}: {\tt T}_{0:n} \vdash {\tt e}: {\tt S}$.

            Choose ${\tt V} = ({\tt z}_{0:n}: {\tt T}_{0:n};\ {\tt S})$.

            From \RInvk, we have
            ${\tt d}_0.{\tt m}(\tbar{d}) \derives {\tt e}[{\tt d}_0,\tbar{d}/\this,\tbar{z}]$.

            By Lemma~\ref{substitution},
            $\Gamma \vdash {\tt e}[{\tt d}_0,\tbar{d}/\this,\tbar{z}] : {\tt V}'$,
            and
            $\Gamma \vdash {\tt V}' \subtype {\tt z}_{0:n} : {\tt T}_{0:n}{\tt ;\ S}$.

   \item
   \RCInvkRecv:
             Follows from the induction hypothesis and application of Lemma~\ref{subtyping-in-existential-lemma}.
        \eat{   
   For the expression ${\tt e_0.m(\tbar{e})}$,
            we have from \TInvk\ that
            $$
            \begin{array}{l}
            \Gamma \vdash {\tt e}_{0:n} : {\tt T}_{0:n}, \\
            \mtype({\tt T}_0,{\tt m},{\tt z}_0) = 
               \tt {\tt z}_{1:n}: {\tt Z}_{1:n}:c \rightarrow {\tt U}, \\
            \Gamma, {\tt z}_{0:n}: {\tt T}_{0:n} \vdash 
                  {\tt T}_{1:n} \subtype {\tt Z}_{1:n}, \mbox{and} \\
            \sigma(\Gamma, {\tt z}_{0:n}: {\tt T}_{0:n})
                  \vdashO {\tt c} \\
            \end{array}$$
            where ${\tt z}_{0:n}$ is fresh.

            Additionally, from \RCInvkRecv\ we have that
            ${\tt e}_0 \derives {{\tt e}}_0'$.
            From the induction hypothesis, we have ${\tt S}_0$ such that
            $\Gamma \vdash {\tt e}_0': {{\tt S}}_0$ and 
            $\Gamma \vdash S_0 \subtype T_0$.
            For all $j>0$, define $S_j = T_j$ and $e_j' = e_j$.
            We have 
            $\Gamma \vdash \tbar{e}': \tbar{S}$ and
            $\Gamma \vdash \tbar{S} \subtype \tbar{T}$.

            From Lemma~\ref{has-lemma}
            and $\Gamma \vdash \tbar{S} \subtype \tbar{T}$, we have
            $$\mtype(S_0,m,z) = \mtype(T_0,m,z).$$

            From Lemma~\ref{subtype-lemma},
            $\Gamma, {\tt z}_{0:n}: {\tt T}_{0:n} \vdash
                  {\tt T}_{1:n} \subtype {\tt Z}_{1:n}$,
            and $\Gamma \vdash \tbar{S} \subtype \tbar{T}$, we have
            $\Gamma, {\tt z}_{0:n}: {\tt S}_{0:n} \vdash
                  {\tt T}_{1:n} \subtype {\tt Z}_{1:n}$,

            From Lemma~\ref{constraint-lemma}, 
            $\Gamma \vdash \tbar{S} \subtype \tbar{T}$, and
            $\sigma(\Gamma, {\tt z}_{0:n}: {\tt T}_{0:n}) \vdashO
                              {\tt c}$
            we have
            $\sigma(\Gamma, {\tt z}_{0:n}: {\tt S}_{0:n}) \vdashO
                              {\tt c}.$

            We now choose 
               $S=({\tt z}_{0:n}: {\tt S}_{0:n}; U)$.
            From 
            $\Gamma \vdash {\tt e}_{0:n}': {\tt S}_{0:n}$,
            $\mtype({\tt S}_0,{\tt m},{\tt z}_0) =
               \tt {\tt z}_{1:n}: {\tt Z}_{1:n}:c \rightarrow {\tt U}$,
            $\Gamma, {\tt z}_{0:n}: {\tt S}_{0:n} \vdash
                  {\tt T}_{1:n} \subtype {\tt Z}_{1:n}$, and
            $\sigma(\Gamma, {\tt z}_{0:n}: {\tt S}_{0:n}) \vdashO {\tt c}$,
            and \TInvk\ we derive
            $\Gamma \vdash {\tt S}\ {\tt e_0.m(e_{1:n}')}$.
            We have 
               $T=({\tt z}_{0:n}: {\tt T}_{0:n}; {\tt U})$.

            From Lemma~\ref{subtyping-in-existential-lemma} and
            $\Gamma \vdash \tbar{S} \subtype \tbar{T}$, we have
            $\Gamma \vdash S \subtype T$.
            }
   \item
   \RCInvkArg:
             Follows from the induction hypothesis and application of Lemma~\ref{subtyping-in-existential-lemma}.
             \eat{
        For the expression ${\tt e_0.m(\tbar{e})}$,
            we have from \TInvk\ that
            $\Gamma \vdash {\tt e}_{0:n} : {\tt T}_{0:n}$,
            $\mtype({\tt T}_0,{\tt m},{\tt z}_0) = 
               \tt {\tt z}_{1:n}: {\tt Z}_{1:n}:c \rightarrow {\tt U}$,
            $\Gamma, {\tt z}_{0:n}: {\tt T}_{0:n} \vdash 
                  {\tt T}_{1:n} \subtype {\tt Z}_{1:n}$, and
            $\sigma(\Gamma, {\tt z}_{0:n}: {\tt Z}_{0:n}) \vdashO {\tt c}$, 
            where ${\tt z}_{0:n}$ is fresh.

            Additionally, from \RCInvkArg\ we have that, for $i>0$,
            ${\tt e}_i \derives {{\tt e}}_i'$.
            From the induction hypothesis, we have ${\tt S}_i$ such that
            $\Gamma \vdash {\tt e}_i': {{\tt S}}_i$ and 
            $\Gamma \vdash S_i \subtype T_i$.
            For all $j$ except $i$, define $S_j = T_j$ and $e_j' = e_j$.
            We have 
            $\Gamma \vdash \tbar{e}': \tbar{S}$ and
            $\Gamma \vdash \tbar{S} \subtype \tbar{T}$.

            From Lemma~\ref{subtype-lemma},
            $\Gamma, {\tt z}_{0:n}: {\tt T}_{0:n} \vdash
                  {\tt T}_{1:n} \subtype {\tt Z}_{1:n}$,
            and $\Gamma \vdash \tbar{S} \subtype \tbar{T}$, we have
            $\Gamma, {\tt z}_{0:n}: {\tt S}_{0:n} \vdash
                  {\tt T}_{1:n} \subtype {\tt Z}_{1:n}$,

            From Lemma~\ref{weakening} and 
            $\Gamma \vdash \tbar{S} \subtype \tbar{T}$, 
            we have 
            $\Gamma, {\tt z}_{0:n}: {\tt S}_{0:n} \vdash 
                  \tbar{S} \subtype \tbar{T}$.

            From \rn{S-Trans},
            $\Gamma, {\tt z}_{0:n}: {\tt S}_{0:n} \vdash 
                  \tbar{S} \subtype \tbar{T}$,
            and
            $\Gamma, {\tt z}_{0:n}: {\tt S}_{0:n} \vdash 
                    {\tt T}_{1:n} \subtype {\tt Z}_{1:n}$, 
            we have
            $\Gamma, {\tt z}_{0:n}: {\tt S}_{0:n} \vdash 
                  \tbar{S} \subtype {\tt Z}_{1:n}$.

            From Lemma~\ref{constraint-lemma}, 
            $\Gamma \vdash \tbar{S} \subtype \tbar{T}$, and
            $\sigma(\Gamma, {\tt z}_{0:n}: {\tt T}_{0:n}) \vdashO
                              {\tt c}$
            we have
            $\sigma(\Gamma, {\tt z}_{0:n}: {\tt S}_{0:n}) \vdashO
                              {\tt c}.$

            We now choose 
               $S=({\tt z}_{0:n}: {\tt S}_{0:n}; U)$.
            From 
            $\Gamma \vdash {\tt e}_{0:n}': {\tt S}_{0:n}$,
            $\mtype({\tt S}_0,{\tt m},{\tt z}_0) =
               \tt {\tt z}_{1:n}: {\tt Z}_{1:n}:c \rightarrow {\tt U}$,
            $\Gamma, {\tt z}_{0:n}: {\tt S}_{0:n} \vdash
                  {\tt S}_{1:n} \subtype {\tt Z}_{1:n}$, and
            $\sigma(\Gamma, {\tt z}_{0:n}: {\tt S}_{0:n}) \vdashO
                  {\tt c}$,
            and \TInvk\ we derive
            $\Gamma \vdash {\tt e_0.m(e_{1:n}')}: {\tt S}$.
            We have 
               $T=({\tt z}_{0:n}: {\tt T}_{0:n}; U)$.

            From Lemma~\ref{subtyping-in-existential-lemma} and
            $\Gamma \vdash \tbar{S} \subtype \tbar{T}$, we have
            $\Gamma \vdash S \subtype T$.
   }
   \end{itemize}
\end{itemize}
\end{proof}

\noindent
Let the normal form of expressions be given by {\em values}
{\tt v} {::=} $\new\ {\tt C(\tbar{v})}$.

\begin{theorem}[Progress] 
\label{progress}
If $\vdash {\tt e: T}$, then one of the following conditions holds:
\begin{enumerate}
\item {\tt e} is a value {\tt v}, 
\item {\tt e} contains a subexpression ${\tt \new\ C(\tbar{v})~as~T}$ such that
$\not\vdash {\tt C} \subtype {\tt T}[{\tt \new\ C(\tbar{v})}/\self]$,
\item there exists ${\tt e}'$ s.t. ${\tt e} \derives {\tt e}'$.
\end{enumerate}
\end{theorem}

\begin{proof}
The proof has a structure that is similar to the proof of Subject Reduction;
we omit the details.
\end{proof}

\begin{theorem}[Type Soundness] 
\label{type-soundness}
If $\vdash {\tt e: T}$ and ${\tt e} \starderives {\tt e}'$, with ${\tt
e}'$ in normal form, then ${\tt e}'$ is either (1)~a value {\tt v}
with $\vdash {\tt v: S}$ and $\vdash {\tt S \subtype T}$,
for some type {\tt S}, or, (2)~ an expression containing
a subexpression ${\tt \new\ {\tt C(\tbar{v})~as~T}}$ where 
$\not\vdash \tt C\subtype T[\new\ C(\tbar{v})/\self]$.

\end{theorem}

\begin{proof}
Combine Theorem~\ref{preservation} and Theorem~\ref{progress}.
\end{proof}



% \appendix
% \onecolumn

% \section{An extended example}
% {\footnotesize
\begin{verbatim}
/**
   A distributed binary tree.
   @author Satish Chandra 4/6/2006
   @author vj
 */
//                             ____P0
//                            |     |
//                            |     |
//                          _P2  __P0
//                         |  | |   |
//                         |  | |   |
//                        P3 P2 P1 P0
//                         *  *  *  *
// Right child is always on the same place as its parent;
// left child is at a different place at the top few levels of the tree,
// but at the same place as its parent at the lower levels.

class Tree(localLeft: boolean,
           left: nullable Tree(& localLeft => loc=here),
           right: nullable Tree(& loc=here),
           next: nullable Tree) extends Object {
    def postOrder:Tree = {
        val result:Tree = this;
        if (right != null) {
            val result:Tree = right.postOrder();
            right.next = this;
            if (left != null) return left.postOrder(tt);
        } else if (left != null) return left.postOrder(tt);
        this
    }
    def postOrder(rest: Tree):Tree = {
        this.next = rest;
        postOrder
    }
    def sum:int = size + (right==null => 0 : right.sum()) + (left==null => 0 : left.sum)
}
value TreeMaker {
    // Create a binary tree on span places.
    def build(count:int, span:int): nullable Tree(& localLeft==(span/2==0)) = {
        if (count == 0) return null;
        {val ll:boolean = (span/2==0);
         new Tree(ll,  eval(ll => here : place.places(here.id+span/2)){build(count/2, span/2)},
           build(count/2, span/2),count)}
    }
}
\end{verbatim}}

\subsection{Places}
{\footnotesize
\begin{verbatim}
/**

 * This class implements the notion of places in X10. The maximum
 * number of places is determined by a configuration parameter
 * (MAX_PLACES). Each place is indexed by a nat, from 0 to MAX_PLACES;
 * thus there are MAX_PLACES+1 places. This ensures that there is
 * always at least 1 place, the 0'th place.

 * We use a dependent parameter to ensure that the compiler can track
 * indices for places.
 *
 * Note that place(i), for i <= MAX_PLACES, can now be used as a non-empty type.
 * Thus it is possible to run an async at another place, without using arays---
 * just use async(place(i)) {...} for an appropriate i.

 * @author Christoph von Praun
 * @author vj
 */

package x10.lang;

import x10.util.List;
import x10.util.Set;

public value class place (nat i : i <= MAX_PLACES){

    /** The number of places in this run of the system. Set on
     * initialization, through the command line/init parameters file.
     */
    config nat MAX_PLACES;

    // Create this array at the very beginning.
    private constant place value [] myPlaces = new place[MAX_PLACES+1] fun place (int i) {
	return new place( i )(); };

    /** The last place in this program execution.
     */
    public static final place LAST_PLACE = myPlaces[MAX_PLACES];

    /** The first place in this program execution.
     */
    public static final place FIRST_PLACE = myPlaces[0];
    public static final Set<place> places = makeSet( MAX_PLACES );

    /** Returns the set of places from first place to last place.
     */
    public static Set<place> makeSet( nat lastPlace ) {
	Set<place> result = new Set<place>();
	for ( int i : 0 .. lastPlace ) {
	    result.add( myPlaces[i] );
	}
	return result;
    }

    /**  Return the current place for this activity.
     */
    public static place here() {
	return activity.currentActivity().place();
    }

    /** Returns the next place, using modular arithmetic. Thus the
     * next place for the last place is the first place.
     */
    public place(i+1 % MAX_PLACES) next()  { return next( 1 ); }

    /** Returns the previous place, using modular arithmetic. Thus the
     * previous place for the first place is the last place.
     */
    public place(i-1 % MAX_PLACES) prev()  { return next( -1 ); }

    /** Returns the k'th next place, using modular arithmetic. k may
     * be negative.
     */
    public place(i+k % MAX_PLACES) next( int k ) {
	return places[ (i + k) % MAX_PLACES];
    }

    /**  Is this the first place?
     */
    public boolean isFirst() { return i==0; }

    /** Is this the last place?
     */
    public boolean isLast() { return i==MAX_PLACES; }
}
\end{verbatim}}
\subsection{$k$-dimensional regions}
{\footnotesize
\begin{verbatim}
package x10.lang;

/** A region represents a k-dimensional space of points. A region is a
 * dependent class, with the value parameter specifying the dimension
 * of the region.
 * @author vj
 * @date 12/24/2004
 */

public final value class region( int dimension : dimension >= 0 )  {

    /** Construct a 1-dimensional region, if low <= high. Otherwise
     * through a MalformedRegionException.
     */
    extern public region (: dimension==1) (int low, int high)
        throws MalformedRegionException;

    /** Construct a region, using the list of region(1)'s passed as
     * arguments to the constructor.
     */
    extern public region( List(dimension)<region(1)> regions );

    /** Throws IndexOutOfBoundException if i > dimension. Returns the
        region(1) associated with the i'th dimension of this otherwise.
     */
    extern public region(1) dimension( int i )
        throws IndexOutOfBoundException;


    /** Returns true iff the region contains every point between two
     * points in the region.
     */
    extern public boolean isConvex();

    /** Return the low bound for a 1-dimensional region.
     */
    extern public (:dimension=1) int low();

    /** Return the high bound for a 1-dimensional region.
     */
    extern public (:dimension=1) int high();

    /** Return the next element for a 1-dimensional region, if any.
     */
    extern public (:dimension=1) int next( int current )
        throws IndexOutOfBoundException;

    extern public region(dimension) union( region(dimension) r);
    extern public region(dimension) intersection( region(dimension) r);
    extern public region(dimension) difference( region(dimension) r);
    extern public region(dimension) convexHull();

    /**
       Returns true iff this is a superset of r.
     */
    extern public boolean contains( region(dimension) r);
    /**
       Returns true iff this is disjoint from r.
     */
    extern public boolean disjoint( region(dimension) r);

    /** Returns true iff the set of points in r and this are equal.
     */
    public boolean equal( region(dimension) r) {
        return this.contains(r) && r.contains(this);
    }

    // Static methods follow.

    public static region(2) upperTriangular(int size) {
        return upperTriangular(2)( size );
    }
    public static region(2) lowerTriangular(int size) {
        return lowerTriangular(2)( size );
    }
    public static region(2) banded(int size, int width) {
        return banded(2)( size );
    }

    /** Return an \code{upperTriangular} region for a dim-dimensional
     * space of size \code{size} in each dimension.
     */
    extern public static (int dim) region(dim) upperTriangular(int size);

    /** Return a lowerTriangular region for a dim-dimensional space of
     * size \code{size} in each dimension.
     */
    extern public static (int dim) region(dim) lowerTriangular(int size);

    /** Return a banded region of width {\code width} for a
     * dim-dimensional space of size {\code size} in each dimension.
     */
    extern public static (int dim) region(dim) banded(int size, int width);


}

\end{verbatim}}

\subsection{Point}
{\footnotesize
\begin{verbatim}
package x10.lang;

public final class point( region region ) {
    parameter int dimension = region.dimension;
    // an array of the given size.
    int[dimension] val;

    /** Create a point with the given values in each dimension.
     */
    public point( int[dimension] val ) {
        this.val = val;
    }

    /** Return the value of this point on the i'th dimension.
     */
    public int valAt( int i) throws IndexOutOfBoundException {
        if (i < 1 || i > dimension) throw new IndexOutOfBoundException();
        return val[i];
    }

    /** Return the next point in the given region on this given
     * dimension, if any.
     */
    public void inc( int i )
        throws IndexOutOfBoundException, MalformedRegionException {
        int val = valAt(i);
        val[i] = region.dimension(i).next( val );
    }

    /** Return true iff the point is on the upper boundary of the i'th
     * dimension.
     */
    public boolean onUpperBoundary(int i)
        throws IndexOutOfBoundException {
        int val = valAt(i);
        return val == region.dimension(i).high();
    }

    /** Return true iff the point is on the lower boundary of the i'th
     * dimension.
     */
    public boolean onLowerBoundary(int i)
        throws IndexOutOfBoundException {
        int val = valAt(i);
        return val == region.dimension(i).low();
    }
}
\end{verbatim}}

\subsection{Distribution}
{\footnotesize
\begin{verbatim}
package x10.lang;

/** A distribution is a mapping from a given region to a set of
 * places. It takes as parameter the region over which the mapping is
 * defined. The dimensionality of the distribution is the same as the
 * dimensionality of the underlying region.

   @author vj
   @date 12/24/2004
 */

public final value class distribution( region region ) {
    /** The parameter dimension may be used in constructing types derived
     * from the class distribution. For instance,
     * distribution(dimension=k) is the type of all k-dimensional
     * distributions.
     */
    parameter int dimension = region.dimension;

    /** places is the range of the distribution. Guranteed that if a
     * place P is in this set then for some point p in region,
     * this.valueAt(p)==P.
     */
    public final Set<place> places; // consider making this a parameter?

    /** Returns the place to which the point p in region is mapped.
     */
    extern public place valueAt(point(region) p);

    /** Returns the region mapped by this distribution to the place P.
        The value returned is a subset of this.region.
     */
    extern public region(dimension) restriction( place P );

    /** Returns the distribution obtained by range-restricting this to Ps.
        The region of the distribution returned is contained in this.region.
     */
    extern public distribution(:this.region.contains(region))
        restriction( Set<place> Ps );

    /** Returns a new distribution obtained by restricting this to the
     * domain region.intersection(R), where parameter R is a region
     * with the same dimension.
     */
    extern public (region(dimension) R) distribution(region.intersection(R))
        restriction();

    /** Returns the restriction of this to the domain region.difference(R),
        where parameter R is a region with the same dimension.
     */
    extern public (region(dimension) R) distribution(region.difference(R))
        difference();

    /** Takes as parameter a distribution D defined over a region
        disjoint from this. Returns a distribution defined over a
        region which is the union of this.region and D.region.
        This distribution must assume the value of D over D.region
        and this over this.region.

        @seealso distribution.asymmetricUnion.
     */
    extern public (distribution(:region.disjoint(this.region) &&
                                dimension=this.dimension) D)
        distribution(region.union(D.region)) union();

    /** Returns a distribution defined on region.union(R): it takes on
        this.valueAt(p) for all points p in region, and D.valueAt(p) for all
        points in R.difference(region).
     */
    extern public (region(dimension) R) distribution(region.union(R))
        asymmetricUnion( distribution(R) D);

    /** Return a distribution on region.setMinus(R) which takes on the
     * same value at each point in its domain as this. R is passed as
     * a parameter; this allows the type of the return value to be
     * parametric in R.
     */
    extern public (region(dimension) R) distribution(region.setMinus(R))
        setMinus();

    /** Return true iff the given distribution D, which must be over a
     * region of the same dimension as this, is defined over a subset
     * of this.region and agrees with it at each point.
     */
    extern public (region(dimension) r)
        boolean subDistribution( distribution(r) D);

    /** Returns true iff this and d map each point in their common
     * domain to the same place.
     */
    public boolean equal( distribution( region ) d ) {
        return this.subDistribution(region)(d)
            && d.subDistribution(region)(this);
    }

    /** Returns the unique 1-dimensional distribution U over the region 1..k,
     * (where k is the cardinality of Q) which maps the point [i] to the
     * i'th element in Q in canonical place-order.
     */
    extern public static distribution(:dimension=1) unique( Set<place> Q );

    /** Returns the constant distribution which maps every point in its
        region to the given place P.
    */
    extern public static (region R) distribution(R) constant( place P );

    /** Returns the block distribution over the given region, and over
     * place.MAX_PLACES places.
     */
    public static (region R) distribution(R) block() {
        return this.block(R)(place.places);
    }

    /** Returns the block distribution over the given region and the
     * given set of places. Chunks of the region are distributed over
     * s, in canonical order.
     */
    extern public static (region R) distribution(R) block( Set<place> s);


    /** Returns the cyclic distribution over the given region, and over
     * all places.
     */
    public static (region R) distribution(R) cyclic() {
        return this.cyclic(R)(place.places);
    }

    extern public static (region R) distribution(R) cyclic( Set<place> s);

    /** Returns the block-cyclic distribution over the given region, and over
     * place.MAX_PLACES places. Exception thrown if blockSize < 1.
     */
    extern public static (region R)
        distribution(R) blockCyclic( int blockSize)
        throws MalformedRegionException;

    /** Returns a distribution which assigns a random place in the
     * given set of places to each point in the region.
     */
    extern public static (region R) distribution(R) random();

    /** Returns a distribution which assigns some arbitrary place in
     * the given set of places to each point in the region. There are
     * no guarantees on this assignment, e.g. all points may be
     * assigned to the same place.
     */
    extern public static (region R) distribution(R) arbitrary();

}
\end{verbatim}}

\subsection{Arrays}
Finally we can now define arrays. An array is built over a
distribution and a base type.

{\footnotesize
\begin{verbatim}
package x10.lang;

/** The class of all  multidimensional, distributed arrays in X10.

    <p> I dont yet know how to handle B@current base type for the
    array.

 * @author vj 12/24/2004
 */

public final value class array ( distribution dist )<B@P> {
    parameter int dimension = dist.dimension;
    parameter region(dimension) region = dist.region;

    /** Return an array initialized with the given function which
        maps each point in region to a value in B.
     */
    extern public array( Fun<point(region),B@P> init);

    /** Return the value of the array at the given point in the
     * region.
     */
    extern public B@P valueAt(point(region) p);

    /** Return the value obtained by reducing the given array with the
        function fun, which is assumed to be associative and
        commutative. unit should satisfy fun(unit,x)=x=fun(x,unit).
     */
    extern public B reduce(Fun<B@?,Fun<B@?,B@?>> fun, B@? unit);


    /** Return an array of B with the same distribution as this, by
        scanning this with the function fun, and unit unit.
     */
    extern public array(dist)<B> scan(Fun<B@?,Fun<B@?,B@?>> fun, B@? unit);

    /** Return an array of B@P defined on the intersection of the
        region underlying the array and the parameter region R.
     */
    extern public (region(dimension) R)
        array(dist.restriction(R)())<B@P>  restriction();

    /** Return an array of B@P defined on the intersection of
        the region underlying this and the parametric distribution.
     */
    public  (distribution(:dimension=this.dimension) D)
        array(dist.restriction(D.region)())<B@P> restriction();

    /** Take as parameter a distribution D of the same dimension as *
     * this, and defined over a disjoint region. Take as argument an *
     * array other over D. Return an array whose distribution is the
     * union of this and D and which takes on the value
     * this.atValue(p) for p in this.region and other.atValue(p) for p
     * in other.region.
     */
    extern public (distribution(:region.disjoint(this.region) &&
                                dimension=this.dimension) D)
        array(dist.union(D))<B@P> compose( array(D)<B@P> other);

    /** Return the array obtained by overlaying this array on top of
        other. The method takes as parameter a distribution D over the
        same dimension. It returns an array over the distribution
        dist.asymmetricUnion(D).
     */
    extern public (distribution(:dimension=this.dimension) D)
        array(dist.asymmetricUnion(D))<B@P> overlay( array(D)<B@P> other);

    extern public array<B> overlay(array<B> other);

    /** Assume given an array a over distribution dist, but with
     * basetype C@P. Assume given a function f: B@P -> C@P -> D@P.
     * Return an array with distribution dist over the type D@P
     * containing fun(this.atValue(p),a.atValue(p)) for each p in
     * dist.region.
     */
    extern public <C@P, D>
        array(dist)<D@P> lift(Fun<B@P, Fun<C@P, D@P>> fun, array(dist)<C@P> a);

    /**  Return an array of B with distribution d initialized
         with the value b at every point in d.
     */
    extern public static (distribution D) <B@P> array(D)<B@P> constant(B@? b);

}
\end{verbatim}}


\begin{example}
 The code for {\tt List} translates as given in Table~\ref{List-translation}.
\end{example}

\begin{figure*}
{\footnotesize
\begin{verbatim}
  public value class List <Node> {
    public final nat n;   // is a parameter
    nullable Node node = null;
    nullable List<Node> rest = null;  // All assignments must check n = this.n-1.

    /** Returns the empty list. Defined only when the parameter n
        has the value 0. Invocation: new List(0)<Node>().
     */
    public List ( final nat n ) {
      assume n==0;
      this.n = n;
    }

    /** Returns a list of length 1 containing the given node.
        Invocation: new List(1)<Node>( node ).
     */
    public List ( final nat n, Node node ) {
      assume n==1;                         // From the constructor precondition.
      assert 0==0 : "DependentTypeError"; // For the constructor call.
      assert n>=1 : "DependentTypeError"; // For the this call.
      this(n, node, new List<Node>(0));
    }

    public List ( final nat n, Node node, List<Node> rest ) {
      assume n>=1;                               // From the constructor precondition
      assume rest.n==n-1 : "DependentTypeError"; // From the argument type.
      this.n = n;
      this.node = node;
      assert rest.n==n-1 : "DependentTypeError"; // For the field assignment.
      this.rest = rest;
    }

    public  List<Node> append( List<Node> arg ) {
      if (n == 0) {
          final List<Node> result = arg;
          assert n+arg.n == result.n : "DependentTypeError"; // For the return value
          return result;
      } else {
          assume rest.n == n-1;
          final List<Node> argval = rest.append(arg);
          assume argval.n == rest.n+arg.n;
          assert n+arg.n-1== argval.n : "DependentTypeError"; // For the constructor call.
          final List<Node> result = new List<Node>(n+arg.n, node, argval);
          assume result.n == n+arg.n;
          assert n+arg.n == result.n : "DependentTypeError"; // For the return value
          return result;
      }
    }

\end{verbatim}}
\caption{Translation of {\tt List} (contd in Table~\ref{List-translation-2}).}\label{List-translation}
\end{figure*}
\begin{figure*}
{\footnotesize
\begin{verbatim}
    public  List<Node> rev() {
      final List<Node> arg = new List<Node>(0);
      assume arg.n = 0;                           // From the constructor call.
      final List<Node> result = rev( arg );
      assume result.n == n+arg.n;                  // From the method signature
      assert n == result.n : "DependentTypeError"; // For the return value.
      return result;
    }

    public  List(n+arg.n)<Node> rev( final List<Node> arg) {
      if (n==0) {
         assert n+arg.n == arg.n : "DependentTypeError"; // For the return value.
         return arg;
      } else {
        assert 1+arg.n-1=arg.n : "DependentTypeError"; // For the argument to the constructor
        final List<Node> arg2 = new List<Node>(1+arg.n,node, arg));
        assume arg2.n==1+arg.n;                      // From the constructor invocation
        final List<Node> restval = rest;             // Read from a mutable field of parametric type
        assume restval.n == n-1;                     // From the field read.
        final List(restval.n+arg2.n)<Node> result = restval.rev( arg2 );
        assume result.n=restval.n+arg2.n
        assert n+arg.n == result.n                   // For the return value
        return result;
    }

    /** Return a list of compile-time unknown length, obtained by filtering
        this with f. */
    public List<Node> filter(fun<Node, boolean> f) {
         if (n==0) return this;
         if (f(node)) {
           final List<Node> l = rest.filter(f);
           assert l.n+1-1==l.n : "DependentTypeError"; // For the constructor call
           return new List<Node>(l.n+1,node, l);
         } else {
           return rest.filter(f);
         }
    }

    /** Return a list of m numbers from o..m-1. */
    public static  List<nat> gen( final nat m ) {
         assert 0 <= m : "DependentTypeError";        // Precondition for method call.
         final List<nat> result = gen(0,m);
         assume result.n=m-0 : "DependentTypeError";  // From the method signature
         assert m == result.n : "DependentTypeError"; // For the return value
         return result;
    }

    /** Return a list of (m-i) elements, from i to m-1. */
    public static List<nat> gen(final nat i, final nat m) {
      assume i <= m;                                   // Method precondition.
      if (i==m) {
        assert m-i == 0 : "DependentTypeError";        // For the constructor call
        final List result = new List<nat>(m-i);
        assume result.n == 0;                          // From the constructor call.
        assert m-i == result.n : "DependentTypeError"; // For the return value.
        return result;
      } else {
        assert i+1 <= m : "DependentTypeError";        // For the method call.
        final List<nat> arg = gen(i+1,m);
        assume arg.n = m-(i+1);                        // From the method call.
        assert m-i-1 = arg.n;                          // For the constructor invocation.
        final List result = new List<nat>(m-i, i, arg);
        assume result.n = m-i;                         // From the constructor invocation.
        assert m-i == result.n : "DependentTypeError"; // For the return value
        return result;
    }
  }
\end{verbatim}}
\caption{Translation of {\tt List} (continued).}\label{List-translation-2}
\end{figure*}

\section{Type-checking dependent classes}

Each programming language---such as \Xten{}---will specify the base
underlying classes (and the operations on them) which can occur as
types in parameter lists. For instance, in the code for {\tt List}
above, the only type that appears in parameter lists is {\tt int}, and
the only operations on {\tt int} are addition, subtraction, {\tt >=},
{\tt ==}, and the only constants are {\tt 0} and {\tt 1}.  (This
language falls within Presburger arithmetic, a decidable fragment of
arithmetic.)  The compiler must come equipped with a constraint solver
(decision procedure) that can answer questions of the form: does one
constraint entail another?  Constraints are atomic formulas built up
from these operations, using variables. For instance, the compiler
must answer each one of:
{\footnotesize
\begin{verbatim}
  n >= 2 |- n-1 >= 0
  n >= 0, m >= 0 |- m+n >= 0
\end{verbatim}}

Ultimately, the only variables that will occur in constraints are
those that correspond to {\tt config} parameters and those that are
defined by implicit parameter definitions. We need to establish that
the verification of any class will generate only a finite number of
constraints, hence only a finite constraint problem for the constraint
solver.

Second, it should be possible for instances of user-defined classes
(and operations on them) to occur as type parameters. For the compiler
to check conditions involving such values, it is necessary that the
underlying constraint solver be extended.

There are two general ways in which the constraint solver may be
extended.  Both require that the programmer single out some classes
and methods on those classes as {\em pure}. (We shall think of
constants as corresponding to zero-ary methods.) Only instances of
pure classes and expressions involving pure methods on these instances
are allowed in parameter expressions.

How shall constraints be generated for such pure methods? First, the
programmer may explicitly supply with each pure method {\tt T m(T1 x1,
..., Tn xn)} a constraint on {\tt n+2} variables in the constraint
system of the underlying solver that is entailed by {\tt y =
o.m(x1,..., xn)}. Whenever the compiler has to perform reasoning on an
expression involving this method invocation, it uses the constraint
supplied by the programmer. A second more ambitious possibility is
that a symbolic evaluator of the language may be run on the body of
the method to automatically generate the corresponding constraint.

Finally an additional possibility is that the constraint solver itself
be made extensible. In this case, when a user writes a class which is
intended to be used in specifying parameters, he also supplies an
additional program which is used to extend the underlying constraint
solver used by the compiler. This program adds more primitive
constraints and knows how to perform reasoning using these
constraints. This is how I expect we will initially implement the
\Xten{} language. As language designers and implementers we will
provide constraint solvers for finite functions and {\tt Herbrand}
terms on top of arithmetic.





\end{document}
