Here we prove a soundness theorem for \CFJ{}.  

\begin{lemma}[Substitution Lemma]
\label{substitution}
Assume $\Gamma \vdash \bar{\tt A}\ \bar{\tt d}$, 
$\Gamma, \bar{A} \bar{x} \vdash \bar{\tt A}\subtype \bar{\tt B}$, and 
$\Gamma, \bar{\tt B}\ \bar{\tt x} \vdash {\tt T}\ {\tt e}$. 
Then there exists a type ${\tt S}$ such that 
$\Gamma \vdash {\tt S}\ {\tt e}[\bar{\tt d}/\bar{\tt x}]$ and
$\Gamma \vdash {\tt S} \subtype \bar{\tt A}\ \bar{\tt x};{\tt T}$.
\end{lemma}

\begin{proof}
Straightforward.
\end{proof}

% Unchanged from FJ
\begin{lemma}[Weakening]
\label{weakening}
If $\Gamma \vdash {\tt T}\ {\tt e}$, then $\Gamma, {\tt S}\ {\tt x}\vdash {\tt T}\ {\tt e}$.
\end{lemma}

\begin{proof}
Straightforward.
\end{proof}

% Unchanged from FJ
\begin{lemma}[Body type]
\label{body-type}
If $\mtype(C(:d),{\tt m},z_0)=\bar{\tt T}\ \bar{\tt x} : {\tt c}
\rightarrow {\tt S}$, and $mbody({\tt m}, C)=\bar{\tt x}.{\tt e}$, 
then there exists ${\tt U}, {\tt V}$ such that
$C \subtype {\tt U}$, ${\tt V}\subtype {\tt S}$, and
$\bar{\tt T}\ \bar{\tt x},{\tt U}\ \this \vdash {\tt V}\ {\tt e}$.
\end{lemma}

\begin{proof}
Straightforward.
\end{proof}

\eat{
\subsection{Erasure}

Constrained types in \CFJ{} are a form of {\em refinement
type}~\cite{refinement-types}.  If constraints are erased from a
well-typed program,
the resulting program will behave identically to the original unerased
program except that the original program might be unable to take
a step on a cast.

Let $\Lb {\tt e} \Rb$ be the erasure of ${\tt e}$ defined as follows:
\begin{align*}
\Lb {\tt x} \Rb &= {\tt x} \\
\Lb {\tt e}.{\tt f} \Rb &= \Lb {\tt e} \Rb.{\tt f} \\
\Lb {\tt e}.{\tt m}(\bar{\tt e}) \Rb &= \Lb {\tt e} \Rb.{\tt m}(\bar{\Lb {\tt e} \Rb}) \\
\Lb {\tt new}~{\tt C}(\bar{\tt e}) \Rb &= {\tt new}~{\tt C}(\bar{\Lb {\tt e} \Rb}) \\
\Lb ({\tt C}(:{\tt c}))~{\tt e} \Rb &= (\tt C)~\bar{\Lb {\tt e} \Rb}
\end{align*}

\begin{theorem}[Erasure]

If $\vdash {\tt C}(:{\tt c})\ {\tt e}$ and ${\tt e} \starderives {\tt v}$,
then $\vdash {\tt C}\ \Lb {\tt e} \Rb$ and $\Lb
{\tt e} \Rb \starderives \Lb {\tt v} \Rb$.

\end{theorem}
}

\begin{lemma}
\label{subtyping}
We have $\Gamma \vdash T \subtype T$.
If $\Gamma \vdash T_1 \subtype T_2$ and $\Gamma \vdash T_2 \subtype T_3$,
then $\Gamma \vdash T_1 \subtype T_3$.
\end{lemma}

\begin{proof}
Straightforward.
\end{proof}

\begin{lemma}
\label{lemmaone} % 1. 
if   $\Gamma \vdash S \subtype T$,
then $\fields(S,z)$ has $\fields(T,z)$ as a prefix.
\end{lemma}

\begin{proof}
Immediate from the definition of $\fields$.
\end{proof}

\begin{lemma}
\label{lemmatwo} % 2. 
if   $S \subtype T$,
then $\mtype(S,m,z) = \mtype(T,m,z)$.
\end{lemma}

\begin{proof}
Immediate from the definition of $\mtype$.
\end{proof}

\begin{lemma}
\label{existential-subtyping}
If   $\Gamma \vdash S \subtype T$,
then $(S z; c_0)[x/\self] \vdash_{\cal C} (T z; c_0)[x/\self]$
where $x$ is fresh.
\end{lemma}

\begin{proof}
Straightforward.
\end{proof}

\begin{lemma}
\label{lemmafour} % 4. 
If   $\Gamma \vdash S \subtype T$,
and  $\sigma(\Gamma,{\tt T}\ {\tt f}) \vdash_{\cal C} c_0$,
then $\sigma(\Gamma,{\tt S}\ {\tt f}) \vdash_{\cal C} c_0$,
\end{lemma}

\begin{proof}
From $\Gamma \vdash S \subtype T$,
we must have $S = C(:c)$ and $T = D(:d)$
where
$C \subtype D$
and
$\sigma(\Gamma, C(:c) x) \vdash_{\cal C} d[x/\self]$
with $x$ fresh.
From the definition of $\sigma(\cdot)$ we have
    $\sigma(\Gamma,{\tt S}\ {\tt f}) = \sigma(\Gamma), c[f/\self], inv(C,f)$,
and $\sigma(\Gamma,{\tt T}\ {\tt f}) = \sigma(\Gamma), d[f/\self], inv(D,f)$.
From $\Gamma \vdash S \subtype T$ we have 
$\sigma(\Gamma, C(:c) x) \vdash_{\cal C} d[f/\self]$.
Additionally, from the definition of $inv(C,f)$ and
from $C \subtype D$, 
we have that $inv(C,F)$ is a constraint that has $inv(D,f)$ as a conjunct so 
$inv(C,f) \vdash_{\cal C} inv(D,f)$.
We conclude
$\sigma(\Gamma,{\tt S}\ {\tt f}) \vdash_{\cal C} 
\sigma(\Gamma,{\tt T}\ {\tt f})$.
We have the $\vdash_{\cal C}$ is transitive so 
from $\sigma(\Gamma,{\tt S}\ {\tt f}) \vdash_{\cal C}
\sigma(\Gamma,{\tt T}\ {\tt f})$
and
$\sigma(\Gamma,{\tt T}\ {\tt f}) \vdash_{\cal C} c_0$,
we have 
$\sigma(\Gamma,{\tt S}\ {\tt f}) \vdash_{\cal C} c_0$.
\end{proof}

\begin{lemma}
\label{lemmathree} % 3. 
if   $\Gamma, {\tt T}\ {\tt f} \vdash U \subtype U'$,
and  $\Gamma \vdash S \subtype T$,
then $\Gamma, {\tt S}\ {\tt f} \vdash U \subtype U'$.
\end{lemma}

\begin{proof}
From $\Gamma, {\tt T}\ {\tt f} \vdash U \subtype U'$
we must have $U = C(:c)$ and $U' = D(:d)$
where
$C \subtype D$
and 
$\sigma(\Gamma, {\tt T}\ {\tt f}, C(:c) x) \vdash_{\cal C} d[x/\self]$ 
with $x$ fresh.
From Lemma~\ref{lemmafour},
$\Gamma \vdash S \subtype T$,
and
$\sigma(\Gamma, {\tt T}\ {\tt f}, C(:c) x) \vdash_{\cal C} d[x/\self]$,
we have
$\sigma(\Gamma, {\tt S}\ {\tt f}, C(:c) x) \vdash_{\cal C} d[x/\self]$.
So we can use 
$C \subtype D$
and
$\sigma(\Gamma, {\tt S}\ {\tt f}, C(:c) x) \vdash_{\cal C} d[x/\self]$
to derive 
$\Gamma, {\tt S}\ {\tt f} \vdash U \subtype U'$.
\end{proof}

\begin{lemma}
\label{lemmafive} % 5. 
if   $\Gamma \vdash S \subtype T$,
then $\Gamma \vdash E(S\ z; c_0) \subtype E(T\ z; c_0)$.
\end{lemma}

\begin{proof}
To prove the desired conclusion $E(S\ z; c_0) \subtype E(T\ z; c_0)$ 
we need to show that
$\sigma(\Gamma, E(:S\ z; c_0) x) \vdash_{\cal C} (T\ z; c_0)[x/\self]$.

We have 
$\sigma(\Gamma, E(:S\ z; c_0) x) = 
 \sigma(\Gamma), (S\ z; c_0)[x/\self], inv(E,x)$.
From Lemma~\ref{existential-subtyping} and
$\Gamma \vdash S \subtype T$, 
we have
$(S z; c_0)[x/\self] \vdash_{\cal C} (T z; c_0)[x/\self]$.
From $\sigma(\Gamma, E(:S\ z; c_0) x) =
 \sigma(\Gamma), (S\ z; c_0)[x/\self], inv(E,x)$
and
$(S z; c_0)[x/\self] \vdash_{\cal C} (T z; c_0)[x/\self]$,
we conclude 
$\sigma(\Gamma, E(:S\ z; c_0) x) \vdash_{\cal C} (T\ z; c_0)[x/\self]$.
\end{proof}

% \begin{lemma}
% \label{lemmasix} % 6. 
% If $\Gamma \vdash \bar{\tt U}\ \bar{\tt d}$,
% $\theta = [\bar{\tt f} / \this.\bar{\tt f}]$,
% $\fields(C,\theta) = \bar{\tt Z} \bar{\tt f}$,
% $\Gamma, \bar{\tt U}\ \bar{\tt f} \vdash \bar{\tt U} \subtype \bar{\tt Z}$,
% $\sigma(\Gamma, \bar{\tt U}\ \bar{\tt f}) \vdash_{\cal C} inv(C,\theta)$,
% $\vdash C \subtype T[new C(\bar{\tt d}) / \self]$,
% then $\Gamma \vdash C(: \bar{\tt U}\ \bar{\tt f}; \self.\bar{\tt f} = 
%      \bar{\tt f}) \subtype T$.
% \end{lemma}
% 
% \begin{proof}
% (Notes)
% It is reasonable to assume that the constraint system {\cal C} satisfies the
% property that if $\bar{\tt d} = \bar{\tt e}$ (where $\bar{\tt d}$ and $\bar{\tt e}$ 
% are sequence of values) 
% then
% for any sequence of constraints $\bar{\tt c}$, $\bar{\tt c}[\bar{\tt d}/\self]$ and 
% $\bar{\tt c} [\bar{\tt e}/\self]$
% are equi-satisfiable, i.e., one holds iff the other holds. 
% (Here by $\bar{\tt c} [\bar{\tt d}/\self]$ we mean $c_1[d1/\self], \ldots, c_n[dn/\self]$.)
% 
% Now when proving subject reduction for this case we take the type $S$ to be 
% $C(: \bar{\tt U}\ \bar{\tt f}; \self.\bar{\tt f} = \bar{\tt f}, \self=o)$.
% Note the addition of the $\self=o$ clause. 
% Note also that from
% 
% $\Gamma \vdash C(: \bar{\tt U}\ \bar{\tt f}; \self.\bar{\tt f} = \bar{\tt f}) o$
% 
% we can derive
% 
% $\Gamma \vdash C(: \bar{\tt U}\ \bar{\tt f}; \self.\bar{\tt f} = \bar{\tt f}, \self=o) o$
% 
% This of course makes complete sense ... all we need to show is that {\em this
% particular object} $o$ is of the type that it has been cast to.
% 
% But we have just checked this condition, i.e. 
% $type(o) \subtype D and \vdash_{\cal C} d[o/\self]$. 
% So we are done.
% 
% Note about the proof:
% the trick of adding $\self=o$ is critical. There is no hope of showing
% that $C(: \bar{\tt U}\ \bar{\tt f}; \self.\bar{\tt f} = \bar{\tt f}) \subtype D(:d)$. 
% All we have checked is the one case that {\em this one object\/} $o$ 
% satisfies the condition $d$, not that *all* objects 
% that satisfy $C(: \bar{\tt U}\ \bar{\tt f}; \self.\bar{\tt f} = \bar{\tt f})$ 
% also satisfy $D(:d)$. 
% So we take advantage of our ability to choose the type $S$ for $o$ 
% to get this done.
%\end{proof}

% \begin{lemma}
% \label{lemmaseven} % 7. 
% If   $\Gamma, \bar{\tt T} \bar{\tt f} \vdash \bar{\tt T} \subtype \bar{\tt Z}$
% and  $\theta = [\bar{\tt f} / \this.\bar{\tt f}]$,
% and  $\fields(C,\theta) = \bar{\tt Z} \bar{\tt f}$,
% and  $\fields(T_0,z_0) = \bar{\tt U}\ \bar{\tt f}$,
% and  $T_0 equiv C(\bar{\tt T} \bar{\tt f}; \self.\bar{\tt f} = \bar{\tt f})$,
% then $\Gamma \vdash T_i \subtype (T_0 z_0; z_0.f_i = \self; U_i)$
% \end{lemma}
% 
% \begin{proof}
% From these we can conclude
%    $\Gamma \vdash T_0 new C(bar e)$
% where $T_0$ is $C(: \bar{\tt T} \bar{\tt p}; \self.\bar{\tt f} = \bar{\tt p})$.
% 
% Now the case we are concerned about is $new C(\bar{\tt e}).f_i --> e_i$.
% 
% We want to show that the static type of $e_i$, namely $T_i$, 
% is a subtype of the static
% type of $new C(\bar{\tt e}).f_i$, which is
% $(T_0 z_0; z_0.f_i=\self, V_i[z_0.\bar{\tt f}/\this.\bar{\tt f}])$
% 
% Let $x$ be an arbitrary element of $T_i$. 
% We wish to show that $x$ is an element of
% the type $(T_0 z_0; z_0.f_i=\self, V_i[z_0.\bar{\tt f}/\this.\bar{\tt f}])$.
% To do this we have
% to show that it is possible to construct from $x$ an object $z_0$ of type 
% $T_0$ such
% that $x$ is the $f_i$'th field of $z_0$. But this is given to us by (4).
% Let $\bar{\tt t}$ be
% a set of values of type $bar T$, with $x$ chosen at index $i$.
% Then per (4), $T_i$ is a
% subtype of $V_i[\bar{\tt t}/\this.\bar{\tt f}]$. Since $x$ is a value of the type $T_i$, 
% it is a value of the type $V_i[\bar{\tt t}/\this.\bar{\tt f}]$. 
% Therefore we have constructed the object $z_0$ 
% which is required to show that $x$ is an element of 
% $(T_0 z_0; z_0.f_i=\self, V_i[z_0.\bar{\tt f}/\this.\bar{\tt f}])$.
% 
% I have implicitly used here the following genericity property of constraint
% systems (see Vijay Saraswat's LICS 91 paper):
% If $\Gamma \vdash A$ then $\Gamma[\bar{\tt t}/\bar{\tt y}] \vdash A [\bar{\tt t}/\bar{\tt y}]$.
% That is the set of axioms of the constraint system may contain free
% variables but are assumed to be closed under instantiation.
% \end{proof}


\begin{theorem}[Subject Reduction] 
\label{preservation}
If $\Gamma \vdash T\ e$ and $e \derives e'$, then for some type $S$,
$\Gamma \vdash S\ e'$ and $\Gamma \vdash S \subtype T$.
\end{theorem}

\begin{proof}
We proceed by induction on the
structure of the derivation of $\Gamma \vdash T\ e$.  We now have five
cases depending on the last rule used in the derivation
of $\Gamma \vdash T\ e$.
\begin{itemize}
\item
\TVar: The expression cannot take a step, so the conclusion is immediate.
\item
\TCast: We have two subcases.
   \begin{itemize}
   \item
   \RCast:  For the expression {\tt (T) o}, where 
            $o = {\tt \new\ {\tt C(\bar{\tt d})}}$,
            we have from \TCast\ that 
            $\Gamma \vdash {\tt C(:\bar{U}\ \bar{\tt f}
                {\tt ;\self.\bar{f}}=\bar{\tt f})\ o}$.
            Additionally, we have from \RCast\ that 
            $\vdash C \subtype T[o/\self]$.
            We now choose 
            $S = {\tt C(:\bar{U}\ \bar{\tt f}
                {\tt ;\self.\bar{f}}=\bar{\tt f}; \self=o)}$.
            From 
            $\Gamma \vdash {\tt C(:\bar{U}\ \bar{\tt f}
                {\tt ;\self.\bar{f}}=\bar{\tt f}) o}$ 
            we get
            $\Gamma \vdash S\ o$.
            From Lemma~\ref{weakening} and $\vdash C \subtype T[o/\self]$ 
            we get 
            $\Gamma \vdash C \subtype T[o/\self]$.
            From $\Gamma \vdash C \subtype T[o/\self]$ we get
            $\Gamma \vdash 
                S \subtype T$.
   \item
   \RCCast: For the expression {\tt (T) e}, we have from \TCast\ that
            $\Gamma \vdash {\tt U}\ {\tt e}$.
            Additionally, we have from \RCCast\ that
            ${\tt e} \derives {{\tt e}}'$.
            From the induction hypothesis, we have ${\tt U}'$ such that
            $\Gamma \vdash {\tt U}'\ e'$ and $\Gamma \vdash U' \subtype U$.
            We now choose $S=T$.
            From $\Gamma \vdash {\tt U}'\ e'$ and \TCast\ we derive
            $\Gamma \vdash {\tt S}\ {\tt (T) e'}$.
            From $S=T$ and Lemma~\ref{subtyping}
            we have $\Gamma \vdash S \subtype T$.
   \end{itemize}
\item
\TNew: We have a single case.
   \begin{itemize}
   \item
   \RCNewArg: For the expression ${\tt new\ C(\bar{e})}$,
            we have from \TNew\ that
            $\Gamma \vdash \bar{\tt T}\ \bar{\tt e}$,
            $\theta=[\bar{\tt f}/\this.\bar{\tt f}]$,
            $\fields(C,\theta)=\bar{\tt Z}\ \bar{\tt f}$,
            $\Gamma, \bar{\tt T}\ \bar{\tt f} \vdash 
                    \bar{\tt T} \subtype \bar{\tt Z}$, and 
            $\sigma(\Gamma, \bar{\tt T}\ \bar{\tt f}) \vdash_{\cal C} 
                    \inv({\tt C},\theta)$.
            Additionally, we have from \RCNewArg\ that
            ${\tt e}_i \derives {{\tt e}}_i'$.
            From the induction hypothesis, we have ${\tt S}_i$ such that
            $\Gamma \vdash {\tt S}_i\ {{\tt e}}_i'$ and 
            $\Gamma \vdash S_i \subtype T_i$.
            For all $j$ except $i$, define $S_j = T_j$ and $e_j' = e_j$.
            We have 
            $\Gamma \vdash \bar{\tt S}\ \bar{\tt e}'$ and
            $\Gamma \vdash \bar{S} \subtype \bar{T}$.
            From Lemma~\ref{lemmathree},
            $\Gamma, \bar{\tt T}\ \bar{\tt f} \vdash
                    \bar{\tt T} \subtype \bar{\tt Z}$,
            and $\Gamma \vdash \bar{S} \subtype \bar{T}$, we have
            $\Gamma, \bar{\tt S}\ \bar{\tt f} \vdash
                    \bar{\tt T} \subtype \bar{\tt Z}$.
            From Lemma~\ref{weakening} and 
            $\Gamma \vdash \bar{S} \subtype \bar{T}$, 
            we have 
            $\Gamma, \bar{\tt S}\ \bar{\tt f} \vdash \bar{S} \subtype \bar{T}$.
            From Lemma~\ref{subtyping},
            $\Gamma, \bar{\tt S}\ \bar{\tt f} \vdash \bar{S} \subtype \bar{T}$,
            and
            $\Gamma, \bar{\tt S}\ \bar{\tt f} \vdash
                    \bar{\tt T} \subtype \bar{\tt Z}$, we have
            $\Gamma, \bar{\tt S}\ \bar{\tt f} \vdash \bar{S} \subtype \bar{Z}$.
            From Lemma~\ref{lemmafour}, 
            $\Gamma \vdash \bar{S} \subtype \bar{T}$, and
            $\sigma(\Gamma, \bar{\tt T}\ \bar{\tt f}) \vdash_{\cal C}
                    \inv({\tt C},\theta)$, 
            we have
            $\sigma(\Gamma, \bar{\tt S}\ \bar{\tt f}) \vdash_{\cal C}
                    \inv({\tt C},\theta)$.
            We now choose 
               $S=C(:\bar{S}\ \bar{\tt f}{\tt ;\self.\bar{f}}=\bar{\tt f})$.
            From 
            $\Gamma \vdash \bar{\tt S}\ \bar{\tt e}$,
            $\theta=[\bar{\tt f}/\this.\bar{\tt f}]$,
            $\fields(C,\theta)=\bar{\tt Z}\ \bar{\tt f}$,
            $\Gamma, \bar{\tt S}\ \bar{\tt f} \vdash
                    \bar{\tt S} \subtype \bar{\tt Z}$,
            $\sigma(\Gamma, \bar{\tt S}\ \bar{\tt f}) \vdash_{\cal C}
                    \inv({\tt C},\theta)$, and \TNew\ we derive
            $\Gamma \vdash {\tt S}\ {\tt new\ C(\bar{e}'}$.
            We have 
               $T=C(:\bar{T}\ \bar{\tt f}{\tt ;\self.\bar{f}}=\bar{\tt f})$.
            From Lemma~\ref{lemmafive} and
            $\Gamma \vdash \bar{S} \subtype \bar{T}$, we have
            $\Gamma \vdash S \subtype T$.
   \end{itemize}
\item
\TField: We have two subcases.
   \begin{itemize}
   \item
   \RField:  For the expression ${\tt (new C(\bar{e})).f_i}$, 
             we have from \TField\ that
             $\Gamma \vdash {\tt T}_0\ {\tt e}$ and
             $\fields({\tt T}_0,{\tt z}_0) = \bar{\tt U}\ \bar{\tt f}$, where
             ${\tt z}_0$ is fresh.
             Additionally, we have from \RField\ that
             $\fields(C)=\bar{C}\ \bar{f}$.
             We have 
             $T_0 = C(:\bar{T}\ \bar{\tt f}{\tt ;\self.\bar{f}}=\bar{\tt f})$
             and, from \TNew, 
             $\Gamma \vdash \bar{\tt T}\ \bar{\tt e}$,
             $\theta=[\bar{\tt f}/\this.\bar{\tt f}]$,
             $\fields(C,\theta)=\bar{\tt Z}\ \bar{\tt f}$,
             $\Gamma, \bar{\tt T}\ \bar{\tt f} \vdash 
                   \bar{\tt T} \subtype \bar{\tt Z}$, and
             $\sigma(\Gamma, \bar{\tt T}\ \bar{\tt f}) \vdash_{\cal C} 
                   \inv({\tt C},\theta)$.
             From $\Gamma \vdash \bar{\tt T}\ \bar{\tt e}$, we have
             $\Gamma \vdash {\tt T}_i\ {\tt e}_i$.
             We now choose $S = {\tt T}_i$.
             %%%% XXX (missing step)
             Finally it is straightforward to show
             $\Gamma \vdash S \subtype (T_0 z_0; z_0.f_i = \self; U_i)$.
   \item
   \RCField: For the expression ${\tt e.f}_i$, we have from \TField\ that
             $\Gamma \vdash {\tt T}_0\ {\tt e}$ and
             $\fields({\tt T}_0,{\tt z}_0)= \bar{\tt U}\ \bar{\tt f}_i$
             where ${\tt z}_0$ is fresh.
             Additionally, we have from \RCField\ that  
             ${\tt e} \derives {{\tt e}}'$.
             From the induction hypothesis, we have $S_0$ such that 
             $\Gamma \vdash S_0\ e'$ and $\Gamma \vdash S_0 \subtype T_0$.
             We now choose 
             $S = 
               ({\tt S}_0\ {\tt z}_0; {\tt z}_0.{\tt f}_i=\self;{\tt U}_i)$.
             From $\Gamma \vdash S_0\ e'$, Lemma~\ref{lemmaone}, and
             \TField, we derive
             $\Gamma \vdash S {{\tt e}}'$.
             From $\Gamma \vdash S_0 \subtype T_0$ and 
             Lemma~\ref{lemmafive}, we have $\Gamma \vdash S \subtype T$.
   \end{itemize}
\item
\TInvk: We have three subcases.
   \begin{itemize}
   \item
   \RInvk:  For the expression 
            $(\new\ {\tt C}(\bar{\tt e})).{\tt m}(\bar{\tt d})$
            we have from $\TInvk$ that
            $\Gamma \vdash {\tt T}_0 \ (\new\ {\tt C}(\bar{\tt e}))$,
            $\Gamma \vdash {\tt T}_{1:n} \ {\tt d}_{1:n}$,
            $\mtype({\tt T}_0,{\tt m},{\tt z}_0) = 
               \tt {\tt Z}_{1:n}\ {\tt z}_{1:n}:c \rightarrow {\tt U}$,
            $\Gamma, {\tt T}_{0:n}\ {\tt z}_{0:n} \vdash 
                  {\tt T}_{1:n} \subtype {\tt Z}_{1:n}$, and
            $\sigma(\Gamma, {\tt T}_{0:n}\ {\tt z}_{0:n}) \vdash_{\cal C}                          {\tt c}$, 
            where ${\tt z}_{0:n}$ is fresh, and
            $T_0 = C(\bar{A} \bar{f}; self.\bar{f} = \bar{f})$ and
            $\Gamma \vdash \bar{A} \bar{e}$.
            For simplicity, define $d_0 = (\new\ {\tt C}(\bar{\tt e}))$.
            Additionally, we have from \RInvk\ that
            ${\mathit{mbody}({\tt m},{\tt C})=\bar{x}. {\tt e}_0}$.
            From Lemma~\ref{body-type}, 
            $\mtype({\tt T}_0,{\tt m},{\tt z}_0) =
               \tt {\tt Z}_{1:n}\ {\tt z}_{1:n}:c \rightarrow {\tt U}$, and
            ${\mathit{mbody}({\tt m},{\tt C})=\bar{x}. {\tt e}_0}$,
            we have $D,V$ such that $C \subtype D$, $V \subtype U$, and
            ${\tt Z}_{1:n}\ \bar{\tt z}_{1:n},{\tt D}\ \this \vdash 
                  {\tt V}\ {\tt e}$.
            From Lemma~\ref{subtyping}, Lemma~\ref{weakening},
            $T_0 = C(\bar{A} \bar{f}; self.\bar{f} = \bar{f})$, 
            $\Gamma \vdash T_0 \subtype C$ and
            $C \subtype D$, 
            we have
            $\Gamma, {\tt T}_{0:n}\ {\tt z}_{0:n} \vdash T_0 \subtype D$.
            From Lemma~\ref{weakening} and
            ${\tt Z}_{1:n}\ \bar{\tt z}_{1:n},{\tt D}\ \this \vdash 
                  {\tt V}\ {\tt e}_0$, 
            we have
            $\Gamma, {\tt Z}_{1:n}\ \bar{\tt z}_{1:n},{\tt D}\ \this \vdash 
                  {\tt V}\ {\tt e}$.
            From Lemma~\ref{substitution}, 
            $\Gamma \vdash {\tt T}_{0:n} \ {\tt d}_{0:n}$,
            $\Gamma,{\tt T}_{0:n}\ {\tt z}_{0:n} \vdash T_0 \subtype D$.
            $\Gamma, {\tt T}_{0:n}\ {\tt z}_{0:n} \vdash
                  {\tt T}_{1:n} \subtype {\tt Z}_{1:n}$,
            $\Gamma, {\tt Z}_{1:n}\ \bar{\tt z}_{1:n},{\tt D}\ \this \vdash 
                    {\tt V}\ {\tt e}_0$,
            we have a type $S$ such that 
            $\Gamma \vdash {\tt S}\ 
                 {\tt e}_0[\bar{\tt d}/{\tt z}_{0:n},\new\ C(\bar{e})/\this]$ 
            and
            $\Gamma \vdash {\tt S} \subtype \bar{\tt T}\ {\tt z}_{0:n};{\tt V}$.

   \item
   \RCInvkRecv: For the expression ${\tt e_0.m(\bar{e})}$,
            we have from \TInvk\ that
            $\Gamma \vdash {\tt T}_{0:n} \ {\tt e}_{0:n}$,
            $\mtype({\tt T}_0,{\tt m},{\tt z}_0) = 
               \tt {\tt Z}_{1:n}\ {\tt z}_{1:n}:c \rightarrow {\tt U}$,
            $\Gamma, {\tt T}_{0:n}\ {\tt z}_{0:n} \vdash 
                  {\tt T}_{1:n} \subtype {\tt Z}_{1:n}$, and
            $\sigma(\Gamma, {\tt T}_{0:n}\ {\tt z}_{0:n}) \vdash_{\cal C}                          {\tt c}$, 
            where ${\tt z}_{0:n}$ is fresh.
            Additionally, we have from \RCInvkRecv\ that
            ${\tt e}_0 \derives {{\tt e}}_0'$.
            From the induction hypothesis, we have ${\tt S}_0$ such that
            $\Gamma \vdash {\tt S}_0\ {{\tt e}}_0'$ and 
            $\Gamma \vdash S_0 \subtype T_0$.
            For all $j>0$, define $S_j = T_j$ and $e_j' = e_j$.
            We have 
            $\Gamma \vdash \bar{\tt S}\ \bar{\tt e}'$ and
            $\Gamma \vdash \bar{S} \subtype \bar{T}$.
            From Lemma~\ref{lemmatwo}
            and $\Gamma \vdash \bar{S} \subtype \bar{T}$, we have
            $\mtype(S_0,m,z) = \mtype(T_0,m,z)$.
            From Lemma~\ref{lemmathree},
            $\Gamma, {\tt T}_{0:n}\ {\tt z}_{0:n} \vdash
                  {\tt T}_{1:n} \subtype {\tt Z}_{1:n}$,
            and $\Gamma \vdash \bar{S} \subtype \bar{T}$, we have
            $\Gamma, {\tt S}_{0:n}\ {\tt z}_{0:n} \vdash
                  {\tt T}_{1:n} \subtype {\tt Z}_{1:n}$,
            From Lemma~\ref{lemmafour}, 
            $\Gamma \vdash \bar{S} \subtype \bar{T}$, and
            $\sigma(\Gamma, {\tt T}_{0:n}\ {\tt z}_{0:n}) \vdash_{\cal C}
                              {\tt c}$
            we have
            $\sigma(\Gamma, {\tt S}_{0:n}\ {\tt z}_{0:n}) \vdash_{\cal C}
                              {\tt c}$
            We now choose 
               $S=({\tt S}_{0:n}\ {\tt z}_{0:n}; U)$.
            From 
            $\Gamma \vdash {\tt S}_{0:n} \ {\tt e}_{0:n}'$,
            $\mtype({\tt S}_0,{\tt m},{\tt z}_0) =
               \tt {\tt Z}_{1:n}\ {\tt z}_{1:n}:c \rightarrow {\tt U}$,
            $\Gamma, {\tt S}_{0:n}\ {\tt z}_{0:n} \vdash
                  {\tt T}_{1:n} \subtype {\tt Z}_{1:n}$, and
            $\sigma(\Gamma, {\tt S}_{0:n}\ {\tt z}_{0:n}) \vdash_{\cal C}
                  {\tt c}$,
            and \TInvk\ we derive
            $\Gamma \vdash {\tt S}\ {\tt e_0.m(e_{1:n}')}$.
            We have 
               $T=({\tt T}_{0:n}\ {\tt z}_{0:n}; U)$.
            From Lemma~\ref{lemmafive} and
            $\Gamma \vdash \bar{S} \subtype \bar{T}$, we have
            $\Gamma \vdash S \subtype T$.
   \item
   \RCInvkArg: For the expression ${\tt e_0.m(\bar{e})}$,
            we have from \TInvk\ that
            $\Gamma \vdash {\tt T}_{0:n} \ {\tt e}_{0:n}$,
            $\mtype({\tt T}_0,{\tt m},{\tt z}_0) = 
               \tt {\tt Z}_{1:n}\ {\tt z}_{1:n}:c \rightarrow {\tt U}$,
            $\Gamma, {\tt T}_{0:n}\ {\tt z}_{0:n} \vdash 
                  {\tt T}_{1:n} \subtype {\tt Z}_{1:n}$, and
            $\sigma(\Gamma, {\tt T}_{0:n}\ {\tt z}_{0:n}) \vdash_{\cal C}                          {\tt c}$, 
            where ${\tt z}_{0:n}$ is fresh.
            Additionally, we have from \RCInvkArg\ that, for $i>0$,
            ${\tt e}_i \derives {{\tt e}}_i'$.
            From the induction hypothesis, we have ${\tt S}_i$ such that
            $\Gamma \vdash {\tt S}_i\ {{\tt e}}_i'$ and 
            $\Gamma \vdash S_i \subtype T_i$.
            For all $j$ except $i$, define $S_j = T_j$ and $e_j' = e_j$.
            We have 
            $\Gamma \vdash \bar{\tt S}\ \bar{\tt e}'$ and
            $\Gamma \vdash \bar{S} \subtype \bar{T}$.
            From Lemma~\ref{lemmathree},
            $\Gamma, {\tt T}_{0:n}\ {\tt z}_{0:n} \vdash
                  {\tt T}_{1:n} \subtype {\tt Z}_{1:n}$,
            and $\Gamma \vdash \bar{S} \subtype \bar{T}$, we have
            $\Gamma, {\tt S}_{0:n}\ {\tt z}_{0:n} \vdash
                  {\tt T}_{1:n} \subtype {\tt Z}_{1:n}$,
            From Lemma~\ref{weakening} and 
            $\Gamma \vdash \bar{S} \subtype \bar{T}$, 
            we have 
            $\Gamma, {\tt S}_{0:n}\ {\tt z}_{0:n} \vdash 
                  \bar{S} \subtype \bar{T}$.
            From Lemma~\ref{subtyping},
            $\Gamma, {\tt S}_{0:n}\ {\tt z}_{0:n} \vdash 
                  \bar{S} \subtype \bar{T}$,
            and
            $\Gamma, {\tt S}_{0:n}\ {\tt z}_{0:n} \vdash
                    {\tt T}_{1:n} \subtype {\tt Z}_{1:n}$, 
            we have
            $\Gamma, {\tt S}_{0:n}\ {\tt z}_{0:n} \vdash 
                  \bar{S} \subtype {\tt Z}_{1:n}$.
            From Lemma~\ref{lemmafour}, 
            $\Gamma \vdash \bar{S} \subtype \bar{T}$, and
            $\sigma(\Gamma, {\tt T}_{0:n}\ {\tt z}_{0:n}) \vdash_{\cal C}
                              {\tt c}$
            we have
            $\sigma(\Gamma, {\tt S}_{0:n}\ {\tt z}_{0:n}) \vdash_{\cal C}
                              {\tt c}$
            We now choose 
               $S=({\tt S}_{0:n}\ {\tt z}_{0:n}; U)$.
            From 
            $\Gamma \vdash {\tt S}_{0:n} \ {\tt e}_{0:n}'$,
            $\mtype({\tt S}_0,{\tt m},{\tt z}_0) =
               \tt {\tt Z}_{1:n}\ {\tt z}_{1:n}:c \rightarrow {\tt U}$,
            $\Gamma, {\tt S}_{0:n}\ {\tt z}_{0:n} \vdash
                  {\tt S}_{1:n} \subtype {\tt Z}_{1:n}$, and
            $\sigma(\Gamma, {\tt S}_{0:n}\ {\tt z}_{0:n}) \vdash_{\cal C}
                  {\tt c}$,
            and \TInvk\ we derive
            $\Gamma \vdash {\tt S}\ {\tt e_0.m(e_{1:n}')}$.
            We have 
               $T=({\tt T}_{0:n}\ {\tt z}_{0:n}; U)$.
            From Lemma~\ref{lemmafive} and
            $\Gamma \vdash \bar{S} \subtype \bar{T}$, we have
            $\Gamma \vdash S \subtype T$.
   \end{itemize}
\end{itemize}
\end{proof}

\noindent
Let the normal form of expressions be given by {\em values}
{\tt v} {::=} $\new\ {\tt C(\bar{\tt v})}$.

\begin{theorem}[Progress] 
\label{progress}
If $\vdash {\tt T\ e}$, then one of the following conditions holds:
\begin{enumerate}
\item {\tt e} is a value {\tt v}, 
\item {\tt e} contains a subexpression ${\tt (T)\new\ C(\bar{\tt
v})}$ such that
$\not\vdash {\tt C} \subtype {\tt T}[{\tt \new\ C(\bar{\tt v})}/\self]$,
\item there exists ${\tt e}'$ s.t. ${\tt e} \derives {\tt e}'$.
\end{enumerate}
\end{theorem}

\begin{proof}
The proof has a structure that is similar to the proof of Subject Reduction;
we omit the details.
\end{proof}

\begin{theorem}[Type Soundness] 
\label{type-soundness}
If $\vdash {\tt T\ e}$ and ${\tt e} \starderives {\tt e}'$, with ${\tt
e}'$ in normal form, then ${\tt e}'$ is either (1)~a value {\tt v}
with $\vdash {\tt S\ v}$ and $\vdash {\tt S
\subtype T}$, for some type {\tt S}, or, (2)~ an expression containing
a subexpression ${\tt (T)\new\ {\tt C(\bar{\tt v})}}$ where 
$\not\vdash \tt C\subtype T[\new\ C(\bar{\tt v})/\self]$.

\end{theorem}

\begin{proof}
Combine Theorem~\ref{preservation} and Theorem~\ref{progress}.
\end{proof}

