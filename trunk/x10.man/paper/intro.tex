
%
%I. Introduction and overview (3 pages)

%Design of X10, concurrency, high productivity, high performance,
%practical language.
%
%Dependent types arise naturally. arrays, regions, distributions, place
%types.
%
%Indeed you can look around and recognize many OO type systems proposed
%in the last decade or so as specific kinds of applied dependent type
%systems.
%
%Our goal is to develop a general framework for dependent types for
%statically typed OO languages ("Java-like languages"). 

\Xten{} is a modern statically typed object-oriented (OO) language
designed for high productivity in the high performance computing (HPC)
domain~\cite{X10}. Built essentially on the imperative sequential
$\mbox{Java}^{\mbox{\scriptsize\sc tm}}$ core, \Xten{} introduces constructs for
distribution and fine-grained concurrency (asynchrony, atomicity,
ordering).

\Xten{}, like most OO languages is designed around
the notion of objects as instances of {\em classes}. However, \Xten{}
places equal emphasis on {\em arrays}, a central data-structure in
high performance computing. In particular, \Xten{} supports dense,
distributed multi-dimensional arrays of value and reference types,
built over index sets known as {\em regions}, and mappings from index
sets to places, known as {\em distributions}.  \Xten{} supports a rich
algebra of operations over regions, distributions and arrays.

A central design goal of \Xten{} is to rule out large classes of error
by design. For instance, the possibility of indexing a 2-d array with 3-d
points should simply be ruled out at compile-time. This means that one
must permit the programmer to express types such as {\tt
region(2)}, the type of all two-dimensional regions, \xcd{int[5]}, the
type of all arrays of \xcd{int} of length \xcd{5}, {\tt
int[region(2)]}, the type of all \xcd{int} arrays over two dimensional
regions, and since \Xten{} supports distributed computation, \xcd{Obj!}, the type of all \xcd{Obj} located on the
current node. For concurrent computations, one needs the ability to
statically check that a method is being invoked by an activity that is
registered with a given clock (i.e., dynamic barrier)~\cite{X10}.

In this paper we describe {\Xten}'s support for {\em
constrained types}, user-defined {\em dependent types}
on top of predicates over the {\em immutable}
state of objects. Such types statically capture many common invariants
that naturally arise in code. For instance, typically the shape of an
array (the number of dimensions (the rank) and the size of each dimension)
is determined at
run time, but is fixed once the array is constructed. Thus, the shape of an
array is part of its immutable state.

Constrained types are instances of {\em dependent
types}, types parametrized by {\em values}
\cite{dependent-types}.
Dependent types have a rich and distinguished history in
logic and functional programming languages (see e.g.,
\cite{xi99dependent,ocrz-ecoop03,aspinall-attapl,cayenne,epigram-matter})
and in the development of logical frameworks
\cite{calc-constructions}. 

\Xten{} provides a framework for specifying and checking constrained types
that achieves certain desirable properties:
\begin{itemize}
\item 
{\bf Ease of use.}  The framework must be easy to use for practicing
programmers. In particular, since \Xten{} is an extension of Java,
the syntax of types is a simple and
natural extension of Java's types.

\item
{\bf Flexibility.}
The framework
permits the development of concrete,
specific type systems tailored to the application area at
hand.  \Xten{}'s compiler permits extension with different constraint systems
via compiler plugins, enabling a kind of pluggable type system~\cite{bracha04-pluggable}.
The framework is parametric in the kinds of
expressions used in the type system, permitting the installed constraint
system to interpret the constraints.

\item
{\bf Modularity.}
The rules for type-checking
are specified once in a way that is independent of the
particular vocabulary of operations used in the dependent type
system.
The type system supports separate compilation.

\item
{\bf Integration with OO languages.}
The framework  works smoothly with Java's nominal type system.
It permits separate compilation.

\item
{\bf Static checking.}  The framework permits mostly static
type-checking. The user is able to escape the confines of
static type-checking using dynamic casts, as is common for Java-like
languages.
\end{itemize}

\subsection{Constrained types}

Our basic approach to introducing constrained types into \Xten{}
is to
follow the spirit of generic types, but to use values instead of
types.

We permit the definition of a class \xcd{C} to specify
a list of typed parameters, or {\em properties},
{\tt (T1 x1, \ldots, Tk xk)} similar in syntactic structure to
a method formal parameter list.
%
Each property in this list is treated as  a \xcd{public} {\tt
final} instance field.
%
We also permit the
specification of a {\em class invariant}, a
{\em where clause}~\cite{where-clauses}
in the class definition. A class invariant
is a boolean expression on the properties of the class.
The compiler ensures that all
instances of the class created at run time satisfy the
invariant.
%
The class invariant is separated from the
property list with a ``\xcd{:}''. 
%
For instance, we may specify a class \xcd{List} with an
\xcd{int length} property as follows:
\begin{displayxten}
    class List(int length: length >= 0) {...}
\end{displayxten}
Given such a definition for a class \xcd{C}, types can be
constructed by {\em constraining} the properties of \xcd{C}.  In
principle, {\em any} boolean expression over the properties
specifies a type: the type of all instances of the class
satisfying the boolean expression. Thus, \xcd{List(:length ==
3)}
is a permissible type, as are \xcd{List(:length <= 41)} and
even \xcd{List(:length * f() >= g())}.
In practice, \xcd{e} is restricted by the particular constraint
system in use.

Accordingly, a {\em constrained type} is of the form {\tt
C(:e)}, the name of a class or interface \xcd{C}, called the
{\em base class}, followed by a {\em condition} \xcd{e},
a boolean expression on the properties of the
base class and the \xcd{final} variables in scope at the type.
Constraints specify (possibly) partial information about the
variables of interest.
Such a type represents a refinement of \xcd{C}: the set of all
instances of \xcd{C} whose immutable state satisfies the
condition
\xcd{c}.

For brevity, we write \xcd{C} as a type as well; it
corresponds to the (vacuously) constrained type \xcd{C(:true)}.
We also permit the syntax {\tt C(t1,\ldots, tk)} for
the type {\tt C (:x1 = t1, \ldots, xk = tk)} (assuming that
the property list for \xcd{C} specifies the \xcd{k} properties
{\tt
x1,\ldots, xk}, and each term \xcd{ti} is of the correct
type).

Thus, using the definitions above, \xcd{List(n)} is the type of
all lists of length \xcd{n}, shown in
Figure~\ref{fig:list-example}.
%
Intuitively, this definition states that a \xcd{List} has a \xcd{int}
property \xcd{n}, which must be non-negative.  The class has two
fields that hold the head and tail of the list.  The properties of a
class are set through the invocation of \xcd{property}{\tt (\ldots);}
(analogously to \xcd{super}{\tt (\ldots);}).  Constructors have ``return
types'' that can specify a where clause satisfied by the object being
returned by the constructor.  The compiler verifies that the
constructor postcondition and the class invariant are implied by the
\xcd{property} statement and any \xcd{super} calls in the constructor
body.
%
The \xcd{List} class has three constructors: the first
constructor returns a empty list; the second returns 
a singleton list of length \xcd{1}; the third
returns a list of length \xcd{m+1}, where \xcd{m} is
the length of the second argument.  If an argument appears in the
return type then the argument must be declared \xcd{final},
ensuring the
argument points to the same object throughout the evaluation of
the constructor body.


\begin{figure}
\begin{xten}
class List(int(:self >= 0) n) {
  Object head = null;
  List(n-1) tail = null;
  List(0)() { property(0); }
  List(1)(Object head) { this(head, new List());}
  List(tail.n+1)(Object head, List tail) {
    property(tail.n+1);
    this.head = head;
    this.tail = tail;
  }
  List(n+arg.n) append(final List arg) {
    return (n == 0) 
      ? arg : new List(head, tail.append(arg));
  }
  List(n) rev() { return rev(new List()); }
  List(n+acc.n) rev(final List acc) {
    return (n == 0) 
     ? acc : tail.rev(new List(head, acc));
  }
  List(:self.n <= this.n) filter(Predicate f) {
    if (n==0) return this;
    List(:self.n <=this.n-1) l = tail.filter(f);
    return (f.isTrue(head)) ?
     new List(head,l) : l;
  }
}
\end{xten}

\caption{
This program implements a mutable list of Objects. The size of a list
does not change through its lifetime, even though at different points
in time its head and tail might point to different structures.}
\label{fig:list-example}
\end{figure}

\subsection{Constraint plugins}


\subsection{Claims}

The paper presents the design of a constrained types framework
in \Xten{}.
THe framework supports
integrating
different constraint solvers into the language.
The design has been implemented
in \Xten{}.  The \Xten{} compiler, available at \xcd{x10.sf.net}
implements a simple
equality-based constraint system.  Constraint solver plugins
have been implemented for Presburger constraints using the Omega
library~\cite{omega} and the CVC3 theorem prover~\cite{cvc}.
The implementation is
discussed in Section~\ref{sec:impl}.

As in staged languages~\cite{nielson-multistage,ts97-multistage}, the
design distinguishes between compile-time and run-time
evaluation. Constrained types are checked (mostly) at compile-time.
The compiler uses a constraint solver to perform universal reasoning
(``for all possible values of method parameters'') for dependent
type-checking.  There is no run-time constraint-solving.  However,
run-time casts to dependent types are permitted; these casts involve
arithmetic, not algebra---the values of all parameters are known.

The design supports separate compilation: a class needs to be
recompiled only when it is modified or when the method
and field signatures or invariants of classes on which it
depends are modified.

We claim that the design is natural and easy to use and useful. Many
example programs have been written using dependent types and are
available at {\tt x10.sf.net}.

We claim that the design is flexible. It is parametric on the
constraint system being used. We are planning on extending the current
implementation to support multiple user-defined constraint systems,
thereby supporting pluggable types. Dependent where clauses are also
the basis for a general user-definable annotation framework we have
implemented separately~\cite{ns07-x10anno}. 

We claim the design is clean and modular. We present a simple core
language \CFJ, extending \FJ{}~\cite{FJ} with constrained types on top
of an arbitrary constraint system. We present rules for type-checking
\CFJ{} programs that are parametric in the constraint system. 
We establish subject reduction and progress theorems. 

%
% XXX contrast with hybrid type checking.

\paragraph{Rest of this paper.}

\ref{TODO} rewrite to agree with the rest of the paper

The next section reviews related work.
Section~\ref{sec:lang} fleshes out the syntactic and semantic details of the
proposal, and presents a formal semantics
and a soundness theorem.
Section~\ref{sec:examples} works through a number of
examples using a variety of constraint systems.
The implementation of constrained types in \Xten{} is described
in Section~\ref{sec:impl}.
Section~\ref{sec:future}
conclude the paper with a discussion of
future work.

