\clearextrapart{Introduction}
\subsection*{Background}
Bigger computational problems need bigger computers capable of
performing a larger number of operations per second. The era of
increasing performance by simply increasing clocking frequency now
seems to be behind us. It is becoming increasingly difficult
to mange chip power and heat.  Instead, computer
designers are starting to look at {\em scale out} systems in which the
system's computational capacity is increased by adding additional
nodes of comparable power to existing nodes, and connecting nodes with
a high-speed communication network.

A central problem with scale out systems is a definition of the {\em
memory model}, that is, a model of the interaction between shared
memory and  simultaneous (read, write) operations on that
memory by multiple processors. The traditional ``one operation at a
time, to completion'' model that underlies Lamport's notion of {\em
sequential consistency} (SC) proves too expensive to implement in
hardware, at scale. Various models of {\em relaxed consistency} have
proven too difficult for programmers to work with.  

One response to this problem has been to move to a {\em fragmented
memory model}. Multiple processors are made to interact via a
relatively language-neutral message-passing format such as MPI
\cite{mpi}. This model has enjoyed some success: several
high-performance applications have been written in this
style. Unfortunately, this model leads to a {\em loss of programmer
productivity}: the mesage-passing format is integrated into the host
language by means of an application-programming interface (API), the
programmer must explicitly represent and manage the interaction
between multiple processes and choreograph their data exchange; large
data-structures (such as distributed arrays, graphs, hash-tables) that
are conceptually unitary must be thought of as fragmented across
different nodes; all processors must generally execute the same code
(in an SPMD fashion) etc.

One response to this problem has been the advent of the {\em
partitioned global address space} (GAS) model underlying languages
such as UPC, Titanium and Co-Array Fortran \cite{pgas,titanium}. These
languages permit the programmer to think of a single computation
running across multiple processors, sharing a common address
space. All data resides at some processors, which is said to have {\em
affinity} to the data.  Each processor may operate directly on the
data it contains but must use some indirect mechanism to access or
update data at other processors. Some kind of global {\em barriers}
are used to ensure that processors remain roughly in lock-step.

\Xten{} is a modern object-oriented programming language
in the GAS family. The fundamental goal of \Xten{} is to enable
scalable, high-performance, high-productivity transformational
programming for high-end computers -- for traditional numerical
computation workloads (such as weather simulation, molecular dynamics,
particle transport problems etc) as well as commercial server
workloads.
%%(By transformational programming we mean computation
%%focused on accepting some input, transforming it in some way and
%%delivering some output. as opposed to computation focused on
%%maintaining an ongoing interaction with the environment, e.g.{} {\em
%%reactive computing}.) 
\Xten{} is based on state-of-the-art object-oriented
programming ideas primarily to take advantage of their proven
flexibility and ease-of-use for a wide spectrum of programming
problems. \Xten{} takes advantage of several years of research (e.g.{}
in the context of the Java$^{TM}$ Grande forum,
\cite{moreira00java,kava}) on how to adapt such languages to the context of
high-performance numerical computing. Thus \Xten{} provides support
for user-defined {\em value types} (such as {\tt int}, {\tt float},
{\tt complex} etc), supports a very
flexible form of multi-dimensional arrays (based on ideas in ZPL
\cite{zpl}) and supports IEEE-standard floating point arithmetic.
Some limited operator overloading is provided for a few ``built in''
classes in the {\cf x10.lang} package.  Future versions of the
language will support user-definable operator overloading.

{}\Xten{} introduces a flexible treatment of concurrency, distribution
and locality, within an integrated type system. \Xten{} extends the
GAS model to the {\em globally asynchronous, locally synchronous}
(GALS) model originally developed in hardware and embedded software
research.  {}\Xten{} introduces {\em places} as an abstraction for a
computational context with a locally synchronous view of shared
memory. An \Xten{} computation runs over a large collection of places.
Each place hosts some data and runs one or more {\em
activities}. Activities are extremely lightweight threads of
execution. An activity may synchronously (and {\em atomically}) use
one or more memory locations in the place in which it resides,
leveraging current symmetric multiprocessor (SMP) technology.  
To access or update memory at other places, it must 
spawn activities asynchronously (either explicitly or implicitly).
\Xten{} provides weaker ordering guarantees for
inter-place data access, enabling applications to scale.  {\em
Immutable} data needs no consistency management and may be freely
copied by the implementation between places.  One or more {\em clocks}
may be used to order activities running in multiple
places. (Multi-dimensional) Arrays may be distributed across multiple
places. Arrays support parallel collective operations. A novel
exception flow model ensures that exceptions thrown by asynchronous
activities can be caught at a suitable parent activity.  The type
system tracks which memory accesses are local. The programmer may
introduce placecasts which verify the access is local at runtime.
Linking with native code is supported.

{}\XtenCurrVer{} adds to Version~0.4 the following features. A
dependent type system is introduced: types may be formed by placing
constraints on the values of {\em properties} ({\tt final} instance
fields) of the type. In particular, reference types must be qualified
with a constraint specifying the place at which instances of the type
are located. Various behavior annotations are introduced {e.g.{} {\tt
local}, {\tt seq}) -- these specify behavioral properties of code (all
accesses to mutable data are to local data, no asynchronous activities
are spawned) and are verified by the compiler. Method inheritance is
required to respect these annotations.

We anticipate a revision of the language in the next several
months. The revision will fully define generics for \Xten{}, and
introduce general closures. We believe both these features will make
code more concise and readable.

{}\Xten{} is an experimental language.  Several representative
concurrent idioms have already found pleasant expression in \Xten. We
intend to develop several full-scale applications to get better
experience with the language, and revisit the design in the light of
this experience. Future versions of the language are expected to
support user-definable operators and permit the specification of
generic classes and methods.

Two papers have been published on \Xten{}. \cite{X10-concur05}
presents a formal semantics for the concurrency and distributed
aspects of \Xten{}. \cite{xten} presents some case studies, and
discusses some \Xten{} programming idioms.
