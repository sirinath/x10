\documentclass[10pt,a4paper]{article}

%% For typesetting theorems and some math symbols.
\usepackage{amssymb}
\usepackage{amsthm}

\usepackage{fullpage}

\title{Featherweight X10}

\author{}

\date{}




\usepackage{xspace}

% Macros for R^nRS.

\def\@makechapterhead#1{%
  \vspace*{50\p@}%
  {\parindent \z@ \raggedright \normalfont
    \ifnum \c@secnumdepth >\m@ne
        \huge\bfseries \thechapter \space\space\space
        \nobreak
    \fi
    \interlinepenalty\@M
    \Huge \bfseries #1\par\nobreak
    \vskip 40\p@
  }}


\makeatletter

\newcommand{\topnewpage}{\@topnewpage}

% Chapters, sections, etc.

\newcommand{\vest}{}
\newcommand{\dotsfoo}{$\ldots\,$}

\newcommand{\sharpfoo}[1]{{\tt\##1}}
\newcommand{\schfalse}{\sharpfoo{f}}
\newcommand{\schtrue}{\sharpfoo{t}}

\newcommand{\singlequote}{{\tt'}}  %\char19
\newcommand{\doublequote}{{\tt"}}
\newcommand{\backquote}{{\tt\char18}}
\newcommand{\backwhack}{{\tt\char`\\}}
\newcommand{\atsign}{{\tt\char`\@}}
\newcommand{\sharpsign}{{\tt\#}}
\newcommand{\verticalbar}{{\tt|}}

\newcommand{\coerce}{\discretionary{->}{}{->}}

% Knuth's \in sucks big boulders
\def\elem{\hbox{\raise.13ex\hbox{$\scriptstyle\in$}}}

\newcommand{\meta}[1]{{\noindent\hbox{\rm$\langle$#1$\rangle$}}}
\let\hyper=\meta
\newcommand{\hyperi}[1]{\hyper{#1$_1$}}
\newcommand{\hyperii}[1]{\hyper{#1$_2$}}
\newcommand{\hyperj}[1]{\hyper{#1$_i$}}
\newcommand{\hypern}[1]{\hyper{#1$_n$}}
\newcommand{\var}[1]{\noindent\hbox{\it{}#1\/}}  % Careful, is \/ always the right thing?
\newcommand{\vari}[1]{\var{#1$_1$}}
\newcommand{\varii}[1]{\var{#1$_2$}}
\newcommand{\variii}[1]{\var{#1$_3$}}
\newcommand{\variv}[1]{\var{#1$_4$}}
\newcommand{\varj}[1]{\var{#1$_j$}}
\newcommand{\varn}[1]{\var{#1$_n$}}

\newcommand{\vr}[1]{{\noindent\hbox{$#1$\/}}}  % Careful, is \/ always the right thing?
\newcommand{\vri}[1]{\vr{#1_1}}
\newcommand{\vrii}[1]{\vr{#1_2}}
\newcommand{\vriii}[1]{\vr{#1_3}}
\newcommand{\vriv}[1]{\vr{#1_4}}
\newcommand{\vrv}[1]{\vr{#1_5}}
\newcommand{\vrj}[1]{\vr{#1_j}}
\newcommand{\vrn}[1]{\vr{#1_n}}


\newcommand{\defining}[1]{\mainindex{#1}{\em #1}}
\newcommand{\ide}[1]{{\schindex{#1}\frenchspacing\tt{#1}}}

\newcommand{\lambdaexp}{{\cf lambda} expression}
\newcommand{\Lambdaexp}{{\cf Lambda} expression}
\newcommand{\callcc}{{\tt call-with-current-continuation}}

% \reallyindex{SORTKEY}{HEADCS}{TYPE}
% writes (index-entry "SORTKEY" "HEADCS" TYPE PAGENUMBER)
% which becomes  \item \HEADCS{SORTKEY} mainpagenumber ; auxpagenumber ...

\global\def\reallyindex#1#2#3{%
\write\@indexfile{"#1" "#2" #3 \thepage}}

\newcommand{\mainschindex}[1]{\label{#1}\reallyindex{#1}{tt}{main}}
\newcommand{\mainindex}[1]{\reallyindex{#1@{\rm #1}{main}}}
\newcommand{\schindex}[1]{\reallyindex{#1}{tt}{aux}}
\newcommand{\sharpindex}[1]{\reallyindex{#1}{sharpfoo}{aux}}
%vj%\renewcommand{\index}[1]{\reallyindex{#1}{rm}{aux}}

\newcommand{\domain}[1]{#1}
\newcommand{\nodomain}[1]{}
%\newcommand{\todo}[1]{{\rm$[\![$!!~#1$]\!]$}}
\newcommand{\todo}[1]{}

% \frobq will make quote and backquote look nicer.
\def\frobqcats{%\catcode`\'=13 %\catcode`\{=13{}\catcode`\}=13{}
\catcode`\`=13{}}
{\frobqcats
\gdef\frobqdefs{%\def'{\singlequote}
\def`{\backquote}}}%\def\{{\char`\{}\def\}{\char`\}}
\def\frobq{\frobqcats\frobqdefs}

% \cf = code font
% Unfortunately, \cf \cf won't work at all, so don't even attempt to
% next constructions which use them...
\newcommand{\cf}{\frenchspacing\tt}

% Same as \obeycr, but doesn't do a \@gobblecr.
{\catcode`\^^M=13 \gdef\myobeycr{\catcode`\^^M=13 \def^^M{\\}}%
\gdef\restorecr{\catcode`\^^M=5 }}

{\catcode`\^^I=13 \gdef\obeytabs{\catcode`\^^I=13 \def^^I{\hbox{\hskip 4em}}}}

{\obeyspaces\gdef {\hbox{\hskip0.5em}}}

\gdef\gobblecr{\@gobblecr}

\def\setupcode{\@makeother\^}

% Scheme example environment
% At 11 points, one column, these are about 56 characters wide.
% That's 32 characters to the left of the => and about 20 to the right.

\newenvironment{x10noindent}{
  % Commands for scheme examples
  \newcommand{\ev}{\>\>\evalsto}
  \newcommand{\lev}{\\\>\evalsto}
  \newcommand{\unspecified}{{\em{}unspecified}}
  \newcommand{\scherror}{{\em{}error}}
  \setupcode
  \small \cf \obeytabs \obeyspaces \myobeycr
  \begin{tabbing}%
\qquad\=\hspace*{5em}\=\hspace*{9em}\=\kill%   was 16em
\gobblecr}{\unskip\end{tabbing}}

%\newenvironment{scheme}{\begin{schemenoindent}\+\kill}{\end{schemenoindent}}
\newenvironment{x10}{
  % Commands for scheme examples
  \newcommand{\ev}{\>\>\evalsto}
  \newcommand{\lev}{\\\>\evalsto}
  \renewcommand{\em}{\rmfamily\itshape}
  \newcommand{\unspecified}{{\em{}unspecified}}
  \newcommand{\scherror}{{\em{}error}}
  \setupcode
  \small \cf \obeyspaces \myobeycr
  \footnotesize
  \begin{tabbing}%
\qquad\=\hspace*{5em}\=\hspace*{9em}\=\+\kill%   was 16em
\gobblecr}{\unskip\end{tabbing}\normalsize}

\newcommand{\evalsto}{$\Longrightarrow$}

% Manual entries

\newenvironment{entry}[1]{
  \vspace{3.1ex plus .5ex minus .3ex}\noindent#1%
\unpenalty\nopagebreak}{\vspace{0ex plus 1ex minus 1ex}}

\newcommand{\exprtype}{syntax}

% Primitive prototype
\newcommand{\pproto}[2]{\unskip%
\hbox{\cf\spaceskip=0.5em#1}\hfill\penalty 0%
\hbox{ }\nobreak\hfill\hbox{\rm #2}\break}

% Parenthesized prototype
\newcommand{\proto}[3]{\pproto{(\mainschindex{#1}\hbox{#1}{\it#2\/})}{#3}}

% Variable prototype
\newcommand{\vproto}[2]{\mainschindex{#1}\pproto{#1}{#2}}

% Extending an existing definition (\proto without the index entry)
\newcommand{\rproto}[3]{\pproto{(\hbox{#1}{\it#2\/})}{#3}}

% Grammar environment

\newenvironment{grammar}{
  \def\:{& \goesto{} &}
  \def\|{& $\vert$& }
  \def\opt{$^?$\ }
  \def\star{$^*$\ }
  \def\plus{$^+$\ }
  \em
  \begin{tabular}{rcl}
  }{\unskip\end{tabular}}

%\newcommand{\unsection}{\unskip}
\newcommand{\unsection}{{\vskip -2ex}}

% Commands for grammars
\newcommand{\arbno}[1]{#1\hbox{\rm*}}  
\newcommand{\atleastone}[1]{#1\hbox{$^+$}}

\newcommand{\goesto}{{\normalfont{::=}}}

% mark modifications (for the grammar) From Igor Pechtchanski/Watson/IBM@IBMUS
\newlength{\modwidth}\setlength{\modwidth}{0.005in}
\newlength{\modskip}\setlength{\modskip}{.4em}
\newlength{\@modheight}
\newlength{\@modpos}
\providecommand{\markmod}[1]{%
  \setlength{\@modheight}{#1}%
  \addtolength{\@modheight}{-0.06in}%
  \setlength{\@modpos}{\linewidth}%
  \addtolength{\@modpos}{0.285in}%         Magic
  \addtolength{\@modpos}{\modwidth}%
  \addtolength{\@modpos}{\modskip}%
  \marginpar{\vspace{-\@modheight}%
             \hspace{-\@modpos}%
             \rule{\modwidth}{#1}}%
}

% The index

\def\theindex{%\@restonecoltrue\if@twocolumn\@restonecolfalse\fi
%\columnseprule \z@
%!! \columnsep 35pt
\clearpage
\@topnewpage[
    \centerline{\large\bf\uppercase{Alphabetic index of definitions of concepts,}}
    \centerline{\large\bf\uppercase{keywords, and procedures}}
    \vskip 1ex \bigskip]
    \markboth{Index}{Index}
    \addcontentsline{toc}{chapter}{Alphabetic index of 
 definitions of concepts, keywords, and procedures}
    \bgroup %\small
    \parindent\z@
    \parskip\z@ plus .1pt\relax\let\item\@idxitem}

\def\@idxitem{\par\hangindent 40pt}

\def\subitem{\par\hangindent 40pt \hspace*{20pt}}

\def\subsubitem{\par\hangindent 40pt \hspace*{30pt}}

\def\endtheindex{%\if@restonecol\onecolumn\else\clearpage\fi
\egroup}

\def\indexspace{\par \vskip 10pt plus 5pt minus 3pt\relax}

\makeatother

%\newcommand{\Xten}{{\sf X10}}
%\newcommand{\XtenCurrVer}{{\sf X10 v1.7}}
%\newcommand{\java}{{\sf Java}}
%\newcommand{\Java}{{\sf Java}}

\newcommand{\Xten}{X10}
\newcommand{\XtenCurrVer}{\Xten{} v1.7}
\newcommand{\Java}{Java}
\newcommand{\java}{\Java{}}

\newcommand{\futureext}[1]{{\em \paragraph{Future Extensions.}#1}}
\newcommand{\tbd}{} % marker for things to be done later.
\newcommand{\limitation}[1]{{\em Limitation: #1}} % marker for things to be done later.


\newcommand\grammarrule[1]{\emph{#1}}

% Rationale

\newenvironment{rationale}{%
\bgroup\noindent{\sc Rationale:}\space}{%
\egroup}

% Notes

\newenvironment{note}{%
\bgroup\noindent{\sc Note:}\space}{%
\egroup}

\newenvironment{staticrule*}{%
\bgroup\noindent{\textsc{Static Semantics Rule:}\space}}{%
\egroup}

\newenvironment{staticrule}[1]{%
\bgroup\noindent{\textsc{Static Semantics Rule} (#1):\space}}{%
\egroup}

\newcommand\Sref[1]{\S\ref{#1}}
\newcommand\figref[1]{Figure~\ref{#1}}
\newcommand\tabref[1]{Table~\ref{#1}}
\newcommand\exref[1]{Example~\ref{#1}}

\newcommand\eat[1]{}



\begin{document}


\maketitle


\lstset{language=java,basicstyle=\ttfamily\small}

%\chapter{Featherweight Ownership and Immutability Generic Java}
\section{Introduction}


We begin with some definitions.

An expression/type is called \emph{closed} if it does not contain \proto nor
    free variables (but it may contain \cooked).

Given a method $\code{U m(}\ol{\hV}~\ol{\hx}\code{) K \lb~return e;~\rb}$ in class~\hC,
    then
\beqst
    \code{\mtype{}(f,C)} &= \ol{\hV}\mapsto\hU\\
    \code{\mtype{}(f,C}^\hK\code{)} &= [\hK/\proto]\code{\mtype{}(f,C)}\\
    \code{\mbody{}(f,C)} &= \he\\
    \code{\mproto{}(f,C)} &= \hK\\
\eeq
We also define \code{\ftype{}(f,C)} and \code{\isVar{}(f,C)}.

%\begin{cases}
%\hl & \he=\hl \\
%\code{error} & \text{otherwise} \\
%\end{cases}

Given an expression~\he, we define~$R(\he)$ to be the set
    of all ongoing constructors in~\he, i.e., all locations in subexpressions~\code{e;return l}.
Formally,
\[
R(\he) =
\begin{cases}
    R(\code{e'}) \cup \{ l \} & \text{if~}\he=(\code{e';return l}) \\
    R(\code{e'}) & \text{if~}\he=(\code{e'.f}) \\
    R(\code{e'}) \cup R(\code{e"}) & \text{if~}\he=(\code{e'.f=e"}) \\
    \cup R(\ol{\he'}) & \text{if~}\he=(\code{new T}\hparen{\ol{\he'}}) \\
    R(\code{e"}) \cup R(\ol{\he'}) & \text{if~}\he=(\code{e".m}\hparen{\ol{\he'}}) \\
    \end{cases}
\]

Summary of syntax used:
Environment $\Gamma ::= \epsilon ~|~ \hx:\hT,\Gamma ~|~ \hl:\hT,\Gamma$.
Proto locations (locations whose constructor is still ongoing) $P::= \epsilon ~|~ \hl,P$.




A heap is a pair $\angular{H,S}$

A \emph{well-typed} heap~$H$ satisfies:
    (i)~there is a linear order~$\Tprec$ over~$\dom{}(H)$ such that for every location~\hl, $H[\hl]=\hC^{\hl'}$,
        we have~$\hl' \Tprec \hl$,
        and
    (ii)~each non-null field location is a subtype (using $\Pst$) of the declared field type.
A heap~$H$ is \emph{well-typed for~\he} if~$H[K \mapsto H[K]\cup R(\he)]$ is well-typed.

\begin{smaller}

\begin{figure*}[htpb!]
\begin{center}
\begin{tabular}{|l|l|}
\hline

$\hK ::= \proto ~|~ \cooked ~|~ \textbf{\hl}$ & cooKer. \\

$\code{T} ::= \code{C<K>}$ & Type. \\

$\code{A} ::= \code{var}~|~\code{val}$ & Assignable (\code{var}) or final (\code{val}) field. \\

$\code{F} ::= \code{A}~\hT~\hf\texttt{;}$ & Field declaration. \\

$\hM ::= \code{T} ~ \hm\hparen{\ol{\code{T}} ~ \ol{\hx}}~\hK ~ \lb\ \hreturn ~ \he\texttt{;}~\rb$
& Method declaration. \\

$\hL ::= \hclass ~ \hC\code{~extends~C'} \lb\ \ol{\code{F}}~\ol{\hM}~\rb$
& cLass declaration. \\


$\hv ::= \code{null} ~|~ \textbf{\hl} $
& Values: either \code{null} or a location~\hl. \\


% No cast: \hparen{\hT} ~ \he ~|~
$\he ::= \hv ~|~ \hx ~|~ \he.\hf ~|~ \he.\hf = \he ~|~ \he.\hm\hparen{\ol{\he}} ~|~ \hnew ~ \hT\hparen{\ol{\he}}  ~|~ \textbf{\he\code{;return l}}$
& Expressions. \\ %: values, parameters, field access\&assignment, invocation, \code{new} start\&finish

\hline
\end{tabular}
\end{center}
\caption{FX10 Syntax. Class declarations in FX10 cannot contain locations~\hl (marked with a boldface).
    Such locations are created during the reduction process (see \RULE{R-New} in \Ref{Figure}{reduction}).}
\label{Figure:syntax}
\end{figure*}


\begin{figure*}[!bt]
\begin{center}
\begin{tabular}{|c|}
\hline


$\typerule{
}{
  \Gamma \vdash \hT \st \hT
}$
~\RULE{(S1)}\quad

$\typerule{
  \Gamma \vdash \code{S} \st \hT
   \gap
  \Gamma \vdash \hT \st \hU
}{
  \Gamma \vdash \code{S} \st \hU
}$
~\RULE{(S2)}\quad
$\typerule{
  \code{class C}_1\code{~extends C}_2^\hK
}{
  \Gamma \vdash \hC_1^\hK \st \code{C}_2^\hK
}$
~\RULE{(S3)}

\\

$\typerule{
  \hl \in \Gamma[K]
}{
  \Gamma \vdash \hl \st \proto
}$
~\RULE{(S4)}\quad
$\typerule{
  \hl \not \in \Gamma[K]
}{
  \Gamma \vdash \hl \st \cooked
}$
~\RULE{(S5)}\quad
$\typerule{
  \hl \not \in \Gamma[K]
}{
  \Gamma \vdash \cooked \st \hl
}$
~\RULE{(S6)}\quad

$\typerule{
    \Gamma \vdash \code{K} \st \code{K}'
}{
  \Gamma \vdash \hC^\code{K} \st \hC^{\code{K}'}
}$
~\RULE{(S7)}\quad
\\


$\typerule{
  \Gamma \vdash \code{T} \st \code{T}'
}{
  \Gamma \vdash \code{T} \Pst \code{T}'
}$
~\RULE{(S8)}\quad
$\typerule{
  \hl \not \in \Gamma[K]
}{
  \Gamma \vdash \code{C}^\hl \Pst \code{C}^\hK
}$
~\RULE{(S9)}\quad
\\

\hline
\end{tabular}
\end{center}
\caption{FX10 Subtyping Rules. The subtyping relation is $\st$, whereas $\Pst$ is the pointing relation (anything can point to a cooked object).}
\label{Figure:subtyping}
\end{figure*}


\begin{figure*}[t]
\begin{center}
\begin{tabular}{|c|}
\hline
$\typerule{
  \Gamma,P \cup \{ \hl \} \vdash \he:\hT
}{
  \Gamma,P \vdash \code{e;return l} : \Gamma(\hl)
}$
\quad \RULE{(T-return)}
\\\\

$\typerule{
\hK'=
\begin{cases}
\bot & \hK=\cooked \\
\hK & \text{otherwise} \\
\end{cases}
    \gap
  \mtype{}(\code{build},\code{C<K'>})=\ol{\code{T}}\rightarrow\code{Object}
    \gap
  \Gamma,P \vdash \ol{\he}:\ol{\code{V}}
    \gap
  \Gamma,P \vdash \ol{\code{V}} \st \ol{\code{T}}
}{
  \Gamma,P \vdash \code{new C<K>(}\ol{\he}\code{)} : \code{C<K>}
}$
\quad \RULE{(T-New)}\\

$\typerule{
}{
  \Gamma,P \vdash \hx : \Gamma(\hx)
}$
\quad \RULE{(T-Var)}
\qquad
$\typerule{
}{
  \Gamma,P \vdash \code{null} : \hT
}$
\quad \RULE{(T-null)}
\qquad
$\typerule{
}{
  \Gamma,P \vdash \hl : \Gamma(\hl)
}$
\quad \RULE{(T-Location)}\\\\

$\typerule{
  \Gamma,P \vdash \he:\code{C<K>}
    \gap
  \isCooked(\hK)
    \gap
  \ftype{}(\hf,\hC)=\code{C'}
}{
  \Gamma,P \vdash \he.\hf : \code{C'<K>}
}$
\quad \RULE{(T-Field-Access)}\\\\


$\typerule{
  \Gamma,P \vdash \he:\code{C<K>}
    \gap
  \ftype{}(\hf,\hC)=\code{C'}
    \gap
  \Gamma,P \vdash \code{e'}:\code{T'}
    \gap
  \Gamma,P \vdash \code{T'} \Pst \code{C'<K>}
    \\
  \big(\isProto(\hK) \text{~~or~~} \isVar{}(\hf,\hC)\big)
}{
  \Gamma,P \vdash \he.\hf = \code{e'} : \code{T'}
}$
\quad \RULE{(T-Field-Assignment)}\\\\

$\typerule{
  \Gamma,P \vdash \he':\code{C<K>}
    \gap
  \mtype{}(\hm,\code{C<K>})=\ol{\code{T}}\rightarrow\code{U}
    \gap
  \Gamma,P \vdash \ol{\he}:\ol{\code{V}}
    \gap
  \Gamma,P \vdash \ol{\code{V}} \st \ol{\code{T}}
    \\
  \isCooked(\hK)=\isCooked(\mproto{}(\hm,\code{C}))
}{
  \Gamma,P \vdash \he'\code{.m(}\ol{\he}\code{)} : \code{U}
}$
\quad \RULE{(T-Invoke)}\\


\hline
\end{tabular}
\end{center}
\caption{FX10 Expression Typing Rules.}
\label{Figure:expressions}
\end{figure*}


\begin{figure*}[t]
\begin{center}
\begin{tabular}{|c|}
\hline

$\typerule{
  \hl \not \in \dom(H)
    \gap
  \code{l'} =
    \begin{cases}
    \hl & \text{if~}\isCooked(\hK) \\
    \hK & \text{otherwise} \\
    \end{cases}
}{
  P \vdash H,\code{new C<K>}\hparen{\ol{\hv}} \rightarrow H[\hl \mapsto \code{C<l'>}\hparen{\ol{\code{null}}}],\hl\code{.build}\hparen{\ol{\hv}}\code{;return l}
}$
\quad \RULE{(R-New)}\\\\

$\typerule{
  H[\hl] = \code{C<K>}\hparen{\ol{\hv}}
    \gap
  \fields{}(\hC)=\ol{\hf}
}{
  P \vdash H,\hl.\hf_i \rightarrow H,\hv_i
}$
\quad \RULE{(R-Field-Access)}
\\\\

$\typerule{
  H[\hl] = \code{C<K>}\hparen{\ol{\hv}}
    \gap
  \fields{}(\hC)=\ol{\hf}
}{
  P \vdash H,\hl.\hf_i = \hv' \rightarrow H[\hl \mapsto \code{C<K>}\hparen{[\hv'/\hv_i]\ol{\hv}}],\hv'
}$
\quad \RULE{(R-Field-Assignment)}\\\\


$\typerule{
}{
  P \vdash H,\code{v;return l} \rightarrow H,\hl
}$
\quad \RULE{(R-return)}
\gap

$\typerule{
  H[\hl] = \code{C<K>}\hparen{\ldots}
    \gap
  \mbody{}(\hm,\code{C})=\ol{\hx}.\he'
}{
  P \vdash H,\hl\code{.m(}\ol{\hv}\code{)} \rightarrow H, [\ol{\hv}/\ol{\hx}, \hl/\this, \hl/\proto]\he'
}$
\quad \RULE{(R-Invoke)}\\\\



$\typerule{
  P \cup \{\hl\} \vdash H,\he \rightarrow H',\code{e'}
}{
  P \vdash H,\code{e;return l} \rightarrow H',\code{e';return l}
}$
\quad \RULE{(R-c1)}
\gap

$\typerule{
  P \vdash H,\he \rightarrow H',\code{e'}
}{
  P \vdash H,\code{e.f} \rightarrow H',\code{e'.f}
}$
\quad \RULE{(R-c2)}
\\\\

$\typerule{
  P \vdash H,\he \rightarrow H',\code{e'}
}{
  P \vdash H,\code{e.f=e"} \rightarrow H',\code{e'.f=e"}
}$
\quad \RULE{(R-c3)}
\gap

$\typerule{
  P \vdash H,\he \rightarrow H',\code{e'}
}{
  P \vdash H,\code{l.f=e} \rightarrow H',\code{l.f=e'}
}$
\quad \RULE{(R-c4)}
\\\\

$\typerule{
  P \vdash H,\he \rightarrow H',\code{e'}
}{
  P \vdash H,\code{new C<K>}\hparen{\ol{\hv},\he,\ol{\code{e"}}} \rightarrow H',\code{new C<K>}\hparen{\ol{\hv},\code{e'},\ol{\code{e"}}}
}$
\quad \RULE{(R-c5)}
\\\\


$\typerule{
  P \vdash H,\he \rightarrow H',\code{e'}
}{
  P \vdash H,\he\code{.m(}\ol{\code{e"}}\code{)} \rightarrow H',\code{e'}\code{.m(}\ol{\code{e"}}\code{)}
}$
\quad \RULE{(R-c6)}
\gap

$\typerule{
  P \vdash H,\he \rightarrow H',\code{e'}
}{
  P \vdash H,\code{l.m(}\ol{\hv},\he,\ol{\code{e"}}\code{)} \rightarrow H',\code{l.m(}\ol{\hv},\code{e'},\ol{\code{e"}}\code{)}
}$
\quad \RULE{(R-c7)}
\gap

\\
\hline
\end{tabular}
\end{center}
\caption{FX10 Reduction Rules. The congruence rules have the initial \RULE{R-c}.}
\label{Figure:reduction}
\end{figure*}

\end{smaller}


Next we describe the syntax (\Ref{Figure}{syntax}),
    subtyping rules (\Ref{Figure}{subtyping}),
    expression typing rules (\Ref{Figure}{expressions}),
    and reduction rules (\Ref{Figure}{reduction}).

\section{Syntax}
Obviously, class declarations cannot contain locations.

\section{Subtyping}


\section{Typing}
\paragraph{Method typing}
If \proto appears in $\mtype{}(\hm,\hC)$ then $\mproto{}(\hm,\hC)=\proto$.

An overriding method must keep the same $\mtype$ and $\mproto$.

In class~\hC, when typing a method:
        $\code{U} ~ \hm\hparen{\ol{\code{V}} ~ \ol{\hx}} ~ \hK~ \lb\ \hreturn ~ \he\texttt{;} \rb$\\
        we use an environment~$\Gamma=\{\ol{\hx}:\ol{\code{T}}, \this:\code{C}^\hK\}$,
        and we must prove that~$\Gamma \vdash \he:\code{S}$
        and~$\Gamma \vdash \code{S} \st \code{U}$.

\paragraph{Expression typing}
See \Ref{Figure}{expressions}.



\end{document}







We next prove that a closed expression has a closed type.
\begin{Lemma}[closed]
If~$\Gamma \vdash \he" : \hT"$ and $\he"$ is closed and $\he"\neq\code{null}$, then $\hT"$ is closed.
\end{Lemma}
\begin{proof}
Note that \code{null} can have any type $\hT"$ (even an open type) according to rule \RULE{T-null},
    therefore we require that~$\he"\neq\code{null}$.

We prove by induction on the structure of~$\he"$.
\begin{description}
  \item[Value~$\he"=\hv$] Because~$\he"\neq\code{null}$, then~$\hv=\hl$,
    and the type of a location is always closed~$\hC\code{<NO,NI>}$.
  \item[Value~$\he"=\code{e;return l}$] Similarly, the type of a location is always closed.
  \item[Method parameter~$\he"=\hx$] We assumed $\he"$ is closed, thus it does not contain parameters.
  \item[Object creation~$\he"= \code{new C<FO,VI>(}\ol{\he}\code{)}$]
    From \RULE{T-New},~$\hT" =\hC\hgn{\code{FO},\code{VI}}$, and because~$\he"$ is closed then~$\hT"$ must be closed.
  \item[Field access~$\he"=\he.\hf$]
    Because~$\Gamma \vdash \he" : \hT"$, from \RULE{T-Field-Access} we have that~$\hT" = \hT$ and
    \[
    \Gamma \vdash \he:\code{C<MO,IP>}
        \gap
    \ftype{}(\he,\hf,\code{C<MO,IP>})=\code{T}
    \]
    By induction,~$\code{C<MO,IP>}$ is closed.
    Therefore~$\hT$ does not contains~\hO nor~\hI.
    Next we show it does not contain \This.
    Because $\ftype$ did not return \code{error},
        then either the field did not contain \This, or it was substituted.
    Because $\he\neq\this$ then~$\he=\hl$, and \This was substituted with \hl.

  \item[Field assignment~$\he"=\he.\hf = \code{e'}$]
    Because~$\Gamma \vdash \he" : \hT"$, from \RULE{T-Field-Assignment} we have that~$\hT" = \hT'$ and
        $
        \Gamma \vdash \code{e'}:\code{T'}
        $
    By induction, \code{T'} is closed.

  \item[Method invocation~$\he"=\he_0\code{.m(}\ol{\he}\code{)}$]
    Because~$\Gamma \vdash \he" : \hT"$, from \RULE{T-Invoke} we have that
    \[
    \Gamma \vdash \he_0:\code{C<MO,IP>}
        \gap
    \mtype{}(\he_0,\hm,\code{C<MO,IP>})=\ol{\code{T}}\rightarrow\code{T"}
    \]
    By induction~$\code{C<MO,IP>}$ is closed, and similar reasoning to field access
        concludes that \code{T"} is closed.
\end{description}
\end{proof}


\section{Heap}

We now prove that if the heap is well-typed, then it remains well-typed even if we remove locations from~$H[K]$,
    i.e., being well-typed is a \emph{stable property}.

\begin{Lemma}[well-typed]
Given a well-typed heap~$H$,
    then for any~$S \subset H[K]$,
    the heap~$H' = H[K \mapsto S]$ is well-typed.
\end{Lemma}
\begin{proof}
This is not trivial, because decreasing~$H[K]$ changes the subtyping relation by turning raw objects into cooked.

First, note that the cookers of each location in $H$ and $H'$ are the same, i.e.,
    the same linear order~$\Tprec$ satisfies (i) for~$H'$.
Suppose to the contrary that $H'$ is not well-typed, i.e.,
    there is some field location~$\code{l.f}$ of type~\hT that points to an object~\ho of type~\code{T"},
    and~$\Gamma_{H'} \not \vdash \code{T"} \Pst \hT$.

\end{proof}


\section{Reduction}
We consider only expressions that when reduced using the erased operational semantics,
    do not result in \emph{null-pointer exceptions}.
Null-pointer exceptions can be handled by adding special reduction rules that return \code{error},
    but we prefer to leave the reduction process ``stuck".

First we prove that a closed expression reduces in one step to another closed expression.
\begin{Lemma}[reduction-closed]
If~\he is closed and~$H,\he \rightarrow H',\he'$, then~$\he'$ is closed.
\end{Lemma}
\begin{proof}
Rules~\RULE{K-New}, \RULE{K-Field-Access}, and \RULE{K-Field-assignment} result in a value, which is closed.
Rule~\RULE{K-Invoke} results in an expression, but all free variables~$\ol{\hx}, \this, \This, \hO, \hI$ are substituted
    with locations, \Immut, $\Immut_\hl$, or \Mutable.
The proof of the congruence rules uses the induction hypothesis.
\end{proof}

\begin{Lemma}[constructors]
If~$H,\he \rightarrow H",\he"$ then
    (i)~$H[K]=H"[K]$,
    (ii)~$\Gamma_H \subseteq \Gamma_{H"}$,
    (iii)~$\Gamma_H \vdash \he' : \hT' \Rightarrow \Gamma_{H'} \vdash \he' : \hT'$, and
    (iv)~$\Gamma_H \vdash \hT \st \hT' \Rightarrow \Gamma_{H'} \vdash \hT \st \hT'$.
\end{Lemma}
\begin{proof}
Proved by induction on the reduction sequence.

(i)~The only reduction rule that mentions~$H[K]$ is~\RULE{K-c1},
    and from the induction hypothesis we have that~$H'[K] = H[K] \cup \{\hl\}$,
    therefore the resulting heap is~$H" = H'[K \mapsto H'[K] \setminus \{\hl\}] = H'[K \mapsto H[K]]$,
        i.e.,~$H"[K] = H[K]$.

(ii)~None of the reduction rules removes locations or changes the type of a location,
    therefore~$H"$ only includes additional locations, and from (i)~$H"[K] = H[K]$.

Parts (iii) and (iv) are trivial from (ii).
\end{proof}


\begin{Theorem}[preservation]
  \textbf{(Progress and Preservation)}
    For every closed expression~$\he \neq \hv$, and a heap~$H$ that is well-typed for~\he,
        if $\Gamma_{H} \vdash \he : \hT$,
        then there exists~$H',\he',\hT'$ such that
        (i)~$H,\he \rightarrow H',\he'$,
        (ii)~$\Gamma_{H'} \vdash \he':\hT'$,
        and~$\Gamma_{H'} \vdash \hT' \st \hT$,
        (iii)~$H'$ is well-typed for~$\he'$,
        (iv)~\hT, \hT', and $\he'$ are closed.
\end{Theorem}
\begin{proof}
Part (iv) is proved from \Ref{Lemma}{reduction-closed} (we know that~$\grave{\he}$ is closed)
    and from \Ref{Lemma}{closed} (we know that $\hT"$ and $\grave{\hT}$ are closed).

We assume that there are no null-pointer exceptions,
    i.e., that for field access, assignment and method invocation, the receiver is never \code{null}.

It is easy to examine the reduction rules
    and verify there is always at most one applicable reduction rule.
We will split the proof into three stages:
    (i)~\textbf{progress:} there is exactly one applicable reduction rule
        (\Ref{Lemma}{part-progress}),
    (ii)~\textbf{preservation:}~$\Gamma_{H'} \vdash \grave{\he}:\grave{\hT}$ and~$\Gamma_{H'} \vdash \grave{\hT} \st \hT"$
        (\Ref{Lemma}{part-subtype}), and
    (iii)~$H'$ is well-typed for~$\he'$ (\Ref{Lemma}{part-well-typed}).
\end{proof}


\begin{Lemma}[part-progress]
  \textbf{(Progress)}
    For every closed expression~$\he" \neq \hv$, and a heap~$H$ that is well-typed for~$\he"$,
        if $\Gamma_{H} \vdash \he" : \hT"$,
        then there exists~$H',\grave{\he}$ such that~$H,\he" \rightarrow H',\grave{\he}$.
\end{Lemma}
\begin{proof}
We prove by examining the structure of~$\he"$.
Because it is closed and not a value, then according to our syntax, it must have one of the following forms:
\[
    \he.\hf ~|~ \he.\hf = \he ~|~ \he.\hm\hparen{\ol{\he}} ~|~ \hnew ~ \hC\hgn{\code{FO},\code{VI}}\hparen{\ol{\he}}  ~|~ \he\code{;return l}
\]
If the subexpressions are not all values, then
    we can always apply (exactly) one of the congruence rules.
For example, if~$\he" = (\he\code{;return l})$ and~$\he$ is not a value,
    then by induction we can apply \RULE{K-c1}.

Therefore,~$\he"$ has one of the following forms:
\[
    \hv.\hf ~|~ \hv.\hf = \hv ~|~ \hv.\hm\hparen{\ol{\hv}} ~|~ \hnew ~ \hC\hgn{\code{FO},\code{VI}}\hparen{\ol{\hv}}  ~|~ \hv\code{;return l}
\]
Because we assumed we do not have \emph{null pointer exceptions} (in field access, assignment or method invocation), then
    $\he"$ has one of the following forms:
\[
    \hl.\hf ~|~ \hl.\hf = \hv ~|~ \hl.\hm\hparen{\ol{\hv}} ~|~ \hnew ~ \hC\hgn{\code{FO},\code{VI}}\hparen{\ol{\hv}}  ~|~ \hv\code{;return l}
\]
We will next examine the matching five reduction rules
    (\RULE{K-Field-Access},
    \RULE{K-Field-Assignment},
    \RULE{K-Invoke},
    \RULE{K-New},
    \RULE{K-return})
    and show that their assumptions hold.

\begin{description}
  \item[Rule~\RULE{K-Field-Access}]
  \[
  \typerule{
    H[\hl] = \code{C<NO,NI>}\hparen{\ol{\hv}}
        \gap
    \fields{}(\hC)=\ol{\hf}
    }{
    H,\hl.\hf_i \rightarrow H,\hv_i
    }
    \]
  We assumed that~$\Gamma_{H} \vdash \hl.\hf_i : \hT"$,
    therefore from \RULE{T-Field-Access}:
    \[
  \Gamma_{H} \vdash \hl:\code{C<MO,IP>}
    \gap
  \ftype{}(\hl,\hf_i,\code{C<MO,IP>})=\hT"
    \]
  From the definitions of~\ftype and~\fields, we know that~$\hf_i \in \fields{}(\hC)$.

  \item[Rule~\RULE{K-Field-Assignment}]
  \[
  \typerule{
  H[\hl] = \code{C<NO,NI>(}\ol{\hv}\code{)}
    \gap
  \fields{}(\hC)=\ol{\hf}
    \gap
  \code{NI}=\Mutable \text{~or~} \Cooker{\hl} \in H[K]
    \gap
  \hv'=\code{null} \text{~or~} \hl \Oprec \Owner{\hv'}
}{
  H,\hl.\hf_i = \hv' \rightarrow H[\hl \mapsto \code{C<NO,NI>(}[\hv'/\hv_i]\ol{\hv}\code{)}],\hv'
}
    \]
Because~$H$ is well-typed, then from \Ref{Lemma}{owner-as-dominator},
    we have that~$\hv'=\code{null} \text{~or~} \hl \Oprec \Owner{\hv'}$.

We assumed that~$\Gamma_{H} \vdash \hl.\hf_i = \code{\hv'} : \hT'$,
      therefore from \RULE{T-Field-Assignment}:
      \[
        \Gamma_{H} \vdash \hl.\hf_i : \hT
            \gap
        \Gamma_{H} \vdash \hl:\code{C<NO,NI>}
            \gap
        \Gamma_{H} \vdash \code{NI} \st \Raw
      \]
Similarly to field access, because~$\Gamma_{H} \vdash \hl.\hf_i : \hT$,
    then~$\hf_i \in \fields{}(\hC)$.
From our syntax~$\code{NI}$ is either \Mutable or $\Immut_{\hl'}$.
We want to show that~$\code{NI}=\Mutable \text{~or~} \Cooker{\hl} \in H[K]$.
Therefore we need to show that if~$\code{NI}=\Immut_{\hl'}$ then~$\hl' \in H[K]$.
Because~$\Gamma_{H} \vdash \code{NI} \st \Raw$, it must be from \RULE{S10} and therefore~$\hl' \in H[K]$.

  \item[Rule~\RULE{K-Invoke}]
  \[
 \typerule{
  H[\hl] = \code{C<NO,NI>}\hparen{\ldots}
    \gap
  \mbody{}(\hm,\code{C})=\ol{\hx}.\he'
}{
  H,\hl\code{.m(}\ol{\hv}\code{)} \rightarrow H, [\ol{\hv}/\ol{\hx}, \hl/\this, \hl/\This, \code{NO}/\hO, \code{NI}/\hI]\he'
}
    \]
  We assumed that~$\Gamma_{H} \vdash \hl\code{.m(}\ol{\hv}\code{)} : \hT"$,
    therefore from \RULE{T-Invoke} we know that \[
    \mtype{}(\hl,\hm,\code{C<NO,NI>})=\ol{\code{T}}\rightarrow\code{T"}
    \]
    Therefore~$\mbody{}(\hm,\code{C})$ is defined.

  \item[Rule~\RULE{K-New}]
  \[
  \typerule{
  \hl \not \in \dom(H)
    \gap
  \code{VI}' =
    \begin{cases}
    \Immut_\hl & \text{if~}\code{VI}=\Immut \text{~or~} (\code{VI}=\Immut_{\hc} \text{~and~} \hc \not \in H[K]) \\
    \code{VI} & \text{otherwise} \\
    \end{cases}
    \gap
  H'=H[\hl \mapsto \code{C<NO,VI'>}\hparen{\ol{\code{null}}}]
}{
  H,\code{new C<NO,VI>}\hparen{\ol{\hv}} \rightarrow H',\hl\code{.build}\hparen{\ol{\hv}}\code{;return l}
}
    \]
    We assumed that~$\Gamma_{H} \vdash \code{new C<NO,VI>}\hparen{\ol{\hv}} : \hT"$,
        therefore from \RULE{T-New} we know that \[
          \mtype{}(\bot,\code{build},\code{C<NO,NI>})=\ol{\code{T}}\rightarrow\code{U}
          \]
    Thus there is a constructor with~$\#(\ol{\hv})$ of arguments.


  \item[Rule~\RULE{K-return}]
    Trivial because there are no assumptions for~\code{v;return l}
\end{description}

\end{proof}

We prove \textbf{preservation} (\Ref{Lemma}{part-subtype}) for method invocation
    by using \Ref{Lemma}{invoke-substitution} that uses
    induction on the size of the method body.

\begin{Lemma}[invoke-substitution]
  \textbf{(Invocation Substitution)}
    For every well-typed heap~$H$,
        location~\hl, values~$\ol{\hv}$, types~$\code{U}$, and guard~\code{IG},
        where \beqs{invoke-substitution1}
            H[\hl] & = \code{C<NO,NI>}\\
            \Gamma_H & \vdash \code{NI} \st \code{IG}\\
            \Gamma_H & \vdash \ol{\hv} : \ol{\hU'}\\
            \Gamma_H & \vdash \ol{\hU'} \st \ol{[\hl/\This,\code{NI}/\hI,\code{NO}/\hO]\code{U}}\\
        \eeq
    Then, for any~$\he"$ such that
        $\Gamma \vdash \he":\hS$,~$\Gamma = \{ \hI:\code{IG}, \ol{\hx}:\ol{\code{U}},\this:\code{C<O,I>} \}$,
        then \beqst
        \Gamma_H & \vdash [\hl/\This,\code{NI}/\hI,\code{NO}/\hO, \ol{\hv}/\ol{\hx}, \hl/\this]\he" : \hS' \\
        \Gamma_H & \vdash \hS' \st [\hl/\This,\code{NI}/\hI,\code{NO}/\hO]\hS\\
        \eeq
\end{Lemma}
\begin{proof}

We will prove by induction on the structure of~$\he"$.

\begin{description}
  \item[$\he"=(\code{e;return l})$] Impossible because $\Gamma$ does not contain locations.
  \item[$\he"=(\hv)$] $\Gamma$ does not contain locations. Therefore,~$\hv=\code{null}$, and we can choose~$\hS' = [\hl/\This,\code{NI}/\hI,\code{NO}/\hO]\hS$.
  \item[$\he"=(\this)$] Then~$\hS = \code{C<O,I>}$,
    and
    \beqst
        & [\hl/\This,\code{NI}/\hI,\code{NO}/\hO, \ol{\hv}/\ol{\hx}, \hl/\this]\he" = \hl \\
        & [\hl/\This,\code{NI}/\hI,\code{NO}/\hO]\hS = \code{C<NO,NI>}\\
        & \hS' = \code{C<NO,NI>}\\
        &\Gamma_H \vdash \hl : \hS' \\
        & \Gamma_H \vdash \hS' \st \code{C<NO,NI>}\\
        \eeq

    \item[$\he"=(\hx_i)$] Then~$\hS = \hU_i$.
    We assumed that \beqst
        \Gamma_H & \vdash \hv_i : \hU'_i\\
        \Gamma_H & \vdash \hU'_i \st [\hl/\This,\code{NI}/\hI,\code{NO}/\hO]\code{U}_i\\
    \eeq
    Therefore,
    \beqst
        & [\hl/\This,\code{NI}/\hI,\code{NO}/\hO, \ol{\hv}/\ol{\hx}, \hl/\this]\he" = \hv_i \\
        & \hS' = \hU'_i\\
        & \hS = \hU_i\\
        &\Gamma_H \vdash \hv_i : \hS' \\
        & \Gamma_H \vdash \hS' \st [\hl/\This,\code{NI}/\hI,\code{NO}/\hO]\hS\\
        \eeq



  \item[$\he"=(\code{new D<FO,VI>}\hparen{\ol{\he}})$]
    We assumed that~$\Gamma \vdash \code{new D<FO,VI>}\hparen{\ol{\he}}:\hS$.
    From \RULE{T-new}, $\hS = \code{D<FO,VI>}$ and \beq{new0}
  \mtype{}(\bot,\code{build},\code{D<FO,VI>})=\ol{\code{T}}\rightarrow\code{U}
    \gap
  \Gamma \vdash \ol{\he}:\ol{\code{T}'}
    \gap
  \Gamma \vdash \ol{\code{T}'} \st \ol{\code{T}}
    \eeq
    By induction on~$\he_i$, we have that \beqs{new1}
        \Gamma_H & \vdash [\hl/\This,\code{NI}/\hI,\code{NO}/\hO, \ol{\hv}/\ol{\hx}, \hl/\this]\he_i : \code{V}_i \\
        \Gamma_H & \vdash \code{V}_i \st [\hl/\This,\code{NI}/\hI,\code{NO}/\hO]\code{T}'_i\\
        \eeq
    From \Ref{Lemma}{subtyping-substitute} and~\eq{new0}, we have that \beq{new2}
        \Gamma_H \vdash [\hl/\This,\code{NI}/\hI,\code{NO}/\hO]\code{T}'_i \st [\hl/\This,\code{NI}/\hI,\code{NO}/\hO]\code{T}_i
    \eeq
    From transitivity,~\eq{new1}, and~\eq{new2}, we have \beq{new3}
        \Gamma_H \vdash \code{V}_i \st [\hl/\This,\code{NI}/\hI,\code{NO}/\hO]\code{T}_i
    \eeq
    From definition of $\mtype$ we have \beq{new4}
        \mtype{}(\bot,\code{build},[\hl/\This,\code{NI}/\hI,\code{NO}/\hO]\code{D<FO,VI>})=[\hl/\This,\code{NI}/\hI,\code{NO}/\hO](\ol{\code{T}}\rightarrow\code{U})
    \eeq
    From \RULE{T-new},~\eq{new3}, and~\eq{new4}, \beqst
        \Gamma_H & \vdash [\hl/\This,\code{NI}/\hI,\code{NO}/\hO, \ol{\hv}/\ol{\hx}, \hl/\this]\code{new D<FO,VI>}\hparen{\ol{\he}} : [\hl/\This,\code{NI}/\hI,\code{NO}/\hO]\code{D<FO,VI>} \\
    \eeq

  \item[$\he"=(\code{e.f})$]
    From \RULE{T-Field-Access}, \beqst
        \Gamma &\vdash \he : \code{D<MO,IP>}\\
        \Gamma &\vdash \code{e.f} : \hS\\
        \hS &= \ftype(\he,\hf,\code{D<MO,IP>})\\
    \eeq
    Recall that~$\hS = \ftype{}(\he,\hf,\code{D<MO,IP>}) = \substitute(\he,\code{D<MO,IP>},\ftype{}(\hf,\code{D}))$.
    Note that~\he is not a location, and thus if \hf is \this-owned, then $\he=\this$.
    Consider first the case that \hf is \this-owned, thus $\he=\this$.
    Let~$\ftype{}(\hf,\code{D})=\code{FT}$, and~$\Ofn{\code{FT}}=\This$.
    We assumed that~$\Gamma \vdash \this.f:\hS$.
    Then,~$\hS=\code{FT}$.
    We need to show that
        \beqst
        \Gamma_H & \vdash [\hl/\This,\code{NI}/\hI,\code{NO}/\hO, \ol{\hv}/\ol{\hx}, \hl/\this]\code{this.f} : \hS' \\
        \Gamma_H & \vdash \hS' \st [\hl/\This,\code{NI}/\hI,\code{NO}/\hO]\code{FT}\\
        \eeq
    From \RULE{T-Field-Access} \[
        \Gamma_H \vdash \code{l.f} : [\hl/\This,\code{NI}/\hI,\code{NO}/\hO]\code{FT}
    \], i.e., $\hS' = [\hl/\This,\code{NI}/\hI,\code{NO}/\hO]\code{FT}$.

    Now suppose that \hf is not \this-owned.
    Therefore,~$\hS = [\code{IP}/\hI,\code{MO}/\hO]\code{FT}$.
    We need to show that
        \beqs{Access1}
        \Gamma_H & \vdash [\hl/\This,\code{NI}/\hI,\code{NO}/\hO, \ol{\hv}/\ol{\hx}, \hl/\this]\code{e.f} : \hS' \\
        \Gamma_H & \vdash \hS' \st [\hl/\This,\code{NI}/\hI,\code{NO}/\hO]\hS \gap \hS = [\code{IP}/\hI,\code{MO}/\hO]\code{FT}\\
        \eeq
    From the induction on~\he, we have that
        \beqs{Access2}
        \Gamma_H & \vdash [\hl/\This,\code{NI}/\hI,\code{NO}/\hO, \ol{\hv}/\ol{\hx}, \hl/\this]\he : \hS" \\
        \Gamma_H & \vdash \hS" \st [\hl/\This,\code{NI}/\hI,\code{NO}/\hO]\code{D<MO,IP>}\\
        \eeq
    From~\eq{Access2},~$\Ofn{\hS"}=[\hl/\This,\code{NO}/\hO]\code{MO}$.
    From~\eq{Access1} and~\eq{Access2}, and \Ref{Lemma}{subtyping2}, we have that \beqst
        &\hS' = \ftype{}(\bot,\hf,\hS") = [\Ifn{\hS"}/\hI,\Ofn{\hS"}/\hO]\code{FT}
            \st [([\code{NI}/\hI]\code{IP})/\hI,([\hl/\This,\code{NO}/\hO]\code{MO})/\hO]\code{FT} =\\
            &= [\hl/\This,\code{NI}/\hI,\code{NO}/\hO] ([\code{IP}/\hI,\code{MO}/\hO]\code{FT})\\
    \eeq

  \item[$\he"=(\code{e.f=e}')$]
    The challenge in field assignment is that (by induction) the type of the substitution of~\he
        changed covariantly (i.e., it is a subtype of the substitution of the type),
        and $\he'$ also changed covariantly.
    However, we will prove that because $\Ifn{\he}$ is \Raw, and \he is either \this or \this-owned, then \he is invariant.

    We assumed that~$\Gamma \vdash \code{e.f=e}':\hS$.
    From \RULE{T-Field-Assignment}, we know that \beqs{assign0}
&\Gamma \vdash \he.\hf : \hT
    \gap
  \Gamma \vdash \he':\hS
    \gap
  \Gamma \vdash \hS \st \hT
    \gap
  \Gamma \vdash \he:\code{D<MO,IP>}
    \\
&  \Gamma \vdash \code{IP} \st \Raw
    \gap
  \isTransitive(\he,\Gamma,\code{D<MO,IP>})
    \gap
  \code{MO} \neq \code{?}\\
  \eeq
  We wish to prove all the assumptions in~\eq{assign0} after substituting~$[\hl/\This,\code{NI}/\hI,\code{NO}/\hO, \ol{\hv}/\ol{\hx}, \hl/\this]$.



  By induction on~$\he'$ we have that
    \beqs{assign1}
        \Gamma_H & \vdash [\hl/\This,\code{NI}/\hI,\code{NO}/\hO, \ol{\hv}/\ol{\hx}, \hl/\this]\he' : \hS' \\
        \Gamma_H & \vdash \hS' \st [\hl/\This,\code{NI}/\hI,\code{NO}/\hO]\hS\\
        \eeq
  By induction on~$\he.\hf$ we have that
    \beqs{assign2}
        \Gamma_H & \vdash [\hl/\This,\code{NI}/\hI,\code{NO}/\hO, \ol{\hv}/\ol{\hx}, \hl/\this]\he.\hf : \hT' \\
        \Gamma_H & \vdash \hT' \st [\hl/\This,\code{NI}/\hI,\code{NO}/\hO]\hT\\
        \eeq
  From the proof of field access above, we see that the class of~$\hT'$ and~\hT is the same.
  By induction on~$\he$ we have that
    \beqs{assign3}
        \Gamma_H & \vdash [\hl/\This,\code{NI}/\hI,\code{NO}/\hO, \ol{\hv}/\ol{\hx}, \hl/\this]\he : \code{D'<MO',IP'>} \\
        \Gamma_H & \vdash \code{D'<MO',IP'>} \st [\hl/\This,\code{NI}/\hI,\code{NO}/\hO]\code{D<MO,IP>}\\
        \eeq
  Because~$\code{MO} \neq \code{?}$, from \Ref{Lemma}{Owners-Invariant} part (i), we have that~$\code{MO'}=\code{MO}$.

  We now prove that the following holds:
  \beqs{assign4}
  & \Gamma_H \vdash [\code{NI}/\hI]\code{IP} \st \Raw\\
  & \isTransitive([\hl/\this]\he,\Gamma,[\hl/\This,\code{NI}/\hI,\code{NO}/\hO]\code{D<MO,IP>})\\
    \eeq
  From~\eq{assign0},~$\Gamma \vdash \code{IP} \st \Raw$.
  From our syntax, and because~$\Gamma$ does not contain locations:\[
    \code{IP} = \ReadOnly ~|~ \Immut ~|~ \Mutable ~|~ \hI
  \]
  If~$\code{IP}=\Mutable$ then we proved~\eq{assign4}.
  Therefore, it must be that~$\code{IP}=\hI$ (thus~$[\code{NI}/\hI]\code{IP}=\code{NI}$)
    and~$\Gamma(\hI)=\code{IG} \st \Raw$.
  From~\eq{invoke-substitution1} ($\Gamma_H \vdash \code{NI} \st \code{IG}$),
    we proved the first part of~\eq{assign4} that~$\Gamma_H \vdash [\code{NI}/\hI]\code{IP} \st \Raw$.
  If~$\code{IG}=\Mutable$ then,
    $\code{NI}=\Mutable$, which proved~\eq{assign4}.
  Therefore~$\code{IG}=\Raw$, and from the definition of~$\isTransitive$ we have
    that~$\code{e=\this{}}\text{~or~~}\code{MO}=\This$.
  If~$\he=\this$ then~$\isTransitive(\hl,\ldots)$, which proved~\eq{assign4}.
  Thus~$\code{MO}=\This$, and
  \beqs{assign5}
    \code{D<MO,IP>} &= \code{D<This,I>}\\
    [\hl/\This,\code{NI}/\hI,\code{NO}/\hO]\code{D<MO,IP>} &= \code{D<\hl,\code{NI}>}\\
  \eeq
  From~\eq{invoke-substitution1},~$H[\hl] = \code{C<NO,NI>}$.
  If~$\code{NI}=\Mutable$ then we proved~\eq{assign4}.
  Otherwise~$\code{NI}=\Immut_{\hl'}$.
  Because $H$ is well-typed, $\hl' \Tprec \hl$, thus~$\hl' \not \OprecNotEqual \hl$,   proving~\eq{assign4}
    (because if~$\code{a} \OprecNotEqual \code{b}$ then~$\code{b} \TprecNotEqual \code{a}$).


  \textbf{(i)~}If~$\he=\this$.
    Let~$\ftype{}(\hf,\code{D})=\code{FT}$.
    We assumed in~\eq{assign0} that~$\Gamma \vdash \this.f:\hT$.
    Then,~$\hT=\code{FT}$.
    From \RULE{T-Field-Access} \[
        \Gamma_H \vdash \code{l.f} : [\hl/\This,\code{NI}/\hI,\code{NO}/\hO]\code{FT}
    \]
    From definition of~$\isTransitive{}$, we have that~$\isTransitive{}(\hl,\ldots)$ holds.
    From \eq{assign0} ($\Gamma \vdash \hS \st \hT$) and \Ref{Lemma}{subtyping-substitute}, we have
    \beqs{this0}
        \Gamma_H \vdash [\hl/\This,\code{NI}/\hI,\code{NO}/\hO]\hS \st [\hl/\This,\code{NI}/\hI,\code{NO}/\hO]\hT
    \eeq
    From \eq{assign1}, \eq{this0}, and transitivity, we have
    \beqs{this1}
    \Gamma_H & \vdash \hS' \st [\hl/\This,\code{NI}/\hI,\code{NO}/\hO]\code{FT}
    \eeq
    To summarize, from \eq{this1}, \eq{assign4} ($\Gamma_H \vdash \code{NI} \st \Raw$), we have that
    \beqs{this2}
    &\Gamma_H \vdash \hl.\hf : [\hl/\This,\code{NI}/\hI,\code{NO}/\hO]\code{FT}
        \gap
    \Gamma_H \vdash \he':\hS'
        \gap
    \Gamma_H \vdash \hS' \st [\hl/\This,\code{NI}/\hI,\code{NO}/\hO]\code{FT}
        \\
    &\Gamma_H \vdash \hl:\code{C<NO,NI>}
        \gap
    \Gamma_H \vdash \code{NI} \st \Raw
        \gap
    \isTransitive(\hl,\ldots)
  \eeq
  Therefore, from~\eq{this2}, and \RULE{T-Field-Assignment}, we proved that \beqst
    \Gamma_H &\vdash \code{l.f=e}' : \hS' \\
    \Gamma_H &\vdash \hS' \st [\hl/\This,\code{NI}/\hI,\code{NO}/\hO]\code{FT}\\
  \eeq


  \textbf{(ii)~}If $\he\neq\this$, then from~\eq{assign4}, we have
  \beqs{not-this1}
  & \isTransitive(\bot,\Gamma,[\hl/\This,\code{NI}/\hI,\code{NO}/\hO]\code{D<MO,IP>})\\
  \eeq
    From~\eq{assign3} and~\eq{not-this1} and \Ref{Lemma}{isTransitive}, we have that~$\code{IP'}=[\code{NI}/\hI]\code{IP}$.
  To summarize, from \eq{assign1}, \eq{assign2}, \eq{this0}, \eq{not-this1},
  \beqs{not-this2}
        \Gamma_H & \vdash [\hl/\This,\code{NI}/\hI,\code{NO}/\hO, \ol{\hv}/\ol{\hx}, \hl/\this]\he.\hf : [\hl/\This,\code{NI}/\hI,\code{NO}/\hO]\hT \\
        \Gamma_H & \vdash [\hl/\This,\code{NI}/\hI,\code{NO}/\hO, \ol{\hv}/\ol{\hx}, \hl/\this]\he' : \hS' \\
        \Gamma_H & \vdash \hS' \st [\hl/\This,\code{NI}/\hI,\code{NO}/\hO]\hS\\
        \Gamma_H & \vdash [\hl/\This,\code{NI}/\hI,\code{NO}/\hO]\hS \st [\hl/\This,\code{NI}/\hI,\code{NO}/\hO]\hT\\
        \Gamma_H & \vdash [\hl/\This,\code{NI}/\hI,\code{NO}/\hO, \ol{\hv}/\ol{\hx}, \hl/\this]\he : \code{D'<MO,IP'>} \\
        \Gamma_H & \vdash \code{IP'} \st \Raw\\
        & \isTransitive(\bot,\Gamma,\code{D'<MO,IP'>})
    \gap
    \code{MO} \neq \code{?}\\
  \eeq
  Thus, from~\eq{not-this2}, and \RULE{T-Field-Assignment},  we proved that \beqst
    \Gamma_H &\vdash \code{e.f=e}' : \hS' \\
    \Gamma_H &\vdash \hS' \st [\hl/\This,\code{NI}/\hI,\code{NO}/\hO]\code{S}\\
  \eeq


  \item[$\he"=(\he_0\code{.m}\hparen{\ol{\he}})$]
    The proof is similar in spirit to field assignment:
        the challenge is that both $\he_0$ and $\he_i$ change covariantly.
    Let~$\code{IG'}$ be the guard of \hm.
    If~$\code{IG'} = \ReadOnly$ then the parameters of \hm cannot include~\hI.
    If~$\code{IG'} = \Mutable ~|~ \Immut$, then~\hI remains with the same bound.
    The challenge is when~$\code{IG'} = \Raw$, then we use either the fact the~$\he_0$ is either \this or \this-owned,
        to prove that $\he_0$ is invariant (like in field assignment).

    With respect to wildcards, if the receiver $\he_0$ has a wildcard, then after the covariant change
        it might no longer be the case.
    Therefore we require that the owner of method parameters in this case must be \World.
        (it cannot be \hO nor \This).


    From \RULE{T-Invoke},
    \beq{invoke1}
    \typerule{
  \Gamma \vdash \he_0:\code{D<MO,IP>}
    \gap
  \mtype{}(\he_0,\hm,\code{D<MO,IP>})=\ol{\code{T}}\rightarrow\code{W}
    \gap
  \Gamma \vdash \ol{\he}:\ol{\code{T'}}
    \gap
  \Gamma \vdash \ol{\code{T'}} \st \ol{\code{T}}
    \gap
  \mguard{}(\hm,\code{D})=\code{IG'}
    \\
  \Gamma \vdash \code{IP} \st \code{IG'}
    \gap
  \code{IG'}=\Raw \Rightarrow \isTransitive(\he_0,\Gamma,\code{D<MO,IP>})
    \gap
  \mtype{}(\hm,\code{D})=\ol{\code{U}}\rightarrow\code{V}
    \gap
  \Ofn{\ol{\code{T}}}=\code{?} \Rightarrow \Ofn{\ol{\code{U}}}=\code{?}
}{
  \Gamma \vdash \he_0\code{.m(}\ol{\he}\code{)} : \code{W}
}
\eeq

By induction on~$\he_0$, we have that
    \beqs{invoke3}
        \Gamma_H & \vdash [\hl/\This,\code{NI}/\hI,\code{NO}/\hO, \ol{\hv}/\ol{\hx}, \hl/\this]\he_0 : \code{D'<MO',IP'>} \\
        \Gamma_H & \vdash \code{D'<MO',IP'>} \st [\hl/\This,\code{NI}/\hI,\code{NO}/\hO]\code{D<MO,IP>}\\
        \eeq
Note that, in contrast with field assignment, here we might have~$\code{MO}=\code{?}$, and then~$\code{MO'}\neq\code{MO}$.
However, from \Ref{Lemma}{Owners-Invariant} part (i),
\beq{invoke-owner-wildcard}
    \code{MO}\neq\code{?} \Rightarrow \code{MO'}=[\hl/\This,\code{NO}/\hO]\code{MO}
\eeq

By induction on~$\he_i$, we have that
    \beqs{invoke4}
        \Gamma_H & \vdash [\hl/\This,\code{NI}/\hI,\code{NO}/\hO, \ol{\hv}/\ol{\hx}, \hl/\this]\he_i : \hS_i' \\
        \Gamma_H & \vdash \hS'_i \st [\hl/\This,\code{NI}/\hI,\code{NO}/\hO]\code{T'}_i\\
        \eeq
From~\eq{invoke4} and~\eq{invoke1} and \Ref{Lemma}{subtyping-substitute}, we have
    \beqs{invoke5}
        &\Gamma_H \vdash [\hl/\This,\code{NI}/\hI,\code{NO}/\hO]\code{T'}_i \st [\hl/\This,\code{NI}/\hI,\code{NO}/\hO]\hT_i\\
    \eeq
From transitivity,~\eq{invoke4}, and~\eq{invoke5},
    \beqs{invoke6}
        &\Gamma_H \vdash \hS'_i \st [\hl/\This,\code{NI}/\hI,\code{NO}/\hO]\hT_i\\
    \eeq


Because method overriding maintains the same signature,
    we have that
\beqs{invoke-signature1}
\mtype{}(\hm,\code{D'})=\mtype{}(\hm,\code{D})=\ol{\code{U}}\rightarrow\code{V}
\eeq
From definition of $\mtype$, \eq{invoke-signature1}, and because~$\he_0$ does not contain locations, we have that
\beqs{invoke-signature2}
\mtype{}(\he_0,\hm,\code{D<MO,IP>})=\ol{\code{T}}\rightarrow\code{W}=[\code{MO}/\hO, \code{IP}/\hI](\ol{\code{U}}\rightarrow\code{V})\\
\mtype{}([\hl/\this]\he_0,\hm,\code{D'<MO',IP'>})=[\hl/\This,\code{MO'}/\hO, \code{IP'}/\hI](\ol{\code{U}}\rightarrow\code{V})\\
\eeq
We will always prove that the parameters are invariant, i.e.,
\beq{invoke-signature-to-prove}
\mtype{}([\hl/\this]\he_0,\hm,\code{D'<MO',IP'>})=\ol{[\hl/\This,\code{NI}/\hI,\code{NO}/\hO]\hT_i}\rightarrow\code{W'}
\eeq
We will also prove that
\beq{invoke-to-prove}
    \Gamma_H \vdash \code{IP'} \st \code{IG'}
    \gap
  \code{IG'}=\Raw \Rightarrow \isTransitive([\hl/\this]\he_0,\Gamma,\code{D'<MO',IP'>})
\eeq
Because there are no wildcards after substitution, from~\eq{invoke-signature-to-prove} and~\eq{invoke-to-prove},
    we will have that
    \beqst
    \Gamma_H & \vdash [\hl/\This,\code{NI}/\hI,\code{NO}/\hO, \ol{\hv}/\ol{\hx}, \hl/\this]\he_0\code{.m(}\ol{\he}\code{)} : \code{W}' \\
    \Gamma_H & \vdash \code{W}' \st [\hl/\This,\code{NI}/\hI,\code{NO}/\hO]\code{W}\\
    \eeq


Next we prove~\eq{invoke-signature-to-prove} just for the owner parameter, i.e., we want to show that (from~\eq{invoke-signature2} and~\eq{invoke-signature-to-prove})
\beq{invoke-owner0}
    \Ofn{[\hl/\This,\code{MO'}/\hO]\code{U}_i} = \Ofn{[\hl/\This,\code{NO}/\hO]\hT_i}
\eeq
From~\eq{invoke-signature2},
\beq{invoke-owner1}
    \Ofn{\hT_i}=\Ofn{[\code{MO}/\hO]\code{U}_i}
\eeq
If~$\Ofn{\code{U}_i}=\This$, then both sides of \eq{invoke-owner0} are~\hl.
If~$\Ofn{\code{U}_i}\neq\hO$, then both sides of \eq{invoke-owner0} are~$\Ofn{\code{U}_i}$.
The last case is that~$\Ofn{\code{U}_i}=\hO$.
From~\eq{invoke1}, we have
\beq{invoke-owner2}
    \Ofn{\code{T}_i}=\code{?} \Rightarrow \Ofn{\code{U}_i}=\code{?}
\eeq
On the one hand, if~$\code{MO}=\code{?}$, then~$\Ofn{\code{T}_i}=\code{?}$, thus~$\Ofn{\code{U}_i}=\code{?}$,
    and both sides of \eq{invoke-owner0} are~$\code{?}$.
On the other hand, if~$\code{MO}\neq\code{?}$, then from \eq{invoke-owner-wildcard} %\code{MO}\neq\code{?} \Rightarrow \code{MO'}=[\hl/\This,\code{NO}/\hO]\code{MO}$.
    \[
    \code{MO'}=[\hl/\This,\code{NO}/\hO]\code{MO}
    \]
which proves~\eq{invoke-owner0}.

From \eq{invoke-owner0},
    in order to prove~\eq{invoke-signature-to-prove}, we just need to show it for the immutability parameter,
    i.e., (from \eq{invoke-signature2})
\beq{invoke-immutability}
\Ifn{[\code{IP'}/\hI]\code{U}_i}=\Ifn{[\code{NI}/\hI]([\code{IP}/\hI]\code{U}_i)}
\eeq
Recall the following:
From~\eq{invoke3}, we know that
        $\Gamma_H \vdash \code{D'<MO',IP'>} \st [\hl/\This,\code{NI}/\hI,\code{NO}/\hO]\code{D<MO,IP>}$.
From~\eq{invoke1},~$\Gamma \vdash \code{IP} \st \code{IG'}$.
From~\eq{invoke-substitution1},~$\Gamma_H \vdash \code{NI} \st \code{IG}$
    and $\hI:\code{IG} \in \Gamma$.

We will split the proof by the four possible values of $\code{IG'} = \ReadOnly ~|~ \Mutable ~|~ \Immut ~|~ \Raw$.
For each case we need to prove~\eq{invoke-to-prove} and~\eq{invoke-immutability}.
%\Gamma_H \vdash \code{IP'} \st \code{IG'}
%    \gap
%  \code{IG'}=\Raw \Rightarrow \isTransitive([\hl/\this]\he_0,\Gamma,\code{D'<MO',IP'>})

\textbf{(i) $\code{IG'} =\ReadOnly$}
Because any immutability is a subtype of \ReadOnly, we proved \eq{invoke-to-prove}.
Furthermore, when $\code{IG'} =\ReadOnly$ the signature of parameters cannot contain~\hI,
    i.e., $\Ifn{\code{U}_i}\neq\hI$, which proved~\eq{invoke-immutability}.

\textbf{(ii) $\code{IG'} =\Mutable$}
From~\eq{invoke1},~$\Gamma \vdash \code{IP} \st \Mutable$,
    thus either $\code{IP}=\Mutable$ or~($\code{IG}=\Mutable$ and~$\code{IP}=\hI$).
If $\code{IP}=\hI$, then from~\eq{invoke-substitution1},~$\Gamma_H \vdash \code{NI} \st \Mutable$, thus~$\code{NI} = \Mutable$.
Thus,~$\Gamma_H \vdash [\code{NI}/\hI]\code{IP} \st \Mutable$.
Therefore, from~\eq{invoke3} and \Ref{Lemma}{Guard} part (i), we have that~$\Gamma_H \vdash \code{IP'} \st \Mutable$,
    which proved \eq{invoke-to-prove}.

We showed that if $\code{IP}=\hI$, then~$\code{NI} = \Mutable$ and~$\code{IP'}=\Mutable$, proving~\eq{invoke-immutability}.
We also showed that if $\code{IP}=\Mutable$ then~$\code{IP'}=\Mutable$, proving~\eq{invoke-immutability}.


\textbf{(iii) $\code{IG'} =\Immut$}
Exactly like part (ii), but we use \Ref{Lemma}{Guard} part (ii) instead of part (i),
    and ($\Immut_{\hl'}$ where~$\hl' \not\in H[K]$) instead of \Mutable.

\textbf{(iv) $\code{IG'} =\Raw$}
Exactly like in field assignment, we prove that:
\beqs{invoke2}
  & \Gamma_H \vdash [\code{NI}/\hI]\code{IP} \st \Raw\\
  & \isTransitive([\hl/\this]\he,\Gamma,[\hl/\This,\code{NI}/\hI,\code{NO}/\hO]\code{D<MO,IP>})\\
\eeq

If $\he=\this$, then $\code{D}=\hC$, $\code{IP}=\hI$ (because~$\Gamma \vdash \this:\code{D<O,I>}$)
    and~$\code{IP'} = \code{NI}$ (because~$\Gamma_H \vdash \hl:\code{D<NO,NI>}$),
    therefore, \[
\Ifn{[\code{NI}/\hI]\code{U}_i}=\Ifn{[\code{NI}/\hI]([\hI/\hI]\code{U}_i)}
\]
which proved \eq{invoke-immutability}.
Furthermore, because~$\Gamma \vdash \code{IP} \st \Raw$ (and~$\code{IP}=\hI$),
    we know that~$\code{IG} \st \Raw$.
Thus from~\eq{invoke-substitution1},~$\Gamma_H \vdash \code{NI} \st \Raw$.
Finally because \this was replaced with \hl, and $\isTransitive(\hl,\ldots)$ always holds,
    then we proved \eq{invoke-to-prove}.

If $\he\neq\this$, then from \eq{invoke2}, we have that
    $\isTransitive(\bot,\Gamma,[\hl/\This,\code{NI}/\hI,\code{NO}/\hO]\code{D<MO,IP>})$,
    thus from definition of $\isTransitive$ we have that~$\code{MO}\neq\code{?}$.
From \Ref{Lemma}{Guard} part (iii) and~\eq{invoke3}, we know that
        $\code{IP'} = [\code{NI}/\hI]\code{IP}$,
        which proved~\eq{invoke-immutability}.
Combined with~\eq{invoke2}, we have that~$\Gamma_H \vdash \code{IP'} \st \Raw$.
From \eq{invoke-owner-wildcard}, we have~$\code{MO'}=[\hl/\This,\code{NO}/\hO]\code{MO}$.
Therefore,~$\code{D'<MO',IP'>} = [\hl/\This,\code{NI}/\hI,\code{NO}/\hO]\code{D<MO,IP>}$,
    which proved \eq{invoke-to-prove}.
\end{description}

\end{proof}


\begin{Lemma}[part-subtype]
  \textbf{(Subtype preservation)}
    For every closed expression~$\he" \neq \hv$, and a heap~$H$ that is well-typed for~$\he"$,
        if $\Gamma_{H} \vdash \he" : \hT"$
        and~$H,\he" \rightarrow H',\grave{\he}$,
        then
        $\Gamma_{H'} \vdash \grave{\he}:\grave{\hT}$ and~$\Gamma_{H'} \vdash \grave{\hT} \st \hT"$.
\end{Lemma}
\begin{proof}
We prove by examining all possible reduction rules.
\begin{description}
  \item[Congruence for field access]
Consider the congruence rule for field access \[
    \typerule{H,\he \rightarrow H',\he'}{H,\he.\hf \rightarrow H',\he'.\hf}
\]
We assumed that $\Gamma_{H} \vdash \he.\hf : \hT"$ and
    by induction~$\Gamma_{H} \vdash \he:\hT$,~$\Gamma_{H'} \vdash \he':\hT'$ and~$\Gamma_{H'} \vdash \hT' \st \hT$.
Let~$\hT = \code{C<MO,IP>}$.
Because $\he \neq \this$ (cause \he is closed) and~$\he \neq \hl$ (cause a location cannot be reduced further),
    then field~$\hf$ is not \this-owned,
    and~$\hT" = \ftype{}(\bot,\hf,\hT)$.

Because~$\Gamma_{H'} \vdash \hT' \st \hT$,
    from \Ref{Lemma}{Owners-Invariant} part (iii),
    then~\code{C'} is a subtype of \code{C}.
Therefore $\fields(\code{C'})$ must contain the same field~$\hf$ which is not \this-owned.
Thus,~$\Gamma_{H'} \vdash \grave{\he}:\grave{\hT}$,
    where~$\grave{\hT} = \ftype{}(\bot,\hf,\hT')$.
The last thing we need to prove is that~$\Gamma_{H'} \vdash \grave{\hT} \st \hT"$,
    which follows from \Ref{Lemma}{subtyping-substitute2}.

  \item[Congruence for method receiver]
Consider the congruence rule for method receiver \[
    \typerule{H,\he_0 \rightarrow H',\he'_0}{H,\he_0\code{.m}\hparen{\ol{\he}} \rightarrow H',\he'_0\code{.m}\hparen{\ol{\he}}}
\]
Similarly to field access, because~$\he_0$ is not a location,
    then none of the parameters or return type of method~\hm is \this-owned.
Proving that~$\Gamma_{H'} \vdash \grave{\hT} \st \hT"$
    (i.e., the return type is preserved)
    is done similarly to field access,
    by noting that method overriding maintains the same return type.
    (The return type could also change covariantly and the proof would still hold.)
However, proving that~$\Gamma_{H'} \vdash \grave{\he}:\grave{\hT}$ is more challenging
    because:
    (i)~we need to show that~$\he'_0$ satisfies the guard, and
    (ii)~the type of method parameters after substitution can change covariantly (as opposed to FGJ, which is invariant).


We will first prove that~$\he'_0$ satisfies the guard.
We assumed that $\Gamma_{H} \vdash \he_0\code{.m}\hparen{\ol{\he}} : \hT"$ and
    by induction~$\Gamma_{H} \vdash \he_0:\hT$,~$\Gamma_{H'} \vdash \he'_0:\hT'$ and~$\Gamma_{H'} \vdash \hT' \st \hT$.
From \RULE{T-Invoke},
    we know that \[
  \Gamma_H \vdash \he_0:\code{C<MO,IP>}
    \gap
  \mguard{}(\hm,\code{C})=\code{IG}
    \gap
  \Gamma_H \vdash \code{IP} \st \code{IG}
    \gap
  \code{IG}=\Raw \Rightarrow \isTransitive(\he_0,\Gamma,\code{C<MO,IP>})
    \]
Because~$\he_0$ is neither \this nor \hl (because it was reduced),
    then~$\code{IG}=\Raw \Rightarrow \isTransitive(\bot,\Gamma,\code{C<MO,IP>})$.
Let~$\hT' = \code{C'<MO,IP'>}$ and~$\mguard{}(\hm,\code{C'})=\code{IG'}$.
Because~$\Gamma_{H'} \vdash \hT' \st \hT$,
    from \Ref{Lemma}{Owners-Invariant} part (iii),
    then~\code{C'} is a subtype of \code{C}.
From the restriction on method overriding with guards,~$\code{IG} \st \code{IG'}$.
From \Ref{Lemma}{Guard} part (iv),
    we have that~$\code{IP'} \st \code{IG}$.
From transitivity,~$\code{IP'} \st \code{IG'}$.

Next we show that~$\code{IG}=\Raw \Rightarrow \isTransitive(\he'_0,\Gamma,\code{C'<MO,IP'>})$.
If~$\code{IG}=\Raw$ then we showed that~$\isTransitive(\bot,\Gamma,\code{C<MO,IP>})$.
From \Ref{Lemma}{isTransitive} we have that~$\code{IP'}=\code{IP}$,
    thus~$\code{IG}=\Raw \Rightarrow \isTransitive(\he'_0,\Gamma,\code{C'<MO,IP'>})$.

Let
\beqst
\mtype{}(\he_0,\hm,\code{C<MO,IP>})=\ol{\code{U}}\rightarrow\code{V} \\
\mtype{}(\he'_0,\hm,\code{C'<MO,IP'>})=\ol{\code{U'}}\rightarrow\code{V'} \\
\Gamma \vdash \ol{\he}:\ol{\code{U"}}\\
\eeq
Because~$\Gamma \vdash \ol{\code{U"}} \st \ol{\code{U}}$,
    from \Ref{Lemma}{invokeSubtype}, we have that~$\Gamma \vdash \ol{\code{U"}} \st \ol{\code{U'}}$.

Therefore, all the assumptions in \RULE{T-Invoke} are fulfilled (the requirement for wildcards is fulfilled because all types are closed),
    and we proved that~$\Gamma_{H'} \vdash \grave{\he}:\grave{\hT}$.


  \item[Congruence for method argument]
    Trivial.

  \item[Congruence for new instance]
    Trivial.

  \item[Congruence for the rvalue of field assignment]
    Trivial.

  \item[Congruence for the receiver of field assignment]
Consider the congruence rule for the receiver of field assignment \[
    \typerule{H,\he \rightarrow H',\he'}{H,\code{e.f=e"} \rightarrow H',\code{e'.f=e"}}
\]
We assumed that $\Gamma_{H} \vdash \code{e.f=e"} : \hT"$ and
    by induction~$\Gamma_{H} \vdash \he:\hT$,~$\Gamma_{H'} \vdash \he':\hT'$ and~$\Gamma_{H'} \vdash \hT' \st \hT$.
We will show that~$\Gamma_{H} \vdash \code{e.f=e"} : \hT"$.
From \RULE{T-Field-Assignment}:
\[
  \Gamma \vdash \he.\hf : \code{F}
    \gap
  \Gamma \vdash \code{e"}:\code{T"}
    \gap
  \Gamma \vdash \code{T"} \st \code{F}
    \gap
  \Gamma \vdash \he:\code{C<MO,IP>}
    \gap
  \Gamma \vdash \code{IP} \st \Raw
    \gap
  \isTransitive(\he,\Gamma,\code{C<MO,IP>})
\]
We need to show that:
\[
  \Gamma \vdash \he'.\hf : \code{F'}
    \gap
  \Gamma \vdash \code{T"} \st \code{F'}
    \gap
  \Gamma \vdash \he':\code{C'<MO,IP'>}
    \gap
  \Gamma \vdash \code{IP'} \st \Raw
    \gap
  \isTransitive(\he,\Gamma,\code{C'<MO,IP'>})
\]
Because \he was reduced, we know it is not a location, so~$\isTransitive(\bot,\Gamma,\code{C<MO,IP>})$.
From \Ref{Lemma}{isTransitive}, we have that~$\code{IP}=\code{IP'}$.
Therefore~$\code{F'}=\code{F}$ (because $\ftype(\hf,\hC)=\ftype(\hf,\hC')$),
    and~$\isTransitive(\he,\Gamma,\code{C'<MO,IP'>})$.



  \item[Congruence for return~\RULE{K-c1}]
Consider the congruence rule for \code{e;return l} \[
\typerule{
  H"=H[K \mapsto H[K] \cup \{\hl\}] \gap H",\he \rightarrow \grave{H},\he' \gap H'= \grave{H}[K \mapsto \grave{H}[K] \setminus \{\hl\}]
}{
  H,\code{e;return l} \rightarrow H',\code{e';return l}
}
\]
We assumed that \[
    \Gamma_{H} \vdash \code{e;return l} : \hT"
    \gap
    \he"=\code{e;return l}
    \gap
    \grave{\he}=\code{e';return l}
    \gap
    \Gamma_{H"} \vdash \he:\hT
    \gap
    \Gamma_{\grave{H}} \vdash \he':\hT'
    \gap
    \Gamma_{\grave{H}} \vdash \hT' \st \hT
    \]
We need to prove that~$\Gamma_{H'} \vdash \grave{\he} :\grave{\hT}$ and~$\Gamma_{H'} \vdash \grave{\hT} \st \hT"$.

From \Ref{Lemma}{constructors}, we have~$H[K] = H'[K]$ and~$H"[K] = \grave{H}[K]$.
Thus~$H'[K] \cup \{\hl\} = \grave{H}[K]$.
According to \RULE{T-return}:
\[
\typerule{
  \Gamma_{H'}[K \mapsto \Gamma_{H'}[K] \cup \{ \hl \}] \vdash \he':\hT'
}{
  \Gamma_{H'} \vdash \code{e';return l} : \Gamma_{H'}(\hl)
}
\]
Because $H'[K] \cup \{\hl\} = \grave{H}[K]$, we have
    that~$\Gamma_{H'}[K \mapsto \Gamma_{H'}[K] \cup \{ \hl \}] =
    \Gamma_{\grave{H}}$.
Because~$\Gamma_{\grave{H}} \vdash \he':\hT'$, we proved that~$\Gamma_{H'} \vdash \code{e';return l} : \Gamma_{H'}(\hl)$,
    i.e.,~$\Gamma_{H'} \vdash \grave{\he} :\grave{\hT}$.

We still need to prove that~$\Gamma_{H'} \vdash \grave{\hT} \st \hT"$.
Because~$\grave{\hT}=\Gamma_{H'}(\hl)$ and~$\hT" = \Gamma_{H}(\hl)$
    then~$\grave{\hT}=\hT"$, and from reflexivity (\RULE{S2}) we have~$\Gamma_{H'} \vdash \grave{\hT} \st \hT"$.


  \item[Rule~\RULE{K-return}] Trivial

  \item[Rule~\RULE{K-New}]
  According to \RULE{K-New} \[
\typerule{
  \hl \not \in \dom(H)
    \gap
  \code{VI}' =
    \begin{cases}
    \Immut_\hl & \text{if~}\code{VI}=\Immut \text{~or~} (\code{VI}=\Immut_{\hc} \text{~and~} \hc \not \in H[K]) \\
    \code{VI} & \text{otherwise} \\
    \end{cases}
    \gap
  H'=H[\hl \mapsto \code{C<NO,VI'>}\hparen{\ol{\code{null}}}]
}{
  H,\code{new C<NO,VI>}\hparen{\ol{\hv}} \rightarrow H',\hl\code{.build}\hparen{\ol{\hv}}\code{;return l}
}
\]
We assumed that \[
    \Gamma_{H} \vdash \he" : \hT"
    \gap
    \he"=\code{new C<NO,VI>}\hparen{\ol{\hv}}
    \gap
    \grave{\he}=\hl\code{.build}\hparen{\ol{\hv}}\code{;return l}
    \]
We need to prove that~$\Gamma_{H'} \vdash \grave{\he} :\grave{\hT}$ and~$\Gamma_{H'} \vdash \grave{\hT} \st \hT"$.

From \RULE{T-New}
\beq{K-new1}
\typerule{
  \mtype{}(\bot,\code{build},\code{C<NO,VI>})=\ol{\code{U}}\rightarrow\code{Z}
    \gap
  \Gamma_{H} \vdash \ol{\hv}:\ol{\code{V}}
    \gap
  \Gamma_{H} \vdash \ol{\code{V}} \st \ol{\code{U}}
}{
  \Gamma_{H} \vdash \code{new C<NO,VI>(}\ol{\hv}\code{)} : \code{C<NO,VI>}
}
\eeq
Thus,~$\hT" = \code{C<NO,VI>}$.
From~\RULE{T-return},~$\grave{\hT} = \code{C<NO,VI'>}$.
Because~$\hl \not \in \Gamma_{H'}[K]$, then~$\Gamma_{H'} \vdash \Immut_\hl \st \Immut$,
    thus~$\Gamma_{H'} \vdash \grave{\hT} \st \hT"$.

We still need to prove that~$\Gamma_{H'} \vdash \grave{\he} :\grave{\hT}$,
    and from~\RULE{T-return} we need to prove that \[
    \Gamma_{H"} = \Gamma_{H'}[K \mapsto \Gamma_{H'}[K] \cup \{ \hl \}] \gap
    \Gamma_{H"} \vdash \hl\code{.build}\hparen{\ol{\hv}}:\code{Z}
    \]
Because~\hl is a new location, all the equations in~\eq{K-new1} are still true
    if we replace~$\Gamma_{H}$ with~$\Gamma_{H"}$.
From \RULE{T-invoke}, and because the guard of \code{build} is \Raw:
\[
\typerule{
  \Gamma_{H"} \vdash \hl:\grave{\hT}
    \gap
  \mtype{}(\hl,\code{build},\grave{\hT})=\ol{\code{W}}\rightarrow\code{Z'}
    \gap
  \Gamma_{H"} \vdash \ol{\hv}:\ol{\code{V}}
    \gap
  \Gamma_{H"} \vdash \ol{\code{V}} \st \ol{\code{W}}
    \\
  \Gamma_{H"} \vdash \code{VI'} \st \Raw
    \gap
  \isTransitive(\hl,\Gamma,\grave{\hT})
}{
  \Gamma_{H"} \vdash \hl\code{.build}\hparen{\ol{\hv}}:\code{Z'}
}
\]
Assumption~$\isTransitive(\hl,\Gamma,\grave{\hT})$ holds because~\hl is a location.
Because~$\hl \in \Gamma_{H"}[K]$ and \code{VI} is either \Mutable or \Immut or $\Immut_{\hl'}$,
    then this assumption holds~$\Gamma_{H"} \vdash \code{VI'} \st \Raw$.
The only assumption left to prove is that~$\Gamma_{H"} \vdash \ol{\code{V}} \st \ol{\code{W}}$.
From~\eq{K-new1} we have that~$\Gamma_{H"} \vdash \ol{\code{V}} \st \ol{\code{U}}$.
If $\code{VI}=\Mutable$ then~$\ol{\code{U}}=\ol{\code{W}}$.
Otherwise $\code{VI}=\Immut$, ~$\grave{\hT} = \code{C<NO,\Immut}_\hl\code{VI>}$,
    and we have that
\beqst
&    \mtype{}(\code{build},\hC) = \ol{\code{FT}}\rightarrow\code{Z"}\\
&  \mtype{}(\hl,\code{build},\code{C<NO,\Immut}_\hl\code{VI>})=\ol{\code{W}}\rightarrow\code{Z'}\\
&  \mtype{}(\bot,\code{build},\code{C<NO,\Immut>})=\ol{\code{U}}\rightarrow\code{Z}\\
&    \code{W}_i = [\code{NO}/\hO,\Immut_\hl/\hI]\code{FT}_i\\
&    \code{U}_i = [\code{NO}/\hO,\Immut/\hI]\code{FT}_i\\
&    \Gamma_{H"} \vdash \code{V}_i \st \code{U}_i\\
\eeq
We want to prove that~$\Gamma_{H"} \vdash \code{V}_i \st \code{W}_i$.
If~$\Ifn{\code{FT}_i}\neq\hI$ then~$\code{W}_i=\code{U}_i$.
Because~$\Ofn{\code{FT}_i}\neq\This$, then it is either \hO or \World,
    thus,~$\Owner{\code{W}_i}$ is either \code{NO} or \World.
Finally note that~$\hl \OprecNotEqual \code{NO}$ (because~$\hl \mapsto \code{C<NO,\ldots>}$),
    and we always have that~$\code{NO} \Oprec \World$, thus~$\hl \OprecNotEqual \World$.
According to subtyping rule~\RULE{S13}, we have that~$\Gamma_{H"} \vdash  \code{U}_i \st \code{W}_i$,
    and from transitivity~$\code{V}_i \st \code{U}_i \st \code{W}_i$.


  \item[Rule~\RULE{K-Field-Access}]
  According to \RULE{K-Field-Access} \[
\typerule{
  H[\hl] = \code{C<NO,NI>}\hparen{\ol{\hv}}
    \gap
  \fields{}(\hC)=\ol{\hf}
}{
  H,\hl.\hf_i \rightarrow H,\hv_i
}
\]
We assumed that~$\Gamma_{H} \vdash \hl.\hf_i : \hT"$, and
    we need to prove that~$\Gamma_{H} \vdash \hv_i :\grave{\hT}$ and~$\Gamma_{H} \vdash \grave{\hT} \st \hT"$.
If~$\hv_i=\code{null}$ then we can choose~$\grave{\hT} = \hT"$,
otherwise~$\hv_i\neq\code{null}$, and because the heap is well-typed, then~$\Gamma_{H} \vdash \grave{\hT} \st \hT"$.

  \item[Rule~\RULE{K-Field-Assignment}]
  According to \RULE{K-Field-Assignment} \[
\typerule{
  \ldots
}{
  H,\hl.\hf_i = \code{\hv'} \rightarrow H',\hv'
}
\]
We assumed that~$\Gamma_{H} \vdash \hl.\hf_i = \code{\hv'} : \hT"$, and
    we need to prove that~$\Gamma_{H'} \vdash \hv' :\grave{\hT}$ and~$\Gamma_{H'} \vdash \grave{\hT} \st \hT"$.
Because we did not add any new locations, we have~$\Gamma_{H} = \Gamma_{H'}$.
From \RULE{T-Field-Assignment}, we have that~$\Gamma_{H} \vdash \code{\hv'} : \hT"$, i.e.,~$\grave{\hT} = \hT"$.


  \item[Rule~\RULE{K-Invoke}]
According to \RULE{K-Invoke} \beqst
\typerule{
  H[\hl] = \code{C<NO,NI>}\hparen{\ldots}
    \gap
  \mbody{}(\hm,\code{C})=\ol{\hx}.\he'
}{
  H,\hl\code{.m(}\ol{\hv}\code{)} \rightarrow H, [\ol{\hv}/\ol{\hx}, \hl/\this, \hl/\This, \code{NO}/\hO, \code{NI}/\hI]\he'
}
\eeq
We assumed that \beqst
    \Gamma_{H} \vdash \he" : \hT"
    \gap
    \he"=\hl\code{.m(}\ol{\hv}\code{)}
    \gap
    \grave{\he}=[\ol{\hv}/\ol{\hx}, \hl/\this, \hl/\This, \code{NO}/\hO, \code{NI}/\hI]\he'
    \eeq
From \RULE{T-Invoke} we have that \beqst
  \mtype{}(\hl,\hm,\code{C<NO,NI>})=\ol{\hT}\rightarrow\hT"
    \gap
  \Gamma_{H} \vdash \ol{\hv}:\ol{\hT'}
    \gap
  \Gamma_{H} \vdash \ol{\hT'} \st \ol{\hT}
    \gap
  \mguard{}(\hm,\code{C})=\code{IG}
    \gap
  \Gamma_{H} \vdash \code{NI} \st \code{IG}
\eeq
We know the method~\hm was typed-checked in~\hC, i.e.,
\beqst
 \Gamma & =\{\hI:\code{IG}, \ol{\hx}:\ol{\code{U}}, \this:\code{C<O,I>}\} \\
 \mtype(\hm,\code{C}) & =\ol{\code{U}}\rightarrow\code{FT} \\
 \Gamma &\vdash \he':\code{S} \\
 \Gamma &\vdash \code{S} \st \code{FT}\\
\eeq
From the definition of \mtype:
\beqst
 \mtype{}(\hl,\hm,\code{C<NO,NI>})&=\substitute{}(\hl,\code{C<NO,NI>},\mtype(\hm,\code{C})) \\
 \hT" &=  [\code{NO}/\hO,\code{NI}/\hI,\hl/\This]\code{FT}\\
 \hT_i &=  [\code{NO}/\hO,\code{NI}/\hI,\hl/\This]\code{U}_i\\
\eeq

We need to prove that~$\Gamma_{H} \vdash \grave{\he} :\grave{\hT}$ and~$\Gamma_{H} \vdash \grave{\hT} \st \hT"$,
    which follows immediately from \Ref{Lemma}{invoke-substitution}.
\end{description}
\end{proof}




\begin{Lemma}[part-well-typed]
  \textbf{(Well-typed heap preservation)}
    For every closed expression~$\he" \neq \hv$, and a heap~$H$ that is well-typed for~$\he"$,
        if $\Gamma_{H} \vdash \he" : \hT"$
        and~$H,\he" \rightarrow H',\grave{\he}$,
        then
        $H'$ is well-typed for~$\grave{\he}$.
\end{Lemma}
\begin{proof}
Recall that a {well-typed} heap~$H$ satisfies:
    (i)~there is a linear order~$\Tprec$ over~$\dom{}(H)$ such that for every location~\hl,
        $\Owner{\hl}=\World$ or $\Owner{\hl} \TprecNotEqual \hl$, and $\Ifn{\hl}=\Mutable$ or $\Cooker{\hl} \Tprec \hl$,
        and
    (ii)~each non-null field location is a subtype of the declared field type.
Recall also that from the definition of a heap~$H$, every location~\hl in~$H$ has the form:
    (iii)~$\hl \mapsto \hC\code{<\underline{NO,NI}>}\hparen{\ol{\hv}}$.
Also recall that a heap~$H$ is {well-typed for~\he} if~$H[K \mapsto H[K]\cup R(\he)]$ is well-typed.

Consider the congruence rules, such as \[
    \typerule{H,\he \rightarrow H',\he'}{H,\he.f \rightarrow H',\he'.f}
\]
By the induction hypothesis~$H'$ is well-typed.

The only rule that changes~$H[K]$ is \RULE{K-c1}:
\[
\typerule{
  H' = H[K \mapsto H[K] \cup \{\hl\}] \gap
  H',\he \rightarrow H",\code{e'}
}{
  H,\code{e;return l} \rightarrow H"[K \mapsto H"[K] \setminus \{\hl\}],\code{e';return l}
}
\]
We need to prove that~$H"[K \mapsto H"[K] \setminus \{\hl\}]$ is well-typed for~\code{e';return l},
    or equivalently that~$H"$ is well-typed for~$\code{e'}$.
Because~$H$ is well-typed for~\code{e;return l},
    then~$H'$ is well-typed for~\code{e},
    and by induction~$H"$ is well-typed for~$\code{e'}$.
From \Ref{Lemma}{constructors}, we know that~$H"[K]=H'[K]$, thus~$\hl \in H"[K]$,
    and we have that~$H"$ is well-typed for~$\code{e';return l}$.
From \Ref{Lemma}{well-typed},~$H"[K \mapsto H"[K] \setminus \{\hl\}]$ is well-typed for~$\code{e';return l}$.

Rules \RULE{K-Field-Access} and \RULE{K-Invoke} do not change the heap.

Rule \RULE{K-return} does not change the heap nor~$H[K]$,
    however~$R(\grave{\he}) = R(\he") \setminus \{ \hl \}$,
According to \Ref{Lemma}{well-typed}, the resulting heap~$H'$
    is well-typed for~$\grave{\he}$.

Rule \RULE{K-New} creates a new object with \code{null} fields:
    \[
    \code{VI}' =
    \begin{cases}
    \Immut_\hl & \text{if~}\code{VI}=\Immut \text{~or~} (\code{VI}=\Immut_{\hc} \text{~and~} \hc \not \in H[K]) \\
    \code{VI} & \text{otherwise} \\
    \end{cases}
        \gap
    \hl \not \in \dom(H)
        \gap
    H' = H[\hl \mapsto \code{C<NO,VI'>}\hparen{\ol{\code{null}}}]
    \]
The fields of the new object are all \code{null}, thus fulfilling demand (ii).

We extend the linear order~$\Tprec$ by adding the new location~\hl at the end.
Its owner~\code{NO} is either \World or an existing object~$\hl'$,
    and either $\code{VI}=\Mutable$ or $\code{VI}=\Immut_{\hl'}$
    (where~$\hl'$ is either an existing location or \hl),
    thus fulfilling demand (i).

Note that~$\he" = \code{new C<\underline{NO,VI}>}\hparen{\ol{\hv}}$, and because~$\he"$ is closed,
    then we have that~$\code{NO}$ and~$\code{VI}$ do not contain~\hO, \hI, nor~\This.
And if~$\code{VI}=\Immut$ then it is substituted with~$\Immut_\hl$, thus fulfilling demand (iii).
Finally, note that~$\grave{\he}=(\hl\code{.build}\hparen{\ol{\hv}}\code{;return l})$,
    i.e.,~$R(\grave{\he}) = R(\he") \cup \{ \hl \}$.
However, because~\hl is a \emph{new} location, it does not change existing subtype relations (it does not affect existing objects
    that do not refer to~\hl).
Therefore, $H'$ is well-typed for~$\grave{\he}$.

Finally, in rule \RULE{K-Field-Assignment},~$\he" = \hl.\hf_i = \code{\hv'}$,
    $\grave{\he} = \code{\hv'}$,
    and~$H' = H[\hl \mapsto \code{C<NO,NI>(}[\hv'/\hv_i]\ol{\hv}\code{)}]$.
Note that~$R(\he") = R(\grave{\he}) = \{\}$, thus a if~$H'$ is well typed then it is well-typed for~$\grave{\he}$.
Because the typing rule \RULE{T-Field-Assignment} require that: \[
  \Gamma_H \vdash \hl.\hf : \hT
    \gap
  \Gamma_H \vdash \code{\hv'}:\code{T'}
    \gap
  \Gamma_H \vdash \code{T'} \st \hT
    \]
    then the heap~$H'$ is well-typed.
\end{proof}



\bibliographystyle{abbrv}
\bibliography{bibstring-abbrev,igj-extra,alex,names,nicks,oopsla,ecoop}


\end{document}
