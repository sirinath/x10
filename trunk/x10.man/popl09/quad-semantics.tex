\def\has{\mbox{\tt has}}
\def\inv{\mathit{inv}}

\def\FX{{\sf FX}}
\def\FXZ{{\sf FX()}}
\def\FXG{{\sf FX(G)}}
\def\FGJ{{\sf FGJ}}
\def\FXD{{\sf FX(D($\cal A$))}}
\def\FXGD{{\sf FX(G,D($\cal A$))}}
\def\FXGDP{{\sf FX(G,D($\cal A$),P)}}
\def\has{\mbox{\tt has}}
\def\TConstr{\mbox{\sc T-Constr}}
\def\TInv{\mbox{\sc T-Inv}}
\def\TVar{\mbox{\sc T-Var}}
\def\TField{\mbox{\sc T-Field}}
\def\TInvk{\mbox{\sc T-Invk}}
\def\TNew{\mbox{\sc T-New}}
\def\TCast{\mbox{\sc T-Cast}}
\def\TUCast{\mbox{\sc T-UCast}}
\def\TDCast{\mbox{\sc T-DCast}}
\def\TSCast{\mbox{\sc T-SCast}}

\def\RField{\mbox{\sc R-Field}}
\def\RCField{\mbox{\sc RC-Field}}
\def\RInvk{\mbox{\sc R-Invk}}
\def\RCInvkRecv{\mbox{\sc RC-Invk-Recv}}
\def\RCInvkArg{\mbox{\sc RC-Invk-Arg}}
\def\RCNewArg{\mbox{\sc RC-New-Arg}}
\def\RCast{\mbox{\sc R-Cast}}
\def\RCCast{\mbox{\sc RC-Cast}}

We now describe the semantics of languages in the \FX{} family.

Each language $\cal L$ in the family is defined over a given input constraint
system $\cal X$. Given a program $P$, we now show how to build a
larger constraint system ${\cal O}({\cal X})$ on top of $\cal X$ which
captures constraints related to the object-oriented structure of
$P$. $\cal O$ is sensitive to $\cal X$ only in that $\cal O$ depends
on the types defined by $\cal L$, and these may depend on $\cal X$.

The static and dynamic semantics of $\cal L$ rests on ${\cal O}({\cal
X})$.

\subsection{The Object constraint system, $\cal O$}\label{sec:O}

Given $\cal L$, its input constraint system $\cal X$ we now show
how to define $\cal O$. The inference relation for $\cal O$ depends
on the object-oriented structure of the input program $P$ in $\cal L$.
For some members of $\cal L$, viz.{} the generic
languages, $\cal X$ itself may use some of the constraints defined by
$\cal O$. Thus we should think of $\cal X$ and $\cal O$ as being
defined simultaneously and recursively.

The constraints of $\cal O$ are given by:

\begin{tabular}{r@{\quad}rcl}
(Member) & {\tt I}&{::=}& $\tt m(\bar{\tt x}:\bar{\tt V})\{c\}:T=e$ \alt {\tt f:V}\\
(Const.) & {\tt c},{\tt d} &{::=}& \klass({\tt C}) \alt ${\tt S}\ \extends\ {\tt T}$ \alt ${\tt S}\ \subtype\ {\tt T}$\\
&&& \alt $\fields({\tt x})= \bar{\tt f}:\bar{\tt V}$ \\
&&& \alt {\tt x\ \has\ I}
\end{tabular}

$\klass({\tt C})$ is intended to be true for all classes {\tt C}
defined in the program. ${\tt S}\ \extends\ {\tt T}$ is intended to
hold if it can be established that {\tt S} extends {\tt T}, for
instance if {\tt S} is a class that extends {\tt T}, or if {\tt S} is
a type variable and {\tt T} its upper bound. ${\tt S}\ \extends\ {\tt
T}$ is intended to hold if {\tt S} is a subtype of {\tt T}.  For a
variable {\tt x}, {\tt \fields({\tt x})} is intended to specify the
(complete) set of typed fields available to {\tt x}.  {\tt x\ \has\ I}
is intended to specify that the member {\tt I} (field or method) is
available to {\tt x}---for instance it is defined at the class at
which {\tt x} is declared or inherited by it, or it is available at
the upper bound of a type variable.


$\cal O$ satisfies the following axioms and inference rules in
Figure~\ref{fig:O}. Since $\cal O$ is a constraint system \cite{cccc},
it also satisfies Identity and Cut.
%% Well, it isnt really since its assumptions are ordered??!

Note that some rules (viz, {\sc S-Exists} rules)
use $\vdash$. 

We assume that the rules given are complete for
defining the predicates $\tt C \subtype D$ and {\tt C\ \has\ I}, for
classes {\tt C}, {\tt D} and members {\tt I}; that is, if the rules
cannot be used to establish $\tt \vdash_{\cal O} C \subtype D$ ($\tt
\vdash_{\cal O} C\ \has\ I$), then it is the case that $\tt
\vdash_{\cal O} C \notsubtype D$ ($\tt \vdash_{\cal O} \neg (C\ \has\
I)$).

Such negative facts are important to establish {\em inconsistency} of
assumptions (for instance, for the programming languages which permits
the user to state constraints on type variables).  

\begin{figure*}
{\footnotesize
\tabcolsep=0pt
\begin{tabular}{p{0.30\textwidth}p{0.24\textwidth}p{0.23\textwidth}p{0.22\textwidth}}
\infrule[Class]
	{\tt \klass\ {\tt C}(\bar{\tt f}:\bar{\tt V})\ {\tt extends}\ {\tt D}\ \ldots \in P}
	{\tt \vdash_{\cal O} \klass({\tt C}),{\tt C} \extends {\tt D}}
&
\infrule[Has-F]
	{\Gamma\vdash_{\cal O} \fields({\tt x})=\bar{\tt f}:\bar{\tt V}} 
	{\Gamma \vdash_{\cal O} {\tt x}\ \has\ {\tt f_i}:{\tt V_i}}
&
\infax[Sel]{\tt \vdash_{\cal O} new\ D(\bar{\tt t}).{\tt f_i=t_i}}
&
\infrule[Inv]
{\Gamma \vdash_{\cal O} {\tt x:C},\klass({\tt C})}
{\Gamma \vdash_{\cal O} \inv({\tt C}, {\tt x})}
\\
\end{tabular}\\[-12pt]
\begin{tabular}{p{0.30\textwidth}p{0.33\textwidth}p{0.36\textwidth}}
\infax[V-Id]{\vdash_{\cal O} T \extends T} 
&
\infrule[Sub-X]
	{\Gamma \vdash_{\cal O} {\tt t}\ \extends\ {\tt T}}
	{\Gamma \vdash_{\cal O} {\tt t} \subtype {\tt T}}
&
\infrule[S-Trans]
	{\Gamma\tt \vdash_{\cal O} T_1 \subtype T_2, T_2 \subtype T_3}
	{\Gamma\tt \vdash_{\cal O} T_1 \subtype T_3}\\[-12pt]
\infrule[S-Const-L]
	{\Gamma\tt,  {\tt c} \vdash_{\cal O} {\tt S}\; \subtype\; {\tt T}}
	{\Gamma\tt  \vdash_{\cal O} {\tt S\{c\}}\;\subtype\; {\tt T}}
&
\infrule[S-Const-R]
	{\Gamma\tt  \vdash_{\cal O} {\tt S}\;\subtype\; {\tt T}\andalso\Gamma,{\tt \self:S} \vdash_{\cal O} {\tt c} }
	{\Gamma\tt  \vdash_{\cal O} {\tt S}\;\subtype\; {\tt T\{c\}}}
&
\infrule[S-Extends]
	{\tt class\ C(\ldots)\ extends\ D\{\ldots\}\in P}
	{\tt\vdash_{\cal O} C \subtype D}
\\[-12pt]
\infrule[S-Exists-L]
	{\Gamma\tt  \vdash {\tt U}\ \type \andalso  
	  \Gamma \vdash_{\cal O} {\tt S}\; \subtype\; {\tt T}\andalso
          \mbox{({\tt x} fresh})}
	{\Gamma\tt  \vdash_{\cal O} {\tt x:U;S}\;\subtype\; {\tt T}}
&
\infrule[S-Exists-R]
	{\Gamma\tt  \vdash t:{\tt U} \andalso \Gamma \vdash_{\cal O} {\tt S}\; \subtype\; {\tt T}[{\tt t}/{\tt x}]}
	{\Gamma\tt  \vdash_{\cal O} {\tt S}\;\subtype\; {\tt x;U:T}}
&
\infrule[Type-equiv]
	{\Gamma\tt  \vdash_{\cal O} S\; \subtype\; T \andalso \Gamma \vdash_{\cal O} T\; \subtype\; S}
	{\Gamma\tt  \vdash_{\cal O} S \equiv T} \\[-12pt]
\infax[Fields-B]{{\tt x}:{\tt Object} \vdash_{\cal O} \fields({\tt x})=\bullet}
&
\infrule[Fields-I]
	{\Gamma, {\tt x:D} \vdash_{\cal O} \fields({\tt x})=\bar{\tt g}:\bar{\tt V}\andalso \\
	  \klass\ {\tt C}(\bar{\tt f}:\bar{\tt U}){\tt \{c\}}\ {\tt extends}\ {\tt D}\{\bar{\tt M}\} \in {\tt C}}
	{\Gamma, {\tt x:C} \vdash \fields({\tt x})=\bar{\tt g}:\bar{\tt V},\bar{\tt f}:\bar{\tt U[x/\this]}}
&
\infrule[Fields-C,E]
	{\Gamma, {\tt x: S} \vdash_{\cal O} \fields({\tt x})=\bar{\tt f}:\bar{\tt V}}
	{\Gamma, {\tt x:S\{d\}} \vdash_{\cal O} \fields({\tt x})=\bar{\tt f}:\bar{\tt V\{d[x/\self]\}} \\
	  \Gamma, {\tt x: (y:U;S)} \vdash_{\cal O} \fields({\tt x})=\bar{\tt f}:\bar{\tt (y:U;V)}
	} \\[-12pt]
\infrule[Method-B]
	{\Gamma, {\tt x}:{\tt C}\vdash_{\cal O}\klass({\tt C})\andalso \theta=[{\tt x}/\this] \andalso \\
	  {\tt def}\ {\tt m(\bar{\tt z}:\bar{\tt V})\{c\}:T=e}\in P}
	{\Gamma\tt, {\tt x:C}\vdash_{\cal O} {\tt x}\ \has\ (m(\bar{\tt z}:\bar{\tt V\theta})\{c\theta\}:T\theta=e)}
&
\infrule[Method-I]
	{\Gamma\tt, {\tt x}:{\tt D}\vdash_{\cal O}{\tt x}\ \has\ m(\bar{\tt z}:\bar{\tt V}){\tt \{c\}:T=e} \\
	  \tt \andalso \klass\ {\tt C}(\ldots)\ {\tt extends}\ {\tt D}\{\bar{\tt M}\} \andalso {\tt m}\not\in \bar{\tt M}}
	{\Gamma\tt, {\tt x:C}\vdash_{\cal O} {\tt x}\ \has\ {\tt m}(\bar{\tt z}:\bar{\tt V}){\tt \{c\}:T=e}}
&
\infrule[Method-C,E]
	{\Gamma\tt, x:S \vdash_{\cal O} x\ \has\ m(\bar{\tt z}:\bar{\tt V})\{c\}:T=e} 
	{\Gamma\tt, x:S\{d\} \vdash_{\cal O} x\ \has\ m(\bar{\tt z}:\bar{\tt V})\{c\}:T\{d[x/\self]\}=e \\
	  \Gamma\tt, x: (y:U;S) \vdash_{\cal O} x\ \has\ m(\bar{\tt z}:\bar{\tt V})\{c\}:(y:U;T)=e}
\end{tabular}}
\caption{The Object constraint system, $\cal O$}\label{fig:O}  
\end{figure*}



\begin{figure*}
\footnotesize
\begin{tabular}{l}
{\bf \FX{} productions:} \\
\quad\\
\begin{tabular}{ll}
\begin{tabular}{r@{\quad}r@{~~}c@{~~}l}
  (Class) & {\tt L} &{::=}& \klass\ {\tt C}($\bar{\tt f}:\bar{\tt T}$)\{{\tt c}\}\ {\tt extends}\ {\tt N}\ \{$\bar{\tt M}$\} \\
  (Method)& {\tt M} &{::=}& {\tt def}\ {\tt m}($\bar{\tt x}:\bar{\tt T}$){\tt \{c\}:T\,=\,e;}\\
  (Exp.)& {\tt e} &{::=}& {\tt x} \alt \this \alt {\tt e.f} \alt {\tt e.m}($\bar{\tt e}$) 
   \alt \new\ {\tt C}($\bar{\tt e}$) \alt {\tt e}\ \mbox{{\tt as} {\tt T}} \\ 
\end{tabular} 
&
\begin{tabular}{r@{\quad}r@{~~}c@{~~}l}
  (Type)& {\tt S},{\tt T},{\tt U},{\tt V}&{::=}& {\tt N}
        \alt {\tt T\{c\}} \alt {\tt x}:{\tt S};\,{\tt T}\\
  (N Type) & {\tt N}&{::=}& {\tt C} \alt {\tt N\{c\}}\\
  (C Term) & {\tt t} &{::=}& {\tt x} \alt \self \alt \this \alt
  {\tt t}.{\tt f} \alt \new\ {\tt C}($\bar{\tt t}$) \alt $\tt f(\bar{\tt t})$ \\
  (Const.) & {\tt c},{\tt d} &{::=}&\true \alt {\tt t==t} \alt {\tt c},{\tt c} \alt {\tt x}:{\tt T};\,{\tt c} \alt ${\tt q(\bar{\tt t})}$
\end{tabular} 
\end{tabular}
\quad \\
\quad\\
{\bf Constraint well-formedness rules:}\\[-12pt]
{\tabcolsep=0pt
\begin{tabular}{p{0.26\textwidth}p{0.33\textwidth}p{0.38\textwidth}}
\infax[True]{\Gamma \vdash {\tt true}: {\tt o}}
&
\infrule[And]
	{\Gamma\tt  \vdash c_0: o \andalso \Gamma \vdash c_1: o}
	{\Gamma\tt \vdash (c_0,c_1):o}
&
\infrule[Exists]
	{\Gamma\tt \vdash t: T \andalso \Gamma \vdash c[t/x]:o}
	{\Gamma\tt \vdash x:T;c: o}
\\
\infrule[Pred]
        {\tt q(\bar{\tt T}):{\tt o}\in {\cal C} \andalso \Gamma \vdash \bar{\tt t}:\bar{\tt T}}
        {\Gamma \vdash \tt q(\bar{\tt T}):o}
&
\infrule[Fun]
        {\tt f(\bar{\tt T}):{\tt T}\in {\cal C} \andalso \Gamma \vdash \bar{\tt t}:\bar{\tt T}}
        {\Gamma\tt \vdash \tt f(\bar{\tt T}):T}
&
\infrule[Equals]
        {\Gamma \vdash {\tt t}_0: {\tt T}_0
        \andalso
        \Gamma \vdash {\tt t}_1: {\tt T}_1 \\
          (\Gamma \vdash {\tt T}_0 \subtype {\tt T}_1 \vee
          \Gamma \vdash {\tt T}_1 \subtype {\tt T}_0)}
        {\Gamma \vdash {\tt t}_0{\tt ==}{\tt t}_1:o}
\\
\end{tabular}}
%\quad\\[-12pt]
\quad\\
{\bf Type well-formedness rules:}\\[-12pt]
{\tabcolsep=0pt
\begin{tabular}{p{0.26\textwidth}p{0.33\textwidth}p{0.38\textwidth}}
\infrule[Class]
	{\Gamma \vdash \klass({\tt C})}
	{\Gamma \vdash {\tt C}\ \type} 
&
\infrule[Exists-T]
	{\Gamma\tt \vdash {\tt S}\ \type, {\tt T}\ \type}
	{\Gamma\tt \vdash {\tt x:S;T}\ {\tt type}} 
&
\infrule[Dep]
	{\Gamma\tt \vdash T\ \type \andalso \Gamma, \self:T \vdash c:o}
	{\Gamma\tt \vdash T\{c\}\ {\tt type}} 
\\
\end{tabular}}
\quad\\
\quad\\
{\bf Type judgment rules:}\\[-12pt]
{\tabcolsep=0pt
\begin{tabular}{l}
\begin{tabular}{p{0.27\textwidth}p{0.28\textwidth}p{0.42\textwidth}}
\infax[T-Var]
      {\Gamma, {\tt x}:{\tt T} \vdash {\tt x}:{\tt T\{\self==x\}}}
&
\infrule[T-Cast]
	{\Gamma \vdash {\tt e}:{\tt U} \andalso \Gamma \vdash {\tt T} \ \type}
	{\Gamma \vdash {\tt e}\ \as\ {\tt T}: {\tt T}} &
\infrule[T-Field]
	{\Gamma \vdash {\tt e}: {\tt S} \andalso \Gamma,{\tt z}:{\tt S}\vdash {\tt z}\ \has\ {\tt f}:{\tt T} \andalso \mbox{({\tt z} fresh)} }
	{\Gamma \vdash {\tt e}.{\tt f}:  ({\tt z}:{\tt S};\,{\tt T\{\self==z.f\}})}
\end{tabular}\\[-12pt]
\begin{tabular}{p{0.50\textwidth}p{0.47\textwidth}} 
\infrule[T-Invk]
	{\Gamma \vdash {\tt e}:{\tt T},\bar{\tt e}:\bar{\tt V} \andalso \\
	  \Gamma,{\tt z}:{\tt T},\bar{\tt z}:\bar{\tt V} \vdash
		{\tt z}\ \has\ ({\tt m}(\bar{\tt z}:\bar{\tt U}), {\tt c} \rightarrow {\tt S}), 
		\bar{\tt V} \subtype \bar{\tt U},{\tt c} 
	  \andalso \mbox{({\tt z},$\bar{\tt z}$ fresh)}}
	{\Gamma \vdash {\tt e}.{\tt m}(\bar{\tt e}):
                ({\tt z}:{\tt T};\,\bar{\tt z}:\bar{\tt V};\,{\tt S})}
	&
\infrule[T-New]
	{\Gamma \vdash \bar{\tt e}:\bar{\tt T} \andalso \vdash \klass({\tt C}) \\ 
	  \Gamma,{\tt z}:{\tt C}\vdash \fields({\tt z})=\bar{\tt f}:\bar{\tt U}  \andalso \mbox{({\tt z},$\bar{\tt z}$\ fresh)}\\
	  \Gamma, {\tt z}:{\tt C}, \bar{\tt z}:\bar{\tt T}, {\tt z}.\bar{\tt f}=\bar{\tt z} 
	  \vdash \bar{\tt T} \subtype \bar{\tt U}, \inv({\tt C},{\tt z})}
	{\Gamma \vdash \new\ {\tt C}(\bar{\tt e}): {\tt C}\{\bar{\tt z}:\bar{\tt T};\,\new\ {\tt C}(\bar{\tt z})=\self, \inv({\tt C},\self)\}}
\\[-12pt]
\infrule[Method OK]
	{\this:{\tt C}\vdash {\tt c}:o \andalso \this:{\tt C},\bar{\tt x}:\bar{\tt U},{\tt c} \vdash {\tt T} \ \type, \bar{\tt U} \ \type, 
	  {\tt e}:{\tt S}, {\tt S} \subtype {\tt T}}
	{{\tt def}\ {\tt m}(\bar{\tt x}:\bar{\tt U})\{{\tt c}\}:{\tt T}= {\tt e};\ \mbox{OK in}\ C}
	&
\infrule[Class OK]
	{\bar{\tt M}\ \mbox{OK in}\ {\tt C}
        \andalso
        \this:{\tt C} \vdash {\tt c}:{\tt o}
        \andalso
        \this:{\tt C},{\tt c} \vdash \bar{\tt T}\ \type, {\tt N}\ \type}
	{\mbox{\tt class}\ {\tt C}(\bar{\tt f}:\bar{\tt T})\{{\tt c}\}\ 
	  \mbox{\tt extends}\ {\tt N}\{\bar{\tt M}\} \ \mbox{OK}}
\end{tabular}
\end{tabular}}\\[-12pt]
\quad\\
{\bf Transition rules:}\\[-12pt]
{\tabcolsep=0pt
\begin{tabular}{p{0.43\textwidth}@{\quad}p{0.54\textwidth}}
\typicallabel{RC-Field}
\infrule[\RField]
	{\tt x:C \vdash \fields(\tt x)=\bar{\tt f}:\bar{\tt T}}
	{(\new\ {\tt C}(\bar{\tt e})).{\tt f_i} \derives {\tt e_i}}

\infrule[\RCField]%
	{{\tt e} \derives {{\tt e}}'}
	{{\tt e}.{\tt f}_i \derives {{\tt e}}'.{\tt f}_i}

\infrule[\RCast]%
	{\vdash {\tt C}\{\self==\new\ {\tt C}(\bar{\tt d})\} \subtype {\tt T}}
	{{\tt \new\ C(\bar{\tt d})~as~T} \derives \new\ C(\bar{\tt d})}

\infrule[\RCCast]%
	{{\tt e} \derives {{\tt e}}'}
	{{\tt e~as~T} \derives {\tt e}'~{\tt as~T}}
	&
	\typicallabel{RC-Invk-Recv}
	\infrule[\RInvk]
		{{\tt x:C}\vdash {\tt x}\ \has\ {\tt m}(\bar{\tt x}:\bar{\tt T}){\tt \{c\}:T=e}}
		{(\new\ {\tt C}(\bar{\tt e})).{\tt m}(\bar{\tt d}) \derives {\tt e[\new\ {\tt C}(\bar{\tt e}),\bar{\tt d}/\this,\bar{\tt x}]}}

	\infrule[\RCInvkRecv]%
		{{\tt e} \derives {{\tt e}}'}
		{{\tt e}.{\tt m}(\bar{\tt e}) \derives {{\tt e}}'.{\tt m}(\bar{\tt e})}

	\infrule[\RCInvkArg]%
		{{\tt e}_i \derives {{\tt e}_i}'}
		{{\tt e}.{\tt m}(\ldots,{\tt e}_i,\ldots) \derives {{\tt e}}.{\tt m}(\ldots,{\tt e}_i',\ldots)} 

	\infrule[\RCNewArg]%
		{{\tt e}_i \derives {{\tt e}_i}'}
		{\new\ {\tt C}(\ldots,{\tt e}_i,\ldots) \derives \new\ {\tt C}(\ldots,{\tt e}_i',\ldots)}
\end{tabular}
}
\end{tabular}  

 \caption{Semantics of \FX}\label{fig:FX}
\end{figure*}


\subsection{Judgments}
%\begin{definition}[Designator] The class of {\em designators}
%is given by:
%\begin{tabular}{r@{\quad}rcl}
%(Designator) & {\tt d} &{::=}& {\tt x}\alt C($\bar{\tt d}$)  \alt {\tt d.f} \\
%\end{tabular}
%\end{definition}

In the following $\Gamma$ is a {\em well-typed context}, i.e.{} a
(finite, possibly empty) sequence of formulas $\tt x:T$, $\tt T\
\type$ and constraints $\tt c$ satisfying:
\begin{enumerate}
  \item for any formula $\phi$ in the sequence all variables $\tt x$
    ({\tt X}) occurring in $\phi$ are defined by a declaration $\tt
    x:T$ ({\tt X\ \type}) in the sequence to the left of $\phi$.

  \item for any variable $\tt x$ ({\tt X}), there is at most one
  formula $\tt x:T$ ({\tt X\ \type})  in $\Gamma$.
\end{enumerate}

The judgments of interest are as follows. 
(1)~Type well-formedness:  $\Gamma \vdash {\tt T} \ {\tt type}$,
(2)~Subtyping: $\Gamma \vdash {\tt S} \subtype {\tt T}$,
(3)~Typing:   $\Gamma   \vdash {\tt e:T}$,
(4)~Method OK (method $\tt M$ is well-defined for the class $\tt C$): $\Gamma \vdash M\ \mbox{OK in $C$}$,
(5)~Field OK (field $\tt f:T$ is well-defined for the class $\tt
   C$): $\Gamma \vdash f:T\ \mbox{OK in $C$}$
(6)~Class OK: $\Gamma \vdash {\tt L} \ OK$ (class definition {\tt L} is well-formed). 

In defining these judgments we will use $\Gamma \vdash_{\cal C} c$,
the judgment corresponding to the underlying constraint system. For
simplicity, we define $\Gamma \vdash {\tt c}$ to mean
$\sigma(\Gamma)\vdash_{\cal C} {\tt c}$, where the {\em constraint
projection}, $\sigma(\Gamma)$ is defined as allows. 

\begin{tabular}{l}
$\sigma(\epsilon)={\tt true}$\\
$\sigma({\tt x}:{\tt T\{c\}}, \Gamma)$={\tt c}[{\tt x}/\self], $\sigma(\Gamma)$\\
$\sigma({\tt x}:{\tt (y:S;T)}, \Gamma)$=$\sigma({\tt y:S}, {\tt x}:{\tt T},\Gamma)$\\
$\sigma(c,\Gamma)$ = c, $\sigma(\Gamma)$
\end{tabular}

\noindent Above, in the third rule we assume that alpha-equivalence is used to
choose the variable {\tt x} from a set of variables that does not
occur in the context $\Gamma$.

We say that a context $\Gamma$ is {\em consistent} if all (finite)
subsets of $\{\sigma(\phi)\alt \Gamma \vdash \phi\}$ are consistent.
In all type judgments presented below ({\sc T-Cast}, {\sc T-Field}
etc) we make the implicit assumption that the context $\Gamma$ is
consistent; if it is inconsistent, the rule cannot be used and the
type of the given expression cannot be established (type-checking
fails).

\subsection{\FXZ}
The semantics of \FXZ{} is presented in Figure~\ref{fig:FX}. 

The syntax is essentially that of \FJ{} with three major
exceptions. First, types may be constrained with a clause {\tt
\{c\}}. Second both classes and methods may have constraint clauses
{\tt c}---in the case of classes, {\tt c} is to be thought of as an
invariant satisfied by all instances of the clause, and in the case of
methods, {\tt c} is an additional condition that must be satisfied by
{\tt this} and the arguments of the method in order for the method to
be applicable. 

Note that we distinguish the category of {\em parameter types} ({\tt
V}) from types ({\tt T}). This is in preparation for the introduction
of type variables; we will introduce a ``type'' {\tt type} and permit
parameters and fields to have this type, thus supporting
genericity.\footnote{But we will not permit the return types of
methods to be {\tt type}. This does indeed make sense, but developing
this theory further is beyond the scope of this paper.}

The syntax for constraints in \FXZ{} is specified in
Figure~\ref{fig:FX}. We distinguish a subset of these constraints as
{\em user constraints}---these are permitted to occur in
programs. For \FXZ{} the only user constraint permitted is the vacuous
{\tt true}. Thus the types occurring in user programs are isomorphic
to class types, and class and method definitions specialize to the
standard class and method definitions of \FJ{}. 

The constraints permitted by the syntax in Figure~\ref{fig:FX} that
are not user constraints are necessary to define the static and
dynamic semantics of the program (see, e.g.{} the rules {\sc T-New}
and {\sc T-Field}, {\sc T-Var} that use equalities, conjunctions and
existential quantifications. The use of this richer constraint set is
not necessary, it simply enables us to present the static and dynamic
semantics once for the entire family of \FX{} languages,
distinguishing different members of the family by varying the
constraint system over which they are defined.

The set of types includes classes {\tt C} and is closed under
constrained types ({\tt T\{c\}}) and existential quantification ({\tt
x:S;T}). An object {\tt o} is of type {\tt C} (for {\tt C} a class)
if it is an instance of a subtype of {\tt C}; it is of type {\tt
T\{c\}} if it is of type {\tt T} and it satisfies the constraint $\tt
c[o/self]$\footnote{Thus the constraint {\tt c} in a type {\tt T\{c\}}
should be thought of as a unary predicate $\lambda\self.{\tt c}$, an
object is of this type if it is of type {\tt T} and satisfies this
predicate.}; it is of type {\tt x:S;T} if there is some object {\tt q}
of type {\tt S} such that {\tt o} is of type {\tt T[q/x]} (treating at
type as a syntactic expression).

The rules for well-formedness of types are straightforward, given 
the assumption that constraints are of a pre-given type $o$.

\paragraph{Type judgment rules.}
{\sc T-Var} is as expected, except that it asserts the constraint {\tt
self==x} which records the fact that any value of this type is known
statically to be equal to {\tt x}. This constraint is actually very
crucial---as we shall see in the other rules once we establish that
an expression {\tt e} is of a given type {\tt T}, we ``transfer'' the
type to a freshly chosen variable {\tt z}. If in fact {\tt e} has a
static ``name'' {\tt x} (i.e.{} {\tt e} is known statically to be
equal to {\tt x}, i.e.{} {\tt e} is of type {\tt T\{self==x\}}), then
{\sc T-Var} lets us assert that {\tt z:T\{self==x\}}, i.e.{} {\tt z}
equals {\tt x}. Thus {\sc T-Var} provides an important base case for
reasoning statically about equality of values in the environment.

We do away with the three casts provided in\FJ{} in favor of a single
cast, requiring only that {\tt e} be of some type {\tt U}. At run time
{\tt e} will be checked to see if it is actually of type {\tt T} (see
Rule~{\sc R-Cast}).

{\sc T-Field} may be understood through ``proxy'' reasoning as
follows.  Given the context $\Gamma$ assume the receiver {\tt e} can
be established to be of type {\tt S}. Now we do not know the run-time
value of {\tt e}, so we shall assume that it is some fixed but unknown
``proxy'' value {\tt z} (of type {\tt S}) that is ``fresh'' in that it
is not known to be related to any known value (i.e.{} those recorded
in $\Gamma$).  If we can establish that {\tt z} has a field {\tt f} of
type {\tt V}\footnote{Note from the definition of
\fields{} in ${\cal O}$ (Figure~\ref{fig:O}) that all occurrences of
\this{} in the declared type of the field {\tt f} will have been replaced
by {\tt z}.}, then we can assert that
{\tt e.f} has type {\tt V} and, further, that it equals {\tt z.f}.
Hence, we can assert that {\tt e.f} has type {\tt
(z:S;V\{\self=z.f\})}.

{\sc T-Invk} has a very similar structure to {\sc T-Field}: we use
``proxy'' reasoning for the receiver and the arguments of the method
call. {\sc T-New} also uses the same proxy reasoning: however in this case
we can establish that the resulting value is equal to ${\tt new\ C(\bar{\tt v})}$
for some values $\bar{\tt v}$ of the given type.

\paragraph{Operational semantics.}

The operational semantics is straightforward and essentially identical
to \FJ \cite{FJ}. It is described in terms of a non-deterministic
reduction relation on expressions. The only novelty is the use of the
subtyping relation to check that the cast is satisfied. In \FXZ, this
test simply involves checking that the class of which the object is an
instance is a subclass of the class specified in the given type; in
richer languages with richer notions of type this operation may
involve run-time constraint solving using the fields of the object.


\subsection{\FXG}
We now turn to showing how \FGJ{} style generics can be supported in the \FX{} family.

\FXG{} is the language obtained by adding the following productions to \FXZ.

\begin{tabular}{r@{\quad}rcl}
  (Par Type)& {\tt V}&{::=}& \type\\
  (Path) & {\tt p} &{::=}& {\tt x} \alt \self \alt \this \alt {\tt p.f}\\
  (Type)& {\tt T}&{::=}& {\tt X} \alt {\tt p}\\
  (C Term) & {\tt t} &{::=}& {\tt T} \\
  (Const.) & {\tt c} &{::=}& ${\tt t } \extends {\tt N}$ \alt {\tt t==t}\\
\end{tabular}

That is, first we introduce the ``type'' \type. \FGJ{} method type
parameters are modeled in \FXG{} as normal parameters of type
\type.\footnote{In concrete \Xten{} syntax type parameters are
distinguished from ordinary value parameters through the use of
``square'' brackets. This is particularly useful in implementing type
inference for generic parameters. We abstract these concerns away in
the abstract syntax presented in this section.}  Generic class
parameters are modeled as ordinary fields of type \type, with
parameter bound information recorded as a constraint in the class
invariant. This decision to use fields rather than parameters is
discussed further in Section~\ref{sec:parameters-vs-fields}. In brief,
it permits powerful idioms using fixed but unknown types without
requiring ``wildcards''.

Once fields of type \type{} are permitted, it is natural to have {\em
path} types (cf \cite{scala}). Such types name type-valued members of
objects.

The set of types is now enhanced to permit some fixed but unknown
types {\tt X}. The key idea is that information about such types can
be accumulated through constraints over $\cal O$.  Specifically we
permit the constraint ${\tt t \extends N}$. It may be used, for
instance, to specify upper bounds on type variables or fields (path
types).\footnote{To support structural typing, we permit the
programmer to use {\tt x\ \has\ I} constraints, see
Section~\ref{sec:structural-typing}.}

The well-formedness and $\vdash_{\cal O}$ rules in
Figure~\ref{fig:fgj} must be added.

\begin{figure*}
{\footnotesize
\tabcolsep=0pt
\begin{tabular}{p{0.33\textwidth}p{0.33\textwidth}p{0.33\textwidth}}
\infrule[Path]
	{\Gamma \vdash {\tt p:T} \andalso \Gamma, {\tt x:T} \vdash {\tt x}\ \has\ {\tt X}:\ \type}
	{\Gamma \vdash {\tt p.X}\ {\tt type}} 
&
\infax[Type-Var]{\Gamma, {\tt X:\type} \vdash {\tt X}\ \type}
&
\infax[Equals]{{\tt S==T} \vdash_{\cal O} S \extends T, T \extends S}
\end{tabular}\\[-12pt]
\begin{tabular}{p{0.16\textwidth}p{0.33\textwidth}p{0.33\textwidth}}
&
\infrule[Inh-p]
	{\Gamma \vdash_{\cal O} {\tt p} \extends {\tt T} \andalso \Gamma, {\tt x}:{\tt T} \vdash_{\cal O} {\tt x}\ \has\ {\tt I}}
	{\Gamma,{\tt x}:{\tt p} \vdash_{\cal O} {\tt x}\ \has\ {\tt I}}
&
\infrule[Inh-X]
	{\Gamma \vdash_{\cal O} {\tt X} \extends {\tt T} \andalso \Gamma, {\tt x}:{\tt T} \vdash_{\cal O} {\tt x}\ \has\ {\tt I}}
	{\Gamma,{\tt x}:{\tt X} \vdash_{\cal O} {\tt x}\ \has\ {\tt I}}
\end{tabular}
}
\caption{\FGJ{} semantics}
\label{fig:fgj}
\end{figure*}

\begin{example}
The \FGJ{} parametric method:

{\footnotesize
\begin{xten} 
 <T> T id(T x) { return x;}
\end{xten}
}
\noindent can be represented as
{\footnotesize
\begin{xten} 
  def id(T:type, x:T):T=x;
\end{xten}
}

The class 
{\footnotesize
\begin{xten} 
  class Comparator<B> {
    int compare(B y) { ...}
  }
  class SortedList<T extends Comparator<T>> { 
    int m(T x, T y) {
       return x.compare(y);
  }
\end{xten}
}
\noindent can be represented as
{\footnotesize
\begin{xten} 
  class Comparator(B: type) {
   def compare(y:B):int = ...;
  }
  class SortedList(T: type){T <: Comparator{self.B==T}} { 
    def m(x:T, y:T):int = x.compare(y);
  }
\end{xten}
}
\end{example}

\subsection{\FXD}
We assume given a constraint system $\cal A$, with a vocabulary of
predicates {\tt q} and functions {\tt f}. These are used to 
augment the constraints expressible in the language. 

\begin{tabular}{r@{\quad}rcl}
  (Type) & {\tt T} & {::=}& \mbox{new base types, e.g., {\tt int}, {\tt boolean}}\\
  (C Term) & {\tt t} &{::=}& $\tt f(\bar{\tt t})$ \\
  (Const.) & {\tt c} &{::=}& ${\tt q(\bar{\tt t})}$ 
\end{tabular}

The obvious rules are needed to ensure that formulas are well-formed.
{\footnotesize
\infrule[Pred]
	{\tt p(\bar{\tt T}):{\tt o}\in {\cal C} \andalso \Gamma \vdash \bar{\tt t}:\bar{\tt T}}
	{\Gamma \vdash \tt p(\bar{\tt T}):o}

\infrule[Fun]
	{\tt f(\bar{\tt T}):{\tt T}\in {\cal C} \andalso \Gamma \vdash \bar{\tt t}:\bar{\tt T}}
	{\Gamma\tt \vdash \tt f(\bar{\tt T}):T}
\infrule[Equals]
	{\Gamma\tt \vdash t_0: T_0 \andalso \Gamma \vdash t_1: T_1 \\
	  (\Gamma\tt \vdash T_0 \subtype T_1 \vee \Gamma \vdash T_1 \subtype T_0)}
	{\Gamma\tt \vdash t_0=t_1:o}
}
\noindent
No additional type judgments or transition rules are needed. 

\subsection{\FXGD} 

No additional rules are needed beyond those of \FXG{} and \FXD{}. This
language permits type and value constraints, supporting \FGJ{} style
generics and value-dependent types.


\subsection{Results}
The following results hold for \FXGD.

\begin{theorem}[Subject Reduction] If $\Gamma \vdash {\tt e:T}$ and ${\tt e} \derives {\tt e'}$ then for some type {\tt S}, $\Gamma \vdash {\tt e':S},{S \subtype T}$.
\end{theorem}

The theorem needs the Substitution Lemma:
\begin{lemma} The following is a derived rule:
\infrule[Subst]
{\Gamma \vdash \bar{\tt d}:\bar{\tt U} \andalso \Gamma, \bar{\tt x}:\bar{\tt U} \vdash \bar{\tt U}\subtype\bar{\tt V} \andalso \Gamma, \bar{\tt x}:\bar{\tt V} \vdash {\tt e:T}}
{\Gamma \vdash {\tt e[\bar{\tt d}/\bar{\tt x}]:S}, {\tt S \subtype \bar{\tt x}:\bar{\tt A};T}}
\end{lemma}

We let values be of the form $\tt v ::= \new\ C(\bar{\tt v})$. 
\begin{theorem}[Progress]
If $\vdash {\tt e:T}$ then one of the following conditions holds:
\begin{enumerate}
\item {\tt e} is a value,
\item {\tt e} contains a cast sub-expression which is stuck,
\item there exists an $\tt e'$ s.t. $\tt e\derives e'$.
\end{enumerate}
\end{theorem}

\begin{theorem}[Type soundness]
If $\vdash {\tt e:T}$ and {\tt e} reduces to a normal form ${\tt e'}$ then
either $\tt e'$ is a value {\tt v} and $\vdash {\tt v:S},{\tt S\subtype T}$ or
${\tt e'}$ contains  a stuck cast sub-expression.
\end{theorem}




