\newcommand\wj[2]{{#1} \vdash_{\cal W}{#2}}
\newcommand\cdecl{{\tt class}~{\tt C}[\tbar{X}]\{{\tt c}\}(\tbar{f}\ty\tbar{F})~{\tt extends}~{\tt D}[\tbar{E}]~\{~\tbar{M}~\}}
\newcommand\msign[5]{{\tt m}[\tbar{#1}](\tbar{#2}\ty\tbar{#3})\{{\tt #4}\}\ty{\tt #5}}
\newcommand\minst[6]{\msign{#1}{#2}{#3}{#4}{#5}={\tt #6}}
\newcommand\mdecl[6]{{\tt def}~\minst{#1}{#2}{#3}{#4}{#5}{#6}}
\newcommand\FXGL[1]{{\sf FXG($\cal #1$)}}
\newcommand{\vdashQ}{\vdash_{\cal Q}}
\newcommand{\vdashS}{\vdash_{\cal S}}
\newcommand{\Dom}{{\sf Dom}}
\newcommand{\Img}{{\sf Img}}

We now describe the semantics of languages in the \FXGL{L} family. We start with a core \FXGL{\cdot} language that adds to the \FX language support for unconstrained type parameters. Variables of type-parameter types in generic classes have therefore no accessible methods and fields in \FXGL{\cdot}. We then introduce constraints on type parameters such as \FGJ-like bounds as a family of extensions to this core language and its type system.


\subsection{\FXGL{\cdot}}

We specify the gammar of \FXGL{\cdot}, its dynamic semantics, and its static type system.

\paragraph{Grammar.} The grammar for \FXGL{\cdot} is shown in Figure~\ref{fig:fxg-grammar}. The syntax is essentially that of \FGJ. Following \FJ{}, we use $\bar{x}$ to denote a list $x_1, \dots, x_n$, and use $\bullet$ to denote the empty list.

A program {\tt P} is a set of class declarations \tbar{L}. Class names {\tt C} range over the declared classes in {\tt P} and {\tt Object}. A class declaration has type parameters \tbar{X}, value properties (i.e., immutable fields) \tbar{f}, a supertype {\tt D}[\tbar{E}], methods \tbar{M}, and a guard {\tt c}---a constraint on its type parameters. It is not used in \FXGL{\cdot}, that is, {\tt c} must be {\tt true}.

Following \FJ{}, we omit user-defined constructors. Each class has a default constructor that requires initial values for all fields of the class including inherited fields. If class {\tt C} has a field {\tt f} of type {\tt F} and inherits from class {\tt D} a field {\tt g} of type {\tt G}, the constructor for {\tt C} takes first an argument of type {\tt G} then an argument of type {\tt F}.

Methods are introduced with the {\tt def} keyword. A method has both type parameters {\tbar{X}} and value parameters {\tbar{x}}. The method guard {\tt c} is to be thought of as an additional condition that must be satisfied by the receiver and the actual type and value arguments of the method call. We do not consider method overloading:\footnote{Constraint-sensitive method overloading is beyond the scope of this paper. Constraint-insensitive method overloading is irrelevant.} we assume each class declares at most one method with name {\tt m}.

The body of a method is an expression {\tt e}. It is built from the value parameters {\tt x} of the method (including the receiver {\tt this}), field access expressions, construtor calls, method invocations, and casts (written {\tt e}~\as~{\tt G}).

The set of types includes class types {\tt C}[\tbar{A}], type parameters {\tt X}, dependent types ($\tt T\{c\}$), and is closed under existential quantification ($\exty{\tt x}{\tt T}{\tt U}$). Existential types arise in typing judgements but are not permitted in programs. Neither casts nor methods invocations are permitted in constraints. A value {\tt v} is of type {\tt C}[\tbar{A}] if it is an instance of class {\tt C}[\tbar{A}]; it is of type $\tt T\{c\}$ if it is of type {\tt T} and it satisfies the constraint $\tt c[v/self]$; it is of type $\exty{\tt x}{\tt T}{\tt U}$ if there exists some value {\tt w} of type {\tt T} such that {\tt v} is of type
$\tt U[w/x]$.

Following \FJ{}, we denote values by means of nested constructor calls, e.g., new Pair[Object](new Object(), new Object()).

\paragraph{Dynamic semantics.}
The operational semantics, shown in Figures~\ref{fig:members} and~\ref{fig:sos}, is straightforward. It is described in terms of a reduction relation on expressions. It enforces a strict left-to-right call-by-value evaluation order.

The only novelty is the use of the subtyping relation to check that the cast is satisfied. This check is the only reason the operational semantics has to propagate type parameters. The fields and methods of a value $\new~{\tt C}[\tbar{A}](\tbar{v})$ do not depend on the types \tbar{A}. See Section~\ref{sec:casts} for a discussion of casts. See below for a discussion of subtyping.

Rule {\sc L-Field-I} prevents classes from overriding inherited fields. Rule {\sc L-Method-I} makes sure method lookup goes bottom up but does not enforce any overriding restrictions. These are not relevant to the operational semantics, hence are introduced later.

\paragraph{Conventions.}
In the following, the context $\Gamma$ is always a (finite, possibly empty) sequence of variable declarations $\tt x\ty T$, type parameter declarations $\tt X\ty*$, and constraints $\tt c$ satisfying:
\begin{enumerate}
  \item for any declaration $\phi=\tt x\ty T$ or constraint $\phi=\tt c$ in the sequence, all free variables\footnote{Variables are bound by existential quantifiers. In addition, the type constructor ${\tt c}\mapsto{\tt T}\{{\tt c}\}$ binds the special variable {\self} in {\tt c}.} $\tt y$ occurring in $\tt T$ or $\tt c$ are declared in the sequence to the left of $\phi$.

  \item a variable or type parameter is declared at most once in $\Gamma$.
\end{enumerate}

In the following, for any formulas $\phi_1$ and $\phi_2$, we adopt the convention that the judgment $\Gamma \vdash \phi_1,~\phi_2$ is shorthand for the judgments $\Gamma \vdash \phi_1$ and $\Gamma \vdash \phi_2$. 

An assumption of the form ``{\tt x} fresh'' in a deduction rule means that {\tt x} does not occur in the premises to the left of this assumption.

A premise of the form $\theta=\tbar{\alpha}/\tbar{\beta}$ requires $\tbar{\alpha}$ and $\tbar{\beta}$ to have the same length $n$ and defines the substitution of $\beta_i$ by $\alpha_i$ for $1\leq i\leq n$. If $\gamma$ is a constraint, a type, an expression, etc., $\gamma\theta$ denotes the application of the substitution $\theta$ to the free variables of $\gamma$.

\paragraph{Type system.} Typing checking a program {\tt P} involves a series of judgments:
\begin{enumerate}
	\item Constraints:\\
	  $\Gamma\vdash {\tt c}$ \hfill $\Gamma$ entails constraint {\tt c}
	\item Well-formedness:\\
	  $\wj{\Gamma}{\tt t}$ \hfill constraint term {\tt t} is well formed in $\Gamma$\\
	  $\wj{\Gamma}{\tt c}$ \hfill constraint {\tt c} is well formed in $\Gamma$\\
	  $\wj{\Gamma}{\tt T}$ \hfill  type {\tt T} is well formed in $\Gamma$
	\item Lookup:\\
	  $\Gamma\vdash {\tt x}~\has~{\tt I}$ \hfill variable {\tt x} has member {\tt I} in $\Gamma$\\
	  $~$ \hfill where ${\tt I}::= {\tt f}\ty{\tt F} \alt \msign{B}{x}{G}{c}{H}$
	\item Subtyping:\\
	  $\Gamma,{\tt x}\ty{\tt S}\vdash{\tt x}\subtype{\tt T}$ \\ $~$ \hfill type ${\tt S}\{\self=={\tt x}\}$ is a 	subtype of type {\tt T} in $\Gamma,{\tt x}\ty{\tt S}$
	\item Typing:\\
	  $\Gamma\vdash {\tt e}\ty{\tt T}$ \hfill expression {\tt e} has type {\tt T} in $\Gamma$\\
	  $\Gamma\vdash {\tt I}~\ll~{\tt J}$ \hfill member {\tt I} overrides member {\tt J} in $\Gamma$\\
	  $\vdash {\tt def}~{\tt m}[\tbar{Y}](\tbar{x}\ty \tbar{G})\{{\tt c}\}\ty {\tt H}={\tt e}~{\rm OK~in}~{\tt C}[\tbar{X}]$ \\ $~$ \hfill method {\tt m} in class ${\tt C}[\tbar{X}]$ type checks\\
	  $\vdash \cdecl~{\rm OK}$ \\ $~$ \hfill class ${\tt C}[\tbar{X}]$ type checks

\end{enumerate}

Notice these judgments depend on the particular program {\tt P}. For readability, we do not make this dependency explicit as we will never consider more than one program at a time.

A program type checks iff all its classes do. We now describe in more detail each of these judgments, in turn. 


\paragraph{1. Constraints.}
Each language in the \FXGL{L} family is defined over a given untyped input constraint system $\mathcal{X}$ that is required to support the constraint permitted by the \FXGL{L} grammar---\true{}, conjunction of constraints, and equality on constraint terms in the \FXGL{\cdot} case. It must also satisfy the inference rule {\sc X-Sel} of Figure~\ref{fig:object}.\footnote{Formally, we consider a constraint system generator $\mathcal{X}$ that given a program {\tt P} produces a constraint system $\mathcal{X}_{\tt P}$ that satisfies {\sc X-Sel}.} Judgments in $\cal X$ are of the form $\tbar{c} \vdashX {\tt d}$.

We define the {\em constraint projection}, $\sigma(\Gamma)$ as follows.
%
\begin{center}
\begin{tabular}{ll}
$\sigma(\epsilon)={\tt true}$\\
$\sigma({\tt c},\Gamma) = {\tt c}, \sigma(\Gamma)$\\
$\sigma({\tt X}\ty*, \Gamma)=\sigma(\Gamma)$\\
$\sigma({\tt x}\ty{\tt C}[\tbar{A}], \Gamma)=\sigma(\Gamma)$\\
$\sigma({\tt x}\ty{\tt X}, \Gamma)=\sigma(\Gamma)$\\
$\sigma({\tt x}\ty{\tt T\{c\}}, \Gamma)={\tt c}\theta,\sigma({\tt x}\ty{\tt T},\Gamma)$ & where $\theta={\tt x}/\self$\\
$\sigma({\tt x}\ty\exty{{\tt y}}{{\tt T}}{{\tt U}}, \Gamma)=\sigma({\tt z}\ty{\tt T}, {\tt x}\ty{\tt U}\theta,\Gamma)$ & where $\theta={\tt z}/{\tt y}$\\
\end{tabular}
\end{center}
%
In the last rule, we assume that alpha-equivalence is used to choose a variable {\tt z} that does not occur in the context under construction.

By rule {\sc X-Proj} in Figure~\ref{fig:object}, we specify that $\Gamma\vdash c$ if the constraint projection of $\Gamma$ entails $c$ in the input constraint system. In \FXGL{\cdot}, there is no other way to prove entailment.

We say that a context $\Gamma$ is {\em consistent} if it is not the case that $\Gamma\vdash\false$.
In all deduction rules presented below, we make the implicit assumption that the context $\Gamma$ of every premise is consistent; if one is inconsistent, the rule cannot be used.

Inconsistency may arise because $\cal X$ itself defines inconsistent constraints, e.g., ``2==3'' for arithmetic constraints. In the sequel, we will also permit type system extensions to mark contexts as inconsistent. For instance, {\tt X} extends class {\tt C}, {\tt X} extends class {\tt D} is inconsistent if both class {\tt C} and {\tt D} extend {\Object}.


\paragraph{2. Well-formedness.} The rules of well-formedness are specified in Figure~\ref{fig:well}. A constraint term, constraint, or type $\alpha$ is well formed in context $\Gamma$, written $\wj{\Gamma}{\alpha}$, iff its free variables are declared in $\Gamma$ and $\Gamma$ entails the guards of all the instances of generic classes in $\alpha$. Notice this depends on $\cal X$ but not on any typing judgments. For \FXGL{\cdot} the later condition is a tautology.

By design of the type system, if the program {\tt P} type checks then all the constraints and types in {\tt P} are well formed in their respective contexts.

\begin{figure*}
\centering
\begin{tabular}{r@{\quad}rcl}
  (Program) & {\tt P} &{::=}& $\tbar{L}$ \\
  (Class declaration) & {\tt L} &{::=}& $ \tt class~C[\tbar{X}]\{c\}(\tbar{f}\ty\tbar{G})~extends~D[\tbar{G}]~\{~\tbar{M}~\}$ \\
  (Method declaration)& {\tt M} &{::=}& $\tt \mdecl{X}{x}{G}{c}{G}{e};$ \\
  (Expression)& {\tt a}, {\tt b}, {\tt e} &{::=}& $\tt x$ \alt $\tt e.f$ \alt $\tt\new~C[\tbar{G}](\tbar{e})$ \alt $\tt e.m[\tbar{G}](\tbar{e})$ \alt $\tt e~\as~G$ \\
  (Constraint term) & {\tt t}, {\tt u} &{::=}& $\tt x$ \alt $\tt t.f$ \alt $\tt\new~C[\tbar{G}](\tbar{t})$ \\
  (Constraint) & {\tt c}, {\tt d} &{::=}& $\true$ \alt $\false$ \alt $\tt c,c$ \alt $\tt t==t$ \\
  (Generic type)& {\tt A}, {\tt B}, {\tt E}, {\tt F}, {\tt G}, {\tt H} &{::=}& $\tt C[\tbar{G}]$ \alt $\tt G\{c\}$ \alt $\tt X$ \\
  (Type)& {\tt S}, {\tt T}, {\tt U} &{::=}& $\tt G$ \alt $\tt T\{c\}$ \alt $\tt \exists x\ty T.~T$ \\
  (Value)& {\tt v}, {\tt w} &{::=}& $\tt\new~C[\tbar{G}](\tbar{v})$ where \tbar{G} contains no type variables \\
\end{tabular}\smallskip

{\tt C}, {\tt D} range over class names, {\tt f}, {\tt g} over field names, {\tt m} over method names, {\tt x}, {\tt y} over variable names, {\tt X}, {\tt Y} over type variables.
\caption{\FXGL{\cdot} productions.}
\label{fig:fxg-grammar}
\end{figure*}


\begin{figure*}
\vspace{-\bigskipamount}
\begin{minipage}{\textwidth}
\quad\typicallabel{XXXXXX}
\infax[L-Fields-B]
  {\fields({\tt Object})=\bullet}

\infrule[L-Fields-I]
  {\cdecl\in{\tt P} \andalso
    \theta=\tbar{A}/\tbar{X} \andalso
    \fields({\tt D}[\tbar{E}\theta])=\tbar{g}\ty\tbar{G} \andalso
    \tbar{f}\cap\tbar{g}=\emptyset}
  {\fields({\tt C}[\tbar{A}])=\tbar{g},\tbar{f}\ty\tbar{G},\tbar{F}\theta}

\infrule[L-Method-B]
  {\cdecl\in{\tt P} \andalso
    \mdecl{Y}{x}{G}{d}{H}{e}\in\tbar{M} \andalso
    \theta=\tbar{A},\tbar{B}/\tbar{X},\tbar{Y}}
  {\methods({\tt C}[\tbar{A}])\ni\minst{B}{x}{G\theta}{d\theta}{H\theta}{e\theta}}

\infrule[L-Method-I]
  {\cdecl\in{\tt P} \andalso
    \theta=\tbar{A}/\tbar{X}
    \andalso
    \methods({\tt D}[\tbar{E}\theta])\ni\minst{B}{x}{G}{d}{H}{e} \andalso
    {\tt m}\not\in\tbar{M}}
  {\methods({\tt C}[\tbar{A}])\ni\minst{B}{x}{G}{d}{H}{e}}
\end{minipage}%
\caption{\FXGL{\cdot} fields and methods.}
\label{fig:members}
\end{figure*}


\begin{figure*}
\vspace{-\bigskipamount}
\begin{minipage}{.33\textwidth}
\quad\typicallabel{XXXXXX}
\infrule[\RField]
	{\fields({\tt C}[\tbar{A}])=\tbar{f}\ty\tbar{F}}
	{\new~{\tt C}[\tbar{A}](\tbar{v}).{\tt f}_i \derives {\tt v}_i}

\infrule[\RCField]
	{{\tt e}\derives {\tt e}'}
	{{\tt e}.{\tt f}\derives {\tt e}'.{\tt f}}

\infrule[\RCInvkRecv]
	{{\tt e}\derives {\tt e}'}
	{{\tt e}.{\tt m}(\tbar{a})\derives {\tt e}'.{\tt m}(\tbar{a})}

\infrule[\RCCast]
	{{\tt e}\derives {\tt e}'}
	{{\tt e}~\as~{\tt G}\derives {\tt e}'~\as~{\tt G}}
\end{minipage}%
\begin{minipage}{.67\textwidth}
\quad\typicallabel{XXXXXX}
\infrule[\RCNewArg]
	{{\tt e}_i\derives {\tt e}'_i}
	{\new~{\tt C}[\tbar{A}]({\tt v}_1,\ldots,{\tt v}_{i-1},{\tt e}_i,\ldots,{\tt e}_n)\derives\new~{\tt C}[\tbar{A}]({\tt v}_1,\ldots,{\tt v}_{i-1},{\tt e}'_i,\ldots,{\tt e}_n)}

\infrule[\RInvk]
	{\methods({\tt C}[\tbar{A}])\ni\minst{B}{x}{G}{d}{H}{e} \andalso
	\theta=\new~{\tt C}[\tbar{A}](\tbar{v}),\tbar{w}/\this,\tbar{x}}
	{\new~{\tt C}[\tbar{A}](\tbar{v}).{\tt m}[\tbar{B}](\tbar{w})\derives {\tt e}\theta}

\infrule[\RCInvkArg]
	{{\tt a}_i\derives {\tt a}'_i}
	{{\tt v}.{\tt m}({\tt w}_1,\ldots,{\tt w}_{i-1},{\tt a}_i,\ldots,{\tt a}_n)\derives {\tt v}.{\tt m}({\tt w}_1,\ldots,{\tt w}_{i-1},{\tt a}'_i,\ldots,{\tt a}_n)}

\infrule[\RCast]
	{{\tt x}\ty{\tt C}[\tbar{A}]\{\self==\new~{\tt C}[\tbar{A}](\tbar{v})\}\vdash{\tt x}\subtype {\tt G}}
	{\new~{\tt C}[\tbar{A}](\tbar{v})~\as~{\tt G}\derives\new~{\tt C}[\tbar{A}](\tbar{v})}
\end{minipage}
\caption{\FXGL{\cdot} operational semantics. \tbar{A} and \tbar{B} are lists of ground types (no type variables, no existentials).}
\label{fig:sos}
\end{figure*}


\begin{figure*}
\vspace{-\bigskipamount}
\begin{minipage}{.5\textwidth}
\quad\typicallabel{XXXXXX}
\infrule[X-Sel]
  {\fields({\tt C}[\tbar{A}])=\tbar{f}\ty\tbar{F}}
  {\vdashX \new~{\tt C}[\tbar{A}](\tbar{t}).{\tt f}_i=={\tt t}_i}
\end{minipage}%
\begin{minipage}{.5\textwidth}
\quad\typicallabel{XXXXXX}
\infrule[X-Proj]
  {\sigma(\Gamma)\vdashX c}
  {\Gamma\vdash c}
\end{minipage}%
\caption{\FXGL{\cdot} object constraint system.}
\label{fig:object}
\end{figure*}


\begin{figure*}
\vspace{-\bigskipamount}
\begin{minipage}{.24\textwidth}
\quad\typicallabel{XXXX}
\infax[W-Var]
  {\wj{{\tt x}\ty{\tt T}}{\tt x}}

\infax[W-True]
  {\wj{}{\true}}

\infax[W-False]
  {\wj{}{\false}}

\infax[W-Object]
  {\wj{}{\Object}}

\infax[W-Type]
	{\wj{{\tt X}\ty*}{\tt X}}
\end{minipage}%
\begin{minipage}{.38\textwidth}
\quad\typicallabel{XXXX}
\infrule[W-Field]
	{\wj{\Gamma}{\tt t}}
	{\wj{\Gamma}{{\tt t}.{\tt f}}}

\infrule[W-And]
	{\wj{\Gamma}{\tt c} \andalso \wj{\Gamma,{\tt c}}{\tt d}}
	{\wj{\Gamma}{{\tt c},{\tt d}}}

\infrule[W-Dep]
  {\wj{\Gamma}{\tt T} \andalso \wj{\Gamma,\self\ty{\tt T}}{\tt c}}
	{\wj{\Gamma}{{\tt T}\{{\tt c}\}}}
\end{minipage}%
\begin{minipage}{.38\textwidth}
\quad\typicallabel{XXXX}
\infrule[W-New]
	{\wj{\Gamma}{\tt C}[\tbar{G}] \andalso \wj{\Gamma}{\tbar{t}}}
	{\wj{\Gamma}{\new~{\tt C}[\tbar{G}](\tbar{t})}}

\infrule[W-Eq]
	{\wj{\Gamma}{\tt t} \andalso \wj{\Gamma}{\tt u}}
	{\wj{\Gamma}{\tt t}=={\tt u}}

\infrule[W-Exists]
  {\wj{\Gamma}{\tt T} \andalso \wj{\Gamma,{\tt x}\ty{\tt T}}{\tt U}}
	{\wj{\Gamma}{\exty{\tt x}{\tt T}{\tt U}}}
\end{minipage}%

\begin{minipage}{\textwidth}
\quad\typicallabel{XXXXXX}
\infrule[W-Class]
  {\cdecl\in{\tt P} \andalso
    \wj{\Gamma}{\tbar{A}} \andalso
    \wj{\tbar{X}\ty*}{{\tt c}} \andalso 
    \theta=\tbar{A}/\tbar{X} \andalso
    \sigma(\Gamma)\vdashX{\tt c}\theta}
  {\wj{\Gamma}{\tt C}[\tbar{A}]}
\end{minipage}%
\caption{\FXGL{\cdot} well-formedness.}
\label{fig:well}
\end{figure*}


\begin{figure*}
\vspace{-\bigskipamount}
\begin{minipage}{\textwidth}
\quad\typicallabel{XXXXXX}
\infrule[H-Field]
  {\wj{\Gamma}{{\tt C}[\tbar{A}]} \andalso
    \fields({\tt C}[\tbar{A}])=\tbar{f}\ty\tbar{F} \andalso \theta={\tt x}/\this}
  {\Gamma,{\tt x}\ty{\tt C}[\tbar{A}]\vdash{\tt x}~\has~{\tt f}_i\ty{\tt F}_i\theta}

\infrule[H-Method]
  {\wj{\Gamma}{{\tt C}[\tbar{A}],\tbar{B}} \andalso
    \methods({\tt C}[\tbar{A}])\ni\minst{B}{y}{G}{d}{H}{e} \andalso
    \theta={\tt x},\tbar{z}/\this,\tbar{y}}
  {\Gamma,{\tt x}\ty{\tt C}[\tbar{A}]\vdash{\tt x}~\has~\msign{B}{z}{G\theta}{d\theta}{H\theta}}
\end{minipage}%

\begin{minipage}{.41\textwidth}
\quad\typicallabel{XXXXXX}
\infrule[H-Dep]
  {\wj{\Gamma}{{\tt T}\{{\tt c}\}} \andalso \Gamma,{\tt x}\ty{\tt T},{\tt c}\vdash {\tt x}~\has~{\tt I}}
  {\Gamma,{\tt x}\ty{\tt T}\{{\tt c}\}\vdash {\tt x}~\has~{\tt I}}
\end{minipage}%
\begin{minipage}{.59\textwidth}
\quad\typicallabel{XXXXXX}
\infrule[H-Exists]
  {\wj{\Gamma}{\exty{\tt y}{\tt U}{\tt T}} \andalso
    \Gamma,{\tt y}\ty{\tt T},{\tt x}\ty{\tt U}\vdash {\tt x}~\has~{\tt I}}
  {\Gamma,{\tt x}\ty\exty{\tt y}{\tt T}{\tt U}\vdash {\tt x}~\has~{\tt I}}
\end{minipage}%
\caption{\FXGL{\cdot} member lookup. {\tt I} ranges over members (fields and methods).}
\label{fig:lookup}
\end{figure*}


\begin{figure*}
\vspace{-\bigskipamount}
\begin{minipage}{.32\textwidth}
\quad\typicallabel{XXXXXX}
\infrule[S-Trans]
	{\Gamma\vdash {\tt x}\subtype {\tt T} \andalso \Gamma,{\tt y}\ty{\tt T}\vdash {\tt y}\subtype {\tt U}}
	{\Gamma\vdash {\tt x}\subtype {\tt U}}
\end{minipage}%
\begin{minipage}{.27\textwidth}
\quad\typicallabel{XXXXXX}
\infrule[S-Const-L]
	{\wj{\Gamma}{{\tt T}\{{\tt c}\}}}
	{\Gamma,{\tt x}\ty{\tt T}\{{\tt c}\}\vdash{\tt x}\subtype {\tt T}}
\end{minipage}%
\begin{minipage}{.41\textwidth}
\quad\typicallabel{XXXXXX}
\infrule[S-Const-R]
	{\wj{\Gamma}{{\tt T}\{{\tt c}\}} \andalso \Gamma\vdash {\tt c}[{\tt x}/\self],{\tt x}\subtype {\tt T}}
	{\Gamma\vdash{\tt x}\subtype {\tt T}\{{\tt c}\}}
\end{minipage}%

\begin{minipage}{.57\textwidth}
\quad\typicallabel{XXXXXX}
\infrule[S-Exists-L]
  {\wj{\Gamma}{\exty{\tt x}{\tt U}{\tt S}} \andalso
    \Gamma,{\tt y}\ty{\tt U},{\tt x}\ty{\tt S}\vdash {\tt x}\subtype {\tt T} \andalso {\tt y}~\rm not~free~in~{\tt T}}
  {\Gamma,{\tt x}\ty\exty{\tt y}{\tt U}{\tt S}\vdash{\tt x}\subtype {\tt T}}
\end{minipage}%
\begin{minipage}{.43\textwidth}
\quad\typicallabel{XXXXXX}
\infrule[S-Exists-R]
  {\wj{\Gamma}{\exty{\tt x}{\tt U}{\tt T}} \andalso
    \Gamma\vdash {\tt t}\ty{\tt U},{\tt y}\subtype {\tt T}[{\tt t}/{\tt x}]}
  {\Gamma\vdash {\tt y}\subtype\exty{\tt x}{\tt U}{\tt T}}
\end{minipage}%

\begin{minipage}{\textwidth}
\quad\typicallabel{XXXXXX}
\infrule[S-Class]
  {\cdecl\in {\tt P} \andalso
    \theta=\tbar{A}/\tbar{X} \andalso
    \wj{\Gamma}{{\tt C}[\tbar{A}],{\tt D}[\tbar{E}\theta]}}
  {\Gamma,{\tt x}\ty{\tt C}[\tbar{A}]\vdash{\tt x}\subtype {\tt D}[\tbar{E}\theta]}      
\end{minipage}%
\caption{\FXGL{\cdot} subtyping rules.}\label{fig:subtyping}
\end{figure*}


\begin{figure*}
\vspace{-\bigskipamount}
\begin{minipage}{.3\textwidth}
\quad\typicallabel{XXXX}
\infrule[T-Var]
  {\wj{\Gamma}{\tt T}}
  {\Gamma,{\tt x}\ty{\tt T}\vdash {\tt x}\ty{\tt T}\{\self=={\tt x}\}}
\end{minipage}%
\begin{minipage}{.25\textwidth}
\quad\typicallabel{XXXX}
\infrule[T-Cast]
	{\Gamma\vdash {\tt e}\ty{\tt T} \andalso \wj{\Gamma}{\tt G}}
	{\Gamma\vdash {\tt e}~\as~{\tt G}\ty{\tt G}}
\end{minipage}%
\begin{minipage}{.45\textwidth}
\quad\typicallabel{XXXX}
\infrule[T-Field]
	{\Gamma\vdash {\tt e}\ty{\tt T} \andalso
	  {\tt x}~{\rm fresh} \andalso
	  \Gamma,{\tt x}\ty{\tt T}\vdash {\tt x}~\has~{\tt f}\ty{\tt F}}
	{\Gamma\vdash {\tt e}.{\tt f}\ty\exty{\tt x}{\tt T}{\tt F}\{\self=={\tt x}.{\tt f}\}}
\end{minipage}

\begin{minipage}{\textwidth}
\quad\typicallabel{XXXXXX}
\infrule[T-New]
	{\Gamma\vdash\tbar{e}\ty\tbar{T} \andalso
	  \wj{\Gamma}{{\tt C}[\tbar{A}]} \andalso
	  \fields({\tt C}[\tbar{A}])=\tbar{f}\ty\tbar{F} \andalso 
	  {\tt y},\tbar{x}~{\rm fresh} \andalso 
	  \theta={\tt y}/\this \andalso
	  \Gamma,{\tt y}\ty{\tt C}[\tbar{A}],\tbar{x}\ty\tbar{T},{\tt y}.\tbar{f}==\tbar{x}\vdash\tbar{x}\subtype\tbar{F}\theta}
	{\Gamma\vdash\new~{\tt C}[\tbar{A}](\tbar{e})\ty\exty{\tbar{x}}{\tbar{T}}{\tt C}[\tbar{A}]\{\self==\new~{\tt C}[\tbar{A}](\tbar{x})\}}
        
\infrule[T-Invk]
	{\Gamma\vdash {\tt e}\ty{\tt T},\tbar{a}\ty\tbar{U} \andalso
	  {\tt x},\tbar{y}~{\rm fresh} \andalso
	  \Gamma,{\tt x}\ty{\tt T},\tbar{y}\ty{\tt U}\vdash {\tt x}~\has~\msign{A}{y}{G}{c}{H},{\tt c},\tbar{y}\subtype\tbar{G}}
	{\Gamma\vdash {\tt e}.{\tt m}[\tbar{A}](\tbar{a})\ty\extyty{\tt x}{\tt T}{\tbar{y}}{\tbar{U}}{\tt H}}

\eat{
\infrule[OK-Method]
  {\cdecl \andalso
    {\tt d}={\tt k},{\tt l} \andalso
    \wj{\tbar{X}\ty*,{\tt c},\tbar{Y}\ty*}{{\tt k}} \\
    \wj{\tbar{X}\ty*,{\tt c},\this\ty{\tt C}[\tbar{X}],\tbar{Y}\ty*,{\tt k},\tbar{x}\ty\tbar{G}}{\tbar{G},{\tt l}} \andalso
    \wj{\tbar{X}\ty*,{\tt c},\this\ty{\tt C}[\tbar{X}],\tbar{Y}\ty*,\tbar{x}\ty\tbar{G},{\tt d}}{{\tt H}} \\
    \tbar{X}\ty*,{\tt c},\this\ty{\tt C}[\tbar{X}],\tbar{Y}\ty*,\tbar{x}\ty\tbar{G},{\tt d}\vdash {\tt e}\ty{\tt E} \andalso
    {\tt y}~{\rm fresh} \andalso
    \tbar{X}\ty*,{\tt c},\this\ty{\tt C}[\tbar{X}],\tbar{Y}\ty*,\tbar{x}\ty\tbar{G},{\tt d},{\tt y}\ty{\tt E}\vdash {\tt y}\subtype {\tt H}}
  {\vdash\mdecl{Y}{x}{G}{d}{H}{e}~{\rm OK~in}~{\tt C}[\tbar{X}]}
}
\end{minipage}%

\begin{minipage}{.3\textwidth}
\quad\typicallabel{XXXXXX}
\infax[O-Field]
  {\Gamma\vdash {\tt f}\ty{\tt F} ~\ll~ {\tt f}\ty{\tt F}}
\end{minipage}%
\begin{minipage}{.7\textwidth}
\quad\typicallabel{XXXXXX}
\infrule[O-Method]
  {\tbar{G}=\tbar{G'} \andalso \Gamma,\tbar{x}\ty\tbar{G},{\tt d'}\vdash{\tt d} \andalso 
    {\tt y}~{\rm fresh} \andalso \Gamma,\tbar{x}\ty\tbar{G},{\tt d},{\tt y}\ty{\tt H}\vdash{\tt y}\subtype{\tt H'}}
  {\Gamma\vdash \msign{Y}{x}{G}{d}{H} ~\ll~ \msign{Y}{x}{G'}{d'}{H'}}
\end{minipage}%

\begin{minipage}{\textwidth}
\quad\typicallabel{XXXXXX}
\infrule[OK-Method]
  {\cdecl \andalso
    \Gamma=\tbar{X}\ty*,{\tt c},\this\ty{\tt C}[\tbar{X}],\tbar{Y}\ty*\\
    \wj{\Gamma,\tbar{x}\ty\tbar{G},{\tt d}}{{\tt d},\tbar{G},{\tt H}} \andalso
    \Gamma,\tbar{x}\ty\tbar{G},{\tt d}\vdash {\tt e}\ty{\tt E} \andalso
    {\tt y}~{\rm fresh} \andalso
    \Gamma,\tbar{x}\ty\tbar{G},{\tt d},{\tt y}\ty{\tt E}\vdash {\tt y}\subtype {\tt H} \\
    \tbar{X}\ty*,{\tt c},\this\ty{\tt D}[\tbar{E}],\tbar{Y}\ty*\vdash\this~\has~\msign{Y}{x}{G'}{d'}{H'} ~\Rightarrow~
    \Gamma\vdash \msign{Y}{x}{G}{d}{H} ~\ll~ \msign{Y}{x}{G'}{d'}{H'}}
  {\vdash\mdecl{Y}{x}{G}{d}{H}{e}~{\rm OK~in}~{\tt C}[\tbar{X}]}

\eat{
\infrule[OK-Method]
  {\cdecl \andalso
    \tbar{X}\ty*,{\tt c},\this\ty{\tt C}[\tbar{X}],\tbar{Y}\ty*,\tbar{x}\ty\tbar{G},{\tt d},{\tt y}\ty{\tt E}\vdash {\tt y}\subtype {\tt H} \\
    \tbar{X}\ty*,{\tt c},\this\ty{\tt D}[\tbar{E}],\tbar{Y}\ty*\vdash\this~\has~\msign{Y}{x}{G'}{d'}{H'}~\Rightarrow
    \left\{\begin{array}{@{}l@{}}
    \tbar{G}=\tbar{G'}\\
    \tbar{X}\ty*,{\tt c},\this\ty{\tt C}[\tbar{X}],\tbar{Y}\ty*,\tbar{x}\ty\tbar{G},{\tt d'}\vdash{\tt d}\\
    {\tt z}~{\rm fresh} \andalso \tbar{X}\ty*,{\tt c},\this\ty{\tt C}[\tbar{X}],\tbar{Y}\ty*,\tbar{x}\ty\tbar{G},{\tt d},{\tt z}\ty{\tt H}\vdash{\tt z}\subtype{\tt H'}
    \end{array}\right.}
  {\vdash\mdecl{Y}{x}{G}{d}{H}{e}~{\rm OK~in}~{\tt C}[\tbar{X}]}
}

\infrule[OK-Class]
  {\fields({\tt D}[\tbar{E}])=\tbar{g}\ty\tbar{G} \andalso
    \tbar{f}\cap\tbar{g}=\emptyset \andalso
    \wj{\tbar{X}\ty*}{{\tt c}} \andalso 
    \wj{\tbar{X}\ty*,{\tt c}}{{\tt D}[\tbar{E}]} \andalso 
    \wj{\tbar{X}\ty*,{\tt c},\this\ty{\tt C}[\tbar{X}]}{\tbar{F}} \andalso
    \tbar{M}~{\rm OK~in}~{\tt C}[\tbar{X}]}
  {\vdash \cdecl~\rm OK}
\end{minipage}%
\caption{\FXGL{\cdot} typing rules.}\label{fig:FX}
\end{figure*}


\begin{figure*}
\begin{minipage}{\textwidth}
\centering
\begin{tabular}{r@{\quad}rcl}
  (Constraint) & {\tt c} &{::=}& $\tt X\extends G$ \\
\end{tabular}
\end{minipage}%

\begin{minipage}{.4\textwidth}
\quad\typicallabel{XXXXXX}
\infrule[W-Bound]
	{\wj{\Gamma}{\tt X} \andalso \wj{\Gamma}{\tt G}}
	{\wj{\Gamma}{{\tt X}\extends{\tt G}}}

\infrule[X-Bound]
	{\Gamma\vdash{\tt X}\extends{\tt G},{\tt X}\extends{\tt H} \andalso
	  {\tt G}\neq{\tt H}}
	{\Gamma\vdash\false}
\end{minipage}%
\begin{minipage}{.6\textwidth}
\quad\typicallabel{XXXXXX}
\infrule[S-Bound]
	{\wj{\Gamma}{{\tt X}\extends{\tt G}} \andalso \Gamma\vdash {\tt X}\extends {\tt G}}
	{\Gamma,{\tt x}\ty{\tt X}\vdash {\tt x}\subtype {\tt G}}

\infrule[H-Bound]
	{\wj{\Gamma}{{\tt X}\extends{\tt G}} \andalso
	  \Gamma\vdash{\tt X}\extends{\tt G}
    \andalso
    \Gamma,{\tt x}\ty{\tt G}\vdash {\tt x}~\has~{\tt I}}
	{\Gamma,{\tt x}\ty{\tt X}\vdash {\tt x}~\has~{\tt I}}
\end{minipage}%
\caption{\FXGL{B}.}
\label{fig:FXGB}
\end{figure*}

\begin{figure*}
\begin{minipage}{\textwidth}
\centering
\begin{tabular}{r@{\quad}rcl}
  (Constraint) & {\tt c} &{::=}& ${\tt X}~\underline{\has}~\msign{Y}{y}{G}{c}{H}}$ \\
\end{tabular}
\end{minipage}%

\begin{minipage}{.45\textwidth}
\quad\typicallabel{XXXXXX}
\infrule[W-Struct]
	{\wj{\Gamma}{\tt X} \andalso \wj{\Gamma,\this\ty{\tt X},\tbar{Y}\ty*,\tbar{y}\ty\tbar{G},{\tt c}}{\tt d},\tbar{G},\tbar{H}}
	{\wj{\Gamma}{{\tt X}~\underline\has~{\tt I}}}

\infrule[X-Struct]
  {\methods({\tt C}[\tbar{A}])\ni\minst{Y}{y}{G}{c}{H}{e}}
  {\vdashS{\tt C}[\tbar{A}]~\underline\has~\msign{Y}{y}{G}{c}{H}}

\end{minipage}%
\begin{minipage}{.55\textwidth}
\quad\typicallabel{XXXXXX}
\infrule[H-Struct]
  {\vdash{\tt X}~\underline\has~{\msign{Y}{y}{G}{c}{H}} \andalso
    \theta={\tt x},\tbar{Z},\tbar{z}/\this,\tbar{Y},\tbar{y}}
  {\Gamma,{\tt x}\ty{\tt X}\vdash {\tt x}~\has~\msign{Z}{z}{G\theta}{c\theta}{H\theta}}

\infrule[X-Struct-False]
	{\Gamma,{\tt x}\ty{\tt X}\vdash{\tt x}~\has~{\tt I},{\tt x}~\has~{\tt J},{\tt I}~\not\ll~{\tt J},{\tt J}~\not\ll~{\tt I}}
	{\Gamma\vdash\false}
\end{minipage}%
\caption{\FXGL{S}.}
\label{fig:FXGS}
\end{figure*}


\begin{figure*}
\begin{minipage}{.34\textwidth}
\centering
\begin{tabular}{r@{\quad}rcl}
  (Generic type)& {\tt G} &{::=}& {\tt Q} \\
  (Expression) & {\tt e} &{::=}& ${\tt q}(\tbar{e})$ \\
  (Values) & {\tt v} &{::=}& ${\tt l}$ \\
  (Constraint term) & {\tt t} &{::=}& ${\tt q}(\tbar{t})$ \alt $\tt l$ \\
  (Constraint) & {\tt c} &{::=}& ${\tt p}(\tbar{t})$ \\  
\end{tabular}
\end{minipage}%
\begin{minipage}{.33\textwidth}
\vspace{-\bigskipamount}\quad\typicallabel{XXXX}
\infrule[W-Fun]
  {\wj{\Gamma}{\tbar{t}} \andalso {\tt q}\in\cal Q}
  {\wj{\Gamma}{{\tt q}(\tbar{t})}}

\infrule[W-Pred]
  {\wj{\Gamma}{\tbar{t}} \andalso {\tt p}\in\cal Q}
  {\wj{\Gamma}{{\tt p}(\tbar{t})}}
\end{minipage}%
\begin{minipage}{.33\textwidth}
\vspace{-\bigskipamount}
\quad\typicallabel{XXXX}
\infrule[W-Lit]
  {{\tt l}\in\cal Q}
  {\wj{}{{\tt l}}}

\infrule[W-Prim]
  {{\tt Q}\in\cal Q}
  {\wj{}{{\tt Q}}}
\end{minipage}%

\begin{minipage}{.34\textwidth}
\quad\typicallabel{XXXX}
\infrule[R-Fun]
	{\vdashQ{\tt q}(\tbar{v})=={\tt l}}
	{{\tt q}(\tbar{v})\derives {\tt l}}
	
\infax[T-Lit]
	{\vdash{{\tt l}\ty\Dom({\tt l})\{\self=={\tt l}\}}}
\end{minipage}%
\begin{minipage}{.66\textwidth}
\quad\typicallabel{XXXX}
\infrule[RC-Fun]
	{{\tt e}_i\derives {\tt e}'_i}
	{{\tt q}({\tt v}_1,\ldots,{\tt v}_{i-1},\ldots,{\tt e}_i,\ldots,{\tt e}_n)\derives {\tt q}({\tt v}_1,\ldots,{\tt v}_{i-1},{\tt e}'_i,\ldots,{\tt e}_n)}

\infrule[T-Fun]
	{\andalso \Gamma\vdash\tbar{e}:\Dom({\tt q})}
	{\Gamma\vdash{{\tt q}(\tbar{e})\ty\exty{\tbar{x}}{\Dom({\tt q})}{\Img({\tt q})}\{\self=={\tt q}(\tbar{x})\}}}
\end{minipage}%
\caption{\FXGL{Q}.}
\label{fig:FXGQ}
\end{figure*}


\paragraph{3. Lookup.} Figure~\ref{fig:lookup} specifies the fields and methods available on each type. In \FXGL{\cdot}, these are exactly those captured by the \fields{} and \methods{} predicates for class types and none for type parameters. This will change when we start considering constraints on type parameters.

\paragraph{4. Subtyping.} The subtyping relation is defined in Figure~\ref{fig:subtyping}.
Because we deal with dependent and existential types, we choose a somewhat unconventional notation for the subtyping relation that enables more compact inference rules. Rather than judgments of the form $\Gamma\vdash{\tt S}\subtype{\tt T}$, we consider judgments of the form $\Gamma,{\tt x}\ty{\tt S}\vdash{\tt x}\subtype{\tt T}$ which permit {\tt T} to depend on {\tt x}, hence make codependent types easier to work with. In fact, one could define classical subtyping as the following:
\infrule
	{\wj{\Gamma}{\tt S} \andalso \Gamma,{\tt x}\ty{\tt S}\vdash{\tt x}\subtype {\tt T} \andalso {\tt x}~\rm not~free~in~{\tt T}}
	{\Gamma\vdash{\tt S}\subtype{\tt T}}
or derive our relation from subtyping:
\infrule
	{\Gamma,{\tt x}\ty{\tt S}\vdash{\tt S}\{\self=={\tt x}\}\subtype{\tt T}}
	{\Gamma,{\tt x}\ty{\tt S}\vdash{\tt x}\subtype {\tt T}}

\paragraph{5. Typing.} The typing rules are specified in Figure~\ref{fig:FX}.

{\sc T-Var} is as expected, except that it asserts the constraint {\tt
self==x}, which records that any value of this type is known
statically to be equal to {\tt x}. This constraint is actually very
crucial. As we will see later, a key condition of the soundness of 
this type system is that constraint terms have singleton types, that is, the
type system retains enough information in the type of a term to permit
reconstructing the term from its type.

\eat{---as we shall see in the other rules, once we establish that
an expression {\tt e} is of a given type {\tt T}, we ``transfer'' the
type to a freshly chosen variable {\tt z}.  If, in fact, {\tt e} has a
static ``name'' {\tt x} (i.e., {\tt e} is known statically to be
equal to {\tt x}; that is, it has type {\tt T\{self==x\}}), then
{\sc T-Var} lets us assert that {\tt z:T\{self==x\}}, i.e., that {\tt z}
equals {\tt x}.
Thus {\sc T-Var} provides an important base case for
reasoning statically about equality of values in the environment.}

We do away with the three cast rules in \FJ{} in favor of a single
cast rule, requiring only that {\tt e} be of some type {\tt T}. At run time,
{\tt e} will be checked to see if it is actually of type {\tt G} (see
{\sc R-Cast} in Figure~\ref{fig:sos}).

{\sc T-Field} may be understood through ``proxy'' reasoning.
Given the context $\Gamma$, assume the receiver {\tt e} can
be established to be of type {\tt T}. Now, we do not know the run-time
value of {\tt e}, so we shall assume that it is some fixed but unknown
``proxy'' value {\tt x} (of type {\tt T}) that is ``fresh'' in that it
is not known to be related to any known value (i.e., those recorded
in $\Gamma$).  If we can establish that {\tt x} has a field {\tt f} of
type {\tt F} then we can assert that
${\tt e}.{\tt f}$ has type {\tt F} and, further, that it equals ${\tt x}.{\tt f}$
for some {\tt x} of type {\tt T}.
Hence, we can assert that ${\tt e}.{\tt f}$ has type 
$\exty{\tt x}{\tt T}{\tt F}\{\self=={\tt x}.{\tt f}\}$.

{\sc T-New} and {\sc T-Invk} have a similar structure to {\sc T-Field}: we use
proxy reasoning for the arguments of the construtor call or for the receiver and the arguments of the method
call.

Fields cannot be overriden thanks to the premise $\tbar{f}\cap\tbar{g}=\emptyset$ in rule {\sc OK-Class}. The {\sc O-Method} rule formalizes method overriding. A method $m1$ may be overriden by a method $m2$ with the same name, type parameters (modulo alpha-renaming), value parameters (modulo alpha-renaming), and value parameter types provided method $m2$ has a return type that is a subtype of $m1$'s return type and $m1$'s guard entails $m2$'s. Rule {\sc O-Field} is redundant at this point. It will matter to type system extensions.

{\sc OK-Method} and {\sc OK-Class} ensure that all types and constraints are well formed, that overriding rules are observed, and that the body of a method has a type that is a subtype of its declared type. Moreover, the guard on a class must entail the well-formedness of its supertype. which includes the guard of the superclass.

\subsection{\FXGL{B}}
We now turn to showing how \FGJ{}-style bounds can be supported in the \FXGL{\cdot} family.
\FXGL{B} is the language obtained by adding to \FXGL{\cdot} the productions and rules of Figure~\ref{fig:FXGB}.
The key idea is that information about type parameters can be accumulated through constraints. Specifically we introduce the ``extends'' constraint \mbox{${\tt X}\extends{\tt G}$}.

\FXGL{B} is built on top of a slightly more expressive constraint system $\cal B$ that permits propagating such constraints viewed as uninterpreted atoms of knowledge. We then add four rules to the type system respectively about well-formedness, inconsitent constraints, subtyping, and member lookup. Rule {\sc X-Bound} ensure we do not type check program that specify mutiple bounds for a given type parameters.\footnote{More exactely, multiple bounds can be specified provided these are syntactically identical.}

While elementary, this extension demonstrates the extensibility of the \FXGL{\cdot} and its mechanisms.

The soundness of the type system (i.e., its proof) is preserved because:
\begin{enumerate}
\item If $\Gamma,{\tt x}\ty{\tt X}\vdash{\tt x}\subtype{\tt G}$ and $\Gamma,{\tt x}\ty{\tt G}\vdash{\tt x}~\has~{\tt I}$ then there exists $\tt J$ such that $\Gamma,{\tt x}\ty{\tt X}\vdash{\tt x}~\has~{\tt J}, {\tt J} ~\ll~ {\tt I}$.
\item
If $\Gamma,{\tt x}\ty{\tt X}\vdash{\tt x}~\has~{\tt I},{\tt x}~\has~{\tt J}$ then $\Gamma,{\tt x}\ty{\tt X}\vdash {\tt I}~\ll~{\tt J},{\tt J}~\ll~{\tt I}$.
\end{enumerate}

\subsection{\FXGL{S}}
Adding structural subtyping constraints to \FXGL{\cdot} is very similar. First, we extend the constraint system: it must satisfy inference rule {\sc X-Struct} so that generic classes and methods may be instantiated upon the actual class types of the program. {\sc H-Struct} is straightforward. Finally, {\sc X-Struct-False} make sure programs with inconsistent structural constraints do not type check.

\subsection{\FXGL{Q}}

Turning \FXGL{\cdot} into a language with value-dependent types is straightforward. First, we assume we are given a constraint system $\cal Q$ with a vocabulary of primitive types ${\tt Q}$, predicates ${\tt p}$, functions ${\tt q}$, and literals ${\tt l}$ of these primitive types. Second, we extend the productions, operational semantics and type system of \FXGL{\cdot} with the productions and rules of Figure~\ref{fig:FXGQ}.


We denote $\Dom({\tt l})$ the primitive type of the literal {\tt l}. For simplicity, we assume each function {\tt q}
is a total mapping from ${\sf Dom}({\tt q})$ (a n-uple of primitive types) to ${\sf Img}({\tt q})$, that is, if $\vdash \tbar{v}\ty{\sf Dom}({\tt q})$ then there exists a unique literal $\tt l$ such that $\vdashQ {\tt q}(\tbar{v})=={\tt l}$ and moreover $\Dom({\tt l})={\sf Img}({\tt q})$.

For instance, if $\mathcal{Q}$ defines the type {\tt int}, integer literals, and the usual arithmetic operators, then we can declare:

\begin{xten}
class Count(n:int) extends Object {
  def inc():Count{self.n==this.n+1} =
  	new Count(this.n+1);
}
\end{xten}



\subsection{Results}
The following results hold for \FXGL{\cdot}.

\begin{theorem}[Subject Reduction] If ${\tt e} \derives {\tt e'}$ and $\Gamma \vdash {\tt e}\ty{\tt T}$ then there exists {\tt S} such that $\Gamma\vdash{\tt e'}\ty{\tt S}$. Morevoer, $\Gamma,{\tt x}\ty{\tt S} \vdash {\tt x}\subtype{\tt T}$.
\end{theorem}

\begin{theorem}[Progress]
If $\vdash {\tt e:T}$ then one of the following conditions holds:
\begin{enumerate}
\item {\tt e} is a value,
\item {\tt e} contains a cast sub-expression that is stuck,
\item there exists $\tt e'$ such that $\tt e\derives e'$.
\end{enumerate}
\end{theorem}

\begin{theorem}[Type soundness]
If $\vdash {\tt e}\ty{\tt T}$ and {\tt e}
reduces to a normal form then either ${\tt e'}$ contains a stuck cast sub-expression or ${\tt e'}$ is a value {\tt v} and there exists {\tt S} such that $\vdash {\tt v:S}$. Moreover, in that case, ${\tt x}\ty{\tt S}\vdash{\tt x}\subtype{\tt T}$.
\end{theorem}












\eat{



\section{Random bits}


In practice, it makes sense to distinguish the functions of the
constraint language from the functions of the base language.
One would define the {\sc T-Fun} typing judgment on a case-by-case
basis to
relate the interpretation of \xcd"q" as an expression to
its interpretation as a constraint term.

\FXD corresponds to the \CFJ calculus presented
in our prior work on constrained types~\cite{constrained-types}.  As described there, \Xten
supports equality constraints and has been extended with constraint
systems for Presburger arithmetic and for set constraints over
\Xten's array index domains (viz., regions).

\subsection{\FXG}
We now turn to showing how \FGJ{}-style generics can be supported in the \FX{} family.
\FXG{} is the language obtained by adding to \FXZ{} the
following productions:
\begin{center}
\begin{tabular}{r@{\quad}rcl}
  (Expression)& {\tt e} &{::=}& ${\tt C}\{{\tt c}\}$ \\
  (Value)& {\tt v} &{::=}& ${\tt C}\{{\tt c}\}$ \\
  (Path)& {\tt p} &{::=}& ${\tt x}$ \alt {\tt p}.{\tt f} \\
  (Type)& {\tt T} &{::=}& ${\tt p}$ \alt * \\
  (Constraint term)& {\tt t} &{::=}& ${\tt T}$ \\
  (Constraint) & {\tt c} &{::=}& ${\tt T}\extends {\tt T}$
\end{tabular}
\end{center}
\noindent
and deduction rules of Figure~\ref{fig:FXGB}.

First we introduce the ``type'' *. \FGJ{} method type
parameters are modeled in \FXG{} as normal parameters of type
*.\footnote{In concrete \Xten{} syntax type parameters are
distinguished from ordinary value parameters through the use of
``square'' brackets. This is particularly useful in implementing type
inference for generic parameters. We abstract these concerns away in
the abstract syntax presented in this section.}  Generic class
parameters are modeled as ordinary fields of type *, with
parameter bound information recorded as a constraint in the class
invariant. This decision to use fields rather than parameters is
discussed further in Section~\ref{sec:parameters-vs-fields}. In brief,
it permits powerful idioms using fixed but unknown types without
requiring ``wildcards''.

The set of well-formed types is now enhanced to permit some fixed but unknown
types {\tt x} as well as \emph{path types} (cf. \cite{scala}),
i.e., type-valued fields of objects as types.\footnote{But we will not permit invocations of methods with return type *\ to be 
used as types. This does indeed make sense, but developing
this theory further is beyond the scope of this paper.} We extend $\sigma$ in the obvious way:
%
\begin{center}
\begin{tabular}{l}
$	\sigma({\tt x}:*, \Gamma)=\sigma(\Gamma)$\\
$\sigma({\tt x}:{\tt y}, \Gamma)=\sigma(\Gamma)$\\
$\sigma({\tt x}:{\tt p}.{\tt f}, \Gamma)=\sigma(\Gamma)$
\end{tabular}
\end{center}
%
Reciprocally, we permit class types ${\tt C}\{{\tt c}\}$ to be
used as expressions. We type them accordingly ({\sc T-Type}). In
contrast, the ``type'' *{} is neither a valid expression nor
a class type: it has no field, method, subclass, or superclass.
It may however be constrained as usual as, for instance, in rule
{\sc T-Type}; that is to say, we permit equality constraints over types.\footnote{Type equality is just equality over uninterpreted functions.}

The key idea is that information about type-valued expressions can
be accumulated through constraints. Specifically we introduce 
the ``extends'' constraint ${\tt T}\extends{\tt U}$. It may be used, for
instance, to specify upper bounds on type variables or fields (path
types). In \FXG{}, users are permitted to specify ``$==$'' and ``$\extends$'' constraints
about type variables, fields, and class types.

\begin{example}
The \FGJ{} parametric method

\begin{xten} 
<T> T id(T x) { return x; }
\end{xten}
\noindent can be represented as
\begin{xten} 
def id(T: type, x: T): T = x;
\end{xten}
\end{example}

\begin{example}
\noindent The \FGJ{} class 
\begin{xten} 
class Comparator<B> {
  int compare(B y) { ... } }
class SortedList<T extends Comparator<T>> { 
  int m(T x, T y) { return x.compare(y); } }
\end{xten}
\noindent can be represented as
\begin{xtenmath} 
class Comparator(B: type) {
  def compare(y:B):int = ...; }
class SortedList(T: type)
    {T $\extends$ Comparator{self.B==T}} { 
  def m(x:T, y:T):int = x.compare(y); }
\end{xtenmath}
\end{example}

We require the underlying constraint system $\mathcal{G}$ to treat ``$\extends$'' as a partial order relation (reflexive, antisymmetric, and transitive). It is possible for a program to specify constraints incompatible with the class hierarchy, e.g., ${\tt x}\extends{\tt C}$ and ${\tt x}\extends{\tt D}$ if both class {\tt C} and class {\tt D} extend {\tt Object}. We therefore require $\mathcal{G}$ to treat as inconsistent all sets of constraints on type-valued variables that admit no valuations where these variables take on types as values.

The ``$\extends$'' constraint is used in two deduction rules. If
type {\tt T} extends type {\tt U}, then
\begin{itemize}
\item{\sc S-Extends}. {\tt T} is a subtype of {\tt U}. A method or constructor with argument type {\tt U} may be passed a parameter of type {\tt T}.
\item{\sc L-Extends}. If {\tt x} has type {\tt U} then {\tt x} has all the members of type {\tt T}. Note we only extend the ``$\underline\has$'' predicate that is used in typing judgments. On the other hand, the ``$\has$'' predicate used for method lookup in the operational semantics is not affected.
\end{itemize}

The modification of the lookup predicate is
necessary to permit typing method invocations with receivers of
generic types. It has the unfortunate side effect that we can no
longer ensure that type derivations---and hence types---are unique.
For instance, given the class definitions:
%
\begin{xten}
class A() extends Object { def m():A = new A(); }
class B() extends A { def m():B new B(); }
class C(f:type){this.f<=A} extends Object {}
class D(){this.f<=B} extends C { ..this.f.m().. }
\end{xten}
%
occurrences of $\this.{\tt f}$ in {\tt D} are bounded both by {\tt A} and {\tt B} hence 
$\this.{\tt f}.{\tt m}()$ may either be typed using the declaration of {\tt m} in {\tt A} or {\tt B}.

Another property of \FXG{} worth noticing is that casts can ``erase'' typing information.
Consider the program:
\begin{xten}
class C() extends Object {}
class D(f:type, g:this.f) extends Object {}
\end{xten}
Class {\tt D} has a type parameter {\tt f} and a value field {\tt g} of type {\tt f}.
Thanks to constraints, if
${\tt e}=\new~{\tt D}({\tt C},\new~{\tt C}())$,
then expression ${\tt e}.{\tt g}$ can be shown to
have type {\tt C}.
In contrast $({\tt e}~\as~{\tt D}).{\tt g}$ has type
$\exists {\tt x}:{\tt D}.{\tt x}.{\tt f}\{\self=={\tt x.g}\}$.
The type of $({\tt e}~\as~{\tt D}).{\tt g}$ is essentially ``unknown''
because the cast erased all information about it. In \Xten, we choose to shield users from existential types and only permit casts of the form $({\tt e}~\as~{\tt D}\{\self.{\tt f}=={\tt t}\})$ where {\tt t} is a type in scope (class type, type parameter, or path type).


\subsection{\FXGD} 

No additional rules are needed beyond those of \FXG{} and \FXD{}. This
language permits type and value constraints, supporting \FGJ{} style
generics and value-dependent types. All constraints but existential constraints are now user constraints.

\subsection{Results}
The following results hold for \FXGD supposing the program {\tt P} type checks.

\begin{theorem}[Subject Reduction] If $\Gamma$ is well formed and $\Gamma \vdash {\tt e:T}$ and ${\tt e} \derives {\tt e'}$, then
for some type {\tt S}, $\Gamma \vdash {\tt {\tt e}':{\tt S}},~{\tt S} \subtype {\tt T}$.
\end{theorem}

Values are of the form $\tt v ::= \new\ C(\bar{\tt v}) \alt {\tt d} \alt C\{c\}$.

\begin{theorem}[Progress]
If $\vdash {\tt e:T}$ then one of the following conditions holds:
\begin{enumerate}
\item {\tt e} is a value,
\item {\tt e} contains a cast sub-expression that is stuck,
\item there exists an $\tt e'$ s.t. $\tt e\derives e'$.
\end{enumerate}
\end{theorem}

\begin{theorem}[Type soundness]
If $\vdash {\tt e:T}$ and {\tt e}
reduces to a normal form ${\tt e'}$, then
either $\tt e'$ is a value {\tt v} and $\vdash {\tt v:S},{\tt S\subtype T}$ or
${\tt e'}$ contains  a stuck cast sub-expression.
\end{theorem}

\paragraph{Proof sketch.} The proof of the same results for a
formal language essentially equivalent to \FXD{} has been
reported in \cite{constrained-types}. We discuss here the key
insights that permit us to revise this proof in order to encompass \FXGD{}.
\begin{itemize}
\item Subject reduction. Having potentially multiple types for
an expression does not make the proof any harder as the subject
reduction theorem lets us choose {\tt S} among the possible types of ${\tt e}'$.

The main novelty of the \FXG{} type system is that it permits
the $\underline\has$ predicate to look for methods in arbitrary
superclasses or upper bounds of the type under scrutiny. This is
not so much a concern for fields as they cannot be overridden.
Because methods can, we must adapt the proof of subject reduction for the execution step corresponding to a method invocation (\RInvk).

First, we observe that the operational semantics rule for method
invocations (\RInvk) is required to employ the ``correct''
method for objects of class {\tt C}, that is, the first method
{\tt m} found on the inheritance path from class {\tt C} to
class {\tt Object} from the bottom up. Second, thanks
to overriding restrictions, we know that this method must have a
return type that is a subtype of any other method {\tt m}
defined in any superclass of {\tt C}. Finally, because
constraint sets incompatible with the class hierarchy are
inconsistent, we also know that the type of
$\new~{\tt C}(\tbar{e})$ cannot be constrained to have any upper
bound that is not {\tt C} itself or one of its superclasses.

We
therefore derive that any method instance one could use to type
the expression $\new~{\tt C}(\tbar{e}).{\tt m}(\tbar{a})$ has a
return type that is a supertype of the return type of the only
method instance that can be used to make a step of execution. We
assume the program type checks; hence, by {\sc OK-Method}, we
know that the actual residue ${\tt b}[\new~{\tt C}(\tbar{e}),\tbar{a}/\this,\tbar{x}]$ is guaranteed to have a
type that is a subtype of its declared type. Therefore, by
transitivity of the subtyping relation, we can derive that if
{\tt T} is a type of $\new~{\tt C}(\tbar{e}).{\tt m}(\tbar{a})$,
then there exists a type {\tt S} of
${\tt b}[\new~{\tt C}(\tbar{e}),\tbar{a}/\this,\tbar{x}]$
that is a subtype of {\tt T}.

\item Progress. \FXGD{} only differs from \FXD{} in that it
admits a new kind of expressions: {\tt C}\{{\tt c}\}. But these are also values, so the proof of progress is essentially unchanged.
\eat{
 {\tt d} ~$|$~ {\tt q}(\tbar{e}) ~$|$~ {\tt C}\{{\tt c}\}.
Both literals and class types are values, we just have to establish progress in the case of function calls (in the context of a proof by induction on the structure of the expression {\tt e}).
Assume ${\tt q}(tbar{e})$ is well typed then if all \tbar{e} are values progress is possible by rule {\sc R-Fun}. If not, then at leat one of the expressions is not a value and by induction hypothesis applied to this expression we can conclude that it is either stuck on a cast or can make a step of execution step. We conclude using rule {\sc RC-Fun} in the second case.}
\item{Type soundness}. Direct consequence of the previous two theorems.
\end{itemize}



\eat{We proceed by induction on the last rule used in the proof of ${\tt e} \derives {\tt e'}$. The key case is {\sc R-Invk}. Because of rule 


\begin{itemize}
\item {\sc RC-Field}. If $\Gamma\vdash S<:T$ then the fields of $T$ are the first fields of $S$.
\item {\sc RC-Invk-Recv}. If $\Gamma\vdash S<:T$ and $\Gamma,x:T\vdash x~\underline\has~m(y:U)\{c\}:V=a$ then there exists $d$ and $W$ such that $\Gamma,x:S\vdash x~\underline\has~m(y:U)\{d\}:W=b$ and $d$ entails $c$ and $W<:V$.
\item {\sc RC-Cast}. Straightforward.
\item {\sc RC-New-Arg}. If $\Gamma\vdash S<:T$ then $\exists x:T.U<:\exists x:S.U$.
\item {\sc RC-Invk-Arg}. Same.
\item {\sc R-Invk}. If $\Gamma,x:C\vdash x~\has~m(y:U)\{c\}:V=a$ then for all $d$ and $W$ such that $\Gamma,x:C\vdash x~\underline\has~m(y:U)\{d\}:W=b$ it is the case that $c$ entails $d$ and $V<:W$.
\item {\sc R-Field}. We have $\Gamma\vdash t:V$ for some $V$, $\Gamma\vdash \new~C(t).f_i:W_i\{\exists x:(\exists y:V.C\{\self==\new~C(y)\}).\self==x.f_i\}$ where $\Gamma\vdash C~\has~f_i:W_i$. We prove $\Gamma\vdash V_i<:W_i\{\exists x:(\exists y:V.C\{\self==\new~C(y)\}).\self==x.f_i\}$.
\item {\sc R-Cast}. Straightforward.
\end{itemize}
}


\eat{




\begin{figure*}
\vspace{-\bigskipamount}
\begin{minipage}{\textwidth}
\quad\typicallabel{XXXXXX}
\infax[W-True]
  {\wj{}{\tt true}}

\infax[W-False]
  {\wj{}{\tt false}}

\infrule[W-Conj]
	{\wj{\Gamma}{\tt c} \andalso 	\wj{\Gamma}{\tt d}}
	{\wj{\Gamma}{{\tt c},{\tt d}}}

\infrule[W-Eq]
	{}
	{\wj{\Gamma}{\tt t}=={\tt u}}

\infrule[W-Dep]
  {\wj{\Gamma}{\tt T} \andalso \wj{\Gamma, {\tt x}:{\tt T}}{\tt c}}
	{\wj{\Gamma}{\exists {\tt x}:{\tt T}.~{\tt c}}}
\end{minipage}
\caption{{\sf FXG} well-formed constraints}
\label{fig:well}
\end{figure*}





\begin{figure*}
\vspace{-\bigskipamount}
\begin{minipage}{\textwidth}
\quad\typicallabel{XXXXXX}
\infax[W-Var]
  {\wj{{\tt x}:{\tt T}}{\tt x}}

\infrule[W-Field]
	{\wj{\Gamma}{\tt t}}
	{\wj{\Gamma}{{\tt t}.{\tt f}}}

\infrule[W-New]
	{\wj{\Gamma}{\tbar{t}} \andalso \wj{\Gamma}{\tbar{G}}}
	{\wj{\Gamma}{\new~{\tt C}[\tbar{G}](\tbar{t})}}
\end{minipage}
\caption{{\sf FXG} well-formed constraint terms}
\label{fig:well}
\end{figure*}
}
\eat{
The syntax for constraints in \FXZ{} is specified in
Figure~\ref{fig:fxg-grammar}. As expected, constraints
relate property fields of objects. Neither casts
nor method invocations are permitted in constraints.

We distinguish a subset of these constraints as
{\em user constraints}---these are permitted to occur in
programs. For \FXZ{} the only user constraint permitted is the vacuous
{\tt true}. Thus the types occurring in user programs are isomorphic
to class types, and class and method definitions specialize to the
standard class and method definitions of \FJ{}. 

The constraints permitted by the syntax in
Figure~\ref{fig:fxg-grammar} that
are not user constraints are used to define the static and
dynamic semantics of \FXZ{} (see, e.g., rule \TField{} in Figure~\ref{fig:FX}).
The use of this richer constraint set as well as constrained and existential types is
not necessary in \FXZ; it simply enables us to present the static and dynamic
semantics once for the entire family of \FX{} languages,
specifying the other members of the family as extensions
to these core semantics.

Existential constraints are introduced for convenience only:
${\tt T}\{\exists {\tt x}:{\tt U}.~{\tt c}\}$ is equivalent to $\exists {\tt y}:{\tt U}.~{\tt T}\{{\tt c}[{\tt y}/{\tt x}]\}$ choosing {\tt y} not free in {\tt T}.
}

\eat{
Because of existential and dependent types we adopt a nonconventional notation for subtyping that we explain and motivate below. 
\{\infrule[S-Sub]
	{\wj{\Gamma}{\tt S} \andalso \Gamma,{\tt x}\ty{\tt S}\vdash{\tt x}\subtype {\tt T} \andalso {\tt x}~\rm not~free~in~{\tt T}}
	{\Gamma\vdash{\tt S}\subtype{\tt T}}

The typing rule for casts ({\sc T-New}) in Figure~\ref{fig:FX} specifies that if the arguments $\tbar{e}$ have type $\tbar{V}$ then $\new~{\tt C}(\tbar{e})$ has type $\exists\tbar{y}:\tbar{V}.~{\tt C}\{\self==\new~{\tt C}(\tbar{y})\}$, therefore {\RCast} requires this particular type to be a subtype of {\tt T}.
In \FXZ, this
test simply involves checking that the class of which the object is an
instance is a subclass of the class specified in the given type; in
languages with richer notions of type this operation may
involve run-time constraint solving using the properties of the object.
See Section~\ref{sec:casts} for further discussion of the casts,
including decidability issues.
}

\eat{
Each language in the family is defined over a given input constraint system $\mathcal{X}$ that is required to support the trivial constraint \true{}, conjunction, and equality on constraint terms. Given a program {\tt P}, we now show how to derive from $\mathcal{X}$ a larger deduction system that captures the object-oriented structure of {\tt P} and lets us decide whether {\tt P} is well typed.

\medskip
}

}