\Xten{} is a class-based object-oriented language.
that provides both dependent and generic types.
The language has a sequential core similar to Java or Scala, but
also
constructs for concurrency and distribution.
This section describes the design of generics for \Xten,
including several alternative designs.

%\footnote{We plan
%to support traits in a future version of the language.}
\todo{emph interactions with data constraints}

\subsection{Handling value types}

\Xten's type system provides both reference types and value
types.  An instance of a reference type is an object on the
heap.  Variables of reference type may be \xcd"null".
In contrast, an instance of a value type might live on the call stack
and can never be \xcd"null".  Java's generics support only 
instantiation on reference types (subtypes of \xcd"Object") and
not primitives.  Languages like C$\sharp$ and  PolyJ~\cite{java-popl97}, in contrast, supports
instantiation on both kinds of type.

\todo{common base type}

\subsection{Type constraints}

To permit genericity, variables \Xcd{X} must be admitted over types.
The choice of type variables is discussed below.  We assume here
that classes have a means of introducing new type variables.
For instance,
the class \xcd"List" introduces a type variable \xcd"X"
representing the list's element type.
Here, we ask: how should type variables be constrained?
\Xten already supports constraints over values, so it is natural
to extend these to constraints over types.

\subsubsection{Subtyping and equality constraints}

In class-based OO languages such as Java,
types are equipped with a partial
order (the \emph{subtyping} order) generated from the user program
through the ``\Xcd{extends}'' relationship.  Further, each type is
associated with member fields and methods, each with their
name and signature.
This motivates a very natural constraint system on types.  For a type
variable \Xcd{X} we should be able to assert the constraint \Xcd{X}
$\extends$ \Xcd{T}: a valuation (mapping from variables to types) realizes
this constraint if it maps \Xcd{X} to a type that extends \Xcd{T}.
Constraints on types can specify either subtype (\xcd"<="),
supertype (\xcd">="), or equality bounds (\xcd"==").

Constraints occur in several places in the \Xten syntax.  They
are of course permitted in constrained types \xcd"C{c}".
Constraints may also be used as \emph{class invariants}, 
which are constraints on the class's properties and other
final variables in scope.
The class invariant must be established by the class's
constructor and subsequently holds for all instances of the
class.

Using subtyping constraints in the class invariant provides a
means to bound the type variables introduced by the class
declaration.  Subtyping constraints in constrained types
\xcd"C{c}", can bound
type-valued members of the base type \xcd"C".

Methods and constructors
may also have additional
constraints, or \emph{guards}, on their parameters.  A
guard must be satisfied by the caller of the method and
will hold throughout its body.  Type constraints used in the
method guard restrict the types of the arguments or the method
receiver.  
This feature is similar to optional methods in CLU~\cite{clu} and to generlized type constraints in C$\sharp$~\cite{emir06}.

For instance, given a list of \xcd"T", one could define a
method \xcd"print" with a guard that requires that \xcd"T" be a
subtype of \xcd"Printable":
\begin{xtenmathnoindent}
  def print(){T <= Printable} {
    head.print();
    Console.OUT.print(" ");
    tail.print();
  }
\end{xtenmathnoindent}
This constraint ensures that the \xcd"head" field of type
\xcd"T" has a \xcd"print()" method.

\subsubsection{Structural consraints}
We considered other constraints on types as well.
One should
be able to require that a type have a
particular member---a field with a given name and type, or a method
with a given name and signature.
We introduce the constraints 
\Xcd{T} \Xcd{has} \Xcd{f:T} and \Xcd{T} \Xcd{has}
\Xcdmath{m($\tbar{x}\ty\tbar{S}$):T} to express this.
These
constraints allow one to define \xcd"List.print" as:
\begin{xtenmathnoindent}
  def print(){T has print(): void} {
    head.print();
    Console.OUT.print(" ");
    tail.print();
  }
\end{xtenmathnoindent}

Structural constraints on types are found in many languages.
For instance,
Haskell supports type
classes~\cite{haskell,haskell-type-classes}.
%
%ML's module system allows modules to be constrained by
%structural signatures~\cite{ml}.
In Modula-3, type equivalence is structural
rather than nominal as in object-oriented languages of the C
family (e.g., C++, Java, and \Xten{}).
Unity~\cite{malayeriIntegrating08}
is a Java-like language with both nominal and structural subtyping.

In the class invariant, a structural constraint can bound the
class's type variables, similarly to 
the language PolyJ~\cite{java-popl97}, which allows type
parameters to be
bounded using
structural \emph{where clauses}~\cite{where-clauses}.
For example, a sorted list class
could be written as follows in PolyJ:
{
\begin{xtennoindent}
class SortedList[T]
    where T {int compareTo(T)} {
  void add(T x) {... x.compareTo(y) ...}
  ...
}
\end{xtennoindent}}
The \xcd"where" clause states that the type parameter
\xcd"T" must have a
method \xcd"compareTo" with the given signature.
\xcd"SortedList" can be instantiated on any type
that provides the method with the given signature.
With nominal bounds, \xcd"SortedList" could only require that
its parameters implement an interface such as \xcd"Comparable".

\subsubsection{Default values}

\Xten's type system provides both reference types and value
types.  An instance of a reference type is an object on the
heap.  Variables of reference type may be \xcd"null".
In contrast, an instance of a value type might live on the stack
and can never be \xcd"null".  In languages like Java with
primitive types, every type has a default value---\xcd"null" for
reference types and \xcd"false" or \xcd"0" for primitive
types---used to initialize arrays of that type.
In \Xten, some types do not have a default.  For example
\xcd"int{self>0}" does not contain the value \xcd"0".
Consequently, a useful constraint is \xcd"T has default", which
holds if the type \xcd"T" has a default value.

\subsubsection{instanceof constraints}

Lastly, we consider constraints of the form \xcd"x" \xcd"instanceof" \xcd"T".
By relating types and values in a single constraint, 
these constraints provide considerable expressive power.
For instance, 
consider the class declaration:
\begin{xtennoindent}
class C {
  def equals[T](x:T) {this instanceof T} = (this==x);
}
\end{xtennoindent}
The \xcd"equals" method can be called with any object
that is a supertype of \xcd"C".

\xcd"instanceof" constraints can be used to build intersection
types, e.g., \xcd"Object{self instanceof A, self instanceof B}"

\subsection{Type variables}
\label{sec:type-properties}
\label{sec:variance}

To permit genericity, variables \Xcd{X} must be admitted over types.
Languages such as Java~\cite{Java3} and
Scala~\cite{scala} introduce \emph{type parameters} on classes
and methods.
Constrained types suggest another approach:
generalizing properties to include type-valued properties.
A \emph{type property}
is a final object member initialized at construction-time with a
concrete type.  

\subsubsection{Type properties}

\label{sec:usability}
\label{sec:parameters-vs-fields}

Like normal value properties, type properties
can be used in constrained types through the variable \xcd"self".
%
This immediately suggests use-site variance constraints on type
properties.
The type of a list of integers, say, can be written as
\xcd"List{self.T==int}".  
Subtyping constraints, then, may be used to
provide \emph{use-site variance} constraints.
Use-site variance based on structural virtual types was proposed by
Thorup and Torgerson~\cite{unifying-genericity} and extended for
parametrized type systems by Igarashi and
Viroli~\cite{variant-parametric-types}.  The latter type system lead
to the development of wildcards in
Java~\cite{Java3,adding-wildcards,wildcards-safe}.  Constrained
type properties
have similar expressive power.

Consider the following subtypes of \xcd"List":
\begin{itemize}
\item \xcd"List".  This type has no constraints on the type
property \xcd"T".
Any type that constrains \xcd"T"
is a subtype of \xcd"List".  The type \xcd"List" is equivalent to
\xcd"List{true}".
%
For a \xcd"List" \xcd"v", the return type of the \xcd"get" method
is \xcd"v.T".
Since the property \xcd"T" is unconstrained,
the caller can only assign the return value of \xcd"get"
to a variable of type \xcd"v.T" or of type \xcd"Object".

\item \xcd"List{T==float}".
The type property \xcd"T" is bound to \xcd"float".
For a final expression \xcd"v" of this type,
\xcd"v.T" and \xcd"float" are equivalent types and can be used
interchangeably.
The syntax \xcd"List[float]" is used as
shorthand for \xcd"List{T==float}".

\item \xcdmath"List{T$\extends$Collection}".
This type constrains \xcd"T" to be a subtype of \xcd"Collection".
All instances of this type must bind \xcd"T" to a subtype of
\xcd"Collection"; for example \xcd"List[Set]" (i.e.,
\xcd"List{T==Set}") is a subtype of
\xcdmath"List{T$\extends$Collection}" because \xcd"T==Set" entails
\xcdmath"T"
\xcdmath"$\extends$"
\xcdmath"Collection".
%
If \xcd"v" has the type \xcdmath"Vector{T$\extends$Collection}",
then the return type of \xcd"get" has type \xcd"v.T", which is an unknown but
fixed subtype of \xcd"Collection"; the return value can be
assigned into a variable of type \xcd"Collection".

\item \xcdmath"Vector{T$\super$String}".  This type bounds the type property
\xcd"T"
from below.  For a \xcd"Vector" \xcd"v" of this type, any
supertype of \xcd"String" may flow into a variable of type \xcd"v.T".
The return type of the \xcd"get"
method is known to be a
supertype of \xcd"String" (and implicitly a subtype of \xcd"Object").
\end{itemize}

However,
type properties have a number of usability issues.
The key difference between type parameters and type properties
is that type properties are
instance \emph{members} bound during object construction.  Type
properties are thus accessible through expressions---\xcd"e.T" is
a legal type (if \xcd"e" is final)---and are inherited by subclasses.
These features gives type properties more expressive power than
type parameters, as we shall describe below; however, because they 
provide similar functionality with often subtle distinctions,
type properties can be difficult to use, especially for novices,
and requires more care in the design.

One usability issue is that
type properties, like normal value properties, are inherited.
The language design needs
to account for ambiguities introduced when the same name is
used for different type properties declared in or inherited into a class.
These can be disambiguated
by ``casting'' the target up to the desired supertype,
e.g., \xcd"(e as C).X" specifies
the property \xcd"X" inherited from \xcd"C", or by introducing a
renaming mechanism.

Another issue is the interaction of type properties and run-time
type representations.  If type properties are not erased from
the run-time representation (as
type parameters as in Java, for instance), then a naive user
might waste storage by unnecessarily declaring properties.
As an example, in the following hypothetical code extended with
type properties (declared as normal properties with the ``type''
\xcd"*"),
\xcd"HashMap"  inherits the properties \xcd"K" and \xcd"V" from
\xcd"AbstractMap".
\begin{xten}
class AbstractMap(K:*, V:*) {
  abstract def get(K): V;
  abstract def put(K, V): V;
}

class HashMap implements Map {
  def get(k: K): V = ...;
  def put(k: K, v: V): V = ...;
}
\end{xten}
A user more familiar with type parameters might declare
\xcd"HashMap" as follows:
\begin{xten}
class HashMap(K:*,V:*) implements Map(K,V) {
  def get(k: K): V = ...;
  def put(k: K, v: V): V = ...;
}
\end{xten}
This declaration would introduce a new pair of type properties
named \xcd"K" and
\xcd"V" that shadow the inherited properties.
A na{\"\i}ve implementation of type properties would store run-time
type information for all four properties in each instance
of \xcd"HashMap".

\paragraph{Virtual types.}

Type properties provide expressive power much like 
\emph{virtual types}~\cite{beta,mp89-virtual-classes,ernst06-virtual}
, but they can also
be constrained at the use-site,
can be refined on a per-object basis without explicit subclassing,
and can be refined contravariantly as well as covariantly.

Thorup~\cite{thorup97}
proposed adding genericity to Java using virtual types.  For example,
a generic \xcd"List" class can be written as follows:
{
\begin{xten}
abstract class List {
  abstract typedef T;
  T get(int i) { ... }
}
\end{xten}}
\noindent
The virtual type \xcd"T" is unbound in \xcd"List", but 
can be refined by binding \xcd"T" in a subclass:
{
\begin{xten}
abstract class NumberList extends List {
  abstract typedef T as Number;
}
class IntList extends NumberList {
  final typedef T as Integer;
}
\end{xten}}
\noindent
Only classes where \xcd"T" is final bound, such as \xcd"IntList",
can be non-abstract.
%
The analogous definition of 
\xcd"List" in \Xten{} using type properties is as follows:
{
\begin{xten}
class List[T] {
  def get(i: int): T { ... }
}
\end{xten}}

\noindent
Unlike the virtual-type version,
the \Xten{} version of \xcd"List" is not abstract;
\xcd"T" need not be instantiated by a subclass because it can be
instantiated on a per-object basis.
Rather than declaring subclasses of \xcd"List",
one uses the constrained subtypes
\xcdmath"List{T$\extends$Number}" and \xcd"List{T==Integer}".

Type properties can also be refined contravariantly.
For instance, one can write the type \xcdmath"List{T$\super$Integer}".

Dependent classes~\cite{dependent-classes} generalize virtual
classes to express similar semantics via parameterization rather
than nesting.  Virtual classes depend only other their enclosing
instance; dependent classes, in contrast, depend on any number
of objects in which they are parameterized.  With type
properties, classes are not parameterized on their values;
rather properties are members and types are constructed by
constraining these properties.  Parameterization can be 
encoded with type properties using equality constraints.

\paragraph{Self types.}

Type properties can also be used to support a form of self
type~\cite{bruce-binary,bsg95}.
%
Self types can be implemented by introducing a
type property \Xcd{type} to the root of the class hierarchy,
\Xcd{Object}:
\begin{xtenmath}
class Object(type:*){type <= Object} { $\dots$ }
\end{xtenmath}

\noindent
For any final path expression \Xcd{p}, the type
$\Xcd{p}.\Xcd{type}$ represents all instances of the fixed,
but statically unknown, run-time class referred to by \Xcd{p}.
Scala's path-dependent types~\cite{scala-book} and J\&'s
dependent classes~\cite{nqm06}
take a similar approach.

Self types are achieved by
constraining types so that if a path expression \Xcd{p}
has type \Xcd{C}, then
$\Xcd{p}.\Xcd{type} \subtype \Xcd{C}$.
In particular, one can add the class invariant
$\Xcd{this}.\Xcd{type} \subtype \Xcd{C}$ to every class \Xcd{C}.
This invariant ensures that
$\Xcd{this}.\Xcd{type}$ is a subtype
of the lexically enclosing class.

The property must be initialized to the given class, so, without
further language support, one must create an instance of
\xcd"Object" with \xcd"new" \xcd"Object(Object)" to initialize
the \xcd"type" property.

\subsubsection{Type parameters}

Most OO languages provide genericity through type parameters on
classes and methods.  The development of a nominal OO type
system with type parameters is now standard (cf.  FGJ~\cite{FJ}).

Scala~\cite{scala} supports definition-site variance
annotations:
a parameter may be declared invariant, covariant, or
contravariant.
If the parameter \xcd"X" of a class \xcd"C" is covariant,
then if \xcd"S" is a subtype of
\xcd"T", the type \xcd"C[S]" is a subtype of \xcd"C[T]".
Similarly, if \xcd"X" is contravariant,
\xcd"C[T]" is a subtype of \xcd"C[S]".
Invariant parameters are the default; a covariant parameter is
declared by prepending ``\xcd"+"'' to the parameter name in the
class header; a contravariant parameter is declared by
prepending ``\xcd"-"''.  The usage of variant parameter types is
restricted to ensure the subtyping relation holds.

Java~\cite{Java3}, by contrast, supports use-site variance through wildcards.
This has a number of usability problems~\cite{wildcards-are-evil},
which also occur with constrained type properties, above.

\subsection{Overloading and dispatch}

The next question to address is the overloading semantics for
methods with constraints on formal parameters and with method
guards.  This issue was considered in non-generic \Xten but was
revisited in light of type constraints.

One option is to ignore constraints when checking for
overloading.  This means that \xcd"m(int{self==0})" and
\xcd"m(int{self==1})", for instance, are considered to have the
same signature; if both occur within the same class, a
compile-time error occurs.

Another option is to allow the
overloading: methods are resolved at compile-time, based on the
constraints.  It is an error if a call could resolve to more
than one method.  One consideration is whether to rule out overlapping
methods (e.g., \xcd"m(int{self>=0})" and \xcd"m(int{self==1})")
or to permit them and either to have the caller resolve the
ambiguity (which might not be possible), or to prioritize the
methods.

\todo{Use type constraints, not value constraints}

Allowing the overloading complicates method overriding by
introducing partial overrides.
Consider:
\begin{xtennoindent}
class A {
  def m(x:int{self>=0}) = ...; // 1
  def m(x:int{self<=0}) = ...; // 2
}
class B {
  def m(x:int{self==0}) = ...; // 3
}
\end{xtennoindent}
\noindent
\xcd"B.m"'s constraint on \xcd"x" partially overrides the
constraint on both \xcd"m" methods of \xcd"A".  Given a variable
\xcd"b" of type \xcd"B": \xcd"b.m(-1)" invokes \xcd"A.m" (method 1),
\xcd"b.m(0)" invokes \xcd"B.m" (method 3), and \xcd"b.m(1)"
invokes \xcd"A.m" (method 2).  Users can be confused by the
semantics.

Finally, one could support a form of predicate
dispatch~\cite{jpred}, selecting the method to invoke by
\emph{dynamically} evaluating the method guard.  Predicate
dispatch generalizes multi-method dispatch.  With type
constraints, multi-method dispatch can be simulated.
\todo{example}

\subsection{Implementation}

Finally, we turn to the implementation of generics.
\Xten's compiler targets both Java and C++ output.
To implement a generic class \xcd"C[X]" one can either generate a single 
class for \xcd"C" in the target language or can generate a class per
instantiation
\xcdmath"C[T$_1$]",
\xcdmath"C[T$_2$]", \dots.
\xcdmath"C[T$_k$]".
The former approach reduces the amount of generated code; the
latter enables specialization based on the type arguments to
\xcd"C".  Hybrid approaches are possible, also.

Java's approach is to erase type parameters and to use the homogenous
translation.  Erasure admits more dynamic errors because
it permits, for instance, a \xcd"C<A>" to be cast to \xcd"C<B>".
Retrieving a field of static type \xcd"B" could cause a run-time
type error when an \xcd"A" is returned instead.

Other languages, such as C++, use the heterogeneous
translation.  Following NextGen~\cite{nextgen},
Fortress~\cite{fortress} takes this approach as well, reducing
(static) code bloat by instantiating generics at run time.

A compromise approach is to specialized for only a few parameter
types, for example the primitive types.

Unlike Java's generics, the \Xten does not erase type
parameters.  This allows \Xten to support run-time casts to 
generic types, including types instantiated on constrained
types.
With
non-generic constrained types, casts like \xcd"r" \xcd"as"
\xcd"Region{rank==k}" can be implemented by
checking the run-time class of the value being
cast---\xcd"r" \xcd"instanceof" \xcd"Region"---and then
evaluating the constraint---\xcd"r.rank==k".

However, the issue is more subtle with generic casts.
For instance, to implement
\xcd"a" \xcd"as" \xcd"Array[int{self>=0}]"
one must check at run time that the concrete type used to instantiate
the \xcd"Array"'s type parameter is equivalent to
\xcd"int{self>=0}".  This check could involve a run-time
entailment check, 
breaking the phase distinction between
compile time and run time for constraint solving.

\subsection{Run-time casts}
\label{sec:casts}

While constraints are normally solved at compile time, 
constraints can be evaluated at run time by using casts.
The expression 
\xcd"xs as List{length==n}" checks not only 
that \xcd"xs"
is an instance of
the \xcd"List" class, but also that \xcd"xs.length" equals \xcd"n".
A \xcd"ClassCastException" is thrown if the check fails.
%
In this example, the test of the constraint does not require
run-time constraint
solving; the constraint can be checked by simply
evaluating the \xcd"length" property of \xcd"xs" and comparing against \xcd"n".

However, the situation is more complicated when casting to a
generic type.  Unlike Java, \Xten does not erase type
parameters at run-time.  Instead each instance of a generic type
contains a description of the types that its parameters are
instantiated upon.  This extra run-time type information
enables checked casts to generic types.

The implementation becomes complicated when variant type
parameters are permitted.
While our formalism does not model
variance, \Xten does support them, as described in
Section~\ref{sec:lang}.
Consider a declaration of class \xcd"C" with a covariant type
parameter \xcd"X":
\begin{xtenmath}
class C[+X](x: X) {
   def this(y: X) { property(y); }
}
\end{xtenmath}
\noindent
Because the static type of an expression may involve one or more
type parameters,
checking if an expression is an instance of, say, \xcd"C[A{c}]"
may require a run-time constraint entailment test.  That is, if a value \xcd"v"
has run-time type \xcd"C[B{d}]", then because \xcd"C"'s
parameter is covariant,
\xcd"v" \xcd"as" \xcd"C[A{c}]" must check that \xcd"B" is a subclass of
\xcd"A" and that \xcd"d" entails \xcd"c".\footnote{If the
\xcd"X" parameter were declared invariant, the cast would only
need to check that \xcd"A" and \xcd"B" are the same class and
that \xcd"d" and \xcd"c" are equivalent, which might be
accomplished by representing constraints at run-time in a
canonical form.}

One approach is to restrict the language 
to rule out casts to type parameters 
and to generic types with subtyping constraints, ensuring that
entailment checks are not needed at run time.

Alternatively, 
the constraint solver could be embedded into the runtime system.
This is the solution used in the \Xten{} implementation; however, this
solution can result in inefficient run-time casts
if entailment checking for the given constraint system is expensive.

We are exploring alternative implementations or future work.

\eat{
A different approach to have the compiler pre-compute the results of
entailment checks.
This might be done by analyzing the program to identify which pairs of
constraints might be tested for entailment at run time and then generating a
graph were each node is a constraint and there is a directed edge between
nodes in an entailment relationship.  Run-time entailment
checking can then be implemented as reachability checking. 
This solution is a whole program analysis; all
constraints must be visible to generate the graph.
\todo{
We leave the design of this analysis for future work.
}

\todo{this is vague.  and very likely wrong.}
If \xcd"e as T" occurs in the program text and \xcd"e" has
type \xcd"U", the analysis identifies all \xcd"new" expressions
$a$,
that create a subtype of \xcd"U" and identifies all type
expressions $t$ that could instantiate \xcd"T".
For each pair $(a,t)$, the analysis checks that the
type of $a$ (determined by the constructor invoked by $a$) is a subtype of $t$.
Since $a$ and $t$ may be in different environments, \xcd"e" must
be substituted for \xcd"this" in $a$ and \xcd"self" in $t$.
}

\eat{
Another option is to test objects cast to \xcd"T" not for membership in the
type \xcd"T", but rather to test against the
\emph{interpretation} of \xcd"T".
Observe that
if an instance of a generic class \xcd"C[X]"
is a member of the type \xcd"C{X==U}", then
all fields ${\tt f}_i$ of the instance with
declared type ${\tt S}_i$ contain values
that are instances of ${\tt S}_i[{\tt U}/{\tt X}]$.
For example, given the following declaration of class \xcd"List":
{
\begin{xten}
class List[X] {
  val head: X;
  val tail: List[X];
}
\end{xten}
}
\noindent
if \xcd"xs" is an instance of \xcd"List{X==String}", then
by checking that \xcd"xs.head" is an instance of \xcd"String"
and \xcd"xs.tail" is, recursively, an instance of \xcd"List{X==String}".
This property can be exploited by implementing cast to check the
types of all fields of the object.
For this check to be sound, it is vital that all fields whose
type depends on the type property \xcd"X" be transitively final;
otherwise, the test is not invariant---the
result of the test could change as the data is mutated.
Care must also be taken to implement the
test so that it terminates for cyclic data structures.
This implementation is inefficient for large data structures.

This solution has a more permissive semantics
than those implemented in \Xten or \FXGL{Q}.
The difference is best illustrated by considering an empty generic class:
{
\begin{xten}
class Nil[X] { }
\end{xten}
}
\noindent
In this case, there is no field of type \xcd"X" to test;
therefore, an object instantiated as \xcd"Nil[int]" 
can be considered a member of \xcd"Nil[String]". 
However, the solution remains sound:
Given a class \xcd"C[X]" and an expression \xcd"e" of type \xcd"t.X", if
a run-time check finds that \xcd"t" has type \xcd"C[T]", 
the compiler \emph{cannot} use this information to derive 
that \xcd"e" has type \xcd"T".
We leave to future work a proof of this claim.
}


\subsection{X10 design decisions}

Nominal bounds.

Use closures if you want structural bounds.

Type variables: parameters.  Properties are just too unfamiliar.
Usability outweights expressive power. 

Erase constraints.

Variance: def site.

