Modern object-oriented languages such as X10 require a rich framework for types
capable of expressing value-dependency, type-dependency and supporting
pluggable, application-specific extensions.

We generalize the {\em constrained types} approach we have presented
earlier to handle type-dependency.  Constraint systems formalize
systems of partial information.  Constrained types are formulas
\Xcd{C\{c\}} where \Xcd{C} is the name of a class or an interface and
\Xcd{c} is a constraint on the immutable instance state of 
\Xcd{C} (the {\em properties}). The framework is parametrized by 
a constraint system $\cal C$ of interest, and permits additional
constraint systems to be added by the programmer.

The basic idea is to formalize the essence of nominal object-oriented
types as a constraint system, and to permit both value and type
properties and parameters.  Type-generic dependence is now expressed
through constraints on these properties and parameters.  Type-valued
properties are required to have a run-time representation---the
run-time semantics is not defined through erasure. Runtime casts are
permitted through dynamic code generation.

Many type systems for object-oriented languages developed over the
last decade can be thought of as constrained type systems in this
formulation.

The paper makes the following contributions: (1) We show how to
accomodate type-dependent object-oriented types within the framework
of constrained types. (2) We develop a core formal calculus \gxx,
illustrating the treatment of type-dependece, and establish type
soundness.  (3) We discuss the specific design and implementation of
the type system for X10 based on constrained types.  The type system
integrates and extends the features of nominal types, virtual types,
self types, and Scala's path-dependent types.


final m 93
gym 100
ap science 94
spanish 94
studio art 92
english 91
global 96
regent sc ... 92
avg is 93.66