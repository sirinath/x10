
\todo{Move some of this to Section~\ref{sec:related}}

\subsection{Type properties versus type parameters}


Type properties are similar, but not identical to type parameters.  The
differences may potentially confuse programmers used to Java generics or C++
templates.  The key difference is that type properties are instance members and
are thus accessible through access paths: \xcd"e.T" is a legal type.

Type properties, unlike type parameters, are inherited.
For example, in the following code, \xcd"T" is defined in \xcd"List"
and inherited into \xcd"Cons".  The property need not be
declared by the \xcd"Cons" class.
\begin{xten}
class List[T] { }
class Cons extends List {
    def head(): T = { ... }
    def tail(): List[T] = { ... }
}
\end{xten}
The analogous code for \xcd"Cons" using type parameters would be:
\begin{xten}
class Cons[T] extends List[T] {
    def head(): T = { ... }
    def tail(): List[T] = { ... }
}
\end{xten}
% This code is perfectly acceptable in X10 as well, but introduces a redundant
% type property \xcd"T" equal to the \xcd"T" inherited from \xcd"List".

We can make the type system behave as if type properties were
type parameters very simply.  We need only make the syntax \xcd"e.T"
illegal and permit type properties to be accessible only
from within the body of their class definition via the implicit \xcd"this"
qualifier.

\subsection{Wildcards}

Wildcards in Java~\cite{Java3,adding-wildcards} were motivated
by the following example (rewritten in X10 syntax)
from \cite{adding-wildcards}.
Sometimes a class needs a field or method
that is a list, but we don't care what the element type is.
For methods, one can give the method a type parameter:
\begin{xten}
def aMethod[T](list: List[T]) = { ... }
\end{xten}
This method can then be called on any \xcd"List" object.
However, there is no way to do this for fields since they
cannot be parametrized.
Java introduced wildcards to allow such fields to be
typed:
\begin{xten}
List<?> list;
\end{xten}
In X10, a similar effect is achieved by not constraining the
type property of \xcd"List".
One can write the following:
\begin{xten}
list: List;
\end{xten}
Similarly, the method can be written without type parameters by
not constraining \xcd"List":
\begin{xten}
def aMethod(list: List) = { ... }
\end{xten}

In X10, \xcd"List"
is a supertype of
\xcd"List[T]" for any \xcd"T",
just as in Java
\xcd"List<?>" is a supertype of
\xcd"List<T>" for any \xcd"T".
This follows directly from the definition of the type \xcd"List"
as \xcd"List{true}", and the type \xcd"List[T]"
as \xcd"List{X==T}", and the definition of subtyping.

Wildcards in Java can also be bounded.
We achieve the same
effect in X10 by using type constraints.
For instance, the following Java declarations:
\begin{xten}
void aMethod(List<? extends Number> list) { ... }
<T extends Number> void aParameterizedMethod(List<T> list) { ... }
\end{xten}
may be written as follows in X10:
\begin{xten}
def aMethod(list: List{T <: Number}) = { ... }
def aParameterizedMethod[T{self <: Number}](list: List[T]) = { ... }
\end{xten}

Wildcard bounds may be covariant, as in the following example:
\begin{xten}
List<? extends Number> list = new ArrayList<Integer>();
Number num = list.get(0);     // legal
list.set(0, new Double(0.0)); // illegal
list.set(0, list.get(1));     // illegal
\end{xten}
This can also be written in X10, but with an important
difference:
\begin{xten}
list: List{T <: Number} = new ArrayList[Integer]();
num: Number = list.get(0);    // legal
list.set(0, new Double(0.0)); // illegal
list.set(0, list.get(1));     // legal! (when list is final)
\end{xten}
Note because \xcd"list.get" has return type \xcd"list.T", the
last call in above is well-typed in X10; the analogous call in
Java is not well-typed.

Finally,
one can also specify lower bounds on types.  These are useful for
comparators:
\begin{xten}
class TreeSet[T] {
  def this[T](cmp: Comparator{T :> this.T}) { ... }
}
\end{xten}
Here, the comparator for any supertype of \xcd"T" can be used as
to compare \xcd"TreeSet" elements.

Another use of lower bounds is for list operations.
The \xcd"map" method below takes a function that maps a supertype
of the class parameter \xcd"T" to the method type parameter \xcd"S":
\begin{xten}
class List[T] {
  def map[S](fun: Object{self :> T} => S) : List[S] = { ... }
}
\end{xten}

\subsection{Proper abstraction}

Consider the following example adapted from \cite{adding-wildcards}:
\begin{xten}
def shuffle[T](list: List[T]) = {
  for (i: int in [0..list.size()-1]) {
    val xi: T = list(i);
    val j: int = Math.random(list.size());
    list(i) = list(j);
    list(j) = xi;
  }
}
\end{xten}
The method is parametrized on \xcd"T" because the method body needs
the element type to declare the variable \xcd"xi".

However, the method parameter can be omitted by using the type \xcd"list.T"
for \xcd"xi".  Thus, the method can be declared with the signature:
\begin{xten}
def shuffle(list: List) { ... }
\end{xten}
This is called \emph{proper abstraction}.

This example illustrates a key difference between type properties
and type parameters:
A type property is a member of its class, whereas a type parameter is
not.  The names of type properties are visible outside the body of
their class declaration.

\if 0
Type properties can be used as the basis of a parametrized type
system.  This is done simply by making type properties private.
Using the syntactic sugar described above,
the resulting system behaves identically to a system with type
parameters.
\fi

In Java,
Wildcard
capture allows the parametrized method to be called with any \xcd"List",
regardless of its parameter type.
However,
the method parameter cannot be omitted: declaring a parameterless version
of shuffle requires delegating to a private parametrized version that
``opens up'' the parameter.

\subsection{Conditional methods and generlized constraints}

For type parameters, method constraints are 
similar to generalized constraints proposed for
\csharp~\cite{emir06}.
%
In the following code, the \xcd"T" parameter is covariant
and so the \xcd"append" methods below are illegal:
\begin{xten}
class List[+T] {
  def append(other: T): List[T] = { ... }
      // illegal
  def append(other: List[T]): List[T] = { ... }
      // illegal
}
\end{xten}
%
However, one can introduce a method parameter and then constrain
the parameter from below by the class's parameter:
For example, in the following code,
\begin{xten}
class List[+T] {
  def append[U](other: U)
      {T <: U}: List[U] = { ... }
  def append[U](other: List[U])
      {T <: U}: List[U] = { ... }
}
\end{xten}

The constraints must be satisfied by the callers of \xcd"append".
For example, in the following code:
\begin{xten}
xs: List[Number];
ys: List[Integer];
xs = ys; // ok
xs.append(1.0); // legal
ys.append(1.0); // illegal
\end{xten}
the call to \xcd"xs.append" is allowed and the result type is \xcd"List[Number]", but
the call to \xcd"ys.append" is not allowed because the caller cannot show that
${\tt Number} \subtype {\tt Double}$.


\subsection{Self types}

The generic constrained type system generalizes
features of type systems from other object-oriented languages.

Type properties can also be used to support a form of self
types~\cite{bruce-binary,bsg95}. 
%
Self types can be implemented by introducing a
type property \Xcd{class} to the root of the class hierarchy, \Xcd{Object}:
\begin{xtenmath}
class Object[class] { $\dots$ }
\end{xtenmath}
Scala's path-dependent types~\cite{scala} and J\&'s
dependent classes~\cite{nqm06}
take a similar approach.

Self types are achieved by
implicitly constraining types so that if an path expression \Xcd{p}
has type \Xcd{C}, then
$\Xcd{p}.\Xcd{class} \subtype \Xcd{C}$.  In particular,
$\Xcd{this}.\Xcd{class}$ is guaranteed to be a subtype
of the lexically enclosing class; the type
$\Xcd{this}.\Xcd{class}$ represents all instances of the fixed,
but statically unknown, run-time class referred to by the \Xcd{this}
parameter.

Self types address the binary method problem~\cite{bruce-binary}.
In the following
example, the class \xcd"BitSet" can be written with a
\xcd"union" method that takes a self type as argument.

\begin{xtenmath}
interface Set {
  def union(s: this.class): void;
}

class BitSet implements Set {
  int bits;
  def union(s: this.class): void {
    this.bits |= s.bits;
  }
}
\end{xtenmath}

\noindent
Since \xcd"s" has type this \Xcd{this}.\Xcd{class}, and the class
invariant of \xcd"BitSet" implies
$\Xcd{this}.\Xcd{class} \subtype \Xcd{BitSet}$,
the implementation of the method is free to access the
\xcd"bits" field of \xcd"s".

Callers of \xcd"BitSet".\xcd"union()" must call the method with
an argument that has the same run-time class as the
receiver.  For a receiver \xcd"p", the
type of the actual argument of the call must have a constraint
that entails \xcd"self".\xcd"class==p".\xcd"class".

\subsection{Virtual types}

Type properties share many similarities with virtual
types~\cite{mp89-virtual-classes,beta,ernst99-gbeta,ernst06-virtual,cdnw07-tribe}
and similar constructs built on path-dependent types found in
languages such as Scala~\cite{scala}, and J\&~\cite{nqm06}.
%
Constrained types are more expressive than virtual
types since they can be constrained at the use-site,
can be refined on a per-object basis without explicit subclassing,
and can be refined contravariantly
as well as covariantly. 

Thorup~\cite{thorup97}
proposed adding genericity to Java using virtual types.  For example,
a generic \xcd"List" class can be written as follows:
\begin{xten}
abstract class List {
    abstract typedef T;
    void add(T element) { ... }
    T get(int i) { ... }
}
\end{xten}
This class can be refined by bounding the virtual type \xcd"T" above:
\begin{xten}
abstract class NumberList extends List {
    abstract typedef T as Number;
}
\end{xten}
And this abstract class can be further refined to \emph{final bind}
\xcd"T" to a particular type:
\begin{xten}
class IntList extends NumberList {
    final typedef T as Integer;
}
\end{xten}
These classes are related by subtyping:
${\tt IntList} \subtype {\tt NumberList} \subtype {\tt List}$.
Only classes where \xcd"T" is final bound can be non-abstract.

The analogous definition of 
\xcd"List" using type properties is as follows:
\begin{xten}
class List[T] {
    def add(element: T) = { ... }
    def get(i: int): T = { ... }
}
\end{xten}

\noindent
\xcd"NumberList" and \xcd"IntList" can be written as follows:
\begin{xten}
class NumberList extends List{T<:Number} { }
class IntList extends NumberList{T==Integer} { }
\end{xten}

However, note that our version of \xcd"List" is not abstract.
Instances of \xcd"List"
can instantiate \xcd"T" with a particular type and there is no
need to declared classes for \xcd"NumberList" and \xcd"IntList".
Instead, one can simply use the types
\xcd"List{T<:Number}" and \xcd"List{T==Integer}".

In addition,
unlike virtual types, type properties can be refined contravariantly.
For instance, one can write the type \xcd"List{T:>Integer}",
and even \xcd"List{Integer<:T, T<:Number}".


\eat{
\subsection{Ownership types}

Consider the following example of generic ownership
derived from Potanin et al.~\cite{ogj-oopsla06}.

\begin{xten}
class Object(owner: Object) { }

// Map inherits Object.owner
// No need to add explicit vOwner and kOwner properties for Key, Value
class Map[Key, Value]{Key <: Comparable, Value <: Object}
{
    private nodes: Vector[Node[Key, Value](this)](this);

    public def put(key: Key, value: Value): Void = {
        nodes.add(new Node[Key, Value](key, value, this)());
    }

    public def get(key: Key): Value = {
        for (mn: Node[Key, Value](this) in nodes) {
            if (mn.key.equals(key))
                return mn.value;
        }
        return null;
    }

    // OGJ will prevent this from being called, since caller
    // can only assign the result to a supertype of Vector(this),
    // which would be only Vector(this) or Object(this)
    // BUT: we have Vector :> Vector(this)
    // Need to require that all class types have an equality constraint
    // on the owner property
    public def exposeVector(): Vector(this) { return nodes; }
}

class Node[Key, Value]
    {Key <: Comparable, Value <: Object}
{
    val key: Key;
    val value: Value;

    public def this[K, V](k: Key, v: Value, o: Object): Node[K, V](o) {
        super(o);               // set the owner
        property[K, V];         // set the type properties
        this.key = k;
        this.value = v;
    }
}
\end{xten}

Restrictions:
\begin{itemize}
\item owner property must be constrained (define this!)
\item owner is always equal to or inside the owner of all other type properties
\item types with an actual owner == this, can only be accessed via this
\end{itemize}

}

\subsection{Structural constraints}
\label{sec:structural}

Type constraints need not be limited to subtyping constraints.
By introducing structural constraints on types, \gxx allows
type properties to be instantiated on any type with a given set
of methods and fields. This feature is useful for reusing code
in separate libraries since it does not require
code of one library to implement an interface to satisfy a
constraint of another library.


\begin{figure}[tp]
\begin{center}
\begin{tabular}{lrcl}
constraints & {\tt c} & ::= & \dots \\
            &        & \bnf & \xcd"T has Sig" \\
signatures  & {\tt Sig} & ::= &
\xcdmath"def this[$\tbar{X}$]($\tbar{x}$: $\tbar{T}$){c}: T" \\
            &        & \bnf &
\xcdmath"def m[$\tbar{X}$]($\tbar{x}$: $\tbar{T}$){c}: T" \\
            &        & \bnf &
\xcdmath"val x{c}: T" \\
            %&  & \bnf & \xcdmath"var x{c}: T" \\
\end{tabular}
\end{center}
\caption{Grammar for structural constraints}
\label{fig:structural2}
\end{figure}

In this section, we consider an extension of the X10 type system
to support structural type constraints.
The type
system need not change except by extending the constraint
system.  The syntax for structural constraints is shown in
Figure~\ref{fig:structural}.  A structural constraint of the
form \xcd"T has Sig" can specify that the type \xcd"T" have a
constructor, method, or field of the given signature.

Structural constraints on types are found in many languages.
Haskell supports type
classes~\cite{haskell,haskell-type-classes}.
%ML's module system allows modules to be constrained by
%structural signatures~\cite{ml}.
In Modula-3, type equivalence and subtyping are structural
rather than nominal as in object-oriented languages of the C
family such as C++, Java, Scala, and X10.
%
The language PolyJ~\cite{java-popl97} allows type parameters to be bounded using
structural where clauses.
For example, the sorted list class from Figure~\ref{fig:sorted}
could be
be written as follows in PolyJ:
\begin{xten}
class SortedList[T] where T { int compareTo(T) } {
    void add(T x) { ... x.compareTo(y) ... }
    ...
}
\end{xten}
The where clause states that the type parameter \xcd"T" must have a
method \xcd"compareTo" with the given signature.

The analogous code for \xcd"SortedList" in the structural
extension of X10 is:
\begin{xten}
class SortedList[T]{T has def compareTo(T): int} {
    def add(x: T) = { ... x.compareTo(y) ... }
    ...
}
\end{xten}

A structural constraint is satisfied if the type has a member of
the appropriate name and with a compatible type.  The constraint
\xcdmath"Z has def m[$\tbar{X}$]($\tbar{x}\ty\tbar{T}$): U"
is satisfied by a type \xcd"T" if it has a method \xcd"m"
with signature
\xcdmath"def m[$\tbar{Y}$]($\tbar{y}\ty\tbar{S}$): V"
and where
(\xcdmath"[$\tbar{Y}$]($\tbar{y}\ty\tbar{S}$) => V")$[\Xcd{T}/\Xcd{Z}]$
is a subtype of
(\xcdmath"[$\tbar{X}$]($\tbar{x}\ty\tbar{T}$) => U")$[\Xcd{T}/\Xcd{Z}]$.
As an example,
the constraint \xcd"X has def compareTo(X): int"
is satisfied by both of the following classes:
\begin{xten}
class C { def compareTo(x: C): int = ...; }
class D { def compareTo(x: Object): int = ...; }
\end{xten}

\todo{The important bit about all this is... don't have to
      change the type system, just the constraints}
