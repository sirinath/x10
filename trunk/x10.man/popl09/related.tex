
Dependent types, constraint-based type systems and generic types have a long
history.

\paragraph{Constraint-based type systems.}

The use of constraints for type inference and subtyping were
developed by
Mitchell~\cite{mitchell84}
and
Reynolds~\cite{reynolds85}.
%
Trifonov and Smith~\cite{trifonov96}
proposed a type system where types are refined by subtyping
constraints, however dependent types are not supported.

% Pottier~\cite{pottier96simplifying,pottier01b}
% presents a constraint-based type system for an ML-like language with
% subtyping.

HM(X)~\cite{sulzmann97type,pottier01a,pottier-remy-attapl}
is a constraint-based framework
for Hindley--Milner-style type systems.
The framework is parameterized on the specific constraint system
X; instantiating X yields extensions of the HM type system.
Constraints in HM(X) are over types, not values.
%
Sulzmann and Stuckey~\cite{sulzmann-hmx-clpx} showed that the
type inference algorithm for HM(X) can be encoded as a
constraint logic program parameterized by the constraint system
X.  Type inference in this framework is thus constraint solving.

Structural constraints on types have been implemented in 
Haskell type classes~\cite{haskell-type-classes}.
PolyJ~\cite{polyj} also uses structual constraints to bound type
parameters.

\paragraph{Dependent types.}

Several systems have been proposed that refine types in a base
type system through constraints on
values~\cite{refinement-types,conditional-types,jones94,sized-types,flanagan-popl06,flanagan-fool06,liquid-types}.

% {\em Refinement types}~\cite{refinement-types} extend the 
% Hindley--Milner type system with intersection, union, and
% constructor types, enabling specification and inference of
% more precise type information.
% %
% {\em Conditional
% types}~\cite{conditional-types} extend refinement types to
% encode control-flow information in the types.
% %
% Jones introduced {\em qualified types}, which permit
% types to be constrained by a finite set of
% predicates~\cite{jones94}.
% %
% {\em Sized types}~\cite{sized-types}
% annotate types with the sizes of recursive data structures.
% Sizes are linear functions of size variables.
% Size inference is performed using a constraint solver for
% Presburger arithmetic~\cite{omega}.
% % constraints on types, support primitive recursion only

% Index objects must be pure.
% Singleton types int(n).
% ML$^{\Pi}_0$:
% Refinement of the ML type system: does not affect the
% operational semantics.  Can erase to ML$_0$.

% Jay and Sekanina 1996: array bounds checking based on shape
% types.

With {\em hybrid type-checking}~\cite{flanagan-popl06,flanagan-fool06},
types can be constrained by arbitrary boolean expressions.
While typing is undecidable, dynamic checks are inserted into
the program when necessary if the type-checker cannot determine
type safety statically.
In X10, dynamic type checks, including tests of dependent
constraints, are inserted only at explicit casts or
\Xcd{instanceof} expressions.

% Ada dependent types.
% Ada has constrained array definitions.  A constraint
% \cite{ada-ref-man}.  Not clear if they're dependent.  Are
% there other dependent types?  Generics are dependent?

        % Used for array bounds by Morrisett et al (I think--need
        % to find paper)

% Singleton types~\cite{aspinall-singletons}.

Logically qualified types, or liquid types~\cite{liquid-types},
permit types in a base HM-style type system to be
qualified with dependent constraints.  The HM type inference
algorithm is extended to infer constraints as well as base
types.  \todo{More on this}

Concoqtion~\cite{concoqtion} extends types in OCaml~\cite{ocaml}
with constraints written as Coq~\cite{coq} rules.
While the types are expressive, supporting the full generality
of the Coq language, proofs must be
provided to satisfy the type checker.
X10 supports only constraints that can be checked by a
constraint solver during compilation.
Concoqtion encodes OCaml types and value to allow reasoning in
the Coq formulae; however, there is an impedence mismatch
caused by the differing syntax, representation, and behavior
of OCaml versus Coq.

% Where clauses for F-bounded polymorphism~\cite{where-clauses}
% Bounded quantification: Cardelli and Wegner.  Bound T with T'
% In F-bounded polymorphism~\cite{f-bounds}, type variables are bounded by a function of 
% the type variable. 
% Not dependent types.



Cayenne~\cite{cayenne} is a Haskell-like language with fully dependent types.
There is no distinction between static and dynamic types.
Type-checking is undecidable.
There is no notion of datatype refinement as in DML.

Epigram~\cite{epigram,epigram-matter}
is a dependently typed functional programming language based on
a type theory with inductive families.
Epigram does not have a phase distinction between values and
types.

\eat{
$\lambda^{\sf Con}$ is a lambda calculus with assertions.
Findler, Felleisen, Contracts for higher-order functions (ICFP02)

  example: int[> 9]

contracts are either simple predicates or function contracts.
defined by (define/contract ...)

enforced at run-time.
}

% Jif~\cite{jif,jflow} is an extension of Java in which
% types are labeled with security policies enforced by the
% compiler.

ESC/Java~\cite{esc-java}
allow programmers to write object invariants and pre- and
post-conditions that are enforced statically
by the compiler using an automated theorem prover.
Static checking is undecidable and, in the presence of loops,
is unsound (but still useful) unless the programmer supplies loop invariants.
ESC/Java can enforce invariants on mutable state.

% and Spec$\sharp$~\cite{specsharp}

\eat{
Pluggable and optional type systems were proposed by
Bracha~\cite{bracha04-pluggable} and provide another means of
specifying refinement types.
Type annotations, implemented in compiler plugins, serve only to
reject programs statically that might otherwise have dynamic
type errors.
CQual~\cite{foster-popl02} extends C with user-defined type
qualifiers.  These
qualifiers may be flow-sensitive and may be inferred. 
CQual supports only a fixed set of typing rules
for all qualifiers.
In contrast, the {\em semantic type qualifiers} of
Chin, Markstrum, and Millstein~\cite{chin05-qualifiers}
allow programmers to define typing rules for qualifiers
in a meta language that allows type-checking rules to be
specified declaratively.
JavaCOP~\cite{javacop-oopsla06} is a pluggable type system
framework for Java.  Annotations are defined in a meta language
that allows type-checking rules to be specified declaratively.
JSR 308~\cite{jsr308} is a proposal for adding user-defined type qualifiers
to Java.
}

% Holt, Cordy, the Turing programming language

% Ou, Tan, Mandelbaum, Walker, Dynamic typing with dependent types
% Separate dependent and simple parts of the program.
% Statically type the dependent parts.
% Dynamic checks when passing values into dependent part.

Our work is closely related to DML, \cite{xi99dependent}, an
extension of ML with dependent types. DML is also built
parametrically on a constraint solver. Types are refinement types;
they do not affect the operational semantics and erasing the
constraints yields a legal ML program.

At a conceptual level the intuitions behind the development of DML
and constrained types are similar. Both are intended for practical
programming by mainstream programmers, both introduce a strict
separation between compile-time and run-time processing, are
parametric on a constraint solver, and deal with mutually recursive
data-structures, mutable state, and higher-order functions (encoded as
objects in the case of constrained types). Both support existential
types.

The most obvious distinction between the two lies in the target
domain: DML is designed for functional programming, specifically
ML, whereas constrained types are designed for imperative, concurrent
OO languages. Hence, technically our development of constrained types
takes the route of an extension to FJ. But there are several other
crucial differences as well.

First, DML achieves its separation by not permitting program
variables to be used in types. Instead, a parallel set of (universally
or existentially quantified) ``index'' variables are
introduced.
Second, DML permits only variables of basic index sorts known to
the constraint solver (e.g., \Xcd{bool}, \Xcd{int}, \Xcd{nat}) to
occur in types. In contrast, constrained types permit program
variables at any type to occur in constrained types. As with DML
only operations specified by the constraint system are permitted in
types. However, these operations always include field selection and
equality on object references.  (As we have seen permitting arbitrary
type/property graphs may lead to undecidability.) Note that DML
style constraints are easily encoded in constrained types.

Third, DML does not permit any runtime checking of constraints
(dynamic casts).

\paragraph{Genericity.}

Genericity in object-oriented languages is usually
suported through
type parameterization.

A number of proposals 
for adding genericity to Java quickly followed
the initial release of
the language~\cite{GJ,Pizza,polyj,thorup97,allen03}.
GJ~\cite{GJ} implements invariant type
parameters via type erasure.
PolyJ~\cite{polyj} supports run-time representation of types
via adapter objects, and also permits instantiation of
parameters on primitive types and structural parameter bounds.
LM~\cite{reflective-generics,type-passing-generics} also supports
a run-time representation of types.
NextGen~\cite{nextgen,allen03} was implemented using run-time 
instantiation.
X10's generics have a hybrid implementation, adopting PolyJ's
adapter object approach for dependent types and for 
type introspection and using NextGen's run-time
instantiation approach for greater efficiency.
% MixGen~\cite{allen04} extends NextGen with mixins.

C\# also supports generics via ~\cite{csharp-generics}.
Type parameters may be declared
with definition-site variance tags.  Compilation does not erase
types; the CLR performs run-time parameter instantiation.
Generalized type constraints were proposed for
C\#~\cite{emir06}.  Methods can be annotated with subtyping
constraints that must be satisfied to invoke the method.
Generic X10 supports these constraints, as well as constraints
on values, with method and constructor where clauses.

\gxx does not support \emph{bivariance}~\cite{variant-parameter-types}; a class
\xcd"C" is bivariant in a type property \xcd"X" if \xcd"C{self.X==S}" is a
subtype of \xcd"C{self.X==T}" for any \xcd"S" and \xcd"T".  Bivariance is
useful for writing code in which the property \xcd"X" is ignored.  One can
achieve  this effect in \gxx simply by leaving \xcd"X" unconstrained.

\eat{
Parametric types with use-site variance are related to existential types:
\xcd"C<+T>" corresponds to the bounded existential $\exists\tcd{X<:T}.C<X>$;
\xcd"C<-T>" corresponds to the bounded existential $\exists\tcd{X:>T}.C<X>$;
\xcd"C<*>" corresponds to the unbounded existential $\exists\tcd{X}.C<X>$.
\gxx has a similar correspondence:
\xcd"C{X<:T}" corresponds to the bounded existential \xcdmath"C\{\exists\tcd{self}:C.self.X<:T\}";
\xcd"C{X:>T}" corresponds to the bounded existential \xcdmath"C\{\exists\tcd{self}.C<X\}";
\xcd"C" corresponds to the unbounded existential \xcdmath"C\{\exists\tcd{self}.C<X\}".
}

\paragraph{Virtual classes and types.}

{\em Virtual classes}~\cite{beta,mp89-virtual-classes,ernst06-virtual}.
are a language-based extensibility
mechanism that where
originally introduced in the language
BETA~\cite{beta} as a mechanism
for supporting genericity.
Virtual classes in BETA are not statically type safe, but this has been
remedied in recent formulations~\cite{ernst99-gbeta,ernst06-virtual} and in
variants of 
virtual classes~\cite{scala,nqm06,cdnw06-tribe,variant-path-types}
using path-dependent types.  

{\em Virtual types}, also introduced in BETA~\cite{beta},
are similar to virtual
classes.  A virtual type is a type binding nested within an
enclosing instance.
Virtual types
may be used to provide genericity; indeed
Thorup~\cite{thorup97} proposed extending Java with virtual types
as a genericity mechanism.  Virtual types influenced Java's 
wildcards~\cite{adding-wildcards,Java3,wildcards-safe}.

Igarashi and Pierce~\cite{ip99-virtual-types}
model the semantics of virtual types
and several variants
in a typed lambda-calculus with subtyping and dependent types.

Use-site variance based on 
structural virtual types
were proposed by Thorup and Torgerson~\cite{unifying-genericity}
and extended for parameterized type systems by Igarashi and
Viroli~\cite{variant-parametric-types}.  The latter type
system lead to the development of wildcards in
Java~\cite{Java3,adding-wildcards,wildcards-safe}.
