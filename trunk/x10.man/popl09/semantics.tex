\section{Formal semantics}
\label{sec:semantics}

We present a core calculus, \gxx, for X10 with generics.
\gxx is based on Constrained Featherweight
Java~\cite{constrained-types}.

\todo{
Add method overriding rules: covariant return, contravariant
args, weaker constraints
}

The grammar for \gxx is shown in 
Figure~\ref{fig:grammar}.  The calculus elides features of the
full X10 language not relevant to this paper.

\begin{figure}[tp]
\begin{center}
\begin{tabular}{lrcl}
program & {\tt P} & ::= & \xbar{L} \\
classes & {\tt L} & ::= &
\xcdmath"class C[$\tbar{X}$]($\tbar{x} \ty \tbar{T}$){c}" \\
& & & \xcdmath"  extends T { $\tbar{M}$ }" \\
base types & {\tt R} \\
\quad classes & & ::= & \xcd"C" \\
\quad type variables  & & \bnf & \xcd"X" \\
\quad type members    & & \bnf & \xcd"e.X" \\
\quad type type       & & \bnf & \xcd"type" \\
types & {\tt T} & ::= & \xcd"R{c}" \\
methods     & {\tt M} & ::= &
\xcdmath"def m[$\tbar{X}$]($\tbar{x}$: $\tbar{T}$){c}: T = e" \\
expressions & {\tt e} & \\
\quad literals        &         & ::=  & \xcd"true" \bnf \xcd"false" \bnf \xcd"null" \bnf $n$ \\
\quad variables       &         & \bnf & \xcd"x" \\
\quad field access    &         & \bnf & \xcdmath"e.x" \\
\quad call            &         & \bnf & \xcdmath"e$_0$.m[$\tbar{T}$]($\tbar{e}$)" \\
%\quad                 &         & \bnf & \xcdmath"e$_0$.m($\tbar{e}$)" \\
\quad new             &         & \bnf & \xcdmath"new C[$\tbar{T}$]($\tbar{e}$)" \\
%\quad                 &         & \bnf & \xcdmath"new C($\tbar{e}$)" \\
\quad cast            &         & \bnf & \xcdmath"e as T" \\
constraint terms & {\tt t} &     & \\
\quad self            &         & ::=  & \xcd"self" \\
\quad variables       &         & \bnf & \xcd"x" \\
\quad properties      &         & \bnf & \xcd"t".\xcd"x" \\
\quad atoms           &         & \bnf & \xcdmath"g(t$_1$,$\dots$,t$_n$)" \\
\quad new             &         & \bnf & \xcdmath"new C(t$_1$,$\dots$,t$_n$)" \\
constraint & {\tt c} &      & \\
\quad true            &  & ::=  & \Xcd{true} \\
\quad equality        &  & \bnf & $\Xcd{t}_1 \equals \Xcd{t}_2$ \\
\quad existentials    &  & \bnf & $\exc{\Xcd{x}}{\Xcd{c}}$ \\
\quad conjunction     &  & \bnf & $\xbar{c}$ \\
\quad predicates      &  & \bnf & \xcdmath"p(t$_1$,$\dots$,t$_n$)" \\
environments & $\Gamma$ & ::=  & $\epsilon$ \\
            &          & \bnf & $\Gamma$, $\Xcd{c}$ \\
            &          & \bnf & $\Gamma$, $\Xcd{x} \ty \Xcd{T}$ \\
            &          & \bnf & $\Gamma$, $\Xcd{X} \ty \Xcd{type}$ \\
\end{tabular}
\end{center}
\caption{\gxx grammar}
\label{fig:grammar}
\end{figure}

Figure~\ref{fig:grammar2} extends the grammar with syntactic
sugar for subtyping constraints and existential types.
The subtyping constraint
                  $\Xcd{t}_1 \subtype \Xcd{t}_2$ 
                  is atomic formula.
The existential type 
$\exty{\Xcd{x}}{\Xcd{T}}{\Xcd{R\{c\}}}$
is sugar for
$\Xcd{R\{}\exc{\Xcd{x}}{\sigma(\Xcd{x}\ty\Xcd{T}),\Xcd{c}}\Xcd{\}}$.

\begin{figure}[tp]
\begin{center}
\begin{tabular}{lrcl}
types & {\tt T} & ::= & \dots \\
            & & \bnf & $\exty{\Xcd{x}}{\Xcd{T}_0}{\Xcd{T}}$ \\
            & & \bnf & $\exty{\Xcd{X}}{\Xcd{type}}{\Xcd{T}}$ \\
constraint terms & {\tt t} & ::= & \dots \\
\quad literals        &         &      & \xcd"true" \bnf $n$ \bnf \xcd"C" \\
\quad type variables       &         & \bnf & \xcd"X" \\
\quad type properties      &         & \bnf & \xcd"t".\xcd"X" \\
constraint & {\tt c} & ::=  & \dots \\
                  &  & \bnf & $\Xcd{t}_1 \subtype \Xcd{t}_2$ \\
                  &  & \bnf & \Xcd{cons(T,z)} \\
\end{tabular}
\end{center}
\caption{\gxx grammar with subtyping constraints}
\label{fig:grammar2}
\end{figure}

We assume a fixed but unknown constraint system ${\cal D}$.
A program \Xcd{P} is written using constraints from ${\cal D}$,
We assume classes defined in \Xcd{P} do not have a cyclic
inheritance structure.

\infrule[Program OK]{
\Xcd{extends}^+~\mbox{acyclic}
}{
\vdash \xbar{L}~\mbox{ok}
}

\subsection{
The object constraint system, ${\cal O}$
}

From \Xcd{P} and ${\cal D}$
we generate an \emph{object constraint system} ${\cal O}$, shown
in Figure~\ref{fig:object-constraints},
as follows.  Let \Xcd{C} and \Xcd{D} range over names of classes
in \Xcd{P}, \Xcd{f} over field names, \Xcd{m} over method names,
\Xcd{T} over types, and \Xcd{c} over constraints in the
underlying data constraint system ${\cal D}$.

\begin{figure}[tp]
\begin{center}
\begin{tabular}{lrcl}
constraint & {\tt c} 
                    & ::=  & \Xcd{class(C)} \\
                  & & \bnf & \Xcd{C}~\Xcd{extends}~\Xcd{D} \\
                  & & \bnf & $\Xcd{fields(x,}\xbar{f}\ty\xbar{T}\Xcd{)}$ \\
                  & & \bnf & $\Xcd{mtype(x,m,[}\xbar{X}\Xcd{(}\xbar{x}\ty\xbar{T}\Xcd{)\{c\}}$ \\
\end{tabular}

\infrule{
\Xcdmath{class C[$\tbar{X}$]($\tbar{f} \ty \tbar{T}$)\{c\} extends D\{d\} \{ $\tbar{M}$\}} \in \Xcd{P}
}{
\vdashO \Xcd{class(C)} \\
\vdashO \Xcd{C}~\Xcd{extends}~\Xcd{C} \\
\vdashO \Xcd{C}~\Xcd{extends}~\Xcd{D}
}

\infrule[Fields]{
\Xcdmath{class C[$\tbar{X}$]($\tbar{f} \ty \tbar{T}$)\{c\} extends Object \{ $\tbar{M}$\}} \in \Xcd{P}
}{
\Gamma, \Xcd{z}\ty \Xcd{C}\{d\} \vdashO \Xcdmath{fields(z,$\tbar{f}\ty\tbar{T}$)}
}

\infrule[Fields-extends]{
\vdashO \Xcd{C}~\Xcd{extends}~\Xcd{D}
\\
\Xcdmath{class C[$\tbar{X}$]($\tbar{f} \ty \tbar{T}$)\{c\} extends D\{d\} \{ $\tbar{M}$\}} \in \Xcd{P}
\\
\Gamma, \Xcd{z}\ty \Xcd{D\{d\}} \vdashO
\Xcdmath{fields(z,$\tbar{f}_0\ty\tbar{T}_0$)}
}{
\Gamma, \Xcd{z}\ty \Xcd{D\{d\}} \vdashO
\Xcdmath{fields(z,$\tbar{f}_0\ty\tbar{T}_0$, $\tbar{f}\ty\tbar{T}$)}
}

\infrule[Mtype]{
\Xcdmath{class C[$\tbar{X}$]($\tbar{f} \ty \tbar{T}$)\{c\} extends Object \{$\tbar{M}$\}} \in \Xcd{P}
\\
\Xcd{M}_i = 
\Xcdmath{def m$_i$[$\tbar{X}$]($\tbar{x}\ty\tbar{T}$)\{c\}: T = e}
}{
\Gamma, \Xcd{z}\ty \Xcd{C\{d\}} \vdashO
\Xcdmath{mtype(z,m$_i$,[$\tbar{X}$]($\tbar{x}\ty\tbar{T}$)\{c\} $\to$T)}
}

\end{center}
\caption{The constraint system ${\cal O}$}
\label{fig:object-constraints}
\end{figure}

In the method signature
                $\Xcd{[}\xbar{X}\Xcd{(}\xbar{x}\ty\xbar{T}\Xcd{)\{c\}}$,
                the type variables \xbar{X} and data variables
                \xbar{x} are considered bound; formulas with
                bound variables are considered equivalent up to
                $\alpha$-renaming.

The constraint system satisfies the axioms and
inference rules in Figure~\ref{fig:object-constraints}.
The \Xcd{class}, \Xcd{extends}, \Xcd{fields}, and \Xcd{mtype}
constraints are given directly from the program \Xcd{P}.

The constraint system ${\cal C}$ is the disjoint conjunction
${\cal D}$, ${\cal O}$
of
the constraint systems
${\cal D}$ and ${\cal O}$.
(This requires the assumption that 
${\cal D}$ does not have any constraints in common with ${\cal O}$.

\eat{
The X10 compiler permits the constraint system to be extended
with compiler plugins.  The base compiler supports equality
constraints over literals and final variables and subtyping
and equality
constraints over types.
The core constraint system is presented here.  We assume a
constraint solver ${\cal X}$ implementing the plugged-in
constraint systems.

The constraint system does not distinguish between values and
types.  Logical variables \xcd"x" may represent program variables
or type variables.  Path terms \Xcd{p.x} may represent field
accesses or member type references.

The constraint system is shown in Figure~\ref{fig:constraints}.
$\xbar{c}$ is a set of constraints.  The constraint system
satisfies the given structural rules, and supports equality and
subtyping constraints over terms.
}%eat


\subsection{Structural and logical rules}

All judgments are intuitionistic.  In particular, this means
that all constraint systems satisfy the rules and axioms
in Figure~\ref{fig:logic}.

\begin{figure}

\infax[Id]{ \Gamma, \Xcd{c} \vdash \Xcd{c} }

\infrule[Cut]{
\Gamma \vdash \Xcd{c}
\andalso
\Gamma, \Xcd{c} \vdash \Xcd{d}
}{
\Gamma \vdash \Xcd{d}
}

% \infrule[Contraction]{ \xbar{c}, \Xcd{c}, \Xcd{c} \vdashC \Xcd{d} }
                     % { \xbar{c}, \Xcd{c} \vdashC \Xcd{d} }
% \infrule[Permutation]{ \xbar{c}, \Xcd{c}, \Xcd{d} \vdashC \Xcd{e} }
                     % { \xbar{c}, \Xcd{d}, \Xcd{c} \vdashC \Xcd{e} }
% \infrule[Extension]{ \xbar{c}, \vdashC \Xcd{c} }
                     % { \xbar{c}, \xbar{c}' \vdashC \Xcd{c} }

\infrule[Weak-1]{
\Gamma \vdash \phi
\andalso
\Gamma \vdash \Xcd{T} \ty \Xcd{type}
\andalso
\Xcd{x} \not\in \mathit{var}(\Gamma)
}{
\Gamma, \Xcd{x} \ty \Xcd{T} \vdash \phi
}

\infrule[Weak-2]{
\Gamma \vdash \phi
\andalso
\Gamma \vdash \Xcd{c} \ty \Xcd{o}
}{
\Gamma, \Xcd{c} \ty \phi
}

\infrule[And-L]{
\Gamma, \psi_0, \psi_1 \vdash \phi
}{
\Gamma, (\psi_0, \psi_1) \vdash \phi
}

\infrule[And-R]{
\Gamma \vdash \psi_0
\andalso
\Gamma \vdash \psi_1
}{
\Gamma \vdash (\psi_0, \psi_1)
}

\infrule[Exists-R]{
\Gamma \vdash \phi[\Xcd{t}/\Xcd{x}]
}{
\Gamma \vdashC \exc{\Xcd{x}}{\phi}
}

\infrule[Exists-L]{
\Gamma, \Xcd{x} \ty \Xcd{T}, \psi \vdash \phi
\andalso
\Xcd{x}~\mbox{fresh}
}{
\Gamma, \exc{\Xcd{x}\ty\Xcd{T}}{\psi} \vdash \phi
}

\caption{Logical rules}
\label{fig:logic}
\end{figure}


\subsection{Well-formedness rules}

We use the judgment for well-typedness for expressions to
represent well-typedness for constraints.  That is, we posit
a special type \Xcd{o} (traditionally the type of propositions),
and regard constraints as expressions of type \Xcd{o}.

Further, we change the formulation slightly so that there are no
constraints of the form \Xcdmath{p(t$_1$,$\dots$,t$_n$)}; rather
instance method invocation syntax is used to express invocation
of
pre-defined constraints.  This logically leads to the step of
simply marking certain classes as ``predicate'' classes---all
the (instance) methods of these classes whose return type is
\Xcd{o} then correspond to ``primitive constraints''.
Syntactically, we continue to use the symmetric syntax
\Xcdmath{p(t$_1$,$\dots$,t$_n$)} rather than
\Xcdmath{t$_1$.p(t$_2$,$\dots$,t$_n$)}.
The alternative is tor introduce static methods and static
method invocations in the expression language.  This is not
difficult, but is annoying to have to repeat most of the
formulation of instance methods.

This means that the only cases left to handle are all the simple
ones, expression the availability of certain constraints and
operations of type \Xcd{o}.


\begin{figure}
\infax[True]{\Gamma \vdash \Xcd{true} \ty \Xcd{o}}

\infrule[Equals]{
\Gamma \vdash \Xcd{t}_0 \ty \Xcd{T}_0
\andalso
\Gamma \vdash \Xcd{t}_1 \ty \Xcd{T}_1
\andalso
(
\Gamma \vdash \Xcd{T}_0 \subtype \Xcd{T}_1
\vee
\Gamma \vdash \Xcd{T}_1 \subtype \Xcd{T}_2
)
}{
\Gamma \vdash \Xcd{t}_0 \equals \Xcd{t}_1 \ty \Xcd{o}
}

\infrule[And]{
\Gamma \vdash \Xcd{c}_0 \ty \Xcd{o}
\andalso
\Gamma \vdash \Xcd{c}_1 \ty \Xcd{o}
}{
\Gamma \vdash (\Xcd{c}_0, \Xcd{c}_1) \ty \Xcd{o}
}

\infrule[Some]{
\Gamma \vdash \Xcd{t} \ty \Xcd{T}
\andalso
\Gamma \vdash \Xcd{c}[\Xcd{t}/\Xcd{x}] \ty \Xcd{o}
}{
\Gamma \vdash \exc{\Xcd{x}\ty\Xcd{T}}{\Xcd{c}} \ty \Xcd{o}
}

\infrule[Type]{
\Gamma \vdash \Xcd{class(C)}
\andalso
\Gamma, \Xcd{self} \ty \Xcd{C} \vdash \Xcd{c} \ty \Xcd{o}
}{
\Gamma \vdash \Xcd{C\{c\}} \ty \Xcd{type}
}

\caption{Well-formedness rules}
\label{fig:wf}
\end{figure}


\subsection{Subtyping constraints}


\begin{figure}

\infrule[Cons-eq]{
\xbar{c} \vdashC \Xcd{T}_1 \equals \Xcd{T}_2
}{
\xbar{c} \vdashC
\Xcd{cons(}\Xcd{T}_1\Xcd{,z)} \equals 
\Xcd{cons(}\Xcd{T}_2\Xcd{,z)} 
}

\infrule[Cons-sub]{
\xbar{c} \vdashC \Xcd{T}_1 \subtype \Xcd{T}_2
}{
\xbar{c},
\Xcd{cons(}\Xcd{T}_1\Xcd{,z)} \vdashC 
\Xcd{cons(}\Xcd{T}_2\Xcd{,z)} 
}

\infax[Cons]{
\vdashC \Xcd{cons(C,z)}
}

\infrule[Eq-atom]
{ \xbar{c} \vdashC \xbar{s} \equals \xbar{t} }
{ \xbar{c} \vdashC \Xcd{f(}\xbar{s}\Xcd{)} \equals \Xcd{f(}\xbar{t}\Xcd{)} }

% \infrule[Eq-field]{ \xbar{c} \vdashC \Xcd{t}_1 \equals \Xcd{t}_2 }
                  % { \xbar{c} \vdashC \Xcd{t}_1.\Xcd{f} \equals \Xcd{t}_2.\Xcd{f} }
% \infrule[Eq-type]{ \xbar{c} \vdashC \Xcd{t}_1 \equals \Xcd{t}_2 }
                  % { \xbar{c} \vdashC \Xcd{t}_1.\Xcd{X} \equals \Xcd{t}_2.\Xcd{X} }

\infax[Eq-refl]{ \xbar{c} \vdashC \Xcd{t} \equals \Xcd{t} }

\infrule[Eq-trans]{
        \xbar{c} \vdashC \Xcd{t}_1 \equals \Xcd{t}_2
        \andalso
        \xbar{c} \vdashC \Xcd{t}_2 \equals \Xcd{t}_3
        }
        { \xbar{c} \vdashC \Xcd{t}_1 \equals \Xcd{t}_3 }

\infrule[Eq-sym]{
        \xbar{c} \vdashC \Xcd{t}_1 \equals \Xcd{t}_2
        }
        { \xbar{c} \vdashC \Xcd{t}_2 \equals \Xcd{t}_1 }

\infrule[Eq-sub]{
\xbar{c} \vdashC \Xcd{T}_1 \subtype \Xcd{T}_2 \\
\xbar{c} \vdashC \Xcd{T}_2 \subtype \Xcd{T}_1
}{
\xbar{c} \vdashC \Xcd{T}_1 \equals \Xcd{T}_2
}

\infrule[Sub-cons]{
\xbar{c} \vdashC \Xcd{C\{c\}} \ty \Xcd{type}
\andalso
\xbar{c}, \Xcd{c} \vdashC \Xcd{d}
}{
\xbar{c} \vdashC \Xcd{C\{c\}} \subtype \Xcd{C\{d\}}
}

\infrule[Sub-super]{
\mbox{\Xcdmath{C[$\tbar{X}$]($\tbar{x}$: $\tbar{T}$)\{c\} ext T \{\ K\ $\tbar{M}$\ $\tbar{F}$\ \}}}
}{
\vdashC \Xcd{C} \subtype \Xcd{T} \\
}

\infax[Sub-object]{
\vdashC \Xcd{T} \subtype \Xcd{Object}
}

\infrule[Sub-eq]{
\xbar{c} \vdashC \Xcd{T}_1 \equals \Xcd{T}_2
}{
\xbar{c} \vdashC \Xcd{T}_1 \subtype \Xcd{T}_2
}

\infrule[Sub-trans]{
\xbar{c} \vdashC \Xcd{T}_1 \subtype \Xcd{T}_2
\andalso
\xbar{c} \vdashC \Xcd{T}_2 \subtype \Xcd{T}_3
}{
\xbar{c} \vdashC \Xcd{T}_1 \subtype \Xcd{T}_3
}

\caption{Equality and subtyping rules}
\label{fig:subtyping}
\end{figure}



\subsection{
Constraint projection
}

First, for a type environment $\Gamma$,
we define the \emph{constraint projection},
$\sigma(\Gamma)$ thus:

\begin{align*}
\sigma(\epsilon) &= \Xcd{true} \\
\sigma(\Gamma, \Xcd{x} \ty \Xcd{T}) &=
        \sigma(\Gamma),
        \cons{\Xcd{T}}{\Xcd{x}}
\\
\sigma(\Gamma, \Xcd{c}) &= \sigma(\Gamma), \Xcd{c} \\
\end{align*}

The auxiliary function $\mathit{cons}$
specifies the constraint for a type \Xcd{T} with \Xcd{self}
bounds to \Xcd{x}.
The constraint projection uses an atomic formula \Xcd{cons},
which is equated to the constraint of \Xcd{T} if \Xcd{T} is not
a type variable.

\begin{align*}
\cons{\Xcd{C}}{\Xcd{z}} &=
    \Xcd{cons(C,z)} \\
\cons{\Xcd{C\{c\}}}{\Xcd{z}} &=
    \Xcd{c}[\Xcd{z}/\Xcd{self}], \Xcd{cons(C\{c\},z)==c}[\Xcd{z}/\Xcd{self}] \\
\cons{\Xcd{p.X}}{\Xcd{z}} &=
    \Xcd{cons(p.X,z)} \\
\cons{\Xcd{X}}{\Xcd{z}} &=
    \Xcd{cons(X,z)} \\
\end{align*}

\noindent
Thus, for example, the constraint projection of the environment:
\begin{quote}
\xcdmath"b: D, a: C{self.X==D{d},self.Y<:b.Z}"
\end{quote}
\noindent is:
\begin{quote}
\xcdmath"a.X==D{d}, a.Y<:b.Z" \\
\end{quote}

\eat{
\subsection{
        Judgments
}%eat

The following judgments will be defined:

\begin{itemize}
\item
     The type {\tt T} is well-formed, given the assumptions $\Gamma$:

    $\Gamma \vdash {\tt T} \ty {\tt type}$

\item
     The type {\tt S} is a subtype of {\tt T}, under the assumption $\Gamma$:

      $\Gamma \vdash {\tt S} \subtype {\tt T}$

    \item The expression {\tt e} is of type {\tt T}, given the assumptions $\Gamma$:

      $\Gamma \vdash {\tt e} \ty {\tt T}$

    \item The method {\tt M} is well-defined for the class {\tt C}
given assumptions $\Gamma$:

      $\Gamma \vdash {\tt M}~\mbox{OK in}~{\tt C}$

    \item The field {\tt f: T} is well defined for the class {\tt C} given assumptions $\Gamma$:

      $\Gamma \vdash {\tt f: T}~\mbox{OK in}~C$

    \item The class definition {\tt L} is well defined given assumptions $\Gamma$:

      $\Gamma \vdash {\tt L}~\mbox{OK}$

\end{itemize}


In what follows we will sometimes think of the family of five
judgments
as a single judgment $\Gamma \vdash \phi$ where $\phi$ ranges over the
formulas 
    ${\tt T} \ty {\tt type}$,
      ${\tt S} \subtype {\tt T}$,
      ${\tt e} \ty {\tt T}$,
      ${\tt M}~\mbox{OK in}~{\tt C}$,
      ${\tt f: T}~\mbox{OK in}~C$, and
      ${\tt L}~\mbox{OK}$.


Now, these judgments need to satisfy certain properties:

\begin{enumerate}

\item
    $\Gamma \vdash {\tt T} \ty {\tt type}$
whenever 
      $\Gamma \vdash {\tt e} \ty {\tt T}$; that is,
if we can conclude that {\tt e}
      has type {\tt T} (under certain assumptions), then under those
      assumptions we must be able to conclude that {\tt T} is well-defined.

\item
    $\Gamma \vdash {\tt S} \ty {\tt type}$ and
    $\Gamma \vdash {\tt T} \ty {\tt type}$ whenever
      $\Gamma \vdash {\tt S} \subtype {\tt T}$.

\item
If 
      $\Gamma \vdash {\tt e} \ty {\tt T}$ and if {\tt x}
is a variable occurring free in ${\tt e} \ty {\tt T}$, then for some
      type {\tt U},
      $\Gamma \vdash {\tt x} \ty {\tt U}$.
That is, all free variables on the right-hand
      side of the judgment are actually defined on the left-hand side.
\end{enumerate}


Keeping in mind these requirements, the rules are as follows. Below,
whenever we use the assertion ``{\tt x} free'' in the antecedent of
a rule we mean
that {\tt x} is not free in the consequent of the rule.


\subsection{
      Structural and Logical Rules
}


First, we present the structural rules for $\vdash$. The
judgment
$\Gamma\vdash {\tt e} \ty {\tt T}$ is
intuitionistic. That is, $\Gamma$ is considered a multiset of assertions, and
the judgment possesses the inference rules:

\infrule{
\Gamma \vdash {\tt e} \ty {\tt T}
\andalso
\Gamma \vdash {\tt S} \ty {\tt type}
\andalso
    \mbox{{\tt x} not in $\mathit{var}(\Gamma)$}
}{
\Gamma, {\tt x} \ty {\tt S} \vdash {\tt e} \ty {\tt T}
}



\infrule{
\Gamma \vdash {\tt e} \ty {\tt T}
\andalso
\Gamma \vdash {\tt c} \ty {\tt boolean}
}{
\Gamma, {\tt c} \vdash {\tt e} \ty {\tt T}
}


We also assume the following rule for conjunctions on the left and right:

\infrule{
\Gamma, \phi_1 , \phi_2 \vdash \phi
}{
\Gamma, (\phi_1 , \phi_2 ) \vdash \phi
}


\infrule{
\Gamma \vdash \phi_1 
\andalso
     \Gamma \vdash \phi_2  
}{
\Gamma \vdash (\phi_1 , \phi_2 )
}



Existential quantification is governed by the following standard rules,
specialized for the particular kinds of formulas we are dealing with:


\infrule{
\Gamma \vdash {\tt e} \ty {\tt T}[{\tt t}/{\tt x}]
\andalso
\Gamma \vdash {\tt t} \ty {\tt S}
}{
\Gamma \vdash {\tt e} \ty ({\tt x} \ty {\tt S};~{\tt T})
}


\infrule{
\Gamma, {\tt x} \ty {\tt S}, {\tt c} \vdash {\tt e} \ty {\tt T}
\andalso
\mbox{{\tt x} fresh}
}{
\Gamma, ({\tt x} \ty {\tt S};~{\tt c}) \vdash {\tt e} \ty {\tt T}
}


\infrule{
\Gamma, {\tt x} \ty {\tt S}, {\tt y} \ty \Xcd{C\{c\}} \vdash {\tt e} \ty {\tt T}
\andalso
\mbox{{\tt x} fresh}
}{
\Gamma, {\tt y} \ty \Xcd{C\{x:S; c\}} \vdash {\tt e} \ty {\tt T}
}

}

\subsection{
Type well-formedness
}

\begin{figure}

\infrule{
\mbox{\Xcdmath{C[$\tbar{X}$]($\tbar{x}$: $\tbar{T}$)\{c\} ext T \{\ $\tbar{M}$\ $\tbar{F}$\ \}}}
}{
\vdash \Xcd{C} \ty \Xcd{type}
}

\infrule{
\Gamma \vdash \Xcd{T} \ty \Xcd{type}
\andalso
\Gamma, \Xcd{self} \ty \Xcd{T} \vdash \Xcd{c} \ty \Xcd{Boolean}
\andalso
\sigma(\Gamma) \vdashC \Xcd{c}~\mbox{OK}
}{
\Gamma \vdash \Xcd{T\{c\}} \ty \Xcd{type}
}

\infrule{
\Gamma \vdash \Xcd{p} \ty \Xcd{T}
\andalso
\Gamma \vdash \Xcd{T}~\Xcd{has}~\Xcd{X}
}{
\Gamma \vdash \Xcd{p.X} \ty \Xcd{type}
}

\infax{
\Gamma, \Xcd{X} \ty \Xcd{type} \vdash \Xcd{X} \ty \Xcd{type}
}

\caption{Type well-formedness}
\label{fig:type-wf}
\end{figure}

\subsection{
      Type inference rules
}

\subsubsection{Constraint rules}

\begin{figure}

\infrule[Has-class]{
\mbox{\Xcdmath{C[$\tbar{X}$]($\tbar{x}$: $\tbar{T}$)\{c\} ext T \{\ K\ $\tbar{M}$\ $\tbar{F}$\ \}}}
}{
\vdash {\tt C}~{\tt has}~{\tt K} \\
\vdash {\tt C}~{\tt has}~{\tt X}_i \\
\vdash {\tt C}~{\tt has}~{\tt x}_i \ty {\tt T}_i \\
\vdash {\tt C}~{\tt has}~{\tt M}_i \\
\vdash {\tt C}~{\tt has}~{\tt F}_i
}

\infrule[Has-sub]{
{\tt Z} \not= {\tt K}
\andalso
\Gamma \vdash {\tt T}_1~{\tt has}~{\tt Z}
\andalso
\sigma(\Gamma) \vdash {\tt T}_2 \subtype {\tt T}_1
}{
\Gamma \vdash {\tt T}_2~{\tt has}~{\tt Z}
}

\caption{Structural constraints}
\label{fig:structural}
\end{figure}

\subsubsection{
        Expression typing judgment
}

\eat{
We define \Xcd{T\{c\}} as follows:

\begin{align*}
\Xcd{D\{c\}} &= \Xcd{D\{c\}} \\
\Xcd{D\{d\}\{c\}} &= \Xcd{D\{d,c\}} \\
\Xcd{X\{c\}} &= \Xcd{X\{c\}} \\
\Xcd{p.X\{c\}} &= \Xcd{p.X\{c\}} \\
\end{align*}

\begin{align*}
\exists\Xcd{x}\ty \Xcd{S}.~\Xcd{C} &= \Xcd{C} \\
\exists\Xcd{x}\ty \Xcd{S}.~\Xcd{C\{c\}} &=
        \Xcd{C\{}\exists\Xcd{x}.~\sigma(\Xcd{x}\ty \Xcd{S})\Xcd{,c\}} \\
\exists\Xcd{x}\ty \Xcd{S}.~\Xcd{p.X} &=
        \Xcd{p.X\{}\exists\Xcd{x}.~\sigma(\Xcd{x}\ty \Xcd{S})\Xcd{,c\}} \\
\exists\Xcd{x}\ty \Xcd{S}.~\Xcd{X} &=
        \Xcd{X\{}\exists\Xcd{x}.~\sigma(\Xcd{x}\ty \Xcd{S})\Xcd{,c\}} \\
\end{align*}
} %eat

\begin{figure}

\infrule[T-sub]{
\Gamma \vdash \Xcd{e} \ty \Xcd{S}
\andalso
\sigma(\Gamma) \vdashC \Xcd{S} \subtype \Xcd{T}
\andalso
\Gamma \vdash \Xcd{T} \ty \Xcd{type}
}{
\Gamma \vdash \Xcd{e} \ty \Xcd{T}
}

\infax[T-bool]{
\vdash \Xcd{true} \ty \Xcd{Boolean\{self==true\}} \\
\vdash \Xcd{false} \ty \Xcd{Boolean\{self==false\}}
}

\infax[T-int]{
\vdash n \ty \Xcd{Int\{self==}n\Xcd{\}}
}

\infrule[T-eq]{
\Gamma \vdash \Xcd{e}_1 \ty \Xcd{T}_1
\andalso
\Gamma \vdash \Xcd{e}_2 \ty \Xcd{T}_2
}{
\Gamma \vdash \Xcd{e}_1 \equals \Xcd{e}_2 \ty
        \extyty{\Xcd{z}_1}{\Xcd{T}_1}{\Xcd{z}_2}{\Xcd{T}_2}
        {\Xcd{Boolean\{self==(}{\Xcd{z}_1}\equals{\Xcd{z}_2}\Xcd{)\}}}
}

\infrule[T-teq]{
\Gamma \vdash \Xcd{T}_1 \ty \Xcd{type}
\andalso
\Gamma \vdash \Xcd{T}_2 \ty \Xcd{type}
}{
\Gamma \vdash \Xcd{T}_1 \equals \Xcd{T}_2 \ty \Xcd{Boolean}
}

\infrule[T-tsub]{
\Gamma \vdash \Xcd{T}_1 \ty \Xcd{type}
\andalso
\Gamma \vdash \Xcd{T}_2 \ty \Xcd{type}
}{
\Gamma \vdash \Xcd{T}_1 \subtype \Xcd{T}_2 \ty \Xcd{Boolean}
}

\infax[T-var]{
\Gamma, \Xcd{x} \ty \Xcd{T} \vdash \Xcd{x} \ty \Xcd{T}
}

\infrule[T-cast]{
\Gamma \vdash \Xcd{e} \ty \Xcd{S}
\andalso
\Gamma \vdash \Xcd{T} \ty \Xcd{type}
}{
\Gamma \vdash \Xcd{e}~\Xcd{as}~\Xcd{T} \ty \Xcd{T}
}

\infrule[T-field]{
\Gamma \vdash \Xcd{e} \ty \Xcd{T}
%\\
%\Xcd{T} = \exty{\Xcd{z}}{\Xcd{S}}{\Xcd{S\{self==z\}}}
\\
\Xcd{T}~\Xcd{has}~\Xcd{f}\Xcd{\{c\}} \ty \Xcd{U}
\\
\sigma(\Gamma, \Xcd{this} \ty \Xcd{T}) \vdashC \Xcd{c}
}{
\Gamma \vdash \Xcd{e}.\Xcd{f} \ty \exty{\Xcd{this}}{\Xcd{T}}{\Xcd{U\{self==this.f\}}}
}

\infrule[T-invk]{
\Gamma \vdash \Xcd{e}_0 \ty \Xcd{T}_0
\andalso
\Gamma \vdash \xbar{e} \ty \xbar{T}
\\
\Xcd{T}_0~\Xcd{has}~\Xcd{def}~\Xcd{m[}\xbar{X}\Xcd{](}\xbar{x} \ty \xbar{S}\Xcd{)\{c\}} \ty {\tt U}~\Xcd{=}~\Xcd{e}
\\
\Gamma' = \Gamma, \xbar{X} \ty \Xcd{type},
        \Xcd{this} \ty \Xcd{T}_0,
        \xbar{x} \ty \xbar{T},
        \xbar{X} \equals \xbar{V}
\\
\sigma(\Gamma') \vdashC \Xcd{c}
\\
\sigma(\Gamma') \vdashC \xbar{T} \subtype \xbar{S}
}{
\Gamma \vdash
\Xcd{e}_0.\Xcd{m[}\xbar{V}\Xcd{](}\xbar{e}\Xcd{)} \ty
\extytyty{\xbar{X}}{\Xcd{type}}{\Xcd{this}}{\Xcd{T}_0}{\xbar{x}}{\xbar{T}}{\Xcd{U}}
}

\infrule[T-new]{
\Gamma \vdash \xbar{e} \ty \xbar{T}
\\
\Xcd{C}~\Xcd{has}~\Xcd{def}~\Xcd{this[}\xbar{X}\Xcd{](}\xbar{x} \ty \xbar{S}\Xcd{)\{c\}} \ty {\tt U}~\Xcd{=}~\dots
\\
\Gamma' = \Gamma, \xbar{X} \ty \Xcd{type}, \Xcd{this} \ty \Xcd{C}, \xbar{x} \ty \xbar{T}, \xbar{V} \equals \xbar{X}
\\
\Gamma'' = \Gamma, \xbar{X} \ty \Xcd{type}, \Xcd{this} \ty \Xcd{U}, \xbar{x} \ty \xbar{T}, \xbar{V} \equals \xbar{X}
\\
\sigma(\Gamma') \vdashC \Xcd{c}
\\
\sigma(\Gamma') \vdashC \xbar{T} \subtype \xbar{S}
\\
\sigma(\Gamma'') \vdashC \mathit{inv}(\Xcd{C}),
}{
\Gamma \vdash
\Xcd{new}~\Xcd{C[}\xbar{V}\Xcd{](}\xbar{e}\Xcd{)} \ty
\extytyty{\xbar{X}}{\Xcd{type}}{\Xcd{this}}{\Xcd{C}}{\xbar{x}}{\xbar{T}}{\Xcd{U}}
}

\caption{Typing rules}
\label{fig:typing}
\end{figure}

The cast rule
\rn{T-cast}
requires that the cast type be well-formed. 

The field access rule \rn{T-field}
differs from the rule in the paper in that there is no need to
substitute a fresh variable for the receiver. Note that {\tt this} may be free
in {\tt S}---that would be a reference to the current object in the code in
which {\tt e.f} occurs, not a reference to the receiver of the {\tt e.f} field
selection (i.e., the object obtained by evaluating {\tt e}).

\noindent
if we allow adding constraints to arbitrary types---do we?

TODO: type parameters!

Now we consider the rule for method invocation. Assume that in a type
environment $\Gamma$ the expressions ${\tt e_0}, \dots, {\tt e_n}$
have the types ${\tt T_0}, \dots, {\tt T_n}$.
Since the
actual values of these expressions are not known, we shall assume that
they take on some fixed but unknown values
                                     ${\tt z_0}, \dots, {\tt z_n}$
of types ${\tt T_0}, \dots, {\tt T_n}$.
Now, for ${\tt z_0}$ as receiver, let us assume that the type
${\tt T_0}$ has a method named ${\tt m}$
with signature
$[\xbar{Z}](\xbar{z} \ty \xbar{S})\Xcd{\{c\}} \to {\tt U}$
(Let ${\tt T_0} = \Xcd{C\{d\}}$.
 If there is no
method named {\tt m} for the class {\tt C} then this method invocation cannot be
type-checked. Without loss of generality, we may assume that the
type parameters of this method are named
                                     ${\tt Z_1}, \dots, {\tt Z_k}$, and
the value parameters are named
                                     ${\tt z_1}, \dots, {\tt z_n}$
since we are free to choose
variable names as we wish.)
Now, for the method to be invokable,
it must be the case that the types
    ${\tt T_1}, \dots, {\tt T_n}$
are subtypes of
    ${\tt S_1}, \dots, {\tt S_n}$.
(Note
that there may be no occurrences of {\tt this} in
    ${\tt S_1}, \dots, {\tt S_n}$---they have been
replaced by ${\tt z_0}$.)
Further, it must be the case that for these parameter
values, the constraint {\tt c} is entailed. Given all these assumptions it
must be the case that the return type is {\tt U}, with all the parameters
    ${\tt z_0}, \dots, {\tt z_n}$
existentially quantified.


\subsubsection{
        Class OK judgment
}

The following rule is modified from what we had in the paper to ensure
that all the types are well-formed (under the assumption {\tt this} \ty {\tt C}).
Note
that the variables $\xbar{x}$ are permitted to occur in the types $\Xcd{T}_0, \xbar{T}$,
hence their typing assertions must be added to $\Gamma$.

\infrule[Method OK]{
\Gamma = \Xcd{this} \ty \Xcd{C\{self==this},\mathit{inv}(\Xcd{C})\Xcd{\}},
        \xbar{x} \ty \xbar{T}\Xcd{\{self==\}}\xbar{x}\Xcd{\}},
        \Xcd{c}
\\
\Gamma \vdash \Xcd{e} \ty \Xcd{U}
\\
\sigma(\Gamma) \vdashC \Xcd{U} \subtype \Xcd{T}
}{
\Xcd{def}~\Xcd{m[}\xbar{X}\Xcd{](}\xbar{x} \ty
\xbar{T}\Xcd{)\{c\}} \ty \Xcd{T}~\Xcd{=}~\Xcd{e}~\mbox{OK in}~\Xcd{C}
}


This rule did not exist in our submission. This is necessary to ensure
that the types of fields are well-formed.

\infrule[Field OK]{
\Xcd{this} \ty \Xcd{C}, \Xcd{c} \vdash \Xcd{T} \ty \Xcd{type}
}{
\Xcd{val}~\Xcd{f}\Xcd{\{c\}} \ty \Xcd{T}~\mbox{OK in}~\Xcd{C}
}


This rule is now modified to ensure that all the types and methods in
the body of the class are well-formed.

\infrule[Class OK]{
K~\mbox{OK in}~{\tt C}
\\
\xbar{M}~\mbox{OK in}~{\tt C}
\\
\xbar{F}~\mbox{OK in}~{\tt C}
\\
{\tt this} \ty {\tt C} \vdash {\tt T} \ty {\tt type}
}{
\mbox{\Xcdmath{C[$\tbar{X}$]($\tbar{x}$: $\tbar{T}$)\{c\} ext T \{\ K\ $\tbar{M}$\ $\tbar{F}$\ \}}}~\mbox{OK}
}

TODO: method overriding


\subsubsection{
        Subtype judgment
}

\infrule{
\sigma(\Gamma) \vdash_{\cal C} {\tt T_1} \subtype {\tt T_2}
}{
\Gamma \vdash {\tt T_1} \subtype {\tt T_2}
}



