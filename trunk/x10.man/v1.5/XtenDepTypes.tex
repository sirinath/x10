\chapter{Dependent types}\label{XtenDepTypes}\index{dependent types}
\def\withmath#1{\relax\ifmmode#1\else{$#1$}\fi}
\def\LL#1{\withmath{\lbrack\!\lbrack #1\rbrack\!\rbrack}}

Dependent types are a fundamental extension to the type system for
Java-like languages. A dependent type records constraints on
{\em properties} (final fields) of the type.  Since types are a fundamental
building block of Java-like languages, the introduction of dependent
types affects many facets of the language simultaneously---the
definition of types, classes, interfaces, methods, constructors,
fields, inheritance, overriding, overloading, and type related
operators (cast, and \xcd"instanceof").

Indeed, dependent types bring substantial expressiveness to Java-like
languages. Just as generic classes permit a single definition for a
class $C$ to be treated as a template for an unbounded number of classes
obtained from $C$ by ``instantiating'' $C$ with type arguments, so also a
dependent type permits a single definition to produce a potentially
unbounded family of types. That is, just as generic types permit a
programmer to express and use {\em functions} from types to types, so also
dependent types permit a programmer to express and use functions from values
to types.  Indeed, the family of types generated from $C$ form a
lattice of subtypes of $C$, one for each constraint on the properties
of $C$ expressible in the underlying constraint system, but all sharing
the same "structure".

In \Xten{} dependent types are checked statically. However, as in
Java-like languages an instanceof relation is available to dynamically
check that an object belongs to a particular (dependent) type. Also a
cast operation is available to force the runtime system to treat an
object o as belonging to a particular dependent type (the operation
throws a cast exception if it is not possible to do so).

Dependent types are also the basis for an implicit syntax for
\Xten{}. This is discussed in the last section.

\section{Properties}\label{DepType:Properties}\index{properties}
The dependent type system is built on the notion of {\em properties}, 
for types (classes and interfaces).

\begin{verbatim}
NormalClassDeclaration ::= 
   ClassModifiersopt class Identifier 
   TypePropertyListopt PropertyListopt WhereClauseopt
   Superopt Interfacesopt ClassBody

NormalInterfaceDeclaration ::= 
   InterfaceModifiersopt interface Identifier 
   TypePropertyListopt PropertyListopt WhereClauseopt
   ExtendsInterfacesopt InterfaceBody

TypePropertyList     ::= [ TypeProperties ]
TypeProperties       ::= TypeProperty
                     | TypeProperties , TypeProperty
TypeProperty         ::= identifier    
TypePropertyListopt  ::= $Empty | TypePropertyList

PropertyList     ::= ( Properties )
Properties       ::= Property
                 | Properties , Property
Property         ::= identifier : Type
PropertyListopt  ::= $Empty | PropertyList
\end{verbatim}

A type property has a name.
A value property has a name and a type. The declaration of a type (class or
interface) introduces a sequence of defined properties for the type. 

\begin{staticrule*}
It is a compile-time error for a class
defining a value property \xcd"x: T" to have an ancestor class that defines
a value property with the name \xcd"x".  
It is a compile-time error for a class
defining a type property \xcd"X" to have an ancestor class that defines
a type property with the name \xcd"X".  
\end{staticrule*}


Each class \xcd"C" defining a property \xcd"x: T" implicitly has a field

\begin{xten}
public val x : T;
\end{xten}

\noindent and a getter method 

\begin{xten}
public final def x(): T { return x; }
\end{xten}

\noindent Each interface \xcd"I" defining a property \xcd"x: T"
implicitly has a getter method

\begin{xten}
public def x(): T;
\end{xten}

\begin{staticrule*}
It is a compile-time error for a class or
interface defining a property \xcd"x :T" to have an existing method with
the signature \xcd"x(): T".   
\end{staticrule*}


Properties are used to build dependent types from classes, as
described below (\S~\ref{DepType:DepType}).

The \xcd"WhereClause" in a class or interface declaration specifies an
explicit condition on the properties of the type, and is discussed further
below (\S~\ref{DepType:Class}, \ref{DepType:Interface}).

\begin{staticrule*}
     Every constructor for a class defining
   properties \xcdmath"x$_1$: T$_1$, $\ldots$, x$_n$: T$_n$" must ensure that each of the fields
   corresponding to the properties is definitely initialized
   (cf. requirement on initialization of final fields in Java) before the
   constructor returns.  
\end{staticrule*}


\begin{example}
 A class representing immutable 2-d points, with two properties \xcd"i" and 
\xcd"j".
\begin{xten}
value class Point(i: int i, j: int) { ... }
value class point(rank: int) { ... }
\end{xten}
  
\end{example}

\section{Dependent types}\label{DepType:DepType}\index{dependent type}

A dependent type (deptype) is of the form \xcd"C{c}" where \xcd"C" is a class
or interface, and \xcd"c" is a {\em constraint}. (\xcd"C" is said to be 
{\em the base type} of the deptype, and \xcd"c" the {\em constraint} of the deptype.)  A
constraint is a boolean expression that can only use a predefined set
of operators and methods. 

\begin{verbatim}
Type  ::= PrimitiveType
        | ClassOrInterfaceType
        | ArrayType DepParamtersopt
        | nullable [ Type ] DepParamtersopt
        | future [ Type ] DepParamtersopt
PrimitiveType ::= NumericType DepParametersopt
                | boolean DepParametersopt
ClassOrInterfaceType   ::= TypeName DepParametersopt
ClassType              ::= TypeName DepParametersopt
InterfaceType          ::= TypeName DepParametersopt
PlaceTypeSpecifier     ::= ! PlaceTypeopt
PlaceType              ::= current
                         | Expression

TypeArgs         ::= [ TypeList ]
TypeList         ::= Type
                   | TypeList , Type
ValueArgs        ::= ( ArgumentList )
ArgumentList     ::= Expression
                   | ArgumentList , Expression
DepClause        ::= { Constraint }
Constraint       ::= Expression
WhereClause      ::= where Constraint
DepParametersopt ::= TypeArgsopt ValueArgsopt
                         DepClauseopt PlaceTypeSpecifieropt
TypeArgsopt      ::= $Empty | TypeArgs
ValueArgsopt     ::= $Empty | ValueArgs
DepClauseopt     ::= $Empty | DepClause
WhereClauseopt   ::= $Empty | WhereClause
PlaceTypeopt     ::= $Empty | PlaceType
ValueInitailizers
\end{verbatim}

\begin{staticrule}{Variable Occurrence}
In a deptype \xcd"T" = \xcd"C{c}", the only variables that may occur in c are (a)
self, (b) properties visible at \xcd"T", (c) final local variables, final
method parameters or final constructor parameters visible at \xcd"T", (d)
final fields visible at \xcd"T"'s lexical place in the source program.  
\end{staticrule}

\begin{staticrule}{Restrictions on \xcd"this"}
  The special variable \xcd"this" may be used in a depclause for a type \xcd"T"
  only if (a)~it occurs in a property declaration for a class, (b)~it
  occurs in an instance method, (c)~it occurs in an instance field, (d)~it
  occurs in an instance initializer.
\end{staticrule}

In particular it may not be used in types that occur in a static
context, or in the arguments, body or return type of a constructor or
in the extends or implements clauses of class and interface
definitions.  In these contexts the object that \xcd"this" would
correspond to has either not been formed or is not well defined.

\begin{staticrule}{Variable visibility}
  If a type \xcd"T" occurs in a field, method or constructor
  declaration, then all variables used in \xcd"T" must have at least the
  same visibility as the declaration.  The relation ``at least the same
  visibility as'' is given by the transitive closure of:

  \begin{xten}
public    > protected
protected > private
public    > package
package   > private
  \end{xten}

All inherited properties of a type \xcd"T" are visible in the property
list of \xcd"T", and the body of \xcd"T".

\end{staticrule}

In general local variables/parameters/properties/fields are visible at
\xcd"T" if they are defined before \xcd"T" in the program. This rule applies to
types in property lists as well as parameter lists (for methods and
constructors).  An exception is made for the return type of a method:
all the arguments to the method are considered to be visible, even
though they occur lexically after the return type (given the \Java{}
syntactic convention that the return type for a method precedes the
argument list for the method).

We permit variable declarations \xcd"v: T" where \xcd"T" is obtained
from a dependent type \xcd"C{c}" by replacing one or more occurrences
of \xcd"self" in \xcd"c" by \xcd"v". (If such a declaration \xcd"v: T"
is type-correct, it must be the case that the variable \xcd"v" is not
visible at the type \xcd"T". Hence we can always recover the
underlying deptype \xcd"C{c}" by replacing all occurrences of \xcd"v"
in the constraint of \xcd"T" by \xcd"self".)

For instance, \xcd"v: int{v > 0}" is shorthand for \xcd"v: int{self > 0}".

\begin{staticrule}{Constraint type}
  The type of a constraint \xcd"c" must be \xcd"boolean".  
\end{staticrule}

A variable occuring in the constraint \xcd"c" of a deptype, other than
\xcd"self" or a property of \xcd"self", is said to be a {\em
parameter} of \xcd"c".\label{DepType:Parameter} \index{parameter}

An instance \xcd"o" of \xcd"C" is said to be of type \xcd"C{c}"
(or: {\em belong to}
\xcd"C{c}") if the predicate \xcd"c" evaluates to \xcd"true" in the current lexical
environment, augmented with the binding \Xcd{self} $\mapsto$ \Xcd{o}. We shall
use the function \LL{\mbox{\Xcd{C\{c\}}}} to denote the set of
objects that belong to \xcd"C{c}". 

\section{Type definition}\label{DepType:Class}
A class definition 
\begin{verbatim}
ClassModifiersopt class identifier 
    TypePropertyListopt PropertyListopt WhereClauseopt
    Superopt Interfacesopt ClassBody  
\end{verbatim}

\noindent and an interface definition
\begin{verbatim}
InterfaceModifiersopt interface identifier 
    TypePropertyListopt PropertyListopt WhereClauseopt
    ExtendsInterfacesopt InterfaceBody  
\end{verbatim}

\noindent may reference several deptypes. The types of properties, the
specification of the super clause and the specification of interfaces
may each involve deptypes.

\begin{verbatim}
Super ::= extends ClassType
Superopt ::= $Empty | Super

Interfaces ::= implements InterfaceTypeList
InterfaceTypeList ::= InterfaceType
       | InterfaceTypeList , InterfaceType
Interfacesopt ::= $Empty | Interfaces
\end{verbatim}

\section{Where clauses}\label{DepType:WhereClauses}\index{where clauses}

There is a general recipe for constructing a list of parameters or
properties \xcdmath"x$_1$: T$_1${c$_1$}, $\dots$, x$_k$: T$_k${c$_k$}" that must satisfy a given
(satisfiable) constraint \xcd"c". 

\begin{xtenmath}
class Foo(x$_1$: T1{x$_2$: T$_2$; ...; x$_k$: T$_k$; c},
          x$_2$: T2{x$_3$: T$_3$; ...; x$_k$: T$_k$; c},
          $\dots$
          x$_k$: T$_k${c}) {
  $\dots$
}
\end{xtenmath}

The first type \xcdmath"x$_1$: T$_1${x$_2$: T$_2$; $\dots$; x$_k$: T$_k$; c}" is consistent iff
\xcdmath"$\exists$x$_1$: T$_1$, x$_2$: T$_2$, $\dots$, x$_k$: T$_k$. c" is consistent. The second is
consistent iff
\begin{xtenmath}
$\forall$x$_1$: T$_1${x$_2$: T$_2$; $\dots$; x$_k$: T$_k$; c}
$\exists$x$_2$: T$_2$. $\exists$x$_3$: T$_3$, $\dots$, x$_k$: T$_k$. c
\end{xtenmath}
\noindent But this is always true. Similarly for the conditions for the other
properties.

Thus logically every satisfiable constraint \xcd"c" on a list of parameters
\xcdmath"x$_1$", \dots, \xcdmath"x$_k$"
can be expressed using the dependent types of 
\xcdmath"x$_i$", provided
that the constraint language is rich enough to permit existential
quantifiers.

Nevertheless we will find it convenient to permit the programmer to
explicitly specify a depclause after the list of properties, thus:
\begin{xten}
class Point(i: int, j: int) { ... }
class Line(start: Point, end: Point) where (end != start)
  { ... }
class Triangle (a: Line, b: Line, c: Line) where
        (a.end == b.start && b.end == c.start &&
         c.end == a.start) { ... }
class SolvableQuad(a: int, b: int, c: int) where 
                   (a*x*x+b*x+c==0)  { ... }
class Circle (r: int, x: int, y: int) where
              (r > 0 && r*r==x*x+y*y){ ... }
class NonEmptyList extends List{n > 0} {...}
\end{xten}

Consider the definition of the class \xcd"Line". This may be thought of as
saying: the class \xcd"Line" has two fields, \xcd"start: Point" and
\xcd"end: Point".
Further, every instance of \xcd"Line" must satisfy the constraint that
\xcd"end != start". Similarly for the other class definitions. 

In the general case, the production for \xcd"NormalClassDeclaration"
specifies that the list of properties may be followed by a \xcd"WhereClause":

\begin{verbatim}
NormalClassDeclaration ::= 
   ClassModifiersopt class Identifier 
   TypePropertyListopt PropertyListopt WhereClauseopt
   Superopt Interfacesopt ClassBody

NormalInterfaceDeclaration ::= 
   InterfaceModifiersopt interface Identifier 
   TypePropertyListopt PropertyListopt WhereClauseopt
   ExtendsInterfacesopt InterfaceBody
\end{verbatim}

All the properties in the list, together with inherited properties,
may appear in the \xcd"WhereClause". A where clause \xcd"c" with
type property list \xcd"X1, ...., Xk"
and value property list \xcd"x1: T1, ...., xn: Tn"
for a class \xcd"Foo" is said to be consistent if each of the \xcd"Ti" are
consistent and the constraint
\begin{xtenmath}
$\exists$X$_1$, $\dots$, X$_k$, x$_1$: T$_1$, $\dots$, x$_n$: T$_n$, self: Foo. c
\end{xtenmath}
\noindent is valid (always true).

\section{Type invariants}\label{DepType:TypeInvariant}\index{Type Invariant}

With every defined class or interface \xcd"T" we associate a {\em type
invariant} $\mathit{inv}($\xcd"T"$)$ as follows. The type invariant associated with
\xcd"x10.lang.Object" is the proposition

\begin{xten}
self.location: nullable[place]
\end{xten}

The type invariant associated with any interface \xcd"I" that extends
interfaces \xcd"I1,..., Ik" and defines properties \xcd"x1: P1, ..., xn: Pn" and
specifies a where clause \xcd"c" is given by:

\begin{xten}
ti(I1) && ... && tk(Ik) && self.x1: P1
&& ... && self.xn: Pn && c  
\end{xten}

Similarly the type invariant associated with any class \xcd"C" that
implements interfaces \xcd"I1,..., Ik", extends class \xcd"D" and defines
properties \xcd"x1: P1,..., xn: Pn" and specifies a where clause \xcd"c" is
given by:
\begin{xten}
i(D) && self.x1: P1 && ... && self.xn: Pn && c  
\end{xten}

The {\sc Int Implements} Static Semantic rule below requires that the
type invariant associated with a class entail the type invariants of
each interface that it implements.

The static semantics rules below guarantee that for any variable \xcd"v" of
type \xcd'T{c}' (where \xcd"T" is an interface name or a class name) the only
objects \xcd"o" that may be stored in \xcd"v" are such that \xcd"o" satisfies
$\mathit{inv}(\mbox{\xcd"T"}[\mbox{\xcd"o"}/\mbox{\xcd"this"}])
\wedge \mbox{\xcd"c"}[\mbox{\xcd"o"}/\mbox{\xcd"self"}]$.

%% XXX check parens

\section{Consistency of deptypes}\label{DepType:Consistency}\index{deptype,consistency}

A dependent type \xcd"C{c}" may contain zero or more parameters. We require
that a type never be empty---so that it is possible for a variable of
the type to contain a value. This is accomplished by requiring that
the constraint \xcd"c" must be satisfiable {\em regardless} of the value assumed
by parameters to the contraint (if any). Formally, consider a type
\xcd"T" = \xcd"C{c}", with the variables \xcd"f1: F1, ..., fk: Fk"
free in \xcd"c".  Let 
\xcd"S={f1: F1, .., fk: Fk, fk+1: Fk+1, ... fn: Fn}" be the smallest set of
declarations containing \xcd"f1: F1, ..., fk: Fk" and closed under the rule:
\xcd"f: F" in \xcd"S" if a reference to variable \xcd"f" (which
is declared as \xcd"f: F") occurs in a type in \xcd"S".

(NOTE: The syntax rules for the language ensure that \xcd"S" is always
finite. The type for a variable \xcd"v" cannot reference a variable whose
type depends on \xcd"v".)

We say that \xcd"T" = \xcd"C{c}" is {\em parametrically consistent} (in brief:
{\em consistent}) if

\begin{itemize}
  \item Each type \xcd"F1, ..., Fn" is (recursively) parametrically consistent, and
\item It can be established that \xcd"forall f1: F1, .., fn: Fn.  exists self: C . c && i(C)".
\end{itemize}
\noindent where \xcd"i(C)" is the invariant associated with the type \xcd"C"
(\S~\ref{DepType:TypeInvariant}).  Note by definition of \xcd"S" the formula on the
above has no free variables.

\begin{staticrule*}
For a declaration \xcd"T v" to be type-correct, \xcd"T" must be parametrically
consistent. The compiler issues an error if it cannot determine
the type is parametrically consistent.
\end{staticrule*}

\begin{example}

A class that represents a line has two distinct points:
\begin{xten}
class Array(rank: int, 
            region: region{rank==this.rank}) {...}  
\end{xten}
\end{example}

One can use deptypes to define other closed geometric figures as well.

\begin{example}
Here is an example:
\begin{xten}
class Point(x: int, y: int) {...}
class Line(start: Point, 
          end: Point{self != this.start}) 
{...}      
\end{xten}
\end{example}


To see that the declaration \xcd"end: Point{self != start}" is
parametrically consistent, note that the following formula is valid:
\begin{xten}
forall this: Line. 
  exists self: Point. self != this.start  
\end{xten}
\noindent since the set of all \xcd"Points" has more than one element.

\begin{example}
A triangle has three lines sharing three vertices.
\begin{xten}
class Triangle 
 (a: Line, 
  b: Line{a.end == b.start}, 
  c: Line{b.end == c.start && c.end == a.start}) 
 { ... }
\end{xten}
\end{example}


Given \xcd"a: Line", the type \xcd"b: Line{a.end == b.start}" is consistent,
and
given the two, the type \xcd"c: Line{b.end == c.start, c.end == a.start}"
is consistent.

%%Similarly:
%%
%%   // A class with properties a, b,c,x satisfying the 
%%   // given constraints.
%%   class SolvableQuad(a: int, b: int, 
%%                      c: int{b*b - 4*a*c >= 0},
%%                      x: int{a*x*x + b*x + c==0}) { 
%%     ...
%%   }
%%
%%  // A class with properties r, x, and y satisfying
%%  // the conditions for (x,y) to lie on a circle with center (0,0)
%%  // and radius r.
%%   class Circle (r: int{r > 0},
%%                 x: int{r*r - x*x >= 0},
%%                 y: int{y*y == r*r -x*x}) { 
%%   ...
%%   }

\section{Equivalence of deptypes}\label{DepType:Equivalence}\label{deptype,equivalence}

Two dependent types \xcd"C{c}" and \xcd"C{d}" are said to be {\em equivalent} if 
\xcd"c" is true whenever \xcd"d" is, and vice versa. Thus, 
$\LL{\mbox{\Xcd{C\{c\}}}} = \LL{\mbox{\Xcd{C\{d\}}}}$.

Note that two deptypes that are syntactically different may be
equivalent. For instance, \xcd"int{self >= 0}" and
\xcd"int{self == 0 || self > 0}" are equivalent though they are syntactically
distinct. The \Java{} type system is essentially a nominal system---two
types are the same if and only if they have the same name. The \Xten{}
type system extends the nominal type system of \Java{} to permit
constraint-based equivalence.

A dependent type \xcd"C{c}" is said to refine a type \xcd"C{d}" if
\xcd"c" implies \xcd"d".  In such a case we have
$\LL{\mbox{\Xcd{C\{c\}}}}$ is a
subset of $\LL{\mbox{\Xcd{C\{d\}}}}$. All dependent types defined on \xcd"C" refine
\xcd"C" since \xcd"C" is equivalent to \xcd"C{true}".

\section{Type checking rules}
\subsection{Class definitions}

Consider a class definition
\begin{xten}
ClassModifiersopt 
 class C (P1 x1,..., Pn xn)  extends D(:d) 
   implements I1(:c1),..., Ik(:ck)
 ClassBody  
\end{xten}

Each of the following static semantics rules must be satisfied:

\begin{staticrule}{Int-implements}
The type invariant \xcd"i(C)" of \xcd"C" must entail
\xcd"ci[this/self]" for each \xcd"i" in \xcd"1:k".  
\end{staticrule}

\begin{staticrule}{Super-extends}
The return type \xcd"c" of each constructor in \xcd"ClassBody"
must entail \xcd"d".
\end{staticrule}

\subsection{Constructor definitions}

A constructor for a class \xcd"C" is guaranteed to return an object of the
class on successful termination. This object must satisfy i(C), the
class invariant associated with \xcd"C" (\S~\ref{DepType:TypeInvariant}). However,
often the objects returned by a constructor may satisfy {\em stronger}
properties than the class invariant. \Xten{}'s dependent type system
permits these extra properties to be asserted with the constructor in
the form of a deptype (the ``return type'' of the constructor):

\begin{verbatim}
ConstructorDeclarator ::=  
  def this TypePropertyListopt ( FormalParameterListopt )
  ReturnTypeopt WhereClauseopt 
ReturnTypeopt ::= $Empty | ReturnType
ReturnType    ::= : Type
\end{verbatim}

As with method declarations, the parameter list for the constructor
may specify a where clause that is to be satisfied by the parameters
to the list.

\begin{example}
Here is another example.
\begin{xten}
public class List[T](n: int{n >= 0}) {
 protected head: nullable[T];
 protected tail: nullable[List[T](n-1)];
 public def this(o: T, t: List[T]) : List[T](t.n+1) = {
     n = t.n+1;
     tail = t;
     head = o;
 }
 public def this() : List[T](0) = {
     n = 0;
     head = null;
     tail = null;
 }
 ...
}
\end{xten}
The second constructor returns a \xcd"List" that is guaranteed to have \xcd"n==0";
the first constructor is guaranteed to return a List with \xcd"n>0"
(in fact, \xcd"n==t.n+1", where the argument to the constructor is \xcd"t"). 
This is recorded by the programmer in the deptype associated with the
constructor.
\end{example}

\begin{staticrule}{Super-invoke}
   Let \xcd"C" be a class with properties \xcd"p1: P1, ..., pn: Pn", invariant \xcd"c"
   extending the deptype \xcd"D{d}" (where \xcd"D" is the name of a class).

   For every constructor in \xcd"C" the compiler checks that the call to
   super invokes a constructor for D whose return type is strong enough
   to entail d. Specifically, if the call to super is of the form 
     \xcd"     super(e1, ..., ek)"
   and the static type of each expression ei is Si, and the invocation
   is statically resolved to a constructor
\xcd"def this(x1: T1,..., xk: Tk): D{d1} where c"
   then it must be the case that 
\begin{xten}
x1: S1, ..., xi: Si |- xi: Ti  (for i in 1:k)
x1: S1, ..., xk: Sk |- c  
d1[a/self] && x1: S1 ... && xk: Sk |- d[a/self]      
\end{xten}
\noindent   where \xcd"a" is a constant that does not appear in 
\xcd"x1: S1 && ... && xk: Sk".
  
\end{staticrule}

\begin{staticrule}{Constructor return}
   The compiler checks that every constructor for \xcd"C" ensures that
   the properties \xcd"p1,..., pn" are initialized with values which satisfy
   \xcd"t(C)", and its own return type \xcd"c'" as follows.  In each constructor, the
   compiler checks that the static types \xcd"Ti" of the expressions \xcd"ei"
   assigned to pi are such that the following is true
\begin{xten}
p1: T1, ...., pn: Tn |- t(C) && c'     
\end{xten}
\end{staticrule}
(Note that for the assignment of ei to pi to be type-correct it must be the
    case that pi: Ti |- pi: Pi.) 


\begin{staticrule}{Constructor invocation}
The compiler must check that every invocation \xcd"C(e1,..., en)" to a
constructor is type correct: each argument \xcd"ei" must have a static type
that is a subtype of the declared type \xcd"Ti" for the \xcd"i"th argument of the
constructor, and the conjunction of static types of the argument must
entail the \xcd"WhereClause" in the parameter-list of the constructor.
\end{staticrule}

\section{Field definitions}

Not every instance of a class needs to have every field defined on the
class. In Java-like languages this is ensured by conditionally setting
fields to a default value, such as \xcd"null", in those instances where the
fields are not needed.  

Consider the class \xcd"List" used earlier.  Here all instances of \xcd"List"
returned by the second constructor do not need the fields \xcd"value" and
\xcd"tail"; their value is set to null.

\Xten{} permits a much cleaner solution that does not require default
values such as null to be stored in such fields. \Xten{} permits fields to
be {\em guarded}, that is defined only if a certain constraint on the
properties of the class, called the \xcd"guard" of that field, is true.

TODO: remove this clause

\begin{verbatim}
FieldDeclaration  ::= 
   FieldModifiersopt ThisClauseopt 
   Type VariableDeclarators ;
ThisClause       ::= this DepParameters
ThisClauseopt    ::= null | ThisClause
\end{verbatim}

It is illegal for code to access a guarded field through a reference
whose static type does not satisfy the associated guard, even
implicitly (i.e.{} through an implicit \xcd"this"). Rather the source
program should contain an explict cast, e.g.{} \xcd"me: C{c} = this as C{c}".

\begin{staticrule*}
Let \xcd"f" be a field defined in class
\xcd"C" with guard \xcd"c".  The compiler declares an error if
field \xcd"f" is accessed through a reference \xcd"o" whose static
type is not a subtype of \xcd"C{c}".
\end{staticrule*}

\begin{example}

We may now rewrite the List example:
\begin{xten}
public class List(n: int{n >=0}) {
  protected val head: Object where n>0;
  protected val tail: List(n-1) where n>0;
  public def this(o: Object, t: List): List(t.n+1) {
     property(t.n+1);
     head=o;
     tail=t;
  }
  public def this(): List(0) {
     property(0);
  }
  ...
}
\end{xten}

The fields \xcd"value" and \xcd"tail" do not exist for instances of the class
\xcd"List(0)".
\end{example}

It is a compile-time error for a class to have two fields of the same
name, even if their \xcd"ThisClauses" are different. A class \xcd"C" with a field
named \xcd"f" is said to {\em hide} a field in a superclass named \xcd"f".

\begin{staticrule*}
     A class may not declare two fields with the same name.
\end{staticrule*}

\subsection{Field hiding}

The definition of field hiding does not take \xcd"ThisClauses" in
account. Suppose a class \xcd"C" has a field

\begin{xten}
var f: Foo where c;
\end{xten}
\noindent and a subclass \xcd"D" of \xcd"C" has a field
\begin{xten}
var f: Fum where d;
\end{xten}

We will say that \xcd"D.f" hides \xcd"C.f", {\em regardless} of the
constraints \xcd"c" and \xcd"d". This is in keeping with \Java, and
permits a naive implementation which always allocates space for a
conditional field.

\begin{rationale}
It might seem attractive to require that \xcd"D.f"
hides \xcd"C.f" only if \xcd"d" entails \xcd"c". This would seem
to necessitate a rather complex implementation structure for classes,
requiring some kind of a heterogenous translation for deptypes of \xcd"C"
and \xcd"D". This bears further investigation.
\end{rationale}

\section{Method definitions}

\Xten{} permits guarded method definitions, similar to guarded
field definitions. Additionally, the parameter list for a method may
contain a WhereClause.

\begin{verbatim}
MethodHeader ::= 
  MethodModifiersopt def identifier TypeParametersopt (
  FormalParameterListopt ) ResultTypeopt WhereClauseopt Throwsopt
ResultType ::= : Type
\end{verbatim}

The guard (specified by \xcd"ThisClause") speciifes a constraint \xcd"c" on the
properties of the class \xcd"C" on which the method is being defined. The
method exists only for those instances of \xcd"C" which satisfy \xcd"c".  It is
illegal for code to invoke the method on objects whose static type is
not a subtype of \xcd"C{c}".

We relax the rules of lexical visibility and finality for variable
references in deptypes for method parameters.  Method
parameters not necessarily declared to be final are permitted to occur
in the types of parameters that occur after them in textual
order. Method parameters may also occur in the ReturnType for the
method, as long as they are declared final. (Even though the ReturnType
occurs lexically before the parameter list, it is considered to lie in
the scope of the declarations in the parameter list.)

\begin{staticrule*}
    The compiler checks that every method invocation \xcd"o.m(e1,..., en)"
    for a method is type correct. Each each argument ei must have a
    static type Si that is a subtype of the declared type Ti for the ith
    argument of the method, and the conjunction of static types
    of the arguments must entail the WhereClause in the parameter-list
    of the method.

    The compiler checks that in every method invocation \xcd"o.m(e1, ..., en)"
    the static type of o, S, is a subtype of \xcd"C{c}", where the method
    is defined in class C and the ThisClause for m is equivalent to
    \xcd"c".

    Finally, if the declared return type of the method is D{d}, the
    return type computed for the call is
    \xcd"D{a: S; x1: S1; ...; xn: Sn; d[a/this]}", where a is a new
    variable that does not occur in
    \xcd"d, S, S1, ..., Sn", and \xcd"x1,...,xn" are the formal
    parameters of the method.
\end{staticrule*}

\begin{example}
Consider the program:
\begin{xten}
public class List(n: int{n >=0}) {
  protected val head: Object where n>0;
  protected val tail: List(n-1) where n>0;
  public def this(o: Object, t: List): List(t.n+1) {
     property(t.n+1);
     head=o;
     tail=t;
  }
  public def this(): List(0) {
     property(0);
  }
  public def append(l: List): List{self.n==this.n+l.n} {
      return (n==0)? l
         : new List(head, tail.append(l)); 
  }
  public def nth(k: int{k >= 1 && k <= n}): Object
    where n > 0 {
      return k==1 ? head : tail.nth(k-1);
  }
}
\end{xten}

The following code fragment
\begin{xten}
List{self.n==3} u = ...
List{self.n==x} t = ...;
List{self.n==x+3} s = t.append(u);
\end{xten}
\noindent will typecheck. The type of the expression \xcd"t.append(u)" is 
\begin{xten}
List{a: List{self.n==x}; 
     l: List{self.n==3}; self.n==a.n+l.n}  
\end{xten}
\noindent and this simplifies to
\begin{xten}
List{a: List{self.n==x}; 
     l: List{self.n==3}; self.n==x+3}  
\end{xten}
\noindent which, after dropping unused local variables, reduces to:
\begin{xten}
List{self.n==x+3}
\end{xten}
\end{example}

\subsection{Method overloading, overriding, hiding, shadowing and obscuring}

The definitions of method overloading, overriding, hiding, shadowing
and obscuring in \Xten{} are the same as in \Java, modulo the following
considerations motivated by dependent types.

The definition of a method declaration \xcd"m1" ``having the same signature
as'' a method declaration \xcd"m2" involves identity of types. Two \Xten{} types
are defined to be identical iff they are equivalent (\S~\ref{DepType:Equivalence}).
Two methods are said to have {\em the same signature} if (a)
they have the same number of formal parameters, (b) for each parameter
their types are equivalent, and (c) the constraints associated with
their ThisTypes are equivalent. It is a compile-time error for there
to be two methods with the same name and same signature in a class
(either defined in that class or in a superclass).

\begin{staticrule*}
  A class \xcd"C" may not have two declarations for a method named \xcd"m"---either
  defined at \xcd"C" or inherited:
\begin{xten}
def m(v1: T1{t1},..., vn: Tn{tn}): T where tc {...}
def m(v1: S1{s1},..., vn: Sn{sn}): S where sc {...}
\end{xten}
\noindent
if it is the case that the types \xcd"C{tc}", \xcd"T1{t1}",
\dots, \xcd"Tn{tn}" are
equivalent to the types \xcd"C{sc}, S1{t1}, ...., Tn{tn}"
respectively.
\end{staticrule*}

A class \xcd"C" inherits from its direct superclass and superinterfaces all
their methods visible according to the access restriction modifiers
public/private/protected/(package) of the superclass/superinterfaces
that are not hidden or overridden. A method \xcd"M1" in a class \xcd"C" overrides
a method \xcd"M2" in a superclass \xcd"D" if \xcd"M1" and \xcd"M2" have the same signature.
Methods are overriden on a signature-by-signature basis.

A method invocation \xcd"o.m(e1,..., en)" is said to have the {\em static
signature} \xcd"<T, T1,...,Tn>" where \xcd"T" is the static type of \xcd"o", and
\xcd"T1,..., Tn" are the static types of \xcd"e1,..., en" respectively.  As in
\Java, it must be the case that the compiler can determine a single
method defined on \xcd"T" with argument type \xcd"T1,..., Tn", otherwise a
compile-time error is declared. However, unlike \Java, the \Xten{} type \xcd"T"
may be a dependent type \xcd"C{c}". Therefore, given a class definition for
\xcd"C" we must determine which methods of \xcd"C" are available at a type
\xcd"C{c}". But the answer to this question is clear: exactly those methods
defined on \xcd"C" are available at the type \xcd"C{c}"
whose guard \xcd"d" is implied by \xcd"c".


\begin{example}
  Consider the definitions:
\begin{xten}
class Point(i: int, j: int) {...}
class Line(s: Point, e: Point{self != i}) {
  // m1: Both points lie in the right half of the plane
  def draw() where (s.i>= 0 && e.i >= 0) {...}
  // m2: Both points lie on the y-axis
  def draw() where (s.i== 0 && e.i == 0) {...}
  // m3: Both points lie in the top half of the plane
  def draw() where (s.j>= 0 && e.j >= 0) {...}
  // m4: The general method
  def draw() {...}
} 
\end{xten} 
\noindent  Three different implementations are given for the draw method, one
  for the case in which the line lies in the right half of the plane,
  one for the case that the line lies on the y-axis and the third for
  the case that the line lies in the top half of the plane.


\noindent  Consider the invocation
\begin{xten}
m: Line{s.i < 0} = ...
m.draw();
\end{xten}

\noindent  This generates a compile time error because there is no applicable
  method definition.

\noindent  Consider the invocation

\begin{xten}
m: Line{s.i>=0 && s.j>=0 && e.i>=0 && e.j>=0} = ...
m.draw();
\end{xten}

\noindent  This generates a compile time error because both
\xcd"m1" and \xcd"m3" are applicable.

\noindent  Consider the invocation
\begin{xten}
m: Line{s.i>=0 && s.j>=0 && e.i>=0} = ...
m.draw();
\end{xten}
  This does not generate any compile-time error since only m1 is
  applicable. 
\end{example}


In the last example, notice that at runtime \xcd"m1" will be invoked
(assuming \xcd"m" contains an instance of the \xcd"Line" class, and not some
subclass of \xcd"Line" that overrides this method). This will be the case
even if \xcd"m" satisfies at runtime the stronger conditions for \xcd"m2" (i.e.,
\xcd"s.i==0 && e.i==0"). That is, dynamic method lookup will not take into
account the  ``strongest'' constraint that the receiver may satisfy, i.e.{}
its ``strongest deptype''. 

\begin{rationale}
  The design decision that dynamic method lookup should ignore
  dependent type information was made to keep the design and the
  implementation simple and to ensure that serious errors such as
  method invocation errors are captured at compile-time.
 
  Consider the above example and the invocation
\begin{xten}
m: Line = ...
m.draw();    
\end{xten}


   Statically the compiler will not report an error because m4 is the
   only method that is applicable. However, if dynamic method lookup
   were to use deptypes then we would face the problem that if m is a
   line that lives in the upper right quadrant then both \xcd"m2"
   and \xcd"m3"
   are applicable and one does not override the other. Hence we must
   report an error dynamically.

   As discussed above, the programmer can write code with \xcd"instanceof"
   and classcasts that perform any application-appropriate
   discrimination.  
\end{rationale}

\section{Interfaces with properties}\label{DepType:Interface}

\Xten{} permits interfaces to have properties and specify an interface
invariant. This is necessary so that programmers can build dependent
types on top of interfaces and not just classes.

\begin{verbatim}
NormalInterfaceDeclaration ::= 
     InterfaceModifiersopt interface identifier 
        PropertyListopt ExtendsInterfacesopt InterfaceBody
PropertyList     ::= ( Properties WhereClauseopt )
Properties       ::= Property
                    | Properties , Property
Property         ::= Type identifier    
PropertyListopt  ::= null | PropertyList
\end{verbatim}
The invariant associated with an interface is the conjunction of the
invariants associated with its superinterfaces and the invariant
defined at the interface. 

\begin{staticrule*}
   The compiler declares an error if this constraint
   is not consistent (\S~\ref{DepType:Consistency}).  
\end{staticrule*}

Each interface implicitly defines a nullary getter method \xcd"T p()" for
each property \xcd"T p". 

\begin{staticrule*}
   The compiler issues a warning if the programmer
   explicitly defines a method with this signature for an interface.
  
\end{staticrule*}

A class \xcd"C" (with properties) is said to implement an interface \xcd"I" if
\begin{itemize}
  \item its properties contains all the properties of I,
\item its class invariant, i(C), implies i(I)
\end{itemize}


\subsection{instanceof} 

\Xten{} permits Types to be used in an in \xcd"instanceof" expression to
determine whether an object is an instance of the given type:

\begin{verbatim}
RelationalExpression ::= 
  RelationalExpression instanceof Type  
\end{verbatim}

In the above expression, \xcd"Type" is any type including deptypes and
``primitive'' types. The expression \xcd"e instanceof T" evaluates to true
if and only if the evaluation of \xcd"e" results in a value \xcd"v" which belongs
to the type \xcd"T". This determination may involve checking that the
constraint associated with the type is true for the value \xcd"v".

\subsection{Class casts}

\Xten{} permits types to be used in a cast expression:

\begin{verbatim}
CastExpression ::= 
  ( Type ) UnaryExpressionNotPlusMinus  
\end{verbatim}

In the above expression, Type is any type including deptypes and
``primitive'' types. The expression \xcd"((T) e)" evaluates \xcd"e" to
a value \xcd"v", and results in \xcd"v" if \xcd"v" is an instance of
the type \xcd"T". Otherwise a \xcd"ClassCastException" is thrown. The
static type of the expression \xcd"(T) e" is \xcd"T".

   \begin{staticrule*}
    The compiler checks that the static type of e is either a subtype
    of T (this situation is sometimes called a "stupid cast"), or a
    supertype of T. If neither is the case, it throws a compile-time
    error since the cast must necessarily fail at runtime.     
   \end{staticrule*}

\subsection{Local variables}

Dependent types may be used to specify the types of local variables,
including loop variables and parameters for catch clauses.

\section{Array types}

\begin{verbatim}
ArrayType ::= Array [ Type ]
            | array [ Type ]
\end{verbatim}

\Xten{} has the following built in types:

\begin{xten}
value class point(rank: int{rank>=0}) {...}
\end{xten}
\begin{xten}
value class region(rank: int{rank>=0},
   // region is a product of rank-1 convex regions.
   rect: boolean,  
   rail: boolean,
   // on each dim, the low bound is 0. 
   lowZero: boolean
  ) { ...}
\end{xten}
\begin{xten}
value class dist(rank: int{rank>=0},
   rect: boolean,
   rail: boolean,
   lowZero: boolean,
   region: region{self.rank==rank
               && self.rect==rect
               && self.lowZero==lowZero},
   onePlace: place 
   )  { ...}
\end{xten}
\begin{xten}
class Array[T](rank: int{rank>=0},
   rect: boolean,
   rail: boolean,
   lowZero: boolean,
   onePlace: place,
   region: region,
   dist: dist{self.rank==rank
           && self.rect=rect
           && self.lowZero==lowZero
           && self.region==region
           && self.onePlace==onePlace}
      ) {...}
\end{xten}
\begin{xten}
value class array[T](rank: int{rank>=0},
   rect: boolean,
   rail: boolean,
   lowZero: boolean,
   onePlace: place,
   region: region,
   dist: dist{self.rank==rank
           && self.rect=rect
           && self.lowZero==lowZero
           && self.region==region
           && self.onePlace==onePlace}
      ) {...}
\end{xten}

For \xcd"R" is a reference type, the type \xcd"array[T](R{c}!current)" is interpreted as the
type of all arrays which are such that the value at a point \xcd"p" in its
region has the type \xcd"R{c}!self.dist(p)". (Recall that in \Xten{} a
distribution is itself a value array that maps a point to a place.)

\begin{example}
  The type \xcd"Array[double]{rail}" is the Java type \xcd"double[]".
  The type \xcd"Array[Array[double]{rail}]{rail}" is the Java type \xcd"double[][]".

  The type \xcd"Array[double]{rank==N}" is the type of all N-dimensional arrays of
  \xcd"double"s.

  The type \xcd"Array[double]{rank==N&&onePlace==here}" is the
  type of all \xcd"N"-dimensional
  arrays of \xcd"double"s that are local (mapped to one place).  
\end{example}

\begin{verbatim}
FUTURE: array[T] implements point => T
\end{verbatim}

\section{Constraint system}

The initial release of \Xten{} has a very simple constraint system,
permitting only conjunctions of equalities between variables and
constants, and existential quantification over typed variables.

Subsquent implementations are intended to support boolean algebra,
arithmetic, relational algebra etc to permit types over regions and
distributions. We envision this as a major step towards removing most
(if not all) dynamic array bounds and places check from \Xten{}.
  
%%\subsection{Normalization of constraints}
%%
%%The constraint system satisfies the following structural rule
%%
%%  Gamma |- c
%%---------------------------------
%%Gamma, S(: c') x |- S(:c' && c) x
%%
%%An additional rule, SELF, relates self to the variable being defined:
%%
%%   T(: c) x |- c[x/self]
%%
%%The rule TI ensures that the type invariant is implicitly available:
%%
%%   T(: c) x |- T (: i(T)[x/this] && c) x
%%
%%Examples
%%
%%Consider the definitions
%%  class region(int rank) { ...}
%%  class intArray(region r, int(:self==r.rank) rank) { ...}
%%
%%Now we have the following invariants:
%%
%%   i(region) =def=  int this.rank
%%   i(intArray) =def= region this.r && int(:self==this.r.rank) this.rank
%%
%%Note that i(intArray) is equivalent to
%%   
%%   region this.r && int(:self==this.r.rank) this.rank && this.rank == this.r.rank
%%
%%Now we have the following derivations
%%
%%  intArray(:self.rank==2) a |- region(:self.rank==2) a.r
%%
%%through the following derivation. 
%%
%%  intArray(:self.rank==2) a |- intArray(:self.rank==2 && i(intArray)) a  (TI)
%%
%%Now |intArray(:self.rank==2 && i(intArray)) a|  is equivalent to
%%
%%  intArray(:self.rank==2 && region a.r && int(:self==a.r.rank) a.rank 
%%            && a.rank == a.r.rank) a 
%%
%%and entails each one of 
%%
%%    a.rank==2
%%    a.rank == a.r.rank
%%
%%which together entail a.r.rank==2.

%\subsection{Syntactic abbreviations}\label{DepType:SyntaxAbbrev}

\section{Place types}\label{DepType:PlaceType}\index{placetype}

\begin{verbatim}
PlaceTypeSpecifier ::= ! PlaceType
PlaceType ::=  any | current | Expression  
\end{verbatim}

All \Xten{} reference classes extend the class \xcd"x10.lang.Ref":

\begin{xten}
package x10.lang;
public class Ref(location: place) { ... }  
\end{xten}

Because of the importance of places in the \Xten{} design, special
syntactic support is provided for deptypes involving places.

We now consider the expansions of deptypes with place information.

Unless a deptype \xcd"T" (whose base is a reference type) has an \xcd"!" suffix,
the constraint for \xcd"T" is implicitly assumed to contain the clause
\xcd"self.location==here".
\begin{xten}
C{c} => C{self.location==here && c}
\end{xten}
\noindent The expansions for a deptype with an ``\xcd"!"'' suffix are:
\begin{xten}
C{c}!  => C{c}  // no self.location clause.
C{c}!p => C{self.location==p && c} 
\end{xten}


\begin{staticrule*}
  It is a compile time error for the \xcd"!"-annotation to
  be used for deptypes whose base type does not extend \xcd"x10.lang.Ref".  
\end{staticrule*}

\section{Implicit Syntax}\label{ImplicitSyntax}\index{implicit syntax}

Recall that the explicit syntax for \Xten{} requires the programmer to use
asyncs/future to ensure the Locality Principle: An activity accesses
only those mutable locations that reside in the same place as the
activity. 

Explicit syntax has the advantage that the performance model for \Xten{}
is explicit from the syntax. It has the disadvantage that the
programmer has to manually reason about the placement of various
objects. If the programmer reasons incorrectly then computation may
abort at runtime with an exception.

The place-based type system enables the compiler to support the
Locality principle. The programmer uses the type system to establish
that the types of various objects are local. These assertions are
checked by the compiler (as a side-effect of checking dependent
types). Additionally, the programmer may now use normal variable
syntax to access (read/write) variables, and invoke methods on
objects. Suppose the type of the variable \xcd"v" is \xcd"C{c}".
If \xcd"c" establishes
\xcd"location==here" then the compiler generates code for performing the relevant
operation on the local variable (read, write, method invocation)
synchronously. 

Otherwise the compiler generates code in explicit syntax as
follows. If the operation is a read, the compiler generates code to
perform a future/force on the variable

\begin{xten}
future(v) { v }.force();  
\end{xten}

If the operation is a write \xcd"v=e", the compiler generates code to perform
the write synchronously:

\begin{xten}
val temp: T = e;
finish async (v) { v = w; }
\end{xten}

If the operation is a read on an array variable a[p] the compiler
generates the code:

\begin{xten}
future(a.dist(p)) { a(p) }.force();  
\end{xten}


If the operation is a write \xcd"a[p]=e", the compiler generates code to perform
the write synchronously:

\begin{xten}
val tp: point = p;
val t: T = e;
finish async (a.dist(tp)) { a(tp) = t; }
\end{xten}

If the operation is a method invocation \xcdmath"e.m(e$_1$,..., e$_n$)" for a void
method, the compiler generates code to perform the method invocation
synchronously:

\begin{xten}
val t: T! = e;
val t1: P1! = e1;
...
val tn: Pn! = en;
finish async (t) {
  t.m(t1,..., tn);
}
\end{xten}


If the operation is a method invocation
\xcdmath"e.m(e$_1$,...,e$_n$)" for a method that returns a value
of type E, then the compiler generates the following code:

XXX remove this -- do not define semantics by translation

\begin{xten}
new Runnable() {
  public E run() {
    val t: T! = e;
    val t1: P1! = e1;
    ...
    val tn: Pn! = en;
    return future(t){ t.m(t1,...tn) }.force();
  }}.run()
\end{xten}

