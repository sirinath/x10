\section{Conversions and promotions}\label{XtenConversions}\label{XtenPromotions}\index{conversions}\index{promotions}

XXX this is out of date.  Olivier...

\XtenCurrVer{} supports the following conversions and
promotions:
\begin{itemize}
\item A subtype may be implicitly coerced to any supertype.
\item A numeric type may be implicitly converted to any wider
numeric type if there is no loss of precision.
\item String conversion.  Any object that is an operand of the binary
\xcd"+" operator may
be converted to \xcd"String" if the other operand is a \xcd"String".
\item Narrowing coercion.  A compile-time constant of type \xcd"int" may be
implicitly coerced to a \xcd"byte", \xcd"short", or \xcd"char"
if the compiler can show that the value can be represented at
the target type.
\item Narrowing coercion.  A compile-time constant of type
\xcd"long" may be
implicitly coerced to a \xcd"byte", \xcd"short", \xcd"char",
or \xcd"int"
if the compiler can show that the value can be represented at
the target type.
\end{itemize}

XXX Java
XXX Coercions

{}\XtenCurrVer{} supports \java's conversions and promotions
(identity, widening, narrowing, value set, assignment, method
invocation, string, casting conversions and numeric promotions)
appropriately modified to support \Xten's built-in numeric classes
rather than \java's primitive numeric types.

This decision may be revisited in a future version of the language in
favor of a streamlined proposal for allowing user-defined
specification of conversions and promotions for value types, as part
of the syntax for user-defined operators.


