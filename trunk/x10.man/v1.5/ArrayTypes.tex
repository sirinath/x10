\subsection{Array types}
\label{ArrayTypeConstructors}\index{array types}

{}\XtenCurrVer{} does not have array class declarations
(\S~\ref{XtenArrays}). This means that user cannot define new array
class types. Instead arrays are created as instances of array types
constructed through the application of {\em array type constructors}
(\S~\ref{XtenArrays}).

XXX 
Arrays are defined via the \xcd"Array" and \xcd"ValArray"
classes in the standard library.

The array type constructor takes as argument a type (the {\em base
type}), an optional distribution (\S~\ref{XtenDistributions}), and
optionally the keyword \xcd"value":
\begin{grammar}
  ArrayType \: \xcd"Array" \xcd"[" Type \xcd"]" ( \xcd"(" Expression \xcd")" )\opt \\
     \| \xcd"ValArray" \xcd"[" Type \xcd"]" ( \xcd"(" Expression \xcd")" )\opt
\end{grammar}

The array type \xcd"Array[T]" is the type of all
reference arrays of base type \xcd"T". Such an array can take on any
distribution, over any region. 

The array type \xcd"ValArray[T]" specifies the type of all
values arrays of base type \xcd"T".
The array elements of a \xcd"ValArray" are
all \xcd"final".\footnote{Note that the base type of a
\xcd"ValArray" can be a value class or a reference class, just as the 
type of a \xcd"final" variable can be a value class or a reference class.}

\XtenCurrVer{} also allows a distribution to be specified 
as a property initializer on the array type.
The distribution must be an expression of type
\xcd"Dist" whose
value does not depend on the value of any mutable variable.

\Xten{} also supports dependent types for arrays,
e.g.,
\xcd"Array[Double]{rank==3}" is the type of all arrays of 
\xcd"Double" of rank \xcd"3".

The array type
\xcd"Array[T](n)" and its synonym
\xcd"Array[T](Region(0..n-1)->here)" corresponds to a local
zero-indexed array, such as in Java.
% \xcd"array[T]{rank==1,dist==0:self.length-1->here,region==0:self.length-1"}
% corresponds to a Java array.
