\documentclass{article}
\def\cS{\mbox{\tt clocked-Statement}}
\def\stmt{\mbox{\tt Statement}}
\def\next{\mbox{\tt next}}
\def\resume{\mbox{\tt resume}}
\def\drop{\mbox{\tt drop}}
\begin{document}
\section {Introduction}
X10 provides a notion of clocks that can be used as a synchronization
barrier.
Each activity in X10, which is a generalization of classical threads, can
get registered to multiple clocks.
An activity can {\em register} extra clocks during its execution, as well
as {\em drop } already registered clocks.
Registering of a new set of clocks for a statement is done by
the structured operation {\tt clocked-Statement}:

        $\cS$ ::= clocked \{clockSet\} $\stmt$

If the current activity is already registered to the set of activities
$S$, then the above construct registers {\tt clockSet} $\cup$ $S$ 
set of clocks to $\stmt$.
During any point of execution, an activity can drop an already registered
clock by invoking the the {\tt drop} method on the explicit clock.
At the end of any {\tt clocked} statement, all the set of clocks that were
additionaly registered at the beginning of the statement are dropped.
An activity may invoke the {\tt drop} operation, only on the registered
set of clocks.

{\bf Synchronization:} X10 provides two operation {\tt next} and {\tt
resume} for synchronization purposes.
Operation $\next$ is a blocking operation, when invoked by an activity
$a$, that waits for synchronization across all activities that are
registered on the same set of clocks on which $a$ is registered.
Any activity can invoke the $\resume$ operation on any clock, that it is
registered on. 
A clock is considered ready to advance to the next phase, when every
activity registered on this clock has invoked a $\resume$ operation on
that clock.

\section{Code generation}
In the general case, we would need maintain program specific state in the
runtime clock objects.
However, there is a fast path: If we can guarrantee that all the
activities that might be running in parallel are registered to the same
set of clocks, then the following simplifications can be affected.
\begin{itemize}
\item The $\next$ operation can be replaced by a {\em global barrier}.
\item The $\cS$ can be replaced by a simple $\stmt$.
\item The $\resume$ operation can be replaced by nop. {\em This needs a
bit more thought. But it looks safe to me. -Krishna.}
\end{itemize}

\section{Fast path Analysis}

{\bf Find all activities running in parallel}:
An async node, inside another async node results in two parallel executing
activities. 
Note: If we have a 

\noindent
loop:\\
async clocked (clk) S\\
end-loop\\

then we can treat async S as one single async for the purpose of our
analysis, if clk is loop invariant.

{\bf Find the set of registered clocks}: For each clocked statement
clocked on set of clocks $C$,
propage the clocks down to all the sub-statements of the statement. 
If we see a c.$\drop$ statement, set the current set of clocks to
$C-\{c\}$. {\em In our case, we will not see any $\drop$ statement.}
At any merge point (including function calls), only if the two incoming
sets of clocks ($C_1$ and $C_2$) match , we will set the registered clocks
for the function to $C = C_1 = C_2$.
        
\end{document}
