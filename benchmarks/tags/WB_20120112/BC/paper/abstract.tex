\begin{abstract}
Betweenness centrality (BC) is an important metric that can be used to
determine the power of a vertex in a graph~\cite{Freeman77,Anthonisse71}.
%
Computing BC for an unweighted, directed graph $G(n,m)$, where $n$ is the
number of vertices and $m$ is the number of edges, is both computationally and
spatially demanding; the best known serial algorithm requires $O(nm)$ time and
$O(n+m)$ space~\cite{brandes01:_mathsoc}.
%
Furthermore, given that the sizes of graphs that need analysis is
ever-increasing, parallelization of the BC computation is a must.
%
To compute BC for massive-scale graphs on modern super computers, which by
nature offer hierarchical parallelism (i.e., support shared- and
distributed-memory parallelism simultaneously), it is important to develop a
parallel algorithm that is also hierarchical for optimal performance.
%
There have been several attempts at parallelizing BC
computations~\cite{Madduri:2009,Santos:2006,edmonds-hipc-2010}; however, these
attempts are specialized for shared- or distributed-memory machines and have
also failed to show significant scaling.
%
Owing to the apparent infeasibility in computing \textit{exact} BC,
approximations have also been implemented~\cite{Bader07:ApproxBC,buluc-2010}.
%
\todoanju{In this paper, we summarize the state-of-the-art algorithms for
computing BC (both serial and parallel).
%
Furthermore, we develop a new hierarchical algorithm for computing BC, which 
improves on existing solutions.}
\end{abstract}
