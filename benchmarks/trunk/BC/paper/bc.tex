\documentclass[conference]{IEEEtran}

\usepackage[ruled,vlined,linesnumbered]{algorithm2e}
\usepackage[cmex10]{amsmath}
\usepackage{epsfig}
\usepackage{xypic}
\usepackage{comment}
\usepackage{listings}
\usepackage{color}
\usepackage{url}
\usepackage[ruled,vlined]{algorithm2e}

% To ensure that C++ symbol comes out right.
\newcommand{\Cpp}{C\kern-0.05em\texttt{+\kern-0.03em+}}
\newcommand{\code}[1]{\lstinline[basicstyle=\sffamily]{#1}}
\newcommand{\func}[1]{\lstinline[basicstyle=\sffamily]{#1()}}
\newcommand{\funcfont}[1]{\footnotesize\sffamily #1}
\newcommand{\algofont}[1]{\footnotesize\sffamily #1}
\newcommand{\commentfont}[1]{\footnotesize\bfseries #1}
\newcommand{\funczero}[1]{\lstinline[basicstyle=\sffamily]{#1()}}
\newcommand{\funcone}[2]{\lstinline[basicstyle=\sffamily]{#1(#2)}}
\newcommand{\functwo}[3]{\lstinline[basicstyle=\sffamily]{#1(#2,#3)}}
\newcommand{\functhree}[4]{\lstinline[basicstyle=\sffamily]{#1(#2,#3,#4)}}
\newcommand{\funcfour}[5]{\lstinline[basicstyle=\sffamily]{#1(#2,#3,#4,#5)}}

% Set how the algorithm comments are displayed
\SetCommentSty{commentfont}
\SetFuncSty{funcfont}

% correct bad hyphenation here
%\hyphenation{op-tical net-works semi-conduc-tor}

\lstdefinestyle{basic}{showstringspaces=false,
                       columns=fullflexible,
                       language=Java,
                       escapechar=@,
                       frame=single,
                       basicstyle=\footnotesize\sffamily,
moredelim=**[is][\color{white}]{~}{~},
morekeywords={concept,model,require,where,reduction},
literate={->}{{$\rightarrow\;$}}1 {<-}{{$\leftarrow\;$}}1 {=>}{{$\Rightarrow\;$}}1,
}

\newcommand{\todoanju}[1]{\textcolor{blue}{Anju: #1}}
\newcommand{\todovijay}[1]{\textcolor{cyan}{Vijay: #1}}

\IEEEoverridecommandlockouts

\begin{document}
%
% paper title
% can use linebreaks \\ within to get better formatting as desired
\title{Computing Betweenness Centrality Of Massive-Scale Unweighted, Directed
Graphs}

\author{\IEEEauthorblockN{Prabhanjan Kambadur}
        \IEEEauthorblockA{I.B.M. T.J. Watson Research Center \\
                          Yorktown Heights, NY 10592, \\
                          pkambadu@us.ibm.com}}
\date{}

% make the title area
\maketitle

Modern object-oriented languages such as \Xten require a rich framework
for types capable of expressing value-dependency, type-dependency
and supporting pluggable, application-specific extensions.

In earlier work, we presented the framework of \emph{constrained
types} for concurrent, object-oriented languages, parametrized by
an underlying constraint system $\cal X$. Constraint systems are a
very expressive framework for partial information. Types are viewed
as formulas \Xcd{C\{c\}} where \Xcd{C} is the name of a class
or an interface and \Xcd{c} is a constraint in $\cal X$ on the
immutable instance state of \Xcd{C} (the \emph{properties}).
Many (value-)dependent type systems for object-oriented languages
can be viewed as constrained types.

This paper extends the constrained types approach to handle
\emph{type-dependency} (``genericity''). The key idea is to extend
the constraint system to include predicates over types such as
\Xcd{X} is a subtype of \Xcd{T}.  Generic types are supported
by introducing type parameters and permitting programs to impose
constraints on such parameters.

To illustrate the underlying theory, we develop a formal family of
programming languages with a common set of sound type-checking rules
parameterized on a constraint system $\cal X$.  By varying $\cal X$
and extending the type system, we obtain languages with the power
of \FJ, \FGJ, and languages that provide dependent types, structural
subtyping, and constraints that relate values and types.  The core
of the \Xten language is a concrete instantiation of the framework.  We
describe the design and implementation of \Xten, which is available
for download at \texttt{x10-lang.org}.




\section{Introduction}
\label{s:intr}

 Graph theoretic problems arise in several traditional and emerging scientific disciplines such as VLSI design, optimization, databases, and computational biology. There are plenty of theoretically fast parallel algorithms, for example, work-time optimal PRAM algorithms, for graph problems; however, in
 practice few parallel implementations beat the best sequential implementations for arbitrary, sparse
 graphs. The mismatch between theory and practice suggests a large gap between algorithmic model and the actual architecture. We observe that the gap is increasing as new diversified architectures emerge. Elegant solutions seem hard to come by from even combined efforts of algorithmic and architectural improvement. What is lacking is an effient way of mapping fine-grained parallelism expressed by the algorithm to target architectures with good performance. X10 is a new parallel programming language that provides expressive programming constructs and efficient runtime support that effectively helps reduce the gap between theory and practice in solving graph problems. In this paper we show that with X10 the fine-grained parallelism for a graph problem can be expressed much easier at a high algorithmic level, and the X10 program, compared with native C implementation, is much simpler and more elegant, and achieves comparable, and sometimes, even better performance. 

 The challenges of solving large-scale graph problems on current and emerging systems come from the irregular and combinatorial nature of the problem. Many of the important real world graphs, for example, internet topology, social interaction network, transfortation network, protein-protein interaction network, and etc., exhibit a ``small-world'' nature, and can be modeled as the so-called ``scale-free'' graph. There is no known efficient technique to partion such graph, which makes it hard to solve on distributed-memory systems. Also compared with the well-known sequential algorithms, for example, depth-first search (DFS) or breadth-first search (BFS) for the spanning tree problem, the parallel graph algorithms take exotic approaches such as ``graft-and-shortcut''. In the absence of efficient scheduling support of parallel activities, fine-grained parallelism incurs large overhead on current systems and oftentimes do not show practical parallel performance advantage. Lastly, graph algorithms tend to be load/store intensive compared with other scientific problems. For example,  They put great pressure on the memory subsystem. The problem obviously gets worse on distributed-memory architectures if necessary task management and memory affinity scheduling are not provided.  
 
 There are several features of X10 that make it extremely helpful in soving large-scale graph problems. X10 provides a shared virtual address space that obviates the need to partition a graph and issue message passing commands explicitly to access remote data. The irregular nature of the graph is also the reason why no SSCA benchmark has been implemented in MPI. X10 provides a wide range of constructs that are de. X10 has a lot of balancing.

 Our target architecure is a cluster of symmetric multiprocessor(SMP) nodes. Each SMP node may further comprise of chip multiprocessors (CMPs).  SMPs and CMPs are becoming very powerful and common place. Most of the high performance
computers are clusters of SMPs and/or CMPs. It is important to solve for them.
It is important to show flexibility but also good support of  PRAM algorithms for graph problems can be emulated much easier and more. 

The problem we consider if the spanning tree problem. It is notoriously hard to achieve good parallel performance.  Several good ones, we show X10 support that can do better. 

 The rest of the paper is organized as follows. Sections~\ref{s:design} describes algorithm design with the X10 language.
 Section~\ref{s:runtime} presents the workstealing runtime support for load-balancing in X10, and compare with other runtime systems, for example, CILK. 
 Section~\ref{s:results} provides our experimental results on current main-stream SMPs.
 In Section~\ref{s:concl} we conclude and give future work. 
 Throughout the paper, we
 use $n$ and $m$ to denote the number of vertices and the number of
 edges of an input graph $G=(V,E)$, respectively. 
  




\section{Background}
\label{sec:background}
% Define BC
Betweenness Centrality (BC) is a graph theoretic metric of a node's importance
in a graph.~\footnote{In this paper, we focus exclusively on the unweighted,
directed graphs.}
%
Informally, a node's BC gives an indication of the ratio of the total number 
of shortest paths between all other nodes taken pair-wise.
%
That is, a node with a high centrality score can reach other nodes with 
relatively fewer hops than another node with a lower centrality score.
%
More formally, let $G(V,E)$ be a graph with the vertex set $V$ and edge set
$E$; let the number of vertices ($\lvert{}V\rvert{})$ be $n$ and the number of 
edges ($\lvert{}E\rvert{}$) be $m$.
%
The BC of a node (vertex) $v\in{V})$ is given by the equation:
%
\begin{equation}
BC_{v} = \sum_{s\ne{}v\ne{}t\in{}V}{\frac{\sigma{}_{st}(v)}{\sigma{}_{st}}}
\label{eq:bc}
\end{equation}
%
Where $\sigma{}_{st}$ is the number of shortest paths from node $s$ to node $t$ 
and $\sigma{}_{st}(v)$ is the number of those shortest paths that go through 
node $v$.
% Get into the simple means of computing BC
There are multiple ways of computing the BC scores of nodes in a graph; in the
following subsections, we highlight a few of these techniques.
%
Please note that regardless of the approach that is used in computing BC, the 
central kernel is enumerating all the shortest paths.

\subsection{Algebraic Approach}
\label{subsec:algebraic}

%%%%%%%%%%%%%%%%%%%%%%%%%%%%%%%%%%%%%%%%%%%%%%%%%%%%%%%%%%%%%%%%%%%%%%%%%%%%%
\begin{figure}
\begin{minipage}{0.20\textwidth}
\begin{center}
\begin{displaymath}
\xymatrix{
a \ar[r] & d \ar[d] \\
b \ar[u] \ar[ur] \ar[r] & c \ar[ul]
}
\end{displaymath}
\end{center}
\end{minipage}
\hspace{10pt}
\begin{minipage}{0.18\textwidth}
\begin{center}
\begin{displaymath}
A = \left[ \begin{array}{cccc}
  0 & 0 & 0 & 1 \\
  1 & 0 & 1 & 1 \\
  1 & 0 & 0 & 0 \\
  0 & 0 & 1 & 0 \\
\end{array} \right]
\end{displaymath}
\end{center}
\end{minipage}
\caption{A sample directed, unweighted graph and its resulting adjacency
matrix. A $1$ in position $a_{ij}$ indicates an edge from node $i$ to node
$j$.}
\label{fig:sample}
\end{figure}
%%%%%%%%%%%%%%%%%%%%%%%%%%%%%%%%%%%%%%%%%%%%%%%%%%%%%%%%%%%%%%%%%%%%%%%%%%%%%

Every graph $G(V,E)$ can be represented as an adjacency matrix $A$, where an 
entry $a_{ij}$ is marked as $1$ \textit{iff} $(i,j)\in{}E$.
%
Figure~\ref{fig:sample} shows a sample four node graph and its related 
adjacency matrix; we will be using this graph as a running example throughout
this section.
%
By inspection, we can tell that the diameter of the graph is $3$; therefore, 
all the possible paths in the matrix (including the number of paths) can be 
enumerated by simply taking the powers of the adjacency matrix $A$; in other 
words, we find the transitive closure of the graph $G$.
%
However, computing the transitive closure is an over-kill; what we want in 
order to compute BC are just the shortest paths between all pair of vertices,
not all paths of length diameter or less.

%%%%%%%%%%%%%%%%%%%%%%%%%%%%%%%%%%%%%%%%%%%%%%%%%%%%%%%%%%%%%%%%%%%%%%%%%%%%%
\begin{figure}
\begin{minipage}{0.2\textwidth}
\begin{center}
\begin{displaymath}
A^2 = \left[ \begin{array}{cccc}
   0 & 0 & 1 & 0 \\
   1 & 0 & 1 & 1 \\
   0 & 0 & 0 & 1 \\
   1 & 0 & 0 & 0 \\
\end{array} \right]
\end{displaymath}
\end{center}
\end{minipage}
\hspace{10pt}
\begin{minipage}{0.2\textwidth}
\begin{center}
\begin{displaymath}
A^3 = \left[ \begin{array}{cccc}
   1 & 0 & 0 & 0 \\
   1 & 0 & 1 & 1 \\
   0 & 0 & 1 & 0 \\
   0 & 0 & 0 & 1 \\
\end{array} \right]
\end{displaymath}
\end{center}
\end{minipage}
\caption{The $2^{nd}$ and $3^{rd}$ powers of the adjacency matrix, $A$, shown
in Figure~\ref{fig:sample}. $A^2$ shows the paths of path length $2$, and $A^3$
shows the paths in the sample graph of path length $3$.}
\label{fig:sample_powers}
\end{figure}
%%%%%%%%%%%%%%%%%%%%%%%%%%%%%%%%%%%%%%%%%%%%%%%%%%%%%%%%%%%%%%%%%%%%%%%%%%%%%

%%%%%%%%%%%%%%%%%%%%%%%%%%%%%%%%%%%%%%%%%%%%%%%%%%%%%%%%%%%%%%%%%%%%%%%%%%%%%
% Floyd-Warshall's Algorithm
\begin{algorithm}
\SetKwFunction{edgeCost}{edgeCost}
\SetKwFunction{minimum}{min}

\caption{FloydWarshall}
\label{alg:floyd_warshall}
\KwIn{$A$: Adjacency matrix of graph $G(V,E)$}

\For{$i=1:n$}{
  \For{$j=1:n$}{
    $path_{ij}$ = \edgeCost{$i$,$j$};\
  }
}

\For{$k=1:n$}{
  \For{$i=1:n$}{
    \For{$j=1:n$}{
      $path_{ij}$ = \minimum{$path_{ij}$, $path_{ik}+path_{kj}$};\
    }
  }
}
\end{algorithm}
%%%%%%%%%%%%%%%%%%%%%%%%%%%%%%%%%%%%%%%%%%%%%%%%%%%%%%%%%%%%%%%%%%%%%%%%%%%%%

Another possible solution was proposed by Batagelj~\cite{Batagelj-1994} and 
involved modifying Floyd/Warshall's algorithm~\cite{floyd62,warshall62} for 
all pairs shortest paths.
%
Briefly, Floyd/Warshall's algorithm computes all pairs shortest paths given an
adjacency matrix representation of a weighted graph in $O(n^3)$ time; the
algorithm is outlined in
Algorithm~\ref{alg:floyd_warshall}.~\footnote{Floyd/Warshall's does not handle
negative cycles, although it can be used to detect such cycles in a graph.}
%
In his solution, Batagelj avoided unnecessary work by using \textit{geodetic
semiring}, an instance of the closed semiring generalization for shortest
paths~\cite{Aho-1974}.
%
We briefly sketch the solution here for the sample graph shown in
Figure~\ref{fig:sample}.
%
First, from the adjacency matrix $A$, we create a new relation matrix
$R=[(d_{u,v}, n_{u,v})]$, where $d$ is the geodesic between $(u,v)\in{}E$ and
$n_{u,v}$ is the number of geodesics between $u$ and $v$; initially
%
\begin{equation}
(d,n)_{u,v} = \left\{ \begin{array}{rcl} 
  (1,1) & \mbox{for} & (u,v)\in{} A \\
  (\infty{},0) & \mbox{for} & (u,v)\notin{} A \\
\end{array}\right.
\label{eq:dnuv}
\end{equation}

Using this transformation on the adjacency matrix shown in
Figure~\ref{fig:sample}, we get the following matrix:
%
\begin{displaymath}
R = \left[ \begin{array}{cccc}
  (\infty{},0) & (\infty{},0) & (\infty{},0) & (1,1) \\
  (1,1) & (\infty{},0) & (1,1) & (1,1) \\
  (1,1) & (\infty{},0) & (\infty{},0) & (\infty{},0) \\
  (\infty{},0) & (\infty{},0) & (1,1) & (\infty{},0) \\
\end{array} \right]
\end{displaymath}
%
From this matrix $R$, we compute the geodesic closure $R^+$ using the semiring
$(R, \oplus{}, \odot{}, (\infty,0), (0,1))$, where:

\begin{equation}
(a,i)\oplus{}(b,j) = (min (a,b), \left\{ \begin{array}{rr} 
  i & a<b \\
  i+j & a=b \\
  j & a>b \\
\end{array}\right.)
\label{eq:oplus}
\end{equation}

\begin{equation}
(a,i)\odot{}(b,j) = (a+b, i\times{}j)
\label{eq:odot}
\end{equation}
%
The key intuition here is that knowing the distance ($d_{u,v}$), and the number
of shortest paths ($n_{u,v}$, it is easy to compute the number of shortest
paths between $(u,v)$ using the following equation:
%
\begin{equation}
n_{u,v}(t) = \left\{ \begin{array}{rr} 
  n_{u,t}\times{}n_{t,v} & d_{u,t}+d_{t,v} \\
  0 & otherwise \\
\end{array}\right.
\label{eq:bcoft}
\end{equation}
%
For a complete proof of $(R, \oplus{}, \odot{}, (\infty,0), (0,1))$ being a 
geodesic semiring, please refer to Batagelj~\cite{Batagelj-1994}.~\footnote{
$(R, \oplus{}, \odot{}, (\infty,0), (0,1))$ is a semiring \textit{iff} all 
distances ($d_{i,j}$) are positive.}
%

%%%%%%%%%%%%%%%%%%%%%%%%%%%%%%%%%%%%%%%%%%%%%%%%%%%%%%%%%%%%%%%%%%%%%%%%%%%%%
% Modified Batagelj's Algorithm
\begin{algorithm}
\SetKwFunction{minimum}{min}

\caption{computeGeodeticSemiRing}
\label{alg:batagelj}
\KwIn{$R$: Relational matrix of graph $G(V,E)$}

\For{$k=1:n$}{
  \For{$i=1:n$}{
    \For{$j=1:n$}{
      $distance$ = \minimum{$\infty{},d_{i,k}+d_{k,j}$}\;
      \If{$d_{i,j}\ge{}distance$}{
        $count = n_{i,k}\times{}n_{k,j}$\;
        \If{$d_{i,j}==distance$}{
          $n_{i,j} = n_{i,j} + count$\;
        }\Else{
          $n_{i,j} = count$\;
          $d_{i,j} = distance$\;
        }
      }
    }
  }
}
\end{algorithm}
%%%%%%%%%%%%%%%%%%%%%%%%%%%%%%%%%%%%%%%%%%%%%%%%%%%%%%%%%%%%%%%%%%%%%%%%%%%%%
%
The modified Floyd-Warshall's algorithm that computes $R^+$ is given in
Algorithm~\ref{alg:batagelj}; after application of this algorithm to the 
matrix $R$, we get:
%
\begin{displaymath}
R^+ = \left[ \begin{array}{cccc}
  (3,1) & (\infty{},0) & (2,1) & (1,1) \\
  (1,1) & (\infty{},0) & (1,1) & (1,1) \\
  (1,1) & (\infty{},0) & (3,1) & (2,1) \\
  (2,1) & (\infty{},0) & (1,1) & (3,1) \\
\end{array} \right]
\end{displaymath}
%
From $R^+$, it is simple to compute the BC of any node using
equations~\ref{eq:bcoft} and~\ref{eq:bc}.
%
For example, BC($d$) in $G$ (from Figure~\ref{fig:sample}) is $1$ as it lies 
on the only shortest path between $c$ and $a$.
%
In this method, computing $R^+$ takes $O(n^3)$ operations, and then computing
BC of any node requires considering all pair-wise entries, which takes a
further $O(n^2)$ computations.
%
Therefore, the overall computational complexity of the algorithm is $O(n^3)$;
space complexity is $O(n^2)$.
%
This is too steep a price to pay in real-world graphs, which are sparse and 
large.

%
% Brandes' approach
%
\subsection{Graph Traversal Approach}
\label{subsec:graph_traversal}

%%%%%%%%%%%%%%%%%%%%%%%%%%%%%%%%%%%%%%%%%%%%%%%%%%%%%%%%%%%%%%%%%%%%%%%%%%%%%
% Brandes' Algorithm
\begin{algorithm}
\SetKwFunction{enqueue}{enqueue}
\SetKwFunction{dequeue}{dequeue}
\SetKwFunction{push}{push}
\SetKwFunction{pop}{pop}
\SetKwFunction{new}{new}
\SetKwFunction{append}{append}
\SetKwFunction{neighbor}{neighbor}

\caption{brandesBC}
\label{alg:brandes}
\KwIn{$G(V,E)$: A graph}
$BC_v = 0$, $v\in{}V$\;

\For{$s\in{}V$}{
  $S \leftarrow{}$ \new{$stack$}\;
  $P_w \leftarrow{} \emptyset{}, w\in{}V$\;
  $\sigma{}_w \leftarrow{} 0, w\in{}V; \sigma{}_s \leftarrow{} 1$\;
  $D_w \leftarrow{} -1, w\in{}V; D_s \leftarrow{} 0$\;
  $Q \leftarrow{}$ \new{$queue$}\;
  \enqueue{$Q$,$s$}\;

  \While{$Q\ne\emptyset{}$}{
    $v \leftarrow{}$ \dequeue{$Q$}\;
    \push{$S$,$v$}\;
    \ForEach{$w\leftarrow{}$\neighbor{$v$}}{
      \If{$D_w<0$}{
        \enqueue{$Q$,$w$}\;
        $D_w = D_v + 1$\;
      }
      \If{$D_w=D_v+1$}{
        \append{$P_w$,$v$}\;
        $\sigma{}_w = \sigma{}_w + \sigma{}_v$\;
      }
    }
  }
  $\delta{}_v \leftarrow{} 0, v\in{}V$\;
  \While{$S\ne\emptyset{}$}{
    $w \leftarrow{}$ \pop{$S$}\;
    \ForEach{$v\in{}P_w$}{
      $\delta{}_v\leftarrow{}\delta{}_v+\frac{\sigma{}_v}{\sigma{}_w}\times{}(1+\delta{}_w)$\;
    }
    \If{$w\ne{}s$}{
      $BC_w \leftarrow{} BC_w + \delta{}_w$\;
    }
  }
}
\end{algorithm}
%%%%%%%%%%%%%%%%%%%%%%%%%%%%%%%%%%%%%%%%%%%%%%%%%%%%%%%%%%%%%%%%%%%%%%%%%%%%%
%
The key to computing efficiently is to exploit the sparsity of the graph 
structure (as opposed to its adjacency matrix).
%
This was the central theme on which Brandes~\cite{brandes01:_mathsoc} algorithm
is based.
%
Brandes recognized the recursive nature of BC computations that were exploited
in equation~\ref{eq:bcoft}.
%
For unweighted graphs, the basic idea is to perform breadth-first searches
(BFS) from all nodes.
%
At each step, the closest set of vertices are added, and during this step, 
cumulative betweenness scores for all vertices are computed by using the 
predecessor relationship.
%
Formally, let us define the \textit{predecessors} of a vertex $v$ on shortest
paths from $s$ to be
%
\begin{equation}
P_v(s)=\{u\in{}V:{u,v}\in{}E,d_G(s,v)=d_G(s,u)+(u,v)\}
\end{equation}
%
Where $d_G(s,v)$ is the shortest path from $s$ to $v$.
%
Now, the number of shortest paths from $s$ to $v$ ($\sigma{}_{sv}$) is exactly
$1$ more than the sum of the number of shortest paths from $s$ to each vertex
$u\in{}P_v(s)$. 
%
\begin{equation}
\sigma{}_{sv} = 1 + \sum_{u\in{}P_v(s)}\sigma{}_{su}
\end{equation}
%
Therefore, in $O(m)$ time, we can compute all the shortest paths and number of
shortest paths from a vertex $s$ to every other vertex; that is, in $O(mn)$,
we can compute all pairs and number of shortest paths.
%
Furthermore, this solution only requires $O(m+n)$ space.
%
The only thing left to do is to determine the contributions to the BC of each
vertex from every other vertex.
%
This information is already embedded in $P_v(s)$; there are
$\lvert{}P_v(s)\rvert{}$ distinct predecessors in the shortest paths from $s$
to $v$, and the number of shortest paths that each predecessor $P_{u,s}(i)$
is on is $\sigma{}_{su}(i)$. 
%
Now, using this information, we can compute the contribution of $(s,v)$ to 
the BC scores of each of the predecessors in $P_v(s)$.
%
\begin{equation}
BC(u) = BC(u) + \frac{\sigma{}_{su}}{\sigma{}_{sv}}, u\in{}P_v(s)
\end{equation}
%
In actual computation, the total number of shortest paths that go from $s$ to 
any vertex $v\in{}V$ through vertex $t\ne{}s,v$ is accumulated for every vertex
$s,v,t$ and finally, the BC scores are computed.
%
This final algorithm is given in Algorithm~\ref{alg:brandes}.

%
Let us consider the execution of Brandes's algorithm~\ref{alg:brandes} on our
sample graph $G$ from Figure~\ref{fig:sample}; let the start vertex be $a$.

\begin{align*}
Initialize\ BCs\ (line\ 1)\\
BC_a\leftarrow{}BC_b\leftarrow{}BC_c\leftarrow{}BC_d\leftarrow{}0\\
Initialize\ for\ BFS\ from\ a\ (lines\ 3\ to\ 8) \\ 
Q \leftarrow{} a, S \leftarrow{} \emptyset{} \\
P_a\leftarrow{}P_b\leftarrow{}P_c\leftarrow{}P_d\leftarrow{}\emptyset{}\\
\sigma{}_b\leftarrow{}\sigma{}_c\leftarrow{}\sigma{}_d\leftarrow{}0, \sigma{}_a\leftarrow{}1\\
D_b\leftarrow{}D_c\leftarrow{}D_d\leftarrow{}-1, D_a\leftarrow{}0
\end{align*}

\begin{align*}
BFS\ from\ a\ (lines\ 9\ to\ 18)\\
Q \leftarrow{} \emptyset{}, S \leftarrow{} (c,d,a) \\
P_a\leftarrow{}P_b\leftarrow{}\emptyset{}, P_c\leftarrow{}d,P_d\leftarrow{}a\\
\sigma{}_b\leftarrow{}0,\sigma{}_c\leftarrow{}\sigma{}_d\leftarrow{}\sigma{}_a\leftarrow{}1\\
D_b\leftarrow{}-1,D_c\leftarrow{}2,D_d\leftarrow{}1,D_a\leftarrow{}0
\end{align*}

\begin{align*}
Compute\ BC\ contributions\ (lines\ 19\ to\ 25)\\
S\leftarrow{}\emptyset{}\\
\delta{}_a\leftarrow{}2,\delta{}_d\leftarrow{}1,\delta{}_b\leftarrow{}\delta{}_c\leftarrow{}0\\
BC_a\leftarrow{}BC_b\leftarrow{}BC_c\leftarrow{}0,BC_d\leftarrow{}1
\end{align*}
%
This change in the $BC_d$ denotes that $d$ is on \textit{all} shortest paths
from $a$ to $d$.
%
Similarly, when shortest paths from $c$ are computed, the BC's are changed 
to:
%
\begin{align*}
BC_b\leftarrow{}BC_c\leftarrow{}0,BC_a\leftarrow{}BC_d\leftarrow{}1
\end{align*}
%
This change in the $BC_a$ denotes that $a$ is on \textit{all} shortest paths
from $c$ to $d$.
%
Similarly, when shortest paths from $d$ are computed, the BC's are changed 
to:
%
\begin{align*}
BC_b\leftarrow{}0,BC_a\leftarrow{}BC_c\leftarrow{}BC_d\leftarrow{}1
\end{align*}
%
This change in the $BC_c$ denotes that $c$ is on \textit{all} shortest paths
from $d$ to $a$.
%
Finally, when shortest paths from $b$ are computed, BC's do not change as 
$b$ is directly connected to $a,c,$ and $d$.

%
% Discuss some parallelization issues
%
\subsubsection{Parallelization}
%
\todoanju{Add information about parallelization.}

%
% Vector formulation of the problem
%
\subsection{Hybrid Formulation}
\label{subsec:hybrid}
%%%%%%%%%%%%%%%%%%%%%%%%%%%%%%%%%%%%%%%%%%%%%%%%%%%%%%%%%%%%%%%%%%%%%%%%%%%%%
% Hybrid Algorithm
\begin{algorithm}
\SetKwFunction{minimum}{min}
\SetKwFunction{matrixy}{matrix}
\SetKwFunction{nnzExists}{nnzExists}
\SetKwFunction{matMult}{matMult}
\SetKwFunction{eltWiseAdd}{eltWiseAdd}
\SetKwFunction{eltWiseMult}{eltWiseMult}
\SetKwFunction{eltWiseAssign}{eltWiseAssign}
\SetKwFunction{eltWiseInvert}{eltWiseInvert}
\SetKwFunction{eltWiseNot}{not}
\SetKwFunction{extractSubMatrix}{extractSubMatrix}

\caption{hybridBC}
\label{alg:hybrid}
\KwIn{$A$: Adjacency matrix of an unweighted graph $G$}
\KwIn{$n$: Dimension of $A$ ($n\times{}n$)}
\KwIn{$nVerts$: Number of BFS' to perform at once}
\tcp{We assume that nVerts is divisible by $n$}
$nPasses = \frac{nVerts}{n}$\;
$BC\leftarrow{}$\matrixy{$1$,$n$,$0$}\;

\ForEach{$p\in{}(1:nPasses)$}{
  $BFS \leftarrow{} \emptyset{}$\;
  $batch = ((p-1)\times{}nVerts+1):$\minimum{$p\times{}nVerts,N$}\;
  $nsp\leftarrow{}$\matrixy{$nVerts$,$n$, $0$}\;
  \ForEach{$row\in{}(1:nVerts)$}{$nsp(row,batch(row))\leftarrow{}1$\;}
  $depth\leftarrow{}0$\;
  \eltWiseAssign{$fringe$,\extractSubMatrix{$A$,$batch$,$:$}}\;

  \tcp{BFS search for all vertices in current batch}
  \While{\nnzExists{$fringe$}}{
    $depth\leftarrow{}depth+1$\;
    $nsp\leftarrow{}$\eltWiseAdd{$nsp$,$fringe$}\;
    $BFS(depth)\leftarrow{}fringe$\;
    $fringe\leftarrow{}$\matMult{$fringe$,$A$}\;
    \tcp{Reset entries for already visited vertices}
    $fringe\leftarrow{}$\eltWiseMult{$fringe$,\eltWiseNot{$nsp$}}\;
  }

  \tcp{Pre-compute BC updates for all but source vertices}
  $BC_{updt}\leftarrow{}$\matrixy{$1$,$n$,$1$}\;
  $nsp^{inv}\leftarrow{}$\eltWiseInvert{$nsp$}\;

  \tcp{Compute BC updates for all but source vertices}
  \For{$d\in{}(depth,depth-1,...,2)$}{
    \tcp{Compute child weights}
    $weights_1\leftarrow{}$\eltWiseMult{$BFS(d)$,$nsp^{inv}$}\;
    $weights\leftarrow{}$\eltWiseMult{$weights_1$,$BC_{updt}$}\;
    \tcp{Apply child weights}
    $temp_1\leftarrow{}$\matMult{$A$,$weights^T$}$^T$\;
    \tcp{Sum them up over parents}
    $temp_2\leftarrow{}$\eltWiseMult{$BFS(d-1)$,$nsp$}\;
    \tcp{Apply weights based on parents values}
    $temp_3\leftarrow{}$\eltWiseMult{$temp_1$,$temp_2$}\;
    $BC_{updt}\leftarrow{}$\eltWiseAdd{$BC_{updt}$,$temp_3$}\;
  }

  \tcp{Update BC scores from each source vertex's BFS}
  \For{$row\in{}(1:nVerts)$}{
    $BC\leftarrow{}$\eltWiseAdd{$BC$,$BC_{updt}(row,:)$}\;
  }
}
\tcp{Subtract additional values added by precomputation}
$BC\leftarrow{}$\eltWiseAdd{$BC$,\matrixy{$1$,$n$,$-nPasses$}}\;
\end{algorithm}
%%%%%%%%%%%%%%%%%%%%%%%%%%%%%%%%%%%%%%%%%%%%%%%%%%%%%%%%%%%%%%%%%%%%%%%%%%%%%
%
Both from a computational and spatial standpoint, there are several
inefficiencies in Algorithm~\ref{alg:brandes}.
%
First, notice that the queue $Q$ and the stack $S$ can be combined into one
array structure that when accessed from one end acts as a queue, and from the
other end, acts as a stack.
%
Second, the distance array $D$ is redundant in a BFS computation as
\textit{all} the shortest paths to a node must be reached during the same
\textit{BFS fringe's} expansion.
%
Any paths that reach a node during a later fringe are \textit{not} shortest
paths; hence, we can replace $D$ with a bit-array with one bit per node to
denote that its shortest path was reached during a particular fringe
expansion.
%
Third, consider the set of predecessors $P_v$ of a particular vertex $v$; since
$G$ is unweighted, these predecessors must have themselves been discovered 
during the previous fringe expansion.
%
That is, to compute the set of predecessors for a particular vertex $v$, we can
look at all its incoming \textit{discovered} edges; we need not store $P$ 
explicitly.
%
These three optimizations decrease the amount of space needed to execute
Brandes' algorithm by a significant constant factor.
%
The final optimization tries to perform BFS fringe expansions in terms of
sparse matrix multiplication with the \textit{fringe} vector (BLAS-2 kernel);
this allows exploration of one full fringe in one operation rather than
looping over all the neighbors of a particular node (lines $12$ to $18$ in
Algorithm~\ref{alg:brandes}).
%
The initial fringe vector contains just one entry, which corresponds to the 
start/source vertex.
%
Furthermore, shortest paths from multiple sources can be determined together by
replacing the sparse matrix vector product operation with a sparse matrix
matrix operation (BLAS-3 kernel); this is a standard trick in linear algebra
called \textit{blocking}.
%
However, note that the space requirements increase linearly with the block 
size.
%
After performing all these optimizations, we end up with
algorithm~\ref{alg:hybrid}, which is given as the sample MATLAB implementation
for the SSCA benchmarks~\cite{ssca_matlab}.
%
For example, in our sample graph $G$ from Figure~\ref{fig:sample}, the first 
fringe expansion starting out from vertices $a$ and $c$ can be expressed as
follows:

%
\begin{minipage}{0.18\textwidth}
\begin{center}
\begin{displaymath}
\left[ \begin{array}{cccc}
  a^T & b^T & c^T & d^T \\
  0 & 1 & 1 & 0 \\
  0 & 0 & 0 & 0 \\
  0 & 1 & 0 & 1 \\
  1 & 1 & 0 & 0 \\
\end{array} \right]
\end{displaymath}
\end{center}
\end{minipage}
\hspace{5pt}
\begin{minipage}{0.12\textwidth}
\begin{center}
\begin{displaymath}
\left[ \begin{array}{cc}
   a & c \\
   1 & 0 \\ 
   0 & 0 \\ 
   0 & 1 \\ 
   0 & 0 \\
\end{array} \right]
\end{displaymath}
\end{center}
\end{minipage}
\begin{minipage}{0.12\textwidth}
\begin{center}
\begin{displaymath}
 = \left[ \begin{array}{cc}
  d & a\\
  0 & 1\\
  0 & 0\\ 
  0 & 0\\
  1 & 0\\
\end{array} \right]
\end{displaymath}
\end{center}
\end{minipage}
\\ % Making sure we start on a new line. Minipage is awful!

%
For formatting purposes, we have shown right multiplication by the starting
vectors, which requires transposing the original adjacency matrix.
%
As can be seen, by starting out with BFS expansion from $a$ and $c$, we end 
up with the new fringe $d$ for $a$, and $a$ for $c$, respectively.
%
In fact, since this is always the case, we can forgo this computation and
simply consider the first fringe to be the starting vertex's adjacency row.
%
Please note that in an actual, high performance implementation, all matrices
are sparsely represented to save space.
%
Let us now work through Algorithm~\ref{alg:hybrid} for the sample graph $G$ 
from Figure~\ref{fig:sample}; for simplicity, we process only one vertex at a
time.
%
As before, let us start processing from vertex $a$.
%
\begin{align*}
(Initialization:\ lines\ 1\ to\ 10) \\
nPasses\leftarrow{}4,
BC\leftarrow{}\left[\begin{array}{cccc}0 & 0 & 0 & 0 \\\end{array} \right]\\
BFS\leftarrow{}\emptyset{}, batch\leftarrow{}1:1, depth\leftarrow{}0\\
nsp\leftarrow{}\left[\begin{array}{cccc}1 & 0 & 0 & 0 \\\end{array} \right],
fringe\leftarrow{}\left[\begin{array}{cccc}0 & 0 & 0 & 1 \\\end{array} \right]
\end{align*}
%
Here, we have initialized $nsp(a)$ to be $1$ and set the fringe to be $a$'s
adjacency ($d$).
%
\begin{align*}
(BFS\ search\ from\ a:\ lines\ 11\ to\ 16,\ first\ pass) \\
depth\leftarrow{}1,
BFS(1)\leftarrow{}\left[\begin{array}{cccc}0 & 0 & 0 & 1 \\\end{array} \right]\\
nsp\leftarrow{}\left[\begin{array}{cccc}1 & 0 & 0 & 1 \\\end{array} \right],
fringe\leftarrow{}\left[\begin{array}{cccc}0 & 0 & 1 & 0 \\\end{array} \right]
\end{align*}
%
At the end of the first pass, we discover vertex $c$; $nsp$ now indicates that
we have found one shortest path $(a,d)$.
%
\begin{align*}
(BFS\ search\ from\ a:\ lines\ 11\ to\ 16,\ second\ pass) \\
depth\leftarrow{}2,
BFS(2)\leftarrow{}\left[\begin{array}{cccc}0 & 0 & 1 & 0 \\\end{array} \right]\\
nsp\leftarrow{}\left[\begin{array}{cccc}1 & 0 & 1 & 1 \\\end{array} \right],
fringe\leftarrow{}\left[\begin{array}{cccc}0 & 0 & 0 & 0 \\\end{array} \right]
\end{align*}
%
At the end of the second pass, we rediscover $a$, hence there is no new
$fringe$, and the BFS ceases; $nsp$ indicates that we discovered two shortest
paths, $(a,d)$ and $(c,d)$.
%
Now, we move on to updating the BC scores for all vertices based on our
exploration from vertex $a$.
%
\begin{align*}
(Initialize\ for\ BC\ updates:\ lines\ 17\ and \ 18) \\
BC_{updt}\leftarrow{}\left[\begin{array}{cccc}1 & 1 & 1 & 1 \\\end{array} \right],
nsp^{inv}\leftarrow{}\left[\begin{array}{cccc}1 & 0 & 1 & 1 \\\end{array} \right]
\end{align*}
%
As in Brandes~\ref{alg:brandes}), we move back from the final fringe to the 
first; in our example, this is $depth=2$.
%
\begin{align*}
(Compute\ and\ update\ child\ weights:\ lines\ 20\ to\ 25)\\
weights_1\leftarrow{}\left[\begin{array}{cccc}0 & 0 & 1 & 0 \\\end{array} \right],
weights\leftarrow{}\left[\begin{array}{cccc}0 & 0 & 1 & 0 \\\end{array} \right]\\
temp_1\leftarrow{}\left[\begin{array}{cccc}0 & 1 & 0 & 1 \\\end{array} \right],
temp_2\leftarrow{}\left[\begin{array}{cccc}0 & 0 & 0 & 1 \\\end{array} \right]\\
temp_3\leftarrow{}\left[\begin{array}{cccc}0 & 0 & 0 & 1 \\\end{array} \right],
BC_{updt}\leftarrow{}\left[\begin{array}{cccc}1 & 1 & 1 & 2 \\\end{array} \right]
\end{align*}
%
Next, we update the BC scores for each vertex with $BC_{updt}$.
%
\begin{align*}
(Update\ the\ BC\ scores:\ lines\ 26\ and\ 27)\\
BC\leftarrow{}\left[\begin{array}{cccc}1 & 1 & 1 & 2 \\\end{array} \right]
\end{align*}
%
Notice that in Algorithm~\ref{alg:hybrid}, we pre-compute $BC_{updt}$ to be $1$
for all vertices; hence, even though at the end of the BFS exploration for 
vertex $a$, BC scores of all vertices show a value $>0$, the only true change
is that of vertex $d$, which lies on the path from $a$ to $c$.
%
For brevity, we will now simply list the values of BC after the end of BFS 
exploration for each of the remaining vertices:
%
\begin{align*}
After\ BFS\ from\ b,\ 
BC\leftarrow{}\left[\begin{array}{cccc}2 & 2 & 2 & 3 \\\end{array} \right]\\
After\ BFS\ from\ c,\ 
BC\leftarrow{}\left[\begin{array}{cccc}4 & 3 & 3 & 4 \\\end{array} \right]\\
After\ BFS\ from\ d,\ 
BC\leftarrow{}\left[\begin{array}{cccc}5 & 4 & 5 & 5 \\\end{array} \right]
\end{align*}
%
The final trick to get the accurate BC scores is to subtract the number of
passes as each pass adds $1$ to each vertex's BC score
(Algorithm~\ref{alg:hybrid}, line $28$).
%
Note that without initializing $BC_{updt}$ with ones, line $24$ in
Algorithm~\ref{alg:hybrid} would not have computed $weights$ accurately;
therefore, it is critical to initialize $BC_{updt}$ with ones.
%
\begin{align*}
Subtract\ 4\ from\ BC\ (line\ 28):\ 
BC\leftarrow{}\left[\begin{array}{cccc}1 & 0 & 1 & 1 \\\end{array} \right]
\end{align*}
%
This gives the final and accurate BC scores for all the vertices.
%

%
% Discuss some parallelization issues
%
\subsubsection{Parallelization}
%
%%%%%%%%%%%%%%%%%%%%%%%%%%%%%%%%%%%%%%%%%%%%%%%%%%%%%%%%%%%%%%%%%%%%%%%%%%%
\begin{table}
\centering
\begin{tabular}{|c|c|c|}
\hline
Variable & Space Requirement & Type \\ \hline
$BC$ & $O(n)$ & \code{double} \\ \hline
$A$ & $O(m)$ & \code{<int,int,int>} \\ \hline
$fringe$ & $O(nVerts\times{}m)$ & \code{<int,int,int>} \\ \hline
$BFS$ & $O(depth\times{}nVerts\times{}m)$ & \code{<int,int,int>}[] \\ \hline
$nsp$ & $O(nVerts\times{}n)$ & \code{<int,int,int>} \\ \hline
$nsp^{inv}$ & Computed on the fly & --- \\ \hline
$BC_{updt}$ & $O(n)$ & \code{double} \\ \hline
$weights_1$ & Computed on the fly & --- \\ \hline
$weights$ & $O(n)$ & \code{double} \\ \hline
$temp_1,temp_2,temp_3$ & Computed on the fly & --- \\ \hline
\end{tabular}
\caption{Table depicting the space requirements of each variable in 
Algorithm~\ref{alg:hybrid}. Triplets indicate storage in either CSR or CSC.}
\label{tbl:hybrid}
\end{table}
%%%%%%%%%%%%%%%%%%%%%%%%%%%%%%%%%%%%%%%%%%%%%%%%%%%%%%%%%%%%%%%%%%%%%%%%%%%

Algorithm~\ref{alg:hybrid} is expressed mostly in terms of computations on 
sparse matrices with sparse or dense vectors, which makes it especially 
suitable for parallelization.
%
In this segment, we discuss some of the obvious factors that affect parallel 
performance of Algorithm~\ref{alg:hybrid}.
%
First, let us recap that we are interested mostly in social network analysis
and the real-world graphs in this particular domain are similar to RMAT
graphs (see Section~\ref{sec:rmat}).
%
Briefly, these graphs follow the power-law, have low average connectivity and
low diameter.
%
First, let us recap that Algorithm~\ref{alg:hybrid} implements BFS as a
matrix-matrix multiplication (line $15$).
%
The \func{matMult} takes two matrices, $A$ and $fringe$; of these, $A$, the
adjacency matrix, is both constant and sparse.
%
On the other hand, $fringe$, might start out sparse depending on the start
vertices chosen for exploration, but becomes dense rapidly owing to the low
diameter of the graphs in question.
%
Table~\ref{tbl:hybrid} lists the space requirements of each of the variables 
used in Algorithm~\ref{alg:hybrid}.
%
Similarly, $BFS$, the array that stores the fringe at each depth-level also 
becomes dense rapidly.
%
For a large graph, $fringe$ and $BFS$ even for a single start vertex (i.e.,
$nVerts==1$) might not fit in memory.

%
%%%%%%%%%%%%%%%%%%%%%%%%%%%%%%%%%%%%%%%%%%%%%%%%%%%%%%%%%%%%%%%%%%%%%%%%%%%
\begin{table*}
\centering
\begin{tabular}{|c|c|c|} \hline
Kernel & Description & Type \\ \hline
\functhree{matrix}{m}{n}{init} & Create an $(m\times{}n)$ matrix initialized to \textit{init} & local \\ \hline
\functhree{extractSubMatrix}{A}{m1:m2}{n1:n2} & Extract a submatrix referred by the rows ($m1:m2$) and cols ($n1:n2$) & global \\ \hline
\functwo{eltWiseAssign}{A}{B} & Assign the matrix $A$ to the matrix $B$ & local \\ \hline
\functwo{eltWiseAdd}{A}{B} & Element-wise addition of two (sparse) matrices & local \\ \hline
\functwo{eltWiseMult}{A}{B} & Element-wise multiplication of two (sparse) matrices & local \\ \hline
\funcone{nnzExists}{A} & Check for the presence of a non-zero in the matrix $A$ & global \\ \hline
\functwo{matMult}{A}{B} & Multiply two (sparse) matrices & global \\ \hline
\end{tabular}
\caption{A Table depicting the different matrix kernels in Algorithm~\ref{alg:hybrid}
and their features.}
\label{tbl:complexity}
\end{table*}
%%%%%%%%%%%%%%%%%%%%%%%%%%%%%%%%%%%%%%%%%%%%%%%%%%%%%%%%%%%%%%%%%%%%%%%%%%%
%
Table~\ref{tbl:complexity} depicts the different kernels used in
Algorithm~\ref{alg:hybrid}; As can be seen, other than \func{extractSubMatrix},
\func{nnzExists}, and \func{matMult}, all the other operations are completely
\textit{local}.
%
That is, these operations do not require and communication in the case of a 
distributed-memory implementation.

%
% Discuss Approximate solutions 
%
\subsection{Approximating BC}
\label{subsec:approximate_bc}
%
Real-world graph sizes are ever increasing and BC computations are expensive;
consequently, approximate measures for BC have been explored.
%
There are three prominent works in this regard.
%
First, Eppstein and Wang~\cite{Eppstein-2004} describe a randomized
approximation algorithm for estimation of \textit{closeness centrality} in
weighted graphs.~\footnote{Closeness centrality (CC) of a vertex is a global
metric that measures the the distance of a vertex to all other vertices in the
graph.
%
Formally $CC_v=\frac{1}{\sum_{u\in{}V}{d(v,u)}}$.}
%
Their method (RAND) randomly chooses $k$ vertices one by one and use these 
vertices as start vertices for the single source shortest paths; that is, 
instead of solving all sources shortest paths problem, $k$ sources shortest
paths problem is solved.
%
They further proved that for $k=\Theta{(\frac{log(n)}{\epsilon{}^2})}$, their
algorithm approximates closeness centrality with an additive error of
$\epsilon{}\Delta{}_{G}$, where $\Delta{}_G$ is the diameter of the graph, and
$\epsilon{}$ is a small constant.
%
Brandes and Pich~\cite{brandes-2007}, after experimenting with various 
deterministic strategies for selecting source vertices for approximating BC,
concluded that a random sampling strategy was superior.
%
Bader et al.~\cite{Bader07:ApproxBC} presented an adaptive sampling technique
that approximates the BC of \textit{a given vertex}.
%
They conclude that for $0<\epsilon{}<\frac{1}{2}$, if the centrality of a vertex
$v$ is $\frac{n^2}{t}$ for some constant factor $t\ge{}1$, then with a 
probability $\ge{}(1-2\epsilon{})$, its centrality can be estimated within a 
factor of $\frac{1}{\epsilon{}}$ by using only $\epsilon{}t$ samples of 
source vertices.


\section{Sparse Matrix Operations}
\label{sec:ops}

\subsection{Matrix-Matrix Multiply}
\label{subsec:matmult}

\subsection{Element-wise Matrix Operations}
\label{subsec:eltwise}


\section{RMAT Graphs}
\label{sec:rmat}
%%%%%%%%%%%%%%%%%%%%%%%%%%%%%%%%%%%%%%%%%%%%%%%%%%%%%%%%%%%%%%%%%%%%%%%%%%%%%%
\begin{table}
\begin{center}
\begin{tabular}{|c|l|}  \hline
Symbol & Meaning \\ \hline
$N$ & Number of nodes in the real graph \\ \hline
$2^n$ & Number of nodes in the generated graph \\ \hline
$E$ & Number of generated edges (without duplicates) \\ \hline
$(a,b,c,d)$ & Probabilities of an edge falling in each quadrant \\ 
 & $(a+b+c+d=1)$ \\ \hline
\end{tabular}
\caption{Table of symbols for RMAT graphs}
\label{tbl:rmat}
\end{center}
\end{table}
%%%%%%%%%%%%%%%%%%%%%%%%%%%%%%%%%%%%%%%%%%%%%%%%%%%%%%%%%%%%%%%%%%%%%%%%%%%%%%
%
As important as it is to be able to compute social networking metrics such as
BC accurately and efficiently, it is also necessary to be able to generate
synthetic graphs that mimic real-world graphs for testing and comparative
purposes.
%
In recent years, RMAT graphs~\cite{Chakrabarti04:Recursive} have been widely
adopted for generating synthetic graphs that mimic the power-law
characteristics demonstrated by several real-world graphs.
%
The idea behind RMAT graphs is simple: a graph $G(V,E)$ is considered to be 
a boolean adjacency matrix $A$, where $A_{ij}=1$ implies the prescence to an 
edge between vertices $(i,j)$.
%
The edge connections are determined using an appropriate psuedo-random number 
generator (PRNG) with range $[0,1)$ and four carefully chosen weights $a,b,c,$
and $d$ ($a+b+c+d=1$).
%
These four weights reflect the probability that any given edge falls within one
of the four equal sized partitions that the adjacency matrix is recursively
divided into.
%
Typically, $a\ge{}b$, $a\ge{}c$, $a\ge{}d$; in many real-world scenarios,
$(a+b)/(c+d)=3$.
%
Recursion ends upon reaching a $1\times{}1$ cell (at some position $(i,j)$,
which cannot be sub-divided any further.
%
If, by chance, that cell is already occupied, the duplicate is either
discarded.~\footnote{Weighted RMAT graphs are generated by accumulating
duplicate edges that fall in the same cell. For example, if an $l$ edges happen
to fall within the same cell $(i,j)$, the edge weight of $(i,j)$ is set to be
$l$.}
%
Table~\ref{tbl:rmat} summarizes the various parameters involved in generating 
an RMAT graph.


Constraint-based type systems, dependent types, and generic types
have been well-studied in the literature.

\paragraph{Constraint-based type systems.}

The use of constraints for type inference and subtyping has a history
going back to Mitchell~\cite{mitchell84} and by
Reynolds~\cite{reynolds85}.  These and subsequent systems are based on
constraints over types, but not over values.  Trifonov and
Smith~\cite{trifonov96} proposed a type system in which types are
refined using subtyping constraints.
Pottier~\cite{pottier96simplifying} presents a constraint-based type
system for an ML-like language with subtyping.  These developments
lead to \hmx~\cite{sulzmann97type}, a constraint-based framework for
Hindley--Milner-style type systems.  The framework is parametrized on
the specific constraint system $X$; instantiating $X$ yields
extensions of the HM type system.  Constraints in \hmx{} are over
types, not values. The \hmx{} approach is an important precursor to
our constrained types approach. The principal difference is that
\hmx{} applies to functional languages and does not integrate
dependent types.

%
Sulzmann and Stuckey~\cite{sulzmann-hmx-clpx} showed that the
type inference algorithm for \hmx can be encoded as a
constraint logic program parametrized by the constraint system
$X$. This is very much in spirit with our approach.
Constrained types open the door to {\em user-defined}
predicates and functions, effectively permitting the user to enrich
$\cal C$ (hence the power of the compile-time type-checker) by
developing application-specific constraints using a constraint
programming language such as CLP($\cal C$) \cite{clp} or the richer
RCC($\cal C$) \cite{DBLP:conf/fsttcs/JagadeesanNS05}.

\paragraph{Dependent types.}

Dependent type
systems~\cite{xi99dependent,calc-constructions,epigram,cayenne}
parametrize types on values.  Refinement type
systems~\cite{refinement-types,conditional-types,jones94,sized-types,flanagan-popl06,flanagan-fool06,liquid-types},
introduced by Freeman and Pfenning~\cite{refinement-types}, are dependent type
systems that extend a base type system through constraints on values.  These
systems do not treat value and type constraints uniformly.

Our work is closely related to DML, \cite{xi99dependent}, an
extension of ML with dependent types. DML is also built
parametrically on a constraint solver. Types are refinement types;
they do not affect the operational semantics and erasing the
constraints yields a legal DML program.  This differs from generic constrained
types, where erasure of subtyping constraints can prevent the program from
type-checking.
DML does not permit any run-time checking of constraints
(dynamic casts).

The most obvious distinction between DML and constrained types
lies in the target
domain: DML is designed for functional programming
whereas constrained types are designed for imperative, concurrent
object-oriented languages. 
But there are several other
crucial differences as well.

DML achieves its separation between compile-time and run-time processing
by not permitting program
variables to be used in types. Instead, a parallel set of (universally
or existentially quantified) ``index'' variables are
introduced.
%
Second, DML permits only variables of basic index sorts known to
the constraint solver (e.g., \Xcd{bool}, \Xcd{int}, \Xcd{nat}) to
occur in types. In contrast, constrained types permit program
variables at any type to occur in constrained types. As with DML
only operations specified by the constraint system are permitted in
types. However, these operations always include field selection and
equality on object references.  Note that DML-style constraints are easily
encoded in constrained types.

% {\em Conditional
% types}~\cite{conditional-types} extend refinement types to
% encode control-flow information in the types.
% %
% Jones introduced {\em qualified types}, which permit
% types to be constrained by a finite set of
% predicates~\cite{jones94}.
% %
% {\em Sized types}~\cite{sized-types}
% annotate types with the sizes of recursive data structures.
% Sizes are linear functions of size variables.
% Size inference is performed using a constraint solver for
% Presburger arithmetic~\cite{omega}.
% % constraints on types, support primitive recursion only

% Index objects must be pure.
% Singleton types int(n).
% ML$^{\Pi}_0$:
% Refinement of the ML type system: does not affect the
% operational semantics.  Can erase to ML$_0$.

% Jay and Sekanina 1996: array bounds checking based on shape
% types.

% Ada dependent types.
% Ada has constrained array definitions.  A constraint
% \cite{ada-ref-man}.  Not clear if they're dependent.  Are
% there other dependent types?  Generics are dependent?

        % Used for array bounds by Morrisett et al (I think--need
        % to find paper)

% Singleton types~\cite{aspinall-singletons}.

Logically qualified types, or liquid types~\cite{liquid-types},
permit types in a base Hindley--Milner-style type system to be refined with
conjunctions of logical qualifiers.  The subtyping relation is similar to
\Xten{}'s, that is, two liquid types are in the subtyping relation if their base
types are and if one type's qualifier implies the other's.
The Hindley--Milner type
inference algorithm is used to infer base types; these types are used as templates for inference of the liquid types.
The types of certain expressions are over-approximated to ensure inference
is decidable.
To improve precision of the inference algorithm, and hence
to reduce the annotation burden on the programmer, 
the type system is path sensitive.

Hybrid type-checking~\cite{flanagan-popl06,flanagan-fool06}
introduced another refinement type system.
While typing is undecidable, dynamic checks are inserted into
the program when necessary if the type-checker (which
includes a constraint solver) cannot determine
type safety statically.
In \FXG{}, dynamic type checks, including tests of dependent
constraints, are inserted only at explicit casts or
\Xcd{instanceof} expressions; constraint solving is performed at compile time.

% Where clauses for F-bounded polymorphism~\cite{where-clauses}
% Bounded quantification: Cardelli and Wegner.  Bound T with T'
% In F-bounded polymorphism~\cite{f-bounds}, type variables are bounded by a function of 
% the type variable. 
% Not dependent types.

Concoqtion~\cite{concoqtion} extends types in OCaml~\cite{ocaml}
with constraints written as Coq~\cite{coq} rules.
While the types are expressive, supporting the full generality
of the Coq language, proofs must be
provided to satisfy the type checker.
\Xten{} supports only constraints that can be checked by a
constraint solver during compilation.
Concoqtion encodes OCaml types and value to allow reasoning in
the Coq formulae; however, there is an impedance mismatch
caused by the differing syntax, representation, and behavior
of OCaml versus Coq.

\eat{
Cayenne~\cite{cayenne} is a Haskell-like language with fully dependent types.
There is no distinction between static and dynamic types.
Type-checking is undecidable.
There is no notion of datatype refinement as in DML.

Epigram~\cite{epigram,epigram-matter}
is a dependently typed functional programming language based on
a type theory with inductive families.
Epigram does not have a phase distinction between values and
types.
}

\eat{
$\lambda^{\sf Con}$ is a lambda calculus with assertions.
Findler, Felleisen, Contracts for higher-order functions (ICFP02)

  example: int[> 9]

contracts are either simple predicates or function contracts.
defined by (define/contract ...)

enforced at run-time.
}

% Jif~\cite{jif,jflow} is an extension of Java in which
% types are labeled with security policies enforced by the
% compiler.

\eat{
ESC/Java~\cite{esc-java}
allow programmers to write object invariants and pre- and
post-conditions that are enforced statically
by the compiler using an automated theorem prover.
Static checking is undecidable and, in the presence of loops,
is unsound (but still useful) unless the programmer supplies loop invariants.
ESC/Java can enforce invariants on mutable state.
}

% and Spec$\sharp$~\cite{specsharp}

\eat{
Pluggable and optional type systems were proposed by
Bracha~\cite{bracha04-pluggable} and provide another means of
specifying refinement types.
Type annotations, implemented in compiler plugins, serve only to
reject programs statically that might otherwise have dynamic
type errors.
CQual~\cite{foster-popl02} extends C with user-defined type
qualifiers.  These
qualifiers may be flow-sensitive and may be inferred. 
CQual supports only a fixed set of typing rules
for all qualifiers.
In contrast, the {\em semantic type qualifiers} of
Chin, Markstrum, and Millstein~\cite{chin05-qualifiers}
allow programmers to define typing rules for qualifiers
in a meta language that allows type-checking rules to be
specified declaratively.
JavaCOP~\cite{javacop-oopsla06} is a pluggable type system
framework for Java.  Annotations are defined in a meta language
that allows type-checking rules to be specified declaratively.
JSR 308~\cite{jsr308} is a proposal for adding user-defined type qualifiers
to Java.
}

% Holt, Cordy, the Turing programming language

% Ou, Tan, Mandelbaum, Walker, Dynamic typing with dependent types
% Separate dependent and simple parts of the program.
% Statically type the dependent parts.
% Dynamic checks when passing values into dependent part.

\paragraph{Genericity.}

Genericity in object-oriented languages is usually
supported through
type parametrization.

A number of proposals 
for adding genericity to Java quickly followed
the initial release of
the language~\cite{GJ,Pizza,java-popl97,thorup97,allen03}.
GJ~\cite{GJ} implements invariant type
parameters via type erasure.
PolyJ~\cite{java-popl97} supports run-time representation of types
via adapter objects, and also permits instantiation of
parameters on primitive types and structural parameter bounds.
Viroli and Natali~\cite{reflective-generics,type-passing-generics}
also support
a run-time representation of types, using Java's reflection API.
NextGen~\cite{nextgen,allen03} was implemented using run-time 
instantiation.
\Xten{}'s generics have a hybrid implementation, adopting PolyJ's
adapter object approach for dependent types and for 
type introspection and using NextGen's run-time
instantiation approach for greater efficiency.
% MixGen~\cite{allen04} extends NextGen with mixins.

\csharp also supports generic types via run-time instantiation in the
CLR~\cite{csharp-generics}.  Type parameters may be declared
with definition-site variance tags.
Generalized type constraints were proposed for
\csharp~\cite{emir06}.  Methods can be annotated with subtyping
constraints that must be satisfied to invoke the method.
Generic \Xten{} supports these constraints, as well as constraints
on values, with method and constructor where clauses.

\eat{
\FXG{} does not support \emph{bivariance}~\cite{variant-parametric-types}; a
class \xcd"C" is bivariant in a type property \xcd"X" if \xcd"C{self.X==S}" is
a subtype of \xcd"C{self.X==T}" for any \xcd"S" and \xcd"T".  Bivariance is
useful for writing code in which the property \xcd"X" is ignored.  One can
achieve  this effect in \FXG{} simply by leaving \xcd"X" unconstrained.
}

\eat{
Parametric types with use-site variance are related to existential types:
\xcd"C<+T>" corresponds to the bounded existential $\exists\tcd{X<:T}.C<X>$;
\xcd"C<-T>" corresponds to the bounded existential $\exists\tcd{X:>T}.C<X>$;
\xcd"C<*>" corresponds to the unbounded existential $\exists\tcd{X}.C<X>$.
\FXG{} has a similar correspondence:
\xcd"C{X<:T}" corresponds to the bounded existential \xcdmath"C\{\exists\tcd{self}:C.self.X<:T\}";
\xcd"C{X:>T}" corresponds to the bounded existential \xcdmath"C\{\exists\tcd{self}.C<X\}";
\xcd"C" corresponds to the unbounded existential \xcdmath"C\{\exists\tcd{self}.C<X\}".
}



\bibliographystyle{plain}

\bibliography{references}

% that's all folks
\end{document}
