\documentclass{article}

\usepackage{fullpage}
\textheight=9.0in
\usepackage[cmex10]{amsmath}
\usepackage{url}

% Squeeze stuff
\makeatletter

\def\@listi{\leftmargin\leftmargini
            \parsep 1.0\p@ \@plus\p@ \@minus\p@
            \topsep 1.25\p@ \@plus2\p@ \@minus2\p@
            \itemsep1.0\p@ \@plus\p@ \@minus\p@}
\let\@listI\@listi
\@listi
\def\@listii {\leftmargin\leftmarginii
              \labelwidth\leftmarginii
              \advance\labelwidth-\labelsep
              \topsep    2.0\p@ \@plus2\p@ \@minus\p@
              \parsep    1.5\p@   \@plus\p@  \@minus\p@
              \itemsep   \parsep}
\def\@listiii{\leftmargin\leftmarginiii
              \labelwidth\leftmarginiii
              \advance\labelwidth-\labelsep
              \topsep    1\p@ \@plus\p@\@minus\p@
              \parsep    \z@
              \partopsep \p@ \@plus\z@ \@minus\p@
              \itemsep   \topsep}

\setlength\abovecaptionskip{2\p@}
\setlength\belowcaptionskip{-2\p@}
\setlength\parskip{0em}
\setlength\parindent{1em}
\setlength\dbltextfloatsep{2pt}
\setlength\dblfloatsep{-5pt}

% Footnote handling
\setlength{\footnotesep}{-8pt}
\setlength{\skip\footins}{10\p@ \@plus 4\p@ \@minus 2\p@}% less 7 pt for rule
%
% footnoterule: let \footins specify the distance between the text
% and the rule (\footins should be at least 7pts), and space a bit 
% before the first footnote so that \footnotesep can be smaller
%
\renewcommand\footnoterule{%
  \kern-7\p@
    \hrule width .4\columnwidth% \hrule is by default .4pt high
      \kern 6.6\p@}
      %

\makeatother


\begin{document}
%
% paper title
% can use linebreaks \\ within to get better formatting as desired
\title{Computing Betweenness Centrality Of Massive-Scale Graphs}

\author{
Prabhanjan Kambadur \\
I.B.M. T.J. Watson Research Center \\
Yorktown Heights, NY 10592, \\
pkambadu@us.ibm.com}
\date{}

% make the title area
\maketitle

Betweenness centrality (BC) is an important metric that can be used to
determine the power of a vertex in a graph~\cite{Freeman77,Anthonisse71}.
%
Its uses range from analyzing social networks to power grids; in fact,
computing BC is one of the core kernels of DARPA's HPCS
SSCA\#2~\cite{ssca_matlab}, a benchmark designed to evaluate the capability of
emerging platforms to perform graph analytics.
%
Computing BC for a directed graph $G(V,E)$, with $n$ vertices and $m$ edges, is
both temporally and spatially demanding; the best known serial algorithm
requires $O(nm)$ time and $O(n+m)$ space if the edges are unweighted, and
$O(nm+n^2log(n))$ time and $O(n+m)$ space if the edges are
weighted~\cite{brandes01:_mathsoc}.
%
Given the burgeoning sizes of graphs that need to be analysed, many efforts
have attempted to parallelize BC computations with varying degrees of
success~\cite{Madduri:2009,Santos:2006,edmonds-hipc-2010,Yang05,buluc-2010};
the best known result~\cite{buluc-2010} has demonstrated scalability up to 
1225 cores on an unweighted graph with $32$ million modes and $\approx{}256$
million edges.
%
As such, an efficient and parallel algorithm for computing BC for massive-scale
weighted/unweighted graphs is a must on Hadoop~\cite{Hadoop}, which is IBM's
planned analytics platform.
%
To aid researchers in both developing and demonstrating scalability of such an
algorithm will require access to a large Hadoop cluster ($\ge{}2048{}$ cores,
$\ge{}4GB$ of memory/core) with either HDFS or GPFS support.

\bibliographystyle{plain}

\bibliography{references}

% that's all folks
\end{document}
