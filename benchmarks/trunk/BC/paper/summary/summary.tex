\documentclass{article}

\usepackage{fullpage}
\textheight=9.0in
\usepackage[cmex10]{amsmath}
\usepackage{url}

% Squeeze stuff
\makeatletter

\def\@listi{\leftmargin\leftmargini
            \parsep 1.0\p@ \@plus\p@ \@minus\p@
            \topsep 1.25\p@ \@plus2\p@ \@minus2\p@
            \itemsep1.0\p@ \@plus\p@ \@minus\p@}
\let\@listI\@listi
\@listi
\def\@listii {\leftmargin\leftmarginii
              \labelwidth\leftmarginii
              \advance\labelwidth-\labelsep
              \topsep    2.0\p@ \@plus2\p@ \@minus\p@
              \parsep    1.5\p@   \@plus\p@  \@minus\p@
              \itemsep   \parsep}
\def\@listiii{\leftmargin\leftmarginiii
              \labelwidth\leftmarginiii
              \advance\labelwidth-\labelsep
              \topsep    1\p@ \@plus\p@\@minus\p@
              \parsep    \z@
              \partopsep \p@ \@plus\z@ \@minus\p@
              \itemsep   \topsep}

\setlength\abovecaptionskip{2\p@}
\setlength\belowcaptionskip{-2\p@}
\setlength\parskip{0em}
\setlength\parindent{1em}
\setlength\dbltextfloatsep{2pt}
\setlength\dblfloatsep{-5pt}

% Footnote handling
\setlength{\footnotesep}{-8pt}
\setlength{\skip\footins}{10\p@ \@plus 4\p@ \@minus 2\p@}% less 7 pt for rule
%
% footnoterule: let \footins specify the distance between the text
% and the rule (\footins should be at least 7pts), and space a bit 
% before the first footnote so that \footnotesep can be smaller
%
\renewcommand\footnoterule{%
  \kern-7\p@
    \hrule width .4\columnwidth% \hrule is by default .4pt high
      \kern 6.6\p@}
      %

\makeatother


\begin{document}
%
% paper title
% can use linebreaks \\ within to get better formatting as desired
\title{Computing Betweenness Centrality Of Massive-Scale Graphs}

\author{
Prabhanjan Kambadur \\
I.B.M. T.J. Watson Research Center \\
Yorktown Heights, NY 10592, \\
pkambadu@us.ibm.com}
\date{}

% make the title area
\maketitle

Betweenness centrality (BC) is an important metric that can be used to
determine the power of a vertex in a graph~\cite{Freeman77,Anthonisse71}.
%
Informally, it is a measure of a node's ability to reach other vertices in 
the graph quickly; formally, for a graph $G(V,E)$, where $V$ is the vertex set
and $E$ is the edge set, the BC of a vertex $v\in{}V$ is $BC_{v} =
\sum_{s\ne{}v\ne{}t\in{}V}{\frac{\sigma{}_{st}(v)}{\sigma{}_{st}}}$.
%
Here, $\sigma{}_{st}$ denotes the number of shortest paths from a vertex $s$ to
a vertex $t$ in $G$, and $\sigma{}_st(v)$ denotes the number of shortest paths
from $s$ to $t$ that go through $v$.
%
BC's uses range from analyzing social networks to power grids; in fact,
computing BC is one of the core kernels of DARPA's HPCS
SSCA\#2~\cite{ssca_matlab}, a benchmark designed to evaluate the capability of
emerging platforms to support graph analytics efficiently.
%
Calculating BC for a graph with $n$ vertices and $m$ edges, is both
computationally and spatially demanding; the best known serial algorithm, by
Brandes~\cite{brandes01:_mathsoc}, requires $O(nm)$ time and $O(n+m)$ space if
the edges are unweighted, and $O(nm+n^2log(n))$ time and $O(n+m)$ space if the
edges are weighted.
%
The workhorse of Brandes' algorithm is the all-pairs-shortest-paths (APSP)
formulation, which is solved using Dijkstra's algorithm~\cite{dijkstra59} for
weighted graphs and breadth-first search (BFS)~\cite{clr90} for unweighted
graphs.
%
Incidentally, massive-scale parallel BFS is itself one of the Graph 500
benchmarks~\cite{graph500}, whose aim (like SSCA) is to guide the design of
next-generation hardware and software that can handle analytics workloads.
%
Given the burgeoning sizes of graphs that need to be analysed, many efforts
have attempted to parallelize BC computations with varying degrees of
success~\cite{Madduri:2009,Santos:2006,edmonds-hipc-2010,Yang05,buluc-2010};
the best known result~\cite{buluc-2010} has demonstrated (using MPI)
scalability up to 1225 cores on an unweighted graph with $32$ million modes and
$\approx{}256$ million edges.
%
Given the emergence of SSCA and Graph 500 as important benchmarks targeted at
the analytics community, it is essential to develop efficient, parallel, and 
highly scalable algorithms for computing BC on Hadoop~\cite{Hadoop}.
%
To aid researchers in developing and demonstrating scalability of these
algorithms, it is necessary to have access to a large Hadoop cluster
($\ge{}2048{}$ cores, $\ge{}4GB$ of memory/core) with HDFS/GPFS support.

\bibliographystyle{plain}

\bibliography{references}

% that's all folks
\end{document}
