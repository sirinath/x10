\subsection{Place constraints}\label{PlaceTypes}\index{place types}
\label{DepType:PlaceType}\index{placetype}

Recall that an \Xten{} computation spans multiple places
(\Sref{XtenPlaces}). Each place constains data and activities that
operate on that data.  \XtenCurrVer{} does not permit the dynamic
creation of a place. Each \Xten{} computation is initiated with a
fixed number of places, as determined by a configuration parameter.
In this section we discuss how the programmer may supply place type
information, thereby allowing the compiler to check data locality,
i.e., that data items being accessed in an atomic section are local.

\begin{grammar}
PlaceConstraint     \: \xcd"!" Place\opt \\
Place              \:  \xcd"current" \\
                        % \| \xcd"any" \\
                        \| Expression \\
\end{grammar}

Because of the importance of places in the \Xten{} design, special
syntactic support is provided for constrained types involving places.

All \Xten{} reference classes extend the class
\xcd"x10.lang.Ref", which defines a property
\xcd"loc" of type
\xcd"Place".

\begin{xten}
package x10.lang;
public class Ref(loc: place) { ... }  
\end{xten}

If a constrained reference type \xcd"T" has an \xcd"!p" suffix,
the constraint for \xcd"T" is implicitly assumed to contain the clause
\xcd"self.loc==p"; that is,
\xcd"C{c}!p" is equivalent to \xcd"C{self.loc==p && c}".

If the place \xcd"p" is ommitted, \xcd"here" is assumed; that is,
\xcd"C{c}!" is equivalent to \xcd"C{self.loc==here && c}".

% The place specifier \xcd"any" specifies that the object can be
% located anywhere.  Thus, the location is unconstrained; that is,
% \xcd"C{c}!any" is equivalent to \xcd"C{c}".

% XXX ARRAY
The place specifier \xcd"current" on an array base type
specifies that an object with that type at point \xcd"p"
in the array 
is located at \xcd"dist(p)".  The \xcd"current" specifier can be
used only with array types.

\begin{staticrule*}
  It is a compile time error for the \xcd"!"-annotation to
  be used for value types.
\end{staticrule*}

