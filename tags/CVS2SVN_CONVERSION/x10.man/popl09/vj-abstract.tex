Genericity through Constrained Types

Abstract

For OO languages, \cite{oopsla08} introduced the general framework of
{\em constrained types}. This framework is parametrized by an
arbitrary {\em constraint system} $\cal C$ of interest.  Constraint
systems (introduced in the late 80s) formalize systems of partial
information. Constrained types are formulas $C\{c\}$ where $C$ is the
name of a class or an interface and $c$ is a constraint on the
immutable state of the object (the {\em properties}). Thus the
framework captures the notion of value-dependent type systems for OO
languages. It permits the development of languages with pluggable type
systems, and supports dynamic code generation to check casts at
run-time.  Many type systems for OO languages developed over the last
decade can be thought of as constrained type systems.

This paper shows that constrained types can also account for generic
(type-dependent) type systems. The basic idea is to formalize the
essence of nominal OO types -- types as names with specified signature
and a partial order (inheritance, implementation) -- as a constraint
system, and to permit type-valued properties and
parameters. Type-generic dependence is now expressed through
constraints on these properties and parameters. Type-valued properties
are required to have a run-time representation (the run-time semantics
is not defined through erasure).

The paper makes the following contributions. (1) We show how to
accommodate generic OO types within the framework of constrained
types. (2) We illustrate with the development of a formal
featherweight language, Generic FJ (GFJ) and establish type
soundness. (3) We discuss the design and implementation of the type
system for X10, a modern OO language, based on constrained types. The
type system integrates and extends the features of Java 1.4 style
nominal types, virtual types, and Scala's path-types, as
well as representing (dynamic) generic types.

