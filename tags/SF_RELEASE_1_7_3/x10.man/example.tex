\extrapart{Example}

This example illustrates Jacobi Iteration over a generic field.

This program assumes that the user will supply the distribution. No
assumptions are made about the distribution. For instance, the
distribution may allocate each item of the array to a different place.

Start with an array {\cf a[1..N,1..N]} of some arithmetic type.  Extend
with {\cf 0} boundary elements on all sides of the square array.  At
each step of the iteration, replace the value of a cell with the
average of its adjacent cells in the {\cf (i,j)} dimensions.  Compute the
error at each iteration as the max of the changes in value across the
whole array.  Continue the iteration until the error falls below a
pre-given bound.

Assume that {\cf Field} defines {\cf 0}, {\cf 1}, {\cf +}, {\cf /} and
a max operation.

\begin{x10}
public 
    class Jacobi<T implements Field, 
                 distribution D[1..N,1..N]> \{
  final T epsilon;
  final T four = 1'T + 1'T + 1'T + 1'T;
  public Jacobi(T eps) \{
    this.epsilon = eps;
  \}
  void T[D] jacobi(final T[D] a) \{
    clock l = new clock();
    clocked(l) final D[0..N+1,0..N+1] b 
       = (D[0,1..N]0'T
         || D[N+1,1..N]0'T
         || D[1..N,0] 0'T 
         || D[1..N,N+1] 0'T 
         || D[1..N,1..N]a);

     do \{
       next l;
       final \_ temp 
         = new T[D[1..N,1..N]] (i,j) \{
           now (l) \{
                 return
                   ! future \{b[i+1,j]\}
                 + ! future \{b[i-1,j]\}
                 + ! future \{b[i,j-1]\} 
                 + ! future \{b[i,j+1]\} /four;
          \}
        \};
       next l;
       // Use a reduce operation over the given 
       // distribution to compute error.
       T err = (b-temp) >> max();

      // Set up the value of the clocked final
      // variable b for the next phase of the clock
      next(l) b = temp;
     \} while (err >= delta);
     return a;
   \}
\}
\end{x10}
