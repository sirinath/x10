\chapter{Activities}\label{XtenActivities}

An \Xten{} computation may have many concurrent {\em activities} ``in
flight'' at any give time. We use the term activity to denote a
dynamic execution instance of a piece of code (with references to
data). An activity is intended to execute in parallel with other
activities. An activity may be thought of as a very light-weight
thread.  In \XtenCurrVer{}, an activity may not be interrupted,
suspended or resumed as the result of actions taken by any other
activity.

An activity is spawned in a given place and stays in that place for
its lifetime.  An activity may be {\em running}, {\em blocked} on some
condition or {\em terminated}. When the statement associated with an
activity terminates normally, the activity terminates normally; when
it terminates abruptly with some reason $R$, the activity terminates
with the same reason (\S~\ref{ExceptionModel}).

An activity may be long-running and may invoke recursive methods (thus
may have a stack associated with it). On the other hand, an activity
may be short-running, involving a fine-grained operation such as a
single read or write.

An activity may have an {\em activitylocal} heap accessible only
to the activity. 

An activity may asynchronously and in parallel launch activities at
other places.

\Xten{} distinguishes between {\em local} termination and {\em global}
termination of a statement. The execution of a statement by an
activity is said to terminate locally when the activity has finished
all its computation related to that statement. (For instance the
creation of an asynchronous activity terminates locally when the
activity has been created.)  It is said to terminate globally when it
has terminated locally and all activities that it may have spawned at
any place (if any) have, recursively, terminated globally.

An \Xten{} computation is initiated as a single activity from the
command line. This activity is the {\em root activity}\index{root
activity} for the entire computation. The entire computation
terminates when (and only when) this activity globally
terminates. Thus \Xten{} does not permit the creation of so called
``daemon threads'' -- threads that outlive the lifetime of the root
activity. We say that an \Xten{} computation is {\em rooted}
(\S~\ref{initial-computation}).

\futureext{ We may permit the initial activity to be a daemon activity
to permit reactive computations, such as webservers, that may not
terminate.}

\section{The \Xten{} rooted exception model}
\label{ExceptionModel}

The rooted nature of \Xten{} computations permits the definition of a
{\em rooted} exception model. In multi-threaded programming languages
there is a natural parent-child relationship between a thread and a
thread that it spawns. Typically the parent thread continues execution
in parallel with the child thread. Therefore the parent thread cannot
serve to catch any exceptions thrown by the child thread. 

The presence of a root activity permits \Xten{} to adopt a different
model.  In any state of the computation, say that an activity $A$ is
{\em a root of} an activity $B$ if $A$ is an ancestor of $B$ and $A$
is suspended at a statement (such as the {\tt finish} statement
\S~\ref{finish}) awaiting the termination of $B$ (and possibly other
activities). For every \Xten{} computation, the {\tt root-of} relation
is guaranteed to be a tree. The root of the tree is the root activity
of the entire computation. If $A$ is the nearest root of $B$, the path
from $A$ to $B$ is called the {\em activation path} for the
activity.\footnote{Note that depending on the state of the computation
the activation path may traverse activities that are running,
suspended or terminated.}

We may now state the exception model for \Xten.  An uncaught exception
propagates up the activation path to its nearest root activity, where
it may be handled locally or propagated up the {\tt root-of} tree when
the activity terminates (based on the semantics of the statement being
executed by the activity).\footnote{In \XtenCurrVer{} the {\tt finish}
statement is the only statement that marks its activity as a root
activity. Future versions of the language may introduce more such
statements.}  Thus, unlike concurrent languages such as \java{} no
exception is ``thrown on the floor''.

\section{Spawning an activity}\label{AsynchronousActivity}\label{AsyncActivity}

Asynchronous activities serve as a single abstraction for supporting a
wide range of concurrency constructs such as message passing, threads,
DMA, streaming, data prefetching. (In general, asynchronous operations
are better suited for supporting scalability than synchronous
operations.)

An activity is created by executing the statement:
\begin{x10}
463 Statement ::= AsyncStatement
473 StatementNoShortIf ::= AsyncStatementNoShortIf
481 AsyncStatement ::= 
      async PlaceExpressionSingleListopt Statement
491 AsyncStatementNoShortIf ::= 
      async PlaceExpressionSingleListopt 
         StatementNoShortIf
524   PlaceExpressionSingleListopt ::=
525       | PlaceExpressionSingleList
499   PlaceExpressionSingleList ::= 
         ( PlaceExpression )
500   PlaceExpression ::= Expression
\end{x10} 

The place expression {\tt e} is expected to be of type {\tt place},
e.g.{} {\tt here} or {\tt place.FIRST\_PLACE} or {\tt d[p]} for some
distribution {\tt d} and point {\tt p} (\S~\ref{XtenPlaces}).  
If not, the compiler replaces
{\tt e} with {\tt e.location}. (Recall that every expression in
\Xten{} has a type; this type is a subtype of the root class {\tt
x10.lang.Object}.  This class has a field {\tt location} of type {\tt
place} recording the place at which the value resides. See
the documentation for {\tt x10.lang.Object}.)

Note specifically that the expression {\tt a[i]} when used as a place
expression will evaluate to {\tt a[i].location}, which may not be
the same place as {\tt a.distribution[i]}. The programmer must be 
careful to choose the right expression, appropriate for the statement.
Accesses to {\tt a[i]} within {\tt Statement} should typically be guarded 
by the place expression {\tt a.distribution[i]}.

In many cases the compiler may infer the unique place at which the
statement is to be executed by an analysis of the types of the
variables occuring in the statement. (The place must be such that the
statement can be executed safetly, without generating a {\tt
BadPlaceException}.) In such cases the programmer may omit the place
designator; the compiler will throw an error if it cannot 
determine the unique designated place.\footnote{\XtenCurrVer{} does 
not specify a particular algorithm; this will be fixed in future versions.}

An activity $A$ executes the statement {\tt async (P) S} by launching
a new activity $B$ at the designated place, to execute the specified
statement. The statement terminates locally as soon as $B$ is
launched.  The activation path for $B$ is that of $A$, augmented with
information about the line number at which $B$ was spawned.  $B$
terminates normally when $S$ terminates normally.  It terminates
abruptly if $S$ throws an (uncaught) exception. The exception is
propagated to $A$ if $A$ is a root activity (see \S~\ref{finish}),
otherwise through $A$ to $A$'s root activity. Note that while an
activity is running, exceptions thrown by activities it has already
generated may propagate through it up to its root activity.

Multiple activities launched by a single activity at another place are
not ordered in any way. They are added to the pool of activities at
the target place and will be executed in sequence or in parallel based
on the local scheduler's decisions. If the programmer wishes to
sequence their execution s/he must use \Xten{} constructs, such as
clocks and {\tt finish} to obtain the desired effect.  Further, the
\Xten{} implementations are not required to have fair schedulers,
though every implementation should make a best faith effort to ensure
that every activity eventually gets a chance to make forward progress.

\paragraph{Static semantics.}
The statement in the body of an {\tt async} is subject to the
restriction that it must be acceptable as the body of a {\cf void}
method for an anoymous inner class declared at that point in the code,
which throws no checked exceptions. As such, it may reference
variables in lexically enclosing scopes (including {\cf clock}
variables, \S~\ref{XtenClocks}) provided that such variables are
(implicitly or explicitly) {\cf final}.

\section{Finish}\index{finish}\label{finish}
The statement {\tt finish S} converts global termination to local
termination and introduces a root activity. 
\begin{x10}
468 Statement ::= FinishStatement
478 StatementNoShortif ::= 
      FinishStatementNoShortIf
488 FinishStatement ::= finish Statement
498 FinishStatementNoShortIf ::= 
      finish StatementNoShortIf
\end{x10}

An activity $A$ executes {\tt finish s} by executing {\tt s}.  The
execution of {\tt s} may spawn other asynchronous activities (here or
at other places).  Uncaught exceptions thrown or propagated by any
activity spawned by {\tt s} are accumulated at {\tt finish s}.  {\tt
finish s} terminates locally when all activities spawned by {\tt s}
terminate globally (either abruptly or normally). If
{\tt s} terminates normally, then {\tt finish s} terminates normally
and $A$ continues execution with the next statement after {\tt finish s}.
If {\tt s} terminates abruptly, then {\tt finish s}
terminates abruptly and throws a single exception formed 
from the collection of exceptions accumulated at {\tt finish s}.

Thus a {\tt finish s} statement serves as a collection point for
uncaught exceptions generated during the execution of {\tt s}.

Note that repeatedly {\tt finish}ing a statement has no effect after
the first {\tt finish}: {\tt finish finish s} is indistinguishable
from {\tt finish s}.

\cbstart 
\paragraph{Interaction with clocks.}\label{sec:finish:clock-rule}
{\tt finish s} interacts with clocks (\S~\ref{XtenClocks}). 

While executing {\tt s}, an activity must not spawn any {\tt clocked}
asyncs. (Asyncs spawned during the execution of {\tt s} may spawn
clocked asyncs.) A {\tt
ClockUseException}\index{clock!ClockUseException} is thrown if (and
when) this condition is violated.

In \XtenCurrVer{} this condition is checked dynamically; future
versions of the language will introduce type qualifiers which permit
this condition to be checked statically.

\cbend
\futureext{
The semantics of {\tt finish S} is conjunctive; it terminates when all
the activities created during the execution of {\tt S} (recursively)
terminate. In many situations (e.g.{} nondeterministic search) it is
natural to require a statement to terminate when any {\em one} of the
activities it has spawned succeeds. The other activities may then be
safely aborted. Future versions of the language may introduce a {\tt
finishone S} construct to support such speculative or nondeterministic
computation.
}
%% Need an example here.

\section{Initial activity}\label{initial-computation}\index{initial activity}

An \Xten{} computation is initiated from the command line on the
presentation of a classname {\tt C}. The class must have a {\tt public
static void main(String[] a)} method, otherwise an exception is thrown
and the computation terminates.  The single statement
\begin{x10}
finish async (place.FIRST\_PLACE) \{ 
   C.main(s);
\}
\end{x10} 
\noindent is executed where {\tt s} is an array of strings created
from command line arguments. This single activity is the root activity
for the entire computation. (See \S~\ref{XtenPlaces} for a discussion of
placs.)

%% Say something about configuration information? 

\section{Asynchronous Expression and Futures}\label{XtenFutures}

\Xten{} provides syntactic support for {\em asynchronous expressions}, also
known as futures:
\begin{x10}
511   Primary ::= FutureExpression
515   FutureExpression ::= 
        future PlaceExpressionSingleListopt 
           { Expression }
\end{x10} 
Intuitively such an expression evaluates its body asynchronously at
the given place. The resulting value may be obtained from the future
returned by this expression, by using the {\tt force} operation.

In more detail, in an expression {\tt future(Q){e}}, the place
expression {\tt Q} is treated as in an {\tt async} statement. {\tt e}
is an expression of some type {\tt T}. {\tt e} may reference only
those variables in the enclosing lexical environment which are
declared to be {\tt final}.

If the type of {\cf e} is {\tt T} then the type of
{\tt future (Q){e}} is {\tt future<T>}.  This 
type {\tt future<T>} is defined as if by:
\begin{x10}
public interface future<T> \{
  T force();
  boolean forced();
\}
\end{x10}

(Here we use the syntax for generic classes. \XtenCurrVer{} does not support
generic classes in their full generality. In particular, the user may
not define generic classes. This is reserved for future extensions to the
language.)

Evaluation of {\tt future (Q){e}} terminates locally with the creation
of a value {\tt f} of type {\tt future<T>}.  This value may be
stored in objects, passed as arguments to methods, returned from
method invocation etc. 

At any point, the method {\tt forced} may be invoked on {\tt f}. This
method returns without blocking, with the value {\tt true} if the
asynchronous evaluation of {\tt e} has terminated globally and with
the value {\tt false} if it has not.

The method invocation {\tt force} on {\tt f} blocks until the
asynchronous evaluation of {\tt e} has terminated globally. If the
evaluation terminates successfully with value {\tt v}, then the method
invocation returns {\tt v}. If the evaluation terminates abruptly with
exception {\tt z}, then the method throws exception {\tt z}. Multiple
invocations of {\tt force} (by this or any other activity) do not
result in multiple evaluations of {\tt e}. The results of the first
evaluation are stored in the future {\tt f} and used to respond to all
{\tt force} queries.

\begin{x10}
future<T> promise 
     = future (a.distribution[3]) \{ a[3] \};
T value = promise.force();
\end{x10}

\subsection{Implementation notes}
Futures are provided in \Xten{} for convenience; they may be
programmed using latches, {\tt async} and {\tt finish} as
described in \S~\ref{future-imp}.

\section{Atomic blocks}\label{AtomicBlocks}\index{atomic blocks}
Languages such as \java{} use low-level synchronization locks to allow
multiple interacting threads to coordinate the mutation of shared
data. \Xten{} eschews locks in favor of a very simple high-level
construct, the {\em atomic block}.

A programmer may use atomic blocks to guarantee that invariants of
shared data-structures are maintained even as they are being accessed
simultaneously by multiple activities running in the same place.

\subsection{Unconditional atomic blocks}
The simplest form of an atomic block is the {\em unconditional
atomic block}:

\begin{x10}
461 Statement ::= AtomicStatement
474 StatementNoShortIf ::= 
       AtomicStatementNoShortIf
482 AtomicStatement ::= atomic  Statement
492 AtomicStatementNoShortIf ::= 
       atomic StatementNoShortIf
445 MethodModifier ::= atomic
\end{x10}

For the sake of efficient implementation \XtenCurrVer{} requires
that the atomic block be {\em analyzable}, that is, the set of
locations that are read and written by the {\cf BlockStatement} are
bounded and determined statically.\footnote{ A static bound is a constant
that depends only on the program text, and is independent 
of any runtime parameters. }
The exact algorithm to be used by
the compiler to perform this analysis will be specified in future
versions of the language.
\tbd{}

Such a statement is executed by an activity as if in a single step
during which all other concurrent activities in the same place are
suspended. If execution of the statement may throw an exception, it is
the programmer's responsibility to wrap the atomic block within a
{\cf try/finally} clause and include undo code in the finally
clause. Thus the {\tt atomic} statement only guarantees atomicity on
successful execution, not on a faulty execution.

%% A compiler is allowed to reorder two atomic blocks that have no
%%data-dependency between them, just as it may reorder any two
%%statements which have no data-dependencies between them. For the
%%purposes of data dependency analysis, an atomic block is deemed to
%%have read and written all data at a single program point, the
%%beginning of the atomic block.
%%%% I dont believe we need to say at some point in the atomic block.
%%
We allow methods of an object to be annotated with {\cf atomic}. Such
a method is taken to stand for a method whose body is wrapped within an
{\cf atomic} statement.

Atomic blocks are closely related to non-blocking synchronization
constructs \cite{herlihy91waitfree}, and can be used to implement 
non-blocking concurrent algorithms.

\paragraph{Static semantics and dynamic checks.}
In {\cf atomic s}, {\cf s} may include method calls, conditionals etc.
It may {\em not} include an {\tt async} activity.
It may {\em not} include any statement that may potentially block at
runtime (e.g.{} {\tt when}, {\tt force} operations, {\tt next}
operations on clocks, {\tt finish}).

\limitation{Not checked in the current implementation.}

All locations accessed in an atomic block must reside {\tt here}
(\S~\ref{Here}). A {\tt
BadPlaceException}\index{place!BadPlaceException} is thrown if (and
when) this condition is violated.

\notfouro{All locations accessed in an atomic block must statically satisfy the
{\em locality condition}: they must belong to the place of the current
activity.\index{locality condition}\label{LocalityCondition} The
compiler checks for this condition by checking whether the statement
could be the body of a {\tt void} method annotated with {\tt local} at
that point in the code (\S~\ref{LocalAnnotation}).}

\paragraph{Consequences.}
Note an important property of an (unconditonal) atomic block:

\begin{eqnarray}
 {\cf atomic\ \{s1\ atomic\ s2\}} &=& {\cf atomic\ \{s1\ s2\}}
\end{eqnarray}

Further, an atomic block will eventually terminate successfully or
thrown an exception; it may not introduce a deadlock.


\subsubsection{Example}

The following class method implements a (generic) compare and swap (CAS) operation:

\begin{x10}
// target defined in lexically enclosing environment.
public atomic boolean CAS( Object old, 
                           Object new) \{
   if (target.equals(old)) \{
     target = new;
     return true;
   \}
   return false;
\}
\end{x10}

\subsection{Conditional atomic blocks}

Conditional atomic blocks are of the form:
\begin{x10}
465 Statement ::= WhenStatement
475 StatementNoShortIf ::= WhenStatementNoShortIf
483 WhenStatement ::= 
          when ( Expression ) Statement
484     | WhenStatement 
          or ( Expression ) Statement
\end{x10}

In such a statement the one or more expressions are called {\em
guards} and must be {\cf boolean} expressions. The statements are the
corresponding {\em guarded statements}. The first pair of expression
and statement is called the {\em main clause} and the additional pairs
are called {\em auxiliary clauses}. A statement must have a main
clause and may have no auxiliary clauses.

An activity executing such a statement suspends until such time as any
one of the guards is true in the current state. In that state, the
statement corresponding to the first guard that is true is executed.
The checking of the guards and the execution of the corresponding
guarded statement is done atomically. 


\Xten{} does not guarantee that a conditional atomic block
will execute if its condition holds only intermmitently. For, based on
the vagaries of the scheduler, the precise instant at which a
condition holds may be missed. Therefore the programmer is advised to
ensure that conditions being tested by conditional atomic blocks are
eventually stable, i.e.{} they will continue to hold until the block
executes (the action in the body of the block may cause the condition
to not hold any more).

%%Fourth, \Xten{} does not guarantees only {\em weak fairness} when executing
%%conditional atomic blocks. Let $c$ be the guard of some conditional
%%atomic block $A$. $A$ is required to make forward progress only if
%%$c$ is {\em eventually stable}. That is, any execution $s_1, s_2,
%%\ldots$ of the program is considered illegal only if there is a $j$
%%such that $c$ holds in all states $s_k$ for $k > j$ and in which $A$
%%does not execute. Specifically, if the system executes in such a way
%%that $c$ holds only intermmitently (that is, for some state in which
%%$c$ holds there is always a later state in which $c$ does not hold),
%%$A$ is not required to be executed (though it may be executed).

\begin{rationale}
The guarantee provided by {\cf wait/notify} in \java{} is no
stronger. Indeed conditional atomic blocks may be thought of as a
replacement for \java's wait/notify functionality.
\end{rationale} 

We note two common abbreviations. The statement {\cf when (true) S} is
behaviorally identical to {\cf atomic S}: it never suspends. Second,
{\cf when (c) \{;\}} may be abbreviated to {\cf await(c);} -- it
simply indicates that the thread must await the occurrence of a
certain condition before proceeding.  Finally note that a {\tt when}
statement with multiple branches is behaviorally identical to a {\tt
when} statement with a single branch that checks the disjunction of
the condition of each branch, and whose body contains an {\cf
if/then/else} checking each of the branch conditions.

\paragraph{Static semantics.}
For the sake of efficient implementation certain restrictions are
placed on the guards and statements in a conditional atomic
block. 

Guards are required not to have side-effects, not to spawn
asynchronous activities and to have a statically determinable upper
bound on their execution. These conditions are expected to be checked
statically by the compiler.

The body of a {\tt when} statement must satisfy the conditions
for the body of an {\tt atomic} block.
%Second, as for unconditional atomic blocks, the set of memory
%locations accessed by a guarded statements are required to be bounded
%and statically anlayzable.

Note that this implies that guarded statements are required to be {\em
flat}, that is, they may not contain conditional atomic blocks. (The
implementation of nested conditional atomic blocks may require
sophisticated operational techniques such as rollbacks.)

\paragraph{Sample usage.} 
There are many ways to ensure that a guard is eventually
stable. Typically the set of activities are divided into those that
may enable a condition and those that are blocked on the
condition. Then it is sufficient to require that the threads that may
enable a condition do not disable it once it is enabled. Instead the
condition may be disabled in a guarded statement guarded by the
condition. This will ensure forward progress, given the weak-fairness
guarantee.

\subsection{Examples}

\paragraph{Bounded buffer.}
The following class shows how to implement a bounded buffer of size
$1$ in \Xten{} for repeated communication between a sender and a
receiver.

\begin{x10}
class OneBuffer \{
  nullable Object datum = null;
  boolean filled = false;
  public 
    void send(Object v) \{
      when ( !filled ) \{
        this.datum = v;
        this.filled = true;
    \}
 \}
  public
    Object receive() \{
      when ( filled ) \{
        Object v  = datum;
        datum = null;
        filled = false;
        return v;
      \}
  \}
\}
\end{x10}

\paragraph{ Implementing a future with a latch.}\label{future-imp}
The following class shows how to implement a {\em latch}. A latch is
an object that is initially created in a state called the {\em
unlatched} state. During its lifetime it may transition once to a {\em
forced} state. Once forced, it stays forced for its lifetime. The
latch may be queried to determine if it is forced, and if so, an
associated value may be retrieved. Below, we will consider a latch set
when some activity invokes a {\tt setValue} method on it. This method
provides two values, a normal value and an exceptional value. The
method {\tt force} blocks until the latch is set. If an exceptional
value was specified when the latch was set, that value is thrown on
any attempt to read the latch. Otherwise the normal value is returned.

\begin{x10}
public interface future \{
   boolean forced();
   Object force();
\}
public class Latch implements future \{
  protected boolean forced = false;
  protected nullable boxed result = null;
  protected nullable exception z = null;

  public atomic 
   boolean setValue( nullable Object val ) \{
   return setValue( val, null);
    \}
   public atomic 
   boolean setValue( nullable exception z ) \{
        return setValue( null, z);
    \}
    public atomic 
    boolean setValue( nullable Object val, 
                      nullable exception z ) \{
        if ( forced ) return false;
        // these assignment happens only once.
        this.result = val;
        this.z = z;
        this.forced = true;
        return true;
    \}
    public atomic boolean forced() \{
        return forced;
    \}
    public Object force() \{
        when ( forced ) \{
            if (z != null) throw z;
            return result;
        \}
    \}
\}
\end{x10}

Latches, {\tt aync} operations and {\tt finish} operations may be used
to implement futures as follows. The expression {\tt future(P) \{e\} }
can be translated to:
\begin{x10}
  new RunnableLatch() \{
      public Latch run() \{
         Latch L = new Latch();
         async ( P ) \{
            Object X;
            try \{
                finish X = e;
                async ( L ) \{
                   L.setValue( X ); 
                \}
            \} catch ( exception Z ) \{
               async ( L ) \{
                 L.setValue( Z );
               \}
            \}
         \}
         return l;
      \}
    \}.run()
\end{x10}

Here we assume that {\tt RunnableLatch} is an interface defined by:
\begin{x10} 
  public interface RunnableLatch \{
     Latch run();
  \}
\end{x10}

We use the standard \java{} idiom of wrapping the core translation
inside an inner class definition/method invocation pair (i.e.{} {\tt
new RunnableLatch() {....}.run()}) so as to keep the resulting
expression completely self-contained, while executing statements
inside the evaluation of an expression.

Execution of a {\tt future(P) \{e\}} causes a new latch to be created,
and an {\tt async} activity spawned at {\tt P}. The activity attempts
to {\tt finish} the assigned {\tt x = e}, where {\tt x} is a local
variable.  This may cause new activities to be spawned, based on {\tt
e}. If the assignment terminates successfully, another activity is
spawned to invoke the {\tt setValue} method on the latch.  Exceptions
thrown by these activities (if any) are accumulated at the {\tt
finish} statement and thrown after global termination of all
activities spawned by {\tt x=e}. The exception will be caught by the 
{\tt catch} clause and stored with the latch. 


\todo{Conditional atomic blocks should be powerful enough to implement clocks as well.}

\paragraph{A future to execute a statement.}
Consider an expression {\tt onFinish \{S\}}. This should return a {\tt
boolean} latch which should be forced when {\tt S} has terminated
globally. Unlike {\tt finish S}, the evaluation of {\tt onFinish
\{S\}} should locally terminate immediately, returning a latch. The
latch may be passed around in method invocations and stored in
objects. An activity may perform {\tt force}/{\tt forced} method
invocations on the latch whenever it desires to determine whether {\tt S}
has terminated.

Such an expression can be written as:
\begin{x10}
  new RunnableLatch() \{
      public Latch run() \{
         Latch L = new Latch();
         async ( here ) \{
            try \{
                finish S;
                L.setValue( true );
            \} catch ( exception Z ) \{
                 L.setValue( Z );
            \}
         \}
         return L;
      \}
    \}.run()
\end{x10}

\notinfouro{\section{Remote Method Invocation}\index{remote method invocation}
 We introduce shorthand for remote method invocation:
 
 \begin{x10}
 507   MethodInvocation ::= 
         Primary -> identifier ( ArgumentListopt )
 \end{x10}
 
 If the method named is {\tt void} the expression {\tt o -> m(a1,...,ak)}
 is equivalent to:
 \begin{x10}
  finish async (o) \{o.m(a1,...,ak);\}
 \end{x10}
 
 Otherwise the expression is equivalent to
 \begin{x10}
  future (o)\{o.m(a1, ..., ak)\}.force()
 \end{x10}

}

\section{Iteration}\index{foreach@{\tt foreach}}\label{ForLoop}
\index{for@{\tt for}}\label{ForAllLoop}

We introduce  $k$-dimensional versions of iteration operations {\cf for} and 
{\cf foreach}:

\begin{x10}
189 Statement ::= ForStatement
206 StatementNoShortIf ::= 
      ForStatementNoShortIf
236 ForStatement ::= EnhancedForStatement
239 ForStatementNoShortIf ::= 
      EnhancedForStatementNoShortIf
466 Statement ::= ForEachStatement
476 StatementNoShortIf ::= 
      ForEachStatementNoShortIf
487 EnhancedForStatement ::= 
       for ( FormalParameter : Expression ) 
         Statement
487 EnhancedForStatementNoShortIf ::= 
       for ( FormalParameter : Expression ) 
           StatementNoShortIf
485  ForEachStatement ::= 
       foreach ( FormalParameter : Expression ) 
           Statement
495  ForEachStatementNoShortIf ::= 
        foreach ( FormalParameter : Expression ) 
          StatementNoShortIf
\end{x10}

In both statements, the expression is intended to be of type {\tt
region}.  Expressions {\tt e} of type {\tt distribution} and {\tt
array} are also accepted, and treated as if they were {\tt
e.region}. The compiler throws a type error in all other cases.

The formal parameter must be of type {\tt point}. Exploded syntax may
be used (\S~\ref{exploded-syntax}). The parameter is considered
implicitly final, as are all the exploded variables. 

An activity executes a {\tt for} statement by enumerating the points
in the region in canonical order. The activity executes the body of
the loop with the formal parameter(s) bound to the given point. If the
body locally terminates successfully, the activity continues with the
next iteration, terminating successfully when all points have been
visited. If an iteration throws an exception then the for statement
throws an exception and terminats abruptly.

An activity executes a {\tt foreach} statement in a similar fashion
except that separate {\tt async} activities are launched in parallel
in the local place for each point in the region. The statement
terminates locally when all the activities have been spawned. It never
throws an exception, though exceptions thrown by the spawned
activities are propagated through to the root activity.

In a similar fashion we introduce the syntax:\index{ateach@{\tt ateach}}
\begin{x10}
467 Statement ::= AtEachStatement
477 StatementNoShortIf ::= 
      AtEachStatementNoShortIf
486 AtEachStatement ::= 
      ateach ( FormalParameter : Expression ) 
         Statement
496 AtEachStatementNoShortIf ::= 
      ateach ( FormalParameter : Expression ) 
          StatementNoShortIf
\end{x10}

Here the expression is intended to be of type {\tt distribution}.
Expressions {\tt e} of type {\tt array} are also accepted, and treated
as if they were {\tt e.distribution}. The compiler throws a type error
in all other cases. This statement differs from {\tt foreach} only in
that each activity is spawned at the place specified by the
distribution for the point. That is, 
{\tt ateach( point p[i1,...,ik]: A) S} may
be thought of as standing for:
\begin{x10}
  foreach (point p[i1,...,ik] : A) 
    async (A.distribution[p]) \{S\}
\end{x10}

