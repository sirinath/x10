

\paragraph{Constraint-based type systems.}

The use of constraints in type systems has a history
going back to Mitchell~\cite{mitchell84} and
Reynolds~\cite{reynolds85}.  These and subsequent systems
are based on
constraints over types, but not over values.
Constraint-based type systems for ML-like
languages~\cite{trifonov96,pottier96simplifying}
% Trifonov and
% Smith~\cite{trifonov96} proposed a type system in which types are
% refined using subtyping constraints.
% Pottier~\cite{pottier96simplifying} presents a constraint-based type
% system for an ML-like language with subtyping.  These developments
lead to \hmx~\cite{sulzmann97type}, a constraint-based framework for
Hindley--Milner-style type systems.
The framework is parametrized on
the specific constraint system $X$; instantiating $X$ yields
extensions of the HM type system.
The \hmx{} approach is an important precursor to
our constrained types approach. The principal difference is that
\hmx{} applies to functional languages and does not integrate
dependent types.

Sulzmann and Stuckey~\cite{sulzmann-hmx-clpx} showed that the
type inference algorithm for \hmx can be encoded as a
constraint logic program parametrized by the constraint system
$X$. This is very much in spirit with our approach.
Constrained types permit {\em user-defined}
predicates and functions, allowing the user to enrich
the constraint system, and hence the power of the compile-time type-checker,
with application-specific constraints using a constraint
programming language such as CLP($\cal C$)~\cite{clp} or 
RCC($\cal C$)~\cite{DBLP:conf/fsttcs/JagadeesanNS05}.

\paragraph{Dependent types.}

Dependent type
systems~\cite{xi99dependent,calc-constructions,epigram,cayenne}
param\-etrize types on values.  Constrained types are a form of refinement
type~\cite{refinement-types,conditional-types,jones94,sized-types,flanagan-popl06,flanagan-fool06,liquid-types}.
Introduced by Freeman and Pfenning~\cite{refinement-types},
refinement types
are dependent types that extend a base type system through constraints on values.

Our work is closely related to DML, \cite{xi99dependent}, an
extension of ML with dependent types. DML is also built
parametrically on a constraint solver. Types are refinement types;
they do not affect the operational semantics and erasing the
constraints yields a legal DML program.
%
The most obvious distinction between DML and constrained types
lies in the target
domain: DML is designed for functional programming
whereas constrained types are designed for imperative, concurrent
object-oriented languages. 
But there are several other
crucial differences as well.

DML achieves its separation between compile-time and run-time processing
by not permitting program
variables to be used in types. Instead, a parallel set of (universally
or existentially quantified) ``index'' variables are
introduced.
%
Second, DML permits only variables of basic index sorts known to
the constraint solver (e.g., \xcd{bool}, \xcd{int}, \xcd{nat}) to
occur in types. In contrast, constrained types permit program
variables at any type to occur in constrained types. As with DML
only operations specified by the constraint system are permitted in
types. However, these operations always include field selection and
equality on object references.  Note that DML-style constraints are easily
encoded in constrained types.

% {\em Conditional
% types}~\cite{conditional-types} extend refinement types to
% encode control-flow information in the types.
% %
% Jones introduced {\em qualified types}, which permit
% types to be constrained by a finite set of
% predicates~\cite{jones94}.
% %
% {\em Sized types}~\cite{sized-types}
% annotate types with the sizes of recursive data structures.
% Sizes are linear functions of size variables.
% Size inference is performed using a constraint solver for
% Presburger arithmetic~\cite{omega}.
% % constraints on types, support primitive recursion only

% Index objects must be pure.
% Singleton types int(n).
% ML$^{\Pi}_0$:
% Refinement of the ML type system: does not affect the
% operational semantics.  Can erase to ML$_0$.

% Jay and Sekanina 1996: array bounds checking based on shape
% types.

% Ada dependent types.
% Ada has constrained array definitions.  A constraint
% \cite{ada-ref-man}.  Not clear if they're dependent.  Are
% there other dependent types?  Generics are dependent?

        % Used for array bounds by Morrisett et al (I think--need
        % to find paper)

% Singleton types~\cite{aspinall-singletons}.

Logically qualified types, or liquid types~\cite{liquid-types},
permit types in a base Hindley--Milner-style type system to be refined with
conjunctions of logical qualifiers.  The subtyping relation is similar to
\Xten{}'s: two liquid types are in the subtyping relation if their base
types are and if one type's qualifier implies the other's.
The Hindley--Milner type
inference algorithm is used to infer base types; these types are used as templates for inference of the liquid types.
The types of certain expressions are over-approximated to ensure inference
is decidable.
To improve precision of the inference algorithm, and hence
to reduce the annotation burden on the programmer, 
the type system is path sensitive.  \Xten{} does not (yet) support type
inference.

Hybrid type-checking~\cite{flanagan-popl06,flanagan-fool06}
introduced another refinement type system.
While typing is undecidable, dynamic checks are inserted into
the program when necessary if the type-checker (which
includes a constraint solver) cannot determine
type safety statically.
In \Xten{}, dynamic type checks, including tests of dependent
constraints, are inserted only at explicit casts or
\xcd{instanceof} expressions; constraint solving is performed at compile time.

% Where clauses for F-bounded polymorphism~\cite{where-clauses}
% Bounded quantification: Cardelli and Wegner.  Bound T with T'
% In F-bounded polymorphism~\cite{f-bounds}, type variables are bounded by a function of 
% the type variable. 
% Not dependent types.

Theorem provers have also been integrated into the programming language.
For instance,
Concoqtion~\cite{concoqtion} extends types in OCaml~\cite{ocaml}
with constraints written as Coq~\cite{coq} rules.
Constraints thus have a different
syntax, representation, and behavior than the rest of the language.
Proofs must be provided to satisfy the type checker.
In contrast, 
\Xten{} supports a more limited constraint language 
that can be checked by a
constraint solver during compilation.

ESC/Java~\cite{esc-java}
allow programmers to write object invariants and pre- and
post-conditions that are enforced statically
by the compiler using an automated theorem prover.
Static checking is undecidable and, in the presence of loops,
is unsound (but still useful) unless the programmer supplies loop invariants.
Unlike \Xten{},
ESC/Java can enforce invariants on mutable state.

Constraints in \Xten{} are over final access paths.  Several other languages
have dependent types defined over final access paths~\cite{ernst99-gbeta,scala-overview,scala-oopsla05,nqm06,oz01,ocrz-ecoop03,cdnw07-tribe,dependent-classes}.
In many of these languages, dependent path types are used to
enforce type soundness for virtual
classes~\cite{beta,mp89-virtual-classes,ernst06-virtual} or
similar mechanisms.  Jif~\cite{myers-popl99,jif3}
uses dependent types over final access paths to enforce
security properties: the security policy of an expression may depend on the
policies of other variables in the program.
Aspects of these type systems can be encoded using equality
constraints in \Xten{}.
% Two dependent types indexed by the same
% access path are guaranteed to have the same run-time type.
For example,
for a final access path \xcd{p}, {\tt p}.{\tt type}
in Scala is the singleton type containing the object \xcd{p}.
% In J\&, {\tt p.class} is a type containing all objects
% whose run-time class is the same as \xcd{p}'s.
Scala's {\tt p}.{\tt type} can be encoded in \Xten{} using an equality
constraint \xcd{C\{self == p\}}, where \xcd{C} is a supertype of
\xcd{p}'s static type.

\eat{
These types can be encoded in CFJ by introducing a
\xcd{type} property.
\rn{T-constr}, as
described in Section~\ref{sec:examples}.
}


\eat{
Cayenne~\cite{cayenne} is a Haskell-like language with fully dependent types.
There is no distinction between static and dynamic types.
Type-checking is undecidable.
There is no notion of datatype refinement as in DML.

Epigram~\cite{epigram,epigram-matter}
is a dependently typed functional programming language based on
a type theory with inductive families.
Epigram does not have a phase distinction between values and
types.
}

\eat{
$\lambda^{\sf Con}$ is a lambda calculus with assertions.
Findler, Felleisen, Contracts for higher-order functions (ICFP02)

  example: int[> 9]

contracts are either simple predicates or function contracts.
defined by (define/contract ...)

enforced at run-time.
}

% Jif~\cite{jif,jflow} is an extension of Java in which
% types are labeled with security policies enforced by the
% compiler.

% and Spec$\sharp$~\cite{specsharp}

\paragraph{Pluggable types.}

In \Xten{}, constraint system plugins can provide a constraint
solver to check consistency and entailment of the extended
constraint language.

Pluggable and optional type systems were proposed by
Bracha~\cite{bracha04-pluggable} and provide a means of
extending the base language's type system.
In Bracha's proposal, type annotations, implemented in compiler plugins,
may only reject programs statically that might otherwise have dynamic
type errors; they may not change the run-time semantics of the
language.
Java annotations~\cite{Java3,jsr308}
may be used to extend the Java type system with compiler plugins.

Other approaches, such as user-defined type
qualifiers~\cite{foster-popl02,chin05-qualifiers}
or JavaCOP~\cite{javacop-oopsla06}
allow programmers to 
declaratively specify new typing rules in a meta language rather
than through plugins.
We have focused on defining
constraint-checking rules for the control constructs in the basic language;
additional rules can be defined for type-checking additional control constructs
in the language.  For instance, rules can be defined to ensure that distributed constructs in \Xten{} are place-type safe.
Such type-checking rules may be
implemented in JavaCOP, provided that it is extended with the underlying
constraint solver.


% Holt, Cordy, the Turing programming language

% Ou, Tan, Mandelbaum, Walker, Dynamic typing with dependent types
% Separate dependent and simple parts of the program.
% Statically type the dependent parts.
% Dynamic checks when passing values into dependent part.
