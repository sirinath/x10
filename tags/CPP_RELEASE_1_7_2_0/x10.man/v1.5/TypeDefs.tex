
\section{Type definitions}

With value arguments, type arguments, and constraints, the
syntax for \Xten{} types can often be verbose;
\Xten{} therefore provides {\em type definitions}
to allow users to define new type constructors.

Type definitions have the following syntax:

\begin{grammar}
TypeDefinition \: 
                \xcd"type"~Identifier
                           ( \xcd"[" TypeParameters \xcd"]" )\opt \\
                        && ( \xcd"(" Formals \xcd")" )\opt
                            Constraint\opt \xcd"=" Type \\
\end{grammar}

\noindent
A type definition can be thought of as a type-valued function,
mapping type parameters and value parameters to a concrete type.
%
The following examples are legal type definitions:
\begin{xten}
type StringSet = Set[String];
type MapToList[K,V] = Map[K,List[V]];
type Nat = Int{self>=0};
type Int(x: Int) = Int{self==x};
type Int(lo: Int, hi: Int) = Int{lo <= self, self <= hi};
\end{xten}

As the two definitions of \xcd"Int" demonstrate, type definitions may 
be overloaded: two type definitions with different numbers of type
parameters or with different types of value
parameters, according to the method overloading rules
(\Sref{MethodOverload}), define distinct type constructors.

Type definitions may appear as a class or interface member or
in a block statement.

Type definitions are applicative, not generative; that is, they
define aliases for types but do not introduce new types.
Thus, the following code is legal:
\begin{xten}
type A = Int;
type B = String;
type C = String;
a: A = 3;
b: B = new C("Hi");
c: C = b + ", Mom!";
\end{xten}
% An instance of a defined type with no type parameters and no
% value parameters may 
If a type definition has no type parameters and no value
parameters and is an alias for a class type, then a \xcd"new"
expression may be used to create an instance of the class using
the type definition's name.
Given the following type definition:
\begin{xtenmath}
type A = C[T$_1$, $\dots$, T$_k$]{c};
\end{xtenmath}
where 
\xcdmath"C[T$_1$, $\dots$, T$_k$]" is a
class type, a constructor of \xcdmath"C" may be invoked with
\xcdmath"new A(e$_1$, $\dots$, e$_n$)", if the
invocation
\xcdmath"new C[T$_1$, $\dots$, T$_k$](e$_1$, $\dots$, e$_n$)" is
legal and if the constructor return type is a subtype of
\xcd"A".

\if 0
All type definitions are members of their enclosing package or
class.  A compilation unit may have one or more type definitions
or class or interface declarations with the same name, as long
as the definitions have distinct parameters according to the
method overloading rules (\Sref{MethodOverload}).
\fi

