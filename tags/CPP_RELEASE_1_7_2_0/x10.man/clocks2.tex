\documentclass{article}
%\usepackage{proof}
\usepackage{fullpage}
\usepackage{amssymb}
\bibliographystyle{abbrv}

%\newcommand \limp {\hbox{ $-\!\!\,\circ $ } }
\newcommand \inp[2] {\hbox{in}~#1~#2}
\newcommand \outp[2] {\hbox{out}~#1~#2}
\newcommand \inpL {\hbox{in}\Lscr}
\newcommand \inpR {\hbox{in}\Rscr}
\newcommand \limp {\multimap}
\newcommand \limpR {\limp\Rscr}
\newcommand \limpL {\limp\Lscr}
\newcommand \nuR {\nu\Rscr}
\newcommand \nuL {\nu\Lscr}
\newcommand \nme {\hbox{\sl name}~}
\newcommand \ws {\hbox{\sl ws}}
\newcommand \sst {\hbox{\sl ss}}
\newcommand \rss {\hbox{\sl rs}}
\newcommand \IMLL {\hbox{\sl IMLL}}
\newcommand \IMLLnd {\hbox{\sl IMLL}_\nu^{\Delta}}
\newcommand \vvec[1]{\stackrel{\rightarrow}{#1}}
\newcommand \en[1]{[\![#1]\!]}
\newcommand \ens[1]{\lceil #1 \rceil}
\newcommand \enf[1]{\langle #1 \rangle}
\newcommand \act[3]{\hbox{\sl act}~#1~#2~#3}
\newcommand \fvar[1] {\hbox{fn}(#1)}
\newcommand{\alt}{{\;{\tt\char`\|}\;}}
\newcommand{\Hat}{{\;{\tt\char`\^}\;}}
\newcommand{\now}{\mbox{\bf\tt now}}
\newcommand{\every}{\mbox{\bf\tt every}}
\newcommand{\then}{\rightarrow}
\newcommand{\Else}{\Rightarrow}
\newcommand{\next}{\mbox{\bf\tt next}}
\newcommand{\sync}{\mbox{\bf\tt sync}}
\newtheorem{example}{Example}
\def\nforall{\mu}
\def\nexists{\nu}
%\def\tr{\triangleright}
%\def\tl{\triangleleft}
\def\tr{\triangleright}
\def\tl{\triangleleft}
\def\emp{\emptyset}
\def\derives{\longrightarrow}
\def\equiderives{\longleftrightarrow}
\def\from#1\infer#2{{{\textstyle #1}\over{\textstyle #2}}}
\def\ccfont{\sf}
\def\cc{{\ccfont cc}}
\def\tcc{{\ccfont tcc}}
\def\x10{{\ccfont x10}}
\def\X10{{\ccfont X10}}
\def\fx10{{\ccfont fx10}}
\def\java{{\sc Java}}
\def\kava{{\sc Kava}}
\newcommand \todo[1] {\begin{quotation}{\bf Todo:} {\footnotesize\em #1}\end{quotation}}
\newcommand \metanote[1] {\begin{quotation}{\bf MetaNote:} {\footnotesize\em #1}\end{quotation}}
\begin{document}
\title{\X10{} Design Notes: Clocks\\
}
\author{{\sc PERCS} Area II\footnote{Contact: Vijay Saraswat, {\tt vsaraswa@us.ibm.com} This note is intended primarily for people working on or evaluating the \x10{} design.}\\ 
{\sc ****DRAFT*** v0.01}\\ 
(Please do not circulate)
}

\maketitle

\begin{abstract}
This note contains a proposal for the definition of clocks in \x10{} for repeated quiescence detection of a programmer-specified, dynamically varying, data-dependent set of activities scattered across multiple places. 
\end{abstract}

\section{Clocks}

\X10{}  has a built-in final reference class, {\tt x10.lang.clock} intended
for repeated quiescence detection of arbitrary, data-dependent
collection of activities.

Clocks may only be stored in {\em flow} variables, i.e.{} local
variables and parameter to methods (that are marked with the {\tt
flow} modifier).  Values received by a method in a {\tt flow} argument
or stored in a {\tt flow} local variable may not be assigned to fields
of classes.

Each activity is initiated with zero or more clocks, namely those
available in its lexically enclosing environment.  Because of the
condition above, an activity cannot acquire any new clocks during its
lifetime (except by creating them). Therefore for each activity we can
statically determine (a superset of) the clocks used by that activity
through a simple flow analysis.

% {We need to have a cost-model for clocks. Should ensure that this
% model reflects hierarchical quiescence detection, cf trees of short
% circuits.)

Two statements use clocks: {\tt now (c) \{S\}} and {\tt next c;},
where {\tt c} is a clock and {\tt S} a statement.

The statement {\tt now (c) \{S\}} is said to be {\em clocked on {\tt
c}}. It behaves like {\tt S}. However {\tt c} cannot advance as long
as {\tt S} is executing. This statement is useful in an activity that
wishes to execute {\tt S} in the current instant and does not require
{\tt c} thereafter. If it does wish to access c (or state related to
c) in subsequent instants, it should use next c.

The statement {\tt next c;} blocks until clock {\tt c} can advance.

An \X10{} computation is said to be {\em quiescent on $c$} if each
activity that has a reference to $c$ is blocked on {\tt next c;}.
All activities that do not have a reference to $c$ do not need to
be quiescent. Note that once the system is quiescent on $c$, it
will remain quiescent on $c$ forever (unless the system takes some
action), since no non-blocked activity has a reference to $c$,
or can acquire a reference to $c$. That is,  quiescence on 
a clock is a {\em stable property}. 

Once the implementation has detected quiecence on $c$, the system
resumes all activities blocked on $c$; we say that the clock $c$ {\em
has advanced}.

\todo{Present a simple example of clock advance.}
\paragraph{Multi clock quiescence.}
The definition of quiescence given above can cause deadlock. Consider
the situation in which only two activities are running in
parallel. The first executes {\tt next c; A; next d;} and the other
exeutes {\tt next d; B; next c;}. Now the configuration is stuck, but
it is not quiescent on $c$ or $d$. Rather it is quiescent on the
{\em clock set} $\{c, d\}$.

But note that the the first activity should be thought of as quiescent
on $c$ {\em and} {\em conditionally quiescent} on $d$. That is, as
long as the clock $c$ does not advance, this activity should be
considered quiescent on $d$. Similarly, the second activity is
quiescent on $d$ and conditionally quiescent on $c$. Further the only
activities that reference $c$ and $d$ are these two activities. From
this we can conclude that the system is quiescent on both $c$ and
$d$. Therefore it makes sense to {\em jointly advance} both clocks $c$
and $d$. 

More explicitly the rule for joint advancement is as follows: Say that
an activity is quiescent on a clock set $s$ if it is blocked on some
clock in $s$. An \X10{} computation is said to be quiescent on $s$ if
every activity that has a reference to some clock $c \in s$ is blocked
on $s$.  In such a state the system may simultaneously advance all the
clocks in $s$, thus releasing all activities blocked on $s$.

Note that the notion of quiescence on a clock set generalizes
quiescence on a clock and is strictly more powerful (i.e.{} it
allows more computations to progress). Indeed the rule ensures that no
computation may deadlock because it uses clocks.

\todo{Need natural programming idioms that require multi-clock synchronization.}

\subsection{Program equivalences}

From the discussion above it should be clear that the following
equivalences hold:

\begin{eqnarray}
 {\tt now(c)\, now(d)\,\{S\}} &=& {\tt now(d)\, now(c)\,\{S\}}\\
 {\tt now(c)\, now(c)\,\{S\}} &=& {\tt now(c)\,\{S\}}\\
 {\tt now(c)\, \{S\}\ next\, c;} &=& {\tt \{S\}\ next\, c;}\\
\end{eqnarray}

Note that {\tt next c; next c;} is not the same as {\tt next c;}. The
first will wait for the clock $c$ to advance twice, and the second
once.  Similarly  {\tt next\, c;\, next\, d;} is not the same as
{\tt next\, d;\, next\, c;}. The first could allow the clock
$c$ to advance by itself, whereas the second could force $c$ to 
advance jointly with $d$. Consider, for instance, the context
{\tt \_ | next\,c;\}.

\subsection{Implementation Notes}
Clocks may be implemented efficiently with message passing, e.g.{} by
using short-circuit ideas in \cite{SaraswatPODC88}.  Recall that every
activity is spawned with references to a fixed number of clocks. Each
reference should be thought of as a global pointer to a location in
some place representing the clock. (We shall discuss a further
optimization below.) Each clock keeps two counters: the total number
of outstanding references to the clock, and the number of activities
that are currently suspended on the clock.

When an activity $A$ spawns another activity $B$ that will reference a
clock $c$ referenced by $A$, $A$ {\em splits} the reference by sending
a message to the clock. Whenever an activity drops a reference to a
clock, or suspends on it, it sends a message to the clock. 

The optimization is that the clock can be represented in a distributed
fashion. Each place keeps a local counter for each clock that is
referenced by an activity in that place. The global location for the
clock simply keeps track of the places that have references and that
are quiescent. This can reduce the inter-place message traffic
significantly.

\paragraph{Clock set quiescence.}
We now discuss how quiescence on clock sets may be detected. For this
it is convenient to assume that are clocks are globally ordered (to
avoid a ``lord of the rings'' computation).  

Each activity keeps track of the set of clocks it references, together
with the phase of each clock (an integer). Each clock $c$ tracks 
\begin{enumerate}
\item the number $n$ of outstanding references to it, 
\item its current phase $m$, 
\item the set $B$ of activities from which it has received a blocking message,
\item the set $A$ of conditional blocking messages is has received from its activities,
\item a set $Z$ of  clock set proposals $q(S_i,Q_i)$ it has received from other clocks, and
\item the phase vector $\phi(d)=m$ which records $c$'s knowledge of the 
current phase of each clock $d$ it knows about.
\end{enumerate}
When a clock is created $n=1$, $m=1$, $b=\emptyset$, $A=\emptyset$,
$Z=\emptyset$, and $\phi$ is undefined for every clock except $c$
($\phi(c)=1$). ($n$ is increased whenever a split reference message is
received by the clock and decreased whenever a reference is dropped.)

When an activity $a$ executes a {\tt next c;} statement, it sends a
message to $c$, which adds $a$ to $b$. $a$ also sends a conditional
blocking message $(c,m)$ to every clock $d$ that it still has a
reference to, thereby informing them that $d$ could be part of a
quiescent clock set involving $(c,m)$.

Whenever $c$ receives a conditional blocking message $(d,m)$ from an
activity $a$ it checks if $\phi(d) > m$. If so, the message is old and is
ignored. If $\phi(d) < m$ then $\phi(d)$ is updated to $m$ and the rest of
the state checked for consistency (see below). If $\phi(d)=m$
(or $\phi(d)$ is undefined, in which case it is set to $m$)  then $a$
together with $(d,m)$ is added to $A$. 

Whenever the phase vector is updated, the state maintained by the
clock is checked for consistency. Any clock set proposal $q(S_i,Q_i)$
which contains a clock at an earlier phase than recorded by $\phi$ is
discarded. Any activity in $A$ with a conditional blocking message
$(d,m)$ is discarded if $m < \phi(d)$. The activity will now be regarded
as being active.

Whenever the clock receives a clock set proposal $q(S,Q)$ from another
clock, it uses the clock information in $S \cup Q$ to update $\phi$
(and its local state, if $\phi$ changes). If any clock phase in $S\cup
Q$ is old (according to $\phi$) then the proposal is discarded;
otherwise it is aded to $Z$. (Note that $\phi$ is updated even if the
proposal is discarded.)

A clock is said to be quiescent if $|A|+|B|=n$.  When a clock becomes
quiescent it constructs the set $Q = Q_1 \cup \ldots \cup Q_k \cup
\{c,m)\}$ and $S = (A \cup S_1 \cup \ldots S_k)\setminus Q$ 
where $Z=\{q(S_1,Q_1),\ldots, q(S_k,Q_k)\}$. This represents its
estimate of the potentially quiescent clock set to which it belongs.
If $S=\emptyset$ then it has discovered a quiescent clock set, and it
informs every clock in $Q$ that it can progress. Otherwise, it
extracts the least clock $(d,m)$ from $S$ and sends a clock set
proposal $q(S, Q)$ to $d$.

When a clock receives a message that it can progress it increments $m$, 
informs every activity in $A$ that it is unblocked, and clears
$A$, $B$ and $Z$.

Note that even after sending a clock set proposal, the clock may
continue to receive conditional blocking messages from
activities. These are treated exactly as above. Thus a clock may
generate multiple clock set proposals for its current phase. This
number is bounded by the number of messages received from activities
referencing the clock before they send an unconditional blocking
message.

This algorithm can be improved if it is possible to devise disjoint
{\em clock groups} and ensure that no activity can be created which
contains clocks from more than one clock group. Now if all the clocks
for one clock group are stored in one place then there may be no need
to keep separate phase estimators $\phi$ at each clock. Instead the
phase of each clock could be read directly from the clock.

The algorithm can also be improved by using a hierarchical scheme with
one ``proxy clock'' per place for a clock. The proxy sends a
conditional or unconditional blocking message to the clock only when
all activities in that place referencing the clock have quiesced.

\subsection{Clocked types}

We allow types to specify clocks, via a {\tt clocked(c)} modifier,
e.g.{}

{\footnotesize
\begin{verbatim}
  clocked(c) int r;
\end{verbatim}}

This declaration asserts that $r$ is accessible (readable/writable) only
by those statements that are clocked on $c$. Thus propagation of the clock provides some control over the ``visibility'' of $r$.

The declaration 

{\footnotesize
\begin{verbatim}
  clocked(c) final int r = 1;
\end{verbatim}}

\noindent asserts additionally that in each clock instant $r$ is final, 
i.e.{} the value of $r$ can be reset in each time instant (and stays
constant in that instant). 

\todo{Generalize the syntax so that aggregate variables can be clocked with an aggregate clock of the same shape.}

We expect that most arrays containing application data will be
declared to be {\tt clocked final}. We support this very powerful type
declaration by providing a new statement:
{\footnotesize
\begin{verbatim}
  next l = r; before c;
\end{verbatim}}

\noindent 
for a variable $l$ declared to be clocked on $c$. The statement
assigns $r$ to the {\em next} value of $l$. There may be multiple such
assignments before the clock advances. The last such assignment
specifies the value of the variable that will be visible after the
clock has advanced.  If the variable is {\tt clocked final} it is
guaranteed that {\em all} readers of the variable throughout this
phase will see the value $r$.

We expect that very efficient {\em clocked cache} hardware techniques
can be developed to update the cache with the new values for clocked
final variables before activities are resumed in the new clock phase.
When activities resume, they will find the data they need available
locally with low latency.

A much richer vocabulary of clocked types is possible and needs to be
investigated further. In particular we believe that the full power of
\tcc{} can be brought to bear to develop types that vary across time. 
Intuitively a temporal type is a finite state machine whose states are
labeled with types and whose edges are labeled with signals which
describe events of interest. The type at each state may describe the
distribution of an aggregate variable across places, or the
co-location of scalar variables. Type checking corresponds to running
\tcc{} programs at compile-time over a constraint system that describes the atemporal type system.

\subsection{Determinate shared memory concurrency}

We should be able to statically guarantee determinate computation if
the program can be typed in such a way that all mutable shared
variables are {\tt clocked final} on a clock $c$, and all accesses to
the variable are performed by statements clocked by $c$. Then,
regardless of the parallelism or distribution across places, all reads
of a variable will be answered with the same value.  \todo{Investigate
variable declarations that allow a variable to be clocked by many
clocks.}

\subsection{Example programs}

\begin{example}[Sweep3d]

Consider the core of the ASCI Benchmark Sweep3D program for computing
solutions to mass transport problems.

In a nutshell the core computation is a triply nested sequential loop
in which the value of a variable in the current iteration is dependent
on the values of neighboring variables in a past iteration. Such a
problem can be parallelized through pipelining. One visualizes a
diagonal wavefront sweeping through the array. An MPI version of the
program may be described as follows. There is a two dimensional grid
of processors which performs the following computation
repeatedly. Each processor synchronously receives a value from the
processor to its west, then to its north, then computes some function
of these values and computes a new value to be sent to the processor
to its east and then to its south.  Ignoring the behavior of the
boundary processors for the moment such a computation may be described
by the following \x10{} program:

{\footnotesize
\begin{verbatim}
region r = [1..n,1..m];
clock[r] w,n;
clock(w) final double [r] A; 
for ( int t : 1..TMax) {
  forall(int i,j:A) {
    double west = A[i-1,j]; next w[i-1,j];           
    double north = A[i,j-1]; next n[i,j-1]
    next A[i,j] = f(west, north);
    next w[i,j];
    next n[i,j];
    }
}
\end{verbatim}
}
\end{example}

\bibliography{master}

\end{document}
