\section{Enumeration type}

%% Why is an int type not an enumeration type?

The basic design to start with is that of Java enums.

enums are final value classes, inheriting from x10.lang.Object, with
no instance variables, and one static final self-typed variable for
each item in the enumeration.

enums may have static variables and static and instance methods. enums
may have constructors.


Hmm.. so this design will not work with the idea of treating 1..25 as
an enum.

Hmm.. what are enums really.. just an enumerated set of literals! Why
cant we use some other enum's constant? i see.. because we would potentially be defining multiple orders.

Ah, so the solution is clear. int is an enum. For any enum X if a and
b are two ordered values with a < b, then a..b is also an enum!

So this gives us all the subranges as enums!

I like this. 9:22am 6/15/2004

We must have a design that allows for us to have enumeration literals.
e.g. we should be able to use 1..25 as an enumeration literal. Such a
literal represents, implicitly an enumeration class declared with the
constants 1, 2, 3... 25, in that particular order.

Interesting. What we are saying is that .. is a type former. It takes
two values and returns a type.

Note that this requires that we be able to use the same literal (e.g. 1) at
different types (e.g. int, etc).
