\chapter{Types}
\label{XtenTypes}\index{types}

{}\Xten{} is a {\em strongly typed} object-oriented language: every
variable and expression has a type that is known at compile-time.
Types limit the values that variables can hold and specify the places
at which these values can lie.


{}\Xten{} supports three kinds of runtime entities, {\em objects},
{\em structs}, and {\em functions}. Objects are instances of {\em
  classes} (\Sref{ReferenceClasses}). They may contain mutable fields
and stay resident in the place in which they were
created. 
Objects are said to be {\em boxed} in that variables of a
class type are implemented through a single memory location that
contains a reference to the memory containing the declared state of
the object (and other meta-information such as the list of methods of
the object). Thus objects are represented through an extra level of
indirection. A consequence of this flexibility is that every class
type contains the value \Xcd{null} corresponding to the invalid
reference. \Xcd{null} is often useful as a default value. Further, two
objects may be compared for identity (\Xcd{==}) in constant time by
simply containing references to the memory used to represent the
objects.

Structs are instances of {\em struct types} (\Sref{StructClasses}). They are a
restricted variant of classes, lacking meta-information; this makes them less
flexible, but in many cases more efficient. When it is semantically
meaningful, converting a class into a struct or vice-versa is quite easy.
Structs are immutable and may be freely copied from place to place. Further,
they may be allocated inline, using only as much memory as necessary to hold
and align the fields of the struct.

Functions, called closures or lambda-expressions in other languages, are
instances of {\em function types|} {\Sref{Functions}). Functions can refer to
%~~exp~~`~~`~~y:Int ~~
variables from the surrounding environment; \eg, \xcd`(x:Int)=>x*y` is a unary
integer function which multiplies its argument by the variable \xcd`y` from
the surrounding block.  
Functions may be freely copied from place to place and may be repeatedly
applied to a set of arguments.

These runtime entities are classified by {\em types}. Types are used in
variable declarations, explicit coercions and conversions, object creation,
array creation, class literals, static state and method accessors, and
\xcd"instanceof" expressions.

The basic relationship between values and types is {\em instantiation}. For
example, \xcd`1` is an instance of type of integers, \xcd`Int`. It is also an
instance of type of all entities \xcd`Any`, and of type of nonzero integers
\xcd`Int{self != 0}`, and many others.

The basic relationship between types is {\em subtyping}: \xcd`T <: U` holds if
every instance of \xcd`T` is also an instance of \xcd`U`. Two important kinds
of subtyping are {\em subclassing} and {\em strengthening}.  Subclassing is a
familiar notion from object-oriented programming.  In a class
hierarchy with classes \xcd`Animal` and \xcd`Cat` arranged in the usual way,
every \xcd`Cat` is an \xcd`Animal`, so \xcd`Cat <: Animal` by subclassing.  
Strengthening is an equally familiar notion from logic.   The instances of
\xcd`Int{self != 0}` are all elements of \xcd`Int{true}` as well, because
\xcd`self != 0` logically implies \xcd`true`; so 
\xcd`Int{self != 0} <: Int{true} == Int` by strengthening.  X10 uses both
notions of subtyping.  See \Sref{DepType:Equivalence} for the full definition
of subtyping in X10.




\subsection*{The Grammar of Types}

Types are described by the following grammar: \\
\bard{Is this still correct?}
\begin{grammar}
Type \: FunctionType \\
    \| ConstrainedType  \\

FunctionType \: TypeParameters\opt \xcd"(" Formals\opt \xcd")"
Constraint\opt Throws\opt \xcd"=>" TypeOrVoid \\
TypeParameters \: \xcd"[" TypeParameter ( \xcd"," TypeParameter )\star \xcd"]" \\
TypeParameter \: Identifier \\
Throws \: \xcd"throws" TypeName ( \xcd"," TypeName )\star \\

ConstrainedType \: Annotation\star BaseType Constraint\opt
PlaceConstraint\opt \\

BaseType \: ClassBaseType \\
     \| InterfaceBaseType \\
     \| PathType \\
     \| \xcd"(" Type \xcd")" \\
TypeOrVoid \: Type \\
     \| \xcd`Void`\\
ClassType \: Annotation\star ClassBaseType Constraint\opt
PlaceConstraint\opt \\
InterfaceType \: Annotation\star InterfaceBaseType Constraint\opt
PlaceConstraint\opt \\

PathType \: Expression \xcd"." Identifier \\

Annotation \: \xcd"@" InterfaceBaseType Constraint\opt \\

ClassOrInterfaceType \: ClassType \\ \| InterfaceType \\
ClassBaseType \: TypeName \\
InterfaceBaseType \: TypeName \\
\end{grammar}

% \section{Type definitions and type constructors}
% 
% Types in \Xten{} are specified through declarations and through
% type constructors:

% \paragraph{Class types.}



\section{Classes and interfaces}
\label{ReferenceTypes}

\subsection{Class types}

\index{types!class types}
\index{class}
\index{class declaration}
\index{declaration!class declaration}
\index{declaration!reference class declaration}

A {\em class declaration} (\Sref{XtenClasses}) introduces a {\em class type}
containing all instances of the class.  The \xcd`Position` class below
could describe the position of a slider control, for example.

%~~gen
% package Types.By.Cripes.Classes;
%~~vis
\begin{xten}
class Position {
  private var x : Int = 0;
  public def move(dx:Int) { x += dx; }
  public def pos() : Int = x;
}
\end{xten}
%~~siv
%
%~~neg

Class instances, also called objects, are created via constructor calls. Class
instances have fields and methods, type members, and value properties bound at
construction time. In addition, classes have static members: constant fields,
type definitions, and member classes and member interfaces.

A class with type parameters is {\em generic}. A class type is instantiatable
only if all of its parameters are instantiated on concrete types.  The
\xcd`Cell[T]` class provides a container capable of holding a value of type
\xcd`T`, or being empty.

%~~gen
% package Types.For.Gripes.Of.Wesley.Snipes;
%~~vis
\begin{xten}
class Cell[T] {
  var empty : Boolean = true;
  var contents : T;
  public def putIn(t:T) { 
    contents = t; empty = false; 
  }
  public def emptyOut() { empty = true; }
  public def isEmpty() = empty;
  public def getOut():T throws Exception {
     if (empty) throw new Exception("Empty!");
     return contents ;
  }
}
\end{xten}
%~~siv
%
%~~neg


\Xten{} does not permit mutable static state. A fundamental principle of the
X10 model of computation is that all mutable state be local to some place
(\Sref{XtenPlaces}), and, as static variables are globally available, they
cannot be mutable. When mutable global state is necessary, programmers should
use singleton classes, putting the state in an object and using place-shifting
commands (\Sref{AtStatement}) and atomicity (\Sref{AtomicBlocks}) as necessary
to mutate it safely.

\index{\Xcd{Object}}
\index{\Xcd{x10.lang.Object}}

Classes are structured in a single-inheritance hierarchy. All classes extend
the class \xcd"x10.lang.Object", directly or indirectly. Each class other than
\xcd`Object` extends a single parent class.  \xcd`Object` provides no behavior
of its own, beyond that required by \xcd`Any`.

\index{class!reference class}
\index{reference class type}
\index{\Xcd{Object}}
\index{\Xcd{x10.lang.Object}}


\index{null}
\bard{We've got to say this better.}
Variables of class type may contain the value \xcd"null". 

\subsection{Interface types}
\label{InterfaceTypes}

\index{types!interface types}
\index{interface}
\index{interface declaration}
\index{declaration!interface declaration}

An {\em interface declaration} (\Sref{XtenInterfaces}) defines an {\em
interface type}, specifying a set of methods, type members, and
properties which must be provided by any class declared to implement the
interface. 


Interfaces can also have static members: constant fields, type definitions,
and member classes and interfaces.  However, interfaces cannot specify that
implementing classes must provide static members.

An interface may extend multiple interfaces.  
%~~gen
%package Types.For.Snipes.Interfaces;
%~~vis
\begin{xten}
interface Named {
  def name():String;
}
interface Mobile {
  def move(howFar:Int):Void;
}
interface Person extends Named, Mobile {}
interface NamedPoint extends Named, Mobile{} 
\end{xten}
%~~siv
%
%~~neg


Classes may be declared to implement multiple interfaces.
Semantically, the interface type is the set of all objects that are
instances of classes that implement the interface. A class implements
an interface if it is declared to and if it implements all the methods
and properties defined in the interface.  For example, \xcd`KimThePoint`
implements \xcd`Person`, and hence \xcd`Named` and \xcd`Mobile`.  It would be
a static error if \xcd`KimThePoint` had no \xcd`name` method.

%~~gen
%interface Named {
%   def name():String;
% }
% interface Mobile {
%   def move(howFar:Int):Void;
% }
% interface Person extends Named, Mobile {}
% interface NamedPoint extends Named, Mobile{} 
%~~vis
\begin{xten}
class KimThePoint implements Person {
   var pos : Int = 0;
   public def name() = "Kim (" + pos + ")";
   public def move(dPos:Int) { pos += dPos; }
}
\end{xten}
%~~siv
%
%~~neg


\subsection{Properties}
\index{properties}
\label{properties}

Classes, interfaces, and structs may have {\em properties}, public \xcd`val` instance
fields bound on object creation. For example, the following code declares a
class named \xcd"Coords" with properties \xcd"x" and \xcd"y" and a \xcd"move"
method. The properties are bound using the \xcd"property" statement in the
constructor.

%~~gen
%package not.x10.lang;
%~~vis
\begin{xten}
class Coords(x: Int, y: Int) {
  def this(x: Int, y: Int) : Coords{self.x==x, self.y==y} 
    = { property(x, y); }
  def move(dx: Int, dy: Int) = new Coords(x+dx, y+dy);
}
\end{xten}
%~~siv
%~~neg

Properties, unlike other public \xcd`val` fields, can be used  
at compile time in {\em constraints}. This allows us
to specify subtypes based on properties, by appending a boolean expression to
the type. For example, the type \xcd"Coords{x==0}" is the set of all points
whose \xcd"x" property is \xcd"0".  Details of this substantial topic are
found in \Sref{ConstrainedTypes}.



\section{Type parameters and Generic Types}
\label{TypeParameters}

\index{types!type parameters}
\index{methods!parametrized methods}
\index{constructors!parametrized constructors}
\index{closures!parametrized closures}
\label{Generics}
\index{types!generic types}

A class, interface, method, closure, or type definition  may have type
parameters.  Type parameters can be used as types, and will be bound to types
on instantiation.  For example, a generic stack class may be defined as 
\xcd`Stack[T]{...}`.  Stacks can hold values of any type; \eg, 
%~~type~~`~~`~~ ~~class Stack[T]{}
\xcd`Stack[Int]` is a stack of integers, and 
%~~type~~`~~`~~ ~~class Stack[T]{}
\xcd`Stack[Point{self!=null}]`is a stack of non-null \xcd`Point`s.
Generics {\em must} be instantiated when they are used: \xcd`Stack`, by
itself, is not a valid type.
Type parameters may be constrained by a guard on the declaration
(\Sref{TypeDefGuard},
\Sref{MethodGuard},\Sref{ClosureGuard}).

\index{types!concrete types}
\index{concrete type}
A {\em generic type} is a class, struct,  interface, or type declared with one or
more type parameters.  When instantiated with concrete (\viz, non-generic)
types for its parameters, a generic type becomes a concrete type and can be
used like any other type. For example,
\xcd`Stack` is a generic type, 
%~~type~~`~~`~~ ~~class Stack[T]{}
\xcd`Stack[Int]` is a concrete type, and can be used as one: 
%~~stmt~~`~~`~~ ~~class Stack[T]{}
\xcd`var stack : Stack[Int];`


A \xcd`Cell[T]` is a generic object, capable of holding a value of type
\xcd`T`.  For example, a \xcd`Cell[Int]` can hold an \xcd`Int`, and a
\xcd`Cell[Cell[Int]{self!=0}]` can hold a \xcd`Cell[Int]` which in turn can
only hold non-zero numbers.  \xcd`Cell`s are actually useful in situations
where values must be bound immutably for one reason, but need to be mutable.
%~~gen
% package ch4;
%~~vis
\begin{xten}
class Cell[T] {
    var x: T;
    def this(x: T) { this.x = x; }
    def get(): T = x;
    def set(x: T) = { this.x = x; }
}
\end{xten}
%~~siv
%~~neg


\xcd"Cell[Int]" is the type of \xcd`Int`-holding cells.  
The \xcd"get" method on a \xcd`Cell[Int]` returns an \xcd"Int"; the
\xcd"set" method takes an \xcd"Int" as argument.  Note that
\xcd"Cell" alone is not a legal type because the parameter is
not bound.

Methods may be generic, even methods in non-generic classes: 
%~~gen
% package Types.For.Cripes.Sake.Generic.Methods;
% import x10.util.*;
%~~vis
\begin{xten}
class NonGeneric {
  static def first[T](x:List[T]!):T = x(0);
}
\end{xten}
%~~siv
%
%~~neg


\subsection{Variance of Type Parameters}
\index{covariant}
\index{contravariant}
\index{invariant}
\index{type parameter!covariant}
\index{type parameter!contravariant}
\index{type parameter!invariant}

Consider classes \xcd`Person :> Child`.  Every child is a person, but there
are people who are not children.  What is the relationship between
\xcd`Cell[Person]` and \xcd`Cell[Child]`?  

\subsubsection{Why Variance Is Necessary}

In this case, \xcd`Cell[Person]` and \xcd`Cell[Child]` should be unrelated.  
If we had \xcd`Cell[Person] :> Cell[Child]`, the following code would let us
assign a \xcd`old` (a \xcd`Person` but not a \xcd`Child`) to a
variable \xcd`young` of type \xcd`Child`, thereby breaking the type system: 
\begin{xten}
// INCORRECTLY assuming Cell[Person] :> Cell[Child]
val cc : Cell[Child] = new Cell[Child]();
val cp : Cell[Person] = cc; // legal upcast
cp.set(old);       // legal since old : Person
val young : Child = cc.get(); 
\end{xten}

Similarly, if \xcd`Cell[Person] <: Cell[Child]`: 
\begin{xten}
// INCORRECTLY assuming Cell[Person] <: Cell[Child]
val cp : Cell[Person] = new Cell[Person];
val cc : Cell[Child] = cp; // legal upcast
val cp.set(old); 
val young : Child = cc.get();
\end{xten}

So, there cannot be a subtyping relationship in either direction between the
two. And indeed, neither of these programs passes the X10 typechecker.


\subsubsection{Legitimate Variance}

The \xcd`Cell[Person]`-vs-\xcd`Cell[Child]` problems occur because it is
possible to both store and retrieve values from the same object. However,
entities with only one of the two capabilities {\em can} sensibly have some
subtyping relations. Furthermore, both sorts of entity are useful. An entity
which can store values but not retrieve them can nonetheless summarize them.
An object which can retrieve values but not store values can be constructed
with an initial value, providing a read-only cell.

So, X10 provides {\em variance} to support these options.  Type parameters
may be defined in one of three forms.  
\begin{enumerate}
\item {\em invariant}: Given a definition \xcd`class C[T]{...}`, \xcd`C[Person]` and
      \xcd`C[Child]` are unrelated classes; neither is a subclass of the
      other.
\item {\em covariant}: Given a definition \xcd`class C[+T]{...}` (the \xcd`+` indicates
      covariance), \xcd`C[Person] :> C[Child]`.  This is appropriate when
      \xcd`C` allows retrieving values but not setting them.
\item {\em contravariant}: Given a definition \xcd`class C[-T]{...}` (the \xcd`-` indicates
      contravariance), \xcd`C[Person] <: C[Child]`.  This is appropriate when
      \xcd`C` allows storing values but not retrieving them.
\end{enumerate}


The \xcd"T" parameter of \xcd"Cell" above is
invariant.  

A typical example of covariance is \xcd`Get`.  As the \xcd`example()` method
shows, a \xcd`Get[T]` must be constructed with its value, and will return that
value whenever desired.
%~~gen
% package ch4;
%~~vis
\begin{xten}
class Get[+T] {
  val x: T;
  def this(x: T) { this.x = x; }
  def get(): T = x;
  static def example() {
     val g : Get[Int]! = new Get[Int](31);
     val n : Int = g.get();
     x10.io.Console.OUT.print("It's " + n);
     x10.io.Console.OUT.print("It's still " + g.get());
  }
}
\end{xten}
%~~siv
%~~neg


A typical example of contravariance is \xcd`Set`.  As the \xcd`example()`
method shows,  a variety of objects\footnote{Objects but no structs.  If we
had wanted structs too, we could have used a \xcd`Cell[Any]`.}  can be put into a
\xcd`Set[Object]`.  While the object itself cannot be retrieved, some summary
information about it -- in this case, its \xcd`typeName` -- can be.  
%~~gen
% package ch4;
%~~vis
\begin{xten}
class Set[-T] {
  var x: T;
  def this(x: T) { this.x = x; }
  def set(x: T) = { this.x = x; } 
  def summary(): String = this.x.typeName();
  static def example() {
    val s : Set[Object]! = new Set[Object](new Throwable());
    s.summary(); // == "x10.lang.Throwable"
    s.set("A String");
    s.summary(); // == "x10.lang.String";
  }    
}
\end{xten}
%~~siv
%
%~~neg


Given types \xcd"S" and \xcd"T": 
\begin{itemize}
\item
If the parameter of \xcd"Get" is covariant, then
\xcd"Get[S]" is a subtype of \xcd"Get[T]" if
\xcd"S" is a {\em subtype} of \xcd"T".

\item
If the parameter of \xcd"Set" is contravariant, then
\xcd"Set[S]" is a subtype of \xcd"Set[T]" if
\xcd"S" is a {\em supertype} of \xcd"T".

\item
If the parameter of \xcd"Cell" is invariant, then
\xcd"Cell[S]" is a subtype of \xcd"Cell[T]" if
\xcd"S" is a {\em equal} to \xcd"T".
\end{itemize}


In order to make types marked as covariant and contravariant semantically
sound, X10 performs extra checks.  
A covariant type parameter is permitted to appear only in covariant type positions,
and a contravariant type parameter in contravariant positions. 
\begin{itemize}
\item The return type of a method is a covariant position.
\item The argument types of a method are contravariant positions.
\item Whether a type argument position of a generic class, interface or struct type \Xcd{C}
is covariant or contravariant is determined by the \Xcd{+} or \Xcd{-} annotation
at that position in the declaration of \Xcd{C}.
\end{itemize}

There are similar restrictions on use of covariant and contravariant values. 
\bard{Get them!  What are they?}


\section{Function Types}

For every sequence of types \xcd"T1,..., Tn,T", and \xcd"n" distinct variables
\xcd"x1,...,xn" and constraint \xcd"c", the expression
\xcd"(x1:T1,...,xn:Tn){c}=>T" is a \emph{function type}. It stands for
 the set of all functions \xcd"f" which can be applied to a
 list of values \xcd"(v1,...,vn)" provided that the constraint
 \xcd"c[v1,...,vn,p/x1,...,xn]" is true, and which returns a value of
 type \xcd"T[v1,...vn/x1,...,xn]". When \xcd"c" is true, the clause \xcd"{c}" can be
 omitted. When \xcd"x1,...,xn" do not occur in \xcd"c" or \xcd"T", they can be
 omitted. Thus the type \xcd"(T1,...,Tn)=>T" is actually shorthand for
 \xcd"(x1:T1,...,xn:Tn){true}=>T", for some variables \xcd"x1,...,xn".




\section{Type definitions}
\label{TypeDefs}

\index{types!type definitions}
\index{declarations!type definitions}

\section{Type definitions}

With value arguments, type arguments, and constraints, the
syntax for \Xten{} types can often be verbose;
\Xten{} therefore provides {\em type definitions}
to allow aliases to be defined for types.
Type definitions have the following syntax:

\begin{grammar}
TypeDefinition \: 
                \xcd"type"~Identifier
                           ( \xcd"[" TypeParameters \xcd"]" )\opt \\
                        && ( \xcd"(" Formals \xcd")" )\opt
                            Constraint\opt \xcd"=" Type \\
\end{grammar}

\noindent
A type definition can be thought of as a type-valued function,
mapping type parameters and value parameters to a concrete type.
%
The following examples are legal type definitions:
\begin{xten}
type StringSet = Set[String];
type MapToList[K,V] = Map[K,List[V]];
type Nat = Int{self>=0};
type Int(x: Int) = Int{self==x};
type Int(lo: Int, hi: Int) = Int{lo <= self, self <= hi};
\end{xten}

As the two definitions of \xcd"Int" demonstrate, type definitions may 
be overloaded: a type definition with a different number of type
parameters or with different types of value
parameters---according to the method overloading rules
(\Sref{MethodOverload}) define distinct types.

Type definitions may appear as class members or in the body of a
method, constructor, or initializer.  Type definitions that are
members of a class are \xcd"static"; type properties can be used
for non-static type definitions.

Type definitions are applicative, not generative; that is, they
are define aliases for types and do not introduce new types.
Thus, the following code is legal:
\begin{xten}
type A = Int;
type B = String;
type C = String;
a: A = 3;
b: B = new C("Hi");
c: C = b + ", Mom!";
\end{xten}
A type defined by a type definition
has the same constructors as its defining type; however, a
constructor may not be invoked using a given type definition
name if the constructor return type is not a subtype of the
defined type.

All type definitions are members of their enclosing package or
class.  A compilation unit may have one or more type definitions
or class or interface declarations with the same name, as long
as the types are unique by overloading.




\section{Constrained types}
\label{ConstrainedTypes}
\label{DepType:DepType}
\label{DepTypes}

\index{dependent types}
\index{constrained types}
\index{generic types}
\index{types!constrained types}
\index{types!dependent types}
\index{types!generic types}


Basic types, like \xcd`Int` and \xcd`List[String]`, provide useful
descriptions of data.  Indeed, most typed programming languages get by with no
more specific descriptions.

However, there are a lot of things that one frequently wants to say about
data.  One might want to know that a \xcd`String` variable is not \xcd`null`,
or that a matrix is square, or that one matrix has the same number of columns
that another has rows (so they can be multiplied).  In the multicore setting,
one might wish to know that two values are located at the same processor, or
that one is located at the same place as the current computation.

In most languages, there is simply no way to say these things statically.
Programmers must made do with comments, \xcd`assert` statements, and dynamic
tests.  X10 can do better, with {\em constraints} on types (and methods and
other things).

A constraint is a boolean expression \xcd`e` attached to a basic type \xcd`T`,
written \xcd`T{e}`.  (Only a limited selection of boolean expressions is
available.)  The values of type \xcd`T{e}` are the values of \xcd`T` for which
\xcd`e` is true.  For example: 

\begin{itemize}
%~~type~~`~~`~~ ~~
\item \xcd`String{self != null}` is the type of non-null strings.  \xcd`self`
      is a special variable available only in constraints; it refers to the
      datum being constrained.   
\item If \xcd`Matrix` has properties \xcd`rows` and \xcd`cols`, 
%~~type~~`~~`~~ ~~class Matrix(rows:Int,cols:Int){}
      \xcd`Matrix{rows == cols}` is the type of square matrices.
\item One way to say that \xcd`a` has the same number of columns that \xcd`b`
      has rows (so that \xcd`a*b` is a valid matrix product), one could say: 
%~~gen
% package Types.cripes.whered.you.get.those.gripes;
% class Matrix(rows:Int, cols:Int){
% public static def someMatrix(): Matrix = null;
% public static def example(){
%~~vis
\begin{xten}
  val a : Matrix = someMatrix() ;
  var b : Matrix{b.rows == a.cols} ;
\end{xten}
%~~siv
%}}
%~~neg

\item Every object has a \xcd`home` property telling where it is located.
      One way to say that objects \xcd`c` and \xcd`d` are located at the same
      place is: 
%~~gen
% package Types.flipes.knipes.shipes.wipes;
% class Exampler {
% static def someObject(): Object = null;
% static def example() {
%~~vis
\begin{xten}
  val a : Object = someObject();
  var b : Object{a.home == b.home};
\end{xten}
%~~siv
%}}
%~~neg

\item  As explained in \Sref{XtenPlaces}, certain operations
can only be performed at an object's home, so having this expressible as a
type is crucial. One way to say that \xcd`e` is located \xcd`here`, \viz, the same place
      as the current computation, is: 
%~~gen
% package Types.cripes.wipes.lipes;
% class Examplorator {
% static def someObject() : Object{self.home==here} = null;
% static def example() {
%~~vis
\begin{xten}
  val e : Object{self.home == here} = someObject();
\end{xten}
%~~siv
%}}
%~~neg
\end{itemize}

\index{self}When constraining a value of type \xcd`T`, \xcd`self` refers to the object of
type \xcd`T` which is being constrained.  For example, \xcd`Int{self == 4}` is
the type of \xcd`Int`s which are equal to 4 -- the best possible description
of \xcd`4`, and a very difficult type to express without using \xcd`self`.  






\xcd"T{e}" is a {\em dependent type}, that is, a type dependent on values. The
type \xcd"T" is called the {\em base type} and \xcd"e" is called the {\em
  constraint}. If the constraint is omitted, it is \xcd`true`---that is, the
  base type is unconstrained.

Constraints may refer to values in the local environment: 
%~~gen
% class ConstraintsMayReferToValues {
% def thoseValues() {
%~~vis
\begin{xten}
     val n = 1;
     var p : Point{rank == n};
\end{xten}
%~~siv
%}}
%~~neg
Indeed, there is technically no need for a constraint to refer to the
properties of its type; it can refer entirely to the environment, thus: 
%~~gen
% class ConstraintsMayReferToValuesTwo {
% def thoseValues() {
%~~vis
\begin{xten}
     val m = 1;
     val n = 2;
     var p : Point{m != n};
\end{xten}
%~~siv
%}}
%~~neg

Constraints on properties induce a natural subtyping relationship:
\xcd"C{c}" is a subtype of
\xcd"D{d}" if \xcd"C" is a subclass of \xcd"D" and
\xcd"c" entails \xcd"d".
For example: 
\begin{itemize}
\item \xcd`Int{self == 3} <: Int{self != 14}`.  The only value of
      \xcd`Int{self ==3}` is 3.  All integers but 14 are members of
      \xcd`Int{self != 14}`, and in particular 3 is.  
\item Suppose we have classes \xcd`Child <: Person`, and \xcd`Person` has a
      long \xcd`ssn` property.  If \xcd`rhys : Child{ssn == 123456789}`, then
      \xcd`rhys` is also a \xcd`Person` and still has \xcd`ssn==123456789`, so 
      \xcd`rhys : Person{ssn==123456789}` as well.  
      So, \xcd`Child{ssn == 123456789} <: Person{ssn == 123456789}`.
\item Furthermore, since \xcd`123456789 != 555555555`, 
      \xcd`rhys : Person{ssn != 555555555}`.  
      So, \xcd`Child{ssn == 123456789} <: Person{ssn != 555555555}`.  
\item \xcd`T{e} <: T` for any type \xcd`T`.  That is, if you have a value
      \xcd`v` of some base type \xcd`T` which satisfied \xcd`e`, then \xcd`v`
      is of that base type \xcd`T` (with the constraint ignored).
\item If \xcd`A <: B`, then \xcd`A{c} <: B{c}` for every constraint \xcd`{c}`
      for which \xcd`A{c}` and \xcd`B{c}` are defined.  That is, if every
      \xcd`A` is also a \xcd`B`, and \xcd`a : A{c}`, then 
      \xcd`a` is an \xcd`A` and \xcd`c` is true of it. So \xcd`a` is also a
      \xcd`B` (and \xcd`c` is still true of 
      it), so \xcd`a : B{c}`.  

\end{itemize}


\subsection{Constraint Expressions}




Only a few kinds of expressions can appear in constraints.  For fundamental
reasons of mathematical logic, the more kinds of expressions that can appear
in constraints, the harder it is to compute the essential properties of
constrained type -- in particular, the harder it is to compute 
\xcd`A{c} <: B{d}`.  It doesn't take much to make this basic fact undecidable.
In order to
make sure that it stays decidable, X10 places quite stringent restrictions on
constraints.  

Only the following forms of expression are allowed in constraints.  

{\bf Value expressions in constraints} may be: 
\begin{enumerate}
\item Literal constants, like \xcd`3` and \xcd`true`;
\item Accessible and immutable variables and parameters;
\item Accessible and immutable fields of the containing object;
\item Properties of the type being constrained;
\item Property methods;
\item \xcd`this`, if the constraint is in a place where \xcd`this` is defined.
\item \xcd`here`
\item \xcd`self`
\end{enumerate}


{\bf Boolean expressions in constraints} may be any of the following, where
all value expressions are of the forms which may appear in constraints: 
\begin{enumerate}
\item Equalities \xcd`e == f`,
\item Inequalities of the form \xcd`e != f`.\footnote{Currently inequalities
      of the form \xcd`e < f` are not supported.}
\item Conjunctions of Boolean expressions
\item Subtyping and supertyping expressions: \xcd`T <: U` and \xcd`T :> U`. 
\item Type equalities and inequalities: \xcd`T == U` and \xcd`T != U`.
\item Boolean value expressions.
\end{enumerate}

All variables appearing in a constraint expression must be visible wherever
that expression can used.  \Eg, properties and public fields of an object are
always permitted, but private fields of an object can only constrain private
members.  (Consider a class \xcd`PriVio` with a private field \xcd`p` and a
public method \xcd`m(x: Int{self != p})`, and a call \xcd`ob.m(10)` made
outside of the class. Since \xcd`p` is only visible inside the class, there is
no way to tell if \xcd`10` is of type \xcd`Int{self != p}` at the call site.)




%%TYPES-CONSTR-EXP%% 
%%TYPES-CONSTR-EXP%% \begin{grammar}
%%TYPES-CONSTR-EXP%% Constraint \: ValueArguments     Guard\opt \\
%%TYPES-CONSTR-EXP%%            \| ValueArguments\opt Guard     \\
%%TYPES-CONSTR-EXP%%            \\
%%TYPES-CONSTR-EXP%% ValueArguments   \:  \xcd"(" ArgumentList\opt \xcd")" \\
%%TYPES-CONSTR-EXP%% ArgumentList     \:  Expression ( \xcd"," Expression )\star \\
%%TYPES-CONSTR-EXP%% Guard            \: \xcd"{" DepExpression \xcd"}" \\
%%TYPES-CONSTR-EXP%% DepExpression    \: ( Formal \xcd";" )\star ArgumentList \\
%%TYPES-CONSTR-EXP%% \end{grammar}
%%TYPES-CONSTR-EXP%% 
%%TYPES-CONSTR-EXP%% In \XtenCurrVer{} value constraints may be equalities (\xcd"=="),
%%TYPES-CONSTR-EXP%% disequalities (\xcd"!=") and conjunctions thereof.  The terms over
%%TYPES-CONSTR-EXP%% which these constraints are specified include literals and
%%TYPES-CONSTR-EXP%% (accessible, immutable) variables and fields, property methods, and the special
%%TYPES-CONSTR-EXP%% constants {\tt here}, {\tt self}, and {\tt this}. Additionally, place
%%TYPES-CONSTR-EXP%% types are permitted (\Sref{PlaceTypes}).
%%TYPES-CONSTR-EXP%% 
%%TYPES-CONSTR-EXP%% \index{self}

%%TYPES-CONSTR-EXP%% Type constraints may be subtyping and supertyping (\xcd"<:" and
%%TYPES-CONSTR-EXP%% \xcd":>") expressions over types.

The static constraint checker approximates computational reality in some
cases.  For example, it assumes that built-in types are infinite. This is a
good approximation for \xcd`Int`.  It is a poor approximation for \xcd`Boolean`,
as the checker believes that \xcd`a != b && a != c && b != c` is satisfiable
over \xcd`Boolean`, which it is not.  However, the checker is always correct
when computing the truth or falsehood of a constraint.


% //, and existential quantification over typed variables.

%%TYPES-CONSTR-EXP%% \emph{
%%TYPES-CONSTR-EXP%% Subsequent implementations are intended to support boolean algebra,
%%TYPES-CONSTR-EXP%% arithmetic, relational algebra, etc., to permit types over regions and
%%TYPES-CONSTR-EXP%% distributions. We envision this as a major step towards removing most,
%%TYPES-CONSTR-EXP%% if not all, dynamic array bounds and place checks from \Xten{}.
%%TYPES-CONSTR-EXP%% }


\subsubsection{Acyclicity restriction}

To ensure that type-checking is decidable, we require that property graphs be
acyclic.  The property graph, at an instant in an X10 execution, is the graph
whose nodes are all objects in existence at that instance, with an edge from
{$x$} to {$y$} if {$x$} is an object with a property whose value is {$y$}{. 

\bard{I think that \xcd`proto` handles this now ...? }

Currently this restriction is not checked by the compiler. Future
versions of the compiler will check this restriction by introducing
rules on escaping of \Xcd{this} (\Sref{protorules}) before the invocation of
\Xcd{property} calls.


\subsection{Place constraints}
\label{PlaceTypes}
\label{PlaceType}
\index{place types}
\label{DepType:PlaceType}\index{placetype}

An \Xten{} computation spans multiple places (\Sref{XtenPlaces}). Much data
can only be accessed from the proper place, and often it is preferable to
determine this statically. So, X10 has special syntax for working with places.
\xcd`T!` is a value of type \xcd`T` located at the right place for the current
computation, and \xcd`T!p` is one located at place \xcd`p`.

\begin{grammar}
PlaceConstraint     \: \xcd"!" Place\opt \\
Place              \:   Expression \\
\end{grammar}

More specifically, All \Xten{} classes extend the class \xcd"x10.lang.Object",
which defines a property \xcd"home" of type \xcd"Place".  \xcd`T!p`, when
\xcd`T` is a class, is \xcd`T{self.home==p}`.  If \xcd`p` is omitted, it
defaults to \xcd`here`.   \xcd`T!` is far and away the most common usage of
\xcd`!`. 

Structs don't have \xcd`home`; they are available everywhere.  For structs, 
\xcd`T!` and \xcd`T!p` are synonyms for \xcd`T`. Since \xcd`T` is available
everywhere, it is available \xcd`here` and at \xcd`p`. 

\xcd`!` may be combined with other constraints.  \xcd`T{c}!` is the type of
values of \xcd`T!` which satisfy \xcd`c`; it is \xcd`T{c && self.home==here}`
for an object type and \xcd`T{c}` for a struct type.  
\xcd`T{c}!p` is the type of
values of \xcd`T!p` which satisfy \xcd`c`; it is \xcd`T{c && self.home==p}`
for an object type and \xcd`T{c}` for a struct type.  



% The place specifier \xcd"any" specifies that the object can be
% located anywhere.  Thus, the location is unconstrained; that is,
% \xcd"C{c}!any" is equivalent to \xcd"C{c}".

% XXX ARRAY
%The place specifier \xcd"current" on an array base type
%specifies that an object with that type at point \xcd"p"
%in the array 
%is located at \xcd"dist(p)".  The \xcd"current" specifier can be
%used only with array types.




\subsection{Variables in Constraints}

X10 permits a \xcd`val` variable to appear in constraints on its own type as
%~~stmt~~`~~`~~ ~~
it is being declared.  For example, \xcd`val nz: Int{nz != 0} = 1;` declares a
non-zero variable \xcd`nz`.
\bard{This will need to be explained further once the language issues are
sorted out..}



%%TYPES-CONSTR-EXP%% We permit variable declarations \xcd"v: T" where \xcd"T" is obtained
%%TYPES-CONSTR-EXP%% from a dependent type \xcd"C{c}" by replacing one or more occurrences
%%TYPES-CONSTR-EXP%% of \xcd"self" in \xcd"c" by \xcd"v". (If such a declaration \xcd"v: T"
%%TYPES-CONSTR-EXP%% is type-correct, it must be the case that the variable \xcd"v" is not
%%TYPES-CONSTR-EXP%% visible at the type \xcd"T". Hence we can always recover the
%%TYPES-CONSTR-EXP%% underlying dependent type \xcd"C{c}" by replacing all occurrences of \xcd"v"
%%TYPES-CONSTR-EXP%% in the constraint of \xcd"T" by \xcd"self".)
%%TYPES-CONSTR-EXP%% 
%%TYPES-CONSTR-EXP%% For instance, \xcd"v: Int{v == 0}" is shorthand for \xcd"v: Int{self == 0}".
%%TYPES-CONSTR-EXP%% 
%%TYPES-CONSTR-EXP%% 
%%TYPES-CONSTR-EXP%% A variable occurring in the constraint \xcd"c" of a dependent type, other than
%%TYPES-CONSTR-EXP%% \xcd"self" or a property of \xcd"self", is said to be a {\em
%%TYPES-CONSTR-EXP%% parameter} of \xcd"c".\label{DepType:Parameter} \index{parameter}

\subsection{Operations on Constrained Types}

An instance \xcd"o" of \xcd"C" is said to be of type \xcd"C{c}" (or: {\em
belong to} \xcd"C{c}") if the constraint \xcd"c" evaluates to \xcd"true" in
the current lexical environment, augmented with the binding \xcd"self"
$\mapsto$ \xcd"o".

The \xcd`instanceof` operation lets programs test type membership.  
\xcd`e instanceof C{c}` returns true if \xcd`e` belongs to \xcd`C{c}`, and
false otherwise.  
%~~exp~~`~~`~~ ~~
\xcd`1 instanceof Int{self != 2}` returns true, 
and 
%~~exp~~`~~`~~x:Int ~~
\xcd`x instanceof Int{self == 1}` returns false if 
\xcd`x==2`.

However, it is a static error if \xcd`e` cannot possibly be an instance of
\xcd`C{c}`; the compiler will reject \xcd`1 instanceof Int{self == 2}` because
\xcd`1` can never satisfy \xcd`Int{self == 2}`. Similarly, \Xcd{1 instanceof
String} is a static error, rather than an expression always returning false. 


\limitation{
X10 does not currently handle \xcd`instanceof` of generics in the way you
%~~exp~~`~~`~~r:ValRail[Int] ~~
might expect.  For example, \xcd`r instanceof ValRail[Int{self != 0}]` does
not test that every element of \xcd`r` is non-zero.
}

The \xcd`as` operation attempts to convert a value to a given constrained
type, as described in \Sref{ClassCast}.   As with \xcd`instanceof`, it may
succeed, fail with a dynamic error, or, in the case where the cast is
impossible, fail to compile.


\subsection{Example of Constraints}

\begin{example}

Constraints can be used to express simple relationships between objects,
enforcing some class invariants statically.  For example, in geometry, a line
is determined by two {\em distinct} points; a \xcd`Line` class can specify the
distinctness in a type constraint:\footnote{We call them
\xcd`Position` to avoid confusion with the built-in class \xcd`Point`}

%~~gen
% 
%~~vis
\begin{xten}
class Position(x: Int, y: Int) {
   def this(x:Int,y:Int){property(x,y);}
   }
class Line(start: Position, 
           end: Position{self != start}) {}
\end{xten}

%~~siv
%~~neg

Extending this concept, a \xcd`Triangle` can be defined as a figure with three
line segments which match up end-to-end.  Note that the degenerate case in
which two or three of the triangle's vertices coincide is excluded by the
constraint on \xcd`Line`.  However, not all degenerate cases can be excluded
by the type system; in particular, it is impossible to check that the three
vertices are not collinear. 

%~~gen
%package triangleExample;
% class Position(x: Int, y: Int) {
%    def this(x:Int,y:Int){property(x,y);}
%    }
% class Line(start: Position, 
%            end: Position{self != start}) {}
% 
%~~vis
\begin{xten}
class Triangle 
 (a: Line, 
  b: Line{a.end == b.start}, 
  c: Line{b.end == c.start && c.end == a.start})  {
   def this(a:Line,
            b: Line{a.end == b.start}, 
            c: Line{b.end == c.start && c.end == a.start}) 
   {property(a,b,c);}
 }
\end{xten}
%~~siv
%
%~~neg



\end{example}

\section{Function types}
\label{FunctionTypes}
\label{FunctionType}
\index{function!types}
\index{types!function types}

X10 functions, like mathematical functions, take some arguments and produce a
result.  X10 functions, like other X10 code, can change mutable state and
throw exceptions.  Closures (\Sref{Closures}) and method
selectors (\Sref{MethodSelectors}) are of function type.
Typical functions are the reciprocal function: 
%~~gen
% package Types.Functions;
% class RecipEx {
% static 
%~~vis
\begin{xten}
val recip = (x : Double) => 1/x;
\end{xten}
%~~siv
%}
%~~neg
and a function which increments  element \xcd`i` of a rail \xcd`r`, or throws an exception
if there is no such element, where, for the sake of example, we constrain the
type of \xcd`i`:  
%~~gen
% package Types.Functions;
% class IncrElEx {
% static def example() throws DoomExn {
%~~vis
\begin{xten}
val inc = (r:Rail[Int]!, i: Int{i != r.length}) => {
  if (i < 0 || i >= r.length) throw new DoomExn();
  r(i)++;
};
\end{xten}
%~~siv
%}}
%class DoomExn extends Exception{}
%~~neg

So, in general, a function type needs to list the types 
\xcdmath"T$_i$"
of all the formal parameters,
and their distinct names \xcdmath"x$_i$" in case other types refer to them; a
constraint 
\xcd"c" on the
function as a whole; a return type \xcd"T"; and the exceptions \xcdmath"EX$_j$"
that the function might throw when applied: 


\limitation{The \xcd`throws` clause is not currently implemented.  Also,
some method modifiers (\xcd`safe`, \xcd`atomic`, etc.) will apply to function
types as well.}

\begin{xtenmath}
(x$_1$: T$_1$, $\dots$, x$_n$: T$_n$){c} => T
        throws EX$_1$, $\dots$, EX$_k$
\end{xtenmath}


The names \xcdmath"x$_i$" of the formal parameters are not relevant.  Types
which differ only in the names of formals (following the usual rules for
renaming of variables, as in {$\alpha$}-renaming in the {$\lambda$} calculus
\bard{cite something}) are considered equal.  \Eg, 
%~~type~~`~~`~~ ~~
\xcd`(a:Int, b:Rail[String]{b.length==a}) => Boolean`
and 
%~~type~~`~~`~~ ~~
\xcd`(b:Int, a:Rail[String]{a.length==b}) => Boolean`
are equivalent types.

The formal parameter names are in scope from the point of definition to the
end of the function type---they may be used in the types of other formal parameters
and in the return type. 
Value parameters names may be
omitted if they are not used; the type of the reciprocal function can be
written as
%~~type~~`~~`~~ ~~
\xcd`(Double)=>Double`. 

\begin{grammar}
FunctionType \: TypeParameters\opt \xcd"(" Formals\opt \xcd")" Constraint\opt
\xcd"=>" Type Throws\opt \\
TypeParameters \: \xcd"[" TypeParameter ( \xcd"," TypeParameter
)\star \xcd"]" \\
TypeParameter \: Identifier \\
Formals \: Formal ( \xcd"," Formal )\star \\
\end{grammar}






A function type is covariant in its result type and contravariant in
each of its argument types. That is, let 
\xcd"S1,...,Sn,S,T1,...Tn,T" be any
types satisfying \xcd"Si <: Ti" and \xcd"S <: T". Then
\xcd"(x1:T1,...,xn:Tn){c}=>S" is a subtype of
\xcd"(x1:S1,...,xn:Sn){c}=>T".



A class or struct definition may use a function type 
\xcd"F = (x1:T1,...,xn:Tn){c}=>T" in its 
implements clause; 
this is equivalent to implementing an interface requiring the single method
\xcd"def apply(x1:T1,...,xn:Tn){c}:T". 
Similarly, an interface
definition may specify a function type \xcd"F" in its \xcd"extends" clause.
Values of a class or struct implementing \xcd`F` 
can be used as functions of type \xcd`F` in all ways.  
In particular, applying one to suitable arguments calls the \xcd`apply`
method. 



A function type \xcd"F" is not a class type in that it does not extend any
type or implement any interfaces, or support equality tests. 
\xcd`F` may be implemented, but not extended, by a class or function type. 
Nor is it a struct type, for it has no predefined notion of equality.




\section{Annotated types}
\label{AnnotatedTypes}

\index{types!annotated types}
\index{annotations!type annotations}

        Any \Xten{} type may be annotated with zero or more
        user-defined \emph{type annotations}
        (\Sref{XtenAnnotations}).  

        Annotations are defined as (constrained) interface types and are
        processed by compiler plugins, which may interpret the
        annotation symbolically.

        A type \xcd"T" is annotated by interface types
        \xcdmath"A$_1$", \dots,
        \xcdmath"A$_n$"
        using the syntax
        \xcdmath"@A$_1$ $\dots$ @A$_n$ T".

\section{Subtyping and type equivalence}\label{DepType:Equivalence}
\index{type equivalence}
\index{subtyping}

Intuitively, type \xcdmath"T$_1$" is a subtype of type \xcdmath"T$_2$", 
written \xcdmath"T$_1$ <: T$_2$", 
if
every instance of \xcdmath"T$_1$" is also an instance of \xcdmath"T$_2$".  For
example, \xcd`Child` is a subtype of \xcd`Person` (assuming a suitably defined
class hierarchy): every child is a person.  Similarly, \xcd`Int{self != 0}`
is a subtype of \xcd`Int` -- every non-zero integer is an integer.  

This section formalizes the concept of subtyping. Subtyping of types depends
on a {\em type context}, \viz. a set of constraints which may say something
about types.   For example: 

%~fails~gen
% package Types.subtyping.cons;
%~fails~vis
\begin{xten}
class ConsTy[T,U] {
   def upcast(t:T){T <: U} :U = t;
}
\end{xten}
%~fails~siv
%
%~fails~neg
\noindent
Inside \xcd`upcast`, \xcd`T` is constrained to be a subtype of \xcd`U`, and so
\xcd`T <: U` is true, and \xcd`t` can be treated as a value of type \xcd`U`.  
Outside of \xcd`upcast`, there is no reason to expect any relationship between
them, and \xcd`T <: U` may be false.
However, subtyping of types that have no free variables does not depend
on the context.    \xcd`Int{self != 0} <: Int` is always
true, regardless of what else is going on.

\limitation{Subtyping of type variables does not currently work.}



\begin{itemize}
\item {\bf Reflexivity:} Every type \xcd`T` is a subtype of itself: \xcd`T <: T`.

\item {\bf Transitivity:} If \xcd`T <: U` and \xcd`U <: V`, then \xcd`T <: V`. 

\iffalse
{\bf Class types:}  
Given the definition 
\xcd`class C[$\vec{X}$] extends D[$\vec{Y}$]{d} implements I1, ..., In {...}`
where {$\vec{X}$} is a vector of type variables, and 
{$\vec{Y$} a vector of types possibly involving variables from {$\vec{X}$}, 
and {$\vec{T$} an instantiation of {$\vec{X$} and {$\vec{U$} the corresponding
instantiation of {$\vec{Y$}, 
then 
\xcdmath"C[$\vec{T}$]`"is a subtype of \xcd`D[$\vec{U}$]{d}`, \xcd`I1`, ..., \xcd`In`. 

\item
{\bf Interface types:}  
Given the definition 
\xcdmath"interface I[$\vec{X}$] extends I1, ... In {...}`"
then \xcdmath"I` is a subtype of \xcd`"1`, ..., \xcd`In`.

\item 
{\bf Struct types:} 
Given the definition 
\xcdmath"struct S implements I1, ..., In {...}`"then \xcd`S` is a 
subtype of \xcd`I1`, ..., \xcd`In`. 
\fi

\item {\bf Direct Subclassing:} 
Let {$\vec{X}$} be a (possibly empty) vector of type variables, and
{$\vec{Y}$}, {$\vec{Y_i}$} be vectors of type terms over {$\vec{X}$}.
Let {$\vec{T}$} be an instantiation of {$\vec{X}$}, 
and {$\vec{U}$}, {$\vec{U_i}$} the corresponding instantiation of 
{$\vec{Y}$}, {$\vec{Y_i}$}.  Let \xcd`c` be a constraint, and \xcdmath"c$'$"
be the corresponding instantiation.
We elide properties, and interpret empty vectors as absence of the relevant
clauses. 
Suppose that \xcd`C` is declared by one of the
forms: 
\begin{enumerate}
\item \xcdmath"class C[$\vec{X}$]{c} extends D[$\vec{Y}$]{d} implements I$_1[\vec{Y_1}]${i$_1$},...,I$_n[\vec{Y_n}]${i$_n$}{"
\item \xcdmath"interface C[$\vec{X}$]{c} extends I$_1[\vec{Y_1}]${i$_1$},...,I$_n[\vec{Y_n}]${i$_n$}{"
\item \xcdmath"struct C[$\vec{X}$]{c} implements I$_1[\vec{Y_1}]${i$_1$},...,I$_n[\vec{Y_n}]${i$_n$}{"
\end{enumerate}
Then: 
\begin{enumerate}
\item \xcdmath"C[$\vec{T}$] <: D[$\vec{U}$]{d}" for a class
\item \xcdmath"C[$\vec{T}$] <: I$_i$[$\vec{U_i}$]{i$_i$}" for all cases.
\item \xcdmath"C[$\vec{T}$] <: C[$\vec{T}$]{c$'$}" for all cases.
\end{enumerate}


\item
{\bf Function types:}
\xcdmath"(x$_1$: T$_1$, $\dots$, x$_n$: T$_n$){c} => T throws EX$_1$, $\dots$,EX$_k$"
is a  subtype of 
\xcdmath"(x$'_1$: T$'_1$, $\dots$, x$'_n$: T$'_n$){c$'$} => T$'$ throws EX$'_1$, $\dots$, EX$'_{k'}$"
if: 
\begin{enumerate}
\item Each \xcdmath"T$_i$ <: T$'_i$";
\item \xcd`c` entails \xcdmath"c$'$";
\item \xcdmath"T$'$ <: T";
\item Each  \xcdmath"EX$_j$ <: EX$'_{j'}$".
\end{enumerate}

\item
{\bf Constrained types:}
\xcd`T{c}` is a subtype of \xcd`T{d}` if \xcd`c` entails \xcd`d`. 

\item {\bf Any:} 
Every type \xcd`T` is a subtype of \xcd`x10.lang.Any`.

\item 
{\bf Type Variables:}
Inside the scope of a constraint \xcd`c` which entails \xcd`A <: B`, we have
\xcd`A <: B`.  \eg, \xcd`upcast` above.


\item 
{\bf Covariant Generic Types:} 
If \xcd`C` is a generic type whose {$i$}th type parameter is covariant, 
and {\xcdmath"T$'_i$ <: T$_i$"}
and  {\xcdmath"T$'_j$ == T$_j$"} for all {$j \ne i$}, 
then {\xcdmath"C[T$'_1$, $\ldots$, T$'_n$] <: C[T$'_1$, $\ldots$, T$'_n$]"}.
\Eg, \xcd`class C[T1, +T2, T3]` with {$i=2$}, and \xcd"U2 <: T2", then
\xcd`C[T1,U2,T3] <: C[T1,T2,T3]`.

\item 
{\bf Contravariant Generic Types:} 
If \xcd`C` is a generic type whose {$i$}th type parameter is contravariant, 
and \xcdmath"T$'_i$ <: T$_i$"
and  \xcdmath"T$'_j$ == T$_j$" for all {$j \ne i$}, 
then \xcdmath"C[T$'_1$, $\ldots$, T$'_n$] :> C[T$'_1$, $\ldots$, T$'_n$]".
\Eg, \xcd`class C[T1, -T2, T3]` with {$i=2$}, and \xcdmath"U2 <: T2", then
\xcd`C[T1,U2,T3] :> C[T1,T2,T3]`.

\end{itemize}


Two types are {\em equivalent}, \xcd`T == U`, if \xcd`T <: U` and \xcd`U <: T`. 


\section{Common ancestors of types}
\label{LCA}

There are several situations where X10 must find a type \xcd`T` that describes
values of two or more different types.  This arises when X10 is trying to find
a good type to describe: 
\begin{itemize}
%~~exp~~`~~`~~test:Boolean ~~
\item Conditional expressions, like \xcd`test ? 0 : "non-zero"` or even 
%~~exp~~`~~`~~test:Boolean ~~
      \xcd`test ? 0 : 1`;
%~~exp~~`~~`~~ ~~
\item ValRail construction, like \xcd`[0, "non-zero"]` and 
%~~exp~~`~~`~~ ~~
      \xcd`[0,1]`;
\item Functions with multiple returns, like
%~~gen
% package Types.For.Gripes.About.Pipes;
% class Examplerator {
%~~vis
\begin{xten}
def f(a:Int) {
  if (a == 0) return 0;
  else return "non-zero";
}
\end{xten}
%~~siv
%}
%~~neg
\end{itemize}

In some cases, there is a unique best type describing the expression.  For
example, if \xcd`B` and \xcd`C` are direct subclasses of \xcd`A`, \xcd`pick`
will have return type \xcd`A`: 
%~~gen
% package Types.For.Gripes.About.Pipes;
%  class A {} class B extends A{} class C extends A{}
% class D {
%~~vis
\begin{xten}
static def pick(t:Boolean, b:B, c:C) = t ? b : c;  
\end{xten}
%~~siv
%}
%~~neg

However, in many common cases, there is no unique best type describing the
expression.  For example, consider the expression {$E=$} \xcd`b ? 0 : 1`.  The
best type of \xcd`0` 
is \xcd`Int{self==0}`, and the best type of 1 is \xcd`Int{self==1}`.
Certainly {$E$} could be given the type \xcd`Int`, or even \xcd`Any`, and that
would describe all possible results.  However, we actually know more.
\xcd`Int{self != 2}` is a better description of the type of {$E$}---certainly
the result of {$E$} can never be \xcd`2`.   \xcd`Int{self != 2, self != 3}` is
an even better description; {$E$} can't be \xcd`3` either.  We can continue
this process forever, adding integers which {$E$} will definitely not return
and getting better and better approximations. (If the constraint
sublanguage had \xcd`||`, we could give it the type 
\xcd`Int{self == 0 || self == 1`, which would be nearly perfect.  But 
\xcd`||` makes typechecking far more expensive, so it is excluded.)
No X10 type is the best description of {$E$}; there is always a better one.

Similarly, consider two unrelated interfaces: 
%~~gen
% package Types.For.Gripes.About.Snipes;
%~~vis
\begin{xten}
interface I1 {}
interface I2 {}
class A implements I1, I2 {}
class B implements I1, I2 {}
class C {
  static def example(t:Boolean, a:A, b:B) = t ? a : b;
}
\end{xten}
%~~siv
%
%~~neg
\xcd`I1` and \xcd`I2` are both perfectly good descriptions of \xcd`t ? a : b`, 
but neither one is better than the other, and there is no single X10 type
which is better than both. (Some languages have {\em conjunctive
    types}, and could say that the return type of \xcd`example` was 
\xcd`I1 && I2`.  This, too, complicates typechecking.)
}

So, when confronted with expressions like this, X10 computes {\em some}
satisfactory type for the expression, but not necessarily the {\em best} type.  
X10 provides certain guarantees about the common type \xcd`V{v}` computed for 
\xcd`T{t}` and \xcd`U{u}`: 
\begin{itemize}
\item If \xcd`T{t} == U{u}`, then \xcd`V{v} == T{t} == U{u}`.  So, if X10's
      algorithm produces an utterly untenable type for \xcd`a ? b : c`, and
      you want the result to have type \xcd`T{t}`, you can 
      (in the worst case) rewrite it to 
      \xcd`a ? b as T{t} : c as T{t}`.
\item If \xcd`T == U`, then \xcd`V == T == U`.  For example, 
      X10 will compute the type of \xcd`b ? 0 : 1` as 
      \xcd`Int{c}` for some constraint \xcd`c`---perhaps simply 
      picking \xcd`Int{true}`, \viz, \xcd`Int`. 
\item X10 preserves place information, because it is so important. If both
      \xcd`t` and \xcd`u` entail \xcd`self.home==p`, then  
      \xcd`v` will also entail \xcd`self.home==p`.  In particular, the common
      type for \xcd`T!` and \xcd`U!` has the form \xcd`V!`. 
\item X10 similarly preserves nullity information.  If \xcd`t` and \xcd`u`
      both entail \xcd`x == null` or \xcd`x != null` for some variable
      \xcd`x`, then \xcd`v` will also entail it as well.

\end{itemize}

%\subsection{Syntactic abbreviations}\label{DepType:SyntaxAbbrev}

\section{Fundamental types}

Certain types are used in fundamental ways by X10.  

\subsection{The interface \xcd`Any`}

It is quite convenient to have a type which all values are instances of; that
is, a supertype of all types.\footnote{Java, for one, suffers a number of
  inconveniences because some built-in types like \xcd`int` and \xcd`char`
  aren't subtypes of anything else.}  X10's universal supertype is the
  interface \xcd`Any`. 

\begin{xten}
package x10.lang;
public interface Any {
  property def home():Place;
  property def at(p:Object):Boolean;
  property def at(p:Place):Boolean;
  global safe def toString():String;
  global safe def typeName():String;
  global safe def equals(Any):Boolean;
  global safe def hashCode():Int;
}
\end{xten}

\xcd`Any` provides a handful of essential methods that make sense and are
useful for everything.\footnote{The behavioral annotation \xcd`property` is
  explained in \Sref{properties}; \xcd`safe` in \Sref{SafeAnnotation}, and
  \xcd`global` in \Sref{GlobalAnnotation}.} \xcd`a.toString()` produces a
string representation of \xcd`a`, and \xcd`a.typeName()` the string
representation of its type; both are useful for debugging.  \xcd`aequals(b)`
is the programmer-overridable equality test, and \xcd`a.hashCode()` an integer
useful for hashing.  \xcd`at()` and \xcd`home()` are used in multi-place
computing. 


\subsection{The class \Xcd{Object}}
\label{Object}
\index{\Xcd{Object}}
\index{\Xcd{x10.lang.Object}}

The class \xcd"x10.lang.Object" is the supertype of all classes.
A variable of this type can hold a reference to any object.
\xcd`Object` implements \xcd`Any`. It also has a property \xcd`home:Place`,
described more in \Sref{XtenPlaces}.



\section{Type inference}
\label{TypeInference}
\index{types!inference}
\index{type inference}

\XtenCurrVer{} supports limited local type inference, permitting
certain variable types and return types to be elided.
It is a static error if an omitted type cannot be inferred or
uniquely determined.

\subsection{Variable declarations}

The type of a \xcd`val` variable declaration can be omitted if the
declaration has an initializer.  The inferred type of the
variable is the computed type of the initializer.
For example, 
%~~stmt~~`~~`~~ ~~
\xcd`val seven = 7;`
is identical to 
%~~stmt~~`~~`~~ ~~
\xcd`val seven: Int{self==7} = 7;`
Note that type inference gives the most precise X10 type, which might be more
specific than the type that a programmer would write.

\limitation{At the moment, only \xcd`val` declarations can have their types
elided in this way.  
}

\subsection{Return types}

The return type of a method can be omitted if the method has a body (\ie, is
not \xcd"abstract" or \xcd"native"). The inferred return type is the computed
type of the body.  In the following example, the return type inferred for
\xcd`isTriangle` is 
%~~type~~`~~`~~ ~~
\xcd`Boolean{self==false}`
%~~gen
% package Types.Inferred.Return;
%~~vis
\begin{xten}
class Shape {
  def isTriangle() = false; 
}  
\end{xten}
%~~siv
%
%~~neg
Note that, as with other type inference, methods are given the most specific
type.  In many cases, this interferes with subtyping.  For example, if one
tried to write: 
\begin{xten}
class Triangle extends Shape {
  def isTriangle() = true;
}
\end{xten}
\noindent
the X10 compiler would reject this program for attempting to override
\xcd`isTriangle()` by a method with the wrong type, \viz,
\xcd`Boolean{self==true}`.  In this case, supply the type that is actually
intended for \xcd`isTriangle`, such as 
\xcd`def isTriangle() :Boolean =false;`. 

The return type of a closure can be omitted.
The inferred return type is the computed type of the body.

The return type of a constructor can be omitted if the
constructor has a body.
The inferred return type is the enclosing class type with
properties bound to the arguments in the constructor's \xcd"property"
statement, if any, or to the unconstrained class type.
For example, the \xcd`Spot` class has two constructors, the first of which has
inferred return type \xcd`Spot{x==0}` and the second of which has 
inferred return type \xcd`Spot{x==xx}`. 
%~~gen
% package Types.Inferred.By.Phone;
%~~vis
\begin{xten}
class Spot(x:Int) {
  def this() {property(0);}
  def this(xx: Int) { property(xx); }
}
\end{xten}
%~~siv
%class Confirm{ 
% static val s0 : Spot{x==0} = new Spot();
% static val s1 : Spot{x==1} = new Spot(1);
%}
%~~neg


\index{Void}

A method or closure that has expression-free \xcd`return` statements
(\xcd`return;` rather than \xcd`return e;`) is said to return \xcd`Void`.
\xcd`Void` is not a type; there are no \xcd`Void` values, nor can \xcd`Void`
be used as the argument of a generic type. However, \xcd`Void` takes the
syntactic place of a type. A method returning \xcd`Void` can be specified by
\xcd`def m():Void`: 

%~~gen
% package Types.Voidd;
% class Voidddd {
% static 
%~~vis
\begin{xten}
val f : () => Void = () => {return;};
\end{xten}
%~~siv
%}
%~~neg

By a convenient abuse of language, \xcd`Void` is sometimes
lumped in with types; \eg, we may say ``return type of a method'' rather than
the formally correct but rather more awkward ``return type of a method, or
\xcd`Void`''.   Despite this informal usage, \xcd`Void` is not a type.  For
example, given 
%~~gen
% package Types.Voidd;
% class EEEEVil {
%~~vis
\begin{xten}
  static def eval[T] (f:()=>T):T = f();
\end{xten}
%~~siv
% }
%~~neg
\noindent
The call \xcd`eval[Void](f)` does {\em not} typecheck; \xcd`Void` is not a
type and thus cannot be used as a type argument.  There is no way in X10 to
write a generic function which works with both functions which return a value
and functions which do not.  In most cases, functions which have no sensible
return value can be provided with a dummy return value.

\subsection{Type arguments}

\limitation{This does not seem to work at all currently.}
\bard{Is this material even currently under consideration?}

A call to a polymorphic method %, closure, or constructor 
may omit the
explicit type arguments.  If the method has a type parameter
\xcd"T", the type argument corresponding to \xcd"T" is inferred
to be the least common ancestor of the types of any formal
parameters of type \xcd"T".
\bard{Not least!}

%TODO--check this!
Consider the following method:
\begin{xten}
def choose[T](a: T, b: T): T { ... }
\end{xten}
%
Given \xcd"Set[T] <: Collection[T]", 
\xcd"List[T] <: Collection[T]",
and \xcd"SubClass <: SuperClass",
in the following snippet, the algorithm will infer the type
\xcd"Collection[Any]" for \xcd"x".
\begin{xten}
def m(intSet: Set[Int], stringList: List[String]) {
  val x = choose(intSet, stringList);
...
}
\end{xten}
%
And in this snippet, the algorithm should infer the type
\xcd"Collection[Int]" for \xcd"y".
\begin{xten}
def m(intSet: Set[Int], intList: List[Int]) {
  val y = choose(intSet, intList);
  ...
}
\end{xten}
%
Finally, in this snippet, the algorithm should infer the type
\xcd"Collection{T <: SuperClass}" for \xcd"z".
\begin{xten}
def m(intSet: Set[SubClass], numList: List{T <: SuperClass}) {
  val z = choose(intSet, numList);
  ...
}
\end{xten}

