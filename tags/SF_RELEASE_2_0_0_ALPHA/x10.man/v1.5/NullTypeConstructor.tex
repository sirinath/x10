\chapter{The Nullary Type Constructor}
\label{NullaryTypeConstructor}\index{nullable@{\tt nullable}}

\Xten{} supports the prefix type constructor {\cf ?}, read as {\cf nullable}. 
For any type {\cf T}, the type {\tt ?T} (read: ``{\cf nullable T}'')
contains all the values of type {\cf T}, and a special {\cf null} value,
unless {\cf T} already contains {\cf null}. This value is designated by
the literal {\cf null}, which is special in that it has the type
{\cf ?T} for all types {\cf T}.\index{null@{\tt null}}

This type constructor can be used in any type expression used to
declare variables (e.g.{} local variable{s}, method parameter{s},
class field{s}, iterator parameter{s}, try/catch parameter{s} etc), or in
a new expression (e.g. {\cf new ?T()}. It may be applied to
value types, reference types, aggregate types or
type parameters. It may not be used in an {\cf extends} clause or an
{\cf implements} clause in a class or interface declaration. If
{\tt T}  is a value (reference) type, then {\cf ?T} is defined to be a
value (reference) type.

An immediate consequence of the definition of {\cf nullable} is that
for any type {\cf T}, the type {\cf ??T} is equal to the type {\cf
?T}. (The type {\cf ??T} can arise when the body of a generic class
contains a type {\cf ?X} where {\cf X} is a type parameter, and the
generic class is instantiated with a type {\cf ?T}.)

Any attempt to access a field or invoke a method on the value {\cf
null} results in a {\cf NullPointerException} being thrown.

An expression {\cf e} of type {\cf ?T} may be checked for nullity using the expression {\cf e==null}. (It is a compile time error for the static type of 
{\cf e} to not be {\cf ?T}, for some {\cf T}.)

\paragraph{Conversions}
{\cf null} can be passed as an argument to a method call whose
corresponding formal parameter is of type {\cf ?T} for some type
{\cf T}. (This is a widening reference conversion, per \cite[Sec
5.1.4]{jls2}.) Similarly it may be returned from a method call of
return type {\cf ?T} for some type {\cf T}.

For any value {\cf v} of type {\cf T}, the class cast expression {\cf
(?T) v} succeeds and specifies a value of type {\cf ?T} that may be
seen as the ``boxed'' version of {\cf v}.

\Xten{} permits the widening reference conversion from any type {\cf T}
to the type {\cf ?T1} if {\cf T} can be widened to the type {\cf
T1}. Thus, the type {\cf T} is a subtype of the type {\cf ?T}.
%in accordance with the LiskovSubstitutionPrinciple.

Correspondingly, a value {\cf e} of type {\cf ?T} can be cast to the
type {\cf T}, resulting in a {\cf NullPointerException} if {\cf e} is
{\cf null} and {\cf ?T} is not equal to {\cf T}, and in the
corresponding value of type {\cf T} otherwise.  If {\cf T} is a value
type this may be seen as the ``unboxing'' operator.

The expression {\cf (T)null} throws a {\cf ClassCastException} if {\cf
T} is not equal to {\cf ?T}; otherwise it returns {\cf null} at type
{\cf T}. Thus it may be used to check whether {\cf T=?T}.

\paragraph{Arrays of nullary type}
The nullary type constructor may also be used in (aggregate) instance
creation expressions (e.g.{} {\cf new ?T[R]}). In such a case {\cf T}
must designate a (possibly generic) class. Each member of the array is
initialized to {\cf null}. (See \S~\ref{XtenClasses} for a discussion of how
type parameters may be specified to have constructors.)

\paragraph{Implementation notes}

A value of type {\cf ?T} may be implemented by boxing a value of
type {\cf T} unless the value is already boxed. The literal {\cf null}
may be represented as the unique null reference.

\paragraph{\Java{} compatibility}

\java{} 
provides a somewhat different treatment of {\cf null}.  A class
definition extends a nullable type to produce a nullable type, whereas
primitive types such as {\tt int} are not nullable --- the programmer
has to explicitly use a boxed version of {\cf int}, {\cf Integer}, to
get the effect of {\cf ?int}. Wherever \Java{} uses a variable at
reference type {\cf S}, and at runtime the variable may carry the
value {\cf null}, the \Xten{} programmer should declare the variable at
type {\cf ?S}. However, there are many situations in \java{} in which
a variable at reference type {\cf S} can be statically determined to
not carry null as a value. Such variables should be declared at type {\cf S}
in \Xten{}

\paragraph{Design rationale}

The need for nullable arose because \Xten{} has value types and
reference types, and arguably the ability to add a {\cf null} value to a
type is orthogonal to whether the type is a value type or a 
reference type. This argues for the notion of nullability as a type
constructor.

The key question that remains is whether it should be possible to
define ``towers'', that is, define the type constructor in such a way
that {\cf ??T} is distinct from {\cf ?T}. Here one would think of
nullable as a disjoint sum type constructor that adds a value {\cf
null} to the interpretation of its argument type even if it already
has that value. Thus {\cf ??T} is distinct from {\cf ?T} because it
has one more {\cf null} value. Explicit injection/projection functions
(of signature {\cf T -> ?T} to {\cf ?T ->T}) would need to be
provided.

The designers of \Xten{} felt that while such a definition might be
mathematically tenable, and programmatically interesting, it was
likely to be too confusing for programmers. More importantly, it would
be a deviation from current practice that is not forced by the core
focus of \Xten{} (concurrency and distribution). Hence the decision to
collapse the tower by requiring that {\cf ??T} be equal to {\tt
?T}. As discussed below, towers can be obtained through explicit
programming.

\paragraph{Examples}

Consider the following class:

\begin{x10}
 final value Box<T> \{ 
   public T datum; 
   public Box(T v) \{ this.datum = v; \}
  \}
\end{x10}

Now one may use a variable {\cf x} at type {\cf ?Box<V>} to
distinguish between the {\cf null} at type {\cf ?Box<V>} and at type
{\cf V} (if {\cf V} is a nullable type). In the first case the value
of {\cf x} will be {\cf null}, in the second case {\cf x.datum} will
be {\cf null}.

Such a type may be used to define efficient generic code for memoization:

\begin{x10}
 abstract class Memo <V> \{
   ?Box<V>[]  values; 
   Memo(int n) \{
     // initialized to all nulls
     values = new ?Box<V>[n]; 
   \}
   V compute(int key); 
   V lookup(int key) \{ 
    if (values[key] != null) 
      return values[key].datum;
    V val = compute(key);
    values[key] = new Box<V>(val);
    return val;
   \}
 \}
\end{x10}


% C#: http://blogs.msdn.com/ericgu/archive/2004/05/27/143221.aspx
% Nice: http://nice.sourceforge.net/cgi-bin/twiki/view/Doc/OptionTypes

 


