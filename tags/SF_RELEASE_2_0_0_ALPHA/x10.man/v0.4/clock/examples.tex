\section{Examples}
\todo{Bring in other examples from Concur paper.}
Consider the core of the ASCI Benchmark Sweep3D program for computing
solutions to mass transport problems.

In a nutshell the core computation is a triply nested sequential loop
in which the value of a variable in the current iteration is dependent
on the values of neighboring variables in a past iteration. Such a
problem can be parallelized through pipelining. One visualizes a
diagonal wavefront sweeping through the array. An MPI version of the
program may be described as follows. There is a two dimensional grid
of processors which performs the following computation
repeatedly. Each processor synchronously receives a value from the
processor to its west, then to its north, then computes some function
of these values and computes a new value to be sent to the processor
to its east and then to its south.  Ignoring the behavior of the
boundary processors for the moment such a computation may be described
by the following \Xten{} program:

\begin{x10}
region R = [1..n0,1..m0];
clock[R] W,N;
clock(W) final double [cyclic(R)] A; 
for (int t : 1..TMax) \{
  ateach( i,j:A) 
    clock (W[i-1,j],N[i,j-1],W[i,j],N[i,j]) \{
      double west = now (W[i-1,j]) future\{A[i-1,j]\}; 
      W[i-1,j].continue();           
      double north = now (N[i,j-1]) future\{A[i,j-1]\}; 
      N[i,j-1].continue();
      next(W[i,j]) A[i,j] = compute(west, north);
      next W[i-1,j],N[i,j-1],W[i,j],N[i,j];
  \}
\}
\end{x10}
