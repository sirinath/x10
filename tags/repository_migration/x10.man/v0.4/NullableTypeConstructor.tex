\section{The Nullable Type Constructor}
\label{NullableTypeConstructor}\index{nullable@{\tt nullable}}

\Xten{} supports the prefix type constructor, {\cf nullable}.  For any
type {\cf T}, the type {\tt nullable T} contains all the values of
type {\cf T}, and a special {\cf null} value, unless {\cf T} already
contains {\cf null}. This value is designated by the literal {\cf
null}, which is special in that it has the type {\cf nullable T} for all types
{\cf T}.\index{null@{\tt null}}

The visibility of the type {\tt nullable T} is the same as the
visibility of {\tt T}. The members of the type {\tt nullable T} are
the same as those of type {\tt T}. Note that because of this {\tt
nullable} may not be regarded as a generic class; rather it is a
special type constructor.

%% TODO: Visibility of nullable^T.

This type constructor can be used in any type expression used to
declare variables (e.g.{} local variable{s}, method parameter{s},
class field{s}, iterator parameter{s}, try/catch parameter{s} etc).
It may be applied to value types, reference types or aggregate types.
It may not be used in an {\cf extends} clause or an {\cf implements}
clause in a class or interface declaration. It may not be used 
in a new expression -- a new expression is used to construct 

If {\tt T} is a value
(respectively, reference) type, then {\cf nullable T} is defined to be
a value (respectively, reference) type.

An immediate consequence of the definition of {\cf nullable} is that
for any type {\cf T}, the type {\cf nullable nullable T} is equal to
the type {\cf nullable T}.

Any attempt to access a field or invoke a method on the value {\cf
null} results in a {\cf NullPointerException}.

An expression {\cf e} of type {\cf nullable T} may be checked for nullity
using the expression {\cf e==null}. (It is a compile time error for
the static type of {\cf e} to not be {\cf nullable T}, for some {\cf T}.)

\paragraph{Conversions}
{\cf null} can be passed as an argument to a method call whose
corresponding formal parameter is of type {\cf nullable T} for some type
{\cf T}. (This is a widening reference conversion, per \cite[Sec
5.1.4]{jls2}.) Similarly it may be returned from a method call of
return type {\cf nullable T} for some type {\cf T}.

For any value {\cf v} of type {\cf T}, the class cast expression {\cf
(nullable T) v} succeeds and specifies a value of type {\cf nullable
T}. This value may be seen as the ``boxed'' version of {\cf v}.

\Xten{} permits the widening reference conversion from any type {\cf T}
to the type {\cf nullable T1} if {\cf T} can be widened to the type {\cf
T1}. Thus, the type {\cf T} is a subtype of the type {\cf nullable T}.
%in accordance with the LiskovSubstitutionPrinciple.

Correspondingly, a value {\cf e} of type {\cf nullable T} can be cast to the
type {\cf T}, resulting in a {\cf NullPointerException} if {\cf e} is
{\cf null} and {\cf nullable T} is not equal to {\cf T}, and in the
corresponding value of type {\cf T} otherwise.  If {\cf T} is a value
type this may be seen as the ``unboxing'' operator.

The expression {\cf (T) null} throws a {\cf ClassCastException} if {\cf
T} is not equal to {\cf nullable T}; otherwise it returns {\cf null} at type
{\cf T}. Thus it may be used to check whether {\cf T=nullable T}.

\paragraph{Arrays of nullary type}
The nullary type constructor may also be used in (aggregate) instance
creation expressions (e.g.{} {\cf new (nullable T)[R]}). In such a
case {\cf T} must designate a class. Each member of the array is
initialized to {\cf null}, unless an explicit array initializer is
specified.

\paragraph{Implementation notes}
A value of type {\cf nullable T} may be implemented by boxing a value of
type {\cf T} unless the value is already boxed. The literal {\cf null}
may be represented as the unique null reference.

\paragraph{\Java{} compatibility}

\java{} provides a somewhat different treatment of {\cf null}.  A
class definition extends a nullable type to produce a nullable type,
whereas primitive types such as {\tt int} are not nullable --- the
programmer has to explicitly use a boxed version of {\cf int}, {\cf
Integer}, to get the effect of {\cf nullable int}. Wherever \Java{} uses a
variable at reference type {\cf S}, and at runtime the variable may
carry the value {\cf null}, the \Xten{} programmer should declare the
variable at type {\cf nullable S}. However, there are many situations
in \java{} in which a variable at reference type {\cf S} can be
statically determined to not carry null as a value. Such variables
should be declared at type {\cf S} in \Xten{}

\paragraph{Design rationale}

The need for {\cf nullable} arose because \Xten{} has value types and
reference types, and arguably the ability to add a {\cf null} value to
a type is orthogonal to whether the type is a value type or a
reference type. This argues for the notion of nullability as a type
constructor.

The key question that remains is whether it should be possible to
define ``towers'', that is, define the type constructor in such a way
that {\cf nullable nullable T} is distinct from {\cf nullable T}. Here
one would think of nullable as a disjoint sum type constructor that
adds a value {\cf null} to the interpretation of its argument type
even if it already has that value. Thus {\cf nullable nullable T} is
distinct from {\cf nullable T} because it has one more {\cf null}
value. Explicit injection/projection functions (of signature {\cf T ->
nullable T} to {\cf nullable T ->T}) would need to be provided.

The designers of \Xten{} felt that while such a definition might be
mathematically tenable, and programmatically interesting, it was
likely to be too confusing for programmers. More importantly, it would
be a deviation from current practice that is not forced by the core
focus of \Xten{} (concurrency and distribution). Hence the decision to
collapse the tower.  As discussed below, this results in no loss of
expressiveness because towers can be obtained through explicit
programming.

\paragraph{Examples}

Consider the following class:

\begin{x10}
final value Box \{ 
  public nullable Object datum; 
  public Box(nullable Object v) \{ this.datum = v; \}
\}
\end{x10}

Now one may use a variable {\cf x} at type {\cf nullable Box} to
distinguish between the {\cf null} at type {\cf nullable Box} and at type
{\cf nullable Object}. In the first case the value
of {\cf x} will be {\cf null}, in the second case the value of {\cf x.datum} will
be {\cf null}.

Such a type may be used to define efficient code for memoization:

\begin{x10}
abstract class Memo  \{
  (nullable Box)[]  values; 
  Memo(int n) \{
    // initialized to all nulls
    values = new (nullable Box)[n]; 
  \}
  nullable Object compute(int key); 
  nullable Object lookup(int key) \{ 
   if (values[key] != null) 
     return values[key].datum;
   V val = compute(key);
   values[key] = new Box(val);
   return val;
  \}
\}
\end{x10}


% C#: http://blogs.msdn.com/ericgu/archive/2004/05/27/143221.aspx
% Nice: http://nice.sourceforge.net/cgi-bin/twiki/view/Doc/OptionTypes

 


