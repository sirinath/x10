\chapter{Clocks}\label{XtenClocks}\index{clocks}

The standard library for \Xten{}, \xcd"x10.lang" defines a
final value class", \xcd"Clock" intended for repeated quiescence detection
of arbitrary, data-dependent collection of activities. Clocks are a
generalization of {\em barriers}. They permit dynamically created
activities to register and deregister. An activity may be registered
with multiple clocks at the same time. In particular, nested clocks
are permitted: an activity may create a nested clock and within one
phase of the outer clock schedule activities to run to completion on
the nested clock.  Nevertheless the design of clocks ensures that
deadlock cannot be introduced by using clock operations, and that
clock operations do not introduce any races.

This chapter describes the syntax and semantics of clocks and
statements in the language that have parameters of type \xcd"Clock". 

The key invariants associated with clocks are as follows.  At any
stage of the computation, a clock has zero or more {\em registered}
activities. An activity may perform operations only on those clocks it
is registered with (these clocks constitute its {\em clock set}).  An
activity is registered with one or more clocks when it is created.
During its lifetime the only additional clocks it is registered with
are exactly those that it creates. In particular it is not possible
for an activity to register itself with a clock it discovers by
reading a data-structure.

An activity may perform the following operations on a clock \xcd"c".
It may {\em unregister} with \xcd"c" by executing \xcd"c.drop();".
After this, it may perform no further actions on \xcd"c"
for its lifetime. It may {\em check} to see if it is unregistered on a
clock. It may {\em register} a newly forked activity with \xcd"c".
%% It may {\em post} a statement \xcd"S" for completion in the current phase
%% of \xcd"c" by executing the statement \xcd"now(c) S". 
It may {\em resume} the clock by executing \xcd"c.resume();". This
indicates to \xcd"c" that it has finished posting all statements it
wishes to perform in the current phase. Finally, it may {\em block}
(by executing \xcd"next;") on all the clocks that it is registered
with. (This operation implicitly \xcd"resume"'s all clocks for the
activity.) It will resume from this statement only when all these
clocks are ready to advance to the next phase.

A clock becomes ready to advance to the next phase when every activity
registered with the clock has executed at least one \xcd"resume"
operation on that clock and all statements posted for completion in
the current phase have been completed.

Though clocks introduce a blocking statement (\xcd"next") an important
property of \Xten{} is that clocks cannot introduce deadlocks. That
is, the system cannot reach a quiescent state (in which no activity is
progressing) from which it is unable to progress. For, before blocking
each activity resumes all clocks it is registered with. Thus if a
configuration were to be stuck (that is, no activity can progress) all
clocks will have been resumed. But this implies that all activities
blocked on \xcd"next" may continue and the configuration is not stuck.
The only other possibility is that an activity may be stuck on
\xcd"finish". But the interaction rule between \xcd"finish" and clocks
(\Sref{sec:finish:clock-rule}) guarantees that this cannot cause a cycle
in the wait-for graph. A more rigorous proof may be found in \cite{X10-concur05}.

\section{Clock operations}\label{sec:clock}
The special statements introduced for clock operations are listed below.
%%479 NowStatement ::= 
%%      now ( Clock ) Statement

\begin{grammar}
Statement \: ClockedStatement \\
ClockedStatement \: \xcd"clocked" \xcd"(" ClockList \xcd")" Statement \\
NextStatement \: next \xcd";" \\
\end{grammar}

Note that \xcd"x10.lang.Clock" provides several useful methods on
clocks (e.g. \xcd"drop").

\subsection{Creating new clocks}\index{clock!creation}\label{sec:clock:create}

Clocks are created using the nullary constructor for \xcd"x10.lang.Clock":

\begin{xten}
timeSynchronizer: Clock = new Clock();
\end{xten}

All clocked variables are implicitly final. The initializer for a
local variable declaration of type \xcd"Clock" must be a new clock
expression. Thus \Xten{} does not permit aliasing of clocks.
Clocks are created in the place global heap and hence outlive the
lifetime of the creating activity.  Clocks are instances of value
classes, hence may be freely copied from place to
place. (Clock instances typically contain references to mutable state
that maintains the current state of the clock.)

The current activity is automatically registered with the newly
created clock.  It may deregister using the \xcd"drop" method on
clocks (see the documentation of \xcd"x10.lang.Clock"). All activities
are automatically deregistered from all clocks they are registered
with on termination (normal or abrupt).

\subsection{Registering new activities on clocks}
\index{clock!clocked statements}\label{sec:clock:register}

The programmer may specify which clocks a new activity is to be
registered with using the \xcd"clocked" clause.

An activity may transmit only those clocks that is registered with and
has not quiesced on (\Sref{resume}). 
A \xcd"ClockUseException"\index{clock!ClockUseException} is
thrown if (and when) this condition is violated.

An activity may check that it is registered on a clock \xcd"c" by
executing:
\begin{xten}
c.registered()
\end{xten}
\noindent This call returns the \xcd"Boolean" value \xcd"true" iff the
activity is registered on \xcd"c"; otherwise it returns \xcd"false".

\begin{note}
\Xten{} does not contain a ``register'' statement that would allow an
activity to discover a clock in a datastructure and register itself on
it. Therefore, while clocks may be stored in a datastructure by one
activity and read from that by another, the new activity cannot
``use'' the clock unless it is already registered with it.
\end{note}

\oldtodo{Add text on arrays of clocks.}

\subsection{Resuming clocks}\index{clock!resume}\label{resume}\label{sec:clock:resume}
\Xten{} permits {\em split phase} clocks. An activity may wish
to indicate that it has completed whatever work it wishes to perform
in the current phase of a  clock \xcd"c" it is registered with, without
suspending all activity. It may do so  by executing the method
invocation:
\begin{xten}
c.resume();
\end{xten}

An activity may invoke this method only on a clock it is registered
with, and has not yet dropped (\Sref{sec:clock:drop}). A \xcd"ClockUseException"\index{clock!ClockUseException} is thrown if (and
when) this condition is violated.  Nothing happens if the activity has
already invoked a \xcd"resume" on this clock in the current phase.
Otherwise execution of this statement indicates that the activity will
not transmit \xcd"c" to an \xcd"async" (through a \xcd"clocked"
clause),
% or invoke \xcd"now" 
until it terminates, drops \xcd"c" or executes a \xcd"next". 

\begin{staticrule*}
The compiler should issue an error if any activity has a potentially
live execution path from a \xcd"resume" statement on a clock \xcd"c"
to a
%\xcd"now" or
async spawn statement (which registers the new
activity on \xcd"c") unless the path goes through a \xcd"next"
statement. (A \xcd"c.drop()" following a \xcd"c.resume()" is legal,
as is \xcd"c.resume()" following a \xcd"c.resume()".)
\end{staticrule*}

\subsection{Advancing clocks}\index{clock!next}\label{sec:clock:next}
An activity may execute the statement
\begin{xten}
next;
\end{xten}

\noindent 
Execution of this statement blocks until all the clocks that the
activity is registered with (if any) have advanced. (The activity
implicitly issues a \xcd"resume" on all clocks it is registered
with before suspending.)

An \Xten{} computation is said to be {\em quiescent} on a clock
\xcd"c" if each activity registered with \xcd"c" has resumed \xcd"c".
Note that once a computation is quiescent on \xcd"c", it will remain
quiescent on \xcd"c" forever (unless the system takes some action),
since no other activity can become registered with \xcd"c".  That is,
quiescence on a clock is a {\em stable property}.

Once the implementation has detected quiecence on \xcd"c", the system
marks all activities registered with \xcd"c" as being able to progress
on \xcd"c". An activity blocked on \xcd"next" resumes execution once
it is marked for progress by all the clocks it is registered with.

\subsection{Dropping clocks}\index{clock!drop}\label{sec:clock:drop}
An activity may drop a clock by executing:
\begin{xten}
c.drop();
\end{xten}

\noindent{} The activity is no longer considered registered with this
clock.  A \xcd"ClockUseException" is thrown if the activity has
already dropped \xcd"c".


%\subsection{Posting statements on a clock}\index{clock!now}\label{sec:clock:now}
\Xten{} provides syntactic support for a common idiom. Often it may be
necessary for an activity $A$ to require that a certain set of
statements be executed to completion before a clock $c$ can move
forward, without $A$ actually waiting for the completion
of the statement. We introduce the syntax:
\begin{x10}
461 Statement ::= NowStatement
471 StatementNoShortIf ::= 
       NowStatementNoShortIf
479 NowStatement ::= 
       now ( Clock ) Statement
489 NowStatementNoShortIf ::= 
       now ( Clock ) StatementNoShortIf
\end{x10}
\noindent 

A statement {\tt now (c) s} may be considered as shorthand for
\begin{x10}
  async clocked(c) \{ 
     finish async s; 
  \}
\end{x10}

\paragraph{Note.} Because of the static semantics of {\tt finish}
it is not possible to nest {\cf now} statements. Instead if it proves
useful, we may introduce a multi-clocked {\tt now} statement,
which permits the statement to be posted on multiple clocks
simultaneously.
\begin{x10}
479' NowStatement ::= 
       now ( ClockList ) Statement
489' NowStatementNoShortIf ::= 
       now ( ClockList ) StatementNoShortIf  
\end{x10}


\subsection{Program equivalences}
From the discussion above it should be clear that the following
equivalences hold:

\begin{eqnarray}
 \mbox{\xcd"c.resume(); next;"}       &=& \mbox{\xcd"next;"}\\
 \mbox{\xcd"c.resume(); d.resume();"} &=& \mbox{\xcd"d.resume(); c.resume();"}\\
 \mbox{\xcd"c.resume(); c.resume();"} &=& \mbox{\xcd"c.resume();"}
\end{eqnarray}

Note that \xcd"next; next;" is not the same as \xcd"next;". The
first will wait for clocks to advance twice, and the second
once.  

%\notinfouro{\subsection{Implementation Notes}
Clocks may be implemented efficiently with message passing, e.g.{} by
using short-circuit ideas in \cite{SaraswatPODC88}.  Recall that every
activity is spawned with references to a fixed number of clocks. Each
reference should be thought of as a global pointer to a location in
some place representing the clock. (We shall discuss a further
optimization below.) Each clock keeps two counters: the total number
of outstanding references to the clock, and the number of activities
that are currently suspended on the clock.

When an activity $A$ spawns another activity $B$ that will reference a
clock $c$ referenced by $A$, $A$ {\em splits} the reference by sending
a message to the clock. Whenever an activity drops a reference to a
clock, or suspends on it, it sends a message to the clock. 

The optimization is that the clock can be represented in a distributed
fashion. Each place keeps a local counter for each clock that is
referenced by an activity in that place. The global location for the
clock simply keeps track of the places that have references and that
are quiescent. This can reduce the inter-place message traffic
significantly.
}
%\notinfouro{\section{Clocked types}\index{types!clocked}

We allow types to specify clocks, via a {\cf clocked(c)} modifier,
e.g.{}

\begin{x10}
  clocked(c) int r;
\end{x10}

This declaration asserts that {\cf r} is accessible
(readable/writable) only by those statements that are clocked on {\cf
c}. Thus propagation of the clock provides some control over the
``visibility'' of {\cf r}.

The declaration 

\begin{x10}
  clocked(c) final int l = r;
\end{x10}

\noindent asserts additionally that in each clock instant {\cf l} is final, 
i.e.{} the value of {\cf l} may be reset at the beginning of each phase
of {\tt c} but stays constant during the phase.

This statement terminates when the computation of {\tt r} has
terminated and the assignment has been performed.

\todo{Generalize the syntax so that aggregate variables can be clocked with an aggregate clock of the same shape.}

\subsection{Clocked assignment}\index{assignment!clocked}
We expect that most arrays containing application data will be
declared to be {\cf clocked final}. We support this very powerful type
declaration by providing a new statement:
{\footnotesize
\begin{verbatim}
  next(c) l = r; 
\end{verbatim}}


\noindent 
for a variable $l$ declared to be clocked on $c$. The statement
assigns $r$ to the {\em next} value of $l$. There may be multiple such
assignments before the clock advances. The last such assignment
specifies the value of the variable that will be visible after the
clock has advanced.  If the variable is {\cf clocked final} it is
guaranteed that {\em all} readers of the variable throughout this
phase will see the value $r$.

The expression {\tt r} is implicitly treated as {\tt now(c) r}. That
is, the clock {\tt c} will not advance until the computation of {\tt r} has
terminated.

}
%\notinfouro{%III. Applied constrained calculi. (3 pages)
%
%For each example below, formal static and dynamic semantics rules for
%new constructs extension over the core CFJ. Subject-reduction and
%type-soundness theorems. Proofs to be found in fuller version of
%paper.
%
%(a) arrays, region, distributions -- type safe implies no arrayoutofbounds
%exceptions, only ClassCastExceptions (when dynamic checks fail).
%
%Use Satish's conditional constraints example.
%-- emphasize what is new over DML. 
%
%(b) places, concurrency -- place types.
%
%(c) ownership types, alias control.
%
The following section presents example uses of constrained types
using several different
constraint systems.
%
\eat{
Many common object-oriented
idioms and
object-oriented type systems can be captured naturally using
constrained types: specifically we discuss types for places,
aliases,
ownership, arrays and clocks.  \ref{TODO}
}

\eat{
\ref{TODO}
Many of these constraint systems are more
expressive than the constraint system implemented in the current
\Xten{} compiler and have not (yet) been implemented.
}

\eat{
\ref{TODO}
In the following we will use the shorthand $\tt C(\bar{t}:c)$ for the
type $\tt C(:\bar{f}=\bar{t},c)$ where the declaration of the class
{\tt C} is $\tt \class\ C(\bar{\tt T}\ \bar{\tt f}:c)\ldots$  Also,
we abbreviate $\tt C(\bar{t}:\true)$ as $\tt C(\bar{t})$.
Finally, we use the shorthand $\tt T\;x=t;~c$ for the constraint
$\tt T\;x;~x=t;~c$.
}

\eat{
Finally, we
will also have need to use the shorthand
${\tt C}_1(\bar{t}_1:{\tt c}_1)\& \ldots {\tt C}_k(\bar{\tt t}_k:{\tt c}_k)$
for the type
${\tt C}_1(:\bar{\tt f}_1=\bar{\tt t}_1, \ldots,
            \bar{\tt f}_k=\bar{\tt t}_k,{\tt c}_1,\ldots,{\tt c}_k)$ 
provided that the ${\tt C}_i$ form a subtype chain
and the declared fields of ${\tt C}_i$ are ${\tt f}_i$.

Constraints naturally allow for partial specification
(e.g., inequalities) or incomplete specification (no constraint on a
variable) with the same simple syntax. In the example below,
the type of {\tt a} does not place any constraint on the second
dimension of {\tt a}, but this dimension can be used in other
types (e.g., the return type).
\begin{xten}
  class Matrix(int m, int n) {
    Matrix(m,a.n) mul(Matrix(:m==this.n) a) {...}
    ...
  }
\end{xten}

Constraints also naturally permit the expression of existential types:
\begin{xten}
  class List(int length) { 
    List(:length <= this.length) filter(Comparator k) {...} 
    ...
  }
\end{xten}
\noindent
Here, the length of the list returned by the \xcd{filter} method is 
unknown, but is bound by the length of the original list.
}

\if 0
\subsection{Presburger constraints: array bounds}

Xi and Pfenning proposed using dependent types for eliminating
array bounds checks~\cite{xi98array}.
\Xten{} does not (yet) support generic types, however XXX
%
In CFJ, an array of type \xcd{T[]} indexed by (signed) integers
can be modeled as a class with the following
signature:\footnote{For this example, we assume generics
are supported.}
\begin{xten}
interface Array<T>(int(:self >= 0) length) {
  T get(int(:0 <= self, self < this.length) i);
  void set(int(:0 <= self, self < this.length) i, T v);
}
\end{xten}

Bounds can be checked using a constraint system based on
Presburger arithmetic~\cite{omega}.  Constraint terms include
integer constraints, scalar multiplication, and addition;
constraints include inequalities:
\fi


\eat{
Some code that iterates over an array (sugaring {\tt get} and {\tt set}):
\begin{xten}
double dot(double[] x, double[] y
         : x.length == y.length) {
  double r = 0.; 
  for (int(:self >= 0, self < x.length)
       i = 0; i < x.length; i++) {
    r += x[i] * y[i];
  }
  return r;
}
\end{xten}
}

\eat{
Another one:
\begin{xten}
double[](:length = x.length) saxpy(double a, double[] x, double[] y : x.length = y.length) {
    double[](:length = x.length) result = new double[x.length];
    for (int(:self >= 0, self < x.length) i = 0; i < x.length; i++) {
        result[i] = a * x[i] + y[i];
    }
    return result;
}
\end{xten}
}

% \subsection{Presburger constraints: blocked LU factorization}

\subsection{Equality constraints}

The \Xten{} compiler includes a simple equality-based constraint
system, described in Section~\ref{sec:lang}.
Equalities constraints
are used throughout \Xten{} programs.  For example, to ensure
$n$-dimensional arrays are indexed only be $n$-dimensional
index points, the array access operation requires that the
array's \xcd{rank} property be equal to the index's \xcd{rank}.

Equality constraints specified in the X10 run-time library are used by the
compiler to generate efficient code.  For instance, an iteration over
the points in a region can be optimized to a set of nested loops
if the constraint on the region's type specifies that the region
is rectangular and of constant rank.


\eat{
\subsection{Equality constraints with disjunction: place types}

This example is due to Satish Chandra. We wish to specify a balanced
distributed tree with the property that its right child is always at
the same place as its parent, and once the left child is at the same
place then the entire subtree is at that place.  In
\Xten{}, every object has a field {\tt location} of type
{\tt place} that specifies the location at which the object is located.
%
The desired property may be specified thus:
\begin{xten}
class Tree(boolean localLeft) extends Object {
  Tree(:!this.localLeft || (location==here && self.localLeft)) left; 
  Tree(:location==here) right);
  ...
}
\end{xten}
The constraint on \xcd{left} states that if the \xcd{localLeft} property is
true for the current node, then the location of the \xcd{left} child must be
\xcd{here} and its \xcd{localLeft} property must be set.  This ensures,
recursively, that the entire left subtree will be located at the same place.
}

\subsection{Presburger constraints}

Presburger constraints are linear integer inequalities.
%A constraint solver plugin was implemented using a port to Java of the
%Omega library.~\cite{omega,scale}
%A separate implementation
%of a Presburger constraint solver was implemented using
%CVC3~\cite{cvc}. 
A Presburger constraint solver plugin was implemented using
CVC3~\cite{cvclite,cvc}.  The list example in
Figure~\ref{fig:list-example} type-checks using this constraint system.

Presburger constraints are particularly useful in a
high-performance computing setting where array operations are
pervasive.
Xi and Pfenning proposed using dependent types for eliminating
array bounds checks~\cite{xi98array}.  A Presburger constraint
system can be used to keep track of array dimensions and array
indices to ensure bounds violations do not occur.

\subsection{Set constraints: region-based arrays}

Rather than using Presburger constraints, 
\Xten{} takes another approach:
following ZPL~\cite{ZPL}, arrays in \Xten{}
are defined over
{\em regions},
sets of $n$-dimensional {\em index points}~\cite{gps06-arrays}.
For instance, the region \xcd{[0:200,}\xcd{1:100]} specifies a
collection of two-dimensional points \xcd{(i,j)} with \xcd{i}
ranging from \xcd{0} to \xcd{200} and \xcd{j} ranging from
\xcd{1} to \xcd{100}.

Regions and points were modeled in CVC3~\cite{cvc} to create a
constraint solver that ensures array bounds
violations do not occur:
an array access type-checks if the index point can be statically
determined to be in the region over which the array is defined.

Region constraints are subset constraints
written as calls to the \xcd{contains}
method of the \xcd{region} class.
The constraint solver does not actually evaluate the calls to
the \xcd{contains} method, rather it interprets these calls
symbolically
as subset constraints at compile time.

Constraints have the following syntax:

{
\small
\begin{tabular}{r@{\quad}rcl}
\\
  (Constraint)   &\xcd{c} &::=& \xcd{r.contains(r)} \bnf \dots \\
  (Region) &\xcd{r} &::=& \xcd{t} \bnf [${\tt b}_1$:${\tt d}_1$,\ldots,${\tt b}_k$:${\tt d}_k$]
           \\
           &        &  \bnf &
           \xcd{r | r} \bnf \xcd{r & r} \bnf \xcd{r - r}
           \\
           &        &  \bnf &
           \xcd{r + p} \bnf \xcd{r - p} \\
  (Point)  &\xcd{p} &::=& \xcd{t} \bnf $[{\tt b}_1,\ldots,{\tt b}_k]$ \\
(Integer)&\xcd{b},\xcd{d} &::=& \xcd{t} \bnf \xcd{n} \\
\\
\end{tabular}
}

\noindent
where \xcd{t} are constraint terms (properties and final variables)
and \xcd{n} are integer literals.

Regions used in constraints are either constraint terms \xcd{t},
region constants, unions (\xcd{|}), intersections (\xcd{&}),
or differences (\xcd{-}), or regions where each point is
offset by another point \xcd{p} using \xcd{+} or \xcd{-}.

% $\xcd{r}_1$\xcd{.contains(}$\xcd{r}_2$\xcd{)}.

\begin{figure}[t]
\footnotesize

\inputxten{sor.x10}

\caption{Successive over-relaxation with regions}
\label{fig:sor}
\end{figure}

For example, the code in Figure~\ref{fig:sor} performs a successive
over-relaxation~\cite{sor} of a matrix \tcd{G} with rank 2.
The function declares a region variable \tcd{outer} as an alias for
\tcd{G}'s region and a region variable \tcd{inner} to be 
the subset of \tcd{outer} that excludes the boundary points,
formed by intersecting the \tcd{outer} region with itself shifted up, down,
left, and right by one.
The function then declares two more regions \tcd{d0} and \tcd{d1},
where ${\tt d}_i$ is set of points ${\tt x}_i$ where
$({\tt x}_0, {\tt x}_1)$ is in \tcd{inner}.  The function
iterates multiple times over points \tcd{i} in \tcd{d0}.
The syntax \tcd{finish} \tcd{foreach} (line 22) tells the
compiler to execute each loop iteration in parallel and to wait
for all concurrent activities to terminate.
The inner loop (lines 24--28) iterates over a subregion of
\tcd{inner}.

The type checker establishes that the \tcd{region} property of
the point \tcd{ij} (line 24) is \tcd{inner}
\xcd{&}~\xcd{[i..i,d1min..d1max]}, and that this region is a
subset of \tcd{inner}, which is in turn a subset of \tcd{outer},
the region of the array \tcd{G}.
Thus, the accesses to the array in the loop body
do not violate the bounds of the array.

A key to making the program type-check is that the region
intersection that defines \tcd{inner} (lines 10--11)
is explicitly intersected with \tcd{outer} so that the 
constraint solver can determine that
the result is a subset of \tcd{outer}.


\eat{
\subsection{AVL trees and red--black trees}

AVL trees and red-black trees can be modeled so that the
data structure invariant is enforced statically.

\begin{xten}
class AVLTree(int(:self >= 0) height) {...}
class Leaf(Object key) extends AVLTree(0) {...}
class Node(Object key, AVLTree l, AVLTree r
         : int d=l.height-r.height; -1 <= d, d <= 1) 
    extends AVLTree(max(l.height,r.height)+1) {...}
\end{xten}

Red--black trees may be modeled similarly. Such trees have the
invariant that (a) all leaves are black, (b) each non-leaf node has
the same number of black nodes on every path to a leaf (the black
height), (c) the immediate children of every red node are black.
\begin{xten}
class RBTree(int blackHeight) {...}
class Leaf extends RBTree(0) { int value; ... }
class Node(boolean isBlack, 
           RBTree(:this.isBlack || isBlack) l, 
           RBTree(:this.isBlack || isBlack,
                   blackHeight=l.blackHeight) r)
    extends RBTree(l.blackHeight+1) { int value; ... }
\end{xten}
}

\eat{
\subsection{Self types and binary methods}

Self types~\cite{bsg95,bfp-ecoop97-match} can be implemented
using a {\tt klass} property on objects.  The {\tt klass}
property represents the run-time class of the object.
Self types can be used to solve the binary method problem \cite{bruce-binary}.

In the example below, the {\tt Set} interface has a {\tt union} method
whose argument must be of the same class as {\tt this}.
\noindent This enables the {\tt IntSet} class's {\tt union}
method to access the {\tt bits} field of its argument {\tt s}.
\begin{xten}
  interface Set(:Class klass) {
    Set(this.klass) union(Set(this.klass) s);
  }
  class IntSet(:Class klass) implements Set(klass) {
    long bits;

    IntSet(IntSet.class)() { property(IntSet.class); }

    IntSet(IntSet.class)(int(:0 <= self, self <= 63) i) {
      property(IntSet.class);
      bits = 1 << i; }

    Set(this.klass) union(Set(this.klass) s) {
      IntSet(this.klass) r = new IntSet(this.klass);
      r.bits = this.bits | s.bits;
      return r; }
  }
\end{xten}
\noindent
The key to ensuring that this code type-checks is the
\rn{T-constr}
rule.
With a constraint system ${\cal C}_{\mathsf{klass}}$ aware of
the {\tt klass} property, the rule 
\rn{T-var} is used to subsume an expression of type
${\tt Set(this.class)}$ to type ${\tt IntSet(this.class)}$
when {\tt this} is known to be an {\tt IntSet}:
{\footnotesize
\[
\from{\begin{array}{c}
{\tt IntSet}~{\tt this}, {\tt Set}({\tt this.klass})~{\tt s}
        \vdash {\tt Set}({\tt this.klass})~{\tt s} \\
{\tt IntSet}~{\tt this}, {\tt Set}({\tt this.klass})~{\tt s}
        \vdash_{{\cal C}_{\mathsf{klass}}} {\tt IntSet}({\tt this.klass})~{\tt s} \\
\end{array}}
\infer{
{\tt IntSet}~{\tt this}, {\tt Set}({\tt this.klass})~{\tt s}
        \vdash {\tt IntSet}({\tt this.klass})~{\tt s}}
\]}
}


\eat{
\subsection{Binary search}

An informal study by Jon Bentley~\cite{programming-pearls}
found that x\% of professional programmers attending in a course
could not correctly implement binary search.

Dependent types can help here by adding the invariants to the
index types.

\subsection{Quicksort}

\begin{xten}
int(:left <= self & self <= right)
partition(T[] array, int left, int right, int pivotIndex : left <= pivotIndex & pivotIndex <= right) {
     T pivotValue = array[pivotIndex];

     // Move pivot to end
     swap(array, pivotIndex, right);

     int(:left <= self & self <= right) storeIndex;
     storeIndex = left;
     for (int(:left <= self & self <= right-1) i = left; i < right; i++) {
         if (array[i] <= pivotValue) {
             swap(array, storeIndex, i);
             storeIndex++;
         }
     }

     // Move pivot to its final place
     swap(array, right, storeIndex)
     return storeIndex;
}

void swap(T[] array,
          int(:0 <= self & self < array.length i,
          int(:0 <= self & self < array.length j) {
    T tmp = array[i];
    array[i] = array[j];
    array[j] = tmp;
}

void quicksort(T[] array, int left, int right : left <= right) {
    if (left < right) {
         // select a pivot index
         int(:left <= self & self <= right) pivotIndex = (left + right) / 2;
         pivotNewIndex = partition(array, left, right, pivotIndex)
         quicksort(array, left, pivotNewIndex-1)
         quicksort(array, pivotNewIndex+1, right)
    }
}
\end{xten}
}


\newif\ifowner
\ownerfalse

\ifowner

\subsection{Ownership constraints}

\begin{figure}[t]
\inputxten{LO.x10}
\caption{Ownership types}
\label{fig:ownership}
\end{figure}

Using a simple extension of \Xten{}'s built-in equality
constraint system,
constrained types can also be used to encode a form of ownership
types~\cite{ownership-types,liskov-popl2003}.

Figure~\ref{fig:ownership} shows a fragment of a \xcd{List}
class with ownership types.
Each \xcd{Owned} object has an \xcd{owner} property.  Objects
also have properties that are used as owner parameters.
%
The \xcd{List} class has an \xcd{owner} property inherited from
\xcd{Owned} and also declares a \xcd{valOwner} property that is
instantiated with the owner of the values in the list, stored in
the \xcd{head} field of each element.  The \xcd{tail} of the
list is owned by the list object itself.

\Xten{}'s equality-based constraint system is sufficient for
tracking object ownership, however is does not capture all
properties of ownership type systems.
Ownership type systems enforce an ``owners as dominators''
property: the ownership relation forms a tree within the object
graph; a reference is not permitted to point directly to objects
with a different owner.
%
To encode this property, the owner of
the values \xcd{valOwner} must be contained within the owner
of the list itself; that is, \xcd{valOwner} must be \xcd{owner}
or \xcd{valOwner}'s owner must be contained in \xcd{owner}.
This is captured by the constraint \xcd{self.owns(owner)} on
\xcd{valOwner}.  Calls to the \xcd{owns} method in constraints
are interpreted by the ownership constraint solver as the
disjunction of conditions shown in the body of \xcd{owns}.
The object \xcd{world} is the root of the ownership tree; 
all objects are transitively owned by \xcd{world}.

For example, the type \xcd{List(:owner==world & valOwner == this)}
is invalid, because
its constraint is interpreted as
\xcd{owner == world & valOwner == this & this.owns(world)},
which is satisfiable only when \xcd{this == world} (which it is not).

An additional check is needed to ensure objects owned by
\xcd{this} are encapsulated.
The \xcd{tail()} method for instance, incorrectly leaks the
list's \xcd{tail} field.  To eliminate this case, the ownership
constraint system must additionally check that owner parameters
are bound only to 
\xcd{this}, \xcd{world}, or an owner property of \xcd{this}.
These conditions ensure that \xcd{tail()} can be called only on
\xcd{this} because its return type is otherwise not valid.
For instance, in the following code, the type of \xcd{ys} is
not valid because the \xcd{owner} property is bound to \xcd{xs}:
\begin{xten}
    final Owned o = ...;
    final List(:owner==o & valOwner==o) xs;
    List(:owner==xs & valOwner==o) ys = xs.tail();
\end{xten}

\fi

\if 0
\subsection{Disequalities: non-null types}

A constraint system that supports disequalities can be used to
enforce a non-null invariant on reference types.
A non-null type \xcd{C} can be written simply as \xcd{C(:self != null)}.
\fi

\eat{
\subsection{Clocked types}

Clocks are barriers that are adapted to a context where activities may be
dynamically created, and are designed so that all clock operations are
determinate.

For each arity $n$, we introduce a {\em Gentzen predicate}
${\tt clocked(\bar{t})}$. A $k$-ary Gentzen predicate $a$ satisfies the
property that $a(t_1,\ldots, t_k) \vdash a(s_1,\ldots,s_n)$ iff $k=n$
and $t_i=s_i$ for $i\leq k$.

Such a \xcd{clocked} atom is added to the context by an \xcd{clocked async}:
$$
\from{\Gamma, {\tt clocked(\bar{\tt v})} \vdash {\tt T}\ {\tt e}}
\infer{\Gamma \vdash {\tt T}\ {\tt async}\ {\tt clocked}(\bar{\tt v}) {\tt e}}
$$

A programmer can require that a method may be invoked only if the
invoking activity is registered on the clocks $\bar{\tt k}$ by adding
a \xcd{clocked} clause. The rule for method elaboration and method invocation then change:
$$
\begin{array}{l}
\from{ \bar{\tt T}\ \bar{\tt x}, {\tt C}\ \this, {\tt c},\clocked(\bar{\tt k}) \vdash {\tt S}\ {\tt e}, {\tt S} \subtype {\tt T} }   
\infer{\tt T\ m(\bar{\tt T}\,\bar{\tt x} : c) \clocked(\bar{\tt  k})\{\return\ e;\}\ \mbox{OK in}\ C} 
\\ \quad\\ 
\rname{T-Invk}%
\from{\begin{array}{l}
\Gamma \vdash {\tt T}_{0:n} \ {\tt e}_{0:n}  \\
\mtype({\tt  T}_0,{\tt  m},{\tt  z}_0)= \tt {\tt  Z}_{1:n}\ {\tt  z}_{1:n}:c,clocked(\bar{\tt  k}) \rightarrow {\tt  S} \\
\Gamma, {\tt  T}_{0:n}\ {\tt  z}_{0:n} \vdash {\tt  T}_{1:n} \subtype {\tt  Z}_{1:n}\\
\sigma(\Gamma, {\tt  T}_{0:n}\ {\tt  z}_{0:n}) \vdash_{\cal C} {\tt  c} \ \ \ 
\mbox {(${\tt  z}_{0:n}$ fresh)} \\
\Gamma \vdash \clocked(\bar{\tt  k})\\
\end{array}}
\infer{\Gamma \vdash ({\tt  T}_{0:n}\ {\tt  z}_{0:n}; S)\ {\tt  e}_0.{\tt  m({\tt  e}_{1:n})}}
\end{array}
$$
}

\eat{
\subsection{Capabilities}

Capabilities (from Radha and Vijay's paper on neighborhoods)
}

\eat{
\subsection{Activity-local objects}

Parallelism in \Xten{} is supported through lightweight asynchronous {\em
activities}, created by {\tt  async} statements.
It is often useful to restrict objects so that they are {\em local} to a
particular activity.
A local object may be accessed only by
the activity that created it or by an ancestor of that activity.
% it may be written only by the activity that created
% it or by a descendant of that activity.
Local objects are declared and created by qualifying their type
with {\tt  local}:
\begin{xten}
  local C o = new local C();
\end{xten}

To encode local objects in \Xten{}, we add an {\tt  activity}
property to objects:
\begin{xten}
  class Object(Activity activity) { ... }
\end{xten}
\noindent
where {\tt  Activity} has a possibly null {\tt  parent} property:
\begin{xten}
  class Activity(Activity parent) { ... }
\end{xten}
\noindent

To track the current activity ({\tt  z}), we augment typing judgments
as follows:
\[
  {\tt  z};~\Gamma \vdash {\tt  T}\ {\tt  e}
\]
\noindent where ${\tt  Activity}({{\tt  z}'})~{\tt  z} \in \Gamma$.
When the current activity is {\tt  z},
we encode the type {\tt  local C} as ${\tt  C}({\tt  z})$.

Spawning a new activity with an {\tt  async} statement
introduces a fresh activity ${\tt  z}'$:
\[
\from{
{\tt  z}';~\Gamma,~{\tt  Activity}({\tt  z})~{\tt  z'} \vdash {\tt  T}\ {\tt  e}\ \ \ 
\mbox{(${\tt  z}'$ fresh)}
}
\infer{
{\tt  z};~\Gamma \vdash {\tt  T}\ ({\tt  async}\ {\tt  e})
}
\]
The rule \rn{T-Field} is strengthened to require that reads 
only be performed on objects whose {\tt  activity} property is a
descendant of the current activity.
%\rname{T-Field-Local}%
\[
\from{
\begin{array}{ll}
{\tt  z};~\Gamma \vdash {\tt  T}_0\ {\tt  e} \\
\mathit{fields}({\tt  T}_0,{\tt  z}_0)= \bar{\tt  U}\ \bar{\tt  f}_i &
\mbox{(${\tt  z}_0$ fresh)} \\
{\tt  z};~\Gamma \vdash {\tt  T}_0 \subtype {\tt  C}(:{\tt  activity} = {\tt  z}') &
\Gamma \vdash {\tt  z}~\mathsf{spawns}~{\tt  z}'
\end{array}
}
\infer{{\tt  z};~\Gamma \vdash ({\tt  T}_0\ {\tt  z}_0; {\tt  z}_0.{\tt  f}_i=\self;{\tt  U}_i)\ {\tt  e.f}_i}
\]

%\Gamma \vdash {\tt  z}_0.{\tt  activity} = {\tt  z}' &

\noindent
where the $\mathsf{spawns}$ relation is defined as follows:
\[
\Gamma \vdash {\tt  z}~\mathsf{spawns}~{\tt  z}
\]
\[
\from{
\Gamma \vdash {\tt  z_1}~\mathsf{spawns}~{\tt  z_2} \ \ \ 
\Gamma \vdash {\tt  z_2}~\mathsf{spawns}~{\tt  z_3}}
\infer{\Gamma \vdash {\tt  z_1}~\mathsf{spawns}~{\tt  z_3}}
\]
\[
\from{\Gamma \vdash {\tt  z_2}.{\tt  parent} = {\tt  z_1}}
\infer{\Gamma \vdash {\tt  z_1}~\mathsf{spawns}~{\tt  z_2}}
\]

\eat{
local objects owned by activity that created it.

locals cannot be read by contained asyncs.

locals can be written by contained asyncs.

locals created by an activity are inherited by the parent when
the activity terminates.

\begin{xten}
C(:thread = current) x = ...;
finish foreach (...) {
  C(:thread = current) y = x; // no!
  x = y;
}
\end{xten}

// can read if thread prop is current, or an ancestor of current
// can write if thread prop is current or a child of current

e : C(:thread = x)
current owns x
fields(...) = Ti fi
-----------------------
e.fi : Ti

extensions:

1. add thread to context
2. strengthen T-field rule
}
}

\eat{
\subsection{Discussion}

\paragraph{Control-flow.}
Tricky to encode.  Need something like {\tt pc} label~\cite{jif}.

\paragraph{Type state.}
Type state depends on the mutable state of the 
objects.  Cannot do in this framework.

Dependent types are of use in annotations~\cite{ns07-x10anno}.
}
}

