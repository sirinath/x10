
%
%I. Introduction and overview (3 pages)

%Design of X10, concurrency, high productivity, high performance,
%practical language.
%
%Dependent types arise naturally. arrays, regions, distributions, place
%types.
%
%Indeed you can look around and recognize many OO type systems proposed
%in the last decade or so as specific kinds of applied dependent type
%systems.
%
%Our goal is to develop a general framework for dependent types for
%statically typed OO languages ("Java-like languages"). 

\Xten{} is a modern statically typed object-oriented (OO) language
designed for high productivity in the high performance computing (HPC)
domain~\cite{X10}. \Xten{}, like most OO languages is designed around the notion
of objects, as instances of {\em classes}. However, \Xten{} places
equal emphasis on {\em arrays}, a central data-structure in high
performance computing. In particular, \Xten{} supports dense,
distributed multi-dimensional arrays of value and reference types,
built over index sets known as {\em regions}, and mappings from index
sets to places, known as {\em distributions}.  \Xten{} supports a rich
algebra of operations over regions, distributions and arrays.

In designing a static type system for \Xten{} a central problem
arises. It becomes necessary to permit types, such as {\tt region(2)},
the type of all two-dimensional regions, {\tt int[5]}, the type of
all arrays of {\tt int} of length {\tt 5}, and {\tt int[region(2)]},
the type of all {\tt int} arrays over two dimensional regions.
The underlying
general idea is that of {\em dependent types}: types
parametrized by {\em values} \cite{dependent-types,xi99dependent}.

XXX emph pluggable nature of the constraint system.

%, just as generic
%types are types paramterized by other types \cite{genericjava}.

In this paper we develop a general syntactic and semantic framework
for {\em constrained types}, user-defined dependent types
in the context of modern class-based
OO languages such as \java{}, \csharp{} and \Xten{}. Our central
insight is that a rich, user-extensible type system can be developed
on top of predicates over the {\em immutable} state of objects. Such
types statically capture many common invariants naturally arising in
code. Given a single dependently typed class or interface {\tt C}, a
user may construct a potentially infinite family of types {\tt C(:c)}
where {\tt c} is a predicate on the immutable state.

%Criteria:
%
%-- easy to use for practicing programmers
%
%-- must be flexible, it must permit the development of concrete,
%   specific type systems tailored to the application area at hand
%   (pluggable types), therefore must be parametric in the kinds of
%   expressions used in the type system.
%
%-- the implementation must be modular, the rules for type-checking
%   must be specified once in a way that is independent of the
%   particular vocabulary of operations used in the dependent type
%   system.
%
%-- must work smoothly with nominal type systems found in Java-like
%   languages, must permit separate compilation.
%
%-- must permit mostly static type-checking. 

In designing this framework, we had the following criteria:
\begin{itemize}
\item 
{\bf Ease of use.}
The framework must be 
easy to use for practicing programmers.

\item
{\bf Flexibility.}
The framework
must permit the development of concrete,
specific type systems tailored to the application area at
hand, enabling a kind of pluggable type system~\cite{bracha04-pluggable}.
Hence, the framework must be parametric in the kinds of
expressions used in the type system.

\item
{\bf Modularity.}
The rules for type-checking
must be specified once in a way that is independent of the
particular vocabulary of operations used in the dependent type
system.

\item
{\bf Integration with OO languages.}
The framework must
work smoothly with nominal type systems found in Java-like
languages, and must permit separate compilation.

\item
{\bf Static checking.}
The framework must permit mostly static type-checking. 
\end{itemize}

\subsection{Overview}

XXX Overview of our design.

Our basic approach to introducing dependent types into
class-based statically typed OO languages is to
follow the spirit of generic types, but use values instead of
types.

We permit the definition of a class {\tt C} to specify
a list of typed parameters, or {\em properties},
{\tt (T1 x1, \ldots, Tk xk)} similar in syntactic structure to
a method argument list. Each
property in this list is treated as a {\tt public final} instance
field of the class.
We also permit the
specification of a {\em class invariant}, a
{\em where clause}~\cite{where-clauses}
in the class definition. A where
clause is a boolean expression on the parameters separated from the
parameter list with a ``{\tt :}''.  The compiler ensures that all
instances of the class created at runtime satisfy the where clause
associated with the class.
Thus, for instance, we may specify a class {\tt List} with an
{\tt int length} property as follows:
\begin{code}
  class List(int length: length >= 0) \{...\}
\end{code}

Given such a definition for a class {\tt C},
types can be constructed by {\em constraining} the properties of
{\tt C}.
In principle, {\em any} boolean expression over the
properties specifies a type: the type of all instances of the
class which satisfy the boolean expression. Thus,
{\tt List(:length == 3)} is a permissible type, as are
{\tt List(:length <= 41)} and even
{\tt List(:length * f() >= g())}.
If {\tt C} has no properties, the only type that can be
constructed is the type {\tt C}.

Accordingly, a {\em constrained type} is of the form {\tt C(:e)}, the name of
a class or interface {\tt C}, called the {\em base class}, followed by a
where clause {\tt e}, called the {\em condition}, a boolean expression
on the properties of the base class. 
The denotation, or
semantic interpretation, of such a type is the set of all instances
of subclasses of the base class whose properties satisfy the
condition.
Clearly, for the denotation of a constrained type $t$ to be
non-empty the condition of $t$ must be consistent with the class
invariant, if any, of the base class of $t$.  The compiler is required to
ensure that the type of any variable declaration is non-empty.

For brevity, we write {\tt C} as a type as well; it
corresponds to the (vacuously) constrained type {\tt C(:true)}.
We also permit the syntax {\tt C(t1,\ldots, tk)} for
the type {\tt C (:x1 == t1 \&\& \ldots \&\& xk == tk)} (assuming that
the property list for {\tt C} specifies the {\tt k} properties {\tt
x1,\ldots, xk}, and each term {\tt ti} is of the correct
type). Thus, using the definition above, {\tt List(n)} is the type of
all lists of length {\tt n}.

\subsection{An example}
\begin{figure*}
{\footnotesize
\begin{verbatim}
class List(int(:self >= 0) length) {
  Object head = null;
  List(length-1) tail = null;
  List(0)() { property(0); }
  List(1)(Object head) { this(head, new List());}
  List(tail.length+1)(Object head, List tail) {
    property(tail.length+1);
    this.head = head;
    this.tail = tail;
  }
  List(length+arg.length) append(List arg) {
    return (length == 0) ? arg : new List(head, tail.append(arg));
  }
  List(length) rev() { return rev(new List()); }
  List(length+acc.length) rev(final List acc) {
    return (length == 0) ? acc : tail.rev(new List(head, acc));
  }
  List(:self.length <= this.length) filter(Predicate f) {
    if (length==0) return this;
    if (f.isTrue(head)) {
      List l = tail.filter(f);
      return new List(l+1)(head, l);
    } else {
      return tail.filter(f);
    }
  }
}
\end{verbatim}}
%% Aha a design choice here!!
%% Have parameter(..) or not.
\caption{List example}
\label{fig:list-example}
\end{figure*}

Figure~\ref{fig:list-example} shows a {\tt List} class.
Intuitively, this definition states that a {\tt List} has a {\tt int}
property {\tt length}, which must be non-negative.  The
class has two fields that hold the head and tail of the list.

The first constructor {\tt List()} has a postcondition that
signifies that it returns a list of length 0.
The constructor body contains a {\tt property} statement
that initializes the {\tt length} property to {\tt 0}.
The compiler verifies that
the constructor postcondition and the class invariant
are implied by the {\tt property} statement
and any {\tt super} calls in the constructor body imply

The 
second constructor returns a singleton list of length {\tt
1}. The third constructor returns a list of length {\tt m+1}, where
{\tt m} is the length of the second argument. 
If an argument appears in the return type then the argument must be
declared {\tt final}. Thus the argument will point to the same object
throughout the evaluation of the constructor body.

Functions that append one list to another or that reverse a list can
be defined quite naturally. In both cases the size of the list
returned is known statically (as a function of the size of the list
and the argument to the method).  The example also illustrates a
method {\tt filter} which returns a list whose length cannot be known
statically since it depends on the argument predicate {\tt f};
the list length can be bounded, however.

\subsection{Claims}

XXX Claims of this paper.

%
%Related work
%

As in staged languages~\cite{nielson-multistage,ts97-multistage},
the design distinguishes between
compile-time and run-time evaluation. Constrained types are
checked (mostly) at compile-time.
The compiler uses a
constraint solver to perform universal reasoning (``for all possible
values of method parameters'') for dependent type-checking.  There is
no run-time constraint-solving.  However,
run-time casts to dependent types are
permitted; these casts involve arithmetic, not
algebra---the values of all parameters are known.

We outline the design for a compiler that can use an extensible
constraint-solver for type-checking.
This design is implemented in the \Xten{} compiler.
No extension of an
underlying virtual machine is necessary, except as may be useful in
improving efficiency (for example, eliminating array bounds checks).
The compiler translates source
programs into target programs without dependent types
but with {\tt assume} and {\tt assert} statements.
A general
constraint propagator that depends only on the operational semantics
of the language and is constrained on the underlying constraint solver
may be run on the program in order to eliminate branches and {\tt
asserts} forced by the assumptions. If all {\tt assert}s cannot be
eliminated at compile-time, some residual constraint-checking
({\em not} solving) may need to be performed at runtime.
%
XXX contrast with hybrid type checking.

\paragraph{Rest of this paper.}

Section~\ref{sec:lang} fleshes out the syntactic and semantic details of the
proposal.
Formal semantics
and a soundness theorem
are presented in Section~\ref{sec:semantics}.
Section~\ref{sec:examples} works through a number of
examples using a variety of constraint systems.
Section~\ref{sec:related} reviews related work.
Last, Sections~\ref{sec:future}
and \ref{sec:conclusions} conclude the paper with a discussion of
future work.

