\subsection {GASNet } 

GASNet (Global-Address Space Networking) is a network-independent and language-independent high-performance communication 
interface for use in implementing the runtime system for global address space languages. It provides
two layers of interface - core and extended. The core API is a narrow interface
based on the Active Messages paradigm. The extended API is an expressive
and flexible interface that provides medium and high-level operations on remote
memory and collective operations. 

The core API provides the active messages functionality. Active message communication is formulated as logically matching request and reply operations. Upon receipt of a request message, a request handler is invoked; likewise, when a reply message is received, the reply handler is invoked. Request handlers can reply at most once to the requesting node and only to the requesting node.
 The active message routines are divided in to three kinds based on the size of the message transfer : gasnet\_AMRequestLongM(), gasnet\_AMRequestMedimM() and
gasnet\_AMRequestShortM(). Corresponding reply routines are also provided. Additionally, an asynchronous version of long request is provided in the
form of gasnet\_AMRequestLongAsyncM().

A key feature in
the active message interface of GASnet is the ability for programmer to
supply several parameters to AM handlers, as  a part of the
AM message routines. The library internally communicates those
parameters to the receiver. This functionality is useful for 
implementing remote {\em async} activities. A remote {\em async} activity is just
a function call that has to executed on a place P. An {\em async} can also
access certain local variables (e.g. final variables) from the parent's
local stack. These variables can be passed as arguments to the function call.

The extended API provides memory to memory transfer operations in the form
of put and get. Both blocking and non-blocking versions are provided. The non-blocking versions do not complete the operation, but return a handle. A synchronization operation on the handle should be performed to complete the operation.
Additionally, register-to-memory transfer operations are also provided. 
