\section{Performance Evaluation}\label{s:results}

\subsection{C implementation}
All the C implementations follow the SPMD programming style.  $P$
threads are created, and each of them is assigned a piece of work .
The DFS implemenation balances the workload through a simple, explicit
work-stealing scheme as described in \cite{BL94}.  Both the BFS and SV
implementaions simply distribute the input array evenly to each
processor to work on.  Ajacency array are used for the input
representation in BFS and DFS.  SV simulates the corresponding PRAM
algorithm, and uses the edge list representation.  In DFS,
mutual-exclusion through atomic instruction is used for
synchronization.  BFS and SV employ only barriers, and the barrier
implemenation is the usual $O(\log P)$ tree implemenation.

DFS in Cilk use the same input represenation as the C implementation.
Concurrency and synchronization are supported by Cilk runtime and Cilk
lock ( an efficient implementation through atomic instructions).

The performance numbers for handwritten C application and for the
\XWS{} code are given below, for psuedo-depth-first search,
breadth-first search and the Shiloach Vishkin algorithm for spanning
tree (from top to bottom). Numbers are presented for two machines,
altair (containing 4 dual-core Opterons) on the left and moxie, a
32-processor Niagara on the right. The best numbers for ten
consecutive runs are presented.

The Java programs were run using the experimental Java 1.7 release on
both machines. C programs were compiled with Sun cc v 5.8 and Cilk
programs with the 5.3.2 compiler.

The graphs show good scaling for the \XWS{} version of applications on
both machines. In all cases the numbers for \XWS{} are better than
that for the C code at higher processor counts.

Cilk does not perform well for BFS. (We were unable to get a Cilk
version of SV running in time for the submission.) 


\begin{figure}
 \begin{tabular}{ccc}
 \pdfimage width 9cm {../IPDPS08/plotdata/altair/BFS/random/altair-bfs-random.jpg}&
 \pdfimage width 9cm {../IPDPS08/plotdata/altair/DFS/random/altair-dfs-random.jpg}\\
 \pdfimage width 9cm {../IPDPS08/plotdata/altair/BFS/kgraph/altair-bfs-kgraph.jpg}&
 \pdfimage width 9cm {../IPDPS08/plotdata/altair/DFS/kgraph/altair-dfs-kgraph.jpg}\\
 \pdfimage width 9cm {../IPDPS08/plotdata/altair/BFS/torus/altair-bfs-torus.jpg}&
 \pdfimage width 9cm {../IPDPS08/plotdata/altair/DFS/torus/altair-dfs-torus.jpg} \\

 \end{tabular}
\caption{Psuedo-DFS and BFS for altair}\label{DFS-altair}\label{BFS-altair}
\end{figure}

\begin{figure}
 \begin{tabular}{ccc}
 \pdfimage width 9cm {../IPDPS08/plotdata/altair/SV/random/altair-sv-random.jpg}&
 \pdfimage width 9cm {../IPDPS08/plotdata/altair/SV/kgraph/altair-sv-kgraph.jpg}\\
 \pdfimage width 9cm {../IPDPS08/plotdata/altair/SV/torus/altair-sv-torus.jpg}\\
 \end{tabular}
\caption{SV for altair}\label{SV-altair}
\end{figure}


