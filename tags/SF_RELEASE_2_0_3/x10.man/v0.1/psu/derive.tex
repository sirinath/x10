\section{Derived expression types}
\label{derivedsection}

This section gives rewrite rules for the derived expression types.  By
the application of these rules, any expression can be reduced to a
semantically equivalent expression in which only the primitive
expression types (literal, variable, call, \ide{lambda}, \ide{if},
\ide{set!}) occur.

% \todo{Expressions such as {\tt (cond)} aren't legal, but they make the
% rewrite rules work.}

\newcommand{\iet}{\hbox to 2em{\hfil $\equiv$}}  % is equivalent to

\begin{schemenoindent}
(cond (\hyper{test} \hyper{sequence})
      \hyperii{clause} \dotsfoo)
\iet  (if \hyper{test}
          (begin \hyper{sequence})
          (cond \hyperii{clause} \dotsfoo))

(cond (\hyper{test})
      \hyperii{clause} \dotsfoo)
\iet  (or \hyper{test} (cond \hyperii{clause} \dotsfoo))

(cond (\hyper{test} => \hyper{recipient})
      \hyperii{clause} \dotsfoo)
\iet  (let ((test-result \hyper{test})
            (thunk2 (lambda () \hyper{recipient}))
            (thunk3 (lambda () (cond \hyperii{clause} \dotsfoo))))
        (if test-result
            ((thunk2) test-result)
            (thunk3)))

(cond (else \hyper{sequence}))
\iet  (begin \hyper{sequence})

(cond)
\iet  \hyper{some expression returning an unspecified value}

(case \hyper{key} 
  ((d1 \dotsfoo) \hyper{sequence})
  \dotsfoo)
\iet  (let ((key \hyper{key})
            (thunk1 (lambda () \hyper{sequence}))
            \dotsfoo)
        (cond ((\hyper{memv} key '(d1 \dotsfoo)) (thunk1))
               \dotsfoo))

(case \hyper{key} 
  ((d1 \dotsfoo) \hyper{sequence})
  \dotsfoo
  (else f1 f2 \dotsfoo))
\iet  (let ((key \hyper{key})
            (thunk1 (lambda () \hyper{sequence}))
            \dotsfoo
            (elsethunk (lambda () f1 f2 \dotsfoo)))
        (cond ((\hyper{memv} key '(d1 \dotsfoo)) (thunk1))
               \dotsfoo
              (else (elsethunk))))%
\end{schemenoindent}
where \hyper{memv} is an expression evaluating to the \ide{memv} procedure.

\begin{schemenoindent}
(and)         \=\iet  \schtrue
(and \hyper{test})\>\iet  \hyper{test}
(and \hyperi{test} \hyperii{test} \dotsfoo)
\iet  (let ((x \hyperi{test})
            (thunk (lambda () (and \hyperii{test} \dotsfoo))))
        (if x (thunk) x))

(or)          \>\iet  \schfalse
(or \hyper{test})\>\iet  \hyper{test}
(or \hyperi{test} \hyperii{test} \dotsfoo)
\iet  (let ((x \hyperi{test})
            (thunk (lambda () (or \hyperii{test} \dotsfoo))))
        (if x x (thunk)))

(let ((\hyperi{variable} \hyperi{init}) \dotsfoo)
  \hyper{body})
\iet  ((lambda (\hyperi{variable} \dotsfoo) \hyper{body}) \hyperi{init} \dotsfoo)

(let* () \hyper{body})
\iet  ((lambda () \hyper{body}))
(let* ((\hyperi{variable} \hyperi{init})
       (\hyperii{variable} \hyperii{init})
       \dotsfoo)
  \hyper{body})
\iet  (let ((\hyperi{variable} \hyperi{init})) 
        (let* ((\hyperii{variable} \hyperii{init})
               \dotsfoo)
          \hyper{body}))

(letrec ((\hyperi{variable} \hyperi{init})
         \dotsfoo)
  \hyper{body})
\iet  (let ((\hyperi{variable} \hyper{undefined})
            \dotsfoo)
         (let ((\hyperi{temp} \hyperi{init})
               \dotsfoo)
           (set! \hyperi{variable} \hyperi{temp})
           \dotsfoo)
         \hyper{body})%
\end{schemenoindent}
where \hyperi{temp}, \hyperii{temp}, \dotsfoo{} are variables, distinct
from \hyperi{variable}, \dotsfoo{}, that do not free occur in the
original \hyper{init} expressions, and \hyper{undefined} is an expression
which returns something that when stored in a location makes it an
error to try to obtain the value stored in the location.  (No such
expression is defined, but one is assumed to exist for the purposes of this
rewrite rule.)  The second \ide{let} expression in the expansion is not
strictly necessary, but it serves to preserve the property that the
\hyper{init} expressions are evaluated in an arbitrary order.


\begin{schemenoindent}

(begin \hyper{sequence})
\iet  ((lambda () \hyper{sequence}))%
\end{schemenoindent}
The following alternative expansion for \ide{begin} does not make use of
the ability to write more than one expression in the body of a lambda
expression.  In any case, note that these rules apply only if
\hyper{sequence} contains no definitions.
\begin{schemenoindent}
(begin \hyper{expression})\iet  \hyper{expression}
(begin \hyper{command} \hyper{sequence})
\iet  ((lambda (ignore thunk) (thunk))
       \hyper{command}
       (lambda () (begin \hyper{sequence})))%
\end{schemenoindent}

The following expansion for \ide{do} is simplified by the assumption
that no \hyper{step} is omitted.  Any \ide{do} expression in which a
\hyper{step} is omitted can be replaced by an equivalent \ide{do}
expression in which the corresponding \hyper{variable} appears as
the \hyper{step}. 

\begin{schemenoindent}
(do ((\hyperi{variable} \hyperi{init} \hyperi{step}) 
     \dotsfoo)
    (\hyper{test} \hyper{sequence})
  \hyperi{command} \dotsfoo)
\iet  (letrec ((\hyper{loop}
                (lambda (\hyperi{variable} \dotsfoo)
                  (if \hyper{test}
                      (begin \hyper{sequence})
                      (begin \hyperi{command}
                             \dotsfoo
                             (\hyper{loop} \hyperi{step} \dotsfoo))))))
        (\hyper{loop} \hyperi{init} \dotsfoo))%
\end{schemenoindent}
where \hyper{loop} is any variable which is distinct from
\hyperi{variable},~\dotsfoo, and which does not occur free in the \ide{do}
expression.

\begin{schemenoindent}
(let \hyper{variable$_0$} ((\hyperi{variable} \hyperi{init}) \dotsfoo)
  \hyper{body})
\iet  ((letrec ((\hyper{variable$_0$} (lambda (\hyperi{variable} \dotsfoo)
                             \hyper{body})))
          \hyper{variable$_0$})
       \hyperi{init} \dotsfoo)
\end{schemenoindent}

% `a                =  Q_1[a]
% `(a b c ... . z)  =  `(a . (b c ...))
% `(a . b)          =  (append Q*_0[a] `b)
% `(a)              =  Q*_0[a]
% Q*_0[a]           =  (list 'a)
% Q*_0[,a]          =  (list a)
% Q*_0[,@a]         =  a
% Q*_0[`a]          =  (list 'quasiquote Q*_1[a])
% `#(a b ...)       =  (list->vector `(a b ...))
%  ugh.
