\section{Functions}
\label{Closures}
\index{Closures}

\Xten{} provides first-class, typed functions, including
\emph{closures},
\emph{operator functions},
and \emph{method selectors}.

\begin{grammar}
ClosureExpression \:
        TypeParameters\opt
        \xcd"("
        Formals\opt
        \xcd")"
\\ &&
        Guard\opt
        ReturnType\opt
        Throws\opt
        \xcd"=>" ClosureBody \\
ClosureBody \:
        Expression \\
        \| \xcd"{" Statement\star \xcd"}" \\
        \| \xcd"{" Statement\star Expression \xcd"}" \\
\end{grammar}

Functions have zero or more type parameters, zero or more
formal parameters, an optional return type and optional set of
exceptions throws by the body.  The body has the same
syntax as a method body; it may be either an expression, a block of
statements, or a block terminated by an expression to return. In
particular, a value may be returned from the body of the function using
a return statement (\Sref{ReturnStatement}). The type of a function is a
function type (\Sref{FunctionType}).

\label{ClosureGuard}

As with methods, a function may declare a guard to
constrain the actual parameters with which it may be invoked.
The guard may refer to the type parameters, formal parameters,
and any final variables in scope at the function expression.

The body of the function is evaluated when the function is
invoked by a call expression (\Sref{Call}), not at the function's
place in the program text.

As with methods, a function with return type \xcd"Void" cannot
have a terminating expression. 
If the return type is omitted, it is inferred, as described in
\Sref{TypeInference}.
It is a static error if the return type cannot be inferred. 


\begin{example}
The following method takes a function parameter and uses it to
test each element of the list, returning the first matching
element.
\begin{xten}
def find[T](f: (T) => Boolean, xs: List[T]): T = {
  for (x: T in xs)
    if (f(x)) return x;
  null
}
\end{xten}

The method may be invoked thus:
\begin{xten}
xs: List[Int] = ...;
x: Int = find((x: Int) => x>0, xs);
\end{xten}
\end{example}

As with a normal method, the function may have a \xcd"throws"
clause. It is a static error if the body of the function throws a
checked exception that is not declared in the function's \xcd"throws"
clause.

\subsection{Outer variable access}

In a function
\xcdmath"(x$_1$: T$_1$, $\dots$, x$_n$: T$_n$){c} => { s }"
the types \xcdmath"T$_i$", the guard \xcd"c" and the body \xcd"s"
may access fields of enclosing classes and local variables and type
parameters declared in an outer scope.

Recall that languages such as \java{} require that methods may
access only those local variables declared in an enclosing scope
(``outer variables'') which are final. This is valuable in
preventing accidental races between multiple functions reading
and writing the same outer variable. At the same time, it is
desirable to support the following common idiom of expression:

\begin{xten}
def allPositive(c: Collection): Boolean {
  shared var result: Boolean = true;
  c.applyToAll((x: Int) => { if (x < 0) atomic {result=false;}});
  return result;
}
\end{xten}

This motivates the following rule:

\begin{staticrule*}
In an expression
\xcdmath"(x$_1$: T$_1$, $\dots$, x$_n$: T$_n$) => e",
any outer local variable accessed by \xcd"e" must be final or
must be declared
as \xcd"shared" (\Sref{Shared}).
\end{staticrule*}

The function body may refer to instances of enclosing classes using
the syntax \xcd"C.this", where \xcd"C" is the name of the
enclosing class.

\begin{note}
The main activity may run in parallel with any
functions it creates. Hence even the read of an outer variable by the
body of a function may result in a race condition. Since functions are
first-class, the analysis of whether a function may execute in parallel
with the activity that created it may be difficult.
\end{note}

%% vj: This should be verified.
%\begin{note}
%The rule for accessing outer variables from function bodies
%should be the same as the rule for accessing outer variables from local
%or anonymous classes.
%\end{note}

\section{Methods selectors}
\label{MethodSelectors}

A method selector expression allows a method to be used as a
first-class function.
\eat{
We can think of a method definition:

\begin{xtenmath}
def m[X$_1$, $\dots$, X$_2$](x$_1$: T$_1$, $\dots$, x$_n$: T$_n$): U = e
\end{xtenmath}

as binding the function

\begin{xtenmath}
[X$_1$, $\dots$, X$_2$](x$_1$: T$_1$, $\dots$, x$_n$: T$_n$): U => e
\end{xtenmath}

to the method-name--argument-list tuple (\xcd"m", \xcd"X",
\xcd"x: T").  That is, underlying every method is a function.

This motivates us to introduce the \emph{method selector}
expression.
}

\begin{grammar}
MethodSelector \:
        Primary \xcd"."
        MethodName \xcd"."
                TypeParameters\opt \xcd"(" Formals\opt \xcd")" \\
      \|
        TypeName \xcd"."
        MethodName \xcd"."
                TypeParameters\opt \xcd"(" Formals\opt \xcd")" \\
\end{grammar}

For a
type \xcd"T", a list of types
(\xcdmath"T$_1$", \dots, \xcdmath"T$_n$"), 
a method name
\xcd"m" and an expression \xcd"e" of type \xcd"T",
\xcdmath"e.m.(T$_1$, $\dots$, T$_n$)" denotes the function,
if any, bound to the instance method named \xcd"m" at type
\xcd"T" whose argument
type is
\xcdmath"(T$_1$, $\dots$, T$_n$), with \xcd"this" in the method
body bound to the value obtained
by evaluating \xcd"e". The
return type of the function is specified in the method declaration.

Thus, the method selector

\begin{xtenmath}
e.m.[X$_1$, $\dots$, X$_m$](T$_1$, $\dots$, T$_n$)
\end{xtenmath}

behaves as if it were the function

\begin{xtenmath}
(X$_1$, $\dots$, X$_m$](x$_1$: T$_1$, $\dots$, x$_n$: T$_n$) => e.m[X$_1$, $\dots$, X$_m$](x$_1$, $\dots$, x$_n$)
\end{xtenmath}

\begin{note}
Because of overloading, a method name is not sufficient to
uniquely identify a function for a given class (in Java-like languages).
One needs the argument type information as well.
The selector syntax (dot) is used to distinguish \xcd"e.m()" (a
method invocation on \xcd"e" of method named \xcd"m" with no arguments)
from \xcd"e.m.()"
(the function bound to the method). 
\end{note}

A static method provides a binding from a name to a function that is
independent of any instance of a class; rather it is associated with the
class itself. The static function selector
\xcdmath"T.m.(T$_1$, $\dots$, T$_n$)" denotes the
function bound to the static method named \xcd"m", with argument types
\xcdmath"(T$_1$, $\dots$, T$_n$) for the type \xcd"T". The return type
of the function is specified by the declaration of \xcd"T.m".

Users of a function type do not care whether a function was defined
directly (using the function syntax), or obtained via (static or
instance) function selectors.

\begin{note}
Design note: The function selector syntax is consistent with the
reinterpretation of the usual method invocation syntax
\xcdmath"e.m(e$_1$,..., e$_n$)"
into a function specifier, \xcd"e.m", applied to a tuple of arguments
\xcdmath"(e$_1$,..., e$_n$)". Note that the receiver is not
treated as ``an extra argument'' to the
function. That would break the above approach.
\end{note}


\section{Operator functions}

Every operator (e.g.,
\xcd"+",
\xcd"-",
\xcd"*",
\xcd"/",
\dots) has a family of functions, one for
each type on which the operator is defined. The function can be
selected using the "." syntax:

\begin{grammar}
OperatorFunction
        \: TypeName \xcd"." Operator \xcd"(" Formals\opt \xcd")" \\
        \| TypeName \xcd"." Operator \\
\end{grammar}

If an operator has more than one arity (e.g., unary and binary
\xcd"-"), the appropriate version may be selected by giving the
formal parameter types.  The binary version is selected by
default.
For example, the following equivalences hold:

\begin{xtenmath}
String.+             $\equiv$ (x: String, y: String): String => x + y
Long.-               $\equiv$ (x: Long, y: Long): Int => x - y
Float.-(Float,Float) $\equiv$ (x: Float, y: Float): Int => x - y
Int.-(Int)           $\equiv$ (x: Int): Int => -x
Boolean.&            $\equiv$ (x: Boolean, y: Boolean): Boolean => x & y
Boolean.!            $\equiv$ (x: Boolean): Boolean => !x
Int.<(Int,Int)       $\equiv$ (x: Int, y: Int): Boolean => x < y
Dist.|(Place)        $\equiv$ (d: Dist, p: Place): Dist => d | p
\end{xtenmath}

Unary and binary promotion (\Sref{XtenPromotions}) is not performed
when invoking these
operations; instead, the operands are coerced individually via implicit
coercions (\Sref{XtenConversions}), as appropriate.

Additionally, for every expression \xcd"e" of a type \xcd"T" at which a binary
operator \xcd"OP" is defined, the expression \xcd"e.OP" or
\xcd"e.OP(T)" represents the function
defined by:

\begin{xten}
(x: T): T => { e OP x }
\end{xten}

\begin{grammar}
Primary \: Expr \xcd"." Operator \xcd"(" Formals\opt \xcd")" \\
        \| Expr \xcd"." Operator \\
\end{grammar}

For every expression \xcd"e" of a type \xcd"T" at which a unary
operator \xcd"OP" is defined, the expression \xcd"e.OP()"
represents the function defined by:

\begin{xten}
(): T => { OP e }
\end{xten}

For example,
one may write an expression that adds one to each member of a
list \xcd"xs" by:

\begin{xten}
xs.map(1.+)
\end{xten}


