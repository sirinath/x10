\subsection{ARMCI}

ARMCI is a portable communication library that provides one-sided communication facility. It provides the functions for
memory to memory transfer operations, accumulation, read-modify-write operations, memory allocation etc. 
The data transfer operations are available in the form of two noncontiguous operations : ARMCI\_PutV and ARMCI\_PutS.
ARMCI\_PutV uses general I/O vectors to describe the source and destination memory locations. This is useful
for transferring any kind of data, with even non-constant stride (e.g. triangular section of an array).
ARMCI\_PutS is useful for transferring strides region with constant strides. For more information on these
functions, refer to ~\cite{armci}.

ARMCI also provides two atomic operations : {\em accumulate} and {\em read-modify-write}. Accumulate operation
combines the local and remote data atomically : {\tt $x = x + a \times y$}. Read-modify-write updates a
remote {\tt integral} variable according to a specified operation and returns the old value.

ARMCI also provides a simple progress and ordering rules. The progress rule is
that all the ARMCI one-sided operations complete regardless of
the actions taken by the receiver. That is, there is no need for the remote
process to make occasional communication calls or poll in order to assure
that communication calls issued by other processes to this process can
complete. 

The ARMCI operations issued to the same destination process complete
in order. Operations issued to different processors can complete in an
arbitrary order. Additionally, when a  put or accumulate operation completes,
the data has been copied out of calling process memory but has not 
necessarily arrived at its destination. This is a local completion. A global
completion can be achieved by calling ARMCI\_Fence operation. 