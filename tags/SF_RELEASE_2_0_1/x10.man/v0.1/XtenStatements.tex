\chapter{Statements and Expressions}\label{XtenStatements}\index{statements}

\Xten{} inherits all the standard statements of \Java{}, with the expected semantics:

\begin{x10}
\em\tt EmptyStatement      LabeledStatement  
\em\tt ExpressionStatement IfStatement
\em\tt SwitchStatement     WhileDo
\em\tt DoWhile             ForLoop           
\em\tt BreakStatement      ContinueStatement  
\em\tt ReturnStatement   ThrowStatement
\em\tt TryStatement
\end{x10}

We focus on the new statements in \Xten. 


\section{Assignment}\index{assignment}\label{AssignmentStatement}

%It is often the case that an \Xten{} variable is assigned to only
%once. The user may declare such variables as {\cf final}. However,
%this is sometimes syntactically cumbersome.
%
%{}\Xten{} supports the syntax {\cf l := r} for assignment to mutable
%variables.  The user is strongly enouraged to use this syntax to
%assign variables that are intended to be assigned to more than
%once. The \Xten{} compiler may issue a warning if it detects code 
%that uses {\cf =} assignment statements on {\cf mutable} variables.

{}\Xten{} supports assignment {\tt l = r} to array variables. In this
case {\tt r} must have the same distribution {\tt D} as {\tt l}. This
statement involves control communication between the sites hosting
{\tt D}. Each site performs the assignment(s) of array components
locally. The assignment terminates when assignment has terminated at
all sites hosting {\tt D}.

%% TODO: Sectional assignment??

\section{Remote Method Invocation}\index{remote method invocation}
We introduce shorthand for asynchronous remote method invocation:

\begin{x10}
   o -> m(a1,...,ak)
\end{x10}
\noindent is taken as shorthand for one of

\begin{x10}
 async (o) \{o.m(a1,...,ak);\}
 future (o)\{o.m(a1, ..., ak);\}
\end{x10}
\noindent based on whether the return type of methodname is {\cf void}.

\section{Iteration}\index{foreach@{\tt foreach}}\label{ForLoop}
\index{for@{\tt for}}\label{ForAllLoop}

We introduce  $k$-dimensional versions of iteration operations {\cf for} and 
{\cf foreach}:

\begin{x10}
   foreach(ind1, ..., indk : R) \{S\}
\end{x10}

Here {\cf R} is a region.\footnote{We also permit {\cf R} to be a
distribution (array); in this case {\tt R} is taken to stand for {\tt
R.region} ({\tt R.distribution.region}).}  Let {\cf [T1,..., Tk]} be
the convex closure of the region.  Then the execution of this
statement results in the parallel execution of an activity

\begin{x10}
   \{ final T1 Ind1 = i1; 
     ...
     final Tk Indk = ik;
     S
   \}
\end{x10}
\noindent for each value {\cf [i1,\ldots, ik]} in the region. Note that
the number of activities is equal to the cardinality of the region,
not its convex closure. 

The syntax for the sequential iterator {\tt for} is the same as for
{\tt forall}. However each statement is executed in the same activity,
one after the other in canonical order.

In a similar fashion we introduce the syntax:\index{ateach@{\tt ateach}}

\begin{x10}
   ateach( Ind1, ..., Indk : A) \{S\}
\end{x10}

\noindent to stand for
\begin{x10}
   foreach (ind1, ..., indk : A) 
    async (A[ind1,...,indk]) \{S\}
\end{x10}

In method definitions, for a region {\tt R} we allow the syntax 
\begin{x10}
     R i1,...,ik; 
\end{x10} 

\noindent to introduce {\cf k} new parameters which range over the
component enumerations of the convex closure of {\cf R}.

\section{Expressions}\label{XtenExpressions}\index{expressions}

{}\Xten{} inherits all the standard expressions of \Java{}
\cite[\S~15]{jls2} -- as modified to permit generics \cite{gjspec} --
with the expected semantics, unless otherwise mentioned below:

\begin{x10}
\em Assignment MethodInvocation 
\em Cast Class
\em ClassInstanceCreationExpression FieldAccessExpression   
\em ArrayCreationExpression ArrayAccessExpression
\em PostfixExpression PrefixExpression 
\em InfixExpression UnaryOperators
\em MultiplicativeOperators AdditiveOperators 
\em ShiftOperators RelationalOperators  
\em EqualityOperators BitwiseOperators
\em ConditionalOperators AssignmentOperators
\end{x10}

Expressions are evaluated in the same order as they would in \java{}
(primarily
left-to-right).\label{FieldAccess}\label{ClassCreation}\label{MethodInvocation}

We focus on the expressions in \Xten{} which have a different
semantics.

\paragraph{The classcast operator}\label{ClassCast}\index{classcast}

The classcast operation may be used to cast an expression to a given type:

\begin{x10}
 {\em\tt{}CastExpression::}
   `('{\em\tt{}ValueType}`)' {\em\tt{}Expression}
   `('{\em\tt{}ReferenceDataType}@{\em\tt{}PlaceType}`)' {\em\tt{}Expression}
\end{x10}

The result of this operation is a value of the given type if the cast
is permissible at runtime. For value types whether or not a cast is
permissible at runtime is determined as for the \java{} language
{}\cite[\S 5.5]{jls2}. For reference types a cast is permissible if
the place type of the expression is the given {\tt PlaceType}, and the
value of the expression can be cast to the given reference data type
per \java{} rules.

Any attempt to cast an expression of a reference type to a value type
(or vice versa) results in a compile-time error. Some casts -- such as
those that seek to cast a value of a subtype to a supertype -- are
known to succeed at compile-time. Such casts should not cause extra
computational overhead at runtime.

\paragraph{{\tt instanceof} operator}\label{instanceOf}\index{instanceof@{\tt instanceof}}

This operator takes two arguments; the first should be a {\tt
RelationalExpression} and the second a {\tt Type}. At run time, the
result of this operator is {\tt true} if the {\tt
RelationalExpression} can be cast to {\tt Type} without a {\tt
ClassCastException} being thrown. Otherwise the result is {\tt false}.

\paragraph{Stable equality.}\label{StableEquality}\index{==}\index{!=}
Reference equality ({\tt ==}, {\tt !=}) is replaced in \Xten{} by the
notion of stable equality so that it can apply uniformly to value and
reference types.

Two values may be compared with the infix predicate {\tt ==}. The call
returns the value {\tt true} if and only if no action taken by any
user program can distinguish between the two values.  In more detail
the rules are as follows.

If the values have a reference type, then both must be references to
the same object. 

If the values have a value type then they must be structurally equal,
that is, they must be instances of the same value class or value array
data type and all their fields or components must be {\tt ==}. 

If one of the values is {\tt null} then the predicate succeeds iff the
other value is also {\tt null}.

The predicate {\tt !=} returns {\tt true} ({\tt false}) on two
arguments if and only if the predicate {\tt ==} returns {\tt false}
({\tt true}) on the same arguments.
 \par  % empty



