\subsection{Structural constraints}
\label{sec:structural}
\label{sec:structural-typing}

Constraints on types need not be limited to subtyping.
By introducing structural constraints on types, one can
instantiate
type properties on any type with a given set
of methods or fields.
A structural constraint is satisfied if the type has a member of
the appropriate name and with a compatible type.
This feature is useful for reusing code
from separate libraries; it does not require
code of one library to implement a named interface of the other
library to interoperate.

Here, we consider an extension, not yet implemented, of the \Xten{} type system
that supports structural type constraints.
Formally, the extension
is straightforward; indeed the \FXG{} family already supports structural
constraints
via the rules for ``{\tt x \underline{has} I}'' in
Figure~\ref{fig:lookup}.

Structural constraints on types are found in many languages.
For instance,
Haskell supports type
classes~\cite{haskell,haskell-type-classes}.
%
%ML's module system allows modules to be constrained by
%structural signatures~\cite{ml}.
In Modula-3, type equivalence is structural
rather than nominal as in object-oriented languages of the C
family (e.g., C++, Java, and \Xten{}).
Unity~\cite{malayeriIntegrating08}
is another language with structural subtyping.

Using structural constraints in \Xten was inspired by 
the language PolyJ~\cite{java-popl97}, which allows type parameters to be
bounded using
structural \emph{where clauses}~\cite{where-clauses}.
For example, a sorted list class % from Figure~\ref{fig:sorted}
could
be written as follows in PolyJ:
{
\begin{xten}
class SortedList[T]
  where T {int compareTo(T)} {
    void add(T x) {... x.compareTo(y) ...}
    ... }
\end{xten}}
The \xcd"where" clause states that the type parameter \xcd"T" must have a
method \xcd"compareTo" with the given signature.

The analogous code for \xcd"SortedList" in the structural
extension of \Xten{} would be:
{
\begin{xten}
class SortedList[T]
  {T has compareTo(T): int} {
    def add(x: T) {... x.compareTo(y) ...}
    ... }
\end{xten}}

Another use of structural constraints is to permit the
introduction of 
CLU-style optional methods~\cite{clu}.  Consider the following
\xcd"Array" class:
{
\begin{xten}
class Array[T] {
  def add(a: Array[S])
    {T has add(S): U}: Array[U] {
    return new Array[U](
      (p:Point) => this(p)+a(p));
  }
  ... }
\end{xten}}

\noindent
The \xcd"Array" class defines an \xcd"add" method that takes 
an array of \xcd"S", adds each element of the array to the
corresponding element of \xcd"this", and returns an array of the
results.  The method guard specifies that the method may
only be invoked if \xcd"T" has an \xcd"add" method of the
appropriate type.  Thus, for example, an \xcd"Array[int]"
can be added to an \xcd"Array[double]" because \xcd"int"
has a method \xcd"add" (corresponding to the \xcd"+" operation)
that adds an \xcd"int" and a \xcd"double", retuning a
\xcd"double".  However, \xcd"Array[Rabbit]", for example, does not support
the \xcd"add" operation because \xcd"Rabbit" does not have an
\xcd"add" method.
%\footnote{It may have a \xcd"multiply" method.}

\eat{
\subsection{Definition-site variance}

\subsection{Conditional methods and generalized constraints}

For type parameters, method constraints are 
similar to generalized constraints proposed for
\csharp~\cite{emir06}.
%
In the following code, the \xcd"T" parameter is covariant
and so the \xcd"append" methods below are illegal:
{\footnotesize
\begin{xten}
class List[+T] {
  def append(other: T): List[T] { ... }
      // illegal
  def append(other: List[T]): List[T] { ... }
      // illegal
}
\end{xten}}
%
However, one can introduce a method parameter and then constrain
the parameter from below by the class's parameter:
For example, in the following code,
{\footnotesize
\begin{xtenmath}
class List[+T] {
  def append[U](other: U)
      {T $\extends$ U}: List[U] { ... }
  def append[U](other: List[U])
      {T $\extends$ U}: List[U] { ... }
}
\end{xten}}

The constraints must be satisfied by the callers of \xcd"append".
For example, in the following code:
{\footnotesize
\begin{xten}
xs: List[Number];
ys: List[Integer];
xs = ys; // ok
xs.append(1.0); // legal
ys.append(1.0); // illegal
\end{xten}}
the call to \xcd"xs.append" is allowed and the result type is \xcd"List[Number]", but
the call to \xcd"ys.append" is not allowed because the caller cannot show that
$\Xcd{Number} \subtype \Xcd{Double}$.

}

\subsection{Run-time casts}
\label{sec:casts}

While constraints are normally solved at compile time, 
constraints can be evaluated at run time by using casts.
The expression 
\xcd"xs as List{length==n}" checks not only 
that \xcd"xs"
is an instance of
the \xcd"List" class, but also that \xcd"xs.length" equals \xcd"n".
A \xcd"ClassCastException" is thrown if the check fails.
%
In this example, the test of the constraint does not require
run-time constraint
solving; the constraint can be checked by simply
evaluating the \xcd"length" property of \xcd"xs" and comparing against \xcd"n".
However, the situation is more complicated when casting to a
type that constrains the value's type properties.

For equality constraints on types, one can simply test equality on
canonical run-time representations of the base types and of the
constraints.
However, with subtyping constraints (e.g., \xcdmath"S$\extends$T"),
the generated code must be able to test if the constraint on the
subtype entails the constraint on the supertype.  Since one or
both of the types involved in the constraint might be type
variables, the entailment cannot be checked at compile time.

One approach is to restrict the language 
to rule out casts to type parameters 
and to generic types with subtyping constraints, ensuring that
entailment checks are not needed at run time.

Alternatively, 
the constraint solver could be embedded into the runtime system.
This is the solution used in the \Xten{} implementation; however, this
solution can result in inefficient run-time casts
if entailment checking for the given constraint system is expensive.

\eat{
A different approach to have the compiler pre-compute the results of
entailment checks.
This might be done by analyzing the program to identify which pairs of
constraints might be tested for entailment at run time and then generating a
graph were each node is a constraint and there is a directed edge between
nodes in an entailment relationship.  Run-time entailment
checking can then be implemented as reachability checking. 
This solution is a whole program analysis; all
constraints must be visible to generate the graph.
\todo{
We leave the design of this analysis for future work.
}

\todo{this is vague.  and very likely wrong.}
If \xcd"e as T" occurs in the program text and \xcd"e" has
type \xcd"U", the analysis identifies all \xcd"new" expressions
$a$,
that create a subtype of \xcd"U" and identifies all type
expressions $t$ that could instantiate \xcd"T".
For each pair $(a,t)$, the analysis checks that the
type of $a$ (determined by the constructor invoked by $a$) is a subtype of $t$.
Since $a$ and $t$ may be in different environments, \xcd"e" must
be substituted for \xcd"this" in $a$ and \xcd"self" in $t$.
}

Another option is to test objects cast to \xcd"T" not for membership in the
type \xcd"T", but rather to test against the
\emph{interpretation} of \xcd"T".
Observe that
if an instance of a generic class \xcd"C[X]"
is a member of the type \xcd"C{X==U}", then
all fields ${\tt f}_i$ of the instance with
declared type ${\tt S}_i$ contain values
that are instances of ${\tt S}_i[{\tt U}/{\tt X}]$.
For example, given the following declaration of class \xcd"List":
{
\begin{xten}
class List[X] {
  val head: X;
  val tail: List[X];
}
\end{xten}
}
\noindent
if \xcd"xs" is an instance of \xcd"List{X==String}", then
by checking that \xcd"xs.head" is an instance of \xcd"String"
and \xcd"xs.tail" is, recursively, an instance of \xcd"List{X==String}".
This property can be exploited by implementing cast to check the
types of all fields of the object.
For this check to be sound, it is vital that all fields whose
type depends on the type property \xcd"X" be transitively final;
otherwise, the test is not invariant---the
result of the test could change as the data is mutated.
Care must also be taken to implement the
test so that it terminates for cyclic data structures.
This implementation is inefficient for large data structures.

This solution has a more permissive semantics
than those implemented in \Xten or \FXGL{Q}.
The difference is best illustrated by considering an empty generic class:
{
\begin{xten}
class Nil[X] { }
\end{xten}
}
\noindent
In this case, there is no field of type \xcd"X" to test;
therefore, an object instantiated as \xcd"Nil[int]" 
can be considered a member of \xcd"Nil[String]". 
However, the solution remains sound:
Given a class \xcd"C[X]" and an expression \xcd"e" of type \xcd"t.X", if
a run-time check finds that \xcd"t" has type \xcd"C[T]", 
the compiler \emph{cannot} use this information to derive 
that \xcd"e" has type \xcd"T".
We leave to future work a proof of this claim.
