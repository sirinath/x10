\documentclass[preprint,nocopyrightspace,9pt]{sigplanconf}
%\documentclass{llncs}

\newif\iflncs
\lncsfalse

\usepackage{times-lite}
\usepackage{mathptm}
\usepackage{txtt}
\usepackage{stmaryrd}
\usepackage{code}
\usepackage{bcprules}
%\usepackage{ttquot}
\usepackage{amsmath}
\usepackage{amssymb}
\usepackage{afterpage}
\usepackage{balance}
\usepackage{floatflt}
\usepackage{defs}
\usepackage{utils}
\usepackage[pdftex]{graphicx}
\usepackage{xspace}
\usepackage{ifpdf}
\usepackage{listings}
\usepackage{x10}

\newif\ifsemantics
%\semanticsfalse
\semanticstrue

\hfuzz=1pt

\pagestyle{plain}


\ifpdf
\setlength{\pdfpagewidth}{8.5in}
\setlength{\pdfpageheight}{11in}
\fi


% \input{../../../../vj/res/pagesizes}
% \input{../../../../vj/res/vijay-macros}
\newcommand\alt{\bnf}

\newcommand\Implies{\Rightarrow}

\iflncs
\else
\newtheorem{example}{Example}[section]
\newtheorem{theorem}{Theorem}[section]
\newtheorem{lemma}[theorem]{Lemma}
\newenvironment{proof}{
\trivlist
\item[\hskip \labelsep \textsc{Proof.}]
\selectfont
\ignorespaces}{$\Box$}

%\newtheorem{proof}[theorem]{Proof}
\fi

\begin{document}

\title{Genericity through Dependent Types}

\iflncs

\author{
Nathaniel Nystom\inst{1}
\and
Igor Peshansky\inst{1}
\and
Vijay Saraswat\inst{1}
}

\institute{IBM T.~J. Watson Research~Center, P.O.~Box~704, Yorktown~Heights NY 10598 USA,
\email{\{nystrom,vsaraswa\}@us.ibm.com}}

\else

\authorinfo{Nathaniel Nystrom\titlenote{IBM T.~J. Watson Research
Center, P.O. Box 704, Yorktown Heights NY 10598 USA}}{}
  {nystrom@us.ibm.com}
\authorinfo{Igor Peshansky$^{\;*}$}{}
  {igorp@us.ibm.com}
\authorinfo{Vijay Saraswat$^{\;*}$}{}
  {vsaraswa@us.ibm.com}

% \conferenceinfo{POPL'08}{XXX}
% \copyrightyear{2008}
% \copyrightdata{[to be supplied]}

\fi

\maketitle

\begin{abstract}
Genericity is a key requirement for modern object-oriented languages.
In this paper, we describe a design for generic types in
the programming language X10.
X10 has an expressive and powerful dependent type system
in which types are specified by constraining the immutable state
of objects.  Generic types are defined
in a natural extension of the
dependent type system.

The immutable state of an object is represented by its
\emph{properties}, public final fields of the object.
A \emph{constrained type} then, is a defined by a class type and
a boolean predicate on the properties of the class.
Generic types are defined by first introducing \emph{type properties} into
the language, and then constraining those properties using
subtyping constraints.

The type system presented  here  subsumes the expressive power
of Java's generic types, virtual types,
and generalized constraints proposed for C\#.
The system also admits an efficient implementation
and eschews the pitfalls of a type erasure semantics.
We describe also a local type inference algorithm for constrained
types that permits type annotations and constraints to be elided
by the programmer.
\end{abstract}

\section{Introduction}

Outline:
\begin{itemize}
\item Classes have value properties and type properties.
\item Constrained types.
\item Pluggable constraint solvers.
\item This combination subsumes many OO type systems:
\begin{itemize}
\item Generics of various flavors:
equality constraints, subtyping constraints, structural
constraints, use-site variance, definition-site variance
\item Translation from Java wildcards to constraints
\item MyType: add \Xcd{class} property, and add
$\Xcd{p} \ty \Xcd{C} \vdash \Xcd{p}.\Xcd{class} \subtype \Xcd{C}$,
and/or add \Xcd{type} property.
\item Virtual types plus contravariance (translation?)
\item Virtual classes: add an \Xcd{outer} property and
the constraint \Xcd{x}.\Xcd{type} == \Xcd{x}.\Xcd{outer}.\Xcd{T}
\item Generalized constraints from C\#
\item Ownership?  Generic ownership?
\end{itemize}
\end{itemize}


\todo{More positioning, relative to: DML, HM(X), constrained
types (Trifonov, Smith) and subtyping constraints, Java
generics, GJ, PolyJ, C\# generics, virtual types, liquid types}

\todo{Possible claim: first type system
that combines genericity and dep types in some vague general way.}

\todo{Incorporate some text from OOPSLA paper on deptypes.}

\todo{Cite liquid types and whatever it cites}

X10 is a statically typed object-oriented language
designed for high-performance computing~\cite{X10}. The language extends a
class-based sequential core language similar to Java
with constructs for distribution and
fine-grained concurrency.  However, X10 does not yet support
generic types, a standard feature of modern object-oriented languages.
This paper presents a design for generics that is a natural
extension of the language's core dependent type system.

The sequential semantics of X10 are similar to Java's
X10 programs and execute on a Java virtual machine.
After evaluating several existing proposals for generic types in
Java-like
languages~\cite{Java3,adding-wildcards,GJ,Pizza,polyj,thorup97,allen03,allen04,csharp,emir06,scala},
we concluded that these proposals were insufficient for our needs.

% Most of these languages support genericity through parameterized
% types.
A problem with many of these proposals, and in particular with
Java5~\cite{Java3} and Scala~\cite{scala}, is that generic types
are implemented via type erasure.
Our design is not implemented via type erasure and, in addition,
supports run-time introspection of generic types.

Another problem with many of these proposals is inadequate support
for primitive types, especially arrays. The performance of primitive arrays is
critical for the high-performance applications for which
X10 is intended. These proposals introduce unnecessary boxing
and unboxing of primitives.
Our design does not require primitives be boxed.

The design of generics in X10 and is based on its existing
dependent type system~\cite{X10,constrained-types}.
To rule out large classes of errors statically,
X10 provides \emph{constrained types},
a form of dependent type
defined on predicates over the immutable state of
objects~\cite{X10,constrained-types}.
%
The immutable state of an object is captured by its
\emph{value properties}: public final fields of the object.
For instance, the following class declares a two-dimensional
point with properties \xcd"x" and \xcd"y" of type \xcd"float":
\begin{xten}
class Point(x: float, y: float) { }
\end{xten}
A constrained type is a type \xcd"C{e}", where \xcd"C" is a
class and \xcd"e" is a boolean predicate on the properties of
\xcd"C" and the final variables in scope at the type.
For example, given the above class definition,
the type \xcd"Point{x*x+y*y<1}" is the type of all
points within the unit circle.

To support genericity these types are generalized
to allow \emph{type properties} and constraints on these properties.
Like a value property, a type property is an instance member.
The type properties of an object are bound to concrete types
when the object is
created.
Types may be defined by constraining the type properties as
well as the value properties of a class.

The following code declares a class \xcd"Cell" with a type
property named \xcd"T".
\begin{xten}
class Cell[T] {
    var x: T;
    def get(): T = x;
    def set(x: T) = { this.x = x: }
}
\end{xten}
The class has a mutable field \xcd"x",
and has \xcd"get" and \xcd"set" methods for accessing the field.

This example shows that type properties are in many ways similar to
type parameters as provided in object-oriented languages such as
Java and Scala.
Type properties are types in their own right:
they may be used in any context a type may be used,
including in \xcd"instanceof" and cast expressions.
However, the key semantic distinction between type properties
and type parameters is that type properties are instance
members.
Thus, for an expression \xcd"e" of type \xcd"Cell", \xcd"e.T" is
a type, equivalent to the concrete type to which \xcd"T" was
initialized when the object \xcd"e" was instantiated.
To ensure
soundness, \xcd"e" is restricted to final access paths.
Within the body of a class, a property name \xcd"T" resolves
to \xcd"this.T" (or to \xcd"C.this.T" if \xcd"T" is a property of
an enclosing class \xcd"C"), just as value properties are
resolved.

As with value properties, type properties may be constrained
by predicates to produce new types.
X10 supports
equality constraints, written \xcdmath"T$_1$==T$_2$", and
subtyping constraints, written \xcdmath"T$_1$<:T$_2$".
For instance, the type \xcd"Cell{T==String}" is the type of
all \xcd"Cell"s that contain a \xcd"String".

In general, the syntax of a constrained type is
\xcd"C{c}", where \xcd"C" is a base class and
\xcd"c" is a predicate on the properties of \xcd"C".
For brevity, a constraint can be written as
a comma-separated list of conjuncts; that is, the constraint
\xcd"c1"~\xcd"&&"~\xcd"c2" can be written
\xcd"c1,"~\xcd"c2".

Constraints on properties induce a natural subtyping relationship:
\xcd"C{c}" is a subtype of
\xcd"D{d}" if \xcd"C" is a subclass of \xcd"D" and
\xcd"c" entails \xcd"d".

We consider here only constraints on type properties.
See Nystrom et al.~\cite{constrained-types} for a more thorough
presentation of constrained types in X10.
The following are legal types:
\begin{itemize}
\item \xcd"Cell".  This type has no constraints on \xcd"T".
Any type that constrains \xcd"T", those below,
is a subtype of \xcd"Cell".  The type \xcd"Cell" is equivalent to
\xcd"Cell{true}".
%
For a \xcd"Cell" \xcd"c", the return type of the \xcd"get" method
is \xcd"c.T".
Since the property \xcd"T" is unconstrained,
the caller can only assign the return value of \xcd"get"
to a variable of type \xcd"c.T" or of type \xcd"Object".
In the following code, \xcd"y" cannot be passed to the \xcd"set" method
because it is not known if \xcd"Object" is a subtype of \xcd"c.T".
\begin{xten}
val x: c.T = c.get();
val y: Object = c.get();
c.set(x); // legal
c.set(y); // illegal
\end{xten}

\item \xcd"Cell{T==float}".
The type property \xcd"T" is bound to \xcd"float".
Assuming \xcd"c" has this type, the following code is legal:
\begin{xten}
val x: float = c.get();
c.set(1.0);
\end{xten}
The type of \xcd"c.get()" is \xcd"c.T", which is equivalent to
\xcd"float".
Similarly, the \xcd"set" method takes a \xcd"float" as argument.

\item \xcd"Cell{T<:int}".
This type constrains \xcd"T" to be a subtype of \xcd"int".
All instances of this type must bind \xcd"T" to a subtype of \xcd"int".
The following expressions have this type:
\begin{xten}
new Cell[int](1);
new Cell[int{self==3}](3);
\end{xten}
The cell in the first expression may contain any \xcd"int".
The cell in the second expression may contain only \xcd"3".
%
If \xcd"c" has the type \xcd"Cell{T<:int}",
then \xcd"c.get()" has type \xcd"c.T", which is an unknown but
fixed subtype of \xcd"int".  The \xcd"set" method of \xcd"c" can
only be called with an object of type \xcd"c.T".

\item \xcd"Cell{T:>String}".  This type bounds the type property
\xcd"T"
from below.  The \xcd"set" method may be called with any
supertype of \xcd"String"; the return type of the \xcd"get"
method is known to be a
supertype of \xcd"String" (and implicitly a subtype of \xcd"Object").
\end{itemize}

For brevity, the constraint may be omitted and
interpreted as \xcd"true".
The syntax
\xcd"C[T1,...,Tm](e1,...,en)" is sugar for
\xcd"C{X1==T1,...,Xm==Tm,x1==e1,...,xn==en}"
where \xcd"Xi" are the type properties and \xcd"xi" are the
value properties of \xcd"C".
If either list of properties is empty, it may be omitted.

In this shortened syntax, a type argument \xcd"T" used may also be annotated
with
a \emph{use-site variance tag}, either \xcd"+" or \xcd"-":
if \xcd"X" is a type property, then
the syntax \xcd"C[+T]" is sugar \xcd"C{X<:T}" and
\xcd"C[-T]" is sugar \xcd"C{X:>T}"; of course,
\xcd"C[T]" is sugar \xcd"C{X==T}".

The rest of the paper\dots

\section{X10}

Before describing the generic type system, we present an
overview of X10's syntax and semantics.
X10 is a class-based language similar to Java or Scala.
Superficially, the language may be thought of as sequential
Java with some elements of Scala syntax and with new constructs
for concurrency and distribution.
Like Java, the language provides both classes and interfaces; it does not
yet support traits, as found in Scala.

Both classes and interfaces may define properties. Value
properties may be considered to be public final fields. Whereas
Java supports only static fields in interfaces, X10
allows interfaces to define value properties. Any class implementing
an interface must declare, and initialize in its
constructor,
the properties inherited from the interface.

Classes may define fields, methods, and constructors. The
declaration syntax,
illustrated in the \xcd"Cell" example
above,
is similar to Scala's.  Fields may be
declared either \xcd"val" or \xcd"var".  A \xcd"val" is n
\emph{final} and must be assigned exactly once.  Methods are
declared with a \xcd"def" keyword.
As in Java, methods may be declared \xcd"static", however fields cannot.
Constructor syntax is
similar to method syntax and X10 adopts Scala's convention of
using the name \xcd"this" for constructors.
In X10, constructors have a return type, which constrains
the properties of the new object.

\if 0
\section{Typing}

As stated above, if \xcd"p" is a final access path of a
type with property \xcd"T", then \xcd"p.T" is a legal type.
Given an expression context \xcd"E"$[\cdot]$, if
the expression \xcd"E"$[{\tt x}]$ has type \xcd"x.T",
then the expression \xcd"E"$[{\tt e}]$ has type \xcd"z.T",
where ${\tt z} = {\tt e}$ if \xcd"e" is a final access path
or else \xcd"z" is a fresh variable.
\fi

\section{Generics}

\subsection{Subtyping}

\xcd"C{c}" is a subtype of \xcd"D{d}" if \xcd"C" is a subclass
of \xcd"D" and \xcd"c" entails \xcd"d".

\subsection{Use-site variance}

\subsection{Class declarations}

Classes may be declared with any number of type properties and
value properties.  These properties can be constrained with a
\emph{class invariant}, specified by a \xcd"where" clause,
a predicate on the properties of any instance of the class.
%
The general form of a class definition is:
\begin{xtenmath}
class C[X$_1$, $\dots$, X$_p$](x$_1$: T$_1$, $\dots$, x$_k$: T$_k$)
      where c
      extends B{c$_0$}
      implements I$_1${c$_1$}, $\dots$, I$_n${c$_n$} {$\dots$}
\end{xtenmath}

\subsection{Definition-site variance}

In a class definition,
a type property may be declared with a \emph{definition-site variance tag}, either \xcd"+" or
\xcd"-".  A \xcd"+" tag indicates that the class is covariant on
the property; that is, given a definition
\xcd"Cell[+T]",
if \xcd"A" $\subtype$ \xcd"B", then
\xcd"Cell[A]" $\subtype$ \xcd"Cell[B]".
Similarly,
\xcd"Cell[-T]" indicates that \xcd"T" is contravariant in \xcd"Cell";
that is, if \xcd"A" $\subtype$ \xcd"B", then
\xcd"Cell[B]" $\subtype$ \xcd"Cell[A]".

A definition-site variance tag changes the meaning of the
syntactic sugar for the type \xcd"Cell[A]".
If the property is covariant (i.e., is declared as \xcd"+T"), \xcd"Cell[A]"
is sugar for \xcd"Cell{T<:A}".
If the property is contravariant (\xcd"-T"), then \xcd"Cell[A]"
is sugar for \xcd"Cell{T:>A}".
Otherwise, the property is invariant and \xcd"Cell[A]"
is sugar for \xcd"Cell{T==A}".

The compiler should issue a warning if
a covariant property is used in a negative position (e.g., in a
method parameter type)
in its class definition,
or if a contravariant property is used in a positive position
(e.g., in a method return type).
Without these restrictions, methods or fields with types
dependent on the property would be safe, but not be accessible
using the default instantiation (e.g., \xcd"Cell[int]").

\if 0
More formally, a type property cannot be used in a position
where its actual variance differs from its declared variance.

\infrule{
\vdash_{+} T
\vdash_{-} T
}{
var x \ty T = e
}

\infrule{
\vdash_{+} T
}{
val x \ty T = e
}

\infrule{
\vdash_{-} T1
\vdash_{+} T2
\vdash_{-} c
}{
def f(x: T1) \ty T2 where c = e
}

\infrule{
\vdash_{+}
}{}
\fi

\if 0
%
A type appears in a negative position if:
\begin{itemize}
\item it is the type of a mutable or immutable field; or,
\item it is the type of a method formal parameter; or,
\item it appears in a constraint on a type that is in a negative position
and the constraint bounds it from above; or,
\item it appears in a constraint on a type that is in a positive position
and the constraint bounds it from below.
\end{itemize}
%
A type appears in a positive position if:
\begin{itemize}
\item it is the type of a mutable field; or,
\item it is a method return type; or,
\item it appears in a constraint on a type that is in a negative position
and the constraint bounds it from below; or,
\item it appears in a constraint on a type that is in a positive position
and the constraint bounds it from above.
\end{itemize}
\fi

\subsection{Class invariant}

\subsection{Method parameters}

Methods and constructors may have type parameters.
For instance, the \xcd"List" class below defines a \xcd"map"
method that maps each element of a list of \xcd"T"
to a value of another type \xcd"S", constructing a new list of
\xcd"S".
\begin{xten}
class List[T] {
    val array: Array[T];
    def map[S](f: T => S): List[S] {
        val newArray = new Array[S](array.length);
        for (i in [0:array.length-1]) {
            newArray(i) = f(array(i));
        }
        return new List(newArray);
    }
}
\end{xten}


A parameterized method can is invoked by giving type arguments before the
expression arguments.  For example, the following code takes a
list of \xcd"String"s and returns a list of string lengths of
type \xcd"int"
\begin{xten}
xs: List[String] = ...;
ys: List[int] = xs.map[int](
        (x: String) => x.length());
\end{xten}

\subsection{Method where clauses}

Method and constructor parameters, both value parameters and
type parameters,
can be constrained with a where clause on the method.
For type parameters,
this feature is similar to generalized constraints proposed for
C\#~\cite{emir06}.
%
In the following code, the \xcd"T" parameter is covariant
and so the \xcd"append" methods below are illegal:
\begin{xten}
class List[+T] {
   def append(other: T): List[T] = { ... }
        // illegal
   def append(other: List[T]): List[T] = { ... }
        // illegal
}
\end{xten}
%
However, one can introduce a method parameter and then constrain
the parameter from below by the class's parameter:
For example, in the following code,
\begin{xten}
class List[+T] {
   def append[U](other: U)
        {T <: U}: List[U] = { ... }
   def append[U](other: List[U])
        {T <: U}: List[U] = { ... }
}
\end{xten}

The constraints must be satisfied by the callers of \xcd"append".
For example, in the following code:
\begin{xten}
xs: List[Number];
ys: List[Integer];
xs = ys; // ok
xs.append(1.0); // legal
ys.append(1.0); // illegal
\end{xten}
the call to \xcd"xs.append" is allowed and the result type is \xcd"List[Number]", but
the call to \xcd"ys.append" is not allowed because the caller cannot show that
${\tt Number} \subtype {\tt Double}$.

\subsection{Method overriding}

Legal if any call to super method can call sub method.

covariant return
contravaraint args
weaker where clause

\subsection{Constructor definitions}

Constructors are defined using the syntax \xcd"def this",
as shown in Figure~\ref{fig:grammar}.
%
Constructors must ensure that all properties of the new object
are initialized and that the class invariants of the object's
class and its superclasses and superinterfaces hold.

Properties are initialized with a \xcd"property" statement.
For instance, the
constructor for \xcd"Cell" ensures that the type property \xcd"T" is bound.
\begin{xten}
    def this[T](x: T) =
      { property[T](); this.x = x; }
\end{xten}
The \xcd"property" statement is used to set all the properties
of the new object simultaneously; the syntax is similar to a \xcd"super"
constructor call.

If the \xcd"property" statement is omitted, the compiler implicitly
initailizes the properties from the formal type and value parameters
of the constructor.  The property statement for \xcd"Cell"'s constructor,
for example, could have been omitted.

Constructors have ``return
types'' that can specify an invariant satisfied by the object being
constructed.  The compiler verifies that the
constructor return type and the class invariant are implied by the
\xcd"property" statement and any \xcd"super"
or \xcd"this" calls in the constructor body.

Classes that do not declare a constructor
have a default constructor with a type parameter for each
type property and a value parameter for each value property.

\section{Formal semantics}

\newcommand\gxx{GenX10\xspace}
\newcommand\xbar[1]{\ensuremath{\bar{\Xcd{#1}}}}
\newcommand\tbar[1]{\ensuremath{\bar{\tt {#1}}}}
\newcommand\exc[2]{\ensuremath{\exists}#1.~#2}
\newcommand\exty[3]{\ensuremath{\exists}#1\ty#2.~#3}
\newcommand\extyty[5]{\ensuremath{\exists}#1\ty#2,#3\ty#4.~#5}
\newcommand\extytyty[7]{\ensuremath{\exists}#1\ty#2,#3\ty#4,#5\ty#6.~#7}

We present a core calculus, \gxx, for X10 with generics.
\gxx is based on Constrained Featherweight
Java~\cite{constrained-types}.

\todo{
Add method overriding rules: covariant return, contravariant
args, weaker constraints
}

The grammar for \gxx is shown in 
Figure~\ref{fig:grammar}.  The calculus elides features of the
full X10 language not relevant to this paper.

\begin{figure}[tp]
\begin{center}
\begin{tabular}{lrcl}
program & {\tt P} & ::= & \xbar{L} \\
classes & {\tt L} & ::= &
\xcdmath"class C[$\tbar{X}$]($\tbar{x} \ty \tbar{T}$){c}" \\
& & & \xcdmath"  extends T { $\tbar{M}$ }" \\
base types & {\tt R} & ::= & \xcd"C" \\
            & & \bnf & \xcd"X" \\
            & & \bnf & \xcd"e.X" \\
types & {\tt T} & ::= & \xcd"R" \\
            & & \bnf & \xcd"R{c}" \\
methods     & {\tt M} & ::= &
\xcdmath"def m[$\tbar{X}$]($\tbar{x}$: $\tbar{T}$){c}: T = e" \\
expressions & {\tt e} & ::= & \\
\quad literals        &         &      & \xcd"true" \bnf \xcd"false" \bnf \xcd"null" \bnf $n$ \\
\quad variables       &         & \bnf & \xcd"x" \\
\quad field access    &         & \bnf & \xcdmath"e.x" \\
\quad call            &         & \bnf & \xcdmath"e$_0$.m[$\tbar{T}$]($\tbar{e}$)" \\
%\quad                 &         & \bnf & \xcdmath"e$_0$.m($\tbar{e}$)" \\
\quad new             &         & \bnf & \xcdmath"new C[$\tbar{T}$]($\tbar{e}$)" \\
%\quad                 &         & \bnf & \xcdmath"new C($\tbar{e}$)" \\
\quad cast            &         & \bnf & \xcdmath"e as T" \\
constraint terms & {\tt t} & ::= & \\
\quad self            &         &      & \xcd"self" \\
\quad variables       &         & \bnf & \xcd"x" \\
\quad properties      &         & \bnf & \xcd"t".\xcd"x" \\
\quad atoms           &         & \bnf & \xcdmath"g(t$_1$,$\dots$,t$_n$)" \\
\quad new             &         & \bnf & \xcdmath"new C(t$_1$,$\dots$,t$_n$)" \\
constraint & {\tt c} & ::=  & \Xcd{true} \\
                  &  & \bnf & $\exc{\Xcd{x}}{\Xcd{c}}$ \\
                  &  & \bnf & $\xbar{c}$ \\
                  &  & \bnf & \xcdmath"p(t$_1$,$\dots$,t$_n$)" \\
environments & $\Gamma$ & ::=  & $\epsilon$ \\
            &          & \bnf & $\Gamma$, $\Xcd{c}$ \\
            &          & \bnf & $\Gamma$, $\Xcd{x} \ty \Xcd{T}$ \\
            &          & \bnf & $\Gamma$, $\Xcd{X} \ty \Xcd{type}$ \\
\end{tabular}
\end{center}
\caption{\gxx grammar}
\label{fig:grammar}
\end{figure}

\begin{figure}[tp]
\begin{center}
\begin{tabular}{lrcl}
types & {\tt T} & ::= & \dots \\
            & & \bnf & $\exty{\Xcd{x}}{\Xcd{T}_0}{\Xcd{T}}$ \\
            & & \bnf & $\exty{\Xcd{X}}{\Xcd{type}}{\Xcd{T}}$ \\
constraint terms & {\tt t} & ::= & \dots \\
\quad literals        &         &      & \xcd"true" \bnf $n$ \bnf \xcd"C" \\
\quad type variables       &         & \bnf & \xcd"X" \\
\quad type properties      &         & \bnf & \xcd"t".\xcd"X" \\
constraint & {\tt c} & ::=  & \dots \\
                  &  & \bnf & $\Xcd{t}_1 \equals \Xcd{t}_2$ \\
                  &  & \bnf & $\Xcd{t}_1 \subtype \Xcd{t}_2$ \\
\end{tabular}
\end{center}
\caption{Extended \gxx grammar}
\label{fig:grammar2}
\end{figure}

The type 
$\exty{\Xcd{x}}{\Xcd{T}}{\Xcd{R\{c\}}}$
is sugar for
$\Xcd{R\{}\exc{\Xcd{x}}{\sigma(\Xcd{x}\ty\Xcd{T}),\Xcd{c}}\Xcd{\}}$.

\subsection{
Constraint system
}

% Handbook of philosophical logic, Gabbay

The X10 compiler permits the constraint system to be extended
with compiler plugins.  The base compiler supports equality
constraints over literals and final variables and subtyping
and equality
constraints over types.
The core constraint system is presented here.  We assume a
constraint solver ${\cal X}$ implementing the plugged-in
constraint systems.

The constraint system does not distinguish between values and
types.  Logical variables \xcd"x" may represent program variables
or type variables.  Path terms \Xcd{p.x} may represent field
accesses or member type references.

The constraint system is shown in Figure~\ref{fig:constraints}.
$\xbar{c}$ is a set of constraints.  The constraint system
satisfies the given structural rules, and supports equality and
subtyping constraints over terms.

\begin{figure}

\infax[I]{ \xbar{c} \vdashC \Xcd{c}_i }

XXX remove
\infrule[Plugin]
{ \xbar{c} \vdashC_{\cal X} \Xcd{d} }
{ \xbar{c} \vdashC \Xcd{d} }

% \infrule[Cut]{ \xbar{c} \vdashC \Xcd{c} \andalso
% \xbar{c}, \Xcd{c} \vdashC \Xcd{d} }
                     % { \xbar{c} \vdashC \Xcd{d} }
% \infrule[Contraction]{ \xbar{c}, \Xcd{c}, \Xcd{c} \vdashC \Xcd{d} }
                     % { \xbar{c}, \Xcd{c} \vdashC \Xcd{d} }
% \infrule[Permutation]{ \xbar{c}, \Xcd{c}, \Xcd{d} \vdashC \Xcd{e} }
                     % { \xbar{c}, \Xcd{d}, \Xcd{c} \vdashC \Xcd{e} }
% \infrule[Extension]{ \xbar{c}, \vdashC \Xcd{c} }
                     % { \xbar{c}, \xbar{c}' \vdashC \Xcd{c} }

XXX check this
\infrule[Ex-I]{
\xbar{c} \vdashC \Xcd{d}[\Xcd{t}/{\Xcd{x}}]
}{
\xbar{c} \vdashC \exc{\Xcd{x}}{\Xcd{d}}
}

XXX check this
\infrule[Ex-E]{
\xbar{c}, \Xcd{d}[\Xcd{t}/\Xcd{x}] \vdashC \Xcd{e}
\andalso
\xbar{c} \vdashC \exc{\Xcd{x}}{\Xcd{d}}
\andalso
\Xcd{x} \not\in \mathit{FV}(\Xcd{e})
}{
\xbar{c} \vdashC \Xcd{e}
}

XXX fix this: cons should be a predicate, not an atom
\infrule[Cons-eq]{
\xbar{c} \vdashC \Xcd{T}_1 \equals \Xcd{T}_2
}{
\xbar{c} \vdashC
\Xcd{cons(}\Xcd{T}_1\Xcd{,z)} \equals 
\Xcd{cons(}\Xcd{T}_2\Xcd{,z)} 
}

\infrule[Cons-sub]{
\xbar{c} \vdashC \Xcd{T}_1 \subtype \Xcd{T}_2
}{
\xbar{c},
\Xcd{cons(}\Xcd{T}_1\Xcd{,z)} \vdashC 
\Xcd{cons(}\Xcd{T}_2\Xcd{,z)} 
}

\infax[Cons]{
\vdashC \Xcd{cons(C,z)}
}

\infrule[Eq-atom]
{ \xbar{c} \vdashC \xbar{s} \equals \xbar{t} }
{ \xbar{c} \vdashC \Xcd{f(}\xbar{s}\Xcd{)} \equals \Xcd{f(}\xbar{t}\Xcd{)} }

% \infrule[Eq-field]{ \xbar{c} \vdashC \Xcd{t}_1 \equals \Xcd{t}_2 }
                  % { \xbar{c} \vdashC \Xcd{t}_1.\Xcd{f} \equals \Xcd{t}_2.\Xcd{f} }
% \infrule[Eq-type]{ \xbar{c} \vdashC \Xcd{t}_1 \equals \Xcd{t}_2 }
                  % { \xbar{c} \vdashC \Xcd{t}_1.\Xcd{X} \equals \Xcd{t}_2.\Xcd{X} }

\infax[Eq-refl]{ \xbar{c} \vdashC \Xcd{t} \equals \Xcd{t} }

\infrule[Eq-trans]{
        \xbar{c} \vdashC \Xcd{t}_1 \equals \Xcd{t}_2
        \andalso
        \xbar{c} \vdashC \Xcd{t}_2 \equals \Xcd{t}_3
        }
        { \xbar{c} \vdashC \Xcd{t}_1 \equals \Xcd{t}_3 }

\infrule[Eq-sym]{
        \xbar{c} \vdashC \Xcd{t}_1 \equals \Xcd{t}_2
        }
        { \xbar{c} \vdashC \Xcd{t}_2 \equals \Xcd{t}_1 }

\infrule[Eq-sub]{
\xbar{c} \vdashC \Xcd{T}_1 \subtype \Xcd{T}_2 \\
\xbar{c} \vdashC \Xcd{T}_2 \subtype \Xcd{T}_1
}{
\xbar{c} \vdashC \Xcd{T}_1 \equals \Xcd{T}_2
}

\infrule[Sub-cons]{
\xbar{c} \vdashC \Xcd{C\{c\}} \ty \Xcd{type}
\andalso
\xbar{c}, \Xcd{c} \vdashC \Xcd{d}
}{
\xbar{c} \vdashC \Xcd{C\{c\}} \subtype \Xcd{C\{d\}}
}

\infrule[Sub-super]{
\mbox{\Xcdmath{C[$\tbar{X}$]($\tbar{x}$: $\tbar{T}$)\{c\} ext T \{\ K\ $\tbar{M}$\ $\tbar{F}$\ \}}}
}{
\vdashC \Xcd{C} \subtype \Xcd{T} \\
}

\infax[Sub-object]{
\vdashC \Xcd{T} \subtype \Xcd{Object}
}

\infrule[Sub-eq]{
\xbar{c} \vdashC \Xcd{T}_1 \equals \Xcd{T}_2
}{
\xbar{c} \vdashC \Xcd{T}_1 \subtype \Xcd{T}_2
}

\infrule[Sub-trans]{
\xbar{c} \vdashC \Xcd{T}_1 \subtype \Xcd{T}_2
\andalso
\xbar{c} \vdashC \Xcd{T}_2 \subtype \Xcd{T}_3
}{
\xbar{c} \vdashC \Xcd{T}_1 \subtype \Xcd{T}_3
}

\caption{Constraints}
\label{fig:subtyping}
\label{fig:constraints}
\end{figure}



\subsection{
Constraint projection
}

First, for a type environment $\Gamma$,
we define the \emph{constraint projection},
$\sigma(\Gamma)$ thus:

\begin{align*}
\sigma(\epsilon) &= \Xcd{true} \\
\sigma(\Gamma, \Xcd{x} \ty \Xcd{T}) &=
        \sigma(\Gamma),
        \cons{\Xcd{T}}{\Xcd{x}}
\\
\sigma(\Gamma, \Xcd{c}) &= \sigma(\Gamma), \Xcd{c} \\
\end{align*}

The auxiliary function $\mathit{cons}$
specifies the constraint for a type \Xcd{T} with \Xcd{self}
bounds to \Xcd{x}.
The constraint projection uses an atomic formula \Xcd{cons},
which is equated to the constraint of \Xcd{T} if \Xcd{T} is not
a type variable.

\begin{align*}
\cons{\Xcd{C}}{\Xcd{z}} &=
    \Xcd{cons(C,z)} \\
\cons{\Xcd{C\{c\}}}{\Xcd{z}} &=
    \Xcd{c}[\Xcd{z}/\Xcd{self}], \Xcd{cons(C\{c\},z)==c}[\Xcd{z}/\Xcd{self}] \\
\cons{\Xcd{p.X}}{\Xcd{z}} &=
    \Xcd{cons(p.X,z)} \\
\cons{\Xcd{X}}{\Xcd{z}} &=
    \Xcd{cons(X,z)} \\
\end{align*}

\noindent
Thus, for example, the constraint projection of the environment:
\begin{quote}
\xcdmath"b: D, a: C{self.X==D{d},self.Y<:b.Z}"
\end{quote}
\noindent is:
\begin{quote}
\xcdmath"a.X==D{d}, a.Y<:b.Z" \\
\end{quote}

\eat{
\subsection{
        Judgments
}

The following judgments will be defined:

\begin{itemize}
\item
     The type {\tt T} is well-formed, given the assumptions $\Gamma$:

    $\Gamma \vdash {\tt T} \ty {\tt type}$

\item
     The type {\tt S} is a subtype of {\tt T}, under the assumption $\Gamma$:

      $\Gamma \vdash {\tt S} \subtype {\tt T}$

    \item The expression {\tt e} is of type {\tt T}, given the assumptions $\Gamma$:

      $\Gamma \vdash {\tt e} \ty {\tt T}$

    \item The method {\tt M} is well-defined for the class {\tt C}
given assumptions $\Gamma$:

      $\Gamma \vdash {\tt M}~\mbox{OK in}~{\tt C}$

    \item The field {\tt f: T} is well defined for the class {\tt C} given assumptions $\Gamma$:

      $\Gamma \vdash {\tt f: T}~\mbox{OK in}~C$

    \item The class definition {\tt L} is well defined given assumptions $\Gamma$:

      $\Gamma \vdash {\tt L}~\mbox{OK}$

\end{itemize}


In what follows we will sometimes think of the family of five
judgments
as a single judgment $\Gamma \vdash \phi$ where $\phi$ ranges over the
formulas 
    ${\tt T} \ty {\tt type}$,
      ${\tt S} \subtype {\tt T}$,
      ${\tt e} \ty {\tt T}$,
      ${\tt M}~\mbox{OK in}~{\tt C}$,
      ${\tt f: T}~\mbox{OK in}~C$, and
      ${\tt L}~\mbox{OK}$.


Now, these judgments need to satisfy certain properties:

\begin{enumerate}

\item
    $\Gamma \vdash {\tt T} \ty {\tt type}$
whenever 
      $\Gamma \vdash {\tt e} \ty {\tt T}$; that is,
if we can conclude that {\tt e}
      has type {\tt T} (under certain assumptions), then under those
      assumptions we must be able to conclude that {\tt T} is well-defined.

\item
    $\Gamma \vdash {\tt S} \ty {\tt type}$ and
    $\Gamma \vdash {\tt T} \ty {\tt type}$ whenever
      $\Gamma \vdash {\tt S} \subtype {\tt T}$.

\item
If 
      $\Gamma \vdash {\tt e} \ty {\tt T}$ and if {\tt x}
is a variable occurring free in ${\tt e} \ty {\tt T}$, then for some
      type {\tt U},
      $\Gamma \vdash {\tt x} \ty {\tt U}$.
That is, all free variables on the right-hand
      side of the judgment are actually defined on the left-hand side.
\end{enumerate}


Keeping in mind these requirements, the rules are as follows. Below,
whenever we use the assertion ``{\tt x} free'' in the antecedent of
a rule we mean
that {\tt x} is not free in the consequent of the rule.


\subsection{
      Structural and Logical Rules
}


First, we present the structural rules for $\vdash$. The
judgment
$\Gamma\vdash {\tt e} \ty {\tt T}$ is
intuitionistic. That is, $\Gamma$ is considered a multiset of assertions, and
the judgment possesses the inference rules:

\infrule{
\Gamma \vdash {\tt e} \ty {\tt T}
\andalso
\Gamma \vdash {\tt S} \ty {\tt type}
\andalso
    \mbox{{\tt x} not in $\mathit{var}(\Gamma)$}
}{
\Gamma, {\tt x} \ty {\tt S} \vdash {\tt e} \ty {\tt T}
}



\infrule{
\Gamma \vdash {\tt e} \ty {\tt T}
\andalso
\Gamma \vdash {\tt c} \ty {\tt boolean}
}{
\Gamma, {\tt c} \vdash {\tt e} \ty {\tt T}
}


We also assume the following rule for conjunctions on the left and right:

\infrule{
\Gamma, \phi_1 , \phi_2 \vdash \phi
}{
\Gamma, (\phi_1 , \phi_2 ) \vdash \phi
}


\infrule{
\Gamma \vdash \phi_1 
\andalso
     \Gamma \vdash \phi_2  
}{
\Gamma \vdash (\phi_1 , \phi_2 )
}



Existential quantification is governed by the following standard rules,
specialized for the particular kinds of formulas we are dealing with:


\infrule{
\Gamma \vdash {\tt e} \ty {\tt T}[{\tt t}/{\tt x}]
\andalso
\Gamma \vdash {\tt t} \ty {\tt S}
}{
\Gamma \vdash {\tt e} \ty ({\tt x} \ty {\tt S};~{\tt T})
}


\infrule{
\Gamma, {\tt x} \ty {\tt S}, {\tt c} \vdash {\tt e} \ty {\tt T}
\andalso
\mbox{{\tt x} fresh}
}{
\Gamma, ({\tt x} \ty {\tt S};~{\tt c}) \vdash {\tt e} \ty {\tt T}
}


\infrule{
\Gamma, {\tt x} \ty {\tt S}, {\tt y} \ty \Xcd{C\{c\}} \vdash {\tt e} \ty {\tt T}
\andalso
\mbox{{\tt x} fresh}
}{
\Gamma, {\tt y} \ty \Xcd{C\{x:S; c\}} \vdash {\tt e} \ty {\tt T}
}

}

\subsection{
Type well-formedness
}

\begin{figure}

\infrule{
\mbox{\Xcdmath{C[$\tbar{X}$]($\tbar{x}$: $\tbar{T}$)\{c\} ext T \{\ $\tbar{M}$\ $\tbar{F}$\ \}}}
}{
\vdash \Xcd{C} \ty \Xcd{type}
}

\infrule{
\Gamma \vdash \Xcd{T} \ty \Xcd{type}
\andalso
\Gamma, \Xcd{self} \ty \Xcd{T} \vdash \Xcd{c} \ty \Xcd{Boolean}
\andalso
\sigma(\Gamma) \vdashC \Xcd{c}~\mbox{OK}
}{
\Gamma \vdash \Xcd{T\{c\}} \ty \Xcd{type}
}

\infrule{
\Gamma \vdash \Xcd{p} \ty \Xcd{T}
\andalso
\Gamma \vdash \Xcd{T}~\Xcd{has}~\Xcd{X}
}{
\Gamma \vdash \Xcd{p.X} \ty \Xcd{type}
}

\infax{
\Gamma, \Xcd{X} \ty \Xcd{type} \vdash \Xcd{X} \ty \Xcd{type}
}

\caption{Type well-formedness}
\label{fig:type-wf}
\end{figure}

\subsection{
      Type inference rules
}

\subsubsection{Constraint rules}

\begin{figure}

\infrule[Has-class]{
\mbox{\Xcdmath{C[$\tbar{X}$]($\tbar{x}$: $\tbar{T}$)\{c\} ext T \{\ K\ $\tbar{M}$\ $\tbar{F}$\ \}}}
}{
\vdash {\tt C}~{\tt has}~{\tt K} \\
\vdash {\tt C}~{\tt has}~{\tt X}_i \\
\vdash {\tt C}~{\tt has}~{\tt x}_i \ty {\tt T}_i \\
\vdash {\tt C}~{\tt has}~{\tt M}_i \\
\vdash {\tt C}~{\tt has}~{\tt F}_i
}

\infrule[Has-sub]{
{\tt Z} \not= {\tt K}
\andalso
\Gamma \vdash {\tt T}_1~{\tt has}~{\tt Z}
\andalso
\sigma(\Gamma) \vdash {\tt T}_2 \subtype {\tt T}_1
}{
\Gamma \vdash {\tt T}_2~{\tt has}~{\tt Z}
}

\caption{Structural constraints}
\label{fig:structural}
\end{figure}

\subsubsection{
        Expression typing judgment
}

\eat{
We define \Xcd{T\{c\}} as follows:

\begin{align*}
\Xcd{D\{c\}} &= \Xcd{D\{c\}} \\
\Xcd{D\{d\}\{c\}} &= \Xcd{D\{d,c\}} \\
\Xcd{X\{c\}} &= \Xcd{X\{c\}} \\
\Xcd{p.X\{c\}} &= \Xcd{p.X\{c\}} \\
\end{align*}

\begin{align*}
\exists\Xcd{x}\ty \Xcd{S}.~\Xcd{C} &= \Xcd{C} \\
\exists\Xcd{x}\ty \Xcd{S}.~\Xcd{C\{c\}} &=
        \Xcd{C\{}\exists\Xcd{x}.~\sigma(\Xcd{x}\ty \Xcd{S})\Xcd{,c\}} \\
\exists\Xcd{x}\ty \Xcd{S}.~\Xcd{p.X} &=
        \Xcd{p.X\{}\exists\Xcd{x}.~\sigma(\Xcd{x}\ty \Xcd{S})\Xcd{,c\}} \\
\exists\Xcd{x}\ty \Xcd{S}.~\Xcd{X} &=
        \Xcd{X\{}\exists\Xcd{x}.~\sigma(\Xcd{x}\ty \Xcd{S})\Xcd{,c\}} \\
\end{align*}
}

\begin{figure}

\infrule[T-sub]{
\Gamma \vdash \Xcd{e} \ty \Xcd{S}
\andalso
\sigma(\Gamma) \vdashC \Xcd{S} \subtype \Xcd{T}
\andalso
\Gamma \vdash \Xcd{T} \ty \Xcd{type}
}{
\Gamma \vdash \Xcd{e} \ty \Xcd{T}
}

\infax[T-bool]{
\vdash \Xcd{true} \ty \Xcd{Boolean\{self==true\}} \\
\vdash \Xcd{false} \ty \Xcd{Boolean\{self==false\}}
}

\infax[T-int]{
\vdash n \ty \Xcd{Int\{self==}n\Xcd{\}}
}

\infrule[T-eq]{
\Gamma \vdash \Xcd{e}_1 \ty \Xcd{T}_1
\andalso
\Gamma \vdash \Xcd{e}_2 \ty \Xcd{T}_2
}{
\Gamma \vdash \Xcd{e}_1 \equals \Xcd{e}_2 \ty
        \extyty{\Xcd{z}_1}{\Xcd{T}_1}{\Xcd{z}_2}{\Xcd{T}_2}
        {\Xcd{Boolean\{self==(}{\Xcd{z}_1}\equals{\Xcd{z}_2}\Xcd{)\}}}
}

\infrule[T-teq]{
\Gamma \vdash \Xcd{T}_1 \ty \Xcd{type}
\andalso
\Gamma \vdash \Xcd{T}_2 \ty \Xcd{type}
}{
\Gamma \vdash \Xcd{T}_1 \equals \Xcd{T}_2 \ty \Xcd{Boolean}
}

\infrule[T-tsub]{
\Gamma \vdash \Xcd{T}_1 \ty \Xcd{type}
\andalso
\Gamma \vdash \Xcd{T}_2 \ty \Xcd{type}
}{
\Gamma \vdash \Xcd{T}_1 \subtype \Xcd{T}_2 \ty \Xcd{Boolean}
}

\infax[T-var]{
\Gamma, \Xcd{x} \ty \Xcd{T} \vdash \Xcd{x} \ty \Xcd{T}
}

\infrule[T-cast]{
\Gamma \vdash \Xcd{e} \ty \Xcd{S}
\andalso
\Gamma \vdash \Xcd{T} \ty \Xcd{type}
}{
\Gamma \vdash \Xcd{e}~\Xcd{as}~\Xcd{T} \ty \Xcd{T}
}

\infrule[T-field]{
\Gamma \vdash \Xcd{e} \ty \Xcd{T}
%\\
%\Xcd{T} = \exty{\Xcd{z}}{\Xcd{S}}{\Xcd{S\{self==z\}}}
\\
\Xcd{T}~\Xcd{has}~\Xcd{f}\Xcd{\{c\}} \ty \Xcd{U}
\\
\sigma(\Gamma, \Xcd{this} \ty \Xcd{T}) \vdashC \Xcd{c}
}{
\Gamma \vdash \Xcd{e}.\Xcd{f} \ty \exty{\Xcd{this}}{\Xcd{T}}{\Xcd{U\{self==this.f\}}}
}

\infrule[T-invk]{
\Gamma \vdash \Xcd{e}_0 \ty \Xcd{T}_0
\andalso
\Gamma \vdash \xbar{e} \ty \xbar{T}
\\
\Xcd{T}_0~\Xcd{has}~\Xcd{def}~\Xcd{m[}\xbar{X}\Xcd{](}\xbar{x} \ty \xbar{S}\Xcd{)\{c\}} \ty {\tt U}~\Xcd{=}~\Xcd{e}
\\
\Gamma' = \Gamma, \xbar{X} \ty \Xcd{type},
        \Xcd{this} \ty \Xcd{T}_0,
        \xbar{x} \ty \xbar{T},
        \xbar{X} \equals \xbar{V}
\\
\sigma(\Gamma') \vdashC \Xcd{c}
\\
\sigma(\Gamma') \vdashC \xbar{T} \subtype \xbar{S}
}{
\Gamma \vdash
\Xcd{e}_0.\Xcd{m[}\xbar{V}\Xcd{](}\xbar{e}\Xcd{)} \ty
\extytyty{\xbar{X}}{\Xcd{type}}{\Xcd{this}}{\Xcd{T}_0}{\xbar{x}}{\xbar{T}}{\Xcd{U}}
}

\infrule[T-new]{
\Gamma \vdash \xbar{e} \ty \xbar{T}
\\
\Xcd{C}~\Xcd{has}~\Xcd{def}~\Xcd{this[}\xbar{X}\Xcd{](}\xbar{x} \ty \xbar{S}\Xcd{)\{c\}} \ty {\tt U}~\Xcd{=}~\dots
\\
\Gamma' = \Gamma, \xbar{X} \ty \Xcd{type}, \Xcd{this} \ty \Xcd{C}, \xbar{x} \ty \xbar{T}, \xbar{V} \equals \xbar{X}
\\
\Gamma'' = \Gamma, \xbar{X} \ty \Xcd{type}, \Xcd{this} \ty \Xcd{U}, \xbar{x} \ty \xbar{T}, \xbar{V} \equals \xbar{X}
\\
\sigma(\Gamma') \vdashC \Xcd{c}
\\
\sigma(\Gamma') \vdashC \xbar{T} \subtype \xbar{S}
\\
\sigma(\Gamma'') \vdashC \mathit{inv}(\Xcd{C}),
}{
\Gamma \vdash
\Xcd{new}~\Xcd{C[}\xbar{V}\Xcd{](}\xbar{e}\Xcd{)} \ty
\extytyty{\xbar{X}}{\Xcd{type}}{\Xcd{this}}{\Xcd{C}}{\xbar{x}}{\xbar{T}}{\Xcd{U}}
}

\caption{Typing rules}
\label{fig:typing}
\end{figure}

The cast rule
\rn{T-cast}
requires that the cast type be well-formed. 

The field access rule \rn{T-field}
differs from the rule in the paper in that there is no need to
substitute a fresh variable for the receiver. Note that {\tt this} may be free
in {\tt S}---that would be a reference to the current object in the code in
which {\tt e.f} occurs, not a reference to the receiver of the {\tt e.f} field
selection (i.e., the object obtained by evaluating {\tt e}).

\noindent
if we allow adding constraints to arbitrary types---do we?

TODO: type parameters!

Now we consider the rule for method invocation. Assume that in a type
environment $\Gamma$ the expressions ${\tt e_0}, \dots, {\tt e_n}$
have the types ${\tt T_0}, \dots, {\tt T_n}$.
Since the
actual values of these expressions are not known, we shall assume that
they take on some fixed but unknown values
                                     ${\tt z_0}, \dots, {\tt z_n}$
of types ${\tt T_0}, \dots, {\tt T_n}$.
Now, for ${\tt z_0}$ as receiver, let us assume that the type
${\tt T_0}$ has a method named ${\tt m}$
with signature
$[\xbar{Z}](\xbar{z} \ty \xbar{S})\Xcd{\{c\}} \to {\tt U}$
(Let ${\tt T_0} = \Xcd{C\{d\}}$.
 If there is no
method named {\tt m} for the class {\tt C} then this method invocation cannot be
type-checked. Without loss of generality, we may assume that the
type parameters of this method are named
                                     ${\tt Z_1}, \dots, {\tt Z_k}$, and
the value parameters are named
                                     ${\tt z_1}, \dots, {\tt z_n}$
since we are free to choose
variable names as we wish.)
Now, for the method to be invocable,
it must be the case that the types
    ${\tt T_1}, \dots, {\tt T_n}$
are subtypes of
    ${\tt S_1}, \dots, {\tt S_n}$.
(Note
that there may be no occurrences of {\tt this} in
    ${\tt S_1}, \dots, {\tt S_n}$---they have been
replaced by ${\tt z_0}$.)
Further, it must be the case that for these parameter
values, the constraint {\tt c} is entailed. Given all these assumptions it
must be the case that the return type is {\tt U}, with all the parameters
    ${\tt z_0}, \dots, {\tt z_n}$
existentially quantified.


\subsubsection{
        Class OK judgment
}

The following rule is modified from what we had in the paper to ensure
that all the types are well-formed (under the assumption {\tt this} \ty {\tt C}).
Note
that the variables $\xbar{x}$ are permitted to occur in the types $\Xcd{T}_0, \xbar{T}$,
hence their typing assertions must be added to $\Gamma$.

\infrule[Method OK]{
\Gamma = \Xcd{this} \ty \Xcd{C\{self==this},\mathit{inv}(\Xcd{C})\Xcd{\}},
        \xbar{x} \ty \xbar{T}\Xcd{\{self==\}}\xbar{x}\Xcd{\}},
        \Xcd{c}
\\
\Gamma \vdash \Xcd{e} \ty \Xcd{U}
\\
\sigma(\Gamma) \vdashC \Xcd{U} \subtype \Xcd{T}
}{
\Xcd{def}~\Xcd{m[}\xbar{X}\Xcd{](}\xbar{x} \ty
\xbar{T}\Xcd{)\{c\}} \ty \Xcd{T}~\Xcd{=}~\Xcd{e}~\mbox{OK in}~\Xcd{C}
}


This rule did not exist in our submission. This is necessary to ensure
that the types of fields are well-formed.

\infrule[Field OK]{
\Xcd{this} \ty \Xcd{C}, \Xcd{c} \vdash \Xcd{T} \ty \Xcd{type}
}{
\Xcd{val}~\Xcd{f}\Xcd{\{c\}} \ty \Xcd{T}~\mbox{OK in}~\Xcd{C}
}


This rule is now modified to ensure that all the types and methods in
the body of the class are well-formed.

\infrule[Class OK]{
K~\mbox{OK in}~{\tt C}
\\
\xbar{M}~\mbox{OK in}~{\tt C}
\\
\xbar{F}~\mbox{OK in}~{\tt C}
\\
{\tt this} \ty {\tt C} \vdash {\tt T} \ty {\tt type}
}{
\mbox{\Xcdmath{C[$\tbar{X}$]($\tbar{x}$: $\tbar{T}$)\{c\} ext T \{\ K\ $\tbar{M}$\ $\tbar{F}$\ \}}}~\mbox{OK}
}

TODO: method overriding


\subsubsection{
        Subtype judgment
}

\infrule{
\sigma(\Gamma) \vdash_{\cal C} {\tt T_1} \subtype {\tt T_2}
}{
\Gamma \vdash {\tt T_1} \subtype {\tt T_2}
}


\section{Odds and ends}

static methods cannot mention T

interfaces can have static methods; a property can implement I,
allowing T.m() static calls

Extensions: structural constraints, optional methods, interfaces
        enabled flag trick


\section{Constraint solver}

The goal of the constraint solver is 
to check an assertion $\xbar{c} \vdashC \Xcd{d}$.

\eat{
Inference

The first step is to normalize constraints
into a set of constraint judgments
$\xbar{c} \vdashC \Xcd{c}$ where $\Xcd{c}$ contains no conjunctions.


Once in normalized form, the inference proceeds as follows:
Select a constraint $\xbar{c} \vdashC \Xcd{c}$.
If not consistent, fail.
If valid, ok.
If not valid, generate assignment of variables that makes it
true, adding the assignment to the assumptions for all
constraints.

The inference algorithm must specify the criteria for:
\begin{itemize}
\item selecting the next constraint to solve
\item generating the variable assignment consistent with all
other constraints (to avoid backtracking)
\end{itemize}

Pick an unassigned variable, find weakest assignment that makes just
this clause true.  Does the weakest assignment exist?

Question: can we ensure each clause involves only one or two
unknowns?
}

We add the following rules to allow type arguments to calls to
be omitted.

\infrule[T-invk-inferred]{
\xbar{Y}~\mbox{fresh}
\\
\Gamma, \xbar{Y} \ty {\tt type}
\vdash
\Xcd{e}_0.\Xcd{m[}\xbar{Y}\Xcd{](}\xbar{e}\Xcd{)} \ty
\Xcd{T}
}{
\Gamma \vdash
\Xcd{e}_0.\Xcd{m(}\xbar{e}\Xcd{)} \ty
\Xcd{T}
}

\infrule[T-new-inferred]{
\xbar{Y}~\mbox{fresh}
\\
\Gamma, \xbar{Y} \ty {\tt type}
\vdash
\Xcd{new}~\Xcd{C[}\xbar{Y}\Xcd{](}\xbar{e}\Xcd{)} \ty
\Xcd{T}
}{
\Gamma \vdash
\Xcd{new}~\Xcd{C(}\xbar{e}\Xcd{)} \ty
\Xcd{T}
}

\subsection{Constraint representation}

%\newcommand\eqedge{\rightleftharpoons}
\newcommand\eqedge{\sim}
\newcommand\flowedge{\to}
\newcommand\treeedge[1]{\mapsto_{#1}}
\newcommand\typeedge{\mapsto_{\tt type}}

Represent a constraint as a graph $G$.
Each node represents a constraint term for a value or a type.
The node for a path $p$ is written $v_p$;
the node for a type $T$ is written $V_T$.
There are four kinds of edges:
\begin{enumerate}
\item undirected equivalence edges,
        $v_p \eqedge v_q$ and $V_S \eqedge V_T$,
\item type edges, $v_p \typeedge V_T$,
\item tree edges, $v_p \treeedge{f} v_{p.f}$
              and $v_p \treeedge{X} V_{p.X}$, and
\item flow edges, $V_S \flowedge V_T$.
\end{enumerate}

First, each constraint term is mapped to a node in the graph as
follows.
Associate each term $t$ with a node
$v_t$.  For each access path {\tt p.x}, add a tree edge
$v_{{\tt p}} \treeedge{{\tt x}} v_{{\tt p.x}}$.
For each path type {\tt p.X}, add a tree edge
$v_{{\tt p}} \treeedge{{\tt X}} V_{{\tt p.X}}$.
For each atomic formula ${\tt f}(\xbar{t})$, add the tree edge
$v_{{\tt f}(\xbar{t})} \treeedge{i} v_{t_i}$ for all $i$.
If term $t$ has type $T$, add $v_t \typeedge V_{t{\tt .type}}$
and
add $V_T \eqedge V_{t{\tt .type}}$ to $G$.

Type nodes are sets of classes.

Next, constraints are incorporated into the graph:

\begin{itemize}
\item
For constraint {\tt p==q}, add $v_{\tt p} \eqedge v_{\tt q}$ to $G$.

\item
For constraint {\tt S==T}, add $V_{\tt S} \eqedge V_{\tt T}$ to $G$.

\item
For constraint {\tt S<:T},
add $V_{\tt S} \flowedge V_{\tt T}$
to $G$.

\end{itemize}

\subsection{Solving}

A flow-path is a path that follows flow and equivalence edges
only.
A type-path is a path that follows type and equivalence edges
only.

Now, we saturate: 
If there is a type-path $v_t \typeedge^* V_{\tt C\{c\}}$,
add $c[t/\Xcd{self}]$ to the worklist.

        Can saturate lazily when doing a lookup.
        EXCEPT: a type may have an arbitrary constraint
                \xcd"C{self.x==3 && y > 7}", so affect is non-local
        EXCEPT: c is x.f==...
                with x: C{c}
                need to avoid infinite loop

To check:

\begin{itemize}
\item To check
constraint {\tt p==q}, check if $v_{\tt p} \eqedge^* v_{\tt q}$.
\item To check
constraint {\tt S<:T}, check if there is a flow-path from $V_{\tt S}$ to
$V_{\tt T}$.  This requires checking entailment of the type constraints and
adding more edges to the graph.  (XXX details!)
Add the flow edge to memoize.
\end{itemize}

\section{Translation}

This section describes an implementation approach for
generic constrained types on a Java virtual machine.
We describe the implementation as a translation to Java.

The design
is a hybrid design based on the implementation of parameterized classes in
NextGen~\cite{allen03,allen04} and the implementation of
PolyJ~\cite{polyj}.
Generic classes are translated into template classes
that are instantiated on demand at run time by binding the type properties
to concrete types.  To implement run-time type checking (e.g.,
casts), type properties are represented at run time
using \emph{adapter objects}.

This design, extended to handle language features
not described in this paper, has been implemented in the X10
compiler.  The X10 compiler is built on the Polyglot framework
and translates X10 source to Java source\footnote{There is also
a translation from X10 to C++ source, not described here.}

\subsection{Method parameters}

The first step in translation is to remove method parameters by
introducing a generic member class for each generic method.
The member class is static iff the method is static.
Constructor type parameters are left unchanged.
After this step, the code consists only of generic classes.
The remaining translation introduces a run-time representation
for the type properties of these classes.

\subsection{Classes}

Each class is translated into a \emph{template class}.
The template class is compiled by a Java compiler (e.g., javac)
to produce a class file.
At run time, when a constrained type \xcd"C{c}" is first referenced, a
class loader loads the template class for \xcd"C" and then transforms the
template class bytecode, specializing it to the constraint
\xcd"c".

For example, consider the following classes.
\begin{xten}
class A[T] {
    var a: T;
}
class C {
    val x: A[Int] = new A[Int]();
    val y: Int = x.a;
}
\end{xten}

The compiler generates the following code:
\begin{xten}
class A {
    // Dummy class needed to type-check uses of T.
    @TypeProperty(1) static class T { }

    T a;

    // Dummy getter and setter; will be eliminated
    // at run time and replaced with actual gets
    // and sets of the field a.
    @Getter("a") <S> S get$a() { return null; }
    @Setter("a") <S> S set$a(S v) { return null; }
}

class C {
    @ActualType("A$Int")
    final A x = Runtime.<A>alloc("A$Int");
    final int y = x.<Integer>get$a();
}
\end{xten}

The member class \xcd"A.T" is used in place of the
type property \xcd"T". 
The \xcd"Runtime.alloc" method is used
used in place of a constructor call.
This code is compiled to Java bytecode.


Then, at run time, suppose the expression \xcd"new C()" is
evaluated.  This causes \xcd"C" to be loaded.
The class loader transforms the bytecode as if it had
been written as follows:

\begin{xten}
class C {
    final A$Int x = new A$Int();
    final int y = x.a;
}
\end{xten}

The \xcd"ActualType" annotation is used to change the
type of the field \xcd"x" from \xcd"A" to \xcd"A$Int".
The call to \xcd"Runtime.alloc" is replaced with a
constructor call.  The call to \xcd"x.get$a()" is
replaced with a field access.

The implementation cannot generate this code directly because
the class \xcd"A$Int" does not yet exist; the Java source compiler
would fail to compile \xcd"C".

Next, as the \xcd"C" object is being constructed, the expression
\xcd"new A$Int()" is evaluated, causing the class \xcd"A$Int" to
be loaded.  The class loader intercepts
this, demangles the name, and loads the bytecode for the
template class \xcd"A".

The bytecode is transformed, replacing the type property \xcd"T"
with the concrete type \xcd"int", the translation of \xcd"Int".

\begin{xten}
class A {
    x10.runtime.Type T;
}

class A$Int extends A {
    int x;
}
\end{xten}

Type properties are mapped to the Java primitive types and to
Object.  Only nine possible instantiations per parameter.
Instantiations used for representation.
Adapter objects used for run time type information.

Could do instantiation eagerly, but quickly gets out of hand without
whole-program analysis to limit the number of instantiations: 9
instantiations for one type property, 81 for two type
properties, 729 for three.

Value constraints are erased from type references.

Constructors are translated to static methods of their enclosing
class.
Constructor calls
are translated to calls to static methods.


Consider the code in Figure~\ref{fig:translation1}.  It contains most of the
features of generics that have to be translated.
\begin{figure*}[tp]
\begin{xten}
class C[T] {
    var x: T;
    def this[T](x: T) { this.x = x; }
    def set(x: T) { this.x = x; }
    def get(): T { return this.x; }
    def map[S](f: T => S): S { return f(this.x); }
    def d() { return new D[T](); }
    def t() { return new T(); }
    def isa(y: Object): boolean { return y instanceof T; }
}

val x : C = new C[String]();
val y : C[int] = new C[int]();
val z : C{T <: Array} = new C[Array[int]]();
x.map[int](f);
new C[int{self==3}]() instanceof C[int{self<4}];
\end{xten}
\caption{Code to translate}
\label{fig:translation1}
\end{figure*}

\subsection{Eliminating method type parameters}

\begin{figure*}[tp]
\begin{xten}
class C[T] where T has T() {
    var x: T;

    def this[T](x: T) { this.x = x; }

    def set(x: T) { this.x = x; }
    def get(): T { return this.x; }

    def d() { return new D[T](); }
    def t() { return new T(); }

    def isa(y: Object): boolean { return y instanceof T; }

    // Translation of map to an inner class
    class map$[T,S] {
        def apply(c: C[T], f: Fun1[T,S]) { return f(c.x); }
    }
}

val x : C = new C[String]();
val y : C[int] = new C[int]();
val z : C{T <: Array} = new C[Array[int]]();
new map$[x.T,int]().apply(x,f);
new C[int{self==3}]() instanceof C[int{self<4}];
\end{xten}
\caption{After removing method parameters}
\label{fig:translation2}
\end{figure*}

\subsection{Translation to Java}

\subsection{Run-time instantiation}

We translate \xcd"instanceof" and cast operations to calls to
methods of a \xcd"Type" because the actual implementation of
the operation may require run-time constraint solving or other
complex code that cannot be easily substituted in when rewriting
the bytecode during instantiation.

\section{Structural constraints}
\label{sec:structural}

XXX this is an extension of the type system

\begin{figure}[tp]
\begin{center}
\begin{tabular}{lrcl}
expressions & {\tt e} & ::= & \dots \\
            &        & \bnf & \xcd"T has Sig" \\
signatures  & {\tt Sig} & ::= &
\xcdmath"def this[$\tbar{X}$]($\tbar{x}$: $\tbar{T}$){c}: T" \\
            &        & \bnf &
\xcdmath"def m[$\tbar{X}$]($\tbar{x}$: $\tbar{T}$){c}: T" \\
            &        & \bnf &
\xcdmath"val x{c}: T" \\
            %&  & \bnf & \xcdmath"var x{c}: T" \\
\end{tabular}
\end{center}
\caption{Grammar for structural constraints}
\label{fig:structural}
\end{figure}

The type system is general enough to support not only subtyping
constraints, but also structural constraints on types.  The type
system need not change except by extending the constraint
system.  The syntax for structural constraints is shown in
Figure~\ref{fig:structural}.

Structural constraints on types are found in many languages.
Haskell~\cite{haskell} supports type classes.
%ML's module system allows modules to be constrained by
%structural signatures~\cite{ml}.
In Modula-3, type equivalence and subtyping are structural
rather than nominal as in object-oriented languages of the C
family such as C++, Java, Scala, and X10.
%
The language PolyJ~\cite{polyj} allows type parameters to be bounded using
structural where clauses, a form of F-bounded
polymorphism~\cite{fbounds}.
For example, a sorted list class in PolyJ can be written as follows:
\begin{xten}
class SortedList[T] where T { int compare(T) } {
    void add(T x) { ... x.compare(y) ... }
}
\end{xten}
The where clause states that the type parameter \xcd"T" must have a
method \xcd"compare" with the given signatures.

To support this, X10 provides structural constraints on types.
The analogous X10 code for \xcd"SortedList" is:
\begin{xten}
class SortedList[T] where T has compare(T): int {
    def add(x: T) = { ... x.compare(y) ... }
}
\end{xten}

A structural constraint is of the form \emph{Type}~\xcd"has"~\emph{Signature}.
A constraint is satisfied if the type has a member of the appropriate name
and with a compatible type.
The constraint \xcd"X has f(T1): T2"
is satisfied by a type \xcd"T" if it has a method \xcd"f"
whose type is a subtype of \xcd"(T1 => T2)"$[{\tt T}/{\tt X}]$.
As an example,
the constraint \xcd"X has equals(X): boolean"
is satisfied by all three of the following classes:
\begin{xten}
class C { def equals(x: C): boolean; }
class D extends C { }
class E { def equals(x: Object): boolean; }
\end{xten}

By using function types and where clauses on constructors,
X10 can go further than PolyJ.
Unlike in PolyJ, where the \xcd"compare" method must be provided by \xcd"T",
in X10 the \xcd"compare" function can be external to \xcd"T".
This is achieved as follows:
\begin{xten}
class SortedList[T] {
    val compare: (T,T) => int;
    def this(cmp: (T,T) => int) = { compare = cmp; }
    def add(x: T) = { ... compare(x,y) ... }
}
\end{xten}

This permits
\xcd"SortedList" to be instantiated using different compare functions:
\begin{xten}
val unixFiles    = new SortedList[String]
                        (String.compareTo.(String));
val windowsFiles = new SortedList[String]
                        (String.compareToIgnoreCase.(String));
\end{xten}

But, a problem with this approach is that the compare function must be
provided to the constructor at each instantiation of \xcd"SortedList".
The problem can be resolved by using constructors with different
structural constraints:
\begin{xten}
class SortedList[T] {
    val compare: (T,T) => int;
    def this[T]() where T has compareTo(T): int = {
        this[T](T.compareTo.(S));
    }
    def this[T](cmp: (T,T) => int) = { compare = cmp; }
    def add(x: T) = { ... compare(x,y) ... }
}
\end{xten}
Now, \xcd"SortedList" can be instantiated with any type that has
a \xcd"compareTo"
method without expliclty specifying the method at each constructor call.

\section{Discussion}

\subsection{Type properties versus type parameters}

Type properties are similar, but not identical to type parameters.  The
differences may potentially confuse programmers used to Java generics or C++
templates.  The key difference is that type properties are instance members and
are thus accessible through access paths: \xcd"e.T" is a legal type.

Type properties, unlike type parameters, are inherited.
For example, in the following code, \xcd"T" is defined in \xcd"List"
and inherited into \xcd"Cons".  The property need not be
declared by the \xcd"Cons" class.
\begin{xten}
class List[T] { }
class Cons extends List {
    def head(): T = { ... }
    def tail(): List[T] = { ... }
}
\end{xten}
The analogous code for \xcd"Cons" using type parameters would be:
\begin{xten}
class Cons[T] extends List[T] {
    def head(): T = { ... }
    def tail(): List[T] = { ... }
}
\end{xten}
% This code is perfectly acceptable in X10 as well, but introduces a redundant
% type property \xcd"T" equal to the \xcd"T" inherited from \xcd"List".

We can make the type system behave as if type properties were
type parameters very simply.  We need only make the syntax \xcd"e.T"
illegal and permit type properties to be accessible only
from within the body of their class definition via the implicit \xcd"this"
qualifier.

\subsection{Wildcards}

Wildcards in Java~\cite{Java3,adding-wildcards} were motivated
by the following example (rewritten in X10 syntax)
from \cite{adding-wildcards}.
Sometimes a class needs a field or method
that is a list, but we don't care what the element type is.
For methods, one can give the method a type parameter:
\begin{xten}
def aMethod[T](list: List[T]) = { ... }
\end{xten}
This method can then be called on any \xcd"List" object.
However, there is no way to do this for fields since they
cannot be parameterized.
Java introduced wildcards to allow such fields to be
typed:
\begin{xten}
List<?> list;
\end{xten}
In X10, a similar effect is achieved by not constraining the
type property of \xcd"List".
One can write the following:
\begin{xten}
list: List;
\end{xten}
Similarly, the method can be written without type parameters by
not constraining \xcd"List":
\begin{xten}
def aMethod(list: List) = { ... }
\end{xten}

In X10, \xcd"List"
is a supertype of
\xcd"List[T]" for any \xcd"T",
just as in Java
\xcd"List<?>" is a supertype of
\xcd"List<T>" for any \xcd"T".
This follows directly from the definition of the type \xcd"List"
as \xcd"List{true}", and the type \xcd"List[T]"
as \xcd"List{X==T}", and the definition of subtyping.

Wildcards in Java can also be bounded.
We achieve the same
effect in X10 by using type constraints.
For instance, the following Java declarations:
\begin{xten}
void aMethod(List<? extends Number> list) { ... }
<T extends Number> void aParameterizedMethod(List<T> list) { ... }
\end{xten}
may be written as follows in X10:
\begin{xten}
def aMethod(list: List{T <: Number}) = { ... }
def aParameterizedMethod[T{self <: Number}](list: List[T]) = { ... }
\end{xten}

Wildcard bounds may be covariant, as in the following example:
\begin{xten}
List<? extends Number> list = new ArrayList<Integer>();
Number num = list.get(0);     // legal
list.set(0, new Double(0.0)); // illegal
list.set(0, list.get(1));     // illegal
\end{xten}
This can also be written in X10, but with an important
difference:
\begin{xten}
list: List{T <: Number} = new ArrayList[Integer]();
num: Number = list.get(0);    // legal
list.set(0, new Double(0.0)); // illegal
list.set(0, list.get(1));     // legal! (when list is final)
\end{xten}
Note because \xcd"list.get" has return type \xcd"list.T", the
last call in above is well-typed in X10; the analogous call in
Java is not well-typed.

Finally,
one can also specify lower bounds on types.  These are useful for
comparators:
\begin{xten}
class TreeSet[T] {
    def this[T](cmp: Comparator{T :> this.T}) { ... }
}
\end{xten}
Here, the comparator for any supertype of \xcd"T" can be used as
to compare \xcd"TreeSet" elements.

Another use of lower bounds is for list operations.
The \xcd"map" method below takes a function that maps a supertype
of the class parameter \xcd"T" to the method type parameter \xcd"S":
\begin{xten}
class List[T] {
    def map[S](fun: Object{self :> T} => S) : List[S] = { ... }
}
\end{xten}

\subsection{Proper abstraction}

Consider the following example adapted from \cite{adding-wildcards}:
\begin{xten}
def shuffle[T](list: List[T]) = {
    for (i: int in [0..list.size()-1]) {
        val xi: T = list(i);
        val j: int = Math.random(list.size());
        list(i) = list(j);
        list(j) = xi;
    }
}
\end{xten}
The method is parameterized on \xcd"T" because the method body needs
the element type to declare the variable \xcd"xi".

However, the method parameter can be omitted by using the type \xcd"list.T"
for \xcd"xi".  Thus, the method can be declared with the signature:
\begin{xten}
def shuffle(list: List) { ... }
\end{xten}
This is called \emph{proper abstraction}.

This example illustrates a key difference between type properties
and type parameters:
A type property is a member of its class, whereas a type parameter is
not.  The names of type properties are visible outside the body of
their class declaration.

\if 0
Type properties can be used as the basis of a parameterized type
system.  This is done simply by making type properties private.
Using the syntactic sugar described above,
the resulting system behaves identically to a system with type
parameters.
\fi

In Java,
Wildcard
capture allows the parameterized method to be called with any \xcd"List",
regardless of its parameter type.
However,
the method parameter cannot be omitted: declaring a parameterless version
of shuffle requires delegating to a private parameterized version that
``opens up'' the parameter.

\subsection{Virtual types}

Type properties share many similarities with virtual types~\cite{mp89-virtual-classes,beta}, particularly
with sound formulations of virtual types using path-dependent types,
as found in gbeta~\cite{ernst99-gbeta}, Scala~\cite{scala},
and J\&~\cite{nqm06}.
%
Constrained types are more expressive than virtual
types since they can be constrained at the use-site,
can be refined on a per-object basis without explicit subclassing,
and can be refined contravariantly
as well as covariantly.

Thorup~\cite{thorup97}
proposed adding genericity to Java using virtual types.  For example,
a generic \xcd"List" class can be written as follows:
\begin{xten}
abstract class List {
    abstract typedef T;
    void add(T element) { ... }
    T get(int i) { ... }
}
\end{xten}
This class can be refined by bounding the virtual type \xcd"T" above:
\begin{xten}
abstract class NumberList extends List {
    abstract typedef T as Number;
}
\end{xten}
And this abstract class can be further refined to \emph{final bind}
\xcd"T" to a particular type:
\begin{xten}
class IntList extends NumberList {
    final typedef T as Integer;
}
\end{xten}
These classes are related by subtyping:
${\tt IntList} \subtype {\tt NumberList} \subtype {\tt List}$.
Only classes where \xcd"T" is final bound can be non-abstract.

In X10, an analogous \xcd"List" class would be written as follows:
\begin{xten}
class List[T] {
    def add(element: T) = { ... }
    def get(i: int): T = { ... }
}
\end{xten}

\xcd"NumberList" and \xcd"IntList" can be written as follows:
\begin{xten}
class NumberList extends List{T<:Number} { }
class IntList extends NumberList{T==Integer} { }
\end{xten}

However, note that X10's \xcd"List" is not abstract.
Instances of \xcd"List"
can instantiate \xcd"T" with a particular type and there is no need to declared classes for \xcd"NumberList" and \xcd"IntList".  Instead, one can use the types
\xcd"List[+Number]" and \xcd"List[Integer]".

Unlike virtual types, type properties can be refined contravariantly.
For instance, one can write the type \xcd"List[-Integer]",
and even \xcd"List{Integer<:T, T<:Number}".

\section{Extensions}

\subsection{Self type}

Add a type property \Xcd{class} to \Xcd{Object}:
\begin{xten}
class Object[class] { }
\end{xten}

The class invariant for all classes \Xcd{C}
implies $\Xcd{this}.\Xcd{class} \subtype \Xcd{C}$.

\subsection{Structural constraints}


\subsection{Ownership}

Consider the following example of generic ownership
derived from Potanin et al.~\cite{ogj-oopsla06}.
\eat{
@inproceedings{ogj-oopsla06,
title = "Generic Ownership for Generic {Java}",
author = "Alex Potanin and James Noble and Dave Clarke and Robert Biddle",
booktitle = "Object-oriented Programming Systems, Languages, and Applications (OOPSLA 2006)"
month = oct,
year = 2006,
location = "Portland, OR"
}
}

\begin{xten}
class Object(owner: Object) { }

// Map inherits Object.owner
// No need to add explicit vOwner and kOwner properties for Key, Value
class Map[Key, Value]{Key <: Comparable, Value <: Object}
{
    private nodes: Vector[Node[Key, Value](this)](this);

    public def put(key: Key, value: Value): Void = {
        nodes.add(new Node[Key, Value](key, value, this)());
    }

    public def get(key: Key): Value = {
        for (mn: Node[Key, Value](this) in nodes) {
            if (mn.key.equals(key))
                return mn.value;
        }
        return null;
    }

    // OGJ will prevent this from being called, since caller
    // can only assign the result to a supertype of Vector(this),
    // which would be only Vector(this) or Object(this)
    // BUT: we have Vector :> Vector(this)
    // Need to require that all class types have an equality constraint
    // on the owner property
    public def exposeVector(): Vector(this) { return nodes; }
}

class Node[Key, Value]
    {Key <: Comparable, Value <: Object}
{
    val key: Key;
    val value: Value;

    public def this[K, V](k: Key, v: Value, o: Object): Node[K, V](o) {
        super(o);               // set the owner
        property[K, V];         // set the type properties
        this.key = k;
        this.value = v;
    }
}
\end{xten}

Restrictions:
\begin{itemize}
\item owner property must be constrained (define this!)
\item owner is always equal to or inside the owner of all other type properties
\item types with an actual owner == this, can only be accessed via this
\end{itemize}


                        


\section{Related work}

\cite{unifying-genericity}
\cite{adding-wildcards}
\cite{emir06}
\cite{myers94}
\cite{polyj}
\cite{allen04}
\cite{allen03}
\cite{beta}
\cite{mp89-virtual-classes}
\cite{thorup97}

\section{Conclusions}

We have presented a preliminary design for supporting genericity
in X10 using type properties.  This type system generalizes the
existing X10 type system.  The use of constraints on type
properties allows
the design to capture many features of generics in languages
like Java 5 and C\# and then to extend these features with new
more expressive power.
We expect that the design admits an efficient
implementation and intend to implement the design shortly.

\section{Acknowledgements} 

The authors thank
Doug Lea, Lex Spoon, Jens Palsberg, Bob Blainey, and Olivier Tardieu
for valuable feedback on versions of the language.
We thank 
Andrew Myers and
Michael Clarkson for providing us with their implementation of
PolyJ, on which our implementation was based, and for many
discussions over the years about parameterized types in Java.

\bibliographystyle{plain}
\bibliography{master}

% \appendix
% \onecolumn

% \section{An extended example}
% {\footnotesize
\begin{verbatim}
/**
   A distributed binary tree.
   @author Satish Chandra 4/6/2006
   @author vj
 */
//                             ____P0
//                            |     |
//                            |     |
//                          _P2  __P0
//                         |  | |   |
//                         |  | |   |
//                        P3 P2 P1 P0
//                         *  *  *  *
// Right child is always on the same place as its parent;
// left child is at a different place at the top few levels of the tree,
// but at the same place as its parent at the lower levels.

class Tree(localLeft: boolean,
           left: nullable Tree(& localLeft => loc=here),
           right: nullable Tree(& loc=here),
           next: nullable Tree) extends Object {
    def postOrder:Tree = {
        val result:Tree = this;
        if (right != null) {
            val result:Tree = right.postOrder();
            right.next = this;
            if (left != null) return left.postOrder(tt);
        } else if (left != null) return left.postOrder(tt);
        this
    }
    def postOrder(rest: Tree):Tree = {
        this.next = rest;
        postOrder
    }
    def sum:int = size + (right==null => 0 : right.sum()) + (left==null => 0 : left.sum)
}
value TreeMaker {
    // Create a binary tree on span places.
    def build(count:int, span:int): nullable Tree(& localLeft==(span/2==0)) = {
        if (count == 0) return null;
        {val ll:boolean = (span/2==0);
         new Tree(ll,  eval(ll => here : place.places(here.id+span/2)){build(count/2, span/2)},
           build(count/2, span/2),count)}
    }
}
\end{verbatim}}

\subsection{Places}
{\footnotesize
\begin{verbatim}
/**

 * This class implements the notion of places in X10. The maximum
 * number of places is determined by a configuration parameter
 * (MAX_PLACES). Each place is indexed by a nat, from 0 to MAX_PLACES;
 * thus there are MAX_PLACES+1 places. This ensures that there is
 * always at least 1 place, the 0'th place.

 * We use a dependent parameter to ensure that the compiler can track
 * indices for places.
 *
 * Note that place(i), for i <= MAX_PLACES, can now be used as a non-empty type.
 * Thus it is possible to run an async at another place, without using arays---
 * just use async(place(i)) {...} for an appropriate i.

 * @author Christoph von Praun
 * @author vj
 */

package x10.lang;

import x10.util.List;
import x10.util.Set;

public value class place (nat i : i <= MAX_PLACES){

    /** The number of places in this run of the system. Set on
     * initialization, through the command line/init parameters file.
     */
    config nat MAX_PLACES;

    // Create this array at the very beginning.
    private constant place value [] myPlaces = new place[MAX_PLACES+1] fun place (int i) {
	return new place( i )(); };

    /** The last place in this program execution.
     */
    public static final place LAST_PLACE = myPlaces[MAX_PLACES];

    /** The first place in this program execution.
     */
    public static final place FIRST_PLACE = myPlaces[0];
    public static final Set<place> places = makeSet( MAX_PLACES );

    /** Returns the set of places from first place to last place.
     */
    public static Set<place> makeSet( nat lastPlace ) {
	Set<place> result = new Set<place>();
	for ( int i : 0 .. lastPlace ) {
	    result.add( myPlaces[i] );
	}
	return result;
    }

    /**  Return the current place for this activity.
     */
    public static place here() {
	return activity.currentActivity().place();
    }

    /** Returns the next place, using modular arithmetic. Thus the
     * next place for the last place is the first place.
     */
    public place(i+1 % MAX_PLACES) next()  { return next( 1 ); }

    /** Returns the previous place, using modular arithmetic. Thus the
     * previous place for the first place is the last place.
     */
    public place(i-1 % MAX_PLACES) prev()  { return next( -1 ); }

    /** Returns the k'th next place, using modular arithmetic. k may
     * be negative.
     */
    public place(i+k % MAX_PLACES) next( int k ) {
	return places[ (i + k) % MAX_PLACES];
    }

    /**  Is this the first place?
     */
    public boolean isFirst() { return i==0; }

    /** Is this the last place?
     */
    public boolean isLast() { return i==MAX_PLACES; }
}
\end{verbatim}}
\subsection{$k$-dimensional regions}
{\footnotesize
\begin{verbatim}
package x10.lang;

/** A region represents a k-dimensional space of points. A region is a
 * dependent class, with the value parameter specifying the dimension
 * of the region.
 * @author vj
 * @date 12/24/2004
 */

public final value class region( int dimension : dimension >= 0 )  {

    /** Construct a 1-dimensional region, if low <= high. Otherwise
     * through a MalformedRegionException.
     */
    extern public region (: dimension==1) (int low, int high)
        throws MalformedRegionException;

    /** Construct a region, using the list of region(1)'s passed as
     * arguments to the constructor.
     */
    extern public region( List(dimension)<region(1)> regions );

    /** Throws IndexOutOfBoundException if i > dimension. Returns the
        region(1) associated with the i'th dimension of this otherwise.
     */
    extern public region(1) dimension( int i )
        throws IndexOutOfBoundException;


    /** Returns true iff the region contains every point between two
     * points in the region.
     */
    extern public boolean isConvex();

    /** Return the low bound for a 1-dimensional region.
     */
    extern public (:dimension=1) int low();

    /** Return the high bound for a 1-dimensional region.
     */
    extern public (:dimension=1) int high();

    /** Return the next element for a 1-dimensional region, if any.
     */
    extern public (:dimension=1) int next( int current )
        throws IndexOutOfBoundException;

    extern public region(dimension) union( region(dimension) r);
    extern public region(dimension) intersection( region(dimension) r);
    extern public region(dimension) difference( region(dimension) r);
    extern public region(dimension) convexHull();

    /**
       Returns true iff this is a superset of r.
     */
    extern public boolean contains( region(dimension) r);
    /**
       Returns true iff this is disjoint from r.
     */
    extern public boolean disjoint( region(dimension) r);

    /** Returns true iff the set of points in r and this are equal.
     */
    public boolean equal( region(dimension) r) {
        return this.contains(r) && r.contains(this);
    }

    // Static methods follow.

    public static region(2) upperTriangular(int size) {
        return upperTriangular(2)( size );
    }
    public static region(2) lowerTriangular(int size) {
        return lowerTriangular(2)( size );
    }
    public static region(2) banded(int size, int width) {
        return banded(2)( size );
    }

    /** Return an \code{upperTriangular} region for a dim-dimensional
     * space of size \code{size} in each dimension.
     */
    extern public static (int dim) region(dim) upperTriangular(int size);

    /** Return a lowerTriangular region for a dim-dimensional space of
     * size \code{size} in each dimension.
     */
    extern public static (int dim) region(dim) lowerTriangular(int size);

    /** Return a banded region of width {\code width} for a
     * dim-dimensional space of size {\code size} in each dimension.
     */
    extern public static (int dim) region(dim) banded(int size, int width);


}

\end{verbatim}}

\subsection{Point}
{\footnotesize
\begin{verbatim}
package x10.lang;

public final class point( region region ) {
    parameter int dimension = region.dimension;
    // an array of the given size.
    int[dimension] val;

    /** Create a point with the given values in each dimension.
     */
    public point( int[dimension] val ) {
        this.val = val;
    }

    /** Return the value of this point on the i'th dimension.
     */
    public int valAt( int i) throws IndexOutOfBoundException {
        if (i < 1 || i > dimension) throw new IndexOutOfBoundException();
        return val[i];
    }

    /** Return the next point in the given region on this given
     * dimension, if any.
     */
    public void inc( int i )
        throws IndexOutOfBoundException, MalformedRegionException {
        int val = valAt(i);
        val[i] = region.dimension(i).next( val );
    }

    /** Return true iff the point is on the upper boundary of the i'th
     * dimension.
     */
    public boolean onUpperBoundary(int i)
        throws IndexOutOfBoundException {
        int val = valAt(i);
        return val == region.dimension(i).high();
    }

    /** Return true iff the point is on the lower boundary of the i'th
     * dimension.
     */
    public boolean onLowerBoundary(int i)
        throws IndexOutOfBoundException {
        int val = valAt(i);
        return val == region.dimension(i).low();
    }
}
\end{verbatim}}

\subsection{Distribution}
{\footnotesize
\begin{verbatim}
package x10.lang;

/** A distribution is a mapping from a given region to a set of
 * places. It takes as parameter the region over which the mapping is
 * defined. The dimensionality of the distribution is the same as the
 * dimensionality of the underlying region.

   @author vj
   @date 12/24/2004
 */

public final value class distribution( region region ) {
    /** The parameter dimension may be used in constructing types derived
     * from the class distribution. For instance,
     * distribution(dimension=k) is the type of all k-dimensional
     * distributions.
     */
    parameter int dimension = region.dimension;

    /** places is the range of the distribution. Guranteed that if a
     * place P is in this set then for some point p in region,
     * this.valueAt(p)==P.
     */
    public final Set<place> places; // consider making this a parameter?

    /** Returns the place to which the point p in region is mapped.
     */
    extern public place valueAt(point(region) p);

    /** Returns the region mapped by this distribution to the place P.
        The value returned is a subset of this.region.
     */
    extern public region(dimension) restriction( place P );

    /** Returns the distribution obtained by range-restricting this to Ps.
        The region of the distribution returned is contained in this.region.
     */
    extern public distribution(:this.region.contains(region))
        restriction( Set<place> Ps );

    /** Returns a new distribution obtained by restricting this to the
     * domain region.intersection(R), where parameter R is a region
     * with the same dimension.
     */
    extern public (region(dimension) R) distribution(region.intersection(R))
        restriction();

    /** Returns the restriction of this to the domain region.difference(R),
        where parameter R is a region with the same dimension.
     */
    extern public (region(dimension) R) distribution(region.difference(R))
        difference();

    /** Takes as parameter a distribution D defined over a region
        disjoint from this. Returns a distribution defined over a
        region which is the union of this.region and D.region.
        This distribution must assume the value of D over D.region
        and this over this.region.

        @seealso distribution.asymmetricUnion.
     */
    extern public (distribution(:region.disjoint(this.region) &&
                                dimension=this.dimension) D)
        distribution(region.union(D.region)) union();

    /** Returns a distribution defined on region.union(R): it takes on
        this.valueAt(p) for all points p in region, and D.valueAt(p) for all
        points in R.difference(region).
     */
    extern public (region(dimension) R) distribution(region.union(R))
        asymmetricUnion( distribution(R) D);

    /** Return a distribution on region.setMinus(R) which takes on the
     * same value at each point in its domain as this. R is passed as
     * a parameter; this allows the type of the return value to be
     * parametric in R.
     */
    extern public (region(dimension) R) distribution(region.setMinus(R))
        setMinus();

    /** Return true iff the given distribution D, which must be over a
     * region of the same dimension as this, is defined over a subset
     * of this.region and agrees with it at each point.
     */
    extern public (region(dimension) r)
        boolean subDistribution( distribution(r) D);

    /** Returns true iff this and d map each point in their common
     * domain to the same place.
     */
    public boolean equal( distribution( region ) d ) {
        return this.subDistribution(region)(d)
            && d.subDistribution(region)(this);
    }

    /** Returns the unique 1-dimensional distribution U over the region 1..k,
     * (where k is the cardinality of Q) which maps the point [i] to the
     * i'th element in Q in canonical place-order.
     */
    extern public static distribution(:dimension=1) unique( Set<place> Q );

    /** Returns the constant distribution which maps every point in its
        region to the given place P.
    */
    extern public static (region R) distribution(R) constant( place P );

    /** Returns the block distribution over the given region, and over
     * place.MAX_PLACES places.
     */
    public static (region R) distribution(R) block() {
        return this.block(R)(place.places);
    }

    /** Returns the block distribution over the given region and the
     * given set of places. Chunks of the region are distributed over
     * s, in canonical order.
     */
    extern public static (region R) distribution(R) block( Set<place> s);


    /** Returns the cyclic distribution over the given region, and over
     * all places.
     */
    public static (region R) distribution(R) cyclic() {
        return this.cyclic(R)(place.places);
    }

    extern public static (region R) distribution(R) cyclic( Set<place> s);

    /** Returns the block-cyclic distribution over the given region, and over
     * place.MAX_PLACES places. Exception thrown if blockSize < 1.
     */
    extern public static (region R)
        distribution(R) blockCyclic( int blockSize)
        throws MalformedRegionException;

    /** Returns a distribution which assigns a random place in the
     * given set of places to each point in the region.
     */
    extern public static (region R) distribution(R) random();

    /** Returns a distribution which assigns some arbitrary place in
     * the given set of places to each point in the region. There are
     * no guarantees on this assignment, e.g. all points may be
     * assigned to the same place.
     */
    extern public static (region R) distribution(R) arbitrary();

}
\end{verbatim}}

\subsection{Arrays}
Finally we can now define arrays. An array is built over a
distribution and a base type.

{\footnotesize
\begin{verbatim}
package x10.lang;

/** The class of all  multidimensional, distributed arrays in X10.

    <p> I dont yet know how to handle B@current base type for the
    array.

 * @author vj 12/24/2004
 */

public final value class array ( distribution dist )<B@P> {
    parameter int dimension = dist.dimension;
    parameter region(dimension) region = dist.region;

    /** Return an array initialized with the given function which
        maps each point in region to a value in B.
     */
    extern public array( Fun<point(region),B@P> init);

    /** Return the value of the array at the given point in the
     * region.
     */
    extern public B@P valueAt(point(region) p);

    /** Return the value obtained by reducing the given array with the
        function fun, which is assumed to be associative and
        commutative. unit should satisfy fun(unit,x)=x=fun(x,unit).
     */
    extern public B reduce(Fun<B@?,Fun<B@?,B@?>> fun, B@? unit);


    /** Return an array of B with the same distribution as this, by
        scanning this with the function fun, and unit unit.
     */
    extern public array(dist)<B> scan(Fun<B@?,Fun<B@?,B@?>> fun, B@? unit);

    /** Return an array of B@P defined on the intersection of the
        region underlying the array and the parameter region R.
     */
    extern public (region(dimension) R)
        array(dist.restriction(R)())<B@P>  restriction();

    /** Return an array of B@P defined on the intersection of
        the region underlying this and the parametric distribution.
     */
    public  (distribution(:dimension=this.dimension) D)
        array(dist.restriction(D.region)())<B@P> restriction();

    /** Take as parameter a distribution D of the same dimension as *
     * this, and defined over a disjoint region. Take as argument an *
     * array other over D. Return an array whose distribution is the
     * union of this and D and which takes on the value
     * this.atValue(p) for p in this.region and other.atValue(p) for p
     * in other.region.
     */
    extern public (distribution(:region.disjoint(this.region) &&
                                dimension=this.dimension) D)
        array(dist.union(D))<B@P> compose( array(D)<B@P> other);

    /** Return the array obtained by overlaying this array on top of
        other. The method takes as parameter a distribution D over the
        same dimension. It returns an array over the distribution
        dist.asymmetricUnion(D).
     */
    extern public (distribution(:dimension=this.dimension) D)
        array(dist.asymmetricUnion(D))<B@P> overlay( array(D)<B@P> other);

    extern public array<B> overlay(array<B> other);

    /** Assume given an array a over distribution dist, but with
     * basetype C@P. Assume given a function f: B@P -> C@P -> D@P.
     * Return an array with distribution dist over the type D@P
     * containing fun(this.atValue(p),a.atValue(p)) for each p in
     * dist.region.
     */
    extern public <C@P, D>
        array(dist)<D@P> lift(Fun<B@P, Fun<C@P, D@P>> fun, array(dist)<C@P> a);

    /**  Return an array of B with distribution d initialized
         with the value b at every point in d.
     */
    extern public static (distribution D) <B@P> array(D)<B@P> constant(B@? b);

}
\end{verbatim}}


\begin{example}
 The code for {\tt List} translates as given in Table~\ref{List-translation}.
\end{example}

\begin{figure*}
{\footnotesize
\begin{verbatim}
  public value class List <Node> {
    public final nat n;   // is a parameter
    nullable Node node = null;
    nullable List<Node> rest = null;  // All assignments must check n = this.n-1.

    /** Returns the empty list. Defined only when the parameter n
        has the value 0. Invocation: new List(0)<Node>().
     */
    public List ( final nat n ) {
      assume n==0;
      this.n = n;
    }

    /** Returns a list of length 1 containing the given node.
        Invocation: new List(1)<Node>( node ).
     */
    public List ( final nat n, Node node ) {
      assume n==1;                         // From the constructor precondition.
      assert 0==0 : "DependentTypeError"; // For the constructor call.
      assert n>=1 : "DependentTypeError"; // For the this call.
      this(n, node, new List<Node>(0));
    }

    public List ( final nat n, Node node, List<Node> rest ) {
      assume n>=1;                               // From the constructor precondition
      assume rest.n==n-1 : "DependentTypeError"; // From the argument type.
      this.n = n;
      this.node = node;
      assert rest.n==n-1 : "DependentTypeError"; // For the field assignment.
      this.rest = rest;
    }

    public  List<Node> append( List<Node> arg ) {
      if (n == 0) {
          final List<Node> result = arg;
          assert n+arg.n == result.n : "DependentTypeError"; // For the return value
          return result;
      } else {
          assume rest.n == n-1;
          final List<Node> argval = rest.append(arg);
          assume argval.n == rest.n+arg.n;
          assert n+arg.n-1== argval.n : "DependentTypeError"; // For the constructor call.
          final List<Node> result = new List<Node>(n+arg.n, node, argval);
          assume result.n == n+arg.n;
          assert n+arg.n == result.n : "DependentTypeError"; // For the return value
          return result;
      }
    }

\end{verbatim}}
\caption{Translation of {\tt List} (contd in Table~\ref{List-translation-2}).}\label{List-translation}
\end{figure*}
\begin{figure*}
{\footnotesize
\begin{verbatim}
    public  List<Node> rev() {
      final List<Node> arg = new List<Node>(0);
      assume arg.n = 0;                           // From the constructor call.
      final List<Node> result = rev( arg );
      assume result.n == n+arg.n;                  // From the method signature
      assert n == result.n : "DependentTypeError"; // For the return value.
      return result;
    }

    public  List(n+arg.n)<Node> rev( final List<Node> arg) {
      if (n==0) {
         assert n+arg.n == arg.n : "DependentTypeError"; // For the return value.
         return arg;
      } else {
        assert 1+arg.n-1=arg.n : "DependentTypeError"; // For the argument to the constructor
        final List<Node> arg2 = new List<Node>(1+arg.n,node, arg));
        assume arg2.n==1+arg.n;                      // From the constructor invocation
        final List<Node> restval = rest;             // Read from a mutable field of parametric type
        assume restval.n == n-1;                     // From the field read.
        final List(restval.n+arg2.n)<Node> result = restval.rev( arg2 );
        assume result.n=restval.n+arg2.n
        assert n+arg.n == result.n                   // For the return value
        return result;
    }

    /** Return a list of compile-time unknown length, obtained by filtering
        this with f. */
    public List<Node> filter(fun<Node, boolean> f) {
         if (n==0) return this;
         if (f(node)) {
           final List<Node> l = rest.filter(f);
           assert l.n+1-1==l.n : "DependentTypeError"; // For the constructor call
           return new List<Node>(l.n+1,node, l);
         } else {
           return rest.filter(f);
         }
    }

    /** Return a list of m numbers from o..m-1. */
    public static  List<nat> gen( final nat m ) {
         assert 0 <= m : "DependentTypeError";        // Precondition for method call.
         final List<nat> result = gen(0,m);
         assume result.n=m-0 : "DependentTypeError";  // From the method signature
         assert m == result.n : "DependentTypeError"; // For the return value
         return result;
    }

    /** Return a list of (m-i) elements, from i to m-1. */
    public static List<nat> gen(final nat i, final nat m) {
      assume i <= m;                                   // Method precondition.
      if (i==m) {
        assert m-i == 0 : "DependentTypeError";        // For the constructor call
        final List result = new List<nat>(m-i);
        assume result.n == 0;                          // From the constructor call.
        assert m-i == result.n : "DependentTypeError"; // For the return value.
        return result;
      } else {
        assert i+1 <= m : "DependentTypeError";        // For the method call.
        final List<nat> arg = gen(i+1,m);
        assume arg.n = m-(i+1);                        // From the method call.
        assert m-i-1 = arg.n;                          // For the constructor invocation.
        final List result = new List<nat>(m-i, i, arg);
        assume result.n = m-i;                         // From the constructor invocation.
        assert m-i == result.n : "DependentTypeError"; // For the return value
        return result;
    }
  }
\end{verbatim}}
\caption{Translation of {\tt List} (continued).}\label{List-translation-2}
\end{figure*}

\section{Type-checking dependent classes}

Each programming language---such as \Xten{}---will specify the base
underlying classes (and the operations on them) which can occur as
types in parameter lists. For instance, in the code for {\tt List}
above, the only type that appears in parameter lists is {\tt int}, and
the only operations on {\tt int} are addition, subtraction, {\tt >=},
{\tt ==}, and the only constants are {\tt 0} and {\tt 1}.  (This
language falls within Presburger arithmetic, a decidable fragment of
arithmetic.)  The compiler must come equipped with a constraint solver
(decision procedure) that can answer questions of the form: does one
constraint entail another?  Constraints are atomic formulas built up
from these operations, using variables. For instance, the compiler
must answer each one of:
{\footnotesize
\begin{verbatim}
  n >= 2 |- n-1 >= 0
  n >= 0, m >= 0 |- m+n >= 0
\end{verbatim}}

Ultimately, the only variables that will occur in constraints are
those that correspond to {\tt config} parameters and those that are
defined by implicit parameter definitions. We need to establish that
the verification of any class will generate only a finite number of
constraints, hence only a finite constraint problem for the constraint
solver.

Second, it should be possible for instances of user-defined classes
(and operations on them) to occur as type parameters. For the compiler
to check conditions involving such values, it is necessary that the
underlying constraint solver be extended.

There are two general ways in which the constraint solver may be
extended.  Both require that the programmer single out some classes
and methods on those classes as {\em pure}. (We shall think of
constants as corresponding to zero-ary methods.) Only instances of
pure classes and expressions involving pure methods on these instances
are allowed in parameter expressions.

How shall constraints be generated for such pure methods? First, the
programmer may explicitly supply with each pure method {\tt T m(T1 x1,
..., Tn xn)} a constraint on {\tt n+2} variables in the constraint
system of the underlying solver that is entailed by {\tt y =
o.m(x1,..., xn)}. Whenever the compiler has to perform reasoning on an
expression involving this method invocation, it uses the constraint
supplied by the programmer. A second more ambitious possibility is
that a symbolic evaluator of the language may be run on the body of
the method to automatically generate the corresponding constraint.

Finally an additional possibility is that the constraint solver itself
be made extensible. In this case, when a user writes a class which is
intended to be used in specifying parameters, he also supplies an
additional program which is used to extend the underlying constraint
solver used by the compiler. This program adds more primitive
constraints and knows how to perform reasoning using these
constraints. This is how I expect we will initially implement the
\Xten{} language. As language designers and implementers we will
provide constraint solvers for finite functions and {\tt Herbrand}
terms on top of arithmetic.





\end{document}
