\documentclass[nocopyrightspace,9pt]{sigplanconf}
%\documentclass{llncs}

\newif\iflncs
\lncsfalse

\usepackage{times-lite}
\usepackage{mathptm}
\usepackage{txtt}
\usepackage{stmaryrd}
\usepackage{code}
\usepackage{bcprules}
%\usepackage{ttquot}
\usepackage{amsmath}
\usepackage{amssymb}
\usepackage{afterpage}
\usepackage{balance}
\usepackage{floatflt}
\usepackage{defs}
\usepackage{utils}
\usepackage[pdftex]{graphicx}
\usepackage{xspace}
\usepackage{ifpdf}
\usepackage{listings}
\usepackage{x10}

\newif\ifsemantics
%\semanticsfalse
\semanticstrue

\hfuzz=1pt

\pagestyle{plain}


\ifpdf
\setlength{\pdfpagewidth}{8.5in}
\setlength{\pdfpageheight}{11in}
\fi


% \input{../../../../vj/res/pagesizes}
% \input{../../../../vj/res/vijay-macros}
\newcommand\alt{\bnf}

\newcommand\Implies{\Rightarrow}

\iflncs
\else
\newtheorem{example}{Example}[section]
\newtheorem{theorem}{Theorem}[section]
\newtheorem{lemma}[theorem]{Lemma}
\newenvironment{proof}{
\trivlist
\item[\hskip \labelsep \textsc{Proof.}]
\selectfont
\ignorespaces}{$\Box$}

%\newtheorem{proof}[theorem]{Proof}
\fi

\begin{document}

\title{Genericity through Dependent Types}

\iflncs

\author{Nathaniel Nystom\inst{1} \and Vijay Saraswat\inst{1}}

\institute{IBM T.~J. Watson Research~Center, P.O.~Box~704, Yorktown~Heights NY 10598 USA,
\email{\{nystrom,vsaraswa\}@us.ibm.com}}

\else

\authorinfo{Nathaniel Nystrom\titlenote{IBM T.~J. Watson Research
Center, P.O. Box 704, Yorktown Heights NY 10598 USA}}{}
  {nystrom@us.ibm.com}
\authorinfo{Vijay Saraswat$^{\;*}$}{}
  {vsaraswa@us.ibm.com}

% \conferenceinfo{POPL'08}{XXX}
% \copyrightyear{2008}
% \copyrightdata{[to be supplied]}

\fi

\maketitle

\begin{abstract}
Genericity is a key requirement for modern object-oriented languages.
In this paper, we describe a design for generic types in the programming language X10.
X10 has an expressive and powerful dependent type system
in which types are specified by constraining the immutable state
of objects.
The immutable state of an object is represented by its
\emph{properties}, public final fields of the object.
A \emph{constrained type} then, is a defined by a class type and
a boolean predicate on the properties of the class.

Generic types are defined in a natural extension of the
dependent type system by first introducing \emph{type properties} into
the language, and then constraining those properties using
subtyping constraints.

The type system presented  here  subsumes the expressive power
of Java's generic types, virtual types,
and generalized constraints proposed for C\#.
The system also admits an efficient implementation 
and eschews the pitfalls of a type erasure semantics.
We describe also a local type inference algorithm for constrained
types that permits type annotations and constraints to be elided
by the programmer.
\end{abstract}

\section{Introduction}

X10 is a statically typed object-oriented language
designed for high-performance computing~\cite{X10}. The language extends a
class-based sequential core language similar to Java
with constructs for distribution and
fine-grained concurrency.  However, X10 does not yet support
generic types, a standard feature of modern object-oriented languages.
This paper presents a design for generics that is a natural
extension of the language's core dependent type system.

The sequential semantics of X10 are similar to Java's
X10 programs and execute on a Java virtual machine.  
After evaluating several existing proposals for generic types in
Java-like
languages~\cite{Java3,adding-wildcards,GJ,Pizza,polyj,thorup97,allen03,allen04,csharp,emir06,scala},
we concluded that these proposals were insufficient for our needs.

% Most of these languages support genericity through parameterized
% types.
A problem with many of these proposals, and in particular with
Java5~\cite{Java3} and Scala~\cite{scala}, is that generic types 
are implemented via type erasure.
Our design is not implemented via type erasure and, in addition,
supports run-time introspection of generic types.

Another problem with many of these proposals is inadequate support
for primitive types, especially arrays. The performance of primitive arrays is
critical for the high-performance applications for which
X10 is intended. These proposals introduce unnecessary boxing
and unboxing of primitives.
Our design does not require primitives be boxed.

The design of generics in X10 and is based on its existing
dependent type system~\cite{X10,constrained-types}.
To rule out large classes of errors statically, 
X10 provides \emph{constrained types},
a form of dependent type
defined on predicates over the immutable state of
objects~\cite{X10,constrained-types}.
%
The immutable state of an object is captured by its
\emph{value properties}: public final fields of the object. 
For instance, the following class declares a two-dimensional
point with properties \xcd"x" and \xcd"y" of type \xcd"float":
\begin{xten}
class Point(x: float, y: float) { }
\end{xten}
A constrained type is a type \xcd"C{e}", where \xcd"C" is a
class and \xcd"e" is a boolean predicate on the properties of
\xcd"C" and the final variables in scope at the type.
For example, given the above class definition,
the type \xcd"Point{x*x+y*y<1}" is the type of all
points within the unit circle.

To support genericity these types are generalized
to allow \emph{type properties} and constraints on these properties.
Like a value property, a type property is an instance member.
The type properties of an object are bound to concrete types
when the object is
created.
Types may be defined by constraining the type properties as
well as the value properties of a class.

The following code declares a class \xcd"Cell" with a type
property named \xcd"T".
\begin{xten}
class Cell[T] {
    var x: T;
    def get(): T = x;
    def set(x: T) = { this.x = x: }
}
\end{xten}
The class has a mutable field \xcd"x",
and has \xcd"get" and \xcd"set" methods for accessing the field.

This example shows that type properties are in many ways similar to
type parameters as provided in object-oriented languages such as
Java and Scala.
Type properties are types in their own right:
they may be used in any context a type may be used,
including in \xcd"instanceof" and cast expressions.
However, the key semantic distinction between type properties
and type parameters is that type properties are instance
members.
Thus, for an expression \xcd"e" of type \xcd"Cell", \xcd"e.T" is
a type, equivalent to the concrete type to which \xcd"T" was
initialized when the object \xcd"e" was instantiated. 
To ensure
soundness, \xcd"e" is restricted to final access paths.
Within the body of a class, a property name \xcd"T" resolves
to \xcd"this.T" (or to \xcd"C.this.T" if \xcd"T" is a property of
an enclosing class \xcd"C"), just as value properties are
resolved.

As with value properties, type properties may be constrained
by predicates to produce new types.
X10 supports
equality constraints, written \xcdmath"T$_1$==T$_2$", and
subtyping constraints, written \xcdmath"T$_1$<:T$_2$".
For instance, the type \xcd"Cell{T==String}" is the type of
all \xcd"Cell"s that contain a \xcd"String".

In general, the syntax of a constrained type is
\xcd"C{c}", where \xcd"C" is a base class and
\xcd"c" is a predicate on the properties of \xcd"C".
For brevity, a constraint can be written as
a comma-separated list of conjuncts; that is, the constraint
\xcd"c1"~\xcd"&&"~\xcd"c2" can be written
\xcd"c1,"~\xcd"c2".

Constraints on properties induce a natural subtyping relationship:
\xcd"C{c}" is a subtype of
\xcd"D{d}" if \xcd"C" is a subclass of \xcd"D" and
\xcd"c" entails \xcd"d".

We consider here only constraints on type properties. 
See Nystrom et al.~\cite{constrained-types} for a more thorough
presentation of constrained types in X10.
The following are legal types:
\begin{itemize}
\item \xcd"Cell".  This type has no constraints on \xcd"T".
Any type that constrains \xcd"T", those below,
is a subtype of \xcd"Cell".  The type \xcd"Cell" is equivalent to
\xcd"Cell{true}".
%
For a \xcd"Cell" \xcd"c", the return type of the \xcd"get" method 
is \xcd"c.T".  
Since the property \xcd"T" is unconstrained,  
the caller can only assign the return value of \xcd"get"
to a variable of type \xcd"c.T" or of type \xcd"Object".
In the following code, \xcd"y" cannot be passed to the \xcd"set" method
because it is not known if \xcd"Object" is a subtype of \xcd"c.T".
\begin{xten}
val x: c.T = c.get();
val y: Object = c.get();
c.set(x); // legal
c.set(y); // illegal
\end{xten}

\item \xcd"Cell{T==float}".
The type property \xcd"T" is bound to \xcd"float".
Assuming \xcd"c" has this type, the following code is legal:
\begin{xten}
val x: float = c.get();
c.set(1.0);
\end{xten}
The type of \xcd"c.get()" is \xcd"c.T", which is equivalent to
\xcd"float".
Similarly, the \xcd"set" method takes a \xcd"float" as argument.

\item \xcd"Cell{T<:int}".
This type constrains \xcd"T" to be a subtype of \xcd"int".
All instances of this type must bind \xcd"T" to a subtype of \xcd"int".
The following expressions have this type:
\begin{xten}
new Cell[int](1);
new Cell[int{self==3}](3);
\end{xten}
The cell in the first expression may contain any \xcd"int".
The cell in the second expression may contain only \xcd"3".
%
If \xcd"c" has the type \xcd"Cell{T<:int}",
then \xcd"c.get()" has type \xcd"c.T", which is an unknown but
fixed subtype of \xcd"int".  The \xcd"set" method of \xcd"c" can
only be called with an object of type \xcd"c.T".

\item \xcd"Cell{T:>String}".  This type bounds the type property
\xcd"T"
from below.  The \xcd"set" method may be called with any
supertype of \xcd"String"; the return type of the \xcd"get"
method is known to be a
supertype of \xcd"String" (and implicitly a subtype of \xcd"Object").
\end{itemize}

For brevity, the constraint may be omitted and
interpreted as \xcd"true".
The syntax 
\xcd"C[T1,...,Tm](e1,...,en)" is sugar for
\xcd"C{X1==T1,...,Xm==Tm,x1==e1,...,xn==en}"
where \xcd"Xi" are the type properties and \xcd"xi" are the
value properties of \xcd"C".  
If either list of properties is empty, it may be omitted.

In this shortened syntax, a type argument \xcd"T" used may also be annotated
with
a \emph{use-site variance tag}, either \xcd"+" or \xcd"-":
if \xcd"X" is a type property, then
the syntax \xcd"C[+T]" is sugar \xcd"C{X<:T}" and
\xcd"C[-T]" is sugar \xcd"C{X:>T}"; of course,
\xcd"C[T]" is sugar \xcd"C{X==T}".

The rest of the paper...

\section{Overview of X10 syntax}

Before describing the generic type system, we present an
overview of X10's syntax and semantics.
X10 is a class-based language similar to Java or Scala.
Superficially, the language may be thought of as sequential
Java with some elements of Scala syntax and with new constructs
for concurrency and distribution.
Like Java, the language provides both classes and interfaces; it does not
yet support traits, as found in Scala.

Both classes and interfaces may define properties. Value
properties may be considered to be public final fields. Whereas
Java supports only static fields in interfaces, X10
allows interfaces to define value properties. Any class implementing
an interface must declare, and initialize in its
constructor,
the properties inherited from the interface.

Classes may define fields, methods, and constructors. The
declaration syntax,
illustrated in the \xcd"Cell" example
above,
is similar to Scala's.  Fields may be
declared either \xcd"val" or \xcd"var".  A \xcd"val" is n
\emph{final} and must be assigned exactly once.  Methods are
declared with a \xcd"def" keyword.
As in Java, methods may be declared \xcd"static", however fields cannot.
Constructor syntax is
similar to method syntax and X10 adopts Scala's convention of
using the name \xcd"this" for constructors.
In X10, constructors have a return type, which constrains
the properties of the new object.

\begin{figure}[tp]
\begin{center}
\begin{tabular}{lrcl}
class & {\tt L} & ::= &
\xcdmath"class C[$\bar{\tt X}$]($\bar{\tt x}$: $\bar{\tt T}$){c}" \\
& & & \xcdmath"  extends T { $\bar{\tt K}$ $\bar{\tt M}$ $\bar{\tt F}$ }" \\
type & {\tt T} & ::= & \xcd"C" \\
            & & \bnf & \xcd"e.X" \\
            & & \bnf & \xcd"T{c}" \\
constructor & {\tt K} & ::= &
\xcdmath"def this[$\bar{\tt X}$]($\bar{\tt x}$: $\bar{\tt T}$){c}: T = e" \\
method      & {\tt M} & ::= &
\xcdmath"def m[$\bar{\tt X}$]($\bar{\tt x}$: $\bar{\tt T}$){c}: T = e" \\
field       & {\tt F} & ::= &
\xcdmath"val x{c}: T = e" \\
            &  & \bnf &
\xcdmath"var x{c}: T = e" \\
constraint & {\tt c} & ::= & \xcd"e" \\
expression & {\tt e} & ::= & \xcd"true" \\
                  &  & \bnf & \xcd"x" \\
                  &  & \bnf & \xcdmath"e$_1$ && e$_2$" \\
                  &  & \bnf & \xcdmath"e$_1$ == e$_2$" \\
                  &  & \bnf & \xcdmath"T$_1$ <: T$_2$" \\
                  &  & \bnf & \xcdmath"T$_1$ == T$_2$" \\
                  &  & \bnf & \dots \\
\end{tabular}
\end{center}
\caption{X10 grammar}
\label{fig:grammar}
\end{figure}

A subset of the grammar for X10 is shown in
Figure~\ref{fig:grammar}.  We elide features of the language
not relevant to this paper.  In the grammar $[ \alpha ]$ denotes
an optional occurrence of the sequence of symbols $\alpha$,
$\alpha^*$ denotes zero or more occurrences of $\alpha$, and
$\alpha^+$ denotes one or more occurrences of $\alpha$.

\if 0
\section{Typing}

As stated above, if \xcd"p" is a final access path of a
type with property \xcd"T", then \xcd"p.T" is a legal type.
Given an expression context \xcd"E"$[\cdot]$, if
the expression \xcd"E"$[{\tt x}]$ has type \xcd"x.T",
then the expression \xcd"E"$[{\tt e}]$ has type \xcd"z.T",
where ${\tt z} = {\tt e}$ if \xcd"e" is a final access path
or else \xcd"z" is a fresh variable.
\fi

\section{Generic types}

\subsection{Use-site variance}

\subsection{Subtyping}

\xcd"C{c}" is a subtype of \xcd"D{d}" if \xcd"C" is a subclass
of \xcd"D" and \xcd"c" entails \xcd"d".

\section{Generic classes}

Classes may be declared with any number of type properties and
value properties.  These properties can be constrained with a
\emph{class invariant}, specified by a \xcd"where" clause,
a predicate on the properties of any instance of the class.
%
The general form of a class definition is:
\begin{xtenmath}
class C[X$_1$, $\dots$, X$_p$](x$_1$: T$_1$, $\dots$, x$_k$: T$_k$)
      where c
      extends B{c$_0$}
      implements I$_1${c$_1$}, $\dots$, I$_n${c$_n$} {$\dots$}
\end{xtenmath}

\subsection{Definition-site variance}

In a class definition, 
a type property may be declared with a \emph{definition-site variance tag}, either \xcd"+" or
\xcd"-".  A \xcd"+" tag indicates that the class is covariant on
the property; that is, given a definition
\xcd"Cell[+T]",
if \xcd"A" $\subtype$ \xcd"B", then
\xcd"Cell[A]" $\subtype$ \xcd"Cell[B]".
Similarly,
\xcd"Cell[-T]" indicates that \xcd"T" is contravariant in \xcd"Cell";
that is, if \xcd"A" $\subtype$ \xcd"B", then
\xcd"Cell[B]" $\subtype$ \xcd"Cell[A]".

A definition-site variance tag changes the meaning of the
syntactic sugar for the type \xcd"Cell[A]".
If the property is covariant (i.e., is declared as \xcd"+T"), \xcd"Cell[A]"
is sugar for \xcd"Cell{T<:A}".
If the property is contravariant (\xcd"-T"), then \xcd"Cell[A]"
is sugar for \xcd"Cell{T:>A}".
Otherwise, the property is invariant and \xcd"Cell[A]"
is sugar for \xcd"Cell{T==A}".

The compiler should issue a warning if 
a covariant property is used in a negative position (e.g., in a
method parameter type)
in its class definition,
or if a contravariant property is used in a positive position
(e.g., in a method return type).
Without these restrictions, methods or fields with types
dependent on the property would be safe, but not be accessible 
using the default instantiation (e.g., \xcd"Cell[int]").

\if 0
More formally, a type property cannot be used in a position
where its actual variance differs from its declared variance.

\infrule{
\vdash_{+} T
\vdash_{-} T
}{
var x : T = e
}

\infrule{
\vdash_{+} T
}{
val x : T = e
}

\infrule{
\vdash_{-} T1
\vdash_{+} T2
\vdash_{-} c
}{
def f(x: T1) : T2 where c = e
}

\infrule{
\vdash_{+}
}{}
\fi

\if 0
%
A type appears in a negative position if:
\begin{itemize}
\item it is the type of a mutable or immutable field; or,
\item it is the type of a method formal parameter; or,
\item it appears in a constraint on a type that is in a negative position 
and the constraint bounds it from above; or,
\item it appears in a constraint on a type that is in a positive position 
and the constraint bounds it from below.
\end{itemize}
%
A type appears in a positive position if:
\begin{itemize}
\item it is the type of a mutable field; or,
\item it is a method return type; or,
\item it appears in a constraint on a type that is in a negative position 
and the constraint bounds it from below; or,
\item it appears in a constraint on a type that is in a positive position 
and the constraint bounds it from above.
\end{itemize}
\fi

\subsection{Class invariant}

\subsection{Method parameters}

Methods and constructors may have type parameters.
For instance, the \xcd"List" class below defines a \xcd"map"
method that maps each element of a list of \xcd"T"
to a value of another type \xcd"S", constructing a new list of
\xcd"S".
\begin{xten}
class List[T] {
    val array: Array[T];
    def map[S](f: T => S): List[S] {
        val newArray = new Array[S](array.length);
        for (i in [0:array.length-1]) {
            newArray(i) = f(array(i));
        }
        return new List(newArray);
    }
}
\end{xten}


A parameterized method can is invoked by giving type arguments before the 
expression arguments.  For example, the following code takes a
list of \xcd"String"s and returns a list of string lengths of
type \xcd"int"
\begin{xten}
xs: List[String] = ...;
ys: List[int] = xs.map[int]((x: String) => x.length());
\end{xten}

\subsection{Method where clauses}

Method and constructor parameters, both value parameters and
type parameters, 
can be constrained with a where clause on the method.
For type parameters,
this feature is similar to generalized constraints proposed for
C\#~\cite{emir06}.
%
In the following code, the \xcd"T" parameter is covariant
and so the \xcd"append" methods below are illegal:
\begin{xten}
class List[+T] {
   def append(other: T): List[T] = { ... }        // illegal
   def append(other: List[T]): List[T] = { ... }  // illegal
}
\end{xten}
%
However, one can introduce a method parameter and then constrain
the parameter from below by the class's parameter:
For example, in the following code,
\begin{xten}
class List[+T] {
   def append[U](other: U): List[U] where T <: U = { ... }
   def append[U](other: List[U]): List[U] where T <: U = { ... }
}
\end{xten}

The constraints must be satisfied by the callers of \xcd"append".
For example, in the following code:
\begin{xten}
xs: List[Number];
ys: List[Integer];
xs = ys; // ok
xs.append(1.0); // legal
ys.append(1.0); // illegal
\end{xten}
the call to \xcd"xs.append" is allowed and the result type is \xcd"List[Number]", but
the call to \xcd"ys.append" is not allowed because the caller cannot show that
${\tt Number} \subtype {\tt Double}$.

\subsection{Method overriding}

Legal if any call to super method can call sub method.

covariant return
contravaraint args
weaker where clause

\subsection{Constructor definitions}

Constructors are defined using the syntax \xcd"def this":

\medskip

\begin{tabular}{rcl}
\emph{ConstructorDef}& ::=  &
                \xcd"def"~\xcd"this"
                        [ \xcd"["~\emph{TypeParameters}~\xcd"]" ]
                        \xcd"("~[\emph{Formals}]~\xcd")"
                        [ \xcd":"~\emph{Type} ]
\\ & & \quad
                        [ \xcd"where"~\emph{Constraint} ]
                        \xcd"="~\emph{Expression}~\xcd";" \\
\end{tabular}

\medskip

Constructors must ensure that all properties of the new object
are initialized and that the class invariants of the object's
class and its superclasses and superinterfaces hold.

Properties are initialized with a \xcd"property" statement.
For instance, the
constructor for \xcd"Cell" ensures that the type property \xcd"T" is bound.
\begin{xten}
    def this[T](x: T) = { property[T](); this.x = x; }
\end{xten}
The \xcd"property" statement is used to set all the properties
of the new object simultaneously; the syntax is similar to a \xcd"super"
constructor call.

If the \xcd"property" statement is omitted, the compiler implicitly
initailizes the properties from the formal type and value parameters
of the constructor.  The property statement for \xcd"Cell"'s constructor,
for example, could have been omitted.

Constructors have ``return
types'' that can specify an invariant satisfied by the object being
constructed.  The compiler verifies that the
constructor return type and the class invariant are implied by the
\xcd"property" statement and any \xcd"super"
or \xcd"this" calls in the constructor body.

Classes that do not declare a constructor 
have a default constructor with a type parameter for each
type property and a value parameter for each value property.

\section{Discussion}

\subsection{Type properties versus type parameters}

Type properties are similar, but not identical to type parameters.  The
differences may potentially confuse programmers used to Java generics or C++
templates.  The key difference is that type properties are instance members and
are thus accessible through access paths: \xcd"e.T" is a legal type.

Type properties, unlike type parameters, are inherited.
For example, in the following code, \xcd"T" is defined in \xcd"List"
and inherited into \xcd"Cons".  The property need not be
declared by the \xcd"Cons" class.
\begin{xten}
class List[T] { }
class Cons extends List {
    def head(): T = { ... }
    def tail(): List[T] = { ... }
}
\end{xten}
The analogous code for \xcd"Cons" using type parameters would be:
\begin{xten}
class Cons[T] extends List[T] {
    def head(): T = { ... }
    def tail(): List[T] = { ... }
}
\end{xten}
% This code is perfectly acceptable in X10 as well, but introduces a redundant
% type property \xcd"T" equal to the \xcd"T" inherited from \xcd"List".

We can make the type system behave as if type properties were
type parameters very simply.  We need only make the syntax \xcd"e.T"
illegal and permit type properties to be accessible only
from within the body of their class definition via the implicit \xcd"this"
qualifier.

\subsection{Wildcards}

Wildcards in Java~\cite{Java3,adding-wildcards} were motivated
by the following example (rewritten in X10 syntax)
from \cite{adding-wildcards}.
Sometimes a class needs a field or method
that is a list, but we don't care what the element type is.
For methods, one can give the method a type parameter:
\begin{xten}
def aMethod[T](list: List[T]) = { ... }
\end{xten}
This method can then be called on any \xcd"List" object.
However, there is no way to do this for fields since they
cannot be parameterized.
Java introduced wildcards to allow such fields to be 
typed:
\begin{xten}
List<?> list;
\end{xten}
In X10, a similar effect is achieved by not constraining the
type property of \xcd"List".
One can write the following:
\begin{xten}
list: List;
\end{xten}
Similarly, the method can be written without type parameters by
not constraining \xcd"List":
\begin{xten}
def aMethod(list: List) = { ... }
\end{xten}

In X10, \xcd"List"
is a supertype of
\xcd"List[T]" for any \xcd"T",
just as in Java
\xcd"List<?>" is a supertype of
\xcd"List<T>" for any \xcd"T".
This follows directly from the definition of the type \xcd"List"
as \xcd"List{true}", and the type \xcd"List[T]"
as \xcd"List{X==T}", and the definition of subtyping.

Wildcards in Java can also be bounded.
We achieve the same
effect in X10 by using type constraints.
For instance, the following Java declarations:
\begin{xten}
void aMethod(List<? extends Number> list) { ... }
<T extends Number> void aParameterizedMethod(List<T> list) { ... }
\end{xten}
may be written as follows in X10:
\begin{xten}
def aMethod(list: List{T <: Number}) = { ... }
def aParameterizedMethod[T{self <: Number}](list: List[T]) = { ... }
\end{xten}

Wildcard bounds may be covariant, as in the following example:
\begin{xten}
List<? extends Number> list = new ArrayList<Integer>();
Number num = list.get(0);     // legal 
list.set(0, new Double(0.0)); // illegal
list.set(0, list.get(1));     // illegal
\end{xten}
This can also be written in X10, but with an important 
difference:
\begin{xten}
list: List{T <: Number} = new ArrayList[Integer]();
num: Number = list.get(0);    // legal 
list.set(0, new Double(0.0)); // illegal
list.set(0, list.get(1));     // legal! (when list is final)
\end{xten}
Note because \xcd"list.get" has return type \xcd"list.T", the
last call in above is well-typed in X10; the analogous call in
Java is not well-typed.

Finally,
one can also specify lower bounds on types.  These are useful for
comparators:
\begin{xten}
class TreeSet[T] {
    def this[T](cmp: Comparator{T :> this.T}) { ... }
}
\end{xten}
Here, the comparator for any supertype of \xcd"T" can be used as
to compare \xcd"TreeSet" elements.

Another use of lower bounds is for list operations.
The \xcd"map" method below takes a function that maps a supertype
of the class parameter \xcd"T" to the method type parameter \xcd"S":
\begin{xten}
class List[T] {
    def map[S](fun: Object{self :> T} => S) : List[S] = { ... }
}
\end{xten}

\subsection{Proper abstraction}

Consider the following example adapted from \cite{adding-wildcards}:
\begin{xten}
def shuffle[T](list: List[T]) = {
    for (i: int in [0..list.size()-1]) {
        val xi: T = list(i);
        val j: int = Math.random(list.size());
        list(i) = list(j);
        list(j) = xi;
    }
}
\end{xten}
The method is parameterized on \xcd"T" because the method body needs
the element type to declare the variable \xcd"xi".

However, the method parameter can be omitted by using the type \xcd"list.T"
for \xcd"xi".  Thus, the method can be declared with the signature:
\begin{xten}
def shuffle(list: List) { ... }
\end{xten}
This is called \emph{proper abstraction}.

This example illustrates a key difference between type properties
and type parameters:
A type property is a member of its class, whereas a type parameter is
not.  The names of type properties are visible outside the body of
their class declaration.

\if 0
Type properties can be used as the basis of a parameterized type
system.  This is done simply by making type properties private.
Using the syntactic sugar described above,
the resulting system behaves identically to a system with type
parameters.
\fi

In Java,
Wildcard
capture allows the parameterized method to be called with any \xcd"List",
regardless of its parameter type.
However,
the method parameter cannot be omitted: declaring a parameterless version
of shuffle requires delegating to a private parameterized version that 
``opens up'' the parameter.

\subsection{Virtual types}

Type properties share many similarities with virtual types~\cite{mp89-virtual-classes,beta}, particularly 
with sound formulations of virtual types using path-dependent types,
as found in gbeta~\cite{ernst99-gbeta}, Scala~\cite{scala}, 
and J\&~\cite{nqm06}.
%
Constrained types are more expressive than virtual
types since they can be constrained at the use-site,
can be refined on a per-object basis without explicit subclassing,
and can be refined contravariantly
as well as covariantly.

Thorup~\cite{thorup97}
proposed adding genericity to Java using virtual types.  For example,
a generic \xcd"List" class can be written as follows:
\begin{xten}
abstract class List {
    abstract typedef T;
    void add(T element) { ... }
    T get(int i) { ... }
}
\end{xten}
This class can be refined by bounding the virtual type \xcd"T" above:
\begin{xten}
abstract class NumberList extends List {
    abstract typedef T as Number;
}
\end{xten}
And this abstract class can be further refined to \emph{final bind}
\xcd"T" to a particular type:
\begin{xten}
class IntList extends NumberList {
    final typedef T as Integer;
}
\end{xten}
These classes are related by subtyping:
${\tt IntList} \subtype {\tt NumberList} \subtype {\tt List}$.
Only classes where \xcd"T" is final bound can be non-abstract.

In X10, an analogous \xcd"List" class would be written as follows:
\begin{xten}
class List[T] {
    def add(element: T) = { ... }
    def get(i: int): T = { ... }
}
\end{xten}

\xcd"NumberList" and \xcd"IntList" can be written as follows:
\begin{xten}
class NumberList extends List{T<:Number} { }
class IntList extends NumberList{T==Integer} { }
\end{xten}

However, note that X10's \xcd"List" is not abstract.
Instances of \xcd"List"
can instantiate \xcd"T" with a particular type and there is no need to declared classes for \xcd"NumberList" and \xcd"IntList".  Instead, one can use the types
\xcd"List[+Number]" and \xcd"List[Integer]".

Unlike virtual types, type properties can be refined contravariantly.
For instance, one can write the type \xcd"List[-Integer]",
and even \xcd"List{Integer<:T, T<:Number}".

\section{Constraint system}

\section{Structural constraints}
\label{sec:structural}

\begin{figure}[tp]
\begin{center}
\begin{tabular}{lrcl}
expressions & {\tt e} & ::= & \dots \\
            &        & \bnf & \xcd"T has Sig" \\
signatures  & {\tt Sig} & ::= & 
\xcdmath"def this[$\bar{\tt X}$]($\bar{\tt x}$: $\bar{\tt T}$){c}: T" \\
            &        & \bnf & 
\xcdmath"def m[$\bar{\tt X}$]($\bar{\tt x}$: $\bar{\tt T}$){c}: T" \\
            &        & \bnf & 
\xcdmath"val x{c}: T" \\
            &  & \bnf &
\xcdmath"var x{c}: T" \\
\end{tabular}
\end{center}
\caption{Grammar for structural constraints}
\label{fig:structural}
\end{figure}

The type system is general enough to support not only subtyping
constraints, but also structural constraints on types.  The type
system need not change except by extending the constraint
system.  The syntax for structural constraints is shown in
Figure~\ref{fig:structural}.

Structural constraints on types are found in many languages.
Haskell~\cite{haskell} supports type classes.
%ML's module system allows modules to be constrained by
%structural signatures~\cite{ml}. 
In Modula-3, type equivalence and subtyping are structural
rather than nominal as in object-oriented languages of the C
family such as C++, Java, Scala, and X10.
%
The language PolyJ~\cite{polyj} allows type parameters to be bounded using
structural where clauses, a form of F-bounded
polymorphism~\cite{fbounds}.
For example, a sorted list class in PolyJ can be written as follows:
\begin{xten}
class SortedList[T] where T { int compare(T) } {
    void add(T x) { ... x.compare(y) ... }
}
\end{xten}
The where clause states that the type parameter \xcd"T" must have a
method \xcd"compare" with the given signatures.

To support this, X10 provides structural constraints on types.
The analogous X10 code for \xcd"SortedList" is:
\begin{xten}
class SortedList[T] where T has compare(T): int {
    def add(x: T) = { ... x.compare(y) ... }
}
\end{xten}

A structural constraint is of the form \emph{Type}~\xcd"has"~\emph{Signature}.
A constraint is satisfied if the type has a member of the appropriate name
and with a compatible type.  
The constraint \xcd"X has f(T1): T2"
is satisfied by a type \xcd"T" if it has a method \xcd"f"
whose type is a subtype of \xcd"(T1 => T2)"$[{\tt T}/{\tt X}]$.
As an example,
the constraint \xcd"X has equals(X): boolean"
is satisfied by all three of the following classes:
\begin{xten}
class C { def equals(x: C): boolean; }
class D extends C { }
class E { def equals(x: Object): boolean; }
\end{xten}

By using function types and where clauses on constructors,
X10 can go further than PolyJ.
Unlike in PolyJ, where the \xcd"compare" method must be provided by \xcd"T",
in X10 the \xcd"compare" function can be external to \xcd"T".
This is achieved as follows:
\begin{xten}
class SortedList[T] {
    val compare: (T,T) => int;
    def this(cmp: (T,T) => int) = { compare = cmp; }
    def add(x: T) = { ... compare(x,y) ... }
}
\end{xten}

This permits 
\xcd"SortedList" to be instantiated using different compare functions:
\begin{xten}
val unixFiles    = new SortedList[String]
                        (String.compareTo.(String));
val windowsFiles = new SortedList[String]
                        (String.compareToIgnoreCase.(String));
\end{xten}

But, a problem with this approach is that the compare function must be
provided to the constructor at each instantiation of \xcd"SortedList".
The problem can be resolved by using constructors with different
structural constraints:
\begin{xten}
class SortedList[T] {
    val compare: (T,T) => int;
    def this[T]() where T has compareTo(T): int = {
        this[T](T.compareTo.(S));
    }
    def this[T](cmp: (T,T) => int) = { compare = cmp; }
    def add(x: T) = { ... compare(x,y) ... }
}
\end{xten}
Now, \xcd"SortedList" can be instantiated with any type that has
a \xcd"compareTo"
method without expliclty specifying the method at each constructor call.


\if 0
\section{Type definitions}

The syntax for constrained types can often be verbose.
X10 supports type definitions
to allow types to be written more succinctly.

Type definitions have the following syntax:
\medskip

\begin{tabular}{rcl}
\emph{TypeDef}& ::=  &
                \xcd"type"~\emph{Identifier}
                        [ \xcd"["~\emph{TypeParameters}~\xcd"]" ]
                        [ \xcd"("~\emph{Formals}~\xcd")" ]~\xcd"="~\emph{Type}
\end{tabular}

\medskip

\noindent
A type definition can be thought of as a type-valued function.
Type definitions may be parameterized on both types and values.
%
The following examples are legal type definitions:
\begin{xten}
type StringSet = Set[String];
type MapToList[K,V] = Map[K,List[V]];
type int(x: int) = int{self==x};
type int(lo: int, hi: int) = int{lo <= self, self <= hi};
\end{xten}

Type definitions may appear as class members or in the body of a
method, constructor, or initializer.  Type definitions that are
members of a class are \xcd"static"; type properties can be used
for non-static type definitions.

Type definitions are applicative, not generative; that is, they
are define aliases for types and do not introduce new types.
Thus, the following code is legal:
\begin{xten}
type A = int;
type B = String;
type C = String;
a: A = 3;
b: B = new C("Hi");
c: C = b + ", Mom!";
\end{xten}
A type defined by a type definition 
has the same constructors as its defining type.

\if 0
\subsection{(Pseudo) generative type definitions}

Annotations~\cite{ns07-x10anno} may be used to make type definitions
generative (upto annotation erasure).  In the following, \xcd"@tag"
annotation restricts how the annotated type can be used: types
with distinct tags are incomparable.
Thus,
\begin{xten}
type Variable = int@tag("Variable");
a: Variable = (Variable) 4; // legal
b: Variable = 3; // illegal
c: int = a; // illegal
\end{xten}
We could introduces variations on \xcd"@tag" that
indicate that the tagged type is a subtype or supertype of the
untagged type, thus allowing the assignments to \xcd"c"
and to \xcd"b" respectively.

\fi

\fi


\section{Type inference}

Because constrained types can be verbose, X10 supports type
inference to reduce the type annotation burden on the
programmer.

The type inference algorithm allows types to be omitted
altogether from many declarations and from method
and constructor invocations.
The algorithm also allows programmers to write a
partially constrained type or just the base type in a
declaration and to have a more precise constrained type
inferred.
For instance, it infers the type \xcd"int{self==3}" for the
local variables \xcd"x", \xcd"y", \xcd"z" in the following code:
\begin{xten}
val x = 3;
val y: int = x;
val z: int{self>0} = x;
\end{xten}
The algorithm is local: method and constructor parameter types,
as well as the types of mutable fields,
must be declared explicitly.  For non-\xcd"private" members,
this requirement is essential for enabling separate compilation.
Limiting the scope of inference also eliminates one cause of
potentially confusing error messages when types cannot be inferred.

The algorithm uses the subtyping constraint system described in
Section~\ref{sec:constraints}.

In general, an expression may have more than one satisfying type.

One requirement of the algorithm
is that it report not only that there exists a satisfying
assignment, but also reports the assignment itself.
The algorithm chooses the most precise assignment.

Because methods can be overridden, the inferred return type may be too
precise, preventing subclasses from overriding the method.

\infrule{
\Gamma \vdash e_0 : C\{c\} \\
\Gamma \vdash \mathit{mtype}(C, m) =
[X_1,\dots,X_k](x_1:T_1,\dots,x_n:T_n)\{d\} \to T \\
\Gamma \vdash e_i : S_i \\
\Gamma \vdash_{\cal C}
\exists {\tt this}: C\{c\} .  \exists x_i: T_i .  S_i \subtype T_i \wedge d
}{
\Gamma \vdash e_0.m(e_1, \ldots, e_n) : T
}

\begin{verbatim}
Types can be inferred if the constraints are satisfied.
Need to materialize the constraints.

1. union-find on equality constraints
2. solve the subtyping constraints -- collapse cycles into union-find
        if SCC has a contradiction, complain
3. materialize bounds

impl: use XConstraint for union-find

        - T <: S
        - whenever we ask if T <: S, ask TypeSystem, then add <:(T,S)
          to constraint if true
        - whenever we ask if T == S, ask TypeSystem, then add ==(T,S)
          to constraint if true

Based on Henglein, TAPOS 99

lower bound of X
        union type of types Ti with Ti --> X
upper bound of X
        interesection type of types Ti with X --> Ti
        
C{c} & D{d} = (C&D){c||d}
C{c} | D{d} = (C|D){c&&d}

C&D = gcd(C,D)
C|D = lca(C,D)

X <: int ||
X <: float
-->
X <: int|float
-->
X <: number

int <: X || float <: X
-->
int&float <: X
-->
void <: X

p==q || p==q  -->  p==q
p==q || p==r  -->  true
S<:T || S<:U  -->  S<:(T|U)
T<:S || U<:S  -->  (T&U)<:S

\end{verbatim}

A key difference is that X10 supports where clauses that
constrain method and constructor type and value arguments.
The algorithm should infer not only the base type of a
constrained type, but also the type and value constraints.


X10 should perform type inference of local variable
types and of type arguments for method and constructor calls.%

\if 0
The algorithm infers a type for a local variable by
type-checking the variable initializer and assigning
that type to the variable.
When checking calls with omitted type arguments,
the algorithm attempts to find an instantiation of the type
parameters for which the call will type check.
\fi

Consider the following method from \cite{adding-wildcards}:
\begin{xten}
def choose[T](a: T, b: T): T { ... }
\end{xten}
%
In the following snippet, the algorithm should infer the type
\xcd"Collection" for \xcd"x".
\begin{xten}
intSet: Set[int];
stringList: List[String];
val x = choose(intSet, stringList);
\end{xten}
%
And in this snippet, the algorithm should infer the type
\xcd"Collection[int]" for \xcd"y".
\begin{xten}
intSet: Set[int];
intList: List[int];
val y = choose(intSet, intList);
\end{xten}
%
Finally, in this snippet, the algorithm should infer the type
\xcd"Collection{T <: Number}" for \xcd"z".
\begin{xten}
intSet: Set[int];
numList: List{T <: Number};
val z = choose(intSet, numList);
\end{xten}
The inference algorithm for Java 5 produces analogous results.

Now, consider the following example:
\begin{xten}
def union[T](a: Set[T], b: Set[T]) : Set[T];
\end{xten}
The union method cannot be called with just arguments of type \xcd"Set".
\begin{xten}
set1: Set;
set2: Set;
val a = union(set1, set2);
\end{xten}
This is illegal because the type system cannot demonstrate that
\xcd"set1.T" and \xcd"set2.T" are equal.
The following, however, is acceptable:
\begin{xten}
set1: Set;
set2: Set[set1.T];
val a = union(set1, set2);
\end{xten}

As another example from \cite{adding-wildcards}, consider
the following method signature:
\begin{xten}
def unmodifiableSet[T](set: Set[T]): Set[T];
\end{xten}

In Java, this method could be called with an argument of
type \xcd"Set<?>".
This instantiates the method on \xcd"?"; that is, the
wildcard is captured by the call, since any element type will be
safe.  A type variable can capture only one wildcard.

In X10, the method can be called with just a \xcd"Set" because there
are no constraints on \xcd"T".  Using desugared syntax, the method is
equivalent to: 
\begin{xten}
def unmodifiableSet[T](set: Set{self.T==T}): Set{self.T==T};
\end{xten}
Any \xcd"Set" can be passed in: for an argument \xcd"e", the method
is instantiated on \xcd"e.T".
%
Note that if this method were defined as:
\begin{xten}
def unmodifiableSet(set: Set): Set;
\end{xten}
then the connection between the element types of the
argument and of the return types would be broken.
However, in X10, one could write use the following signature to keep the
connection:
\begin{xten}
def unmodifiableSet(set: Set): Set[set.T];
\end{xten}

\if 0
\section{Introspection}

\xcd"Cell[Cell[T<:A]]"

If not invariant, need run-time constraint solving.
\fi

\ifsemantics

\section{Semantics}
Featherweight X10 (FX10) is a formal calculus for X10 intended to  complement Featherweight Java
(FJ).  It models imperative aspects of X10 including the concurrency
constructs \hfinish{} and \hasync{}.


\paragraph{Overview of formalism}
\Subsection{Syntax}

\begin{figure}[htpb!]
\begin{center}
\begin{tabular}{|l|l|}
\hline

$\hP ::= \ol{\hL},\hS$ & Program. \\

$\hL ::= \hclass ~ \hC~\hextends~\hD~\lb~\ol{\hF};~\ol{\hM}~\rb$
& cLass declaration. \\

$\hF ::= \hvar\,\hf:\hC$
& Field declaration. \\

$\hM ::= \hdef\ \hm(\ol{\hx}:\ol{\hC}):\hC\{\hS\}$
& Method declaration. \\

$\hp ::= \hl ~~|~~ \hx$
& Path. \\ %(location or parameter)

$\he ::=  \hp.\hf  ~|~ \hnew{\hC} ~|~ \hnew{\hAcc(\hr,\hz)}$
& Expressions. \\ %: locations, parameters, field access\&assignment,  %invocation, \code{new}

$\hS ::=  \hp.\hf = \hp; ~|~ \hp.\hm(\ol{\hp});  ~|~ \hval{\hx}{\he}{\hS}$ &\\
$~~~~|~\acc{\hx}{\hnew{\hAcc(\hr,\hz)}}{\hS} ~|~ \ha \leftarrow \hp$ &\\
$~~~~|~ \pclocked~\finish{\hS}$&\\
$~~~~|~ \pclocked~\async{\hS} ~|~ \hS~\hS$
& Statements. \\ %: locations, parameters, field access\&assignment, invocation, \code{new}

\hline
\end{tabular}
\end{center}
\caption{FX10 Syntax.
    The terminals are locations (\hl), parameters and \hthis (\hx), field name (\hf), method name (\hm), class name (\hB,\hC,\hD,\hObject),
        and keywords (\hhnew, \hfinish, \hasync, \code{val}).
    The program source code cannot contain locations (\hl), because locations are only created during execution/reduction in \RULE{R-New} of \Ref{Figure}{reduction}.
    }
\label{Figure:syntax}
\end{figure}

\Ref{Figure}{syntax} shows the syntax of FX10.
%(\Ref{Section}{val} will later add the \hval and \hvar field modifiers.)
Expression~$\hval{\hx}{\he}{\hS}$ evaluates $\he$, assigns it to a
new variable $\hx$, and then evaluates \hS. The scope of \hx{} is \hS.

The syntax is similar to the real X10 syntax with the following difference:
%Non-escaping methods are marked with \code{@NonEscaping}, such methods
%can be invoked on raw objects (and can be used to initialize them).
FX10 does not have constructors; instead, an object is initialized by assigning to its fields or
    by calling
    non-escaping methods.

\Subsection{Reduction}
A {\em heap}~$H$ is a mapping from a given set of locations to {\em
  objects}. An object is a pair $C(F)$ where $C$ is a class (the exact
class of the object), and $F$ is a partial map from the fields of $C$
to locations.
%We say the object~\hl is {\em total/cooked} (written~$\cooked_H(\hl)$)
%    if its map is total, i.e.,~$H(\hl)=\hC(F) \gap \dom(F)=\fields(\hC)$.

%We say that a heap~$H$ {\em satisfies} $\phi$ (written~$H \vdash \phi$)
%    if the plus assertions in~$\phi$ (ignoring the minus assertions) are true in~$H$,
%    i.e., if~$\phi \vdash +\hl$ then~$\hl$ is cooked in~$H$
%    and if~$\phi \vdash +\hl.\hf$ then~$H(\hl)=\hC(F)$ and~$F(\hf)$ is cooked in~$H$.


%An {\em annotation} $N$ for a heap $H$ maps each $l \in \dom(H)$ to a
%possibly empty set of fields $a(H(l))$ of the class of $H(l)$ disjoint
%from $\dom(H(l))$. (These are the fields currently being
%asynchronously initialized.) The logic of initialization described
%above is clearly sound for the obvious interpretation of formulas over
%annotated heaps. For future reference, we say that that a heap $H$
%{\em satisfies} $\phi$ if there is some annotation $N$ (and some
%valuation $v$ assigning locations in $\dom(H)$ to free variables of
%$\phi$) such that $\phi$ evaluates to true.

The reduction relation is described in
Figure~\ref{Figure:reduction}. An S-configuration is of the form
$\hS,H$ where \hS{} is a statement and $H$ is a heap (representing a
computation which is to execute $\hS$ in the heap $H$), or $H$
(representing terminated computation). An
E-configuration is of the form $\he,H$ and represents the
computation which is to evaluate $\he$ in the configuration $H$. The
set of {\em values} is the set of locations; hence E-configurations of
the form $\hl,H$ are terminal.

Two transition relations $\preduce{}$ are defined, one over
S-configurations and the other over E-configurations. Here $\pi$ is a
set of locations which can currently be asynchronously accessed.  Thus
each transition is performed in a context that knows about the current
set of capabilities.  For $X$ a partial function, we use the notation
$X[v \mapsto e]$ to represent the partial function which is the same
as $X$ except that it maps $v$ to $e$.

The rules for termination, step, val, new, invoke, access and assign
are standard.  The only minor novelty is in how \hasync{} is
defined. The critical rule is the last rule in~\RULE{(R-Step)} -- it
specifies the ``asynchronous'' nature of \hasync{} by permitting \hS{}
to make a step even if it is preceded by $\async{\hS_1}$. Further,
each async records its set of synchronous capabilities. When
descending into an async the async's own capabiliites are added to
those obtained from the environment.

%
The rule~\RULE{(R-New)} returns a new location that is bound to a new
object that is an instance of \hC{} with none of its fields initialized.
%
The rule~\RULE{(R-Access)} ensures that the field is initialized before it is
read ($\hf_i$ is contained in $\ol{\hf}$).
%
The rule~\RULE{(R-Acc-N)} adds the new location bound to the
accumulator as a synchronous access capability to the current async.
%

The rule~\RULE{(R-Acc-A-R)} permits the accumulator to be read by an
unclocked async provided that the current set of capabilities permit
it, and provided that there is no nested async prior to the read
(i.e.{} if there were any, they have terminated).

The rule~\RULE{(R-Acc-CA-R)} permits the accumulator to be read by 
clocked async provided that the current set of capabilities permit
it, and provided that the only statements prior to the read are
clocked asyncs that are stuck at an \hadvance. By convention we regard
the empty statement as stuck, hence this rule can be applied even if
there is no statement preceding the read. 

The rule~\RULE{(R-Acc-W)} updates the current contents of the
accumulator provided that the current set of capabilities permit
asynchronous access to the accumulator.

The rule~\RULE{(R-Advance)} permits a \code{clocked finish} to advance
only if all the top-level \code{clocked asyncs} in the scope of the
\code{clocked finish} and before an \code{advance} are stuck at an
\code{advance}. A \code{clocked finish} can also advance if all the
the top-level \code{clocked asyncs} in its scope are stuck. In this
case, the statement in the body of hte \code{clocked finish} has
terminated, and left behind only possibly clocked \code{async}. This
rule corresponds to the notion that the body of a \code{clocked
  finish} deregisters itself from the clock on local termination.
Note that in both these rules unlocked \code{aync} may exist in the
scope of the \code{clocked finish}; they do not come in the way of the
\code{clocked finish} advancing.

\begin{figure*}[t]
\begin{center}
\begin{tabular}{|c|}
\hline
$\typerule{
 \hS,H \preduce H'
}{
  \begin{array}{l}
    \pclocked~\finish{\hS},H \preduce H'\\
    \pclocked~\asynct{\pi_1}{\hS},H \ptreduce H'\ \ \ \pi=\pi_1,\pi_2\\
    \hS~\hS', H \preduce \hS',H'\\
  \end{array}
}$~\RULE{(R-Term)}
\\\\
$\typerule{
  \hS,H \preduce \hS', H'
}{
  \begin{array}{l}
    \pclocked~\finish{\hS},H \preduce \pclocked~\finish{\hS'},H'\\
    \pclocked~\asynct{\pi_1}{\hS},H \ptreduce  \pclocked~\asynct{\pi_1}{\hS'}, H'\ \ \ \pi=\pi_1,\pi_2\\
    \hS~\hS_1, H \preduce \hS'~\hS_1,H'\\
    \asynct{\pi_1}{\hS_1}~\hS, H \preduce \asynct{\pi_1}{\hS_1}~\hS', H'\\
  \end{array}
}$~\RULE{(R-Step)}
\\\\

$\typerule{
  \he,H \reduce \hl,H'
}{
  \hval{\hx}{\he}{\hS},H \preduce \hS[\hl/\hx], H'
}$~\RULE{(R-Val)}
~
$\typerule{
    \hl' \not \in \dom(H)
}{
  \hnew{\hC},H \preduce \hl',H[ \hl' \mapsto \hC()]
}$~\RULE{(R-New)}
~
$\typerule{
    H(\hl')=\hC(\ldots)
        \gap
    \mbody{}(\hm,\hC)=\ol{\hx}.\hS
}{
  \hl'.\hm(\ol{\hl}),H \preduce \hS[\ol{\hl}/\ol{\hx},\hl'/\hthis],H
}$~\RULE{(R-Invoke)}
\\\\
$\typerule{
    H(\hl)=\hC(\ol{\hf}\mapsto\ol{\hl'})
}{
  \hl.\hf_i,H \preduce \hl_i',H
}$~\RULE{(R-Access)}
\quad
$\typerule{
    H(\hl)=\hC(F) 
}{
  \hl.\hf=\hl',H \preduce H[ \hl \mapsto \hC(F[ \hf \mapsto \hl'])]
}$~\RULE{(R-Assign)}
\\\\
$\typerule{
    \hl \not \in \dom(H)
}{
 \asynct{\pi_1}{\acc{\hx}{\hnew{\hAcc(\hr,\hz)}}~\hS},H \preduce 
   \asynct{\pi_1,l}{\hS[\hl/\hx]},H[ \hl \mapsto \hAcc(\hr,\hz)]
}$~\RULE{(R-Acc-N)}
\\\\
$\typerule{
   \ha\in \pi \gap H(a)=\hAcc(\hr,\hv)\gap \hx \in \pi_1
}{
\asynct{\pi_1}{\hval{\hx}{\ha()}{\hS}},H \preduce 
\asynct{\pi_1}{\hS[\hv/\hx]}, H'
}$~\RULE{(R-Acc-A-R)}
\\\\
$\typerule{}{
    \hclocked~\asynct{\pi}{\xadvance~\hS} \stuck
}$~\RULE{(R-Stuck-CA)}
~
$\typerule{
  \hS_1 \stuck \gap \hS_2 \stuck
}{
   \hS_1~\hS_2 \stuck
}$~\RULE{(R-Stuck-S)}
\\\\
$\typerule{
   \ha\in \pi \gap H(a)=\hAcc(\hr,\hv)\gap \hx \in \pi_1 \gap S\stuck
}{
\hclocked~\asynct{\pi_1}{\hS~\hval{\hx}{\ha()}{\hS_1}},H \preduce 
\hclocked~\asynct{\pi_1}{\hS~\hS_1[\hv/\hx]}, H'
}$~\RULE{(R-Acc-CA-R)}
\\\\
$\typerule{
  \ha\in \pi\gap H(\ha)=\hAcc(\hr,\hv)\gap \hw=\hr(\hv,\hp)  
}{
  \ha \leftarrow \hp,H \preduce H[\ha \mapsto \hAcc(\hr,\hw)]
}$~\RULE{(R-Acc-W)}
\\\\

$\typerule{}{
    \asynct{\pi}{\hS}\ureduce \asynct{\pi}{\hS}\\
    \hclocked~\asynct{\pi}{\xadvance~\hS} \ureduce
    \hclocked~\asynct{\pi}{\hS} \\
    \hclocked~\finish{\hS}\ureduce \hclocked~\finish{\hS}\\
}$~\RULE{(R-Advance-A,CA,CF)}
\quad
$\typerule{
    \hS_1 \ureduce \hS_1'\ \     \hS_2 \ureduce \hS_2'
}{
  \hS_1~\hS_2\ureduce \hS_1'~\hS_2'
}$~\RULE{(R-Advance-S)}
\\\\
$\typerule{
    \hS \ureduce \hS'
}{
  \begin{array}{l}
    \hclocked~\finish{\hS~\xadvance~\hS_1},H\preduce \hclocked~\finish{\hS'~\hS_1},H\\
    \hclocked~\finish{\hS},H\preduce  \hclocked~\finish{\hS'},H
  \end{array}
}$~\RULE{(R-Advance)}
\\
\hline
\end{tabular}
\end{center}
\caption{FX10 Reduction Rules ($\hS,H \preduce \hS',H' ~|~H'$ and~$\he,H \preduce \hl,H'$).}
\label{Figure:reduction}
\end{figure*}


\Subsection{Results}


\fi

\section{Implementation}

This section describes a possible implementation approach that
performs
run-time instantiation of classes, similar to the implementation
of parameterized classes in NextGen \cite{allen03,allen04}.
We describe only the translation to Java; an obvious
translation strategy for the C++ backend would use templates,
but the details have not been worked out.
Another possible translation, 
based on the implementation of PolyJ~\cite{polyj},
uses \emph{adapter objects} to allow generic code to invoke
methods of instances of the its type properties.

The translation is described by example.  A more thorough and complete
description will be given after some experience with the actual
implementation.

Consider the code in Figure~\ref{fig:translation1}.  It contains most of the 
features of generics that have to be translated.
\begin{figure*}[tp]
\begin{xten}
class C[T] {
    var x: T;
    def this[T](x: T) { this.x = x; }
    def set(x: T) { this.x = x; }
    def get(): T { return this.x; }
    def map[S](f: T => S): S { return f(this.x); }
    def d() { return new D[T](); }
    def t() { return new T(); }
    def isa(y: Object): boolean { return y instanceof T; }
}

val x : C = new C[String]();
val y : C[int] = new C[int]();
val z : C{T <: Array} = new C[Array[int]]();
x.map[int](f);
new C[int{self==3}]() instanceof C[int{self<4}];
\end{xten}
\caption{Code to translate}
\label{fig:translation1}
\end{figure*}

\subsection{Eliminating method type parameters}

\begin{figure*}[tp]
\begin{xten}
class C[T] where T has T() {
    var x: T;

    def this[T](x: T) { this.x = x; }

    def set(x: T) { this.x = x; }
    def get(): T { return this.x; }

    def d() { return new D[T](); }
    def t() { return new T(); }

    def isa(y: Object): boolean { return y instanceof T; }

    // Translation of map to an inner class
    class map$[T,S] {
        def apply(c: C[T], f: Fun1[T,S]) { return f(c.x); }
    }
}

val x : C = new C[String]();
val y : C[int] = new C[int]();
val z : C{T <: Array} = new C[Array[int]]();
new map$[x.T,int]().apply(x,f);
new C[int{self==3}]() instanceof C[int{self<4}];
\end{xten}
\caption{After removing method parameters}
\label{fig:translation2}
\end{figure*}

The first step in translation is to remove method parameters by
introducing generic inner class for each generic method.
Constructor type parameters are not changed.
In addition, function type \xcd"T => S" is translated to \xcd"Fun1[T,S]".
This step produces the code in Figure~\ref{fig:translation2}:
At this point, the code consists only of generic classes.
The remaining translation introduces a run-time representation
for the type properties of these classes.

\begin{figure*}[tp]
\begin{xten}
class C{0} implements C {
    final Type T = {0}$Type.it;
    {0} x;
    C{0}({0} x) { this.x = x; }

    void set$(Object x) { set(({0}) x); }
    void set({0} x) { this.x = x; }

    Object get$() { return ({0}) get(); }
    {0} get() { return this.x; }

    D d$() { return d(); }
    D{0} d() { return new D{0}(); }

    Object t$() { return t(); }
    {0} t() { return new {0}(); }

    boolean isa(Object y) { return T.instanceof$(y); }

    static class map$Type extends Type {
        ...
        static map$Type instantiate$(Type T, Type S) { ... }
    }

    static class map$Type{0}{1} extends Map$Type {
        map$ new$() { return new map${0}{1}(); }
    }
    
    interface map$ {
        Object apply$(C c, Fun1 f);
    }

    class map${0}{1} implements map$ {
        final Type T = {0}$Type.it;
        final Type S = {1}$Type.it;
        Object apply$(C c, Fun1 f)
          { return apply((C{0}) c, (Fun1{0}{1}) f); }
        {1} apply(C{0} c, Fun1{0}{1} f) { return f(c.x); }
    }
}

C x = new C$String();
C$int y = new C$int();
C z = new C$Array$int();
C.map$Type.instantiate$(x.T, int$Type.it).new$().apply$(x,f);
C$int$self$lt$4.instanceof$(new C$int$self$eqeq$3());
\end{xten}

\caption{Translation to Java}
\label{fig:translation4}
\end{figure*}

\begin{figure*}[tp]
\begin{xten}
class C$Type implements Type {
    static Type it = new C$Type();
    boolean instanceof$(Object x) { return x instanceof C; }

    static Map<Type,Type> instantiations;

    static Type instantiate$(Type T) {
        instantiations.get(T);
    }
}

class C{0}$Type implements Type {
    static Type it = new C{0}$Type();
    boolean instanceof$(Object x) { return x instanceof C{0}; }
}

interface C {
    void set$(Object x);
    Object get$();
    D d$();
    Object t$();
    boolean isa$(Object y);
}
\end{xten}

\caption{Translation to Java}
\label{fig:translation3}
\end{figure*}

\subsection{Run-time instantiation}

In this translation the type properties are represented as
instances of a \xcd"Type" class, analogous to \xcd"java.lang.Class". 
Each generic class has a \xcd"Type"-typed field for each of
its type properties initialized by the class's constructor.
The \xcd"Type" objects
are used to implement {\tt instanceof} and cast operations.
\begin{xten}
interface Type {
    boolean instanceof$(Object x);
    <T> T cast$(Object x);
}
\end{xten}


In this translation, which is partially based on the
NextGen~\cite{allen03,allen04} translation,
a generic class is translated into a \emph{base interface} and
a \emph{template class} that implements the base interface.
At runtime, the first time a generic class is instantiated
a class loader loads \emph{template class}, rewriting the
bytecode to instantiate the type properties as appropriate.

For example, the 
code for class {\tt C} above is translated into the template
class in Figure~\ref{fig:translation4}
with supporting classes Figure~\ref{fig:translation3}.
When instantiating the template, the string ``{\tt \{0\}}'' is
substituted with the name of the actual type
property.\footnote{In a real implementation, the names would be
mangled as appropriate.}
Since methods of {\tt C} can be called in a context where the
property instantiation is not known, 
each method in the template class has to be implemented twice:
once with an Object interface and once with an instantiated
interface.

We translate \xcd"instanceof" and cast operations to calls to
methods of a \xcd"Type" because the actual implementation of
the operation may require run-time constraint solving or other
complex code that cannot be easily substituted in when rewriting
the bytecode during instantiation.

\section{Conclusions}

We have presented a preliminary design for supporting genericity
in X10 using type properties.  This type system generalizes the
existing X10 type system.  The use of constraints on type
properties allows
the design to capture many features of generics in languages
like Java 5 and C\# and then to extend these features with new
more expressive power.
We expect that the design admits an efficient
implementation and intend to implement the design shortly.

\nocite{unifying-genericity}
\nocite{adding-wildcards}
\nocite{emir06}
\nocite{myers94}
\nocite{polyj}
\nocite{allen04}
\nocite{allen03}
\nocite{beta}
\nocite{mp89-virtual-classes}
\nocite{thorup97}

\newpage

\bibliographystyle{plain}
\bibliography{master}

% \appendix
% \onecolumn

% \section{An extended example}
% {\footnotesize
\begin{verbatim}
/**
   A distributed binary tree.
   @author Satish Chandra 4/6/2006
   @author vj
 */
//                             ____P0
//                            |     |
//                            |     |
//                          _P2  __P0
//                         |  | |   |
//                         |  | |   |
//                        P3 P2 P1 P0
//                         *  *  *  *
// Right child is always on the same place as its parent;
// left child is at a different place at the top few levels of the tree,
// but at the same place as its parent at the lower levels.

class Tree(localLeft: boolean,
           left: nullable Tree(& localLeft => loc=here),
           right: nullable Tree(& loc=here),
           next: nullable Tree) extends Object {
    def postOrder:Tree = {
        val result:Tree = this;
        if (right != null) {
            val result:Tree = right.postOrder();
            right.next = this;
            if (left != null) return left.postOrder(tt);
        } else if (left != null) return left.postOrder(tt);
        this
    }
    def postOrder(rest: Tree):Tree = {
        this.next = rest;
        postOrder
    }
    def sum:int = size + (right==null => 0 : right.sum()) + (left==null => 0 : left.sum)
}
value TreeMaker {
    // Create a binary tree on span places.
    def build(count:int, span:int): nullable Tree(& localLeft==(span/2==0)) = {
        if (count == 0) return null;
        {val ll:boolean = (span/2==0);
         new Tree(ll,  eval(ll => here : place.places(here.id+span/2)){build(count/2, span/2)},
           build(count/2, span/2),count)}
    }
}
\end{verbatim}}

\subsection{Places}
{\footnotesize
\begin{verbatim}
/**

 * This class implements the notion of places in X10. The maximum
 * number of places is determined by a configuration parameter
 * (MAX_PLACES). Each place is indexed by a nat, from 0 to MAX_PLACES;
 * thus there are MAX_PLACES+1 places. This ensures that there is
 * always at least 1 place, the 0'th place.

 * We use a dependent parameter to ensure that the compiler can track
 * indices for places.
 *
 * Note that place(i), for i <= MAX_PLACES, can now be used as a non-empty type.
 * Thus it is possible to run an async at another place, without using arays---
 * just use async(place(i)) {...} for an appropriate i.

 * @author Christoph von Praun
 * @author vj
 */

package x10.lang;

import x10.util.List;
import x10.util.Set;

public value class place (nat i : i <= MAX_PLACES){

    /** The number of places in this run of the system. Set on
     * initialization, through the command line/init parameters file.
     */
    config nat MAX_PLACES;

    // Create this array at the very beginning.
    private constant place value [] myPlaces = new place[MAX_PLACES+1] fun place (int i) {
	return new place( i )(); };

    /** The last place in this program execution.
     */
    public static final place LAST_PLACE = myPlaces[MAX_PLACES];

    /** The first place in this program execution.
     */
    public static final place FIRST_PLACE = myPlaces[0];
    public static final Set<place> places = makeSet( MAX_PLACES );

    /** Returns the set of places from first place to last place.
     */
    public static Set<place> makeSet( nat lastPlace ) {
	Set<place> result = new Set<place>();
	for ( int i : 0 .. lastPlace ) {
	    result.add( myPlaces[i] );
	}
	return result;
    }

    /**  Return the current place for this activity.
     */
    public static place here() {
	return activity.currentActivity().place();
    }

    /** Returns the next place, using modular arithmetic. Thus the
     * next place for the last place is the first place.
     */
    public place(i+1 % MAX_PLACES) next()  { return next( 1 ); }

    /** Returns the previous place, using modular arithmetic. Thus the
     * previous place for the first place is the last place.
     */
    public place(i-1 % MAX_PLACES) prev()  { return next( -1 ); }

    /** Returns the k'th next place, using modular arithmetic. k may
     * be negative.
     */
    public place(i+k % MAX_PLACES) next( int k ) {
	return places[ (i + k) % MAX_PLACES];
    }

    /**  Is this the first place?
     */
    public boolean isFirst() { return i==0; }

    /** Is this the last place?
     */
    public boolean isLast() { return i==MAX_PLACES; }
}
\end{verbatim}}
\subsection{$k$-dimensional regions}
{\footnotesize
\begin{verbatim}
package x10.lang;

/** A region represents a k-dimensional space of points. A region is a
 * dependent class, with the value parameter specifying the dimension
 * of the region.
 * @author vj
 * @date 12/24/2004
 */

public final value class region( int dimension : dimension >= 0 )  {

    /** Construct a 1-dimensional region, if low <= high. Otherwise
     * through a MalformedRegionException.
     */
    extern public region (: dimension==1) (int low, int high)
        throws MalformedRegionException;

    /** Construct a region, using the list of region(1)'s passed as
     * arguments to the constructor.
     */
    extern public region( List(dimension)<region(1)> regions );

    /** Throws IndexOutOfBoundException if i > dimension. Returns the
        region(1) associated with the i'th dimension of this otherwise.
     */
    extern public region(1) dimension( int i )
        throws IndexOutOfBoundException;


    /** Returns true iff the region contains every point between two
     * points in the region.
     */
    extern public boolean isConvex();

    /** Return the low bound for a 1-dimensional region.
     */
    extern public (:dimension=1) int low();

    /** Return the high bound for a 1-dimensional region.
     */
    extern public (:dimension=1) int high();

    /** Return the next element for a 1-dimensional region, if any.
     */
    extern public (:dimension=1) int next( int current )
        throws IndexOutOfBoundException;

    extern public region(dimension) union( region(dimension) r);
    extern public region(dimension) intersection( region(dimension) r);
    extern public region(dimension) difference( region(dimension) r);
    extern public region(dimension) convexHull();

    /**
       Returns true iff this is a superset of r.
     */
    extern public boolean contains( region(dimension) r);
    /**
       Returns true iff this is disjoint from r.
     */
    extern public boolean disjoint( region(dimension) r);

    /** Returns true iff the set of points in r and this are equal.
     */
    public boolean equal( region(dimension) r) {
        return this.contains(r) && r.contains(this);
    }

    // Static methods follow.

    public static region(2) upperTriangular(int size) {
        return upperTriangular(2)( size );
    }
    public static region(2) lowerTriangular(int size) {
        return lowerTriangular(2)( size );
    }
    public static region(2) banded(int size, int width) {
        return banded(2)( size );
    }

    /** Return an \code{upperTriangular} region for a dim-dimensional
     * space of size \code{size} in each dimension.
     */
    extern public static (int dim) region(dim) upperTriangular(int size);

    /** Return a lowerTriangular region for a dim-dimensional space of
     * size \code{size} in each dimension.
     */
    extern public static (int dim) region(dim) lowerTriangular(int size);

    /** Return a banded region of width {\code width} for a
     * dim-dimensional space of size {\code size} in each dimension.
     */
    extern public static (int dim) region(dim) banded(int size, int width);


}

\end{verbatim}}

\subsection{Point}
{\footnotesize
\begin{verbatim}
package x10.lang;

public final class point( region region ) {
    parameter int dimension = region.dimension;
    // an array of the given size.
    int[dimension] val;

    /** Create a point with the given values in each dimension.
     */
    public point( int[dimension] val ) {
        this.val = val;
    }

    /** Return the value of this point on the i'th dimension.
     */
    public int valAt( int i) throws IndexOutOfBoundException {
        if (i < 1 || i > dimension) throw new IndexOutOfBoundException();
        return val[i];
    }

    /** Return the next point in the given region on this given
     * dimension, if any.
     */
    public void inc( int i )
        throws IndexOutOfBoundException, MalformedRegionException {
        int val = valAt(i);
        val[i] = region.dimension(i).next( val );
    }

    /** Return true iff the point is on the upper boundary of the i'th
     * dimension.
     */
    public boolean onUpperBoundary(int i)
        throws IndexOutOfBoundException {
        int val = valAt(i);
        return val == region.dimension(i).high();
    }

    /** Return true iff the point is on the lower boundary of the i'th
     * dimension.
     */
    public boolean onLowerBoundary(int i)
        throws IndexOutOfBoundException {
        int val = valAt(i);
        return val == region.dimension(i).low();
    }
}
\end{verbatim}}

\subsection{Distribution}
{\footnotesize
\begin{verbatim}
package x10.lang;

/** A distribution is a mapping from a given region to a set of
 * places. It takes as parameter the region over which the mapping is
 * defined. The dimensionality of the distribution is the same as the
 * dimensionality of the underlying region.

   @author vj
   @date 12/24/2004
 */

public final value class distribution( region region ) {
    /** The parameter dimension may be used in constructing types derived
     * from the class distribution. For instance,
     * distribution(dimension=k) is the type of all k-dimensional
     * distributions.
     */
    parameter int dimension = region.dimension;

    /** places is the range of the distribution. Guranteed that if a
     * place P is in this set then for some point p in region,
     * this.valueAt(p)==P.
     */
    public final Set<place> places; // consider making this a parameter?

    /** Returns the place to which the point p in region is mapped.
     */
    extern public place valueAt(point(region) p);

    /** Returns the region mapped by this distribution to the place P.
        The value returned is a subset of this.region.
     */
    extern public region(dimension) restriction( place P );

    /** Returns the distribution obtained by range-restricting this to Ps.
        The region of the distribution returned is contained in this.region.
     */
    extern public distribution(:this.region.contains(region))
        restriction( Set<place> Ps );

    /** Returns a new distribution obtained by restricting this to the
     * domain region.intersection(R), where parameter R is a region
     * with the same dimension.
     */
    extern public (region(dimension) R) distribution(region.intersection(R))
        restriction();

    /** Returns the restriction of this to the domain region.difference(R),
        where parameter R is a region with the same dimension.
     */
    extern public (region(dimension) R) distribution(region.difference(R))
        difference();

    /** Takes as parameter a distribution D defined over a region
        disjoint from this. Returns a distribution defined over a
        region which is the union of this.region and D.region.
        This distribution must assume the value of D over D.region
        and this over this.region.

        @seealso distribution.asymmetricUnion.
     */
    extern public (distribution(:region.disjoint(this.region) &&
                                dimension=this.dimension) D)
        distribution(region.union(D.region)) union();

    /** Returns a distribution defined on region.union(R): it takes on
        this.valueAt(p) for all points p in region, and D.valueAt(p) for all
        points in R.difference(region).
     */
    extern public (region(dimension) R) distribution(region.union(R))
        asymmetricUnion( distribution(R) D);

    /** Return a distribution on region.setMinus(R) which takes on the
     * same value at each point in its domain as this. R is passed as
     * a parameter; this allows the type of the return value to be
     * parametric in R.
     */
    extern public (region(dimension) R) distribution(region.setMinus(R))
        setMinus();

    /** Return true iff the given distribution D, which must be over a
     * region of the same dimension as this, is defined over a subset
     * of this.region and agrees with it at each point.
     */
    extern public (region(dimension) r)
        boolean subDistribution( distribution(r) D);

    /** Returns true iff this and d map each point in their common
     * domain to the same place.
     */
    public boolean equal( distribution( region ) d ) {
        return this.subDistribution(region)(d)
            && d.subDistribution(region)(this);
    }

    /** Returns the unique 1-dimensional distribution U over the region 1..k,
     * (where k is the cardinality of Q) which maps the point [i] to the
     * i'th element in Q in canonical place-order.
     */
    extern public static distribution(:dimension=1) unique( Set<place> Q );

    /** Returns the constant distribution which maps every point in its
        region to the given place P.
    */
    extern public static (region R) distribution(R) constant( place P );

    /** Returns the block distribution over the given region, and over
     * place.MAX_PLACES places.
     */
    public static (region R) distribution(R) block() {
        return this.block(R)(place.places);
    }

    /** Returns the block distribution over the given region and the
     * given set of places. Chunks of the region are distributed over
     * s, in canonical order.
     */
    extern public static (region R) distribution(R) block( Set<place> s);


    /** Returns the cyclic distribution over the given region, and over
     * all places.
     */
    public static (region R) distribution(R) cyclic() {
        return this.cyclic(R)(place.places);
    }

    extern public static (region R) distribution(R) cyclic( Set<place> s);

    /** Returns the block-cyclic distribution over the given region, and over
     * place.MAX_PLACES places. Exception thrown if blockSize < 1.
     */
    extern public static (region R)
        distribution(R) blockCyclic( int blockSize)
        throws MalformedRegionException;

    /** Returns a distribution which assigns a random place in the
     * given set of places to each point in the region.
     */
    extern public static (region R) distribution(R) random();

    /** Returns a distribution which assigns some arbitrary place in
     * the given set of places to each point in the region. There are
     * no guarantees on this assignment, e.g. all points may be
     * assigned to the same place.
     */
    extern public static (region R) distribution(R) arbitrary();

}
\end{verbatim}}

\subsection{Arrays}
Finally we can now define arrays. An array is built over a
distribution and a base type.

{\footnotesize
\begin{verbatim}
package x10.lang;

/** The class of all  multidimensional, distributed arrays in X10.

    <p> I dont yet know how to handle B@current base type for the
    array.

 * @author vj 12/24/2004
 */

public final value class array ( distribution dist )<B@P> {
    parameter int dimension = dist.dimension;
    parameter region(dimension) region = dist.region;

    /** Return an array initialized with the given function which
        maps each point in region to a value in B.
     */
    extern public array( Fun<point(region),B@P> init);

    /** Return the value of the array at the given point in the
     * region.
     */
    extern public B@P valueAt(point(region) p);

    /** Return the value obtained by reducing the given array with the
        function fun, which is assumed to be associative and
        commutative. unit should satisfy fun(unit,x)=x=fun(x,unit).
     */
    extern public B reduce(Fun<B@?,Fun<B@?,B@?>> fun, B@? unit);


    /** Return an array of B with the same distribution as this, by
        scanning this with the function fun, and unit unit.
     */
    extern public array(dist)<B> scan(Fun<B@?,Fun<B@?,B@?>> fun, B@? unit);

    /** Return an array of B@P defined on the intersection of the
        region underlying the array and the parameter region R.
     */
    extern public (region(dimension) R)
        array(dist.restriction(R)())<B@P>  restriction();

    /** Return an array of B@P defined on the intersection of
        the region underlying this and the parametric distribution.
     */
    public  (distribution(:dimension=this.dimension) D)
        array(dist.restriction(D.region)())<B@P> restriction();

    /** Take as parameter a distribution D of the same dimension as *
     * this, and defined over a disjoint region. Take as argument an *
     * array other over D. Return an array whose distribution is the
     * union of this and D and which takes on the value
     * this.atValue(p) for p in this.region and other.atValue(p) for p
     * in other.region.
     */
    extern public (distribution(:region.disjoint(this.region) &&
                                dimension=this.dimension) D)
        array(dist.union(D))<B@P> compose( array(D)<B@P> other);

    /** Return the array obtained by overlaying this array on top of
        other. The method takes as parameter a distribution D over the
        same dimension. It returns an array over the distribution
        dist.asymmetricUnion(D).
     */
    extern public (distribution(:dimension=this.dimension) D)
        array(dist.asymmetricUnion(D))<B@P> overlay( array(D)<B@P> other);

    extern public array<B> overlay(array<B> other);

    /** Assume given an array a over distribution dist, but with
     * basetype C@P. Assume given a function f: B@P -> C@P -> D@P.
     * Return an array with distribution dist over the type D@P
     * containing fun(this.atValue(p),a.atValue(p)) for each p in
     * dist.region.
     */
    extern public <C@P, D>
        array(dist)<D@P> lift(Fun<B@P, Fun<C@P, D@P>> fun, array(dist)<C@P> a);

    /**  Return an array of B with distribution d initialized
         with the value b at every point in d.
     */
    extern public static (distribution D) <B@P> array(D)<B@P> constant(B@? b);

}
\end{verbatim}}


\begin{example}
 The code for {\tt List} translates as given in Table~\ref{List-translation}.
\end{example}

\begin{figure*}
{\footnotesize
\begin{verbatim}
  public value class List <Node> {
    public final nat n;   // is a parameter
    nullable Node node = null;
    nullable List<Node> rest = null;  // All assignments must check n = this.n-1.

    /** Returns the empty list. Defined only when the parameter n
        has the value 0. Invocation: new List(0)<Node>().
     */
    public List ( final nat n ) {
      assume n==0;
      this.n = n;
    }

    /** Returns a list of length 1 containing the given node.
        Invocation: new List(1)<Node>( node ).
     */
    public List ( final nat n, Node node ) {
      assume n==1;                         // From the constructor precondition.
      assert 0==0 : "DependentTypeError"; // For the constructor call.
      assert n>=1 : "DependentTypeError"; // For the this call.
      this(n, node, new List<Node>(0));
    }

    public List ( final nat n, Node node, List<Node> rest ) {
      assume n>=1;                               // From the constructor precondition
      assume rest.n==n-1 : "DependentTypeError"; // From the argument type.
      this.n = n;
      this.node = node;
      assert rest.n==n-1 : "DependentTypeError"; // For the field assignment.
      this.rest = rest;
    }

    public  List<Node> append( List<Node> arg ) {
      if (n == 0) {
          final List<Node> result = arg;
          assert n+arg.n == result.n : "DependentTypeError"; // For the return value
          return result;
      } else {
          assume rest.n == n-1;
          final List<Node> argval = rest.append(arg);
          assume argval.n == rest.n+arg.n;
          assert n+arg.n-1== argval.n : "DependentTypeError"; // For the constructor call.
          final List<Node> result = new List<Node>(n+arg.n, node, argval);
          assume result.n == n+arg.n;
          assert n+arg.n == result.n : "DependentTypeError"; // For the return value
          return result;
      }
    }

\end{verbatim}}
\caption{Translation of {\tt List} (contd in Table~\ref{List-translation-2}).}\label{List-translation}
\end{figure*}
\begin{figure*}
{\footnotesize
\begin{verbatim}
    public  List<Node> rev() {
      final List<Node> arg = new List<Node>(0);
      assume arg.n = 0;                           // From the constructor call.
      final List<Node> result = rev( arg );
      assume result.n == n+arg.n;                  // From the method signature
      assert n == result.n : "DependentTypeError"; // For the return value.
      return result;
    }

    public  List(n+arg.n)<Node> rev( final List<Node> arg) {
      if (n==0) {
         assert n+arg.n == arg.n : "DependentTypeError"; // For the return value.
         return arg;
      } else {
        assert 1+arg.n-1=arg.n : "DependentTypeError"; // For the argument to the constructor
        final List<Node> arg2 = new List<Node>(1+arg.n,node, arg));
        assume arg2.n==1+arg.n;                      // From the constructor invocation
        final List<Node> restval = rest;             // Read from a mutable field of parametric type
        assume restval.n == n-1;                     // From the field read.
        final List(restval.n+arg2.n)<Node> result = restval.rev( arg2 );
        assume result.n=restval.n+arg2.n
        assert n+arg.n == result.n                   // For the return value
        return result;
    }

    /** Return a list of compile-time unknown length, obtained by filtering
        this with f. */
    public List<Node> filter(fun<Node, boolean> f) {
         if (n==0) return this;
         if (f(node)) {
           final List<Node> l = rest.filter(f);
           assert l.n+1-1==l.n : "DependentTypeError"; // For the constructor call
           return new List<Node>(l.n+1,node, l);
         } else {
           return rest.filter(f);
         }
    }

    /** Return a list of m numbers from o..m-1. */
    public static  List<nat> gen( final nat m ) {
         assert 0 <= m : "DependentTypeError";        // Precondition for method call.
         final List<nat> result = gen(0,m);
         assume result.n=m-0 : "DependentTypeError";  // From the method signature
         assert m == result.n : "DependentTypeError"; // For the return value
         return result;
    }

    /** Return a list of (m-i) elements, from i to m-1. */
    public static List<nat> gen(final nat i, final nat m) {
      assume i <= m;                                   // Method precondition.
      if (i==m) {
        assert m-i == 0 : "DependentTypeError";        // For the constructor call
        final List result = new List<nat>(m-i);
        assume result.n == 0;                          // From the constructor call.
        assert m-i == result.n : "DependentTypeError"; // For the return value.
        return result;
      } else {
        assert i+1 <= m : "DependentTypeError";        // For the method call.
        final List<nat> arg = gen(i+1,m);
        assume arg.n = m-(i+1);                        // From the method call.
        assert m-i-1 = arg.n;                          // For the constructor invocation.
        final List result = new List<nat>(m-i, i, arg);
        assume result.n = m-i;                         // From the constructor invocation.
        assert m-i == result.n : "DependentTypeError"; // For the return value
        return result;
    }
  }
\end{verbatim}}
\caption{Translation of {\tt List} (continued).}\label{List-translation-2}
\end{figure*}

\section{Type-checking dependent classes}

Each programming language---such as \Xten{}---will specify the base
underlying classes (and the operations on them) which can occur as
types in parameter lists. For instance, in the code for {\tt List}
above, the only type that appears in parameter lists is {\tt int}, and
the only operations on {\tt int} are addition, subtraction, {\tt >=},
{\tt ==}, and the only constants are {\tt 0} and {\tt 1}.  (This
language falls within Presburger arithmetic, a decidable fragment of
arithmetic.)  The compiler must come equipped with a constraint solver
(decision procedure) that can answer questions of the form: does one
constraint entail another?  Constraints are atomic formulas built up
from these operations, using variables. For instance, the compiler
must answer each one of:
{\footnotesize
\begin{verbatim}
  n >= 2 |- n-1 >= 0
  n >= 0, m >= 0 |- m+n >= 0
\end{verbatim}}

Ultimately, the only variables that will occur in constraints are
those that correspond to {\tt config} parameters and those that are
defined by implicit parameter definitions. We need to establish that
the verification of any class will generate only a finite number of
constraints, hence only a finite constraint problem for the constraint
solver.

Second, it should be possible for instances of user-defined classes
(and operations on them) to occur as type parameters. For the compiler
to check conditions involving such values, it is necessary that the
underlying constraint solver be extended.

There are two general ways in which the constraint solver may be
extended.  Both require that the programmer single out some classes
and methods on those classes as {\em pure}. (We shall think of
constants as corresponding to zero-ary methods.) Only instances of
pure classes and expressions involving pure methods on these instances
are allowed in parameter expressions.

How shall constraints be generated for such pure methods? First, the
programmer may explicitly supply with each pure method {\tt T m(T1 x1,
..., Tn xn)} a constraint on {\tt n+2} variables in the constraint
system of the underlying solver that is entailed by {\tt y =
o.m(x1,..., xn)}. Whenever the compiler has to perform reasoning on an
expression involving this method invocation, it uses the constraint
supplied by the programmer. A second more ambitious possibility is
that a symbolic evaluator of the language may be run on the body of
the method to automatically generate the corresponding constraint.

Finally an additional possibility is that the constraint solver itself
be made extensible. In this case, when a user writes a class which is
intended to be used in specifying parameters, he also supplies an
additional program which is used to extend the underlying constraint
solver used by the compiler. This program adds more primitive
constraints and knows how to perform reasoning using these
constraints. This is how I expect we will initially implement the
\Xten{} language. As language designers and implementers we will
provide constraint solvers for finite functions and {\tt Herbrand}
terms on top of arithmetic.





\end{document}
