\documentclass[preprint,nocopyrightspace,9pt]{sigplanconf}
%\documentclass{llncs}

\newif\iflncs
\lncsfalse

\usepackage{times-lite}
\usepackage{mathptm}
\usepackage{txtt}
\usepackage{stmaryrd}
\usepackage{code}
\usepackage{bcprules}
%\usepackage{ttquot}
\usepackage{amsmath}
\usepackage{amssymb}
\usepackage{afterpage}
\usepackage{balance}
\usepackage{floatflt}
\usepackage{defs}
\usepackage{utils}
\usepackage[pdftex]{graphicx}
\usepackage{xspace}
\usepackage{ifpdf}
\usepackage{listings}
\usepackage{x10}

\newif\ifsemantics
%\semanticsfalse
\semanticstrue

\hfuzz=1pt

\pagestyle{plain}


\ifpdf
\setlength{\pdfpagewidth}{8.5in}
\setlength{\pdfpageheight}{11in}
\fi

\newcommand\gxx{GFX\xspace}

\newcommand\xbar[1]{\ensuremath{\bar{\Xcd{#1}}}}
\newcommand\tbar[1]{\ensuremath{\bar{\tt {#1}}}}
\newcommand\exc[2]{\ensuremath{\exists}#1.~#2}
\newcommand\exty[3]{\ensuremath{\exists}#1\ty#2.~#3}
\newcommand\extyty[5]{\ensuremath{\exists}#1\ty#2,#3\ty#4.~#5}
\newcommand\extytyty[7]{\ensuremath{\exists}#1\ty#2,#3\ty#4,#5\ty#6.~#7}


% \input{../../../../vj/res/pagesizes}
% \input{../../../../vj/res/vijay-macros}
\newcommand\alt{\bnf}

\newcommand\Implies{\Rightarrow}

\iflncs
\else
\newtheorem{example}{Example}[section]
\newtheorem{theorem}{Theorem}[section]
\newtheorem{lemma}[theorem]{Lemma}
\newenvironment{proof}{
\trivlist
\item[\hskip \labelsep \textsc{Proof.}]
\selectfont
\ignorespaces}{$\Box$}

%\newtheorem{proof}[theorem]{Proof}
\fi

\begin{document}

\title{Genericity through Dependently Constrained Types}

\iflncs

\author{
Nathaniel Nystrom\inst{1}
\and
Igor Peshansky\inst{1}
\and
Vijay Saraswat\inst{1}
}

\institute{IBM T.~J. Watson Research~Center,
P.O.~Box~704, Yorktown~Heights NY 10598 USA,
\email{\{nystrom,igorp,vsaraswa\}@us.ibm.com}}

\else

\authorinfo{Nathaniel Nystrom\titlenote{IBM T.~J. Watson Research
Center, P.O. Box 704, Yorktown Heights NY 10598 USA}}{}
  {nystrom@us.ibm.com}
\authorinfo{Igor Peshansky$^{\;*}$}{}
  {igorp@us.ibm.com}
\authorinfo{Vijay Saraswat$^{\;*}$}{}
  {vsaraswa@us.ibm.com}

% \conferenceinfo{POPL'08}{XXX}
% \copyrightyear{2008}
% \copyrightdata{[to be supplied]}

\fi

\maketitle

\begin{abstract}
We present a general framework for \emph{generic constrained types}
that captures the notion of value-dependent and
type-dependent (generic) type systems for object-oriented
languages.  Constraint systems formalize systems of partial
information.  Constrained types are formulas \Xcd{C\{c\}} where
\Xcd{C} is the name of a class or an interface and \Xcd{c} is a
constraint on the immutable state of an instance of \Xcd{C} (the
{\em properties}).

The basic idea is to formalize the essence of nominal
object-oriented types as a constraint system, and to permit both
value and type properties and parameters.  Type-generic
dependence is now expressed through constraints on these
properties and parameters.  Type-valued properties are required
to have a run-time representation---the run-time semantics is
not defined through erasure.

Many type systems for object-oriented languages developed over
the last decade can be thought of as constrained type systems in
this formulation.  This framework is parametrized by an
arbitrary constraint system $\cal C$ of interest.  It permits
the development of languages with pluggable type systems, and
supports dynamic code generation to check casts at run-time.

The paper makes the following contributions: (1) We show how to
accommodate generic object-oriented types within the framework of
constrained types. (1) We illustrate the type system with the
development of a formal calculus \gxx and establish type
soundness. (3) We discuss the design and implementation of the
type system for X10, a modern object-oriented language, based on
constrained types.  The type system integrates and extends the
features of nominal types, virtual types, and
Scala's path-dependent types, as well as representing generic
types.
\end{abstract}

\section{Introduction}
\label{sec:intro}

\todo{Awkward, repetitive}

\todo{More positioning, relative to: DML, HM(X), constrained
types (Trifonov, Smith) and subtyping constraints, Java
generics, GJ, PolyJ, C\# generics, virtual types, liquid types}

\todo{Possible claim: first type system
that combines genericity and dep types in some vague general way.}

\todo{Incorporate some text from OOPSLA paper on deptypes.}

\todo{Cite liquid types and whatever it cites}

Modern object-oriented type systems provide 
many features to improve productivity by 
allowing programmers to express program
invariants as types that are
checked by the compiler, without sacrificing the ability to
reuse code.
We present a dependent type system
that extends a class-based language with
statically-enforced constraints on types and values.
This type system 
supports several features of modern object-oriented 
language through natural extensions of the core dependent type
system: generic types, virtual types, and self types
among them.

The key idea is to define \emph{constrained types},
a form of dependent type
defined on predicates over types and over the immutable state of
the program.
The type system is parametrized on a constraint system.
We have formalized the type system in an extension of
Featherweight Java~\cite{FJ}
and provide proof of soundness.
By augmenting the default constraint system,
the type system can serve as a core calculus 
for formalizing extensions of a core object-oriented language. 

This work is done in the context of the X10
programming language~\cite{X10}.
In X10, objects may have both value members (fields)
and type members.
The immutable state of an object is captured by its
\emph{value properties}: public final fields of the object.
For instance, the following class declares a two-dimensional
point with properties \xcd"x" and \xcd"y" of type \xcd"float":
\begin{xten}
class Point(x: float, y: float) { }
\end{xten}

A constrained type is a type \xcd"C{e}", where \xcd"C" is a
class---called the \emph{base class}---and \xcd"e" is a
\emph{constraint}, or list of constraints, on the properties of
\xcd"C" and the final variables in scope at the type.
For example, given the above class definition,
the type \xcd"Point{x*x+y*y<1}" is the type of all
points within the unit circle.

% For brevity, a constraint can be written as
% a comma-separated list of conjuncts; that is, the constraint
% \xcd"c1"~\xcd"&&"~\xcd"c2" can be written
% \xcd"c1,"~\xcd"c2".

Constraints on properties induce a natural subtyping relationship:
\xcd"C{c}" is a subtype of
\xcd"D{d}" if \xcd"C" is a subclass of \xcd"D" and
\xcd"c" entails \xcd"d".  Thus, \xcd"Point{x==1,y==1}"
is a subtype of \xcd"Point{x>0}", which in turn is a subtype of
\xcd"Point{true}"---written simply as \xcd"Point".

In previous
work~\cite{X10,constrained-types}, we considered
only value properties.
In this paper,
to support genericity these types are generalized
to allow \emph{type properties}, type-valued instance
members of an object.
Types may be defined by constraining the type properties as
well as the value properties of a class.

The following code declares a class \xcd"Cell" with a type
property named \xcd"T".
\begin{xten}
class Cell[T] {
    var value: T;
    def get(): T = value;
    def set(v: T) = { value = v; }
}
\end{xten}
The class has a mutable field \xcd"value" of type \xcd"T",
and has \xcd"get" and \xcd"set" methods for accessing the field.

This example shows that type properties are in many ways similar to
type parameters as provided in object-oriented languages such as
Java~\cite{Java3} and Scala~\cite{scala} and in functional
languages such as ML~\cite{ml} and
Haskell~\cite{haskell}.

As the example illustrates,
type properties are types in their own right:
they may be used in any context a type may be used,
including in \xcd"instanceof" and cast expressions.
%
However, the key distinction between type properties
and type parameters is that type properties are instance
members.
Thus, for an expression \xcd"e" of type \xcd"Cell", \xcd"e.T" is
a type, equivalent to the concrete type to which \xcd"T" was
initialized when the object \xcd"e" was instantiated.
To ensure
soundness, \xcd"e" is restricted to final access paths.
Within the body of a class, the unqualified property name \xcd"T" resolves
to \xcd"this.T".

All properties of an object, both type and value, must be bound at object
instantiation and are immutable once bound.  Thus, the type
property \xcd"T" of a given \xcd"Cell" instance must be bound
by the constructor
to a concrete type such as \xcd"String" or \xcd"Point{x>=0}".

As with value properties, type properties may be constrained
by predicates to produce \emph{constrained types}.
Many features of modern object-oriented type systems fall out
naturally from this type system.

\paragraph{Generic types.}
Constraints on type properties lead directly to a form of generic class.
The \xcd"Cell" defined class above is a generic class.
X10 supports
equality constraints, written \xcdmath"T$_1$==T$_2$", and
subtyping constraints, written \xcdmath"T$_1$<:T$_2$", on types.
For instance,
the type \xcd"Cell{T==float}" is the type of all \xcd"Cell"s
containing a \xcd"float".  For an instance \xcd"c" of this type,
the types \xcd"c.T" and \xcd"float" are equivalent.  Thus, the
following code is legal.
\begin{xten}
val x: float = c.get();
c.set(1.0);
\end{xten}

Subtyping constraints enable \emph{use-site variance}~\cite{variance}.
The type \xcd"Cell{T<:Collection}"
constrains \xcd"T" to be a subtype of \xcd"Collection".
All instances with this type must bind \xcd"T" to a subtype of
\xcd"Collection".
Variables of this type may contain \xcd"Cell"s of
\xcd"Collection", \xcd"Cell" of \xcd"List",
or \xcd"Cell" of \xcd"Set", etc.
Subtyping constraints provide similar expressive power as Java
wildcards.  We describe an encoding of wildcards in
Section~\ref{sec:wildcards}. 

\paragraph{Other type systems.}

While this paper focuses mainly on supporting generic types 
through dependent types, the formalism generalizes 
other object-oriented type systems, e.g., self
types~\cite{bruce-binary,bsg95}, virtual
types~\cite{mp89-virtual-classes,beta,ernst06-virtual},
and structually constrained
types~\cite{polyj,haskell-type-classes}.

\subsection{Contributions}

\todo{We need some!}

\subsection{Implementation}

Type properties are a powerful mechanism
for providing genericity in X10.
Unlike existing 
existing proposals for generic types in
Java-like
languages~\cite{Java3,GJ,Pizza,polyj,thorup97,scala},
which 
are implemented via type erasure,
our design supports run-time introspection of generic types.

Another problem with many of these proposals is inadequate support
for primitive types, especially arrays. The performance of primitive arrays is
critical for the high-performance applications for which
X10 is intended. These proposals introduce unnecessary boxing
and unboxing of primitives.
Our design does not require primitives be boxed.

\paragraph{Outline.}
The rest of the paper is organized as follows.
%
Section~\ref{sec:related} discusses related work.
%
An informal overview of generic constrained types in X10
is presented in
Section~\ref{sec:lang}.  
%
Section~\ref{sec:semantics} presents a formal semantics and a
proof of soundness.
%
The implementation of generics in X10 by translation to Java is described in 
Section~\ref{sec:translation}.
%
Finally, Section~\ref{sec:conclusions} concludes.

\todo{Fix this}

\section{X10 language overview}
\label{sec:lang}

This section presents an informal description of the 
generic constrained types in X10.  The type system is formalized
in a simplified version of X10, \gxx (Generic Featherweight X10), in
Section~\ref{sec:semantics}.

X10 is a class-based object-oriented language.
The language has a sequential core similar to Java or Scala, but 
constructs
for concurrency and distribution, as well as constrained types,
described here.
Like Java, the language provides single class inheritance and
multiple interface inheritance.

\subsection{Classes}

Classes in X10 may be declared with any number of type properties and
value properties.  These properties can be constrained with a
\emph{class invariant},
a predicate on the properties of any instance of the class.
%
The general form of a class declaration is:
\begin{xtenmath}
class C[$\tbar{X}$]($\tbar{x} \ty \tbar{T}$){c} extends D{d}
      implements $\tbar{I}${$\tbar{c}$} { $\dots$ }
\end{xtenmath}
This declaration defines a class \xcd"C" with zero or more type properties
          \xbar{X}, zero of more value properties \xbar{x} of
          types \xbar{T}, and a class invariant \xcd"c".
          The class \xcd"C" is a subclass of \xcd"D"
          (constrained by \xcd"d")
          and implements the constrained
          interfaces \xbar{I}\xcd"{"\xbar{c}\xcd"}".

Both classes and interfaces may define properties. Value
properties may be considered to be public final instance fields.
Whereas
Java supports only static fields in interfaces, X10
allows interfaces to define value properties. Any class implementing
an interface must declare or inherit from a superclass 
the properties inherited from the interface.  All properties of
a class,
both type and value, must be initialized by the class's
constructors.

Classes may define fields, methods, and constructors.
Consider the example in Figure~\ref{fig:list}.
The
declaration syntax
is similar to Scala's.  Fields may be
declared either \xcd"val" or \xcd"var".  A \xcd"val" field is
\emph{final} and must be assigned exactly once by the
constructor.  Methods are
declared with a \xcd"def" keyword.
Methods in classes and interfaces may be declared \xcd"static".
Mutable static fields 
are not permitted.
Constructor syntax is
similar to method syntax; X10 adopts Scala's syntax,
using the name \xcd"this" for constructors.
In X10, constructors have a return type, which constrains
the properties of the new object.

\begin{figure}
\begin{xtennoindent}
class List[T](length: int){length >= 0} {
    var head: T;
    var tail: List[T];

    def this[S](): List[S](0) = property[S](0);
    def this[S](hd: S): List[S](1) = {
        property[S](1); head = hd;
    }
    def this[S](hd: S, tl: List[S])
        : List[S](tl.length+1) = {
        property[S](tl.length+1);
        head = hd; tail = tl;
    }

    def map[S](f: (T)=>S): List[S] = ...;

    // XXX
    def get(i: int){0 <= i, i < length} = {
        if (i == 0) return head;
        if (tail != null) return tail.get(i-1);
        throw new IndexOutOfBoundsException();
    }
}
\end{xtennoindent}
\caption{List example}
\label{fig:list}
\end{figure}

\subsection{Constrained types}

A constrained type is written \xcd"C{e}", where \xcd"C" is the
name of a class and \xcd"e" is a constraint on the
properties of \xcd"C" and the final variables in scope at the
type.  

For brevity, the constraint may be omitted and
interpreted as \xcd"true".
The syntax
\xcdmath"C[T$_1$,$\dots$,T$_m$](e$_1$,$\dots$,e$_n$)" is sugar for
\xcdmath"C{X$_1$==T$_1$,$\dots$,X$_m$==T$_m$,x$_1$==e$_1$,$\dots$,x$_n$==e$_n$}"
where \xcd"X$_i$" are the type properties and \xcd"x$_i$" are the
value properties of \xcd"C".
If either list of properties is empty, it may be omitted.

In this shortened syntax, a type argument \xcd"T" used may also
be annotated
with
a \emph{use-site variance tag}, either \xcd"+" or \xcd"-":
if \xcd"X" is a type property, then
the syntax \xcd"C[+T]" is sugar \xcd"C{X<:T}" and
\xcd"C[-T]" is sugar \xcd"C{X:>T}"; of course,
\xcd"C[T]" is sugar \xcd"C{X==T}".
Use-site variance is discussed in more detail in
Section~\ref{sec:variance}

The compiler checks that constraints are expressions
of type \xcd"boolean" and that they can be statically checked by
the compiler's constraint solver.  X10 supports conjunctions of equality and
subtyping consraints.  Compiler plugins may be installed to
handle richer constraint systems such as Presburger arithmetic
or set constraints.

\subsection{Generics}

Type properties and subtyping constraints are used in X10 to 
provide genericity.

Unlike existing 
existing proposals for generic types in
Java-like
languages~\cite{Java3,adding-wildcards,GJ,Pizza,polyj,thorup97,allen03,allen04,csharp,emir06,scala},
which 
are implemented via type erasure,
our design supports run-time introspection of generic types.

Another problem with many of these proposals is inadequate support
for primitive types, especially arrays. The performance of primitive arrays is
critical for the high-performance applications for which
X10 is intended. These proposals introduce unnecessary boxing
and unboxing of primitives.
Our design does not require primitives be boxed.

\subsection{Type constraints and variance}
\label{sec:variance}

Type properties and subtyping constraints are used in X10 to 
provide genericity.

The \xcd"List" class in Figure~\ref{fig:list}.
Consider the following subtypes  of \xcd"List".
\begin{itemize}
\item \xcd"List".  This type has no constraints on the type
property \xcd"T".
Any type that constrains \xcd"T",
is a subtype of \xcd"List".  The type \xcd"List" is equivalent to
\xcd"List{true}".
%
For a \xcd"List" \xcd"l", the return type of the \xcd"get" method
is \xcd"l.T".
Since the property \xcd"T" is unconstrained,
the caller can only assign the return value of \xcd"get"
to a variable of type \xcd"l.T" or of type \xcd"Object".
In the following code, \xcd"y" cannot be passed to the \xcd"set" method
because it is not known if \xcd"Object" is a subtype of \xcd"c.T".
\begin{xten}
val x: l.T = l.get(0);
val y: Object = l.get(1);
l.set(x); // legal
l.set(y); // illegal
\end{xten}

\item \xcd"List{T==float}".
The type property \xcd"T" is bound to \xcd"float".
Assuming \xcd"l" has this type, then following code is legal:
\begin{xten}
val x: float = l.get();
l.set(1.0);
\end{xten}
The type of \xcd"l.get()" is \xcd"l.T", which is equivalent to
\xcd"float".

\item \xcd"List{T<:Collection}".
This type constrains \xcd"T" to be a subtype of \xcd"Collection".
All instances of this type must bind \xcd"T" to a subtype of
\xcd"Collection"; for example \xcd"List[Set]" (i.e.,
\xcd"List{T==Set}" is a subtype of
\xcd"List{T<:Collection}" because \xcd"T==Set" entails
\xcd"T<:Collection".
%
If \xcd"l" has the type \xcd"List{T<:Collection}",
then \xcd"l.get(0)" has type \xcd"l.T", which is an unknown but
fixed subtype of \xcd"Collection"; the return value can be
assigned into a variable of type \xcd"Collection".

\item \xcd"List{T:>String}".  This type bounds the type property
\xcd"T"
from below.  The \xcd"set" method may be called with any
supertype of \xcd"String"; the return type of the \xcd"get"
method is known to be a
supertype of \xcd"String" (and implicitly a subtype of \xcd"Object").
\end{itemize}

\subsection{Type properties}

Type properties may be declared invariant, covariant, or
contravariant.
If a property \xcd"X" of a class \xcd"C" is covariant,
then if \xcd"S" is a subtype of
\xcd"T", the type \xcd"C{X==S}" is a subtype of \xcd"C{X==T}".
Similarly, if \xcd"X" is contravariant, 
                  \xcd"C{X==T}" is a subtype of \xcd"C{X==S}".
It is illegal for a covariant property to occur in a negative
position in its class declaration and for a contravariant
property to occur in a positive position.  A position is
negative if it is a formal parameter type, or occurs in a method
where clause.  A position is positive if it is a return type or
occurs in a method where clause.

\subsection{Constructors}

Objects in X10 are initialized with constructors. 
Constructors are defined using the syntax \xcd"def this",
illustrated with the three \xcd"List"
constructors in Figure~\ref{fig:list}.

Constructors must ensure that all properties of the new object
are initialized and that the class invariants of the object's
class and its superclasses and superinterfaces hold.

Constructors can take zero or more type parameters and zero or
more value parameters.

Properties are initialized with a \xcd"property" statement.
The \xcd"property" statement is used to set all the properties
of the new object simultaneously; the syntax is similar to a \xcd"super"
constructor call.
The first constructor takes zero arguments and initializes the
type property \xcd"T" to the type parameter \xcd"S"
length to \xcd"0".  The second constructor initializes the
length to \xcd"1", the third to one plus the length of the tail.

Constructors have ``return
types'' that can specify an invariant satisfied by the object being
constructed.  This type 
is used as the type of the \xcd"new" expression that
invoked the constructor.
The compiler verifies that the
constructor return type and the class invariant are implied by the
\xcd"property" statement and any \xcd"super"
or \xcd"this" calls in the constructor body.

Classes that do not declare a constructor
have a default constructor with a type parameter for each
type property and a value parameter for each value property.

\subsection{Methods}

Methods in X10 are declared with the \xcd"def" keyword.
The \xcd"List" class in Figure~\ref{fig:list} declares methods
\xcd"get" and \xcd"map".

Like Java, X10 supports both instance and static methods.
Since a type property is an instance member, a static method may
not refer to a type property of the class.

Interfaces are also permitted to have static methods.  Classes
implementing the interface must provide an implementation of the
static methods of the interface.
This feature is
useful when a type property \xcd"T" is constrained to implement
an interface \xcd"I"; static methods of \xcd"I" can be invoked
through \xcd"T".

Methods may have both type and value parameters.  
For instance, the \xcd"map" method in Figure~\ref{fig:list} 
has a type parameter \xcd"S" and a value parameter that is a
function from \xcd"T" to \xcd"S".

A parametrized method can is invoked by giving type arguments before the
expression arguments.  The following code takes a
list of \xcd"String"s and returns a list of string lengths of
type \xcd"int"
\begin{xten}
xs: List[String] = ...;
ys: List[int] = xs.map[int](
        (x: String) => x.length());
\end{xten}

\paragraph{Conditional methods.}

Methods and constructor may also have \emph{where clauses},
constraints on how
the method may be invoked.  The where clause is written after
the
method parameters and before the return type.  The \xcd"get" method in
Figure~\ref{fig:list} requires that the argument \xcd"i" is
within the list bounds.  A method with a where clause
is called a \emph{conditional method}.

For type parameters, method where clauses are 
similar to generalized constraints proposed for
C\#~\cite{emir06}.
%
In the following code, the \xcd"T" parameter is covariant
and so the \xcd"append" methods below are illegal:
\begin{xten}
class List[+T] {
   def append(other: T): List[T] = { ... }
        // illegal
   def append(other: List[T]): List[T] = { ... }
        // illegal
}
\end{xten}
%
However, one can introduce a method parameter and then constrain
the parameter from below by the class's parameter:
For example, in the following code,
\begin{xten}
class List[+T] {
   def append[U](other: U)
        {T <: U}: List[U] = { ... }
   def append[U](other: List[U])
        {T <: U}: List[U] = { ... }
}
\end{xten}

The constraints must be satisfied by the callers of \xcd"append".
For example, in the following code:
\begin{xten}
xs: List[Number];
ys: List[Integer];
xs = ys; // ok
xs.append(1.0); // legal
ys.append(1.0); // illegal
\end{xten}
the call to \xcd"xs.append" is allowed and the result type is \xcd"List[Number]", but
the call to \xcd"ys.append" is not allowed because the caller cannot show that
${\tt Number} \subtype {\tt Double}$.

\paragraph{Method overriding.}

Method overriding is similar to Java: a method of a subclass
with the same name and parameter types overrides a method of the
superclass.  An overridden method may have a return type that is
a subtype of the superclass method's return type.
A method where clause may be weakened by an overriding
method; that is, the where clause of the superclass must entail the  
where clause of the subclass.

\eat{
\subsection{Interfaces}

optional interfaces
value properties in interfaces
static methods in interfaces

\subsubsection{Optional methods and interfaces}

Method where clauses also provide support for optional methods.

\begin{xten}
class List[T] {
    ...
    def print(){T <: Printable} = {
        for (x: T in this)
            x.print();
    }
}
\end{xten}

\xcd"List.print" may only be called on lists instantiated on
subtypes of the \xcd"Printable" interface.

Optional methods generalize to optional interfaces.

\begin{xten}
interface Printable { def print(); }

class List[T] implements Printable if {T <: Printable} {
    ...
    def print(){T <: Printable} = {
        for (x: T in this)
            x.print();
    }
}
\end{xten}

In this case \xcd"List" implements the \xcd"Printable" interface
only if \xcd"List.T" implements \xcd"Printable".
Thus \xcd"List{T==String}"
and \xcd"List{T==List[String]}"
are subtypes of \xcd"Printable", but
\xcd"List{T" \xcd"==DirtyWord}" is not.


Without optional interfaces, \xcd"List" cannot be a subtype
of \xcd"Printable".  The constraint \xcd"{T <: Printable}" on
the \xcd"print" method is more restrictive than the 
constraint (i.e., \xcd"true") on 
\xcd"Printable.print".
}

\subsection{Function-typed properties}

\begin{figure}

\begin{xtenmathnoindent}
class SortedList(compare: (T,T)=>int) extends List {
    def this[T](hd: T, tl: List[T],
                compare: (T,T)=>int)
        : SortedList[T](compare) = {
        super[T](hd, tl);
        property(compare);
    }

    def this[T](hd: T, tl: List[T]){T <: Comparable}
        : SortedList[T](T.compare.(Object)) = {
        this[T](hd, tl, T.compareTo.(Object));
    }

    def add(x: T) = {
        $\dots$ compare(x, y) $\dots$
    }
}
\end{xtenmathnoindent}

\caption{A \Xcd{SortedList} class with function-typed value properties}
\label{fig:sorted}
\end{figure}


X10 supports first-class functions.
Function-typed properties are a useful feature for generic
collection classes.  Consider the definition of the
\xcd"SortedList" class in Figure~\ref{fig:sorted}.
The class has a property \xcd"compare" of type
\xcd"(T,T)=>int"---a function that takes two \xcd"T"s and
returns an \xcd"int".  The class declares two constructors,
one that takes a function to bind to the \xcd"compare"
property, and another that binds \xcd"T"'s \xcd"compare" method to
the property.  The \xcd"compare" method uses the \xcd"equals"
function to compare elements. 

Using this definition, one can create lists with distinct types
of, for example, case-sensitive and case-insensitive strings:
\begin{xten}
val unixFiles
  = new SortedList[String]
        (String.compareTo.(String));
val windowsFiles
  = new SortedList[String]
        (String.compareToIgnoreCase.(String));
\end{xten}

\noindent
The lists \xcd"unixFiles" and \xcd"windowsFiles" are constrained
by different comparison functions.  This allows the programmer
to write code, for instance, in which it is illegal to pass a list of UNIX
files into a function that expects a list of Windows files, and
vice versa.

\section{Structural constraints}
\label{sec:structural}

Type constraints need not be limited to subtyping constraints.
By introducing structural constraints on types, \gxx allows
type properties to be instantiated on any type with a given set
of methods and fields. This feature is useful for reusing code
in separate libraries since it does not require
code of one library to implement an interface to satisfy a
constraint of another library.


\begin{figure}[tp]
\begin{center}
\begin{tabular}{lrcl}
constraints & {\tt c} & ::= & \dots \\
            &        & \bnf & \xcd"T has Sig" \\
signatures  & {\tt Sig} & ::= &
\xcdmath"def this[$\tbar{X}$]($\tbar{x}$: $\tbar{T}$){c}: T" \\
            &        & \bnf &
\xcdmath"def m[$\tbar{X}$]($\tbar{x}$: $\tbar{T}$){c}: T" \\
            &        & \bnf &
\xcdmath"val x{c}: T" \\
            %&  & \bnf & \xcdmath"var x{c}: T" \\
\end{tabular}
\end{center}
\caption{Grammar for structural constraints}
\label{fig:structural}
\end{figure}

In this section, we consider an extension of the X10 type system
to support structural type constraints.
The type
system need not change except by extending the constraint
system.  The syntax for structural constraints is shown in
Figure~\ref{fig:structural}.  A structural constraint of the
form \xcd"T has Sig" can specify that the type \xcd"T" have a
constructor, method, or field of the given signature.

Structural constraints on types are found in many languages.
Haskell supports type
classes~\cite{haskell,haskell-type-classes}.
%ML's module system allows modules to be constrained by
%structural signatures~\cite{ml}.
In Modula-3, type equivalence and subtyping are structural
rather than nominal as in object-oriented languages of the C
family such as C++, Java, Scala, and X10.
%
The language PolyJ~\cite{polyj} allows type parameters to be bounded using
structural where clauses.
For example, the sorted list class from Figure~\ref{fig:sorted}
could be
be written as follows in PolyJ:
\begin{xten}
class SortedList[T] where T { int compareTo(T) } {
    void add(T x) { ... x.compareTo(y) ... }
    ...
}
\end{xten}
The where clause states that the type parameter \xcd"T" must have a
method \xcd"compareTo" with the given signature.

The analogous code for \xcd"SortedList" in the structural
extension of X10 is:
\begin{xten}
class SortedList[T]{T has def compareTo(T): int} {
    def add(x: T) = { ... x.compareTo(y) ... }
    ...
}
\end{xten}

A structural constraint is satisfied if the type has a member of
the appropriate name and with a compatible type.  The constraint
\xcdmath"Z has def m[$\tbar{X}$]($\tbar{x}\ty\tbar{T}$): U"
is satisfied by a type \xcd"T" if it has a method \xcd"m"
with signature
\xcdmath"def m[$\tbar{Y}$]($\tbar{y}\ty\tbar{S}$): V"
and where
(\xcdmath"[$\tbar{Y}$]($\tbar{y}\ty\tbar{S}$) => V")$[\Xcd{T}/\Xcd{Z}]$
is a subtype of
(\xcdmath"[$\tbar{X}$]($\tbar{x}\ty\tbar{T}$) => U")$[\Xcd{T}/\Xcd{Z}]$.
As an example,
the constraint \xcd"X has def compareTo(X): int"
is satisfied by both of the following classes:
\begin{xten}
class C { def compareTo(x: C): int = ...; }
class D { def compareTo(x: Object): int = ...; }
\end{xten}

\todo{The important bit about all this is... don't have to
      change the type system, just the constraints}

\section{Self types}
\label{sec:self}

X10 itself does not support self 
types~\cite{bruce-binary,bsg95}
directly, but type properties can
be used to encode them.

%
We introduce a 
type property \Xcd{class} to the root of the class hierarchy, \Xcd{Object}:
\begin{xtenmath}
class Object[class] { $\dots$ }
\end{xtenmath}
Scala's path-dependent types~\cite{scala} and J\&'s
dependent classes~\cite{nqm06}
take a similar approach.

\noindent
Self types are achieved by
implicitly constraining types so that if a path expression \Xcd{p}
has type \Xcd{C}, then
$\Xcd{p}.\Xcd{class} \subtype \Xcd{C}$.  In particular,
$\Xcd{this}.\Xcd{class}$ is guaranteed to be a subtype
of the lexically enclosing class; the type
$\Xcd{this}.\Xcd{class}$ represents all instances of the fixed,
but statically unknown, run-time class referred to by the \Xcd{this}
parameter.

Self types address the binary method problem~\cite{bruce-binary}.
In the following
example, the class \xcd"BitSet" can be written with a
\xcd"union" method that takes a self type as argument.

\begin{xtenmath}
interface Set {
    def union(s: this.class): void;
}

class BitSet implements Set {
    int bits;
    def union(s: this.class): void {
        this.bits |= s.bits;
    }
}
\end{xtenmath}

\noindent
The implementation of the method is free to access the
\xcd"bits" field of the argument since the constraint
$\Xcd{this}.\Xcd{class} \subtype \Xcd{BitSet}$ ensures the field is
accessible.



\section{Formal semantics}
\label{sec:semantics}

We present a core calculus, \gxx, for X10 with generics.
\gxx is based on Constrained Featherweight
Java~\cite{constrained-types}.

\todo{
Add method overriding rules: covariant return, contravariant
args, weaker constraints
}

The grammar for \gxx is shown in 
Figure~\ref{fig:grammar}.  The calculus elides features of the
full X10 language not relevant to this paper.

\begin{figure}[tp]
\begin{center}
\begin{tabular}{lrcl}
program & {\tt P} & ::= & \xbar{L} \\
classes & {\tt L} & ::= &
\xcdmath"class C[$\tbar{X}$]($\tbar{x} \ty \tbar{T}$){c}" \\
& & & \xcdmath"  extends T { $\tbar{M}$ }" \\
base types & {\tt R} \\
\quad classes & & ::= & \xcd"C" \\
\quad type variables  & & \bnf & \xcd"X" \\
\quad type members    & & \bnf & \xcd"e.X" \\
\quad type type       & & \bnf & \xcd"type" \\
types & {\tt T} & ::= & \xcd"R{c}" \\
methods     & {\tt M} & ::= &
\xcdmath"def m[$\tbar{X}$]($\tbar{x}$: $\tbar{T}$){c}: T = e" \\
expressions & {\tt e} & \\
\quad literals        &         & ::=  & \xcd"true" \bnf \xcd"false" \bnf \xcd"null" \bnf $n$ \\
\quad variables       &         & \bnf & \xcd"x" \\
\quad field access    &         & \bnf & \xcdmath"e.x" \\
\quad call            &         & \bnf & \xcdmath"e$_0$.m[$\tbar{T}$]($\tbar{e}$)" \\
%\quad                 &         & \bnf & \xcdmath"e$_0$.m($\tbar{e}$)" \\
\quad new             &         & \bnf & \xcdmath"new C[$\tbar{T}$]($\tbar{e}$)" \\
%\quad                 &         & \bnf & \xcdmath"new C($\tbar{e}$)" \\
\quad cast            &         & \bnf & \xcdmath"e as T" \\
constraint terms & {\tt t} &     & \\
\quad self            &         & ::=  & \xcd"self" \\
\quad variables       &         & \bnf & \xcd"x" \\
\quad properties      &         & \bnf & \xcd"t".\xcd"x" \\
\quad atoms           &         & \bnf & \xcdmath"g(t$_1$,$\dots$,t$_n$)" \\
\quad new             &         & \bnf & \xcdmath"new C(t$_1$,$\dots$,t$_n$)" \\
constraint & {\tt c} &      & \\
\quad true            &  & ::=  & \Xcd{true} \\
\quad equality        &  & \bnf & $\Xcd{t}_1 \equals \Xcd{t}_2$ \\
\quad existentials    &  & \bnf & $\exc{\Xcd{x}}{\Xcd{c}}$ \\
\quad conjunction     &  & \bnf & $\xbar{c}$ \\
\quad predicates      &  & \bnf & \xcdmath"p(t$_1$,$\dots$,t$_n$)" \\
environments & $\Gamma$ & ::=  & $\epsilon$ \\
            &          & \bnf & $\Gamma$, $\Xcd{c}$ \\
            &          & \bnf & $\Gamma$, $\Xcd{x} \ty \Xcd{T}$ \\
            &          & \bnf & $\Gamma$, $\Xcd{X} \ty \Xcd{type}$ \\
\end{tabular}
\end{center}
\caption{\gxx grammar}
\label{fig:grammar}
\end{figure}

Figure~\ref{fig:grammar2} extends the grammar with syntactic
sugar for subtyping constraints and existential types.
The subtyping constraint
                  $\Xcd{t}_1 \subtype \Xcd{t}_2$ 
                  is atomic formula.
The existential type 
$\exty{\Xcd{x}}{\Xcd{T}}{\Xcd{R\{c\}}}$
is sugar for
$\Xcd{R\{}\exc{\Xcd{x}}{\sigma(\Xcd{x}\ty\Xcd{T}),\Xcd{c}}\Xcd{\}}$.

\begin{figure}[tp]
\begin{center}
\begin{tabular}{lrcl}
types & {\tt T} & ::= & \dots \\
            & & \bnf & $\exty{\Xcd{x}}{\Xcd{T}_0}{\Xcd{T}}$ \\
            & & \bnf & $\exty{\Xcd{X}}{\Xcd{type}}{\Xcd{T}}$ \\
constraint terms & {\tt t} & ::= & \dots \\
\quad literals        &         &      & \xcd"true" \bnf $n$ \bnf \xcd"C" \\
\quad type variables       &         & \bnf & \xcd"X" \\
\quad type properties      &         & \bnf & \xcd"t".\xcd"X" \\
constraint & {\tt c} & ::=  & \dots \\
                  &  & \bnf & $\Xcd{t}_1 \subtype \Xcd{t}_2$ \\
                  &  & \bnf & \Xcd{cons(T,z)} \\
\end{tabular}
\end{center}
\caption{\gxx grammar with subtyping constraints}
\label{fig:grammar2}
\end{figure}

We assume a fixed but unknown constraint system ${\cal D}$.
A program \Xcd{P} is written using constraints from ${\cal D}$,
We assume classes defined in \Xcd{P} do not have a cyclic
inheritance structure.

\infrule[Program OK]{
\Xcd{extends}^+~\mbox{acyclic}
}{
\vdash \xbar{L}~\mbox{ok}
}

\subsection{
The object constraint system, ${\cal O}$
}

From \Xcd{P} and ${\cal D}$
we generate an \emph{object constraint system} ${\cal O}$, shown
in Figure~\ref{fig:object-constraints},
as follows.  Let \Xcd{C} and \Xcd{D} range over names of classes
in \Xcd{P}, \Xcd{f} over field names, \Xcd{m} over method names,
\Xcd{T} over types, and \Xcd{c} over constraints in the
underlying data constraint system ${\cal D}$.

\begin{figure}[tp]
\begin{center}
\begin{tabular}{lrcl}
constraint & {\tt c} 
                    & ::=  & \Xcd{class(C)} \\
                  & & \bnf & \Xcd{C}~\Xcd{extends}~\Xcd{D} \\
                  & & \bnf & $\Xcd{fields(x,}\xbar{f}\ty\xbar{T}\Xcd{)}$ \\
                  & & \bnf & $\Xcd{mtype(x,m,[}\xbar{X}\Xcd{(}\xbar{x}\ty\xbar{T}\Xcd{)\{c\}}$ \\
\end{tabular}

\infrule{
\Xcdmath{class C[$\tbar{X}$]($\tbar{f} \ty \tbar{T}$)\{c\} extends D\{d\} \{ $\tbar{M}$\}} \in \Xcd{P}
}{
\vdashO \Xcd{class(C)} \\
\vdashO \Xcd{C}~\Xcd{extends}~\Xcd{C} \\
\vdashO \Xcd{C}~\Xcd{extends}~\Xcd{D}
}

\infrule[Fields]{
\Xcdmath{class C[$\tbar{X}$]($\tbar{f} \ty \tbar{T}$)\{c\} extends Object \{ $\tbar{M}$\}} \in \Xcd{P}
}{
\Gamma, \Xcd{z}\ty \Xcd{C}\{d\} \vdashO \Xcdmath{fields(z,$\tbar{f}\ty\tbar{T}$)}
}

\infrule[Fields-extends]{
\vdashO \Xcd{C}~\Xcd{extends}~\Xcd{D}
\\
\Xcdmath{class C[$\tbar{X}$]($\tbar{f} \ty \tbar{T}$)\{c\} extends D\{d\} \{ $\tbar{M}$\}} \in \Xcd{P}
\\
\Gamma, \Xcd{z}\ty \Xcd{D\{d\}} \vdashO
\Xcdmath{fields(z,$\tbar{f}_0\ty\tbar{T}_0$)}
}{
\Gamma, \Xcd{z}\ty \Xcd{D\{d\}} \vdashO
\Xcdmath{fields(z,$\tbar{f}_0\ty\tbar{T}_0$, $\tbar{f}\ty\tbar{T}$)}
}

\infrule[Mtype]{
\Xcdmath{class C[$\tbar{X}$]($\tbar{f} \ty \tbar{T}$)\{c\} extends Object \{$\tbar{M}$\}} \in \Xcd{P}
\\
\Xcd{M}_i = 
\Xcdmath{def m$_i$[$\tbar{X}$]($\tbar{x}\ty\tbar{T}$)\{c\}: T = e}
}{
\Gamma, \Xcd{z}\ty \Xcd{C\{d\}} \vdashO
\Xcdmath{mtype(z,m$_i$,[$\tbar{X}$]($\tbar{x}\ty\tbar{T}$)\{c\} $\to$T)}
}

\end{center}
\caption{The constraint system ${\cal O}$}
\label{fig:object-constraints}
\end{figure}

In the method signature
                $\Xcd{[}\xbar{X}\Xcd{(}\xbar{x}\ty\xbar{T}\Xcd{)\{c\}}$,
                the type variables \xbar{X} and data variables
                \xbar{x} are considered bound; formulas with
                bound variables are considered equivalent up to
                $\alpha$-renaming.

The constraint system satisfies the axioms and
inference rules in Figure~\ref{fig:object-constraints}.
The \Xcd{class}, \Xcd{extends}, \Xcd{fields}, and \Xcd{mtype}
constraints are given directly from the program \Xcd{P}.

The constraint system ${\cal C}$ is the disjoint conjunction
${\cal D}$, ${\cal O}$
of
the constraint systems
${\cal D}$ and ${\cal O}$.
(This requires the assumption that 
${\cal D}$ does not have any constraints in common with ${\cal O}$.

\eat{
The X10 compiler permits the constraint system to be extended
with compiler plugins.  The base compiler supports equality
constraints over literals and final variables and subtyping
and equality
constraints over types.
The core constraint system is presented here.  We assume a
constraint solver ${\cal X}$ implementing the plugged-in
constraint systems.

The constraint system does not distinguish between values and
types.  Logical variables \xcd"x" may represent program variables
or type variables.  Path terms \Xcd{p.x} may represent field
accesses or member type references.

The constraint system is shown in Figure~\ref{fig:constraints}.
$\xbar{c}$ is a set of constraints.  The constraint system
satisfies the given structural rules, and supports equality and
subtyping constraints over terms.
}


\subsection{Structural and logical rules}

All judgments are intuitionistic.  In particular, this means
that all constraint systems satisfy the rules and axioms
in Figure~\ref{fig:logic}.

\begin{figure}

\infax[Id]{ \Gamma, \Xcd{c} \vdash \Xcd{c} }

\infrule[Cut]{
\Gamma \vdash \Xcd{c}
\andalso
\Gamma, \Xcd{c} \vdash \Xcd{d}
}{
\Gamma \vdash \Xcd{d}
}

% \infrule[Contraction]{ \xbar{c}, \Xcd{c}, \Xcd{c} \vdashC \Xcd{d} }
                     % { \xbar{c}, \Xcd{c} \vdashC \Xcd{d} }
% \infrule[Permutation]{ \xbar{c}, \Xcd{c}, \Xcd{d} \vdashC \Xcd{e} }
                     % { \xbar{c}, \Xcd{d}, \Xcd{c} \vdashC \Xcd{e} }
% \infrule[Extension]{ \xbar{c}, \vdashC \Xcd{c} }
                     % { \xbar{c}, \xbar{c}' \vdashC \Xcd{c} }

\infrule[Weak-1]{
\Gamma \vdash \phi
\andalso
\Gamma \vdash \Xcd{T} \ty \Xcd{type}
\andalso
\Xcd{x} \not\in \mathit{var}(\Gamma)
}{
\Gamma, \Xcd{x} \ty \Xcd{T} \vdash \phi
}

\infrule[Weak-2]{
\Gamma \vdash \phi
\andalso
\Gamma \vdash \Xcd{c} \ty \Xcd{o}
}{
\Gamma, \Xcd{c} \ty \phi
}

\infrule[And-L]{
\Gamma, \psi_0, \psi_1 \vdash \phi
}{
\Gamma, (\psi_0, \psi_1) \vdash \phi
}

\infrule[And-R]{
\Gamma \vdash \psi_0
\andalso
\Gamma \vdash \psi_1
}{
\Gamma \vdash (\psi_0, \psi_1)
}

\infrule[Exists-R]{
\Gamma \vdash \phi[\Xcd{t}/\Xcd{x}]
}{
\Gamma \vdashC \exc{\Xcd{x}}{\phi}
}

\infrule[Exists-L]{
\Gamma, \Xcd{x} \ty \Xcd{T}, \psi \vdash \phi
\andalso
\Xcd{x}~\mbox{fresh}
}{
\Gamma, \exc{\Xcd{x}\ty\Xcd{T}}{\psi} \vdash \phi
}

\caption{Logical rules}
\label{fig:logic}
\end{figure}


\subsection{Well-formedness rules}

We use the judgment for well-typedness for expressions to
represent well-typedness for constraints.  That is, we posit
a special type \Xcd{o} (traditionally the type of propositions),
and regard constraints as expressions of type \Xcd{o}.

Further, we change the formulation slightly so that there are no
constraints of the form \Xcdmath{p(t$_1$,$\dots$,t$_n$)}; rather
instance method invocation syntax is used to express invocation
of
pre-defined constraints.  This logically leads to the step of
simply marking certain classes as ``predicate'' classes---all
the (instance) methods of these classes whose return type is
\Xcd{o} then correspond to ``primitive constraints''.
Syntactically, we continue to use the symmetric syntax
\Xcdmath{p(t$_1$,$\dots$,t$_n$)} rather than
\Xcdmath{t$_1$.p(t$_2$,$\dots$,t$_n$)}.
The alternative is tor introduce static methods and static
method invocations in the expression language.  This is not
difficult, but is annoying to have to repeat most of the
formulation of instance methods.

This means that the only cases left to handle are all the simple
ones, expression the availability of certain constraints and
operations of type \Xcd{o}.


\begin{figure}
\infax[True]{\Gamma \vdash \Xcd{true} \ty \Xcd{o}}

\infrule[Equals]{
\Gamma \vdash \Xcd{t}_0 \ty \Xcd{T}_0
\andalso
\Gamma \vdash \Xcd{t}_1 \ty \Xcd{T}_1
\andalso
(
\Gamma \vdash \Xcd{T}_0 \subtype \Xcd{T}_1
\vee
\Gamma \vdash \Xcd{T}_1 \subtype \Xcd{T}_2
)
}{
\Gamma \vdash \Xcd{t}_0 \equals \Xcd{t}_1 \ty \Xcd{o}
}

\infrule[And]{
\Gamma \vdash \Xcd{c}_0 \ty \Xcd{o}
\andalso
\Gamma \vdash \Xcd{c}_1 \ty \Xcd{o}
}{
\Gamma \vdash (\Xcd{c}_0, \Xcd{c}_1) \ty \Xcd{o}
}

\infrule[Some]{
\Gamma \vdash \Xcd{t} \ty \Xcd{T}
\andalso
\Gamma \vdash \Xcd{c}[\Xcd{t}/\Xcd{x}] \ty \Xcd{o}
}{
\Gamma \vdash \exc{\Xcd{x}\ty\Xcd{T}}{\Xcd{c}} \ty \Xcd{o}
}

\infrule[Type]{
\Gamma \vdash \Xcd{class(C)}
\andalso
\Gamma, \Xcd{self} \ty \Xcd{C} \vdash \Xcd{c} \ty \Xcd{o}
}{
\Gamma \vdash \Xcd{C\{c\}} \ty \Xcd{type}
}

\caption{Well-formedness rules}
\label{fig:wf}
\end{figure}


\subsection{Subtyping constraints}


\begin{figure}

\infrule[Cons-eq]{
\xbar{c} \vdashC \Xcd{T}_1 \equals \Xcd{T}_2
}{
\xbar{c} \vdashC
\Xcd{cons(}\Xcd{T}_1\Xcd{,z)} \equals 
\Xcd{cons(}\Xcd{T}_2\Xcd{,z)} 
}

\infrule[Cons-sub]{
\xbar{c} \vdashC \Xcd{T}_1 \subtype \Xcd{T}_2
}{
\xbar{c},
\Xcd{cons(}\Xcd{T}_1\Xcd{,z)} \vdashC 
\Xcd{cons(}\Xcd{T}_2\Xcd{,z)} 
}

\infax[Cons]{
\vdashC \Xcd{cons(C,z)}
}

\infrule[Eq-atom]
{ \xbar{c} \vdashC \xbar{s} \equals \xbar{t} }
{ \xbar{c} \vdashC \Xcd{f(}\xbar{s}\Xcd{)} \equals \Xcd{f(}\xbar{t}\Xcd{)} }

% \infrule[Eq-field]{ \xbar{c} \vdashC \Xcd{t}_1 \equals \Xcd{t}_2 }
                  % { \xbar{c} \vdashC \Xcd{t}_1.\Xcd{f} \equals \Xcd{t}_2.\Xcd{f} }
% \infrule[Eq-type]{ \xbar{c} \vdashC \Xcd{t}_1 \equals \Xcd{t}_2 }
                  % { \xbar{c} \vdashC \Xcd{t}_1.\Xcd{X} \equals \Xcd{t}_2.\Xcd{X} }

\infax[Eq-refl]{ \xbar{c} \vdashC \Xcd{t} \equals \Xcd{t} }

\infrule[Eq-trans]{
        \xbar{c} \vdashC \Xcd{t}_1 \equals \Xcd{t}_2
        \andalso
        \xbar{c} \vdashC \Xcd{t}_2 \equals \Xcd{t}_3
        }
        { \xbar{c} \vdashC \Xcd{t}_1 \equals \Xcd{t}_3 }

\infrule[Eq-sym]{
        \xbar{c} \vdashC \Xcd{t}_1 \equals \Xcd{t}_2
        }
        { \xbar{c} \vdashC \Xcd{t}_2 \equals \Xcd{t}_1 }

\infrule[Eq-sub]{
\xbar{c} \vdashC \Xcd{T}_1 \subtype \Xcd{T}_2 \\
\xbar{c} \vdashC \Xcd{T}_2 \subtype \Xcd{T}_1
}{
\xbar{c} \vdashC \Xcd{T}_1 \equals \Xcd{T}_2
}

\infrule[Sub-cons]{
\xbar{c} \vdashC \Xcd{C\{c\}} \ty \Xcd{type}
\andalso
\xbar{c}, \Xcd{c} \vdashC \Xcd{d}
}{
\xbar{c} \vdashC \Xcd{C\{c\}} \subtype \Xcd{C\{d\}}
}

\infrule[Sub-super]{
\mbox{\Xcdmath{C[$\tbar{X}$]($\tbar{x}$: $\tbar{T}$)\{c\} ext T \{\ K\ $\tbar{M}$\ $\tbar{F}$\ \}}}
}{
\vdashC \Xcd{C} \subtype \Xcd{T} \\
}

\infax[Sub-object]{
\vdashC \Xcd{T} \subtype \Xcd{Object}
}

\infrule[Sub-eq]{
\xbar{c} \vdashC \Xcd{T}_1 \equals \Xcd{T}_2
}{
\xbar{c} \vdashC \Xcd{T}_1 \subtype \Xcd{T}_2
}

\infrule[Sub-trans]{
\xbar{c} \vdashC \Xcd{T}_1 \subtype \Xcd{T}_2
\andalso
\xbar{c} \vdashC \Xcd{T}_2 \subtype \Xcd{T}_3
}{
\xbar{c} \vdashC \Xcd{T}_1 \subtype \Xcd{T}_3
}

\caption{Equality and subtyping rules}
\label{fig:subtyping}
\end{figure}



\subsection{
Constraint projection
}

First, for a type environment $\Gamma$,
we define the \emph{constraint projection},
$\sigma(\Gamma)$ thus:

\begin{align*}
\sigma(\epsilon) &= \Xcd{true} \\
\sigma(\Gamma, \Xcd{x} \ty \Xcd{T}) &=
        \sigma(\Gamma),
        \cons{\Xcd{T}}{\Xcd{x}}
\\
\sigma(\Gamma, \Xcd{c}) &= \sigma(\Gamma), \Xcd{c} \\
\end{align*}

The auxiliary function $\mathit{cons}$
specifies the constraint for a type \Xcd{T} with \Xcd{self}
bounds to \Xcd{x}.
The constraint projection uses an atomic formula \Xcd{cons},
which is equated to the constraint of \Xcd{T} if \Xcd{T} is not
a type variable.

\begin{align*}
\cons{\Xcd{C}}{\Xcd{z}} &=
    \Xcd{cons(C,z)} \\
\cons{\Xcd{C\{c\}}}{\Xcd{z}} &=
    \Xcd{c}[\Xcd{z}/\Xcd{self}], \Xcd{cons(C\{c\},z)==c}[\Xcd{z}/\Xcd{self}] \\
\cons{\Xcd{p.X}}{\Xcd{z}} &=
    \Xcd{cons(p.X,z)} \\
\cons{\Xcd{X}}{\Xcd{z}} &=
    \Xcd{cons(X,z)} \\
\end{align*}

\noindent
Thus, for example, the constraint projection of the environment:
\begin{quote}
\xcdmath"b: D, a: C{self.X==D{d},self.Y<:b.Z}"
\end{quote}
\noindent is:
\begin{quote}
\xcdmath"a.X==D{d}, a.Y<:b.Z" \\
\end{quote}

\eat{
\subsection{
        Judgments
}

The following judgments will be defined:

\begin{itemize}
\item
     The type {\tt T} is well-formed, given the assumptions $\Gamma$:

    $\Gamma \vdash {\tt T} \ty {\tt type}$

\item
     The type {\tt S} is a subtype of {\tt T}, under the assumption $\Gamma$:

      $\Gamma \vdash {\tt S} \subtype {\tt T}$

    \item The expression {\tt e} is of type {\tt T}, given the assumptions $\Gamma$:

      $\Gamma \vdash {\tt e} \ty {\tt T}$

    \item The method {\tt M} is well-defined for the class {\tt C}
given assumptions $\Gamma$:

      $\Gamma \vdash {\tt M}~\mbox{OK in}~{\tt C}$

    \item The field {\tt f: T} is well defined for the class {\tt C} given assumptions $\Gamma$:

      $\Gamma \vdash {\tt f: T}~\mbox{OK in}~C$

    \item The class definition {\tt L} is well defined given assumptions $\Gamma$:

      $\Gamma \vdash {\tt L}~\mbox{OK}$

\end{itemize}


In what follows we will sometimes think of the family of five
judgments
as a single judgment $\Gamma \vdash \phi$ where $\phi$ ranges over the
formulas 
    ${\tt T} \ty {\tt type}$,
      ${\tt S} \subtype {\tt T}$,
      ${\tt e} \ty {\tt T}$,
      ${\tt M}~\mbox{OK in}~{\tt C}$,
      ${\tt f: T}~\mbox{OK in}~C$, and
      ${\tt L}~\mbox{OK}$.


Now, these judgments need to satisfy certain properties:

\begin{enumerate}

\item
    $\Gamma \vdash {\tt T} \ty {\tt type}$
whenever 
      $\Gamma \vdash {\tt e} \ty {\tt T}$; that is,
if we can conclude that {\tt e}
      has type {\tt T} (under certain assumptions), then under those
      assumptions we must be able to conclude that {\tt T} is well-defined.

\item
    $\Gamma \vdash {\tt S} \ty {\tt type}$ and
    $\Gamma \vdash {\tt T} \ty {\tt type}$ whenever
      $\Gamma \vdash {\tt S} \subtype {\tt T}$.

\item
If 
      $\Gamma \vdash {\tt e} \ty {\tt T}$ and if {\tt x}
is a variable occurring free in ${\tt e} \ty {\tt T}$, then for some
      type {\tt U},
      $\Gamma \vdash {\tt x} \ty {\tt U}$.
That is, all free variables on the right-hand
      side of the judgment are actually defined on the left-hand side.
\end{enumerate}


Keeping in mind these requirements, the rules are as follows. Below,
whenever we use the assertion ``{\tt x} free'' in the antecedent of
a rule we mean
that {\tt x} is not free in the consequent of the rule.


\subsection{
      Structural and Logical Rules
}


First, we present the structural rules for $\vdash$. The
judgment
$\Gamma\vdash {\tt e} \ty {\tt T}$ is
intuitionistic. That is, $\Gamma$ is considered a multiset of assertions, and
the judgment possesses the inference rules:

\infrule{
\Gamma \vdash {\tt e} \ty {\tt T}
\andalso
\Gamma \vdash {\tt S} \ty {\tt type}
\andalso
    \mbox{{\tt x} not in $\mathit{var}(\Gamma)$}
}{
\Gamma, {\tt x} \ty {\tt S} \vdash {\tt e} \ty {\tt T}
}



\infrule{
\Gamma \vdash {\tt e} \ty {\tt T}
\andalso
\Gamma \vdash {\tt c} \ty {\tt boolean}
}{
\Gamma, {\tt c} \vdash {\tt e} \ty {\tt T}
}


We also assume the following rule for conjunctions on the left and right:

\infrule{
\Gamma, \phi_1 , \phi_2 \vdash \phi
}{
\Gamma, (\phi_1 , \phi_2 ) \vdash \phi
}


\infrule{
\Gamma \vdash \phi_1 
\andalso
     \Gamma \vdash \phi_2  
}{
\Gamma \vdash (\phi_1 , \phi_2 )
}



Existential quantification is governed by the following standard rules,
specialized for the particular kinds of formulas we are dealing with:


\infrule{
\Gamma \vdash {\tt e} \ty {\tt T}[{\tt t}/{\tt x}]
\andalso
\Gamma \vdash {\tt t} \ty {\tt S}
}{
\Gamma \vdash {\tt e} \ty ({\tt x} \ty {\tt S};~{\tt T})
}


\infrule{
\Gamma, {\tt x} \ty {\tt S}, {\tt c} \vdash {\tt e} \ty {\tt T}
\andalso
\mbox{{\tt x} fresh}
}{
\Gamma, ({\tt x} \ty {\tt S};~{\tt c}) \vdash {\tt e} \ty {\tt T}
}


\infrule{
\Gamma, {\tt x} \ty {\tt S}, {\tt y} \ty \Xcd{C\{c\}} \vdash {\tt e} \ty {\tt T}
\andalso
\mbox{{\tt x} fresh}
}{
\Gamma, {\tt y} \ty \Xcd{C\{x:S; c\}} \vdash {\tt e} \ty {\tt T}
}

}

\subsection{
Type well-formedness
}

\begin{figure}

\infrule{
\mbox{\Xcdmath{C[$\tbar{X}$]($\tbar{x}$: $\tbar{T}$)\{c\} ext T \{\ $\tbar{M}$\ $\tbar{F}$\ \}}}
}{
\vdash \Xcd{C} \ty \Xcd{type}
}

\infrule{
\Gamma \vdash \Xcd{T} \ty \Xcd{type}
\andalso
\Gamma, \Xcd{self} \ty \Xcd{T} \vdash \Xcd{c} \ty \Xcd{Boolean}
\andalso
\sigma(\Gamma) \vdashC \Xcd{c}~\mbox{OK}
}{
\Gamma \vdash \Xcd{T\{c\}} \ty \Xcd{type}
}

\infrule{
\Gamma \vdash \Xcd{p} \ty \Xcd{T}
\andalso
\Gamma \vdash \Xcd{T}~\Xcd{has}~\Xcd{X}
}{
\Gamma \vdash \Xcd{p.X} \ty \Xcd{type}
}

\infax{
\Gamma, \Xcd{X} \ty \Xcd{type} \vdash \Xcd{X} \ty \Xcd{type}
}

\caption{Type well-formedness}
\label{fig:type-wf}
\end{figure}

\subsection{
      Type inference rules
}

\subsubsection{Constraint rules}

\begin{figure}

\infrule[Has-class]{
\mbox{\Xcdmath{C[$\tbar{X}$]($\tbar{x}$: $\tbar{T}$)\{c\} ext T \{\ K\ $\tbar{M}$\ $\tbar{F}$\ \}}}
}{
\vdash {\tt C}~{\tt has}~{\tt K} \\
\vdash {\tt C}~{\tt has}~{\tt X}_i \\
\vdash {\tt C}~{\tt has}~{\tt x}_i \ty {\tt T}_i \\
\vdash {\tt C}~{\tt has}~{\tt M}_i \\
\vdash {\tt C}~{\tt has}~{\tt F}_i
}

\infrule[Has-sub]{
{\tt Z} \not= {\tt K}
\andalso
\Gamma \vdash {\tt T}_1~{\tt has}~{\tt Z}
\andalso
\sigma(\Gamma) \vdash {\tt T}_2 \subtype {\tt T}_1
}{
\Gamma \vdash {\tt T}_2~{\tt has}~{\tt Z}
}

\caption{Structural constraints}
\label{fig:structural}
\end{figure}

\subsubsection{
        Expression typing judgment
}

\eat{
We define \Xcd{T\{c\}} as follows:

\begin{align*}
\Xcd{D\{c\}} &= \Xcd{D\{c\}} \\
\Xcd{D\{d\}\{c\}} &= \Xcd{D\{d,c\}} \\
\Xcd{X\{c\}} &= \Xcd{X\{c\}} \\
\Xcd{p.X\{c\}} &= \Xcd{p.X\{c\}} \\
\end{align*}

\begin{align*}
\exists\Xcd{x}\ty \Xcd{S}.~\Xcd{C} &= \Xcd{C} \\
\exists\Xcd{x}\ty \Xcd{S}.~\Xcd{C\{c\}} &=
        \Xcd{C\{}\exists\Xcd{x}.~\sigma(\Xcd{x}\ty \Xcd{S})\Xcd{,c\}} \\
\exists\Xcd{x}\ty \Xcd{S}.~\Xcd{p.X} &=
        \Xcd{p.X\{}\exists\Xcd{x}.~\sigma(\Xcd{x}\ty \Xcd{S})\Xcd{,c\}} \\
\exists\Xcd{x}\ty \Xcd{S}.~\Xcd{X} &=
        \Xcd{X\{}\exists\Xcd{x}.~\sigma(\Xcd{x}\ty \Xcd{S})\Xcd{,c\}} \\
\end{align*}
}

\begin{figure}

\infrule[T-sub]{
\Gamma \vdash \Xcd{e} \ty \Xcd{S}
\andalso
\sigma(\Gamma) \vdashC \Xcd{S} \subtype \Xcd{T}
\andalso
\Gamma \vdash \Xcd{T} \ty \Xcd{type}
}{
\Gamma \vdash \Xcd{e} \ty \Xcd{T}
}

\infax[T-bool]{
\vdash \Xcd{true} \ty \Xcd{Boolean\{self==true\}} \\
\vdash \Xcd{false} \ty \Xcd{Boolean\{self==false\}}
}

\infax[T-int]{
\vdash n \ty \Xcd{Int\{self==}n\Xcd{\}}
}

\infrule[T-eq]{
\Gamma \vdash \Xcd{e}_1 \ty \Xcd{T}_1
\andalso
\Gamma \vdash \Xcd{e}_2 \ty \Xcd{T}_2
}{
\Gamma \vdash \Xcd{e}_1 \equals \Xcd{e}_2 \ty
        \extyty{\Xcd{z}_1}{\Xcd{T}_1}{\Xcd{z}_2}{\Xcd{T}_2}
        {\Xcd{Boolean\{self==(}{\Xcd{z}_1}\equals{\Xcd{z}_2}\Xcd{)\}}}
}

\infrule[T-teq]{
\Gamma \vdash \Xcd{T}_1 \ty \Xcd{type}
\andalso
\Gamma \vdash \Xcd{T}_2 \ty \Xcd{type}
}{
\Gamma \vdash \Xcd{T}_1 \equals \Xcd{T}_2 \ty \Xcd{Boolean}
}

\infrule[T-tsub]{
\Gamma \vdash \Xcd{T}_1 \ty \Xcd{type}
\andalso
\Gamma \vdash \Xcd{T}_2 \ty \Xcd{type}
}{
\Gamma \vdash \Xcd{T}_1 \subtype \Xcd{T}_2 \ty \Xcd{Boolean}
}

\infax[T-var]{
\Gamma, \Xcd{x} \ty \Xcd{T} \vdash \Xcd{x} \ty \Xcd{T}
}

\infrule[T-cast]{
\Gamma \vdash \Xcd{e} \ty \Xcd{S}
\andalso
\Gamma \vdash \Xcd{T} \ty \Xcd{type}
}{
\Gamma \vdash \Xcd{e}~\Xcd{as}~\Xcd{T} \ty \Xcd{T}
}

\infrule[T-field]{
\Gamma \vdash \Xcd{e} \ty \Xcd{T}
%\\
%\Xcd{T} = \exty{\Xcd{z}}{\Xcd{S}}{\Xcd{S\{self==z\}}}
\\
\Xcd{T}~\Xcd{has}~\Xcd{f}\Xcd{\{c\}} \ty \Xcd{U}
\\
\sigma(\Gamma, \Xcd{this} \ty \Xcd{T}) \vdashC \Xcd{c}
}{
\Gamma \vdash \Xcd{e}.\Xcd{f} \ty \exty{\Xcd{this}}{\Xcd{T}}{\Xcd{U\{self==this.f\}}}
}

\infrule[T-invk]{
\Gamma \vdash \Xcd{e}_0 \ty \Xcd{T}_0
\andalso
\Gamma \vdash \xbar{e} \ty \xbar{T}
\\
\Xcd{T}_0~\Xcd{has}~\Xcd{def}~\Xcd{m[}\xbar{X}\Xcd{](}\xbar{x} \ty \xbar{S}\Xcd{)\{c\}} \ty {\tt U}~\Xcd{=}~\Xcd{e}
\\
\Gamma' = \Gamma, \xbar{X} \ty \Xcd{type},
        \Xcd{this} \ty \Xcd{T}_0,
        \xbar{x} \ty \xbar{T},
        \xbar{X} \equals \xbar{V}
\\
\sigma(\Gamma') \vdashC \Xcd{c}
\\
\sigma(\Gamma') \vdashC \xbar{T} \subtype \xbar{S}
}{
\Gamma \vdash
\Xcd{e}_0.\Xcd{m[}\xbar{V}\Xcd{](}\xbar{e}\Xcd{)} \ty
\extytyty{\xbar{X}}{\Xcd{type}}{\Xcd{this}}{\Xcd{T}_0}{\xbar{x}}{\xbar{T}}{\Xcd{U}}
}

\infrule[T-new]{
\Gamma \vdash \xbar{e} \ty \xbar{T}
\\
\Xcd{C}~\Xcd{has}~\Xcd{def}~\Xcd{this[}\xbar{X}\Xcd{](}\xbar{x} \ty \xbar{S}\Xcd{)\{c\}} \ty {\tt U}~\Xcd{=}~\dots
\\
\Gamma' = \Gamma, \xbar{X} \ty \Xcd{type}, \Xcd{this} \ty \Xcd{C}, \xbar{x} \ty \xbar{T}, \xbar{V} \equals \xbar{X}
\\
\Gamma'' = \Gamma, \xbar{X} \ty \Xcd{type}, \Xcd{this} \ty \Xcd{U}, \xbar{x} \ty \xbar{T}, \xbar{V} \equals \xbar{X}
\\
\sigma(\Gamma') \vdashC \Xcd{c}
\\
\sigma(\Gamma') \vdashC \xbar{T} \subtype \xbar{S}
\\
\sigma(\Gamma'') \vdashC \mathit{inv}(\Xcd{C}),
}{
\Gamma \vdash
\Xcd{new}~\Xcd{C[}\xbar{V}\Xcd{](}\xbar{e}\Xcd{)} \ty
\extytyty{\xbar{X}}{\Xcd{type}}{\Xcd{this}}{\Xcd{C}}{\xbar{x}}{\xbar{T}}{\Xcd{U}}
}

\caption{Typing rules}
\label{fig:typing}
\end{figure}

The cast rule
\rn{T-cast}
requires that the cast type be well-formed. 

The field access rule \rn{T-field}
differs from the rule in the paper in that there is no need to
substitute a fresh variable for the receiver. Note that {\tt this} may be free
in {\tt S}---that would be a reference to the current object in the code in
which {\tt e.f} occurs, not a reference to the receiver of the {\tt e.f} field
selection (i.e., the object obtained by evaluating {\tt e}).

\noindent
if we allow adding constraints to arbitrary types---do we?

TODO: type parameters!

Now we consider the rule for method invocation. Assume that in a type
environment $\Gamma$ the expressions ${\tt e_0}, \dots, {\tt e_n}$
have the types ${\tt T_0}, \dots, {\tt T_n}$.
Since the
actual values of these expressions are not known, we shall assume that
they take on some fixed but unknown values
                                     ${\tt z_0}, \dots, {\tt z_n}$
of types ${\tt T_0}, \dots, {\tt T_n}$.
Now, for ${\tt z_0}$ as receiver, let us assume that the type
${\tt T_0}$ has a method named ${\tt m}$
with signature
$[\xbar{Z}](\xbar{z} \ty \xbar{S})\Xcd{\{c\}} \to {\tt U}$
(Let ${\tt T_0} = \Xcd{C\{d\}}$.
 If there is no
method named {\tt m} for the class {\tt C} then this method invocation cannot be
type-checked. Without loss of generality, we may assume that the
type parameters of this method are named
                                     ${\tt Z_1}, \dots, {\tt Z_k}$, and
the value parameters are named
                                     ${\tt z_1}, \dots, {\tt z_n}$
since we are free to choose
variable names as we wish.)
Now, for the method to be invokable,
it must be the case that the types
    ${\tt T_1}, \dots, {\tt T_n}$
are subtypes of
    ${\tt S_1}, \dots, {\tt S_n}$.
(Note
that there may be no occurrences of {\tt this} in
    ${\tt S_1}, \dots, {\tt S_n}$---they have been
replaced by ${\tt z_0}$.)
Further, it must be the case that for these parameter
values, the constraint {\tt c} is entailed. Given all these assumptions it
must be the case that the return type is {\tt U}, with all the parameters
    ${\tt z_0}, \dots, {\tt z_n}$
existentially quantified.


\subsubsection{
        Class OK judgment
}

The following rule is modified from what we had in the paper to ensure
that all the types are well-formed (under the assumption {\tt this} \ty {\tt C}).
Note
that the variables $\xbar{x}$ are permitted to occur in the types $\Xcd{T}_0, \xbar{T}$,
hence their typing assertions must be added to $\Gamma$.

\infrule[Method OK]{
\Gamma = \Xcd{this} \ty \Xcd{C\{self==this},\mathit{inv}(\Xcd{C})\Xcd{\}},
        \xbar{x} \ty \xbar{T}\Xcd{\{self==\}}\xbar{x}\Xcd{\}},
        \Xcd{c}
\\
\Gamma \vdash \Xcd{e} \ty \Xcd{U}
\\
\sigma(\Gamma) \vdashC \Xcd{U} \subtype \Xcd{T}
}{
\Xcd{def}~\Xcd{m[}\xbar{X}\Xcd{](}\xbar{x} \ty
\xbar{T}\Xcd{)\{c\}} \ty \Xcd{T}~\Xcd{=}~\Xcd{e}~\mbox{OK in}~\Xcd{C}
}


This rule did not exist in our submission. This is necessary to ensure
that the types of fields are well-formed.

\infrule[Field OK]{
\Xcd{this} \ty \Xcd{C}, \Xcd{c} \vdash \Xcd{T} \ty \Xcd{type}
}{
\Xcd{val}~\Xcd{f}\Xcd{\{c\}} \ty \Xcd{T}~\mbox{OK in}~\Xcd{C}
}


This rule is now modified to ensure that all the types and methods in
the body of the class are well-formed.

\infrule[Class OK]{
K~\mbox{OK in}~{\tt C}
\\
\xbar{M}~\mbox{OK in}~{\tt C}
\\
\xbar{F}~\mbox{OK in}~{\tt C}
\\
{\tt this} \ty {\tt C} \vdash {\tt T} \ty {\tt type}
}{
\mbox{\Xcdmath{C[$\tbar{X}$]($\tbar{x}$: $\tbar{T}$)\{c\} ext T \{\ K\ $\tbar{M}$\ $\tbar{F}$\ \}}}~\mbox{OK}
}

TODO: method overriding


\subsubsection{
        Subtype judgment
}

\infrule{
\sigma(\Gamma) \vdash_{\cal C} {\tt T_1} \subtype {\tt T_2}
}{
\Gamma \vdash {\tt T_1} \subtype {\tt T_2}
}



\section{Constraint solver}
\label{sec:solver}

The goal of the constraint solver is 
to check an assertion $\xbar{c} \vdashC \Xcd{d}$.

\eat{
Inference

The first step is to normalize constraints
into a set of constraint judgments
$\xbar{c} \vdashC \Xcd{c}$ where $\Xcd{c}$ contains no conjunctions.


Once in normalized form, the inference proceeds as follows:
Select a constraint $\xbar{c} \vdashC \Xcd{c}$.
If not consistent, fail.
If valid, ok.
If not valid, generate assignment of variables that makes it
true, adding the assignment to the assumptions for all
constraints.

The inference algorithm must specify the criteria for:
\begin{itemize}
\item selecting the next constraint to solve
\item generating the variable assignment consistent with all
other constraints (to avoid backtracking)
\end{itemize}

Pick an unassigned variable, find weakest assignment that makes just
this clause true.  Does the weakest assignment exist?

Question: can we ensure each clause involves only one or two
unknowns?
}

We add the following rules to allow type arguments to calls to
be omitted.

\infrule[T-invk-inferred]{
\xbar{Y}~\mbox{fresh}
\\
\Gamma, \xbar{Y} \ty {\tt type}
\vdash
\Xcd{e}_0.\Xcd{m[}\xbar{Y}\Xcd{](}\xbar{e}\Xcd{)} \ty
\Xcd{T}
}{
\Gamma \vdash
\Xcd{e}_0.\Xcd{m(}\xbar{e}\Xcd{)} \ty
\Xcd{T}
}

\infrule[T-new-inferred]{
\xbar{Y}~\mbox{fresh}
\\
\Gamma, \xbar{Y} \ty {\tt type}
\vdash
\Xcd{new}~\Xcd{C[}\xbar{Y}\Xcd{](}\xbar{e}\Xcd{)} \ty
\Xcd{T}
}{
\Gamma \vdash
\Xcd{new}~\Xcd{C(}\xbar{e}\Xcd{)} \ty
\Xcd{T}
}

\subsection{Constraint representation}

%\newcommand\eqedge{\rightleftharpoons}
\newcommand\eqedge{\sim}
\newcommand\flowedge{\to}
\newcommand\treeedge[1]{\mapsto_{#1}}
\newcommand\typeedge{\mapsto_{\tt type}}

Represent a constraint as a graph $G$.
Each node represents a constraint term for a value or a type.
The node for a path $p$ is written $v_p$;
the node for a type $T$ is written $V_T$.
There are four kinds of edges:
\begin{enumerate}
\item undirected equivalence edges,
        $v_p \eqedge v_q$ and $V_S \eqedge V_T$,
\item type edges, $v_p \typeedge V_T$,
\item tree edges, $v_p \treeedge{f} v_{p.f}$
              and $v_p \treeedge{X} V_{p.X}$, and
\item flow edges, $V_S \flowedge V_T$.
\end{enumerate}

First, each constraint term is mapped to a node in the graph as
follows.
Associate each term $t$ with a node
$v_t$.  For each access path {\tt p.x}, add a tree edge
$v_{{\tt p}} \treeedge{{\tt x}} v_{{\tt p.x}}$.
For each path type {\tt p.X}, add a tree edge
$v_{{\tt p}} \treeedge{{\tt X}} V_{{\tt p.X}}$.
For each atomic formula ${\tt f}(\xbar{t})$, add the tree edge
$v_{{\tt f}(\xbar{t})} \treeedge{i} v_{t_i}$ for all $i$.
If term $t$ has type $T$, add $v_t \typeedge V_{t{\tt .type}}$
and
add $V_T \eqedge V_{t{\tt .type}}$ to $G$.

Type nodes are sets of classes.

Next, constraints are incorporated into the graph:

\begin{itemize}
\item
For constraint {\tt p==q}, add $v_{\tt p} \eqedge v_{\tt q}$ to $G$.

\item
For constraint {\tt S==T}, add $V_{\tt S} \eqedge V_{\tt T}$ to $G$.

\item
For constraint {\tt S<:T},
add $V_{\tt S} \flowedge V_{\tt T}$
to $G$.

\end{itemize}

\subsection{Solving}

A flow-path is a path that follows flow and equivalence edges
only.
A type-path is a path that follows type and equivalence edges
only.

Now, we saturate: 
If there is a type-path $v_t \typeedge^* V_{\tt C\{c\}}$,
add $c[t/\Xcd{self}]$ to the worklist.

        Can saturate lazily when doing a lookup.
        EXCEPT: a type may have an arbitrary constraint
                \xcd"C{self.x==3 && y > 7}", so affect is non-local
        EXCEPT: c is x.f==...
                with x: C{c}
                need to avoid infinite loop

To check:

\begin{itemize}
\item To check
constraint {\tt p==q}, check if $v_{\tt p} \eqedge^* v_{\tt q}$.
\item To check
constraint {\tt S<:T}, check if there is a flow-path from $V_{\tt S}$ to
$V_{\tt T}$.  This requires checking entailment of the type constraints and
adding more edges to the graph.  (XXX details!)
Add the flow edge to memoize.
\end{itemize}

\section{Translation}
\label{sec:translation}

This section describes an implementation approach for
generic constrained types on a Java virtual machine.
We describe the implementation as a translation to Java.

The design
is a hybrid design based on the implementation of parametrized classes in
NextGen~\cite{allen03,allen04} and the implementation of
PolyJ~\cite{polyj}.
Generic classes are translated into template classes
that are instantiated on demand at run time by binding the type properties
to concrete types.  To implement run-time type checking (e.g.,
casts), type properties are represented at run time
using \emph{adapter objects}.

This design, extended to handle language features
not described in this paper, has been implemented in the X10
compiler.  The X10 compiler is built on the Polyglot framework
and translates X10 source to Java source\footnote{There is also
a translation from X10 to C++ source, not described here.}

\subsection{Classes}

Each class is translated into a \emph{template class}.
The template class is compiled by a Java compiler (e.g., javac)
to produce a class file.
At run time, when a constrained type \xcd"C{c}" is first referenced, a
class loader loads the template class for \xcd"C" and then transforms the
template class bytecode, specializing it to the constraint
\xcd"c".

For example, consider the following classes.
\begin{xten}
class A[T] {
    var a: T;
}
class C {
    val x: A[Int] = new A[Int]();
    val y: Int = x.a;
}
\end{xten}

The compiler generates the following code:
\begin{xten}
class A {
    // Dummy class needed to type-check uses of T.
    @TypeProperty(1) static class T { }

    T a;

    // Dummy getter and setter; will be eliminated
    // at run time and replaced with actual gets
    // and sets of the field a.
    @Getter("a") <S> S get$a() { return null; }
    @Setter("a") <S> S set$a(S v) { return null; }
}

class C {
    @ActualType("A$Int")
    final A x = Runtime.<A>alloc("A$Int");
    final int y = x.<Integer>get$a();
}
\end{xten}

The member class \xcd"A.T" is used in place of the
type property \xcd"T". 
The \xcd"Runtime.alloc" method is used
used in place of a constructor call.
This code is compiled to Java bytecode.


Then, at run time, suppose the expression \xcd"new C()" is
evaluated.  This causes \xcd"C" to be loaded.
The class loader transforms the bytecode as if it had
been written as follows:

\begin{xten}
class C {
    final A$Int x = new A$Int();
    final int y = x.a;
}
\end{xten}

The \xcd"ActualType" annotation is used to change the
type of the field \xcd"x" from \xcd"A" to \xcd"A$Int".
The call to \xcd"Runtime.alloc" is replaced with a
constructor call.  The call to \xcd"x.get$a()" is
replaced with a field access.

The implementation cannot generate this code directly because
the class \xcd"A$Int" does not yet exist; the Java source compiler
would fail to compile \xcd"C".

Next, as the \xcd"C" object is being constructed, the expression
\xcd"new A$Int()" is evaluated, causing the class \xcd"A$Int" to
be loaded.  The class loader intercepts
this, demangles the name, and loads the bytecode for the
template class \xcd"A".

The bytecode is transformed, replacing the type property \xcd"T"
with the concrete type \xcd"int", the translation of \xcd"Int".

\begin{xten}
class A {
    x10.runtime.Type T;
}

class A$Int extends A {
    int x;
}
\end{xten}

Type properties are mapped to the Java primitive types and to
Object.  Only nine possible instantiations per parameter.
Instantiations used for representation.
Adapter objects used for run time type information.

Could do instantiation eagerly, but quickly gets out of hand without
whole-program analysis to limit the number of instantiations: 9
instantiations for one type property, 81 for two type
properties, 729 for three.

Value constraints are erased from type references.

Constructors are translated to static methods of their enclosing
class.
Constructor calls
are translated to calls to static methods.


Consider the code in Figure~\ref{fig:translation1}.  It contains most of the
features of generics that have to be translated.
\begin{figure*}[tp]
\begin{xten}
class C[T] {
    var x: T;
    def this[T](x: T) { this.x = x; }
    def set(x: T) { this.x = x; }
    def get(): T { return this.x; }
    def map[S](f: T => S): S { return f(this.x); }
    def d() { return new D[T](); }
    def t() { return new T(); }
    def isa(y: Object): boolean { return y instanceof T; }
}

val x : C = new C[String]();
val y : C[int] = new C[int]();
val z : C{T <: Array} = new C[Array[int]]();
x.map[int](f);
new C[int{self==3}]() instanceof C[int{self<4}];
\end{xten}
\caption{Code to translate}
\label{fig:translation1}
\end{figure*}

\subsection{Eliminating method type parameters}

\subsection{Translation to Java}

\subsection{Run-time instantiation}

We translate \xcd"instanceof" and cast operations to calls to
methods of a \xcd"Type" because the actual implementation of
the operation may require run-time constraint solving or other
complex code that cannot be easily substituted in when rewriting
the bytecode during instantiation.

\section{Discussion}
\label{sec:discussion}

\todo{Move some of this to Section~\ref{sec:related}}

\subsection{Type properties versus type parameters}


Type properties are similar, but not identical to type parameters.  The
differences may potentially confuse programmers used to Java generics or C++
templates.  The key difference is that type properties are instance members and
are thus accessible through access paths: \xcd"e.T" is a legal type.

Type properties, unlike type parameters, are inherited.
For example, in the following code, \xcd"T" is defined in \xcd"List"
and inherited into \xcd"Cons".  The property need not be
declared by the \xcd"Cons" class.
\begin{xten}
class List[T] { }
class Cons extends List {
    def head(): T = { ... }
    def tail(): List[T] = { ... }
}
\end{xten}
The analogous code for \xcd"Cons" using type parameters would be:
\begin{xten}
class Cons[T] extends List[T] {
    def head(): T = { ... }
    def tail(): List[T] = { ... }
}
\end{xten}
% This code is perfectly acceptable in X10 as well, but introduces a redundant
% type property \xcd"T" equal to the \xcd"T" inherited from \xcd"List".

We can make the type system behave as if type properties were
type parameters very simply.  We need only make the syntax \xcd"e.T"
illegal and permit type properties to be accessible only
from within the body of their class definition via the implicit \xcd"this"
qualifier.

\subsection{Wildcards}

Wildcards in Java~\cite{Java3,adding-wildcards} were motivated
by the following example (rewritten in X10 syntax)
from \cite{adding-wildcards}.
Sometimes a class needs a field or method
that is a list, but we don't care what the element type is.
For methods, one can give the method a type parameter:
\begin{xten}
def aMethod[T](list: List[T]) = { ... }
\end{xten}
This method can then be called on any \xcd"List" object.
However, there is no way to do this for fields since they
cannot be parametrized.
Java introduced wildcards to allow such fields to be
typed:
\begin{xten}
List<?> list;
\end{xten}
In X10, a similar effect is achieved by not constraining the
type property of \xcd"List".
One can write the following:
\begin{xten}
list: List;
\end{xten}
Similarly, the method can be written without type parameters by
not constraining \xcd"List":
\begin{xten}
def aMethod(list: List) = { ... }
\end{xten}

In X10, \xcd"List"
is a supertype of
\xcd"List[T]" for any \xcd"T",
just as in Java
\xcd"List<?>" is a supertype of
\xcd"List<T>" for any \xcd"T".
This follows directly from the definition of the type \xcd"List"
as \xcd"List{true}", and the type \xcd"List[T]"
as \xcd"List{X==T}", and the definition of subtyping.

Wildcards in Java can also be bounded.
We achieve the same
effect in X10 by using type constraints.
For instance, the following Java declarations:
\begin{xten}
void aMethod(List<? extends Number> list) { ... }
<T extends Number> void aParameterizedMethod(List<T> list) { ... }
\end{xten}
may be written as follows in X10:
\begin{xten}
def aMethod(list: List{T <: Number}) = { ... }
def aParameterizedMethod[T{self <: Number}](list: List[T]) = { ... }
\end{xten}

Wildcard bounds may be covariant, as in the following example:
\begin{xten}
List<? extends Number> list = new ArrayList<Integer>();
Number num = list.get(0);     // legal
list.set(0, new Double(0.0)); // illegal
list.set(0, list.get(1));     // illegal
\end{xten}
This can also be written in X10, but with an important
difference:
\begin{xten}
list: List{T <: Number} = new ArrayList[Integer]();
num: Number = list.get(0);    // legal
list.set(0, new Double(0.0)); // illegal
list.set(0, list.get(1));     // legal! (when list is final)
\end{xten}
Note because \xcd"list.get" has return type \xcd"list.T", the
last call in above is well-typed in X10; the analogous call in
Java is not well-typed.

Finally,
one can also specify lower bounds on types.  These are useful for
comparators:
\begin{xten}
class TreeSet[T] {
    def this[T](cmp: Comparator{T :> this.T}) { ... }
}
\end{xten}
Here, the comparator for any supertype of \xcd"T" can be used as
to compare \xcd"TreeSet" elements.

Another use of lower bounds is for list operations.
The \xcd"map" method below takes a function that maps a supertype
of the class parameter \xcd"T" to the method type parameter \xcd"S":
\begin{xten}
class List[T] {
    def map[S](fun: Object{self :> T} => S) : List[S] = { ... }
}
\end{xten}

\subsection{Proper abstraction}

Consider the following example adapted from \cite{adding-wildcards}:
\begin{xten}
def shuffle[T](list: List[T]) = {
    for (i: int in [0..list.size()-1]) {
        val xi: T = list(i);
        val j: int = Math.random(list.size());
        list(i) = list(j);
        list(j) = xi;
    }
}
\end{xten}
The method is parametrized on \xcd"T" because the method body needs
the element type to declare the variable \xcd"xi".

However, the method parameter can be omitted by using the type \xcd"list.T"
for \xcd"xi".  Thus, the method can be declared with the signature:
\begin{xten}
def shuffle(list: List) { ... }
\end{xten}
This is called \emph{proper abstraction}.

This example illustrates a key difference between type properties
and type parameters:
A type property is a member of its class, whereas a type parameter is
not.  The names of type properties are visible outside the body of
their class declaration.

\if 0
Type properties can be used as the basis of a parametrized type
system.  This is done simply by making type properties private.
Using the syntactic sugar described above,
the resulting system behaves identically to a system with type
parameters.
\fi

In Java,
Wildcard
capture allows the parametrized method to be called with any \xcd"List",
regardless of its parameter type.
However,
the method parameter cannot be omitted: declaring a parameterless version
of shuffle requires delegating to a private parametrized version that
``opens up'' the parameter.

\section{Extensions}

\paragraph{Self types.}
Type properties can also be used to support a form of self
types~\cite{bruce-binary,bsg95}. 
%
Self types can be implemented by introducing a
type property \Xcd{class} to the root of the class hierarchy, \Xcd{Object}:
\begin{xtenmath}
class Object[class] { $\dots$ }
\end{xtenmath}
Scala's path-dependent types~\cite{scala} and J\&'s
dependent classes~\cite{nqm06}
take a similar approach.

Self types are achieved by
implicitly constraining types so that if an path expression \Xcd{p}
has type \Xcd{C}, then
$\Xcd{p}.\Xcd{class} \subtype \Xcd{C}$.  In particular,
$\Xcd{this}.\Xcd{class}$ is guaranteed to be a subtype
of the lexically enclosing class; the type
$\Xcd{this}.\Xcd{class}$ represents all instances of the fixed,
but statically unknown, run-time class referred to by the \Xcd{this}
parameter.

\paragraph{Virtual types.}

Type properties share many similarities with virtual
types~\cite{mp89-virtual-classes,beta,ernst99-gbeta,ernst06-virtual,cdnw07-tribe}
and similar constructs built on path-dependent types found in
languages such as Scala~\cite{scala}, and J\&~\cite{nqm06}.
%
Constrained types are more expressive than virtual
types since they can be constrained at the use-site,
can be refined on a per-object basis without explicit subclassing,
and can be refined contravariantly
as well as covariantly. 

\paragraph{Structural constraints.}

\eat{


\paragraph{Ownership types.}

Consider the following example of generic ownership
derived from Potanin et al.~\cite{ogj-oopsla06}.

\begin{xten}
class Object(owner: Object) { }

// Map inherits Object.owner
// No need to add explicit vOwner and kOwner properties for Key, Value
class Map[Key, Value]{Key <: Comparable, Value <: Object}
{
    private nodes: Vector[Node[Key, Value](this)](this);

    public def put(key: Key, value: Value): Void = {
        nodes.add(new Node[Key, Value](key, value, this)());
    }

    public def get(key: Key): Value = {
        for (mn: Node[Key, Value](this) in nodes) {
            if (mn.key.equals(key))
                return mn.value;
        }
        return null;
    }

    // OGJ will prevent this from being called, since caller
    // can only assign the result to a supertype of Vector(this),
    // which would be only Vector(this) or Object(this)
    // BUT: we have Vector :> Vector(this)
    // Need to require that all class types have an equality constraint
    // on the owner property
    public def exposeVector(): Vector(this) { return nodes; }
}

class Node[Key, Value]
    {Key <: Comparable, Value <: Object}
{
    val key: Key;
    val value: Value;

    public def this[K, V](k: Key, v: Value, o: Object): Node[K, V](o) {
        super(o);               // set the owner
        property[K, V];         // set the type properties
        this.key = k;
        this.value = v;
    }
}
\end{xten}

Restrictions:
\begin{itemize}
\item owner property must be constrained (define this!)
\item owner is always equal to or inside the owner of all other type properties
\item types with an actual owner == this, can only be accessed via this
\end{itemize}

}


\section{Related work}
\label{sec:related}

Constraint-based type systems, dependent types, and generic types
have been well-studied in the literature.

\paragraph{Constraint-based type systems.}

The use of constraints for type inference and subtyping has a history
going back to Mitchell~\cite{mitchell84} and by
Reynolds~\cite{reynolds85}.  These and subsequent systems are based on
constraints over types, but not over values.  Trifonov and
Smith~\cite{trifonov96} proposed a type system in which types are
refined using subtyping constraints.
Pottier~\cite{pottier96simplifying} presents a constraint-based type
system for an ML-like language with subtyping.  These developments
lead to \hmx~\cite{sulzmann97type}, a constraint-based framework for
Hindley--Milner-style type systems.  The framework is parametrized on
the specific constraint system $X$; instantiating $X$ yields
extensions of the HM type system.  Constraints in \hmx{} are over
types, not values. The \hmx{} approach is an important precursor to
our constrained types approach. The principal difference is that
\hmx{} applies to functional languages and does not integrate
dependent types.

%
Sulzmann and Stuckey~\cite{sulzmann-hmx-clpx} showed that the
type inference algorithm for \hmx can be encoded as a
constraint logic program parametrized by the constraint system
$X$. This is very much in spirit with our approach.
Constrained types open the door to {\em user-defined}
predicates and functions, effectively permitting the user to enrich
$\cal C$ (hence the power of the compile-time type-checker) by
developing application-specific constraints using a constraint
programming language such as CLP($\cal C$) \cite{clp} or the richer
RCC($\cal C$) \cite{DBLP:conf/fsttcs/JagadeesanNS05}.

\paragraph{Dependent types.}

Dependent type
systems~\cite{xi99dependent,calc-constructions,epigram,cayenne}
parametrize types on values.  Refinement type
systems~\cite{refinement-types,conditional-types,jones94,sized-types,flanagan-popl06,flanagan-fool06,liquid-types},
introduced by Freeman and Pfenning~\cite{refinement-types}, are dependent type
systems that extend a base type system through constraints on values.  These
systems do not treat value and type constraints uniformly.

Our work is closely related to DML, \cite{xi99dependent}, an
extension of ML with dependent types. DML is also built
parametrically on a constraint solver. Types are refinement types;
they do not affect the operational semantics and erasing the
constraints yields a legal DML program.  This differs from generic constrained
types, where erasure of subtyping constraints can prevent the program from
type-checking.
DML does not permit any run-time checking of constraints
(dynamic casts).

The most obvious distinction between DML and constrained types
lies in the target
domain: DML is designed for functional programming
whereas constrained types are designed for imperative, concurrent
object-oriented languages. 
But there are several other
crucial differences as well.

DML achieves its separation between compile-time and run-time processing
by not permitting program
variables to be used in types. Instead, a parallel set of (universally
or existentially quantified) ``index'' variables are
introduced.
%
Second, DML permits only variables of basic index sorts known to
the constraint solver (e.g., \Xcd{bool}, \Xcd{int}, \Xcd{nat}) to
occur in types. In contrast, constrained types permit program
variables at any type to occur in constrained types. As with DML
only operations specified by the constraint system are permitted in
types. However, these operations always include field selection and
equality on object references.  Note that DML-style constraints are easily
encoded in constrained types.

% {\em Conditional
% types}~\cite{conditional-types} extend refinement types to
% encode control-flow information in the types.
% %
% Jones introduced {\em qualified types}, which permit
% types to be constrained by a finite set of
% predicates~\cite{jones94}.
% %
% {\em Sized types}~\cite{sized-types}
% annotate types with the sizes of recursive data structures.
% Sizes are linear functions of size variables.
% Size inference is performed using a constraint solver for
% Presburger arithmetic~\cite{omega}.
% % constraints on types, support primitive recursion only

% Index objects must be pure.
% Singleton types int(n).
% ML$^{\Pi}_0$:
% Refinement of the ML type system: does not affect the
% operational semantics.  Can erase to ML$_0$.

% Jay and Sekanina 1996: array bounds checking based on shape
% types.

% Ada dependent types.
% Ada has constrained array definitions.  A constraint
% \cite{ada-ref-man}.  Not clear if they're dependent.  Are
% there other dependent types?  Generics are dependent?

        % Used for array bounds by Morrisett et al (I think--need
        % to find paper)

% Singleton types~\cite{aspinall-singletons}.

Logically qualified types, or liquid types~\cite{liquid-types},
permit types in a base Hindley--Milner-style type system to be refined with
conjunctions of logical qualifiers.  The subtyping relation is similar to
\Xten{}'s, that is, two liquid types are in the subtyping relation if their base
types are and if one type's qualifier implies the other's.
The Hindley--Milner type
inference algorithm is used to infer base types; these types are used as templates for inference of the liquid types.
The types of certain expressions are over-approximated to ensure inference
is decidable.
To improve precision of the inference algorithm, and hence
to reduce the annotation burden on the programmer, 
the type system is path sensitive.

Hybrid type-checking~\cite{flanagan-popl06,flanagan-fool06}
introduced another refinement type system.
While typing is undecidable, dynamic checks are inserted into
the program when necessary if the type-checker (which
includes a constraint solver) cannot determine
type safety statically.
In \FXG{}, dynamic type checks, including tests of dependent
constraints, are inserted only at explicit casts or
\Xcd{instanceof} expressions; constraint solving is performed at compile time.

% Where clauses for F-bounded polymorphism~\cite{where-clauses}
% Bounded quantification: Cardelli and Wegner.  Bound T with T'
% In F-bounded polymorphism~\cite{f-bounds}, type variables are bounded by a function of 
% the type variable. 
% Not dependent types.

Concoqtion~\cite{concoqtion} extends types in OCaml~\cite{ocaml}
with constraints written as Coq~\cite{coq} rules.
While the types are expressive, supporting the full generality
of the Coq language, proofs must be
provided to satisfy the type checker.
\Xten{} supports only constraints that can be checked by a
constraint solver during compilation.
Concoqtion encodes OCaml types and value to allow reasoning in
the Coq formulae; however, there is an impedance mismatch
caused by the differing syntax, representation, and behavior
of OCaml versus Coq.

\eat{
Cayenne~\cite{cayenne} is a Haskell-like language with fully dependent types.
There is no distinction between static and dynamic types.
Type-checking is undecidable.
There is no notion of datatype refinement as in DML.

Epigram~\cite{epigram,epigram-matter}
is a dependently typed functional programming language based on
a type theory with inductive families.
Epigram does not have a phase distinction between values and
types.
}

\eat{
$\lambda^{\sf Con}$ is a lambda calculus with assertions.
Findler, Felleisen, Contracts for higher-order functions (ICFP02)

  example: int[> 9]

contracts are either simple predicates or function contracts.
defined by (define/contract ...)

enforced at run-time.
}

% Jif~\cite{jif,jflow} is an extension of Java in which
% types are labeled with security policies enforced by the
% compiler.

\eat{
ESC/Java~\cite{esc-java}
allow programmers to write object invariants and pre- and
post-conditions that are enforced statically
by the compiler using an automated theorem prover.
Static checking is undecidable and, in the presence of loops,
is unsound (but still useful) unless the programmer supplies loop invariants.
ESC/Java can enforce invariants on mutable state.
}

% and Spec$\sharp$~\cite{specsharp}

\eat{
Pluggable and optional type systems were proposed by
Bracha~\cite{bracha04-pluggable} and provide another means of
specifying refinement types.
Type annotations, implemented in compiler plugins, serve only to
reject programs statically that might otherwise have dynamic
type errors.
CQual~\cite{foster-popl02} extends C with user-defined type
qualifiers.  These
qualifiers may be flow-sensitive and may be inferred. 
CQual supports only a fixed set of typing rules
for all qualifiers.
In contrast, the {\em semantic type qualifiers} of
Chin, Markstrum, and Millstein~\cite{chin05-qualifiers}
allow programmers to define typing rules for qualifiers
in a meta language that allows type-checking rules to be
specified declaratively.
JavaCOP~\cite{javacop-oopsla06} is a pluggable type system
framework for Java.  Annotations are defined in a meta language
that allows type-checking rules to be specified declaratively.
JSR 308~\cite{jsr308} is a proposal for adding user-defined type qualifiers
to Java.
}

% Holt, Cordy, the Turing programming language

% Ou, Tan, Mandelbaum, Walker, Dynamic typing with dependent types
% Separate dependent and simple parts of the program.
% Statically type the dependent parts.
% Dynamic checks when passing values into dependent part.

\paragraph{Genericity.}

Genericity in object-oriented languages is usually
supported through
type parametrization.

A number of proposals 
for adding genericity to Java quickly followed
the initial release of
the language~\cite{GJ,Pizza,java-popl97,thorup97,allen03}.
GJ~\cite{GJ} implements invariant type
parameters via type erasure.
PolyJ~\cite{java-popl97} supports run-time representation of types
via adapter objects, and also permits instantiation of
parameters on primitive types and structural parameter bounds.
Viroli and Natali~\cite{reflective-generics,type-passing-generics}
also support
a run-time representation of types, using Java's reflection API.
NextGen~\cite{nextgen,allen03} was implemented using run-time 
instantiation.
\Xten{}'s generics have a hybrid implementation, adopting PolyJ's
adapter object approach for dependent types and for 
type introspection and using NextGen's run-time
instantiation approach for greater efficiency.
% MixGen~\cite{allen04} extends NextGen with mixins.

\csharp also supports generic types via run-time instantiation in the
CLR~\cite{csharp-generics}.  Type parameters may be declared
with definition-site variance tags.
Generalized type constraints were proposed for
\csharp~\cite{emir06}.  Methods can be annotated with subtyping
constraints that must be satisfied to invoke the method.
Generic \Xten{} supports these constraints, as well as constraints
on values, with method and constructor where clauses.

\eat{
\FXG{} does not support \emph{bivariance}~\cite{variant-parametric-types}; a
class \xcd"C" is bivariant in a type property \xcd"X" if \xcd"C{self.X==S}" is
a subtype of \xcd"C{self.X==T}" for any \xcd"S" and \xcd"T".  Bivariance is
useful for writing code in which the property \xcd"X" is ignored.  One can
achieve  this effect in \FXG{} simply by leaving \xcd"X" unconstrained.
}

\eat{
Parametric types with use-site variance are related to existential types:
\xcd"C<+T>" corresponds to the bounded existential $\exists\tcd{X<:T}.C<X>$;
\xcd"C<-T>" corresponds to the bounded existential $\exists\tcd{X:>T}.C<X>$;
\xcd"C<*>" corresponds to the unbounded existential $\exists\tcd{X}.C<X>$.
\FXG{} has a similar correspondence:
\xcd"C{X<:T}" corresponds to the bounded existential \xcdmath"C\{\exists\tcd{self}:C.self.X<:T\}";
\xcd"C{X:>T}" corresponds to the bounded existential \xcdmath"C\{\exists\tcd{self}.C<X\}";
\xcd"C" corresponds to the unbounded existential \xcdmath"C\{\exists\tcd{self}.C<X\}".
}


                        

\section{Conclusions}
\label{sec:conclusions}

We have presented a preliminary design for supporting genericity
in X10 using type properties.  This type system generalizes the
existing X10 type system.  The use of constraints on type
properties allows
the design to capture many features of generics in languages
like Java 5 and C\# and then to extend these features with new
more expressive power.
We expect that the design admits an efficient
implementation and intend to implement the design shortly.

\section*{Acknowledgments} 

The authors thank Bob Blainey, 
Doug Lea, Jens Palsberg, Lex Spoon, and Olivier Tardieu
for valuable feedback on versions of the language.
We thank 
Andrew Myers and
Michael Clarkson for providing us with their implementation of
PolyJ, on which our implementation was based, and for many
discussions over the years about parametrized types in Java.

\bibliographystyle{plain}
\bibliography{master}

% \appendix
% \onecolumn

% \section{An extended example}
% {\footnotesize
\begin{verbatim}
/**
   A distributed binary tree.
   @author Satish Chandra 4/6/2006
   @author vj
 */
//                             ____P0
//                            |     |
//                            |     |
//                          _P2  __P0
//                         |  | |   |
//                         |  | |   |
//                        P3 P2 P1 P0
//                         *  *  *  *
// Right child is always on the same place as its parent;
// left child is at a different place at the top few levels of the tree,
// but at the same place as its parent at the lower levels.

class Tree(localLeft: boolean,
           left: nullable Tree(& localLeft => loc=here),
           right: nullable Tree(& loc=here),
           next: nullable Tree) extends Object {
    def postOrder:Tree = {
        val result:Tree = this;
        if (right != null) {
            val result:Tree = right.postOrder();
            right.next = this;
            if (left != null) return left.postOrder(tt);
        } else if (left != null) return left.postOrder(tt);
        this
    }
    def postOrder(rest: Tree):Tree = {
        this.next = rest;
        postOrder
    }
    def sum:int = size + (right==null => 0 : right.sum()) + (left==null => 0 : left.sum)
}
value TreeMaker {
    // Create a binary tree on span places.
    def build(count:int, span:int): nullable Tree(& localLeft==(span/2==0)) = {
        if (count == 0) return null;
        {val ll:boolean = (span/2==0);
         new Tree(ll,  eval(ll => here : place.places(here.id+span/2)){build(count/2, span/2)},
           build(count/2, span/2),count)}
    }
}
\end{verbatim}}

\subsection{Places}
{\footnotesize
\begin{verbatim}
/**

 * This class implements the notion of places in X10. The maximum
 * number of places is determined by a configuration parameter
 * (MAX_PLACES). Each place is indexed by a nat, from 0 to MAX_PLACES;
 * thus there are MAX_PLACES+1 places. This ensures that there is
 * always at least 1 place, the 0'th place.

 * We use a dependent parameter to ensure that the compiler can track
 * indices for places.
 *
 * Note that place(i), for i <= MAX_PLACES, can now be used as a non-empty type.
 * Thus it is possible to run an async at another place, without using arays---
 * just use async(place(i)) {...} for an appropriate i.

 * @author Christoph von Praun
 * @author vj
 */

package x10.lang;

import x10.util.List;
import x10.util.Set;

public value class place (nat i : i <= MAX_PLACES){

    /** The number of places in this run of the system. Set on
     * initialization, through the command line/init parameters file.
     */
    config nat MAX_PLACES;

    // Create this array at the very beginning.
    private constant place value [] myPlaces = new place[MAX_PLACES+1] fun place (int i) {
	return new place( i )(); };

    /** The last place in this program execution.
     */
    public static final place LAST_PLACE = myPlaces[MAX_PLACES];

    /** The first place in this program execution.
     */
    public static final place FIRST_PLACE = myPlaces[0];
    public static final Set<place> places = makeSet( MAX_PLACES );

    /** Returns the set of places from first place to last place.
     */
    public static Set<place> makeSet( nat lastPlace ) {
	Set<place> result = new Set<place>();
	for ( int i : 0 .. lastPlace ) {
	    result.add( myPlaces[i] );
	}
	return result;
    }

    /**  Return the current place for this activity.
     */
    public static place here() {
	return activity.currentActivity().place();
    }

    /** Returns the next place, using modular arithmetic. Thus the
     * next place for the last place is the first place.
     */
    public place(i+1 % MAX_PLACES) next()  { return next( 1 ); }

    /** Returns the previous place, using modular arithmetic. Thus the
     * previous place for the first place is the last place.
     */
    public place(i-1 % MAX_PLACES) prev()  { return next( -1 ); }

    /** Returns the k'th next place, using modular arithmetic. k may
     * be negative.
     */
    public place(i+k % MAX_PLACES) next( int k ) {
	return places[ (i + k) % MAX_PLACES];
    }

    /**  Is this the first place?
     */
    public boolean isFirst() { return i==0; }

    /** Is this the last place?
     */
    public boolean isLast() { return i==MAX_PLACES; }
}
\end{verbatim}}
\subsection{$k$-dimensional regions}
{\footnotesize
\begin{verbatim}
package x10.lang;

/** A region represents a k-dimensional space of points. A region is a
 * dependent class, with the value parameter specifying the dimension
 * of the region.
 * @author vj
 * @date 12/24/2004
 */

public final value class region( int dimension : dimension >= 0 )  {

    /** Construct a 1-dimensional region, if low <= high. Otherwise
     * through a MalformedRegionException.
     */
    extern public region (: dimension==1) (int low, int high)
        throws MalformedRegionException;

    /** Construct a region, using the list of region(1)'s passed as
     * arguments to the constructor.
     */
    extern public region( List(dimension)<region(1)> regions );

    /** Throws IndexOutOfBoundException if i > dimension. Returns the
        region(1) associated with the i'th dimension of this otherwise.
     */
    extern public region(1) dimension( int i )
        throws IndexOutOfBoundException;


    /** Returns true iff the region contains every point between two
     * points in the region.
     */
    extern public boolean isConvex();

    /** Return the low bound for a 1-dimensional region.
     */
    extern public (:dimension=1) int low();

    /** Return the high bound for a 1-dimensional region.
     */
    extern public (:dimension=1) int high();

    /** Return the next element for a 1-dimensional region, if any.
     */
    extern public (:dimension=1) int next( int current )
        throws IndexOutOfBoundException;

    extern public region(dimension) union( region(dimension) r);
    extern public region(dimension) intersection( region(dimension) r);
    extern public region(dimension) difference( region(dimension) r);
    extern public region(dimension) convexHull();

    /**
       Returns true iff this is a superset of r.
     */
    extern public boolean contains( region(dimension) r);
    /**
       Returns true iff this is disjoint from r.
     */
    extern public boolean disjoint( region(dimension) r);

    /** Returns true iff the set of points in r and this are equal.
     */
    public boolean equal( region(dimension) r) {
        return this.contains(r) && r.contains(this);
    }

    // Static methods follow.

    public static region(2) upperTriangular(int size) {
        return upperTriangular(2)( size );
    }
    public static region(2) lowerTriangular(int size) {
        return lowerTriangular(2)( size );
    }
    public static region(2) banded(int size, int width) {
        return banded(2)( size );
    }

    /** Return an \code{upperTriangular} region for a dim-dimensional
     * space of size \code{size} in each dimension.
     */
    extern public static (int dim) region(dim) upperTriangular(int size);

    /** Return a lowerTriangular region for a dim-dimensional space of
     * size \code{size} in each dimension.
     */
    extern public static (int dim) region(dim) lowerTriangular(int size);

    /** Return a banded region of width {\code width} for a
     * dim-dimensional space of size {\code size} in each dimension.
     */
    extern public static (int dim) region(dim) banded(int size, int width);


}

\end{verbatim}}

\subsection{Point}
{\footnotesize
\begin{verbatim}
package x10.lang;

public final class point( region region ) {
    parameter int dimension = region.dimension;
    // an array of the given size.
    int[dimension] val;

    /** Create a point with the given values in each dimension.
     */
    public point( int[dimension] val ) {
        this.val = val;
    }

    /** Return the value of this point on the i'th dimension.
     */
    public int valAt( int i) throws IndexOutOfBoundException {
        if (i < 1 || i > dimension) throw new IndexOutOfBoundException();
        return val[i];
    }

    /** Return the next point in the given region on this given
     * dimension, if any.
     */
    public void inc( int i )
        throws IndexOutOfBoundException, MalformedRegionException {
        int val = valAt(i);
        val[i] = region.dimension(i).next( val );
    }

    /** Return true iff the point is on the upper boundary of the i'th
     * dimension.
     */
    public boolean onUpperBoundary(int i)
        throws IndexOutOfBoundException {
        int val = valAt(i);
        return val == region.dimension(i).high();
    }

    /** Return true iff the point is on the lower boundary of the i'th
     * dimension.
     */
    public boolean onLowerBoundary(int i)
        throws IndexOutOfBoundException {
        int val = valAt(i);
        return val == region.dimension(i).low();
    }
}
\end{verbatim}}

\subsection{Distribution}
{\footnotesize
\begin{verbatim}
package x10.lang;

/** A distribution is a mapping from a given region to a set of
 * places. It takes as parameter the region over which the mapping is
 * defined. The dimensionality of the distribution is the same as the
 * dimensionality of the underlying region.

   @author vj
   @date 12/24/2004
 */

public final value class distribution( region region ) {
    /** The parameter dimension may be used in constructing types derived
     * from the class distribution. For instance,
     * distribution(dimension=k) is the type of all k-dimensional
     * distributions.
     */
    parameter int dimension = region.dimension;

    /** places is the range of the distribution. Guranteed that if a
     * place P is in this set then for some point p in region,
     * this.valueAt(p)==P.
     */
    public final Set<place> places; // consider making this a parameter?

    /** Returns the place to which the point p in region is mapped.
     */
    extern public place valueAt(point(region) p);

    /** Returns the region mapped by this distribution to the place P.
        The value returned is a subset of this.region.
     */
    extern public region(dimension) restriction( place P );

    /** Returns the distribution obtained by range-restricting this to Ps.
        The region of the distribution returned is contained in this.region.
     */
    extern public distribution(:this.region.contains(region))
        restriction( Set<place> Ps );

    /** Returns a new distribution obtained by restricting this to the
     * domain region.intersection(R), where parameter R is a region
     * with the same dimension.
     */
    extern public (region(dimension) R) distribution(region.intersection(R))
        restriction();

    /** Returns the restriction of this to the domain region.difference(R),
        where parameter R is a region with the same dimension.
     */
    extern public (region(dimension) R) distribution(region.difference(R))
        difference();

    /** Takes as parameter a distribution D defined over a region
        disjoint from this. Returns a distribution defined over a
        region which is the union of this.region and D.region.
        This distribution must assume the value of D over D.region
        and this over this.region.

        @seealso distribution.asymmetricUnion.
     */
    extern public (distribution(:region.disjoint(this.region) &&
                                dimension=this.dimension) D)
        distribution(region.union(D.region)) union();

    /** Returns a distribution defined on region.union(R): it takes on
        this.valueAt(p) for all points p in region, and D.valueAt(p) for all
        points in R.difference(region).
     */
    extern public (region(dimension) R) distribution(region.union(R))
        asymmetricUnion( distribution(R) D);

    /** Return a distribution on region.setMinus(R) which takes on the
     * same value at each point in its domain as this. R is passed as
     * a parameter; this allows the type of the return value to be
     * parametric in R.
     */
    extern public (region(dimension) R) distribution(region.setMinus(R))
        setMinus();

    /** Return true iff the given distribution D, which must be over a
     * region of the same dimension as this, is defined over a subset
     * of this.region and agrees with it at each point.
     */
    extern public (region(dimension) r)
        boolean subDistribution( distribution(r) D);

    /** Returns true iff this and d map each point in their common
     * domain to the same place.
     */
    public boolean equal( distribution( region ) d ) {
        return this.subDistribution(region)(d)
            && d.subDistribution(region)(this);
    }

    /** Returns the unique 1-dimensional distribution U over the region 1..k,
     * (where k is the cardinality of Q) which maps the point [i] to the
     * i'th element in Q in canonical place-order.
     */
    extern public static distribution(:dimension=1) unique( Set<place> Q );

    /** Returns the constant distribution which maps every point in its
        region to the given place P.
    */
    extern public static (region R) distribution(R) constant( place P );

    /** Returns the block distribution over the given region, and over
     * place.MAX_PLACES places.
     */
    public static (region R) distribution(R) block() {
        return this.block(R)(place.places);
    }

    /** Returns the block distribution over the given region and the
     * given set of places. Chunks of the region are distributed over
     * s, in canonical order.
     */
    extern public static (region R) distribution(R) block( Set<place> s);


    /** Returns the cyclic distribution over the given region, and over
     * all places.
     */
    public static (region R) distribution(R) cyclic() {
        return this.cyclic(R)(place.places);
    }

    extern public static (region R) distribution(R) cyclic( Set<place> s);

    /** Returns the block-cyclic distribution over the given region, and over
     * place.MAX_PLACES places. Exception thrown if blockSize < 1.
     */
    extern public static (region R)
        distribution(R) blockCyclic( int blockSize)
        throws MalformedRegionException;

    /** Returns a distribution which assigns a random place in the
     * given set of places to each point in the region.
     */
    extern public static (region R) distribution(R) random();

    /** Returns a distribution which assigns some arbitrary place in
     * the given set of places to each point in the region. There are
     * no guarantees on this assignment, e.g. all points may be
     * assigned to the same place.
     */
    extern public static (region R) distribution(R) arbitrary();

}
\end{verbatim}}

\subsection{Arrays}
Finally we can now define arrays. An array is built over a
distribution and a base type.

{\footnotesize
\begin{verbatim}
package x10.lang;

/** The class of all  multidimensional, distributed arrays in X10.

    <p> I dont yet know how to handle B@current base type for the
    array.

 * @author vj 12/24/2004
 */

public final value class array ( distribution dist )<B@P> {
    parameter int dimension = dist.dimension;
    parameter region(dimension) region = dist.region;

    /** Return an array initialized with the given function which
        maps each point in region to a value in B.
     */
    extern public array( Fun<point(region),B@P> init);

    /** Return the value of the array at the given point in the
     * region.
     */
    extern public B@P valueAt(point(region) p);

    /** Return the value obtained by reducing the given array with the
        function fun, which is assumed to be associative and
        commutative. unit should satisfy fun(unit,x)=x=fun(x,unit).
     */
    extern public B reduce(Fun<B@?,Fun<B@?,B@?>> fun, B@? unit);


    /** Return an array of B with the same distribution as this, by
        scanning this with the function fun, and unit unit.
     */
    extern public array(dist)<B> scan(Fun<B@?,Fun<B@?,B@?>> fun, B@? unit);

    /** Return an array of B@P defined on the intersection of the
        region underlying the array and the parameter region R.
     */
    extern public (region(dimension) R)
        array(dist.restriction(R)())<B@P>  restriction();

    /** Return an array of B@P defined on the intersection of
        the region underlying this and the parametric distribution.
     */
    public  (distribution(:dimension=this.dimension) D)
        array(dist.restriction(D.region)())<B@P> restriction();

    /** Take as parameter a distribution D of the same dimension as *
     * this, and defined over a disjoint region. Take as argument an *
     * array other over D. Return an array whose distribution is the
     * union of this and D and which takes on the value
     * this.atValue(p) for p in this.region and other.atValue(p) for p
     * in other.region.
     */
    extern public (distribution(:region.disjoint(this.region) &&
                                dimension=this.dimension) D)
        array(dist.union(D))<B@P> compose( array(D)<B@P> other);

    /** Return the array obtained by overlaying this array on top of
        other. The method takes as parameter a distribution D over the
        same dimension. It returns an array over the distribution
        dist.asymmetricUnion(D).
     */
    extern public (distribution(:dimension=this.dimension) D)
        array(dist.asymmetricUnion(D))<B@P> overlay( array(D)<B@P> other);

    extern public array<B> overlay(array<B> other);

    /** Assume given an array a over distribution dist, but with
     * basetype C@P. Assume given a function f: B@P -> C@P -> D@P.
     * Return an array with distribution dist over the type D@P
     * containing fun(this.atValue(p),a.atValue(p)) for each p in
     * dist.region.
     */
    extern public <C@P, D>
        array(dist)<D@P> lift(Fun<B@P, Fun<C@P, D@P>> fun, array(dist)<C@P> a);

    /**  Return an array of B with distribution d initialized
         with the value b at every point in d.
     */
    extern public static (distribution D) <B@P> array(D)<B@P> constant(B@? b);

}
\end{verbatim}}


\begin{example}
 The code for {\tt List} translates as given in Table~\ref{List-translation}.
\end{example}

\begin{figure*}
{\footnotesize
\begin{verbatim}
  public value class List <Node> {
    public final nat n;   // is a parameter
    nullable Node node = null;
    nullable List<Node> rest = null;  // All assignments must check n = this.n-1.

    /** Returns the empty list. Defined only when the parameter n
        has the value 0. Invocation: new List(0)<Node>().
     */
    public List ( final nat n ) {
      assume n==0;
      this.n = n;
    }

    /** Returns a list of length 1 containing the given node.
        Invocation: new List(1)<Node>( node ).
     */
    public List ( final nat n, Node node ) {
      assume n==1;                         // From the constructor precondition.
      assert 0==0 : "DependentTypeError"; // For the constructor call.
      assert n>=1 : "DependentTypeError"; // For the this call.
      this(n, node, new List<Node>(0));
    }

    public List ( final nat n, Node node, List<Node> rest ) {
      assume n>=1;                               // From the constructor precondition
      assume rest.n==n-1 : "DependentTypeError"; // From the argument type.
      this.n = n;
      this.node = node;
      assert rest.n==n-1 : "DependentTypeError"; // For the field assignment.
      this.rest = rest;
    }

    public  List<Node> append( List<Node> arg ) {
      if (n == 0) {
          final List<Node> result = arg;
          assert n+arg.n == result.n : "DependentTypeError"; // For the return value
          return result;
      } else {
          assume rest.n == n-1;
          final List<Node> argval = rest.append(arg);
          assume argval.n == rest.n+arg.n;
          assert n+arg.n-1== argval.n : "DependentTypeError"; // For the constructor call.
          final List<Node> result = new List<Node>(n+arg.n, node, argval);
          assume result.n == n+arg.n;
          assert n+arg.n == result.n : "DependentTypeError"; // For the return value
          return result;
      }
    }

\end{verbatim}}
\caption{Translation of {\tt List} (contd in Table~\ref{List-translation-2}).}\label{List-translation}
\end{figure*}
\begin{figure*}
{\footnotesize
\begin{verbatim}
    public  List<Node> rev() {
      final List<Node> arg = new List<Node>(0);
      assume arg.n = 0;                           // From the constructor call.
      final List<Node> result = rev( arg );
      assume result.n == n+arg.n;                  // From the method signature
      assert n == result.n : "DependentTypeError"; // For the return value.
      return result;
    }

    public  List(n+arg.n)<Node> rev( final List<Node> arg) {
      if (n==0) {
         assert n+arg.n == arg.n : "DependentTypeError"; // For the return value.
         return arg;
      } else {
        assert 1+arg.n-1=arg.n : "DependentTypeError"; // For the argument to the constructor
        final List<Node> arg2 = new List<Node>(1+arg.n,node, arg));
        assume arg2.n==1+arg.n;                      // From the constructor invocation
        final List<Node> restval = rest;             // Read from a mutable field of parametric type
        assume restval.n == n-1;                     // From the field read.
        final List(restval.n+arg2.n)<Node> result = restval.rev( arg2 );
        assume result.n=restval.n+arg2.n
        assert n+arg.n == result.n                   // For the return value
        return result;
    }

    /** Return a list of compile-time unknown length, obtained by filtering
        this with f. */
    public List<Node> filter(fun<Node, boolean> f) {
         if (n==0) return this;
         if (f(node)) {
           final List<Node> l = rest.filter(f);
           assert l.n+1-1==l.n : "DependentTypeError"; // For the constructor call
           return new List<Node>(l.n+1,node, l);
         } else {
           return rest.filter(f);
         }
    }

    /** Return a list of m numbers from o..m-1. */
    public static  List<nat> gen( final nat m ) {
         assert 0 <= m : "DependentTypeError";        // Precondition for method call.
         final List<nat> result = gen(0,m);
         assume result.n=m-0 : "DependentTypeError";  // From the method signature
         assert m == result.n : "DependentTypeError"; // For the return value
         return result;
    }

    /** Return a list of (m-i) elements, from i to m-1. */
    public static List<nat> gen(final nat i, final nat m) {
      assume i <= m;                                   // Method precondition.
      if (i==m) {
        assert m-i == 0 : "DependentTypeError";        // For the constructor call
        final List result = new List<nat>(m-i);
        assume result.n == 0;                          // From the constructor call.
        assert m-i == result.n : "DependentTypeError"; // For the return value.
        return result;
      } else {
        assert i+1 <= m : "DependentTypeError";        // For the method call.
        final List<nat> arg = gen(i+1,m);
        assume arg.n = m-(i+1);                        // From the method call.
        assert m-i-1 = arg.n;                          // For the constructor invocation.
        final List result = new List<nat>(m-i, i, arg);
        assume result.n = m-i;                         // From the constructor invocation.
        assert m-i == result.n : "DependentTypeError"; // For the return value
        return result;
    }
  }
\end{verbatim}}
\caption{Translation of {\tt List} (continued).}\label{List-translation-2}
\end{figure*}

\section{Type-checking dependent classes}

Each programming language---such as \Xten{}---will specify the base
underlying classes (and the operations on them) which can occur as
types in parameter lists. For instance, in the code for {\tt List}
above, the only type that appears in parameter lists is {\tt int}, and
the only operations on {\tt int} are addition, subtraction, {\tt >=},
{\tt ==}, and the only constants are {\tt 0} and {\tt 1}.  (This
language falls within Presburger arithmetic, a decidable fragment of
arithmetic.)  The compiler must come equipped with a constraint solver
(decision procedure) that can answer questions of the form: does one
constraint entail another?  Constraints are atomic formulas built up
from these operations, using variables. For instance, the compiler
must answer each one of:
{\footnotesize
\begin{verbatim}
  n >= 2 |- n-1 >= 0
  n >= 0, m >= 0 |- m+n >= 0
\end{verbatim}}

Ultimately, the only variables that will occur in constraints are
those that correspond to {\tt config} parameters and those that are
defined by implicit parameter definitions. We need to establish that
the verification of any class will generate only a finite number of
constraints, hence only a finite constraint problem for the constraint
solver.

Second, it should be possible for instances of user-defined classes
(and operations on them) to occur as type parameters. For the compiler
to check conditions involving such values, it is necessary that the
underlying constraint solver be extended.

There are two general ways in which the constraint solver may be
extended.  Both require that the programmer single out some classes
and methods on those classes as {\em pure}. (We shall think of
constants as corresponding to zero-ary methods.) Only instances of
pure classes and expressions involving pure methods on these instances
are allowed in parameter expressions.

How shall constraints be generated for such pure methods? First, the
programmer may explicitly supply with each pure method {\tt T m(T1 x1,
..., Tn xn)} a constraint on {\tt n+2} variables in the constraint
system of the underlying solver that is entailed by {\tt y =
o.m(x1,..., xn)}. Whenever the compiler has to perform reasoning on an
expression involving this method invocation, it uses the constraint
supplied by the programmer. A second more ambitious possibility is
that a symbolic evaluator of the language may be run on the body of
the method to automatically generate the corresponding constraint.

Finally an additional possibility is that the constraint solver itself
be made extensible. In this case, when a user writes a class which is
intended to be used in specifying parameters, he also supplies an
additional program which is used to extend the underlying constraint
solver used by the compiler. This program adds more primitive
constraints and knows how to perform reasoning using these
constraints. This is how I expect we will initially implement the
\Xten{} language. As language designers and implementers we will
provide constraint solvers for finite functions and {\tt Herbrand}
terms on top of arithmetic.





\end{document}
