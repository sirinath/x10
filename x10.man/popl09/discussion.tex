\subsection{Type-valued properties}
\label{sec:type-properties}
\label{sec:variance}

We explored a 
different approach to adding generics to \Xten based on a
generlization of properties.
A \emph{type property}
is a final object member initialized at construction-time with a
concrete type.  Type properties and other type-valued variables
are declared just as normal variables but with type \xcd"*".
Like normal value properties, type properties
can be used in constrained types through the variable \xcd"self".

For example, the \xcd"Vector" class of Figure~\ref{fig:vector}
could be written as shown in Figure~\ref{fig:vprop}.
The class has a type-valued property \xcd"T" and a value property \xcd"len".
\begin{figure}
{
\begin{xtennoindent}
class List(T: *, len: int) {
  val data: Rail(T);

  def this(S: *, n: int, init: (int) => T):
      Vector{self.T==S,self.len==n} {
    super();
    property(S, n);
    ...
  }

  def map(S: *, f: T => S):
      Vector{self.T==S,self.len==this.len} { ... }

  def get(i: int{0 <= self, self < len}): this.T { ... }

  ...
}
\end{xtennoindent}
}
\caption{Vector example, with type properties}
\label{fig:vprop}
\end{figure}

The constructor of \xcd"Vector" takes an additional type-valued
argument \xcd"S", which is used to initialize the type property
\xcd"T".  The \xcd"map" method takes a type argument \xcd"S" among its
formal parameters and returns a \xcd"Vector" whose \xcd"T"
property is constrained to be equal to \xcd"S".

Subtyping constraints on type properties may be used to
provide \emph{use-site variance} constraints.
Use-site variance based on structural virtual types was proposed by
Thorup and Torgerson~\cite{unifying-genericity} and extended for
parametrized type systems by Igarashi and
Viroli~\cite{variant-parametric-types}.  The latter type system lead
to the development of wildcards in
Java~\cite{Java3,adding-wildcards,wildcards-safe}.  Constrained
type properties
have similar expressive power.

Consider the following subtypes of \xcd"Vector":
\begin{itemize}
\item \xcd"Vector".  This type has no constraints on the type
property \xcd"T".
Any type that constrains \xcd"T"
is a subtype of \xcd"Vector".  The type \xcd"Vector" is equivalent to
\xcd"Vector{true}".
%
For a \xcd"Vector" \xcd"v", the return type of the \xcd"get" method
is \xcd"v.T".
Since the property \xcd"T" is unconstrained,
the caller can only assign the return value of \xcd"get"
to a variable of type \xcd"v.T" or of type \xcd"Object".

\item \xcd"Vector{T==float}".
The type property \xcd"T" is bound to \xcd"float".
For a final expression \xcd"v" of this type,
\xcd"v.T" and \xcd"float" are equivalent types and can be used
interchangeably.
The syntax \xcd"Vector[float]" is used as
shorthand for \xcd"Vector{T==float}".

\item \xcdmath"Vector{T$\extends$Collection}".
This type constrains \xcd"T" to be a subtype of \xcd"Collection".
All instances of this type must bind \xcd"T" to a subtype of
\xcd"Collection"; for example \xcd"Vector[Set]" (i.e.,
\xcd"Vector{T==Set}") is a subtype of
\xcdmath"Vector{T$\extends$Collection}" because \xcd"T==Set" entails
\xcdmath"T"
\xcdmath"$\extends$"
\xcdmath"Collection".
%
If \xcd"v" has the type \xcdmath"Vector{T$\extends$Collection}",
then the return type of \xcd"get" has type \xcd"v.T", which is an unknown but
fixed subtype of \xcd"Collection"; the return value can be
assigned into a variable of type \xcd"Collection".

\item \xcdmath"Vector{T$\super$String}".  This type bounds the type property
\xcd"T"
from below.  For a \xcd"Vector" \xcd"v" of this type, any
supertype of \xcd"String" may flow into a variable of type \xcd"v.T".
The return type of the \xcd"get"
method is known to be a
supertype of \xcd"String" (and implicitly a subtype of \xcd"Object").
\end{itemize}

\label{sec:usability}
\label{sec:parameters-vs-fields}

The key difference between type parameters and type properties
is that type properties are
instance \emph{members} bound during object construction.  Type
properties are thus accessible through expressions---\xcd"e.T" is
a legal type (if \xcd"e" is final)---and are inherited by subclasses.
These features gives type properties more expressive power than
type parameters, but because they 
provide similar functionality with often subtle distinctions
type properties
can be difficult to use, especially for novices.
For this reason, \Xten uses type parameter rather than type properties.

Since type properties are inherited, the language design needs
to account for ambiguities introduced when the same name is
used for different type properties declared in or inherited into a class.
These can be disambiguated
by ``casting'' the target up to the desired supertype,
e.g., \xcd"(e as C).X" specifies
the property \xcd"X" inherited from \xcd"C".

As an example, in the following \Xten code extended with type properties, the
\xcd"HashMap"  class inherits the properties \xcd"K" and \xcd"V" from the
\xcd"Map" interface.
\begin{xten}
interface Map(K:*, V:*) {
  def get(K): V;
  def put(K, V): V;
}

class HashMap implements Map {
  def get(k: K): V = ...;
  def put(k: K, v: V): V = ...;
}
\end{xten}
A user more used to type parameters might declare \xcd"HashMap" as follows:
\begin{xten}
class HashMap(K:*,V:*) implements Map(K,V) {
  def get(k: K): V = ...;
  def put(k: K, v: V): V = ...;
}
\end{xten}
This declaration would introduce a new pair of type properties
named \xcd"K" and
\xcd"V" that shadow the inherited properties.
A na{\"\i}ve implementation of type properties would store run-time
type information for all four properties in each instance
of \xcd"HashMap".


\subsection{Virtual types}

\emph{Virtual types}~\cite{beta,mp89-virtual-classes,ernst06-virtual}
are a language-based extensibility
mechanism 
originally introduced in the language
BETA~\cite{beta} that---along with
similar constructs built on path-dependent types found in
languages such as Scala~\cite{scala}, J\&~\cite{nqm06},
and Tribe~\cite{cdnw07-tribe}---share many similarities with type properties.
A virtual type is a type binding nested within an enclosing instance.
Subclasses are permitted to override the binding of inherited virtual types. 

Virtual types
may be used to provide genericity; indeed,
one of the first proposals for genericity in Java was based on
virtual types~\cite{thorup97}, and
Java
wildcards (i.e., parameters with use-site variance)
were developed from a line of work beginning with virtual
types~\cite{unifying-genericity,variant-parametric-types,adding-wildcards}.

%Igarashi and Pierce~\cite{ip99-virtual-types}
%model the semantics of virtual types
%and several variants
%in a typed lambda-calculus with subtyping and dependent types.

Constrained types are more expressive than virtual
types in that they can be constrained at the use-site,
can be refined on a per-object basis without explicit subclassing,
and can be refined contravariantly as well as covariantly.

Thorup~\cite{thorup97}
proposed adding genericity to Java using virtual types.  For example,
a generic \xcd"List" class can be written as follows:
{
\begin{xten}
abstract class List {
  abstract typedef T;
  T get(int i) { ... }
}
\end{xten}}
\noindent
The virtual type \xcd"T" is unbound in \xcd"List", but 
can be refined by binding \xcd"T" in a subclass:
{
\begin{xten}
abstract class NumberList extends List {
  abstract typedef T as Number;
}
class IntList extends NumberList {
  final typedef T as Integer;
}
\end{xten}}
\noindent
Only classes where \xcd"T" is final bound, such as \xcd"IntList",
can be non-abstract.
%
The analogous definition of 
\xcd"List" in \Xten{} using type properties is as follows:
{
\begin{xten}
class List[T] {
  def get(i: int): T { ... }
}
\end{xten}}

\noindent
Unlike the virtual-type version,
the \Xten{} version of \xcd"List" is not abstract;
\xcd"T" need not be instantiated by a subclass because it can be
instantiated on a per-object basis.
Rather than declaring subclasses of \xcd"List",
one uses the constrained subtypes
\xcdmath"List{T$\extends$Number}" and \xcd"List{T==Integer}".

Type properties can also be refined contravariantly.
For instance, one can write the type \xcdmath"List{T$\super$Integer}".

Dependent classes~\cite{dependent-classes} generalize virtual
classes to express similar semantics via parametrization rather
than nesting.  Virtual classes depend only other their enclosing
instance; dependent classes, in contrast, depend on any number
of objects in which they are parametrized.  With type
properties, classes are not parametrized on their values;
rather properties are members and types are constructed by
constraining these properties.  Parametrization can be 
encoded with type properties using equality constraints.

\subsection{Wildcards}
\label{sec:wildcards}

Constraints on type properties can also provide a characterization of
wildcards in Java~\cite{Java3,adding-wildcards,wildcards-safe}.
We leave to future work a formal proof of this
characterization.
Wildcards can be  motivated
by the following example from Torgersen et al.~\cite{adding-wildcards}:
Consider a \xcd"Set" class and a variable \xcd"EMPTY" containing
the empty set.  What should be the type of \xcd"EMPTY"?
In Java, one can use a wildcard and 
assign the type \xcd"Set<?>", i.e., the type of all \xcd"Set"
instantiated on \emph{some} parameter.  Clients of this
type do not know what parameter to which the actual instance of \xcd"Set"
is bound.
With type properties,
a similar effect is achieved simply by leaving the
element type property of \xcd"Set" unconstrained.

Wildcards can
also be bounded above and below with
``\Xcd{?} \Xcd{extends} \Xcd{T}'' and ``\Xcd{?} \Xcd{super} \Xcd{T}'',
respectively.
%
Again, with type properties, 
a similar effect is achieved by constraining
the
element type property \xcd"X" of \xcd"Set" with subtyping constraints,
e.g., \xcd"Set{X<=T}" and \xcd"Set{X>=T}".

\eat{
Finally,
one can also specify lower bounds on types.  These are useful for
comparators:
{\footnotesize
\begin{xtenmath}
class TreeSet[T] {
  def this[T](cmp: Comparator{T $\super$ this.T}) { ... }
}
\end{xtenmath}}
Here, the comparator for any supertype of \xcd"T" can be used as
to compare \xcd"TreeSet" elements.

Another use of lower bounds is for list operations.
The \xcd"map" method below takes a function that maps a supertype
of the class parameter \xcd"T" to the method type parameter \xcd"S":
{\footnotesize
\begin{xtenmath}
class List[T] {
  def map[S](fun: Object{self $\super$ T} => S) : List[S] { ... }
}
\end{xtenmath}}
}

Like wildcards,
constrained types support
\emph{proper abstraction}~\cite{adding-wildcards}.  To illustrate, a
\xcd"reverse"
operation can operate on \xcd"List" of any type:
{
\begin{xten}
def reverse(list: List) {
  for (i: int in 0..list.length-1)
    swap(list, i, list.length-1-i);
}
\end{xten}}

\noindent
The client of \xcd"reverse" need not provide the concrete type
on which the list is instantiated; the \xcd"list" itself
provides the element type---it is stored in the \xcd"list"
to implement run-time type introspection.

In Java, this method would be written with a type parameter on
the method; the type system permits it to be called with any
\xcd"List".
However,
the method parameter cannot be omitted: declaring a parameterless version
of \xcd"reverse" requires delegating to a private parametrized version that
``opens up'' the parameter.

%\subsection{Self types}

Type properties can also be used to support a form of self
types~\cite{bruce-binary,bsg95}.
%
Self types can be implemented by introducing a
type property \Xcd{class} to the root of the class hierarchy, \Xcd{Object}:
{\footnotesize
\begin{xtenmath}
class Object[class] { $\dots$ }
\end{xtenmath}}

\noindent
Scala's path-dependent types~\cite{scala} and J\&'s
dependent classes~\cite{nqm06}
take a similar approach.

Self types are achieved by
implicitly constraining types so that if a path expression \Xcd{p}
has type \Xcd{C}, then
$\Xcd{p}.\Xcd{class} \subtype \Xcd{C}$.  In particular,
$\Xcd{this}.\Xcd{class}$ is guaranteed to be a subtype
of the lexically enclosing class; the type
$\Xcd{this}.\Xcd{class}$ represents all instances of the fixed,
but statically unknown, run-time class referred to by the \Xcd{this}
parameter.

Self types address the binary method problem~\cite{bruce-binary}.
In the following
example, the class \xcd"BitSet" can be written with a
\xcd"union" method that takes a self type as argument.

{\footnotesize
\begin{xtenmath}
interface Set {
  def union(s: this.class): void;
}

class BitSet implements Set {
  int bits;
  def union(s: this.class): void {
    this.bits |= s.bits;
  }
}
\end{xtenmath}}

\noindent
Since \xcd"s" has type this \Xcd{this}.\Xcd{class}, and the class
invariant of \xcd"BitSet" implies
$\Xcd{this}.\Xcd{class} \subtype \Xcd{BitSet}$,
the implementation of the method is free to access the
\xcd"bits" field of \xcd"s".

Callers of \xcd"BitSet".\xcd"union()" must call the method with
an argument that has the same run-time class as the
receiver.  For a receiver \xcd"p", the
type of the actual argument of the call must have a constraint
that entails \xcd"self".\xcd"class==p".\xcd"class".



\subsection{Self types}

Type properties can also be used to support a form of self
types~\cite{bruce-binary,bsg95}.
%
Self types can be implemented by introducing a
type property \Xcd{type} to the root of the class hierarchy, \Xcd{Object}:
\begin{xtenmath}
class Object(type:*){type <= Object} { $\dots$ }
\end{xtenmath}

\noindent
For any final path expression \Xcd{p}, the type
$\Xcd{p}.\Xcd{type}$ represents all instances of the fixed,
but statically unknown, run-time class referred to by \Xcd{p}.
Scala's path-dependent types~\cite{scala} and J\&'s
dependent classes~\cite{nqm06}
take a similar approach.

Self types are achieved by
constraining types so that if a path expression \Xcd{p}
has type \Xcd{C}, then
$\Xcd{p}.\Xcd{type} \subtype \Xcd{C}$.
In particular, one can add the class invariant
$\Xcd{this}.\Xcd{type} \subtype \Xcd{C}$ to every class \Xcd{C}.
This invariant ensures that
$\Xcd{this}.\Xcd{type}$ is a subtype
of the lexically enclosing class.

\eat{
\subsection{Ownership types}

Consider the following example of generic ownership
derived from Potanin et al.~\cite{ogj-oopsla06}.

\begin{xtenmath}
class Object(owner: Object) { }

// Map inherits Object.owner
// No need to add explicit vOwner and kOwner properties for Key, Value
class Map[Key, Value]{Key $\extends$ Comparable, Value $\extends$ Object}
{
    private nodes: Vector[Node[Key, Value](this)](this);

    public def put(key: Key, value: Value): Void {
        nodes.add(new Node[Key, Value](key, value, this)());
    }

    public def get(key: Key): Value {
        for (mn: Node[Key, Value](this) in nodes) {
            if (mn.key.equals(key))
                return mn.value;
        }
        return null;
    }

    // OGJ will prevent this from being called, since caller
    // can only assign the result to a supertype of Vector(this),
    // which would be only Vector(this) or Object(this)
    // BUT: we have Vector $\super$ Vector(this)
    // Need to require that all class types have an equality constraint
    // on the owner property
    public def exposeVector(): Vector(this) { return nodes; }
}

class Node[Key, Value]
    {Key $\extends$ Comparable, Value $\extends$ Object}
{
    val key: Key;
    val value: Value;

    public def this[K, V](k: Key, v: Value, o: Object): Node[K, V](o) {
        super(o);               // set the owner
        property[K, V];         // set the type properties
        this.key = k;
        this.value = v;
    }
}
\end{xten}

Restrictions:
\begin{itemize}
\item owner property must be constrained (define this!)
\item owner is always equal to or inside the owner of all other type properties
\item types with an actual owner == this, can only be accessed via this
\end{itemize}

}


\subsection{Run-time casts}
\label{sec:casts}

While constraints are normally solved at compile time, 
constraints can be evaluated at run time by using casts.
The expression 
\xcd"xs as List{length==n}" checks not only 
that \xcd"xs"
is an instance of
the \xcd"List" class, but also that \xcd"xs.length" equals \xcd"n".
A \xcd"ClassCastException" is thrown if the check fails.
%
In this example, the test of the constraint does not require
run-time constraint
solving; the constraint can be checked by simply
evaluating the \xcd"length" property of \xcd"xs" and comparing against \xcd"n".

However, the situation is more complicated when casting to a
generic type.  Unlike Java, \Xten does not erase type
parameters at run-time.  Instead each instance of a generic type
contains a description of the types that its parameters are
instantiated upon.  This extra run-time type information
enables checked casts to generic types.

The implementation becomes complicated when variant type
parameters are permitted.
While our formalism does not model
variance, \Xten does support them, as described in
Section~\ref{sec:lang}.
Consider a declaration of class \xcd"C" with a covariant type
parameter \xcd"X":
\begin{xtenmath}
class C[+X](x: X) {
   def this(y: X) { property(y); }
}
\end{xtenmath}
\noindent
Because the static type of an expression may involve one or more
type parameters,
checking if an expression is an instance of, say, \xcd"C[A{c}]"
may require a run-time constraint entailment test.  That is, if a value \xcd"v"
has run-time type \xcd"C[B{d}]", then because \xcd"C"'s
parameter is covariant,
\xcd"v" \xcd"as" \xcd"C[A{c}]" must check that \xcd"B" is a subclass of
\xcd"A" and that \xcd"d" entails \xcd"c".\footnote{If the
\xcd"X" parameter were declared invariant, the cast would only
need to check that \xcd"A" and \xcd"B" are the same class and
that \xcd"d" and \xcd"c" are equivalent, which might be
accomplished by representing constraints at run-time in a
canonical form.}

One approach is to restrict the language 
to rule out casts to type parameters 
and to generic types with subtyping constraints, ensuring that
entailment checks are not needed at run time.

Alternatively, 
the constraint solver could be embedded into the runtime system.
This is the solution used in the \Xten{} implementation; however, this
solution can result in inefficient run-time casts
if entailment checking for the given constraint system is expensive.

We are exploring alternative implementations or future work.

\eat{
A different approach to have the compiler pre-compute the results of
entailment checks.
This might be done by analyzing the program to identify which pairs of
constraints might be tested for entailment at run time and then generating a
graph were each node is a constraint and there is a directed edge between
nodes in an entailment relationship.  Run-time entailment
checking can then be implemented as reachability checking. 
This solution is a whole program analysis; all
constraints must be visible to generate the graph.
\todo{
We leave the design of this analysis for future work.
}

\todo{this is vague.  and very likely wrong.}
If \xcd"e as T" occurs in the program text and \xcd"e" has
type \xcd"U", the analysis identifies all \xcd"new" expressions
$a$,
that create a subtype of \xcd"U" and identifies all type
expressions $t$ that could instantiate \xcd"T".
For each pair $(a,t)$, the analysis checks that the
type of $a$ (determined by the constructor invoked by $a$) is a subtype of $t$.
Since $a$ and $t$ may be in different environments, \xcd"e" must
be substituted for \xcd"this" in $a$ and \xcd"self" in $t$.
}

\eat{
Another option is to test objects cast to \xcd"T" not for membership in the
type \xcd"T", but rather to test against the
\emph{interpretation} of \xcd"T".
Observe that
if an instance of a generic class \xcd"C[X]"
is a member of the type \xcd"C{X==U}", then
all fields ${\tt f}_i$ of the instance with
declared type ${\tt S}_i$ contain values
that are instances of ${\tt S}_i[{\tt U}/{\tt X}]$.
For example, given the following declaration of class \xcd"List":
{
\begin{xten}
class List[X] {
  val head: X;
  val tail: List[X];
}
\end{xten}
}
\noindent
if \xcd"xs" is an instance of \xcd"List{X==String}", then
by checking that \xcd"xs.head" is an instance of \xcd"String"
and \xcd"xs.tail" is, recursively, an instance of \xcd"List{X==String}".
This property can be exploited by implementing cast to check the
types of all fields of the object.
For this check to be sound, it is vital that all fields whose
type depends on the type property \xcd"X" be transitively final;
otherwise, the test is not invariant---the
result of the test could change as the data is mutated.
Care must also be taken to implement the
test so that it terminates for cyclic data structures.
This implementation is inefficient for large data structures.

This solution has a more permissive semantics
than those implemented in \Xten or \FXGL{Q}.
The difference is best illustrated by considering an empty generic class:
{
\begin{xten}
class Nil[X] { }
\end{xten}
}
\noindent
In this case, there is no field of type \xcd"X" to test;
therefore, an object instantiated as \xcd"Nil[int]" 
can be considered a member of \xcd"Nil[String]". 
However, the solution remains sound:
Given a class \xcd"C[X]" and an expression \xcd"e" of type \xcd"t.X", if
a run-time check finds that \xcd"t" has type \xcd"C[T]", 
the compiler \emph{cannot} use this information to derive 
that \xcd"e" has type \xcd"T".
We leave to future work a proof of this claim.
}

