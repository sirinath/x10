\def\has{\mbox{\tt has}}
\def\inv{\mathit{inv}}

\def\FX{{\sf FX}}
\def\FXG{{\sf FX(G)}}
\def\FXD{{\sf FX(D($\cal A$))}}
\def\FXGD{{\sf FX(G,D($\cal A$))}}
\def\FXGDP{{\sf FX(G,D($\cal A$),P)}}
\def\TConstr{\mbox{\sc T-Constr}}
\def\TInv{\mbox{\sc T-Inv}}
\def\TVar{\mbox{\sc T-Var}}
\def\TField{\mbox{\sc T-Field}}
\def\TInvk{\mbox{\sc T-Invk}}
\def\TNew{\mbox{\sc T-New}}
\def\TCast{\mbox{\sc T-Cast}}
\def\TUCast{\mbox{\sc T-UCast}}
\def\TDCast{\mbox{\sc T-DCast}}
\def\TSCast{\mbox{\sc T-SCast}}

\def\RField{\mbox{\sc R-Field}}
\def\RCField{\mbox{\sc RC-Field}}
\def\RInvk{\mbox{\sc R-Invk}}
\def\RCInvkRecv{\mbox{\sc RC-Invk-Recv}}
\def\RCInvkArg{\mbox{\sc RC-Invk-Arg}}
\def\RCNewArg{\mbox{\sc RC-New-Arg}}
\def\RCast{\mbox{\sc R-Cast}}
\def\RCCast{\mbox{\sc RC-Cast}}

We now describe the semantics of languages in the \FX{} family.

Each language $\cal L$ in the family is defined over a given input constraint
system $\cal X$. Given a program $P$, we now show how to build a
larger constraint system ${\cal O}({\cal X})$ on top of $\cal X$ which
captures constraints related to the object-oriented structure of
$P$. $\cal O$ is sensitive to $\cal X$ only in that $\cal O$ depends
on the types defined by $\cal L$, and these may depend on $\cal X$.

The static and dynamic semantics of $\cal L$ rests on ${\cal O}({\cal
X})$.

\subsection{The Object constraint system, $\cal O$}\label{sec:O}

Given $\cal L$, its input constraint system $\cal X$ we now show
how to define $\cal O$. The inference relation for $\cal O$ depends
on the object-oriented structure of the input program $P$ in $\cal L$.
For some members of $\cal L$, viz.{} the generic
languages, $\cal X$ itself may use some of the constraints defined by
$\cal O$. Thus we should think of $\cal X$ and $\cal O$ as being
defined simultaneously and recursively.

The constraints of $\cal O$ are given by:

\begin{tabular}{r@{\quad}rcl}
(Member) & {\tt M}&{::=}& $\tt m(\bar{\tt x}:\bar{\tt V})\{c\}:T=e$ \alt {\tt f:V}\\
(Const.) & {\tt c},{\tt d} &{::=}& ${\tt S}\ \extends\ {\tt T}$ \alt ${\tt S}\ \subtype\ {\tt T}$\\
&&& \alt $\fields({\tt x})= \bar{\tt f}:\bar{\tt V}$ \\
&&& \alt {\tt x\ \has\ M}
\end{tabular}

Intuitively, ${\tt S}\ \extends\ {\tt T}$ is intended to hold if it
can be established that {\tt S} is extends {\tt T}. For a
variable {\tt x}, {\tt \fields({\tt x})} is intended to specify the
(complete) set of typed fields available to {\tt x}. 
{\tt x\ \has\ M} is intended to specify that the member {\tt M}
(field or method) is availabele to {\tt x}.


$\cal O$ satisfies the following axioms and inference rules:
{\footnotesize
\infrule[Class]
{\tt \klass\ {\tt C}(\bar{\tt f}:\bar{\tt V})\ {\tt extends}\ {\tt D}\ \ldots \in P}
{\tt \vdash_{\cal O} \klass({\tt C}),{\tt C} \extends {\tt D}}

\infax[V-Id]{\vdash_{\cal O} T \extends T}
\infrule[S-Trans]
{\tt \Gamma \vdash_{\cal O} T_1 \subtype T_2, T_2 \subtype T_3}
{\tt \Gamma \vdash_{\cal O} T_1 \subtype T_3}
\infrule[\tt Sub-X]
{\Gamma \vdash_{\cal O} {\tt t}\ \extends\ {\tt T}}
{\Gamma \vdash_{\cal O} {\tt t} \subtype {\tt T}}

\infax[Fields-B]{x:{\tt Object} \vdash_{\cal O} \fields({\tt x})=\bullet}
\infrule[Fields-I]
{\Gamma, {\tt x:D} \vdash_{\cal O} \fields({\tt x})=\bar{\tt g}:\bar{\tt V}\andalso 
\klass\ {\tt C}(\bar{\tt f}:\bar{\tt U}){\tt \{c\}}\ {\tt extends}\ {\tt D}\{\bar{\tt M}\} \in {\tt C}}
{\Gamma, {\tt x:C} \vdash \fields({\tt x})=\bar{\tt g}:\bar{\tt V},\bar{\tt f}:\bar{\tt U[x/\this]}}
\infrule[Fields-C,E]
{\Gamma, {\tt x: S} \vdash_{\cal O} \fields({\tt x})=\bar{\tt f}:\bar{\tt V}}
{\Gamma, {\tt x:S\{d\}} \vdash_{\cal O} \fields({\tt x})=\bar{\tt f}:\bar{\tt V\{d[x/\self]\}} \\
 \Gamma, {\tt x: (y:U;S)} \vdash_{\cal O} \fields({\tt x})=\bar{\tt f}:\bar{\tt (y:U;V)}
}
\infrule[\tt has-F]
{\Gamma\vdash_{\cal O} \fields({\tt x})=\bar{\tt f}:\bar{\tt V}} 
{\Gamma \vdash_{\cal O} {\tt x}\ \has\ {\tt f_i}:{\tt V_i}}

\infax[Sel]{\tt \vdash_{\cal O} new\ D(\bar{\tt t}).{\tt f_i=t_i}}

\infrule[Method-B]
{\Gamma, {\tt x}:{\tt C}\vdash_{\cal O}\klass({\tt C})\andalso \theta=[{\tt x}/\this] \andalso
{\tt def}\ {\tt m(\bar{\tt z}:\bar{\tt V})\{c\}:T=e}\in P}
{\tt\Gamma, {\tt x:C}\vdash_{\cal O} {\tt x}\ \has\ (m(\bar{\tt z}:\bar{\tt V\theta})\{c\theta\}:T\theta=e)}
\infrule[Method-I]
{\Gamma, {\tt x}:{\tt D}\vdash_{\cal O}{\tt x}\ \has\ m(\bar{\tt z}:\bar{\tt V})\{c\}:T=e
\andalso \klass\ {\tt C}(\ldots)\ {\tt extends}\ {\tt D}\{\bar{\tt M}\} \andalso {\tt m}\not\in \bar{\tt M}}
{\Gamma, {\tt x:C}\vdash_{\cal O} {\tt x}\ \has\ m(\bar{\tt z}:\bar{\tt V})\{c\}:T=e}
\infrule[Method-C,E]
{\tt\Gamma, x:S \vdash_{\cal O} x\ \has\ m(\bar{\tt z}:\bar{\tt V})\{c\}:T=e}
{\tt\Gamma, x:S\{d\} \vdash_{\cal O} x\ \has\ m(\bar{\tt z}:\bar{\tt V})\{c\}:T\{d[x/\self]\}=e \\
 \tt\Gamma, x: (y:U;S) \vdash_{\cal O} x\ \has\ m(\bar{\tt z}:\bar{\tt V})\{c\}:(y:U;T)=e}
}%\footnotesize

{\em Note: Figure out whether consistency checks need to be added.}

\begin{figure*}
\footnotesize
\begin{tabular}{l}
{\bf \FX{} productions:} \\
\quad\\
\begin{tabular}{ll}
\begin{tabular}{r@{\quad}r@{~~}c@{~~}l}
  (Class) & {\tt L} &{::=}& \klass\ {\tt C}($\bar{\tt f}:\bar{\tt T}$)\{{\tt c}\}\ {\tt extends}\ {\tt N}\ \{$\bar{\tt M}$\} \\
  (Method)& {\tt M} &{::=}& {\tt def}\ {\tt m}($\bar{\tt x}:\bar{\tt T}$){\tt \{c\}:T\,=\,e;}\\
  (Exp.)& {\tt e} &{::=}& {\tt x} \alt \this \alt {\tt e.f} \alt {\tt e.m}($\bar{\tt e}$) 
   \alt \new\ {\tt C}($\bar{\tt e}$) \alt {\tt e}\ \mbox{{\tt as} {\tt T}} \\ 
\end{tabular} 
&
\begin{tabular}{r@{\quad}r@{~~}c@{~~}l}
  (Type)& {\tt S},{\tt T},{\tt U},{\tt V}&{::=}& {\tt N}
        \alt {\tt T\{c\}} \alt {\tt x}:{\tt S};\,{\tt T}\\
  (N Type) & {\tt N}&{::=}& {\tt C} \alt {\tt N\{c\}}\\
  (C Term) & {\tt t} &{::=}& {\tt x} \alt \self \alt \this \alt
  {\tt t}.{\tt f} \alt \new\ {\tt C}($\bar{\tt t}$) \alt $\tt f(\bar{\tt t})$ \\
  (Const.) & {\tt c},{\tt d} &{::=}&\true \alt {\tt t==t} \alt {\tt c},{\tt c} \alt {\tt x}:{\tt T};\,{\tt c} \alt ${\tt q(\bar{\tt t})}$
\end{tabular} 
\end{tabular}
\quad \\
\quad\\
{\bf Constraint well-formedness rules:}\\[-12pt]
{\tabcolsep=0pt
\begin{tabular}{p{0.26\textwidth}p{0.33\textwidth}p{0.38\textwidth}}
\infax[True]{\Gamma \vdash {\tt true}: {\tt o}}
&
\infrule[And]
	{\Gamma\tt  \vdash c_0: o \andalso \Gamma \vdash c_1: o}
	{\Gamma\tt \vdash (c_0,c_1):o}
&
\infrule[Exists]
	{\Gamma\tt \vdash t: T \andalso \Gamma \vdash c[t/x]:o}
	{\Gamma\tt \vdash x:T;c: o}
\\
\infrule[Pred]
        {\tt q(\bar{\tt T}):{\tt o}\in {\cal C} \andalso \Gamma \vdash \bar{\tt t}:\bar{\tt T}}
        {\Gamma \vdash \tt q(\bar{\tt T}):o}
&
\infrule[Fun]
        {\tt f(\bar{\tt T}):{\tt T}\in {\cal C} \andalso \Gamma \vdash \bar{\tt t}:\bar{\tt T}}
        {\Gamma\tt \vdash \tt f(\bar{\tt T}):T}
&
\infrule[Equals]
        {\Gamma \vdash {\tt t}_0: {\tt T}_0
        \andalso
        \Gamma \vdash {\tt t}_1: {\tt T}_1 \\
          (\Gamma \vdash {\tt T}_0 \subtype {\tt T}_1 \vee
          \Gamma \vdash {\tt T}_1 \subtype {\tt T}_0)}
        {\Gamma \vdash {\tt t}_0{\tt ==}{\tt t}_1:o}
\\
\end{tabular}}
%\quad\\[-12pt]
\quad\\
{\bf Type well-formedness rules:}\\[-12pt]
{\tabcolsep=0pt
\begin{tabular}{p{0.26\textwidth}p{0.33\textwidth}p{0.38\textwidth}}
\infrule[Class]
	{\Gamma \vdash \klass({\tt C})}
	{\Gamma \vdash {\tt C}\ \type} 
&
\infrule[Exists-T]
	{\Gamma\tt \vdash {\tt S}\ \type, {\tt T}\ \type}
	{\Gamma\tt \vdash {\tt x:S;T}\ {\tt type}} 
&
\infrule[Dep]
	{\Gamma\tt \vdash T\ \type \andalso \Gamma, \self:T \vdash c:o}
	{\Gamma\tt \vdash T\{c\}\ {\tt type}} 
\\
\end{tabular}}
\quad\\
\quad\\
{\bf Type judgment rules:}\\[-12pt]
{\tabcolsep=0pt
\begin{tabular}{l}
\begin{tabular}{p{0.27\textwidth}p{0.28\textwidth}p{0.42\textwidth}}
\infax[T-Var]
      {\Gamma, {\tt x}:{\tt T} \vdash {\tt x}:{\tt T\{\self==x\}}}
&
\infrule[T-Cast]
	{\Gamma \vdash {\tt e}:{\tt U} \andalso \Gamma \vdash {\tt T} \ \type}
	{\Gamma \vdash {\tt e}\ \as\ {\tt T}: {\tt T}} &
\infrule[T-Field]
	{\Gamma \vdash {\tt e}: {\tt S} \andalso \Gamma,{\tt z}:{\tt S}\vdash {\tt z}\ \has\ {\tt f}:{\tt T} \andalso \mbox{({\tt z} fresh)} }
	{\Gamma \vdash {\tt e}.{\tt f}:  ({\tt z}:{\tt S};\,{\tt T\{\self==z.f\}})}
\end{tabular}\\[-12pt]
\begin{tabular}{p{0.50\textwidth}p{0.47\textwidth}} 
\infrule[T-Invk]
	{\Gamma \vdash {\tt e}:{\tt T},\bar{\tt e}:\bar{\tt V} \andalso \\
	  \Gamma,{\tt z}:{\tt T},\bar{\tt z}:\bar{\tt V} \vdash
		{\tt z}\ \has\ ({\tt m}(\bar{\tt z}:\bar{\tt U}), {\tt c} \rightarrow {\tt S}), 
		\bar{\tt V} \subtype \bar{\tt U},{\tt c} 
	  \andalso \mbox{({\tt z},$\bar{\tt z}$ fresh)}}
	{\Gamma \vdash {\tt e}.{\tt m}(\bar{\tt e}):
                ({\tt z}:{\tt T};\,\bar{\tt z}:\bar{\tt V};\,{\tt S})}
	&
\infrule[T-New]
	{\Gamma \vdash \bar{\tt e}:\bar{\tt T} \andalso \vdash \klass({\tt C}) \\ 
	  \Gamma,{\tt z}:{\tt C}\vdash \fields({\tt z})=\bar{\tt f}:\bar{\tt U}  \andalso \mbox{({\tt z},$\bar{\tt z}$\ fresh)}\\
	  \Gamma, {\tt z}:{\tt C}, \bar{\tt z}:\bar{\tt T}, {\tt z}.\bar{\tt f}=\bar{\tt z} 
	  \vdash \bar{\tt T} \subtype \bar{\tt U}, \inv({\tt C},{\tt z})}
	{\Gamma \vdash \new\ {\tt C}(\bar{\tt e}): {\tt C}\{\bar{\tt z}:\bar{\tt T};\,\new\ {\tt C}(\bar{\tt z})=\self, \inv({\tt C},\self)\}}
\\[-12pt]
\infrule[Method OK]
	{\this:{\tt C}\vdash {\tt c}:o \andalso \this:{\tt C},\bar{\tt x}:\bar{\tt U},{\tt c} \vdash {\tt T} \ \type, \bar{\tt U} \ \type, 
	  {\tt e}:{\tt S}, {\tt S} \subtype {\tt T}}
	{{\tt def}\ {\tt m}(\bar{\tt x}:\bar{\tt U})\{{\tt c}\}:{\tt T}= {\tt e};\ \mbox{OK in}\ C}
	&
\infrule[Class OK]
	{\bar{\tt M}\ \mbox{OK in}\ {\tt C}
        \andalso
        \this:{\tt C} \vdash {\tt c}:{\tt o}
        \andalso
        \this:{\tt C},{\tt c} \vdash \bar{\tt T}\ \type, {\tt N}\ \type}
	{\mbox{\tt class}\ {\tt C}(\bar{\tt f}:\bar{\tt T})\{{\tt c}\}\ 
	  \mbox{\tt extends}\ {\tt N}\{\bar{\tt M}\} \ \mbox{OK}}
\end{tabular}
\end{tabular}}\\[-12pt]
\quad\\
{\bf Transition rules:}\\[-12pt]
{\tabcolsep=0pt
\begin{tabular}{p{0.43\textwidth}@{\quad}p{0.54\textwidth}}
\typicallabel{RC-Field}
\infrule[\RField]
	{\tt x:C \vdash \fields(\tt x)=\bar{\tt f}:\bar{\tt T}}
	{(\new\ {\tt C}(\bar{\tt e})).{\tt f_i} \derives {\tt e_i}}

\infrule[\RCField]%
	{{\tt e} \derives {{\tt e}}'}
	{{\tt e}.{\tt f}_i \derives {{\tt e}}'.{\tt f}_i}

\infrule[\RCast]%
	{\vdash {\tt C}\{\self==\new\ {\tt C}(\bar{\tt d})\} \subtype {\tt T}}
	{{\tt \new\ C(\bar{\tt d})~as~T} \derives \new\ C(\bar{\tt d})}

\infrule[\RCCast]%
	{{\tt e} \derives {{\tt e}}'}
	{{\tt e~as~T} \derives {\tt e}'~{\tt as~T}}
	&
	\typicallabel{RC-Invk-Recv}
	\infrule[\RInvk]
		{{\tt x:C}\vdash {\tt x}\ \has\ {\tt m}(\bar{\tt x}:\bar{\tt T}){\tt \{c\}:T=e}}
		{(\new\ {\tt C}(\bar{\tt e})).{\tt m}(\bar{\tt d}) \derives {\tt e[\new\ {\tt C}(\bar{\tt e}),\bar{\tt d}/\this,\bar{\tt x}]}}

	\infrule[\RCInvkRecv]%
		{{\tt e} \derives {{\tt e}}'}
		{{\tt e}.{\tt m}(\bar{\tt e}) \derives {{\tt e}}'.{\tt m}(\bar{\tt e})}

	\infrule[\RCInvkArg]%
		{{\tt e}_i \derives {{\tt e}_i}'}
		{{\tt e}.{\tt m}(\ldots,{\tt e}_i,\ldots) \derives {{\tt e}}.{\tt m}(\ldots,{\tt e}_i',\ldots)} 

	\infrule[\RCNewArg]%
		{{\tt e}_i \derives {{\tt e}_i}'}
		{\new\ {\tt C}(\ldots,{\tt e}_i,\ldots) \derives \new\ {\tt C}(\ldots,{\tt e}_i',\ldots)}
\end{tabular}
}
\end{tabular}  

 \caption{Semantics of \FX}\label{fig:FX}
\end{figure*}


\subsection{Extensions of \FX}
\subsubsection{\FXG}
Generics as in {\sf FGJ} are supported by adding the following productions:

\begin{tabular}{r@{\quad}rcl}
  (Par Type)& {\tt V}&{::=}& \type\\
  (Path) & {\tt p} &{::=}& {\tt x} \alt {\tt p.f}\\
  (Type)& {\tt T}&{::=}& {\tt X} \alt p\\
  (C Term) & {\tt t} &{::=}& {\tt T} \\
  (Const.) & {\tt c} &{::=}& ${\tt t } \extends {\tt N}$ \alt {\tt t==t}\\
\end{tabular}

\noindent and the following rule:
{\footnotesize
\infrule[Path]
	{\Gamma \vdash {\tt p:T} \andalso \Gamma, {\tt x:T} \vdash {\tt x}\ \has\ {\tt X}:\ \type}
	{\Gamma \vdash {\tt p.X}\ {\tt type}}
\infax[\tt Type Var]{\Gamma, {\tt X:\type} \vdash {\tt X}\ \type}
\infrule[\tt Inh-p]
	{\Gamma \vdash_{\cal O} {\tt p} \extends {\tt T} \andalso \Gamma, {\tt x}:{\tt T} \vdash_{\cal O} {\tt x}\ \has\ {\tt M}}
	{\Gamma,{\tt x}:{\tt p} \vdash_{\cal O} {\tt x}\ \has\ {\tt M}}
\infrule[\tt Inh-X]
	{\Gamma \vdash_{\cal O} {\tt X} \extends {\tt T} \andalso \Gamma, {\tt x}:{\tt T} \vdash_{\cal O} {\tt x}\ \has\ {\tt M}}
	{\Gamma,{\tt x}:{\tt X} \vdash_{\cal O} {\tt x}\ \has\ {\tt M}}
}

\subsubsection{\FXD}
Only the following rules are needed; no additional rules are needed for
sub-typing, type-equivalence, expression typing or dynamic semantics.
Below, {\tt q} ({\tt f}) ranges over all predicates (functions) specified by $\cal C$:

\begin{tabular}{r@{\quad}rcl}
  (Type) & T & {::=}& \mbox{new base types, e.g. {\tt int}, {\tt boolean}}\\
  (C Term) & {\tt t} &{::=}& {\tt x} \alt $\tt f(\bar{\tt t})$ \alt t.f \alt $\tt \new\ C(\bar{\tt t})$\\
  (Const.) & {\tt c} &{::=}& {\tt t==t}  \alt ${\tt q(\bar{\tt t})}$ 
\end{tabular}

We need the following rules:
{\footnotesize
\infrule[Pred]
	{\tt p(\bar{\tt T}):{\tt o}\in {\cal C} \andalso \Gamma \vdash \bar{\tt t}:\bar{\tt T}}
	{\Gamma \vdash \tt p(\bar{\tt T}):o}

\infrule[Fun]
	{\tt f(\bar{\tt T}):{\tt T}\in {\cal C} \andalso \Gamma \vdash \bar{\tt t}:\bar{\tt T}}
	{\tt\Gamma \vdash \tt f(\bar{\tt T}):T}
\infrule[Equals]
	{\tt\Gamma \vdash t_0: T_0 \andalso \Gamma \vdash t_1: T_1 \\
	  (\tt\Gamma \vdash T_0 \subtype T_1 \vee \Gamma \vdash T_1 \subtype T_0)}
	{\tt\Gamma \vdash t_0=t_1:o}
}

\subsubsection{\FXGDP} 
Only the following rules are needed beyond those of \FXG{} and \FXD{}.





\subsection{Judgements}
%\begin{definition}[Designator] The class of {\em designators}
%is given by:
%\begin{tabular}{r@{\quad}rcl}
%(Designator) & {\tt d} &{::=}& {\tt x}\alt C($\bar{\tt d}$)  \alt {\tt d.f} \\
%\end{tabular}
%\end{definition}

In the following $\Gamma$ is a {\em well-typed context}, i.e.{} a
(finite, possibly empty) sequence of formulas $\tt x:T$, $\tt T\
\type$ and constraints $\tt c$ satisfying:
\begin{enumerate}
  \item for any formula $\phi$ in the sequence all variables $\tt x$
    ({\tt X}) occuring in $\phi$ are defined by a declaration $\tt
    x:T$ ({\tt X\ \type}) in the sequence to the left of $\phi$.

  \item for any variable $\tt x$ ({\tt X}), there is at most one
  formula $\tt x:T$ ({\tt X\ \type})  in $\Gamma$.
\end{enumerate}

The judgements of interest are:
\begin{description}
  \item[Type well-formedness]  $\Gamma \vdash {\tt T} \ {\tt type}$
  \item[Subtyping] $\Gamma \vdash {\tt S} \subtype {\tt T}$
  \item[Typing]    $\Gamma   \vdash {\tt e:T}$
  \item[Method ok] $\Gamma \vdash M\ \mbox{OK in $C$}$ (method $\tt M$ is well-defined for the class $\tt C$). 
  \item[Field ok]  $\Gamma \vdash f:T\ \mbox{OK in $C$}$ (field $\tt f:T$ is well-defined for the class $\tt C$).
  \item[Class ok] $\Gamma \vdash {\tt Cl} \ OK$ (class definition {\tt Cl} is ok). 
\end{description}


In defining these judgements we will use $\Gamma \vdash_{\cal C} c$,
the judgement corresponding to the underlying constraint system. For simplicity,
we define $\Gamma \vdash {\tt c}$ to mean $\sigma(\Gamma)\vdash_{\cal C} {\tt c}$,
where the {\em constraint projection}, $\sigma(\Gamma)$ is defined thus:

\begin{tabular}{l}
$\sigma(\epsilon)={\tt true}$\\
$\sigma(x:C\{c\}, \Gamma)$=c[x/\self], $\sigma(\Gamma)$\\
$\sigma(c,\Gamma)$ = c, $\sigma(\Gamma)$
\end{tabular}

\subsection{Well formedness rules}

We posit a special type {\tt o} (traditionally the type of
propositions), and regard constraints as expressions of type {\tt
o}. See Figure~\ref{Figure:Wff}.


\section{Type inference rules}

\subsection{Expression typing judgement}

Now we consider the rule for method invocation. Assume that in a type
environment $\Gamma$ the expressions $\bar{e}$ have the types
$\bar{T}$. Since the actual values of these expressions are not known,
we shall assume that they take on some fixed but unknown values
$\bar{z}$ of type $\bar{T}$.  Now for $z$ as receiver, let us assume
that the type $T\equiv C\{d\}$ has a method named {\tt m} with
signature $\bar{z}: \bar{Z}, c \rightarrow U$.  If there is no method
named {\tt m} for the class {\tt C} then this method invocation cannot
be type-checked. Without loss of generality we may assume that the
parameters of this method are named $\bar{z}$, since we are free to
choose variable names as we wish because of $\alpha$-equivalence. Now
in order for the method to be invocable, it must be the case that the
types $\bar{T}$ are subtypes of $\bar{Z}$. (Note that there are no
occurrences of {\tt this} in $\bar{Z}$; they have been replaced by
$z$ -- see Section~\ref{sec:O}) Further, it must be the case that for these parameter values,
the constraint $c$ is entailed. Given all these assumptions it must be
the case that the return type is $U$ --- with all the parameters
$\bar{z}$ existentially quantified.





