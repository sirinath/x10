\newcommand{\constraint}{{\tt constraint}}
\newcommand\cj[2]{{#1} \vdash {#2}~\constraint}
\newcommand\cjj[3]{{#1} \vdash {#2}~\constraint, {#3}~\constraint}
\newcommand\wj[2]{{#1} \vdash {#2}~\type}
\newcommand\tj[3]{{#1} \vdash {#2} \ty {#3}}
\newcommand\stj[3]{{#1} \vdash {#2} \subtype {#3}}

We now describe the semantics of languages in the \FX{} family.
We start with a core \FXZ{} language that support simple
\FJ-like types, then add value-dependent types, and type-dependent
types, separately, then finally add both. For uniformity we declare type-valued parameters and properties to be of ``type'' \type, instead of using square brackets to demarcate them.

\subsection{\FXZ}

The grammar for \FXZ{} is shown in Figure~\ref{fig:fx-grammar}.
The syntax is essentially that of \FJ{}.
Following the convention of \FJ{}, we use $\bar{x}$ to denote a
list $x_1, \dots, x_n$, and use $\bullet$ to denote the empty
list.

A program {\tt P} is a set of class declarations \tbar{L}.
Class names {\tt C} range over the declared classes in {\tt P} 
and {\tt Object}.
Classes have
properties (i.e., immutable fields) \tbar{f} and methods \tbar{M}.  We omit constructors
and require that the \new{} expression provide initializers
for all fields, including inherited fields. 
Methods are introduced with the {\tt def} keyword.

Both classes and methods may have constraint clauses
{\tt c}.  In the case of classes, {\tt c} is to be thought of as an
{\em invariant} satisfied by all instances of the class; in the case of
methods, {\tt c} is an additional condition, or {\em guard},
that must be satisfied by
the receiver 
and the actual arguments of the method in order for the method to
be applicable.

Expressions {\tt e} are either parameters {\tt x} (including the implicit
method parameter {\tt this}), field accesses, method invocations, \new{}
expressions, or casts (written {\tt e}~\as~{\tt T}).

The set of types includes classes {\tt C} and is closed under
constrained types ($\tt T\{c\}$) and existential
quantification ($\exists \tt x:T.~U$).
An object {\tt o} is of type {\tt C} if it is an instance of class {\tt C}; it is of type $\tt
T\{c\}$ if it is of type {\tt T} and it satisfies the constraint $\tt
c[o/self]$\footnote{Thus the constraint {\tt c} in a type {\tt T\{c\}}
should be thought of as a unary predicate $\lambda\self.{\tt c}$, an
object is of this type if it is of type {\tt T} and satisfies this
predicate.}; it is of type $\exists \tt x:T.~U$
if there is some object {\tt q}
of type {\tt T} such that {\tt o} is of type
$\tt U[q/x]$.

The syntax for constraints in \FXZ{} is specified in
Figure~\ref{fig:fx-grammar}. As expected, constraints
relate property fields of objects. Neither casts
nor method invocations are permitted in constraints.

We distinguish a subset of these constraints as
{\em user constraints}---these are permitted to occur in
programs. For \FXZ{} the only user constraint permitted is the vacuous
{\tt true}. Thus the types occurring in user programs are isomorphic
to class types, and class and method definitions specialize to the
standard class and method definitions of \FJ{}. 

The constraints permitted by the syntax in
Figure~\ref{fig:fx-grammar} that
are not user constraints are used to define the static and
dynamic semantics of \FXZ{} (see, e.g., rule \TField{} in Figure~\ref{fig:FX}).
The use of this richer constraint set as well as constrained and existential types is
not necessary in \FXZ; it simply enables us to present the static and dynamic
semantics once for the entire family of \FX{} languages,
specifying the other members of the family as extensions
to these core semantics.

Existential constraints are introduced for convenience only:
${\tt T}\{\exists {\tt x}:{\tt U}.~{\tt c}\}$ is equivalent to $\exists {\tt y}:{\tt U}.~{\tt T}\{{\tt c}[{\tt y}/{\tt x}]\}$ choosing {\tt y} not free in {\tt T}.

\paragraph{Dynamic semantics.}
The operational semantics, shown in Figure~\ref{fig:sos},
is straightforward and essentially identical
to \FJ \cite{FJ}. It is described in terms of a non-deterministic
reduction relation on expressions.\footnote{
For simplicity, we enforce call-by-value-like semantics:
\RField, \RInvk, and \RCast{} require receivers of the form
``$\new~{\tt C}(\tbar{t})$'' instead of ``$\new~{\tt C}(\tbar{e})$''.
Otherwise, we would have to distinguish compile-time constraints over constraint terms from run-time constraints over expressions.}
The only novelty is the use of the
subtyping relation to check that the cast is satisfied. In \FXZ, this
test simply involves checking that the class of which the object is an
instance is a subclass of the class specified in the given type; in
languages with richer notions of type this operation may
involve run-time constraint solving using the properties of the object.

\paragraph{Static Semantics.}
Each language in the family is defined over a given input constraint system $\mathcal{X}$ that is required to support the trivial constraint \true{}, conjunction, existential quantification, and equality on constraint terms. Given a program {\tt P}, we now show how to derive from $\mathcal{X}$ a larger deduction system that captures the object-oriented structure of {\tt P} and lets us decide whether {\tt P} is well-typed.

In the following, the context $\Gamma$ is always a
(finite, possibly empty) sequence of formulas $\tt x:T$ and constraints $\tt c$ satisfying:
\begin{enumerate}
  \item for any formula $\phi=\tt x:T$ or constraint $\phi=\tt c$ in the sequence all free variables $\tt y$
  occurring in $\tt T$ or $\tt c$ are declared by a formula $\tt
  y:U$ in the sequence to the left of $\phi$.

  \item for any variable $\tt x$, there is at most one
  formula $\tt x:T$ in $\Gamma$.
\end{enumerate}

\medskip

In the judgments that follow, 
for any formulas $\phi_1$ and
$\phi_2$, we adopt the convention that the rule $\Gamma \vdash \phi_1, \phi_2$
is shorthand for the rules
$\Gamma \vdash \phi_1$
and
$\Gamma \vdash \phi_2$.

The judgments of interest are as follows:
\begin{enumerate}
	\item Well-formedness:\\
	  $\cj{\Gamma}{\tt c}$ \hfill constraint {\tt c} is well-formed\\
	  $\wj{\Gamma}{\tt T}$ \hfill  type {\tt T} is well-formed
	\item Member lookup:\\
	  $\fields({\tt C})=\tbar{f}:\tbar{T}$ \hfill class {\tt C} has fields \tbar{f} of type \tbar{T}\\
	  $\Gamma\vdash {\tt C}~\has~{\tt I}$ \hfill class {\tt C} has member {\tt I}\\
	  $\Gamma\vdash {\tt x}~\underline\has~{\tt I}$ \hfill variable {\tt x} has member {\tt I}\\
	  $~$ \hfill where ${\tt I}::= {\tt f}:{\tt T} \alt {\tt m}(\tbar{x}:\tbar{T})\{{\tt c}\}:{\tt U}={\tt e}$
	\item Constraints:\\
	  $\Gamma\vdash {\tt c}$ \hfill constraint {\tt c} holds
	\item Typing:\\
	  $\Gamma\vdash {\tt e}:{\tt T}$ \hfill expression {\tt e} has type {\tt T}\\
	  $\vdash {\tt def}~{\tt m}(\tbar{x}:\tbar{T})\{{\tt c}\}:{\tt U}={\tt e}~{\rm OK~in}~{\tt C}$ \\ $~$ \hfill method {\tt m} in class {\tt C} type checks\\
	  $\vdash {\tt class}~{\tt C}(\tbar{f}:\tbar{T})\{{\tt c}\}~{\tt extends}~{\tt D}~\{~\tbar{M}~\}~{\rm OK}$ \\ $~$ \hfill class {\tt C} type checks
	\item Subtyping:\\
	  $\Gamma \vdash {\tt S} \subtype {\tt T}$ \hfill type {\tt S} is a subtype of type {\tt T}
\end{enumerate}

\medskip

A program type checks iff all its classes do.  We now describe
in more detail each of these judgments, in turn.

\paragraph{1. Well-formedness.} A constraint {\tt c} is well-formed in context $\Gamma$, written $\cj{\Gamma}{\tt c}$, iff its free variables are declared in $\Gamma$. The rules in Figure~\ref{fig:well} extend this requirement to types.

We say a program, context, or judgment is well-formed iff all the constraints and types involved are well-formed. By design, every judgment derived from a well-formed context is also well-formed. As a consequence, if a program type checks, it is well-formed. 

\paragraph{2. Member lookup.} Figure~\ref{fig:lookup} specifies the fields and methods available on each class. We impose restrictions on inheritance as follows. Classes may only be extended by classes with stronger invariants. Fields cannot be overridden.
A method may only be overridden by a method with the same name,
arity, parameter types, a return type that is a subtype of the
overridden return type, and a weaker guard (i.e., the superclass
method guard must entail the subclass's).
\footnote{To avoid cluttering the inference rules, we
define overriding only informally; a formal definition is
straightforward.}

To prepare for the introduction of generic types later, we
distinguish members that are available on variables from members
available on classes.  For now,
${\tt x}:{\tt C}~\underline\has~{\tt I}[{\tt x}/\this]$ iff ${\tt C}~\has~{\tt I}$.

\paragraph{3. Constraints.}
In defining these judgments we will use \mbox{$\Gamma \vdashX {\tt c}$}, the judgment corresponding to the underlying constraint system. Formally, $\cal X$ permits judgments of the form $\Gamma \vdashX {\tt c}$ where $\Gamma::=\tbar{c}$ is a (finite, possibly empty) sequence of constraints. We define the {\em constraint
projection}, $\sigma(\Gamma)$ as follows.
%
\begin{center}
\begin{tabular}{l}
$\sigma(\epsilon)={\tt true}$\\
$\sigma({\tt x}:{\tt C}, \Gamma)=\sigma(\Gamma)$\\
$\sigma({\tt x}:{\tt T\{c\}}, \Gamma)={\tt c}[{\tt x}/\self], \sigma({\tt x}:{\tt T},\Gamma)$\\
$\sigma({\tt x}:\exists {\tt y}:{\tt T}.~{\tt U}, \Gamma)=\sigma({\tt z:T}, {\tt x}:{\tt U[{\tt z}/{\tt y}]},\Gamma)$\\
$\sigma({\tt c},\Gamma) = {\tt c}, \sigma(\Gamma)$
\end{tabular}
\end{center}
%
Above, in the fourth rule, 
we assume that alpha-equivalence is used to
choose a variable {\tt z} that does not
occur in the context under construction.

We define $\Gamma\vdash {\tt c}$ in Figure~\ref{fig:object}. {\sc X-Sel} permits the underlying constraint system $\mathcal{X}$ to interpret field accesses. {\sc X-Inv} handles class invariants. Note the use of subtyping here to prepare for the introduction of bounds on generics types later.

We say that a context $\Gamma$ is {\em consistent} if all (finite)
subsets of $\{{\tt c}\alt \sigma(\Gamma) \vdash {\tt c}\}$ are consistent.
In all type judgments presented below ({\sc T-Cast}, {\sc T-Field},
etc.), we make the implicit assumption that the context $\Gamma$ is
consistent; if it is inconsistent, the rule cannot be used and the
type of the given expression cannot be established (type-checking
fails).

\eat{
The following rules govern existential constraints:
\infrule[Exists-R]
  {\Gamma\vdash {\tt t}:{\tt T} \andalso \Gamma\vdash{\tt c}[{\tt t}/{\tt x}]}
  {\Gamma\vdash \exists {\tt x}:{\tt T}.~{\tt c}}

\infrule[Exists-L]
  {\Gamma,{\tt x}:{\tt T},{\tt c}\vdash {\tt d} \andalso {\tt x}~\rm fresh}
  {\Gamma,\exists {\tt x}:{\tt T}.~{\tt c}\vdash {\tt d}}
}

\paragraph{4. Typing.} The typing rules are specified in Figure~\ref{fig:FX}.

{\sc T-Var} is as expected, except that it asserts the constraint {\tt
self==x}, which records that any value of this type is known
statically to be equal to {\tt x}. This constraint is actually very
crucial---as we shall see in the other rules, once we establish that
an expression {\tt e} is of a given type {\tt T}, we ``transfer'' the
type to a freshly chosen variable {\tt z}.  If, in fact, {\tt e} has a
static ``name'' {\tt x} (i.e., {\tt e} is known statically to be
equal to {\tt x}; that is, it has type {\tt T\{self==x\}}), then
{\sc T-Var} lets us assert that {\tt z:T\{self==x\}}, i.e., that {\tt z}
equals {\tt x}.
Thus {\sc T-Var} provides an important base case for
reasoning statically about equality of values in the environment.

We do away with the three cast rules in \FJ{} in favor of a single
cast rule, requiring only that {\tt e} be of some type {\tt U}.  At run time,
{\tt e} will be checked to see if it is actually of type {\tt T} (see
{\sc R-Cast} in Figure~\ref{fig:sos}).

{\sc T-Field} may be understood through ``proxy'' reasoning as
follows:  Given the context $\Gamma$, assume the receiver {\tt e} can
be established to be of type {\tt T}. Now, we do not know the run-time
value of {\tt e}, so we shall assume that it is some fixed but unknown
``proxy'' value {\tt x} (of type {\tt T}) that is ``fresh'' in that it
is not known to be related to any known value (i.e., those recorded
in $\Gamma$).  If we can establish that {\tt x} has a field {\tt f} of
type {\tt U}\footnote{Note from the definition of
\fields{} in Figure~\ref{fig:lookup} that all occurrences of
\this{} in the declared type of the field {\tt f} will have been replaced
by {\tt x}.}, then we can assert that
{\tt e.f} has type {\tt U} and, further, that it equals {\tt x.f}.
Hence, we can assert that {\tt e.f} has type 
$\exists {\tt x}:{\tt T}.~{\tt U}\{\self={\tt x}.{\tt f}\}$.

{\sc T-Invk} has a similar structure to {\sc T-Field}: we use
proxy reasoning for the receiver and the arguments of the method
call. {\sc T-New} also uses the same proxy reasoning: however in this case
we can establish that the resulting value is equal to $\new~{\tt C}(\tbar{y})$
for some values $\bar{\tt y}$ of the given types.

{\sc OK-Method} and {\sc OK-Class} ensure that the types and constraints occurring in a program are well-formed. Following from \FJ{}, these rules do not preclude the existence of cycles in the type declarations. We assume they are acyclic. {\sc OK-Method} checks that the actual type {\tt W} of the method body {\tt e} is a subtype of its declared return type {\tt V}. {\sc OK-Class} makes sure all methods of the class type check.


\paragraph{5. Subtyping.} The subtyping rules in Figure~\ref{fig:subtyping} are straightforward.


\begin{figure*}
\centering
\begin{tabular}{r@{\quad}rcl}
  (Program) & {\tt P} &{::=}& $\tbar{L}$ \\
  (Class declaration) & {\tt L} &{::=}& $\tt class~C(\tbar{f}:\tbar{T})\{c\}~extends~D~\{~\tbar{M}~\}$ \\
  (Method declaration)& {\tt M} &{::=}& $\tt def~m(\tbar{x}:\tbar{T})\{c\}:T=e;$ \\
  (Expression)& {\tt e} &{::=}& $\tt x$ \alt $\tt e.f$ \alt $\tt\new~C(\tbar{e})$ \alt $\tt e.m(\tbar{e})$ \alt $\tt e~\as~T$ \\
  (Type)& {\tt T} &{::=}& $\tt C$ \alt $\tt T\{c\}$ \alt $\tt \exists x:T.~T$ \\
  (Constraint term) & {\tt t} &{::=}& $\tt x$ \alt $\tt t.f$ \alt $\tt\new~C(\tbar{t})$ \\
  (Constraint) & {\tt c} &{::=}& $\true$ \alt $\tt t==t$ \alt $\tt c,c$ \alt $\tt \exists x:T.~c$ \\
\end{tabular} 
\caption{\FX{} productions.
{\tt C} ranges over class names, {\tt f} over field names, {\tt m} over method names, {\tt x} over variable names.}
\label{fig:fx-grammar}
\end{figure*}


\begin{figure*}
\begin{minipage}{.5\textwidth}
\quad\typicallabel{XXXXXX}
\infrule[\RField]
	{\fields({\tt C})=\tbar{f}:\tbar{T}}
	{\new~{\tt C}(\tbar{t}).{\tt f}_i \derives {\tt t}_i}

\infrule[\RCField]
	{{\tt e}\derives {\tt e}'}
	{{\tt e}.{\tt f}\derives {\tt e}'.{\tt f}}

\infrule[\RCast]
	{\vdash {\tt C}\{\self==\new~{\tt C}(\tbar{t})\}\subtype {\tt T}}
	{\new~{\tt C}(\tbar{t})~\as~{\tt T}\derives\new~{\tt C}(\tbar{t})}

\infrule[\RCCast]
	{{\tt e}\derives {\tt e}'}
	{{\tt e}~\as~{\tt T}\derives {\tt e}'~\as~{\tt T}}

\end{minipage}%
\begin{minipage}{.5\textwidth}
\quad\typicallabel{XXXXXX}
\infrule[\RInvk]
	{{\tt C}~\has~{\tt m}(\tbar{x}:\tbar{T})\{{\tt c}\}:{\tt U}={\tt e}}
	{\new~{\tt C}(\tbar{t}).{\tt m}(\tbar{a})\derives {\tt b}[\new~{\tt C}(\tbar{t}),\tbar{a}/\this,\tbar{x}]}

\infrule[\RCInvkRecv]
	{{\tt e}\derives {\tt e}'}
	{{\tt e}.{\tt m}(\tbar{a})\derives {\tt e}'.{\tt m}(\tbar{a})}

\infrule[\RCInvkArg]
	{{\tt a}_i\derives {\tt a}'_i}
	{{\tt e}.{\tt m}(\ldots,{\tt a}_i,\ldots)\derives {\tt e}.{\tt m}(\ldots,{\tt a}'_i,\ldots)}

\infrule[\RCNewArg]
	{{\tt e}_i\derives {\tt e}'_i}
	{\new~{\tt C}(\ldots,{\tt e}_i,\ldots)\derives\new~{\tt C}(\ldots,{\tt e}'_i,\ldots)}
\end{minipage}
\caption{\FX{} operational semantics}
\label{fig:sos}
\end{figure*}


\begin{figure*}
\begin{minipage}{.5\textwidth}
\quad\typicallabel{XXXXXX}
\infax[W-Object]
  {\wj{}{\tt Object}}

\infrule[W-Class]
  {{\tt class}~{\tt C}(\tbar{f}:\tbar{T})\{{\tt c}\}~{\tt extends}~{\tt D}~\{~\tbar{M}~\} \in {\tt P}}
  {\wj{}{\tt C}}
\end{minipage}%
\begin{minipage}{.5\textwidth}
\quad\typicallabel{XXXXXX}
\infrule[W-Dep]
        {\wj{\Gamma}{\tt T} \andalso \cj{\Gamma, \self \ty {\tt T}}{\tt c}}
	{\wj{\Gamma}{{\tt T}\{{\tt c}\}}}

\infrule[W-Exists]
        {\wj{\Gamma}{\tt T} \andalso \wj{\Gamma, {\tt x} \ty {\tt T}}{\tt U}}
        {\wj{\Gamma}{\exists {\tt x} \ty {\tt T}.~{\tt U}}}
\end{minipage}
\caption{\FX{} well-formed types}
\label{fig:well}
\end{figure*}

\begin{figure*}
\begin{minipage}{.5\textwidth}
\quad\typicallabel{XXXXXX}
\infax[L-Fields-Object]
  {\vdash\fields({\tt Object})=\bullet}
\end{minipage}%
\begin{minipage}{.5\textwidth}
\quad\typicallabel{XXXXXX}
\infrule[L-Field-B]
  {\vdash\fields({\tt C})=\tbar{f}:\tbar{T}}
  {{\tt C}~\has~{\tt f}_i:{\tt T}_i}

\end{minipage}

\begin{minipage}{\textwidth}
\quad\typicallabel{XXXXXX}
\infrule[L-Fields-I]
  {{\tt class}~{\tt C}(\tbar{f}:\tbar{T})\{{\tt c}\}~{\tt extends}~{\tt D}~\{~\tbar{M}~\}\in {\tt P}
   \andalso
   \vdash\fields({\tt D})=\tbar{g}:\tbar{U}}
   {\vdash\fields({\tt C})=\tbar{g}: \tbar{U}, \tbar{f}: \tbar{T}}

\infrule[L-Method-B]
  {{\tt class}~{\tt C}(\tbar{f}:\tbar{T})\{{\tt c}\}~{\tt extends}~{\tt D}~\{~\tbar{M}~\}\in {\tt P}
   \andalso
   {\tt def}~{\tt m}(\tbar{x}\ty \tbar{U})\{{\tt d}\}\ty {\tt V}={\tt e}\in \tbar{M}}
  {\vdash {\tt C}~\has~{\tt m}(\tbar{x}:\tbar{U})\{{\tt d}\}:{\tt V}={\tt e}}

\infrule[L-Method-I]
  {{\tt class}~{\tt C}(\tbar{f}:\tbar{T})\{{\tt c}\}~{\tt extends}~{\tt D}~\{~\tbar{M}~\}\in {\tt P}
   \andalso
   \vdash {\tt D}~\has~{\tt m}(\tbar{x}:\tbar{U})\{{\tt d}\}:{\tt V}={\tt e}
   \andalso
   {\tt m}\not\in\tbar{M}}
  {\vdash {\tt C}~\has~{\tt m}(\tbar{x}:\tbar{U})\{{\tt d}\}:{\tt V}={\tt e}}
\end{minipage}

\begin{minipage}{.5\textwidth}
\quad\typicallabel{XXXXXX}
\infrule[L-Member-C,E]
  {\Gamma,{\tt x}:{\tt T}\vdash {\tt x}~\underline\has~{\tt I} \andalso {\tt y}~\rm fresh}
  {\Gamma,{\tt x}:{\tt T}\{{\tt c}\}\vdash {\tt x}~\underline\has~{\tt I} \\
   \Gamma,{\tt x}:\exists {\tt y}:{\tt U}.~{\tt T}\vdash {\tt x}~\underline\has~{\tt I}}
\end{minipage}%
\begin{minipage}{.5\textwidth}
\quad\typicallabel{XXXXXX}
\infrule[L-Member-B]
  {{\tt C}~\has~{\tt I}}
  {{\tt x}:{\tt C}\vdash {\tt x}~\underline\has~{\tt I}[{\tt x}/\this]}
\end{minipage}
\caption{\FX{} member lookup}
\label{fig:lookup}
\end{figure*}

\begin{figure*}
\begin{minipage}{\textwidth}
\quad\typicallabel{XXXXXX}
\infrule[X-Proj]
  {\sigma(\Gamma)\vdashX{\tt c}}
  {\Gamma\vdash {\tt c}}

\infrule[X-Sel]
  {\vdash\fields({\tt C})=\tbar{f}:\tbar{T}
  	\andalso
  	\Gamma\vdash\new~{\tt C}(\tbar{t}):{\tt U}
  	\andalso
  	\Gamma, \new~{\tt C}(\tbar{t}).{\tt f}_i=={\tt t}_i\vdash {\tt c}}
  {\Gamma\vdash {\tt c}}

\infrule[X-Inv]
  {{\tt class}~{\tt C}(\tbar{f}:\tbar{T})\{{\tt c}\}~{\tt extends}~{\tt D}~\{~\tbar{M}~\}\in {\tt P}
   \andalso
   \Gamma\vdash {\tt t}:{\tt U},{\tt U}\subtype{\tt C}\andalso \Gamma,{\tt c}[{\tt t}/\this]\vdash {\tt d}}
  {\Gamma\vdash {\tt d}}
\end{minipage}%
\caption{\FX{} object constraint system}
\label{fig:object}
\end{figure*}

\begin{figure*}
\begin{minipage}{\textwidth}
\quad\typicallabel{XXXXXX}
\infax[T-Var]
  {\Gamma,{\tt x}:{\tt T}\vdash {\tt x}:{\tt T}\{\self=={\tt x}\}}

\infrule[T-Cast]
	{\Gamma\vdash {\tt e}:{\tt U}\andalso\wj{\Gamma}{\tt T}}
	{\Gamma\vdash {\tt e}~\as~{\tt T}:{\tt T}}

\infrule[T-Field]
	{\Gamma\vdash {\tt e}:{\tt T} \andalso \Gamma,{\tt x}:{\tt T}\vdash {\tt x}~\underline\has~{\tt f}:{\tt U} \andalso {\tt x}~\rm fresh}
	{\Gamma\vdash {\tt e}.{\tt f}:\exists {\tt x}:{\tt T}.~{\tt U}\{\self=={\tt x}.{\tt f}\}}

\infrule[T-Invk]
	{\Gamma\vdash {\tt e}:{\tt T},\tbar{a}:\tbar{U} \andalso 
	  \Gamma,{\tt x}:{\tt T},\tbar{y}:\tbar{U}\vdash {\tt x}~\underline\has~{\tt m}(\tbar{y}:\tbar{V})\{{\tt d}\}:{\tt W}={\tt b},
		\tbar{U}\subtype\tbar{V},{\tt d} \andalso {\tt x},\tbar{y}~\rm fresh}
	{\Gamma\vdash {\tt e}.{\tt m}(\tbar{a}):\exists {\tt x}:{\tt T}.~\exists\tbar{y}:\tbar{U}.~{\tt W}}

\infrule[T-New]
	{{\tt class}~{\tt C}(\tbar{f}:\tbar{T})\{{\tt c}\}~{\tt extends}~{\tt D}~\{~\tbar{M}~\}\in {\tt P}
		\andalso
		\fields({\tt C})=\tbar{g}:\tbar{U} \\
	  \Gamma\vdash\tbar{e}:\tbar{V} \andalso
    \Gamma,{\tt x}:{\tt C},\tbar{y}:\tbar{V},{\tt x}.\tbar{g}==\tbar{y}\vdash
    \tbar{V}\subtype\tbar{U}[{\tt x}/\this],{\tt c}[{\tt x}/\this] \andalso {\tt x},\tbar{y}~\rm fresh}
	{\Gamma\vdash\new~{\tt C}(\tbar{e}):\exists\tbar{y}:\tbar{V}.~{\tt C}\{\self==\new~{\tt C}(\tbar{y})\}}
        
\infrule[OK-Method]
  {{\tt class}~{\tt C}(\tbar{f}:\tbar{T})\{{\tt c}\}~{\tt extends}~{\tt D}~\{~\tbar{M}~\}\in {\tt P} \andalso
    {\tt def}~{\tt m}(\tbar{x}:\tbar{U})\{{\tt d}\}:{\tt V}={\tt e}\in\tbar{M}\\
    \this:{\tt C},\tbar{x}:\tbar{U}\vdash\tbar{U}~\type,{\tt d}~\constraint
    \andalso
    \this:{\tt C},\tbar{x}:\tbar{U},{\tt d}\vdash {\tt e}:{\tt W},{\tt W}\subtype {\tt V}}
  {\vdash {\tt def}~{\tt m}(\tbar{x}:\tbar{U})\{{\tt d}\}:{\tt V}={\tt e}~{\rm OK~in}~{\tt C}}

\infrule[OK-Class]
  {\this:{\tt C},\tbar{f}:\tbar{T}\vdash\tbar{T}~\type,{\tt c}~\constraint \andalso \wj{}{\tt D} \andalso \tbar{M}~{\rm OK~in}~{\tt C}}
  {\vdash {\tt class}~{\tt C}(\tbar{f}:\tbar{T})\{{\tt c}\}~{\tt extends}~{\tt D}~\{~\tbar{M}~\}~\rm OK}
\end{minipage}
\caption{\FX{} typing rules}\label{fig:FX}
\end{figure*}


\begin{figure*}
\begin{minipage}{.333\textwidth}
\quad\typicallabel{XXX}
\infrule[S-Refl]
  {\wj{\Gamma}{\tt T}}
  {\Gamma\vdash {\tt T}\subtype {\tt T}}
\end{minipage}%
\begin{minipage}{.333\textwidth}
\quad\typicallabel{XXX}
\infrule[S-Trans]
	{\Gamma\vdash {\tt T}\subtype {\tt U}, {\tt U}\subtype {\tt V}}
	{\Gamma\vdash {\tt T}\subtype {\tt V}}
\end{minipage}%
\begin{minipage}{.333\textwidth}
\quad\typicallabel{XXX}
\infrule[S-Const-L]
	{\Gamma,{\tt c}\vdash{\tt T}\subtype {\tt U}}
	{\Gamma\vdash {\tt T}\{{\tt c}\}\subtype {\tt U}}
\end{minipage}%

\begin{minipage}{.5\textwidth}
\quad\typicallabel{XXXXX}
\infrule[S-Class]
  {{\tt class}~{\tt C}(\tbar{f}:\tbar{T})\{{\tt c}\}~{\tt extends}~{\tt D}~\{~\tbar{M}~\}\in {\tt P}}
  {\Gamma\vdash {\tt C}\subtype {\tt D}}

\infrule[S-Exists-L]
  {\wj{\Gamma}{\tt T} \andalso \Gamma,{\tt x}:{\tt T}\vdash {\tt U}\subtype {\tt V} \andalso {\tt x}~\rm fresh}
  {\Gamma\vdash\exists {\tt x}:{\tt T}.~{\tt U}\subtype {\tt V}}
\end{minipage}%
\begin{minipage}{.5\textwidth}
\quad\typicallabel{XXXXX}
\infrule[S-Const-R]
	{\wj{\Gamma}{{\tt U}\{{\tt c}\}}\andalso\Gamma,\self:{\tt T}\vdash {\tt c},{\tt T}\subtype {\tt U} }
	{\Gamma\vdash {\tt T}\subtype {\tt U}\{{\tt c}\}}

\infrule[S-Exists-R]
  {\Gamma\vdash {\tt t}:{\tt T} \andalso \Gamma\vdash {\tt U}\subtype {\tt V}[{\tt t}/{\tt x}]}
  {\Gamma\vdash {\tt U}\subtype\exists {\tt x}:{\tt T}.~{\tt V}}
\end{minipage}        
\caption{\FX{} subtyping rules}\label{fig:subtyping}
\end{figure*}


\begin{figure*}
\begin{minipage}[b]{.5\textwidth}
\centering
\begin{tabular}{r@{\quad}rcl}
  (Expression)& {\tt e} &{::=}& ${\tt C}\{{\tt c}\}$ \\
  (Path)& {\tt p} &{::=}& ${\tt x}$ \alt {\tt p}.{\tt f} \\
  (Type)& {\tt T} &{::=}& ${\tt p}$ \alt \type \\
  (Constraint term)& {\tt t} &{::=}& ${\tt T}$ \\
  (Constraint) & {\tt c} &{::=}& ${\tt T}\extends {\tt T}$
\end{tabular}
\typicallabel{XXX}
\infrule[L-Extends]
	{\Gamma\vdash {\tt T}\extends{\tt U}
         \andalso
         \Gamma,{\tt x}:{\tt U}\vdash {\tt x}~\underline\has~{\tt I}}
	{\Gamma,{\tt x}:{\tt T}\vdash {\tt x}~\underline\has~{\tt I}}

\infrule[S-Extends]
	{\Gamma\vdash {\tt T}\extends {\tt U} \andalso \Gamma\vdash{\tt T}:\type,{\tt U}:\type}
	{\Gamma\vdash {\tt T}\subtype {\tt U}}
\end{minipage}%
\begin{minipage}[b]{.5\textwidth}
\quad\typicallabel{XXX}
\infax[W-Type]
	{\wj{}{\type}}

\infax[W-Var]
	{\wj{\Gamma,{\tt x}:\type}{\tt x}}

\infrule[W-Path]
	{\Gamma\vdash {\tt p}:{\tt T}
         \andalso
         \Gamma,{\tt x}:{\tt T}\vdash {\tt x}~\underline\has~{\tt f}:\type}
	{\wj{\Gamma}{{\tt p}.{\tt f}}}

\infrule[T-Type]
        {\wj{\Gamma}{\tt C\{{\tt c}\}}}
  {\Gamma\vdash{\tt C}\{{\tt c}\}:\type\{\self=={\tt C}\{{\tt c}\}\}}
\end{minipage}
\caption{\FXG}
\label{fig:FXG}
\end{figure*}

\subsection{\FXD}

Turning \FXZ{} into a language with value-dependent types is straightforward
since the construction of the previous section is parametric in the underlying constraint system $\mathcal{X}$ and constraint propagation is already built into the typing rules.

First, we assume we are given a constraint system $\cal A$ with a vocabulary of primitive types ${\tt A}$,
predicates ${\tt p}$, and functions ${\tt f}$.  Zero arity
functions encode constants.

Second, we extend the productions of \FXZ{} as follows.
\begin{center}
\begin{tabular}{r@{\quad}rcl}
  (Type)& {\tt T} &{::=}& {\tt A} \\
  (Expression) & {\tt e} &{::=}& ${\tt f}(\tbar{e})$ \\
  (Constraint term) & {\tt t} &{::=}& ${\tt f}(\tbar{t})$ \\
  (Constraint) & {\tt c} &{::=}& ${\tt p}(\tbar{t})$ \\  
\end{tabular}
\end{center}

The obvious rule is needed to type function calls.

\infrule[T-Fun]
	{{\tt f}{\rm~has~type~\tbar{A} \rightarrow {\tt B}{\rm~in~}\mathcal{A}} \andalso \Gamma\vdash\tbar{e}:\tbar{A}} 
	{\Gamma\vdash{{\tt f}(\tbar{e}):\exists\tbar{x}:\tbar{A}.~{\tt B}\{\self=={\tt f}(\tbar{x})\}}}

Third, we authorize users to write constraints in $\cal A$ (except for existential constraints) in programs.

For instance, if $\mathcal{A}$ defines the type {\tt int}, the constant {\tt 0} and the successor function {\tt s}, we can declare:

\begin{xten}
class Count(n:int) extends Object {
  def inc():Count{self.n==s(this.n)} =
  	new Count(s(this.n));
}
\end{xten}

\FXD corresponds to the \CFJ calculus presented
in prior work~\cite{constrained-types}.  As described there, \Xten
supports equality constraints and has been extended with constraint
systems for Presburger arithmetic and for set constraints over
\Xten's array sublanguage.

\subsection{\FXG}
We now turn to showing how \FGJ{}-style generics can be supported in the \FX{} family.
\FXG{} is the language obtained by adding to \FXZ{} the productions and deduction rules of Figure~\ref{fig:FXG}.

First we introduce the ``type'' \type. \FGJ{} method type
parameters are modeled in \FXG{} as normal parameters of type
\type.\footnote{In concrete \Xten{} syntax type parameters are
distinguished from ordinary value parameters through the use of
``square'' brackets. This is particularly useful in implementing type
inference for generic parameters. We abstract these concerns away in
the abstract syntax presented in this section.}  Generic class
parameters are modeled as ordinary fields of type \type, with
parameter bound information recorded as a constraint in the class
invariant. This decision to use fields rather than parameters is
discussed further in Section~\ref{sec:parameters-vs-fields}. In brief,
it permits powerful idioms using fixed but unknown types without
requiring ``wildcards''.

The set of well-formed types is now enhanced to permit some fixed but unknown
types {\tt x} as well as \emph{path types} (cf. \cite{scala}), that is use type-valued fields of objects as types.\footnote{But we will not permit invocations of methods with return type \type\ to be 
used as types. This does indeed make sense, but developing
this theory further is beyond the scope of this paper.} We extend $\sigma$ in the obvious way:
%
\begin{center}
\begin{tabular}{l}
$	\sigma({\tt x}:\type, \Gamma)=\sigma(\Gamma)$\\
$\sigma({\tt x}:{\tt y}, \Gamma)=\sigma(\Gamma)$\\
$\sigma({\tt x}:{\tt p}.{\tt f}, \Gamma)=\sigma(\Gamma)$
\end{tabular}
\end{center}
%
Reciprocally, we permit class types ${\tt C}\{{\tt c}\}$ to be used as expressions. We type them accordingly ({\sc T-Type}). In contrast, the ``type'' \type{} is neither a valid expression nor a class type: it has no field, method, subclass, or superclass. It may however be constrained as usual as for instance in rule ({\sc T-Type}), that is to say we permit equality constraints over types.

The key idea is that information about type-valued expressions can
be accumulated through constraints. Specifically we introduce 
the constraint ${\tt T}\extends{\tt U}$. It may be used, for
instance, to specify upper bounds on type variables or fields (path
types).\eat{\footnote{To support structural typing, we permit the
programmer to use {\tt x\ \has\ I} constraints, see
Section~\ref{sec:structural-typing}.}} We require the underlying constraint system $\mathcal{G}$ to treat $\extends$ as a partial order relation (reflexive, antisymmetric, and transitive).

As in the previous section, the final step of the construction consists in deciding which constraints are user constraints. In \FXG{}, users are permitted to specify equality and inequality constraints about type variables, fields, and class types.


\begin{example}
The \FGJ{} parametric method

\begin{xten} 
<T> T id(T x) { return x; }
\end{xten}
\noindent can be represented as
\begin{xten} 
def id(T: type, x: T): T = x;
\end{xten}
\end{example}

\begin{example}
\noindent The \FGJ{} class 
\begin{xten} 
class Comparator<B> {
  int compare(B y) { ... } }
class SortedList<T extends Comparator<T>> { 
  int m(T x, T y) {
     return x.compare(y); } }
\end{xten}
\noindent can be represented as
\begin{xtenmath} 
class Comparator(B: type) {
  def compare(y:B):int = ...; }
class SortedList(T: type)
    {T $\extends$ Comparator{self.B==T}} { 
  def m(x:T, y:T):int = x.compare(y); }
\end{xtenmath}
\end{example}

\paragraph{Type casts.} 
Consider the program:
\begin{xten}
class C() extends Object {}
class D(f:type, g:this.f) extends Object {}
\end{xten}
Class {\tt D} has a type parameter {\tt f} and a value field {\tt g} of type {\tt f}.
Thanks to contraints, if
${\tt e}=\new~{\tt D}({\tt C},\new~{\tt C}())$,
then expression ${\tt e}.{\tt g}$ can be shown to
have type {\tt C}.
In contrast $({\tt e}~\as~{\tt D}).{\tt g}$ has type
$\exists {\tt x}:{\tt D}.{\tt x}.{\tt f}\{\self=={\tt x.g}\}$.
The type of $({\tt e}~\as~{\tt D}).{\tt g}$ is essentially ``unknown''
because the cast erased all information about it. In \Xten, we choose to shield users from existential types and permit casts of the form $({\tt e}~\as~{\tt D}\{\self.{\tt f}=={\tt t}\})$ where {\tt t} is a type in scope (class type, type parameter, or path type).


\subsection{\FXGD} 

No additional rules are needed beyond those of \FXG{} and \FXD{}. This
language permits type and value constraints, supporting \FGJ{} style
generics and value-dependent types.

\subsection{Results}
The following results hold for \FXGD supposing the program {\tt P} type checks.

\begin{theorem}[Subject Reduction] If $\Gamma$ is well-formed and $\Gamma \vdash {\tt e:T}$ and ${\tt e} \derives {\tt e'}$, then
for some type {\tt S}, $\Gamma \vdash {\tt e':S},{S \subtype T}$.
\end{theorem}

We let values be of the form $\tt v ::= \new\ C(\bar{\tt v}) \alt C\{c\}$.

\begin{theorem}[Progress]
If $\vdash {\tt e:T}$ then one of the following conditions holds:
\begin{enumerate}
\item {\tt e} is a value,
\item {\tt e} contains a cast sub-expression that is stuck,
\item there exists an $\tt e'$ s.t. $\tt e\derives e'$.
\end{enumerate}
\end{theorem}

\begin{theorem}[Type soundness]
If $\vdash {\tt e:T}$ and {\tt e}
reduces to a normal form ${\tt e'}$, then
either $\tt e'$ is a value {\tt v} and $\vdash {\tt v:S},{\tt S\subtype T}$ or
${\tt e'}$ contains  a stuck cast sub-expression.
\end{theorem}






\eat{
\begin{figure*}
\centering
\begin{tabular}{r@{\quad}rcl}
  (Member) & {\tt I} &{::=}& ${\tt f}: {\tt T}$ \alt ${\tt m}(\tbar{x} : \tbar{T})\{{\tt c}\}: {\tt T} = {\tt e}$ \\
  (Hypothesis) & {\tt h} &{::=}& ${\tt c}$ \alt ${\tt x}:{\tt T}$ \\
  (Consequence) & {\tt k} &{::=}& ${\tt c}$ \alt ${\tt e}:{\tt T}$ \alt ${\tt x\ \has\ I}$ \alt $\fields({\tt x})=\tbar{f}:\tbar{T}$ \alt ${\tt T}\subtype{\tt T}$ \\
  (Judgment) & {\tt j} &{::=}& $\tbar{h}\vdash\tbar{k}$
\end{tabular} 
\caption{\FX{} judgments}
\end{figure*}
}




\paragraph{Proof sketch (Subject Reduction).} We proceed by induction on the last rule used in the proof of ${\tt e} \derives {\tt e'}$. The key case is {\sc R-Invk}. Because of rule 


\begin{itemize}
\item {\sc RC-Field}. If $\Gamma\vdash S<:T$ then the fields of $T$ are the first fields of $S$.
\item {\sc RC-Invk-Recv}. If $\Gamma\vdash S<:T$ and $\Gamma,x:T\vdash x~\underline\has~m(y:U)\{c\}:V=a$ then there exists $d$ and $W$ such that $\Gamma,x:S\vdash x~\underline\has~m(y:U)\{d\}:W=b$ and $d$ entails $c$ and $W<:V$.
\item {\sc RC-Cast}. Straightforward.
\item {\sc RC-New-Arg}. If $\Gamma\vdash S<:T$ then $\exists x:T.U<:\exists x:S.U$.
\item {\sc RC-Invk-Arg}. Same.
\item {\sc R-Invk}. If $\Gamma,x:C\vdash x~\has~m(y:U)\{c\}:V=a$ then for all $d$ and $W$ such that $\Gamma,x:C\vdash x~\underline\has~m(y:U)\{d\}:W=b$ it is the case that $c$ entails $d$ and $V<:W$.
\item {\sc R-Field}. We have $\Gamma\vdash t:V$ for some $V$, $\Gamma\vdash \new~C(t).f_i:W_i\{\exists x:(\exists y:V.C\{\self==\new~C(y)\}).\self==x.f_i\}$ where $\Gamma\vdash C~\has~f_i:W_i$. We prove $\Gamma\vdash V_i<:W_i\{\exists x:(\exists y:V.C\{\self==\new~C(y)\}).\self==x.f_i\}$.
\item {\sc R-Cast}. Straightforward.
\end{itemize}
