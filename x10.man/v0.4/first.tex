% First page

\thispagestyle{empty}

% \todo{"another" report?}

\topnewpage[{
\begin{center}   
{\huge\bf Report on the Experimental Language \Xten{}}
\vskip 1ex
$$
\begin{tabular}{l@{\extracolsep{.5in}}lll}
\multicolumn{4}{c}{\sc ***DRAFT v 0.407: Please do not cite.***}\\
\multicolumn{4}{c}{\sc Please send comments to 
V\authorsc{IJAY} S\authorsc{ARASWAT} at 
{\tt vsaraswa@us.ibm.com}}\\
\multicolumn{4}{c}{({\sc IBM Confidential})}

%\ldots
\end{tabular}
$$
\vskip 2ex
% {\it Dedicated to the Memory of APL} % vj
{\bf April 9, 2005}
\vskip 2.6ex
\end{center}


}]


\chapter*{Summary}
This draft report provides an initial description of the programming
language \Xten. \Xten{} is a single-inheritance class-based object-oriented
(OO) programming language designed for high-performance, high-productivity
computing on high-end computers supporting $O(10^5)$ hardware threads
and $O(10^{15})$ operations per second. 

{}\Xten{} is based on state-of-the-art object-oriented programming
languages and deviates from them only as necessary to support its
design goals. The language is intended to have a simple and clear
semantics and be readily accessible to mainstream OO programmers. It
is intended to support a wide variety of concurrent programming
idioms.
%, incuding data parallelism, task parallelism, pipelining.
%producer/consumer and divide and conquer.

This document provides an initial description of the language and
corresponds to the first implementation of the language.  We expect to
revise this document in the light of experience gained in implementing
and using this language.

The \Xten{} design team consists of 
D\authorsc{AVID} B\authorsc{ACON}, 
B\authorsc{OB} B\authorsc{LAINEY}, 
P\authorsc{HILIPPE} C\authorsc{HARLES}, 
P\authorsc{ERRY} C\authorsc{HENG}, 
C\authorsc{HRISTOPHER} D\authorsc{ONAWA}, 
J\authorsc{ULIAN} D\authorsc{OLBY}, 
K\authorsc{EMAL} E\authorsc{BCIO\u{G}LU},
P\authorsc{ATRICK} G\authorsc{ALLOP}, 
C\authorsc{HRISTIAN} G\authorsc{ROTHOFF}, 
A\authorsc{LLAN} K\authorsc{IELSTRA}, 
R\authorsc{OBERT} O'\authorsc{CALLAHAN}, 
F\authorsc{ILIP} P\authorsc{IZLO}, 
V.T.~R\authorsc{AJAN}, 
V\authorsc{IJAY} S\authorsc{ARASWAT} (contact author), 
V\authorsc{IVEK} S\authorsc{ARKAR},
C\authorsc{HRISTOPH von} P\authorsc{RAUN}
and 
J\authorsc{AN} V\authorsc{ITEK}.

For extended discussions and support we would like to thank: Calin
Cascaval, Elmootaz Elnozahy, John Field, Bob Fuhrer, Orren Krieger,
John McCalpin, Paul McKenney, Ram Rajamony, Frank Tip and Mandana
Vaziri.

We also thank Jonathan Rhees and William Clinger with help in
obtaining the \LaTeX{} style file and macros used in producing the
Scheme report, after which this document is based. We also acknowledge
the influence of the Java Language Specification \cite{jls2} document,
as evidenced by the numerous citations in the text.

This document is a revision of the {\cf 0.32} version of the Report,
released on 17 July 2004. It documents the language corresponding to
the first version of the implementation.  The implementation was done
by P\authorsc{HILIPPE} C\authorsc{HARLES}, 
C\authorsc{HRISTOPHER} D\authorsc{ONAWA}, 
C\authorsc{HRISTIAN} G\authorsc{ROTHOFF}, 
V\authorsc{IJAY} S\authorsc{ARASWAT},
C\authorsc{HRISTOPH von} P\authorsc{RAUN}, and 
K\authorsc{EMAL} E\authorsc{BCIO\u{G}LU}.



%\vfill
%\begin{center}
%{\large \bf
%*** DRAFT*** \\
%%August 31, 1989
%\today
%}\end{center}

\vfill
\eject


\chapter*{Contents}
\addvspace{3.5pt}                  % don't shrink this gap
\renewcommand{\tocshrink}{-3.5pt}  % value determined experimentally
{\footnotesize
\tableofcontents
}

\vfill
\eject


