\chapter{Names and packages}
\label{packages} \index{names}\index{packages}

\Xten{} supports \java's mechanisms for names and packages \cite[\S
6,\S 7]{jls2}, including \xcd"public", \xcd"protected", \xcd"private"
and package-specific access control.

\begin{grammar}
TypeName   \: Identifier \\
        \| TypeName \xcd"." Identifier \\
        \| PackageName \xcd"." Identifier \\
PackageName   \: Identifier \\
        \| PackageName \xcd"." Identifier \\
\end{grammar}


\section{Naming conventions}

While not enforced by the compiler, classes and interfaces
in the \Xten{} library support the following naming conventions.
Names of types---including classes,
\iftypeparams\else
type properties,
\fi
type
parameters, and types specified by type definitions---are in
CamelCase and begin with an uppercase letter.  For backward
compatibility with languages such as C and \java{}, type
definitions are provided to allow primitive value types
such as \xcd"int" and \xcd"boolean" to be written in lowercase.
Names of methods, fields, value properties, and packages are in camelCase and begin with a lowercase letter.
Names of \xcd"const" fields are in all uppercase with words
separated by an ``\xcd"_"''.


