\chapter{Places}
\label{XtenPlaces}\index{places}

An \Xten{} place is a repository for data and activities. Each place is to
be thought of as a locality boundary: the activities running in a
place may access data items located at that place with the efficiency
of on-chip access. Accesses to remote places may take orders of
magnitude longer.

In \XtenCurrVer{}, the set of places available to a computation is
determined at the time that the program is run and remains fixed
through the run of the program. The number of places available to a
computation may be determined by querying a run-time int constant
({\cf place.MAX\_PLACES}).

All scalar objects created during program execution are located in one
place, though they may be accessed from other places. Aggregate
objects (arrays) may be distributed across multiple places using
distributions.

The set of all places in a running instance of an \Xten{} program is
denoted by {\cf place.places}. (This set may be used to define
distributions, for instance, \S~\ref{XtenDistributions}.)

\todo{{}\Xten{} provides a built-in value class, {\tt x10.lang.place}; all
places are instances of this class.  This class is {\tt final} in
{}\XtenCurrVer; future versions of the language may permit
user-definable places.  Since places play a dual role (values
as well as types), variables of type {\cf place} must be initialized
and are implicitly {\tt final}.}

\todo{Introduce {\cf placeSet}? Alternative is to define distributions.places as a built-in unique distribution and use distributions wherever sets of places are expected.}

The set of all places is totally ordered. Places may be used as keys
in hash-tables. Like a value object, a place is ``unlocated''.

\Xten{} permits user-definable place constants (= final variables =
variables that must be assigned before use, and can be assigned only
once). Place constants may be used in type expressions after the {\cf @}
sign. For instance, consider the following class definition:

\begin{x10}
 public class Cell<BaseType@P> \{
  BaseType@P value;

  public Cell(BaseType@P value) \{  
     this.value := value;
  \}

  public BaseType@P getValue() \{
     return this.value;
  \}

  public void setValue(BaseType@P value) \{
     this.value := value;
  \}
 \}
\end{x10}

This class may be used thus:

\begin{x10}
Cell<Point@Q> cell = 
   new Cell<Point@Q>(new Point@Q());
\end{x10}

{\cf Cell} is a generic class whose single located type parameter
specifies type and location information. At runtime, {\cf BaseType}
will be replaced by an unlocated type (e.g.{} a class or an interface)
and {\cf P} will be replaced by a place constant (e.g. {\cf
here}). {\cf P} may be used in the body of {\cf Cell} anywhere a place
expression may be used. See \S~\ref{XtenTypes}.

