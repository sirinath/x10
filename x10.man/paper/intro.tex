
%
%I. Introduction and overview (3 pages)

%Design of X10, concurrency, high productivity, high performance,
%practical language.
%
%Dependent types arise naturally. arrays, regions, distributions, place
%types.
%
%Indeed you can look around and recognize many OO type systems proposed
%in the last decade or so as specific kinds of applied dependent type
%systems.
%
%Our goal is to develop a general framework for dependent types for
%statically typed OO languages ("Java-like languages"). 

\Xten{} is a modern statically typed object-oriented (OO) language
designed for high productivity in the high performance computing (HPC)
domain~\cite{X10}. Built essentially on the imperative sequential
$\mbox{Java}^{\mbox{\scriptsize\sc tm}}$ core, \Xten{} introduces constructs for
distribution and fine-grained concurrency (asynchrony, atomicity,
ordering).

\Xten{}, like most OO languages is designed around
the notion of objects, as instances of {\em classes}. However, \Xten{}
places equal emphasis on {\em arrays}, a central data-structure in
high performance computing. In particular, \Xten{} supports dense,
distributed multi-dimensional arrays of value and reference types,
built over index sets known as {\em regions}, and mappings from index
sets to places, known as {\em distributions}.  \Xten{} supports a rich
algebra of operations over regions, distributions and arrays.

A central design goal of \Xten{} is to rule out large classes of error
by design. Thus the possibility of indexing a 2-d array with 3-d
points should simply be ruled out at compile-time. This means that one
must permit the programmer to express types such as such as {\tt
region(2)}, the type of all two-dimensional regions, {\tt int[5]}, the
type of all arrays of {\tt int} of length {\tt 5}, {\tt
int[region(2)]}, the type of all {\tt int} arrays over two dimensional
regions, and {\tt Obj!}, the type of all {\tt Obj} located on the
current node. For concurrent computations, one needs the ability to
statically check that a method is being invoked by an activity that is
registered with a given clock (i.e., dynamic barrier)~\cite{X10}.

Thus one needs a practical and usable framework for {\em dependent
types} for concurrent, imperative, object-oriented
languages. Dependent types are types parametrized by {\em values}
\cite{dependent-types}  and have a rich and distinguished history in
logic and functional programming languages (see e.g.,
\cite{xi99dependent,ocrz-ecoop03,aspinall-attapl,cayenne,epigram-matter}, 
e.g., in the development of logical frameworks
\cite{calc-constructions}.  In the current setting, several questions
must be answered: How are dependent types defined? What vocabulary of
functions and predicates can be used in forming dependent types? Is
this vocabulary extensible?  How do dependent types interact with
inheritance, method overloading and overriding, type casting? Are
types checked statically? Can classes continue to be compiled separately in the presence of dependent types?

We believe these questions must be answered in a way that achieves
certain desirable properties:
\begin{itemize}
\item 
{\bf Ease of use.}  The framework must be easy to use for practicing
programmers. In particular, the syntax of types should be a simple and
natural extension of existing syntax for types.

\item
{\bf Flexibility.}
The framework
must permit the development of concrete,
specific type systems tailored to the application area at
hand, enabling a kind of pluggable type system~\cite{bracha04-pluggable}.
Hence, the framework must be parametric in the kinds of
expressions used in the type system.

\item
{\bf Modularity.}
The rules for type-checking
must be specified once in a way that is independent of the
particular vocabulary of operations used in the dependent type
system.

\item
{\bf Integration with OO languages.}
The framework must work smoothly with nominal type systems found in
Java-like languages. It must permit separate compilation.

\item
{\bf Static checking.}  The framework must permit mostly static
type-checking. (The user should be able to escape the confines of
static type-checking using dynamic casts, as is common for Java-like
languages.)
\end{itemize}

\subsection{Overview}

In this paper we develop a general syntactic and semantic framework
for {\em constrained types}, user-defined constraint-based types for
modern class-based OO languages such as \java{}, \csharp{} and
\Xten{}. Our central insight is that a rich, user-extensible type
system can be developed on top of predicates over the {\em immutable}
state of objects. Such types statically capture many common invariants
naturally arising in code. For instance, typically the shape of an
array (number of dimensions, size of each dimension) is determined at
runtime but fixed once the array is constructed. Thus the shape of an
array is part of its immutable state.

\paragraph{Properties.}
Our first step is to permit a class specification to specify {\em
properties}. A property is a {\tt public} final instance field of the
class that cannot be overridden by subclassing. Like any other field,
a property is typed, and its type need not necessarily be
primitive. Thus properties capture the immutable public state of
object---initialized when the object is created---that can be
classified by constrained types. Syntactically, properties are
specified in a parameter list right after the name of the class in a
class definition. The class body may contain specifications of other
fields; these fields are considered mutable.

\paragraph{Constraints.}
Given a class {\tt C} with properties {\tt T1 p1}, \ldots, {\tt Tn
pn}, a user may construct a potentially infinite family of types {\tt
C(:c)}, by using a {\em constraint} {\tt c} on the properties of {\tt
C} and final variables in scope at the type. ({\tt C} is called the
{\em base class} of the type, and {\tt c} its condition or {\em where}
clause.)  Constraints specify (possibly) partial information about the
variables of interest. 

Such a type represents a refinement of {\tt C}---the set of all
instances of {\tt C} whose immutable state satisfies the condition
{\tt c}.  The constraint is specified in terms of an underlying
constraint system~\cite{CCCC}---a pre-defined logical vocabulary of
functions and predicates with algorithms for consistency and
entailment.  It may use the special variable {\tt self} to stand for
the object whose type is being defined. Thus {\tt int(:self >= 0)} is
the set of natural numbers, and {\tt point(: x*x + y*y <= 1)}
represents the interior of a circle (for a class {\tt point} with two
{\tt int} properties {\tt x} and {\tt y}). The type {\tt C(:self !=
null)} represents all instances of {\tt C}.\footnote{When there is no
ambiguity a property reference {\tt self.x} may be abbreviated to {\tt
x}.} The type {\tt int(:self==v)} represents a ``singleton'' type, an
{\tt int} is of this type only if it has the same value as {\tt v}.

\begin{example}	
Consider the class definition:
{\footnotesize
\begin{verbatim}
class List(int(:self >= 0) n) {...}
\end{verbatim}}
The type {\tt List(:n==3)} is permissible and represents the set of
all lists with length {\tt 3}. Similarly for {\tt List(:n <= 41)}
and even {\tt List(:n * f() >= g())}.
\end{example}

\paragraph{Interface properties.}
Since properties play a central role in the specification of
refinements of a type, it makes sense to permit interfaces to specify
properties. Any class implementing the interface is required to define
at least the same properties as the interface. It is an error for a
class to inherit a property with the same name but different types
from two different interfaces.  Thus an object {\tt o} satisfies a
type {\tt I(:c)} if it is an instance of a class {\tt C} implementing
{\tt I} and it satisfies {\tt c}.

\paragraph{Properties are typed.}
Properties may be of arbitrary type. For instance, the class {\tt
region} has an {\tt int} property called {\tt rank}. In turn, the
class {\tt dist} has a {\tt region} property, called {\tt region}, and
also an {\tt int} property {\tt rank}.  The invariant for {\tt dist}
ensures that {\tt this.region.rank=this.rank}. Similarly, an array has
properties {\tt dist dist}, {\tt region region}, and {\tt int rank}
and appropriate constraints ensuring that the statically available
information about them is consistent.\footnote{All constraint
languages used in constrained types permit object referencs, field
selection and equality. Such constraint systems have been studied
extensively under the name of ``feature structures''
\cite{feature-structures}.}

In this way a class invariant (and constraint clauses in constrained
types) may specify fairly rich constraints on the immutable portion of
the object reference graph rooted at the current object, and utilizing
objects at user-defined types.

\paragraph{Subtyping.} 
Constrained types are naturally endowed with a subtype relation that
combines the nominal subtyping relation of classes and interfaces with
the logical entailment relation of the constraint system. Namely, a
constraint {\tt C(:c)} is a subtype of {\tt D(:d)} if {\tt C} is a
subtype of {\tt D} and every value in {\tt C} that satisfies {\tt c}
also satisfies {\tt d}. Thus the set of constrained types on a base
type {\tt C} inherits a lattice structure from the underlying
constraint system. The maximal element is {\tt C(:true)}, which is
just {\tt C} and the minimal element is the unsatisfiable constraint.

Constrained types may occur wherever normal types occur. In
particular, they may be used to specify the types of (possibly
mutable) local variables, properties, (possibly mutable) fields,
arguments to methods, return types of methods, in casts etc.  Note
that final local variables as well as final parameters of methods may
be used to define types. Specifically all the final parameters of a
method are in scope at the return type of the method and can be used
to construct the return type.

\begin{example}
A method that multiplies two matrices may be specified by:
{\footnotesize
\begin{verbatim}
class Matrix(int I, int J) {
  Matrix(I, arg.J) mul(final Matrix(:self.I==this.J) arg) 
  {...}
  ...
}
\end{verbatim}}

\end{example}
\paragraph{Method overloading.}
Several design alternatives are possible for integrating method
overloading, overriding, shadowing and obscuring~\cite{Java3} with
dependent types. We discuss in more detail in
Section~\ref{method-sec}. In summary, our current implementation
erases dependent type information when compiling to Java. Therefore it
must be the case that a class does not have two different method
definitions which conflict with each other when the constrained
clauses in their types are erased.

\paragraph{Invariants.}
For additional expressiveness, we permit the specification of a {\em
class invariant}, a {\em where clause}~\cite{where-clauses} in the
class definition. A where clause is a boolean expression on the
parameters separated from the parameter list with a ``{\tt :}''.  The
compiler ensures that all instances of the class created at runtime
satisfy the where clause associated with the class.

Where clauses may also be specified in argument lists of methods and
constructors. They place an additional constraint on the actual
arguments to the method or constructor.

For brevity, we write {\tt C} as a type as well; it
corresponds to the (vacuously) constrained type {\tt C(:true)}.
We also permit the syntax {\tt C(t1,\ldots, tk)} for
the type {\tt C (:x1 = t1, \ldots, xk = tk)} (assuming that
the property list for {\tt C} specifies the {\tt k} properties {\tt
x1,\ldots, xk}, and each term {\tt ti} is of the correct
type). 

Thus, using the definition above, {\tt List(n)} is the type of
all lists of length {\tt n}.

\begin{example}[List]
This program implements a mutable list of Objects. The size of a list
does not change through its lifetime, even though at different points
in time its head and tail might point to different structures.
{\footnotesize
\begin{verbatim}
class List(int(:self >= 0) n) {
  Object head = null;
  List(n-1) tail = null;
  List(0)() { property(0); }
  List(1)(Object head) { this(head, new List());}
  List(tail.n+1)(Object head, List tail) {
    property(tail.n+1);
    this.head = head;
    this.tail = tail;
  }
  List(n+arg.n) append(final List arg) {
    return (n == 0) 
      ? arg : new List(head, tail.append(arg));
  }
  List(n) rev() { return rev(new List()); }
  List(n+acc.n) rev(final List acc) {
    return (n == 0) 
     ? acc : tail.rev(new List(head, acc));
  }
  List(:self.n <= this.n) filter(Predicate f) {
    if (n==0) return this;
    List(:self.n <=this.n-1) l = tail.filter(f);
    return (f.isTrue(head)) ?
     new List(head,l) : l;
  }
}
\end{verbatim}}
\label{fig:list-example}
Intuitively, this definition states that a {\tt List} has a {\tt int}
property {\tt n}, which must be non-negative.  The class has two
fields that hold the head and tail of the list.  The properties of a
class are set through the invocation of {\tt property(\ldots);}
(analogously to {\tt super(\ldots);}).  Constructors have ``return
types'' that can specify a where clause satisfied by the object being
returned by the constructor.  The compiler verifies that the
constructor postcondition and the class invariant are implied by the
{\tt property} statement and any {\tt super} calls in the constructor
body imply

The second constructor returns a singleton list of length {\tt 1}. The
third constructor returns a list of length {\tt m+1}, where {\tt m} is
the length of the second argument.  If an argument appears in the
return type then the argument must be declared {\tt final}. Thus the
argument will point to the same object throughout the evaluation of
the constructor body.
\end{example}

\paragraph{Final variables.}
The use of final variables (local variables, method arguments) in
types has proven to be particularly valuable in practice. The same
variable that is being used in computation can also be used to specify
types. There is no need to introduce separate, universally and
existentially quantified ``index'' variables.

During type-checking, final variables are turned into symbolic
variables---some fixed but unknown value---of the same type.
Computation is performed in a constraint-based fashion on such
variables.

\paragraph{Separation between compile-time and run-time computation}
Our design distinguishes between compile-time execution (performed
during type-checking) and run-time execution. At compile-time, the
compiler processes the abstract syntax tree of the program generating
queries to the constraint solver. The only computation engine running
is the constraint-solver, which operates on its own vocabulary of
predicates and functions. Program variables (such as local variables)
that occur in types are dealt with symbolically. They are replaced
with logical variables---some fixed, but unknown value---of the same
type. The constraint-solver knows how to process pieces of partial
information about these logical variables in order to determine
whether some constraint is entailed. At runtime, the same program
variable will have a concrete value and will perform ``arithmetic''
(calculations) where the compiler performed ``algebra'' (symbolic
analysis).

\paragraph{Query evaluation.}

Because OO languages permit arbitrary mutual recursion between classes
-- class {\tt A} and {\tt B} may have fields of type {\tt B} and {\tt
A} respectively---the type/property graph may have loops. The nodes
in this graph are base types (class and interface names). There is an
edge from node {\tt A} to node {\tt B} if {\tt A} has a property whose
base type is {\tt B}.

Let us define the real-clause of a constrained type {\tt C(:c)} to be
the set of constraints that must be satisfied by any instance of {\tt
C(:c)}. This includes {\tt c} but also includes constraints that hold
for all instances of {\tt C}, as determined by the definition of {\tt
C}. Let us use the notation $\rc({\tt C})$ for the {\em real clause} of {\tt
C}. Since we consider only top-level classes, the only free variable in 
$\rc({\tt C})$ is {\tt self}.

What is $\rc({\tt C}, {\tt X})$ (we have drawn out {\tt X} as the
formal variable)? Consider a general class definition:
$$\class\ {\tt C}(\bar{\tt T}\ \bar{\tt x} : c)\extends\ D(:d)\ \{\bar{M}\}$$

\noindent Clearly, from this we get:

$$\rc({\tt C},{\tt X}) \leftrightarrow 
\begin{array}{l}
\quad ({\tt c},{\tt d})[{\tt X}/\this] \&   \rc({\tt D}) \& \\
\quad \rc({\tt P}_1, {\tt X.x}_1)\& \ldots, \& \rc({\tt P}_k, {\tt X.x}_k)
\end{array}
$$
  
That is, given a program $P$ with classes ${\tt C}_1,\ldots, {\tt
C}_k$, the set of real-clauses for ${\tt C}_1,\ldots, {\tt C}_k$ are
defined in a mutually recursive fashion through the Clark completion
of a Horn clause theory (over an underlying constraint system).

The central algorithmic question now becomes whether given a
constrained clause {\tt d}, does $\rc({\tt C},{\tt X}$ entail {\tt d}? 

From the above formulation the question is clearly semi-decidable. It
is not clear however whether it is decidable. This is a direction for
further work.

In practice, many data-structures have non-cyclic dependency
graphs. For such programs the real-clause can be computed quickly and
only a bounded number of questions to the constraint-solver are
generated during type-checking.

\paragraph{Existential Types.}
The use of constraints makes existential types very natural.  Consider
the return type of {\tt filter} above---it specifies that the list
returned is of some unknown length. The only thing known about it is
that its size is bounded by {\tt n}. Thus constrained types naturally
subsume existential dependent types. Indeed, every base type {\tt C}
is an ``existential'' constrained type since it does not specify any
constraint on its properties. Thus code written with constrained types
can interact seamlessly with legacy library code---using just base
types wherever appropriate.

\paragraph{Parametric Consistency.}
Consider the set of final variables that are referenced in a type {\tt
T=C(:c)}. These are the {\em parameters} of the type. A type is said
to be {\em parametrically consistent} if its where clause is solvable
for each possible assignment of values to parameters.  A
parametrically consistent type has the property that its extension
will never be empty. 

Types are required to be parametrically consistent.

\begin{example}[List, revisited] Consider  a variation of List:
{\footnotesize
\begin{verbatim}
class List(int(:self >= 0) n) {
  Object head;
  List(n-1: self!= null) tail;
  ...
}
\end{verbatim}}
The type of the field {\tt tail} is not parametrically
consistent. There exists a value for its parameter {\tt n}, namely,
{\tt 0} for which the where clause {\tt self != null =>
(self.n=this.n-1, self.n >= 0), self!=null} is not satisfiable.

The compiler will throw a type error when it encounters the
initializer for this field in a constructor since it will not be able
to prove that the initial value is of the given type.

\end{example}

\paragraph{Runtime type casts.} 
Constrained types may occur in a class cast {\tt (T) o}.  Code is
generated to check at runtime that {\tt o} satisfies {\tt T}.


\subsection{Claims}
The paper presents a framework for integrating dependent types into
Java-like OO languages by augmenting nominal types with constraints on
properties of the type. This framework is the basis for the design of
the dependent type system for \Xten{}. The design has been implemented
in \Xten{} version 1.0 (for a simple equality-based constraint
system), available at {\tt x10.sf.net}.  The implementation is
discussed in Section~\ref{sec:impl}.

As in staged languages~\cite{nielson-multistage,ts97-multistage}, the
design distinguishes between compile-time and run-time
evaluation. Constrained types are checked (mostly) at compile-time.
The compiler uses a constraint solver to perform universal reasoning
(``for all possible values of method parameters'') for dependent
type-checking.  There is no run-time constraint-solving.  However,
run-time casts to dependent types are permitted; these casts involve
arithmetic, not algebra---the values of all parameters are known.

We claim that the design is natural and easy to use and useful. Many
example programs have been written using dependent types and are
available at {\tt x10.sf.net}.

We claim that the design is flexible. It is parametric on the
constraint system being used. We are planning on extending the current
implementation to support multiple user-defined constraint systems,
thereby supporting pluggable types. Dependent where clauses are also
the basis for a general user-definable annotation framework we have
implemented separately~\cite{ns07-x10anno}. 

We claim the design is clean and modular. We present a simple core
language \CFJ, extending \FJ{}~\cite{FJ} with constrained types on top
of an arbitrary constraint system. We present rules for type-checking
\CFJ{} programs that are parameteric in the constraint system. 
We establish subject reduction and progress theorems. 

%
% XXX contrast with hybrid type checking.

\paragraph{Rest of this paper.}

The next section reviews related work.
Section~\ref{sec:lang} fleshes out the syntactic and semantic details of the
proposal.
Formal semantics
and a soundness theorem
are presented in Section~\ref{sec:semantics}.
Section~\ref{sec:examples} works through a number of
examples using a variety of constraint systems.
Sections~\ref{sec:future}
and \ref{sec:conclusions} conclude the paper with a discussion of
future work.

