We have presented a simple design for dependent types in \java-like
languages. The design considerably enriches the space of (mostly)
statically checkable types expressible in the language. This is
particularly important for data-structurs such as lists and arrays. We
have shown a simple translation scheme for dependent types into an
underlying language with {\tt assert} and {\tt assume} statements.
The assert and assume statements generated by this translation have
the important property of state invariance. This enables a very simple
notion of simplification for such programs. A general constraint
propagator can simplify programs by using ask and tell operations on
the underlying constraint system. Assert statements are removed if they
are entailed by the conjunction of {\tt assume}s on each path to the
statement.

Our treatment is parametric in that the underlying constraint system
can vary. Indeed the constraint system is not required to be complete;
any incompleteness results merely in certain asserts being relegated
to runtime. Some of these asserts may throw runtime exceptions if they
are violated.

In future work we plan to investigate optimizations (such as array
bounds check elimination) enabled by dependent types. We also plan to
pursue much richer constraint systems, e.g., those necessary to deal
with regions, cyclic and block-cyclic distributions etc.
