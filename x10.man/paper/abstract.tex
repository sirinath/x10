\Xten{} is a modern object-oriented language designed for productivity
and performance in concurrent and distributed systems, such as
(heterogeneous) multicores and clusters. In this context, dependent
types arise naturally: objects may be located at one of many places,
arrays may be multidimensional, activities may be associated with one
or more clocks, variables may be marked as shared or private following
an ownership discipline, etc.  A framework for dependent types offers
significant opportunities for detecting design errors statically,
documenting design decisions, eliminating costly runtime checks
(e.g., for array bounds, null values), and improving the quality of
generated code.

We present the design and implementation of constraint-based
dependent types in \Xten{}.
The system is parametric on an underlying constraint
system {\cal C}: the compiler provides a framework that supports extension
with new constraint systems using compiler 
plugins.
Classes and interfaces are associated with {\em
properties} (= final instance fields). A type \xcd{C(:c)} names a class
or interface \xcd{C} and a {\em constraint} \xcd{c} on the
properties of \xcd{C} and in-scope final variables.  Constraints
may also be associated with class definitions (representing
class invariants) and with method and constructor definitions
(representing preconditions). Dynamic casting is permitted.

In conclusion, we believe that constrained types offer a natural,
simple, clean, and expressive extension to OO programming.
