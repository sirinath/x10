In this section we formalize a small fragment of \Xten{} to illustrate
the basic concepts behind constrained type-checking. In fact a very
tiny language is chosen -- a small extension of \FJ{} with constrained
types. 

The language is functional in that assignment is not
admitted. However, it is not difficult to introduce the notion of
mutable fields, and assignment to such fields. Since constrained types
may onl refer to immutable state, the validity of these types is not
compromised by the introduction of state.

Further, we do not formalize overloading of methods. Rather, with
\FJ{}, we simply require that the input program be such that the class
name {\tt C} and method name {\tt m} uniquely select the associated
method on the class. 

We do model properties, constrained clauses, class invariants, where
clauses in methods and constructors, and dependent type casts.


\begin{figure*}

\paragraph{Syntax for \CFJ.}
The definitions are based on those in Featherweight Java~\cite{FJ}. 

\begin{tabular}{rrcl}
&&&\\
(Class) & {\tt L} &{::=}& $\tt\class \ C(\bar{T}\ \bar{f}:c)\  \extends\ T\ \{\bar{M}\}$ \\
(Method)& {\tt M} &{::=}& $\tt T\ m(\bar{T}\ \bar{x}:c)\{\return\ e;\}$\\
(Expr)& {\tt e} &{::=}& $\tt x \alt e.f \alt e.m(\bar{e})\alt \new\ C(\bar{e})\alt (T)e$ \\
(Type)& {\tt S},{\tt T},{\tt U},{\tt Z}&{::=}& $\tt C(:d)$\\
&&&\\
\end{tabular}

Above, ``,'' binds tighter than ``;''. We use the syntax {\tt
{\tt T\;x};\;c} for the constraint obtained by existentially quantifying the
variable {\tt x} of type {\tt T} in {\tt c}. {\tt p} ranges over
the collection of predicates supplied by the underlying constraint
system, and {\tt g} over the collection of functions.

The {\em base type} of a type {\tt C(:c)} (read as {\em {\tt C} with
{\tt c}}) is {\tt C}.  We use the following shorthand for types: For a
type {\tt T} equal to {\tt C(:c)}, we will write {\tt S\ x; T} for
{\tt C(:S\ x; c)}, and {\tt d,T} for {\tt C(:d,c)}.
Application of substitutions is extended to
types by: ${\tt C(:c)\theta}={\tt C(:c\theta)}$.

\paragraph{\CFJ{} subtyping judgment.}\label{CFJ-subtyping}
We add a single rule to the rules of \FJ:
$$
\begin{array}{llll}
 {\tt C} \subtype {\tt C}
&
\from{\class\ {\tt C(\ldots)}\ \extends\ {\tt D(\ldots)}\{\ldots\}}
\infer{{\tt C} \subtype {\tt D}}
& 
\from{{\tt C} \subtype {\tt D} \ \ \ {\tt D} \subtype {\tt E}}
\infer{{\tt C} \subtype {\tt E}} &
\from{
\begin{array}{ll}
{\tt C} \subtype {\tt D} &
\sigma(\Gamma, {\tt C(:c)}\ {\tt x}) \vdash_{\cal C} {\tt d}[{\tt x}/\self] \ \ \mbox{({\tt x} fresh)}
\end{array}}
\infer{\Gamma \vdash {\tt C(:c) \subtype D(:d)}}
\end{array}
$$

(Whenever we state an assumption of the form ``{\tt x} is
fresh'' in a rule we mean it is not free in the consequent of the
rule.)

\paragraph{\CFJ{} typing judgment.}\label{CFJ-typing}
We let $\Gamma$ stand for multisets of type assertions, of the form
${\tt T\ x}$,\footnote{We use the non-standard notation ${\tt T\ x}$
rather than the more familiar ${\tt x} : {\tt T}$ since {\tt :} is
used in the syntax of a type.}  and constraints. Typing judgments are
of the form $\Gamma\vdash {\tt S\ t}$ When $\Gamma$ is empty, it is
omitted. 

Let {\tt C} be a class declared as ${\tt \class\ C(\bar{\tt T}\
\bar{\tt f}:c)\ extends\ D(:d)\{\bar{\tt M}\}}$. Let
$\theta$ be a substitution and the type {\tt T} be based on {\tt C}.
We define $\inv(T,\theta)$
as the conjunction ${\tt c\theta,d\theta}$ and (recursively)
$\inv({\tt D},\theta)$.  We bottom out with $\inv({\tt
Object},\theta)=\true$. For a variable {\tt x}, we use the shorthand
$\inv({\tt C},{\tt x})$ to mean $\inv({\tt C},[{\tt x}/\self])$.

The definition of {\mtype({\tt C},{\tt m})} (the signature of a method
named {\tt m} in class {\tt C}), {\mbody({\tt C},{\tt m})}, (the body
associated with method {\tt m} in type {\tt C}) and \fields(C) (the
sequence of fields and their types inherited or defined at {\tt C}) is
essentially as specified in FJ~\cite{FJ} with the difference that the
method of a signature is taken to be of the form $\bar{\tt S}\ \bar{\tt
x}: {\tt c} \rightarrow {\tt T}$.  The variables {\tt x} are permitted
to occur in the types $\bar{\tt S},{\tt T}$, and are considered bound,
and subject to alpha-renaming.  The definitions of \mtype, \mbody,
\fields{} are extended to apply to constrained types by ignoring the
constraint.  For a substitution $\theta$ we define $\mtype({\tt
T},{\tt m},\theta)$ as the signature obtained by applying $\theta$ to
$\mtype({\tt T},{\tt m})$, renaming bound variables as necessary.
Similarlly, for a substitution $\theta$ we define $\fields({\tt
T},\theta)$ to be $\bar{S}\theta\ \bar{f}$, if the sequence of
inherited and defined fields of the class underlying the type {\tt T}
is $\bar{S}\ \bar{f}$. We let $\fields({\tt T},{\tt x})$ stand for
$\fields({\tt T},[{\tt x}/\self]))$.

We define $\sigma(\Gamma)$ to be the set of
constraints obtained from $\Gamma$ by replacing each type assertion
${\tt C(:d)\ x}$ in $\Gamma$ with ${\tt d}[{\tt x}/\self],\inv(C,x)$
and retaining any constraint in $\Gamma$.

$$
\begin{array}{l}
\begin{array}{lll}
\rname{T-Var}%
\from{\sigma(\Gamma, {\tt C(:c)}\ {\tt x}) \vdash_{\cal C} {\tt d}[{\tt x}/\self]}
\infer{\Gamma, {\tt C(:c)\ x} \vdash {\tt C(:d)}\ {\tt x}} &
\rname{T-Field}%
\from{\Gamma \vdash {\tt T}_0\ {\tt e} \ \ \ \fields({\tt T}_0,{\tt z}_0)= \bar{\tt U}\ \bar{\tt f}_i \ \ \ \mbox{(${\tt z}_0$ fresh)}} 
\infer{\Gamma \vdash ({\tt T}_0\ {\tt z}_0; {\tt z}_0.{\tt f}_i=\self;{\tt U}_i)\ {\tt e.f}_i} 
& 
\rname{T-Cast}%
\from{\Gamma \vdash {\tt S}\ {\tt e}}
\infer{\Gamma \vdash {\tt T}\ {\tt (T) e}}
\end{array}
\\  \quad \\
\begin{array}{ll}
\rname{T-Invk}%
\from{\begin{array}{ll}
\Gamma \vdash {\tt T}_{0:n} \ {\tt e}_{0:n}  &
\mtype({\tt T}_0,{\tt m},{\tt z}_0)= \tt {\tt Z}_{1:n}\ {\tt z}_{1:n}:c \rightarrow {\tt S} \\
\Gamma, {\tt T}_{0:n}\ {\tt z}_{0:n} \vdash {\tt T}_{1:n} \subtype {\tt Z}_{1:n}&
\sigma(\Gamma, {\tt T}_{0:n}\ {\tt z}_{0:n}) \vdash_{\cal C} {\tt c} \ \ \ 
\mbox {(${\tt z}_{0:n}$ fresh)}
\end{array}}
\infer{\Gamma \vdash ({\tt T}_{0:n}\ {\tt z}_{0:n}; S)\ {\tt e}_0.{\tt m({\tt e}_{1:n})}}&
\rname{T-New}%
\from{
  \begin{array}{ll}
    \Gamma \vdash \bar{\tt T}\ \bar{\tt e} \ \ \
  \theta=[\bar{\tt f}/\this.\bar{\tt f}] & 
    \fields(C,\theta)=\bar{\tt Z}\ \bar{\tt f} \\
    \Gamma, \bar{\tt T}\ \bar{\tt f} \vdash \bar{\tt T} \subtype \bar{\tt Z} &
    \sigma(\Gamma, \bar{\tt T}\ \bar{\tt f}) \vdash_{\cal C} \inv({\tt C},\theta) 
  \end{array}
}
\infer{\Gamma \vdash {\tt C(:\bar{T}\ \bar{\tt f}{\tt ;\self.\bar{f}}=\bar{\tt f})\ \new\ {\tt C(\bar{\tt e})}}} \\
\end{array}
\end{array}
$$
\paragraph{Method and class typing.}
$$
\begin{array}{ll}
\from{ \bar{\tt T}\ \bar{\tt x}, {\tt C}\ \this, {\tt c} \vdash {\tt S}\ {\tt e}, {\tt S} \subtype {\tt T} }   
\infer{\tt T\ m(\bar{\tt T}\,\bar{\tt x} : c)\{\return\ e;\}\ \mbox{OK in}\ C} &
\from{\bar{M}\ \mbox{OK in}\ C}
\infer{\tt \class\ C(\bar{\tt T}\ \bar{\tt f}:c)\ \extends\ D(:d)\ \{\bar{M}\}\ \mbox{OK}} 
\end{array}
$$

\paragraph{Computation.}
$$
\begin{array}{ccc}
\rname{{\sc R-Field}}%
\from{\fields(C)=\bar{C}\ \bar{f}}
\infer{(\new\ {\tt C}(\bar{\tt e})).{\tt f}_i \derives {\tt e}_i} &
\rname{{\sc R-Invk}}%
\from{mbody({\tt m},{\tt C})=\bar{x}. {\tt e}_0}
\infer{(\new\ {\tt C}(\bar{\tt e})).{\tt m}(\bar{\tt d}) \derives 
[\bar{d}/\bar{x},\new\ C(\bar{e})/\this]{\tt e}_0} &
\rname{{\sc R-Cast}}%
\from{\vdash C \subtype T[\new\ C(\bar{\tt d})/\self]}
\infer{{\tt (T)(\new\ C(\bar{\tt d}))} \derives \new\ C(\bar{\tt d})}
\end{array}
$$
\paragraph{Congruence.}
$$
\begin{array}{l}
\begin{array}{ccc}
\rname{{\sc RC-Field}}%
\from{{\tt e}_0 \derives {{\tt e}_0}'}
\infer{{\tt e}_0.{\tt f} \derives {{\tt e}_0}'.{\tt f}} &
\rname{{\sc RC-Invk-Recv}}%
\from{{\tt e}_0 \derives {{\tt e}_0}'}
\infer{{\tt e}_0.{\tt m}(\bar{\tt e}) \derives {{\tt e}_0}'.{\tt m}(\bar{\tt e})} &
\rname{{\sc RC-Invk-Arg}}%
\from{{\tt e}_i \derives {{\tt e}_i}'}
\infer{{\tt e}_0.{\tt m}(\ldots,{\tt e}_i,\ldots) \derives {{\tt e}_0}.{\tt m}(\ldots,{\tt e}_i',\ldots)} 
\end{array}
\\ \quad \\
\begin{array}{cc}
\rname{{\sc RC-New-Arg}}%
\from{{\tt e}_i \derives {{\tt e}_i}'}
\infer{\new\ {\tt C}(\ldots,{\tt e}_i,\ldots) \derives \new\ {\tt C}(\ldots,{\tt e}_i',\ldots)} &
\rname{{\sc RC-Cast}}%
\from{{\tt e}_0 \derives {{\tt e}_0}'}
\infer{{\tt (C) e}_0 \derives {{\tt (C) e}_0}'}
\end{array}
\end{array}
$$
\caption{The system Constrained \FJ}\label{CFJ-red-rules}
\end{figure*}

\paragraph{Constraint system.}
Constraints are assumed to be drawn from a fixed constraint system,
$\cal C$, with inference relation $\vdash_{\cal C}$ \cite{CCCC}.
All
constraint systems are required to support the trivial constraint
\true, conjunction, existential quantification and equality on
constraint terms. Constraint terms include (final) variables, the
special variable {\tt self} (which may occur only in constraints {\tt
c} which occur in a constrained type {\tt C(:c)}), and field
selections {\tt t.f}. 

We summarize here properties of constraint systems described in
\cite{CCCC} that are needed for the proofs: constraint systems may be
thought of as presented via an intuitionistic Gentzen proof system
supporting Identity, affine and exchange on the left, existential
quantification and conjunction on the left and right, and closure
under substitution of terms. We denote the application of the
substitution $\theta=[\bar{{\tt t}}/\bar{{\tt x}}]$ to a constraint ${\tt c}$ by
${\tt c}[\bar{{\tt t}}/\bar{{\tt x}}]$. 

\begin{tabular}{rrcl}
&&&\\
(C Term) & {\tt t} &{::=}& {\tt  x}\alt \self \alt \this \alt {\tt t.f} \\
&&& \alt \new\ {\tt C($\bar{\tt t}) \alt {\tt g}(\bar{\tt t})$}\\
(Constraint) & {\tt c},{\tt d} &{::=}&$\true\alt {\tt p}(\bar{\tt t})$\\
&&& $\alt {\tt t=t}\alt {\tt c,c}\alt{\tt  T\,x;c}$\\
&&&\\
\end{tabular}

All constraint systems are required to satisfy: $\new\ {\tt C(\bar{\tt
t})}.{\tt f}_i={\tt t}_i $ provided that $\fields({\tt C})=\bar{\tt
T}\ \bar{\tt f}$ (for some sequence of types $\bar{\tt T}$).

\paragraph{Syntax.}
The syntax for the language is specified in Figure~\ref{CFJ-red-rules}.

A type is taken to be of the form {\tt C(:c)} where {\tt C} is the
name of a class or interface and {\tt c} is a constraint; we say that
{\tt C} is the {\em base} of the type {\tt C(:c)}.

A type assertion {\tt C(:c) x} constrains the variable {\tt x} to
contain references to only those objects {\tt o} that are instances of
(subclasses of) {\tt C} and for which the constraint {\tt c} is true
provided that occurrences of {\tt self} in {\tt c} are replaced by
{\tt o}. Thus in the constraint {\tt c} of a constrained type {\tt
C(:c)}, {\tt self} may be used to reference the object whose type is
being specified. Note that {\tt self} is distinct from
{\tt this}---{\tt this} is permitted to occur in the clause of
a type {\tt T} only
if {\tt T} occurs in an instance field declaration or instance method
declaration of a class; as usual, \this{} is considered bound to the
instance of the class to which the field or method declaration
applies.

A {\em class declaration} $\tt\class \ C(\bar{T}\ \bar{f}:c)\ \extends\ D(:d)\
\{\bar{M}\}$ is thought of as declaring a class {\tt C} with the
fields $\bar{\tt f}$ (of type $\bar{\tt T}$), a {\em declared class
invariant} {\tt c}, a {\em super-class invariant} {\tt d} and a
collection of methods $\bar{\tt M}$. The constraints {\tt c} and {\tt
d} are true for all instances of the class {\tt C} (this is verified
in the rule for type-checking constructors, T-New).  In these
constraints, {\tt this} may be used to reference the current object;
\self{} does not have any meaning and must not be used.

A {\em method declaration} ${\tt T}_0\ {\tt m(\bar{\tt T}\ \bar{\tt x} :
c)\{\ldots\}}$ specifies the type of the arguments and the result, as
usual.  The method arguments $\bar{\tt x}$ may occur in the argument
types $\bar{\tt T}$ and the return type ${\tt T}_0$.  The constraint
{\tt c} specifies additional constraints on the arguments $\bar{\tt
x}$ and
\this{} that must hold for a method invocation to be legal. Note that
\self{} does not make sense in {\tt c} (no type is being defined), and must not occur in {\tt c}.

\paragraph{Type judgments.}
Typing judgments are of the form $\Gamma \vdash {\tt T}\ {\tt e}$
where $\Gamma$ is a multiset of type assertions ${\tt T}\ {\tt x}$ and
constraints ${\tt c}$. 
%The constraint entailment relation
%$\vdash_{\cal C}$ is lifted to type assertions through the definition:
%$\Gamma \vdash_{\cal C} {\tt D(:d)}\ {\tt x}$ provided that $\{ {\tt
%c}[{\tt x}/\self] \alt {\tt C(:c)}\ x \in \Gamma\} \cup
%\{{\tt c} \alt  {\tt c} \in \Gamma\} \vdash_{\cal C} {\tt d}[{\tt x}/\self]
%$ and $\Gamma \vdash {\tt D}\ {\tt x}$. Intuitively, $\Gamma \vdash
%\tt D(:d)\ {\tt x}$ if $\Gamma$ constrains {\tt x} to be of type {\tt
%D} and there is enough information in the constraints in $\Gamma$ to
%entail {\tt d} for {\tt x}.

%$\sigma(\Gamma)$ is the set of constraints on the variables whose type
%assertions are specified by $\Gamma$, generated by replacing each type
%assertion {\tt C(:c) x} in $\Gamma$ with ${\tt c}[{\tt x}/\self]$.
%The rule T-Constr permits a type {\tt C(:c)} for a variable {\tt x} to
%be strengthened with information entailed per $\cal C$ from the
%information about {\tt x} and other variables specified in $\Gamma$.
\def\TConstr{\mbox{\sc T-Constr}}
\def\TInv{\mbox{\sc T-Inv}}
\def\TVar{\mbox{\sc T-Var}}
\def\TField{\mbox{\sc T-Field}}
\def\TInvk{\mbox{\sc T-Invk}}
\def\TNew{\mbox{\sc T-New}}
\def\TCast{\mbox{\sc T-Cast}}
\def\TUCast{\mbox{\sc T-UCast}}
\def\TDCast{\mbox{\sc T-DCast}}
\def\TSCast{\mbox{\sc T-SCast}}
%\TConstr{} is a form of cut which permits information obtained through
%constraint entailment to enrich the type of an expression.

\TVar{} extends the identity rule ($\Gamma, x:C \vdash x:C$) of {\sf FJ} to take into account the constraint entailment relation.

\TCast{} encapsulates the three inference rules of {\sf FJ}:
\TUCast{}, \TDCast{} and \TSCast{} for upwards cast, downwards cast, and ``stupid'' cast respectively. 

%\TInv{} is a form of contraction that permits the class invariant {\tt c}
%of a class {\tt C} to enrich the type of any variable of type {\tt C}.

In \TField, we postulate the existence of a receiver object {\tt o} of
the given static type (${\tt T}_0$). $\fields({\tt T}_0,{\tt o})$ is
the set of typed fields for ${\tt T}_0$ with all occurrences of 
\this{} replaced  by {\tt o}. We record in the resulting
constraint that ${\tt o.f}_i=\self$.\footnote{A new name {\tt o} is
necessary to name this object since {\tt e} cannot be used. Arbitrary
term expressions {\tt e} are not permitted in constraints; the
functions used in {\tt e} may not be known to the constraint system,
and {\tt e} may have side-effects.}  This permits transfer of
information that may have been recorded in ${\tt T}_0$ about the field
${\tt f}_i$. 

Similarly, in \TInvk{} we postulate the existence of a receiver object
{\tt o} of the given static type. For any type $T$, object {\tt o} of
type $T$ and method name {\tt m}, let $\mtype({\tt T},{\tt m},{\tt
o})$ be a copy of the signature of the method with \this{} replaced by
{\tt o}. We establish (under the assumption that the formals
($\bar{\tt z}$) have the static type of the actuals)\footnote{This is
stronger than assuming $\bar{\tt Z}$.}  that actual types are subtypes
of the formal types, and the method constraint is satisfied. This
permits us to record the constraint {\tt d} on the return type, with
the formal variables $\bar{\tt z}$ existentially
quantified.\footnote{Recall that the $\bar{\tt z}$ may occur in {\tt
d} but must not occur in a type in the calling environment; hence they
must be existentially quantified in the resulting constraint.}

In \TNew, similarly, we establish that the static types of the actual
arguments to the constructor are subtypes of the declared types of the
field, and contain enough information to satisfy the class invariant,
{\tt c}. The declared types (and {\tt c}) contain references to ${\tt
this.\bar{\tt f}}$; these must be replaced by the formals $\bar{\tt
f}$, which carry information about the static type of the
actuals. Note that the object {\tt o} we hypothesized in an analogous
situation in \TInvk{} does not exist; it will exist on successful
invocation of the constructor. The constrained clause of the \new{}
expression contains all the information that can be gleaned from the
static types of the actuals by assigning them to the corresponding
fields of the object being created.

\begin{theorem}[Subject Reduction] 

If $\Gamma \vdash T\ e$ and $e \derives e'$, then for some type $S$,
$\Gamma \vdash S\ e'$ and $\Gamma \vdash S \subtype T$.

\end{theorem}

Let the normal form of expressions be given by {\em values},
i.e.{} expressions:

\begin{tabular}{rrcl}
&&&\\
(Values) & {\tt v} &{::=}& $\new\ {\tt C(\bar{\tt v})}$
\end{tabular}

\begin{theorem}[Progress] If $\vdash {\tt T\ e}$, then one of the following conditions holds:
\begin{enumerate}
\item {\tt e} is a value {\tt v}, 
\item {\tt e} contains a subexpression ${\tt (T)\new\ C(\bar{\tt
v})}$ such that
$\not\vdash {\tt C} \subtype {\tt T}[{\tt \new\ C(\bar{\tt v})}/\self]$,
\item there exists ${\tt e}'$ s.t. ${\tt e} \derives {\tt e}'$.
\end{enumerate}
\end{theorem}

\begin{theorem}[Type Soundness] 

If $\vdash {\tt T\ e}$ and ${\tt e} \starderives {\tt e}'$, then ${\tt e}'$ is
either (1)~a value {\tt v} with $\vdash {\tt S\ v}$ and $\vdash {\tt S
\subtype T}$, for some type {\tt S}, or, (2)~ an expression containing
a subexpression ${\tt (T)\new\ {\tt C(\bar{\tt v})}}$ where 
$\not\vdash \tt C\subtype T[\new\ C(\bar{\tt v})/\self]$.

\end{theorem}

\begin{lemma}[Substitution Lemma]
Assume $\Gamma \vdash \bar{\tt A}\ \bar{\tt d}$, $\Gamma \vdash \bar{\tt A}\subtype \bar{\tt B}$, and $\Gamma, \bar{\tt B}\ \bar{\tt x} \vdash {\tt T}\ {\tt e}$. Then for some type ${\tt S}$ s.t. $\Gamma \vdash {\tt S} \subtype \bar{\tt A}\ \bar{\tt x};{\tt T}$ it is the case that $\Gamma \vdash {\tt S}\ {\tt e}[\bar{\tt d}/\bar{\tt x}]$.
\end{lemma}

% Unchanged from FJ
\begin{lemma}[Weakening]
If $\Gamma \vdash {\tt T}\ {\tt e}$, then $\Gamma, {\tt S}\ {\tt x}\vdash {\tt T}\ {\tt e}$.
\end{lemma}

% Unchanged from FJ
\begin{lemma}[Body type]
If $\mtype({\tt T}_0,{\tt m})=\bar{\tt T}\ \bar{\tt x} : {\tt c}
\rightarrow {\tt S}$, and $mbody({\tt m}, {\tt T}_0)=\bar{\tt x}.{\tt
e}$, then for some ${\tt U}_0$ with ${\tt T}_0 \subtype {\tt U}_0$,
there exists ${\tt V}\subtype {\tt S}$ such that
$\bar{\tt T}\ \bar{\tt x},{\tt U}_0\ \this \vdash {\tt V}\ {\tt e}$
\end{lemma}


\subsection{Erasure}

Constrained types in CFJ are a form of {\em refinement
type}~\cite{refinement-types}.  If constraints are erased from a
well-typed program,
the resulting program will behave identically to the unerased
program except that the unerased program might be unable to take
a step on a cast.

Let $\Lb {\tt e} \Rb$ be the erasure of ${\tt e}$ defined as follows:
\begin{align*}
\Lb {\tt x} \Rb &= {\tt x} \\
\Lb {\tt e}.{\tt f} \Rb &= \Lb {\tt e} \Rb.{\tt f} \\
\Lb {\tt e}.{\tt m}(\bar{\tt e}) \Rb &= \Lb {\tt e} \Rb.{\tt m}(\bar{\Lb {\tt e} \Rb}) \\
\Lb {\tt new}~{\tt C}(\bar{\tt e}) \Rb &= {\tt new}~{\tt C}(\bar{\Lb {\tt e} \Rb}) \\
\Lb ({\tt C}(:{\tt c}))~{\tt e} \Rb &= (\tt C(: {\tt true}))~\bar{\Lb {\tt e} \Rb}
\end{align*}

\begin{theorem}[Erasure]

If $\vdash {\tt C}(:{\tt c})\ {\tt e}$ and ${\tt e} \starderives {\tt v}$,
then $\vdash {\tt C}(:{\tt true})\ \Lb {\tt e} \Rb$ and $\Lb
{\tt e} \Rb \starderives \Lb {\tt v} \Rb$.

\end{theorem}

