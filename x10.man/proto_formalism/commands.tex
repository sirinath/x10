\usepackage{relsize}
\usepackage{amsmath}
\usepackage{url}


% http://en.wikibooks.org/wiki/LaTeX/Packages/Listings
\usepackage{listings}


\usepackage[ruled]{algorithm} % [plain]
\usepackage[noend]{algorithmic} % [noend]
\renewcommand\algorithmiccomment[1]{// \textit{#1}} %


% For fancy end of line formatting.
\usepackage{microtype}


% For smaller font.
\usepackage{pslatex}


\usepackage{xspace}
\usepackage{yglabels}
\usepackage{yglang}
\usepackage{ygequation}
\usepackage{graphicx}
%\usepackage{epstopdf}

\newcommand{\formalrule}[1]{\mbox{\textsc{\scriptsize #1}}}
\newcommand{\myrule}[1]{\textsc{\codesmaller #1 Rule}}
\newcommand{\umyrule}[1]{\textbf{\underline{\textsc{\codesmaller #1 Rule}}}}


% \small \footnotesize \scriptsize \tiny
\newcommand{\mysubsection}[1]{\medskip\noindent {\textbf{#1.}~~}} %\paragraph
% \codesize and \scriptsize seem to do the same thing.
% \newcommand{\code}[1]{\texttt{\textup{\footnotesize #1}}}
% \newcommand{\code}[1]{\texttt{\textup{\codesize #1}}}
\newcommand{\normalcode}[1]{\texttt{\textup{#1}}}
\def\codesmaller{\small}
\newcommand{\myCOMMENT}[1]{\COMMENT{\small #1}}
\newcommand{\code}[1]{\texttt{\textup{\codesmaller #1}}}
% \newcommand{\code}[1]{\ifmmode{\mbox{\smaller\ttfamily{#1}}}
%                       \else{\smaller\ttfamily #1}\fi}
\newcommand{\smallcode}[1]{\texttt{\textup{\scriptsize #1}}}
\newcommand{\myparagraph}[1]{\noindent\textit{\textbf{#1}}~} %\vspace{-1mm}\paragraph{#1}}

% See: \usepackage{bold-extra} if you want to do \textbf
\newcommand{\keyword}[1]{\code{#1}}

% For new, method invocation, and cast:
\newcommand{\hparen}[1]{\code{(}#1\code{)}}
\newcommand{\angular}[1]{\code{<}#1\code{>}}

\newcommand{\hgn}[1]{\lt#1\gt} % type parameters and generic method parameters

\newcommand\Abs[1]{\ensuremath{\left\lvert#1\right\rvert}}
\newcommand{\erase}[1]{\ensuremath{\Abs{#1}}}


\newcommand{\todo}[1]{\textbf{[[#1]]}}
%% To disable, just uncomment this line:
%\renewcommand{\todo}[1]{\relax}

%% Additional todo commands:
\newcommand{\TODO}[1]{\todo{TODO: #1}}

\newcommand\xX[1]{$\textsuperscript{\textit{\text{#1}}}$}


\newcommand{\ol}[1]{\overline{#1}}
\newcommand{\nounderline}[1]{{#1}}


%% Commands used to typeset the FIGJ type system.
\newcommand{\typerule}[2]{
\begin{array}{c}
  #1 \\
\hline
  #2
\end{array}}


%% Commands used to typeset the FOIGJ type system.
\newcommand{\inside}{\prec}
\newcommand{\visible}{{\it visible}}
\newcommand{\placeholderowners}{{\it placeholderowners}}
\newcommand{\nullexpression}{{\tt null}}
\newcommand{\errorexpression}{{\tt error}}
\newcommand{\locations}{{\it locations}} %% \mathop{\mathit{locations}}}
\newcommand{\I}{\it I}


\newcommand\mynewcommand[2]{\newcommand{#1}{#2\xspace}}

\mynewcommand{\hA}{\code{A}} % inVariant definition
\mynewcommand{\hB}{\code{B}} % inVariant definition

\mynewcommand{\hI}{\code{I}} % iparam

% In the syntax: \hI or ReadOnly or Mutable or Immut
\mynewcommand{\hJ}{\code{J}}
\mynewcommand{\hO}{\code{O}}
\mynewcommand{\ho}{\code{o}}

\mynewcommand{\hK}{\code{K}}
\mynewcommand{\hX}{\code{X}} % vars
\mynewcommand{\hY}{\code{Y}} % vars
\mynewcommand{\hC}{\code{C}} % class
\mynewcommand{\hc}{\code{c}} % cooker
\mynewcommand{\hL}{\code{L}} % class decl
\mynewcommand{\hM}{\code{M}} % Method decl
\mynewcommand{\hN}{\code{N}} % Non-variable type
\mynewcommand{\hm}{\code{m}} % method
\mynewcommand{\he}{\code{e}} % expression
\mynewcommand{\hv}{\code{v}} % value
\mynewcommand{\hl}{\code{l}} % location in the store
\mynewcommand{\lroot}{\code{l}_\top} % root
\mynewcommand{\lthis}{\code{l}_\smallcode{this}} % this
\mynewcommand{\hx}{\code{x}} % method parameter
\mynewcommand{\hf}{\code{f}} % field
\mynewcommand{\hF}{\code{F}} % types (vars or non vars) of a field
\mynewcommand{\hT}{\code{T}} % types (vars or non vars)
\mynewcommand{\hU}{\code{U}} % types (vars or non vars)
\mynewcommand{\hV}{\code{V}} % closed types
\mynewcommand{\hH}{\code{H}} % Heap
\mynewcommand{\hS}{\code{S}}
\mynewcommand{\hsub}{\code{/}} % substitute (reduction rules)
\mynewcommand{\hthis}{\code{this}} % this
\mynewcommand{\hclass}{\code{class}}
\mynewcommand{\hreturn}{\code{return}}
\mynewcommand{\hnew}{\code{new}}
\newcommand{\lt}{\code{<}}%{\mathop{\textrm{\tt <}}}
\newcommand{\gt}{\code{>}}%{\mathop{\textrm{\tt >}}}

\mynewcommand{\proto}{\keyword{proto}}
\mynewcommand{\cooked}{\keyword{cooked}}
\mynewcommand{\this}{\keyword{this}}
\mynewcommand{\Object}{\code{Object}}
\mynewcommand{\const}{\keyword{const}} %C++ keyword
\mynewcommand{\mutable}{\keyword{mutable}} %C++ keyword
\mynewcommand{\romaybe}{\keyword{romaybe}} %Javari keyword

%% Define the behaviour of the theorem package.
%% Use http://math.ucsd.edu/~jeggers/latex/amsthdoc.pdf for reference.

\newtheorem{theorem}{Theorem}[section]
\newtheorem{definition}[theorem]{Definition}
\newtheorem{lemma}[theorem]{Lemma}
\newtheorem{corollary}[theorem]{Corollary}
\newtheorem{fact}[theorem]{Fact}
\newtheorem{example}[theorem]{Example}
\newtheorem{remark}[theorem]{Remark}


\mynewcommand{\IP}{\code{I}}   % formal type parameter
\mynewcommand{\JP}{\code{J}}   % formal type parameter (for soundness proofs)



\mynewcommand{\This}{\code{This}}
\mynewcommand{\World}{\code{World}}


% Our annotations
\mynewcommand{\OMutable}{\code{@OMutable}}
\mynewcommand{\OI}{\code{@OI}}


\mynewcommand{\InVariantAnnot}{\code{@InVariant}}


\newcommand{\func}[1]{\text{\textnormal{\textit{\codesmaller #1}}}}

\mynewcommand{\Fdef}{\ensuremath{\triangleq}} %=_{\code{def}}
\newcommand{\Tprec}[0]{\ensuremath{\leqslant^T}}
\newcommand{\Ppoints}[2]{\ensuremath{#1 \rightarrow_P #2}}
\newcommand{\Pequals}[2]{\ensuremath{#1 \equiv_P #2}}
\mynewcommand{\st}{\ensuremath{\mathrel{{\preccurlyeq}}}} %{\mathop{\textrm{\tt <:}}}
\mynewcommand{\notst}{\mathrel{\st\hspace{-1.5ex}\rule[-.25em]{.4pt}{1em}~}}
\mynewcommand{\gap}{~ ~ ~ ~ ~ ~}


\newcommand{\RULE}[1]{\textsc{\scriptsize{}#1}} %\RULEhape\scriptsize}
\newcommand{\closed}[1]{\ensuremath{\func{closed}(#1)}}

\mynewcommand{\NPE}{\func{NPE}}
\mynewcommand{\mtype}{\func{mtype}}
\mynewcommand{\mbody}{\func{mbody}}
\mynewcommand{\mcooker}{\func{mcooker}}
\mynewcommand{\isVar}{\func{isVar}}
\newcommand{\cooker}[1]{\ensuremath{\func{cooker}(P,#1)}}
\mynewcommand{\bound}{\func{bound}_\Delta}
\mynewcommand{\substitute}{\func{substitute}}
\mynewcommand{\fields}{\func{fields}}
\mynewcommand{\fclass}{\func{fclass}}
\DeclareMathOperator{\dom}{dom}


\mynewcommand\xth{\xX{th}}
\mynewcommand\xrd{\xX{rd}}
\mynewcommand\xnd{\xX{nd}}
\mynewcommand\xst{\xX{st}}
\mynewcommand\ith{$i$\xth}
\mynewcommand\jth{$j$\xth}


%\mynewcommand{\emptyline}{\vspace{\baselineskip}}
\mynewcommand{\myindent}{~~}


% Add line between figure and text
\makeatletter
\def\topfigrule{\kern3\p@ \hrule \kern -3.4\p@} % the \hrule is .4pt high
\def\botfigrule{\kern-3\p@ \hrule \kern 2.6\p@} % the \hrule is .4pt high
\def\dblfigrule{\kern3\p@ \hrule \kern -3.4\p@} % the \hrule is .4pt high
\makeatother

\setlength{\textfloatsep}{.75\textfloatsep}


% Remove line between figure and its caption.  (The line is prettier, and
% it also saves a couple column-inches.)
\makeatletter
%\@setflag \@caprule = \@false
\makeatother


% http://www.tex.ac.uk/cgi-bin/texfaq2html?label=bold-extras
\usepackage{bold-extra}


% Left and right curly braces in tt font
\newcommand{\ttlcb}{\texttt{\char "7B}}
\newcommand{\ttrcb}{\texttt{\char "7D}}
\newcommand{\lb}{\ttlcb}
\newcommand{\rb}{\ttrcb}


\setlength{\leftmargini}{.75\leftmargini}
