\documentclass[10pt,a4paper]{article}

%% For typesetting theorems and some math symbols.
\usepackage{amssymb}
\usepackage{amsthm}

\usepackage{fullpage}

\title{Featherweight X10}

\author{}

\date{}




\usepackage{xspace}

% Macros for R^nRS.

\def\@makechapterhead#1{%
  \vspace*{50\p@}%
  {\parindent \z@ \raggedright \normalfont
    \ifnum \c@secnumdepth >\m@ne
        \huge\bfseries \thechapter \space\space\space
        \nobreak
    \fi
    \interlinepenalty\@M
    \Huge \bfseries #1\par\nobreak
    \vskip 40\p@
  }}


\makeatletter

\newcommand{\topnewpage}{\@topnewpage}

% Chapters, sections, etc.

\newcommand{\vest}{}
\newcommand{\dotsfoo}{$\ldots\,$}

\newcommand{\sharpfoo}[1]{{\tt\##1}}
\newcommand{\schfalse}{\sharpfoo{f}}
\newcommand{\schtrue}{\sharpfoo{t}}

\newcommand{\singlequote}{{\tt'}}  %\char19
\newcommand{\doublequote}{{\tt"}}
\newcommand{\backquote}{{\tt\char18}}
\newcommand{\backwhack}{{\tt\char`\\}}
\newcommand{\atsign}{{\tt\char`\@}}
\newcommand{\sharpsign}{{\tt\#}}
\newcommand{\verticalbar}{{\tt|}}

\newcommand{\coerce}{\discretionary{->}{}{->}}

% Knuth's \in sucks big boulders
\def\elem{\hbox{\raise.13ex\hbox{$\scriptstyle\in$}}}

\newcommand{\meta}[1]{{\noindent\hbox{\rm$\langle$#1$\rangle$}}}
\let\hyper=\meta
\newcommand{\hyperi}[1]{\hyper{#1$_1$}}
\newcommand{\hyperii}[1]{\hyper{#1$_2$}}
\newcommand{\hyperj}[1]{\hyper{#1$_i$}}
\newcommand{\hypern}[1]{\hyper{#1$_n$}}
\newcommand{\var}[1]{\noindent\hbox{\it{}#1\/}}  % Careful, is \/ always the right thing?
\newcommand{\vari}[1]{\var{#1$_1$}}
\newcommand{\varii}[1]{\var{#1$_2$}}
\newcommand{\variii}[1]{\var{#1$_3$}}
\newcommand{\variv}[1]{\var{#1$_4$}}
\newcommand{\varj}[1]{\var{#1$_j$}}
\newcommand{\varn}[1]{\var{#1$_n$}}

\newcommand{\vr}[1]{{\noindent\hbox{$#1$\/}}}  % Careful, is \/ always the right thing?
\newcommand{\vri}[1]{\vr{#1_1}}
\newcommand{\vrii}[1]{\vr{#1_2}}
\newcommand{\vriii}[1]{\vr{#1_3}}
\newcommand{\vriv}[1]{\vr{#1_4}}
\newcommand{\vrv}[1]{\vr{#1_5}}
\newcommand{\vrj}[1]{\vr{#1_j}}
\newcommand{\vrn}[1]{\vr{#1_n}}


\newcommand{\defining}[1]{\mainindex{#1}{\em #1}}
\newcommand{\ide}[1]{{\schindex{#1}\frenchspacing\tt{#1}}}

\newcommand{\lambdaexp}{{\cf lambda} expression}
\newcommand{\Lambdaexp}{{\cf Lambda} expression}
\newcommand{\callcc}{{\tt call-with-current-continuation}}

% \reallyindex{SORTKEY}{HEADCS}{TYPE}
% writes (index-entry "SORTKEY" "HEADCS" TYPE PAGENUMBER)
% which becomes  \item \HEADCS{SORTKEY} mainpagenumber ; auxpagenumber ...

\global\def\reallyindex#1#2#3{%
\write\@indexfile{"#1" "#2" #3 \thepage}}

\newcommand{\mainschindex}[1]{\label{#1}\reallyindex{#1}{tt}{main}}
\newcommand{\mainindex}[1]{\reallyindex{#1@{\rm #1}{main}}}
\newcommand{\schindex}[1]{\reallyindex{#1}{tt}{aux}}
\newcommand{\sharpindex}[1]{\reallyindex{#1}{sharpfoo}{aux}}
%vj%\renewcommand{\index}[1]{\reallyindex{#1}{rm}{aux}}

\newcommand{\domain}[1]{#1}
\newcommand{\nodomain}[1]{}
%\newcommand{\todo}[1]{{\rm$[\![$!!~#1$]\!]$}}
\newcommand{\todo}[1]{}

% \frobq will make quote and backquote look nicer.
\def\frobqcats{%\catcode`\'=13 %\catcode`\{=13{}\catcode`\}=13{}
\catcode`\`=13{}}
{\frobqcats
\gdef\frobqdefs{%\def'{\singlequote}
\def`{\backquote}}}%\def\{{\char`\{}\def\}{\char`\}}
\def\frobq{\frobqcats\frobqdefs}

% \cf = code font
% Unfortunately, \cf \cf won't work at all, so don't even attempt to
% next constructions which use them...
\newcommand{\cf}{\frenchspacing\tt}

% Same as \obeycr, but doesn't do a \@gobblecr.
{\catcode`\^^M=13 \gdef\myobeycr{\catcode`\^^M=13 \def^^M{\\}}%
\gdef\restorecr{\catcode`\^^M=5 }}

{\catcode`\^^I=13 \gdef\obeytabs{\catcode`\^^I=13 \def^^I{\hbox{\hskip 4em}}}}

{\obeyspaces\gdef {\hbox{\hskip0.5em}}}

\gdef\gobblecr{\@gobblecr}

\def\setupcode{\@makeother\^}

% Scheme example environment
% At 11 points, one column, these are about 56 characters wide.
% That's 32 characters to the left of the => and about 20 to the right.

\newenvironment{x10noindent}{
  % Commands for scheme examples
  \newcommand{\ev}{\>\>\evalsto}
  \newcommand{\lev}{\\\>\evalsto}
  \newcommand{\unspecified}{{\em{}unspecified}}
  \newcommand{\scherror}{{\em{}error}}
  \setupcode
  \small \cf \obeytabs \obeyspaces \myobeycr
  \begin{tabbing}%
\qquad\=\hspace*{5em}\=\hspace*{9em}\=\kill%   was 16em
\gobblecr}{\unskip\end{tabbing}}

%\newenvironment{scheme}{\begin{schemenoindent}\+\kill}{\end{schemenoindent}}
\newenvironment{x10}{
  % Commands for scheme examples
  \newcommand{\ev}{\>\>\evalsto}
  \newcommand{\lev}{\\\>\evalsto}
  \renewcommand{\em}{\rmfamily\itshape}
  \newcommand{\unspecified}{{\em{}unspecified}}
  \newcommand{\scherror}{{\em{}error}}
  \setupcode
  \small \cf \obeyspaces \myobeycr
  \footnotesize
  \begin{tabbing}%
\qquad\=\hspace*{5em}\=\hspace*{9em}\=\+\kill%   was 16em
\gobblecr}{\unskip\end{tabbing}\normalsize}

\newcommand{\evalsto}{$\Longrightarrow$}

% Manual entries

\newenvironment{entry}[1]{
  \vspace{3.1ex plus .5ex minus .3ex}\noindent#1%
\unpenalty\nopagebreak}{\vspace{0ex plus 1ex minus 1ex}}

\newcommand{\exprtype}{syntax}

% Primitive prototype
\newcommand{\pproto}[2]{\unskip%
\hbox{\cf\spaceskip=0.5em#1}\hfill\penalty 0%
\hbox{ }\nobreak\hfill\hbox{\rm #2}\break}

% Parenthesized prototype
\newcommand{\proto}[3]{\pproto{(\mainschindex{#1}\hbox{#1}{\it#2\/})}{#3}}

% Variable prototype
\newcommand{\vproto}[2]{\mainschindex{#1}\pproto{#1}{#2}}

% Extending an existing definition (\proto without the index entry)
\newcommand{\rproto}[3]{\pproto{(\hbox{#1}{\it#2\/})}{#3}}

% Grammar environment

\newenvironment{grammar}{
  \def\:{& \goesto{} &}
  \def\|{& $\vert$& }
  \def\opt{$^?$\ }
  \def\star{$^*$\ }
  \def\plus{$^+$\ }
  \em
  \begin{tabular}{rcl}
  }{\unskip\end{tabular}}

%\newcommand{\unsection}{\unskip}
\newcommand{\unsection}{{\vskip -2ex}}

% Commands for grammars
\newcommand{\arbno}[1]{#1\hbox{\rm*}}  
\newcommand{\atleastone}[1]{#1\hbox{$^+$}}

\newcommand{\goesto}{{\normalfont{::=}}}

% mark modifications (for the grammar) From Igor Pechtchanski/Watson/IBM@IBMUS
\newlength{\modwidth}\setlength{\modwidth}{0.005in}
\newlength{\modskip}\setlength{\modskip}{.4em}
\newlength{\@modheight}
\newlength{\@modpos}
\providecommand{\markmod}[1]{%
  \setlength{\@modheight}{#1}%
  \addtolength{\@modheight}{-0.06in}%
  \setlength{\@modpos}{\linewidth}%
  \addtolength{\@modpos}{0.285in}%         Magic
  \addtolength{\@modpos}{\modwidth}%
  \addtolength{\@modpos}{\modskip}%
  \marginpar{\vspace{-\@modheight}%
             \hspace{-\@modpos}%
             \rule{\modwidth}{#1}}%
}

% The index

\def\theindex{%\@restonecoltrue\if@twocolumn\@restonecolfalse\fi
%\columnseprule \z@
%!! \columnsep 35pt
\clearpage
\@topnewpage[
    \centerline{\large\bf\uppercase{Alphabetic index of definitions of concepts,}}
    \centerline{\large\bf\uppercase{keywords, and procedures}}
    \vskip 1ex \bigskip]
    \markboth{Index}{Index}
    \addcontentsline{toc}{chapter}{Alphabetic index of 
 definitions of concepts, keywords, and procedures}
    \bgroup %\small
    \parindent\z@
    \parskip\z@ plus .1pt\relax\let\item\@idxitem}

\def\@idxitem{\par\hangindent 40pt}

\def\subitem{\par\hangindent 40pt \hspace*{20pt}}

\def\subsubitem{\par\hangindent 40pt \hspace*{30pt}}

\def\endtheindex{%\if@restonecol\onecolumn\else\clearpage\fi
\egroup}

\def\indexspace{\par \vskip 10pt plus 5pt minus 3pt\relax}

\makeatother

%\newcommand{\Xten}{{\sf X10}}
%\newcommand{\XtenCurrVer}{{\sf X10 v1.7}}
%\newcommand{\java}{{\sf Java}}
%\newcommand{\Java}{{\sf Java}}

\newcommand{\Xten}{X10}
\newcommand{\XtenCurrVer}{\Xten{} v1.7}
\newcommand{\Java}{Java}
\newcommand{\java}{\Java{}}

\newcommand{\futureext}[1]{{\em \paragraph{Future Extensions.}#1}}
\newcommand{\tbd}{} % marker for things to be done later.
\newcommand{\limitation}[1]{{\em Limitation: #1}} % marker for things to be done later.


\newcommand\grammarrule[1]{\emph{#1}}

% Rationale

\newenvironment{rationale}{%
\bgroup\noindent{\sc Rationale:}\space}{%
\egroup}

% Notes

\newenvironment{note}{%
\bgroup\noindent{\sc Note:}\space}{%
\egroup}

\newenvironment{staticrule*}{%
\bgroup\noindent{\textsc{Static Semantics Rule:}\space}}{%
\egroup}

\newenvironment{staticrule}[1]{%
\bgroup\noindent{\textsc{Static Semantics Rule} (#1):\space}}{%
\egroup}

\newcommand\Sref[1]{\S\ref{#1}}
\newcommand\figref[1]{Figure~\ref{#1}}
\newcommand\tabref[1]{Table~\ref{#1}}
\newcommand\exref[1]{Example~\ref{#1}}

\newcommand\eat[1]{}



\begin{document}


\maketitle


\lstset{language=java,basicstyle=\ttfamily\small}

%\chapter{Featherweight Ownership and Immutability Generic Java}
\section{Introduction}


We begin with some definitions.

FX10 program consists of class declarations followed by the program's expression.
The source language does not contain


An expression/type is called \emph{closed} if it does not contain \proto nor
    free variables (but it may contain \cooked or other locations).
For example, the expression~$\code{new Foo<l>()}$ is closed, but $\code{new Foo<proto>()}$ is not closed.

Given a field $\code{A}~\hT~\hf$ and a method $\code{U m(}\ol{\hV}~\ol{\hx}\code{) K \lb~return e;~\rb}$ in class~\hC,
    we define
\beqst
    \code{\mtype{}(m,C)} &= \ol{\hV}\mapsto\hU\\
    \code{\mtype{}(m,C<K>)} &= [\hK/\proto]\code{\mtype{}(f,C)}\\
    \code{\mbody{}(m,C)} &= \he\\
    \code{\mproto{}(m,C)} &= \hK\\
    \code{\ftype{}(f,C)} &= \hT\\
    \code{\isVar{}(f,C)} &= (\code{A}=\code{var})\\
\eeq

Given a proto set~$P$, functions receive a cooker~$\hK ::= \proto ~|~ \cooked ~|~ {\hl}$,
    and return whether it is cooked or proto:
\beqst
\isCooked(\hK) &= \hK=\code{cooked} \text{~~or~~} (\hK=\hl \text{~~and~~}\hl \not \in P)\\
\isProto(\hK) &= \hK=\code{proto} \text{~~or~~} (\hK=\hl \text{~~and~~}\hl \in P)\\
\eeq

Note that for every cooker~\hK, either $\isCooked(\hK)$ or $\isProto(\hK)$ (and never both).

%\begin{cases}
%\hl & \he=\hl \\
%\code{error} & \text{otherwise} \\
%\end{cases}

Given an expression~\he, we define~$R(\he)$ to be the set
    of all ongoing constructors in~\he, i.e., all locations in subexpressions~\code{e;return l}.
Formally,
\[
R(\he) =
\begin{cases}
    R(\code{e'}) \cup \{ l \} & \text{if~}\he=(\code{e';return l}) \\
    R(\code{e'}) & \text{if~}\he=(\code{e'.f}) \\
    R(\code{e'}) \cup R(\code{e"}) & \text{if~}\he=(\code{e'.f=e"}) \\
    \cup R(\ol{\he'}) & \text{if~}\he=(\code{new T}\hparen{\ol{\he'}}) \\
    R(\code{e"}) \cup R(\ol{\he'}) & \text{if~}\he=(\code{e".m}\hparen{\ol{\he'}}) \\
    \end{cases}
\]

Summary of syntax used:
The comma operator~$,$ represents disjoint union.
Environment $\Gamma ::= \epsilon ~|~ \hx:\hT,\Gamma ~|~ \hl:\hT,\Gamma$.
Proto locations (locations whose constructor is still ongoing) $P::= \epsilon ~|~ \hl,P$.

A \emph{location}~\hl is a pointer to an object on the heap.
An \emph{object} has the form~$\code{C<l>}\hparen{\ol{\hv}}$, where~\hC is a class,~$\hl$ is the object's cooker, and~$\ol{\hv}$ are the values of the object's fields.
A \emph{heap} is a pair $\angular{H,P}$, where $H$ maps locations to objects, and $P \in \dom(H)$ is the set of proto locations.

A \emph{well-typed} heap~$\angular{H,P}$ satisfies:
    (i)~there is a linear order~$\Tprec$ over~$\dom{}(H)$ such that for every location~\hl, $H[\hl]=\code{C<l'>}\hparen{\ol{\hv}}$,
        we have~$\hl' \Tprec \hl$,
        and
    (ii)~each non-null field location is a subtype (using $\Pst$) of the declared field type.
A heap~$\angular{H,P}$ is \emph{well-typed for~\he} if~$\angular{H,P\cup R(\he)}$ is well-typed.

Summary of judgements:
\beqst
\Gamma,P & \vdash \hT \st \code{T'}\\
\Gamma,P & \vdash \hT \Pst \code{T'}\\
\Gamma,P & \vdash \he : \hT\\
P & \vdash H,\he \rightarrow H',\code{e'}\\
& \vdash \angular{H,P} \text{~is well-type}\\
& \vdash \angular{H,P} \text{~is well-type for~\he}\\
\eeq

\begin{smaller}

\begin{figure*}[htpb!]
\begin{center}
\begin{tabular}{|l|l|}
\hline

$\hK ::= \proto ~|~ \cooked ~|~ \textbf{\hl}$ & cooKer. \\

$\code{T} ::= \code{C<K>}$ & Type. \\

$\code{A} ::= \code{var}~|~\code{val}$ & Assignable (\code{var}) or final (\code{val}) field. \\

$\code{F} ::= \code{A}~\hT~\hf\texttt{;}$ & Field declaration. \\

$\hM ::= \code{T} ~ \hm\hparen{\ol{\code{T}} ~ \ol{\hx}}~\hK ~ \lb\ \hreturn ~ \he\texttt{;}~\rb$
& Method declaration. \\

$\hL ::= \hclass ~ \hC\code{~extends~C'} \lb\ \ol{\code{F}}~\ol{\hM}~\rb$
& cLass declaration. \\


$\hv ::= \code{null} ~|~ \textbf{\hl} $
& Values: either \code{null} or a location~\hl. \\


% No cast: \hparen{\hT} ~ \he ~|~
$\he ::= \hv ~|~ \hx ~|~ \he.\hf ~|~ \he.\hf = \he ~|~ \he.\hm\hparen{\ol{\he}} ~|~ \hnew ~ \hT\hparen{\ol{\he}}  ~|~ \textbf{\he\code{;return l}}$
& Expressions. \\ %: values, parameters, field access\&assignment, invocation, \code{new} start\&finish

\hline
\end{tabular}
\end{center}
\caption{FX10 Syntax. Class declarations in FX10 cannot contain locations~\hl (marked with a boldface).
    Such locations are created during the reduction process (see \RULE{R-New} in \Ref{Figure}{reduction}).}
\label{Figure:syntax}
\end{figure*}


\begin{figure*}[!bt]
\begin{center}
\begin{tabular}{|c|}
\hline


$\typerule{
}{
  \Gamma \vdash \hT \st \hT
}$
~\RULE{(S1)}\quad

$\typerule{
  \Gamma \vdash \code{S} \st \hT
   \gap
  \Gamma \vdash \hT \st \hU
}{
  \Gamma \vdash \code{S} \st \hU
}$
~\RULE{(S2)}\quad
$\typerule{
  \code{class C}_1\code{~extends C}_2^\hK
}{
  \Gamma \vdash \hC_1^\hK \st \code{C}_2^\hK
}$
~\RULE{(S3)}

\\

$\typerule{
  \hl \in \Gamma[K]
}{
  \Gamma \vdash \hl \st \proto
}$
~\RULE{(S4)}\quad
$\typerule{
  \hl \not \in \Gamma[K]
}{
  \Gamma \vdash \hl \st \cooked
}$
~\RULE{(S5)}\quad
$\typerule{
  \hl \not \in \Gamma[K]
}{
  \Gamma \vdash \cooked \st \hl
}$
~\RULE{(S6)}\quad

$\typerule{
    \Gamma \vdash \code{K} \st \code{K}'
}{
  \Gamma \vdash \hC^\code{K} \st \hC^{\code{K}'}
}$
~\RULE{(S7)}\quad
\\


$\typerule{
  \Gamma \vdash \code{T} \st \code{T}'
}{
  \Gamma \vdash \code{T} \Pst \code{T}'
}$
~\RULE{(S8)}\quad
$\typerule{
  \hl \not \in \Gamma[K]
}{
  \Gamma \vdash \code{C}^\hl \Pst \code{C}^\hK
}$
~\RULE{(S9)}\quad
\\

\hline
\end{tabular}
\end{center}
\caption{FX10 Subtyping Rules. The subtyping relation is $\st$, whereas $\Pst$ is the pointing relation (anything can point to a cooked object).}
\label{Figure:subtyping}
\end{figure*}


\begin{figure*}[t]
\begin{center}
\begin{tabular}{|c|}
\hline
$\typerule{
  \Gamma,P \cup \{ \hl \} \vdash \he:\hT
}{
  \Gamma,P \vdash \code{e;return l} : \Gamma(\hl)
}$
\quad \RULE{(T-return)}
\\\\

$\typerule{
\hK'=
\begin{cases}
\bot & \hK=\cooked \\
\hK & \text{otherwise} \\
\end{cases}
    \gap
  \mtype{}(\code{build},\code{C<K'>})=\ol{\code{T}}\rightarrow\code{Object}
    \gap
  \Gamma,P \vdash \ol{\he}:\ol{\code{V}}
    \gap
  \Gamma,P \vdash \ol{\code{V}} \st \ol{\code{T}}
}{
  \Gamma,P \vdash \code{new C<K>(}\ol{\he}\code{)} : \code{C<K>}
}$
\quad \RULE{(T-New)}\\

$\typerule{
}{
  \Gamma,P \vdash \hx : \Gamma(\hx)
}$
\quad \RULE{(T-Var)}
\qquad
$\typerule{
}{
  \Gamma,P \vdash \code{null} : \hT
}$
\quad \RULE{(T-null)}
\qquad
$\typerule{
}{
  \Gamma,P \vdash \hl : \Gamma(\hl)
}$
\quad \RULE{(T-Location)}\\\\

$\typerule{
  \Gamma,P \vdash \he:\code{C<K>}
    \gap
  \isCooked(\hK)
    \gap
  \ftype{}(\hf,\hC)=\code{C'}
}{
  \Gamma,P \vdash \he.\hf : \code{C'<K>}
}$
\quad \RULE{(T-Field-Access)}\\\\


$\typerule{
  \Gamma,P \vdash \he:\code{C<K>}
    \gap
  \ftype{}(\hf,\hC)=\code{C'}
    \gap
  \Gamma,P \vdash \code{e'}:\code{T'}
    \gap
  \Gamma,P \vdash \code{T'} \Pst \code{C'<K>}
    \\
  \big(\isProto(\hK) \text{~~or~~} \isVar{}(\hf,\hC)\big)
}{
  \Gamma,P \vdash \he.\hf = \code{e'} : \code{T'}
}$
\quad \RULE{(T-Field-Assignment)}\\\\

$\typerule{
  \Gamma,P \vdash \he':\code{C<K>}
    \gap
  \mtype{}(\hm,\code{C<K>})=\ol{\code{T}}\rightarrow\code{U}
    \gap
  \Gamma,P \vdash \ol{\he}:\ol{\code{V}}
    \gap
  \Gamma,P \vdash \ol{\code{V}} \st \ol{\code{T}}
    \\
  \isCooked(\hK)=\isCooked(\mproto{}(\hm,\code{C}))
}{
  \Gamma,P \vdash \he'\code{.m(}\ol{\he}\code{)} : \code{U}
}$
\quad \RULE{(T-Invoke)}\\


\hline
\end{tabular}
\end{center}
\caption{FX10 Expression Typing Rules.}
\label{Figure:expressions}
\end{figure*}


\begin{figure*}[t]
\begin{center}
\begin{tabular}{|c|}
\hline

$\typerule{
  \hl \not \in \dom(H)
    \gap
  \code{l'} =
    \begin{cases}
    \hl & \text{if~}\isCooked(\hK) \\
    \hK & \text{otherwise} \\
    \end{cases}
}{
  P \vdash H,\code{new C<K>}\hparen{\ol{\hv}} \rightarrow H[\hl \mapsto \code{C<l'>}\hparen{\ol{\code{null}}}],\hl\code{.build}\hparen{\ol{\hv}}\code{;return l}
}$
\quad \RULE{(R-New)}\\\\

$\typerule{
  H[\hl] = \code{C<K>}\hparen{\ol{\hv}}
    \gap
  \fields{}(\hC)=\ol{\hf}
}{
  P \vdash H,\hl.\hf_i \rightarrow H,\hv_i
}$
\quad \RULE{(R-Field-Access)}
\\\\

$\typerule{
  H[\hl] = \code{C<K>}\hparen{\ol{\hv}}
    \gap
  \fields{}(\hC)=\ol{\hf}
}{
  P \vdash H,\hl.\hf_i = \hv' \rightarrow H[\hl \mapsto \code{C<K>}\hparen{[\hv'/\hv_i]\ol{\hv}}],\hv'
}$
\quad \RULE{(R-Field-Assignment)}\\\\


$\typerule{
}{
  P \vdash H,\code{v;return l} \rightarrow H,\hl
}$
\quad \RULE{(R-return)}
\gap

$\typerule{
  H[\hl] = \code{C<K>}\hparen{\ldots}
    \gap
  \mbody{}(\hm,\code{C})=\ol{\hx}.\he'
}{
  P \vdash H,\hl\code{.m(}\ol{\hv}\code{)} \rightarrow H, [\ol{\hv}/\ol{\hx}, \hl/\this, \hl/\proto]\he'
}$
\quad \RULE{(R-Invoke)}\\\\



$\typerule{
  P \cup \{\hl\} \vdash H,\he \rightarrow H',\code{e'}
}{
  P \vdash H,\code{e;return l} \rightarrow H',\code{e';return l}
}$
\quad \RULE{(R-c1)}
\gap

$\typerule{
  P \vdash H,\he \rightarrow H',\code{e'}
}{
  P \vdash H,\code{e.f} \rightarrow H',\code{e'.f}
}$
\quad \RULE{(R-c2)}
\\\\

$\typerule{
  P \vdash H,\he \rightarrow H',\code{e'}
}{
  P \vdash H,\code{e.f=e"} \rightarrow H',\code{e'.f=e"}
}$
\quad \RULE{(R-c3)}
\gap

$\typerule{
  P \vdash H,\he \rightarrow H',\code{e'}
}{
  P \vdash H,\code{l.f=e} \rightarrow H',\code{l.f=e'}
}$
\quad \RULE{(R-c4)}
\\\\

$\typerule{
  P \vdash H,\he \rightarrow H',\code{e'}
}{
  P \vdash H,\code{new C<K>}\hparen{\ol{\hv},\he,\ol{\code{e"}}} \rightarrow H',\code{new C<K>}\hparen{\ol{\hv},\code{e'},\ol{\code{e"}}}
}$
\quad \RULE{(R-c5)}
\\\\


$\typerule{
  P \vdash H,\he \rightarrow H',\code{e'}
}{
  P \vdash H,\he\code{.m(}\ol{\code{e"}}\code{)} \rightarrow H',\code{e'}\code{.m(}\ol{\code{e"}}\code{)}
}$
\quad \RULE{(R-c6)}
\gap

$\typerule{
  P \vdash H,\he \rightarrow H',\code{e'}
}{
  P \vdash H,\code{l.m(}\ol{\hv},\he,\ol{\code{e"}}\code{)} \rightarrow H',\code{l.m(}\ol{\hv},\code{e'},\ol{\code{e"}}\code{)}
}$
\quad \RULE{(R-c7)}
\gap

\\
\hline
\end{tabular}
\end{center}
\caption{FX10 Reduction Rules. The congruence rules have the initial \RULE{R-c}.}
\label{Figure:reduction}
\end{figure*}

\end{smaller}


Next we describe the syntax (\Ref{Figure}{syntax}),
    subtyping rules (\Ref{Figure}{subtyping}),
    expression typing rules (\Ref{Figure}{expressions}),
    and reduction rules (\Ref{Figure}{reduction}).

\section{Syntax}
Obviously, class declarations cannot contain locations.

\section{Subtyping}


\section{Typing}
\paragraph{Method typing}
If \proto appears in $\mtype{}(\hm,\hC)$ then $\mproto{}(\hm,\hC)=\proto$.

An overriding method must keep the same $\mtype$ and $\mproto$.

In class~\hC, when typing a method:
        $\code{U} ~ \hm\hparen{\ol{\code{V}} ~ \ol{\hx}} ~ \hK~ \lb\ \hreturn ~ \he\texttt{;} \rb$\\
        we use an environment~$\Gamma=\{\ol{\hx}:\ol{\code{T}}, \this:\code{C<K>}\}$, $P=\{\}$,
        and we must prove that~$\Gamma,P \vdash \he:\code{S}$
        and~$\Gamma,P \vdash \code{S} \st \code{U}$.

\paragraph{Expression typing}
See \Ref{Figure}{expressions}.



\end{document}
