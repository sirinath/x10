\section{Functions}
\label{Functions}
\label{functions}
\index{functions}

The runtime entities in \Xten{} are of three kinds: \emph{structs},
\emph{objects}, and \emph{functions}. This section is concerned with
functions and their types -- how they are created, and what operations
can be performed on them.

Intuitively, a function is a piece of code which can be applied to a
set of arguments to produce a value. (As in most programming
languages, functions in \Xten{} are partial.)

A \emph{closure literal expression} 
\xcd"(x1:T1,..,xn:Tn){c}:T=>e"
creates a function of type \xcd"(x1:T1,...,xn:Tn){c}=>T" (\Sref{FunctionType}). The constraint
clause \xcd"{c}" is optional and defaults to \xcd"{true}". In some cases the
return type \Xcd{T} is also optional and defaults to the type of \Xcd{e}. If a
formal \Xcd{xi} does not occur in \Xcd{Ti},\Xcd{c},\Xcd{T} or \Xcd{e}, the declaration \Xcd{xi:Ti} may be
replaced by just \Xcd{Ti}. The body \Xcd{e} of such an expression is type-checked
in an environment in which \Xcd{c} is true. At runtime, function execution
results in the evaluation of \Xcd{e} in an environment in which each formal
is bound to the given actual parameter.

The \emph{method selector expression} \Xcd{e.m(T1,...,Tn)} is type
correct only if it is the case that the static type of \Xcd{e} is a
class or struct or interface with a method
\Xcd{m(x1:T1,...xn:Tn)\{c\}:T} defined on it (for some
\Xcd{x1,...,xn,c,T)}. At runtime the evaluation of this expression
evaluates \Xcd{e} to a value \Xcd{v} and creates a function \Xcd{f}
which, when applied to an argument list \Xcd{(a1,...,an)} (of the right
type) yields the value obtained by evaluating \Xcd{v.m(a1,...,an)}.


An object \Xcd{o} that is an instance of a class implementing a function
type \Xcd{(x1:T1,...,xn:Tn)\{c\}:T} also behaves like a function. Indeed, an
object may behave like any (fixed) number of functions, since the
class it is an instance of may implement any (fixed) number of
function types. Note again that while structs and objects have a
notion of equality (\Xcd{==}), functions do not. Hence if one desires to use
a function \Xcd{f} in an \Xcd{==} operation, it is necessary to first cast it to a
class or struct type.


  
%\section{Implementation Notes}
%\begin{itemize}
%
%\item Note that e.m.(T1,...,Tn) will evaluate e to create a
%  function. This function will be applied later to given
%  arguments. Thus this syntax can be used to evaluate the receiver of
%  a method call ahead of the actual invocation. The resulting function
%  can be used multiple times, of course.
%\end{itemize}


