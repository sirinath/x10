\chapter{Places}\label{XtenPlaces}\index{places}

An \Xten{} place is a repository for data and activities. Each place
is to be thought of as a locality boundary: the activities running in
a place may access data items located at that place with the
efficiency of on-chip access. Accesses to remote places may take
orders of magnitude longer.

{}\Xten{} provides a built-in struct, \xcd"x10.lang.Place"; all
places are instances of this struct. 

In \XtenCurrVer{}, the set of places available to a computation is
determined at the time that the program is run and remains fixed
through the run of the program. The number of places available 
may be determined by reading (\xcd"Place.MAX_PLACES"). (This number
is specified from the command line/configuration information; 
see associated {\tt README} documentation.)

All scalar objects created during program execution are located in one
place, though they may be referenced from other places. Aggregate
objects (arrays) may be distributed across multiple places using
distributions.

The set of all places in a running instance of an \Xten{} program may
be obtained through the \xcd"const" field \xcd"Place.places".  (This
set may be used to define distributions, for instance,
\Sref{XtenDistributions}.) 


The set of all places is totally ordered.  The first place may be
obtained by reading \xcd"Place.FIRST_PLACE". The initial activity for
an \Xten{} computation starts in this place
(\Sref{initial-computation}). For any place, the operation \xcd"next()"
returns the next place in the total order (wrapping around at the
end). Further details on the methods and fields available on this
class may be obtained by consulting the API documentation.

\begin{note}
Future versions of the language may permit user-definable
places, and the ability to dynamically create places.
\end{note}

\begin{staticrule*}
Variables of type \xcd"Place" must be initialized and are implicitly
\xcd"val".  
\end{staticrule*}

\section{Place expressions}
Any expression of type \xcd"Place" is called a place expression. 
Examples of place expressions are \xcd"this.home" (the place
at which the current object lives), \xcd"here"
(the place where the current activity is executing), etc.

Place expressions are used in the following contexts: 
\begin{itemize}
%\item As a  place type in a type (\Sref{PlaceTypes}).
\item As a target for an \xcd"async" activity or a future
(\Sref{AsyncActivity}).
\item In a cast expression (\Sref{ClassCast}).
\item In an \xcd"instanceof" expression (\Sref{instanceOf}).
\item In stable equality comparisons, at type \xcd"Place".
\end{itemize}

Like values of any other type, places may be passed as arguments
to methods, returned from methods, stored in fields etc.

\section{\Xcd{here}}\index{here}\label{Here}
\Xten{} supports a special indexical constant\footnote{ An indexical constant
  is one whose value depends on its context of use, like \xcd`this`.}
\xcd"here", which evaluates to the place at which the current activity is
running.

\begin{grammar}
Expression \: \xcd"here" \\
\end{grammar}

\begin{example}
For example, the following method looks through a collection of \xcd`Thing`s
for ones which belong in the current place \xcd`here`, and deals with the
things which do.  Note that every object \xcd`thing` has a property
\xcd`thing.home` giving its home location.
%~~gen
%package Places.Are.For.Graces;;
%abstract class Thing {}
%class DoMine {
%  static def dealWith(Thing) {}	
%~~vis
\begin{xten}
  public static def dealWithLocal(things: Rail[Thing]) {
     for(thing in things) {
    	 if (thing.home == here) 
            dealWith(thing);
     }	  
  }
\end{xten}
%~~siv
%}
%~~neg



\end{example}

\xcd`here` is frequently used in constraints, quite often of the form
\xcd`obh.home == here`. Such constraints are necessary to check that a
non-global method can be called on \xcd`ob`: 


%TODO~~gen
% package Places.Are.For.Aces.Of.Spaces;
% class Thing {
% def nonGlobalMethod():Void{}
% static def example() {
%TODO~~vis
\begin{xten}
val ob : Thing{self.home == here} = new Thing();
ob.nonGlobalMethod();
\end{xten}
%TODO~~siv
%}}
%TODO~~neg

This idiom is so common and useful that the constraint
\xcd`T{self.home==here}` can be abbreviated as \xcd`T!`: 

%TODO~~gen
% package Places.Are.For.Aces.Of.Graces;
% class Thing {
% def nonGlobalMethod():Void{}
% static def example() {
%TODO~~vis
\begin{xten}
val ob : Thing! = new Thing();
ob.nonGlobalMethod();
\end{xten}
%TODO~~siv
%}}
%TODO~~neg

%%PLACES-REMOVED-EXAMPLE%%\begin{example}
%%PLACES-REMOVED-EXAMPLE%%
%%PLACES-REMOVED-EXAMPLE%%%%TODO -- redo this example to something that's useful.
%%PLACES-REMOVED-EXAMPLE%%
%%PLACES-REMOVED-EXAMPLE%%The code:
%%PLACES-REMOVED-EXAMPLE%%
%%PLACES-REMOVED-EXAMPLE%%%~-~gen
%%PLACES-REMOVED-EXAMPLE%%% package Places.Are.For.Graces.Or.Maybe.Aces;
%%PLACES-REMOVED-EXAMPLE%%%~-~vis
%%PLACES-REMOVED-EXAMPLE%%\begin{xten}
%%PLACES-REMOVED-EXAMPLE%%class F {
%%PLACES-REMOVED-EXAMPLE%%  public def m(a: F) {
%%PLACES-REMOVED-EXAMPLE%%    val OldHere: Place = here;
%%PLACES-REMOVED-EXAMPLE%%    async (a) {
%%PLACES-REMOVED-EXAMPLE%%      Console.OUT.println("OldHere == here:" 
%%PLACES-REMOVED-EXAMPLE%%                         + (OldHere == here));
%%PLACES-REMOVED-EXAMPLE%%    }
%%PLACES-REMOVED-EXAMPLE%%  }
%%PLACES-REMOVED-EXAMPLE%%  public static def main(Rail[String]) {
%%PLACES-REMOVED-EXAMPLE%%    new F().m( (at(Place.FIRST_PLACE.next()) new F()));
%%PLACES-REMOVED-EXAMPLE%%  }
%%PLACES-REMOVED-EXAMPLE%%}  
%%PLACES-REMOVED-EXAMPLE%%\end{xten}
%%PLACES-REMOVED-EXAMPLE%%%~-~siv
%%PLACES-REMOVED-EXAMPLE%%%
%%PLACES-REMOVED-EXAMPLE%%%~-~neg
%%PLACES-REMOVED-EXAMPLE%%
%%PLACES-REMOVED-EXAMPLE%%
%%PLACES-REMOVED-EXAMPLE%%\noindent will print out \xcd"true" iff the computation was configured
%%PLACES-REMOVED-EXAMPLE%%to start with the number of places set to \xcd"1". 
%%PLACES-REMOVED-EXAMPLE%%\end{example}

