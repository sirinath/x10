
\section{Type definitions}

With value arguments, type arguments, and constraints, the
syntax for \Xten{} types can often be verbose;
\Xten{} therefore provides {\em type definitions}
to allow users to define new type constructors.

Type definitions have the following syntax:

\begin{grammar}
TypeDefinition \: 
                \xcd"type"~Identifier
                           ( \xcd"[" TypeParameters \xcd"]" )\opt \\
                        && ( \xcd"(" Formals \xcd")" )\opt
                            Constraint\opt \xcd"=" Type \\
\end{grammar}

\noindent
A type definition can be thought of as a type-valued function,
mapping type parameters and value parameters to a concrete type.
%
The following examples are legal type definitions:
\begin{xten}
type StringSet = Set[String];
type MapToList[K,V] = Map[K,List[V]];
type Nat = Int{self>=0};
type Int(x: Int) = Int{self==x};
type Int(lo: Int, hi: Int) = Int{lo <= self, self <= hi};
\end{xten}

As the two definitions of \xcd"Int" demonstrate, type definitions may 
be overloaded: two type definitions with different numbers of type
parameters or with different types of value
parameters, according to the method overloading rules
(\Sref{MethodOverload}), define distinct types.

Type definitions may appear as class members or in the body of a
method, constructor, or initializer.

Type definitions are applicative, not generative; that is, they
define aliases for types and do not introduce new types.
Thus, the following code is legal:
\begin{xten}
type A = Int;
type B = String;
type C = String;
a: A = 3;
b: B = new C("Hi");
c: C = b + ", Mom!";
\end{xten}
An instance of a defined type with no type parameters and no
value parameters may 
A \xcd"new" expression may be used to instantiate an instance
of a type with a defining class type
by a type definition.
The type
has the same constructors with the same signature as its defining type;
however, a
constructor may not be invoked using a given defined type 
name if the constructor return type is not a subtype of the
defined type.

All type definitions are members of their enclosing package or
class.  A compilation unit may have one or more type definitions
or class or interface declarations with the same name, as long
as the definitions have distinct parameters according to the
method overloading rules (\Sref{MethodOverload}).

