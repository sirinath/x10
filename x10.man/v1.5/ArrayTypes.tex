\section{Array Type Constructors}
\label{ArrayTypeConstructors}\index{array types}

{}\XtenCurrVer{} does not have array class declarations
(\S~\ref{XtenArrays}). This means that user cannot define new array
class types. Instead arrays are created as instances of array types
constructed through the application of {\em array type constructors}
(\S~\ref{XtenArrays}).

The array type constructor takes as argument a type (the {\em base
type}), an optional distribution (\S~\ref{XtenDistributions}), and
optionally the keyword \xcd"value":
\begin{verbatim}
439   ArrayType ::= array [ Type ]
441     | value array [ Type ]
442     | array [ Type ] ( Expression )
444     | value array [ Type ] ( Expression )
\end{verbatim}

The array type \xcd"array[T]" is the type of all
reference arrays of base type \xcd"T". Such an array can take on any
distribution, over any region. 

The qualifier \xcd"value" specifies that the array
is a \xcd"value" array. The array elements of a \xcd"value" array are
all \xcd"final".\footnote{Note that the base type of a \xcd"value" array can be a value class or a reference class, just as the 
type of a \xcd"final" variable can be a value class or a reference class.}
If the qualifier is not specified, the array is a reference array.

\XtenCurrVer{} also allows a distribution to be specified 
as a property initializer on the array type.
The distribution must be an expression of type
\xcd"distribution" (e.g., a \xcd"final" variable) whose
value does not depend on the value of any mutable variable.

\Xten{} also supports dependent types for arrays,
e.g.,
\xcd"array[double]{rank==3}" is the type of all arrays of 
\xcd"double" of rank \xcd"3".

The array type
\xcd"array[T](n)" and its synonym
\xcd"array[T]([0:n-1]->here)" corresponds to a local
zero-indexed array, such as in Java.
% \xcd"array[T]{rank==1,dist==0:self.length-1->here,region==0:self.length-1"}
% corresponds to a Java array.

\if 0

Arrays are as if defined by the following class:
\begin{xten}
class array[T](dist: distribution, region: region, rank: int, zeroBased: boolean, onePlace: boolean, rail: boolean)
\end{array}

Array constructors:
\begin{xten}
type array[T](D: distribution) = array[T]{dist=D, region=D.region, zeroBased=???, onePlace=???, rail=???}
type array[T](R: region) = array[T]{dist=D, region=D.region, zeroBased=D.region.zeroBased, onePlace=D.onePlace, rail=R.rail}
type array[T](n: int) = array[T]([0:n]->here);
\end{xten}

\fi
