\subsection{Implementation Notes}
Clocks may be implemented efficiently with message passing, e.g.{} by
using short-circuit ideas in \cite{SaraswatPODC88}.  Recall that every
activity is spawned with references to a fixed number of clocks. Each
reference should be thought of as a global pointer to a location in
some place representing the clock. (We shall discuss a further
optimization below.) Each clock keeps two counters: the total number
of outstanding references to the clock, and the number of activities
that are currently suspended on the clock.

When an activity $A$ spawns another activity $B$ that will reference a
clock $c$ referenced by $A$, $A$ {\em splits} the reference by sending
a message to the clock. Whenever an activity drops a reference to a
clock, or suspends on it, it sends a message to the clock. 

The optimization is that the clock can be represented in a distributed
fashion. Each place keeps a local counter for each clock that is
referenced by an activity in that place. The global location for the
clock simply keeps track of the places that have references and that
are quiescent. This can reduce the inter-place message traffic
significantly.
