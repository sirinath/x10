\documentclass{article}
\extrapart{Appendix: Macros}

This appendix describes an extension to Scheme that allows programs
to define and use new derived expression types.
A derived expression type that has been defined using this extension
is called a {\em macro}.\mainindex{macro}

Derived expression types introduced using this extension have
the syntax
\begin{scheme}
(\hyper{keyword} \arbno{\hyper{datum}})%
\end{scheme}%
where \hyper{keyword} is an identifier that uniquely determines the
expression type.  This identifier is called the {\em syntactic
keyword}\index{syntactic keyword}, or simply {\em
keyword}\index{keyword}, of the macro\index{macro keyword}.  The
number of the \hyper{datum}s, and their syntax, depends on the
expression type.

Each instance of a macro is called a {\em use}\index{macro use}
of the macro.
The set of rules, or more generally the procedure, that specifies
how a use of a macro is transcribed into a more primitive expression
is called the {\em transformer}\index{macro transformer}
of the macro.

The extension described here consists of three parts:

\begin{itemize}

\item A set of expressions used to establish that certain identifiers
are macro keywords, associate them with macro transformers, and control
the scope within which a macro is defined,

\item a convenient pattern language that makes it easy to write
transformers for most macros, and

\item a compatible low-level macro facility for writing macro
transformers that cannot be expressed by the pattern language.

\end{itemize}

With this extension, there are no reserved identifiers.  The syntactic
keyword of a macro may shadow variable bindings, and local variable
bindings may shadow keyword bindings.  \index{keyword} All macros
defined using the pattern language are ``hygienic'' and
``referentially transparent'':\mainindex{hygienic}
\mainindex{referentially transparent}

\begin{itemize}

\item If a macro transformer inserts a binding for an identifier
(variable or keyword), the identifier will in effect be renamed
throughout its scope to avoid conflicts with other identifiers.

\item If a macro transformer inserts a free reference to an
identifier, the reference refers to the binding that was visible
where the transformer was specified, regardless of any local
bindings that may surround the use of the macro.

\end{itemize}

%The low-level facility permits non-hygienic macros to be written,
%and may be used to implement the high-level pattern language.

This appendix is divided into three major sections.  The first
section describes the expressions and definitions used to
introduce macros, i.e.~to bind identifiers to macro
transformers.

The second section describes the pattern language.  This pattern
language is sufficient to specify most macro transformers, including
those for all the derived expression types from
section~\ref{derivedexps}.  The primary limitation of the pattern
language is that it is thoroughly hygienic, and thus cannot express
macros that bind identifiers implicitly.

The third section describes a low-level macro facility that could be
used to implement the pattern language described in the second
section.  This low-level facility is also capable of expressing
non-hygienic macros and other macros whose transformers cannot be
described by the pattern language, and is important as an example of a
more powerful facility that can co-exist with the high-level pattern
language.

The particular low-level facility described in the third
section is but one of several low-level facilities that have been
designed and implemented to complement the pattern language described
in the second section.  The design of such low-level macro facilities
remains an active area of research, and descriptions of alternative
low-level facilities will be published in subsequent documents.

% The fourth section describes some features that would make the
% low-level macro facility easier to use directly.

\subsection*{Binding syntactic keywords}
\label{bindsyntax}

{\cf Define-syntax}, {\cf let-syntax}, and {\cf letrec-syntax} are
analogous to {\cf define}, {\cf let}, and {\cf letrec}, but they bind
syntactic keywords to macro transformers instead of binding variables
to locations that contain values.  Furthermore, there is no {\cf
define-syntax} analogue of the internal definitions described in
section~\ref{internaldefines}.

\begin{rationale}
As discussed below, the syntax and scope rules for definitions
give rise to syntactic ambiguities when syntactic keywords are
not reserved.
Further ambiguities would arise if {\cf define-syntax}
were permitted at the beginning of a \meta{body}, with scope
rules analogous to those for internal definitions.
\end{rationale}

These new expression types and the pattern language described in
section~\ref{patternlanguage} are added to Scheme by augmenting the
BNF in section~\ref{BNF} with the following new productions.  Note
that the identifier {\cf ...} used in some of these productions is not
a metasymbol.

\begin{grammar}%
\meta{expression} \: \meta{macro use}
\>  \| \meta{macro block}

\meta{macro use} \: (\meta{keyword} \arbno{\meta{datum}})
\meta{keyword} \: \meta{identifier}

\meta{macro block} \:
\>  \> (let-syntax (\arbno{\meta{syntax spec}}) \meta{body})
\>  \| (letrec-syntax (\arbno{\meta{syntax spec}}) \meta{body})
\meta{syntax spec} \: (\meta{keyword} \meta{transformer spec})
\meta{transformer spec} \:
\>  \> (syntax-rules (\arbno{\meta{identifier}}) \arbno{\meta{syntax rule}})
\meta{syntax rule} \: (\meta{pattern} \meta{template})
\meta{pattern} \: \meta{pattern identifier}
\>  \| (\arbno{\meta{pattern}})
\>  \| (\atleastone{\meta{pattern}} . \meta{pattern})
\>  \| (\arbno{\meta{pattern}} \meta{pattern} \meta{ellipsis})
\>  \| \meta{pattern datum}
\meta{pattern datum} \: \meta{vector}
\>  \| \meta{string}
\>  \| \meta{character}
\>  \| \meta{boolean}
\>  \| \meta{number}
\meta{template} \: \meta{pattern identifier}
\>  \| (\arbno{\meta{template element}})
\>  \| (\atleastone{\meta{template element}} . \meta{template})
\>  \| \meta{template datum}
\meta{template element} \: \meta{template}
\>  \| \meta{template} \meta{ellipsis}
\meta{template datum} \: \meta{pattern datum}
\meta{pattern identifier} \: \meta{any identifier except {\cf ...}}
\meta{ellipsis} \: \meta{the identifier {\cf ...}}

\meta{command or definition} \: \meta{syntax definition}
\meta{syntax definition} \:
\>  \> (define-syntax \meta{keyword} \meta{transformer spec})
\>  \| (begin \arbno{\meta{syntax definition}})%
\end{grammar}

% It is an error for a program to contain more than one top-level
% \meta{definition} or \meta{syntax definition} of any identifier.
%
% [I flushed this because it isn't an error for a program to
% contain more than one top-level definition of an identifier,
% and I didn't want to introduce any gratuitous incompatibilities
% with the existing Scheme language. -- Will]

Although macros may expand into definitions in any context that permits
definitions, it is an error for a definition to shadow a syntactic
keyword whose meaning is needed to determine whether some definition in
the group of top-level or internal definitions that contains the
shadowing definition is in fact a definition, or is needed to determine
the boundary between the group and the expressions that follow the
group.  For example, the following are errors:

\begin{scheme}
(define define 3)

(begin (define begin list))

(let-syntax
  ((foo (syntax-rules ()
          ((foo (proc args ...) body ...)
           (define proc
             (lambda (args ...)
               body ...))))))
  (let ((x 3))
    (foo (plus x y) (+ x y))
    (define foo x)
    (plus foo x)))
\end{scheme}


\begin{entry}{%
\proto{let-syntax}{ \hyper{bindings} \hyper{body}}{\exprtype}}

\syntax
\hyper{Bindings} should have the form
\begin{scheme}
((\hyper{keyword} \hyper{transformer spec}) \dotsfoo)%
\end{scheme}
%where each \hyper{keyword} is an identifier,
%each \hyper{transformer spec} is an instance of {\cf syntax-rules}, and
Each \hyper{keyword} is an identifier,
each \hyper{transformer spec} is an instance of {\cf syntax-rules}, and
\hyper{body} should be a sequence of one or more expressions.  It is an error
for a \hyper{keyword} to appear more than once in the list of keywords
being bound.

\semantics
The \hyper{body} is expanded in the syntactic environment
obtained by extending the syntactic environment of the
{\cf let-syntax} expression with macros whose keywords are
the \hyper{keyword}s, bound to the specified transformers.
Each binding of a \hyper{keyword} has \hyper{body} as its region.

\begin{scheme}
(let-syntax ((when (syntax-rules ()
                     ((when test stmt1 stmt2 ...)
                      (if test
                          (begin stmt1
                                 stmt2 ...))))))
  (let ((if \schtrue))
    (when if (set! if 'now))
    if))                           \ev  now

(let ((x 'outer))
  (let-syntax ((m (syntax-rules () ((m) x))))
    (let ((x 'inner))
      (m))))                       \ev  outer%
\end{scheme}

\end{entry}

\begin{entry}{%
\proto{letrec-syntax}{ \hyper{bindings} \hyper{body}}{\exprtype}}

\syntax
Same as for {\cf let-syntax}.

\semantics
 The \hyper{body} is expanded in the syntactic environment obtained by
extending the syntactic environment of the {\cf letrec-syntax}
expression with macros whose keywords are the
\hyper{keyword}s, bound to the specified transformers.
Each binding of a \hyper{keyword} has the \hyper{bindings}
as well as the \hyper{body} within its region,
so the transformers can
transcribe expressions into uses of the macros
introduced by the {\cf letrec-syntax} expression.

\begin{scheme}
(letrec-syntax
  ((or (syntax-rules ()
         ((or) \schfalse)
         ((or e) e)
         ((or e1 e2 ...)
          (let ((temp e1))
            (if temp
                temp
                (or e2 ...)))))))
  (let ((x \schfalse)
        (y 7)
        (temp 8)
        (let odd?)
        (if even?))
    (or x
        (let temp)
        (if y)
        y)))        \ev  7%
\end{scheme}

\end{entry}

\begin{entry}{%
\proto{define-syntax}{ \hyper{keyword} \hyper{transformer spec}}{}}

\syntax
The \hyper{keyword} is an identifier, and the \hyper{transformer
spec} should be an instance of {\cf syntax-rules}.

\semantics
The top-level syntactic environment is extended by binding the
\hyper{keyword} to the specified transformer.

\begin{scheme}
(define-syntax let*
  (syntax-rules ()
    ((let* () body1 body2 ...)
     (let () body1 body2 ...))
    ((let* ((name1 val1) (name2 val2) ...)
       body1 body2 ...)
     (let ((name1 val1))
       (let* ((name2 val2) ...)
         body1 body2 ...)))))
\end{scheme}

\end{entry}

\subsection*{Pattern language}
\label{patternlanguage}

\begin{entry}{%
\proto{syntax-rules}{ \hyper{literals} \hyper{syntax rule} \dotsfoo}{}}

\syntax
\hyper{Literals} is a list of identifiers, and each \hyper{syntax rule}
should be of the form
\begin{scheme}
(\hyper{pattern} \hyper{template})%
\end{scheme}
where the \hyper{pattern} and \hyper{template} are as in the grammar
above.

\semantics An instance of {\cf syntax-rules} produces a new macro
transformer by specifying a sequence of hygienic rewrite rules.  A use
of a macro whose keyword is associated with a transformer specified by
{\cf syntax-rules} is matched against the patterns contained in the
\hyper{syntax rule}s, beginning with the leftmost \hyper{syntax rule}.
When a match is found, the macro use is transcribed hygienically
according to the template.

Each pattern begins with the keyword for the macro.  This keyword
is not involved in the matching and is not considered a pattern
variable or literal identifier.

\begin{rationale}
The scope of the keyword is determined by the expression or syntax
definition that binds it to the associated macro transformer.
If the keyword were a pattern variable or literal identifier, then
the template that follows the pattern would be within its scope
regardless of whether the keyword were bound by {\cf let-syntax}
or by {\cf letrec-syntax}.
\end{rationale}

An identifier that appears in the pattern of a \hyper{syntax rule} is
a pattern variable, unless it is the keyword that begins the pattern,
is listed in \hyper{literals}, or is the identifier ``{\cf ...}''.
Pattern variables match arbitrary input elements and
are used to refer to elements of the input in the template.  It is an
error for the same pattern variable to appear more than once in a
\hyper{pattern}.

Identifiers that appear in \hyper{literals} are interpreted as literal
identifiers to be matched against corresponding subforms of the input.
A subform
in the input matches a literal identifier if and only if it is an
identifier
and either both its occurrence in the macro expression and its
occurrence in the macro definition have the same lexical binding, or
the two identifiers are equal and both have no lexical binding.

% [Bill Rozas suggested the term "noise word" for these literal
% identifiers, but in their most interesting uses, such as a setf
% macro, they aren't noise words at all. -- Will]

A subpattern followed by {\cf ...} can match zero or more elements of the
input.  It is an error for {\cf ...} to appear in \hyper{literals}.
Within a pattern the identifier {\cf ...} must follow the last element of
a nonempty sequence of subpatterns.

More formally, an input form $F$ matches a pattern $P$ if and only if:

\begin{itemize}
\item $P$ is a pattern variable; or

\item $P$ is a literal identifier and $F$ is an identifier with the same
      binding; or

\item $P$ is a pattern list {\cf ($P_1$ $\dots$ $P_n$)} and $F$ is a
      list of $n$
      forms that match $P_1$ through $P_n$, respectively; or

\item $P$ is an improper pattern list
      {\cf ($P_1$ $P_2$ $\dots$ $P_n$ . $P_{n+1}$)}
      and $F$ is a list or
      improper list of $n$ or more forms that match $P_1$ through $P_n$,
      respectively, and whose $n$th ``cdr'' matches $P_{n+1}$; or

\item $P$ is % a pattern list
      of the form
      {\cf ($P_1$ $\dots$ $P_n$ $P_{n+1}$ \meta{ellipsis})}
      where \meta{ellipsis} is the identifier {\cf ...}
      and $F$ is
      a proper list of at least $n$ elements, the first $n$ of which match
      $P_1$ through $P_n$, respectively, and each remaining element of $F$
      matches $P_{n+1}$; or

\item $P$ is a pattern datum and $F$ is equal to $P$ in the sense of
      the {\cf equal?} procedure.
\end{itemize}

It is an error to use a macro keyword, within the scope of its
binding, in an expression that does not match any of the patterns.

When a macro use is transcribed according to the template of the
matching \hyper{syntax rule}, pattern variables that occur in the
template are replaced by the subforms they match in the input.
Pattern variables that occur in subpatterns followed by one or more
instances of the identifier
{\cf ...} are allowed only in subtemplates that are
followed by as many instances of {\cf ...}.
They are replaced in the
output by all of the subforms they match in the input, distributed as
indicated.  It is an error if the output cannot be built up as
specified.

%%% This description of output construction is very vague.  It should
%%% probably be formalized, but that is not easy...

Identifiers that appear in the template but are not pattern variables
or the identifier
{\cf ...} are inserted into the output as literal identifiers.  If a
literal identifier is inserted as a free identifier then it refers to the
binding of that identifier within whose scope the instance of
{\cf syntax-rules} appears.
If a literal identifier is inserted as a bound identifier then it is
in effect renamed to prevent inadvertent captures of free identifiers.

\begin{scheme}
(define-syntax let
  (syntax-rules ()
    ((let ((name val) ...) body1 body2 ...)
     ((lambda (name ...) body1 body2 ...)
      val ...))
    ((let tag ((name val) ...) body1 body2 ...)
     ((letrec ((tag (lambda (name ...)
                      body1 body2 ...)))
        tag)
      val ...))))

(define-syntax cond
  (syntax-rules (else =>)
    ((cond (else result1 result2 ...))
     (begin result1 result2 ...))
    ((cond (test => result))
     (let ((temp test))
       (if temp (result temp))))
    ((cond (test => result) clause1 clause2 ...)
     (let ((temp test))
       (if temp
           (result temp)
           (cond clause1 clause2 ...))))
    ((cond (test)) test)
    ((cond (test) clause1 clause2 ...)
     (or test (cond clause1 clause2 ...)))
    ((cond (test result1 result2 ...))
     (if test (begin result1 result2 ...)))
    ((cond (test result1 result2 ...)
           clause1 clause2 ...)
     (if test
         (begin result1 result2 ...)
         (cond clause1 clause2 ...)))))

(let ((=> \schfalse))
  (cond (\schtrue => 'ok)))           \ev ok%
\end{scheme}

The last example is not an error because the local variable {\cf =>}
is renamed in effect, so that its use is distinct from uses of the top
level identifier {\cf =>} that the transformer for {\cf cond} looks
for.  Thus, rather than expanding into

\begin{scheme}
(let ((=> \schfalse))
  (let ((temp \schtrue))
    (if temp ('ok temp))))%
\end{scheme}

which would result in an invalid procedure call, it expands instead
into

\begin{scheme}
(let ((=> \schfalse))
  (if \schtrue (begin => 'ok)))%
\end{scheme}

\end{entry}

\subsection*{A compatible low-level macro facility}
\label{lowlevelmacros}

Although the pattern language provided by {\cf syntax-rules} is the
preferred way to specify macro transformers, other low-level
facilities may be provided to specify more complex macro transformers.
In fact, {\cf syntax-rules} can itself be defined as a macro using the
low-level facilities described in this section.

The low-level macro facility described here introduces {\cf syntax}
as a new syntactic keyword analogous to {\cf quote}, and allows a
\meta{transformer spec} to be any expression.  This is accomplished by
adding the following two productions to the productions in
section~\ref{BNF} and in section~\ref{bindsyntax} above.

\begin{grammar}%
\meta{expression} \: (syntax \hyper{datum})
\meta{transformer spec} \: \meta{expression}%
\end{grammar}

The low-level macro system also adds the following procedures:

\begin{scheme}
unwrap-syntax          identifier->symbol
identifier?            generate-identifier
free-identifier=?      construct-identifier
bound-identifier=?
\end{scheme}

Evaluation of a program proceeds in two logical steps.  First the
program is converted into an intermediate language via macro-expansion,
and then the result of macro expansion is evaluated.  When it is
necessary to distinguish the second stage of this process from the
full evaluation process, it is referred to as ``execution.''

Syntax definitions, either lexical or global, cause an identifier to
be treated as a keyword within the scope of the binding.  The keyword
is associated with a transformer, which may be created implicitly
using the pattern language of {\cf syntax-rules} or explicitly using
the low-level facilities described below.

Since a transformer spec must be fully evaluated during the
course of expansion, it is necessary to specify the environment in
which this evaluation takes place.  A transformer spec is
expanded in the same environment as that in which the program is being
expanded, but is executed in an environment that is distinct from the
environment in which the program is executed.  This execution
environment distinction is important only for the resolution of global
variable references and assignments.  In what follows, the environment
in which transformers are executed is called the standard transformer
environment and is assumed to be a standard Scheme environment.

Since part of the task of hygienic macro expansion is to resolve
identifier references, the fact that transformers are expanded in the
same environment as the program means that identifier bindings in the
program can shadow identifier uses within transformers.  Since
variable bindings in the program are not available at the time the
transformer is executed, it is an error for a transformer to reference
or assign them.  However, since keyword bindings are available during
expansion, lexically visible keyword bindings from the program may be
used in macro uses in a transformer.

When a macro use is encountered, the macro transformer associated with
the macro keyword is applied to a representation of the macro
expression.  The result returned by the macro transformer replaces the
original expression and is expanded once again.  Thus macro expansions
may themselves be or contain macro uses.

The syntactic representation passed to a macro transformer
encapsulates information about the structure of the represented form
and the bindings of the identifiers it contains.  These syntax objects
can be traversed and examined using the procedures described below.
The output of a transformer may be built up using the usual Scheme
list constructors, combining pieces of the input with new syntactic
structures.

\begin{entry}{%
\proto{syntax}{ \hyper{datum}}{\exprtype}}

\syntax
The \hyper{datum} may be any external representation of a Scheme
object.

\semantics
{\cf Syntax} is the syntactic analogue of {\cf quote}.  It creates a
syntactic representation of \hyper{datum} that, like an argument to a
transformer, contains information about the bindings for identifiers
contained in \hyper{datum}.  The binding for an identifier introduced
by {\cf syntax} is the closest lexically visible binding.  All
variables and keywords introduced by transformers must be created by
{\cf syntax}.  It is an error to insert a symbol in the output of a
transformation procedure unless it is to be part of a quoted datum.

\begin{scheme}
(symbol? (syntax x))                               \ev \schfalse%

(let-syntax ((car (lambda (x) (syntax car))))
  ((car) '(0)))                                    \ev 0%

(let-syntax
  ((quote-quote
    (lambda (x) (list (syntax quote) 'quote))))
  (quote-quote))                                   \ev quote%

(let-syntax
  ((quote-quote
    (lambda (x) (list 'quote 'quote))))
  (quote-quote))                                   \ev \scherror%
\end{scheme}

The second {\cf quote-quote} example results in an error because two raw
symbols are being inserted in the output.  The quoted {\cf quote} in the
first {\cf quote-quote} example does not cause an error because it will
be a quoted datum.

\begin{scheme}
(let-syntax ((quote-me
              (lambda (x)
                (list (syntax quote) x))))
  (quote-me please))                              \ev (quote-me please)

(let ((x 0))
  (let-syntax ((alpha (lambda (e) (syntax x))))
    (alpha)))                                     \ev 0

(let ((x 0))
  (let-syntax ((alpha (lambda (x) (syntax x))))
    (alpha)))                                     \ev \scherror

(let-syntax ((alpha
              (let-syntax ((beta
                            (syntax-rules ()
                              ((beta) 0))))
                (lambda (x) (syntax (beta))))))
  (alpha))                                        \ev \scherror%
\end{scheme}


The last two examples are errors because in both cases a lexically
bound identifier is placed outside of the scope of its binding.
In the first case, the variable {\cf x} is placed outside its scope.
In the second case, the keyword {\cf beta} is placed outside its
scope.

\begin{scheme}
(let-syntax ((alpha (syntax-rules ()
                      ((alpha) 0))))
  (let-syntax ((beta (lambda (x) (alpha))))
    (beta)))                                      \ev 0

(let ((list 0))
  (let-syntax ((alpha (lambda (x) (list 0))))
    (alpha)))                                     \ev \scherror%
\end{scheme}

The last example is an error because the reference to {\cf list} in the
transformer is shadowed by the lexical binding for {\cf list}.  Since the
expansion process is distinct from the execution of the program,
transformers cannot reference program variables.  On the other hand,
the previous example is not an error because definitions for keywords
in the program do exist at expansion time.

\begin{note}
It has been suggested that {\cf \#'\hyper{datum}} and
{\cf \#`\hyper{datum}} would be
felicitous abbreviations for {\cf (syntax \hyper{datum})}
and {\cf (quasisyntax \hyper{datum})}, respectively,
where {\cf quasisyntax}, which is not described in this
appendix, would bear the same relationship to {\cf syntax}
that {\cf quasiquote} bears to {\cf quote}.
\end{note}

\end{entry}

\begin{entry}{%
\proto{identifier?}{ syntax-object}{procedure}}

Returns \schtrue{} if \var{syntax-object} represents an identifier,
otherwise returns \schfalse{}.

\begin{scheme}
(identifier? (syntax x))       \ev \schtrue
(identifier? (quote x))        \ev \schfalse
(identifier? 3)                \ev \schfalse%
\end{scheme}

\end{entry}


\begin{entry}{%
\proto{unwrap-syntax}{ syntax-object}{procedure}}

If \var{syntax-object} is an identifier, then it is returned unchanged.
Otherwise {\cf unwrap-syntax} converts the outermost structure of
\var{syntax-object} into a
data object whose external representation is the same as that of
\var{syntax-object}.  The result is either an identifier, a pair whose
car
and cdr are syntax objects, a vector whose elements are syntax
objects, an empty list, a string, a boolean, a character, or a number.

\begin{scheme}
(identifier? (unwrap-syntax (syntax x)))
              \ev \schtrue
(identifier? (car (unwrap-syntax (syntax (x)))))
              \ev \schtrue
(unwrap-syntax (cdr (unwrap-syntax (syntax (x)))))
              \ev ()%
\end{scheme}

\end{entry}


\begin{entry}{%
\proto{free-identifier=?}{ \vari{id} \varii{id}}{procedure}}

Returns \schtrue{} if the original occurrences of \vari{id}
and \varii{id} have
the same binding, otherwise returns \schfalse.
{\cf free-identifier=?}
is used to look for a literal identifier in the argument to a
transformer, such as {\cf else} in a {\cf cond} clause.
A macro
definition for {\cf syntax-rules} would use {\cf free-identifier=?}
to look for literals in the input.

\begin{scheme}
(free-identifier=? (syntax x) (syntax x))
          \ev \schtrue
(free-identifier=? (syntax x) (syntax y))
          \ev \schfalse

(let ((x (syntax x)))
  (free-identifier=? x (syntax x)))
          \ev \schfalse

(let-syntax
  ((alpha
    (lambda (x)
      (free-identifier=? (car (unwrap-syntax x))
                         (syntax alpha)))))
  (alpha))                                        \ev \schfalse

(letrec-syntax
  ((alpha
    (lambda (x)
      (free-identifier=? (car (unwrap-syntax x))
                         (syntax alpha)))))
  (alpha))                                        \ev \schtrue%
\end{scheme}

\end{entry}


\begin{entry}{%
\proto{bound-identifier=?}{ \vari{id} \varii{id}}{procedure}}

Returns \schtrue{} if a binding for one of the two identifiers
\vari{id} and \varii{id} would shadow free references to the other,
otherwise returns \schfalse{}.
Two identifiers can be {\cf free-identifier=?} without being
{\cf bound-identifier=?}  if they were introduced at different
stages in the
expansion process.
{\cf Bound-identifier=?} can be used, for example, to
detect duplicate identifiers in bound-variable lists.  A macro
definition of {\cf syntax-rules} would use {\cf bound-identifier=?}
to look for
pattern variables from the input pattern in the output template.

\begin{scheme}
(bound-identifier=? (syntax x) (syntax x))
        \ev \schtrue

(letrec-syntax
  ((alpha
    (lambda (x)
      (bound-identifier=? (car (unwrap-syntax x))
                          (syntax alpha)))))
  (alpha))                                         \ev \schfalse%
\end{scheme}

\end{entry}


\begin{entry}{%
\proto{identifier->symbol}{ \var{id}}{procedure}}

Returns a symbol representing the original name of \var{id}.
{\cf Identifier->symbol} is used to examine identifiers that appear in
literal contexts, i.e., identifiers that will appear in quoted
structures.

\begin{scheme}
(symbol? (identifier->symbol (syntax x)))
   \ev \schtrue
(identifier->symbol (syntax x))          
   \ev x%
\end{scheme}

\end{entry}


\begin{entry}{%
\proto{generate-identifier}{}{procedure}
\proto{generate-identifier}{ \var{symbol}}{procedure}}

Returns a new identifier.
The optional argument to {\cf generate-identifier} specifies the
symbolic name of the resulting identifier.  If no argument is
supplied the name is unspecified.

{\cf Generate-identifier} is used to
introduce bound identifiers into the output of a transformer.  Since
introduced bound identifiers are automatically renamed, {\cf
generate-identifier} is necessary only for distinguishing introduced
identifiers when an indefinite number of them must be generated by a
macro.

The optional argument to {\cf generate-identifier} specifies the
symbolic name of the resulting identifier.  If no argument is
supplied the name is unspecified.  The procedure {\cf
identifier->symbol} reveals the symbolic name of an identifier.


\begin{scheme}
(identifier->symbol (generate-identifier 'x))
  \ev x

(bound-identifier=? (generate-identifier 'x)
                    (generate-identifier 'x))
  \ev \schfalse

(define-syntax set*!
  ; (set*! (<identifier> <expression>) ...)
  (lambda (x)
    (letrec
      ((unwrap-exp
        (lambda (x)
          (let ((x (unwrap-syntax x)))
            (if (pair? x)
                (cons (car x)
                      (unwrap-exp (cdr x)))
                x)))))
      (let ((sets (map unwrap-exp
                       (cdr (unwrap-exp x)))))
        (let ((ids (map car sets))
              (vals (map cadr sets))
              (temps (map (lambda (x)
                            (generate-identifier))
                          sets)))
          `(,(syntax let) ,(map list temps vals)
            ,@(map (lambda (id temp)
                     `(,(syntax set!) ,id ,temp))
                   ids
                   temps)
            \schfalse))))))
\end{scheme}

\end{entry}



\begin{entry}{%
\proto{construct-identifier}{ \var{id} \var{symbol}}{procedure}}

Creates and returns an identifier named by \var{symbol} that behaves
as if it had been introduced where the identifier \var{id} was
introduced.

{\cf Construct-identifier} is used to circumvent hygiene by
creating an identifier that behaves as though it had been
implicitly present in some expression.  For example, the
transformer for a structure
definition macro might construct the name of a field accessor
that does not explicitly appear in a use of the macro,
but can be
constructed from the names of the structure and the field.
If a binding for the field accessor were introduced
by a hygienic transformer, then it would be renamed automatically,
so that the introduced binding would fail to capture any
references to the field accessor that were present in the
input and were intended to be
within the scope of the introduced binding.

Another example is a macro that implicitly binds {\cf exit}:

\begin{scheme}
(define-syntax loop-until-exit
  (lambda (x)
    (let ((exit (construct-identifier
                 (car (unwrap-syntax x))
                 'exit))
          (body (car (unwrap-syntax
                      (cdr (unwrap-syntax x))))))
      `(,(syntax call-with-current-continuation)
        (,(syntax lambda)
         (,exit)
         (,(syntax letrec)
          ((,(syntax loop)
            (,(syntax lambda) ()
               ,body
               (,(syntax loop)))))
          (,(syntax loop))))))))

(let ((x 0) (y 1000))
  (loop-until-exit
   (if (positive? y)
       (begin (set! x (+ x 3))
              (set! y (- y 1)))
       (exit x)))) \evalsto 3000
\end{scheme}

\end{entry}


\subsection*{Acknowledgements}

The extension described in this appendix is the most
sophisticated macro facility that has ever been proposed
for a block-structured programming language.  The main ideas
come from
Eugene Kohlbecker's PhD thesis on hygienic macro expansion
\cite{Kohlbecker86}, written under the direction of Dan
Friedman \cite{hygienic}, and from the work by Alan Bawden
and Jonathan Rees on syntactic closures \cite{Bawden88}.
Pattern-directed macro facilities were popularized by Kent
Dybvig's non-hygienic implementation of {\cf extend-syntax}
\cite{Dybvig87}.

At the 1988 meeting of this report's authors at Snowbird,
a macro committee consisting of Bawden, Rees, Dybvig,
and Bob Hieb was charged with developing a hygienic macro
facility akin to {\cf extend-syntax} but based on syntactic closures.
Chris Hanson implemented a prototype and wrote a paper on his
experience, pointing out that an implementation based on
syntactic closures must determine the syntactic roles of some
identifiers before macro expansion based on textual pattern
matching can make those roles apparent.  William Clinger
observed that Kohlbecker's algorithm amounts to a technique
for delaying this determination, and proposed a more efficient
version of Kohlbecker's algorithm.  Pavel Curtis spoke up for
referentially transparent local macros.  Rees merged syntactic
environments with the modified Kohlbecker's algorithm and
implemented it all, twice \cite{macrosthatwork}.

Dybvig and Hieb designed and implemented the low-level
macro facility described above.
Recently Hanson and Bawden have extended syntactic closures
to obtain an alternative low-level macro facility.
The macro committee has not endorsed any particular
low-level facility, but does endorse the general concept of
a low-level facility that is compatible with the
high-level pattern language described in this appendix.

Several other people have contributed by working on macros
over the years.  Hal Abelson contributed by holding this
report hostage to the appendix on macros.


\usepackage{graphicx}


\begin{document}

\title{The Design and Implementation of Data-Centric Sychronization for Structured Parallel Program}
\author{Sai Zhang (szhang@cs.washington.edu)}

\maketitle

\begin{abstract}
This document describes the design and implementation details of supporting data-centric sychronization for the X10 programming language. It includes: (1) a detailed description of the current design and implementation, and (2) tricks, tips and commons pitfalls in modifying the X10 compiler infrastracture for future extensions.
\end{abstract}

\section{Current Status}

As of Sep 2011, the implementation fully supports the initial design as described in Section~\ref{sec:design}. However, the current implementation 
 \textit{only} supports Java backend. It has been tested on 6 small examples from the X10 release package plus 1 medium-size subject called \textit{fmm} from the anuchem application (\CodeIn{http://cs.anu.edu.au/~Josh.Mil-\\thorpe/anuchem.html}).

The source code and tested programs can be found at the following svn repository: \CodeIn{http://x10.svn.sourceforge.net/viewvc/x10/branches/atomic-sets/}.  

%I will do my best to assist you to comprehend the code, fix existing bugs, and perform evaluations.

The following sections will illustrate the implemented data-centric sychronization features (Section~\ref{sec:design}),  all major implementation details (Section~\ref{sec:details}), and a few summarized tricks, tips and common pitfalls in modifying the compiler for future extensions (Section~\ref{sec:tips}).

\section{Data-centric Sychronization for X10}
\label{sec:design}

This sections gives a high-level overview of what has been implemented. Most of the content is summarized from emails between Mandana Vaziri and Sai Zhang.

\subsection{Programming model}

The programming model consists of declaring a collection of data as being part of a data group, and indicating the transactional boundaries for that data group (or unit of work, atomic section). A unit of work is such that the elements of the data group to which it belongs are manipulated atomically. The units of work provide mutual exclusion for accesses to elements of the data group.

Each instance of a class has a single implicit data group labeled this, and all fields of a class belong to this data group. A data group may contain only data that is located in the same X10 place.

Units of work are indicated using the construct \CodeIn{atomic($var_1$, ..., $var_n$) \{ ... \}}, which means that the object referenced by $var_i$ is manipulated atomically, if $var_i$ is an object reference, and that  $var_i$ is accessed atomically if  $var_i$ is of primitive type. More formally, if  $var_i$ refers to an object, \CodeIn{atomic($var_i$)\{ ... \}} declares a unit of work for the data group of the object referenced by  $var_i$. In this case,  $var_i$ may be a field, a formal parameter, or a local variable.
If  $var_i$ is a primitive type,  \CodeIn{atomic($var_i$)\{ ... \}} declares a unit of work for an implicitly locally defined data group that only contains $var_i$. In this case, $var_i$ can only be a local variable or a primitive formal parameter.

So far atomic sections look very much like Java's synchronized block. What differentiate them is that multiple objects may share the same data group. This is indicated using the \CodeIn{linked} keyword on a field, formal parameter, or local variable. This keyword means that the referenced object has the same data group as the data group of 'this'. It is not allowed to label variables of primitive type with the keyword \CodeIn{linked}.

\subsection{High-level implementation}


We add to each class, an additional field that holds the lock for the data groups of instances of that class.
 Each class is equipped with a getter method for the lock.

Constructors are modified to take an additional lock. When the \CodeIn{linked} keyword is used for a constructor call, the lock for \CodeIn{this} is passed to the newly created object.

The construct \CodeIn{atomic($var_1$, ..., $var_n$) \{ ... \}} grabs the lock for every $var_i$ by calling the getter method for its lock, if $var_i$ is a reference type.
If $var_i$ is a primitive value, a local lock is declared after the declaration of the local variable $var_i$ and this lock is grabbed.

\section{Implementation Details}
\label{sec:details}

This sections describes the most important implementation details. Section~\ref{sec:code} gives an overview of the X10 compiler workflow and which parts  have been modified to implement data-centric synchronization. Section~\ref{sec:keyword} shows how is the new syntax added to the existing compiler framework (lexer and parser parts). Section~\ref{sec:typechecking} illustrates how to perform type checking in the presence of the  data-centric features, and Section~\ref{sec:codegen} presents how the X10 compiler performs code generation. Changes to the runtime library are described in Section~\ref{sec:runtime}, and limitations and possible improvement space is summarized in Section~\ref{sec:limitation}.

\subsection{Code structure and Compiler workflow}
\label{sec:code}

All code changes reside in two sub-projects \CodeIn{x10.compiler} and
\CodeIn{x10.runtime}. The project \CodeIn{x10.compiler} also contains a modified version of polyglot, which is under the package \CodeIn{polyglot}.

In \CodeIn{x10.compiler},  code for adding new syntax  is in the \CodeIn{x10.parser} package. Code for type-checking is in the \CodeIn{x10.ast} and \CodeIn{polygloat.ast} packages. More specifically, class \CodeIn{x10.visit.X10AtomicityChecker} is the entry class for type-checking , and class \CodeIn{x10.visit.X10LockMapAtomicityTranslator} performs  code generation. All changes to the runtime library are in the package \CodeIn{x10.util.concurrent}. 

\textbf{Configuration option: } the only new configuration option is \CodeIn{DATA\_CENTRIC} (with default value \CodeIn{true}) in class \CodeIn{x10.Configuration}. Setting this value to \CodeIn{false} will turn off the data-centric sychronization features. \textbf{Note:} when compiling the x10 runtime lib using command \CodeIn{ant dist}, this flag \textbf{must} be turned off; since the current data-centric implementation lacks the lack of C++ backend support.

Here is the general work flow of compiling an X10 program:

\begin{enumerate}

\Item Parse the X10 program text into AST tokens. The parser is automatically generated based on the grammar files located in package  \CodeIn{x10.parser}. When a rule defined in the \CodeIn{x10.g} grammar file is matched, the parser will invoke corresponding action method in class \CodeIn{x10.parser.X10SemanticRules} to create AST nodes.  At this stage, almost AST nodes do not contain any type information. The parser merely uses a \CodeIn{polyglot.parse.ParsedName} object to represent each type node.

\Item Translate a \CodeIn{ParsedName} object into a \CodeIn{TypeNode}. This step is done by the \CodeIn{ParsedName.toType} method. This method creates a \CodeIn{AmbTypeNode} node for each \CodeIn{ParsedName} object to represent a type node. However, the created \CodeIn{AmbTypeNode} still needs to be dis-ambiguated before type-checking.

\Item Dis-ambiguate each \CodeIn{AmbTypeNode} node. This step is performed by the class \CodeIn{x10.visit.X10TypeChecker}. Method \CodeIn{X10TypeChecker.leaveCall} first calls \CodeIn{disambiguate(tc)} before performing type-checking. The \CodeIn{disambiguate} method is overriden in every ambiguous type node to resolve ambiguity and infer types. Thus, if a new AST node type is added,  be aware of overriding the \CodeIn{disambiguate} method.

\Item After disambuigiating each AST node, the following compiler workflow is essentially applying a set of passes to the AST tree. Each pass is implemented as a visitor. All visitor classes are under packages  \CodeIn{x10.visit} and \CodeIn{polyglot.visit}.  A visitor can manipulate each AST node,  delete, or add needed information to it. The visiting order of each visitor is defined in class \CodeIn{x10.ExtensionInfo} (and \CodeIn{x10c.ExtensionInfo} and \CodeIn{x10cpp.ExtensionInfo} for Java and C++ specific passes). A good example to refer is the  \CodeIn{goals(Job)} method.

\Item After type-checking, many optimization and code generation tasks in X10 is implemented as a X10-to-X10 source-code-level transformation. A good example is the \CodeIn{x10.visit.Lowerer} class. When implementing new features, normally you only need to define a similar visitor (as the \CodeIn{Lowerer} class) to perform X10- to-X10 source transformation instead of modifying the backend translation from X10 to Java (C++).

\Item The last step is generating native code (Java bytecode and C++ binary code) from the X10 AST. Normally, you do not need to touch this phase.


\end{enumerate}

\subsection{Adding new syntax}
\label{sec:keyword}

The syntax changes to the X10 language are:

\begin{itemize}

\Item Add a new keyword \CodeIn{linked} as a type modifier.

\Item Add a new rule for type identifier:  \textit{Type} = \CodeIn{linked} \textit{Type}. This permits programmers to link the atomic set of a variable to the current \CodeIn{this} atomic set by declaring: \CodeIn{var a:linked A = new linked A()}.

\Item Add a new rule for atomic section: \CodeIn{atomic}(\textit{identifier\_list})  \CodeIn{statement}. The \textit{identifier\_list} is  for programmers to specify which atomic sets need to be protected. For example, \CodeIn{atomic (var1,  this, formal1) \{ ... \}} indicates that atomic sets to which \CodeIn{var1}, \CodeIn{this}, and \CodeIn{formal1} belongs are updated atomically in the atomic section.

\end{itemize}

Here are the detailed steps in implementing the above syntax changes:

\textbf{Adding the \CodeIn{linked} keyword}

\begin{enumerate}

\Item Go to \CodeIn{X10KWLexer.gi} file, and modify two places. First, add \CodeIn{linked} as keyword by adding a new entry under the \textit{\%Export} declaration. Then, add a new entry for the \CodeIn{linked} modifier by adding a new entry to the \textit{\%Rules} declaration.

\Item Go to \CodeIn{polyglot.types.Flags} class. Add a new static field declaration like \CodeIn{public static final Flags LINKED = createFlag("linked", null)}. Then, add three corresponding methods: \CodeIn{Flags linked()}, \CodeIn{Flags clearLinked()}, and \CodeIn{boolean isLinked}.  

\Item Go to class \CodeIn{x10.parser.X10SemanticRules.FlagModifier}. Add a new field declaration: \CodeIn{public static int LINKED = 19}, and change the field \CodeIn{NUM\_FLAGS} correspondingly.  Change the \CodeIn{FlagModifier.flags()} method by adding an extra if condition like \CodeIn{if(flag == LINKED) } \CodeIn{return Flags.LINKED}.  Add a new entry in method \CodeIn{FlagModifier.name} like \CodeIn{if(flag = LINKED)} \CodeIn{return "linked"}. Finally, add a new rule for the modifier \CodeIn{linked} in the \CodeIn{X10SemanticRules} class:

\CodeIn{void rule\_Modifier13() \{\\  setResult( new FlagModifier(pos(), FlagModifier.LINKED));\\ \}}


\Item Depending on how the new keyword should be used, you may also need to modify a few declarations in classes \CodeIn{FlagModifier} 
and \CodeIn{TypeSystem\_c}. For the \CodeIn{linked} case, the new keyword can only be used to decorate a type, thus  a new entry in the \CodeIn{typeModifiers} declaration is added.

\end{enumerate}

\textbf{Adding new grammar rules}

\begin{enumerate}

\Item Go to \CodeIn{x10.parser.x10.g} file. Add corresponding productions as well as their action methods in class \CodeIn{X10SemanticRules}.

\Item For the \CodeIn{linked} keyword, first add a production rule under the \textit{Modifier :: =} declaration, then add a production rule under the \textit{TypeName ::= } declaration. For each added production rule, corresponding action method must be added in class \CodeIn{X10SemanticRules}.

\Item For the new \CodeIn{atomic} section syntax, one additional rule needs to be added to the \CodeIn{AtomicStatement} declaration. Similarly, corresponding action method should be added in class \CodeIn{X10SemanticRules}.

\end{enumerate}

\subsubsection{Building the new parser}

You need to first download the \CodeIn{lpg.generator}. The easiest way is to download it from its CVS repository (\textit{lpg.cvs.sf.net} with \textit{anonymous} user, and repository path: \textit{/cvsroot/lpg}).
Be aware of choosing the a correct version for your environment. In my environment, I chose two projects \CodeIn{lpg.generator} and \CodeIn{lpg.generator.linux\_x86\_64}. The first project must be used by any version, and the second project is platform-specific.

Run the \textit{grammar} task in the \CodeIn{x10.compiler/build.xml} configuration file, and remember to refresh the whole project.

\subsubsection{Define new AST nodes and propagate type information}

%If the changes to compiler result to new kind of AST node,  you must define the corresponding AST node class, which is normally placed in pakcage \CodeIn{x10.ast}.

Using data-centric synchronization features, two variable can be declared as:

\CodeIn{var a1:A = new A();}

\CodeIn{var a2: linked A = new linked A();}

The above variable \CodeIn{a1} and \CodeIn{a2}  technically have different types. \CodeIn{a1} is a raw \CodeIn{A} object, and \CodeIn{a2} is a \CodeIn{linked A} object. Thus,  the compiler must
keep this \textit{linked} information through the whole compiling process,  propagating from the initial parsing phase to the  type-checking phase to the code generation phase.

To achieve the above goal, the following changes are made  (\textbf{Note}: the following changes can work,  but may not be the optimal way for implementation):

\begin{enumerate}

\Item Add a \CodeIn{FlagsNode flags} field to the \CodeIn{TypeNode\_c} class. This field indicates whether a type node is linked to other's atomic set. The value of \CodeIn{flags} is \CodeIn{null} by default, and is set to \CodeIn{linked} if the current object is \textit{linked} to somewhere else. \textbf{Note:} the \CodeIn{copy} method must be overriden or modified, to make sure this new \CodeIn{flags} field will also be copied.

\Item Add a \CodeIn{FlagsNode flags} field to the \CodeIn{ParsedName} class to represent whether the type object is \CodeIn{linked} or not. The \CodeIn{flags} field can only be \CodeIn{null} (the default value) or \CodeIn{linked}.

\Item Change the \CodeIn{ParsedName.toType} method. It checks, if the \CodeIn{flags} field is \CodeIn{linked}, the compiler needs to create a different AST node of type \CodeIn{X10AmbTypeNodeLinked\_c} for it.

\Item The newly added \CodeIn{X10AmbTypeNodeLinked\_c} type represent a \textit{linked} ambiguious AST node. Its \CodeIn{flags} field is set to \CodeIn{linked} inside method \CodeIn{ParsedName.toType}.

\Item When dis-ambiguating a type node, the \CodeIn{linked} flags must be preserved and propagated correctly from the \CodeIn{X10AmbTypeNodeLinked\_c} node.  The code for preserving the \CodeIn{linked} flag is in \CodeIn{X10AmbTypeNodeLinked\_c}.\CodeIn{disambiguate},  and \CodeIn{X10Disamb\_c}.\CodeIn{disambiguateNoPrefix} methods. The \textbf{most} important notice here is: when setting a type node as \CodeIn{linked},  that type node must be copied and then re-set the field value on the copied node (see the code in \CodeIn{X10Disamb\_c}.\CodeIn{disambiguateNoPrefix} as an example). This is because \textit{all} variables with the same type are sharing the \textit{same} type node object; thus, a linked  node must have a different object (with the same type value but an additional \CodeIn{linked} field).

\Item Finally, the \CodeIn{AbstractNodeFactory\_c}, \CodeIn{NodeFactory}, and \CodeIn{X10NodeFactory\_c} should also be modified by adding additional factory methods to create the new \CodeIn{X10AmbTypeNodeLinked\_c} nodes.

\end{enumerate}



\subsection{Performing type checking}
\label{sec:typechecking}

The visitor \CodeIn{X10AtomicityChecker} performs type-checking for the new grammar rules. As indicated in its \CodeIn{leaveCall(Node, Node, NodeVisitor)} method, the type-checking is essentially invoking the \CodeIn{checkAtomicity} and \CodeIn{checkLinkProperty} methods on each AST node as the visitor traverses the whole tree.

Two methods  \CodeIn{checkAtomicity} and \CodeIn{checkLinkProperty} are added to a few related places, namely, classes \CodeIn{Node\_c}, \CodeIn{JL\_c}, and \CodeIn{NodeOps}. The default behavior of these two methods are doing nothing. So, if needed, an AST node can override these two methods to check certain properties.

In general,  \CodeIn{checkAtomicity} fetches the \CodeIn{linked} flags from the \CodeIn{TypeNode} (that is associated with some AST nodes that are translated from \CodeIn{X10AmbTypeNodeLinked\_c}), and add \textit{atomic context} to its type. The \textit{atomic context} here represents the atomic set to which the declared var is linked to.  To keep the \textit{atomic context} information,  I added a  field \CodeIn{Type} \CodeIn{atomicContext} to class \CodeIn{X10ParsedClassType\_c} to record the linked object type. This field is set inside the \CodeIn{checkAtomicity} method (for a few special cases, it is set inside the \CodeIn{typeCheck} method). After fetching the \CodeIn{atomicContext}, the visitor checks the linked property against the typing rules. (\textbf{Note} that, for most cases, \CodeIn{checkAtomicity} and \CodeIn{checkLinkProperty} can be merged into one method).

During type-checking, we not only need to check the type compatibility as the normal X10 type-checking does, but also need to check the consistency of the \CodeIn{atomicContext} field to see whether a variable is always linked to the same atomic set.

I next use a few examples to show how the type checking is performed:

\begin{enumerate}

\Item checking assignment

\CodeIn { class C \{}

\hspace{3mm}\CodeIn{public def foo() \{}

\hspace{6mm}\CodeIn{var a1:A = new A();}

\hspace{6mm}\CodeIn{var a2: linked A = new linked A();}

\hspace{6mm}\CodeIn{a2 = a1; } //type check this assignment

\hspace{3mm} \CodeIn{\}}

\CodeIn{\}}



The type of \CodeIn{a1} is \CodeIn{A}, with \CodeIn{atomicContext = null}

The type of \CodeIn{a2} is \CodeIn{A}, with \CodeIn{atomicContext = C}, indicating variable \CodeIn{a2}'s atomic set is linked to \CodeIn{C}.\CodeIn{this}.atomic set.

Thus, when checking the assignment \CodeIn{a2 = a1}, the type checker will issue an error, saying that \CodeIn{a2} is linked to somewhere else, and can not be assigned to a raw object \CodeIn{a1} which is not linked to any other atomic set.

\Item checking field access.

Consider the following example (just for illustration purpose. we may make field as strongly private later):

\CodeIn { class C \{}

\hspace{3mm}\CodeIn{var f: linked C =  new linked C();}

\CodeIn{\}}

\CodeIn { class B \{}

\hspace{3mm}\CodeIn{public def foo() \{}

\hspace{6mm}\CodeIn{var c1: linked C = new linked C();}

\hspace{6mm}\CodeIn{c1.f = new C(); } //type check this assignment

\hspace{3mm} \CodeIn{\}}

\CodeIn{\}}

The type of \CodeIn{f} field is: \CodeIn{C} with \CodeIn{atomicContext = C}

The type of \CodeIn{c1} is: \CodeIn{C} with \CodeIn{atomicContext = B}

The type of \CodeIn{new linked C()} inside method \CodeIn{foo()}
is: \CodeIn{C} with \CodeIn{atomicContext = B}

The \textbf{tricky} part is that expression \CodeIn{c1.f} has type: \CodeIn{C} but with \CodeIn{atomicContext = B}, since as indicated by the typing rule the accessed field's \CodeIn{atomicContext} equals the receiver's \CodeIn{atomicContext} if both are linked.

Thus, this assignment  type checks.


\end{enumerate}



\subsection{Code generation}
\label{sec:codegen}

The code generation is a source-code-level X10-to-X10 code translation process. All relevant code is in class \CodeIn{X10LockMapAtomicityTranslator}.

The major code transformation consists of the following phases:

\begin{enumerate}

\Item Class-level transformation:
\begin{itemize}
\Item Let each compiled class (interface) implement (inherit) \CodeIn{x10.util.concurrent.Atomic}.
\Item Associate each class with a lock by inserting  a unique lock id field. The lock id can be used to find the corresponding lock in a global lock map.  Then, add corresponding getter method for the lock field.
\Item For each constructor, create a new constructor by adding an additional lock field formal parameter, then add the new constructor to the class declaration.

Here is one transformation example:
\begin{CodeOut}
\begin{alltt}
public class A \{
   this() \{...\}
   this(v:Int) \{\}
\}
\end{alltt}
\end{CodeOut}

\hspace{10mm}$\Downarrow$

\begin{CodeOut}
\begin{alltt}
public class A \underline{implements Atomic} \{
   \underline{var lockid:Int = -1;}
   \underline{public def OrderedLock getOrderedLock() \{ return OrderedLock.getLock(lockid);\}}
   \underline{static var static\_lockid:Int = OrderedLock.createNewLockID();}
   \underline{public static def OrderedLock getStaticOrderedLock() \{ return static\_lockid;\}}
   this() \{...\}
   this(\underline{lock:OrderedLock}) \{... \underline{this.lockid = lock.getIndex();}\}
   this(v:Int) \{...\}
   this(v:Int, \underline{lock:OrderedLock}) \{... \underline{this.lockid = lock.getIndex();}\}
\}
\end{alltt}
\end{CodeOut}
\end{itemize}

\Item Method-level transformation

\begin{itemize}
\Item Add additional local locks for parameters which are not associated with a lock (e.g., lib code, and primitive types)
\Item Transform atomic method to acquire locks

Here is an example:
\begin{CodeOut}
\begin{alltt}
public def foo(b:Array[Int]) \{
    finish \{
       async \{atomic(b) \{...update b... \}\}
       async \{atomic(b) \{...update b... \}\}
    \}
\}
\end{alltt}
\end{CodeOut}

\hspace{10mm}$\Downarrow$

\begin{CodeOut}
\begin{alltt}
public def foo(b:Array[Int]) \{
    \underline{var lockid\_for\_b:Int = OrderedLock.createNewLockID();}
    finish \{
       async \{atomic(b) \{...update b... \}\}
       async \{atomic(b) \{...update b... \}\}
    \}
\}
\end{alltt}
\end{CodeOut}
\end{itemize}

Here is an example for atomic method (\textbf{note:} \CodeIn{atomic} method is the syntactic sugar of \CodeIn{atomic(this)}, and an atomic method will also protect the atomic sets of its formal parameters):

\begin{CodeOut}
\begin{alltt}
public \textbf{atomic} def foo(a:A) \{
    ...//do something
\}
\end{alltt}
\end{CodeOut}

\hspace{10mm}$\Downarrow$

\begin{CodeOut}
\begin{alltt}
public \textbf{atomic} def foo(a:A) \{
  try\{
    \underline{OrderedLock.acquireLocks(this.getOrderedLock(), a.getOrderedLock());}
    ...//do something
  \} finally \{
    \underline{OrderedLock.releaseLocks(this.getOrderedLock(), a.getOrderedLock());}
  \}
\}
\end{alltt}
\end{CodeOut}

\Item Block-level transformation

This phase primarily declares locks to protect local variables that are accessed inside an atomic section. Here is an example (in which the local \CodeIn{value} must be protected):

\begin{CodeOut}
\begin{alltt}
public def count() \{
    var value:Int = 0;
    \textbf{finish} for (var i:Int = 0; i < 100; i++) \textbf{async} \{ \textbf{atomic}(value) value ++; \}
\}
\end{alltt}
\end{CodeOut}

\hspace{10mm}$\Downarrow$

\begin{CodeOut}
\begin{alltt}
public def count() \{
    var value:Int = 0;
    \underline{var local\_lockid\_for\_value = OrderedLock.createNewLockID();}
    \textbf{finish} for (var i:Int = 0; i < 100; i++) \textbf{async} \{ \textbf{atomic}(value) value ++; \}
\}
\end{alltt}
\end{CodeOut}


\Item Atomic-section-level transformation

This phase primarily grabs suitable locks for each atomic section.  Here is an example which covers almost all locking cases:

\begin{CodeOut}
\begin{alltt}
public def foo(a:Array[Int], c:C) \{
    var value:Int = 0;
    \textbf{finish} for (var i:Int = 0; i < 100; i++) \textbf{async} \{ \textbf{atomic}(value, a, c, this) \{ ... do something\} \}
\}
\end{alltt}
\end{CodeOut}

\hspace{10mm}$\Downarrow$

\begin{CodeOut}
\begin{alltt}
public def foo(a:Array[Int], c:C) \{
    \underline{var local\_lockid\_for\_a = OrderedLock.createNewLockID();}    
    var value:Int = 0;
    \underline{var local\_lockid\_for\_value = OrderedLock.createNewLockID();}
    \textbf{finish} for (var i:Int = 0; i < 100; i++) \textbf{async} \{
        try \{
             \underline{OrderedLock.acquireLocks( local\_lockid\_for\_value,  local\_lockid\_for\_a, }
                \underline{c.getOrderedLock(), this.getOrderedLock());}
             ...do something
        \} finally \{
              \underline{OrderedLock.releaseLocks(local\_lockid\_for\_value,  local\_lockid\_for\_a, }
                \underline{c.getOrderedLock(), this.getOrderedLock());}
        \}
   \}
\}
\end{alltt}
\end{CodeOut}

\Item Other transformations.

In particular, do \textbf{remember}  you must manually update the captured environment vars of \CodeIn{async}, \CodeIn{at}, \CodeIn{ateach}, and \CodeIn{athome} code block after performing transformation. Please see the \CodeIn{X10LockMapAtomicityTranslator.visitAsync\_c} as an example.

\end{enumerate}

\subsection{Runtime library}
\label{sec:runtime}

Two classes are added to the \CodeIn{x10.runtime} project:

1. \CodeIn{x10.util.concurrent.Atomic}. An interface that every compiled class (interface) will implement (inherit) for data-centric sychronization.

2. \CodeIn{x10.util.concurrent.OrderedLock}. A class wrapping a \CodeIn{lock} field and a unique lock id identifier. This class contains all lock operations used in the compiler, such as \CodeIn{createNewLock}, \CodeIn{acquireLocks}, and \CodeIn{releaseLocks}. It also maintains a global lock map.

\subsubsection{Utility methods}

A few useful utility classes I added:

1. \CodeIn{x10.util.X10TypeUtils} contains a few utility methods for processing type information.

2. Two visitor classes: \CodeIn{x10.visit.AtomicLocalAndFieldAccessVisitor} and
\CodeIn{x10.visit.X10AtomicLockLocalCollector} are used to fetch referred variables inside the atomic sections. Please see the code documentation for more details.

3. A few common error messages are organized in the \CodeIn{Errors} class.

\subsection{Limitations and possible solutions}
\label{sec:limitation}

The section summarizes some  known limitations in the current design and implementation:

\begin{enumerate}

\Item The global lock map  in class \CodeIn{x10.util.concurrent.OrderedLock} may lead to potential memory leak. This lock map maps an \CodeIn{Integer} lock id to an \CodeIn{OrderedLock} object. When the object associated with a lock id has been recycled, this corresponding map entry should be deleted. Furthermore, if the program is running on multiple places, there will be one copy of lock map per place. Thus, the lock map will no longer be a globally  one. This will lead to problems like lock id conflicts, and how to deal with a sychronized object passed from one place to anther ( which lock should be used to protect it? ).

Here are a few possible solutions. First, replace the global lock map with a \CodeIn{WeakHashMap}. This \CodeIn{WeakHashMap} maps each object to its associated lock object, so that when the (Java) object has been recycled, the corresponding entry will be automatically deleted. Second, override the \CodeIn{finalize} method in class \CodeIn{x10.lang.Object} to manually delete the corresponding entry in the lock map. The above two solutions can only be applied to Java backend, and there is still no clear solution for the C++ backend. Third, improve the lock id allocation mechanism to avoid conflict id from different places. A possible way is to combine the \textit{place\_id} with a \textit{place-unique} integer as the lock id to ensure its global uniqueness. Another way is to use a separate service (running in a separate place) to allocate locks upon the request.

\Item Arrays are not well supported in the current implementation. For example, you can not declare an array like: \CodeIn{var linkedArray:} \CodeIn{Array[linked C] = }\CodeIn{new Array[linked C]()}. Implementing this support requires to change a few places.  First, change  class \CodeIn{AmbMacroTypeNode\_c} to capture the \CodeIn{linked} modifier on the parameterized type. Second, change the \CodeIn{disambiguate} method in class \CodeIn{AmbMacroTypeNode\_c}  to propagate the linked information to each type node. Third, implement the \CodeIn{checkAtomicity} and \CodeIn{checkLinkProperty} methods in all array-related AST node classes like \CodeIn{polyglot.ast.ArrayAccess\_c} and \CodeIn{polyglot.ast.ArrayInit\_c} for type-checking.

\Item  The current design treats \CodeIn{linked} as a type modifier. This may unncessarily complicate the implementation (as seen above, multiple code places need to be changed to gurantee the \textit{linked} information is correctly propagated). A more natural solution can be integrating the \CodeIn{linked} keyword seamlessly into the \textit{constraint types} in X10. In that way, a linked var can be declared as: \CodeIn{var c:C\{linked\}} = \CodeIn{new C\{linked\}()}. Doing so can leverage the existing powerful constraint solver in X10 for type checking.

\Item A few code issues (pure engineering improvement):

\begin{itemize}
\Item There are fairly code repetition in the \CodeIn{X10LockMapAtomicityTranslator} class. It is possible (but not easy) to reduce the code clones.

\Item There are some classes and methods annotated with  \CodeIn{@Deprecade} in the code base. Such classes and methods are not used in the current implementation, and thus can be safely removed (in certain cases, you may need to resolve all compilation errors; but that is straightford such as removing all references). The reason I still kept them is  those code can be used as in experiments for comparison purpose. For example, the deprecaded class \CodeIn{X10MixedAtomicityTranslaotr} implements a different way of code generation. It \textit{infers} all accessed variables inside each atomic section.

\Item When visitor \CodeIn{X10LockMapAtomicityTranslator} adds new code (i.e., field declarations, constructors, field access) to the existing class declaration,  it needs to make sure different instances of the same variable should share the same \textit{def} object. However, this is not fully preserved in the current implementation (see Section~\ref{sec:tips}). For example, as the visitor  visits  a \CodeIn{New\_c} statement, it needs to add an additional lock argument value to it. However, at this point, the new constructor with the additional lock formal parameter has not been inserted to the class declaration yet. Therefore, the \CodeIn{visitNew\_c} method need to create another \CodeIn{ConstructorDef} object. This created \CodeIn{ConstructorDef} object is different from the def object used for the new constructor (which is created when the visitor leaves the class declaration). This issue can lead to a few problems. Particularly, in the \CodeIn{ClosureRemover} and  \CodeIn{X10InnerClassRemover} classes, when the constructor def is updated, the def referred by the \CodeIn{New\_c} statement will not be updated correspondingly.I temporarily work around that problem by manually updating each constructor def.

\Item A static option \CodeIn{compile} can be removed in class \CodeIn{Configuration}. This option should be set to \CodeIn{true} when compiling all x10 runtime library.  As I found during my (incomplete) testing, this \CodeIn{compile} can be superseded by the \CodeIn{DATA\_CENTRIC} option (set \CodeIn{DATA\_CENTRIC} to \CodeIn{false} is sufficient for compiling x10 runtime lib). I leave it in the code base in case I missed some corner cases that need to manually set this flag.
\end{itemize}


\end{enumerate}

\section{Tricks, tips and pitfalls}
\label{sec:tips}

I finally summarize a few useful tricks, tips and common pitfalls.

\begin{enumerate}

\Item \textbf{How to test your code.} According to the standard user manual on the  X10 website, you can run \CodeIn{ant dist} to build the whole compiler,  use \CodeIn{x10c} to compile the code, and use \CodeIn{x10} to run the code.  The command \CodeIn{ant dist} will compile all compiler code as well as the runtime lib, and cost over 4 minutes. Normally, you do not need to run this command everytime after making some changes only to the \CodeIn{x10.compiler} project. Instead, running \CodeIn{ant compiler-jar} is much more faster (completed in 5 seconds). If you are using eclipse, a more convenient way to test the compiler is to run the code directly inside eclipse as follows: (1) select an X10 class file on the explorer view, and (2) click the run $\rightarrow$ x10c launch option. The embedded \CodeIn{x10c} will automatically compile the \textit{selected} X10 file. The output Java file is located in the \CodeIn{out} folder. 

\Item \textbf{How to debug your code.} Debugging X10 code is not an easy task. Here are a few tips:

\begin{itemize}

\Item \textbf{Use the generated Java file}. Each Java file contains  line numbers in the original X10 code for the transformed Java code. Those line numbers are very useful to trace back to the original X10 file for fault localization.

\Item \textbf{Pretty-print an AST Node.} There are two useful methods: \CodeIn{Node.prettyPrint(System.out)}, and \CodeIn{PrettyPrinter.printAST}. The first method prints an AST node in a text form (what you see in a code file), while the second one prints an AST node in a tree structure (you can see each Node types from the result).

\end{itemize}


\Item \textbf{Add a compiler configuration option.} It is very easy: just add two static field declarations to file \CodeIn{x10.Configuration} class.  The first one declares the option name, such as \CodeIn{public boolean OPTIMIZE = false}, and the second one is a \CodeIn{String} type that must end with a fixed suffix \CodeIn{\_desc} as an explanation message, like \CodeIn{private static final String} \CodeIn{OPTIMIZE\_desc } \CodeIn{= "Generate optimized code";}.


\Item \textbf{Be aware of the visitor order.} Normally, all visitors override the \CodeIn{leaveCall} method to manipulate the AST. You can treat this method to visit a given AST in a bottom-up manner.  Roughly speaking, for the code snippet below, it will visit code places in the order of \textit{A, B, C,  D, E, and F}.

\begin{CodeOut}
\begin{alltt}
public class C \{                          //\textit{F}
    public def foo() \{                      //\textit{E}
        \textbf{finish}                                //\textit{D}
            for (var i:Int = 0; i < 100; i++)   //\textit{C}
                \textbf{async} \{                          //\textit{B}
                      var c:C = new C();           //\textit{A}
                \}
    \}
\}
\end{alltt}
\end{CodeOut}

\Item \textbf{Examples for reference.} People who are new to x10.compiler often need to find existing code examples for reference when hacking the compiler infrastructure. Here are a few good places:

\begin{itemize}

\Item \CodeIn{x10.visit.Lowerer} contains many examples for X10 $\rightarrow$ X10 code transformation.

\Item \CodeIn{x10.visit.X10InnerClassRemover} contains a few more advanced transformation code.

\Item \CodeIn{x10.ast.X10MethodDecl\_c.typeCheck} gives a quick idea on how type-checking is performed in X10.

\Item \CodeIn{x10.visit.X10TypeChecker} gives a quick idea on how to write a visit to manipulate the AST.

\end{itemize}

\Item \textbf{What I have changed.} In case I missed some important changes I made, please search "data-centric" in the whole eclipse project. Normally, places that I editted  are associated with comments with the above keyword.

\end{enumerate}


\section{Acknowlegement}

This is the joint work with Mandana Vaziri and Olivier Tardieu. Igor Peshansky provided very useful guidance in hacking the x10 compiler infrastructure. David Grove, Yoav Zibin, and Nate Clinger gave insightful comments on many implementation issues.



\end{document}
