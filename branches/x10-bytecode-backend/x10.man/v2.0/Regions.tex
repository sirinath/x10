\section{Regions}\label{XtenRegions}\index{region}

A region is a set of points.  {}\Xten{}
provides a built-in class, \xcd"x10.lang.Region", to allow the
creation of new regions and to perform operations on regions. 

Each region \xcd"R" has a constant integer rank, \xcd"R.rank".
%% TODO: Should be uint.

Here are several examples of region declarations:
\begin{xten}
val MAX_HEIGHT=20;
val Null = Region.makeUnit();  // Empty 0-dimensional region          
val N = 10;
val K = 2;
val R1 = 1..100; // 1-dim region with extent 1..100
val R2 = [1..100] as Region(1); // same as R1
val R3 = (0..99) * (-1..MAX_HEIGHT);   
val R4 = [0..99, -1..MAX_HEIGHT] as Region(2); // same as R3  
val R5 = Region.makeUpperTriangular(N);
val R7 = R4 && R5; // intersection of two regions
val R8 = R4 || R5; // union of two regions
\end{xten}

The expression \xcdmath"a$_1$..a$_2$"
is shorthand for the rectangular, rank-1 region
consisting of the points
$\{$\xcdmath"[a$_1$]", \dots, \xcdmath"[a$_2$]"$\}$.
Each subexpression of \xcdmath"a$_i$" must be of type \xcd"Int".
If \xcdmath"a$_1$"
is greater than \xcdmath"a$_2$", the region is empty.

A region may be constructed by converting from a rail of
regions or a rail of points, typically using a rail constructor
(\Sref{RailConstructors})
(e.g., \xcd"R4" above).
The region constructed from a rail of points represents the
region containing just those points.
The region constructed from a rail of regions
represents
the Cartesian product of each of the arguments.
\XtenCurrVer{} does not (yet) support hierarchical regions.

\index{region!upperTriangular}
\index{region!lowerTriangular}\index{region!banded}

Various built-in regions are provided through  factory
methods on \xcd"Region".  For instance:
\begin{itemize}
\item \xcd"Region.makeUpperTriangular(N)" returns a region corresponding
to the non-zero indices in an upper-triangular \xcd"N x N" matrix.
\item \xcd"Region.makeLowerTriangular(N)" returns a region corresponding
to the non-zero indices in a lower-triangular \xcd"N x N" matrix.
\end{itemize}

All the points in a region are ordered canonically by the
lexicographic total order. Thus the points of a region \xcd"R=(1..2)*(1..2)"
are ordered as 
\begin{xten}
(1,1), (1,2), (2,1), (2,2)
\end{xten}
Sequential iteration statements such as \xcd"for" (\Sref{ForAllLoop})
iterate over the points in a region in the canonical order.

A region is said to be {\em rectangular}\index{region!convex} if it is of
the form \xcdmath"(T$_1$ * $\cdots$ * T$_k$)" for some set of regions
\xcdmath"T$_i$". Such a
region satisfies the property that if two points $p_1$ and $p_3$ are
in the region, then so is every point $p_2$ between them (that is, it is {\em convex}). 
(Note that \xcd"||" may produce non-convex regions from convex regions, e.g.,
\xcd"[1,1] || [3,3]" is a non-convex region.  The operation
\xcd`R.boundingBox()` gives the smallest rectangular region containing
\xcd`R`.)  



%%RECT.CLOSURE  For each region \xcd"R", the {\em rectangular closure} of \xcd"R" is the
%%RECT.CLOSURE  smallest rectangular region enclosing \xcd"R".  For each integer \xcd"i"
%%RECT.CLOSURE  less than \xcd"R.rank", the term \xcd"R(i)" represents the enumeration
%%RECT.CLOSURE  in the \xcd"i"th dimension of the rectangular closure of \xcd"R". It may be
%%RECT.CLOSURE  used in a type expression wherever an enumeration may be used.



\subsection{Operations on regions}
Various non side-effecting operators (i.e., pure functions) are
provided on regions. These allow the programmer to express sparse as
well as dense regions.

Let \xcd"R" be a region. A subset of \xcd"R" is also called a
{\em sub-region}.\index{region!sub-region}

Let \xcdmath"R$_1$" and \xcdmath"R$_2$" be two regions whose type
establishes that they are of the same rank. Let 
\xcdmath"S" be a region of unrelated rank.

\xcdmath"R$_1$ && R$_2$" is the intersection of \xcdmath"R$_1$" and
\xcdmath"R$_2$". 

\index{region!intersection}

\xcdmath"R$_1$ || R$_2$" is the union of the \xcdmath"R$_1$" and
\xcdmath"R$_2$".\index{region!union}

\xcdmath"R$_1$ - R$_2$" is the set difference of \xcdmath"R$_1$" and
\xcdmath"R$_2$".\index{region!set difference}

\xcdmath"R$_1$ * S" is the Cartesian product of \xcdmath"R$_1$" and
\xcdmath"S",  formed by pairing each point in \xcdmath"R$_1$" with every the point in \xcdmath"S".
\index{region!product}
Thus, \xcd"([1..2,3..4] as Region 2) * (5..6)"
is the region of rank \Xcd{3} containing the points \Xcd{(x,y,z)}
where \Xcd{x} is \Xcd{1} or \Xcd{2}, 
\Xcd{y} is \Xcd{3} or \Xcd{4}, and
\Xcd{z} is \Xcd{5} or \Xcd{6}. 


For a region \xcdmath"R" and point \xcdmath"p" of the same rank 
\xcdmath"R+p" and \xcdmath"R-p" represent the translation of the region
with \xcdmath"p". That is, point \xcdmath"q" is in 
\xcdmath"R" if and only if point \xcdmath"q+p" is in \xcdmath"R+p". (And similarly
for \xcdmath"R-p".)

%%TODO: Determine how equality is actually implemented. This should not be the definition of ==. 
%%  This could be the definition of .equals(..).

%% Two regions are equal (\xcd"==") if they represent the same set of
%% points.\index{region!==}

For more details on the available methods on \xcdmath"Region", please
consult the API documentation.
