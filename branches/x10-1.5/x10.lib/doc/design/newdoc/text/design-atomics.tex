\subsection{Atomic Sections}
Atomic sections are implemented using locks. Each place will have a set
of locks.  An activity may request to perform an atomic operation at a
place which in turn will be translated to obtaining the locks at the
designated place before performing the operation. These locks are
globally ordered, precluding deadlocks. Obtained locks cannot be passed
from one activity to another, and all locks required by an activity need
to be obtained together. 

On a cluster of Power5 SMP nodes, we primarily have two types of
atomics: {\it primitive} atomics and {\it compound} atomics. {\it
Primitive} atomics do not block within the same SMP node and use {\tt
lwarx} and {\tt stwcx} instructions to atomically update within a SMP
node. However, across SMP nodes, we need {\it compound} atomics which
explicitly obtains lock at the designated place to perform the operation
and are blocking in nature. It can be observed that primitive atomics
designed specifically to improve performance on a cluster of Power5 SMP
nodes.

Consider the RandomAccess code fragment given in
Section~\ref{sec:deploy:example}. The spawn of asynchronous activities
in Statement 6 can be optimized to leverage the {\it primitive} atomics
concept on Power5 and can get rid of the cost of creating new
asynchronous activities and obtaining locks within the same SMP node.
