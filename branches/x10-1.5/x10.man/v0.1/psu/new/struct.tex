% 1. Structure of the language

\chapter{Overview of Scheme}

\section{Semantics}
\label{semanticsection}

This section gives an overview of Scheme's semantics.  A
detailed informal semantics is the subject of
chapters~\ref{basicchapter} through~\ref{builtinchapter}.  For reference
purposes, appendix~\ref{formalsemanticssection} provides a formal
semantics of Scheme.

\vest Following Algol, Scheme is a statically-scoped\mainindex{static scoping}
programming language.  Each use of a variable is associated with a
lexically-apparent binding of that variable; for this reason static
scoping is also referred to as \defining{lexical scoping}.

\vest Scheme has latent\mainindex{latent types} as opposed to manifest types.
Types are associated with values (also called
objects\mainindex{object}) rather than with variables.  (Some authors
refer to languages with latent types as weakly typed or dynamically
typed languages.)  Other languages with latent types are APL, Snobol,
and other dialects of Lisp.  Languages with manifest types (sometimes
referred to as strongly typed or statically typed languages) include
Algol 60, Pascal, and~C.

\vest All objects created in the course of a Scheme computation, including
procedures and continuations, have unlimited extent.
No Scheme object is ever destroyed.  The reason that
implementations of Scheme do not (usually!)\ run out of storage is that
they are permitted to reclaim the storage occupied by an object if
they can prove that the object cannot possibly matter to any future
computation.  Other languages in which most objects have unlimited
extent include APL and other Lisp dialects.

Implementations of Scheme are required to be properly
tail-recursive\cite{Rabbit}.  This allows the execution of an
iterative computation in constant space, even if the iterative
computation is described by a syntactically recursive procedure.  Thus
with a tail-recursive implementation, iteration can be expressed using
the ordinary procedure-call mechanics, so that special iteration
constructs are useful only as syntactic sugar.

\vest Scheme procedures are objects in their own right.  Procedures can be
created dynamically, stored in data structures, returned as results of
procedures, and so on.  Other languages with these properties include
Common Lisp and ML. \todo{Rozas: Scheme had them first.}

\vest One distinguishing feature of Scheme is that continuations, which
in most other languages only operate behind the scenes, also have
``first-class'' status.  Continuations are useful for implementing a
wide variety of advanced control constructs, including non-local exits,
backtracking, and coroutines (see~\ref{continuations}).

\vest Arguments to Scheme procedures are always passed by value, which
means that the actual argument expressions are evaluated before the
procedure gains control, whether the procedure needs the result of the
evaluation or not.  ML, C, and APL are three other languages that always
pass arguments by value.
This is distinct from the lazy-evaluation semantics of Haskell~\cite{Haskell},
or the call-by-name semantics of Algol 60, where an argument
expression is not evaluated unless its value is needed by the
procedure.

\todo{Lisp's call by value should be explained more
accurately.  What's funny is that all values are references.}

\vest Scheme's model of arithmetic is designed to remain as independent as
possible of the particular ways in which numbers are represented within a
computer.  In Scheme, every integer is a rational number, every rational is a
real, and every real is a complex number.  Thus the distinction between integer
and real arithmetic, so important to many programming languages, does not
appear in Scheme.  In its place is a distinction between exact arithmetic,
which corresponds to the mathematical ideal, and inexact arithmetic on
approximations.  As in Common Lisp, exact arithmetic is not limited to
integers.


\section{Syntax}

Scheme, like most dialects of Lisp, employs a fully parenthesized prefix
notation for programs and (other) data; the grammar of Scheme generates a
sublanguage of the language used for data.  An important
consequence of this simple, uniform representation is the susceptibility of
Scheme programs and data to uniform treatment by other Scheme programs.

The \ide{read} procedure performs syntactic as well as lexical decomposition of
the data it reads.  The \ide{read} procedure parses its input as data
(see~\ref{datumsyntax}), not as program.

The formal syntax of Scheme is described in~\ref{BNF}.


\section{Notation and terminology}


\subsection{Error situations and unspecified behavior}

\mainindex{error}
When speaking of an error situation, this Standard uses the phrase ``an
error is signalled'' to indicate that implementations shall detect and
report the error.  If such wording does not appear in the discussion of
an error, then implementations are not required to detect or report the
error, though they are encouraged to do so.  An error situation that
implementations are not required to detect is usually referred to simply
as ``an error.''

\vest For example, it is an error for a procedure to be passed an argument that
the procedure is not explicitly specified to handle, even though such
domain errors are seldom mentioned in this Standard.  Implementations may
extend a procedure's domain of definition to include such arguments.

\vest This Standard uses the phrase ``may report a violation of an
implementation restriction'' to indicate circumstances under which an
implementation is permitted to report that it is unable to continue
execution of a correct program because of some restriction imposed by the
implementation.  Implementation restrictions are of course discouraged,
but implementations are encouraged to report violations of implementation
restrictions.\mainindex{implementation restriction}

\vest For example, an implementation should report a violation of an
implementation restriction if it does not have enough storage to run a
program.

\vest If the value of an expression is said to be ``unspecified,'' then
the expression shall evaluate to some object without signalling an error,
but the value depends on the implementation; this Standard explicitly does
not say what value should be returned. \mainindex{unspecified}

\todo{Talk about unspecified behavior vs. unspecified values.}

\todo{Look at KMP's situations paper.}


\subsection{Entry format}

Chapters~\ref{expressionchapter} and~\ref{builtinchapter} are organized
into entries.  Each entry describes one language feature or a group of
related features, where a feature is either a syntactic construct or a
built-in procedure.  An entry begins with one or more header lines of the form

\noindent\pproto{\var{template}}{\var{category}}\unpenalty

If \var{category} is ``\exprtype'', the entry describes an expression
type, and the header line gives the syntax of the expression type.
Components of expressions are designated by syntactic variables, which
are written using angle brackets, for example, \hyper{expression},
\hyper{variable}.  Syntactic variables should be understood to denote segments of
program text; for example, \hyper{expression} stands for any string of
characters which is a syntactically valid expression.  The notation
\begin{tabbing}
\qquad \hyperi{thing} $\ldots$
\end{tabbing}
indicates zero or more occurrences of a \hyper{thing}, and
\begin{tabbing}
\qquad \hyperi{thing} \hyperii{thing} $\ldots$
\end{tabbing}
indicates one or more occurrences of a \hyper{thing}.

If \var{category} is ``procedure'', then the entry describes a procedure, and
the header line gives a template for a call to the procedure.  Argument
names in the template are \var{italicized}.  Thus the header line

\noindent\pproto{(vector-ref \var{vector} \var{k})}{procedure}\unpenalty

indicates that the built-in procedure {\tt vector-ref} takes
two arguments, a vector \var{vector} and an exact non-negative integer
\var{k} (see below).

\label{typeconventions}
It is an error for an operation to be presented with an argument that
it is not specified to handle.  For succinctness, we follow the
convention that if an argument name is also the name of a type listed
in~\ref{disjointness}, then that argument must be of the named type.
For example, the header line for {\tt vector-ref} given above dictates
that the first argument to {\tt vector-ref} must be a vector.  The
following naming conventions also imply type restrictions:
\newcommand{\foo}[1]{\vr{#1}, \vri{#1}, $\ldots$ \vrj{#1}, $\ldots$}
$$
\begin{tabular}{ll}
\var{obj}&any object\\
\foo{list}&list (see~\ref{listsection})\\
\foo{z}&complex number\\
\foo{x}&real number\\
\foo{y}&real number\\
\foo{q}&rational number\\
\foo{n}&integer\\
\foo{k}&exact non-negative integer\\
\end{tabular}
$$

\todo{If it is decided that these argument types do not directly
correspond to the type predicates of the same name, we should make a
note of that here.}

\todo{Provide an example entry??}


\subsection{Evaluation examples}

The symbol ``\evalsto'' used in program examples should be read
``evaluates to.''  For example,

\begin{scheme}
(* 5 8)      \ev  40%
\end{scheme}

means that the expression {\tt(* 5 8)} evaluates to the object {\tt 40}.
Or, more precisely:  the expression given by the sequence of characters
``{\tt(* 5 8)}'' evaluates, in the initial environment, to an object
that may be represented externally by the sequence of characters ``{\tt
40}''.  See~\ref{externalreps} for a discussion of external
representations of objects.

\subsection{Naming conventions}

By convention, the names of procedures that always return a boolean
value usually end
in ``\ide{?}''.  Such procedures are called predicates.\schindex{?}

By convention, the names of procedures that store values into previously
allocated locations (see~\ref{storagemodel}) usually end in
``\ide{!}''.
Such procedures are called mutation procedures.
By convention, the value returned by a mutation procedure is unspecified\index{unspecified}.
\schindex{!}

By convention, ``\ide{->}'' appears within the names of procedures that
take an object of one type and return an analogous object of another type.
For example, \ide{list->vector} takes a list and returns a vector whose
elements are the same as those of the list.\schindex{->}


	
\todo{Terms that need defining: thunk, command (what else?).}
