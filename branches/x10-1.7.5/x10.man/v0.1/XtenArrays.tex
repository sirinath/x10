\chapter{Arrays}\label{XtenArrays}\index{arrays}

An array is a mapping from a distribution to a range data
type. Multiple arrays may be declared with the same underlying
distribution. 

Each array has a field {\tt a.distribution} which may be used to
obtain the underlying distribution.\index{arrays!distribution@\tt distribution}

The distribution underlying an array {\tt a} may be obtained 

\section{Regions}\label{XtenRegions}\index{region}

A region is a set of indices (called {\em points}).  
{}\Xten{} provides a built-in value class, {\tt
x10.lang.region}, to allow the creation of new regions and
to perform operations on regions. This class is {\tt final} in
{}\XtenCurrVer; future versions of the language may permit
user-definable regions. Since regions play a dual role
(values as well as types), variables of type {\cf region} 
must be initialized and are implicitly {\tt final}.

Each region {\cf R} has a constant rank, {\cf R.rank}, which is a
non-negative integer. The literal {\cf []} represents the {\em null
region} and has rank {\cf 0}.

For instance:
\begin{x10}
   range  E = 1..100;
   region R = [0..99:2, -1..MAX\_HEIGHT];   
   region R = region.upperTriangular(N);
   region R = region.banded(N, K);
     // A square region.
   region R = [E, E];           
     // Same region as above.
   region R = [100, 100];       
     // A representation for 52*7 days.
   region W = [Week, Weekday];  
     // Represents  the empty region
   region Null = [];            
     // Represents the intersection of two regions
   region AandB = A \&\& B;       
     // represents the union of two regions
   region AOrB = A || B;        
\end{x10}

A region may be constructed using a comma-separated list of {\cf
Ranges} (\S~\ref{XtenRanges}) within square brackets, as above and
represents the Cartesian product of each of the arguments.  For
convenience we allow an integer {\cf n} to stand for the enumeration
{\cf 1..n}.  The bound of a dimension may be any
final variable of a fixed-point numeric type. \XtenCurrVer{} does not
support hierarchical regions.

Various built-in regions are provided through {\tt static} factory
methods on {\tt region}.  For instance:\index{region!upperTriangular}
\index{region!lowerTriangular}\index{region!banded}
\begin{itemize}
{}\item {\cf region.upperTriangular(N)} returns a region corresponding
to the non-zero indices in an upper-triangular {\cf N x N} matrix.
{}\item {\cf region.lowerTriangular(N)} returns a region corresponding
to the non-zero indices in a lower-triangular {\cf N x N} matrix.
{}\item {\cf region.banded(N, K)} returns a region corresponding to
the non-zero indices in a banded {\cf N x N} matrix where the width of
the band is {\cf K}
\end{itemize}

All the points in a region are ordered canonically by the lexicographic total order. Thus the points of a region {\cf R=[1..2,1..2]} are ordered as 
\begin{x10}
  (1,1), (1,2), (2,1), (2,2)
\end{x10}
Sequential iteration statements such as {\cf for} (\S~\ref{ForAllLoop})
iterate over the points in a region in the canonical order.

A region is said to be {\em convex}\index{region!convex} if it is of
the form {\cf [T1,..., Tk]} for some set of enumerations {\cf Ti}. Such a
region satisfies the property that if two points $p_1$ and $p_3$ are
in the region, then so is every point $p_2$ between them. (Note that
{\cf ||} may produce non-convex regions from convex regions, e.g.{}
{\cf [1,1] || [3,3]} is a non-convex region.)

For each region {\cf R}, the {\em convex closure} of {\cf R} is the
smallest convex region enclosing {\cf R}.  For each integer {\cf i}
less than {\cf R.rank}, the term {\cf R.i} represents the enumeration
in the {\cf i}th dimension of the convex closure of {\cf R}. It may be
used in a type expression wherever an enumeration may be used.

Region variables can be declared and used within user programs. They
are implicitly {\tt final} since they can be used within type
expressions (and hence must not take on different values at
runtime). That is, \Xten{} does not permit the declaration of mutable
{\tt region} variables.

\subsection{Operations on Regions}
Various non side-effecting operators (i.e.{} pure functions) are
provided on regions. These allow the programmer to express sparse as
well as dense regions.

Let {\cf R} be a region. A subset of {\cf R} is also called a {\em
sub-region}.\index{region!sub-region}

Let {\cf R1} and {\cf R2} be two regions.

{\cf R1 \&\& R2} is the intersection of {\cf R1} and {\cf R2}.\index{region!intersection}

{\cf R1 || R2} is the union of the {\cf R1} and {\cf R2}.\index{region!union}

{\cf R1 - R2} is the set difference of {\cf R1} and {\cf R2}.\index{region!set difference}

Two regions are {\tt ==} if they represent the same set of points.\index{region!==}


\todo{ Need to determine if regions can be passed to and returned from methods.}

\todo{Can objects have region fields?}

\todo{ Need to determine if Xten control constructs already provide the nesting of regions of ZPL.}

\todo{ Need to determine if directions (and "of", wrap, reflect) make sense and should be included in Xten.}

\todo{ Need to add translations (ZPL @). Check whether they are widely useful.}

\todo{ Need to determine if {\tt index<d>} arrays are useful enough to include them.}




\section{Distributions}\label{XtenDistributions}
\index{distribution}

A {\em distribution} is a mapping from a region to a set of places.
{}\Xten{} provides a built-in value class, \xcd"x10.lang.Dist", to allow the creation of new distributions and
to perform operations on distributions. This class is \xcd"final" in
{}\XtenCurrVer; future versions of the language may permit
user-definable distributions. Since distributions play a dual role
(values as well as types), variables of type \xcd"Dist" must
be initialized and are implicitly \xcd"final".

The {\em rank} of a distribution is the rank of the underlying region.

%Recall that each program runs in a fixed number of places, determined
%by runtime parameters. The static constant place.MAX_PLACES specifies
%the maximum number of places. The collection of places is assumed to
%be totally ordered.


\begin{xten}
R: region = [1:100];
D: distribution = distribtion.factory.block(R);
D: distribution = distribution.factory.cyclic(R);
D: distribution = R -> here;
D: distribution = distribution.factory.random(R);
\end{xten}

Let \xcd"D" be a distribution. \xcd"D.region" denotes the underlying
region. \xcd"D.places" is the set of places constituting the range of
\xcd"D" (viewed as a function). Given a point \xcd"p", the expression
\xcd"D[p]" represents the application of \xcd"D" to \xcd"p", that is,
the place that \xcd"p" is mapped to by \xcd"D". The evaluation of the
expression \xcd"D[p]" throws an \xcd"ArrayIndexOutofBoundsException"
if \xcd"p" does not lie in the underlying region.

When operated on as a distribution, a region \xcd"R" implicitly
behaves as the distribution mapping each item in \xcd"R" to \xcd"here"
(i.e., \xcd"R->here", see below). Conversely, when used in a context
expecting a region, a distribution \xcd"D" should be thought of as
standing for \xcd"D.region".

{}\todo{Allan: We do not specify how the values of an array at a place
are stored, e.g. in row-major or column major order. Need to work this
out.}

\subsection{Operations returning distributions}

Let \xcd"R" be a region, \xcd"Q" a set of places \{\xcd"p1", \dots, \xcd"pk"\}
(enumerated in canonical order), and \xcd"P" a place. All the operations
described below may be performed on \xcd"Dist.factory".

\paragraph{Unique distribution} \index{distribution!unique}
The distribution \xcd"unique(Q)" is the unique distribution from the
region \xcd"1:k" to \xcd"Q" mapping each point \xcd"i" to \xcd"pi".

\paragraph{Constant distributions.} \index{distribution!constant}
The distribution \xcd"R->P" maps every point in \xcd"R" to \xcd"P".

\paragraph{Block distributions.}\index{distribution!block}
The distribution \xcd"block(R, Q)" distributes the elements of \xcd"R"
(in order) over the set of places \xcd"Q" in blocks  as
follows. Let $p$ equal \xcd"|R| div N" and $q$ equal \xcd"|R| mod N",
where \xcd"N" is the size of \xcd"Q", and 
\xcd"|R|" is the size of \xcd"R".  The first $q$ places get
successive blocks of size $(p+1)$ and the remaining places get blocks of
size $p$.

The distribution \xcd"block(R)" is the same distribution as {\cf
block(R, place.places)}.

\todo{Check into block distributions per dimension.}
\paragraph{Cyclic distributions.} \index{distribution!cyclic}
The distribution \xcd"cyclic(R, Q)" distributes the points in \xcd"R"
cyclically across places in \xcd"Q" in order.

The distribution \xcd"cyclic(R)" is the same distribution as \xcd"cyclic(R, place.places)".

Thus the distribution \xcd"cyclic(place.MAX_PLACES)" provides a 1--1
mapping from the region \xcd"place.MAX_PLACES" to the set of all
places and is the same as the distribution \xcd"unique(place.places)".

\paragraph{Block cyclic distributions.}\index{distribution!block cyclic}
The distribution \xcd"blockCyclic(R, N, Q)" distributes the elements
of \xcd"R" cyclically over the set of places \xcd"Q" in blocks of size
\xcd"N".

\paragraph{Arbitrary distributions.} \index{distribution!arbitrary}
The distribution \xcd"arbitrary(R,Q)" arbitrarily allocates points in {\cf
R} to \xcd"Q". As above, \xcd"arbitrary(R)" is the same distribution as
\xcd"arbitrary(R, place.places)".

\todo{Determine which other built-in distributions to provide.}

\paragraph{Domain Restriction.} \index{distribution!restriction!domain}

If \xcd"D" is a distribution and \xcd"R" is a sub-region of {\cf
D.domain}, then \xcd"D | R" represents the restriction of \xcd"D" to
\xcd"R".  The compiler throws an error if it cannot determine that
\xcd"R" is a sub-region of \xcd"D.domain".

\paragraph{Range Restriction.}\index{distribution!restriction!range}

If \xcd"D" is a distribution and \xcd"P" a place expression, the term
\xcd"D | P" denotes the sub-distribution of \xcd"D" defined over all the
points in the domain of \xcd"D" mapped to \xcd"P".

Note that \xcd"D | here" does not necessarily contain adjacent points
in \xcd"D.region". For instance, if \xcd"D" is a cyclic distribution,
\xcd"D | here" will typically contain points that are \xcd"P" apart,
where \xcd"P" is the number of places. An implementation may find a
way to still represent them in contiguous memory, e.g., using a
complex arithmetic function to map from the region index to an index
into the array.

\subsection{User-defined distributions}\index{distribution!user-defined}

Future versions of \Xten{} may provide user-defined distributions, in
a way that supports static reasoning.

\todo{TBD. Make distribution a value type. What is the API? Return a
set of indices. For each index point, return the place. A method to
return a subdistribution given a subregion. A method to check if a
given distribution is a subdistribution. But may need to provide methods that the compiler can use to reason about the distribution.

Postpone to 0.7.}

\subsection{Operations on distributions}

A {\em sub-distribution}\index{sub-distribution} of \xcd"D" is any
distribution \xcd"E" defined on some subset of the domain of \xcd"D",
which agrees with \xcd"D" on all points in its domain. We also say
that \xcd"D" is a {\em super-distribution} of \xcd"E". A distribution
\xcd"D1" {\em is larger than} \xcd"D2" if \xcd"D1" is a
super-distribution of \xcd"D2".

Let \xcd"D1" and \xcd"D2" be two distributions.  


\paragraph{Intersection of distributions.}\index{distribution!intersection}
\xcd"D1 && D2", the intersection of \xcd"D1" and \xcd"D2", is the
largest common sub-distribution of \xcd"D1" and \xcd"D2".

\paragraph{Asymmetric union of distributions.}\index{distribution!union!asymmetric}
\xcd"D1.overlay(D2", the asymmetric union of \xcd"D1" and \xcd"D2", is the
distribution whose domain is the union of the regions of \xcd"D1" and
\xcd"D2", and whose value at each point \xcd"p" in its domain is \xcd"D2[p]"
if \xcd"p" lies in \xcd"D.domain" otherwise it is \xcd"D1[p]". (\xcd"D1" provides the defaults.)

\paragraph{Disjoint union of distributions.}\index{distribution!union!disjoint}
\xcd"D1 || D2", the disjoint union of \xcd"D1" and \xcd"D2", is
defined only if the domains of \xcd"D1" and \xcd"D2" are disjoint. Its
value is \xcd"D1.overlay(D2)" (or equivalently \xcd"D2.overlay(D1)".
(It is the least super-distribution of \xcd"D1" and \xcd"D2".)

\paragraph{Difference of distributions.}\index{distribution!difference}
\xcd"D1 - D2" is the largest sub-distribution of \xcd"D1" whose domain is
disjoint from that of \xcd"D2".


\subsection{Example}
\begin{xten}
def dotProduct(a: array[T](D), b: array[T](D)): array[double](D) {
  return (new array[T]([1:D.places]) (j: point) => (
      (new array[T](D | here) (i: point) => a[i]*b[i]).sum();
  )).sum();
}
\end{xten}

This code returns the inner product of two \xcd"T" vectors defined
over the same (otherwise unknown) distribution. The result is the sum
reduction of an array of \xcd"T" with one element at each place in the
range of \xcd"D". The value of this array at each point is the sum
reduction of the array formed by multiplying the corresponding
elements of \xcd"a" and \xcd"b" in the local sub-array at the current
place.




\section{Array initializer}\label{ArrayInitializer}\label{array!creation}
An array initializer creating a new array with distribution {\cf D} may
optionally take a parametrized block of the form {\cf (ind1,...,
indk)\{S\}}. Here, {\cf k} may be zero; in this case the statement is written
as just {\cf \{S\}}. For instance:
\begin{x10}
\_ data = new int value [1000] 
    (i)\{ return i*i; \};
\_ data2 = new int[1000->current]@threadlocal 
    \{ return 1; \};
\end{x10}

{}\noindent The first declaration stores in {\tt data} an (immutable)
array whose distribution is {\tt (1..1000)-> here}, which is created
{\tt here}, and which is initialized with the value {\tt i*i} at index
{\tt i}. 

The second declaration stores in {\tt data2} a reference to a mutable
array (allocated in the {\tt threadlocal} region of the current
activity) with {\tt 1000} elements each of which is located in the
same place as the array (hence is {\tt threadlocal}). Each array
component is initialized to {\tt 1}.


In general the expression
\begin{x10}
    \_ data =  new T[D]@P (ind1, ..., indk) \{ S \}
\end{x10}

\noindent should be thought of as creating a new array located at {\tt P} with a {\tt k}-dimensional distribution {\tt D} such that the elements of 
the array are initialized as if by execution of the code:

\begin{x10}
   ateach(ind1, ..., indk : D) \{
       A[ind1, ..., indk] = 
     (new Object \{ T val(D ind1,...,indk) \{S\}\})
     .val(ind1, ..., indk);
   \}
\end{x10}

Notice that in the method declaration {\cf D} is used as a type. 

Other examples:
\begin{x10}
\_ data = new int[1000](i)\{return i*i; \};
float[D] d = new float[D] (i)\{return 10.0*i; \}; 
float[D] d2 = new float[D] (i)\{return i*i; \};
float[D] result = new float[D] 
      (i) \{return d[i] + d2[i]; \};
\end{x10}

\section{Operations on arrays}
In the following let {\tt a} be an array with distribution {\tt D} and
base type {\tt T}. {\tt a} may be mutable or immutable, unless
indicated otherwise.

\subsection{Element operations}\index{array!access}
The value of {\tt a} at a point {\tt p} in its region of definition is
obtained by using the indexing operation {\tt a[p]}. This operation
may be used on the left hand side of an assignment operation to update
the value.

\subsection{Constant promotion}\label{ConstantArray}\index{arrays!constant promotion}

For a distribution {\tt D} and a constant or final variable {\tt v} of
type {\tt T} the expression {\tt D v} denotes the mutable array with
distribution {\tt D} and base type {\tt T} initialized with {\tt v}.

\subsection{Restriction of an array}\index{array!restriction}

Let {\tt D1} be a sub-distribution of {\tt D}. Then {\tt a[D1]}
represents the sub-array of {\tt a} with the distribution {\tt D1}.

Recall that a rich set of operators are available on distributions
(\S~\ref{XtenDistributions}) to obtain sub-distributions
(e.g. restricting to a sub-region, to a specific place etc).

\subsection{Assembling an array}
Let {\tt a1,a2} be arrays of the same base type {\tt T} defined over
distributions {\tt D1} and {\tt D2} respectively. Assume that both
arrays are value or reference arrays. 
\paragraph{Assembling arrays over disjoint regions}\index{array!union!disjoint}

If {\tt D1} and {\tt D2} are disjoint then the expression {\tt a1 ||
a2} denotes the unique array of base type {\tt T} defined over the
distribution {\tt D1 || D2} such that its value at point {\tt p} is
{\tt a1[p]} if {\tt p} lies in {\tt D1} and {\tt a2[p]}
otherwise. This array is a reference (value) array if {\tt a1} is.

\paragraph{Overlaying an array on another}\index{array!union!asymmetric}
The expression
{\tt a1.over(a2)} (read: the array {\tt a1} {\em overlaid on} {\tt a2})
represents an array whose underlying region is the union of that of
{\tt a1} and {\tt a2} and whose distribution maps each point {\tt p}
in this region to {\tt D1[p]} if that is defined and to {\tt D2[p]}
otherwise. The value {\tt a1.over(a2)[p]} is {\tt a1[p]} if it is defined and {\tt a2[p]} otherwise.

This array is a reference (value) array if {\tt a1} is.

\todo{Define Flooding of arrays}

\todo{Wrapping an array}

\todo{Extending an array in a given direction.}

\subsection{Global operations }
\paragraph{Pointwise operations}\label{ArrayPointwise}\index{array!pointwise operations}
Suppose that {\tt m} is an operation defined on type {\tt T} that
takes an argument of type {\tt S} and returns a value of type {\tt
R}. Such an operation can be lifted pointwise to operate on a {\tt T}
array and an {\tt S} array defined over the same distribution {\tt D}
to return an {\tt R} array defined over {\tt D}.

The syntax for such pointwise application is {\tt a.m(b)} where {\tt a} and
{\tt b} are {\tt D} arrays.

\paragraph{Reductions}\label{ArrayReductions}\index{array!reductions}

Let {\tt m} be a reduction operator (\S~\ref{ReductionOperator})
defined on type {\tt T}. Let {\tt a} be a value or reference array
over base type {\tt T}. Then the operation {\tt a>>m()} returns a
value of type {\tt T} obtained by performing {\tt m} on all points in
{\tt a} in some order.

This operation involves communication between the places over which
the array is distributed. The \Xten{} implementation guarantees that
only one value of type {\tt T} is communicated from a place as part of
this reduction process.

\paragraph{Scans}\label{ArrayScans}\index{array!scans}

Let {\tt m} be a reduction operator (\S~\ref{ReductionOperator})
defined on type {\tt T}. Let {\tt a} be a value or reference array
over base type {\tt T} and distribution {\tt D}. Then the operation
{\tt a||m()} returns an array of base type {\tt T} and distribution
{\tt D} whose {\tt i}th element (in canonical order) is obtained by
performing the reduction {\tt m} on the first {\tt i} elements of {\tt
a} (in canonical order).

This operation involves communication between the places over which
the array is distributed. The \Xten{} implementation will endeavour to
minimize the communication between places to implement this operation.

