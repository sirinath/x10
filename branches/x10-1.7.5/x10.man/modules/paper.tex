%\documentclass[preprint,nocopyrightspace,10pt]{sigplanconf}
\documentclass{article}

\usepackage{charter}
%\usepackage{times-lite}
%\usepackage{mathptm}
\usepackage{txtt}
\usepackage{stmaryrd}
\usepackage{code}
\usepackage{bcprules}
%\usepackage{ttquot}
\usepackage{amsmath}
\usepackage{amssymb}
\usepackage{afterpage}
\usepackage{balance}
\usepackage{floatflt}
\usepackage{defs}
\usepackage{utils}
\usepackage{graphicx}
\usepackage{xspace}
\usepackage{ifpdf}
\usepackage{listings}
\usepackage{x10}
\usepackage{color}

\hfuzz=1pt

\pagestyle{plain}

\ifpdf
\setlength{\pdfpagewidth}{8.5in}
\setlength{\pdfpageheight}{11in}
\fi

\newcommand\Xten{{\sf X10}\xspace}
\newcommand\xbar[1]{\ensuremath{\bar{\Xcd{#1}}}}
\newcommand\tbar[1]{\ensuremath{\bar{\tt {#1}}}}

\newcommand\Figref[1]{Figure~\ref{fig:#1}}
\newcommand\Secref[1]{Section~\ref{sec:#1}}

\newcommand\RED[1]{\textcolor{red}{#1}}
\newcommand\BLUE[1]{\textcolor{blue}{#1}}
\newcommand\GREEN[1]{\textcolor{green}{#1}}
\newcommand\TODO[1]{\RED{#1}}

\begin{document}

\title{Proposal: A Module System for X10}

\author{Nathaniel Nystrom \\
\small{IBM T.~J. Watson Research~Center,
P.O.~Box~704, Yorktown~Heights NY 10598 USA} \\
\small{\texttt{nystrom@us.ibm.com}}}

\date{\today}

\maketitle

\eat{
\begin{abstract}
We propose a module system for \Xten.
\end{abstract}
}

\section{Introduction}
\label{sec:intro}

As \Xten scales up to programming of large, high-performance applications,
the need for a robust module system becomes apparent.
Modules provide a mechanism for high-level code reuse,
encapsulation, and abstraction.
Modules give users control over sharing and separation of data.
Modules in \Xten are motivated by the following requirements:

\begin{itemize}
\item
\Xten classes and interfaces are currently organized into
packages, which provide limited access control features,
inadequate for large-scale code reuse.
\emph{Modules} allow the programmer higher-level control
over the loading and linking of code.

\item
\Xten code is intended to be compiled to multiple architectures
and run on multiple runtime configurations.
\Xten needs to expose runtime configuration information 
into the programming language.  \emph{Configuration constants} 
are link-time or load-time constants.

\item
For effective static analysis and optimization, the
\Xten compiler needs to be able to
statically bind method references to their definitions in order to
inline or specialize code.
\emph{Sealed modules} allow the compiler to be made aware of all code
that refers to a given definition.

\item
The \Xten specification needs to clarify the 
initialization order of static data.

\item
The specification needs to clarify how static state is
replicated (or not replicated) across places.

\item
\Xten needs a system for packaging and distributing code spanning
multiple classes.
\end{itemize}

We propose a module system for \Xten.  The design is based on
the module system for Thorn
designed by Rok Strni\v{s}a~\cite{thorn}, which is in turn based
on the proposed Java Module System (JAM~\cite{SSP:07}).

The module system is not backward compatible with the current 
\Xten semantics.  For instance, we remove packages and the semantics for
\c{import} differ.

The basic unit of the system is the \emph{module}.
Modules hold \emph{members}---classes, interfaces,
type definitions, constants, functions.
Module definitions control access to their members:
\emph{public} members are \emph{exported} to other
modules;
\emph{module-private} members are visible only
to other members of the module.

Modules can \emph{import} other modules, making the public
members of those modules visible in the importing module.
Imported members are not re-exported by default.

\subsection{Things to revisit}

\begin{itemize}
\item Modules do not nest.
\item Modules are not inheritable.
\item Modules are not parameterized on other modules.
\item Module imports are acyclic, defining initialization order completely.
\item Consts are only allowed at the module level.
(Or else define initialization order between classes in a
module.)
\item Static methods are only allowed at the module level.
\end{itemize}

\section{Module definitions}

Any \Xten construct (except modules) with a
name (class, constant, variable, method,
function, etc.) is called an \emph{entity}.
The different kinds of entities nest.  Entities whose parent is
the module itself are called \emph{members}.
\TODO{Fix these definitions.}
Modules may contain the following members:
\begin{itemize}
\item class definitions
\item interface definitions
\item type definitions\footnote{Note: this will fix the limitation in the current language that type definitions cannot be top-level.}
\item constants (final static variables)
\item static functions\footnote{And operators if we add operator
overloading}
\end{itemize}

A module definition is instantiated at load-time as a \emph{module instance}.
Each module instance has its own copy of the
\c{const} variables defined by the module.
A given module definition may be instantiated more than once.
All entities in \Xten belong to exactly one module instance.

Modules are specified in a module file, 
given by the grammar in \Figref{modfile}.
An \Xten source file is given by the grammar in \Figref{srcfile}.

\begin{figure}
\begin{centering}
\begin{tabular}{rcl}
\emph{ModuleFile} 
  & ::= &
    \emph{ModuleDeclaration}
    \emph{ModuleMember}$^*$
    \\
\emph{ModuleDeclaration} 
  & ::= &
    \emph{ModuleFlags} \c{module}
    \emph{ModuleName} \c{;}
    \\
\emph{ModuleFlags}
  & ::= &
    \c{sealed}
  \bnf  
    $\epsilon$
    \\
\emph{ModuleMember}
  & ::= &
    \emph{Import}
    \\
  & \bnf &
    \emph{ClassDeclaration}
    \bnf  
    \emph{InterfaceDeclaration}
    \bnf  
    \emph{TypeDefinition}
    \\
  & \bnf &
    \emph{ConstantDefinition}
    \bnf  
    \emph{FunctionDeclaration}
    \\
  & \bnf &
    \emph{RequireDeclaration}
    \bnf  
    \emph{ExportDeclaration}
    \bnf  
    \emph{IncludeDeclaration}
    \\
  & \bnf &
    \c{;}
    \\
\emph{Import}
  & ::= &
    \emph{ImportMemberFlags}
    \c{import} \emph{ImportFlags} \emph{ModuleAlias}
    \c{.} \emph{Members}
    \emph{ImportQualifiers}
    \c{;}
  \\
  & \bnf &
    \emph{ImportMemberFlags}
    \c{import} \emph{ImportFlags} \emph{ImportSpecifier}
    \emph{ImportQualifiers}
    \c{;}
  \\
  & \bnf &
    \c{with} \emph{ImportFlags} \emph{ModuleAlias}
    \emph{ImportQualifiers}
    \c{\{}
    \emph{SimpleImport}$^*$
    \c{\}}
  \\
\emph{ImportQualifiers}
  & ::= &
    ( \c{from} \emph{URL} )$^?$
    ( \c{with} \emph{Initializer} ( \c{,} \emph{Initializer} )$^*$ )$^?$
    \\
\emph{Initializer}
  & ::= &
    \emph{Identifier} \c{=} \emph{Expression}
  \\
  & \bnf &
    \emph{Identifier} \c{=} \emph{Type}
  \\
\emph{SimpleImport}
  & ::= &
    \emph{ImportMemberFlags}
    \c{import} \emph{Members} \c{;}
  \\
  & \bnf &
    \emph{ImportMemberFlags}
    \c{import} \emph{MemberSpecifier} \c{;}
  \\
\emph{ImportFlags}
  & ::= &
    \c{own}
  \bnf
    $\epsilon$
    \\
\emph{ImportMemberFlags}
  & ::= &
    \c{public}
  \bnf
    $\epsilon$
    \\
\emph{Members}
  & ::= &
    \c{*} ( \c{except}
           \emph{Identifier} ( \c{,} \emph{Identifier} )$^*$ )$^?$
    \\
  & \bnf &
    \c{\{} \emph{MemberSpecifier} ( \c{,} \emph{MemberSpecifier} )$^*$ \c{\}}
    \\
\emph{ImportSpecifier}
  & ::= &
    \emph{ModuleAlias} \c{.} \emph{Identifier}
    ( \c{as} \emph{Identifier} )$^?$
    \\
  & \bnf &
    \c{class} \emph{ModuleAlias} \c{.} \emph{Identifier}
    ( \c{as} \emph{Identifier} )$^?$
    \\
  & \bnf &
    \c{interface} \emph{ModuleAlias} \c{.} \emph{Identifier}
    ( \c{as} \emph{Identifier} )$^?$
    \\
  & \bnf &
    \c{type} \emph{ModuleAlias} \c{.} \emph{Identifier}
    ( \c{as} \emph{Identifier} )$^?$
    \\
  & \bnf &
    \c{def} \emph{ModuleAlias} \c{.} \emph{Identifier}
    ( \c{as} \emph{Identifier} )$^?$
    \\
  & \bnf &
    \c{const} \emph{ModuleAlias} \c{.} \emph{Identifier}
    ( \c{as} \emph{Identifier} )$^?$
    \\
\emph{MemberSpecifier}
  & ::= &
    \emph{Identifier}
    ( \c{as} \emph{Identifier} )$^?$
    \\
  & \bnf &
    \c{class} \emph{Identifier}
    ( \c{as} \emph{Identifier} )$^?$
    \\
  & \bnf &
    \c{interface} \emph{Identifier}
    ( \c{as} \emph{Identifier} )$^?$
    \\
  & \bnf &
    \c{type} \emph{Identifier}
    ( \c{as} \emph{Identifier} )$^?$
    \\
  & \bnf &
    \c{def} \emph{Identifier}
    ( \c{as} \emph{Identifier} )$^?$
    \\
  & \bnf &
    \c{const} \emph{Identifier}
    ( \c{as} \emph{Identifier} )$^?$
    \\
\emph{RequireDeclaration}
  & ::= &
    \c{require} \emph{RequireSpecifier} \c{;}
    \\
  & \bnf &
    \c{require} \emph{Guard} \c{;}
    \\
\emph{RequireSpecifier}
  & ::= &
    \c{class} \emph{Identifier}
      \emph{TypeParameters}$^?$
      \emph{TypeConstraint}$^?$
    \\
  & \bnf &
    \c{interface} \emph{Identifier}
      \emph{TypeParameters}$^?$
      \emph{TypeConstraint}$^?$
    \\
  & \bnf &
    \c{type} \emph{Identifier}
      \emph{TypeParameters}$^?$
      \emph{FormalParameters}$^?$
      \emph{Guard}$^?$
      \emph{TypeConstraint}$^?$
    \\
  & \bnf &
    \c{def} \emph{Identifier}
      \emph{TypeParameters}$^?$
      \emph{FormalParameters}
      \emph{Guard}$^?$
      \c{:}
      \emph{Type}
      \emph{Throws}$^?$
    \\
  & \bnf &
    \c{const} \emph{Identifier}
      \emph{Guard}$^?$
      \c{:}
      \emph{Type}
    \\
\emph{TypeConstraint}
  & ::= &
    \c{<:} \emph{Type}
  \bnf
    \c{:>} \emph{Type}
      \\
\emph{ExportDeclaration}
  & ::= &
    \c{public} \emph{MemberSpecifier} \c{;}
    \\
\emph{IncludeDeclaration}
  & ::= &
    \c{include} \emph{URL} \c{;}
    \\
\emph{ModuleAlias}
  & ::= &
    \emph{ModuleName}
    ( \c{as} \emph{Identifier} )$^?$
    \\
\emph{ModuleName}
  & ::= &
    \emph{ModuleName} \c{.} \emph{Identifier}
  \bnf
    \emph{Identifier}
    \\
\end{tabular}
\end{centering}
\caption{Module file grammar}
\label{fig:modfile}
\end{figure}

\begin{figure}
\begin{centering}
\begin{tabular}{rcl}
\emph{SourceFile} 
  & ::= &
    \emph{ModuleDeclaration}$^?$
    \emph{SourceFileMember}$^+$
    \\
\emph{SourceFileMember}
  & ::= &
    \emph{Import}
    \\
  & \bnf &
    \emph{ClassDeclaration}
    \\
  & \bnf &
    \emph{InterfaceDeclaration}
    \\
  & \bnf &
    \emph{TypeDefinition}
    \\
  & \bnf &
    \emph{ConstantDefinition}
    \\
  & \bnf &
    \emph{FunctionDeclaration}
    \\
  & \bnf &
    \emph{ExportDeclaration}
    \\
  & \bnf &
    \c{;}
    \\
\end{tabular}
\end{centering}
\caption{Source file grammar}
\label{fig:srcfile}
\end{figure}


\eat{
Example 1:
\begin{xten}
module M;
import N as O.{x,y,z} from URL;
\end{xten}

Example 2 is equivalent to Example 1:
\begin{xten}
module M;
with N as O from URL {
    import x;
    import y;
    import z;
}
\end{xten}

Example 3:
\begin{xten}
module M;
import N.* from URL;
\end{xten}

Example 4 is equivalent to Example 3:
\begin{xten}
module M;
with N from URL import *;
\end{xten}
}

% Also:
% require Id [ Bound ];
% and:
% import Module { id = Foo }
% and:
% export Name as Id; // hides Name, makes Id public
% (same as alias Id = Name; public Id;)
% (or: private Name; public Id = Name;}

By default, a module definition includes all declarations in the module
file and in all
source files in the current directory.  This can be overridden
by using \c{include} declarations in the module file to specify
URLs for source files.  If the module file contains an
\c{include} declaration, the module contains only the definitions in the
module file and in the source files 
explicitly \c{include}d.\footnote{\TODO{Revisit this.}}

These semantics allow the compiler to identify
precisely which declarations are contained in a given module.
Combined with \emph{module sealing}, described in
\Secref{sealed}, this permits the compiler to identify
all uses of a given declaration.

Modules import \c{public} members from other modules using
either a module import or a member import statement.
Both module files and source files 
may contain both kinds of import statements.
When compiling a module, the imports in the module file and in
all source files are collected together into the \emph{module
dependencies}.

The scope of an import in a module file is the
entire module; the scope of an import in a source file
is the source file only.
Source files do not define their own namespaces.
Definitions in source files do not shadow definitions in the
module file, including (explicitly) imported definitions.
It is an error for a module to contain two members of the same
name and signature.

The following defines a module for \c{x10.array}.

\begin{xten}
sealed module x10.array;
import x10.lang.*;
public class Array[T];
public class Region;
public class Dist;
public class Point;
public class NoSuchElementException;
public class RankMismatchException;
\end{xten}

This module includes all source files in the directory containing
the module file (e.g., \texttt{x10/array}).
The module imports the public members of \c{x10.lang} and
exports 6 classes of \c{x10.array}.
The module is sealed: other modules cannot extend any of the
classes or interfaces defined in the module.

\section{Name resolution}

References to module members can be qualified with a module name,
e.g, \c{M.x}.  The definition of \c{x} is searched for within
module \c{M}.

\Xten source can also refer to entities via unqualified names (e.g.,
\c{x}).  If the name is not resolved in a 
local scope (e.g., as a local variable, as a field of
a lexically enclosing class or inherited into the class),
the module system performs name resolution.
Name resolution maps unqualified names
to fully qualified names.  All
references are resolved at compile time.

If a name in file $F$ in module $M$ is not fully qualified, 
it is searched for in the following order:
\begin{enumerate}
\item within the members of $M$.
\item within the module imported by $F$ and by $M$.
\end{enumerate}
This ordering is chosen to be robust against changes outside the
source file or module.

Publicly visible namespaces of imported modules are allowed to
clash.  It is only an error if an imported name that clashes is actually used.
It is an error if an explicitly imported member's signature clashes
with another explicitly imported member
or with a member defined in the importing module.

A qualified reference is unambiguous if the name
of its prefix is unambiguous.

Modules and members can be aliased on import, 
e.g., \c{import M as N.*}, \c{import M.C as D}.
Aliases in a source file are visible only in that file.
Aliases in a module file are visible to all files in the module.
Aliases introduce names in the same namespace as module
members.  It is an error for an alias to clash with the name
of any member defined in the module or any other alias.

Imported names are invisible to importing modules; that is,
imports are not transitive.
Members by default are not exported.
A module can export the name of a member or of an alias by declaring 
it \c{public}.
A module can re-export an imported name \c{x} with the export
statement 
(\c{public x}) or by qualifying the \c{import} itself with \c{public}.
Exported names must not clash.  It is illegal to \c{import}
\c{*} with a \c{public} flag; only explicitly named members can be
imported \c{public}.

The module prefix \c{M} of a name \c{M.x} must be the name of a
directly imported module or its alias.
It is an error to refer to re-exported 
entities by using the name of the defining module.  This
localizes name ambiguity issues.

\section{Imports}

The module import statement \c{import M.*} (and variants)
adds all public members of \c{M} into the namespace
of the importing module, and adds a module dependency on \c{M}.
Module imports subsume
\Xten 1.7's
on-demand imports (\c{import p.*}).

A name clash is \emph{not} introduced if an imported module
contains a member with the same name as another member
(whether declared in the module or imported); it is only an
error if that name is used.

Names of imported modules in overlapping scopes must not clash.
For instance, if a module file imports module \c{M},
the module file or a source file in the module cannot import a
module \c{M}.
This rule allows the user always to be able of resolving ambiguities
by prefixing a name with the module name.
Module aliases can be used to avoid clashes.

The member import statement \c{import M.x} (and variants)
adds only the public member \c{x} in module \c{M} to the namespace
of the importing module, and adds a module dependency on \c{M}.
It is a link-time error if \c{M.x} does not exist or is not public.

Member imports are similar to imports in \Xten 1.7 or earlier.
If a module 
An \emph{on-demand import} \c{import M.*} adds a dependency on module \c{M}
and adds all exported members of \c{M} to the namespace of the
importing module.
Multiple names can be explicitly imported, e.g., \c{import M.\{x,y,z\}}.

An explicit import may optionally alias the imported entity.
The statement \c{import y = M.x} adds a dependency on module \c{M}
and adds \c{y} to the namespace as an alias for \c{x}.
\c{x} is not added to the namespace.

\subsection{Shared vs. own imports}

A module can define constant static state via \c{const}
definitions.  Each module instance has its own copy of the
\c{const} variables defined by the module.

By default all module instances are shared.
As part of an import statement, a module may be unshared
by prefixing the imported module with \c{own}.

If a module \c{N} requires its own instance of \c{M}, a new
instance of \c{M} is created, which recursively creates
instances of any modules it imports.  Thus, if \c{M} imports
another module \c{P} as shared, that instance of \c{P} is shared
with other instances of \c{P} imported transitively by \c{M},
but not with instances of \c{P} imported by \c{M}'s importer, \c{N}.

Two shared (i.e., non-\c{own}) import statements referring to the same module
name always refer to the same module instance. 
Two \c{own} import statements (or an \c{own} import and a
non-\c{own} import) referring to the same module
name always refer to different module instances. 

A type defined in one module instance is valid in any
other; it is only state that is not shared.

Consider the following module:
\begin{xten}
module M;
const x: Object = new Object();
\end{xten}

In the following, \c{M} is imported twice, shared.  There is a
single copy of \c{M.x} accessible as \c{N.x} and \c{O.x}.
\begin{xten}
module M1;
import M as N.*; // import shared
import M as O.*; // import shared
assert N.x == O.x;
\end{xten}

In the following, \c{M} is imported twice, unshared.  There are two
copies of \c{M.x} accessible as \c{N.x} and \c{O.x}.
\begin{xten}
module M2;
import own M as N.*;
import own M as O.*;
assert N.x != O.x;
\end{xten}

\subsection{Distributed vs. global imports}

\TODO{Should we have global vs. distributed \emph{modules} instead?}
\TODO{Should we have global vs. distributed \c{const} instead?}

A module can be imported either
\emph{distrubted}\footnote{Alternatively: place-local.} or \emph{global}.
The default is global. 
The distribution decision is orthogonal to the sharing decision.

If a module is imported \emph{global}, all places share the
module instance.
When a module variable (constant) is accessed, the same value is
returned regardless of the place from which it is accessed.
Initialization code must be place-independent.
In \Xten 2.0, global methods are place-independent; the same
restrictions apply here.
Initializer code for a module runs once per module instance.

If a module is imported
\emph{distributed}, each place (or
a set of places) has its own module instance; that is, the
module's state is place-indexed.
Each place has a replica of the module, each initialized
separately.
The runtime system can share
module instances across multiple places; for example, one could
map each core of a multicore processor to a different place, but have one 
module instance shared across all cores.
Accessing the module's state returns the replca
for the place that is running the activity doing the access.

\TODO{Distributed modules will not work with dynamic places:
modules with place-local data cannot be initialized at load-time.}

The name of a constant in a distributed module is accessible
everywhere, but may contain different values depending on the
place from which it is accessed.  
One does not need to \c{async} to access a constant; the
place-local value is always returned.

\eat{
\TODO{Fix this.  Really just to simplify things.  I had a better
reason for this, but forgot it.}
Note that modules are distributed rather than variables.
This is to avoid initialization issues: a module constant may
depend only on other state in the module instance, or in state imported
from other module instances.  Since imported state is also
replicated across places, module instance initialization can be entirely
place-local.  If individual variables are distributed,
the initialization would have to access both place-local and
global state.
}

Consider the following module:
\begin{xten}
module M;
const x: Object = new Object();
\end{xten}

Module \c{M1} imports \c{M}, shared.
\c{M.x} at place \c{p1}
and
\c{M.x} at place \c{p2}
are identical.
\begin{xten}
module M1;
import M.*;
assert (at (p1) M.x) == (at (p2) M.x);
\end{xten}

Module \c{M2} imports \c{M}, shared.
\c{M.x} at place \c{p1}
and
\c{M.x} at place \c{p2}
are different values.
\begin{xten}
module M2;
require const S: Partition[Place];
import distributed(S) M.*;
assert (at (p1) M.x) != (at (p2) M.x);
\end{xten}

In this example, \c{M} is parametrized on the constant \c{S}.
Parametrized modules are discussed further in \Secref{generics}.

Distributed modules are replicated across multiple places
according to the given partition.   Note the partition
can be provided by the runtime system or on the command-line via
configuration constants, discussed in \Secref{config}  

\TODO{We could also just say \c{import distributed M}.  In this
case, the partitioning is determined by the runtime.  This is
simpler and may be sufficient.}

\section{Module generics}
\label{sec:generics}

Modules can be parameterized on members using a \c{require}
declaration.  Modules importing a parameterized module must
provide a value for the parameter.  If no value is provided, a
link error occurs.

Module type parameters have the same restrictions as other type
parameters.  For instance,
it is an error to extend or implement a parameter type;
it is an error to \c{new} a parameter type.

Example:

\begin{xten}
module M;
require class A[T];
require A[T] <: Array[T];

// error: class C extends A[int] { }
// error: new A[T]();

def sort(a: A[int]) = ...
\end{xten}

\begin{xten}
module N;
import x10.array.FastArray;
import M.* with A = FastArray;
sort(Array.makeFast[int](100));
\end{xten}

Constant parameters and function parameters must specify a type
for the constant or function.  Class, interface, or type parameters may
optionally specify a bound on the type.

Constant parameters have an optional \c{default} value.
It is a load-time error if no value is provided by the runtime
system and there is no \c{default}.

The following defaults \c{SIZE} to 1000, but sets \c{SIZE} to 10000
at the import.
\begin{xten}
module M;
require const SIZE: int default 1000;
... 

module :;
import M.* with SIZE = 10000;
\end{xten}

\subsection{Configuration constants}
\label{sec:config}

A special kind of \c{const} parameter is the \emph{configuration constant},
a constant defined by the runtime system, usually (but not
necessarily) either on the
command-line or via a configuration file.

Example:
\begin{xten}
module N;
config SZ: int default 1000;
import M.* with SIZE = SZ;
\end{xten}

The \Xten runtime, \texttt{x10}, is invoked with a
command-line option to set the value
\texttt{x10} \texttt{-D:N.SZ=10000}.\footnote{This only
works for some constants that have a literal syntax.}

\c{config} constants can be provided at \c{import} also:
\begin{xten}
module P;
import N.* with SZ = 10000;
\end{xten}

\eat{
\section{Place definitions}

Places in \Xten are associated with module instances.

Need to statically know the capabilities of each place.

A \c{Place} is little more than an identifier.
The \c{PlaceDescriptor} class associates information with
a place.

\begin{xten}
class PlaceDescriptor {
  def place(): Place;
}
\end{xten}

\begin{xten}
class PPUPlaceDescriptor extends PlaceDescriptor {
  def spus(): Rail[Place]; // the SPUs on the same processor
}
\end{xten}

\begin{xten}
class SPUPlaceDescriptor extends PlaceDescriptor {
  def ppu(): Place;        // the PPU on the same processor
  def spus(): Rail[Place]; // the SPUs on the same processor
}
\end{xten}

OR:

\begin{xten}
module x10.cell;

import x10.{Rail,Place};

const PPU: Place;
const SPUs: Rail[Place];
\end{xten}

The boot-up code on Cell creates
separate module instances of \c{x10.cell} for each processor.
There are shared instances of other modules (e.g., \c{Math}).
But, how does that appear to the programmer?  The main module
runs on the PPU

Could also use different module instances to define \c{here}.
\begin{xten}
module x10.places;
const here: Place;
\end{xten}

\begin{xten}
module x10.lang;
import const x10.places.here distributed with Partition.Unique;
public const here: Place; // reexport
\end{xten}
}

\section{Sealed modules}
\label{sec:sealed}

A module can be declared \c{sealed}.  It is a link-time error if a class or
interface extends a class or interface in another module that is sealed.
This feature generalizes final classes: a class in a sealed module 
cannot be extended by classes in other modules, but can be extended by modules
in the current module.

\section{Loading and linking}

\paragraph{Linking.}

Linking begins with the \emph{main module}.  Module linking triggers, first,
recursive linking of module dependencies.
Dependencies are
processed in their order in the module definition.  It is a link-time error if
there is a cycle in module dependencies.  The order of linking defines the
\emph{link order}.  The main module is the last module in link order.

\paragraph{Loading.}

When an \Xten program starts, the runtime system loads the main module,
specified on the command-line.  In the Java implementation, module linking
now occurs.  Note that all link errors occur on program startup before any
user-level code is executed.  In the C++ implementation, linking is done
statically.

After linking, modules are initialized in their link order.
Any \c{const}
variables in the module are initialized, in order of definition.

\paragraph{Running.}

After module linking and loading, the runtime invokes the module-level
function \c{main(Rail[String])} in the main module, passing in the
command-line arguments.

\section{Default module}

If a source directory does not contain a module file, the following module
definition is used.

\begin{xtenmath}
module $\mbox{\normalfont\emph{fresh\_module\_name}}$;

import x10.lang.*;
import x10.io.Console;
import x10.arrays.*;
import x10.math.*;
\end{xtenmath}

\section{Tin cans}

\Xten sources are compiled into Tincode.
One or more modules are compiled into a \emph{can}.
If a module is in a can, then the can should contain all compiled
sources in the module; otherwise, it is a load-time error.

\eat{
\section{Properties}

\TODO{Future}

Modules may have properties, for instance a version number.
These are given by \c{const} fields of the module annotated with
the \c{property} flag.

When looking up a name, other modules can be filtered by the property.

\begin{xten}
module M;
property const Version: String = "1.0";
\end{xten}

\begin{xten}
module N;
import M with V1;

class V1 implements ModuleConstraint {
    def filter(M: Module) {
        return M.get("Version") as String == "1.0";
    }
}
\end{xten}
}

\eat{
\section{Place-indexed constants}

\TODO{Future}

A \c{const} variable may be declared \c{distributed} over a
partition of places.   For instance, we can define \c{here} using
the unique partition.

\begin{xten}
module x10.lang;
distributed with Partition.Unique const here: Place;
\end{xten}

On the Cell, we want to map each place representing a core to
the other cores on the same processor.
\begin{xten}
module x10.cell;
distributed with Partition.Nodes const ppu: Place;
distributed with Partition.Nodes const spu: Rail[Place]{length==8};
\end{xten}
}

\bibliographystyle{plain}
\bibliography{master}

\end{document}
