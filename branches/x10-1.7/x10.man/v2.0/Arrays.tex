\chapter{Arrays}\label{XtenArrays}\index{arrays}

An array is a mapping from a distribution to a range data
type. Multiple arrays may be declared with the same underlying
distribution. 

Each array has a field \xcd"a.dist" which may be used to
obtain the underlying
distribution.\index{arrays!distribution@{\tt distribution}}

The distribution underlying an array \xcd"a" may be obtained through
the field \xcd"a.dist".

\section{Points}\label{point-syntax}\index{point syntax}

Arrays are indexed by points--$n$-dimensional tuples of
integers, implemented by the value class \xcd"x10.lang.Point".
\Xten{} specifies a simple syntax for the construction of points.
A rail constructor (\Sref{RailConstructors}) of type \xcd"ValRail[Int]"
%or
%\xcd"ValRail[Long]" array
of length $n$
can be implicitly coerced to a \xcd"Point" of rank $n$.  For
example, the following code initializes \xcd"p" to a point of
rank two using a rail constructor:

\begin{xten}
p: Point = [1,2];
\end{xten}

The \xcd"Point" constructor can take a rail constructor as
argument.  The assignment above can be written, without
implicit coercion, as:

\begin{xten}
p: Point = new Point([1,2]);
\end{xten}

Points implement the function type \xcd"(Int) => Int"; thus, the
\xcd"i"the element of a point \xcd"p" may be accessed as \xcd"p(i)".

\section{Regions}\label{XtenRegions}\index{region}

A region is a set of points.  {}\Xten{}
provides a built-in class, \xcd"x10.lang.Region", to allow the
creation of new regions and to perform operations on regions. 

Each region \xcd"R" has a constant integer rank, \xcd"R.rank".
%% TODO: Should be uint.

Here are several examples of region declarations:
\begin{xten}
val MAX_HEIGHT=20;
val Null = Region.makeUnit();  // Empty 0-dimensional region          
val N = 10;
val K = 2;
val R1 = 1..100; // 1-dim region with extent 1..100
val R2 = [1..100] as Region(1); // same as R1
val R3 = (0..99) * (-1..MAX_HEIGHT);   
val R4 = [0..99, -1..MAX_HEIGHT] as Region(2); // same as R3  
val R5 = Region.makeUpperTriangular(N);
val R7 = R4 && R5; // intersection of two regions
val R8 = R4 || R5; // union of two regions
\end{xten}

The expression \xcdmath"a$_1$..a$_2$"
is shorthand for the rectangular, rank-1 region
consisting of the points
$\{$\xcdmath"[a$_1$]", \dots, \xcdmath"[a$_2$]"$\}$.
Each subexpression of \xcdmath"a$_i$" must be of type \xcd"Int".
If \xcdmath"a$_1$"
is greater than \xcdmath"a$_2$", the region is empty.

A region may be constructed by converting from a rail of
regions or a rail of points, typically using a rail constructor
(\Sref{RailConstructors})
(e.g., \xcd"R4" above).
The region constructed from a rail of points represents the
region containing just those points.
The region constructed from a rail of regions
represents
the Cartesian product of each of the arguments.
\XtenCurrVer{} does not (yet) support hierarchical regions.

\index{region!upperTriangular}
\index{region!lowerTriangular}\index{region!banded}

Various built-in regions are provided through  factory
methods on \xcd"Region".  For instance:
\begin{itemize}
\item \xcd"Region.makeUpperTriangular(N)" returns a region corresponding
to the non-zero indices in an upper-triangular \xcd"N x N" matrix.
\item \xcd"Region.makeLowerTriangular(N)" returns a region corresponding
to the non-zero indices in a lower-triangular \xcd"N x N" matrix.
\end{itemize}

All the points in a region are ordered canonically by the
lexicographic total order. Thus the points of a region \xcd"R=(1..2)*(1..2)"
are ordered as 
\begin{xten}
(1,1), (1,2), (2,1), (2,2)
\end{xten}
Sequential iteration statements such as \xcd"for" (\Sref{ForAllLoop})
iterate over the points in a region in the canonical order.

A region is said to be {\em rectangular}\index{region!convex} if it is of
the form \xcdmath"(T$_1$ * $\cdots$ * T$_k$)" for some set of regions
\xcdmath"T$_i$". Such a
region satisfies the property that if two points $p_1$ and $p_3$ are
in the region, then so is every point $p_2$ between them (that is, it is {\em convex}). 
(Note that \xcd"||" may produce non-convex regions from convex regions, e.g.,
\xcd"[1,1] || [3,3]" is a non-convex region.)

For each region \xcd"R", the {\em rectangular closure} of \xcd"R" is the
smallest rectangular region enclosing \xcd"R".  For each integer \xcd"i"
less than \xcd"R.rank", the term \xcd"R(i)" represents the enumeration
in the \xcd"i"th dimension of the rectangular closure of \xcd"R". It may be
used in a type expression wherever an enumeration may be used.

\subsection{Operations on regions}
Various non side-effecting operators (i.e., pure functions) are
provided on regions. These allow the programmer to express sparse as
well as dense regions.

Let \xcd"R" be a region. A subset of \xcd"R" is also called a
{\em sub-region}.\index{region!sub-region}

Let \xcdmath"R$_1$" and \xcdmath"R$_2$" be two regions whose type
establishes that they are of the same rank. Let 
\xcdmath"S" be a region of unrelated rank.

\xcdmath"R$_1$ && R$_2$" is the intersection of \xcdmath"R$_1$" and
\xcdmath"R$_2$". 

\index{region!intersection}

\xcdmath"R$_1$ || R$_2$" is the union of the \xcdmath"R$_1$" and
\xcdmath"R$_2$".\index{region!union}

\xcdmath"R$_1$ - R$_2$" is the set difference of \xcdmath"R$_1$" and
\xcdmath"R$_2$".\index{region!set difference}

\xcdmath"R$_1$ * S" is the Cartesian product of \xcdmath"R$_1$" and
\xcdmath"S",  formed by pairing each point in \xcdmath"R$_1$" with every the point in \xcdmath"S".
\index{region!product}
Thus, \xcd"([1..2,3..4] as Region 2) * (5..6)"
is the region of rank \Xcd{3} containing the points \Xcd{(x,y,z)}
where \Xcd{x} is \Xcd{1} or \Xcd{2}, 
\Xcd{y} is \Xcd{3} or \Xcd{4}, and
\Xcd{z} is \Xcd{5} or \Xcd{6}. 


For a region \xcdmath"R" and point \xcdmath"p" of the same rank 
\xcdmath"R+p" and \xcdmath"R-p" represent the translation of the region
with \xcdmath"p". That is, point \xcdmath"q" is in 
\xcdmath"R" if and only if point \xcdmath"q+p" is in \xcdmath"R+p". (And similarly
for \xcdmath"R-p".)

%%TODO: Determine how equality is actually implemented. This should not be the definition of ==. 
%%  This could be the definition of .equals(..).

%% Two regions are equal (\xcd"==") if they represent the same set of
%% points.\index{region!==}

For more details on the available methods on \xcdmath"Region", please
consult the API documentation.

\section{Distributions}\label{XtenDistributions}
\index{distribution}

A {\em distribution} is a mapping from a region to a set of places.
{}\Xten{} provides a built-in class, \xcd"x10.lang.Dist", to allow the creation of new distributions and
to perform operations on distributions. This class is \xcd"final" in
{}\XtenCurrVer; future versions of the language may permit
user-definable distributions. Since distributions play a dual role
(values as well as types), variables of type \xcd"Dist" must
be initialized and are implicitly \xcd"val".
\bard{Is this true?}

The {\em rank} of a distribution is the rank of the underlying region.

%Recall that each program runs in a fixed number of places, determined
%by runtime parameters. The static constant Place.MAX_PLACES specifies
%the maximum number of places. The collection of places is assumed to
%be totally ordered.


\begin{xten}
R: Region = 1..100;
D: Dist = Dist.makeBlock(R);
D: Dist = Dist.makeCyclic(R);
D: Dist = R -> here;
D: Dist = Dist.random(R);
\end{xten}

Let \xcd"D" be a distribution. \xcd"D.region" denotes the underlying
region. \xcd"D.places" is the set of places constituting the range of
\xcd"D" (viewed as a function). Given a point \xcd"p", the expression
\xcd"D(p)" represents the application of \xcd"D" to \xcd"p", that is,
the place that \xcd"p" is mapped to by \xcd"D". The evaluation of the
expression \xcd"D(p)" throws an \xcd"ArrayIndexOutofBoundsException"
if \xcd"p" does not lie in the underlying region.

When operated on as a distribution, a region \xcd"R" implicitly
behaves as the distribution mapping each item in \xcd"R" to \xcd"here"
(i.e., \xcd"R->here", see below). Conversely, when used in a context
expecting a region, a distribution \xcd"D" should be thought of as
standing for \xcd"D.region".

{}\oldtodo{Allan: We do not specify how the values of an array at a place
are stored, e.g. in row-major or column major order. Need to work this
out.}

\subsection{Operations returning distributions}

Let \xcd"R" be a region, \xcd"Q" a set of places \{\xcd"p1", \dots, \xcd"pk"\}
(enumerated in canonical order), and \xcd"P" a place. All the operations
described below may be performed on \xcd"Dist.factory".

\paragraph{Unique distribution} \index{distribution!unique}
The distribution \xcd"unique(Q)" is the unique distribution from the
region \xcd"1..k" to \xcd"Q" mapping each point \xcd"i" to \xcd"pi".

\paragraph{Constant distributions.} \index{distribution!constant}
The distribution \xcd"R->P" maps every point in \xcd"R" to \xcd"P".

\paragraph{Block distributions.}\index{distribution!block}
The distribution \xcd"block(R, Q)" distributes the elements of \xcd"R"
(in order) over the set of places \xcd"Q" in blocks  as
follows. Let $p$ equal \xcd"|R| div N" and $q$ equal \xcd"|R| mod N",
where \xcd"N" is the size of \xcd"Q", and 
\xcd"|R|" is the size of \xcd"R".  The first $q$ places get
successive blocks of size $(p+1)$ and the remaining places get blocks of
size $p$.

The distribution \xcd"block(R)" is the same distribution as {\cf
block(R, Place.places)}.

\oldtodo{Check into block distributions per dimension.}
\paragraph{Cyclic distributions.} \index{distribution!cyclic}
The distribution \xcd"cyclic(R, Q)" distributes the points in \xcd"R"
cyclically across places in \xcd"Q" in order.

The distribution \xcd"cyclic(R)" is the same distribution as \xcd"cyclic(R, Place.places)".

Thus the distribution \xcd"cyclic(Place.MAX_PLACES)" provides a 1--1
mapping from the region \xcd"Place.MAX_PLACES" to the set of all
places and is the same as the distribution \xcd"unique(Place.places)".

\paragraph{Block cyclic distributions.}\index{distribution!block cyclic}
The distribution \xcd"blockCyclic(R, N, Q)" distributes the elements
of \xcd"R" cyclically over the set of places \xcd"Q" in blocks of size
\xcd"N".

\paragraph{Arbitrary distributions.} \index{distribution!arbitrary}
The distribution \xcd"arbitrary(R,Q)" arbitrarily allocates points in {\cf
R} to \xcd"Q". As above, \xcd"arbitrary(R)" is the same distribution as
\xcd"arbitrary(R, Place.places)".

\oldtodo{Determine which other built-in distributions to provide.}

\paragraph{Domain Restriction.} \index{distribution!restriction!region}

If \xcd"D" is a distribution and \xcd"R" is a sub-region of {\cf
D.region}, then \xcd"D | R" represents the restriction of \xcd"D" to
\xcd"R".  The compiler throws an error if it cannot determine that
\xcd"R" is a sub-region of \xcd"D.region".

\paragraph{Range Restriction.}\index{distribution!restriction!range}

If \xcd"D" is a distribution and \xcd"P" a place expression, the term
\xcd"D | P" denotes the sub-distribution of \xcd"D" defined over all the
points in the region of \xcd"D" mapped to \xcd"P".

Note that \xcd"D | here" does not necessarily contain adjacent points
in \xcd"D.region". For instance, if \xcd"D" is a cyclic distribution,
\xcd"D | here" will typically contain points that are \xcd"P" apart,
where \xcd"P" is the number of places. An implementation may find a
way to still represent them in contiguous memory, e.g., using a
complex arithmetic function to map from the region index to an index
into the array.

\subsection{User-defined distributions}\index{distribution!user-defined}

Future versions of \Xten{} may provide user-defined distributions, in
a way that supports static reasoning.

\subsection{Operations on distributions}

A {\em sub-distribution}\index{sub-distribution} of \xcd"D" is
any distribution \xcd"E" defined on some subset of the region of
\xcd"D", which agrees with \xcd"D" on all points in its region.
We also say that \xcd"D" is a {\em super-distribution} of
\xcd"E". A distribution \xcdmath"D$_1$" {\em is larger than}
\xcdmath"D$_2$" if \xcdmath"D$_1$" is a super-distribution of
\xcdmath"D$_2$".

Let \xcdmath"D$_1$" and \xcdmath"D$_2$" be two distributions.  


\paragraph{Intersection of distributions.}\index{distribution!intersection}
\xcdmath"D$_1$ && D$_2$", the intersection of \xcdmath"D$_1$"
and \xcdmath"D$_2$", is the largest common sub-distribution of
\xcdmath"D$_1$" and \xcdmath"D$_2$".

\paragraph{Asymmetric union of distributions.}\index{distribution!union!asymmetric}
\xcdmath"D$_1$.overlay(D$_2$)", the asymmetric union of
\xcdmath"D$_1$" and \xcdmath"D$_2$", is the distribution whose
region is the union of the regions of \xcdmath"D$_1$" and
\xcdmath"D$_2$", and whose value at each point \xcd"p" in its
region is \xcdmath"D$_2$(p)" if \xcdmath"p" lies in
\xcdmath"D$_2$.region" otherwise it is \xcdmath"D$_1$(p)".
(\xcdmath"D$_1$" provides the defaults.)

\paragraph{Disjoint union of distributions.}\index{distribution!union!disjoint}
\xcdmath"D$_1$ || D$_2$", the disjoint union of \xcdmath"D$_1$"
and \xcdmath"D$_2$", is defined only if the regions of
\xcdmath"D$_1$" and \xcdmath"D$_2$" are disjoint. Its value is
\xcdmath"D$_1$.overlay(D$_2$)" (or equivalently
\xcdmath"D$_2$.overlay(D$_1$)".  (It is the least
super-distribution of \xcdmath"D$_1$" and \xcdmath"D$_2$".)

\paragraph{Difference of distributions.}\index{distribution!difference}
\xcdmath"D$_1$ - D$_2$" is the largest sub-distribution of
\xcdmath"D$_1$" whose region is disjoint from that of
\xcdmath"D$_2$".


\subsection{Example}
\begin{xten}
def dotProduct(a: Array[T](D), b: Array[T](D)): Array[Double](D) =
  (new Array[T]([1:D.places],
      (Point) => (new Array[T](D | here,
                    (i): Point) => a(i)*b(i)).sum())).sum();
\end{xten}

This code returns the inner product of two \xcd"T" vectors defined
over the same (otherwise unknown) distribution. The result is the sum
reduction of an array of \xcd"T" with one element at each place in the
range of \xcd"D". The value of this array at each point is the sum
reduction of the array formed by multiplying the corresponding
elements of \xcd"a" and \xcd"b" in the local sub-array at the current
place.




\section{Array initializer}\label{ArrayInitializer}\label{array!creation}

Arrays are instantiated by invoking the constructor for the
classes
\xcd"Array" or \xcd"ValArray".

An array instantiation may be annotated
\xcd"unsafe"
if it is intended to be
allocated in an unmanaged region (e.g., for communication with native
code). A value array is an immutable array. An array creation
must take either an \xcd"Int" as an argument or a \xcd"Dist". In the first
case an array is created over the distribution \xcd"[0:N-1]->here";
in the second over the given distribution. 

An array creation operation may also specify an initializer
function.
The function is applied in parallel
at all points in the domain of the distribution. The array
construction operation terminates locally only when the array has been
fully created and initialized (at all places in the range of the
distribution).

For instance:
\begin{xten}
val data : Array[Int]
    = new Array[Int](1000->here, Point(i) => i);
val data2 : Array[Int]
    = new ValArray[Int]([1:1000,1:1000]->here, Point(i,j) => i*j);
\end{xten}

{}\noindent 
The first declaration stores in \xcd"data" a reference to a mutable
array with \xcd"1000" elements each of which is located in the
same place as the array. Each array component is initialized to \xcd"i".

The second declaration stores in \xcd"data2" an (immutable)
2-d array over \xcd"[1:1000, 1:1000]" initialized with \xcd"i*j"
at point \xcd"[i,j]". It uses a more abbreviated form to specify 
the array initializer function.

Other examples:
\begin{xten}
val data : Array[Int]
    = new Array[Int](1000) ((i): Point => i*i);
val d : Array[Float](D)
   = new Array[Float](D) ((i): point) => 10.0*i);
val result : Array[Float](D)
   = new Array[Float](D) ((i,j): point) => i+j);
\end{xten}

\section{Operations on arrays}
In the following let \xcd"a" be an array with distribution \xcd"D" and
base type \xcd"T". \xcd"a" may be mutable or immutable, unless
indicated otherwise.

\subsection{Element operations}\index{array!access}
The value of \xcd"a" at a point \xcd"p" in its region of definition is
obtained by using the indexing operation \xcd"a(p)". This operation
may be used on the left hand side of an assignment operation to update
the value. The operator assignments \xcd"a(i) op= e" are also available
in \Xten{}.

For array variables, the right-hand-side of an assignment must
have the same distribution \xcd"D" as an array being assigned. This
assignment involves
control communication between the sites hosting \xcd"D". Each
site performs the assignment(s) of array components locally. The
assignment terminates when assignment has terminated at all
sites hosting \xcd"D".

\subsection{Constant promotion}\label{ConstantArray}\index{arrays!constant promotion}

For a distribution \xcd"D" and a constant or final variable \xcd"v" of
type \xcd"T" the expression \xcd"new Array[T](D, (p: Point) => v)"
denotes the mutable array with
distribution \xcd"D" and base type \xcd"T" initialized with \xcd"v"
at every point.

\subsection{Restriction of an array}\index{array!restriction}

Let \xcd"D1" be a sub-distribution of \xcd"D". Then \xcd"a | D1"
represents the sub-array of \xcd"a" with the distribution \xcd"D1".

Recall that a rich set of operators are available on distributions
(\Sref{XtenDistributions}) to obtain sub-distributions
(e.g. restricting to a sub-region, to a specific place etc).

\subsection{Assembling an array}
Let \xcd"a1,a2" be arrays of the same base type \xcd"T" defined over
distributions \xcd"D1" and \xcd"D2" respectively. Assume that both
arrays are value or reference arrays. 
\paragraph{Assembling arrays over disjoint regions}\index{array!union!disjoint}

If \xcd"D1" and \xcd"D2" are disjoint then the expression \xcd"a1 || a2" denotes the unique array of base type \xcd"T" defined over the
distribution \xcd"D1 || D2" such that its value at point \xcd"p" is
\xcd"a1(p)" if \xcd"p" lies in \xcd"D1" and \xcd"a2(p)"
otherwise. This array is a reference (value) array if \xcd"a1" is.

\paragraph{Overlaying an array on another}\index{array!union!asymmetric}
The expression
\xcd"a1.overlay(a2)" (read: the array \xcd"a1" {\em overlaid with} \xcd"a2")
represents an array whose underlying region is the union of that of
\xcd"a1" and \xcd"a2" and whose distribution maps each point \xcd"p"
in this region to \xcd"D2(p)" if that is defined and to \xcd"D1(p)"
otherwise. The value \xcd"a1.overlay(a2)(p)" is \xcd"a2(p)" if it is defined and \xcd"a1(p)" otherwise.

This array is a reference (value) array if \xcd"a1" is.

The expression \xcd"a1.update(a2)" updates the array \xcd"a1" in place
with the result of \xcd"a1.overlay(a2)".

\oldtodo{Define Flooding of arrays}

\oldtodo{Wrapping an array}

\oldtodo{Extending an array in a given direction.}

\subsection{Global operations }

\paragraph{Pointwise operations}\label{ArrayPointwise}\index{array!pointwise operations}
The unary \xcd"lift" operation applies a function to each element of
an array, returning a new array with the same distribution.
The \xcd"lift" operation is implemented by the following method
in \xcd"Array[T]":
\begin{xten}
def lift[S](f: (T) => S): Array[S](dist);
\end{xten}

The binary \xcd"lift" operation takes a binary function and
another
array over the same distribution and applies the function
pointwise to corresponding elements of the two arrays, returning
a new array with the same distribution.
The \xcd"lift" operation is implemented by the following method
in \xcd"Array[T]":
\begin{xten}
def lift[S,R](f: (T,S) => R, Array[S](dist)): Array[R](dist);
\end{xten}

\paragraph{Reductions}\label{ArrayReductions}\index{array!reductions}

Let \xcd"f" be a function of type \xcd"(T,T)=>T".  Let
\xcd"a" be a value or reference array over base type \xcd"T".
Let \xcd"unit" be a value of type \xcd"T".
Then the
operation \xcd"a.reduce(f, unit)" returns a value of type \xcd"T" obtained
by performing \xcd"f" on all points in \xcd"a" in some order, and in
parallel.  The function \xcd"f" must be associative and
commutative.  The value \xcd"unit" should satisfy
\xcd"f(unit,x)" \xcd"==" \xcd"x" \xcd"==" \xcd"f(x,unit)".

This operation involves communication between the places over which
the array is distributed. The \Xten{} implementation guarantees that
only one value of type \xcd"T" is communicated from a place as part of
this reduction process.

\paragraph{Scans}\label{ArrayScans}\index{array!scans}

Let \xcd"f" be a reduction operator defined on type \xcd"T". Let
\xcd"a" be a value or reference array over base type \xcd"T" and
distribution \xcd"D". Then the operation \xcd"a||f()" returns an array
of base type \xcd"T" and distribution \xcd"D" whose $i$th element
(in canonical order) is obtained by performing the reduction \xcd"f"
on the first $i$ elements of \xcd"a" (in canonical order).

This operation involves communication between the places over which
the array is distributed. The \Xten{} implementation will endeavour to
minimize the communication between places to implement this operation.

Other operations on arrays may be found in \xcd"x10.lang.Array" and
other related classes.
