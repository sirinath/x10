\section{Statements}
First, we introduce  $k$-dimensional versions of iteration operations {\tt for} and {\tt forall}:

\begin{code}
   forall (  ind1, ..., indk : A) {S}
\end{code}

Assume that {\tt A} is an array and {\tt [T1,..., Tk]} is the convex closure of its underlying region. 
Then the execution of this statement results in the parallel execution of an activity

\begin{code}
   { final T1 Ind1 = i1; 
     ...
     final Tk Indk = ik;
     S
   }
\end{code}
\noindent for each value {\tt [i1,\ldots, ik]} in the region underlying {\tt A}.

In a similar fashion we introduce the syntax:
\begin{code}
   ateach( Ind1, ..., Indk : A) {S}
\end{code}
\noindent to stand for
\begin{code}
   forall (  ind1, ..., indk : A ) async (A[ind1,...,indk]) {S}
\end{code}

In method definitions, for a region {\tt R} we allow the syntax 
\begin{code}
     R i1,...,ik; 
\end{code} 

\noindent to introduce {\tt k} new parameters which range over the component enumerations of the convex closure of
{\tt R}.
