Modern object-oriented languages such as \Xten require a rich framework
for types capable of expressing value-dependency, type-dependency
and supporting pluggable, application-specific extensions.

In earlier work, we presented the framework of \emph{constrained
types} for concurrent, object-oriented languages, parametrized by
an underlying constraint system $\cal X$. Constraint systems are a
very expressive framework for partial information. Types are viewed
as formulas \Xcd{C\{c\}} where \Xcd{C} is the name of a class
or an interface and \Xcd{c} is a constraint in $\cal X$ on the
immutable instance state of \Xcd{C} (the \emph{properties}).
Many (value-)dependent type systems for object-oriented languages
can be viewed as constrained types.

This paper extends the constrained types approach to handle
\emph{type-dependency} (``genericity''). The key idea is to extend
the constraint system to introduce \emph{constrained kinds}: in
the same way that constraints on values can be used to define
constrained types, constraints on types can
define constrained kinds.
Generic types are supported
by introducing type variables and permitting programs to impose
constraints on such variables.

To illustrate the underlying theory, we develop a formal family of
programming languages with a common set of sound type-checking rules
parametrized on a constraint system.  By varying the constraint system
and by extending the typing rules in a simple way, we obtain
languages with the power
of \FJ, \FGJ, and languages that provide dependent types, structural
subtyping, and constraints that relate values and types.  The core
of the \Xten language is a concrete instantiation of the framework.  We
describe the design of \Xten, which is available
for download at \texttt{x10-lang.org}.

