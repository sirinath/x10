\documentclass{article}
%\usepackage{proof}
\usepackage{fullpage}
\usepackage{amssymb}
\bibliographystyle{abbrv}

%\newcommand \limp {\hbox{ $-\!\!\,\circ $ } }
\newcommand \inp[2] {\hbox{in}~#1~#2}
\newcommand \outp[2] {\hbox{out}~#1~#2}
\newcommand \inpL {\hbox{in}\Lscr}
\newcommand \inpR {\hbox{in}\Rscr}
\newcommand \limp {\multimap}
\newcommand \limpR {\limp\Rscr}
\newcommand \limpL {\limp\Lscr}
\newcommand \nuR {\nu\Rscr}
\newcommand \nuL {\nu\Lscr}
\newcommand \nme {\hbox{\sl name}~}
\newcommand \ws {\hbox{\sl ws}}
\newcommand \sst {\hbox{\sl ss}}
\newcommand \rss {\hbox{\sl rs}}
\newcommand \IMLL {\hbox{\sl IMLL}}
\newcommand \IMLLnd {\hbox{\sl IMLL}_\nu^{\Delta}}
\newcommand \vvec[1]{\stackrel{\rightarrow}{#1}}
\newcommand \en[1]{[\![#1]\!]}
\newcommand \ens[1]{\lceil #1 \rceil}
\newcommand \enf[1]{\langle #1 \rangle}
\newcommand \act[3]{\hbox{\sl act}~#1~#2~#3}
\newcommand \fvar[1] {\hbox{fn}(#1)}
\newcommand{\alt}{{\;{\tt\char`\|}\;}}
\newcommand{\Hat}{{\;{\tt\char`\^}\;}}
\newcommand{\now}{\mbox{\bf\tt now}}
\newcommand{\every}{\mbox{\bf\tt every}}
\newcommand{\then}{\rightarrow}
\newcommand{\Else}{\Rightarrow}
\newcommand{\next}{\mbox{\bf\tt next}}
\newcommand{\sync}{\mbox{\bf\tt sync}}

\def\nforall{\mu}
\def\nexists{\nu}
%\def\tr{\triangleright}
%\def\tl{\triangleleft}
\def\tr{\triangleright}
\def\tl{\triangleleft}
\def\emp{\emptyset}
\def\derives{\longrightarrow}
\def\equiderives{\longleftrightarrow}
\def\from#1\infer#2{{{\textstyle #1}\over{\textstyle #2}}}
\def\ccfont{\sf}
\def\cc{{\ccfont cc}}
\def\tcc{{\ccfont tcc}}
\def\x10{{\ccfont x10}}
\def\X10{{\ccfont X10}}
\def\fx10{{\ccfont fx10}}
\def\java{{\sc Java}}
\def\kava{{\sc Kava}}
\newcommand \todo[1] {\begin{quotation}{\bf Todo:} {\footnotesize\em #1}\end{quotation}}
\newcommand \metanote[1] {\begin{quotation}{\bf MetaNote:} {\footnotesize\em #1}\end{quotation}}
\begin{document}
\title{\X10{} Design Notes: Values\\
}
\author{{\sc PERCS} Area II\footnote{Contact: Vijay Saraswat, {\tt vsaraswa@us.ibm.com} This note is intended primarily for people working on or evaluating the \x10{} design.}\\ 
{\sc ****DRAFT*** v0.02}\\ 
(Please do not circulate)
}

\maketitle

This note contains a proposal for the introduction of values into
\x10, based on the {\sc Kava} design, and on the need to have
high-performance user-defined types for numerical computation, e.g.{}
a variety of {\tt Complex} numbers.

Below, by a {\em scalar class} we mean a class defined with the usual
{\tt class} definition. Such a definition explicitly specifies a
sequence of fields for the class by naming them.  By an {\em array
class} we mean classes defined implicitly through array declarations
(e.g. {\tt Complex[]}).\footnote{Note that \x10{} arrays are much more
sophisticated than \java{} arrays.} Such a definition implicitly
specifies the sequence of fields for the class, one for each element
in the domain of the array, together with some {\tt final} fields
representing the domain of the array.

\section{Values} 
This section is normative.

\X10{} singles out a certain set of classes for additional support. A
(scalar or array) class is said to be {\em stateless} if all of its
fields are declared to be {\tt final}, otherwise it is {\em
stateful}. (\X10{} has syntax for specifying an array class with {\tt
final} fields, unlike \java.) A {\em stateless (stateful) object} is
an instance of a stateless (stateful) class.

A final variable satisfies two conditions: (1)~it can be assigned to
at most once, (2)~it must be assigned to before use. \X10{} follows
\java{} language rules in this respect \cite[Sections
4.5.4,8.3.1.2,Chapter 16]{jls}. Briefly, the compiler must undertake a
specific analysis to statically guarantee the two properties above.\footnote{Todo: Check if this analysis needs to be revisited.}

\X10{} allows the programmer to signify that a class (and all its
descendents) are stateless. Such a class is called a {\em value
class}.  The programmer specifies a value class by prefixing the
modifier {\tt value} before the keyword {\tt class} in a class
declaration.  (A class not declared to be a value class will be called
a {\em reference class}.)  Each instance field of a value class is
treated as {\tt final}. It is legal (but neither required nor
recommended) for fields in a value class to be declared {\tt final}.
It is a compile-time error for a value class to inherit from a
stateful class or for a reference class to inherit from a value class.
For brevity, the \x10{} compiler allows the programmer to omit the
keyword {\tt class} after {\tt value} in a value class declaration.

The infix operator {\tt ==} tests {\em stable equality}. Its behavior
cannot be overridden by user code.  In more detail, it checks to see
if its two arguments are equal and will stay equal regardless of any
operation that can be performed by a program. For reference types,
this translates to checking if the two arguments are a reference to
the same object. For value types, this translates to checking that
both objects are of the same type, and their corresponding fields are
{\tt ==}. For aggregate types, this translates to checking that the
domain of the two objects is the same, and their values at each point
in the domain is {\tt ==}.

The method {\tt .equals(Object o)} is defined on {\tt x10.lang.Object}
to behave like {\tt ==}, but may be overridden on subclasses.

%% Aggregate and scalar?

%% Have to determine behavior of ==, equals, and hashcode.

The {\tt nullary} type constructor
(Section~\ref{NullaryTypeConstructor}) can be used to declare
variables whose value may be null or a variable type.

Each type has a nullary constructor. If the programmer does not supply
one, one is provided by default. This default constructor
automatically initializes each field (if any) as follows. If the field
is of {\tt nullable} type, it is initialized to {\tt null}. Otherwise,
if it is of scalar type it is initialized with the value returned by
invoking the nullary constructor of that type.  If the field is an
aggregate type with a dynamically determined region, the region is
chosen to be the smallest region of the given type. If the region is
non-empty then each variable in that region is intialized with the
value returned by calling the nullary constructor of the declared base
type of the aggregate.

Variables declared at a value or reference type can never be observed
in an uninitialized state outside the constructor of that type.

A value class may implement interfaces, just like a reference
class. Indeed the same interface may be implemented by a value class
and a reference class. 

% The modifier {\tt value} may be used in the
% type-specified of a method argument to indicate that the argument must
% be an instance of a value type implementing that interface.
% How would that help the programmer or the implementation?

\paragraph{Representation.}
Since value objects do not contain any updatable locations, they can
be freely copied from place to place. An implementation may use
copying techniques even within a place to implement value types,
rather than references. This is transparent to the programmer.

More explicitly, \X10{} guarantees that an implementation must always
behave as if a variable of a reference type takes up as much space as
needed to store a reference that is either null or is bound to an
object allocated on the (appropriate) heap. However, \X10{} makes no
such guarantees about the representation of a variable of value
type. The implementation is free to behave as if the value is stored
``inline'', allocated on the heap (and a reference stored in the
variable) or use any other scheme (such as structure-sharing) it may
deem appropriate. Indeed, an implementation may even dynamically
change the representation of an object of a value type, or dynamically
use different representations for different instances (that is,
implement automatic box/unboxing of values).

Implementations are strongly encouraged to implement value types as
space-efficiently as possible (e.g.{} inlining them or passing them in
registers, as appropriate).  Implementations are expected to cache
values of remote final value variables by default. If a value is
large, the programmer may wish to consider spawning a remote activity
(at the place the value was created) rather than referencing the
containing variable (thus forcing it to be cached).

\todo{vj: Need to figure out whether we should let the programmer be
aware of lazy pull vs full-value push of value objects. This is the
idea of introducing a $\star$-annotation. Need to make a decision on
this.

Could leave this for resolution in v0.7.
}

\paragraph{The value class {\tt x10.lang.Value}}

{\tt x10.lang.Value} is a value class extending {\tt x10.lang.Object},
an immutable reference class.  All value classes directly or
indirectly extend {\tt x10.lang.Value}.

In Version 0.5 of the language no methods are defined on {\tt
x10.lang.Value}. The class has no fields.

\todo{vj: Need to consider syntax for allowing literals to be
 introduced for value-defined classes.}

\paragraph{Built-in types.}
\X10{} provides built in implementations for the {\tt final} value classes:
\begin{enumerate}
\item {\tt bit}: 1-bit value (either {\tt 0} or {\tt 1})
\item {\tt nibble}: 4-bit unsigned values, ({\tt 0}, {\tt 1}, {\tt 2}, {\tt 3})
\item {\tt byte}: 8-bit unsigned values
\item {\tt short}: 16-bit integers
\item {\tt int}: 32-bit integers
\item {\tt long}: 64-bit integers
\item {\tt float}: 32-bit (IEEE???) floating point numbers
\item {\tt double}: 64-bit (IEEE???) floating point numbers
\item {\tt doubledouble}: 128-bit (IEEE???) floating point numbers
\item {\tt boolean}: with two values {\tt false} and {\tt true}
\end{enumerate}

\metanote{Perhaps all the int types should be thought of as enums.}

\todo{TBD if there is an interface of the same name.}

\todo{Possibly define a {\tt toString} method for value types whose
default implementation produces some parsable serialized value.

Maybe there should be two forms -- a toXML, and a toString?
}


\subsection{Idioms of usage}
This section is descriptive.

\subsection{Design rationale}
Logically, one could allow a stateless reference class to inherit from
a value class. But this will cause more confusion than its worth,
because we are thinking of a value class as an explicit assertion by
the programmer that all its descendants will also be value classes. A
stateless reference class is a reference class that happens to be
stateless (e.g. because the programmer just deleted a stateful field
while debugging the program); a value class is {\em intentionally}
stateless.

\metanote{Describe why value classes are allowed to be as general as
reference classes. X10 only has pass by value, like \java; the value
of a reference variable is a reference to the object. In contrast,
\csharp{} has pass by value and pass by reference; a programmer must
make this distinction when writing code for the class and must know
whether a type is a value type or a reference type when using a value
of that type.  
}

\paragraph{The null value.}
\csharp{} permits {\tt null} as a legal value for every reference
type, like \java{}. \x10{} uses a different, orthogonal design
(Section~\ref{NullTypeConstructor}) which permits value types to have
the value {\tt null}.

In an alternative design, value classes may optionally be declared to
permit a null value, e.g.{} through a {\tt nullable}
qualifier. Reference classes are always implicitly {\tt
nullable}. Variables declared at a nullable class may optionally be
qualified as nullable. Only such variables may be assigned a nullable
value. 

\subsection{\java{} compatibility issues.}
Note that \X10{} objects do not have any associated lock and the
associated \java{} methods ({\tt wait}, {\tt notify}, {\tt
synchronize}) are not avaiable in \x10. 

Define a {\em concrete} value class to be a final value class which
implements no interfaces which have members and none of whose fields
are dynamically sized arrays.

\todo{vj: Need to check if anything special needs to be done for inner
or anonymous classes.  Is there any need to distinguish an anoymous
value class from an anonymous stateless reference class?  
}
\todo{vj: Determine if we need finalizers for reference
objects. finalizers should never be called for value objects, since
they can be copied freely.}

\todo{vj: Determine whether objects will have a hashcode field, and
how the default behavior of {\tt ==} is to be specified.}

\todo{vj: Specify infix, prefix and postfix operations, and compound assignment operations on (value {\em and} reference) classes.}

\subsection{\csharp{} compatibility}

Unlike \csharp{}, \x10{} value classes are immutable, may be abstract,
and may be subclassable. {\tt final} \x10{} value classes correspond
to \csharp{} structs all of whose fields are {\tt readOnly} ({\tt
final}).


\subsection{Implementation notes}
Boxing/unboxing for upcalls. 

The central characteristic of value objects is that they are 
immutable. Once an object is created the pattern of bits representing
the object is completely known and will never change for the lifetime
of the object.\footnote{In contrast, the pattern of bits representing
an object with a non-final field could change with every assignment to
that field.} Therefore an assignment at a value type may always be
correctly implemented by simply copying the entire bit representation
of the source object into the target location (in exactly the same
fashion as, say, the assignment of an integer to a register in {\sc
Fortran} or {\sc C}).\footnote{Alternatively, the compiler could use
advanced techniques (such as structure-sharing, developed in {\sc
Prolog} and functional language implementations) for a compact
representation for a set of such values.}  Note that this is a correct
implementation technique even if a field is typed at a reference class
or an array value class with dynamic size (in which case the field is
represented by a reference). The technique is incorrect for a
reference type, which must use (and copy) a reference to correctly
represent sharing of updatable locations.

To facilitate such efficient direct-copying implementations, it is
often necessary for the compiler to determine whether a value type is
{\em explicit}, i.e.{} whether the size of memory needed to represent
the entire state of its instance is statically known.  For instance,
enumeration types are explicit. Recursive classes
\footnote{Informally, a class is recursive if it has a self-typed field. More formally, consider a directed graph created from the set of all class declarations, with each node labeled with a class and with an edge from $A$ to $B$ if
the class $A$ has a field of type $B$. Now a class is recursive if the
graph has a loop involving the node labeled with the class.}  are not
explicit. A class with an array field whose size is not known
statically is not explicit.\footnote{\x10{} allows the declaration of
aggregates with known size, e.g. {\tt int[32] a;} } An example of an
explicit class is {\tt Complex} declared with two {\tt double} fields.

In general, it is not hard to see that a value class is explicit iff
the type of any field with a value type is explicit, and the size of
any array field with a value type is statically known. (The size of
a field with a reference type is considered to be the size of the
reference.)

Instances of an explicit type may be {\em inlined}, often resulting in
considerable efficiency. Note however that when such an instance is
assigned to a variable of a super type the instance must be boxed.

Not all the conditions for explicitness need to be met for the class
to be efficiently implementable. For instance it may be possible to
efficiently accomodate arrays of dynamic size via indirection through
a size word.

%% Hmm..why not represent the size in there explicitly, as we did in the presence and instant messaging protocol?

Further, the instance needs a pointer to the virtual function table
for the class only if the class is not final or it implements an
interface which defines at least one method.

\todo{vj: Allow graphs of values to be created.}


\end{document}

\todo{vj: Results of referencing remote clocked final values are cached locally. The cache is flushed when the clock ticks.}

A final variable (also called a {\em constant}) satisfies two conditions:
(1)~it can be assigned to at most once, (2)~it must be assigned to
before use.

Every compiler must attempt to check these two conditions. The
compiler will accept the program if it can be shown to satisfy these
conditions, reject if it if it can be shown to not satisfy these
conditions, and issue a warning if it is not sure. In the last case,
it must generate code to ensure that an UseBeforeInitializationError
or SecondAssignmentToFinalError error is thrown at runtime (as
appropriate).

Design alternative is to specify a particular analysis that must be
run by the compiler.

Any class all of whose fields are final is said to be a {\em value} class.

We must introduce the concept of a constant array as well, an array of
final locations.

final final int[1..3] a = {41,42,43}

Formally one reads this as: a is a final location that is bound to an
array of final ints with domain 1..3.

Informally, this is read as saying a is a final variable (so its
binding does not change). And it is going to be bound to an unchanging
map from the domain 1..3 to int. Any attempt to assign to one of the
array locations, e.g.  a[1]=45 is a compile-time error.

OK good.. so here is a thought...we introduce generalized arrays. A
generalized array is a map from a region to a value domain. A region
can be a cross product. And there are other operations on a region:
extend a region with a vector. (ZPL operations). 


One can iterate over a region.

Regions can be mapped to places.


%% Need to support operator overloading. See Kava. 






