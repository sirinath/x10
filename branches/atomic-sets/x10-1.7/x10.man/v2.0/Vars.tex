
\chapter{Variables}\label{XtenVariables}\index{variables}

A variable is a storage location. All
variables are initialized with a value and cannot be observed
without
a value.

Variables whose value may not be changed after initialization
are called {\em final variables} (or sometimes {\em constants}).
The programmer indicates that a variable is final by declaring
it with the \xcd"val" keyword rather than the \xcd"var" keyword.
Variables that are annotated neither \xcd"val" nor \xcd"var"
are considered final.

A variable of a reference data type \xcd"T" where \xcd"T" is the name
of a reference class (possibly with type arguments) always holds a
reference to an instance of the class \xcd"T" or a class that is a
subclass of \xcd"T", or a \xcd"null" reference.

A variable of a reference rail type \xcd"Rail[T]" has
as many variables as the size of the rail.

%XXX
%A variable of a nullable (reference or value) data type \xcd"nullable[T]"
%always holds either the value (named by) \xcd"null" or a value of
%type \xcd"T" (these cases are not mutually exclusive).

A variable of an interface type \xcd"I" always holds either a
reference to a reference class implementing \xcd"I" (including possibly
a boxed value class that implements \xcd"I"), or a \xcd"null"
reference.

A variable of a value data type \xcd"T" where \xcd"T" is the name of a
value class always holds a reference to an instance of \xcd"T" (or a
class that is a subclass of \xcd"T") or to an instance of \xcd"T" (or a
class that is a subclass of \xcd"T"). No program can distinguish
between the two cases.

A variable of an function type always holds an instance
of a value class with implementing the function type.

A variable of a reference rail type \xcd"ValRail[T]" has
as many variables as the size of the rail.
Each of these variables is immutable and has the type \xcd"T".

\Xten{} supports seven kinds of variables: constant {\em class
variables} (static variables), {\em instance variables} (the instance
fields of a class), {\em array components}, {\em method parameters},
{\em constructor parameters}, {\em exception-handler parameters} and
{\em local variables}.

\section{Final variables}\label{FinalVariable}\index{variable!final}\index{final variable}
A final variable satisfies two conditions: 
\begin{itemize}
\item it can be assigned to at most once, 
\item it must be assigned to before use. 
\end{itemize}

\Xten{} follows \java{} language rules in this respect \cite[\S
4.5.4,8.3.1.2,16]{jls2}. Briefly, the compiler must undertake a
specific analysis to statically guarantee the two properties above.

Final local variables and fields are defined by the \xcd"val"
keyword.  Elements of value arrays are also final.

\oldtodo{Check if this analysis needs to be revisited.}

\section{Initial values of variables}
\label{NullaryConstructor}\index{nullary constructor}

Every variable declared at a type must always contain a value of that type.

Every class variable must be initialized before it is read, through
the execution of an explicit initializer or a static block. Every
instance variable must be initialized before it is read, through the
execution of an explicit initializer or a constructor.
Non-final instance variables of reference type are 
assumed to have an initializer that sets the value to \xcd"null".
Non-final instance variables of value type are 
assumed to have an initializer that sets the value to the
result of invoking the nullary constructor on the class. 
An initializer is required if the default initial value is not
assignable to the variable's type.

Each method and constructor parameter is initialized to the
corresponding argument value provided by the invoker of the method. An
exception-handling parameter is initialized to the object thrown by
the exception. A local variable must be explicitly given a value by
initialization or assignment, in a way that the compiler can verify
using the rules for definite assignment \cite[\S~16]{jls2}.

