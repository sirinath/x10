\section{Clocks}

\X10{}  has a built-in final reference class, {\tt x10.lang.clock} intended
for repeated quiescence detection of arbitrary, data-dependent
collection of activities.

Clocks may only be stored in {\em flow} variables, i.e.{} local
variables and parameter to methods (that are marked with the {\tt
flow} modifier).  Values received by a method in a {\tt flow} argument
or stored in a {\tt flow} local variable may not be assigned to fields
of classes.

Each activity is initiated with zero or more clocks, namely those
available in its lexically enclosing environment.  Because of the
condition above, an activity cannot acquire any new clocks during its
lifetime (except by creating them). Therefore for each activity we can
statically determine (a superset of) the clocks used by that activity
through a simple flow analysis.

% {We need to have a cost-model for clocks. Should ensure that this
% model reflects hierarchical quiescence detection, cf trees of short
% circuits.)

Two statements use clocks: {\tt now (c) \{S\}} and {\tt next c;},
where {\tt c} is a clock and {\tt S} a statement.

The statement {\tt now (c) \{S\}} is said to be {\em clocked on {\tt
c}}. It behaves like {\tt S}. However {\tt c} cannot advance as long
as {\tt S} is executing. This statement is useful in an activity that
wishes to execute {\tt S} in the current instant and does not require
{\tt c} thereafter. If it does wish to access c (or state related to
c) in subsequent instants, it should use next c.

The statement {\tt next c;} blocks until clock {\tt c} can advance.

An \X10{} computation is said to be {\em quiescent on $c$} if each
activity that has a reference to $c$ is blocked on {\tt next c;}.
All activities that do not have a reference to $c$ do not need to
be quiescent. Note that once the system is quiescent on $c$, it
will remain quiescent on $c$ forever (unless the system takes some
action), since no non-blocked activity has a reference to $c$,
or can acquire a reference to $c$. That is,  quiescence on 
a clock is a {\em stable property}. 

Once the implementation has detected quiecence on $c$, the system
resumes all activities blocked on $c$; we say that the clock $c$ {\em
has advanced}.

The definition of quiescence given above can cause deadlock. Consider
the situation in which only two activities are running in
parallel. The first executes {\tt next c; A; next d;} and the other
exeutes {\tt next d; B; next c;}. Now the configuration is stuck, but
it is not quiescent on $c$ or $d$. Rather it is quiescent on the
{\em clock set} $\{c, d\}$.

But note that the the first activity should be thought of as quiescent
on $c$ {\em and} {\em conditionally quiescent} on $d$. That is, as
long as the clock $c$ does not advance, this activity should be
considered quiescent on $d$. Similarly, the second activity is
quiescent on $d$ and conditionally quiescent on $c$. Further the only
activities that reference $c$ and $d$ are these two activities. From
this we can conclude that the system is quiescent on both $c$ and
$d$. Therefore it makes sense to {\em jointly advance} both clocks $c$
and $d$. 

More explicitly the rule for joint advancement is as follows: Say that
an activity is quiescent on a clock set $s$ if it is blocked on some
clock in $s$. An \X10{} computation is said to be quiescent on $s$ if
every activity that has a reference to some clock $c \in s$ is blocked
on $s$.  In such a state the system may simultaneously advance all the
clocks in $s$, thus releasing all activities blocked on $s$.

Note that the notion of quiescence on a clock set generalizes
quiescence on a clock and is more strictly more powerful (i.e.{} it
allows more computations to progress). Indeed the rule ensures that no
computation may deadlock because it uses clocks.

\subsection{Implementation Notes}
Clocks may be implemented efficiently with message passing, e.g.{} by
using short-circuit ideas in \cite{SaraswatPODC87}.  Recall that every
activity is spawned with references to a fixed number of clocks. Each
reference should be thought of as a global pointer to a location in
some place representing the clock. (We shall discuss a further
optimization below.) Each clock keeps two counters: the total number
of outstanding references to the clock, and the number of activities
that are currently suspended on the clock.

When an activity $A$ spawns another activity $B$ that will reference a
clock $c$ referenced by $A$, $A$ {\em splits} the reference by sending
a message to the clock. Whenever an activity drops a reference to a
clock, or suspends on it, it sends a message to the clock. 

The optimization is that the clock can be represented in a distributed
fashion. Each place keeps a local counter for each clock that is
referenced by an activity in that place. The global location for the
clock simply keeps track of the places that have references and that
are quiescent. This can reduce the inter-place message traffic
significantly.

\paragraph{Clock set quiescence.}
We now discuss how quiescence on clock sets may be detected. For this
it is convenient to assume that are clocks are globally ordered (to
avoid a ``lord of the rings'' computation).  

Each activity keeps track of the set of clocks it references, together
with the phase of each clock (an integer). Each clock $c$ tracks 
\begin{enumerate}
\item the number $n$ of outstanding references to it, 
\item its current phase $m$, 
\item the set $b$ of activities from which it has received a blocking message,
\item the set $A$ of conditional blocking messages is has received from its activities,
\item a set of  clock set proposals $q(S_i,Q_i)$ it has received from other clocks,

\end{enumerate}

$b$ of activities from which it has received a blocking message, and a
set of quiescent clock set proposals $q(S_i,Q_i)$ it has received from
other clocks, and its current knowledge $p(d)$ of the phase of each
clock $d$ that it knows about.

and two clock
phase sets $S$ and $Q$. $S \cup Q$ represents the set of clocks that
may be in a quiescent set with $c$; $Q$ is further known to be
quiescent, whereas $S$ is not known to be quiescent.

When an activity quiesces on {\tt next c;} it sends a {\bf Q()}
message to $c$ and an $E(d,k)$ message to each remaining clock $d$
(with value $k$) that it references.

A clock {\em advances} itself (see below) on receiving a {\bf Q()}
message from all activities referencing it. (Single clock rule.)

A clock $c$ with value $k$ drops any message $E_c(S \cup
\{(c,k')\},Q)$ it receives if $k\not=k'$. If $k

Suppose an activity
executes {\tt next c;} and has references to the set $s$ of clocks
(excluding $c$). It sends an {\bf AmQuiescent($s$)} message to $c$
and a {\bf CondQuiesecent} message to every clock in $s$.

When a clock has received a (conditional) quiescence messages from all
activities that reference it, it checks to see if the conditional set
(the set of all clocks it has received in a conditional quiescence
message, if any) is empty. If so, it advances itself (see below). If
not it sends a {\bf DepQuiescent(s)} message to the clock $d$
determined as follows. $s \cup \{d\}$ is the union of all sets
received in {\bf DepQuiescent} messages together with the clocks
received in conditional quiescent messages. $d$ is the least clock in
$s\cup\{d\}$.

A clock is advanced by sending a resume message to all activities that
have quiesced on it, as well as a {\bf NowQuiescent} message to any
clock which has sent it a {\bf DepQuiescent} message with an empty
clock set. If the clock has received {\bf DepQuiescent} messages 

A clock receiving a {\bf NowQuiescent} message from some other clock
has determined that it is part of a quiescent clock set and 
advances itself.

\subsection{Usage of clocks}

We allow types to specify clocks, via a clocked{c} modifier, e.g.

  clocked(c) int r;

This declaration asserts that r is accessible (readable/writable) only
by those statements that are clocked on c. 

The declaration 

  clocked(c) final int r = 1;

asserts additionally that in each clock instant r is final, i.e. the
value of r can be reset in each time instant (and stays constant in
that instant).

I expect that most arrays containing application data will be declared
to be phased final.

We should prolly allow a variable to be clocked by multiple
clocks.  For instance

  clocked(c,d) final int r = 1;

6/16/2004

Now we should be able to get determinate computation with shared
memory, if all shared variables are clocked final, and all accesses
are from clocked statements.
