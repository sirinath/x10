%\vfill\eject
\chapter{Expressions}
\label{expressionchapter}

\newcommand{\syntax}{{\em Syntax: }}
\newcommand{\semantics}{{\em Semantics: }}

A Scheme expression is a construct that returns a value, such as a
variable reference, literal, procedure call, or conditional.

Expression types are categorized as \mainindex{primitive
expression}{\em primitive} or \mainindex{derived expression}{\em
derived}.  Primitive expression types include variables and procedure
calls.  Derived expression types are not semantically primitive, but
can instead be explained in terms of the primitive constructs as
in~\ref{derivedsection}.  They are redundant in the strict sense of
the word, but they capture common patterns of usage, and are therefore
provided as convenient abbreviations.

\section{Primitive expression types}
\label{primitivexps}

\subsection{Variable references}\unsection

\begin{entry}{%
\pproto{\hyper{variable}}{\exprtype}}

An expression consisting of a variable\index{variable}
(see~\ref{variablesection}) is a variable reference.  The value of
the variable reference is the value stored in the location to which the
variable is bound.  It is an error\index{error} to reference an
unbound\index{unbound} variable.

\begin{scheme}
(define x 28)
x   \ev  28%
\end{scheme}
\end{entry}

\subsection{Literal expressions}\unsection
\label{literalsection}

\begin{entry}{%
\proto{quote}{ \hyper{datum}}{\exprtype}
\pproto{\singlequote\hyper{datum}}{\exprtype}
\pproto{\hyper{constant}}{\exprtype}}

{\cf (quote \hyper{datum})} evaluates to \hyper{datum}.\mainschindex{'}
\hyper{Datum}
may be any external representation of a Scheme object
(see~\ref{externalreps}).  This notation is used to include literal
constants in Scheme code. \index{constant}

\begin{scheme}%
(quote a)                     \ev  a
(quote \sharpsign(a b c))     \ev  \#(a b c)
(quote (+ 1 2))               \ev  (+ 1 2)%
\end{scheme}

{\cf (quote \hyper{datum})} may be abbreviated as
\singlequote\hyper{datum}.  The two notations are equivalent in all
respects.  The external syntax generated by \ide{write} for
two-element lists whose car is the symbol {\cf quote} may vary between
implementations.

\begin{scheme}
'a                   \ev  a
'\#(a b c)           \ev  \#(a b c)
'()                  \ev  ()
'(+ 1 2)             \ev  (+ 1 2)
'(quote a)           \ev  (quote a)
''a                  \ev  (quote a)%
\end{scheme}

Numerical constants, string constants, character constants, and boolean
constants evaluate ``to themselves''; they need not be quoted.

\begin{scheme}
'"abc"     \ev  "abc"
"abc"      \ev  "abc"
'145932    \ev  145932
145932     \ev  145932
'\schtrue  \ev  \schtrue
\schtrue   \ev  \schtrue%
\end{scheme}

As noted in~\ref{storagemodel}, it is an error\index{error} to alter a
constant (i.e.~the value of a literal expression\mainindex{literal
expression}) using a mutation procedure like
\ide{set-car!}\ or \ide{string-set!}. \index{constant}

\end{entry}


\subsection{Procedure calls}\unsection

\begin{entry}{%
\pproto{(\hyper{operator} \hyperi{operand} \dotsfoo)}{\exprtype}}

A procedure call is written by simply enclosing in parentheses
expressions for the procedure to be called and the arguments to be
passed to it.  The operator and operand expressions are evaluated (in an
unspecified\index{unspecified} order) and the resulting procedure is passed the resulting
arguments.\mainindex{call}\mainindex{procedure call}
\begin{scheme}%
(+ 3 4)                          \ev  7
((if \schfalse + *) 3 4)         \ev  12%
\end{scheme}

A number of procedures are available as the values of variables in the
initial environment; for example, the addition and multiplication
procedures in the above examples are the values of the variables \ide{+}
and \ide{*}.  New procedures are created by evaluating \lambdaexp{}s
(see~\ref{lambda}).
\todo{At Friedman's reuest, flushed mention of other ways.}
% or definitions (see~\ref{define}).

Procedure calls are also called \mainindex{combination}{\em
combinations}.

\begin{note} In contrast to other dialects of Lisp, the order of
evaluation is unspecified\index{unspecified}, and the operator expression and the operand
expressions are always evaluated with the same evaluation rules.
\end{note}

\begin{note}
Although the order of evaluation is otherwise unspecified, the effect of
any concurrent evaluation of the operator and operand expressions is
constrained to be consistent with some sequential order of evaluation.
The order of evaluation may be chosen differently for each procedure call.
\end{note}

\begin{note} In many dialects of Lisp, the empty combination, {\tt
()}, is a legitimate expression.  In Scheme, combinations must have at
least one subexpression, so {\tt ()} is not a syntactically valid
expression.  \todo{Dybvig: ``it should be obvious from the syntax.''}
\end{note}

\todo{Freeman:
I think an explanation as to why evaluation order is not specified
should be included.  It should not include any reference to parallel
evaluation.  Does any existing compiler generate better code because
the evaluation order is unspecified?  Clinger: yes: T3, MacScheme v2,
probably MIT Scheme and Chez Scheme.  But that's not the main reason
for leaving the order unspecified.}

\end{entry}


\subsection{\Lambdaexp{}s}\unsection
\label{lamba}

\begin{entry}{%
\proto{lambda}{ \hyper{formals} \hyper{body}}{\exprtype}}

\syntax
\hyper{Formals} shall be a formal arguments list as described below,
and \hyper{body} shall be a sequence of one or more expressions.

\semantics
\vest A \lambdaexp{} evaluates to a procedure.  The environment in
effect when the \lambdaexp{} was evaluated is remembered as part of the
procedure.  When the procedure is later called with some actual
arguments, the environment in which the \lambdaexp{} was evaluated will
be extended by binding the variables in the formal argument list to
fresh locations, the corresponding actual argument values will be stored
in those locations, and the expressions in the body of the \lambdaexp{}
will be evaluated sequentially in the extended environment.  The result
of the last expression in the body will be returned as the result of
the procedure call.

\begin{scheme}
(lambda (x) (+ x x))      \ev  {\em{}a procedure}
((lambda (x) (+ x x)) 4)  \ev  8

(define reverse-subtract
  (lambda (x y) (- y x)))
(reverse-subtract 7 10)         \ev  3

(define add4
  (let ((x 4))
    (lambda (y) (+ x y))))
(add4 6)                        \ev  10%
\end{scheme}

\hyper{Formals} shall have one of the following forms:

\begin{itemize}
\item {\tt(\hyperi{variable} \dotsfoo)}:
The procedure takes a fixed number of arguments; when the procedure is
called, the arguments will be stored in the bindings of the
corresponding variables.

\item \hyper{variable}:
The procedure takes any number of arguments; when the procedure is
called, the sequence of actual arguments is converted into a newly
allocated list, and the list is stored in the binding of the
\hyper{variable}.

\item {\tt(\hyperi{variable} \dotsfoo{} \hyper{variable$_{n-1}$} {\bf.}\ \hypern{variable})}:
If a space-delimited period precedes the last variable, then
the value stored in the binding of the last variable will be a
newly allocated
list of the actual arguments left over after all the other actual
arguments have been matched up against the other formal arguments.
\end{itemize}

\begin{scheme}
((lambda x x) 3 4 5 6)          \ev  (3 4 5 6)
((lambda (x y . z) z)
 3 4 5 6)                       \ev  (5 6)%
\end{scheme}

It is an error\index{error} for a \hyper{variable} to appear more than once in
\hyper{formals}.

Each procedure created as the result of evaluating a \lambdaexp{} is tagged
with a storage location, in order to make \ide{eqv?} and
\ide{eq?} work on procedures (see~\ref{equivalencesection}).

\end{entry}


\subsection{Conditionals}\unsection

\begin{entry}{%
\proto{if}{ \hyper{test} \hyper{consequent} \hyper{alternate}}{\exprtype}
\protox{if}{ \hyper{test} \hyper{consequent}}{\exprtype}}  %\/ if hyper = italic

\syntax
\hyper{Test}, \hyper{consequent}, and \hyper{alternate} may be arbitrary
expressions.

\semantics
An \ide{if} expression is evaluated as follows: first, \hyper{test} is
evaluated.  If it yields a true value\index{true}
(see~\ref{booleansection}), then \hyper{consequent} is evaluated and
its value is returned.  Otherwise \hyper{alternate} is evaluated and
its value is returned.  If \hyper{test} yields a false value and no
\hyper{alternate} is specified, then the result of the expression is
unspecified\index{unspecified}.

\begin{scheme}
(if (> 3 2) 'yes 'no)           \ev  yes
(if (> 2 3) 'yes 'no)           \ev  no
(if (> 3 2)
    (- 3 2)
    (/ 1 0))                    \ev  1%
\end{scheme}

\end{entry}


\subsection{Assignments}\unsection
\mainindex{assignments}
\label{assignments}

\begin{entry}{%
\proto{set!}{ \hyper{variable} \hyper{expression}}{\exprtype}}

\hyper{Expression} is evaluated, and the resulting value is stored in
the location to which \hyper{variable} is bound.  \hyper{Variable} shall
be bound either in some region\index{region} enclosing the \ide{set!}\ expression
or in the top level environment.  The result of the \ide{set!} expression is
unspecified\index{unspecified}.

\begin{scheme}
(define x 2)
(+ x 1)                 \ev  3
(set! x 4)              \ev  \unspecified
(+ x 1)                 \ev  5%
\end{scheme}

\end{entry}


\section{Derived expression types}

For reference purposes, Section~\ref{derivedsection} gives rewrite rules
that will convert constructs described in this section into the
primitive constructs described in the previous section.


\subsection{Conditionals}\unsection

\begin{entry}{%
\proto{cond}{ \hyperi{clause} \hyperii{clause} \dotsfoo}{\exprtype}}

\syntax
Each \hyper{clause} shall be of the form
\begin{scheme}
(\hyper{test} \hyper{expression} \dotsfoo)%
\end{scheme}
where \hyper{test} is any expression.  Alternately, a \hyper{clause}
may be of the form
\begin{scheme}
(\hyper{test} => \hyper{expression})%
\end{scheme}
where again, \hyper{test} is any expression.  The last \hyper{clause}
may be an ``else clause,'' which has the form
\begin{scheme}
(else \hyperi{expression} \hyperii{expression} \dotsfoo)\rm.%
\end{scheme}
\mainschindex{else}
\mainschindex{=>}

\semantics
A \ide{cond} expression is evaluated by evaluating the \hyper{test}
expressions of successive \hyper{clause}s in order until one of them
evaluates to a true value\index{true} (see~\ref{booleansection}).
When a \hyper{test} evaluates to a true value, then the remaining
\hyper{expression}s in its \hyper{clause} are evaluated in order, and
the result of the last \hyper{expression} in the
\hyper{clause} is returned as the result of the entire \ide{cond}
expression.
If the selected \hyper{clause} contains only the \hyper{test} and no
\hyper{expression}s, then the value of the \hyper{test} is returned as
the result.
If the \hyper{clause} uses the {\tt =>} alternate form, then
\hyper{expression} shall evaluate to a procedure, which is invoked with
the value of \hyper{test} as its single argument; the value returned
by this procedure is returned as the value of the entire \ide{cond}
expression.
If all \hyper{test}s evaluate
to false values, and there is no else clause, then the result of
the conditional expression is unspecified\index{unspecified}; if there is an else
clause, then its \hyper{expression}s are evaluated, and the value of
the last one is returned.

\begin{scheme}
(cond ((> 3 2) 'win)
      ((/ 1 0) 'lose))         \ev  win%

(cond ((> 3 3) 'greater)
      ((< 3 3) 'less)
      (else 'equal))            \ev  equal%

(cond ((assv 'b '((a 1) (b 2))) => cadr)
      (else \schfalse{}))         \ev  2%
\end{scheme}
\end{entry}


\begin{entry}{%
\proto{case}{ \hyper{key} \hyperi{clause} \hyperii{clause} \dotsfoo}{\exprtype}}

\syntax
\hyper{Key} may be any expression.  Each \hyper{clause} shall have
the form
\begin{scheme}
((\hyperi{datum} \dotsfoo) \hyperi{expression} \hyperii{expression} \dotsfoo)\rm,%
\end{scheme}
where each \hyper{datum} is an external representation of some object.
The last \hyper{clause} may be an ``else clause,'' which has the form
\begin{scheme}
(else \hyperi{expression} \hyperii{expression} \dotsfoo)\rm.%
\end{scheme}
\schindex{else}

\semantics
A \ide{case} expression is evaluated as follows.  \hyper{Key} is
evaluated and then each successive \hyper{clause} is processed in
order.  If the result of evaluating \hyper{key} is equivalent (in the
sense of \ide{eqv?}; see~\ref{eqv?}) to any \hyper{datum} in the
\hyper{clause}, then the expressions in that \hyper{clause} are
evaluated from left to right and the result of the last expression in
the \hyper{clause} is returned as the result of the \ide{case}
expression.  If no \hyper{datum} in any \hyper{clause} is equivalent
to the result of evaluating \hyper{key}, then if there is an else
clause its expressions are evaluated and the result of the last is the
result of the \ide{case} expression; otherwise the result of the
\ide{case} expression is unspecified\index{unspecified}.

\begin{scheme}
(case (* 2 3)
  ((2 3 5 7) 'prime)
  ((1 4 6 8 9) 'composite))     \ev  composite
(case (car '(c d))
  ((a) 'a)
  ((b) 'b))                     \ev  \unspecified
(case (car '(c d))
  ((a e i o u) 'vowel)
  ((w y) 'semivowel)
  (else 'consonant))            \ev  consonant%
\end{scheme}

\todo{Isn't "r" a semivowel?}

\end{entry}


\begin{entry}{%
\proto{and}{ \hyperi{test} \dotsfoo}{\exprtype}}

The \hyper{test} expressions are evaluated from left to right, and the
value of the first expression that evaluates to a false value
(see~\ref{booleansection}) is returned.  Any remaining expressions are
not evaluated.  If all the expressions evaluate to true values, the
value of the last expression is returned.  If there are no expressions
then \schtrue{} is returned.

\begin{scheme}
(and (= 2 2) (> 2 1))           \ev  \schtrue
(and (= 2 2) (< 2 1))           \ev  \schfalse
(and 1 2 'c '(f g))             \ev  (f g)
(and)                           \ev  \schtrue%
\end{scheme}

\end{entry}


\begin{entry}{%
\proto{or}{ \hyperi{test} \dotsfoo}{\exprtype}}

The \hyper{test} expressions are evaluated from left to right, and the
value of the first expression that evaluates to a true value
(see~\ref{booleansection}) is returned.  Any remaining expressions are
not evaluated.  If all the expressions evaluate to false values, the value
of the last expression is returned.  If there are no expressions then
\schfalse{} is returned.

\begin{scheme}
(or (= 2 2) (> 2 1))            \ev  \schtrue
(or (= 2 2) (< 2 1))            \ev  \schtrue
(or \schfalse \schfalse \schfalse) \ev  \schfalse
(or (memq 'b '(a b c)) 
    (/ 3 0))                    \ev  (b c)%
\end{scheme}

\end{entry}


\subsection{Binding constructs}
\label{bindingconstructs}

The three binding constructs \ide{let}, \ide{let*}, and \ide{letrec}
give Scheme a block structure, like Algol 60.  The syntax of the three
constructs is similar, but they differ in the regions\index{region} they establish
for their variable bindings.  In a \ide{let} expression, the initial
values are computed before any of the variables become bound; in a
\ide{let*} expression, the bindings and evaluations are performed
sequentially; while in a \ide{letrec} expression, all the bindings are in
effect while their initial values are being computed, thus allowing
mutually recursive definitions.

\begin{entry}{%
\proto{let}{ \hyper{bindings} \hyper{body}}{\exprtype}}

\syntax
\hyper{Bindings} shall have the form
\begin{scheme}
((\hyperi{variable} \hyperi{init}) \dotsfoo)\rm,%
\end{scheme}
where each \hyper{init} is an expression, and \hyper{body} shall be a
sequence of one or more expressions.  It is
an error\index{error} for a \hyper{variable} to appear more than once in the list of variables
being bound.

\semantics
The \hyper{init}s are evaluated in the current environment (in some
unspecified\index{unspecified} order), the \hyper{variable}s are bound to fresh locations
holding the results, the \hyper{body} is evaluated in the extended
environment, and the value of the last expression of \hyper{body} is
returned.  Each binding of a \hyper{variable} has \hyper{body} as its
region.\index{region}

\begin{scheme}
(let ((x 2) (y 3))
  (* x y))                      \ev  6

(let ((x 2) (y 3))
  (let ((x 7)
        (z (+ x y)))
    (* z x)))                   \ev  35%
\end{scheme}

See also~\ref{namedlet}, named \ide{let}.

\end{entry}


\begin{entry}{%
\proto{let*}{ \hyper{bindings} \hyper{body}}{\exprtype}}\nobreak

\nobreak
\syntax
\hyper{Bindings} shall have the form
\begin{scheme}
((\hyperi{variable} \hyperi{init}) \dotsfoo)\rm,%
\end{scheme}
and \hyper{body} shall be a sequence of
one or more expressions.

\semantics
\ide{Let*} is similar to \ide{let}, but the bindings are performed
sequentially from left to right, and the region\index{region} of a binding indicated
by {\cf(\hyper{variable} \hyper{init})} is that part of the \ide{let*}
expression to the right of the binding.  Thus the second binding is done
in an environment in which the first binding is visible, and so on.

\begin{scheme}
(let ((x 2) (y 3))
  (let* ((x 7)
         (z (+ x y)))
    (* z x)))             \ev  70%
\end{scheme}

\end{entry}


\begin{entry}{%
\proto{letrec}{ \hyper{bindings} \hyper{body}}{\exprtype}}

\syntax
\hyper{Bindings} shall have the form
\begin{scheme}
((\hyperi{variable} \hyperi{init}) \dotsfoo)\rm,%
\end{scheme}
and \hyper{body} shall be a sequence of
one or more expressions. It is an error\index{error} for a \hyper{variable} to appear more
than once in the list of variables being bound.

\semantics
The \hyper{variable}s are bound to fresh locations holding undefined
values, the \hyper{init}s are evaluated in the resulting environment (in
some unspecified\index{unspecified} order), each \hyper{variable} is assigned to the result
of the corresponding \hyper{init}, the \hyper{body} is evaluated in the
resulting environment, and the value of the last expression in
\hyper{body} is returned.  Each binding of a \hyper{variable} has the
entire \ide{letrec} expression as its region\index{region}, making it possible to
define mutually recursive procedures.

\begin{scheme}
%(letrec ((x 2) (y 3))
%  (letrec ((foo (lambda (z) (+ x y z))) (x 7))
%    (foo 4)))                   \ev  14
%
(letrec ((even?
          (lambda (n)
            (if (zero? n)
                \schtrue
                (odd? (- n 1)))))
         (odd?
          (lambda (n)
            (if (zero? n)
                \schfalse
                (even? (- n 1))))))
  (even? 88))   
                \ev  \schtrue%
\end{scheme}

One restriction on \ide{letrec} is very important: it shall be possible
to evaluate each \hyper{init} without assigning or referring to the value of any
\hyper{variable}.  If this restriction is violated, then it is an error\index{error}.  The
restriction is necessary because Scheme passes arguments by value rather than by
name.  In the most common uses of \ide{letrec}, all the \hyper{init}s are
\lambdaexp{}s and the restriction is satisfied automatically.

% \todo{use or uses?  --- Jinx.}

\end{entry}


\subsection{Sequencing}\unsection
\label{sequencingsection}

\begin{entry}{%
\proto{begin}{ \hyperi{expression} \hyperii{expression} \dotsfoo}{\exprtype}}

The \hyper{expression}s are evaluated sequentially from left to right,
and the value of the last \hyper{expression} is returned.  This
expression type is used to sequence side effects such as input and
output.  (See also~\ref{definitionsection}.)

\begin{scheme}
(define x 0)

(begin (set! x 5)
       (+ x 1))                  \ev  6

(begin (display "4 plus 1 equals ")
       (display (+ 4 1)))      \ev  \unspecified
 \>{\em and prints}  4 plus 1 equals 5%
\end{scheme}

\end{entry}


\subsection{Iteration}%\unsection

\noindent%
\pproto{(do ((\hyperi{variable} \hyperi{init} \hyperi{step})}{\exprtype}
\mainschindex{do}{\tt\obeyspaces%
     \dotsfoo)\\
    (\hyper{test} \hyper{expression} \dotsfoo)\\
  \hyper{command} \dotsfoo)}

\ide{Do} is an iteration construct.  It specifies a set of variables to
be bound, how they are to be initialized at the start, and how they are
to be updated on each iteration.  When a termination condition is met,
the loop exits with a specified result value.

\ide{Do} expressions are evaluated as follows:
The \hyper{init} expressions are evaluated (in some unspecified\index{unspecified} order),
the \hyper{variable}s are bound to fresh locations, the results of the
\hyper{init} expressions are stored in the bindings of the
\hyper{variable}s, and then the iteration phase begins.

\vest Each iteration begins by evaluating \hyper{test}; if the result is
false (see~\ref{booleansection}), then the \hyper{command}
expressions are evaluated in order for effect, the \hyper{step}
expressions are evaluated in some unspecified\index{unspecified} order, the
\hyper{variable}s are bound to fresh locations, the results of the
\hyper{step}s are stored in the bindings of the
\hyper{variable}s, and the next iteration begins.

\vest If \hyper{test} evaluates to a true value, then the
\hyper{expression}s are evaluated from left to right and the value of
the last \hyper{expression} is returned as the value of the \ide{do}
expression.  If no \hyper{expression}s are present, then the value of
the \ide{do} expression is unspecified\index{unspecified}.

\vest The region\index{region} of the binding of a \hyper{variable}
consists of the entire \ide{do} expression except for the \hyper{init}s.
It is an error\index{error} for a \hyper{variable} to appear more than once in the
list of \ide{do} variables.

\vest A \hyper{step} may be omitted, in which case the effect is the
same as if {\cf(\hyper{variable} \hyper{init} \hyper{variable})} had
been written instead of {\cf(\hyper{variable} \hyper{init})}.

\begin{scheme}
(do ((vec (make-vector 5))
     (i 0 (+ i 1)))
    ((= i 5) vec)
  (vector-set! vec i i))          \ev  \#(0 1 2 3 4)

(let ((x '(1 3 5 7 9)))
  (do ((x x (cdr x))
       (sum 0 (+ sum (car x))))
      ((null? x) sum)))             \ev  25%
\end{scheme}

%\end{entry}


\begin{entry}{%
\proto{let}{ \hyper{variable} \hyper{bindings} \hyper{body}}{\exprtype}}

\label{namedlet}
This is a variant on the syntax of \ide{let} called ``named
\ide{let}''\index{named let} which provides a more general looping construct than
\ide{do}, and may also be used to express recursions.

Named \ide{let} has the same syntax and semantics as ordinary \ide{let}
except that \hyper{variable} is bound within \hyper{body} to a procedure
whose formal arguments are the bound variables and whose body is
\hyper{body}.  Thus the execution of \hyper{body} may be repeated by
invoking the procedure named by \hyper{variable}.

%                                              |  <-- right margin
\begin{scheme}
(let loop ((numbers '(3 -2 1 6 -5))
           (nonneg '())
           (neg '()))
  (cond ((null? numbers) (list nonneg neg))
        ((negative? (car numbers))
         (loop (cdr numbers)
               nonneg
               (cons (car numbers) neg)))
        (else
         (loop (cdr numbers)
               (cons (car numbers) nonneg)
               neg)))) %
  \lev  ((6 1 3) (-5 -2))%
\end{scheme}

\end{entry}


\subsection{Quasiquotation}\unsection
\label{quasiquotesection}

\begin{entry}{%
\proto{quasiquote}{ \hyper{template}}{\exprtype} \nopagebreak
\pproto{\backquote\hyper{template}}{\exprtype}}

``Backquote'' or ``quasiquote''\index{backquote} expressions are useful
for constructing a list or vector structure when most but not all of the
desired structure is known in advance.  If no
commas\index{comma} appear within the \hyper{template}, the result of evaluating
\backquote\hyper{template} is equivalent (in the sense of
\ide{equal?}; see~\ref{equal?}) to the result of evaluating
\singlequote\hyper{template}.  If a comma\mainschindex{,} appears within the
\hyper{template}, however, the expression following the comma is
evaluated (``unquoted'') and its result is inserted into the structure
instead of the comma and the expression.  If a comma appears followed
immediately by an at-sign (\atsign),\mainschindex{,@}
then the following expression shall evaluate to a list; the opening
and closing parentheses of the list are then ``stripped away'' and the
elements of the list are inserted in place of the comma at-sign
expression sequence.

% struck: "(in the sense of \ide{equal?})" after "equivalent"

\begin{scheme}
`(list ,(+ 1 2) 4)  \ev  (list 3 4)
(let ((name 'a)) `(list ,name ',name)) %
          \lev  (list a (quote a))
`(a ,(+ 1 2) ,@(map abs '(4 -5 6)) b) %
          \lev  (a 3 4 5 6 b)
`((foo ,(- 10 3)) ,@(cdr '(c)) . ,(car '(cons))) %
          \lev  ((foo 7) . cons)
`\#(10 5 ,(sqrt 4) ,@(map sqrt '(16 9)) 8) %
          \lev  \#(10 5 2 4 3 8)
`,(+ 2 3)            \ev  5%
\end{scheme}

Quasiquote forms may be nested.  Substitutions are made only for
unquoted components appearing at the same nesting level
as the outermost backquote.  The nesting level increases by one inside
each successive quasiquotation, and decreases by one inside each
unquotation.

\begin{scheme}
`(a `(b ,(+ 1 2) ,(foo ,(+ 1 3) d) e) f) %
          \lev  (a `(b ,(+ 1 2) ,(foo 4 d) e) f)
(let ((name1 'x)
      (name2 'y))
  `(a `(b ,,name1 ,',name2 d) e)) %
          \lev  (a `(b ,x ,'y d) e)%
\end{scheme}

The notations
 \backquote\hyper{template} and {\tt (quasiquote \hyper{template})}
 are identical in all respects.
 {\cf,\hyper{expression}} is identical to {\cf (unquote \hyper{expression})},
 and
 {\cf,@\hyper{expression}} is identical to {\cf (unquote-splicing \hyper{expression})}.
The external syntax generated by \ide{write} for two-element lists whose
car is one of these symbols
({\cf quasiquote}, {\cf unquote}, or {\cf unquote-splicing})
may vary between implementations.
\mainschindex{`}

\begin{scheme}
(quasiquote (list (unquote (+ 1 2)) 4)) %
          \lev  (list 3 4)
'(quasiquote (list (unquote (+ 1 2)) 4)) %
          \lev  `(list ,(+ 1 2) 4)
     {\em{}i.e.,} (quasiquote (list (unquote (+ 1 2)) 4))%
\end{scheme}

Unpredictable behavior can result if any of the symbols
\ide{quasiquote}, \ide{unquote}, or \ide{unquote-splicing} appear in
positions within a \hyper{template} otherwise than as described above.

\end{entry}
