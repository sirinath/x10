Many features of modern object-oriented type systems can be
cast as specializations of the generic constrained types
framework.  
Generic constrained types generalize virtual types and 
have a connection to parametric types with use-site variance
annotations, such as Java's wildcards.  

\subsection{Virtual types}

\emph{Virtual types}~\cite{beta,mp89-virtual-classes,ernst06-virtual}
are a language-based extensibility
mechanism 
originally introduced in the language
BETA~\cite{beta} that---along with
similar constructs built on path-dependent types found in
languages such as Scala~\cite{scala}, J\&~\cite{nqm06},
and Tribe~\cite{cdnw07-tribe}---share many similarities with type properties.
A virtual type is a type binding nested within an enclosing instance.
Subclasses are permitted to override the binding of inherited virtual types. 

Virtual types
may be used to provide genericity; indeed,
one of the first proposals for genericity in Java was based on
virtual types~\cite{thorup97}, and
Java
wildcards (i.e., parameters with use-site variance)
were developed from a line of work beginning with virtual
types~\cite{unifying-genericity,variant-parametric-types,adding-wildcards}.

%Igarashi and Pierce~\cite{ip99-virtual-types}
%model the semantics of virtual types
%and several variants
%in a typed lambda-calculus with subtyping and dependent types.

Constrained types are more expressive than virtual
types in that they can be constrained at the use-site,
can be refined on a per-object basis without explicit subclassing,
and can be refined contravariantly as well as covariantly.

Thorup~\cite{thorup97}
proposed adding genericity to Java using virtual types.  For example,
a generic \xcd"List" class can be written as follows:
{
\begin{xten}
abstract class List {
  abstract typedef T;
  T get(int i) { ... }
}
\end{xten}}
\noindent
The virtual type \xcd"T" is unbound in \xcd"List", but 
can be refined by binding \xcd"T" in a subclass:
{
\begin{xten}
abstract class NumberList extends List {
  abstract typedef T as Number;
}
class IntList extends NumberList {
  final typedef T as Integer;
}
\end{xten}}
\noindent
Only classes where \xcd"T" is final bound, such as \xcd"IntList",
can be non-abstract.
%
The analogous definition of 
\xcd"List" in \Xten{} using type properties is as follows:
{
\begin{xten}
class List[T] {
  def get(i: int): T { ... }
}
\end{xten}}

\noindent
Unlike the virtual-type version,
the \Xten{} version of \xcd"List" is not abstract;
\xcd"T" need not be instantiated by a subclass because it can be
instantiated on a per-object basis.
Rather than declaring subclasses of \xcd"List",
one uses the constrained subtypes
\xcdmath"List{T$\extends$Number}" and \xcd"List{T==Integer}".

Type properties can also be refined contravariantly.
For instance, one can write the type \xcdmath"List{T$\super$Integer}",
and even \xcdmath"List{Integer$\extends$T, T$\extends$Number}".

\subsection{Wildcards}
\label{sec:wildcards}

As discussed in Section~\ref{sec:parameters-vs-fields},
type properties are similar to type
parameters.
Constrained types can provide a characterization of
wildcards in Java~\cite{Java3,adding-wildcards,wildcards-safe}.
Wildcards can be motivated
by the following example from Torgersen et al.~\cite{adding-wildcards}:
Consider a \xcd"Set" class and a variable \xcd"EMPTY" containing
the empty set.  What should be the type of \xcd"EMPTY"?
In Java, one can use a wildcard and 
assign the type \xcd"Set<?>", i.e., the type of all \xcd"Set"
instantiated on \emph{some} parameter.  Clients of this
type do not know what parameter to which the actual instance of \xcd"Set"
is bound.
With constrained types,
a similar effect is achieved simply by leaving the
element type property of \xcd"Set" unconstrained.

Wildcards can
also be bounded above and below with
``\Xcd{?} \Xcd{extends} \Xcd{T}'' and ``\Xcd{?} \Xcd{super} \Xcd{T}'',
respectively.
%
A similar effect is achieved with constrained types by constraining
the
element type property of \xcd"Set" with a subtyping constraint.

We can define the following straightforward translation
from Java wildcards to \Xten{}.
Type parameters are translated to type properties whose name
encodes their position in the parameter list.
Types are translated as follows:
\begin{align*}
\SB{ \Xcd{X} } &= \Xcd{_}i ~~\hfill \mbox{where \xcd{X} is the $i$th type param} \\
\SB{ \Xcdmath{C<?, $\dots$>} } &= \Xcdmath{C\{$\dots$\}} \\
\SB{ \Xcdmath{C<? extends T, $\dots$>} } &= \Xcdmath{C\{_1 } \ \extends\Xcdmath{ $\ \SB{\tt{T}}$, $\dots$\}} \\
\SB{ \Xcdmath{C<? super T, $\dots$>} } &= \Xcdmath{C\{_1 } \ \super\Xcdmath{ $\ \SB{\tt{T}}$, $\dots$\}} \\
\SB{ \Xcdmath{C<T, $\dots$>} } &= \Xcdmath{C\{_1 == $\ \SB{\tt{T}}$, $\dots$\}}
\end{align*}
\noindent
Through such a translation, the \FXG\ calculus in
Section~\ref{sec:semantics} captures the essence of Java's
wildcards, but extended with support for run-time type
introspection.  We leave to future work a formal proof of this
characterization.

\eat{
Finally,
one can also specify lower bounds on types.  These are useful for
comparators:
{\footnotesize
\begin{xtenmath}
class TreeSet[T] {
  def this[T](cmp: Comparator{T $\super$ this.T}) { ... }
}
\end{xtenmath}}
Here, the comparator for any supertype of \xcd"T" can be used as
to compare \xcd"TreeSet" elements.

Another use of lower bounds is for list operations.
The \xcd"map" method below takes a function that maps a supertype
of the class parameter \xcd"T" to the method type parameter \xcd"S":
{\footnotesize
\begin{xtenmath}
class List[T] {
  def map[S](fun: Object{self $\super$ T} => S) : List[S] { ... }
}
\end{xtenmath}}
}

Like wildcards,
constrained types support
\emph{proper abstraction}~\cite{adding-wildcards}.  To illustrate, a
\xcd"reverse"
operation can operate on \xcd"List" of any type:
{
\begin{xten}
def reverse(list: List) {
  for (i: int in 0..list.length-1)
    swap(list, i, list.length-1-i);
}
\end{xten}}

\noindent
The client of \xcd"reverse" need not provide the concrete type
on which the list is instantiated; the \xcd"list" itself
provides the element type---it is stored in the \xcd"list"
to implement run-time type introspection.

In Java, this method would be written with a type parameter on
the method; the type system permits it to be called with any
\xcd"List".
However,
the method parameter cannot be omitted: declaring a parameterless version
of \xcd"reverse" requires delegating to a private parametrized version that
``opens up'' the parameter.

%\subsection{Self types}

Type properties can also be used to support a form of self
types~\cite{bruce-binary,bsg95}.
%
Self types can be implemented by introducing a
type property \Xcd{class} to the root of the class hierarchy, \Xcd{Object}:
{\footnotesize
\begin{xtenmath}
class Object[class] { $\dots$ }
\end{xtenmath}}

\noindent
Scala's path-dependent types~\cite{scala} and J\&'s
dependent classes~\cite{nqm06}
take a similar approach.

Self types are achieved by
implicitly constraining types so that if a path expression \Xcd{p}
has type \Xcd{C}, then
$\Xcd{p}.\Xcd{class} \subtype \Xcd{C}$.  In particular,
$\Xcd{this}.\Xcd{class}$ is guaranteed to be a subtype
of the lexically enclosing class; the type
$\Xcd{this}.\Xcd{class}$ represents all instances of the fixed,
but statically unknown, run-time class referred to by the \Xcd{this}
parameter.

Self types address the binary method problem~\cite{bruce-binary}.
In the following
example, the class \xcd"BitSet" can be written with a
\xcd"union" method that takes a self type as argument.

{\footnotesize
\begin{xtenmath}
interface Set {
  def union(s: this.class): void;
}

class BitSet implements Set {
  int bits;
  def union(s: this.class): void {
    this.bits |= s.bits;
  }
}
\end{xtenmath}}

\noindent
Since \xcd"s" has type this \Xcd{this}.\Xcd{class}, and the class
invariant of \xcd"BitSet" implies
$\Xcd{this}.\Xcd{class} \subtype \Xcd{BitSet}$,
the implementation of the method is free to access the
\xcd"bits" field of \xcd"s".

Callers of \xcd"BitSet".\xcd"union()" must call the method with
an argument that has the same run-time class as the
receiver.  For a receiver \xcd"p", the
type of the actual argument of the call must have a constraint
that entails \xcd"self".\xcd"class==p".\xcd"class".



\eat{
\subsection{Ownership types}

Consider the following example of generic ownership
derived from Potanin et al.~\cite{ogj-oopsla06}.

\begin{xtenmath}
class Object(owner: Object) { }

// Map inherits Object.owner
// No need to add explicit vOwner and kOwner properties for Key, Value
class Map[Key, Value]{Key $\extends$ Comparable, Value $\extends$ Object}
{
    private nodes: Vector[Node[Key, Value](this)](this);

    public def put(key: Key, value: Value): Void {
        nodes.add(new Node[Key, Value](key, value, this)());
    }

    public def get(key: Key): Value {
        for (mn: Node[Key, Value](this) in nodes) {
            if (mn.key.equals(key))
                return mn.value;
        }
        return null;
    }

    // OGJ will prevent this from being called, since caller
    // can only assign the result to a supertype of Vector(this),
    // which would be only Vector(this) or Object(this)
    // BUT: we have Vector $\super$ Vector(this)
    // Need to require that all class types have an equality constraint
    // on the owner property
    public def exposeVector(): Vector(this) { return nodes; }
}

class Node[Key, Value]
    {Key $\extends$ Comparable, Value $\extends$ Object}
{
    val key: Key;
    val value: Value;

    public def this[K, V](k: Key, v: Value, o: Object): Node[K, V](o) {
        super(o);               // set the owner
        property[K, V];         // set the type properties
        this.key = k;
        this.value = v;
    }
}
\end{xten}

Restrictions:
\begin{itemize}
\item owner property must be constrained (define this!)
\item owner is always equal to or inside the owner of all other type properties
\item types with an actual owner == this, can only be accessed via this
\end{itemize}

}

\subsection{Structural constraints}
\label{sec:structural}
\label{sec:structural-typing}

Constraints on types need not be limited to subtyping.
By introducing structural constraints on types, one can
instantiate
type properties on any type with a given set
of methods or fields.
A structural constraint is satisfied if the type has a member of
the appropriate name and with a compatible type.
This feature is useful for reusing code
from separate libraries; it does not require
code of one library to implement a named interface of the other
library to interoperate.

Here, we consider an extension, not yet implemented, of the \Xten{} type system
that supports structural type constraints.
Formally, the extension
is straightforward; indeed the \FX{} family already supports structural
constraints
via the rules for ``{\tt x \underline{has} I}'' in
Figure~\ref{fig:lookup}.

Structural constraints on types are found in many languages.
For instance,
Haskell supports type
classes~\cite{haskell,haskell-type-classes}.
%
%ML's module system allows modules to be constrained by
%structural signatures~\cite{ml}.
In Modula-3, type equivalence is structural
rather than nominal as in object-oriented languages of the C
family (e.g., C++, Java, and \Xten{}).
Unity~\cite{malayeriIntegrating08}
is another language with structural subtyping.

Using structural constraints in \Xten was inspired by 
the language PolyJ~\cite{java-popl97}, which allows type parameters to be
bounded using
structural \emph{where clauses}~\cite{where-clauses}.
For example, a sorted list class % from Figure~\ref{fig:sorted}
could
be written as follows in PolyJ:
{
\begin{xten}
class SortedList[T]
  where T {int compareTo(T)} {
    void add(T x) {... x.compareTo(y) ...}
    ... }
\end{xten}}
The \xcd"where" clause states that the type parameter \xcd"T" must have a
method \xcd"compareTo" with the given signature.

The analogous code for \xcd"SortedList" in the structural
extension of \Xten{} would be:
{
\begin{xten}
class SortedList[T]
  {T has compareTo(T): int} {
    def add(x: T) {... x.compareTo(y) ...}
    ... }
\end{xten}}

\eat{
A structural constraint is satisfied if the type has a member of
the appropriate name and with a compatible type.
The constraint
\xcdmath"Z has def m[$\tbar{X}$]($\tbar{x}\ty\tbar{T}$): U"
is satisfied by a type \xcd"T" if it has a method \xcd"m"
with signature
\xcdmath"def m[$\tbar{Y}$]($\tbar{y}\ty\tbar{S}$): V"
and where
(\xcdmath"[$\tbar{Y}$]($\tbar{y}\ty\tbar{S}$) => V")$[\Xcd{T}/\Xcd{Z}]$
is a subtype of
(\xcdmath"[$\tbar{X}$]($\tbar{x}\ty\tbar{T}$) => U")$[\Xcd{T}/\Xcd{Z}]$.
As an example,
the constraint \xcd"X has def compareTo(X): int"
is satisfied by both of the following classes:
{\footnotesize
\begin{xten}
class C { def compareTo(x: C): int = ...; }
class D { def compareTo(x: Object): int = ...; }
\end{xten}}
}

\subsubsection{Function-typed properties}

\Xten{} supports first-class functions.
Function-typed properties provide a useful alternative to
structural constraints.  In the
following version of the
\xcd"SortedList" class:
{
\begin{xten}
class SortedList(compare: (T,T)=>int)
    extends List {
  def add(x: T) {... compare(x, y) ...}
  ... }
\end{xten}}
\noindent
the class has a property \xcd"compare" of type
\xcd"(T,T)=>int", a function that takes a pair of \xcd"T"s and
returns an \xcd"int".  

Using this definition, one can create lists with distinct types
using different constraints on the \xcd"compare" property.
For example, one can define lists of
both
case-sensitive and case-insensitive strings:
\bigskip
{
\begin{xten}
val unixFiles
  = new SortedList[String]
      (String.compareTo.(String));
val windowsFiles
  = new SortedList[String]
      (String.compareToIgnoreCase.(String));
\end{xten}}

\noindent
\xcd"String.compareTo.(String)" selects the
method named
\xcd"compareTo" with a \xcd"String" argument from
class \xcd"String".
%
The lists \xcd"unixFiles" and \xcd"windowsFiles" are constrained
by different comparison functions.  This allows the programmer
to write code, for instance, in which it is illegal to pass a list of UNIX
file names into a function that expects a list of Windows file names, and
vice versa.

\subsubsection{Optional methods}

Structural method constraints permit the introduction of
CLU-style optional methods~\cite{clu}.  Consider the following
\xcd"Array" class:
{
\begin{xten}
class Array[T] {
  def add(a: Array[S])
    {T has add(S): U}: Array[U] {
    return new Array[U](
      (p:Point) => this(p)+a(p));
  }
  ... }
\end{xten}}

\noindent
The \xcd"Array" class defines an \xcd"add" method that takes 
an array of \xcd"S", adds each element of the array to the
corresponding element of \xcd"this", and returns an array of the
results.  The method constraint specifies that the method may
only be invoked if \xcd"T" has an \xcd"add" method of the
appropriate type.  Thus, for example, an \xcd"Array[int]"
can be added to an \xcd"Array[double]" because \xcd"int"
has a method \xcd"add" (corresponding to the \xcd"+" operation)
that adds an \xcd"int" and a \xcd"double", retuning a
\xcd"double".  However, \xcd"Array[Rabbit]", for example, does not support
the \xcd"add" operation because \xcd"Rabbit" does not have an
\xcd"add" method.
%\footnote{It may have a \xcd"multiply" method.}




\eat{
\subsection{Definition-site variance}

\subsection{Conditional methods and generalized constraints}

For type parameters, method constraints are 
similar to generalized constraints proposed for
\csharp~\cite{emir06}.
%
In the following code, the \xcd"T" parameter is covariant
and so the \xcd"append" methods below are illegal:
{\footnotesize
\begin{xten}
class List[+T] {
  def append(other: T): List[T] { ... }
      // illegal
  def append(other: List[T]): List[T] { ... }
      // illegal
}
\end{xten}}
%
However, one can introduce a method parameter and then constrain
the parameter from below by the class's parameter:
For example, in the following code,
{\footnotesize
\begin{xtenmath}
class List[+T] {
  def append[U](other: U)
      {T $\extends$ U}: List[U] { ... }
  def append[U](other: List[U])
      {T $\extends$ U}: List[U] { ... }
}
\end{xten}}

The constraints must be satisfied by the callers of \xcd"append".
For example, in the following code:
{\footnotesize
\begin{xten}
xs: List[Number];
ys: List[Integer];
xs = ys; // ok
xs.append(1.0); // legal
ys.append(1.0); // illegal
\end{xten}}
the call to \xcd"xs.append" is allowed and the result type is \xcd"List[Number]", but
the call to \xcd"ys.append" is not allowed because the caller cannot show that
$\Xcd{Number} \subtype \Xcd{Double}$.

}

\subsection{Run-time casts}
\label{sec:casts}

While constraints are normally solved at compile time, 
constraints can be evaluated at run time by using casts.
The expression 
\xcd"xs as List{length==n}" checks not only 
that \xcd"xs"
is an instance of
the \xcd"List" class, but also that \xcd"xs.length" equals \xcd"n".
A \xcd"ClassCastException" is thrown if the check fails.
%
In this example, the test of the constraint does not require
run-time constraint
solving; the constraint can be checked by simply
evaluating the \xcd"length" property of \xcd"xs" and comparing against \xcd"n".
However, the situation is more complicated when casting to a
type that constrains the value's type properties.

For equality constraints on types, one can simply test equality on
canonical run-time representations of the base types and of the
constraints.
However, with subtyping constraints (e.g., \xcdmath"S$\extends$T"),
the generated code must be able to test if the constraint on the
subtype entails the constraint on the supertype.  Since one or
both of the types involved in the constraint might be type
variables, the entailment cannot be checked at compile time.

One approach is to restrict the language 
to rule out casts to type parameters 
and to generic types with subtyping constraints, ensuring that
entailment checks are not needed at run time.

Alternatively, 
the constraint solver could be embedded into the runtime system.
This is the solution used in the \Xten{} implementation; however, this
solution can result in inefficient run-time casts
if entailment checking for the given constraint system is expensive.

\eat{
A different approach to have the compiler pre-compute the results of
entailment checks.
This might be done by analyzing the program to identify which pairs of
constraints might be tested for entailment at run time and then generating a
graph were each node is a constraint and there is a directed edge between
nodes in an entailment relationship.  Run-time entailment
checking can then be implemented as reachability checking. 
This solution is a whole program analysis; all
constraints must be visible to generate the graph.
\todo{
We leave the design of this analysis for future work.
}

\todo{this is vague.  and very likely wrong.}
If \xcd"e as T" occurs in the program text and \xcd"e" has
type \xcd"U", the analysis identifies all \xcd"new" expressions
$a$,
that create a subtype of \xcd"U" and identifies all type
expressions $t$ that could instantiate \xcd"T".
For each pair $(a,t)$, the analysis checks that the
type of $a$ (determined by the constructor invoked by $a$) is a subtype of $t$.
Since $a$ and $t$ may be in different environments, \xcd"e" must
be substituted for \xcd"this" in $a$ and \xcd"self" in $t$.
}

Another option is to test objects cast to \xcd"T" not for membership in the
type \xcd"T", but rather to test against the
\emph{interpretation} of \xcd"T".
Observe that
if an instance of a generic class \xcd"C[X]"
is a member of the type \xcd"C{X==U}", then
all fields ${\tt f}_i$ of the instance with
declared type ${\tt S}_i$ contain values
that are instances of ${\tt S}_i[{\tt U}/{\tt X}]$.
For example, given the following declaration of class \xcd"List":
{
\begin{xten}
class List[X] {
  val head: X;
  val tail: List[X];
}
\end{xten}
}
\noindent
if \xcd"xs" is an instance of \xcd"List{X==String}", then
by checking that \xcd"xs.head" is an instance of \xcd"String"
and \xcd"xs.tail" is, recursively, an instance of \xcd"List{X==String}".
This property can be exploited by implementing cast to check the
types of all fields of the object.
For this check to be sound, it is vital that all fields whose
type depends on the type property \xcd"X" be transitively final;
otherwise, the test is not invariant---the
result of the test could change as the data is mutated.
Care must also be taken to implement the
test so that it terminates for cyclic data structures.
This implementation is inefficient for large data structures.

This solution has a more permissive semantics
than those implemented in \Xten or \FXGD.
The difference is best illustrated by considering an empty generic class:
{
\begin{xten}
class Nil[X] { }
\end{xten}
}
\noindent
In this case, there is no field of type \xcd"X" to test;
therefore, an object instantiated as \xcd"Nil[int]" 
can be considered a member of \xcd"Nil[String]". 
However, the solution remains sound:
Given a class \xcd"C[X]" and an expression \xcd"e" of type \xcd"t.X", if
a run-time check finds that \xcd"t" has type \xcd"C[T]", 
the compiler \emph{cannot} use this information to derive 
that \xcd"e" has type \xcd"T".
We leave to future work a proof of this claim.
