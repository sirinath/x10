
%
%I. Introduction and overview (3 pages)

%Design of X10, concurrency, high productivity, high performance,
%practical language.
%
%Dependent types arise naturally. arrays, regions, distributions, place
%types.
%
%Indeed you can look around and recognize many OO type systems proposed
%in the last decade or so as specific kinds of applied dependent type
%systems.
%
%Our goal is to develop a general framework for dependent types for
%statically typed OO languages ("Java-like languages"). 

\Xten{} is a modern statically typed object-oriented
language designed for high productivity in the high performance
computing (HPC) domain~\cite{X10}. Built essentially on
sequential imperative object-oriented core
similar to
Scala or
$\mbox{Java}^{\mbox{\scriptsize\sc tm}}$,
\Xten{} introduces constructs for distribution and
fine-grained concurrency (asynchrony, atomicity, ordering).

The design of \Xten{} requires a rich type system to permit a
large variety of errors to be ruled out at compile time and to 
generate efficient code.  
Like most object-oriented languages, \Xten{} supports classes;
however, it places
equal emphasis on {\em arrays}, a central data structure in high
performance computing.
In particular, \Xten{} supports dense,
distributed multi-dimensional arrays of value and reference types,
built over index sets known as {\em regions}.%, and mappings from index
%sets to places, known as {\em distributions}.  \Xten{} supports a rich
%algebra of operations over regions, distributions and arrays.

A key goal of \Xten{} is to rule out large classes of error
by design. For instance, the possibility of indexing a 2-d array with 3-d
points should simply be ruled out at compile-time. This means that one
must permit the programmer to express types such as \xcd{Region(2)},
the type of all two-dimensional regions, and
\xcd{Array[int](r)}, the type of all \xcd{int} arrays
defined over a given region \xcd{r}.
For concurrent computations, one needs the ability to
statically check that a method is being invoked by an activity that is
registered with a given clock (i.e., dynamic barrier)~\cite{X10}.
For distributed computations, the compiler needs to be aware of
the location of objects: for instance,
\xcd{Tree\{loc==here\}} is the type of all \xcd{Tree} objects located on the
current node.

For performance, it is necessary that array accesses are
bounds-checked statically.  Further, certain regions (e.g.,
rectangular regions) may be represented particularly
efficiently.  Arrays defined over sparse regions may be
implemented with less memory overhead.
Hence, if a variable is to range only over
rectangular regions, it is important that this information is
conveyed through the type system to the code generator.

In this paper we describe {\Xten}'s support for {\em
constrained types},
a form of {\em dependent
type}~\cite{dependent-types,xi99dependent,ocrz-ecoop03,aspinall-attapl,cayenne,epigram-matter,calc-constructions}---types parametrized by values---defined 
on predicates over the {\em immutable}
state of objects. Constrained types statically capture many common invariants
that naturally arise in code. For instance, typically the shape of an
array (the number of dimensions (the rank) and the size of each dimension)
is determined at
run time, but is fixed once the array is constructed. Thus, the shape of an
array is part of its immutable state.
Both mutable and immutable variables may have a constrained
type: the constraint specifies an invariant on the immutable state
of the object referenced by the variable. 

\Xten{} provides a framework for specifying and checking constrained types
that achieves certain desirable properties:
\begin{itemize}
\item 
{\bf Ease of use.}  
The syntax of constrained types is a simple and
natural extension of nominal class types.
% Constrained types in
% \Xten{} interoperate smoothly with Java libraries.

\item
{\bf Flexibility.}
The framework
permits the development of concrete,
specific type systems tailored to the application area at
hand.  \Xten{}'s compiler permits extension with different constraint systems
via compiler plugins, enabling a kind of pluggable type system~\cite{bracha04-pluggable}.
The framework is parametric in the kinds of
expressions used in the type system, permitting the installed constraint
system to interpret the constraints.

\item
{\bf Modularity.}
The rules for type-checking
are specified once in a way that is independent of the
particular vocabulary of operations used in the dependent type
system.
The type system supports separate compilation.

\item
{\bf Static checking.}  The framework permits mostly static
type-checking. The user is able to escape the confines of
static type-checking using dynamic casts.
%, as is common for Java-like
%languages.
\end{itemize}

\subsection{Constrained types}

\begin{figure}[t]
{
\footnotesize
\begin{xtenlines}
class List(n: int{self >= 0}) {
  var head: Object = null;
  var tail: List(n-1) = null;

  def this(): List(0) { property(0); }

  def this(head: Object, tail: List): List(tail.n+1) {
    property(tail.n+1);
    this.head = head;
    this.tail = tail;
  }

  def append(arg: List): List(n+arg.n) {
    return n==0
      ? arg : new List(head, tail.append(arg));
  }

  def reverse(): List(n) = rev(new List());
  def rev(acc: List): List(n+acc.n) {
    return n==0
      ? acc : tail.rev(new List(head, acc));
  }

  def filter(f: Predicate): List{self.n <= this.n} {
    if (n==0) return this;
    val l: List{self.n <= this.n-1} = tail.filter(f);
    return (f.isTrue(head)) ? new List(head,l) : l;
  }
}
\end{xtenlines}
}
\caption{
This program implements a mutable list of Objects. The size of a list
does not change through its lifetime, even though at different points
in time its head and tail might point to different structures.}
\label{fig:list-example}
\end{figure}

\Xten's sequential syntax is similar to Scala's~\cite{scala}.
We permit the definition of a class \xcd{C} to specify
a list of typed parameters or {\em properties},
${\tt f}_1: {\tt T}_1, \dots, {\tt f}_k: {\tt T}_k$,
similar in syntactic structure to a method formal parameter list.
%
Each property in this list is treated as a public final instance field.
%
We also permit the
specification of a {\em class invariant}
%, a {\em where clause}~\cite{where-clauses}
in the class definition,
a boolean expression on the properties of the class.
The compiler ensures that all
instances of the class created at run time satisfy the invariant.
%
For instance, we may specify a class \xcd{List} with an
\xcd{int} \xcd{length} property as follows:
\begin{xtennoindent}
  class List(length: int){length >= 0} {...}
\end{xtennoindent}
The class invariant states that the length of the list
is greater than zero.

Given such a definition for a class \xcd{C}, types
can be
constructed by {\em constraining} the properties of \xcd{C} with
a boolean expression.  In
principle, {\em any} boolean expression over the properties
specifies a type: the type of all instances of the class
satisfying the boolean expression. Thus, \xcd{List\{length == 3\}},
\xcd{List\{length <= 42\}} and
even \xcd{List\{length * f() >= 0\}} (where \xcd{f} 
is function on the immutable state of the \xcd{List} object)
are permissible types.
In practice, the constraint expression is restricted by the
particular constraint system in use.

Our basic approach to introducing constrained types into \Xten{}
is to follow the spirit of generic types, but to use values
instead of types.
%
In general, a {\em constrained type} is of the form \xcd{C\{e\}},
the name of a class or interface\footnote{In \Xten{}, primitive
types such as \xcd{int} and \xcd{double} are object types; thus,
for example, \xcd{int\{self==0\}} is a legal constrained type.}
\xcd{C}, called the {\em base class}, followed
by a {\em condition} \xcd{e}.
The condition may refer to the properties of the base class
and also to any final variables in scope where the type appears.
Such a type represents a refinement of \xcd{C}: the set of all
instances of \xcd{C} whose immutable state satisfies the
condition \xcd{e}.
%
We write \xcd{C} for 
the vacuously constrained type \xcd{C\{true\}}, and
write
\xcd{C(${\tt e}_1,\ldots,{\tt e}_k$)} for
the type
\xcd{C\{${\tt f}_1$==${\tt e}_1,\ldots,{\tt f}_k$==${\tt e}_k$\}}
where \xcd{C} declares the $k$ properties
${\tt f}_1,\ldots,{\tt f}_k$.  Also, for brevity, a constraint
may be written as a comma-separated list of conjuncts, e.g.,
\xcd{Point\{x>0,y<=0\}}.

Constrained types may occur wherever normal types occur. In
particular, they may be used to specify the types of properties,
(possibly mutable) local variables or fields,
arguments to methods, return types of methods; they may also be
used in casts, etc.

Using the definitions above, \xcd{List(n)}, shown in
Figure~\ref{fig:list-example}, is the type of all lists of
length \xcd{n}.
%
Intuitively, this definition states that a \xcd{List} has an \xcd{int}
property \xcd{n}, which must be non-negative.
The properties
of the
class are set through the invocation of \xcd{property}\tcd{(\ldots)}
(analogously to \xcd{super}\tcd{(\ldots)}) in the constructors
of the class.

In a constraint, the name \xcd{self} is bound and refers to the type being
constrained.  The name \xcd{this}, by contrast, is a free
variable in the
constraint and refers to the receiver parameter of the current
method or constructor.  Use of \xcd{this} is not permitted in static
methods.

The \xcd{List} class has two
fields (lines 2--3) that hold the head and tail of the list.  The fields are
declared with the \xcd{var} keyword, indicating that they are
not final.  Variables declared with the \xcd{val} keyword, or
without a keyword (e.g., \xcd{length}: \xcd{int}) are final.

Constructors have ``return
types'' that can specify an invariant satisfied by the object being
constructed.  The compiler verifies that the
constructor return type and the class invariant are implied by the
\xcd{property} statement and any \xcd{super} calls in the constructor
body.
A constructor must either invoke another constructor of the same
class via a
\xcd{this} call
or must have a \xcd{property} statement on every
non-exceptional path
to ensure the properties are initialized.
The \xcd{List} class has two constructors: the first
constructor returns an empty list;
the second
returns a list of length \xcd{m+1}, where \xcd{m} is the length
of the second argument. 

In the second constructor (lines 7--11), as well as 
the \xcd{append} (line 13) and \xcd{rev} (line 20) methods,
the return type
depends on properties of the formal parameters. 
If an argument appears in a
return type then the parameter must be final,
ensuring the
argument points to the same object throughout the evaluation of
the method or constructor body.  A parameter may also depend on
another parameter in the argument list.

The use of constraints makes existential types very natural.
Consider the return type of \xcd{filter} (line 24): it specifies
that the list returned is of some unknown length. The only thing
known about it is that its size is bounded by \tcd{n}.
Thus,
constrained types naturally subsume existential dependent types.
Indeed, every base type \xcd{C} is an ``existential''
constrained type since it does not specify any constraint on its
properties. Thus, code written with constrained types can
interact seamlessly with legacy library code, using just base
types wherever appropriate.

The return type of \xcd{filter} also illustrates the difference
between \xcd{self} and \xcd{this}.  Here, \xcd{self} refers to
the \xcd{List} being returned by the method; \xcd{this} refers
to a different \xcd{List}: the method's receiver.

\subsection{Constraint system plugins}

The \Xten{} compiler allows  
programmers to extend the semantics of the language with
compiler plugins.  Plugins may be used to support different constraint
systems~\cite{cccc} to be used in constrained types.
Constraint systems provide code for checking consistency and
entailment.

The condition of a constrained type is parsed and type-checked
as a normal boolean expression over properties and
the final variables in scope at the type.  Installed
constraint systems translate the expression into an internal
form, rejecting expressions that cannot be represented.
%
A given condition may be a conjunction of constraints from
multiple constraint systems.
A Nelson--Oppen procedure~\cite{nelson-oppen} is used to check
consistency of the constraints.

The \Xten{} compiler
implements a simple
equality-based constraint system.  Constraint solver plugins
have been implemented for inequality constraints, for
Presburger constraints using
the CVC3 theorem prover~\cite{cvc}, and for
set-based constraints also using CVC3.
These constraint systems are described in
Section~\ref{sec:examples} and the implementation is
discussed in Section~\ref{sec:impl}.

\subsection{Claims}

The paper presents constrained types in the \Xten{} programming
language.
We claim that the design is natural, easy to use, and useful. Many
example programs have been written using constrained types and are
available at {\tt x10.sf.net/\allowbreak applications/\allowbreak examples}.

As in staged languages~\cite{nielson-multistage,ts97-multistage}, the
design distinguishes between compile-time and run-time
evaluation. Constrained types are checked (mostly) at compile-time.
The compiler uses a constraint solver to perform universal reasoning
(e.g., ``for all possible values of method parameters'') for dependent
type-checking.  There is no run-time constraint-solving.  However,
run-time casts and \xcd{instanceof} checks involving dependent types
are permitted; these tests involve
arithmetic, not algebra---the values of all parameters are known.

The design supports separate compilation: a class needs to be
recompiled only when it is modified or when the method
and field signatures or invariants of classes on which it
depends are modified.

We claim that the design is flexible. The language design is
parametric on the constraint system being used.
%We are planning on extending the current
%implementation to support multiple user-defined constraint systems,
%thereby supporting pluggable types.
The compiler supports
integration of
different constraint solvers into the language.
Dependent clauses  also form
the basis of a general user-definable annotation framework we have
implemented separately~\cite{ns07-x10anno}. 

We claim the design is clean and modular. We present a simple core
language \CFJ, extending \FJ{}~\cite{FJ} with constrained types on top
of an arbitrary constraint system. We present rules for type-checking
\CFJ{} programs that are parametric in the constraint system
and establish subject reduction and progress theorems. 

%
% XXX contrast with hybrid type checking.

\paragraph{Rest of this paper.}

Section~\ref{sec:lang} describes the syntax and semantics of
constrained types.
Section~\ref{sec:examples} works through a number of
examples using a variety of constraint systems.
The compiler implementation, including support for constraint
system plugins, is described Section~\ref{sec:impl}.
A formal semantics for a core language with constrained types 
is presented in Section~\ref{sec:semantics}, and a soundness
proof is presented in the appendix.
Section~\ref{sec:related} reviews related work.
The paper concludes in Section~\ref{sec:future}
with a discussion of future work.


