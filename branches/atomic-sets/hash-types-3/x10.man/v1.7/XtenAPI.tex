\chapter{API}\label{XtenAPI}

The \Xten{} implementation is required to implement a number of
classes in the package \xcd"x10.lang".

\section{Base types}

\begin{xten}
package x10.lang;

/** The top of the type hierarchy. */
public interface Object {
    public def equals(o: Object): Boolean;
    public def hashCode(): Int;
    public def toString(): String;
    /** Return the full name of the class, e.g., "x10.lang.String" */
    public def className(): String;
}

/**
 * Base class of all value classes; implemented by any class
 * declared value; a Ref class cannot implement Value.
 */
public value Value { }

/** Base class of all reference classes, except Object. */
public class Ref(location: Place) {
    public def this() { property(here); }
}
\end{xten}

\section{Boolean}

\begin{xten}
package x10.lang;

public value Boolean { ... }
\end{xten}

\section{Char}

\begin{xten}
package x10.lang;

public value Char { ... }
\end{xten}

\section{Numeric values types}

\begin{xten}
package x10.lang;

public value Byte   { ... }
public value Short  { ... }
public value Int    { ... }
public value Long   { ... }
public value UByte  { ... }
public value UShort { ... }
public value UInt   { ... }
public value ULong  { ... }
public value Float  { ... }
public value Double { ... }

// Type aliases
public type boolean = Boolean;
public type char = Char;
public type byte = Byte;
public type short = Short;
public type int = Int;
public type long = Long;
public type ubyte = UByte;
public type ushort = UShort;
public type uint = UInt;
public type ulong = ULong;
public type float = Float;
public type double = Double;

public type Int8    = Byte;
public type Int16   = Short;
public type Int32   = Int;
public type Int64   = Long;

public type UInt8   = UByte;
public type UInt16  = UShort;
public type UInt32  = UInt;
public type UInt64  = ULong;

public type Float32 = Float;
public type Float64 = Double;

public type Boolean(x: Boolean) = Boolean{self == x};
public type Byte   (x: Byte)    = Byte   {self == x};
public type Short  (x: Short)   = Short  {self == x};
public type Char   (x: Char)    = Char   {self == x};
public type Int    (x: Int)     = Int    {self == x};
public type Long   (x: Long)    = Long   {self == x};
public type Float  (x: Float)   = Float  {self == x};
public type Double (x: Double)  = Double {self == x};
\end{xten}

\section{String}

\begin{xten}
package x10.lang;

public value String implements (Int) => Char {
    // methods of java.lang.String
    def apply(i: Int) = charAt(i);
}

public type String (x: String) = String {self == x};

public type string = String;

public class StringBuffer implements (Int) => Char {
    // methods of java.lang.StringBuilder
    def apply(i: Int) = charAt(i);
}
\end{xten}

\section{Native rails}

\begin{xten}
package x10.lang;

public interface Indexable[T](base: Region) extends (Point(base)) => T { } 

public interface Settable[T](base: Region) extends (Point(base)) => T {
    public def set(p: Point(base), v: T): Void;
} 

interface NativeRail[T](length: Nat) extends Indexable[T](0..length-1) { }

interface NativeValRail[T](length: Nat) extends NativeRail[T](length) { }

interface NativeVarRail[T](length: Nat) extends NativeRail[T](length),
    Settable[T](0..length-1) { }
\end{xten}

\section{Rails}


A rail is an array over the distribution \xcd"[0..n-1]-> here".
A \xcd"ValRail" is an immutable rail of \xcd"T"s.
A \xcd"Rail" is a mutable rail of \xcd"T"s.

\begin{xten}
public value Rail[T](length: Nat) 
    implements Array[T], Arithmetic[Rail[T](length)]{T <: Arithmetic[T]}
{
}
\end{xten}



\section{Arrays}

\begin{xten}
abstract class AbstractArray[T,R](dist: Dist){R <: NativeRail[T]]
    implements Indexable[T](dist.region)
{
    def this[T,R](dist: Dist, init: (Point(dist.region)) => T):
        AbstractArray[T,R](dist);

    def get(p: Point(dist.region)): T;
    def apply(p: Point(dist.region)) = get(p);

    def map[S](op: (T) => S): ValArray[S](dist);
    def map[S,U](b: Array[S](dist), op: (T,S) => U): ValArray[U](dist);
    def map[S,U](b: ValArray[S](dist), op: (T,S) => U): ValArray[U](dist);

    def reduce(z: S, op: (T,T) => S): S;

    def andReduce(op: (T) => Boolean): Boolean;
    def andReducePoint(op: (Point(base)) => Boolean): Boolean;

    def orReduce(op: (T) => Boolean): Boolean;
    def orReducePoint(op: (Point(base)) => Boolean): Boolean;
}
        
public class ValArray[T](dist: Dist) extends AbstractArray[T,NativeValRail[T]]
    implements Arithmetic[ValArray[T]]
{
    def this[T](dist: Dist, init: (Point(dist.region)) => T):
        ValArray[T](dist);
    def this[T](n: Int, init: (Int) => T):
        ValArray[T]([0..n-1]->here);

    def add(x: ValArray[T]){T <: Arithmetic[T]}: ValArray[T];
}

public class Array[T](dist: Dist) extends AbstractArray[T,NativeVarRail[T]]
    implements Arithmetic[ValArray[T]], Settable[T](dist.region)
{
    def this[T](dist: Dist, init: (Point(dist.region)) => T):
        ValArray[T](dist);
    def this[T](n: Int, init: (Int) => T):
        ValArray[T]([0..n-1]->here);

    def set(p: Point(dist.region), x: T);
}
\end{xten}

\section{Points, regions, distributions}

\begin{xten}
public value Point(rank: Int{self >= 0}) implements
    (Int{0 <= self, self < rank}) => Int
{ }

public value Region(rank: Int{self >= 0},
                    rect: Boolean) {
}

public value Dist(region: Region, onePlace: Box[Place]) extends ValArray[Place](region) {
    property rank = region.rank;
    property zeroBased = region.zeroBased;
    property rect = region.rect;
    property rail = zeroBased && rect && onePlace != null;
}
\end{xten}

\section{Clocks}

\begin{xten}
public value Clock { }
\end{xten}

\section{Places}

\begin{xten}
public value Place { }
\end{xten}

\section{Boxed types}

\begin{xten}
public class Box[T](value: T) { }
\end{xten}

\if 0
\section{Function types}

The library contains the following family of interfaces.

\begin{xtenmath}
public interface Fun_$k$_$n$[X$_1$, $\dots$, X$_k$] {
    def apply(x$_1$: T$_1$, $\dots$, x$_n$: T$_n$): T;
}
\end{xtenmath}

Closures implement the appropriate \xcd"Fun" interface.  The
function type \xcdmath"[X$_1$, $\dots$, X$_k$](x$_1$: T$_1$, $\dots$, x$_n$: T$_n$) => T"
is equivalent to the interface \xcdmath"Fun_$k$_$n$".

A closure call \xcd"e1(e2)" is equivalent to 
the method call \xcd"e1.apply(e2)".
\fi
