\section{Distributions}\label{XtenDistributions}
\index{distribution}

A {\em distribution} is a mapping from a region to a set of places.
{}\Xten{} provides a built-in class, \xcd"x10.lang.Dist", to allow the creation of new distributions and
to perform operations on distributions. This class is \xcd"final" in
{}\XtenCurrVer; future versions of the language may permit
user-definable distributions. Since distributions play a dual role
(values as well as types), variables of type \xcd"Dist" must
be initialized and are implicitly \xcd"val".
\bard{Is this true?}

The {\em rank} of a distribution is the rank of the underlying region.

%Recall that each program runs in a fixed number of places, determined
%by runtime parameters. The static constant Place.MAX_PLACES specifies
%the maximum number of places. The collection of places is assumed to
%be totally ordered.


\begin{xten}
R: Region = 1..100;
D: Dist = Dist.makeBlock(R);
D: Dist = Dist.makeCyclic(R);
D: Dist = R -> here;
D: Dist = Dist.random(R);
\end{xten}

Let \xcd"D" be a distribution. \xcd"D.region" denotes the underlying
region. \xcd"D.places" is the set of places constituting the range of
\xcd"D" (viewed as a function). Given a point \xcd"p", the expression
\xcd"D(p)" represents the application of \xcd"D" to \xcd"p", that is,
the place that \xcd"p" is mapped to by \xcd"D". The evaluation of the
expression \xcd"D(p)" throws an \xcd"ArrayIndexOutofBoundsException"
if \xcd"p" does not lie in the underlying region.

When operated on as a distribution, a region \xcd"R" implicitly
behaves as the distribution mapping each item in \xcd"R" to \xcd"here"
(i.e., \xcd"R->here", see below). Conversely, when used in a context
expecting a region, a distribution \xcd"D" should be thought of as
standing for \xcd"D.region".

{}\oldtodo{Allan: We do not specify how the values of an array at a place
are stored, e.g. in row-major or column major order. Need to work this
out.}

\subsection{Operations returning distributions}

Let \xcd"R" be a region, \xcd"Q" a set of places \{\xcd"p1", \dots, \xcd"pk"\}
(enumerated in canonical order), and \xcd"P" a place. All the operations
described below may be performed on \xcd"Dist.factory".

\paragraph{Unique distribution} \index{distribution!unique}
The distribution \xcd"unique(Q)" is the unique distribution from the
region \xcd"1..k" to \xcd"Q" mapping each point \xcd"i" to \xcd"pi".

\paragraph{Constant distributions.} \index{distribution!constant}
The distribution \xcd"R->P" maps every point in \xcd"R" to \xcd"P".

\paragraph{Block distributions.}\index{distribution!block}
The distribution \xcd"block(R, Q)" distributes the elements of \xcd"R"
(in order) over the set of places \xcd"Q" in blocks  as
follows. Let $p$ equal \xcd"|R| div N" and $q$ equal \xcd"|R| mod N",
where \xcd"N" is the size of \xcd"Q", and 
\xcd"|R|" is the size of \xcd"R".  The first $q$ places get
successive blocks of size $(p+1)$ and the remaining places get blocks of
size $p$.

The distribution \xcd"block(R)" is the same distribution as {\cf
block(R, Place.places)}.

\oldtodo{Check into block distributions per dimension.}
\paragraph{Cyclic distributions.} \index{distribution!cyclic}
The distribution \xcd"cyclic(R, Q)" distributes the points in \xcd"R"
cyclically across places in \xcd"Q" in order.

The distribution \xcd"cyclic(R)" is the same distribution as \xcd"cyclic(R, Place.places)".

Thus the distribution \xcd"cyclic(Place.MAX_PLACES)" provides a 1--1
mapping from the region \xcd"Place.MAX_PLACES" to the set of all
places and is the same as the distribution \xcd"unique(Place.places)".

\paragraph{Block cyclic distributions.}\index{distribution!block cyclic}
The distribution \xcd"blockCyclic(R, N, Q)" distributes the elements
of \xcd"R" cyclically over the set of places \xcd"Q" in blocks of size
\xcd"N".

\paragraph{Arbitrary distributions.} \index{distribution!arbitrary}
The distribution \xcd"arbitrary(R,Q)" arbitrarily allocates points in {\cf
R} to \xcd"Q". As above, \xcd"arbitrary(R)" is the same distribution as
\xcd"arbitrary(R, Place.places)".

\oldtodo{Determine which other built-in distributions to provide.}

\paragraph{Domain Restriction.} \index{distribution!restriction!region}

If \xcd"D" is a distribution and \xcd"R" is a sub-region of {\cf
D.region}, then \xcd"D | R" represents the restriction of \xcd"D" to
\xcd"R".  The compiler throws an error if it cannot determine that
\xcd"R" is a sub-region of \xcd"D.region".

\paragraph{Range Restriction.}\index{distribution!restriction!range}

If \xcd"D" is a distribution and \xcd"P" a place expression, the term
\xcd"D | P" denotes the sub-distribution of \xcd"D" defined over all the
points in the region of \xcd"D" mapped to \xcd"P".

Note that \xcd"D | here" does not necessarily contain adjacent points
in \xcd"D.region". For instance, if \xcd"D" is a cyclic distribution,
\xcd"D | here" will typically contain points that are \xcd"P" apart,
where \xcd"P" is the number of places. An implementation may find a
way to still represent them in contiguous memory, e.g., using a
complex arithmetic function to map from the region index to an index
into the array.

\subsection{User-defined distributions}\index{distribution!user-defined}

Future versions of \Xten{} may provide user-defined distributions, in
a way that supports static reasoning.

\subsection{Operations on distributions}

A {\em sub-distribution}\index{sub-distribution} of \xcd"D" is
any distribution \xcd"E" defined on some subset of the region of
\xcd"D", which agrees with \xcd"D" on all points in its region.
We also say that \xcd"D" is a {\em super-distribution} of
\xcd"E". A distribution \xcdmath"D$_1$" {\em is larger than}
\xcdmath"D$_2$" if \xcdmath"D$_1$" is a super-distribution of
\xcdmath"D$_2$".

Let \xcdmath"D$_1$" and \xcdmath"D$_2$" be two distributions.  


\paragraph{Intersection of distributions.}\index{distribution!intersection}
\xcdmath"D$_1$ && D$_2$", the intersection of \xcdmath"D$_1$"
and \xcdmath"D$_2$", is the largest common sub-distribution of
\xcdmath"D$_1$" and \xcdmath"D$_2$".

\paragraph{Asymmetric union of distributions.}\index{distribution!union!asymmetric}
\xcdmath"D$_1$.overlay(D$_2$)", the asymmetric union of
\xcdmath"D$_1$" and \xcdmath"D$_2$", is the distribution whose
region is the union of the regions of \xcdmath"D$_1$" and
\xcdmath"D$_2$", and whose value at each point \xcd"p" in its
region is \xcdmath"D$_2$(p)" if \xcdmath"p" lies in
\xcdmath"D$_2$.region" otherwise it is \xcdmath"D$_1$(p)".
(\xcdmath"D$_1$" provides the defaults.)

\paragraph{Disjoint union of distributions.}\index{distribution!union!disjoint}
\xcdmath"D$_1$ || D$_2$", the disjoint union of \xcdmath"D$_1$"
and \xcdmath"D$_2$", is defined only if the regions of
\xcdmath"D$_1$" and \xcdmath"D$_2$" are disjoint. Its value is
\xcdmath"D$_1$.overlay(D$_2$)" (or equivalently
\xcdmath"D$_2$.overlay(D$_1$)".  (It is the least
super-distribution of \xcdmath"D$_1$" and \xcdmath"D$_2$".)

\paragraph{Difference of distributions.}\index{distribution!difference}
\xcdmath"D$_1$ - D$_2$" is the largest sub-distribution of
\xcdmath"D$_1$" whose region is disjoint from that of
\xcdmath"D$_2$".


\subsection{Example}
\begin{xten}
def dotProduct(a: Array[T](D), b: Array[T](D)): Array[Double](D) =
  (new Array[T]([1:D.places],
      (Point) => (new Array[T](D | here,
                    (i): Point) => a(i)*b(i)).sum())).sum();
\end{xten}

This code returns the inner product of two \xcd"T" vectors defined
over the same (otherwise unknown) distribution. The result is the sum
reduction of an array of \xcd"T" with one element at each place in the
range of \xcd"D". The value of this array at each point is the sum
reduction of the array formed by multiplying the corresponding
elements of \xcd"a" and \xcd"b" in the local sub-array at the current
place.



