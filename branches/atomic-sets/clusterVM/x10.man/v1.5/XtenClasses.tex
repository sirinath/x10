\chapter{Classes}
\label{XtenClasses}\index{class}

{}\Xten{} classes are essentially the same as \java{} classes \cite[\S
8]{jls2}. Classes are structured in a single-inheritance code
hierarchy, may implement multiple interfaces, may have static and
instance fields, may have static and instance methods, may have
constructors, may have static and instance initializers, may have
static and instance inner classes and interfaces. \Xten{} does not
permit mutable static state, so the role of static methods and
initializers is quite limited. Instead programmers should use
singleton classes to carry mutable static state.

Method signatures may specify checked exceptions. Method definitions
may be overridden by subclasses; the overriding definition may have a
declared return type that is a subclass of the return type of the
definition being overridden. Multiple methods with the same name but
different signatures may be provided on a class (ad hoc
polymorphism). The public/private/protected/package-protected access
modification framework may be used.

\todo{Add the new rule for preventing leakage of this from a constructor.}

Because of its different concurrency model, \Xten{} does not support
{\tt transient} and {\tt volatile} field modifiers.

\todo{Figure out class modifiers. Figure out which new ones need to be added to support IEEE FP.}

\section{Reference classes}\index{class!reference class}\label{ReferenceClasses}
A reference class is declared with the optional keyword {\tt
reference} preceding {\tt class} in a class declaration. Reference
class declarations may be used to construct reference types
(\S~\ref{ReferenceTypes}). Reference classes may have mutable
fields. Instances of a reference class are always created in a fixed
place and in \XtenCurrVer{} stay there for the lifetime of the
object. (Future versions of \Xten{} may support object migration.)
Variables declared at a reference type always store a reference to the
object, regardless of whether the object is local or remote.

\section{Value classes}\index{class!value class}\label{ValueClasses}

{}\Xten{} singles out a certain set of classes for additional
support. A class is said to be {\em stateless} if all of its fields
are declared to be {\cf final} (\S~\ref{FinalVariable}), otherwise it
is {\em stateful}. (\Xten{} has syntax for specifying an array class
with final fields, unlike \java{}.) A {\em stateless (stateful)
object} is an instance of a stateless (stateful) class.

{}\Xten{} allows the programmer to signify that a class (and all its
descendents) are stateless. Such a class is called a {\em value
class}.  The programmer specifies a value class by prefixing the
modifier {\cf value} before the keyword {\cf class} in a class
declaration.  (A class not declared to be a value class will be called
a {\em reference class}.)  Each instance field of a value class is
treated as {\cf final}. It is legal (but neither required nor recommended)
for fields in a value class to be declared final. For brevity, the \Xten{}
compiler allows the programmer to omit the keyword {\tt class} after
{\tt  value} in a value class declaration.

\begin{x10}
447  ClassDeclaration ::= ValueClassDeclaration
448  ValueClassDeclaration ::= 
       ClassModifiersopt value identifier Superopt 
          Interfacesopt ClassBody
449  | ClassModifiersopt value class identifier 
          Superopt Interfacesopt ClassBody
\end{x10}

The {\cf nullable} type-constructor (\S~\ref{NullableTypeConstructor}) can
be used to declare variables whose value may be {\cf null} or a value
type.

\cbstart
Stable equality for value types is defined through a deep walk,
bottoming out in fields of reference types (\S~\ref{StableEquality}).
\cbend

\paragraph{Static semantics.}
It is a compile-time error for a value class to inherit from a
stateful class or for a reference class to inherit from a value
class. All fields of a value class are implicitly declared {\tt
final}.

\subsection{Representation}

Since value objects do not contain any updatable locations, they can
be freely copied from place to place. An implementation may use
copying techniques even within a place to implement value types,
rather than references. This is transparent to the programmer.

More explicitly, \Xten{} guarantees that an implementation must always
behave as if a variable of a reference type takes up as much space as
needed to store a reference that is either null or is bound to an
object allocated on the (appropriate) heap. However, \Xten{} makes no
such guarantees about the representation of a variable of value
type. The implementation is free to behave as if the value is stored
``inline'', allocated on the heap (and a reference stored in the
variable) or use any other scheme (such as structure-sharing) it may
deem appropriate. Indeed, an implementation may even dynamically
change the representation of an object of a value type, or dynamically
use different representations for different instances (that is,
implement automatic box/unboxing of values).

Implementations are strongly encouraged to implement value types as
space-efficiently as possible (e.g.{} inlining them or passing them in
registers, as appropriate).  Implementations are expected to cache
values of remote final value variables by default. If a value is
large, the programmer may wish to consider spawning a remote activity
(at the place the value was created) rather than referencing the
containing variable (thus forcing it to be cached).

\todo{ Need to figure out whether we should let the programmer be
aware of lazy pull vs full-value push of value objects. This is the
idea of introducing a *-annotation. Need to make a decision on
this. Could leave this for 0.7.}

\subsection{Example}
\cbstart

A functional {\tt LinkedList} program may be written as follows:

\cbend

\begin{x10}
value LinkedList  \{ 
  Object first;
  nullable LinkedList rest;
  public
     LinkedList(Object first) \{
     this(first, null);
  \}
  public
    LinkedList(Object first,  
               nullable LinkedList rest) \{
    this.first = first;
    this.rest = rest;
  \}
  public 
    Object first() \{
    return first;
  \}
  public 
    nullable LinkedList rest() \{
    return rest;
  \} 
  public
    LinkedList append(LinkedList l) \{
    return (this.rest == null) 
        ? new LinkedList(this.first, l) 
        : this.rest.append(l);
  \}
\}
\end{x10}

Similarly, a {\tt Complex} class may be implemented as follows:
\begin{x10}
value Complex  \{ 
  double re;
  double im;
  public
     Complex(double re, double im) \{
     this.re=re;
     this.im=im;
  \}
  public Complex add(Complex other) \{
    return new Complex(this.re+other.re,
                       this.im+other.im);
  \}
  public Complex mult(Complex other) \{
    return new Complex(this.re^2-other.re^2,
                       2*this.im*other.im);
  \}
  ...
\}
\end{x10}


\section{Method annotations}

\subsection{{\tt atomic} annotation}
A method may be declared {\tt atomic}.
\begin{x10}
445   MethodModifier ::= atomic  
\end{x10}

Such a method is treated as if the statement in its body is wrapped 
implicitly in an {\tt atomic} statement.

\notfouro{
\subsection{{\tt local} annotation}\label{LocalAnnotation}\index{local!{\tt local}}
A method may be declared {\tt local}.
\begin{x10}
445   MethodModifier ::= local
\end{x10}

By declaring a method {\tt local} the programmer asserts that while
executing this method an activity will only access local memory.

The compiler implements the following rules to guarantee this property.

Let {\tt o} be any expression occurring in the body of the
method. Assume its static datatype is {\tt F}. 

\begin{itemize}
\item Local methods can only be overridden by local methods. 

\item If the body of the method contains any field access {\tt o.e}, then
the static placetype of {\tt o} must be {\tt here}. 

The programmer can always ensure that this condition is satisfied
(albeit at the risk of introducing a runtime exception) by replacing
each field access {\tt o.e} with {\tt ((F!here) o).e}.

\item If the body of the method contains any assignments to fields
(e.g. {\tt o.e Op= t}, or {\tt Op o.e} or {\tt o.e Op}) then the
static placetype of {\tt o} must be {\tt here}.

The programmer can always ensure that this condition is satisfied by
replacing {\tt o.e Op= t} by {\tt o1.e Op=t} and preceding it (in the
same basic block) with the local variable declaration {\tt F!here o1 =
(F!here) o} (for some new local variable {\tt o1}). Similarly for
{\tt Op o.e} and {\tt o.e Op}.

\item Recall that the static placetype of an array access {\tt o[e]}
is {\tt o.distribution[e]}. Therefore, any read/write array access
{\tt o[e]} must be guarded by the condition {\tt o.distribution[e] ==
here}.  (Since  {\tt e} may have side-effects, the compiler must
ensure that the place check uses the value returned by the same
expression evaluation that is used to access the array element.)

\item If the body of the method contains any method invocation {\tt
o.m(t1,...,tk)} then the method invoked must be local. Additionally,
the static place type of {\tt o} must be {\tt here}. 
As above, the programmer can always ensure the second
condition is satisfied by writing such a method invocation
as {\tt ((F!here) o).m(t1,...,tk)}.
\end{itemize}

Note that reads/writes to local variables or method parameters are
always local, hence the compiler does not have to check any extra
conditions.

A method declared {\tt atomic} is automatically declared
to be {\tt local}.
}
