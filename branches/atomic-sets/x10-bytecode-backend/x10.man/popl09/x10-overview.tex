
\Xten{} is a class-based object-oriented language.
that provides both dependent and generic types.
The language has a sequential core similar to Java or Scala, but
also
constructs for concurrency and distribution.
Like Java, the language provides single class
inheritance and multiple interface inheritance.
%\footnote{We plan
%to support traits in a future version of the language.}

To illustrate the features of \Xten{}, we
develop a simple \xcd"Vector"
class, shown in Figure~\ref{fig:vector}.
The \xcd"Vector" class has a type parameter \xcd"T" and an \xcd"int"
property \xcd"len" (line 1).
Classes in \Xten{} may be declared with any number of type
parameters and properties.

\xcd"Vector" has a single
immutable (final) \xcd"data" field (line 2), of type \xcd"Rail[T]"---a ``rail'' in \Xten is a zero-based, one-dimensional array similar to Java's \xcd"T[]".
The type of \xcd"data" is
has a constraint specifying that the length of
the \xcd"data" rail is the same as the vector length.
Note that \xcd"this" occurring
in the constraint refers to the instance of the enclosing
\xcd"Vector" class,
and \xcd"self" refers to the value being
constrained---\xcd"this.data" in this case.

\begin{figure}
{\footnotesize
\begin{numberedxten}
class Pair[X,Y](x: X, y: Y) {
  def this(xx: X, yy: Y): Pair[X,Y]{x==xx,y==yy} {
    super();
    property(xx, yy);
  }

  def reduce[Z](f: (X,Y)=>Z): Z = f(x,y);

  def get(){x==y} = x;

  def print(){X $\extends$ Printable, Y $\extends$ Printable} {
      x.print();
      Console.OUT.print(", ");
      y.print();
    }
  }
}
\end{numberedxten}}
\caption{A Vector class in \Xten}
\label{fig:vector}
\end{figure}

Structural typing with print.



Constraints may be used as \emph{class invariants}, 
which are constraints on the class's properties and type
parameters.  The invariant must be established by the class's
constructor and subsequently holds for all instances of the
class.
\xcd"Vector"'s class invariant (line 1, in curly braces)
specifies that the length of
the vector be non-negative.
The class invariant may also place constraints on the type
parameters, as discussed below.

Objects in \Xten{} are initialized with constructors, which
must ensure that all properties of the new object
are initialized and that the class invariants of the object's
class and its superclasses and superinterfaces hold.
\Xten{} uses the syntax \xcd"def" \xcd"this" for constructors.
In \Xten{}, constructors have a ``return type'', which constrains
the properties of the new object.  The constructor in
Figure~\ref{fig:vector} (lines 4--8) takes
two arguments---an integer \xcd"n", and a function from integers
to \xcd"T", \xcd"init"---and 
initializes the
object to have type \xcd"Vector[T]{len==n}".

The body of the constructor
invokes the superclass constructor through a \xcd"super" call.
In this case, the superclass is
\xcd"Object" and its constructor takes no arguments.
A \xcd"property" statement then initializes the
properties of the new instance.  All properties are initialized
simultaneously and it is required that the property assignment
entails the constraint in the constructor return type.
The remainder of the constructor assigns the fields of the
instance from the constructor arguments.

Methods are declared with the \xcd"def" keyword.
Methods in \Xten may have type parameters.  
For instance, the \xcd"map" method (lines 10--12)
has a type parameter \xcd"S" and a value parameter that is a
function from \xcd"T" to \xcd"S".
A parametrized method is invoked by giving type arguments before the
expression arguments, e.g., \xcd"v.map[int](f)".\footnote{In \Xten, actual type
arguments can be inferred from the types of the value arguments.  However, type
inference is out of the scope of this paper.}

The method \xcd"get" (lines 14--16) takes an integer argument \xcd"i"
and returns the element at that position.  The type of \xcd"i"
is constrained to be within the bounds of the vector. 

Method and constructors
may also have additional
constraints, or \emph{guards}, on their parameters.  A
guard must be satisfied by the caller of the method and
holds throughout the body.
For instance, 
\xcd"get"
has a guard that requires that the 
requires that the actual
receiver's
\xcd"len" property be positive---calls to \xcd"get" on empty
vectors are not permitted.
The guard may constrain not only the value and type parameters of
the method, but also other types in
scope at the declaration, including the enclosing class's type parameters.

Method guards can also constrain type parameters.
For instance, the \xcd"print" method (lines 18--24) can be invoked only if
the type parameter \xcd"T" is instantiated on a type that implements
\xcd"Printable".  Since the guard holds throughout the body of the
method, the \xcd"print" method can be invoked on each element of \xcd"data"---it
is guaranteed to implement \xcd"Printable".
This feature is similar to optional methods in CLU~\cite{clu} and to generlized type constraints in C$\sharp$~\cite{emir06}.

Method overriding is similar to Java: a method of a subclass
with the same name and parameter types overrides a method of the
superclass.  An overridden method may have a return type that is
a subtype of the superclass method's return type.
A method guard may be weakened by an overriding
method; that is, the guard in the superclass must entail the  
guard in the subclass.

Method overloading in \Xten is not constraint-sensitive.  It is
illegal for a class to contain two methods with the same name
and the same parameter types when constraints are erased from the 
types.  Constraints are also not evaluated during method
dispatch, as in predicate dispatch~\cite{jpred}.

\eat{
\xcd"List" also defines three constructors: the first
constructor takes no value arguments and initializes
the length to \xcd"0".  Note that \xcd"head" and \xcd"tail" are
not assigned since they are inaccessible.
The second constructor takes an argument for the head of the
list; the third takes both a head and tail.
}

\paragraph{Type constraints and variance}
\label{sec:variance-overview}

For generic types, the class invariant may also be used to provide 
bounds on the type parameters.
For example, in the following code,
\xcd"SortedList" 
specifies that the element type \xcd"T"
be a subtype of \xcd"Comparable[T]":
\begin{xtenmathnoindent}
 class SortedList[T] {T$\extends$Comparable[T]} extends List[T] {
   def sort() { ... x.compare(y) ... }
 }
\end{xtenmathnoindent}
Constraints can specify either subtype (\xcd"<="), supertype (\xcd">="),
or equality bounds (\xcd"==").

\Xten supports definition-site variance annotations.
Parameters may be declared invariant, covariant, or
contravariant.
If a parameter \xcd"X" of a class \xcd"C" is covariant,
then if \xcd"S" is a subtype of
\xcd"T", the type \xcd"C[S]" is a subtype of \xcd"C[T]".
Similarly, if \xcd"X" is contravariant, 
\xcd"C[T]" is a subtype of \xcd"C[S]".
Invariant parameters are the default; a covariant parameter is
declared by prepending ``\xcd"+"'' to the parameter name in the
class header; a contravariant parameter is declared by
prepending ``\xcd"-"''.

It is illegal for a covariant parameter to occur in a negative
position in its class declaration and for a contravariant
parameter to occur in a positive position.  A position is
negative if it is a formal parameter type, or occurs in a method
where clause.  A position is positive if it is a return type or
occurs in a method constraint.

