\chapter{Interfaces}
\label{XtenInterfaces}\index{interfaces}

{}\XtenCurrVer{} interfaces are generally modelled on Java interfaces \cite[\S
  9]{jls2}. An interface specifies signatures for public methods, properties,
\xcd`static val`s, and an invariant. It may extend several interfaces, giving
X10 a large fraction of the power of multiple inheritance at a tiny fraction
of the cost.

The following puny example illustrates all these features: 
%~~gen
%
%~~vis
\begin{xten}
interface Pushable(text:String, prio:Int) {
  def push(): Void;
  static val MAX_PRIO = 100;
}
class MessageButton(text:String, prio:Int) 
  implements Pushable{self.prio==Pushable.MAX_PRIO} {
  public def push() { 
    x10.io.Console.OUT.println(text + " pushed");
  }
}
\end{xten}
%~~siv
%
%~~neg
\noindent
\xcd`Pushable` defines two properties, a method, and a static value.  
\xcd`MessageButton` implements a constrained version of \xcd`Pushable`,
\viz\ one with maximum priority.  It also has \xcd`Pushable`'s properties.  It
defines the \xcd`push()` method given in the interface, as a \xcd`public`
method---interface methods are implicitly \xcd`public`.

A concrete type---a class or struct---can {\em implement} an interface,
typically by having all the methods and properties that the interface
requires.

A variable may be declared to be of interface type.  Such a variable has all
the fields and methods declared (directly or indirectly) by the interface;
nothing else is statically available.  Values of a concrete type which
implement the interface may be stored in the variable.  


\label{DepType:Interface}


\begin{grammar}
NormalInterfaceDeclaration \:
      InterfaceModifiers\opt \xcd"interface" Identifier  \\
   && TypePropertyList\opt PropertyList\opt Constraint\opt \\
   && ExtendsInterfaces\opt InterfaceBody \\
\end{grammar}
\noindent
The invariant associated with an interface is the conjunction of the
invariants associated with its superinterfaces and the invariant
defined at the interface. 

\begin{staticrule*}
The compiler declares an error if this constraint
is not consistent.  
\end{staticrule*}

Each interface implicitly defines a nullary getter method
\xcd"def p(): T" for each property \xcd"p: T". The interface may not have
another definition of a method \xcd`p()`. 



A class \xcd"C" is said to implement an interface \xcd"I" if
\begin{itemize}
\item \xcd`I`, or a subtype of \xcd`I`, appears in the \xcd`implements` list
      of \xcd`C`, 
\item \xcd`C`'s properties include all the properties of \xcd"I",
\item \xcd`C`'s class invariant $\mathit{inv}($\xcd"C"$)$ implies
$\mathit{inv}($\xcd"I"$)$.
\item Each method \xcd`m` defined by \xcd`I` is also a method of \xcd`C` --
      {\em with the \xcd`public`} modifier added.   These methods may be
      \xcd`abstract` if \xcd`C` is \xcd`abstract`.
\end{itemize}

\section{Field Definitions}

An interface may declare a \xcd`val` field, with a value.  This field is implicitly
\xcd`public static val`: 
%~~gen
% package Interface.Field;
%~~vis
\begin{xten}
interface KnowsPi {
  PI = 3.14159265358;
}
\end{xten}
%~~siv
%
%~~neg

Classes and structs implementing such an interface get the interface's fields as
\xcd`public static` fields.  Unlike properties and methods, there is no need
for the implementing class to declare them. 
%~~gen
% package Interface.Field.Two;
% interface KnowsPi {PI = 3.14159265358;}
%~~vis
\begin{xten}
class Circle implements KnowsPi {
  static def area(r:Double) = PI * r * r;
}
\end{xten}
%~~siv
%
%~~neg

\subsection{Fine Points of Fields}

It can happen that two parent interfaces give fields of the same name.  In
that case, those fields must be referred to by qualified names.
%~fails~gen
%
%~fails~vis
\begin{xten}
interface E1 {static val a = 1;}
interface E2 {static val a = 2;}
interface E3 extends E1, E2{}
class Example implements E3 {
  def example() = E1.a + E2.a;
}
\end{xten}
%~fails~siv
%
%~fails~neg

If the {\em same} field \xcd`a` is inherited through many paths, there is no need to
disambiguate it:
%~fails~gen
% package Interfaces.Mult.Inher.Field;
%~fails~vis
\begin{xten}
interface I1 { static val a = 1;} 
interface I2 extends I1 {}
interface I3 extends I1 {}
interface I4 extends I2,I3 {}
class Example implements I4 {
  def example() = a;
}
\end{xten}
%~fails~siv
%
%~fails~neg

\section{Interfaces Specifying Properties}

Interfaces may specify properties.  


\bard{Methods}
\bard{Properties}
\bard{Invariannt}


