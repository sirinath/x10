\chapter{Names and packages}
\label{packages} \index{names}\index{packages}\index{public}\index{protected}\index{private}

\Xten{} supports mechanisms for names and packages in the style of Java
\cite[\S 6,\S 7]{jls2}, including \xcd"public", \xcd"protected", \xcd"private"
and package-specific access control.

\section{Packages}

A package is a named collection of top-level type declarations, \viz, class,
interface, and struct declarations. Package names are sequences of
identifiers, like \xcd`x10.lang` and \xcd`com.ibm.museum`. The multiple names
are simply a convenience. Packages \xcd`a`, \xcd`a.b`, and \xcd`a.c` have only
a very tenuous relationship, despite the similarity of their names.

Packages and protection modifiers determine which top-level names can be used
where. Only the \xcd`public` members of package \xcd`pack.age` can be accessed
outside of \xcd`pack.age` itself.  
%~~gen~~Stimulus
%
%~~vis
\begin{xten}
package pack.age;
class Deal {
  public def make() {}
}
public class Stimulus {
  private def taxCut() = true;
  protected def benefits() = true;
  public def jobCreation() = true;
  /*package*/ def jumpstart() = true;
}
\end{xten}
%~~siv
%
%~~neg

The class \xcd`Stimulus` can be referred to from anywhere outside of
\xcd`pack.age` by its full name of \xcd`pack.age.Stimulus`, or can be imported
and referred to simply as \xcd`Stimulus`.  The public \xcd`jobCreation()`
method of a \xcd`Stimulus` can be referred to from anywhere as well; the other
methods have smaller visibility.  The non-\xcd`public` class \xcd`Deal` cannot
be used from outside of \xcd`pack.age`.  



\subsection{Name Collisions}

It is a static error for a package to have two members, or apparent members,
with the same name.  For example, package \xcd`pack.age` cannot define two
classes both named \xcd`Crash`, nor a class and an interface with that name.

Furthermore, \xcd`pack.age` cannot define a member \xcd`Crash` if there is
another package named \xcd`pack.age.Crash`, nor vice-versa. (This prohibition
is the only actual relationship between the two packages.)  This prevents the
ambiguity of whether \xcd`pack.age.Crash` refers to the class or the package.  
Note that the naming convention that package names are lower-case and package
members are capitalized prevents such collisions.


\section{\xcd`import` Declarations}

Any public member of a package can be referred to from anywhere through a
fully-qualified name: \xcd`pack.age.Stimulus`.    

Often, this is too awkward.  X10 has two ways to allow code outside of a class
to refer to the class by its short name (\xcd`Stimulus`): single-type imports
and on-demand imports.   

Imports of either kind appear at the start of the file, immediately after the
\xcd`package` directive if there is one; their scope is the whole file.

\subsection{Single-Type Import}

The declaration \xcd`import ` {\em TypeName} \xcd`;` imports a single type
into the current namespace.  The type it imports must be a fully-qualified
name of an extant type, and it must either be in the same package (in which
case the \xcd`import` is redundant) or be declared \xcd`public`.  

Furthermore, when importing \xcd`pack.age.T`, there must not be another type
named \xcd`T` at that point: neither a  \xcd`T` declared in \xcd`pack.age`,
nor a \xcd`inst.ant.T` imported from some other package.

\subsection{Automatic Import}

The automatic import \xcd`import pack.age.*;`, loosely, imports all the public
members of \xcd`pack.age`.  In fact, it does so somewhat carefully, avoiding
certain errors that could occur if it were done naively.  Types defined in the
current package, and those imported by single-type imports, shadow those
imported by automatic imports.  

\subsection{Implicit Imports}

The packages \xcd`x10.lang` and \xcd`x10.array` are imported in all files
without need for further specification.

%%BARD-HERE



\section{Conventions on Type Names}

\begin{grammar}
TypeName   \: Identifier \\
        \| TypeName \xcd"." Identifier \\
        \| PackageName \xcd"." Identifier \\
PackageName   \: Identifier \\
        \| PackageName \xcd"." Identifier \\
\end{grammar}


While not enforced by the compiler, classes and interfaces
in the \Xten{} library follow the following naming conventions.
Names of types---including classes,
type parameters, and types specified by type definitions---are in
CamelCase and begin with an uppercase letter.  (Type variables are often
single capital letters, such as \xcd`T`.)
For backward
compatibility with languages such as C and \java{}, type
definitions are provided to allow primitive types
such as \xcd"int" and \xcd"boolean" to be written in lowercase.
Names of methods, fields, value properties, and packages are in camelCase and
begin with a lowercase letter. 
Names of \xcd"static val" fields are in all uppercase with words
separated by `\xcd"_"''s.


