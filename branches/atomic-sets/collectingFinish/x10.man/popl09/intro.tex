% Problem statement
% X10 ... new language design
% OO, imperative, concurrent
% catch errors -> types
% dependent types, generics
% practical, pluggable

% @@@ Cut this.  Plagiarizes from OOPSLA now.
% Jump to motivation for dep types.
% Then motivation for generic types
% This work...

Modern architectural advances are leading to the development of
complex computational systems, such as heterogeneous multi-core,
hardware accelerators, and large CPU-count hybrid clusters.
The \Xten{} programming language \cite{X10-concur,x10-oopsla05,X10} was
designed to address the challenge of developing
high-performance applications for such machines, building on the
productivity gains of modern object-oriented languages.

\Xten{} requires a rich type system
to enable code reuse, rule out a large variety of errors at
compile-time, and to generate efficient code.
A central
data structure in \Xten{} is the dense, distributed, multi-dimensional
array.  Arrays are defined over a set of indices known as \emph{regions},
and support arbitrary base types and accesses through {\em
points} that must lie in the underlying region. For performance it is
necessary that array accesses are
bounds-checked statically as far as possible.
Further, certain regions (such as
polyhedral regions) may be represented particularly
efficiently.  Hence, if a variable is to range only over
polyhedral regions, it is important that this information be conveyed
statically (through the type system) to the code 
generator. To support $P$-way data parallelism it is often necessary
to logically partition an array into $P$ pieces. A type system that
can establish that a given division is a partition can ensure that no
race conditions arise due to simultaneous accesses by
multiple activities to different pieces.

\subsection{Constrained Types}

These requirements motivated us to develop a framework for dependent
types in object-oriented languages \cite{constrained-types}.
\emph{Dependent type
systems}~\cite{dependent-types,xi99dependent,calc-constructions} have
been extensively developed over the past few decades in the context of
logic and functional programming---they permit types to be
parametrized by \emph{values}.

The key idea behind our approach is to focus on the notion of a
\emph{constraint system}~\cite{cccc}. Constraint systems were originally developed
to provide a simple framework for a large variety of
inference systems used in programming languages, in particular as a
foundation for constraint programming languages.  Patterned after
Scott's information systems, a constraint system is
organized around the notion of \emph{constraints} or tokens of partial
information (e.g., \Xcd{x+y>z*3}), together with an entailment
relation $\vdash$.  Tokens may have first-order structure; existential
quantification is supported. The entailment relation is required to
support a certain set of inference rules arising from a Gentzen-style
formulation of intuitionistic logic.

In applying constraint systems to object-oriented
languages\footnote{The use of constraints for types has a
distinguished history going back to Mitchell~\cite{mitchell84}.
Our work is closely related to the \hmx{} approach
\cite{sulzmann97type}---see Section~\ref{sec:related} for
details.}, the principal insight
is that objects typically have some immutable state, and constraints on
this state are of interest to the application.  For instance, in \Java{}
the length of an array might not be statically known, but is fixed once
the array is created. Hence, we can enrich the notion of a type: for a
class \Xcd{C} we permit a \emph{constrained type} \Xcd{C\{c\}} where \Xcd{c}
is a
constraint on the immutable fields, or \emph{properties}, of the
class as well as any immutable variables and constants in
scope~\cite{constrained-types}.
Constraints are drawn from a constraint language that,
syntactically, is a subset of the boolean expressions of \Xten{}.  For
brevity, the constraint may be omitted and interpreted as \xcd"true".
Thus, 
\Xcd{Point\{self.rank==N\}} is a type satisfied by any
\Xcd{N}-dimensional point, that is, any
instance of \xcd"Point" whose \xcd"rank" property
is \Xcd{N}.  \Xcd{N}, here, is a (final) variable whose value may be unknown
statically. In a constraint, \Xcd{self} refers to 
a value of the base type being constrained, in this case \Xcd{Array}.
Subtyping is easily defined: a type \Xcd{C\{c\}} is a
subtype of \Xcd{D\{d\}} 
provided that \Xcd{C} is a subclass of \Xcd{D} and \Xcd{c}
entails \Xcd{d} in the underlying constraint system.

Constrained types maintain a phase distinction between compile time
(entailment checking in the underlying constraint system) and run time
(computation).  Dynamic type casting is permitted---code is generated
to check at run time that the properties of the given object satisfy
the given constraint. We note that while constraints may reference
only immutable state, a constrained type can classify mutable
variables. For instance the mutable field \Xcd{tail} of the class
\Xcd{List} may be typed with \Xcd{List\{self.length==N\}} and will be statically checked to ensure that at runtime it can only contain lists that satisfy the given property.


The constrained types approach enjoys many nice properties in contrast
to similar approaches such as DML~\cite{xi99dependent}.  Constrained
types are a natural extension to OO languages, and quite easy to
use. Constraints may also be used to specify class invariants, and
conditions on the accessibility of fields and methods (conditional
fields and methods).  Final variables in the computation can be used
directly in types; there is no need to define a separate
parallel language of index expressions to be used in the type system.
Constrained types always
permit field selection and equality at object types; hence the
programmer may specify constraints at any user-specified object type,
not just over the built-in constraint system.  

\subsection{Constrained Kinds}
The notion of \emph{generic types}---types such as \Xcd{List<T>} in
Java that are parametrized by other types---is now widely
established~\cite{clu,ada,GJ,java-popl97,thorup97,Java3,csharp-generics}.
Generic types are vital for implementing type-safe, reusable
libraries, especially collections classes.  For instance, the data
type \Xcd{Array} discussed above is generic on its element type.

This paper lays out a framework to extend constrained types to handle
(type-)genericity. The general outlines of the approach seem clear
enough. First, some mechanism must be found to introduce {\em type
variables}, for instance as {\em type parameters} (cf.
Java~\cite{Java3}) or as
{\em type members} (cf. BETA~\cite{beta}). 

Second, a suitable vocabulary of constraints over types must be
chosen. This raises the fundamental question: What is the structure of
types in (nominal) OO languages?  The answer to this is fairly
clear. A type is a {\em name} (e.g., class name or interface name),
there is a partial order (the {\em inheritance} relation) on such
names, and an association of (typed) members (fields, methods) to
these names. This structure raises some natural candidates for
constraints: e.g., \xcd"X <= T" (realized by any type \xcd"S" that
inherits from \xcd"T"), \xcd"X" \xcdmath"$\uhas$"
\xcdmath"m(T$_1$,$\ldots$,T$_n$):T", (realized by
any type \xcd"S" that has a method named \xcd"m" with the given
signature), \xcd"x" \xcd"instanceof" \xcd"X" (realized by any type
\xcd"S" to which the final variable 
\xcd"x" can be cast), etc. 

Third, these constraints can be used to specify {\em constrained
kinds} in the same way that constraints over values can be used to
specify {\em constrained types}. Thus \xcd"X:*{self <= Printable}"
would declare a type variable \xcd"X" which could only be bound to
those types \xcd"S" which satisfy \xcd"S <= Printable".

Fourth, a type variable could then be declared as a (kinded) class
property, field, or method parameter. Within the scope of its
declaration, the type variable can be used wherever a type can (e.g.,
to specify the type of method parameters and return values, local
variables, fields, as targets of cast etc).

While the outline is clear, a number of important issues remain untouched:
\begin{itemize}
\item What is the difference in expressiveness between type parameters and type properties? 

\item How do constraints on types interact with constraints on values?

\item Should constrained kinds be used to differentiate overloaded methods? During dynamic method lookup?

\item Can constrained kinds be used to express variance declarations (e.g., contravariance, covariance, invariance)? Java ``wildcards''?

\item How are constrained kinds implemented? What kind of runtime representation of types is needed, and how does it interact with the constraint solver?
\end{itemize}

\subsection{Contributions}

This paper develops the framework of constrained kinds. In the next
section we address answers to the questions above, and illustrate with
programming examples. We then motivate the choices made in the design
of the \Xten{} language, and outline the implementation strategy. An
implementation of \Xten{} (realizing these features) can be downloaded
from \xcd"x10-lang.org". 

We present a formalization of these ideas through an extension of the
development in \cite{constrained-types}. We present a core calculus,
\FXG{}, parametrized on an underlying data constraint system. The
calculus can express \Xcd{X <= T} constraints on type variables. We
present subject reduction and type soundness properties for the
calculus. This calculus embeds the calculus presented in
\cite{constrained-types}. 
\todo{What can we say about relationship with FJG.}

We show how the framework can be extended in a simple fashion to
handle structural typing constraints.

We summarize our contributions:
\begin{itemize}

\item We extend the notion of constrained types \cite{constrained-types} with {\em constrained kinds}. These permit type variables to be classified based on constraints over types and values. 
\item We explore the design space for an OO language with constrained kinds and types.
\item We formalize a core ``featherweight'' calculus for constrained kinds, permitting subtype constraints, conservatively extending \cite{constrained-types} and establish subject reduction and type soundness.
\item 
We show how the framework can be extended in a simple fashion by adding other constraints on types (and associated typing rules).
\end{itemize}
\todo{Revisit once the proof is entered, need to highlight something key in the lemma.}

\paragraph{Outline.}

The rest of the paper is organized as follows.
%
Design considerations for generics in \Xten{}
and related work are
discussed in Section~\ref{sec:lang}.
%
Section~\ref{sec:semantics} presents a formal semantics and a
proof of soundness.
%\todo{implementation}
%
Finally, Section~\ref{sec:conclusions} concludes.


