\subsection{Array types}
\label{ArrayTypeConstructors}\index{array types}
\index{\Xcd{Array}}
\index{\Xcd{ValArray}}
\index{\Xcd{x10.lang.Array}}

Arrays in \Xten{} are instances of the class
\xcd"x10.lang.Array".
Because of the importance of arrays in \Xten{}, the language
supports more concise syntax for accessing array elements and
performing operations on arrays.

% The array type constructor takes as argument a type (the {\em base
% type}), an optional distribution (\Sref{XtenDistributions}), and
% optionally the keyword \xcd"value":

The array type \xcd"Array[T]" is the type of all
reference arrays of base type \xcd"T". Such an array can take on any
distribution, over any region. 

\eat{
The array type \xcd"ValArray[T]" specifies the type of all
values arrays of base type \xcd"T".
The array elements of a \xcd"ValArray" are
all final.\footnote{Note that the base type of a
\xcd"ValArray" can be a value class or a reference class, just as the 
type of a final variable can be a value class or a reference class.}
}

The array class implements the function type
\xcd"(Point) => T"; the element of array \xcd"A" at point
\xcd"p" may be accessed using the 
syntax \xcd"A(p)".  The \xcd"Array" class 
also implements the \xcd"Settable[Point,T]" interface 
permitting assignment to an array element using the syntax
\xcd"A(p) = v".

\Xten{} also supports dependent types for arrays,
e.g.,
\xcd"Array[Double]{rank==3}" is the type of all arrays of 
\xcd"Double" of rank \xcd"3".
The \xcd"Array" class has distribution, region, and rank
properties. 
\XtenCurrVer{} defines type definitions that allow a distribution,
region, or rank to be specified with on the array type.

\begin{xten}
package x10.lang;
type Array[T](n: Int) = Array[T]{rank==n};
type Array[T](d: Dist) = Array[T]{dist==d};
type Array[T](r: Region) = Array[T]{region==r};
\end{xten}

\subsection{Rails}

A \emph{rail} is a one-dimensional, zero-based, local array. 
It is more primitive than the \xcd"Array" class.
Rails are indexed by integers rather than multi-dimensional
points.  Rails have a single \xcd"length" property of type
\xcd"Int".  Rails can be mutable or immutable and are defined
by the following class definitions:

\begin{xten}
package x10.lang;
public value class ValRail[T](length: Int) extends (Int)=>T { }
public class Rail[T](length: Int) extends (Int)=>T, Settable[Int,T] { }
\end{xten}

\Xten{} supports shorthand syntax for rail construction
(\Sref{RailConstructors}).

