\section{Linking with native code}\label{extern}\index{extern}
\XtenCurrVer{} supports a simple facility to permit the efficient
intra-thread communication of an array of primitive type to code
written in the language {\tt C}.  The array must be a ``local''
array. The primary intent of this design is to permit the reuse of
native code that efficiently implements some numeric array/matrix
calculation.

Future language releases are expected to support similar bindings to
{\sc Fortran}, and to support parallel native processing of
distributed \Xten{} arrays. 

The interface consists of two parts. First, an array intended to be
communicated to native code must be created as an {\tt unsafe} array:
\begin{x10}
450 ArrayCreationExpression ::= 
      new ArrayBaseType Unsafeopt [ ] 
        ArrayInitializer
451   | new ArrayBaseType Unsafeopt [ Expression ]
452   | new ArrayBaseType Unsafeopt 
          [ Expression ] Expression
453   | new ArrayBaseType Unsafeopt [ Expression ] 
          ( FormalParameter ) MethodBody
454   | new ArrayBaseType value 
           Unsafeopt [ Expression ]
455   | new ArrayBaseType value 
           Unsafeopt [ Expression ] Expression
456   | new ArrayBaseType value 
        Unsafeopt [ Expression ] 
          ( FormalParameter ) MethodBody
530   Unsafeopt ::=
531     | unsafe
\end{x10}
Unsafe arrays can be of any dimension. However, \XtenCurrVer{}
requires that unsafe arrays be of a primitive type, and local (i.e.{}
with an underlying distribution that maps all elements in its region
to {\tt here}).

Unsafe arrays are allocated in a special array of memory that permits
their efficient transmission to natively linked code.
%% Comment about when this memory is freed.

Second, the \Xten{} programmer may specify that certain methods are to
be implemented natively by using the keyword {\tt extern}:
\begin{x10}
446   MethodModifier ::= extern
\end{x10}
Such a method must have the statement ``{\tt ;}'' as its body.
\XtenCurrVer{} requires that the method be {\tt static}; this
restriction is likely to be lifted in the future.  Primitive types in
the method argument are translated to their corresponding JNI type
(e.g.{} {\tt float} is translated to {\tt jfloat}, {\tt double} to
{\tt jdouble} etc).  The only non-primitive type permitted in an {\tt
extern} method is an (unsafe) array. This is passed at type {\tt
jlong} as an eight byte address into the unsafe region which contains
the data for the array. ({\tt jlong} is not the same as {\tt long} on
32-bit machines.)


Since only the starting address of an array is passed, if the array is
multidimensional, the user must explicitly communicate (or have a
guarantee of) the rank of the passed array, and must either typecast
or explicitly code the address calculation.  Note that all \Xten{}
arrays are created in row-major order, and so any native routine must
also access them in the same order.

For each class {\tt C} that contains an {\tt extern} method, the
\Xten{} compiler generates a text file {\tt C\_x10stub.c}.  This file
contains generated {\tt C} stub functions which are called from the
{\tt extern} routines.  The name of the stub function is derived from
the name of the {\tt extern} method. If the method is {\tt
C.process()}, the stub function will be {\tt
Java\_C\_C\_process()}. The name is suffixed with the signature of the
method if the method is overloaded.

The programmer must write {\tt C} code to implement the native method,
using the methods in the {\tt C} stub file to call the actual native
method.  The programmer must compile these files and link them into a
dynamically linked library (DLL).  Note that the {\tt jni.h} header file
must be in the include path.  The programmer must ensure this library
is loaded by the program before the method is called e.g.{} add a {\tt
System.loadlibrary} call (in a static initializer of the
\Xten{} class).

\paragraph{Example.}
The following class illustrates the use of {\tt unsafe} and native
linking. 
\begin{x10}
public class IntArrayExternUnsafe \{
  public static extern 
      void process(int [.] yy, int size);
  static {System.loadLibrary("IntArrayExternUnsafe");}
  public static void main(String args[]) \{
     boolean b= (new IntArrayExternUnsafe()).run();
     System.out.println("++++++ Test "
                         +(b?"succeeded.":"failed."));
     System.exit(b?0:1);
  \}
  public boolean run()\{
    int high = 10;
    boolean verified=false;
    distribution d= (0:high) -> here;
    int [.] y = new int unsafe[d]; 
    for( int j=0;j < 10;++j)
        y[j] = j;
    process(y,high);
    for(int j=0;j < 10;++j)\{
      int expected = j+100;
      if(y[j] != expected)\{
        System.out.println("y["+j+"]="
                           +y[j]+" != "+expected);
        return false;
       \}
    \}
    return true;
  \}
\}
\end{x10}

The programmer may then write the {\tt C} code thus:
\begin{x10}
void IntArrayExternUnsafe\_process(jlong yy, 
                                signed int size)\{
  int i;
  int* array = (int *)(long)yy;
  for(i = 0;i < size;++i)\{
    array[i] += 100;
  \}
\}
/* automatically generated in \_x10stub.c*/
void 
 Java\_IntArrayExternUnsafe\_IntArrayExternUnsafe\_process
 (JNIEnv *env,  jobject obj,jlong yy,jint size)\{
   IntArrayExternUnsafe\_process(yy,size);
\}
\end{x10}

This code may be linked with the stub file (or textually placed in
it). The programmer must then compile and link the {\tt C} code and
ensure that the DLL is on the appropriate classpath. 

