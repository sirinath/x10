In this section we formalize a small fragment of \Xten{}, \CFJ---an
extension of Featherweight Java (\FJ{})~\cite{FJ} with
constrained types---to illustrate the basic concepts behind
constrained type-checking.  A proof of soundness is given in the
appendix.

The language is functional in that assignment is not admitted.
However, it is not difficult to introduce the notion of mutable
fields, and assignment to such fields. Since constrained types
may only refer to immutable state, the validity of these types
is not compromised by the introduction of state.
%
Further, we do not formalize overloading of methods. Rather, as
with \FJ{}, we simply require that the input program be such
that the class name {\tt C} and method name {\tt m} uniquely
select the associated method on the class.

The language is defined over a constraint system $\cal C$
that includes equality constraints over final access paths,
conjunction, existential quantification, and a vocabulary of
formulas and predicates.

\subsection{The Object constraint system}\label{sec:O}

\newcommand\vdashO{\vdash_{\cal O}}

Given a program $P$, we now show how to build a larger
constraint system ${\cal O}({\cal C})$
on top of $\cal C$ which
captures constraints related to the object-oriented structure of
$P$.
$\cal O$ includes the inference rules shown in
Figure~\ref{fig:O} for structual and subtyping constraints.
In addition,
$\cal O(\cal C)$ subsumes $\cal C$ in that if
$\Gamma \vdash_{\cal C} {\tt c}$
then
$\Gamma \vdash_{\cal O} {\tt c}$.

The constraint $\klass({\tt C})$ is intended to be true for all
classes {\tt C} defined in the program.  
For a
variable {\tt x}, {\tt \fields({\tt x})} is intended to specify the
(complete) set of typed fields available to {\tt x}.  {\tt x}~\has~{\tt I}
is intended to specify that the member {\tt I} (field or method) is
available to {\tt x}---for instance it is defined at the class at
which {\tt x} is declared or inherited by it, or it is available at
the upper bound of a type variable.
%
The judgment 
$\Gamma \vdashO {\tt S} \subtype {\tt T}$
is intended to hold if {\tt S} is a subtype of {\tt T} in the
environment $\Gamma$.

We assume that the rules given are complete for
defining the predicates $\tt C \subtype D$ and {\tt C\ \has\ I}, for
classes {\tt C}, {\tt D} and members {\tt I}; that is, if the rules
cannot be used to establish $\tt \vdashO C \subtype D$ ($\tt
\vdashO C\ \has\ I$), then it is the case that $\tt
\vdashO C \notsubtype D$ ($\tt \vdashO \neg (C\ \has\ I)$).
%
Such negative facts are important to establish {\em inconsistency} of
assumptions (for instance, for the programming languages which permits
the user to state constraints on type variables).  

\begin{figure*}
{\footnotesize
\tabcolsep=0pt
{\bf Structural constraints:}\\[-12pt]
\begin{tabular}{p{0.30\textwidth}p{0.23\textwidth}p{0.22\textwidth}p{0.24\textwidth}}
\infrule[Class]
	{{\tt class\ C(\ldots)\ extends\ D\{\ldots\}}\in P}
	{\tt \vdashO \klass({\tt C})}
&
\infax[Sel]{\tt \vdashO new\ D(\bar{\tt t}).{\tt f}_i{\tt ==}{\tt t_i}}
&
\infrule[Inv]
{\Gamma \vdashO {\tt x:C},\klass({\tt C})}
{\Gamma \vdashO \inv({\tt C}, {\tt x})}
&
\infrule[Field]
	{\Gamma \vdashO \fields({\tt x})=\bar{\tt f}:\bar{\tt T}} 
	{\Gamma \vdashO {\tt x}\ \has\ {\tt f}_i:{\tt T}_i}
\\
\end{tabular}
\\[-12pt]
\begin{tabular}{p{0.30\textwidth}p{0.33\textwidth}p{0.36\textwidth}}
\infax[Fields-B]{{\tt x}:{\tt Object} \vdashO \fields({\tt x})=\bullet}
&
\infrule[Fields-I]
	{\Gamma, {\tt x:D} \vdashO \fields({\tt x})=\bar{\tt g}:\bar{\tt V}\andalso \\
	  \klass\ {\tt C}(\bar{\tt f}:\bar{\tt U}){\tt \{c\}}\ {\tt extends}\ {\tt D}\{\bar{\tt M}\} \in {\tt C}}
	{\Gamma, {\tt x:C} \vdashO \fields({\tt x})=\bar{\tt g}:\bar{\tt V},\bar{\tt f}:\bar{\tt U[x/\this]}}
&
\infrule[Fields-C,E]
	{\Gamma, {\tt x: S} \vdashO \fields({\tt x})=\bar{\tt f}:\bar{\tt V}}
	{\Gamma, {\tt x:S\{d\}} \vdashO \fields({\tt x})=\bar{\tt f}:\bar{\tt V\{d[x/\self]\}} \\
	  \Gamma, {\tt x: (y:U;S)} \vdashO \fields({\tt x})=\bar{\tt f}:\bar{\tt (y:U;V)}
	} \\[-12pt]
\infrule[Method-B]
	{\Gamma, {\tt x}:{\tt C}\vdashO\klass({\tt C})\andalso \theta=[{\tt x}/\this] \\
	  {\tt def}\ {\tt m(\bar{\tt z}:\bar{\tt V})\{c\}:T=e}\in P}
	{\Gamma\tt, {\tt x:C}\vdashO {\tt x}\ \has\ (m(\bar{\tt z}:\bar{\tt V\theta})\{c\theta\}:T\theta=e)}
&
\infrule[Method-I]
	{\Gamma\tt, {\tt x}:{\tt D}\vdashO{\tt x}\ \has\ m(\bar{\tt z}:\bar{\tt V}){\tt \{c\}:T=e} \\
	  \tt \klass\ {\tt C}(\ldots)\ {\tt extends}\ {\tt D}\{\bar{\tt M}\} \andalso {\tt m}\not\in \bar{\tt M}}
	{\Gamma\tt, {\tt x:C}\vdashO {\tt x}\ \has\ {\tt m}(\bar{\tt z}:\bar{\tt V}){\tt \{c\}:T=e}}
&
\infrule[Method-C,E]
	{\Gamma\tt, x:S \vdashO x\ \has\ m(\bar{\tt z}:\bar{\tt V})\{c\}:T=e} 
	{\Gamma\tt, x:S\{d\} \vdashO x\ \has\ m(\bar{\tt z}:\bar{\tt V})\{c\}:T\{d[x/\self]\}=e \\
	  \Gamma\tt, x: (y:U;S) \vdashO x\ \has\ m(\bar{\tt z}:\bar{\tt V})\{c\}:(y:U;T)=e}
\end{tabular}
{\bf Subtyping:}\\[-12pt]
\begin{tabular}{p{0.16\textwidth}p{0.25\textwidth}p{0.33\textwidth}p{0.23\textwidth}}
\infax[S-Id]{\vdashO {\tt T} \subtype {\tt T}} 
&
\infrule[S-Trans]
	{\Gamma\tt \vdashO T_1 \subtype T_2, T_2 \subtype T_3}
	{\Gamma\tt \vdashO T_1 \subtype T_3}
&
\infrule[S-Extends]
	{{\tt class\ C(\ldots)\ extends\ D\{\ldots\}}\in P}
	{\tt\vdashO C \subtype D}
&
\infrule[S-Const-L]     %%% XXX changed from POPL, was : G, c |- S <: T  ==>  G |- S{c} <: T
                        %%% XXX couldn't get constraint-lemma to go thru with that rule
	{\Gamma \vdash {\tt T}\{{\tt c}\}~\type}
	{\Gamma\tt  \vdashO {\tt T\{c\}}\subtype {\tt T}}
\end{tabular}
\quad\\[-12pt]
\begin{tabular}{p{0.32\textwidth}p{0.38\textwidth}p{0.31\textwidth}}
\infrule[S-Const-R]
	{\Gamma\tt  \vdashO {\tt S}\subtype {\tt T}
        \andalso
        \Gamma,{\tt \self:S} \vdashO {\tt c} }
	{\Gamma\tt  \vdashO {\tt S}\subtype {\tt T\{c\}}}
&
\infrule[S-Exists-L]
	{\Gamma\tt  \vdash {\tt U}\ \type \andalso  
	  \Gamma \vdashO {\tt S} \subtype {\tt T}
          \andalso
          \mbox{({\tt x} fresh})}
	{\Gamma\tt  \vdashO {\tt x:U; S} \subtype {\tt T}}
&
\infrule[S-Exists-R]
	{\Gamma\tt  \vdash t:{\tt U} \andalso \Gamma \vdashO {\tt S}\; \subtype\; {\tt T}[{\tt t}/{\tt x}]}
	{\Gamma\tt  \vdashO {\tt S} \subtype {\tt x:U; T}}
\end{tabular}}

For a class {\tt C} and variable {\tt x}, ${\tt \inv(C,x)}$ stands for the conjunction of class invariants for {\tt C} and its supertypes, with {\tt this} replaced by {\tt x}.
\caption{The Object constraint system, ${\cal O}$}\label{fig:O}  
\end{figure*}


\begin{figure*}
\footnotesize
\begin{tabular}{l}
{\bf \FX{} productions:} \\
\quad\\
\begin{tabular}{ll}
\begin{tabular}{r@{\quad}r@{~~}c@{~~}l}
  (Class) & {\tt L} &{::=}& \klass\ {\tt C}($\bar{\tt f}:\bar{\tt T}$)\{{\tt c}\}\ {\tt extends}\ {\tt N}\ \{$\bar{\tt M}$\} \\
  (Method)& {\tt M} &{::=}& {\tt def}\ {\tt m}($\bar{\tt x}:\bar{\tt T}$){\tt \{c\}:T\,=\,e;}\\
  (Exp.)& {\tt e} &{::=}& {\tt x} \alt \this \alt {\tt e.f} \alt {\tt e.m}($\bar{\tt e}$) 
   \alt \new\ {\tt C}($\bar{\tt e}$) \alt {\tt e}\ \mbox{{\tt as} {\tt T}} \\ 
\end{tabular} 
&
\begin{tabular}{r@{\quad}r@{~~}c@{~~}l}
  (Type)& {\tt S},{\tt T},{\tt U},{\tt V}&{::=}& {\tt N}
        \alt {\tt T\{c\}} \alt {\tt x}:{\tt S};\,{\tt T}\\
  (N Type) & {\tt N}&{::=}& {\tt C} \alt {\tt N\{c\}}\\
  (C Term) & {\tt t} &{::=}& {\tt x} \alt \self \alt \this \alt
  {\tt t}.{\tt f} \alt \new\ {\tt C}($\bar{\tt t}$) \alt $\tt f(\bar{\tt t})$ \\
  (Const.) & {\tt c},{\tt d} &{::=}&\true \alt {\tt t==t} \alt {\tt c},{\tt c} \alt {\tt x}:{\tt T};\,{\tt c} \alt ${\tt q(\bar{\tt t})}$
\end{tabular} 
\end{tabular}
\quad \\
\quad\\
{\bf Constraint well-formedness rules:}\\[-12pt]
{\tabcolsep=0pt
\begin{tabular}{p{0.26\textwidth}p{0.33\textwidth}p{0.38\textwidth}}
\infax[True]{\Gamma \vdash {\tt true}: {\tt o}}
&
\infrule[And]
	{\Gamma\tt  \vdash c_0: o \andalso \Gamma \vdash c_1: o}
	{\Gamma\tt \vdash (c_0,c_1):o}
&
\infrule[Exists]
	{\Gamma\tt \vdash t: T \andalso \Gamma \vdash c[t/x]:o}
	{\Gamma\tt \vdash x:T;c: o}
\\
\infrule[Pred]
        {\tt q(\bar{\tt T}):{\tt o}\in {\cal C} \andalso \Gamma \vdash \bar{\tt t}:\bar{\tt T}}
        {\Gamma \vdash \tt q(\bar{\tt T}):o}
&
\infrule[Fun]
        {\tt f(\bar{\tt T}):{\tt T}\in {\cal C} \andalso \Gamma \vdash \bar{\tt t}:\bar{\tt T}}
        {\Gamma\tt \vdash \tt f(\bar{\tt T}):T}
&
\infrule[Equals]
        {\Gamma \vdash {\tt t}_0: {\tt T}_0
        \andalso
        \Gamma \vdash {\tt t}_1: {\tt T}_1 \\
          (\Gamma \vdash {\tt T}_0 \subtype {\tt T}_1 \vee
          \Gamma \vdash {\tt T}_1 \subtype {\tt T}_0)}
        {\Gamma \vdash {\tt t}_0{\tt ==}{\tt t}_1:o}
\\
\end{tabular}}
%\quad\\[-12pt]
\quad\\
{\bf Type well-formedness rules:}\\[-12pt]
{\tabcolsep=0pt
\begin{tabular}{p{0.26\textwidth}p{0.33\textwidth}p{0.38\textwidth}}
\infrule[Class]
	{\Gamma \vdash \klass({\tt C})}
	{\Gamma \vdash {\tt C}\ \type} 
&
\infrule[Exists-T]
	{\Gamma\tt \vdash {\tt S}\ \type, {\tt T}\ \type}
	{\Gamma\tt \vdash {\tt x:S;T}\ {\tt type}} 
&
\infrule[Dep]
	{\Gamma\tt \vdash T\ \type \andalso \Gamma, \self:T \vdash c:o}
	{\Gamma\tt \vdash T\{c\}\ {\tt type}} 
\\
\end{tabular}}
\quad\\
\quad\\
{\bf Type judgment rules:}\\[-12pt]
{\tabcolsep=0pt
\begin{tabular}{l}
\begin{tabular}{p{0.27\textwidth}p{0.28\textwidth}p{0.42\textwidth}}
\infax[T-Var]
      {\Gamma, {\tt x}:{\tt T} \vdash {\tt x}:{\tt T\{\self==x\}}}
&
\infrule[T-Cast]
	{\Gamma \vdash {\tt e}:{\tt U} \andalso \Gamma \vdash {\tt T} \ \type}
	{\Gamma \vdash {\tt e}\ \as\ {\tt T}: {\tt T}} &
\infrule[T-Field]
	{\Gamma \vdash {\tt e}: {\tt S} \andalso \Gamma,{\tt z}:{\tt S}\vdash {\tt z}\ \has\ {\tt f}:{\tt T} \andalso \mbox{({\tt z} fresh)} }
	{\Gamma \vdash {\tt e}.{\tt f}:  ({\tt z}:{\tt S};\,{\tt T\{\self==z.f\}})}
\end{tabular}\\[-12pt]
\begin{tabular}{p{0.50\textwidth}p{0.47\textwidth}} 
\infrule[T-Invk]
	{\Gamma \vdash {\tt e}:{\tt T},\bar{\tt e}:\bar{\tt V} \andalso \\
	  \Gamma,{\tt z}:{\tt T},\bar{\tt z}:\bar{\tt V} \vdash
		{\tt z}\ \has\ ({\tt m}(\bar{\tt z}:\bar{\tt U}), {\tt c} \rightarrow {\tt S}), 
		\bar{\tt V} \subtype \bar{\tt U},{\tt c} 
	  \andalso \mbox{({\tt z},$\bar{\tt z}$ fresh)}}
	{\Gamma \vdash {\tt e}.{\tt m}(\bar{\tt e}):
                ({\tt z}:{\tt T};\,\bar{\tt z}:\bar{\tt V};\,{\tt S})}
	&
\infrule[T-New]
	{\Gamma \vdash \bar{\tt e}:\bar{\tt T} \andalso \vdash \klass({\tt C}) \\ 
	  \Gamma,{\tt z}:{\tt C}\vdash \fields({\tt z})=\bar{\tt f}:\bar{\tt U}  \andalso \mbox{({\tt z},$\bar{\tt z}$\ fresh)}\\
	  \Gamma, {\tt z}:{\tt C}, \bar{\tt z}:\bar{\tt T}, {\tt z}.\bar{\tt f}=\bar{\tt z} 
	  \vdash \bar{\tt T} \subtype \bar{\tt U}, \inv({\tt C},{\tt z})}
	{\Gamma \vdash \new\ {\tt C}(\bar{\tt e}): {\tt C}\{\bar{\tt z}:\bar{\tt T};\,\new\ {\tt C}(\bar{\tt z})=\self, \inv({\tt C},\self)\}}
\\[-12pt]
\infrule[Method OK]
	{\this:{\tt C}\vdash {\tt c}:o \andalso \this:{\tt C},\bar{\tt x}:\bar{\tt U},{\tt c} \vdash {\tt T} \ \type, \bar{\tt U} \ \type, 
	  {\tt e}:{\tt S}, {\tt S} \subtype {\tt T}}
	{{\tt def}\ {\tt m}(\bar{\tt x}:\bar{\tt U})\{{\tt c}\}:{\tt T}= {\tt e};\ \mbox{OK in}\ C}
	&
\infrule[Class OK]
	{\bar{\tt M}\ \mbox{OK in}\ {\tt C}
        \andalso
        \this:{\tt C} \vdash {\tt c}:{\tt o}
        \andalso
        \this:{\tt C},{\tt c} \vdash \bar{\tt T}\ \type, {\tt N}\ \type}
	{\mbox{\tt class}\ {\tt C}(\bar{\tt f}:\bar{\tt T})\{{\tt c}\}\ 
	  \mbox{\tt extends}\ {\tt N}\{\bar{\tt M}\} \ \mbox{OK}}
\end{tabular}
\end{tabular}}\\[-12pt]
\quad\\
{\bf Transition rules:}\\[-12pt]
{\tabcolsep=0pt
\begin{tabular}{p{0.43\textwidth}@{\quad}p{0.54\textwidth}}
\typicallabel{RC-Field}
\infrule[\RField]
	{\tt x:C \vdash \fields(\tt x)=\bar{\tt f}:\bar{\tt T}}
	{(\new\ {\tt C}(\bar{\tt e})).{\tt f_i} \derives {\tt e_i}}

\infrule[\RCField]%
	{{\tt e} \derives {{\tt e}}'}
	{{\tt e}.{\tt f}_i \derives {{\tt e}}'.{\tt f}_i}

\infrule[\RCast]%
	{\vdash {\tt C}\{\self==\new\ {\tt C}(\bar{\tt d})\} \subtype {\tt T}}
	{{\tt \new\ C(\bar{\tt d})~as~T} \derives \new\ C(\bar{\tt d})}

\infrule[\RCCast]%
	{{\tt e} \derives {{\tt e}}'}
	{{\tt e~as~T} \derives {\tt e}'~{\tt as~T}}
	&
	\typicallabel{RC-Invk-Recv}
	\infrule[\RInvk]
		{{\tt x:C}\vdash {\tt x}\ \has\ {\tt m}(\bar{\tt x}:\bar{\tt T}){\tt \{c\}:T=e}}
		{(\new\ {\tt C}(\bar{\tt e})).{\tt m}(\bar{\tt d}) \derives {\tt e[\new\ {\tt C}(\bar{\tt e}),\bar{\tt d}/\this,\bar{\tt x}]}}

	\infrule[\RCInvkRecv]%
		{{\tt e} \derives {{\tt e}}'}
		{{\tt e}.{\tt m}(\bar{\tt e}) \derives {{\tt e}}'.{\tt m}(\bar{\tt e})}

	\infrule[\RCInvkArg]%
		{{\tt e}_i \derives {{\tt e}_i}'}
		{{\tt e}.{\tt m}(\ldots,{\tt e}_i,\ldots) \derives {{\tt e}}.{\tt m}(\ldots,{\tt e}_i',\ldots)} 

	\infrule[\RCNewArg]%
		{{\tt e}_i \derives {{\tt e}_i}'}
		{\new\ {\tt C}(\ldots,{\tt e}_i,\ldots) \derives \new\ {\tt C}(\ldots,{\tt e}_i',\ldots)}
\end{tabular}
}
\end{tabular}  

 \caption{Semantics of \FX}\label{fig:FX}
\end{figure*}


\subsection{Judgments}
%\begin{definition}[Designator] The class of {\em designators}
%is given by:
%\begin{tabular}{r@{\quad}rcl}
%(Designator) & {\tt d} &{::=}& {\tt x}\alt C($\bar{\tt d}$)  \alt {\tt d.f} \\
%\end{tabular}
%\end{definition}

In the following $\Gamma$ is a {\em well-typed context}, i.e., a
finite, possibly empty sequence of formulas $\tt x:T$
%, $\tt T\ \type$
and constraints $\tt c$ satisfying:
\begin{enumerate}
  \item for any formula $\phi$ in the sequence all variables $\tt x$
    occurring in $\phi$ are defined by a declaration $\tt x:T$
    in the sequence to the left of $\phi$.

  \item for any variable $\tt x$ , there is at most one
  formula $\tt x:T$ in $\Gamma$.
\end{enumerate}

The judgments of interest are as follows. 
(1)~Type well-formedness:  $\Gamma \vdash {\tt T} \ {\tt type}$,
(2)~Subtyping: $\Gamma \vdash {\tt S} \subtype {\tt T}$,
(3)~Typing:   $\Gamma   \vdash {\tt e:T}$,
(4)~Method OK (method $\tt M$ is well-defined for the class $\tt
C$): $\Gamma \vdash {\tt M}\ \mbox{OK in {\tt C}}$,
(5)~Field OK (field $\tt f:T$ is well-defined for the class $\tt
C$): $\Gamma \vdash {\tt f}:{\tt T}\ \mbox{OK in {\tt C}}$
(6)~Class OK: $\Gamma \vdash {\tt L} \ \mbox{OK}$ (class definition {\tt L} is well-formed). 

In defining these judgments we will use $\Gamma \vdash_{\cal O} c$,
the judgment corresponding to the Object constraint system.
Recall that $\cal O$ subsumes the underlying constraint system $\cal C$.
For
simplicity, we define $\Gamma \vdash {\tt c}$ to mean
$\sigma(\Gamma)\vdash_{\cal O} {\tt c}$, where the {\em constraint
projection}, $\sigma(\Gamma)$ is defined as allows. 

\begin{tabular}{l}
$\sigma(\epsilon) = {\tt true}$\\
$\sigma({\tt x}:{\tt C}, \Gamma)$ = $\sigma(\Gamma)$\\
$\sigma({\tt x}:{\tt T\{c\}}, \Gamma)$ = {\tt c}[{\tt x}/\self], $\sigma({\tt x}:{\tt T}, \Gamma)$\\
$\sigma({\tt x}:{\tt (y:S;T)}, \Gamma)$ = $\sigma({\tt y:S}, {\tt x}:{\tt T},\Gamma)$\\
$\sigma({\tt c},\Gamma)$ = {\tt c}, $\sigma(\Gamma)$
\end{tabular}

\noindent Above, in the third rule we assume that
alpha-equivalence is used to choose the variable {\tt x} from a
set of variables that does not occur in the context $\Gamma$.

We say that a context $\Gamma$ is {\em consistent} if all (finite)
subsets of $\{\sigma(\phi)\alt \Gamma \vdash \phi\}$ are consistent.
In all type judgments presented below ({\sc T-Cast}, {\sc T-Field}
etc) we make the implicit assumption that the context $\Gamma$ is
consistent; if it is inconsistent, the rule cannot be used and the
type of the given expression cannot be established (type-checking
fails).

\subsection{\FXZ{}}

The syntax and semantics of \FXZ{} is presented in Figure~\ref{fig:FX}. 
%
The syntax is essentially that of \FJ{} with the following major
exceptions. First, types may be constrained with a clause {\tt
\{c\}}. Second both classes and methods may have constraint clauses
{\tt c}---in the case of classes, {\tt c} is to be thought of as an
invariant satisfied by all instances of the class, and in the case of
methods, {\tt c} is an additional condition that must be satisfied by
the receiver and the arguments of the method in order for the method to
be invoked.

We assume a constraint system $\cal C$, with a vocabulary of
predicates {\tt q} and functions {\tt f}.
Constraints include {\tt true}, conjunctions, existentials,
predicates, and term equality.
Constraints are well-formed if they are of the pre-given
type {\tt o}.
The rules \rn{Pred} and \rn{Fun}
ensure that predicates and formulas are well-formed and appeal
to the constraint system $\cal C$.

\eat{
The constraints permitted by the syntax in Figure~\ref{fig:FX} that
are not user constraints are necessary to define the static and
dynamic semantics of the program (see, e.g., the rules {\sc T-New}
and {\sc T-Field}, {\sc T-Var} that use equalities, conjunctions and
existential quantifications. The use of this richer constraint set is
not necessary, it simply enables us to present the static and dynamic
semantics once for the entire family of \FX{} languages,
distinguishing different members of the family by varying the
constraint system over which they are defined.\footnote{Another way of
saying this is that the language ${\cal L}=\FX({\cal C})$ is actually
defined over the constraint system ${\cal O}({\cal C})$--the constraints
of $\cal O$ are uniformly added to every \FX{} member.}
}

The set of types includes classes {\tt C} and is closed under
constrained types ({\tt T\{c\}}) and existential quantification ({\tt
x:S;T}). An object {\tt o} is of type {\tt C} (for {\tt C} a class)
if it is an instance of a subtype of {\tt C}; it is of type {\tt
T\{c\}} if it is of type {\tt T} and it satisfies the constraint $\tt
c[o/self]$\footnote{Thus the constraint {\tt c} in a type {\tt T\{c\}}
should be thought of as a unary predicate $\lambda\self.{\tt c}$, an
object is of this type if it is of type {\tt T} and satisfies this
predicate.}; it is of type {\tt x:S;T} if there is some object {\tt q}
of type {\tt S} such that {\tt o} is of type {\tt T[q/x]} (treating at
type as a syntactic expression).

The rules for well-formedness of types are straightforward, given 
the assumption that constraints are of type {\tt o}.

\paragraph{Typing rules.}
{\sc T-Var} is as expected, except that it asserts the constraint {\tt
self==x} which records the fact that any value of this type is known
statically to be equal to {\tt x}. This constraint is actually 
crucial---as we shall see in the other rules once we establish that
an expression {\tt e} is of a given type {\tt T}, we ``transfer'' the
type to a freshly chosen variable {\tt z}. If in fact {\tt e} has a
static ``name'' {\tt x} (i.e., {\tt e} is known statically to be
equal to {\tt x}; that is, {\tt e} is of type {\tt T\{self==x\}}), then
{\sc T-Var} lets us assert that {\tt z:T\{self==x\}}, i.e., {\tt z}
equals {\tt x}. Thus {\sc T-Var} provides an important base case for
reasoning statically about equality of values in the environment.

We do away with the three casts provided in \FJ{} in favor of a single
cast, requiring only that {\tt e} be of some type {\tt U}. At run time
{\tt e} will be checked to see if it is actually of type {\tt T} (see
Rule~{\sc R-Cast}).

{\sc T-Field} may be understood through ``proxy'' reasoning as
follows.  Given the context $\Gamma$ assume the receiver {\tt e} can
be established to be of type {\tt S}. Now we do not know the run-time
value of {\tt e}, so we shall assume that it is some fixed but unknown
``proxy'' value {\tt z} (of type {\tt S}) that is ``fresh'' in that it
is not known to be related to any known value (i.e., those recorded
in $\Gamma$).  If we can establish that {\tt z} has a field {\tt f} of
type {\tt T}\footnote{Note from the definition of \fields{} in
${\cal O}$ (Figure~\ref{fig:O}) that all occurrences of \this{}
in the declared type of the field {\tt f} will have been
replaced by {\tt z}.}, then we can assert that {\tt e.f} has
type {\tt T} and, further, that it equals {\tt z.f}.  Hence, we
can assert that {\tt e.f} has type {\tt (z:S;\,T\{\self==z.f\})}.

{\sc T-Invk} has a very similar structure to {\sc T-Field}: we use
``proxy'' reasoning for the receiver and the arguments of the method
call. {\sc T-New} also uses the same proxy reasoning; however, in this case
we can establish that the resulting value is equal to ${\tt new\ C(\bar{\tt v})}$
for some values $\bar{\tt v}$ of the given type.  The rule
requires that the class invariant of {\tt C} be established.

\paragraph{Operational semantics.}

The operational semantics is essentially identical
to \FJ~\cite{FJ}. It is described in terms of a non-deterministic
reduction relation on expressions. The only novelty is the use of the
subtyping relation to check that the cast is satisfied. In \FXZ, this
test simply involves checking that the class of which the object is an
instance is a subclass of the class specified in the given type; in
richer languages with richer notions of type this operation may
involve run-time constraint solving using the fields of the object.


\subsection{Results}
The following results hold for \FXD.

\begin{theorem}[Subject Reduction] If $\Gamma \vdash {\tt e:T}$
and ${\tt e} \derives {\tt e'}$ then for some type {\tt S}, $\Gamma \vdash {\tt e':S},{\tt S \subtype T}$.
\end{theorem}

The theorem needs the Substitution Lemma:
\begin{lemma}
If 
$\Gamma \vdash \tbar{d}:\tbar{U}$, and
$\Gamma, \bar{\tt x}:\tbar{U} \vdash \tbar{U}\subtype\tbar{V}$,
and
$\Gamma, \tbar{x}:\tbar{V} \vdash {\tt e:T}$,
then for some type {\tt S}, it is the case that
$\Gamma \vdash {\tt e[\tbar{d}/\tbar{x}]:S}, {\tt S \subtype
\tbar{x}:\tbar{V};T}.$
\end{lemma}

We let values be of the form $\tt v ::= \new\ C(\bar{\tt v})$. 
\begin{theorem}[Progress]
If $\vdash {\tt e:T}$ then one of the following conditions holds:
\begin{enumerate}
\item {\tt e} is a value,
\item {\tt e} contains a cast sub-expression which is stuck,
\item there exists an $\tt e'$ s.t. $\tt e\derives e'$.
\end{enumerate}
\end{theorem}

\begin{theorem}[Type soundness]
If $\vdash {\tt e:T}$ and {\tt e} reduces to a normal form ${\tt e'}$ then
either $\tt e'$ is a value {\tt v} and $\vdash {\tt v:S},{\tt S\subtype T}$ or
${\tt e'}$ contains  a stuck cast sub-expression.
\end{theorem}




