\chapter{Program structure}
\label{programchapter}

\section{Programs}

A Scheme program consists of a sequence of expressions and definitions.
Expressions are described in Chapter~\ref{expressionchapter};
definitions are the subject of the rest of the present chapter.

Programs are typically stored in files or entered interactively to a
running Scheme system, although other paradigms are possible;
questions of user interface lie outside the scope of this report.
(Indeed, Scheme would still be useful as a notation for expressing
computational methods even in the absence of a mechanical
implementation.)

Definitions occurring at the top level of a program can be interpreted
declaratively.  They cause bindings to be created in the top level
environment.  Expressions occurring at the top level of a program are
interpreted imperatively; they are executed in order when the program is
invoked or loaded, and typically perform some kind of initialization.

\todo{Cromarty, etc.: disclaimer about top level?}


\section{Definitions}
\label{definitionsection}

Definitions are valid in some, but not all, contexts where expressions
are allowed.  They are valid only at the top level of a \hyper{program}
and at the beginning of a \hyper{body}.
\mainindex{definition}

A definition shall have one of the following forms:\mainschindex{define}

\begin{itemize}

\item {\tt(define \hyper{variable} \hyper{expression})}

\item{\tt(define (\hyper{variable} \hyper{formals}) \hyper{body})}

\hyper{Formals} shall be either a
sequence of zero or more variables, or a sequence of one or more
variables followed by a space-delimited period and another variable (as
in a lambda expression).  This form is equivalent to
\begin{scheme}
(define \hyper{variable}
  (lambda (\hyper{formals}) \hyper{body}))\rm.%
\end{scheme}

\item{\tt(define (\hyper{variable} .\ \hyper{formal}) \hyper{body})}

\hyper{Formal} shall be a single
variable.  This form is equivalent to
\begin{scheme}
(define \hyper{variable}
  (lambda \hyper{formal} \hyper{body}))\rm.%
\end{scheme}

\item {\tt(begin \hyperi{definition} \dotsfoo)}

This form is equivalent to the set of
definitions that form the body of the \ide{begin}.
See also~\ref{sequencingsection}.

\end{itemize}


\subsection{Top level definitions}
\label{topleveldefinitions}

At the top level of a program, a definition
\begin{scheme}
(define \hyper{variable} \hyper{expression})%
\end{scheme}
has essentially the same effect as the assignment expression
\begin{scheme}
(\ide{set!}\ \hyper{variable} \hyper{expression})%
\end{scheme}
if \hyper{variable} is bound (but see~\ref{safebehavior}).  If
\hyper{variable} is not bound, however, then the definition will bind
\hyper{variable} to a new location before performing the assignment,
whereas it would be an error to perform a \ide{set!}\ on an
unbound\index{unbound} variable.

\begin{scheme}
(define add3
  (lambda (x) (+ x 3)))
(add3 3)                            \ev  6
(define first car)
(first '(1 2))                      \ev  1%
\end{scheme}

It is permissible for an implementation to use an initial
environment\index{initial environment} in which all possible variables
are bound to locations, most of which contain undefined values.


\subsection{Internal definitions}
\label{internaldefines}

Definitions are also permitted to occur at the
beginning of a \hyper{body} (that is, the body of a \ide{lambda},
\ide{let}, \ide{let*}, \ide{letrec}, or \ide{define} expression).  Such
definitions are known as {\em internal definitions} \mainindex{internal
definition} as opposed to the top level definitions described above.
The variable defined by an internal definition is local to the
\hyper{body}.  That is, \hyper{variable} is bound rather than assigned,
and the region of the binding is the entire \hyper{body}.  For example,

\begin{scheme}
(let ((x 5))
  (define foo (lambda (y) (bar x y)))
  (define bar (lambda (a b) (+ (* a b) a)))
  (foo (+ x 3)))                \ev  45%
\end{scheme}

A \hyper{body} containing internal definitions can always be converted
into a completely equivalent \ide{letrec} expression.  For example, the
\ide{let} expression in the above example is equivalent to

\begin{scheme}
(let ((x 5))
  (letrec ((foo (lambda (y) (bar x y)))
           (bar (lambda (a b) (+ (* a b) a))))
    (foo (+ x 3))))%
\end{scheme}

Just as for the equivalent \ide{letrec} expression, it shall be
possible to evaluate each \hyper{expression} of every internal
definition in a \hyper{body} without assigning or referring to
the value of any \hyper{variable} being defined.
