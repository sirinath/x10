% First page

\thispagestyle{empty}

% \todo{"another" report?}

\topnewpage[{
\begin{center}   {\huge\bf
	Revised$^{\bf 4}$ Report on the Algorithmic Language \\
			      \vskip 3pt
				Scheme}
\vskip 1ex
$$
\begin{tabular}{l@{\extracolsep{.5in}}lll}
\multicolumn{4}{c}{W{\sc ILLIAM} C{\sc LINGER AND} J{\sc ONATHAN} R{\sc EES}
 ({\it Editors\/})} \\
H. A{\sc BELSON}     &
R. K. D{\sc YBVIG}   &
C. T. H{\sc AYNES}   &
G. J. R{\sc OZAS}    \\
N. I. A{\sc DAMS IV} &
D. P. F{\sc RIEDMAN} &
E. K{\sc OHLBECKER}  &
G. L. S{\sc TEELE} J{\sc R}. \\
D. H. B{\sc ARTLEY}  &
R. H{\sc ALSTEAD}    &
D. O{\sc XLEY}	     &
G. J. S{\sc USSMAN}  \\
G. B{\sc ROOKS}	     &
C. H{\sc ANSON}	     &
K. M. P{\sc ITMAN}   &
M. W{\sc AND}	     \\
\end{tabular}
$$
\vskip 2ex
{\it Dedicated to the Memory of ALGOL 60}
\vskip 2.6ex
\end{center}
}]

\chapter*{Summary}

The report gives a defining description of the programming language
Scheme.  Scheme is a statically scoped and properly tail-recursive
dialect of the Lisp programming language invented by Guy Lewis
Steele~Jr.\ and Gerald Jay~Sussman.  It was designed to have an
exceptionally clear and simple semantics and few different ways to
form expressions.  A wide variety of programming paradigms, including
imperative, functional, and message passing styles, find convenient
expression in Scheme.

\vest The introduction offers a brief history of the language and of
the report.

\vest The first three chapters present the fundamental ideas of the
language and describe the notational conventions used for describing the
language and for writing programs in the language.

\vest Chapters~\ref{expressionchapter} and~\ref{programchapter} describe
the syntax and semantics of expressions, programs, and definitions.

\vest Chapter~\ref{builtinchapter} describes Scheme's built-in
procedures, which include all of the language's data manipulation and
input/output primitives.

\vest Chapter~\ref{formalchapter} provides a formal syntax for Scheme
written in extended BNF, along with a formal denotational semantics.
An example of the use of the language follows the formal syntax and
semantics.

\vest The appendix describes a macro facility that may be used to
extend the syntax of Scheme.

\vest The report concludes with a bibliography and an
alphabetic index.

\todo{expand the summary so that it fills up the column.}

%\vfill
%\begin{center}
%{\large \bf
%*** DRAFT*** \\
%August 31, 1989%\today
%}\end{center}

\vfill
\eject

{\footnotesize
\tableofcontents
}

\vfill
\eject
