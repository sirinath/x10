\section{Array Type Constructors}
\label{ArrayypeConstructors}\index{array types}

{}\XtenCurrVer{} does not have array class declarations
(\S~\ref{XtenArrays}). This means that user cannot define new array
class types. Instead arrays are created as instances of array types
constructed through the application of {\em array type constructors}
(\S~\ref{XtenArrays}).

The array type constructor takes as argument a type (the {\em base
type}), an optional distribution (\S~\ref{XtenDistributions}), and
optionally either the keyword {\cf reference} or {\cf value} (the
default is reference):
\begin{x10}
18    ArrayType ::= Type [ ]
438   ArrayType ::= X10ArrayType
439   X10ArrayType ::= Type [ . ]
440     | Type reference [ . ]
441     | Type value [ . ]
442     | Type [ DepParameterExpr ]
443     | Type reference [ DepParameterExpr ]
444     | Type value [ DepParameterExpr ]
\end{x10}

The array type {\cf Type[ ] } is the type of all arrays of base
type {\tt Type} defined over the distribution {\tt 0:N -> here}
for some positive integer {\tt N}.

The qualifier {\tt value} ({\tt reference}) specifies that the array
is a {\tt value}({\tt reference}) array. The array elements of a {\tt
value} array are all {\tt final}.\footnote{Note that the base type of a {\tt value} array can be a value class or a reference class, just as the 
type of a {\tt final} variable can be a value class or a reference class.
}If the qualifier is not specified,
the array is a {\tt reference} array.

The array type {\cf Type reference [.]} is the type of all (reference)
arrays of basetype {\tt Type}. Such an array can take on any
distribution, over any region. Similarly, {\cf Type value [.]} is the
type of all value arrays of basetype {\tt Type}.

\XtenCurrVer{} also allows a distribution to be specified between {\tt
[} and {\tt ]}. The distribution must be an expression of type
{\tt distribution} (e.g.{} a {\tt final} variable) whose
value does not depend on the value of any mutable variable.

Future extensions to \Xten{} will support a more general syntax for
arrays which allows for the specification of dependent types, 
e.g.{} {\tt double[:rank 3]}, the type of all arrays of 
{\tt double} of rank {\tt 3}.


