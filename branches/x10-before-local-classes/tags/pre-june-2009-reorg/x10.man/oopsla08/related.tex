Constraint-based type systems enjoy a long history.
The use of constraints for type inference and subtyping were
developed by
Mitchell~\cite{mitchell84}
and
Reynolds~\cite{reynolds85}.
%
Trifonov and Smith~\cite{trifonov96}
proposed a type system where types are refined by subtyping
constraints.  Dependent types are not supported.
%
Pottier~\cite{pottier96simplifying,pottier01b}
presents a constraint-based type system for an ML-like language with
subtyping.

HM(X)~\cite{sulzmann97type,pottier01a,pottier-remy-attapl}
is a constraint-based framework
for Hindley--Milner style type systems.
The framework is parameterized on the specific constraint system
X; instantiating X yields extensions of the HM type system.
Constraints in HM(X) are over types, not values.
%
% XXX HM(X) introduced {\em term constraint systems}; constraints in
% CFJ are term constraints?

% Cardelli~\cite{cardelli86}, type checking dependent types and
% subtypes.

% Russell
% \cite{fuh88}
% \cite{curtis90}
% \cite{aiken93}

% Aiken, Wimmers, and Lakshman proposed {\em conditional
% types}~\cite{conditional-types}, which have the ability to
% encode control-flow analysis of {\tt case} expressions.
% Conditional types are not dependent.

% \cite{smith94}



% \cite{palsberg95}
% constraint-based inference algorithm for object calculus, 

% Henglein (TAPOS) set constraints for OO language type-inference.

% Bane~\cite{fahndrich99}

% Pottier

% CLP(X) framework in constraint logic programming (JM94)
% HM(X)

Several systems have been proposed that refine types in a base
type system through constraints.
%
{\em Refinement types}~\cite{refinement-types} extend the 
Hindley--Milner type system with intersection, union, and
constructor types, enabling specification and inference of
more precise type information.
%
{\em Conditional
types}~\cite{conditional-types} extend refinement types to
encode control-flow information in the types.
%
Jones introduced {\em qualified types}, which permit
types to be constrained by a finite set of
predicates~\cite{jones94}.
%
{\em Sized types}~\cite{sized-types}
annotate types with the sizes of recursive data structures.
Sizes are linear functions of size variables.
Size inference is performed using a constraint solver for
Presburger arithmetic~\cite{omega}.
% constraints on types, support primitive recursion only

% Indexed types~\cite{indexed-types}

% Index objects must be pure.
% Singleton types int(n).
% ML$^{\Pi}_0$:
% Refinement of the ML type system: does not affect the
% operational semantics.  Can erase to ML$_0$.

% Jay and Sekanina 1996: array bounds checking based on shape
% types.

With hybrid type-checking~\cite{flanagan-popl06,flanagan-fool06},
types can be constrained by arbitrary boolean expressions.
While typing is undecidable, dynamic checks are inserted into
the program when necessary if the type-checker cannot determine
type safety statically.
In \Xten{} dynamic type checks, including tests of dependent
clauses, are inserted only at explicit casts.

% Ada dependent types.
% Ada has constrained array definitions.  A constraint
% \cite{ada-ref-man}.  Not clear if they're dependent.  Are
% there other dependent types?  Generics are dependent?

Singleton types~\cite{aspinall-singletons,stone00} are dependent
types containing only one value.  
In Stone's formulation~\cite{stone00},
$S(e : \tau)$
is the type of all values of type $\tau$ that are equal to $e$.
Term equivalence is
constructed so that type-checking is decidable.
The singleton $S(e: \tau)$ can be encoded in \Xten{} as
$\tau$\xcd{(:self ==}~$e$\xcd{)}.

        % Used for array bounds by Morrisett et al (I think--need
        % to find paper)

% Singleton types~\cite{aspinall-singletons}.

Several languages---gbeta~\cite{ernst99-gbeta},
Scala~\cite{scala-overview,scala-oopsla05}, J\&~\cite{nqm06}, and
others~\cite{oz01,ocrz-ecoop03,dependent-classes}---provide {\em path-dependent
types}.  For a final access path \xcd{p}, {\tt p.type}
in Scala is the singleton type containing the object \xcd{p}.
In J\&, {\tt p.class} is a type containing all objects
whose run-time class is the same as \xcd{p}'s.
Scala's {\tt p.type} can be encoded in \Xten{} using an equality
constraint \xcd{C(:self == p)}, where \xcd{C} is a supertype of
\xcd{p}'s static type.
\eat{
These types can be encoded in CFJ by introducing a
\xcd{type} property.
\rn{T-constr}, as
described in Section~\ref{sec:examples}.
}

% Where clauses for F-bounded polymorphism~\cite{where-clauses}
% Bounded quantification: Cardelli and Wegner.  Bound T with T'
% In F-bounded polymorphism~\cite{f-bounds}, type variables are bounded by a function of 
% the type variable. 
% Not dependent types.

\eat{
conditional types:

type of an expression can be constrained using information about
the results of run-time tests in the context surrounding the
expression.

e.g., can express that e2 is evaluated only if e1 is true

\begin{verbatim}
\y. case y of true => e1 | false => e2 :
        'a -> (true ? ('a ^ typeof(e1)) U (false ? ('a ^ typeof(e2))
\end{verbatim}

Types include type constructors applied to types.

\begin{verbatim}
        so,  true      : true
        but, (\x . x)  : 'a -> 'a
             node(l,r) : node('a tree, 'a tree)
\end{verbatim}


when checking a case branch, type of the expression being
matched refined to the include the type constructor for that
branch

captures some control flow analysis in the types

types
\begin{verbatim}
        t ::= t1 -> t2
                | c(..ti..) <-- type constructor
                | alpha
                | t1 U t2
                | t1 ^ t2
                | t1 ? t2
                | 0
                | 1

        sigma ::= t | \forall ..alpha.. t where ..ti <= tj..
\end{verbatim}
}


Cayenne~\cite{cayenne} is a Haskell-like language with fully dependent types.
There is no distinction between static and dynamic types.
Type-checking is undecidable.
There is no notion of datatype refinement as in DML.

Epigram~\cite{epigram,epigram-matter}
is a dependently typed functional programming language based on
a type theory with inductive families.
Epigram does not have a phase distinction between values and
types.

\eat{
$\lambda^{\sf Con}$ is a lambda calculus with assertions.
Findler, Felleisen, Contracts for higher-order functions (ICFP02)

  example: int[> 9]

contracts are either simple predicates or function contracts.
defined by (define/contract ...)

enforced at run-time.
}

% Jif~\cite{jif,jflow} is an extension of Java in which
% types are labeled with security policies enforced by the
% compiler.

ESC/Java~\cite{esc-java}
allow programmers to write object invariants and pre- and
post-conditions that are enforced statically
by the compiler using an automated theorem prover.
Static checking is undecidable and, in the presence of loops,
is unsound (but still useful) unless the programmer supplies loop invariants.
ESC/Java can enforce invariants on mutable state.

% and Spec$\sharp$~\cite{specsharp}

Pluggable and optional type systems were proposed by
Bracha~\cite{bracha04-pluggable} and provide another means of
specifying refinement types.
Type annotations, implemented in compiler plugins, serve only to
reject programs statically that might otherwise have dynamic
type errors.
CQual~\cite{foster-popl02} extends C with user-defined type
qualifiers.  These
qualifiers may be flow-sensitive and may be inferred. 
CQual supports only a fixed set of typing rules
for all qualifiers.
In contrast, the {\em semantic type qualifiers} of
Chin, Markstrum, and Millstein~\cite{chin05-qualifiers}
allow programmers to define typing rules for qualifiers
in a meta language that allows type-checking rules to be
specified declaratively.
JavaCOP~\cite{javacop-oopsla06} is a pluggable type system
framework for Java.  Annotations are defined in a meta language
that allows type-checking rules to be specified declaratively.
JSR 308~\cite{jsr308} is a proposal for adding user-defined type qualifiers
to Java.

% Holt, Cordy, the Turing programming language

% Ou, Tan, Mandelbaum, Walker, Dynamic typing with dependent types
% Separate dependent and simple parts of the program.
% Statically type the dependent parts.
% Dynamic checks when passing values into dependent part.

Our work is most closely related to \DML{}, \cite{xi99dependent}, an
extension of ML with dependent types. \DML{} is also built
parametrically on a constraint solver. Types are refinement types;
they do not affect the operational semantics and erasing the
constraints yields a legal ML program.

At a conceptual level the intuitions behind the development of \DML{}
and constrained types are similar. Both are intended for practical
programming by mainstream programmers, both introduce a strict
separation between compile-time and run-time processing, are
parametric on a constraint solver, and deal with mutually recursive
data-structures, mutable state, and higher-order functions (encoded as
objects in the case of constrained types). Both support existential
types.

The most obvious distinction between the two lies in the target
domain: \DML{} is designed for functional programming, specifically
ML, whereas constrained types are designed for imperative, concurrent
OO languages. Hence technically our development of constrained types
takes the route of an extension to \FJ. But there are several other
crucial differences as well.

\lstnewenvironment{displayml}
  {\lstset{language=ML,basicstyle=\tt,tabsize=4,columns=fullflexible,captionpos=b,xleftmargin=1em,xrightmargin=1em,keywordstyle=,keepspaces}}
  {}

First, \DML{} achieves its separation by not permitting program
variables to be used in types. Instead, a parallel set of (universally
or existentially quantified) ``index'' variables are
introduced. For instance the typing of the \xcd{append} operation on
lists is provided by:
\begin{displayml}
fun('a)
  append(nil, ys) = ys
| append(cons(x, xs), ys)
    = cons(x, append(xs,ys))
where append <| {m:nat}{n:nat} 
    'a list(m) * 'a list(n) -> 'a list(m+n)  
\end{displayml}
\noindent in contrast with the direct embedded expression with constrained types:
\begin{xten}
class List(int(:self >= 0) n) {
  Object item;
  List(n-1) tail;
  List(n+a.n) append(final List a) { 
    return n==0
      ? a : new List(item, tail.app(a));
  }
  ...
}
\end{xten}

Second, \DML{} permits only variables of basic index sorts known to
the constraint solver (e.g., \xcd{bool}, \xcd{int}, \xcd{nat}) to
occur in types. In contrast, constrained types permit program
variables at any type to occur in constrained types. As with \DML,
only operations specified by the constraint system are permitted in
types. However, these operations always include field selection and
equality on object references.  (As we have seen permitting arbitrary
type/property graphs may lead to undecidability.) Note that \DML{}
style constraints are easily encoded in constrained types.

% A reviewer says:
% The third criticism of DML is technically correct but highly
% misleading.  Instead of casts, DML allows "if tests" or case
% analysis as dynamic tests that then yield static information
% about the type in the appropriate branch of the if or case.
% Either omit this criticism or describe how DML does the same
% thing--or if DML's system is weaker in some way, give a
% particular example to justify that.

% Third, \DML{} does not permit any runtime checking of constraints
% (dynamic casts).
