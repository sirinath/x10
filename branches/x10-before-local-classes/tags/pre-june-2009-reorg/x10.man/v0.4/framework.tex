\documentclass{article}
\input{../../../../vj/res/pagesizes}
\def\csharp{{\sf C\#}}
\usepackage[pdftex]{graphicx}
\def\Xten{{\sf X10}}
\def\java{{\sf Java}}
\newtheorem{example}{Example}[section]

\begin{document}
\title{A parametric framework for static/dynamic checking of assertions}
\author{
Vijay Saraswat \\
IBM TJ Watson Research Center \\
PO Box 704, Yorktown Heights\\
NY 10598\\
({\sc Draft Version 0.21})\\
(Please do not cite)\\
(Send comments to {\tt vsaraswa@us.ibnm.com}.)
}

\date{30 December 2004}
\maketitle

\begin{abstract}
We consider the problem of making it easy for programmers and language
designers to enrich the type system for \java-like class-based OO
languages. 

Let $L$ be such a language.  We show that under some simple
assumptions it is possible to translate programs written in $L$
augmented with an ``extended'' type system into programs written in
a language $L_a$ which is 
$L$  augmented with {\tt assert} and {\tt assume} statements over a
particular assertion language.

We show that programs in $L_a$ may be evaluated by a ``universal''
transformer (a single program in {\sf Axa}) which knows only about the
operational semantics of $L$ and works parametrically with a
constraint solver for the assertion language. The type-checker
transforms an input program into a semantically equivalent output
program that may be less expensive to run because it simplifies
conditionals whose condition can be evaluated statically and removes
{\tt assert}s that are statically shown to be valid. (Assertions not
statically removed will generate checks at runtime and may throw
exceptions if the checks fail.)

To implement an extended type system, a programming language designer
or programmer must supply a translator for programs in the extended
type system to programs with asserts and assumes, and must specify the
underlying constraint solver. No separate type-checker needs to be provided.

We illustrate the use of this framework by showing how to implement the dependent
types of \cite{saraswat-dp}.

\end{abstract}

\section{Introduction}


An enriched language is a \java{} like language in which the type
system is richer than some base type system.

The problem at hand is making it easy for programming language
designers (and programmers) to design and implement new, richer type
systems on top of the base type system.

\subsection{Specifying \java{} like languages}
We now discuss what it means for a language to be ``\java-like''.  The
language must be organized around a notion of classes, which support
an inheritance structure, carry instance and static state (through
fields), and code, organized in {\em methods} which are named, and may
take multiple arguments and return values. 

$L$ must ensure that objects are encapsulated, and are created through
the invocation of {\em constructors} which may take zero or more
arguments and throw exceptions. Each constructor is associated with a
precondition and a postcondition. We require that any constructor
invocation $I$ must be uniquely matched with a constructor definition
$D$, and $L$ must check that the precondition of $D$ is satisfied
statically at the site of the call. After the call, the calling site
may {\tt assume} the postcondition of $D$.

Every variable must be declared with a type. The language rules must
guarantee that only objects that satisfy the type can be stored in the
variable.  Variables are assumed to be mutable, unless they are
declared to be {\tt final}. The language may support different kinds
of variables with different scope and extent -- e.g.{} arguments to
methods, local variables in method bodies, arguments of catch clauses,
loop variables in for loops, instance and static fields of
classes, and members of arrays, the implicit variable representing the
return value from a method, etc.

We require that {\em the type of all accesses to a variable be
statically determinable}. That is, given a statement, it should be
possible to statically determine the type of any variable that the
statement reads or writes. This property is true of \java{}; it is not
true, for example, for a language that permits the name of the field
being read/written to be computed by an expression whose value cannot
be determined statically, and permits different fields of an object to
have different types.

We require that all methods be associated with a {\em precondition}
and a {\em postcondition}.  Each must be a predicate in the assertion
language on the arguments and the return value (if any) of the method.
We require that given a particular method invocation $o.m(t_1,\ldots,
t_n)$ the particular method declaration on the type of $o$ which will
be invoked at runtime be determined statically. Recall that because of
subtyping in OO languages it is possible that the actual method that
is invoked at runtime may not be the method that is statically
determined by a method on a subtype of the type of $o$ which overrides
this method. Therefore we require that overriding respect the pre/post
assertions of the method. That is, if a method $M_1$ on a subclass
override a method $M_2$ on a superclass then it must be the case that
$M_1$'s precondition is implied by $M_2$'s precondition, and $M_1$'s
postcondition implies $M_2$'s postcondition. We require that the
language ensure that every method invocation statically satisfies the
precondition of the method it is invoking. The call site may {\tt assume}
the postcondition of the method after the call.

As an example \java{} satisfies this property. The precondition of a
method is simply the assertion that each argument to the method has
the given type. The postcondition of a method is simply the assertion
that the result has the given type. \java{} forces the postcondition
of a method to be uniquely determined by the precondition of the
method (the return type of the method is uniquely detrmined by the
input types of the arguments to the method). It allows a method $M_1$
to override a method $M_2$ only if the preconditions of both are the
same.  It checks that the type of actuals in a method invocation matches
the type of formals in the associated method declaration.

\subsection{$L$ with {\tt assert} and {\tt assume}}

Let $L$ be such a language.  For the purposes of this paper, let us
define {\tt assert} and {\tt assume} statements as follows.  For {\tt
e} a boolean expression the statement:
{\footnotesize
\begin{verbatim}
   assume e; 
\end{verbatim}}
asserts that {\tt e} will be true whenever control reaches that point
in the code. This assertion is categorical -- it does not depend on
any property of the code other than what is expressed through the type
system and hence checked statically. Thus it is a mere comment -- it
does not translate into any runtime code. 

On the contrary the statement: 
{\footnotesize
\begin{verbatim}
   assert e : s; 
\end{verbatim}}
(where {\tt e} is a boolean expression and {\tt s} a string) is an
assertion that says {\tt e} {\em should} be true at runtime whenever
control reaches that point. In case it is not, a runtime error should be thrown
with the given string as argument. Thus it translates into the code:
{\footnotesize
\begin{verbatim}
   if (! e) throw new RuntimeError( s );
\end{verbatim}}

The predications occurring in these statements must belong to a
special class of {\tt boolean} valued methods defined in the source
language called {\em operations}. An operation satisfies the property
that the set of all tuples of objects on which it succeeds is closed
under equivalence of objects. Two objects (of the same class) are said
to be {\em equivalent} if they are instances of the same class and
their immutable state is identical. Further, the operation must be
associated with an {\em invariant}, a constraint (in the underlying
constraint language) of the same arity as the number of arguments to
the operation.  The constraint must be co-extensive with the
operation, that is it must succeed on the same set of tuples of
objects.  Thus a constraint is a symbolic representation of the
operation and can be used to reason symbolically about the effect of
the operation on unknown arguments.

Programs in the enriched source language -- with the enriched type
system -- are translated into programs in the base language, augmented
with {\tt assert} and {\tt assume} statements. 

When is this possible? The enriched type system must satisfy some
properties.  To each enriched type $t$ must correspond a type, the
base type $b(t)$ in the base language which is a coarse approximation
to $t$ (is implied by $b(t)$). 


\bibliographystyle{plain}
\bibliography{master}
\end{document}