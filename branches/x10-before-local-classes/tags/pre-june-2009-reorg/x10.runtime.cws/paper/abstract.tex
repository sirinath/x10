Fine-grained concurrency is a premier technique for the productive exploitation
of shared memory concurrency as realized in symmetric multi-processors and
multicore architectures. In this approach programmers focus on expressing the
natural parallelism of the application in as fine-grained
a manner as reasonable, leaving it  up to compilation and runtime mechanisms to
schedule such activities efficiently.

Cilk work-stealing is a premier mechanism for implementing fully strict,
series-parallel  task graphs. We present several techniques that improve the
performance of work-stealing and extend its range of applicability.
These techniques have been implemented in the X10 runtime, both
as a Java 5 library and as part of X10lib, a C++-based runtime and can be used
directly by programs written in Java and C++ (respectively). (A front-end for
X10 is currently under development.) We show that stack-allocation of frames
leads to significant improvement in performance. We show that for several
applications (e.g. depth-first graph search, n-queens) the implicit join
required  by Cilk in every procedure frame can be relaxed in favor of global
termination detection (the detection that no worker has any more task left to
execute). Global termination detection can be  implemented with an overhead
proportional to the number of workers (rather than the number of spawned
tasks). Additionally, we show that globally clocked computations (e.g. as
needed by breadth-first search) can be implemented efficiently through
(repeated) global quiescence detection.

Finally, we show the benefits of introducing  area-specific  operations. An
area is a portion  of X10's place that can be written into only by its owning
worker; some objects may be marked also as being read only by the area's owning
worker. All objects in an area are accessible to the tasks being executed by
the owning worker. Area-specific operations may be used to implement global
reductions using local counters (e.g. N-queens) as well as worker-specific
dynamic scheduling policies (e.g. for LU).
