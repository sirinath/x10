\clearextrapart{Introduction}

\subsection*{Objectives}


\vest This Standard specifies the form and meaning of programs written
in the Scheme programming language.

\vest Scheme was initially conceived as a vehicle for programming
language research founded on a few key concepts that together
critically distinguish it from others in the related Algol and LISP
families of languages.  Through several independent implementations,
Scheme has evolved to include a commonly accepted base
set of data types and procedures as well as other features that are
considered experimental or controversial.

\vest The principal objective of this Standard is to specify a
language containing exactly those features for which there has been
extensive experience and thus a conservative consensus for
standardization.  This provides users with a basis for constructing
programs that are portable across implementations without jeopardizing
research and future evolution of the language.


\subsection*{Scope}

\vest This Standard specifies the representation of Scheme programs,
their syntax, the semantic rules for interpreting them, and the
representation of data to be input or output by them.

\vest This Standard does not specify the mechanisms by which Scheme
programs are transferred to and from system memory and placed into
execution, or the size or complexity of a program and its data that
will exceed the capacity of a particular implementation.


\subsection*{Future Directions}

\vest Future revisions to this Standard may be anticipated by a series
of informal language specifications collectively known as the
``Revised Reports on the Algorithmic Language Scheme''
(see~\cite{RRRS}, \cite{R3RS}, \cite{R4RS}). These reports extend the
standard language to include features that are considered mature
enough to merit wide-spread implementation and experimentation.
Addition of such a feature to the report suggests that Scheme
implementors should examine the addition and carefully consider it as
a candidate for formal standardization.


\subsection*{Compliance}

\vest A {\it conforming program\/} shall use only those features of
the language that are specified in this Standard.  It shall not
contain dependencies on any unspecified or undefined aspect of this
specification.

\vest A {\it conforming implementation\/} shall accept any conforming
program, and execute it as specified in this Standard.  Conforming
implementations may have extensions, provided they do not alter the
behavior of any conforming program.


\subsection*{Organization of the document}

\vest This document is divided into the following major sections:

\begin{enumerate}

\item this introduction;

\item three chapters that present the fundamental ideas of the
language and describe the notational conventions used for describing
the language and writing programs in the language;

\item two chapters that specify the syntax and semantics of
expressions, programs, and definitions;

\item a chapter describing Scheme's built-in procedures, which include
all of the language's data manipulation and input/output primitives;

\item a chapter providing a formal syntax for Scheme written in
extended Backus-Naur Form;

\item an appendix providing a formal denotational semantics for
Scheme;

\item two appendices outlining some issues in the implementation of
Scheme's arithmetic;

\item an index.

\end{enumerate}


\subsection*{Definitions of terms}

\vest In this Standard, \defining{shall} is to be interpreted as a
requirement on an implementation or a program; conversely,
\defining{shall not} is to be interpreted as a prohibition.

\vest The word \defining{should} is to be interpreted as a strong
recommendation in keeping with the intent of the Standard; conversely,
\defining{should not} is to be interpreted as a strong discouragement.

The following terms are used in this document.  Other terms are
defined at their first appearance, indicated by {\it italic\/} type.

\defining{applicative order}. A property of a programming language or
procedure: the arguments to a procedure call are evaluated before the
procedure is invoked, and the result of each evaluation is passed to
the procedure in place of its argument expression.

\defining{bignum}. A multiple-precision computer representation for very
large integers (cf.~{\it fixnum}).

\defining{extent}. A period of time, usually referring to the lifetime of
an object.  Once created, an object with \defining{unlimited extent}
exists forever.

\defining{first-class object}. An object that can be the value of a
variable or can be stored in a data structure.  In Scheme, first-class
objects have {\it unlimited extent}.

\defining{fixnum}. A limited-precision computer representation
for integers, where the limitation is imposed by machine-architecture
constraints (cf.~{\it bignum}).

\defining{flonum}. A floating-point number.

\defining{procedure}. A parameterized program fragment, called a
``subroutine'' or ``function'' in some programming languages.

\defining{recursive}. Self-referential.  In common usage, a
``recursive procedure'' is one that calls itself; similarly a
\defining{recursion} is a call by a procedure to itself.  A set of procedures
is \defining{mutually recursive} if they refer to one another.

\defining{scope}. The region of a program's source text that is associated
with a linguistic construct.  Normally used with ``variable'' to
describe the region over which a variable is bound: ``the {\it scope}
of a variable''.

\defining{side effect}.  Loosely speaking, an expression has a side
effect if it performs some observable action in addition to returning
a value.  For example, a variable assignment is a side effect.

\defining{tail recursive}. A property of the implementation of a
programming language.  In a {\it tail-recursive} implementation,
iterative processes can be expressed by means of procedure calls.
(The process described by a program is iterative if and only if the
order of its space growth is constant, aside from that used for the
values of the program's variables.)

\defining{top level environment}.  The environment used to resolve
free variable references in a Scheme program.


\subsection*{Examples}

\vest This Standard describes in detail the form and meaning of the
features of the Scheme language.  To assist the reader, the document
provides examples to clarify the intent or operation of certain
features.  Because many aspects of the language are purposely left
undefined, these examples often merely typify the results to be
expected and should not be construed as implying further constraints
on the language specification.


\subsection*{Base Document}

\vest This document is derived from ``Revised$^4$ Report on the
Algorithmic Language Scheme,''~\cite{R4RS} edited by William Clinger
and Jonathan Rees.


\subsection*{References}

%\extrapart{Bibliography and references}

% My reference for proper reference format is:
%    Mary-Claire van Leunen.
%    {\em A Handbook for Scholars.}
%    Knopf, 1978.
% I think the references list would look better in ``open'' format,
% i.e. with the three blocks for each entry appearing on separate
% lines.  I used the compressed format for SIGPLAN in the interest of
% space.  In open format, when a block runs over one line,
% continuation lines should be indented; this could probably be done
% using some flavor of latex list environment.  Maybe the right thing
% to do in the long run would be to convert to Bibtex, which probably
% does the right thing, since it was implemented by one of van
% Leunen's colleagues at DEC SRC.
%  -- Jonathan

% This is just a personal remark on your question on the RRRS:
% The language CUCH (Curry-Church) was implemented by 1964 and 
% is a practical version of the lambda-calculus (call-by-name).
% One reference you may find in Formal Language Description Languages
% for Computer Programming T.~B.~Steele, 1965 (or so).
%  -- Matthias Felleisen


\begin{thebibliography}{99}

\bibitem{SICP}
Harold Abelson and Gerald Jay Sussman with Julie Sussman.
{\em Structure and Interpretation of Computer Programs.}
MIT Press, Cambridge, 1985.

\bibitem{readfloat}
William Clinger.
How to read floating point numbers accurately.
In {\em Proceedings of the 1990 ACM SIGPLAN Conference on Programming
  Language Design and Implementation}.  Forthcoming.

University of Oregon Technical Report CIS-TR-90-01.

\bibitem{RRRS}
William Clinger, editor.
The revised revised report on Scheme, or an uncommon Lisp.
MIT Artificial Intelligence Memo 848, August 1985.
Also published as Computer Science Department Technical Report 174,
  Indiana University, June 1985.

\bibitem{R4RS}
William Clinger and Jonathan Rees, editors.
The revised$^4$ report on the algorithmic language Scheme.
University of Oregon Technical Report CIS-TR-90-02.

\bibitem{Scheme311}
Carol Fessenden, William Clinger, Daniel P.~Friedman, and Christopher Haynes.
Scheme 311 version 4 reference manual.
Indiana University Computer Science Technical Report 137, February 1983.
Superceded by~\cite{Scheme84}.

\bibitem{Scheme84}
D.~Friedman, C.~Haynes, E.~Kohlbecker, and M.~Wand.
Scheme 84 interim reference manual.
Indiana University Computer Science Technical Report 153, January 1985.

\bibitem{CFractions}
G.~H.~Hardy and E.~M.~Wright.
{\em An Introduction to the Theory of Numbers.} 5th ed.
Oxford University Press, New York NY, 1979.

\bibitem{Haskell}
Paul Hudak and Philip Wadler, editors.
Report on the Functional Programming Language Haskell.
Yale University Research Report YALEU/DCS/RR-666, December 1988.

\bibitem{IEEE}
{\em IEEE Standard 754-1985.  IEEE Standard for Binary Floating-Point
Arithmetic.}  IEEE, New York, 1985.

\bibitem{Knuth}
Donald E. Knuth.
The Art of Computer Programming, volume 2: Seminumerical Algorithms.
Addison-Wesley, Reading MA, 1969.

\bibitem{Landin65}
Peter Landin.
A correspondence between Algol 60 and Church's lambda notation: Part I.
{\em Communications of the ACM} 8(2):89--101, February 1965.

\bibitem{Matula68}
David W. Matula.
In-and-Out Conversions.
{\em Communications of the ACM} 11(1):47--50, January 1968.

\bibitem{Matula70}
David W. Matula.
A Formalization of Floating-Point Numeric Base Conversion.
{\em IEEE Transactions on Computers} C-19, 8:681-692, August 1970.

\bibitem{MITScheme}
MIT Department of Electrical Engineering and Computer Science.
Scheme manual, seventh edition.
September 1984.

\bibitem{Penfield81}
Paul Penfield, Jr.
Principal values and branch cuts in complex APL.
In {\em APL '81 Conference Proceedings,} pages 248--256.
ACM SIGAPL, San Francisco, September 1981.
Proceedings published as {\em APL Quote Quad} 12(1), ACM, September 1981.

\bibitem{Pitman83}
Kent M.~Pitman.
The revised MacLisp manual (Saturday evening edition).
MIT Laboratory for Computer Science Technical Report 295, May 1983.

\bibitem{Rees82}
Jonathan A.~Rees and Norman I.~Adams IV.
T: A dialect of Lisp or, lambda: The ultimate software tool.
In {\em Conference Record of the 1982 ACM Symposium on Lisp and
  Functional Programming}, pages 114--122.

\bibitem{R3RS}
Jonathan Rees and William Clinger, editors.
The revised$^3$ report on the algorithmic language Scheme.
In {\em ACM SIGPLAN Notices} 21(12), ACM, December 1986.

\bibitem{Reynolds72}
John Reynolds.
Definitional interpreters for higher order programming languages.
In {\em ACM Conference Proceedings}, pages 717--740.
ACM, \todo{month?}~1972.

\bibitem{Rabbit}
Guy Lewis Steele Jr.
Rabbit: a compiler for Scheme.
MIT Artificial Intelligence Laboratory Technical Report 474, May 1978.

\bibitem{CLtL}
Guy Lewis Steele Jr.
{\em Common Lisp: The Language.}
Digital Press, Burlington MA, 1984.

\bibitem{CLtL2}
Guy Lewis Steele Jr.
{\em Common Lisp: The Language.} 2d ed.
Digital Press, Bedford MA, 1990.

\bibitem{Scheme78}
Guy Lewis Steele Jr.~and Gerald Jay Sussman.
The revised report on Scheme, a dialect of Lisp.
MIT Artificial Intelligence Memo 452, January 1978.

\bibitem{Heuristic}
Guy Lewis Steele Jr.~and Jon L White.
How to Print Floating-Point Numbers Accurately.
In {\em Proceedings of the 1990 ACM SIGPLAN Conference on Programming
  Language Design and Implementation}.  Forthcoming.

\bibitem{Stoy77}
Joseph E.~Stoy.
{\em Denotational Semantics: The Scott-Strachey Approach to
  Programming Language Theory.}
MIT Press, Cambridge, 1977.

\bibitem{Scheme75}
Gerald Jay Sussman and Guy Lewis Steele Jr.
Scheme: an interpreter for extended lambda calculus.
MIT Artificial Intelligence Memo 349, December 1975.

\bibitem{Vuillemin}
Jean Vuillemin.
Exact real computer arithmetic with continued fractions.
In {\em Proceedings of the 1988 ACM Conference on Lisp and
  Functional Programming}, pages 14--27.

\end{thebibliography}

