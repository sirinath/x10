\subsection{Posting statements on a clock}\index{clock!now}\label{sec:clock:now}
\Xten{} provides syntactic support for a common idiom. Often it may be
necessary for an activity $A$ to require that a certain set of
statements be executed to completion before a clock $c$ can move
forward, without $A$ actually waiting for the completion
of the statement. We introduce the syntax:
\begin{x10}
461 Statement ::= NowStatement
471 StatementNoShortIf ::= 
       NowStatementNoShortIf
479 NowStatement ::= 
       now ( Clock ) Statement
489 NowStatementNoShortIf ::= 
       now ( Clock ) StatementNoShortIf
\end{x10}
\noindent 

A statement {\tt now (c) s} may be considered as shorthand for
\begin{x10}
  async clocked(c) \{ 
     finish async s; 
  \}
\end{x10}

\paragraph{Note.} Because of the static semantics of {\tt finish}
it is not possible to nest {\cf now} statements. Instead if it proves
useful, we may introduce a multi-clocked {\tt now} statement,
which permits the statement to be posted on multiple clocks
simultaneously.
\begin{x10}
479' NowStatement ::= 
       now ( ClockList ) Statement
489' NowStatementNoShortIf ::= 
       now ( ClockList ) StatementNoShortIf  
\end{x10}
