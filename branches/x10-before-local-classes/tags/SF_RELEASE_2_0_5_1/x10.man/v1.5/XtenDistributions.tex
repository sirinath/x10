\section{Distributions}\label{XtenDistributions}
\index{distribution}

A {\em distribution} is a mapping from a region to a set of places.
{}\Xten{} provides a built-in value class, {\tt
x10.lang.distribution}, to allow the creation of new distributions and
to perform operations on distributions. This class is {\tt final} in
{}\XtenCurrVer; future versions of the language may permit
user-definable distributions. Since distributions play a dual role
(values as well as types), variables of type {\cf distribution} must
be initialized and are implicitly {\tt final}.

The {\em rank} of a distribution is the rank of the underlying region.

%Recall that each program runs in a fixed number of places, determined
%by runtime parameters. The static constant place.MAX_PLACES specifies
%the maximum number of places. The collection of places is assumed to
%be totally ordered.


\begin{x10}
region R = [1:100]
distribution D = distribtion.factory.block(R);
distribution D = distribution.factory.cyclic(R);
distribution D = R -> here;
distribution D = distribution.factory.random(R);
\end{x10}

Let {\cf D} be a distribution. {\cf D.region} denotes the underlying
region. {\cf D.places} is the set of places constituting the range of
{\cf D} (viewed as a function). Given a point {\cf p}, the expression
{\cf D[p]} represents the application of {\cf D} to {\cf p}, that is,
the place that {\tt p} is mapped to by {\tt D}. The evaluation of the
expression {\tt D[p]} throws an {\tt ArrayIndexOutofBoundsException}
if {\tt p} does not lie in the underlying region.

When operated on as a distribution, a region {\cf R} implicitly
behaves as the distribution mapping each item in {\cf R} to {\cf here}
(i.e.{} {\cf R->here}, see below). Conversely, when used in a context
expecting a region, a distribution {\tt D} should be thought of as
standing for {\tt D.region}.

{}\todo{Allan: We do not specify how the values of an array at a place
are stored, e.g. in row-major or column major order. Need to work this
out.}

\subsection{Operations returning distributions}

Let {\cf R} be a region, {\cf Q} a set of places {\cf {}\{p1,...,pk\}}
(enumerated in canonical order), and {\cf P} a place. All the operations
described below may be performed on {\tt distribution.factory}.

\paragraph{Unique distribution} \index{distribution!unique}
The distribution {\cf unique(Q)} is the unique distribution from the
region {\cf 1:k} to {\cf Q} mapping each point {\cf i} to {\cf pi}.

\paragraph{Constant distributions.} \index{distribution!constant}
The distribution {\cf R->P} maps every point in {\cf R} to {\cf P}.

\paragraph{Block distributions.}\index{distribution!block}
The distribution {\cf block(R, Q)} distributes the elements of {\cf R}
(in order) over the set of places {\cf Q} in blocks  as
follows. Let $p$ equal {\cf |R| div N} and $q$ equal {\cf |R| mod N},
where {\cf N} is the size of {\cf Q}, and 
{\cf |R|} is the size of {\tt R}.  The first $q$ places get
successive blocks of size $(p+1)$ and the remaining places get blocks of
size $p$.

The distribution {\cf block(R)} is the same distribution as {\cf
block(R, place.places)}.

\todo{Check into block distributions per dimension.}
\paragraph{Cyclic distributions.} \index{distribution!cyclic}
The distribution {\cf cyclic(R, Q)} distributes the points in {\cf R}
cyclically across places in {\cf Q} in order.

The distribution {\cf cyclic(R)} is the same distribution as {\cf 
cyclic(R, place.places)}.

Thus the distribution {\cf cyclic(place.MAX\_PLACES)} provides a $1-1$
mapping from the region {\cf place.MAX\_PLACES} to the set of all
places and is the same as the distribution {\cf unique(place.places)}.

\paragraph{Block cyclic distributions.}\index{distribution!block cyclic}
The distribution {\cf blockCyclic(R, N, Q)} distributes the elements
of {\cf R} cyclically over the set of places {\cf Q} in blocks of size
{\cf N}.

\paragraph{Arbitrary distributions.} \index{distribution!arbitrary}
The distribution {\cf arbitrary(R,Q)} arbitrarily allocates points in {\cf
R} to {\cf Q}. As above, {\cf arbitrary(R)} is the same distribution as
{\cf arbitrary(R, place.places)}.

\todo{Determine which other built-in distributions to provide.}

\paragraph{Domain Restriction.} \index{distribution!restriction!domain}

If {\cf D} is a distribution and {\cf R} is a sub-region of {\cf
D.domain}, then {\cf D | R} represents the restriction of {\cf D} to
{\cf R}.  The compiler throws an error if it cannot determine that
{\cf R} is a sub-region of {\cf D.domain}.

\paragraph{Range Restriction.}\index{distribution!restriction!range}

If {\cf D} is a distribution and {\cf P} a place expression, the term
{\cf D | P} denotes the sub-distribution of {\cf D} defined over all the
points in the domain of {\cf D} mapped to {\cf P}.

\cbstart
Note that {\cf D | here} does not necessarily contain adjacent points
in {\cf D.region}. For instance, if {\cf D} is a cyclic distribution,
{\cf D | here} will typically contain points that are {\cf P} apart,
where {\cf P} is the number of places. An implementation may find a
way to still represent them in contiguous memory, e.g.{} using a
complex arithmetic function to map from the region index to an index
into the array.
\cbend

\subsection{User-defined distributions}\index{distribution!user-defined}

Future versions of \Xten{} may provide user-defined distributions, in
a way that supports static reasoning.

\todo{TBD. Make distribution a value type. What is the API? Return a
set of indices. For each index point, return the place. A method to
return a subdistribution given a subregion. A method to check if a
given distribution is a subdistribution. But may need to provide methods that the compiler can use to reason about the distribution.

Postpone to 0.7.}

\subsection{Operations on Distributions}

A {\em sub-distribution}\index{sub-distribution} of {\cf D} is any
distribution {\cf E} defined on some subset of the domain of {\cf D},
which agrees with {\cf D} on all points in its domain. We also say
that {\cf D} is a {\em super-distribution} of {\cf E}. A distribution
{\cf D1} {\em is larger than} {\cf D2} if {\cf D1} is a
super-distribution of {\cf D2}.

Let {\cf D1} and {\cf D2} be two distributions.  


\paragraph{Intersection of distributions.}\index{distribution!intersection}
{\cf D1 \&\& D2}, the intersection of {\cf D1} and {\cf D2}, is the
largest common sub-distribution of {\cf D1} and {\cf D2}.

\paragraph{Asymmetric union of distributions.}\index{distribution!union!asymmetric}
{\cf D1.overlay(D2}, the asymmetric union of {\cf D1} and {\cf D2}, is the
distribution whose domain is the union of the regions of {\cf D1} and
{\cf D2}, and whose value at each point {\cf p} in its domain is {\cf D2[p]}
if {\cf p} lies in {\cf D.domain} otherwise it is {\cf D1[p]}. ({\tt D1} provides the defaults.)

\paragraph{Disjoint union of distributions.}\index{distribution!union!disjoint}
{\cf D1 || D2}, the disjoint union of {\cf D1} and {\cf D2}, is
defined only if the domains of {\cf D1} and {\cf D2} are disjoint. Its
value is {\cf D1.overlay(D2)} (or equivalently {\cf D2.overlay(D1)}.
(It is the least super-distribution of {\cf D1} and {\cf D2}.)

\paragraph{Difference of distributions.}\index{distribution!difference}
{\cf D1 - D2} is the largest sub-distribution of {\cf D1} whose domain is
disjoint from that of {\cf D2}.


\subsection{Example}
\begin{x10}
    double[D] dotProduct(T[D] a, T[D] b) \{
    return (new T[1:D.places] (point j) \{
        return (new T[D | here] (point i) \{
            return a[i]*b[i];
        \}).sum();
   \}).sum();
\}
\end{x10}

This code returns the inner product of two {\cf T} vectors defined
over the same (otherwise unknown) distribution. The result is the sum
reduction of an array of {\cf T} with one element at each place in the
range of {\cf D}. The value of this array at each point is the sum
reduction of the array formed by multiplying the corresponding
elements of {\cf a} and {\cf b} in the local sub-array at the current
place.



