\chapter{Interfaces}
\label{XtenInterfaces}\index{interfaces}

{}\XtenCurrVer{} interfaces are essentially the same \java{}
interfaces \cite[\S 9]{jls2}. An interface primarily specifies
signatures for public methods. It may extend multiple interfaces. 
%The
%need for magic constants in interfaces is lessened with the
%introduction of {\tt enum} (\Sref{XtenEnums}).


Future version of \Xten{} will introduce additional structure in
interface definitions that will allow the programmer to state
additional properties of classes that implement that interface. For
instance a method may be declared {\tt pure} to indicate that its
evaluation cannot have any side-effects. A method may be declared {\tt
local} to indicate that its execution is confined purely to the
current place (no communication with other places). Similarly,
behavioral properties of the method as they relate to the usage of
clocks of the current activity may be specified.

\section{Interfaces with properties}\label{DepType:Interface}

\Xten{} permits interfaces to have properties and specify an interface
invariant. This is necessary so that programmers can build dependent
types on top of interfaces and not just classes.

\begin{grammar}
NormalInterfaceDeclaration \:
      InterfaceModifiers\opt \xcd"interface" Identifier  \\
   && TypeParameterList\opt PropertyList\opt Constraint\opt \\
   && ExtendsInterfaces\opt InterfaceBody \\
\end{grammar}

\noindent
The invariant associated with an interface is the conjunction of the
invariants associated with its superinterfaces and the invariant
defined at the interface. 


\begin{staticrule*}
   The compiler declares an error if this constraint
   is not consistent (\Sref{DepType:Consistency}).  
\end{staticrule*}

Each interface implicitly defines a nullary getter method
\xcd"def p(): T" for each property \xcd"p: T". 

\begin{staticrule*}
   The compiler issues a warning if the programmer
   explicitly defines a method with this signature for an interface.
\end{staticrule*}

A class \xcd"C" (with properties) is said to implement an interface \xcd"I" if
\begin{itemize}
  \item its properties contains all the properties of \xcd"I",
\item its class invariant
$\mathit{inv}($\xcd"C"$)$ implies
$\mathit{inv}($\xcd"I"$)$.
\end{itemize}

