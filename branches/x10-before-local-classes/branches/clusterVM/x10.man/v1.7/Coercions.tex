\section{Coercions and conversions}
\label{XtenConversions}
\index{conversions}\index{coercions}

\XtenCurrVer{} supports the following coercions and conversions

\subsection{Coercions}

A {\em coercion} does not change object identity;
a coerced object may be explicitly coerced back to its original
type through a cast.

\paragraph{Subsumption coercion.}
A subtype may be implicitly coerced to any supertype.
\index{coercions!subsumption coercion}

\paragraph{Explicit coercion (casting with \xcd"as")}
A reference type may be explicitly coerced to any other
reference type using the \xcd"as" operation.
If the value coerced is not an instance of the target type,
a \xcd"TypeCastException" is thrown.  Casting to a constrained
type may require a run-time check that the constraint is
satisfied.
\index{coercions!explicit coercion}
\index{casting}
\index{\Xcd{as}}

\subsection{Conversions}

A {\em conversion} may change object identity if the type being
converted to is not the same as the type converted from.

\paragraph{Explicit conversions.}

A value of type \xcd"T" may be explicity converted to a value of type
\xcd"U" using an \xcd"as" operation if it may be implicitly
converted (below), or if there is an explicit numeric conversion defined
from \xcd"T" to \xcd"U",
or if there is a \xcd"static" conversion operator defined
on the target type \xcd"U" that takes a value of the source type
\xcd"T" and returns a value of type \xcd"U".

\paragraph{Implicit conversions.}

A value of type \xcd"T" may be implicity converted to a value of type
\xcd"U" using an \xcd"as" operation if there is an implicit numeric conversion defined
from \xcd"T" to \xcd"U",
or if there is a \xcd"static" implicit conversion operator defined
on the target type \xcd"U" that takes a value of the source type
\xcd"T" and returns a value of type \xcd"U".

\paragraph{Numeric conversion.}

\index{conversions!numeric conversions}

Conversions of certain numeric types are handled specially by the
compiler.
The \emph{numeric types} are the integer types and the floating
point types.
The \emph{integer types} the signed and unsigned integer types.
The \emph{signed integer types} are the subtypes of
\xcd"Byte", \xcd"Short", \xcd"Int", and \xcd"Long".
The \emph{unsigned integer types} are the subtypes of
\xcd"UByte", \xcd"UShort", \xcd"UInt", and \xcd"ULong".
The \emph{floating point types} are the subtypes of
\xcd"Float" and \xcd"Double".

A numeric type may be explicitly converted to any other numeric
type using the \xcd"as" operation.

An integer type may be implicitly converted to a wider numeric
type of the same signedness.
In particular, an implicit conversion may be performed between
the following types, transitively:
\begin{itemize}
\item \xcd"Byte" to \xcd"Short"
\item \xcd"Short" to \xcd"Int"
\item \xcd"Int" to \xcd"Long"
\item \xcd"UByte" to \xcd"UShort"
\item \xcd"UShort" to \xcd"UInt"
\item \xcd"UInt" to \xcd"ULong"
\end{itemize}

An integer type may be implicitly converted to a wider floating point
type.  
In particular, an implicit conversion may be performed between
the following types, transitively:
\begin{itemize}
\item \xcd"Short" to \xcd"Float"
\item \xcd"UShort" to \xcd"Float"
\item \xcd"Int" to \xcd"Double"
\item \xcd"UInt" to \xcd"Double"
\end{itemize}

There is no implicit conversion from \xcd"Int" or
\xcd"UInt" to
\xcd"Float" since not not all integers can be 
exactly represented as a
\xcd"Float".
Similarly, there is no implicit conversion from \xcd"Long" or
\xcd"ULong" to
\xcd"Double".

A \xcd"Float" may be implicitly converted to a \xcd"Double".

Implicit conversions are also allowed if the constraint on the
source type ensures the value can be represented exactly in the
target type.  In particular,
the following implicit conversions
are permitted for a value of any integer type:
\begin{itemize}
\item
An integer can be converted 
to \xcd"Byte" if it is between $-2^{7}$ and $2^{7}-1$, inclusive. 

\item
An integer can be converted 
to \xcd"Short" if it is between $-2^{15}$ and $2^{15}-1$, inclusive. 

\item
An integer can be converted 
to \xcd"Int" if it is between $-2^{31}$ and $2^{31}-1$, inclusive. 

\item
An integer can be converted 
to \xcd"Long" if it is between $-2^{63}$ and $2^{63}-1$, inclusive. 

\item
An integer can be converted 
to \xcd"UByte" if it is between $0$ and $2^{8}-1$, inclusive. 

\item
An integer can be converted 
to \xcd"UShort" if it is between $0$ and $2^{16}-1$, inclusive. 
\item

An integer can be converted 
to \xcd"UInt" if it is between $0$ and $2^{32}-1$, inclusive. 
\item

An integer can be converted 
to \xcd"ULong" if it is between $0$ and $2^{64}-1$, inclusive. 

\item
An integer can be converted 
to \xcd"Float" if it is between $-2^{24}$ and $2^{24}$, inclusive. 

\item
An integer can be converted 
to \xcd"Double" if it is between $-2^{53}$ and $2^{53}$, inclusive.
\end{itemize}

A floating point number cannot be implicitly converted to any 
integer type.


\paragraph{Boxing conversion.}
% @@@ KZ intro Box type earlier, along with Object, Int, etc
A type \xcd"T" may be implicitly converted to the
boxed type \xcd"Box[T]".

\index{conversions!boxing conversions}
\index{autoboxing}
\index{boxing}

\paragraph{Unboxing conversion.}
% @@@ KZ intro Box type earlier, along with Object, Int, etc
A value type \xcd"Box[T]" may be implicitly converted to the
type \xcd"T".

\index{conversions!unboxing conversions}
\index{autoboxing}
\index{unboxing}

\paragraph{Explicit conversion (casting with \xcd"as")}
A type may be explicitly converted to any type to which it can be
implicitly converted or implicitly or explicitly coerced.

\index{conversions!explicit conversion}
\index{casting}
\index{\Xcd{as}}

\paragraph{String conversion.}
Any object that is an operand of the binary
\xcd"+" operator may
be converted to \xcd"String" if the other operand is a \xcd"String".
A conversion to \xcd"String" is performed by invoking the \xcd"toString()"
method of the object.

\index{conversions!string conversion}

