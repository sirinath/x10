\section{Place types}\label{PlaceTypes}\index{place types}
Recall that an \Xten{} computation spans multiple places
(\S~\ref{XtenPlaces}). Each place constains data and activities that
operate on that data.  \XtenCurrVer{} does not permit the dynamic
creation of a place. Each \Xten{} computation is initiated with a
fixed number of places, as determined by a configuration parameter.
In this section we discuss how the programmer may supply place type
information, thereby allowing the compiler to check data locality,
i.e.{} that data items being accessed in an atomic section are local.

\begin{x10}
422   Type ::= DataType PlaceTypeSpecifieropt
428   PlaceTypeSpecifier ::= AT PlaceType
429   PlaceType ::= ?
431     | current
432     | PlaceExpression
\end{x10}

\paragraph{The anywhere place type}\label{anywhere}\index{? place type}
The simplest piece of place information that a programmer can specify
is no information at all. The type {\tt F!?} expresses this: it is
satisfied by an object of datatype {\tt F} located at any place.  This
type is particularly useful in specifying collections of objects at
heterogenous places. For intance the type {\tt F!?[.]} is the type of
all distributed arrays whose elements can contain references to
objects of datatype {\tt F} located at any place. Thus this type specifies
no static relationship between the places of two different array elements.

A datatype {\tt F} occurring in a place where a type is expected is
always taken as shorthand for {\tt F!?}. This is compatible with the
Principle of Least Disclosure: the programmer may quickly get his code
running without specifying where objects are located. The compiler
will insert dynamic runtime checks to ensure that locality conditions
are satisfied within atomic sections. As the programmer discovers more
information about the locality of data elements s/he may supply more
refined place type information.

\paragraph{Place expression in place types.}
The programmer may use the indexical place expression {\tt here} as a
place type. Recall that in any runtime context, {\tt here} stands for
the place of the current activity (\S~\ref{Here}). {\tt here} cannot
be used to specify the placetype of any object field because the
type of a field cannot vary with the activity accessing the field.

Consider the example:
\begin{x10}
public class F \{
  int datum;
  int m(F!here a, F!here b) \{
    return a.datum + b.datum;
 \}
\}
\end{x10}
This code satisfies \Xten{}'s typing rules. Recall that 
an object's fields can be accessed only if the object is located in 
the same place as the activity. Hence in the body of the method
{\tt m}, the compiler must check that {\tt a} and {\tt b} are statically
known to be local. But this follows from the type declarations
of the method parameters. 

The programmer may also specify the place information {\tt P} for any
final variable {\tt P} of type place. This may allow the compiler to
check locality conditions for remote activities at place {\tt P}. For
example:
\begin{x10}
public class F \{
  int datum;
  place P;
  public F(place P) \{
    this.P = P;
  \}
  future<int> m(final F!P a, final F!P b) \{
    return future(P) {a.datum + b.datum;}
  \}
\}
\end{x10}
In this class the constructor for {\tt F} takes a place argument and
stores it in a final field. This field may be used in types in the
body of the class, e.g.{} to specify the types of method parameters,
as stated above. The body of the method {\tt m} typechecks because the
compiler can verify that at the place {\tt P} the values contained in
the variables {\tt a} and {\tt b} are local, hence it is legal to 
access their fields. 


In many circumstances it becomes necessary to use final variables of type
{\tt place} as 
\Xten{} distinguishes two kinds of places: {\em shared
places}\index{places!shared}\label{SharedPlaces}, which correspond to
the memory of individual processors which is shared across multiple
activities, and {\em scoped
places}\index{places!scoped}\label{ScopedPlaces} such as threadlocal
memory and method memory which is available only in limited
scope. \XtenCurrVer{} supports only {\tt threadlocal} scoped memory,
i.e.{} memory accessible only to the current activity.  Future
versions may support {\tt methodlocal} and {\tt blocklocal} memory.

\paragraph{ Future work }

In future versions, \Xten{} will support value and type parameters for
classes, interfaces and methods. This will permit code of the form:

\begin{x10}
public class F \{
  int datum;
  <P instance place> future<int> m(F!P a, F!P b) \{
    return future(P) {a.datum + b.datum;}
  \}
\}
\end{x10}
The class {\tt F} specifies a generic method {\tt m} which requires that 
its two arguments must be of data type {\tt F} and must both reside
at the same unknown place {\tt P}. Now the body of the {\cf future} expression
can be type-checked by the compiler, without knowing the identity 
of the place {\tt P}.

Similarly in the body of a generic class with a place parameter {\tt
P} one may construct the type {\tt F!P[.]} to represent the type of
distributed arrays each of whose elements must contain a reference to
an object of datatype {\tt F} located at the place {\tt P}.


An object can be cast to a particular place. If the object is not at
the right place, a {\tt BadPlaceException} is thrown:

\begin{x10}
m(Object obj) \{ // accept an object at any place
     // Cast it to here. This may throw
     // an exception if the object is not local.
     Object!here h = (Object!here)obj;
     // The object is local, invoke a
     // method on it synchronously.
     h.m1();
 \}
\end{x10}

Places can be checked for equality. One can view this as the analog of
the {\tt instanceof} operator for places.

\begin{x10}
m(Object obj) \{
    if (here == obj.location) \{
       // will never throw an exception
      Object!here h = (Object!here)obj;
      h.m();
   \}
\}
\end{x10}


\paragraph{{\tt current}}
Similarly the object reference type {\tt current} can be used in
%%(i)~constructing a distribution for a reference array or (ii)
specifying the location of the base type of a reference array. 
%% In the first case points mapped to {\tt current} by the distribution will
%% reside in the same place as the array object itself, and in the second
%% case 
The value of the array at a particular point is an object in the
same place as that array component. Example:
\begin{x10}
    Object!current[] objects;
\end{x10}

\Xten{} permits user-definable place constants (= final variables =
variables that must be assigned before use, and can be assigned only
once). Place constants may be used in type expressions after the {\cf !}
sign. For instance, consider the following class definition:

\begin{x10}
 public class Cell \{
  Object!P value;

  public Cell( Object!P value ) \{  
     this.value = value;
  \}

  public Object!P getValue() \{
     return this.value;
  \}

  public void setValue( Object!P value ) \{
     this.value = value;
  \}
 \}
\end{x10}

This class may be used thus:

\begin{x10}
Cell cell = 
   new Cell(new Point!Q());
\end{x10}


