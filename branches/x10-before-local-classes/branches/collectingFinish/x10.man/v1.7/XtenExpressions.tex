\chapter{Expressions}\label{XtenExpressions}\index{expressions}

\Xten{} supports a rich expression language similar to
\Java{}'s.
Evaluating an expression produces a value, which may be either
an instance of a value class or an instance of a reference
class. Expressions may also be \xcd"void"; that is, they produce no
value.
Expression evaluation may have side effects: assignment to a
variable, allocation, method calls, or exceptional control-flow.
Evaluation is strict and is performed left to right.

\section{Literals}

\Xten{} supports the following literal expressions: 
\begin{itemize}
\item A 32-bit integer literal is a value of type \xcd"x10.lang.Int".
\item A 64-bit long literal is a value of type \xcd"x10.lang.Long".
\item A 32-bit unsigned integer literal is a value of type \xcd"x10.lang.UInt".
\item A 64-bit unsigned long literal is a value of type \xcd"x10.lang.ULong".
\item A 32-bit floating-point literal is a value of type \xcd"x10.lang.Float".
\item A 64-bit floating-point literal is a value of type \xcd"x10.lang.Double".
\item A character literal is a value of type \xcd"x10.lang.Char".
\item A string literal is a value of type \xcd"x10.lang.String".
\item The boolean literals \xcd"true" and \xcd"false" are of type
\xcd"x10.lang.Boolean".
\item The \xcd"null" literal is of the null type, a subtype of all reference types.
\end{itemize}

\section{\Xcd{this}}

\begin{grammar}
ThisExpression \: \xcd"this" \\
\| ClassName \xcd"." \xcd"this" \\
\end{grammar}

The expression \xcd"this" is a final local variable containing a reference
to an instance of the lexically enclosing class.
It may be used only within the body of an instance method, a
constructor, or in the initializer of a instance field.

Within an inner class, \xcd"this" may be qualified with the
name of a lexically enclosing class.  In this case, it
represents an instance of that enclosing class.

The type of a \xcd"this" expression is the
innermost enclosing class, or the qualifying class,
constrained by the class invariant and the
method guard, if any.

The \xcd"this" expression may also be used within constraints in
a class or interface header (the class invariant and
\xcd"extends" and \xcd"implements" clauses).  Here, the type of
\xcd"this" is restricted so that only properties declared in the
class header itself, and specifically not any members declared in the class
body or in supertypes, are accessible through \xcd"this".

\section{Local variables}


\begin{grammar}
LocalExpression \: Identifier \\
\end{grammar}

A local variable expression consists simply of the name of the
local variable.


\section{Field access}
\label{FieldAccess}


\begin{grammar}
FieldExpression \: Expression \xcd"." Identifier \\
                \| \xcd"super" \xcd"." Identifier \\
                \| ClassName \xcd"." Identifier \\
                \| ClassName \xcd"." \xcd"super" \xcd"." Identifier \\
\end{grammar}

A field of an object instance may be  accessed
with a field access expression.

The type of the access is the declared type of the field with the
actual target substituted for \xcd"this" in the type.  If the actual
target is not a final access path (\Sref{FinalAccessPath}),
an anonymous path is substituted for \xcd"this".

The field accessed is selected from the fields and value properties
of the static type of the target and its superclasses.

If the field target is given by the keyword \xcd"super",
the target's type is
the superclass of the enclosing class, as
constrained by the superclass's class invariant, if any.

If the field target is \xcd"null", a \xcd"NullPointerException"
is thrown.

If the field target is a class name, a static field is selected.

It is illegal to access  a field that is not visible from
the current context.
It is illegal to access a non-static field
through a static field access expression.

\section{Closures}
\label{Closures}

\Xten{} provides first-class, typed functions, including
\emph{closures},
\emph{operator functions},
and \emph{method selectors}.

\begin{grammar}
ClosureExpression \:
        TypeParameters\opt
        \xcd"("
        Formals\opt
        \xcd")"
        ReturnType\opt
        Throws\opt
        \xcd"=>" ClosureBody \\
ClosureBody \:
        Expression \\
        \| \xcd"{" Statement\star \xcd"}" \\
        \| \xcd"{" Statement\star Expression \xcd"}" \\
\end{grammar}

Closure expressions have zero or more type parameters,
zero or more formal parameters, an an optional return type and
optional set of exceptions throws by the closure body.  The
closure body has the same syntax as a method body; it may be
either an expression, a block of statements, or a block
terminated by an expression to return.

The body of the closure is evaluated with the closure is
invoked by a call expression (\Sref{Call}).

As with methods, a closure with return type \xcd"Void" cannot
have a terminating expression. 
If the return type is omitted, it is inferred, as described in
\Sref{TypeInference}.
It is a static error if the return type cannot be inferred.

\begin{example}
The following method takes a function parameter and uses it to
test each element of the list, returning the first matching
element.
\begin{xten}
def find[T](f: T => Boolean, xs: List[T]): T {
  for (x: T in xs)
    if (f(x)) return x;
  null
}
\end{xten}

The method may be invoked with a closure:
\begin{xten}
xs: List[Int] = ...;
x: Int = find((x: Int) => (x>0), xs);
\end{xten}
\end{example}

As with a normal method, the closure may have a \xcd"throws" clause. It is a
static error if a checked exception is thrown that is not declared in
the \xcd"throws" clause.

\subsection{Closures are objects}

Closures, like all first-class functions in \Xten{} are
objects
(\Sref{FunctionsAreObjects}).
As objects, the closure body may refer to
\xcd"this", which is a reference to the current function,
and use it to invoke the closure recursively.

\begin{example}
The following method uses a closure to compute the $n$th
Fibonacci number.
\begin{xten}
def fib(n: Int): Int {
    val f = (n: Int) : Point {
        if (n < 1) return new Point(1,0);
        val (n,n1): Point = this(n-1);
        new Point(n+n1,n)
    }
    val (n,n1): Point = f(n);
    n
}
\end{xten}
\end{example}

\subsection{Outer variable access}

The body \xcd"s" in a function
\xcdmath"(x$_1$: T$_1$, $\dots$, x$_n$: T$_n$) => { s }"
may access fields
of enclosing classes and local variable declared in an outer scope.

Recall that languages such as \java{} require that methods may
access only those local variables declared in an enclosing scope
("outer variables") which are final. This is valuable in
preventing accidental races between multiple closures reading
and writing the same outer variable. At the same time, it is
desirable to support the following common idiom of expression:

\begin{xten}
def allPositive(c: Collection): Boolean {
  shared var result: Boolean = true;
  c.applyToAll((x: Int) => { if (x < result) atomic {result=false;}});
  return result;
}
\end{xten}

This motivates the following rule:

\begin{staticrule*}
In an expression
\xcdmath"(x$_1$: T$_1$, $\dots$, x$_n$: T$_n$) => e",
any outer local variable accessed by \xcd"e" must be final or
must be declared
as \xcd"shared" (\Sref{Shared}).
\end{staticrule*}

The closure body may refer to instances of enclosing classes using
the syntax \xcd"C.this", where \xcd"C" is the name of the
enclosing class.

\begin{note}
The main activity may run in parallel with any
closures it creates. Hence even the read of an outer variable by the
body of a closure may result in a race condition, Since closures are
first-class, the analysis of whether a closure may execute in parallel
with the activity that created it is very difficult.
\end{note}

\begin{note}
The rule for accessing outer variables from closure bodies
should be the same as the rule for accessing outer variables from local
or anonymous classes.
\end{note}

\section{Methods selectors}
\label{MethodSelectors}

A method selector expression allows a method to be used as a
first-class function.
\eat{
We can think of a method definition:

\begin{xtenmath}
def m[X$_1$, $\dots$, X$_2$](x$_1$: T$_1$, $\dots$, x$_n$: T$_n$): U = e
\end{xtenmath}

as binding the closure

\begin{xtenmath}
[X$_1$, $\dots$, X$_2$](x$_1$: T$_1$, $\dots$, x$_n$: T$_n$): U => e
\end{xtenmath}

to the method-name--argument-list tuple (\xcd"m", \xcd"X",
\xcd"x: T").  That is, underlying every method is a function.

This motivates us to introduce the \emph{method selector}
expression.
}

\begin{grammar}
MethodSelector \:
        Primary \xcd"."
        MethodName \xcd"."
                TypeParameters\opt \xcd"(" Formals\opt \xcd")" \\
      \|
        TypeName \xcd"."
        MethodName \xcd"."
                TypeParameters\opt \xcd"(" Formals\opt \xcd")" \\
\end{grammar}

For a
type \xcd"T", a list of types
(\xcdmath"T$_1$", \dots, \xcdmath"T$_n$"), 
a method name
\xcd"m" and an expression \xcd"e" of type \xcd"T",
\xcdmath"e.m.(T$_1$, $\dots$, T$_n$)" denotes the function,
if any, bound to the instance method named \xcd"m" at type
\xcd"T" whose argument
type is
\xcdmath"(T$_1$, $\dots$, T$_n$), with \xcd"this" in the method
body bound to the value obtained
by evaluating \xcd"e". The
return type of the function is specified in the method declaration.

Thus, the method selector

\begin{xtenmath}
e.m.[X$_1$, $\dots$, X$_m$](T$_1$, $\dots$, T$_n$)
\end{xtenmath}

behaves as if it were the closure

\begin{xtenmath}
(X$_1$, $\dots$, X$_m$](x$_1$: T$_1$, $\dots$, x$_n$: T$_n$) => e.m[X$_1$, $\dots$, X$_m$](x$_1$, $\dots$, x$_n$)
\end{xtenmath}

\begin{note}
Because of overloading, a method name is not sufficient to
uniquely identify a function for a given class (in Java-like languages).
One needs the argument type information as well.
The selector syntax (dot) is used to distinguish \xcd"e.m()" (a
method invocation on \xcd"e" of method named \xcd"m" with no arguments)
from \xcd"e.m.()"
(the function bound to the method). 
\end{note}

A static method provides a binding from a name to a function that is
independent of any instance of a class; rather it is associated with the
class itself. The static function selector
\xcdmath"T.m.(T$_1$, $\dots$, T$_n$)" denotes the
function bound to the static method named \xcd"m", with argument types
\xcdmath"(T$_1$, $\dots$, T$_n$) for the type \xcd"T". The return type
of the function is specified by the declaration of \xcd"T.m".

Users of a function type do not care whether a function was defined
directly (using the closure syntax), or obtained via (static or
instance) function selectors.

\begin{note}
Design note: The function selector syntax is consistent with the
reinterpretation of the usual method invocation syntax
\xcdmath"e.m(e$_1$,..., e$_n$)"
into a function specifier, \xcd"e.m", applied to a tuple of arguments
\xcdmath"(e$_1$,..., e$_n$)". Note that the receiver is not
treated as ``an extra argument'' to the
function. That would break the above approach.
\end{note}


\section{Operator functions}

Every operator (e.g.,
\xcd"+",
\xcd"-",
\xcd"*",
\xcd"/",
\dots) has a family of functions, one for
each type on which the operator is defined. The function can be
selected using the "." syntax:

\begin{grammar}
OperatorFunction
        \: TypeName \xcd"." Operator \xcd"(" Formals\opt \xcd")" \\
        \| TypeName \xcd"." Operator \\
\end{grammar}

If an operator has more than one arity (e.g., unary and binary
\xcd"-"), the appropriate version may be selected by giving the
formal parameter types.  The binary version is selected by
default.
For example, the following equivalences hold:

\begin{xtenmath}
String.+             $\equiv$ (x: String, y: String): String => x + y
Long.-               $\equiv$ (x: Long, y: Long): Int => x - y
Float.-(Float,Float) $\equiv$ (x: Float, y: Float): Int => x - y
Int.-(Int)           $\equiv$ (x: Int): Int => -x
Boolean.&            $\equiv$ (x: Boolean, y: Boolean): Boolean => x & y
Boolean.!            $\equiv$ (x: Boolean): Boolean => !x
Int.<(Int,Int)       $\equiv$ (x: Int, y: Int): Boolean => x < y
Dist.|(Place)        $\equiv$ (d: Dist, p: Place): Dist => d | p
\end{xtenmath}

Unary and binary promotion (\Sref{XtenPromotions}) is not performed
when invoking these
operations; instead, the operands are coerced individually via implicit
coercions (\Sref{XtenConversions}), as appropriate.

Additionally, for every expression \xcd"e" of a type \xcd"T" at which a binary
operator \xcd"OP" is defined, the expression \xcd"e.OP" or
\xcd"e.OP(T)" represents the function
defined by:

\begin{xten}
(x: T): T => { e OP x }
\end{xten}

\begin{grammar}
Primary \: Expr \xcd"." Operator \xcd"(" Formals\opt \xcd")" \\
        \| Expr \xcd"." Operator \\
\end{grammar}

For every expression \xcd"e" of a type \xcd"T" at which a unary
operator \xcd"OP" is defined, the expression \xcd"e.OP()"
represents the function defined by:

\begin{xten}
(): T => { OP e }
\end{xten}

For example,
one may write an expression that adds one to each member of a
list \xcd"xs" by:

\begin{xten}
xs.map(1.+)
\end{xten}




\section{Calls}
\label{Call}
\label{MethodInvocation}
\label{MethodInvocationSubstitution}

\begin{grammar}
MethodCall \: TypeName \xcd"." Identifier TypeArguments\opt \xcd"(" ArgumentList\opt \xcd")" \\
           \| \xcd"super" \xcd"." Identifier TypeArguments\opt \xcd"(" ArgumentList\opt \xcd")" \\
           \| ClassName \xcd"." \xcd"super" \xcd"." Identifier TypeArguments\opt \xcd"(" ArgumentList\opt \xcd")" \\
Call \: Primary TypeArguments\opt \xcd"(" ArgumentList\opt \xcd")" \\
TypeArguments \: \xcd"[" Type ( \xcd"," Type )\star \xcd"]" \\
\end{grammar}

A \grammarrule{MethodCall} may be to either \xcd"static" or to
instance methods.  A \grammarrule{Call} may to either a method
or a closure.  The syntax is ambiguous; the target must be
type-checked to determine if it is the name of a method or if it
refers to a closure.

It is a static error if a call may resolve to both a closure
call or to a method call.  

A closure call \xcdmath"e($\dots$)" is shorthand for a method call \xcdmath"e.apply($\dots$)".

Method selection rules are similar to that of \java{}.
For a call with no explicit type arguments, a method with 
no parameters is considered more specific than a method with one or more
type parameters that would have to be inferred.

Type arguments may be omitted and inferred, as described in
\Sref{TypeInference}.

It is a static error if a method's \grammarrule{Guard} is not satisfied by the caller.

\section{Assignment}\index{assignment}\label{AssignmentStatement}

\begin{grammar}
Expression \: Assignment \\
Assignment \: SimpleAssignment \\
           \| OpAssignment \\
SimpleAssignment \: LeftHandSide \xcd"=" Expression \\
OpAssignment \: LeftHandSide \xcd"+=" Expression \\
             \: LeftHandSide \xcd"-=" Expression \\
             \: LeftHandSide \xcd"*=" Expression \\
             \: LeftHandSide \xcd"/=" Expression \\
             \: LeftHandSide \xcd"%=" Expression \\
             \: LeftHandSide \xcd"&=" Expression \\
             \: LeftHandSide \xcd"|=" Expression \\
             \: LeftHandSide \xcd"^=" Expression \\
             \: LeftHandSide \xcd"<<=" Expression \\
             \: LeftHandSide \xcd">>=" Expression \\
             \: LeftHandSide \xcd">>>=" Expression \\
LeftHandSide \: Identifier \\
             \| Primary \xcd"." Identifier \\
             \| Primary \xcd"(" Expression \xcd")" \\
\end{grammar}

%It is often the case that an \Xten{} variable is assigned to only
%once. The user may declare such variables as \xcd"final". However,
%this is sometimes syntactically cumbersome.
%
%{}\Xten{} supports the syntax \xcd"l := r" for assignment to mutable
%variables.  The user is strongly enouraged to use this syntax to
%assign variables that are intended to be assigned to more than
%once. The \Xten{} compiler may issue a warning if it detects code 
%that uses \xcd"=" assignment statements on \xcd"mutable" variables.

The assignment expression \xcd"x = e" assigns a value given by
expression \xcd"e"
to a mutable variable \xcd"x".  There are three forms of
assignment: \xcd"x" may be a local variable, it may be a field
\xcd"y.f", or it may be the variable obtained by evaluating the expression \xcd"a(i)".
In the last case, \xcd"a" must be an instance of a class implementing the interface
 \xcd"x10.lang.Settable[S,T]", where 
\xcd"S" and \xcd"T" are the types of \xcd"i" and \xcd"e", respectively.

This interface is defined thus:
\begin{xten}
package x10.lang;
public interface Settable[S,T] {
    def set[S,T](i: S, v: T): T;
}
\end{xten}
The assignment \xcd"a(i) = e" is equivalent to the call
\xcd"a.set(i, e)".

\eat{
For array variables, the right-hand-side
expression \xcd"e" 
must have the same distribution \xcd"a(i)". This
statement involves control communication between the sites hosting
\xcd"D". Each site performs the assignment(s) of array components
locally. The assignment terminates when assignment has terminated at
all sites hosting \xcd"D".
}

%% vj: fix. Use some other word for a mutable memory location, not ``location''. 
For a binary operator
$\mathit{op}$,
the
$\mathit{op}$-assignment expression \xcdmath"x $\mathit{op}$= e"
evaluates \xcd"x" to a memory location, then evaluates \xcd"e",
applies the operation \xcdmath"x \mathit{op} e", and assigns the
result into the location computed for \xcd"x".
The expression is equivalent to \xcdmath"x = x $\mathit{op}$ e"
except that any subexpressions of \xcd"x" are evaluated only
once.

%% TODO: Sectional assignment??

%\section{Remote Method Invocation}\index{remote method invocation}
 We introduce shorthand for remote method invocation:
 
 \begin{x10}
 507   MethodInvocation ::= 
         Primary -> identifier ( ArgumentListopt )
 \end{x10}
 
 If the method named is {\tt void} the expression {\tt o -> m(a1,...,ak)}
 is equivalent to:
 \begin{x10}
  finish async (o) \{o.m(a1,...,ak);\}
 \end{x10}
 
 Otherwise the expression is equivalent to
 \begin{x10}
  future (o)\{o.m(a1, ..., ak)\}.force()
 \end{x10}




\section{Increment and decrement}

The operators \xcd"++" and \xcd"--" increment and decrement
a variable, respectively.  The variable must be non-final
and of numeric type.

When the operator is prefix, the variable is
incremented or decremented by \xcd"1" and the result of the expression is
the new value of the variable.
When the operator is postfix, the variable is incremented or
decremented by \xcd"1" and the result of the expression is the old value of
the variable.

The new value of the variable $v$ is identical to the result of
the expressions
\xcdmath"$v$+1" or \xcdmath"$v$-1", as appropriate.

\section{Numeric promotion}
\label{XtenPromotions}
\index{promotion}
\index{numeric promotion}

The unary and binary operators promote their operands as
follows.
Values are sign extended and converted to instances of the
promoted type.

\begin{itemize}
\item  The unary promotion of \xcd"Byte", \xcd"Short", and \xcd"Int" is \xcd"Int".
\item  The unary promotion of \xcd"Long" is \xcd"Long".
\item  The unary promotion of \xcd"Float" is \xcd"Float".
\item  The unary promotion of \xcd"Double" is \xcd"Double".
\item  The unary promotion of \xcd"UByte", \xcd"UShort", and \xcd"UInt" is \xcd"UInt".
\item  The unary promotion of \xcd"ULong" is \xcd"ULong".
\item The binary promotion of two types is the  
greater of the unary promotion of each type
according to the following partial order:
\begin{itemize}
\item \xcd"Int" $<$ \xcd"Long"
\item \xcd"UInt" $<$ \xcd"ULong"
\item \xcd"Long" $<$ \xcd"Float"
\item \xcd"ULong" $<$ \xcd"Float"
\item \xcd"Float" $<$ \xcd"Double"
\end{itemize}
\end{itemize}

\section{Unary plus and unary minus}

The unary \xcd"+" operator applies unary numeric promotion to
its operand.   The operand must be of numeric type.

The unary \xcd"-" operator
applies unary numeric promotion to its operand
and then
subtracts the promoted operand from \xcd"0".
The operand must be of numeric type.
The type of the result is promoted type.

\section{Bitwise complement}

 The unary \xcd"~" operator
applies unary numeric promotion to its operand
and then
 evaluates to the bitwise complement of
 the promoted operand.
  The operand must be of integral type.
The type of the result is promoted type.

\section{Binary arithmetic operations} 

The binary arithmetic operations apply binary numeric promotion
to their operands. The operands must be of numeric type.
The type of the result is the promoted type.
The
\xcd"+" operator adds the promoted operands. The \xcd"-" operator
subtracts the second operand from the first. The \xcd"*" operator
multiplies the  promoted  operands. The \xcd"/" operator
divides the
first  operand  by the second.
The \xcd"%" operator evaluates to
the remainder of the division of the first operand by the
second.

Floating point operations are determined by the IEEE 754
standard. 
The integer \xcd"/" and \xcd"%" throw a \xcd"DivideByZeroException"
if the right operand is zero.

\section{Binary shift operations}

Unary promotion is performed on each operand separately. 
The operands must be of integral type.
The type of the result is the promoted type of the left operand.

If the promoted type of the left operand is \xcd"Int",
the right operand is masked with \xcd"0x1f" using the bitwise
AND (\xcd"&") operator.
If the promoted type of the left operand is \xcd"Long",
the right operand is masked with \xcd"0x3f" using the bitwise
AND (\xcd"&") operator.

The \xcd"<<" operator left-shifts the left operand by the number of
bits given by the right operand.

The \xcd">>" operator right-shifts the left operand by the number of
bits given by the right operand.  The result is sign extended;
that is, if the right operand is $k$,
the most significant $k$ bits of the result are set to the most
significant bit of the operand.

The \xcd">>>" operator right-shifts the left operand by the number of
bits given by the right operand.  The result is not sign extended;
that is, if the right operand is $k$,
the most significant $k$ bits of the result are set to \xcd"0".

\section{Binary bitwise operations}

The binary bitwise operations apply binary numeric promotion
to their operands. The operands must be of integral type.
The type of the result is the promoted type.
The \xcd"&" operator  performs the bitwise AND of the promoted operands.
The \xcd"|" operator  performs the bitwise inclusive OR of the promoted operands.
The \xcd"^" operator  performs the bitwise exclusive OR of the promoted operands.

\section{String concatenation}

The \xcd"+"  operator is used for string concatenation 
 as well as addition.
If either operand is of static type \xcd"x10.lang.String",
 the other operand is converted to a \xcd"String", if needed,
  and  the two strings  are concatenated.

 String conversion of a non-\xcd"null" value is  performed by invoking the
 \xcd"toString()" method of the value.
  If the value is \xcd"null", the value is converted to 
  \xcd'"null"'.

The type of the result is \xcd"String".

\section{Logical negation}

The operand of the  unary \xcd"!" operator 
must be of type \xcd"x10.lang.Boolean".
The type of the result is \xcd"Boolean".
If the value of the operand is \xcd"true", the result is \xcd"false"; if
if the value of the operand  is \xcd"false", the result is \xcd"true".

\section{Boolean logical operations}

Operands of the binary boolean logical operators must be of type \xcd"Boolean".
The type of the result is \xcd"Boolean"

The \xcd"&" operator  evaluates to \xcd"true" if both of its
operands evaluate to \xcd"true"; otherwise, the operator
evaluates to \xcd"false".

The \xcd"|" operator  evaluates to \xcd"false" if both of its
operands evaluate to \xcd"false"; otherwise, the operator
evaluates to \xcd"true".

\section{Boolean conditional operations}

Operands of the binary boolean conditional operators must be of type \xcd"Boolean".
The type of the result is \xcd"Boolean"

The \xcd"&&" operator  evaluates to \xcd"true" if both of its
operands evaluate to \xcd"true"; otherwise, the operator
evaluates to \xcd"false".
Unlike the logical operator \xcd"&",
if the first operand is \xcd"false",
the second operand is not evaluated.

The \xcd"||" operator  evaluates to \xcd"false" if both of its
operands evaluate to \xcd"false"; otherwise, the operator
evaluates to \xcd"true".
Unlike the logical operator \xcd"||",
if the first operand is \xcd"true",
the second operand is not evaluated.

\section{Relational operations} 

The relational operations apply binary numeric promotion
to their operands. The operands must be of numeric type.
The type of the result is \xcd"Boolean".

The \xcd"<" operator evaluates to \xcd"true" if the left operand is
less than the right.
The \xcd"<=" operator evaluates to \xcd"true" if the left operand is
less than or equal to the right.
The \xcd">" operator evaluates to \xcd"true" if the left operand is
greater than the right.
The \xcd">=" operator evaluates to \xcd"true" if the left operand is
greater than or equal to the right.

Floating point comparison is determined by the IEEE 754
standard.  Thus,
if either operand is NaN, the result is \xcd"false".
Negative zero and positive zero are considered to be equal.
All finite values are less than positive infinity and greater
than negative infinity.

\section{Conditional expressions}
\label{Conditional}

\begin{grammar}
ConditionalExpression \: Expression
                \xcd"?" Expression
                \xcd":" Expression \\
\end{grammar}

A conditional expression evaluates its first subexpression (the
condition); if \xcd"true"
the second subexpression (the consequent) is evaluated; otherwise,
the third subexpression (the alternative) is evaluated.

The type of the condition must be \xcd"Boolean".
The type of the conditional expression is the least common
ancestor (\Sref{LCA}) of the types of the consequent and the alternative.

\section{Stable equality}\label{StableEquality}\index{==}\index{!=}

\begin{grammar}
EqualityExpression \: Expression \xcd"==" Expression \\
\| Expression \xcd"!=" Expression \\
\end{grammar}

The \xcd"==" and \xcd"!=" operators provide \emph{stable
equality}; that is, the result of the equality operation is not affected
by the mutable state of the program.

Two operands may be compared with the infix predicate \xcd"==".
The operation
evaluates to \xcd"true" if and only if no action taken by any
user program can distinguish between the two operands.  In more detail,
the rules are as follows.

If the operands both have reference type, then the operation
evaluates to \xcd"true" if both are references to
the same object (even if the object has no mutable fields). 

If one operand evaluates to \xcd"null" then the predicate
evaluates to \xcd"true" if and only if the
other operand is also \xcd"null".

If the operands both have value type, then they must be structurally equal;
that is, they must be instances of the same value class or value array
data type and all their fields or components must be \xcd"==". 

If one operand is of reference type and the other is of value type,
the result is \xcd"false".

If the operands both have numeric type, binary promotion
(\Sref{XtenPromotions}) is performed on the operands before the
comparison.  If the operands do not have numeric type, no
conversions are performed.

The predicate \xcd"!=" returns \xcd"true" (\xcd"false") on two
arguments if and only if the operand \xcd"==" returns \xcd"false"
(\xcd"true") on the same operands.

The predicates \xcd"==" and \xcd"!=" may not be overridden by the
programmer.

\section{Allocation}
\label{ClassCreation}

\begin{grammar}
NewExpression \: \xcd"new" ClassName TypeArguments\opt \xcd"(" ArgumentList\opt \xcd")"
        ClassBody\opt \\
  \| \xcd"new" InterfaceName TypeArguments\opt \xcd"(" ArgumentList\opt \xcd")"
        ClassBody
\end{grammar}

An allocation expression creates a new instance of a class and
invokes a constructor of the class.
The expression designates the class name and passes
type and value arguments to the constructor.

The allocation expression may have an optional class body.
In this case, an anonymous subclass of the given class is
allocated.   An anonymous class allocation may also specify a
single super-interface rather than a superclass; the superclass
of the anonymous class is \xcd"x10.lang.Object".

If the class is anonymous---that is, if a class body is
provided---then the constructor is selected from the superclass.
The constructor to invoke is selected using the same rules as
for method invocation (\Sref{MethodInvocation}).

The type of an allocation expression
is the return type of the constructor invoked, with appropriate
substitutions  of actual arguments for formal parameters, as
specified in \Sref{MethodInvocationSubstitution}.

It is illegal to allocate an instance of an \xcd"abstract" class.
It is illegal to allocate an instance of a class or to invoke a
constructor that is not visible at
the allocation expression.


\section{Casts}\label{ClassCast}\index{classcast}

The cast operation may be used to cast an expression to a given type:

\begin{grammar}
UnaryExpression \: CastExpression \\
CastExpression \: UnaryExpression \xcd"as" Type \\
\end{grammar}

The result of this operation is a value of the given type if the cast
is permissible at run time.

The \xcd"as" operation converts the value to the given type or
throws \xcd"x10.lang.ClassCastException" if the value cannot be 
converted.
Object identity need not be preserved by the conversion.
Type conversion is checked according to the
rules of the \java{} language (e.g., \cite[\S 5.5]{jls2}).
For constrained types, both the base
type and the constraint are checked.
If the
value cannot be cast to the appropriate type, a
\xcd"ClassCastException"
is thrown.
%If the value cannot be cast to the
%appropriate place type a \xcd"BadPlaceException" is thrown. 

% Any attempt to cast an expression of a reference type to a value type
% (or vice versa) results in a compile-time error. Some casts---such as
% those that seek to cast a value of a subtype to a supertype---are
% known to succeed at compile-time. Such casts should not cause extra
% computational overhead at run time.

\section{\Xcd{instanceof}}\label{instanceOf}\index{instanceof@\xcd"instanceof"}

\Xten{} permits types to be used in an in instanceof expression
to determine whether an object is an instance of the given type:

\begin{grammar}
RelationalExpression \: RelationalExpression \xcd"instanceof" Type
\end{grammar}

In the above expression, \grammarrule{Type} is any type including
constrained types and value types. 
At run time, the result of this operator is
\xcd"true" if the \grammarrule{RelationalExpression} can be
coerced
to \grammarrule{Type} without a \xcd"TypeCastException" being
thrown.  Otherwise the result is \xcd"false".
This determination may involve checking
that the
constraint, if any, associated with the type is true for the
given expression.

\section{Subtyping expressions}

\begin{grammar}
SubtypingExpression \: Expression \xcd"<:" Expression \\
                    \| Expression \xcd":>" Expression \\
                    \| Expression \xcd"==" Expression \\
\end{grammar}

The subtyping expression \xcdmath"T$_1$ <: T$_2$"
evaluates to  \xcd"true"  
                         \xcdmath"T$_1$"
                         is a subtype of
                         \xcdmath"T$_2$".

The expression \xcdmath"T$_1$ :> T$_2$"
evaluates to  \xcd"true"  
                         \xcdmath"T$_2$"
                         is a subtype of
                         \xcdmath"T$_1$".

The expression \xcdmath"T$_1$ == T$_2$"
evaluates to  \xcd"true"  
                         \xcdmath"T$_1$"
                         is a subtype of
                         \xcdmath"T$_2$"
                         and if
                         \xcdmath"T$_2$"
                         is a subtype of
                         \xcdmath"T$_1$".

                         Subtyping expressions are used in
                         subtyping constraints for generic
                         types.

\section{Contains expressions}

\begin{grammar}
ContainsExpression \: Expression \xcd"in" Expression \\
\end{grammar}

The expression \xcd"p in r" tests if a value \xcd"p" is in a collection
\xcd"r"; it evaluates to \xcd"r.contains(p)".
The collection \xcd"r"
must be of type \xcd"Collection[T]" and the value \xcd"p" must
be of type \xcd"T".

\section{Rail constructors}
\label{RailConstructors}

\begin{grammar}
RailConstructor \: \xcd"[" Expressions \xcd"]" \\
Expressions \: Expression ( \xcd"," Expression )\star \\
\end{grammar}

The rail constructor \xcdmath"[a$_0$, $\dots$, a$_{k-1}$]"
creates an instance of \xcd"ValRail" with length $k$
where the $i$th element is
\xcdmath"a$_i$".  The element type of the array (\xcd"T") is
bound to the least common ancestor of the types of the
\xcdmath"a$_i$" (\Sref{LCA}).

Since arrays are subtypes of \xcd"(Point) => T",
rail constructors can be passed into the \xcd"Array" and
\xcd"ValArray" constructors as initializer functions.

Rail constructors of type \xcd"ValRail[Int]" and length \xcd"n" 
may be implicitly converted to type \xcd"Point{rank==n}".
Rail constructors of type \xcd"ValRail[Region]" and length \xcd"n" 
may be implicitly converted to type \xcd"Region{rank==n}".

