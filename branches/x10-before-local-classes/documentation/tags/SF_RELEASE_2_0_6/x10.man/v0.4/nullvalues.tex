\section{Nullable Type Constructor} 

\X10{} supports the type constructor {\tt nullable}. 
For any type {\tt T}, the type {\tt nullable T} contains all the
values of type {\tt T}, and a special {\em null} value, unless {\tt T}
already contains {\em null}. This value is designated by the literal
{\tt null}. 

This literal is special in that it has the type {\tt nullable $\tau$}
for all types $\tau$. Thus it can be passed as an argument to a method
call whose corresponding formal parameter is of type {\tt nullable
$\tau$} for some type $\tau$. (This is a widening reference
conversion, per \cite[Sec 5.1.4]{jls}.)

An immediate consequence of the definition of {\tt nullable} is that
for any type $\tau$, the type {\tt nullable nullable $\tau$} is equal
to the type {\tt nullable $\tau$}.

Any attempt to access a field or invoke a method on the value {\em null}
results in a {\tt NullPointerException} being thrown.

For any value {\tt v} of type $\tau$, the class cast expression $\tt
(nullable \tau) v$ succeeds and specifies a value 
of type $\tt nullable \tau$ that may be seen as the ``boxed'' version
of {\tt v}. 

\X10{} permits the widening reference conversion from any type $\tau$
to the type $\tt nullable \tau'$ if $\tau$ can be widened
to the type $\tau'$. 


%% Can two variables of different reference types be compared 
%% by the == operation without a static type error being thrown?

Correspondingly, a value {\tt e} of type $\tt nullable \tau$ can be
cast to the type $\tau$, resulting in a {\tt ClassCastException} if
{\tt e} is {\tt null} and $nullable \tau$ is not equal to $\tau$, and
in the corresponding value of type $\tau$ otherwise.  If $\tau$ is a
value type this may be seen as the ``unboxing'' operator.



\paragraph{Implementaiton notes.}

A value of type {\tt nullable T} may be implemented by boxing a value of
type {\tt T} unless the value is already boxed. The literal {\tt null}
may be represented as the unique null reference.

