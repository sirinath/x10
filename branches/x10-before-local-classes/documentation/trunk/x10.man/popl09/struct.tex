\subsection{Structural constraints}
\label{sec:structural}
\label{sec:structural-typing}

Constraints on types need not be limited to subtyping.
By introducing structural constraints on types, one can
instantiate
type properties on any type with a given set
of methods or fields.
A structural constraint is satisfied if the type has a member of
the appropriate name and with a compatible type.
This feature is useful for reusing code
from separate libraries; it does not require
code of one library to implement a named interface of the other
library to interoperate.

Here, we consider an extension, not yet implemented, of the \Xten{} type system
that supports structural type constraints.
Formally, the extension
is straightforward; indeed the \FXG{} family already supports structural
constraints
via the rules for ``{\tt x \underline{has} I}'' in
Figure~\ref{fig:lookup}.

Structural constraints on types are found in many languages.
For instance,
Haskell supports type
classes~\cite{haskell,haskell-type-classes}.
%
%ML's module system allows modules to be constrained by
%structural signatures~\cite{ml}.
In Modula-3, type equivalence is structural
rather than nominal as in object-oriented languages of the C
family (e.g., C++, Java, and \Xten{}).
Unity~\cite{malayeriIntegrating08}
is another language with structural subtyping.

Using structural constraints in \Xten was inspired by 
the language PolyJ~\cite{java-popl97}, which allows type parameters to be
bounded using
structural \emph{where clauses}~\cite{where-clauses}.
For example, a sorted list class % from Figure~\ref{fig:sorted}
could
be written as follows in PolyJ:
{
\begin{xten}
class SortedList[T]
  where T {int compareTo(T)} {
    void add(T x) {... x.compareTo(y) ...}
    ... }
\end{xten}}
The \xcd"where" clause states that the type parameter \xcd"T" must have a
method \xcd"compareTo" with the given signature.

The analogous code for \xcd"SortedList" in the structural
extension of \Xten{} would be:
{
\begin{xten}
class SortedList[T]
  {T has compareTo(T): int} {
    def add(x: T) {... x.compareTo(y) ...}
    ... }
\end{xten}}

Another use of structural constraints is to permit the
introduction of 
CLU-style optional methods~\cite{clu}.  Consider the following
\xcd"Array" class:
{
\begin{xten}
class Array[T] {
  def add(a: Array[S])
    {T has add(S): U}: Array[U] {
    return new Array[U](
      (p:Point) => this(p)+a(p));
  }
  ... }
\end{xten}}

\noindent
The \xcd"Array" class defines an \xcd"add" method that takes 
an array of \xcd"S", adds each element of the array to the
corresponding element of \xcd"this", and returns an array of the
results.  The method guard specifies that the method may
only be invoked if \xcd"T" has an \xcd"add" method of the
appropriate type.  Thus, for example, an \xcd"Array[int]"
can be added to an \xcd"Array[double]" because \xcd"int"
has a method \xcd"add" (corresponding to the \xcd"+" operation)
that adds an \xcd"int" and a \xcd"double", retuning a
\xcd"double".  However, \xcd"Array[Rabbit]", for example, does not support
the \xcd"add" operation because \xcd"Rabbit" does not have an
\xcd"add" method.
%\footnote{It may have a \xcd"multiply" method.}

\eat{
\subsection{Definition-site variance}

\subsection{Conditional methods and generalized constraints}

For type parameters, method constraints are 
similar to generalized constraints proposed for
\csharp~\cite{emir06}.
%
In the following code, the \xcd"T" parameter is covariant
and so the \xcd"append" methods below are illegal:
{\footnotesize
\begin{xten}
class List[+T] {
  def append(other: T): List[T] { ... }
      // illegal
  def append(other: List[T]): List[T] { ... }
      // illegal
}
\end{xten}}
%
However, one can introduce a method parameter and then constrain
the parameter from below by the class's parameter:
For example, in the following code,
{\footnotesize
\begin{xtenmath}
class List[+T] {
  def append[U](other: U)
      {T $\extends$ U}: List[U] { ... }
  def append[U](other: List[U])
      {T $\extends$ U}: List[U] { ... }
}
\end{xten}}

The constraints must be satisfied by the callers of \xcd"append".
For example, in the following code:
{\footnotesize
\begin{xten}
xs: List[Number];
ys: List[Integer];
xs = ys; // ok
xs.append(1.0); // legal
ys.append(1.0); // illegal
\end{xten}}
the call to \xcd"xs.append" is allowed and the result type is \xcd"List[Number]", but
the call to \xcd"ys.append" is not allowed because the caller cannot show that
$\Xcd{Number} \subtype \Xcd{Double}$.

}
