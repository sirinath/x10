\chapter{Activities}\label{XtenActivities}

An \Xten{} computation may have many concurrent {\em activities} ``in
flight'' at any give time. We use the term activity to denote a piece
of code (with references to data) that is intended to execute in
parallel with other pieces of code. Activities are much lighter-weight
than threads. In particular, there is no object associated with an
activity. Activities may not be interrupted, suspended or resumed from
outside. There is no notion of ``activity groups''. 

An activity is spawned in a given place and stays in that place for
its lifetime.  An activity may be {\em running}, {\em blocked} on some
condition or {\em terminated}.

An activity may be long-running and may invoke recursive methods (thus
may have a stack associated with it). (On the other hand, many
activities may contain a single atomic section with a dozen or so
operations.)

An activity may have local variables (just like a stack frame) that
are not accessible from outside the activity.

An activity may asynchronously and in parallel launch activities at
other places.

\section{Spawning an activity}\label{AsynchronousActivity}\label{AsyncActivity}
An activity is created by executing the statement:
\begin{x10}
{\cf\em{}AsyncStatement::}
   async {\cf\em{}PlaceDesignator} {\cf\em{}BlockStatement}
{\cf\em{}BlockStatement::}
  `\{'{\cf\em{}Statement}`\}' 
{\cf\em{}PlaceDesignator::}
    {\cf\em{}empty}
    `('\_`)'
    `('{\cf\em{}PlaceExpr}`)'
    `('{\cf\em{}NonFinalVariable}`)'
\end{x10} 

In many cases the compiler may infer the unique place at which the
statement is to be executed by an analysis of the types of the
variables occuring in the statement. In such cases the programmer may
omit the place designator; the compiler will throw an error if it
cannot uniquely determine the designated place. The programmer may use
the anonymous place syntax {\tt (\_)} to explicitly indicate that the
compiler should infer the designated place. In all other circumstances
the designated place is determined by examining the specified
expression.

The place designator may be any place expression (\S~\ref{XtenPlaces})
designating a place (e.g.{} {\tt here}, or a place constant). Or it
may be any non-final variable; the designated place is then the
location of this variable.

The statement is subject to the restriction that it must be acceptable
as the body of a {\cf void} method for an anoymous inner class
declared at that point in the code. As such, it may reference
variables in lexically enclosing scopes (including {\cf clock}
variables, \S~\ref{XtenClocks}) provided that such variables
are declared {\cf final}.


Such a statement is executed by launching a new activity at the
designated place, to execute the specified statement. Any exception
thrown by the statement is trapped and an error message printed out in
the standard error file. (Programmers would typically wish to wrap the 
statement in a {\cf try/catch} clause which catches those exceptions
the code should explicitly manage.)


\section{Asynchronous Expression and Futures}\label{XtenFutures}

An asynchronous expression is of the form:
\begin{x10}
 {\cf\em AsyncExpr::} 
    future {\cf\em{}PlaceDesignator} `\{'{\cf\em{}Expr}`\}' 
\end{x10} 
\def\Hat{{\tt \char`\^}}
If {\tt T@P} is the type of {\cf Expr} then the type of this
expression is {\tt\Hat T@P}, read as {\em future {\cf T} at {\cf P}}.
The expression is evaluated at the designated place
asynchronously. The calling activity continues immediately with a 
future\index{future} for the real value (of the given type). 

A future\index{future} may be stored in variables, communicated in
method calls and returned from method invocations. The value computed
by the future may be retrieved by invoking the method {\cf force}:

\begin{x10}
 ^T@P promise = future (P) \{ a[3];\};
 T@P value = promise.force();
\end{x10}
The invocation of this method suspends until the value has been
computed by the asynchronous activity. This value is returned by {\tt
force}, and memoized so that subsequent invocations return the same
value. We also provide the alternate notation {\tt ! p} 
as shorthand for {\tt p.promise()}.

\subsection{Implementation notes}
Futures are provided in \Xten{} for convenience; they may be
programmed using conditional atomic sections as follows. Nevertheless
introducing futures directly is useful because they use conditional
atomic sections in a circumscribed way that is guaranteed not to
create deadlocks.

Consider the class:
\begin{x10}
public class Future<T@P> \{
    ?Box<T> b := null;

    public Future(Runnable<Runnable<T@P>> r) \{
        r.run(new Runnable<T@P>() \{
            public void run(T@P t) \{
                // respond with the value.
                asynch (Future.this) atomic \{
                    Future.this.b := new Box<T@P>(t);
                \}
            \}
        \});
    \}
    public T force() \{
        when (b != null) \{
            return b.value;
        \}
    \}
\}
\end{x10}
\noindent Assume the interface {\cf Runnable} is defined by:
\begin{x10}
 value interface Runnable<T@P> \{
   void run(T@P value);
 \}
\end{x10}

\noindent An object of the class {\cf Future} is created with a 
runnable object {\cf r} which represents the computation to be run to
determine the underlying value. {\cf r} is given a runnable object
which represents the capability to write the boxed value enapsulated
in the future. An attempt to force a future is blocked until such time as the value is known; this value is then returned.

Now an expression {\cf future (P) E}, where the type of {\cf E} is
{\cf T@Q} may be translated as follows (assuming that {\cf reply} is
not a free variable in {\cf E}). Return a future created with a
runnable object (say {\tt q}) which expects a reply object {\tt reply}
and which on being run will evaluate the expression {\tt E} at the
location {\tt P} and use the {\cf reply} object to communicate the
result back to the future.

\begin{x10}
 new Future<T@Q>(new Runable<Runnable<T@Q>>() \{
     public void run(Runnable<T@Q> reply) \{
         async (P) \{ 
             T@Q t = E;
             reply.run(t);
         \}
     \}
 \})
\end{x10}

\section{Atomic sections}\label{AtomicSections}\index{atomic sections}
Languages such as \java{} use low-level synchronization locks to allow
multiple interacting threads to coordinate the mutation of shared
data. \Xten{{ eschews locks in favor of a very simple high-level
construct, the {\em atomic section}.

A programmer may use atomic sections to guarantee that invariants of
shared data-structures are maintained even as they are being accessed
simultaneously by multiple activities running in the same place.

\subsection{Unconditional atomic sections}
The simplest form of an atomic section is the {\em unconditional
atomic section}:

\begin{x10}
  {\cf\em{}UnconditionalAtomicSection::}
    atomic {\cf\em{}BlockStatement}
\end{x10}

{\cf BlockStatement} may include method calls, conditionals,
asynchronous method invocations etc. However, it should not include
invocations of conditional atomic sections (see below) which may
suspend. (Recall that the invocation of {\cf force} on a future may
invoke a conditional atomic section.) Also for the sake of efficient
implementation \XtenCurrVer{} requires that the atomic section be {\em
analyzable}, that is, the set of locations that are read and written
by the {\cf BlockStatement} are bounded and determined statically.

Such a statement is executed by an activity as if in a single step
during which all other concurrent activities in the same place are
suspended. If execution of the statement may throw an exception, it is
the programmer's responsibility to wrap the atomic section within a
{\cf try/finally} clause and include undo code in the finally clause.


The \Xten{} compiler guarantees that if a program compiles correctly
then either all memory locations accessed within an atomic section are
local or the runtime will throw a {\cf BadPlaceException} because of a
failed classcast.

We allow methods of an object to be annotated with {\cf atomic}. Such
a method is taken to stand for a method whose body is wrapped within an
{\cf atomic} statement.

Note an important property of an (unconditonal) atomic section:

\begin{eqnarray}
 {\cf atomic \{ atomic \{ S\}\}} &=& {\cf atomic \{S\}}
\end{eqnarray}

Further, an atomic section will eventually terminate successfully or
thrown an exception; it may not introduce a deadlock.

Atomic sections are closely related to non-blocking synchronization
constructs \cite{Herlihy-non-block}, and can be used to implement 
non-blocking concurrent algorithms.

\subsubsection{Example}

The following class method implements a (generic) compare and swap (CAS) operation:
\begin{x10}
  public atomic <T> boolean CAS(T target, T old, T new) \{
     if (target.equals(old)) \{
       target = new;
       return true;
     \}
     return false;
  \}
\end{x10}
\subsection{Conditional atomic sections}

Conditional atomic sections are of the form:
\begin{x10}
  {\cf\em{}ConditionalAtomicSection}::
    `when' `('{\cf\em{}Expression}`)'{\cf\em{}BlockStatement}
    [`or' `('{\cf\em{}Expression}`)' {\cf\em{}BlockStatement}]*
\end{x10}

In such a statement the one or more expressions are called {\em
guards} and must be {\cf boolean} expressions. The statements are the
corresponding {\em guarded statements}. The first pair of expression
and statement is called the {\em main clause} and the additional pairs
are called {\em auxiliary clauses}. A statement must have a main
clause and may have no auxiliary clauses.

An activity executing such a statement suspends until such time as any
one of the guards is true in the current state. In that state, the
statement corresponding to the first guard that is true is executed.
The checking of the guards and the execution of the corresponding
guarded statement is done atomically. 

We note two common abbreviations. The statement {\cf when (true) S} is
behaviorally identical to {\cf atomic S}: it never suspends. Second,
{\cf when (c) \{;\}} may be abbreviated to {\cf await(c);} -- it
simply indicates that the thread must await the occurrence of a
certain condition before proceeding.  

\paragraph{Conditions on {\tt when} clauses.} 

For the sake of efficient implementation certain restrictions are
placed on the guards and statements in a conditional atomic
section. First, guards are required not to have side-effects, not to
spawn asynchronous activities and to have a statically determinable
upper bound $k$ on their execution. These conditions are expected to
be checked statically by the compiler which may impose additionl
restrictions (e.g.{} all method invocations are recursion-free).

Second, as for unconditional atomic sections, guarded statements are
required to be bounded and statically anlayzable.

Third, guarded statements are required to be {\em flat}, that is, they
may not contain conditional atomic sections. (The implementation of
nested conditional atomic sections may require sophisticated
operational techniques such as rollbacks.)

Third, \Xten{} guarantees only {\em weak fairness} when executing
conditional atomic sections. Let $c$ be the guard of some conditional
atomic section $A$. $A$ is required to make forward progress only if
$c$ is {\em eventually stable}. That is, any execution $s_1, s_2,
\ldots$ of the system is considered illegal only if there is a $j$
such that $c$ holds in all states $s_k$ for $k > j$ and in which $A$
does not execute. Specifically, if the system executes in such a way
that $c$ holds only intermmitently (that is, for some state in which
$c$ holds there is always a later state in which $c$ does not hold),
$A$ is not required to be executed.

\begin{rationale}
The guarantee provided by {\cf wait/notify} in \java{} is no
stronger. Indeed conditional atomic sections may be thought of as a
replacement for \java's wait/notify functionality.
\end{rationale} 

\paragraph{Sample usage.} 
There are many ways to ensure that a guard is eventually
stable. Typically the set of activities are divided into those that
may enable a condition and those that are blocked on the
condition. Then it is sufficient to require that the threads that may
enable a condition do not disable it once it is enabled. Instead the
condition may be disabled in a guarded statement guarded by the
condition. This will ensure forward progress, given the weak-fairness
guarantee.

\paragraph{Example.}

The following class shows how to implement a bounded buffer of size
$1$ in \Xten{} for repeated communication between a sender and a
receiver.

\begin{x10}
class OneBuffer<value T> \{
  ?Box<T> datum = null;
  public 
     void send(T v) \{
     when (this.datum == null) \{
       this.datum := new Box<T>(datum);
    \}
 \}
  public
    T receive() \{
      when (this.datum != null) \{
        T v  = datum.datum;
        value := null;
        return v;
      \}
  \}
\}
\end{x10}

Similar techniques may be used to implement semaphores and other
inter-process communication mechanisms.


\todo{Conditional atomic sections should be powerful enough to implement clocks as well.}