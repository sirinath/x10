% First page

\thispagestyle{empty}

% \todo{"another" report?}

\topnewpage[{
\begin{center}   
{\huge\bf Report on the Experimental Language \Xten{}}
\vskip 1ex
$$
\begin{tabular}{l@{\extracolsep{.5in}}lll}
\multicolumn{4}{c}{\sc ***DRAFT v 0.32: Please do not cite.***}\\
\multicolumn{4}{c}{\sc Please send comments to {\tt vsaraswa@us.ibm.com}}\\
\multicolumn{4}{c}{({\sc IBM Confidential})}

%V\authorsc{IJAY} S\authorsc{ARASWAT}({\itshape Editor\/})} \\
%\ldots
\end{tabular}
$$
\vskip 2ex
% {\it Dedicated to the Memory of APL} % vj
\vskip 2.6ex
\end{center}
}]


\chapter*{Summary}

This draft report provides an initial description of the programming
language \Xten. \Xten{} is a single-inheritance class-based object-oriented
programming language designed for high-performance, high-productivity
computing on high-end computers supporting $O(10^5)$ hardware threads
and $O(10^{15})$ operations per second. 

{}\Xten{} is based on state-of-the-art object-oriented programming
languages and deviates from them only as necessary to support its
design goals. The language is intended to have a simple and clear
semantics and be readily accessible to mainstream object-oriented
programmers. It is intended to support a wide variety of concurrent
programming idioms, incuding data parallelism, task parallelism,
pipelining, producer/consumer and divide and conquer.

This document provides an initial description of the language. We
expect to revise this document in several months in the light of
experience gained in implementing and using this language.

The \Xten{} design team consists of 
D\authorsc{AVID} B\authorsc{ACON}, 
B\authorsc{OB} B\authorsc{LAINEY}, 
P\authorsc{ERRY} C\authorsc{HENG}, 
J\authorsc{ULIAN} D\authorsc{OLBY}, 
K\authorsc{EMAL} E\authorsc{BCIOGLU}, 
A\authorsc{LLAN} K\authorsc{IELSTRA}, 
R\authorsc{OBERT} O'\authorsc{CALLAHAN}, 
F\authorsc{ILIP} P\authorsc{IZLO}, 
V.T.~R\authorsc{AJAN}, 
V\authorsc{IJAY} S\authorsc{ARASWAT} (contact author), 
V\authorsc{IVEK} S\authorsc{ARKAR}
and 
J\authorsc{AN} V\authorsc{ITEK}.

For extended discussions and support we would like to thank: Calin
Cascaval, Elmootaz Elnozahy, Orren Krieger, John McCalpin, Paul
McKenney, Ram Rajamony.

We also thank Jonathan Rhees and William Clinger with help in
obtaining the \LaTeX{} style file and macros used in producing the
Scheme report, after which this document is based. We also acknowledge
the influence of the Java Language Specification \cite{jls2} document,
as evidenced by the numerous citations in the text.

\todo{expand the summary so that it fills up the column.}

%\vfill
%\begin{center}
%{\large \bf
%*** DRAFT*** \\
%%August 31, 1989
%\today
%}\end{center}

\vfill
\eject


\chapter*{Contents}
\addvspace{3.5pt}                  % don't shrink this gap
\renewcommand{\tocshrink}{-3.5pt}  % value determined experimentally
{\footnotesize
\tableofcontents
}

\vfill
\eject


