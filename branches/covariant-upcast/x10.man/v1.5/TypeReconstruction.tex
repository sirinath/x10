\subsubsection{Type reconstruction}\index{type!reconstruction}

{}\Xten{} permits the use of the generic method syntax in variable
declarations. Any variable declaration may be prefixed with a type
parameter list and may use these parameters in type expressions in the
declaration. Such a parameterized variable declaration succeeds at
compile time if it is possible for the compiler to assign unique types
to the parameters in such a way that the declaration type-checks.  The
scope of the parameter is the scope of the variable introduced by the
declaration. Throughout this scope the parameter has the value
inferred by the compiler.

For instance:
\begin{x10}
  // This introduces P as a constant place, the
  // location of objects[0]
 <P> Object@P obj = objects[0];
 async (P) { obj.blah(); };
\end{x10}

Often it is the case that a type parameter is not referenced after it
is introduced. In such cases \Xten{} permits the use of ``\_'' (the
{\em anonymous
parameter})\label{AnonymousParameter}\index{parameter!anonymous@{\tt\_}}
as a parameter. Multiple occurrences of ``\_'' are taken to stand for
``fresh names'' in each occurrence.  For instance:
\begin{x10}
 <P> \_@P obj = objects[0];
 async (P) { obj.blah(); };
\end{x10}
\noindent should be taken as shorthand for
\begin{x10}
 <P,Q> Q@P obj = objects[0];
 async (P) { obj.blah(); };
\end{x10}
{}\noindent where {\tt Q} does not occur anywhere else in the
program. If a declaration only uses the anoymous parameter the angle
brackets may be omitted. Thus for example the often used special case:
\begin{x10}
  \_ obj = new int[D];
\end{x10}
\noindent is shorthand for:
\begin{x10}
  <> \_@\_ obj = new int[D];
\end{x10}

{}\noindent By default, objects are created {\tt here}. An object can
be created in thread local storage by using an {\tt @threadlocal}
suffix to the data type name:
\begin{x10}
   new Foo@threadlocal<here, here>();
\end{x10}


Note that the effect of creating an object at a place {\tt P} and
returning a reference to it may be obtained by:
\begin{x10}
    Foo@P x = future (P) { new Foo();}
\end{x10}

{}\Xten{} imposes the rule that the lifetime of places passed as place
parameters to objects must be no shorter than the lifetime of the
object itself. It also maintains the invariant that all place variables
in scope are guaranteed to last longer than this method. This implies
that given an invocation:
\begin{x10}
    new Foo@P<..., Q, ...>
\end{x10}
\noindent the following constraints must be true
\begin{itemize}
\item {\cf P} is {\cf threadlocal},  or
\item {\cf P} is {\cf here} (and not in an {\cf async}) and {\cf Q} is a parameter to this class, or {\cf Q} is {\cf shared} or {\cf P} is {\cf Q}
\end{itemize}

In a generic object constructor invocation, a type parameter is always
replaced with a type which includes the place. For instance

\begin{x10}
    \_ s = new Stack<Point@P>();
\end{x10}

Constructor calls, method calls and field accesses have the following place constraints.
\begin{x10}
  Foo@P foo = new Foo@P();
  int[]@P data;
  foo.m();
  int i = foo.f;
  foo.f = i;
  int i = data[j];
  data[j] = i;
\end{x10}
Here {\tt P} is scoped or {\tt here} or {\tt accessible}. {\tt f}
must be a final field.

\Xten{} also provides place inference for {\tt asyncs} and {\tt futures} (\S~\ref{AsyncActivity}):

\begin{x10}
  async \{{\cf\em{}Statement;}\} 
  async (\_) \{{\cf\em{}Statement;}\}
  future \{{\cf\em{}Statement;}\} 
  future (\_) \{{\cf\em{}Statement;}\} 
\end{x10}

In the {\tt async} ({\tt future}) statement, the target place of the
{\tt async} ({\tt future}) is the unique place that will satisfy the
place constraints of the body; if there is more than one such place,
an error is thrown at compile-time. We permit the second form so as to
allow the programmer to document that s/he intends the compiler to
infer the location.

Place variables outside an async are accessible inside the async.

\paragraph{Examples}
Consider a {\tt Stack} containing elements of some type {\tt T} which must
all be located at a given place {\tt P}:
\begin{x10}
 class Stack<T@P> \{
     ?T@P[] elements;
     int size;
     Stack() \{
         // filled with nulls.
         elements = new ?T@P[1000]; 
         size = 0;
     \}
     void push(T@P t) \{ elements[size++] = t; \}
     T@P pop() \{ return elements[--size]; \}
     void fill(T@P v) \{ 
         for (int i : elements) \{ 
            elements[i] = v; 
         \} 
     \}
 \}
\end{x10}

Consider the following array initializers (\S~\ref{ArrayInitializer}):
\begin{x10}
  distribution D = block(1000);
  // region 1..1000 treated as 1..1000 here
  \_ data = new int@current[1000](i){ return i*i; }; 
    // gives in P the place to which data[2] was 
    // mapped, i.e. the first place.
  <P> int@P q = data[2]; 
    // initialize the array with 10 times 
    // the index value
  float[D] d = new float[D] (i){ return 10.0*i; };
  float[D] d2 = new float[D] (i){ return i*i; };
  float[D] result = 
    new float[D] (i){ return di] + d2[i]; };
\end{x10}

The code fragments type-check because the compiler may make the
inference that {\tt here} inside the {\tt ateach} is {\tt D[i]}, and
that the place of the elements {\tt d[i]} and {\tt d2[i]} is also 
{\tt D[i]}. The \Xten{} compiler uses only syntactic equality and simple
intraprocedural dataflow identities to determine which places are the
same.  

The place associated with a particular distribution element may be
accessed using array syntax.
\begin{x10}
  place P = D[i];
  Object@P obj = async { new Object@P(); };
  int x = async (D[77]) { return data[77]; };
\end{x10}
Distributions can be passed as parameters much as places can.

\begin{x10}
 <P> void m(Object@P f) \{
    \_ stack = new Stack@local <Object@P>();
    fill(stack);
 \}
\end{x10}

In the following, the type specifier for the argument expands to
{\tt Stack<Object@P>@here}:

\begin{x10}
 <P> void fill(Stack<Object@P> stack) \{
    for (int i : 1000 ) \{
        stack.push( new Object@P(););
    \}
 \}

 <P> void unordered\_fill(Stack<Object@P> stack) \{
    ^Object@P[] futures = new ^Object@P[1000] 
          \{return new Object@P();\}; 
    for (^Object@P f : futures) \{
        stack.push(f);
    \}
 \}
\end{x10}

\paragraph{Examples with type inference}
\begin{x10}
 <P> void m() \{
    \_ stack = new Stack@threadlocal<Object, P>();
    fill(stack);
\}

 <P> void fill(Stack@threadlocal<Object@P> stack) \{
    for (int i = 0; i < 1000; i++) \{
        stack.push(async \{return new Object@P();\});
    \}
\}

<P> void unordered\_fill(Stack@local<Object@P> stack) \{
    \_ values = new \_[1000] \{return new Object@P();\};
    for (\_ f : values) \{
        stack.push(f);
    \}
\}

 // Use of anywhere types and newplace 
 // to create heterogenous collections
 void anywhere\_test() \{
    // create 1000 objects at 
    // 1000 different shared places
   \_ objs = new Object@?[1000]@local 
          (i)\{ async(?) \{ return new Object();\}\};
    for (\_ o : objs) \{
        <P> Object@P v = o;
        async \{v.blah(); \};
    \}
 \}
\end{x10}
