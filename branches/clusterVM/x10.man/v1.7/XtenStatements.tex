\chapter{Statements}\label{XtenStatements}\index{statements}

This chapter describes the statements in the sequential core of
\Xten{}.  Statements involving concurrency and distribution
are described in \Sref{XtenActivities}.

\section{Empty statement}

\begin{grammar}
Statement \: \xcd";" \\
\end{grammar}

The empty statement \xcd";" does nothing.  It is useful when a
loop header is evaluated for its side effects.  For example,
the following code sums the elements of an array.
\begin{xten}
var sum: Int = 0;
for (i: Int = 0; i < a.length; i++, sum += a[i])
    ;
\end{xten}

\section{Local variable declaration}

\begin{grammar}
Statement \: LocalVariableDeclarationStatement \\
             LocalVariableDeclarationStatement \:
             LocalVariableDeclaration \xcd";" \\
\end{grammar}

The syntax of local variables declarations is described in
\Sref{VariableDeclarations}.

Local variables may be declared only within a block statement
(\Sref{Blocks}).
The scope of a local variable declaration is the 
statement itself and the subsequent statements in the block.

\section{Block statement}
\label{Blocks}

\begin{grammar}
Statement \: BlockStatement \\
BlockStatement \: \xcd"{" Statement\star \xcd"}" \\
\end{grammar}

A block statement consists of a sequence of statements delimited
by ``\xcd"{"'' and ``\xcd"}"''.  Statements are evaluated in
order.  The scope of local variables introduced within the block  
is the remainder of the block following the variable declaration.

\section{Expression statement}

\begin{grammar}
Statement \: ExpressionStatement \\
ExpressionStatement \: StatementExpression \xcd";" \\
StatementExpression \: Assignment \\
          \| Allocation \\
          \| Call \\
\end{grammar}

The expression statement evaluates an expression, ignoring the
result.  The expression must be either an assignment, an
allocation, or a call.

\section{Labeled statement}

\begin{grammar}
Statement \: LabeledStatement \\
LabeledStatement \: Identifier \xcd":" Statement \\
\end{grammar}

Statements may be labeled.  The label may be used as the target
of a break or continue statement.  The scope of a label is the
statement labeled.

\section{Break statement}

\begin{grammar}
Statement \: BreakStatement \\
BreakStatement \: \xcd"break" Identifier\opt \\
\end{grammar}

An unlabeled break statement exits the currently enclosing loop
or switch statement.

An labeled break statement exits the enclosing loop
or switch statement with the given label.

It is illegal to break out of a loop not defined in the current
method, constructor, initializer, or closure.

The following code searches for an element of a two-dimensional
array and breaks out of the loop when found:

\begin{xten}
var found: Boolean = false;
for (i: Int = 0; i < a.length; i++)
    for (j: Int = 0; j < a(i).length; j++)
        if (a(i)(j) == v) {
            found = true;
            break;
        }
\end{xten}

\section{Continue statement}

\begin{grammar}
Statement \: ContinueStatement \\
ContinueStatement \: \xcd"continue" Identifier\opt \\
\end{grammar}

An unlabeled continue statement branches to the top of the
currently enclosing loop.

An labeled break statement branches to the top of the enclosing loop
with the given label.

It is illegal to continue a loop not defined in the current
method, constructor, initializer, or closure.

\section{If statement}

\begin{grammar}
Statement \: IfThenStatement \\
          \| IfThenElseStatement \\
IfThenStatement \: \xcd"if" \xcd"(" Expression \xcd")" Statement \\
IfThenElseStatement \: \xcd"if" \xcd"(" Expression \xcd")" Statement \xcd"else" Statement \\
\end{grammar}

An if statement comes in two forms: with and without an else
clause.

The if-then statement evaluates a condition expression and 
evaluates the consequent expression if the condition is
\xcd"true".  If the 
condition is \xcd"false",
the if-then statement completes normally.

The if-then-else statement evaluates a condition expression and 
evaluates the consequent expression if the condition is
\xcd"true"; otherwise, the alternative statement is evaluated.

The condition must be of type \xcd"Boolean".

\section{Switch statement}

\begin{grammar}
Statement \: SwitchStatement \\
SwitchStatement \: \xcd"switch" \xcd"(" Expression \xcd")" \xcd"{" Case\plus \xcd"}" \\
Case \: \xcd"case" Expression \xcd":" Statement\star \\
     \| \xcd"default" \xcd":" Statement\star \\
\end{grammar}

A switch statement evaluates an index expression and then branches to
a case whose value equal to the value of the index expression.
If no such case exists, the switch branches to the 
\xcd"default" case, if any.

Statements in each case branch evaluated in sequence.  At the
end of the branch, normal control-flow falls through to the next case, if
any.  To prevent fall-through, a case branch may be exited using
a \xcd"break" statement.

The index expression must be of type \xcd"Int".

Case labels must be of type \xcd"Int" and must be compile-time
constants.  Case labels cannot be duplicated within the
\xcd"switch" statement.

\section{While statement}

\index{while@\xcd"while"}

\begin{grammar}
Statement \: WhileStatement \\
WhileStatement \: \xcd"while" \xcd"(" Expression \xcd")" Statement \\
\end{grammar}

A while statement evaluates a condition and executes a loop body
if \xcd"true".  If the loop body completes normally (either by reaching
the end or via a \xcd"continue" statement with the loop header
as target), the condition is reevaluated and the loop repeats if
\xcd"true".  If the condition is \xcd"false", the loop
exits.

The condition must be of type \xcd"Boolean".

\section{Do--while statement}

\index{do@\xcd"do"}

\begin{grammar}
Statement \: DoWhileStatement \\
DoWhileStatement \: \xcd"do" Statement \xcd"while" \xcd"(" Expression \xcd")" \xcd";" \\
\end{grammar}


A do-while statement executes a loop body, and then evaluates a
condition expression.  If \xcd"true", the loop repeats.
Otherwise, the loop exits.

The condition must be of type \xcd"Boolean".

\section{For statement}

\index{for@\xcd"for"}

\begin{grammar}
Statement \: ForStatement \\
          \| EnhancedForStatement \\
ForStatement \: \xcd"for" \xcd"("
        ForInit\opt \xcd";" Expression\opt \xcd";" ForUpdate\opt
        \xcd")" Statement \\
ForInit \:
        StatementExpression ( \xcd"," StatementExpression )\star
        \\
      \| LocalVariableDeclaration \\
EnhancedForStatement \: \xcd"for" \xcd"("
        Formal \xcd"in" Expression 
        \xcd")" Statement \\
\end{grammar}

\Xten{} provides two forms of for statement: a basic for
statement and an enhanced for statement.

A basic for statement consists of an initializer, a condition, an
iterator, and a body.  First, the initializer is evaluated.
The initializer may introduce local variables that are in scope
throughout the for statement.  An empty initializer is
permitted.
Next, the condition is evaluated.  If \xcd"true", the loop body
is executed; otherwise, the loop exits.
The condition may be omitted, in which case the condition is
considered \xcd"true".
If the loop completes normally (either by reaching the end
or via a \xcd"continue" statement with the loop header as
target),
the iterator is evaluated and then the condition is reevaluated
and the loop repeats if
\xcd"true".  If the condition is \xcd"false", the loop
exits.

The condition must be of type \xcd"Boolean".
The initializer and iterator are statements, not expressions
and so do not have types.

\label{ForAllLoop}

% XXX REGION

An enhanced for statement is used to iterate over a collection.
If the formal parameter is of type \xcd"T",
the collection expression must be of type \xcd"Iterable[T]".
Exploded
syntax may
be used for the formal parameter (\Sref{exploded-syntax}).
Each iteration of the loop
binds the parameter to another element of the collection.
If the parameter is final, it may not be assigned within the
loop body.

In a common case, the
the collection is intended to be of type
\xcd"Region" and the formal parameter is of type \xcd"Point".
Expressions \xcd"e" of type \xcd"Dist" and
\xcd"Array" are also accepted, and treated as if they were
\xcd"e.region".
If the collection is a region, the \xcd"for" statement
enumerates the points in the region in canonical order.



\section{Throw statement}
\index{throw}

\begin{grammar}
Statement \: ThrowStatement \\
ThrowStatement \: \xcd"throw" Expression \xcd";"
\end{grammar}

The \xcd"throw" statement throws an exception.  The exception
must be a subclass of the value class \xcd"x10.lang.Throwable". 
% null not allowed since a value class;
% If the exception is
% \xcd"null", a \xcd"NullPointerException" is thrown.

\begin{example}
The following statement checks if an index is in range and
throws an exception if not.
\begin{xten}
if (i < 0 || i > x.length)
    throw new IndexOutOfBoundsException();
\end{xten}
\end{example}

\section{Try--catch statement}

\begin{grammar}
Statement \: TryStatement \\
TryStatement \: \xcd"try" BlockStatement Catch\plus Finally\opt \\
             \| \xcd"try" BlockStatement Catch\star Finally \\
Catch \: \xcd"catch" \xcd"(" Formal \xcd")" BlockStatement \\
Finally \: \xcd"finally" BlockStatement \\
\end{grammar}

Exceptions are handled with a \xcd"try" statement.
A \xcd"try" statement consists of a \xcd"try" block, zero or more
\xcd"catch" blocks, and an optional \xcd"finally" block.

First, the \xcd"try" block is evaluated.  If the block throws an
exception, control transfers to the first matching \xcd"catch"
block, if any.  A \xcd"catch" matches if the value of the
exception thrown is a subclass of the \xcd"catch" block's formal
parameter type.

The \xcd"finally" block, if present, is evaluated on all normal
and exceptional control-flow paths from the \xcd"try" block.
If the \xcd"try" block completes normally
or via a \xcd"return", a \xcd"break", or a
\xcd"continue" statement, 
the \xcd"finally"
block is evaluated, and then control resumes at
the statement following the \xcd"try" statement, at the branch target, or at
the caller as appropriate.
If the \xcd"try" block completes
exceptionally, the \xcd"finally" block is evaluated after the
matching \xcd"catch" block, if any, and then the
exception is rethrown.

\section{Return statement}
\label{ReturnStatement}
\index{ReturnStatement}
\begin{grammar}
Statement \: ReturnStatement \\
ReturnStatement \: \xcd"return" Expression \xcd";" \\
             \| \xcd"return" \xcd";" \\
\end{grammar}

Methods and closures may return values using a return statement.
If the method's return type is expliclty declared \xcd"Void",
the method may return without a value; otherwise, it must return
a value of the appropriate type.
