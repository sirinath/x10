\if 0
\chapter{Linking with Java code}
\label{JavaLink}

On some platforms,
\XtenCurrVer{} allows \java{} code to be called directly from
\Xten{} code.  \java{} classes and packages are in the same namespace as
\Xten{} classes and packages.  When a \java{} class is loaded by
the \Xten{} run-time system, types are mapped from \java{} types
to \Xten{} types as follows.

\begin{tabular}{|l|l|}
\hline
\multicolumn{1}{|c|}{\bf \java{} types} &
\multicolumn{1}{|c|}{\bf \Xten{} type} \\
\hline
\hline
\xcd"java.lang.Object" & \xcd"x10.lang.Object" \\
\hline
\xcd"java.lang.String" & \xcd"x10.lang.String" \\
\hline
\xcd"boolean", \xcd"java.lang.Boolean" & \xcd"x10.lang.Boolean" \\
\hline
\xcd"byte", \xcd"java.lang.Byte" & \xcd"x10.lang.Byte" \\
\hline
\xcd"short", \xcd"java.lang.Short" & \xcd"x10.lang.Short" \\
\hline
\xcd"char", \xcd"java.lang.Character" & \xcd"x10.lang.Char" \\
\hline
\xcd"int", \xcd"java.lang.Integer" & \xcd"x10.lang.Int" \\
\hline
\xcd"long", \xcd"java.lang.Long" & \xcd"x10.lang.Long" \\
\hline
\xcd"float", \xcd"java.lang.Float" & \xcd"x10.lang.Float" \\
\hline
\xcd"double", \xcd"java.lang.Double" & \xcd"x10.lang.Double" \\
\hline
\xcd"void" & \xcd"x10.lang.Void" \\
\hline
\xcd"T[]" & \xcd"x10.lang.NativeRail[T]" \\
\hline
\end{tabular}

This mapping allows \Xten{} objects
to be stored in fields of \java{} objects or in \java{} arrays.
\Xten{} objects are treated as subtypes of \xcd"java.lang.Object"
by \java{} code

%XXX TODO

\java{} objects imported into \Xten{} are augmented with a \xcd"location" field.
Java code can determine the location of an object by
invoking the static method
\xcd"x10java.Place.getLocation(java.lang.Object)".

\java{} generics are not supported.
\java{}'s threading and synchronization are not supported.

\fi

\chapter{Linking with native code}\label{extern}\index{extern}

On some platforms,
\XtenCurrVer{} supports a simple facility to permit the efficient
intra-thread communication of an array of primitive type to code
written in the language {\tt C}.  The array must be a ``local''
array. The primary intent of this design is to permit the reuse of
native code that efficiently implements some numeric array/matrix
calculation.

Future language releases are expected to support similar bindings to
{\sc Fortran}, and to support parallel native processing of
distributed \Xten{} arrays. 

The interface consists of two parts. First, an array intended to be
communicated to native code must be created as an unsafe array
via the \xcd"unsafe" annotation.
\begin{xten}
new unsafe Array[T](dist)
\end{xten}
Unsafe arrays can be of any dimension. However, \XtenCurrVer{}
requires that unsafe arrays be of a primitive type, and local
(i.e.,
with an underlying distribution that maps all elements in its region
to \xcd"here").

Unsafe arrays are allocated in a special array of memory that permits
their efficient transmission to natively linked code.
%% Comment about when this memory is freed.

Second, the \Xten{} programmer may specify that certain methods are to
be implemented natively by using the modifier \xcd"extern":

\begin{grammar}
   MethodModifier \: \xcd"extern"
\end{grammar}

Such a method must be declared \xcd"static" and may not have a
method body.\footnote{This
restriction is likely to be lifted in the future.}  Primitive types in
the method argument are translated to their corresponding JNI type
(e.g., {\tt Float} is translated to {\tt jfloat}, {\tt Double} to
{\tt jdouble}, etc.).  The only non-primitive type permitted in an
\xcd"extern" method is an unsafe array. This is passed at type
\xcd"jlong" as an eight-byte address into the unsafe region that contains
the data for the array.  Note that {\tt jlong} is not the same as {\tt long} on
32-bit machines.


Since only the starting address of an array is passed, if the array is
multidimensional, the user must explicitly communicate (or have a
guarantee of) the rank of the passed array, and must either typecast
or explicitly code the address calculation.  Note that all \Xten{}
arrays are created in row-major order, and so any native routine must
also access them in the same order.

For each class \xcd"C" that contains an \xcd"extern" method, the
\Xten{} compiler generates a text file {\tt C\_x10stub.c}.  This file
contains generated \xcd"C" stub functions that are called from the
\xcd"extern" routines.  The name of the stub function is derived from
the name of the \xcd"extern" method. If the method is
{\tt C.process()}, the stub function will be
{\tt Java\_C\_C\_process()}. The name is suffixed with the signature of the
method if the method is overloaded.

The programmer must write C code to implement the native method,
using the methods in the \xcd"C" stub file to call the actual native
method.  The programmer must compile these files and link them into a
dynamically linked library (DLL).  Note that the {\tt jni.h} header file
must be in the include path.  The programmer must ensure this library
is loaded by the program before the method is called, e.g., by
adding a
{\tt x10.lang.System.loadLibrary} call (in a static initializer of the
\Xten{} class).

\begin{example}
The following class illustrates the use of \xcd"unsafe" and native
linking. 
\begin{xten}
public class IntArrayExternUnsafe {
  public static extern 
      def process(yy: unsafe Rail[Int], size: Int);
  static { System.loadLibrary("IntArrayExternUnsafe"); }
  public static def main(args: Rail[String]) {
     val b = (new IntArrayExternUnsafe()).run();
     System.out.println("++++++ Test "
                         +(b?"succeeded.":"failed."));
     System.exit(b?0:1);
  }
  public def run() : Boolean {
    val high = 10;
    val d = (0..high) -> here;
    val y: Array[Int] = new unsafe Array[Int](d);
    for (val (j) in y.region) {
        y(j) = j;
    process(y,high);
    for (val (j) in y.region) {
      val expected = j+100;
      if(y(j) != expected) {
        System.out.println("y("+j+")="
                           +y(j)+" != "+expected);
        return false;
       }
    }
    return true;
  }
}
\end{xten}

The programmer may then write the {\tt C} code thus:
\begin{xten}
void IntArrayExternUnsafe_process(jlong yy, signed int size) {
  int i;
  int* array = (int*) (long) yy;
  for (i = 0; i < size; i++) {
    array[i] += 100;
  }
}
/* automatically generated in C_x10stub.c */
void 
Java_IntArrayExternUnsafe_IntArrayExternUnsafe_process
(JNIEnv *env, jobject obj, jlong yy, jint size) {
   IntArrayExternUnsafe_process(yy, size);
}
\end{xten}

This code may be linked with the stub file (or textually placed
in it). The programmer must then compile and link the C code and
ensure that the DLL is on the appropriate classpath. 

\end{example}
