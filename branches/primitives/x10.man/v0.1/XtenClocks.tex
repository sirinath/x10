\chapter{Clocks}\label{XtenClocks}\index{clocks}

The standard library for \Xten{}, {\cf x10.lang} defines a {\cf final
value class}, {\tt clock} intended for repeated quiescence detection
of arbitrary, data-dependent collection of activities.

This chapter describes the syntax and semantics of clocks and
statements in the language that have parameters of type {\cf clock}. 

At any stage of the computation, a clock has zero or more {\em
registered} activities. An activity may use only those clocks it is
registered with. An activity may be subscribed to zero or more
clocks. An activity is registered with a clock when it is created.
During its lifetime the only additional clocks it is registered with
are exactly those that it creates. In particular it is not possible
for an activity to obtain access to a clock by reading a
data-structure. This is accomplished by requiring that clocks cannot
be stored in objects, only in ``flow'' variables that live on the
stack.

An activity may perform the following operations on a clock {\cf
c}. It may {\em unregister} with {\cf c}; after this, it may perform
no further actions on {\cf c}. It may {\em check} to see if it is
unregistered on a clock. It may {\em register} a newly forked activity
with {\cf c}.  It may {\em mark} a statement {\cf S} for completion in
the current phase by executing the statement {\cf now(c) S}. It may
{\em continue} the clock by executing {\cf c.continue();}. This
indicate to {\cf c} that it has finished marking all statements it
wishes to perform in the current phase. Finally, it may {\em block}
(through the statement {\cf next}) on all the clocks that it is
registered with. (This operation implicitly {\cf continue}'s all
clocks for the activity.) It will resume from this statement only when
all these clocks are ready to advance to the next phase.

A clock becomes ready to advance to the next phase when every activity
registered with the clock has executed at least one {\cf continue}
operation on that clock and all statements marked for completion in
the current phase have been completed.

Though clocks introduce a blocking statement ({\cf next}) an important
property of \Xten{} is that clocks cannot introduce deadlocks. That is,
the system cannot reach a quiescent state (in which no activity is
progressing) from which it is unable to progress. For, before blocking
each activity continues all clocks it is registered with. Thus if a
configuration were to be stuck (that is, no activity can progress) all
clocks will have been continued. But this implies that all activities
blocked on {\cf next} may continue and the configuration is not stuck.

\section{Clock operations}
\subsection{Creating new clocks}\index{clock!creation}

Clocks are created using the nullary constructor for {\cf clock}

\begin{x10}
clock timeSynchronizer = new clock();
\end{x10}

The current activity is automatically registered with the clock.

\subsection{Transmission of clocks}
Clocks may only be stored in {\em flow} variables\index{flow
variables}, i.e.{} local variables and parameters to methods (that are
marked with the {\cf flow} modifier).  Values received by a method in
a {\cf flow} argument or stored in a {\cf flow} local variable may not
be assigned to fields of objects.

\todo{Perhaps we can say: may not be assigned to fields of objects that are not {\tt threadlocal}?}

Each activity is initiated with zero or more clocks, namely those
available in its lexically enclosing environment that are referenced
by the activity.  Because of the condition above, an activity cannot
acquire any new clocks during its lifetime (except by creating
them). Therefore for each activity we can statically determine (a
superset of) the clocks used by that activity through a simple flow
analysis.

\todo{We need to have a cost-model for clocks. Should ensure that this
model reflects hierarchical quiescence detection, cf trees of short
circuits.}

\subsection{Marking clocks referenced in statements}\index{clock!clocked statements}
The programmer may explicitly indicate the set of currently visible
clocks that are to be used inside a statement using the
clocked statement:
\begin{x10}
  {\tt\em{}Statement::}
     clock (c1,...,ck) {\tt\em{}Statement}
\end{x10}

The \Xten{} compiler throws an error if the statement statically
references any clock other than the ones enumerated. The \Xten{}
compiler also performs ``type reconstruction'' on such clock
assertions so that often the user may omit the {\cf clock (c1,...,ck)}
prefix. Nevertheless the programmer may wish to use this form of the
statement to explicitly document the programmer's intention.

The statement may create and use clocks internally.

\subsection{Clocking statements}\index{clock!now}
An activity may execute the statement
\begin{x10}
  now (c) S;
\end{x10}
\noindent (for {\cf c} a clock and {\cf S} is a statement). Execution
of this statement causes {\cf S} to be executed. The clock {\cf c}
cannot advance as long as {\cf S} (and any asynchronous activity it
has spawned) is executing. The statement {\cf S} is said to be {\em
clocked} on {\cf c}. 

{\tt S} may not suspend on {\tt c} or any other clock
visible to {\tt now (c) S}. (It may create new clocks and
suspend on them.) If it is desired to spawn a new activity and allow
it to suspend on {\tt c} (or any other clock visible to {\tt now (c)
S} then that activity should not be spawned within the scope of {\tt
now (c)}. Instead it should internally use {\tt now(c)} as appropriate.

\subsection{Continuing clocks}\index{clock!continue}
An activity may execute the statement
\begin{x10}
  c.continue();
\end{x10}
\noindent on a clock it is registered with. Execution of this
statement indicates that the activity will clock no further statements
on {\cf c} until the clock has advanced.

The compiler should issue an error if any activity has a potentially
live execution path from a {\cf continue} statement to a {\cf now}
statement on the same clock that does not go through a {\cf next}
statement.

\subsection{Advancing clocks}\index{clock!next}
An activity may execute the statement
\begin{x10}
  next c1,...,ck;
\end{x10}

\noindent 
It is a compile time error for the body of an activity to contain such
a statement if the clocks registered for that activity are not the clocks
{\tt c1,...,ck} (in some order).


Execution of this statement blocks until all the clocks that the
activity is registered with (if any) have advanced. The activity
implicitly issues a {\cf continue} on all clocks it is registered
with, before suspending.

An \Xten{} computation is said to be {\em quiescent} on a clock {\cf
c} if each activity registered with {\cf c} has continued {\cf c}.
Note that once the system is quiescent on {\cf c}, it will remain
quiescent on $c$ forever (unless the system takes some action), since
no other activity can become registered with {\cf c}.  That is,
quiescence on a clock is a {\em stable property}.

Once the implementation has detected quiecence on {\cf c}, the system
marks all activities registered with {\cf c} as being able to progress
on {\cf c}. An activity blocked on {\cf next} resumes execution once
it is marked for progress by all the clocks it is registered with.

\todo{Present a simple example of clock advance.}

\subsection{Dropping clocks}\index{clock!drop}
An activity may drop a clock by executing:
\begin{x10}
c.drop();
\end{x10}

\noindent{} 
This statement does nothing if the activity has already dropped {\cf c}. 
The compiler must ensure conservatively that after dropping {\cf c} no
activity can mark a statement current on {\tt c} or continue {\cf c}.

\subsection{Checking for dropped clocks}\index{clock!dropped}
An activity may check that a clock is dropped by executing:
\begin{x10}
c.dropped()
\end{x10}
\noindent This call returns a {\cf boolean} value: {\cf true} iff the activity has already executed {\cf c.drop()}.

\subsection{Program equivalences}

From the discussion above it should be clear that the following
equivalences hold:

\begin{eqnarray}
 {\cf now(c)\, now(d)\,S} &=& {\cf now(d)\, now(c)\,S}\\
 {\cf now(c)\, now(c)\,S} &=& {\cf now(c)\,S}\\
 {\cf c.continue();  next;} &=& {\cf next;}\\
\begin{array}{l}
 {\cf c.continue();}\\
 {\cf d.continue();}
\end{array}  &=& 
\begin{array}{l}
{\cf d.continue();}\\
{\cf c.continue;}
\end{array}
\end{eqnarray}

Note that {\cf next; next;} is not the same as {\cf next;}. The
first will wait for clocks to advance twice, and the second
once.  

\subsection{Implementation Notes}
Clocks may be implemented efficiently with message passing, e.g.{} by
using short-circuit ideas in \cite{SaraswatPODC88}.  Recall that every
activity is spawned with references to a fixed number of clocks. Each
reference should be thought of as a global pointer to a location in
some place representing the clock. (We shall discuss a further
optimization below.) Each clock keeps two counters: the total number
of outstanding references to the clock, and the number of activities
that are currently suspended on the clock.

When an activity $A$ spawns another activity $B$ that will reference a
clock $c$ referenced by $A$, $A$ {\em splits} the reference by sending
a message to the clock. Whenever an activity drops a reference to a
clock, or suspends on it, it sends a message to the clock. 

The optimization is that the clock can be represented in a distributed
fashion. Each place keeps a local counter for each clock that is
referenced by an activity in that place. The global location for the
clock simply keeps track of the places that have references and that
are quiescent. This can reduce the inter-place message traffic
significantly.


\section{Clocked types}\index{types!clocked}

%We allow types to specify clocks, via a {\cf clocked(c)} modifier,
%e.g.{}

%\begin{x10}
%  clocked(c) int r;
%\end{x10}

%This declaration asserts that {\cf r} is accessible
%(readable/writable) only by those statements that are clocked on {\cf
%c}. Thus propagation of the clock provides some control over the
%``visibility'' of {\cf r}.

The declaration 

\begin{x10}
  clocked(c) final int l = r;
\end{x10}

\noindent asserts additionally that in each clock instant {\cf l} is final, 
i.e.{} the value of {\cf l} may be reset at the beginning of each phase
of {\tt c} but stays constant during the phase.

This statement terminates when the computation of {\tt r} has
terminated and the assignment has been performed.

\todo{Generalize the syntax so that aggregate variables can be clocked with an aggregate clock of the same shape.}

\subsection{Clocked assignment}\index{assignment!clocked}
We expect that most arrays containing application data will be
declared to be {\cf clocked final}. We support this very powerful type
declaration by providing a new statement:
{\footnotesize
\begin{verbatim}
  next(c) l = r; 
\end{verbatim}}


\noindent 
for a variable $l$ declared to be clocked on $c$. The statement
assigns $r$ to the {\em next} value of $l$. There may be multiple such
assignments before the clock advances. The last such assignment
specifies the value of the variable that will be visible after the
clock has advanced.  If the variable is {\cf clocked final} it is
guaranteed that {\em all} readers of the variable throughout this
phase will see the value $r$.

The expression {\tt r} is implicitly treated as {\tt now(c) r}. That
is, the clock {\tt c} will not advance until the computation of {\tt r} has
terminated.

\section{Examples}

Consider the core of the ASCI Benchmark Sweep3D program for computing
solutions to mass transport problems.

In a nutshell the core computation is a triply nested sequential loop
in which the value of a variable in the current iteration is dependent
on the values of neighboring variables in a past iteration. Such a
problem can be parallelized through pipelining. One visualizes a
diagonal wavefront sweeping through the array. An MPI version of the
program may be described as follows. There is a two dimensional grid
of processors which performs the following computation
repeatedly. Each processor synchronously receives a value from the
processor to its west, then to its north, then computes some function
of these values and computes a new value to be sent to the processor
to its east and then to its south.  Ignoring the behavior of the
boundary processors for the moment such a computation may be described
by the following \Xten{} program:

\begin{x10}
region R = [1..n0,1..m0];
clock[R] W,N;
clock(W) final double [cyclic(R)] A; 
for (int t : 1..TMax) \{
  ateach( i,j:A) 
    clock (W[i-1,j],N[i,j-1],W[i,j],N[i,j]) \{
      double west = now (W[i-1,j]) future\{A[i-1,j]\}; 
      W[i-1,j].continue();           
      double north = now (N[i,j-1]) future\{A[i,j-1]\}; 
      N[i,j-1].continue();
      next(W[i,j]) A[i,j] = compute(west, north);
      next W[i-1,j],N[i,j-1],W[i,j],N[i,j];
  \}
\}
\end{x10}

