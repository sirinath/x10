Here we prove a soundness theorem for \CFJ{}.  

\begin{lemma}[Substitution Lemma]
\label{substitution}
If
$\Gamma \vdash \tbar{d}:\tbar{U}$,
$\Gamma, \bar{\tt x}:\tbar{U} \vdash \tbar{U}\subtype\tbar{V}$,
and
$\Gamma, \tbar{x}:\tbar{V} \vdash {\tt e:T}$,
then for some type {\tt S},
$\Gamma \vdash {\tt e[\tbar{d}/\tbar{x}]:S}, {\tt S \subtype \tbar{x}:\tbar{V};T}.$
\end{lemma}

\begin{proof}
Straightforward.
\end{proof}

\eat{
\begin{proof}
Assume the premises.  The proof is by structural induction on
{\tt e}.

\begin{itemize}
\item \xcd{x}.
Then $\Gamma, \xcd{x}_i \ty \xcd{V}_i \vdash {\tt T}$.
There are two subcases.

\begin{itemize}
\item
If \xcd{x} is \xcd{x$_i$}, then 
${\tt e[\tbar{d}/\tbar{x}]}$ = \xcd{d$_i$}.
$\Gamma \vdash {\tt d}_i \ty {\tt U}_i$.

$\Gamma \vdash {\tt U}_i \subtype {\tt x}_i \ty {\tt V}_i; {\tt T}$
follows from \rn{S-Exists-R}.

and
$\Gamma \vdash {\tt U}_i \subtype {\tt T}[{\tt d}_i/{\tt x}_i]$

Since,
$\Gamma \vdash {\tt d}_i \ty {\tt U}_i$
by \rn{T-sub} we have
$\Gamma \vdash {\tt d}_i \ty {\tt V}_i$.

\item
Otherwise, 
${\tt e[\tbar{d}/\tbar{x}] = e}$
and the case
follows from \rn{S-Exists-R}.
\end{itemize}

\item $\xcd{e}.\xcd{f}$
\item $\xcd{e}.\xcd{m}(\tbar{e})$
\item $\xcd{new}~\xcd{C}(\tbar{e})$
\item $\xcd{e}~\xcd{as}~\xcd{T}$
\end{itemize}
\end{proof}
}

% Unchanged from FJ
\begin{lemma}[Weakening]
\label{weakening}
If $\Gamma \vdash {\tt e} \ty {\tt T}$, then
$\Gamma, {\tt x} \ty {\tt S} \vdash {\tt e} \ty {\tt T}$.
\end{lemma}

\begin{proof}
Straightforward.
\end{proof}

% Unchanged from FJ
\begin{lemma}[Method body type]
\label{body-type}
If
$$\Gamma, {\tt z}:{\tt T} \vdash {\tt z}\ \has\ {\tt m}(\bar{\tt z} \ty \bar{\tt U})\{{\tt c}\}: {\tt S} = {\tt e},$$
and 
$\Gamma, {\tt z}:{\tt T},\bar{\tt z}: \bar{\tt T} \vdash \bar{\tt T} \subtype \bar{\tt U}$,
then for some type ${\tt S}'$ it is the case that 
$\Gamma, {\tt z}:{\tt T},\bar{\tt z}: \bar{\tt T} \vdash {\tt e}: {\tt S}',{\tt S}'\ty {\tt S}$.
\end{lemma}

%%vj. Updated above formulation to use bars. Proof below needs to be fixed.
\begin{proof}
\eat{
From
$\Gamma, {\tt z}: {\tt T} \vdash {\tt z}\ \has\ {\tt m}(\bar{\tt z} \ty \bar{\tt U})\{{\tt c}\}: {\tt S} = {\tt e}$,
it is easy to show that for some {\tt C} such that $\Gamma \vdash {\tt T} \subtype {\tt C}$,
${\tt def}\ {\tt m}(\tbar{z}:\tbar{U}[\this/{\tt z}])\{{\tt c[\this/{\tt z}]}\}:{\tt S}[\this/{\tt z}] = {\tt e} \in P$.

By \rn{Method OK},
$\this:{\tt C}\vdash {\tt c}:{\tt o}$
and
$\this:{\tt C},\tbar{z}:\tbar{U}[\this/{\tt z}],{\tt c}[\this/{\tt z}] \vdash {\tt S}[\this/{\tt z}] \ \type, \tbar{U}[\this/{\tt z}] \ \type, {\tt e}:{\tt V}, {\tt V} \subtype {\tt S}[\this/{\tt z}]$.
}

Straightforward.
\end{proof}

\begin{lemma}
\label{existential-subtyping}
If   $\Gamma \vdash {\tt S} \subtype {\tt T}$,
then $$({\tt z}:{\tt S};~{\tt c}_0)[{\tt x}/\self] \vdashO ({\tt z}:{\tt T};~{\tt c}_0)[{\tt x}/\self]$$
where {\tt x} is fresh.
\end{lemma}

\begin{proof}
Straightforward.
\end{proof}

\eat{
\begin{lemma}
\label{erased-subtyping}
If   $\Gamma \vdash \xcd{C\{c\}} \subtype \xcd{D\{d\}}$,
then $\Gamma \vdash \xcd{C} \subtype \xcd{D}$.
\end{lemma}

\begin{proof}
From \rn{S-Id} and \rn{S-Const-L}, we have
$\Gamma \vdash \xcd{D\{d\}} \subtype \xcd{D}$.
From \rn{S-Const-R}
$\Gamma \vdash \xcd{C\{c\}} \subtype \xcd{D}$.
\end{proof}
}

\begin{lemma}
\label{subst-in-assumption}
If   $\Gamma, {\tt x}: {\tt U}, {\tt c} \vdashO {\tt c}_0$
and  $\Gamma \vdash {\tt t}: {\tt U}$,
then $$\Gamma, {\tt c}[{\tt t}/{\tt x}] \vdashO {\tt c}_0[{\tt t}/{\tt x}].$$
\end{lemma}

\begin{proof}
Straightforward.
\end{proof}

\begin{lemma}
\label{constraint-lemma} % 4. 
If   $\Gamma \vdash {\tt S} \subtype {\tt T}$,
and  $\sigma(\Gamma,{\tt f}: {\tt T}) \vdashO {\tt c}_0$,
then $\sigma(\Gamma,{\tt f}: {\tt S}) \vdashO {\tt c}_0$.
\end{lemma}

\begin{proof}
By induction on the derivation of 
     $\Gamma \vdash {\tt S} \subtype {\tt T}$.
We proceed by cases for the last judgment in the derivation.
\begin{itemize}
\item \textsc{S-Id}.
Trivial.
\item \textsc{S-Trans}.
Straightforward from the induction hypothesis.
\item \textsc{S-Extends}.
We have
${\tt S} = {\tt C}$ and ${\tt T} = {\tt D}$.
From the definition of $\sigma(\cdot)$ we have
    $$\sigma(\Gamma,{\tt f}: {\tt C}) =
    \sigma(\Gamma,{\tt f}: {\tt D}) = \sigma(\Gamma).$$
The conclusion follows easily.

\item \textsc{S-Const-L}.
We have
${\tt S} = {\tt T\{c\}}$.
Assume
$\sigma(\Gamma,{\tt f}: {\tt T}) \vdashO {\tt c}_0$.
From the definition of $\sigma(\cdot)$ we have
    $$\sigma(\Gamma,{\tt f}: {\tt T\{c\}}) = 
      \sigma(\Gamma, {\tt f}: {\tt T}), {\tt c}[{\tt f}/\self].$$
Since
$\sigma(\Gamma,{\tt f}: {\tt T}) \vdashO {\tt c}_0$, it follows immediately that
$\sigma(\Gamma,{\tt f}: {\tt T}), {\tt c}[{\tt f}/\self] \vdashO {\tt c}_0$.

\item \textsc{S-Const-R}.
We have
${\tt T} = {\tt U\{c\}}$ and $\Gamma \vdash {\tt S} \subtype {\tt U}$
and $\Gamma, \self: {\tt S} \vdash {\tt c}$.

Assume
$\sigma(\Gamma,{\tt f}: {\tt U\{c\}}) \vdashO {\tt c}_0$.
From the definition of $\sigma(\cdot)$ we have
    $$\sigma(\Gamma,{\tt f}: {\tt U\{c\}}) = 
      \sigma(\Gamma, {\tt f}: {\tt U}), {\tt c}[{\tt f}/\self].$$
Thus,
    $\sigma(\Gamma, {\tt f}: {\tt U}), {\tt c}[{\tt f}/\self] \vdashO {\tt c}_0$.

Since
$\Gamma, \self: {\tt S} \vdash {\tt c}$,
we have
$\Gamma, {\tt f}: {\tt S} \vdash {\tt c}[{\tt f}/\self]$,
and hence
$\sigma(\Gamma, {\tt f}: {\tt S}) \vdashO {\tt c}[{\tt f}/\self]$.

Since 
$\Gamma \vdash {\tt S} \subtype {\tt U}$,
applying the induction hypothesis to ${\tt c}[{\tt f}/\self]$, we have
$\sigma(\Gamma, {\tt f}: {\tt U}) \vdashO {\tt c}[{\tt f}/\self]$.
%
Therefore, in the judgment
$\sigma(\Gamma, {\tt f}: {\tt U}), {\tt c}[{\tt f}/\self] \vdashO {\tt c}_0$,
${\tt c}[{\tt f}/\self]$ is redundant and we can conclude
    $\sigma(\Gamma, {\tt f}: {\tt U}) \vdashO {\tt c}_0$.

Finally,
applying the induction hypothesis to ${\tt c}_0$, we have
$\sigma(\Gamma,{\tt f}: {\tt S}) \vdashO {\tt c}_0$.

\item \textsc{S-Exists-L}.
We have
${\tt S} = {\tt x}: {\tt U};~{\tt V}$
    and
    $\Gamma \vdash {\tt U}~\type$
    and
    $\Gamma \vdashO {\tt V} \subtype {\tt T}$
    where {\tt x} is fresh.

Assume $\sigma(\Gamma,{\tt f}: {\tt T}) \vdashO {\tt c}_0$.
%
By the induction hypothesis,
$\sigma(\Gamma,{\tt f}: {\tt V}) \vdashO {\tt c}_0$.
%
Adding ${\tt x}: {\tt U}$ to the assumptions,
we can conclude
$\sigma(\Gamma, {\tt x}: {\tt U}, {\tt f}: {\tt V}) \vdashO {\tt c}_0$.

From the definition of $\sigma(\cdot)$ we have
    $$\sigma(\Gamma,{\tt f}: ({\tt x}: {\tt U};~{\tt V})) =
      \sigma(\Gamma, {\tt x}: {\tt U}, {\tt f}: {\tt V}).$$
%
Thus,
$\sigma(\Gamma,{\tt f}: {\tt S}) \vdashO {\tt c}_0$.

\item \textsc{S-Exists-R}.
We have
${\tt T} = {\tt x}: {\tt U}; {\tt V}$
    and
    $\Gamma \vdash {\tt t}: {\tt U}$
    and
    $\Gamma \vdashO {\tt S} \subtype {\tt V}[{\tt t}/{\tt x}]$.

Assume
$\sigma(\Gamma,{\tt f}: {\tt T}) \vdashO {\tt c}_0$.
%
From the definition of $\sigma(\cdot)$ we have
    $$\sigma(\Gamma,{\tt f}: ({\tt x}: {\tt U};~{\tt V})) =
      \sigma(\Gamma, {\tt x}: {\tt U}, {\tt f}: {\tt V}).$$
%
Thus,
$\sigma(\Gamma, {\tt x}: {\tt U}, {\tt f}: {\tt V}) \vdashO {\tt c}_0$.
Since
$\Gamma \vdash {\tt x}: {\tt U}$, we can show via Lemma~\ref{subst-in-assumption} that
$\sigma(\Gamma, {\tt f}: {\tt V[{\tt t}/{\tt x}]}) \vdashO {\tt c}_0[{\tt t}/{\tt x}]$.

% XXX is this right?
Since $\Gamma \vdash {\tt c}_0 : {\tt o}$, {\tt x} is not free in ${\tt c}_0$.
Thus ${\tt c}_0[{\tt t}/{\tt x}] = {\tt c}_0$
and
$\sigma(\Gamma, {\tt f}: {\tt V[{\tt t}/{\tt x}]}) \vdashO {\tt c}_0$.
%
Since,
$\Gamma \vdashO {\tt S} \subtype {\tt V}[{\tt t}/{\tt x}$
and
$\sigma(\Gamma, {\tt f}: {\tt V[{\tt t}/{\tt x}}) \vdashO {\tt c}_0$,
by the induction hypothesis, we have
$\sigma(\Gamma, {\tt f}: {\tt S}) \vdashO {\tt c}_0$.
\end{itemize}
\end{proof}

\begin{lemma}
\label{subtype-lemma} % 3. 
if   $\Gamma, {\tt f}: {\tt T} \vdash {\tt U} \subtype {\tt U}'$,
and  $\Gamma \vdash {\tt S} \subtype {\tt T}$, then $\Gamma, {\tt f}: {\tt S} \vdash {\tt U} \subtype {\tt U}'$.
\end{lemma}

\begin{proof}
Follows From Lemma~\ref{constraint-lemma}.
\end{proof}

\begin{lemma}
\label{subtyping-in-existential-lemma} % 5. 
if   $\Gamma \vdash {\tt S} \subtype {\tt T}$,
then $\Gamma \vdash {\tt E}\{{\tt z}: {\tt S}; {\tt c}_0\} \subtype {\tt E}\{{\tt z}: {\tt T}; {\tt c}_0\}$.
\end{lemma}

\begin{proof}
To prove the desired conclusion 
$$\Gamma \vdash {\tt E}\{{\tt z}: {\tt S}; {\tt c}_0\} \subtype {\tt E}\{{\tt z}: {\tt T}; {\tt c}_0\},$$
we need to show that
$$\sigma(\Gamma, {\tt x}: {\tt E}\{{\tt z}: {\tt S}; {\tt c}_0\}) \vdashO ({\tt z}: {\tt T}; {\tt c}_0)[{\tt x}/\self].$$
%
We have 
$$
\sigma(\Gamma, {\tt x}: {\tt E}\{{\tt z}: {\tt S}; {\tt c}_0\})
                                                        = \sigma(\Gamma, ({\tt z}: {\tt S}; {\tt c}_0)[{\tt x}/\self])
$$
From Lemma~\ref{existential-subtyping} and
$\Gamma \vdash {\tt S} \subtype {\tt T}$, 
we have
$$({\tt z}: {\tt S}; {\tt c}_0)[{\tt x}/\self] \vdashO ({\tt z}: {\tt T}; {\tt c}_0)[{\tt x}/\self].$$
From $$
\sigma(\Gamma, {\tt x}: {\tt E}\{{\tt z}: {\tt S}; {\tt c}_0\})
                                                        = \sigma(\Gamma, ({\tt z}: {\tt S}; {\tt c}_0)[{\tt x}/\self])$$
and
$$({\tt z}: {\tt S}; {\tt c}_0)[{\tt x}/\self] \vdashO ({\tt z}: {\tt T}; {\tt c}_0)[{\tt x}/\self],$$
we conclude 
$$\sigma(\Gamma, {\tt x}: {\tt E}\{{\tt z}: {\tt S}; {\tt c}_0\}) \vdashO ({\tt z}: {\tt T}; {\tt c}_0)[{\tt x}/\self].$$
\end{proof}

% \begin{lemma}
% \label{lemmasix} % 6. 
% If $\Gamma \vdash \tbar{U}\ \tbar{d}$,
% $\theta = [\tbar{f} / \this.\tbar{f}]$,
% $\fields(C,\theta) = \tbar{Z} \tbar{f}$,
% $\Gamma, \tbar{U}\ \tbar{f} \vdash \tbar{U} \subtype \tbar{Z}$,
% $\sigma(\Gamma, \tbar{U}\ \tbar{f}) \vdashO inv(C,\theta)$,
% $\vdash C \subtype T[new C(\tbar{d}) / \self]$,
% then $\Gamma \vdash C(: \tbar{U}\ \tbar{f}; \self.\tbar{f} = 
%      \tbar{f}) \subtype T$.
% \end{lemma}
% 
% \begin{proof}
% (Notes)
% It is reasonable to assume that the constraint system {\cal C} satisfies the
% property that if $\tbar{d} = \tbar{e}$ (where $\tbar{d}$ and $\tbar{e}$ 
% are sequence of values) 
% then
% for any sequence of constraints $\tbar{c}$, $\tbar{c}[\tbar{d}/\self]$ and 
% $\tbar{c} [\tbar{e}/\self]$
% are equi-satisfiable, i.e., one holds iff the other holds. 
% (Here by $\tbar{c} [\tbar{d}/\self]$ we mean $c_1[d1/\self], \ldots, c_n[dn/\self]$.)
% 
% Now when proving subject reduction for this case we take the type $S$ to be 
% $C(: \tbar{U}\ \tbar{f}; \self.\tbar{f} = \tbar{f}, \self=o)$.
% Note the addition of the $\self=o$ clause. 
% Note also that from
% 
% $\Gamma \vdash C(: \tbar{U}\ \tbar{f}; \self.\tbar{f} = \tbar{f}) o$
% 
% we can derive
% 
% $\Gamma \vdash C(: \tbar{U}\ \tbar{f}; \self.\tbar{f} = \tbar{f}, \self=o) o$
% 
% This of course makes complete sense ... all we need to show is that {\em this
% particular object} $o$ is of the type that it has been cast to.
% 
% But we have just checked this condition, i.e. 
% $type(o) \subtype D and \vdashO d[o/\self]$. 
% So we are done.
% 
% Note about the proof:
% the trick of adding $\self=o$ is critical. There is no hope of showing
% that $C(: \tbar{U}\ \tbar{f}; \self.\tbar{f} = \tbar{f}) \subtype D(:d)$. 
% All we have checked is the one case that {\em this one object\/} $o$ 
% satisfies the condition $d$, not that *all* objects 
% that satisfy $C(: \tbar{U}\ \tbar{f}; \self.\tbar{f} = \tbar{f})$ 
% also satisfy $D(:d)$. 
% So we take advantage of our ability to choose the type $S$ for $o$ 
% to get this done.
%\end{proof}

% \begin{lemma}
% \label{lemmaseven} % 7. 
% If   $\Gamma, \tbar{T} \tbar{f} \vdash \tbar{T} \subtype \tbar{Z}$
% and  $\theta = [\tbar{f} / \this.\tbar{f}]$,
% and  $\fields(C,\theta) = \tbar{Z} \tbar{f}$,
% and  $\fields(T_0,z_0) = \tbar{U}\ \tbar{f}$,
% and  $T_0 equiv C(\tbar{T} \tbar{f}; \self.\tbar{f} = \tbar{f})$,
% then $\Gamma \vdash T_i \subtype (T_0 z_0; z_0.f_i = \self; U_i)$
% \end{lemma}
% 
% \begin{proof}
% From these we can conclude
%    $\Gamma \vdash T_0 new C(bar e)$
% where $T_0$ is $C(: \tbar{T} \tbar{p}; \self.\tbar{f} = \tbar{p})$.
% 
% Now the case we are concerned about is $new C(\tbar{e}).f_i --> e_i$.
% 
% We want to show that the static type of $e_i$, namely $T_i$, 
% is a subtype of the static
% type of $new C(\tbar{e}).f_i$, which is
% $(T_0 z_0; z_0.f_i=\self, V_i[z_0.\tbar{f}/\this.\tbar{f}])$
% 
% Let $x$ be an arbitrary element of $T_i$. 
% We wish to show that $x$ is an element of
% the type $(T_0 z_0; z_0.f_i=\self, V_i[z_0.\tbar{f}/\this.\tbar{f}])$.
% To do this we have
% to show that it is possible to construct from $x$ an object $z_0$ of type 
% $T_0$ such
% that $x$ is the $f_i$'th field of $z_0$. But this is given to us by (4).
% Let $\tbar{t}$ be
% a set of values of type $bar T$, with $x$ chosen at index $i$.
% Then per (4), $T_i$ is a
% subtype of $V_i[\tbar{t}/\this.\tbar{f}]$. Since $x$ is a value of the type $T_i$, 
% it is a value of the type $V_i[\tbar{t}/\this.\tbar{f}]$. 
% Therefore we have constructed the object $z_0$ 
% which is required to show that $x$ is an element of 
% $(T_0 z_0; z_0.f_i=\self, V_i[z_0.\tbar{f}/\this.\tbar{f}])$.
% 
% I have implicitly used here the following genericity property of constraint
% systems (see Vijay Saraswat's LICS 91 paper):
% If $\Gamma \vdash A$ then $\Gamma[\tbar{t}/\tbar{y}] \vdash A [\tbar{t}/\tbar{y}]$.
% That is the set of axioms of the constraint system may contain free
% variables but are assumed to be closed under instantiation.
% \end{proof}

\begin{lemma}
\label{fields-lemma} % 1. 
If   $\Gamma \vdash {\tt S} \subtype {\tt T}$,
and  $\Gamma, {\tt z}: {\tt S} \vdash \fields({\tt z}) = \tbar{F}_1$,
and  $\Gamma, {\tt z}: {\tt T} \vdash \fields({\tt z}) = \tbar{F}_2$,
then $\tbar{F}_2$ is a prefix of $\tbar{F}_1$.
\end{lemma}

\begin{proof}
Follows from Lemma~\ref{constraint-lemma} and rules for $\fields$  in ${\cal O}$.
\end{proof}

\begin{lemma}
\label{has-lemma} % 2. 
If   $\Gamma \vdash {\tt S} \subtype {\tt T}$,
and  $\Gamma, {\tt z}: {\tt T} \vdash {\tt z}\ \has\ {\tt I}$,
then $\Gamma, {\tt z}: {\tt S} \vdash {\tt z}\ \has\ {\tt I}$.
\end{lemma}

\begin{proof}
Follows from Lemma~\ref{constraint-lemma}.
\end{proof}

\begin{lemma}\label{existslemma}
If 
$\Gamma \vdash ({\tt x}: {\tt S}; \xcd{T\{c\}})\ \type$, then
$\Gamma \vdash ({\tt x}: {\tt S}; \xcd{T\{c\}}) \subtype \xcd{T\{x: S; c\}})$.
\end{lemma}

\begin{proof}
% From S-Const-L
% G, x:S |- T{c} <: T
% Then from S-Exists-L
% G |- x:S; T{c} <: T
% By definition of \sigma and c |- c,
% G,self:(x:S; T{c}) |-O {x: S; c},
% From
% G |- x:S; T{c} <: T
% and
% G,self:(x:S; T{c}) |-O {x: S; c},
% by S-Const-R we have
% G |- x: S; T{c} <: T{x: S; c}
Straightforward.
\end{proof}


\begin{theorem}[Subject Reduction] 
\label{preservation}
If $\Gamma \vdash {\tt e}: {\tt V}$ and ${\tt e} \derives {\tt e}'$, then for some type ${\tt V}'$,
$\Gamma \vdash {\tt e}': {\tt V}'$ and $\Gamma \vdash {\tt V}' \subtype {\tt V}$.
\end{theorem}

\begin{proof}
We proceed by induction on the
structure of the derivation of $\Gamma \vdash {\tt e}: {\tt T}$.  We now have five
cases depending on the last rule used in the derivation
of $\Gamma \vdash {\tt e}: {\tt T}$.
\begin{itemize}
\item
\TVar: The expression cannot take a step, so the conclusion is immediate.
\item
\TCast: We have two subcases.
   \begin{itemize}
   \item
   \RCast:  For the expression {\tt o~as~V}, where 
            ${\tt o} = {\tt \new\ {\tt C(\tbar{d})}}$,
            we have from \TNew\ that 
            $$\Gamma \vdash {\tt o: {\tt C}\{\tbar{z}: \tbar{T}; \new\ {\tt C}(\tbar{z}) = \self; \inv({\tt C},\self)\}}.$$
            Additionally, we have from \RCast\ that 
            $$\vdash {\tt C}\{\new\ {\tt C}(\tbar{d}) = \self\} \subtype {\tt V}.$$
            We now choose 
            $${\tt V}' = {{\tt C}\{\tbar{z}: \tbar{T}; \new\ {\tt C}(\tbar{z}) = \self; \inv({\tt C},\self)\}}.$$

            From \rn{S-Exists-L},
            $$
            \begin{array}{r}
            \Gamma \vdash \tbar{z}: \tbar{T}; {{\tt C}\{\new\ {\tt C}(\tbar{z}) = \self; \inv({\tt C},\self)\}} \\
            \subtype {\tt C}\{\new\ {\tt C}(\tbar{d}) = \self\}.
            \end{array}
            $$

            From Lemma~\ref{existslemma},
            $$
            \begin{array}{r}
            \Gamma \vdash
            \tbar{z}: \tbar{T}; {{\tt C}\{\new\ {\tt C}(\tbar{z}) = \self; \inv({\tt C},\self)\}} \\
            \subtype
                {{\tt C}\{\tbar{z}: \tbar{T}; \new\ {\tt C}(\tbar{z}) = \self; \inv({\tt C},\self)\}}
            \end{array}
            $$

        Thus,
            from \rn{S-Trans}, $\Gamma \vdash {\tt V}' \subtype {\tt V}$.
   \item
   \RCCast: For the expression {\tt o~as~V}, we have from \TCast\ that
            $\Gamma \vdash {\tt o}: {\tt U}$.
            Additionally, we have from \RCCast\ that
            ${\tt o} \derives {{\tt o}}'$.
            From the induction hypothesis, we have ${\tt U}'$ such that
            $\Gamma \vdash {\tt o}': {\tt U}'$ and
            $\Gamma \vdash {\tt U}' \subtype {\tt U}$.
            We now choose ${\tt V}'={\tt V}$.
            From $\Gamma \vdash {\tt o}': {\tt U}'$ and \TCast\ we derive
            $\Gamma \vdash {\tt o}'{\tt ~as~V} : {\tt V}$.
            From ${\tt V}'={\tt V}$ and \rn{S-Id}
            we have $\Gamma \vdash {\tt V}' \subtype {\tt V}$.
   \end{itemize}
\item
\TNew: We have a single case.
   \begin{itemize}
   \item
   \RCNewArg: For the expression ${\new\ {\tt C}(\tbar{e})}$,
            we have from \TNew\ that
            $$
            \begin{array}{l}
            \Gamma \vdash \tbar{e}: \tbar{T}, \\
            \vdash \class({\tt C}), \\
            \Gamma \vdash {\tt z}: {\tt C} \vdash \fields({\tt z}) = \tbar{f}: \tbar{S}, \\
            \Gamma \vdash {\tt z}: {\tt C}, \tbar{z}: \tbar{T}, {\tt z}.\tbar{f} = \tbar{z} \vdash \tbar{T} \subtype \tbar{S}, \inv({\tt C}, {\tt z}).
            \end{array}$$

            We choose
            ${\tt V} = {\tt C}\{ \tbar{z}: \tbar{T}; \new\ {\tt C}(\tbar{z}) = \self, \inv({\tt C}, \self)\}$.

            Additionally, we have from \RCNewArg\ that
            ${\tt e}_i \derives {{\tt e}}_i'$.
            From the induction hypothesis, we have ${\tt U}_i$ such that
            $\Gamma \vdash {\tt e}_i': {{\tt U}}_i$ and 
            $\Gamma \vdash U_i \subtype T_i$.

            For all $j$ except $i$, define ${\tt U}_j = {\tt T}_j$ and ${\tt e}_j' = {\tt e}_j$.
            We have 
            $\Gamma \vdash \tbar{e}': \tbar{U}$ and
            $\Gamma \vdash \tbar{U} \subtype \tbar{T}$.

            From Lemma~\ref{weakening}, we have
            $$\Gamma \vdash {\tt z}: {\tt C}, \tbar{z}: \tbar{U}, {\tt z}.\tbar{f} = \tbar{z} \vdash \tbar{U} \subtype \tbar{T}.$$

            % From Lemma~\ref{subtype-lemma},
            % $\Gamma, \tbar{f}: \tbar{T} \vdash \tbar{T} \subtype \tbar{Z}$,
            % and $\Gamma \vdash \tbar{S} \subtype \tbar{T}$, we have $\Gamma, \tbar{f}: \tbar{S} \vdash \tbar{T} \subtype \tbar{Z}.$

            % From Lemma~\ref{weakening}, we have
            % $\Gamma \vdash {\tt z}: {\tt C}, \tbar{z}: \tbar{U}, {\tt z}.\tbar{f} = \tbar{z} \vdash \tbar{U} \subtype \tbar{S}$.

            From \TNew,
            we have
            $$\Gamma \vdash {\tt z}: {\tt C}, \tbar{z}: \tbar{T}, {\tt z}.\tbar{f} = \tbar{z} \vdash \tbar{T} \subtype \tbar{S}.$$
            From Lemma~\ref{subtype-lemma},
            $$\Gamma \vdash {\tt z}: {\tt C}, \tbar{z}: \tbar{U}, {\tt z}.\tbar{f} = \tbar{z} \vdash \tbar{T} \subtype \tbar{S}.$$

            From \rn{S-Trans},
            we have
            $$\Gamma \vdash {\tt z}: {\tt C}, \tbar{z}: \tbar{U}, {\tt z}.\tbar{f} = \tbar{z} \vdash \tbar{U} \subtype \tbar{S}.$$

            From \TNew,
            $$\Gamma \vdash {\tt z}: {\tt C}, \tbar{z}: \tbar{T}, {\tt z}.\tbar{f} = \tbar{z} \vdash \inv({\tt C}, {\tt z}).$$
            From Lemma~\ref{subtype-lemma},
            $$\Gamma \vdash {\tt z}: {\tt C}, \tbar{z}: \tbar{U}, {\tt z}.\tbar{f} = \tbar{z} \vdash \inv({\tt C}, {\tt z}).$$

            Thus, by \TNew,
            $$\Gamma \vdash {\new\ {\tt C}(\tbar{e}')} : {\tt C}\{ \tbar{z}: \tbar{U}; \new\ {\tt C}(\tbar{z}) = \self, \inv({\tt C}, \self)\}$$
            and we choose
            $${\tt V}' = {\tt C}\{ \tbar{z}: \tbar{U}; \new\ {\tt C}(\tbar{z}) = \self, \inv({\tt C}, \self)\}.$$

            From Lemma~\ref{existential-subtyping}, we have
            $\Gamma \vdash {\tt V}' \subtype {\tt V}$.
   \end{itemize}
\item
\TField: We have two subcases.
   \begin{itemize}
   \item
   \RField:  For the expression ${\tt (\new\ C(\tbar{e})).f_i}$, 
             we have from \TField\ that
             $$
             \begin{array}{l}
             \Gamma \vdash {\tt e}: {\tt S}, \\
             \Gamma, {\tt z}: {\tt S} \vdash {\tt z}\ \has\ {\tt f}_i: {\tt U}_i. \\
             \end{array}$$
             Let ${\tt V} = ({\tt z}: {\tt S}; {\tt U}_i\{\self{\tt ==}{\tt z}.{\tt f}_i\})$.
             ${\tt z}$ is fresh.

             % Additionally, we have from \RField\ that
             % ${\tt z}: {\tt C} \vdash \fields({\tt z}) = \tbar{f}: \tbar{T}$.

             We have 
             $${\tt S} = {\tt C}\{\tbar{z}: \tbar{T}{\tt ; \new\ {\tt C}(\tbar{z}) = \self, \inv({\tt C},\self)}\}.$$

             From \TNew, we have
             $\Gamma \vdash \tbar{e}: \tbar{T}$
             %,
             %$\vdash \class(C)$,
             %$\Gamma \vdash {\tt z}: {\tt C} \vdash \fields({\tt z}) = \tbar{f}: \tcd{U}_i$,
             and
             $$\Gamma \vdash {\tt z}: {\tt C}, \tbar{z}: \tbar{T}, {\tt z}.\tbar{f} = \tbar{z} \vdash \tcd{T}_i \subtype \tcd{U}_i.$$

             From $\Gamma \vdash \tbar{e}: \tbar{T}$, we have
             $\Gamma \vdash {\tt e}_i: {\tt T}_i$.
             We now choose ${\tt V}' = {\tt T}_i$.

             By \TNew,
             $$\Gamma, {\tt z}: {\tt C}, \tbar{z}: \tbar{T}, {\tt z}.\tbar{f} = \tbar{z} \vdash \tcd{T}_i \subtype \tcd{U}_i.$$

             By \rn{S-Const-R},
             $$\Gamma, {\tt z}: {\tt C}, \tbar{z}: \tbar{T}, {\tt z}.\tbar{f} = \tbar{z} \vdash \tcd{T}_i \subtype \tcd{U}_i\{\self = {\tt z}_i\}.$$

             Since ${\tt z}.{\tt f}_i = {\tt z}_i$, by application of \rn{S-Id}, \rn{S-Const-L}, and \rn{S-Const-R}, we have
             $$\Gamma, {\tt z}: {\tt C}, \tbar{z}: \tbar{T}, {\tt z}.\tbar{f} = \tbar{z} \vdash \tcd{T}_i \subtype \tcd{U}_i\{\self = {\tt z}.{\tt f}_i\}.$$

             % XXX leap of logic: need to simplify the environment; it should be fine since the zs are fresh and not used
             We can then show via \rn{S-Exists-R} that
             $$\Gamma \vdash {\tt T}_i \subtype ({\tt z}: {\tt S}; {\tt U}_i\{\self = {\tt z}.{\tt f}_i\}),$$
             or more simply
             $\Gamma \vdash {\tt V}' \subtype {\tt V}$.

   \item
   \RCField:
             Follows from the induction hypothesis and application of Lemma~\ref{subtyping-in-existential-lemma}.
        \eat{   
             For the expression ${\tt e.f}_i$, we have from \TField\ that
             $\Gamma \vdash {\tt e}: {\tt S}$ and
             $\Gamma, {\tt z}: {\tt S} \vdash {\tt z}\ \has\ {\tt f}: {\tt T}$
             where ${\tt z}$ is fresh.
             Let ${\tt V} = ({\tt z}: {\tt S}; {\tt T}\{\self = {\tt z}.{\tt f}\})$.
             Additionally, we have from \RCField\ that  
             ${\tt e} \derives {{\tt e}}'$.
             From the induction hypothesis, we have ${\tt S}'$ such that 
             $\Gamma \vdash {\tt e}': {\tt S}'$ and $\Gamma \vdash {\tt S}' \subtype {\tt S}$.

             We now choose 
             ${\tt V'} = ({\tt z}: {\tt S}'; {\tt z}.{\tt f}_i=\self;{\tt U}_i)$.
             From $\Gamma \vdash {\tt e}': {\tt S}'$, and
             Lemma~\ref{fields-lemma},
             $\Gamma \vdash {\tt e}' : {\tt V}'$.

             From $\Gamma \vdash {\tt S} \subtype {\tt S}'$ and 
             Lemma~\ref{subtyping-in-existential-lemma}, we have $\Gamma \vdash {\tt V} \subtype {\tt V}'$.
             }
   \end{itemize}
\item
\TInvk: We have three subcases.
   \begin{itemize}
   \item
   \RInvk:  
            For simplicity, define ${\tt d}_0 = \new\ {\tt C}(\tbar{e})$.
            For the expression 
            ${\tt d}_0.{\tt m}(\tbar{d})$
            we have from $\TInvk$ that
            $$
            \begin{array}{l}
            \Gamma \vdash {\tt d}_0: {\tt T}_0 \\
            \Gamma \vdash {\tt d}_{1:n}: {\tt T}_{1:n} \\
            \Gamma, {\tt z}_{0:n}: {\tt T}_{0:n} \vdash {\tt z}_0\ \has\ {\tt m}({\tt z}_{1:n} \ty {\tt U}_{1:n})\{c\}: {\tt S} = {\tt e} \\
            \Gamma, {\tt z}_{0:n}: {\tt T}_{0:n} \vdash {\tt T}_{1:n} \subtype {\tt U}_{1:n} \\
            \Gamma, {\tt z}_{0:n}: {\tt T}_{0:n} \vdash {\tt c} \\
            \end{array}
            $$
            \noindent
            where ${\tt z}_{0:n}$ is fresh.  By \TNew, we have
            $\Gamma \vdash \tbar{e}: \tbar{A}$
            and
            $${\tt T}_0 = {\tt C}\{\tbar{z}: \tbar{A}; \self = \new\ {\tt C}(\tbar{z}), \inv({\tt C},\self)\}.$$

            Since
            $$\Gamma, {\tt z}_{0:n}: {\tt T}_{0:n} \vdash {\tt z}_0\ \has\ {\tt m}({\tt z}_{1:n} \ty {\tt U}_{1:n})\{c\}: {\tt S} = {\tt e},$$
            and 
            $\Gamma, {\tt z}_{0:n}: {\tt T}_{0:n} \vdash {\tt T}_{1:n} \subtype {\tt U}_{1:n}$,
            by Lemma~\ref{body-type}, we have for some ${\tt S}'$
            $\Gamma {\tt z}_{0:n}: {\tt T}_{0:n} \vdash {\tt e}: {\tt S}',{\tt S}' \subtype S$.

            Choose ${\tt V} = ({\tt z}_{0:n}: {\tt T}_{0:n};\ {\tt S}')$.

            From \RInvk, we have
            ${\tt d}_0.{\tt m}(\tbar{d}) \derives {\tt e}[{\tt d}_0,\tbar{d}/\this,\tbar{z}]$.

            By Lemma~\ref{substitution},
            $\Gamma \vdash {\tt e}[{\tt d}_0,\tbar{d}/\this,\tbar{z}] : {\tt V}'$,
            and
            $\Gamma \vdash {\tt V}' \subtype {\tt z}_{0:n} : {\tt T}_{0:n}{\tt ;\ S}$.

   \item
   \RCInvkRecv:
             Follows from the induction hypothesis and application of Lemma~\ref{subtyping-in-existential-lemma}.
        \eat{   
   For the expression ${\tt e_0.m(\tbar{e})}$,
            we have from \TInvk\ that
            $$
            \begin{array}{l}
            \Gamma \vdash {\tt e}_{0:n} : {\tt T}_{0:n}, \\
            \mtype({\tt T}_0,{\tt m},{\tt z}_0) = 
               \tt {\tt z}_{1:n}: {\tt Z}_{1:n}:c \rightarrow {\tt U}, \\
            \Gamma, {\tt z}_{0:n}: {\tt T}_{0:n} \vdash 
                  {\tt T}_{1:n} \subtype {\tt Z}_{1:n}, \mbox{and} \\
            \sigma(\Gamma, {\tt z}_{0:n}: {\tt T}_{0:n})
                  \vdashO {\tt c} \\
            \end{array}$$
            where ${\tt z}_{0:n}$ is fresh.

            Additionally, from \RCInvkRecv\ we have that
            ${\tt e}_0 \derives {{\tt e}}_0'$.
            From the induction hypothesis, we have ${\tt S}_0$ such that
            $\Gamma \vdash {\tt e}_0': {{\tt S}}_0$ and 
            $\Gamma \vdash S_0 \subtype T_0$.
            For all $j>0$, define $S_j = T_j$ and $e_j' = e_j$.
            We have 
            $\Gamma \vdash \tbar{e}': \tbar{S}$ and
            $\Gamma \vdash \tbar{S} \subtype \tbar{T}$.

            From Lemma~\ref{has-lemma}
            and $\Gamma \vdash \tbar{S} \subtype \tbar{T}$, we have
            $$\mtype(S_0,m,z) = \mtype(T_0,m,z).$$

            From Lemma~\ref{subtype-lemma},
            $\Gamma, {\tt z}_{0:n}: {\tt T}_{0:n} \vdash
                  {\tt T}_{1:n} \subtype {\tt Z}_{1:n}$,
            and $\Gamma \vdash \tbar{S} \subtype \tbar{T}$, we have
            $\Gamma, {\tt z}_{0:n}: {\tt S}_{0:n} \vdash
                  {\tt T}_{1:n} \subtype {\tt Z}_{1:n}$,

            From Lemma~\ref{constraint-lemma}, 
            $\Gamma \vdash \tbar{S} \subtype \tbar{T}$, and
            $\sigma(\Gamma, {\tt z}_{0:n}: {\tt T}_{0:n}) \vdashO
                              {\tt c}$
            we have
            $\sigma(\Gamma, {\tt z}_{0:n}: {\tt S}_{0:n}) \vdashO
                              {\tt c}.$

            We now choose 
               $S=({\tt z}_{0:n}: {\tt S}_{0:n}; U)$.
            From 
            $\Gamma \vdash {\tt e}_{0:n}': {\tt S}_{0:n}$,
            $\mtype({\tt S}_0,{\tt m},{\tt z}_0) =
               \tt {\tt z}_{1:n}: {\tt Z}_{1:n}:c \rightarrow {\tt U}$,
            $\Gamma, {\tt z}_{0:n}: {\tt S}_{0:n} \vdash
                  {\tt T}_{1:n} \subtype {\tt Z}_{1:n}$, and
            $\sigma(\Gamma, {\tt z}_{0:n}: {\tt S}_{0:n}) \vdashO {\tt c}$,
            and \TInvk\ we derive
            $\Gamma \vdash {\tt S}\ {\tt e_0.m(e_{1:n}')}$.
            We have 
               $T=({\tt z}_{0:n}: {\tt T}_{0:n}; {\tt U})$.

            From Lemma~\ref{subtyping-in-existential-lemma} and
            $\Gamma \vdash \tbar{S} \subtype \tbar{T}$, we have
            $\Gamma \vdash S \subtype T$.
            }
   \item
   \RCInvkArg:
             Follows from the induction hypothesis and application of Lemma~\ref{subtyping-in-existential-lemma}.
             \eat{
        For the expression ${\tt e_0.m(\tbar{e})}$,
            we have from \TInvk\ that
            $\Gamma \vdash {\tt e}_{0:n} : {\tt T}_{0:n}$,
            $\mtype({\tt T}_0,{\tt m},{\tt z}_0) = 
               \tt {\tt z}_{1:n}: {\tt Z}_{1:n}:c \rightarrow {\tt U}$,
            $\Gamma, {\tt z}_{0:n}: {\tt T}_{0:n} \vdash 
                  {\tt T}_{1:n} \subtype {\tt Z}_{1:n}$, and
            $\sigma(\Gamma, {\tt z}_{0:n}: {\tt Z}_{0:n}) \vdashO {\tt c}$, 
            where ${\tt z}_{0:n}$ is fresh.

            Additionally, from \RCInvkArg\ we have that, for $i>0$,
            ${\tt e}_i \derives {{\tt e}}_i'$.
            From the induction hypothesis, we have ${\tt S}_i$ such that
            $\Gamma \vdash {\tt e}_i': {{\tt S}}_i$ and 
            $\Gamma \vdash S_i \subtype T_i$.
            For all $j$ except $i$, define $S_j = T_j$ and $e_j' = e_j$.
            We have 
            $\Gamma \vdash \tbar{e}': \tbar{S}$ and
            $\Gamma \vdash \tbar{S} \subtype \tbar{T}$.

            From Lemma~\ref{subtype-lemma},
            $\Gamma, {\tt z}_{0:n}: {\tt T}_{0:n} \vdash
                  {\tt T}_{1:n} \subtype {\tt Z}_{1:n}$,
            and $\Gamma \vdash \tbar{S} \subtype \tbar{T}$, we have
            $\Gamma, {\tt z}_{0:n}: {\tt S}_{0:n} \vdash
                  {\tt T}_{1:n} \subtype {\tt Z}_{1:n}$,

            From Lemma~\ref{weakening} and 
            $\Gamma \vdash \tbar{S} \subtype \tbar{T}$, 
            we have 
            $\Gamma, {\tt z}_{0:n}: {\tt S}_{0:n} \vdash 
                  \tbar{S} \subtype \tbar{T}$.

            From \rn{S-Trans},
            $\Gamma, {\tt z}_{0:n}: {\tt S}_{0:n} \vdash 
                  \tbar{S} \subtype \tbar{T}$,
            and
            $\Gamma, {\tt z}_{0:n}: {\tt S}_{0:n} \vdash 
                    {\tt T}_{1:n} \subtype {\tt Z}_{1:n}$, 
            we have
            $\Gamma, {\tt z}_{0:n}: {\tt S}_{0:n} \vdash 
                  \tbar{S} \subtype {\tt Z}_{1:n}$.

            From Lemma~\ref{constraint-lemma}, 
            $\Gamma \vdash \tbar{S} \subtype \tbar{T}$, and
            $\sigma(\Gamma, {\tt z}_{0:n}: {\tt T}_{0:n}) \vdashO
                              {\tt c}$
            we have
            $\sigma(\Gamma, {\tt z}_{0:n}: {\tt S}_{0:n}) \vdashO
                              {\tt c}.$

            We now choose 
               $S=({\tt z}_{0:n}: {\tt S}_{0:n}; U)$.
            From 
            $\Gamma \vdash {\tt e}_{0:n}': {\tt S}_{0:n}$,
            $\mtype({\tt S}_0,{\tt m},{\tt z}_0) =
               \tt {\tt z}_{1:n}: {\tt Z}_{1:n}:c \rightarrow {\tt U}$,
            $\Gamma, {\tt z}_{0:n}: {\tt S}_{0:n} \vdash
                  {\tt S}_{1:n} \subtype {\tt Z}_{1:n}$, and
            $\sigma(\Gamma, {\tt z}_{0:n}: {\tt S}_{0:n}) \vdashO
                  {\tt c}$,
            and \TInvk\ we derive
            $\Gamma \vdash {\tt e_0.m(e_{1:n}')}: {\tt S}$.
            We have 
               $T=({\tt z}_{0:n}: {\tt T}_{0:n}; U)$.

            From Lemma~\ref{subtyping-in-existential-lemma} and
            $\Gamma \vdash \tbar{S} \subtype \tbar{T}$, we have
            $\Gamma \vdash S \subtype T$.
   }
   \end{itemize}
\end{itemize}
\end{proof}

\noindent
Let the normal form of expressions be given by {\em values}
{\tt v} {::=} $\new\ {\tt C(\tbar{v})}$.

\begin{theorem}[Progress] 
\label{progress}
If $\vdash {\tt e: T}$, then one of the following conditions holds:
\begin{enumerate}
\item {\tt e} is a value {\tt v}, 
\item {\tt e} contains a subexpression ${\tt \new\ C(\tbar{v})~as~T}$ such that
$\not\vdash {\tt C} \subtype {\tt T}[{\tt \new\ C(\tbar{v})}/\self]$,
\item there exists ${\tt e}'$ s.t. ${\tt e} \derives {\tt e}'$.
\end{enumerate}
\end{theorem}

\begin{proof}
The proof has a structure that is similar to the proof of Subject Reduction;
we omit the details.
\end{proof}

\begin{theorem}[Type Soundness] 
\label{type-soundness}
If $\vdash {\tt e: T}$ and ${\tt e} \starderives {\tt e}'$, with ${\tt
e}'$ in normal form, then ${\tt e}'$ is either (1)~a value {\tt v}
with $\vdash {\tt v: S}$ and $\vdash {\tt S \subtype T}$,
for some type {\tt S}, or, (2)~ an expression containing
a subexpression ${\tt \new\ {\tt C(\tbar{v})~as~T}}$ where 
$\not\vdash \tt C\subtype T[\new\ C(\tbar{v})/\self]$.

\end{theorem}

\begin{proof}
Combine Theorem~\ref{preservation} and Theorem~\ref{progress}.
\end{proof}

