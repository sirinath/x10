%\vfill\eject
\chapter{Basic concepts}
\label{basicchapter}

\section{Variables and regions}
\label{specialformsection}
\label{variablesection}

\vest Any identifier that is not a syntactic keyword\index{keyword}
(see~\ref{keywordsection}) may be used as a
variable.\index{syntactic
keyword}\index{identifier}\mainindex{variable} A variable may name a
location where a value can be stored.  A variable that does so is said
to be \defining{bound} to the location.  The set of all visible
bindings\mainindex{binding} in effect at some point in a program is
known as the \defining{environment} in effect at that point.  The value
stored in the location to which a variable is bound is called the
variable's value.  By abuse of terminology, the variable is sometimes
said to name the value or to be bound to the value.  This is not quite
accurate, but confusion rarely results from this practice.

\todo{Define ``assigned'' and ``unassigned'' perhaps?}

\todo{In programs without side effects, one can safely pretend that the
variables are bound directly to the arguments.  Or:
In programs without \ide{set!}, one can safely pretend that the
variable is bound directly to the value. }

\vest Certain expression types are used to create new locations and to
bind variables to those locations.  The most fundamental of these {\em
binding constructs}\mainindex{binding construct} is the
\lambdaexp{}\index{\lambdaexp{}}, because all other binding constructs
can be explained in terms of \lambdaexp{}s.  The other binding
constructs are \ide{let}, \ide{let*}, \ide{letrec}, and \ide{do}
expressions (see~\ref{lambda}, \ref{letrec}, and \ref{do}).

%Note: internal definitions not mentioned here.

\vest Like Algol and Pascal, and unlike most other dialects of Lisp
except for Common Lisp, Scheme is a statically scoped language with
\defining{block structure}.  To each place where a variable is bound
in a program there corresponds a \defining{region} of the program text
within which the binding is effective.  The region is determined by
the particular binding construct that establishes the binding; if the
binding is established by a \lambdaexp{}, for example, then its region
is the entire \lambdaexp{}.  Every reference to or assignment of a
variable refers to the binding of the variable that established the
innermost of the regions containing the use.  If there is no binding
of the variable whose region contains the use, then the use refers to
the binding for the variable in the top level environment, if any
(see chapter~\ref{initialenv}); if there is no binding for the identifier, it
is said to be \defining{unbound}.\mainindex{bound}\index{top level
environment}

\todo{Mention that some implementations have multiple top level environments?}

\todo{Pitman sez: needs elaboration in case of {\tt(let ...)}}

\todo{Pitman asks: say something about vars created after scheme starts?
{\tt (define x 3) (define (f) x) (define (g) y) (define y 4)}
Clinger replies: The language was explicitly
designed to permit a view in which no variables are created after
Scheme starts.  In files, you can scan out the definitions beforehand.
I think we're agreed on the principle that interactive use should
approximate that behavior as closely as possible, though we don't yet
agree on which programming environment provides the best approximation.}


\section{True and false}

\vest Any Scheme value can be used as a boolean value for the purpose
of a conditional test.  As explained in~\ref{booleansection}, all
values count as true in such a test except for \schfalse{}.  This
Standard uses the word ``true'' to refer to any Scheme value except
\schfalse{}, and the word ``false'' to refer to \schfalse{}.
\mainindex{true} \mainindex{false}

\todo{Bartley: tighten this up.}


\section{External representations}
\label{externalreps}

An important concept in Scheme is that of the \defining{external
representation} of an object as a sequence of characters.  For
example, an external representation of the integer 28 is the sequence
of characters ``{\tt 28}'', and an external representation of a list
consisting of the integers 8 and 13 is the sequence of characters
``{\tt(8 13)}''.

\vest The external representation of an object is not necessarily
unique.  The integer 28 also has representations ``{\tt \#e28.000}''
and ``{\tt\#x1c}'', and the list in the previous paragraph also has
the representations ``{\tt( 08 13 )}'' and ``{\tt(8 .\ (13 .\ ()))}''
(see~\ref{listsection}).

\vest Many objects have standard external representations, but some,
such as procedures and circular data structures, do not have standard
representations (although particular implementations may define
representations for them).

\vest An external representation may be written in a program to obtain
the corresponding object (see~\ref{quote}, \ide{quote}).

\vest External representations can also be used for input and output.
The procedure \ide{read} (see~\ref{read}) parses external
representations, and the procedure \ide{write} (see~\ref{write})
generates them.  Together, they provide an elegant and powerful
input/output facility.

\vest Note that the sequence of characters ``{\tt(+ 2 6)}'' is {\em
not} an external representation of the integer 8, even though it {\em
is} an expression evaluating to the integer 8; rather, it is an
external representation of a three-element list, the elements of which
are the symbol {\tt +} and the integers 2 and 6.  Scheme's syntax has
the property that any sequence of characters which is an expression is
also the external representation of some object.  This can lead to
confusion, since it may not be obvious out of context whether a given
sequence of characters is intended to denote data or program, but it
is also a source of power, since it facilitates writing programs such
as interpreters and compilers which treat programs as data (or vice
versa).

\vest The syntax of external representations of various kinds of
objects accompanies the description of the primitives for manipulating
the objects in the appropriate sections of chapter~\ref{initialenv}.


\section{Disjointness of types}
\label{disjointness}

Every object satisfies at most one of the following predicates:

\begin{scheme}
boolean?          pair?             null?
symbol?           number?           char?
string?           vector?           procedure?%
\end{scheme}

These predicates define the types \defining{boolean}, \defining{pair},
\defining{empty list}, \defining{symbol}, \defining{number},
\defining{char} (or \defining{character}), \defining{string},
\defining{vector}, and \defining{procedure}.\mainindex{type}


\section{Storage model}
\label{storagemodel}

Variables and objects such as pairs, vectors, and strings implicitly
denote locations\mainindex{location} or sequences of locations.  A
string, for example, denotes as many locations as there are characters
in the string.  (These locations need not correspond to a full machine
word.) A new value may be stored into one of these locations using the
{\tt string-set!} procedure, but the string continues to denote the
same locations as before.
  
An object fetched from a location, by a variable reference or by
a procedure such as \ide{car}, \ide{vector-ref}, or \ide{string-ref}, is
equivalent in the sense of \ide{eqv?} (see~\ref{equivalencesection})
to the object last stored in the location before the fetch.

Every location is marked to show whether it is in use.
No variable or object ever refers to a location that is not in use.
Whenever this Standard speaks of storage being allocated for a variable
or object, what is meant is that an appropriate number of locations are
chosen from the set of locations that are not in use, and the chosen
locations are marked to indicate that they are now in use before the variable
or object is made to denote them.

In many systems it is desirable for constants\index{constant} (i.e. the values of
literal expressions) to reside in read-only-memory.  To express this, it is
convenient to imagine that every object that denotes locations is associated
with a flag telling whether that object is mutable\index{mutable} or
immutable\index{immutable}.  The constants and the strings returned by
\ide{symbol->string} are then the immutable objects, while all objects created by
the other procedures listed in this Standard are mutable.  It is an error\index{error} to
attempt to store a new value into a location that is denoted by an immutable
object.
