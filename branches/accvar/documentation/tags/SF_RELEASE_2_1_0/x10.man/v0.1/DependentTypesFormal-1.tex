\documentclass{article}
\input{../../../../vj/res/pagesizes}
\input{../../../../vj/res/vijay-macros}
\def\csharp{{\sf C\#}}
\usepackage[pdftex]{graphicx}
\def\Xten{{\sf X10}}
\def\java{{\sf Java}}
\def\FXten{{\sf FX10}}
\begin{document}
\title{Formalizing Dependent Types for \Xten}
\author{
({\sc Draft Version 0.04})\\
(Please do not cite)\\
(Send comments to {\tt vsaraswa@us.ibm.com}.)
}

\date{April 19 2006}
\maketitle

\begin{abstract}
We formalize the basic ideas of dependent-types for Java like
languages (introduced in \cite{DependentTypes}), in the context of
\FXten, a Featherweight version of \Xten{} \cite{X10}.  For now, we
focus on a sequential language; distribution and concurrency
constructs {\em a la} \Xten{} are to be added later. This note should
be taken as a companion to \cite{X10-concur} which formalizes the
dynamic semantics for \FXten. It is intended to provide the basis for
a place-based type-system for scalars and arrays.
\end{abstract}

\section{Introduction}

\section{Basic \FXten}
The abstract syntax for \FXten{} is specified in
Table~\ref{Table:AST}. We consider a parametric version of the
language, with an underlying constraint system $\cal C$ \cite{cccc} being used to
specify dependent types. 

%% TODO: The formal type-checking rules are defined only in the case in which 
%% the subject and arguments of a method call are final variables/parameters.
%% Need to think through why this is the case, and motivate and adjust.

\begin{table}
  \begin{tabular}{|llll|}
\hline
 (Classes) & CL &{::=}& {\tt class C($\tt \bar{\tt f}:\bar{\tt X}$, var $\tt\bar{\tt g}:\bar{\tt Y}$)extends D\{K\ $\bar{\tt M}$\}} \\
 (Constructor) & K &{::=}& {\tt def C($\tt \bar{\tt u}:\bar{\tt U},\bar{\tt f}:\bar{\tt X},\bar{\tt g}:\bar{Y}$ \&c):T
                       = \{super($\bar{\tt u}$); this.$\bar{\tt f},\bar{\tt g}=\bar{\tt f},\bar{\tt g}$;\}} \\
 (Method) & M &{::=}& {\tt def m($\tt\bar{\tt x}:\bar{\tt X}$ \&c):T = e}\\
 (Expression) & e,r,s &{::=}& {\tt null} \alt {\tt n} \alt {\tt x} \alt {\tt new C($\tt \bar{e}$)}  \\
    &&& \alt {\tt \{val $\tt\bar{\tt x}:\bar{\tt X}=\bar{\tt e}$; e\}} \alt {\tt e;e} \\
    &&& \alt {\tt x.f} \alt {\tt x.f=e} \alt {\tt x.m($\tt \bar{\tt x}$)} \alt {\tt (T) e} \\
    &&& \alt {\tt c?e:e}\alt \ldots\\
 (Type) & T,U,V,X,Y &{::=}& {\tt C(\&c)} \alt \tt nullable\ C(\&c)\\
 (Constraint) & c,d &{::=}& {\tt true} \alt {\tt c\&c} \alt \alt ce==ce \alt ce !=ce \alt $\tt (\exists x:T)\, c$\ldots \\
 (Term) & ce &{::=}& {\tt self} \alt {\tt f} \alt x \alt \ldots \\
\hline
  \end{tabular}
\caption{Abstract syntax for \FXten.}\label{Table:AST}
\end{table}

We follow \cite{FJ,MJ} in our treatment. Meta-variables {\tt C}, {\tt
D}, {\tt E} range over class names; {\tt f} and {\tt g} range over
field names; {\tt m} ranges over method names; {\tt x}, {\tt y}, {\tt
z} range over parameter and local variable names; other meta-variables are specified in
Table~\ref{Table:AST}.

We write $\bar{e}$ as shorthand for $\tt e_1,\ldots, e_n$
(comma-separated sequence); this sequence may be empty ({\tt n=0}).
Similarly for $\bar{x}$.
$\bar{\tt M}$ and $\bar{\tt K}$ are similar except that no commas
separate the items in the sequence. We use the obvious abbreviation:
$\tt\bar{\tt f}:\bar{X},$ abbreviates $\tt f_1:X_1,\ldots, f_n:X_n$
({\tt n} may be zero).  {\tt var $\tt\bar{\tt g}:\bar{\tt Y}$}
abbreviates the sequence {\tt var $\tt g_1:Y_1, \ldots, g_n:Y_n$} if
$\tt n > 1$ and the empty sequence otherwise. Empty parameter
sequences may be omitted (like Scala, unlike Java). 

The phrase ``{\tt \&c}'' is called a {\em where clause}. We abbreviate
{\tt \&true} to the empty string. Similarly {\tt this.$\bar{\tt
f}$=$\bar{\tt e}$;} abbreviates {$\tt this.f_1=e_1;\ldots
this.f_n=e_n$}.  {\tt \{val $\tt\bar{\tt x}:\bar{\tt X}=\bar{\tt e};\}$}
abbreviates {\tt \{val $\tt x_1: X_1= e_1;\ldots; x_n:X_n=e_n;\}$}.
Sequences of field declarations, parameter declarations, local
variable declarations, are assumed to not contain any duplicates.
Sequences of constructors in a class must not contain two constructors
with the same sequence of parameters types; similarly for
methods. (\FXten{} permits {\em ad hoc} polymorphism.)

\paragraph{Expressions.}
For expressions, we assume the following precedence order (from less
tight to more tight): sequencing, type-cast, assignment, conditional,
field invocation, method invocation.

We also reserve the local variable name ``{\tt this}'' and ``{\tt
self}''. That is, no program may define a local variable or parameter
named {\tt this} or {\tt self}. 

We note that, somewhat unusually, field selection, assignment, method
invocation and constructor invocation take constants as arguments,
rather than expressions. This is necessary because we need a name for
the arguments so that the name can be substituted for the formal
argument in the resulting type. The version of these operations which
takes arbitrary expressionas as arguments can be obtained by combining
with the local variable combinator. Thus, {\tt e.m(e1)} is simply {\tt
\{val x:X=e;x\}.m(\{val y:Y=e1;y\})}, where {\tt x} and {\tt y} are
new local variables, and the type of the expressions {\tt e} and {\tt
e1} is {\tt X} and {\tt Y} respectively. Below, when writing actual
programs we shall feel free to use the {\tt e.m(e1)} form.

The two-armed conditional expression {\tt c?e:e} does not take an
arbitrary expression as a test; rather it takes as argument a
constraint which must satisfy the property that its negation is also a
constraint. (Note that we do not require that constraints are closed
under negation in general.) {\tt c?e:e1} is best thought of as a
``{\tt typecase}'' expression. It permits the compiler to reason
conditionally about the expression, by propagating the
constraint down the positive branch and its negation down the negative
branch.

\paragraph{Types.}
We reserve the class names ``{\tt Object}'' and ``{\tt
int}''.\footnote{In a subsequent version of this document we will
introduce value types, and then {\tt int} will be just another value
type, defined with native methods.}

A type is of the form {\tt C(\&c)} where {\tt c} is a constraint.
Intuitively, a type {\tt C(\&c)} is the type of all objects that are
instances of {\tt C} and satisfy the condition {\tt c}. Note that if
the condition {\tt c} is unsatisfiable, then the type is
empty. Variables/parameters cannot be declared at empty types.

From the abbreviation rules above, the type {\tt C(\&true)} may be
abbreviated to {\tt C}.  {\tt c} may contain references to parameters
and variables visible at the point of declaration of the type
(including {\tt this}), and the special variable {\tt self} which
refers to the object to which the type applies.  The fields of {\tt C}
may occur unqualified in {\tt c}, they are presumed to be qualified with the
special variable {\tt self}.\footnote{
Note that in general {\tt this} is
different from {\tt self}.  For instance the type {\tt B(\& f =
this.g)} appearing in the body of the definition of the class {\tt A}
is the type of all instances of {\tt B} whose {\tt f} field has the
same value as the {\tt g} field of the current {\tt A} object. (It is
the same as the type {\tt B(\& self.f = this.g}).)}

The special variable {\tt this} may {\em not} be referenced in a type
in the definition of a constructor (e.g.{} in the constraint in the
parameter list of the constructor, or in the return type of the
constructor). This recognizes the fact that an object does not exist
until it is created. 

The type {\tt nullable C(\&c)} is the type {\tt C(\&c)} together with
the special value {\tt null}. Thus the type {\tt nullable C(\&c)} is
never empty: if {\tt c} is inconsistent it permits precisely the value
{\tt null}.

The terms {\tt ce} in a constraint are drawn from an underlying
constraint system, $\cal C$ \cite{cccc}. {\em For now we take the
constraint system to be fixed, but it makes sense to permit the
programmer to extend the constraint system with new value types, and
operations over them, provided that a constraint solver is supplied
for these types.}

Recall that for readability, we permit the expression {\tt self.f} to
be abbreviated to {\tt f} in the constraint {\tt c} of a type {\tt
C(\&c)}.  {\tt self} is often absent in types. It is particularly
useful in {\em singleton types}, e.g. {\tt Point(\&self=p)} which is
satisfied by any object that is an instance of {\tt Point} and is the
same as {\tt p}.

We also permit the shorthand {\tt C($\overline{\tt ce}$)} for {\tt
C(\&$\overline{\tt f}=\overline{\tt ce}$)} where $\tt \overline f$ is the textual
order enumeration of the {\tt val} fields of {\tt C}. If the class has
no field, then we use the shorthand {\tt C(ce)} for {\tt
C(\&self=ce)}. Thus {\tt int(0)} is satisfied precisely by the value
{\tt 0}.

Finally, we permit the shorthand $\tt (\exists x:T) C(\&c)$ for
$\tt C(\exists x:T)c)$.

{\em In a later version of this document we will introduce type definitions.
This will let us use, for instance, the abbreviation {\tt nat}
for the type {\tt int(\&self >= 0)}. 
}

\paragraph{Program.}
A {\em program} is a pair of a set of classes and an expression. For
simplicity we shall leave the set of classes implicit. We shall assume
that every class name used in the program (except {\tt Object}) is
defined exactly once in the program. We assume that the class
hierarchy, defined by $\leq$ in the next section is acyclic
(anti-symmetric).

\subsection{Remarks on the syntax}
In a class declaration, $\tt\bar{\tt f}$ represents the val (final)
fields of the class, $\tt\bar{\tt g}$ the var (mutable)
fields.\footnote{ The addition of {\tt var} constructor and method
parameters and local variables is routine and omitted for
brevity. Similarly for initializers of fields, static fields and
methods. We may add exceptions later since they play an integral part in the
semantics of concurrent constructs.} 

%% $\tt c$ is the {\em class
%% invariant} (a constraint on the fields of the class true for all
%% instances of the class); this must be checked when validating the
%% constructor.\footnote{In the current formulation there is only one
%% constructor, so the class invariant can be folded into the return type
%% of the constructor. However, it makes sense to permit multiple
%% constructors in the future, so we shall retain the separate syntax for
%% the class invariant.}

After Scala \cite{scala}, we have chosen the ML-style variable
declaration (type comes after the variable, and is separated by a
colon), as opposed to conventional Java-style declarations, so that
the return type of a method may appear after the declaration of the
parameters of the method. This permits parameters to appear in the
return type, while respecting the ``define before use''
meta-rule. This syntax also permits types (for parameters, variables
and methods) to be optional, while still retaining readability.  Note
that the language above does not permit mutable constructor or method
parameters.

In a constructor declaration, {\tt c} is a constraint on the
constructor parameters $\bar{\tt x}$ that must be true for the
constructor to be invoked.  A constructor may specify the return type
(useful for specifying constraints on the fields of the object
returned by the constructor).  In a method definition, {\tt c} is a
constraint on the parameters $\bar{\tt x}$ and the final fields of the
object on which the method is being invoked that must be true for this
method to be invoked. In a type specification, {\tt c} is a constraint
on the final fields of the class {\tt C}. A constraint is a boolean
valued expression over final fields and local variables written using
the operations provided in the underlying constraint system.

The abstract syntax differs from the syntax in \cite{DependentTypes}
in many notational respects. It differs substantively in that there is
no distinction between parameters of classes and fields, and
parameters of methods and arguments to the method. All arguments to a
method are considered final. There are no implicit parameters for
classes.  All final fields of a class may be used in defining a
dependent type on that class.

\subsection{Example}
\begin{example}[List]\label{List}
Consider the class {\tt List}. We shall use generic syntax for
type parameters; this will be formalized in a subsequent version 
of this note in a fashion similar to \cite{FJ}. For now, 
the reader should understand generic syntax in the spirit of \cite{FJ}.

{\footnotesize
\begin{verbatim}
  class List<X>( n:int(&self>=0), 
                 var node: nullable X, 
                     rest: nullable List<X>(n-1)) {
    /** Returns the empty list. Defined only when the parameter n 
        has the value 0. Invocation: new List(0)<X>().
     */
    def List(n:int(&self>=0), node: nullable X, rest: nullable List<X>(n-1)) = {
      super(); this.n=n; this.node=node; this.rest=rest;
    }
    def makeList:List<X>(0)=new List(0,null, null)
    def makeList(node:X):List<X>(1)=new List(1,node,makeList)
    def makeList{node:X, rest:nullable List<X>):List<X>(rest.n+1)=
     rest==null ? makeList(node):new List(rest.n+1,node,rest)

    def append(arg:List<X>):List<X>(n+arg.n)  =
         (n == 0) ? arg : new List<X>(node, rest.append(arg))
    def rev : List<X>(n) = rev(new List<X>())
    def rev(arg:List<X>):List<X>(n+arg.n) = 
         (n == 0) ? arg : rest.rev(new List<X>( node, arg))
    /** Return a list of compile-time unknown length, obtained by filtering
        this with f. */
    def filter(f: fun<X,boolean>):List<X> = 
         (n==0) ? this
         : (f(node) ? 
           new List<X>(node, rest.filter(f));
            : rest.filter(f))
    /** Return a list of m numbers from o..m-1. */
    def gen(m:int(&self>= 0)):List<int(&self>=0)>(m) = gen(0,m)
    /** Return a list of (m-i) elements, from i to m-1. */
    def gen(i:int(&self>=0), m:int(&self>=i)):List<int(&self>=0)>(m-i) = 
       (i == m)? new List<nat>() : new List<nat>(i, gen(i+1,m))
  }  
\end{verbatim}}

The class {\tt List} has three fields, the {\tt val} field {\tt n},
and the {\tt var} fields {\tt node} and {\tt rest}. {\tt rest} is
required to be a {\tt List} whose {\tt n} field has the value {\tt
this.n-1}.  {\tt n} is required to be non-negative -- this is checked
statically for each constructor.

Three constructors are provided. The first takes no arguments and
returns instances of {\tt List} which satisfy the constraint {\tt
n=0}. The second takes a single argument and returns a list of size
{\tt 1}. The third takes two arguments, {\tt node} (of type {\tt X})
and {\tt rest} of type {\tt List<X>}, and returns a list of size {\tt
rest.n+1}. That is, the length of the list returned depends on the
value of a parameter to the constructor. 

Functions that append one list to another or that reverse a list can
be defined quite naturally. In both cases the size of the
list returned is known statically (as a function of the size of the
list and the argument to the method).  The example also illustrates a
method {\tt filter} which returns a list whose size cannot be known
statically (it depends on properties of the argument function {\tt f}
which are not captured statically).

The {\tt gen} methods\footnote{These should really be {\tt static}.}
illustrate ``{\tt self}''-constraints. The first {\tt gen} method
takes a single argument {\tt m} that is required to be a non-negative
{\tt int}.  The second {\tt gen} method illustrates that the type of a
parameter can depend on the value of another parameter: {\tt m} is
required to be no less than {\tt i}. This assumption are necessary in
order to guarantee that the result type of the method is not empty,
that is, to guarantee {\tt m-i >= 0}.
\end{example}

{\em Note:
Language extensions to be made to get a fuller subset of \Xten:

\begin{itemize}
  \item Add multiple constructors.
  \item Permit arbitrary code in constructors.
  \item Add static state -- or at least distinguish between objects and traits.
  \item Permit fields to be overridden?
  \item Add functions.
  \item Add exceptions.
  \item Add places.
  \item Add async, finish, future.
  \item Add regions, distributions and arrays.
  \item Add generics with declaration-time
  covariance/contravariance/invariance, a la Scala.  Generics should
  be instantiable with arbitrary value types.
  \item Consider adding generic values (universal quantification).
  \item Arrays should be generic.
\end{itemize}
}

\subsection{Type system}
A {\em typing environment}, $\Gamma$, is a collection of variable
typings $x:T$ together with zero or more constraints on variables. The
constraint system satisfies the axiom:

$$
\tt \Gamma, x:C(\&c) \vdash c[x/self]
$$

\noindent Thus, for instance, we have:
$$
\tt x:List(\& n=3) \vdash x.n=3
$$
\noindent (asuming {\tt n} is a field in {\tt List}, and hence an abbreviation for {\tt self.n}).

Typing judgements for expressions are of the form $\tt \Gamma \vdash
e:T$, read as ``Under the assumptions $\Gamma$, {\tt e} has type {\tt
T}.'' We use sequence notation in the obvious way: $\tt \Gamma \vdash
\bar{x}:\bar{X}$ is shorthand for the collection of typing judgements
$\tt \Gamma \vdash x_1:X_1 \ldots \Gamma \vdash x_n:X_n$.

\subsubsection{Subtyping rules}
We will also use the sub-typing judgement $\tt \Gamma \vdash X \leq Y$
on types, read as ``Under the assumptions $\Gamma$ the type $\tt X$ is
a sub-type of $\tt Y$.

$$
\begin{array}{llll}
\tt \Gamma \vdash C \leq C
& \tt 
\from{\tt \Gamma \vdash C \leq D \ \ \ \Gamma \vdash D \leq E}
\infer{\tt\Gamma \vdash C \leq E}
&
\from{\tt class\ C(\ldots)\ extends\ D\ \ldots}
\infer{\tt\Gamma \vdash C \leq D}
&
\from{\tt \Gamma \vdash C \leq D \ \ \ \Gamma, c \vdash d}
\infer{\tt\Gamma \vdash C(\&c) \leq D(\&d)}
\end{array}
$$

Under the assumptions of well-formed programs, this relation is acyclic.

\subsubsection{Static semantics rules}
$$
\begin{array}{|l|}\hline
\begin{array}{ll}
\rname{\sc\tt (T-Null)}\from{}\infer{\tt \Gamma \vdash null: nullable\ T}  &
\rname{\sc\tt (T-Null-2)}\from{\tt\Gamma \vdash e:T}\infer{\tt \Gamma \vdash e: nullable\ T}  
\end{array}\\
\quad\\
\begin{array}{ll}
\rname{\sc\tt (T-Lit)}\from{}\infer{\tt \Gamma \vdash n:int(n)}  
&  \rname{\sc\tt (T-Var)}\from{}\infer{\tt \Gamma,x:T \vdash x:T}
\end{array} \\
\quad\\
\begin{array}{ll}
\rname{\sc\tt (T-Field-r)}\from{
  \begin{array}{l}
  \tt  \Gamma \vdash x:T \\
  \tt f:U \in fields(T) \\
  \tt \theta=[x/this] \\
 \end{array}}
\infer{\tt \Gamma \vdash x.f:U \theta}
&
\rname{\sc\tt (T-Field-w)}
\from{
  \begin{array}{l}
  \tt  \Gamma \vdash x:T \\
  \tt f:U \in varfields(T) \\
  \tt \theta=[x/this] \\
  \begin{array}{ll}
  \tt \Gamma \vdash g:V & \tt \Gamma \vdash V \leq U\theta
  \end{array}
 \end{array}}
\infer{\tt \Gamma \vdash x.f=g:U\theta}
\end{array}
\\
\quad\\
\begin{array}{ll}
\rname{\sc\tt (T-New)}\from{
  \begin{array}{l}
  \tt \Gamma \vdash \bar{y}:\bar{Y} \\
  \tt C(\bar{z}:\bar{Z}\&c):T \in ctor(C) \\
  \tt \theta=[\bar{y}/\bar{z}] \\
  \begin{array}{ll}
  \tt \Gamma \vdash \bar{Y} \leq \bar{Z}\theta &
  \tt \Gamma \vdash c\theta 
  \end{array}
 \end{array}}
\infer{\tt \Gamma \vdash new\ C(\bar{y}):T\theta}
&
\rname{\sc\tt (T-Invoke)}
\from{
  \begin{array}{l}
   \tt \Gamma \vdash x,\bar{y}:X,\bar{Y} \\
   \tt m(\bar{z}:\bar{Z}\&c):T \in mType(X) \\
   \tt \theta=[x/this,\bar{y}/\bar{z}] \\
   \begin{array}{ll}
   \tt \Gamma \vdash \bar{Y} \leq \bar{Z}\theta &
    \tt \Gamma \vdash c\theta
   \end{array}
 \end{array}}
\infer{\tt \Gamma \vdash x.m(\bar{y}):T\theta}
\end{array}
\\
\quad\\
\begin{array}{ll}
\rname{\sc\tt (T-Local)}
\from{
  \begin{array}{l}
   \tt \Gamma[\bar{x}:\bar{X}] \vdash \bar{d}:\bar{Y} \\
   \tt \Gamma[\bar{x}:\bar{X}] \vdash \bar{Y} \leq \bar{X} \\
   \tt \Gamma[\bar{x}:\bar{X}] \vdash e:Z
 \end{array}}
\infer{\tt \Gamma \vdash \{val\ \bar{x}:\bar{X}=\bar{d};e\}:\exists \bar{\tt x}:\bar{\tt X}.Z}
&
\rname{\sc\tt (T-Cast)}\from{}\infer{\tt \Gamma \vdash (T) e: T}
\end{array}
\\
\quad\\
\begin{array}{ll}
\rname{\sc\tt (T-Seq)}
\from{
  \begin{array}{ll}
   \tt \Gamma \vdash d:S   & 
   \tt \Gamma \vdash e:T
 \end{array}}
\infer{\tt \Gamma \vdash d;e:T}
&
\rname{\sc\tt (T-Cond)}
\from{
  \begin{array}{l}
   \tt \Gamma \vdash c:boolean \\
  \begin{array}{ll}
   \tt \Gamma, c \vdash r:T &  \tt \Gamma, ! c \vdash s:T
 \end{array}
 \end{array}}
\infer{\tt \Gamma \vdash (c?r:s) :T}
\end{array}\\
\quad\\
\rname{\sc\tt (T-Method)}
\from{
  \begin{array}{l}
   \tt class\ C(\bar{\tt f}:\bar{\tt X},var\ \bar{\tt g}:\bar{\tt Y})\,extends\ D\{K\ \bar{\tt M}\}\\
   \tt M_i = def\ m(\bar{\tt z}:\bar{\tt Z}\&d):T=e\\
   \tt\Gamma, \bar{\tt z}:\bar{\tt Z}, this:C,d \vdash e:T
 \end{array}}
\infer{\tt \Gamma \vdash M_i\ OK\ in\ C}\\
\quad\\
\rname{\sc\tt (T-Class)}\from{
  \begin{array}{l}
\tt   K=def\ C(\overline{\tt u,f,g}:\overline{\tt U,X,Y}\& c):T
                       = \{super(\bar{\tt u}); this.\overline{\tt f,g}=\overline{\tt f,g};\} \\
\tt  D\ OK \ \ \ \ \  K_d=ctor(D)= def\ D(\bar{\tt u}:\bar{\tt U}\&d):Z=\ldots \\
  \tt \overline{\tt u,f,g}:\overline{\tt U,X,Y},c\vdash d \\
  \tt \overline{\tt u,f,g}:\overline{\tt U,X,Y},c,this:Z,this.\overline{f,g}=\overline{f,g}\vdash this:T \\
  \tt \bar{\tt M}\ OK \ in \ C\\
 \end{array}}
\infer{\tt class\ C(\bar{\tt f}:\bar{\tt X},\, var\, \tt\bar{\tt g}:\bar{\tt Y})\,extends\ D\{K\ \bar{\tt M}\}\ OK}\\
\quad\\ \hline
\end{array}
$$

Note that {\tt (T-Cond)} applies to conditionals {\tt c?r:s} where the
test is a constraint term.  This means it can be built up using the
primitive functions supplied with the constraint system, and must not
reference any mutable variable.

\subsubsection{Auxiliary definitions}
The rules use some auxiliary definitions. 
For any type {\tt T}, {\tt fields(T)} is the set of typed fields for that type:
$$
\begin{array}{l}
\begin{array}{ll}
\tt fields(Object)=\emptyset & \tt fields(C(\&c))=fields(C)
\end{array} \\
\tt fields(C)=fields(D),\bar{\tt x}:\bar{\tt X},\bar{\tt y}:\bar{\tt Y}
\ \ \mbox{\em if 
{\tt class C($\tt \bar{\tt x}:\bar{\tt X},var\ \bar{\tt y}:\bar{\tt Y}$\&c) extends D\ldots}}  
\end{array}
$$

%%\noindent Given two sets $S$ and $T$ of field type assertions, the set $S,T$
%%has all the elements of $T$ and all the elements of $S$ except those
%%whose variable is the same as a variable of an element in $T$. Thus
%%the above definitions implement field overriding: fields of {\tt C}
%%with the same name hide the fields of {\tt D} with the same name (even
%%if they have different types).

For any type {\tt T}, {\tt varfields(T)} is the set of mutable fields for that type:
$$
\begin{array}{l}
\begin{array}{ll}
\tt varfields(Object)=\emptyset & \tt varfields(C(\&c))=varfields(C)
\end{array} \\
\tt varfields(C)=varfields(D)[\bar{y}:\bar{Y}]\ \ \mbox{\em if 
{\tt class C($\tt \bar{\tt x}:\bar{\tt X},var\ \bar{\tt y}:\bar{\tt Y}$\&c) extends D\ldots}}  
\end{array}
$$

The set of constructors of a type {\tt T=C(\&c)} ({\tt ctor(C)}) is
the set of all elements {$\tt C(\bar{\tt x}:\bar{\tt X}\&c):T$} such
that the constructor {$\tt def C(\bar{\tt x}:\bar{\tt
X}\&c):T$=\ldots} occurs in the definition of the class {\tt C}. We omit a formal definition.
The set of methods of a type {\tt T=C(\&c)}, {\tt mType(T)}, is the
set of all elements {$\tt m(\bar{z}:\bar{Z}\&c):T$}  such that the method
{$\tt def m(\bar{z}:\bar{Z}\&c):T=e$} occurs in the definition of the class
{\tt C} or an ancestor class. We omit a formal definition.

\subsection{Dynamic semantics}

{\em Establish theorems: Subject reduction, well-typed program can do no wrong.}
{\em Adapt from \cite{X10-concur}.}

\subsection{Example revisited}

Here we consider a concrete constraint system with the type {\tt int} (with arithmetic operations).

{\em Show the List example type-checks.}

\section{\FXten{} with places}

{\em
  \begin{itemize}
    \item Follow the outline of my recent set of slides.
    \item Every reference object now has a final {\tt place} field, {\tt loc}.
    \item The constant {\tt here} evaluates to the current place.
    \item {\tt place} is a value type, with a fixed but unknown set of
      operations, we can assume {\tt .next:place}.
    \item The type {\tt T@!} is represented by {\tt T(\&loc=here}), and
      the type {\tt T@x} by {\tt T(\&loc=x.loc)}.
    \item The rules for field read/write and method invocation are changed to require that
      the subject be of type {\tt \_(\&loc=here)}.
    \item The new place-shifting control construct {\tt eval(p)e} is introduced.
  \end{itemize}
}

%\section{\FXten{} with regions}

%\section{\FXten{} with distributions}

\bibliographystyle{plain}
\bibliography{master}
\end{document}
