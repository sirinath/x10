\chapter{Structs}
\label{XtenStructs}
\label{StructClasses}
\label{Structs}
\index{struct}

X10 objects are a powerful general-purpose programming tool. However, the
power must be paid for in space and time. In space, a typical object
implementation requires some extra memory for run-time class information, as
well as a pointer for each reference to the object. In time, a typical object
requires an extra indirection to read or write data, and some run-time
computation to figure out which method body to call.

For high-performance computing, this overhead may not be acceptable for all
objects. X10 provides structs, which are stripped-down objects. They are less
powerful than objects; in particular they lack inheritance and mutable fields.
Without inheritance, method calls do not need to do any lookup; they can be
implemented directly. Accordingly, structs can be implemented and used more
cheaply than objects, potentially avoiding the space and time overhead.
(Currently, the C++ back end avoids the overhead, but the Java back end
implements structs as Java objects and does not avoid it.)

Structs and classes are interoperable. Both can implement interfaces (in
particular, like all X10 values they implement \xcd`Any`), and subprocedures
whose arguments are defined by interfaces can take both structs and classes.
(Some caution is necessary here: referring to a struct through an interface
requires overhead similar to that required for an object.)

In many cases structs can be converted to classes or classes to structs,
within the constraints of structs. If you start off defining a struct and
decide you need a class instead, the code change required is simply changing
the keyword \xcd`struct` to \xcd`class`. If you have a class that does not use
inheritance or mutable fields, it can be converted to a struct by changing its
keyword. Client code using the struct that was a class will need certain
changes: \eg, the \xcd`new` keyword must be added in constructor calls, and
structs (unlike classes) do not have default values.



\section{Struct declaration}
\index{struct!declaration}

%##(StructDecl TypeParamsWithVariance Properties WhereClause Interfaces ClassBody
\begin{bbgrammar}
%(FROM #(prod:StructDecl)#)
          StructDecl \: Mods\opt \xcd"struct" Id TypeParamsWithVariance\opt Properties\opt WhereClause\opt Interfaces\opt ClassBody & (\ref{prod:StructDecl}) \\
%(FROM #(prod:TypeParamsWithVariance)#)
TypeParamsWithVariance \: \xcd"[" TypeParamWithVarianceList \xcd"]" & (\ref{prod:TypeParamsWithVariance}) \\
%(FROM #(prod:Properties)#)
          Properties \: \xcd"(" PropertyList \xcd")" & (\ref{prod:Properties}) \\
%(FROM #(prod:WhereClause)#)
         WhereClause \: DepParams & (\ref{prod:WhereClause}) \\
%(FROM #(prod:Interfaces)#)
          Interfaces \: \xcd"implements" InterfaceTypeList & (\ref{prod:Interfaces}) \\
%(FROM #(prod:ClassBody)#)
           ClassBody \: \xcd"{" ClassBodyDecls\opt \xcd"}" & (\ref{prod:ClassBody}) \\
\end{bbgrammar}
%##)



All fields of a struct must be \xcd`val`.

A struct \Xcd{S} cannot contain a field of type \Xcd{S}, or a field of struct
type \Xcd{T} which, recursively, contains a field of type \Xcd{S}.  This
restriction is necessary to permit \xcd`S` to be implemented as a contiguous
block of memory of size equal to the sum of the sizes of its fields.  


Values of a struct \Xcd{C} type can be created by invoking a constructor
defined in \Xcd{C}.  Unlike for classes, the  \Xcd{new} keyword is optional
for struct constructors.

\begin{ex}
%~~gen ^^^ Structs10
% package Structs.For.Muckts;
%~~vis
\begin{xten}
struct Polar(r:Double, theta:Double){
  def this(r:Double, theta:Double) {property(r,theta);}
  static val Origin = Polar(0,0);
  static val x0y1   = Polar(1, 3.14159/2);
  static val x1y0   = new Polar(1, 0); 
}
\end{xten}
%~~siv
%
%~~neg
\end{ex}
Structs support the same notions of generics, properties, and constrained
types that classes do.  

\begin{ex}The \xcd`Pair` type below provides pairs
of values; the \xcd`diag()` method applies only when the two elements of the
pair are equal, and returns that common value: 
%~~gen ^^^ Structs20
% package Structs20;
%~~vis
\begin{xten}
struct Pair[T,U](t:T, u:U) {
  def this(t:T, u:U) { property(t,u); }
  def diag(){T==U && t==u} = t;
}
\end{xten}
%~~siv
% class Hook{ def run() {
%   val p = Pair(3,3); 
%   return p.diag() == 3;
% }}
%~~neg
\end{ex}

\section{Boxing of structs}
\index{auto-boxing!struct to interface}
\index{struct!auto-boxing}
\index{struct!casting to interface}
\label{auto-boxing} 
If a struct \Xcd{S} implements an interface \Xcd{I} (\eg, \Xcd{Any}),
a value \Xcd{v} of type \Xcd{S} can be assigned to a variable of type
\Xcd{I}. The implementation creates an object \Xcd{o} that is an
instance of an anonymous class implementing \Xcd{I} and containing
\Xcd{v}.  The result of invoking a method of \Xcd{I} on \Xcd{o} is the
same as invoking it on \Xcd{v}. This operation is termed {\em auto-boxing}.
It allows full interoperability of structs and objects---at the cost of losing
the extra efficiency of the structs when they are boxed.

In a generic class or struct obtained by instantiating a type parameter
\Xcd{T} with a struct \Xcd{S}, variables declared at type \Xcd{T} in the body
of the class are not boxed. They are implemented as if they were declared at
type \Xcd{S}.

\section{Optional Implementation of {\tt Any} methods}
\label{StructAnyMethods}
\index{Any!structs}

Two
structs are equal (\Xcd{==}) if and only if their corresponding fields
are equal (\Xcd{==}). 

All structs implement \Xcd{x10.lang.Any}. 
Structs are required to implement the following methods from \xcd`Any`.  
Programmers need not provide them; X10 will produce them automatically if 
the program does not include them. 
\begin{xten}
  public def equals(Any):Boolean;
  public def hashCode():Int;
  public def typeName():String;
  public def toString():String;  
\end{xten}


A programmer who provides an explicit implementation
of \Xcd{equals(Any)} for a struct \Xcd{S} should also consider
supplying a definition for \Xcd{equals(S):Boolean}. This will often
yield better performance since the cost of an upcast to \Xcd{Any} and
then a downcast to \Xcd{S} can be avoided.

\section{Primitive Types}
\index{types!primitive}
\index{primitive types}
\index{Int}
\index{UInt}
\index{Long}
\index{ULong}
\index{Char}
\index{Byte}
\index{UByte}
\index{Boolean}
\index{Short}
\index{UShort}
\index{Float}
\index{Double}

Certain types that might be built in to other languages are in fact
implemented as structs in package \xcd`x10.lang` in X10. Their methods and
operations are often provided with \xcd`@Native` (\Sref{NativeCode}) rather
than X10 code, however. These types are:
\begin{xten}
Boolean, Char, Byte, Short, Int, Long
Float, Double, UByte, UShort, UInt, ULong
\end{xten}

\subsection{Signed and Unsigned Integers}
\index{types!unsigned}
\index{integers!unsigned}
\index{unsigned}

X10 has an unsigned integer type corresponding to each integer type:
\xcd`UInt` is an unsigned \xcd`Int`, and so on. These types can be used for
binary programming, or when an extra bit of precision for counters or other
non-negative numbers is needed in integer arithmetic. However, X10 does not
otherwise encourage the use of unsigned arithmetic.




 
\section{Generic programming with structs}
\index{struct!generic}
\index{generic!struct}



\noo{The rest of this section is under discussion.  The example is wrong; it
ignores the fact that values can be functions.}
The programmer must be aware of the different interpretations of
equality for structs and classes and ensure that the code is correctly
written for both cases. If necessary the programmer can write code
that distinguishes between the two cases (a type parameter \Xcd{X} is
instantiated to a struct or not) as follows:

\begin{xten}
val x:X = ...;
if (x instanceof Object) { // x is a real object
   val x2 = x as Object; // this cast will always succeed.
   ...
} else { // x is a struct
   ...
}
\end{xten}
 
  
\section{Example structs}

\xcd`x10.lang.Complex` provides a detailed example of a practical struct,
suitable for use in a library.  For a shorter example, we define the
\xcd`Pair` struct---available in \xcd`x10.util.Pair`.  A \xcd`Pair` packages
two values of possibly unrelated type together in a single value, \eg, to
return two values from a function.  

\xcd`divmod` computes the quotient and remainder of \xcd`a \div b` (naively).
It returns both, packaged as a \xcd`Pair[UInt, UInt]`.  Note that the
constructor uses type inference, and that the quotient and remainder are
accessed through the \xcd`first` and \xcd`second` fields.
%~~gen ^^^ Structs30
% package Structs30Pair;
%~~vis
\begin{xten}
struct Pair[T,U] {
    public val first:T;
    public val second:U;
    public def this(first:T, second:U):Pair[T,U] {
        this.first = first;
        this.second = second;
    }
    public def toString() = "(" + first + ", " + second + ")";
}
class Example {
  static def divmod(var a:UInt, b:UInt): Pair[UInt, UInt] {
     assert b > 0;
     var q : UInt = 0;
     while (a > b) {q++; a -= b;}
     return Pair(q, a); 
  }
  static def example() {
     val qr = divmod(22, 7);
     assert qr.first == 3 && qr.second == 1;
  }
}
\end{xten}
%~~siv
%class Hook{ def run() { Example.example(); return true; } } 
%~~neg

\section{Nested Structs}
\index{struct!static nested}
\index{static nested struct}

Static nested structs may be defined, essentially as static nested classes
except for making them structs
(\Sref{StaticNestedClasses}).  Inner structs may be defined, essentially as
inner classes except making them structs (\Sref{InnerClasses}).

\section{Converting Between Classes And Structs}

Code written using structs can be modified to use classes, or vice versa.
Caution must be used in certain places. 

Class and struct {\em definitions} are syntactically nearly identical:
change the \xcd`class` keyword to \xcd`struct` or vice versa.  Of course,
certain important class features can't be used with structs, such as
inheritance and \xcd`var` fields. 

Converting code that {\em uses} the class or struct requires a certain amount
of caution.
Suppose, in particular, that we want to convert the class \xcd`Class2Struct`
to a struct, and \xcd`Struct2Class` to a class.
%~~gen ^^^ Structs40
%package Structs.Converting;
%~~vis
\begin{xten}
class Class2Struct {
  val a : Int;
  def this(a:Int) { this.a = a; }
  def m() = a;
}
struct Struct2Class { 
  val a : Int;
  def this(a:Int) { this.a = a; }
  def m() = a;
}
\end{xten}
%~~siv
%
%~~neg

\begin{enumerate}

\item Class constructors require the \xcd`new` keyword; struct constructors
      allow  it but do not require it.  
      \xcd`Struct2Class(3)` to will need to be converted to 
      \xcd`new Struct2Class(3)`.

\item Objects and structs have different notions of \xcd`==`.  
      For objects, \xcd`==` means ``same object''; for structs, it means
      ``same contents''. Before conversion, both \xcd`assert`s in the
      following program succeed.  After converting and fixing constructors,
      both of them fail.
%~~gen ^^^ Structs50
%package Structs.Converting.And.Conniving;
% class Class2Struct {
%   val a : Int;
%   def this(a:Int) { this.a = a; }
%   def m() = a;
% }
% struct Struct2Class { 
%   val a : Int;
%   def this(a:Int) { this.a = a; }
%   def m() = a;
% }
%class Example {
% static def Examplle() {
%~~vis
\begin{xten}
val a = new Class2Struct(2);
val b = new Class2Struct(2);
assert a != b;
val c = Struct2Class(3);
val d = Struct2Class(3);
assert c==d;
\end{xten}
%~~siv
%}}
%~~neg

\item Objects can be set to \xcd`null`.  Structs cannot.  There is no
      default value like \xcd`null` available for structs.  


\end{enumerate}
