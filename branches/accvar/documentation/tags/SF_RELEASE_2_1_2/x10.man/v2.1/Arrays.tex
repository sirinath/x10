\chapter{Local and Distributed Arrays}\label{XtenArrays}\index{array}

\Xcd{Array}s provide indexed access to data at a single \Xcd{Place}, {\em via}
\Xcd{Point}s---indices of any dimensionality. \Xcd{DistArray}s is similar, but
spreads the data across multiple \xcd`Place`s, {\em via} \Xcd{Dist}s.  
We refer to arrays either sort as ``general arrays''.  


This chapter provides an overview of the \Xcd{x10.array} classes \Xcd{Array}
and \Xcd{DistArray}, and their supporting classes \Xcd{Point}, \Xcd{Region}
and \Xcd{Dist}.  


\section{Points}\label{point-syntax}
\index{point}
\index{point!syntax}


General arrays are indexed by \xcd`Point`s, which are $n$-dimensional tuples of
integers.  The \xcd`rank`
property of a point gives its dimensionality.  Points can be constructed from
integers or \xcd`Array[Int](1)`s by
the \xcd`Point.make` factory methods:
%~~gen ^^^ArraysPointsExample1
% package Arrays.Points.Example1;
% class Example1 {
% def example1() {
%~~vis
\begin{xten}
val origin_1 : Point{rank==1} = Point.make(0);
val origin_2 : Point{rank==2} = Point.make(0,0);
val origin_5 : Point{rank==5} = Point.make([0,0,0,0,0]);
\end{xten}
%~~siv
% } } 
%~~neg

%~~type~~`~~`~~ ~~ ^^^ Arrays10
There is an implicit conversion from \xcd`Array[Int](1)` to 
%~~type~~`~~`~~ ~~ ^^^ Arrays20
\xcd`Point`, giving
a convenient syntax for constructing points: 

%~~gen ^^^ Arrays30
% package Arrays.Points.Example2;
% class Example{
% def example() {
%~~vis
\begin{xten}
val p : Point = [1,2,3];
val q : Point{rank==5} = [1,2,3,4,5];
val r : Point(3) = [11,22,33];
\end{xten}
%~~siv
% } } 
%~~neg

The coordinates of a point are available by subscripting; \xcd`p(i)` is the
\xcd`i`th coordinate of the point \xcd`p`. 
\xcdmath`Point($n$)` is a \Xcd{type}-defined shorthand  for 
\xcdmath`Point{rank==$n$}`.


\section{Regions}\label{XtenRegions}\index{region}
\index{region!syntax}

A region is a set of points of the same rank.  {}\Xten{}
provides a built-in class, \xcd`x10.array.Region`, to allow the
creation of new regions and to perform operations on regions. 
Each region \xcd`R` has a property \xcd`R.rank`, giving the dimensionality of
all the points in it.

%~~gen ^^^ Arrays40
% package Arrays40;
% class Example {
% static def example() {
%~~vis
\begin{xten}
val MAX_HEIGHT=20;
val Null = Region.makeUnit();  // Empty 0-dimensional region
val R1 = 1..100; // 1-dim region with extent 1..100
val R2 = (1..100) as Region(1); // same as R1
val R3 = (0..99) * (-1..MAX_HEIGHT);
val R4 = Region.makeUpperTriangular(10);
val R5 = R4 && R3; // intersection of two regions
\end{xten}
%~~siv
% } } 
%~~neg

The expression \xcdmath`m..n`, for integer expressions \Xcd{m} and \Xcd{n},
evaluates to the rectangular, rank-1 region consisting of the points
$\{$\xcdmath`[m]`, \dots, \xcdmath`[n]`$\}$. If \xcdmath`m` is greater than
\xcdmath`n`, the region \Xcd{m..n} is empty.

%%MAYBE%% A region may be constructed by converting from a rail of
%%MAYBE%% regions (\eg, \xcd`R4` above).
%%MAYBE%% The region constructed from a rail of regions represents the Cartesian product
%%MAYBE%% of the arguments. \Eg, \Xcd{R8} is a region of {$100 \times 16 \times 78$}
%%MAYBE%% points, in
%%MAYBE%% %~s~gen
%%MAYBE%% %package Arrays.Region.RailOfRegions;
%%MAYBE%% %class Example{
%%MAYBE%% %def example(){
%%MAYBE%% %~s~vis
%%MAYBE%% \begin{xten}
%%MAYBE%%   val R8 = [1..100, 3..18, 1..78] as Region(3);
%%MAYBE%% \end{xten}
%%MAYBE%% %~s~siv
%%MAYBE%% %}}
%%MAYBE%% %~s~neg


\index{region!upperTriangular}
\index{region!lowerTriangular}\index{region!banded}

Various built-in regions are provided through  factory
methods on \xcd`Region`.  
\begin{itemize}
\item \Xcd{Region.makeEmpty(n)} returns an empty region of rank \Xcd{n}.
\item \Xcd{Region.makeFull(n)} returns the region containing all points of
      rank \Xcd{n}.  
\item \Xcd{Region.makeUnit()} returns the region of rank 0 containing the
      unique point of rank 0.  It is useful as the identity for Cartesian
      product of regions.
\item \Xcd{Region.makeHalfspace(normal:Point, k:Int)} returns the unbounded
      half-space of rank \Xcd{normal.rank}, consisting of all points \Xcd{p}
      satisfying \xcdmath`p$\cdot$normal $\le$ k`.
\item \Xcd{Region.makeRectangular(min, max)}, where \Xcd{min} and \Xcd{max}
      are \Xcd{Int} rails or valrails of length \Xcd{n}, returns a
      \Xcd{Region(n)} equal to: 
      \xcdmath`[min(0) .. max(0), $\ldots$, min(n-1)..max(n-1)]`.
\item \Xcd{Region.make(regions)} constructs the Cartesian product of the
      \Xcd{Region(1)}s in \Xcd{regions}.
\item \Xcd{Region.makeBanded(size, upper, lower)} constructs the
      banded \Xcd{Region(2)} of size \Xcd{size}, with \Xcd{upper} bands above
      and \Xcd{lower} bands below the diagonal.
\item \Xcd{Region.makeBanded(size)} constructs the banded \Xcd{Region(2)} with
      just the main diagonal.
\item \xcd`Region.makeUpperTriangular(N)` returns a region corresponding
to the non-zero indices in an upper-triangular \xcd`N x N` matrix.
\item \xcd`Region.makeLowerTriangular(N)` returns a region corresponding
to the non-zero indices in a lower-triangular \xcd`N x N` matrix.
\item 
  If \xcd`R` is a region, and \xcd`p` a Point of the same rank, then 
%~~exp~~`~~`~~R:Region, p:Point(R.rank) ~~ ^^^ Arrays50
  \xcd`R+p` is \xcd`R` translated forwards by 
  \xcd`p` -- the region whose
%~~exp~~`~~`~~r:Point, p:Point(r.rank) ~~ ^^^ Arrays60
  points are \xcd`r+p` 
  for each \xcd`r` in \xcd`R`.
\item 
  If \xcd`R` is a region, and \xcd`p` a Point of the same rank, then 
%~~exp~~`~~`~~R:Region, p:Point(R.rank) ~~ ^^^ Arrays70
  \xcd`R-p` is \xcd`R` translated backwards by 
  \xcd`p` -- the region whose
%~~exp~~`~~`~~r:Point, p:Point(r.rank) ~~ ^^^ Arrays80
  points are \xcd`r-p` 
  for each \xcd`r` in \xcd`R`.

\end{itemize}

All the points in a region are ordered canonically by the
lexicographic total order. Thus the points of the region \xcd`(1..2)*(1..2)`
are ordered as 
\begin{xten}
(1,1), (1,2), (2,1), (2,2)
\end{xten}
Sequential iteration statements such as \xcd`for` (\Sref{ForAllLoop})
iterate over the points in a region in the canonical order.

A region is said to be {\em rectangular}\index{region!convex} if it is of
the form \xcdmath`(T$_1$ * $\cdots$ * T$_k$)` for some set of intervals
\xcdmath`T$_i = $ l$_i$ .. h$_i$ `. Such a
region satisfies the property that if two points $p_1$ and $p_3$ are
in the region, then so is every point $p_2$ between them (that is, it is {\em convex}). 
(Banded and triangular regions are not rectangular.)
The operation
%~~exp~~`~~`~~R:Region ~~ ^^^ Arrays90
\xcd`R.boundingBox()` gives the smallest rectangular region containing
\xcd`R`.

\subsection{Operations on regions}
\index{region!operations}

Let \xcd`R` be a region. A {\em sub-region} is a subset of \Xcd{R}.
\index{region!sub-region}

Let \xcdmath`R1` and \xcdmath`R2` be two regions whose types establish that
they are of the same rank. Let \xcdmath`S` be another region; its rank is
irrelevant. 

\xcdmath`R1 && R2` is the intersection of \xcdmath`R1` and
\xcdmath`R2`, \viz, the region containing all points which are in both
\Xcd{R1} and \Xcd{R2}.  \index{region!intersection}
%~~exp~~`~~`~~ ~~ ^^^ Arrays100
For example, \xcd`1..10 && 2..20` is \Xcd{2..10}.


\xcdmath`R1 * S` is the Cartesian product of \xcdmath`R1` and
\xcdmath`S`,  formed by pairing each point in \xcdmath`R1` with every  point in \xcdmath`S`.
\index{region!product}
%~~exp~~`~~`~~ ~~ ^^^ Arrays110
Thus, \xcd`(1..2)*(3..4)*(5..6)`
is the region of rank \Xcd{3} containing the eight points with coordinates
\xcd`[1,3,5]`, \xcd`[1,3,6]`, \xcd`[1,4,5]`, \xcd`[1,4,6]`,
\xcd`[2,3,5]`, \xcd`[2,3,6]`, \xcd`[2,4,5]`, \xcd`[2,4,6]`.


For a region \xcdmath`R` and point \xcdmath`p` of the same rank,
%~~exp~~`~~`~~R:Region, p:Point{p.rank==R.rank} ~~ ^^^ Arrays120
\xcd`R+p` 
and
%~~exp~~`~~`~~R:Region, p:Point{p.rank==R.rank} ~~ ^^^ Arrays130
\xcd`R-p` 
represent the translation of the region
forward 
and backward 
by \xcdmath`p`. That is, \Xcd{R+p} is the set of points
\Xcd{p+q} for all \Xcd{q} in \Xcd{R}, and \Xcd{R-p} is the set of \Xcd{q-p}.

More \Xcd{Region} methods are described in the API documentation.

\section{Arrays}
\index{array}

Arrays are organized data, arranged so that it can be accessed by subscript.
An \xcd`Array[T]` \Xcd{A} has a \Xcd{Region} \Xcd{A.region}, telling which
\Xcd{Point}s are in \Xcd{A}.  For each point \Xcd{p} in \Xcd{A.region},
\Xcd{A(p)} is the datum of type \Xcd{T} associated with \Xcd{p}.  X10
implementations should 
attempt to store \xcd`Array`s efficiently, and to make array element accesses
quick---\eg, avoiding constructing \Xcd{Point}s when unnecessary.

This generalizes the concepts of arrays appearing in many other programming
languages.  A \Xcd{Point} may have any number of coordinates, so an
\xcd`Array` can have, in effect, any number of integer subscripts.  

Indeed, it is possible to write code that works on \Xcd{Array}s regardless 
of dimension.  For example, to add one \Xcd{Array[Int]} \Xcd{src} into another
\Xcd{dest}, 
%~~gen ^^^ Arrays140
%package Arrays.Arrays.Arrays.Example;
% class Example{
%~~vis
\begin{xten}
static def addInto(src: Array[Int], dest:Array[Int])
  {src.region == dest.region}
  = {
    for (p in src.region) 
       dest(p) += src(p);
  }
\end{xten}
%~~siv
%}
% class Hook{
%   def run() { 
%     val a = [1,2,3]; val b = [10, 20, 30] as Array[Int]{self.region == a.region};
%     Example.addInto(a, b);
%     return b(0) == 11 && b(1) == 22 && b(2) == 33;
% }}
%~~neg
\noindent
Since \Xcd{p} is a \Xcd{Point}, it can hold as many coordinates as are
necessary for the arrays \Xcd{src} and \Xcd{dest}.

The basic operation on arrays is subscripting: if \Xcd{A} is an \Xcd{Array[T]}
and \Xcd{p} a point with the same rank as \xcd`A.region`, then
%~~exp~~`~~`~~A:Array[Int], p:Point{self.rank == A.region.rank} ~~ ^^^ Arrays150
\xcd`A(p)`
is the value of type \Xcd{T} associated with point \Xcd{p}.

Array elements can be changed by assignment. If \Xcd{t:T}, 
%~~gen ^^^ Arrays160
%package Arrays.Arrays.Subscripting.Is.From.Mars;
%class Example{
%def example[T](A:Array[T], p: Point{rank == A.region.rank}, t:T){
%~~vis
\begin{xten}
A(p) = t;
\end{xten}
%~~siv
%} } 
%~~neg
modifies the value associated with \Xcd{p} to be \Xcd{t}, and leaves all other
values in \Xcd{A} unchanged.

An \Xcd{Array[T]} \Xcd{A} has: 
\begin{itemize}
%~~exp~~`~~`~~A:Array[Int] ~~ ^^^ Arrays170
\item \xcd`A.region`: the \Xcd{Region} upon which \Xcd{A} is defined.
%~~exp~~`~~`~~A:Array[Int] ~~ ^^^ Arrays180
\item \xcd`A.size`: the number of elements in \Xcd{A}.
%~~exp~~`~~`~~A:Array[Int] ~~ ^^^ Arrays190
\item \xcd`A.rank`, the rank of the points usable to subscript \Xcd{A}.
      Identical to \Xcd{A.region.rank}.
\end{itemize}

\subsection{Array Constructors}
\index{array!constructor}

To construct an array whose elements all have the same value \Xcd{init}, call
\Xcd{new Array[T](R, init)}. 
For example, an array of a thousand \xcd`"oh!"`s can be made by:
%~~exp~~`~~`~~ ~~ ^^^ Arrays200
\xcd`new Array[String](1..1000, "oh!")`.


To construct and initialize an array, call the two-argument constructor. 
\Xcd{new Array[T](R, f)} constructs an array of elements of type \Xcd{T} on
region \Xcd{R}, with \Xcd{A(p)} initialized to \Xcd{f(p)} for each point
\Xcd{p} in \Xcd{R}.  \Xcd{f} must be a function taking a point of rank
\Xcd{R.rank} to a value of type \Xcd{T}.  \Eg, to construct an array of a
hundred zero values, call
%~~exp~~`~~`~~ ~~ ^^^ Arrays210
\xcd`new Array[Int](1..100, (Point(1))=>0)`. 
To construct a multiplication table, call
%~~exp~~`~~`~~ ~~ ^^^ Arrays220
\xcd`new Array[Int]((0..9)*(0..9), (p:Point(2)) => p(0)*p(1))`.

Other constructors are available; see the API documentation and
\Sref{sect:ArrayCtors}. 

\subsection{Array Operations}
\index{array!operations on}

The basic operation on \Xcd{Array}s is subscripting.  If \Xcd{A:Array[T]} and 
\xcd`p:Point{rank == A.rank}`, then \Xcd{a(p)} is the value of type \Xcd{T}
appearing at position \Xcd{p} in \Xcd{A}.    The syntax is identical to
function application, and, indeed, arrays may be used as functions.
\Xcd{A(p)} may be assigned to, as well, by the usual assignment syntax
%~~exp~~`~~`~~A:Array[Int], p:Point{rank == A.rank}, t:Int ~~ ^^^ Arrays230
\xcd`A(p)=t`.
(This uses the application and setting syntactic sugar, as given in \Sref{set-and-apply}.)

Sometimes it is more convenient to subscript by integers.  Arrays of rank 1-4
can, in fact, be accessed by integers: 
%~~gen ^^^ Arrays240
%package Arrays240;
%class Example{
%def example(){
%~~vis
\begin{xten}
val A1 = new Array[Int](1..10, 0);
A1(4) = A1(4) + 1;
val A4 = new Array[Int]((1..2)*(1..3)*(1..4)*(1..5), 0);
A4(2,3,4,5) = A4(1,1,1,1)+1;
\end{xten}
%~~siv
%}}
%~~neg



Iteration over an \Xcd{Array} is defined, and produces the \Xcd{Point}s of the
array's region.  If you want to use the values in the array, you have to
subscript it.  For example, you could double every element of an
\Xcd{Array[Int]} by: 
%~~gen ^^^ Arrays250
%package Arrays250;
%class Example{
%static def example(A:Array[Int]) {
%~~vis
\begin{xten}
for (p in A) A(p) = 2*A(p);
\end{xten}
%~~siv
%}}
% class Hook{ def run() { val a = [1,2]; Example.example(a); return a(0)==2 && a(1)==4; }}
%~~neg



\section{Distributions}\label{XtenDistributions}
\index{distribution}

Distributed arrays are spread across multiple \xcd`Place`s.  
A {\em distribution}, a mapping from a region to a set of places, 
describes where each element of a distributed array is kept.
Distributions are embodied by the class \Xcd{x10.array.Dist}.
This class is \xcd`final` in
{}\XtenCurrVer; future versions of the language may permit
user-definable distributions. 
The {\em rank} of a distribution is the rank of the underlying region, and
thus the rank of every point that the distribution applies to.



%~~gen ^^^ Arrays260
%package Arrays.Dist_example_a;
% class Example{
% def example() {
%~~vis
\begin{xten}
val R  <: Region = 1..100;
val D1 <: Dist = Dist.makeBlock(R);
val D2 <: Dist = R -> here;
\end{xten}
%~~siv
% } } 
%~~neg

Let \xcd`D` be a distribution. 
%~~exp~~`~~`~~D:Dist ~~ ^^^ Arrays270
\xcd`D.region` 
denotes the underlying
region. 
Given a point \xcd`p`, the expression
%~~exp~~`~~`~~ D:Dist, p:Point{p.rank == D.rank}~~ ^^^ Arrays280
\xcd`D(p)` represents the application of \xcd`D` to \xcd`p`, that is,
the place that \xcd`p` is mapped to by \xcd`D`. The evaluation of the
expression \xcd`D(p)` throws an \xcd`ArrayIndexOutofBoundsException`
if \xcd`p` does not lie in the underlying region.
%%NO-R2D2-CONV%% 
%%NO-R2D2-CONV%% When operated on as a distribution, a region \xcd`R` implicitly
%%NO-R2D2-CONV%% behaves as the distribution mapping each item in \xcd`R` to \xcd`here`
%%NO-R2D2-CONV%% (\ie, \xcd`R->here`, see below). Conversely, when used in a context
%%NO-R2D2-CONV%% expecting a region, a distribution \xcd`D` should be thought of as
%%NO-R2D2-CONV%% standing for \xcd`D.region`.


\subsection{{\tt PlaceGroup}s}

A \xcd`PlaceGroup` represents an ordered set of \xcd`Place`s.
\xcd`PlaceGroup`s exist for performance and scaleability: they are more
efficient, in certain critical places, than general collections of
\xcd`Place`. \xcd`PlaceGroup` implements \xcd`Sequence[Place]`, and thus
provides familiar operations -- \xcd`pg.size()` for the number of places,
\xcd`pg.iterator()` to iterate over them, etc.  

\xcd`PlaceGroup` is an abstract class.  The concrete class
\xcd`SparsePlaceGroup` is intended for a small group of places: 
in particular, 
%~~exp~~`~~`~~ somePlace:Place ~~ ^^^Arrays1j6q
\xcd`new SparsePlaceGroup(somePlace)` is a good \xcd`PlaceGroup` containing
one place.  
%~~exp~~`~~`~~ seqPlaces: Sequence[Place] ~~ ^^^Arrays9g6f
\xcd`new SparsePlaceGroup(seqPlaces)`
constructs a sparse place group from a sorted sequence of places.

\subsection{Operations returning distributions}
\index{distribution!operations}



Let \xcd`R` be a region, \xcd`Q` 
a \xcd`PlaceGroup`, and \xcd`P` a place.

\paragraph{Unique distribution} \index{distribution!unique}
%~~exp~~`~~`~~Q:PlaceGroup ~~ ^^^ Arrays290
The distribution \xcd`Dist.makeUnique(Q)` is the unique distribution from the
region \xcd`1..k` to \xcd`Q` mapping each point \xcd`i` to \xcd`pi`.

\paragraph{Constant distributions.} \index{distribution!constant}
%~~exp~~`~~`~~R:Region, P:Place ~~ ^^^ Arrays300
The distribution \xcd`R->P` maps every point in region \xcd`R` to place \xcd`P`, as does
%~~exp~~`~~`~~R:Region, P:Place ~~ ^^^ Arrays310
\xcd`Dist.makeConstant(R,P)`. 

\paragraph{Block distributions.}\index{distribution!block}
%~~exp~~`~~`~~R:Region ~~ ^^^ Arrays320
The distribution \xcd`Dist.makeBlock(R)` distributes the elements of \xcd`R`,
in order, over all the places available to the program. 
Let $p$ equal \xcd`|R| div N` and $q$ equal \xcd`|R| mod N`,
where \xcd`N` is the size of \xcd`Q`, and 
\xcd`|R|` is the size of \xcd`R`.  The first $q$ places get
successive blocks of size $(p+1)$ and the remaining places get blocks of
size $p$.

There are other \xcd`Dist.makeBlock` methods capable of controlling the
distribution and the set of places used; see the API documentation.


%%NO-CYCLIC-DIST%%  \paragraph{Cyclic distributions.} \index{distribution!cyclic}
%%NO-CYCLIC-DIST%%  The distribution \xcd`Dist.makeCyclic(R, Q)` distributes the points in \xcd`R`
%%NO-CYCLIC-DIST%%  cyclically across places in \xcd`Q` in order.
%%NO-CYCLIC-DIST%%  
%%NO-CYCLIC-DIST%%  The distribution \xcd`Dist.makeCyclic(R)` is the same distribution as
%%NO-CYCLIC-DIST%%  \xcd`Dist.makeCyclic(R, Place.places)`. 
%%NO-CYCLIC-DIST%%  
%%NO-CYCLIC-DIST%%  Thus the distribution \xcd`Dist.makeCyclic(Place.MAX_PLACES)` provides a 1--1
%%NO-CYCLIC-DIST%%  mapping from the region \xcd`Place.MAX_PLACES` to the set of all
%%NO-CYCLIC-DIST%%  places and is the same as the distribution \xcd`Dist.makeCyclic(Place.places)`.
%%NO-CYCLIC-DIST%%  
%%NO-CYCLIC-DIST%%  \paragraph{Block cyclic distributions.}\index{distribution!block cyclic}
%%NO-CYCLIC-DIST%%  The distribution \xcd`Dist.makeBlockCyclic(R, N, Q)` distributes the elements
%%NO-CYCLIC-DIST%%  of \xcd`R` cyclically over the set of places \xcd`Q` in blocks of size
%%NO-CYCLIC-DIST%%  \xcd`N`.
%%NO-ARB-DIST%%  
%%NO-ARB-DIST%%  \paragraph{Arbitrary distributions.} \index{distribution!arbitrary}
%%NO-ARB-DIST%%  The distribution \xcd`Dist.makeArbitrary(R,Q)` arbitrarily allocates points in {\cf
%%NO-ARB-DIST%%  R} to \xcd`Q`. As above, \xcd`Dist.makeArbitrary(R)` is the same distribution as
%%NO-ARB-DIST%%  \xcd`Dist.makeArbitrary(R, Place.places)`.
%%NO-ARB-DIST%%  
%%NO-ARB-DIST%%  \oldtodo{Determine which other built-in distributions to provide.}
%%NO-ARB-DIST%%  
\paragraph{Domain Restriction.} \index{distribution!restriction!region}

If \xcd`D` is a distribution and \xcd`R` is a sub-region of {\cf
%~~exp~~`~~`~~D:Dist,R :Region{R.rank==D.rank} ~~ ^^^ Arrays330
D.region}, then \xcd`D | R` represents the restriction of \xcd`D` to
\xcd`R`---that is, the distribution that takes each point \xcd`p` in \xcd`R`
to 
%~~exp~~`~~`~~D:Dist, p:Point{p.rank==D.rank} ~~ ^^^ Arrays340
\xcd`D(p)`, 
but doesn't apply to any points but those in \xcd`R`.

\paragraph{Range Restriction.}\index{distribution!restriction!range}

If \xcd`D` is a distribution and \xcd`P` a place expression, the term
%~~exp~~`~~`~~ D:Dist, P:Place~~ ^^^ Arrays350
\xcd`D | P` 
denotes the sub-distribution of \xcd`D` defined over all the
points in the region of \xcd`D` mapped to \xcd`P`.

Note that \xcd`D | here` does not necessarily contain adjacent points
in \xcd`D.region`. For instance, if \xcd`D` is a cyclic distribution,
\xcd`D | here` will typically contain points that differ by the number of
places. 
An implementation may find a
way to still represent them in contiguous memory, \eg, using a
complex arithmetic function to map from the region index to an index
into the array.

%%NO-USER-DIST%%  \subsection{User-defined distributions}\index{distribution!user-defined}
%%NO-USER-DIST%%  
%%NO-USER-DIST%%  Future versions of \Xten{} may provide user-defined distributions, in
%%NO-USER-DIST%%  a way that supports static reasoning.

%%NO-flinking-operations-on-DIST%%  \subsection{Operations on distributions}
%%NO-flinking-operations-on-DIST%%  
%%NO-flinking-operations-on-DIST%%  A {\em sub-distribution}\index{sub-distribution} of \xcd`D` is
%%NO-flinking-operations-on-DIST%%  any distribution \xcd`E` defined on some subset of the region of
%%NO-flinking-operations-on-DIST%%  \xcd`D`, which agrees with \xcd`D` on all points in its region.
%%NO-flinking-operations-on-DIST%%  We also say that \xcd`D` is a {\em super-distribution} of
%%NO-flinking-operations-on-DIST%%  \xcd`E`. A distribution \xcdmath`D1` {\em is larger than}
%%NO-flinking-operations-on-DIST%%  \xcdmath`D2` if \xcdmath`D1` is a super-distribution of
%%NO-flinking-operations-on-DIST%%  \xcdmath`D2`.
%%NO-flinking-operations-on-DIST%%  
%%NO-flinking-operations-on-DIST%%  Let \xcdmath`D1` and \xcdmath`D2` be two distributions with the same rank.  
%%NO-flinking-operations-on-DIST%%  

%%NO-&&-DIST%%  \paragraph{Intersection of distributions.}\index{distribution!intersection}
%%NO-&&-DIST%%  ~~exp~~`~~`~~D1:Dist, D2:Dist{D1.rank==D2.rank} ~~
%%NO-&&-DIST%%  \xcdmath`D1 && D2`, the intersection 
%%NO-&&-DIST%%  of \xcdmath`D1`
%%NO-&&-DIST%%  and \xcdmath`D2`, is the largest common sub-distribution of
%%NO-&&-DIST%%  \xcdmath`D1` and \xcdmath`D2`.

%%NO-overlay-DIST%%  \paragraph{Asymmetric union of distributions.}\index{distribution!union!asymmetric}
%%NO-overlay-DIST%%  ~~exp~~`~~`~~D1:Dist, D2:Dist{D1.rank==D2.rank} ~~
%%NO-overlay-DIST%%  \xcdmath`D1.overlay(D2)`, the asymmetric union of
%%NO-overlay-DIST%%  \xcdmath`D1` and \xcdmath`D2`, is the distribution whose
%%NO-overlay-DIST%%  region is the union of the regions of \xcdmath`D1` and
%%NO-overlay-DIST%%  \xcdmath`D2`, and whose value at each point \xcd`p` in its
%%NO-overlay-DIST%%  region is \xcdmath`D2(p)` if \xcdmath`p` lies in
%%NO-overlay-DIST%%  \xcdmath`D2.region` otherwise it is \xcdmath`D1(p)`.
%%NO-overlay-DIST%%  (\xcdmath`D1` provides the defaults.)

%%NO-flinking-operations-on-DIST%%  \paragraph{Disjoint union of distributions.}\index{distribution!union!disjoint}
%%NO-flinking-operations-on-DIST%%  ~~exp~~`~~`~~D1:Dist, D2:Dist{D1.rank==D2.rank} ~~
%%NO-flinking-operations-on-DIST%%  \xcdmath`D1 || D2`, the disjoint union of
%%NO-flinking-operations-on-DIST%%  \xcdmath`D1`
%%NO-flinking-operations-on-DIST%%  and \xcdmath`D2`, is defined only if the regions of
%%NO-flinking-operations-on-DIST%%  \xcdmath`D1` and \xcdmath`D2` are disjoint. Its value is
%%NO-flinking-operations-on-DIST%%  \xcdmath`D1.overlay(D2)` (or equivalently
%%NO-flinking-operations-on-DIST%%  \xcdmath`D2.overlay(D1)`.  (It is the least
%%NO-flinking-operations-on-DIST%%  super-distribution of \xcdmath`D1` and \xcdmath`D2`.)
%%NO-flinking-operations-on-DIST%%  
%%NO-flinking-operations-on-DIST%%  \paragraph{Difference of distributions.}\index{distribution!difference}
%%NO-flinking-operations-on-DIST%%  \xcdmath`D1 - D2` is the largest sub-distribution of
%%NO-flinking-operations-on-DIST%%  \xcdmath`D1` whose region is disjoint from that of
%%NO-flinking-operations-on-DIST%%  \xcdmath`D2`.
%%NO-flinking-operations-on-DIST%%  
%%NO-flinking-operations-on-DIST%%  
%%What-Is-This-Example%% \subsection{Example}
%%What-Is-This-Example%% \begin{xten}
%%What-Is-This-Example%% def dotProduct(a: Array[T](D), b: Array[T](D)): Array[Double](D) =
%%What-Is-This-Example%%   (new Array[T]([1:D.places],
%%What-Is-This-Example%%       (Point) => (new Array[T](D | here,
%%What-Is-This-Example%%                     (i): Point) => a(i)*b(i)).sum())).sum();
%%What-Is-This-Example%% \end{xten}
%%What-Is-This-Example%% 
%%What-Is-This-Example%% This code returns the inner product of two \xcd`T` vectors defined
%%What-Is-This-Example%% over the same (otherwise unknown) distribution. The result is the sum
%%What-Is-This-Example%% reduction of an array of \xcd`T` with one element at each place in the
%%What-Is-This-Example%% range of \xcd`D`. The value of this array at each point is the sum
%%What-Is-This-Example%% reduction of the array formed by multiplying the corresponding
%%What-Is-This-Example%% elements of \xcd`a` and \xcd`b` in the local sub-array at the current
%%What-Is-This-Example%% place.
%%What-Is-This-Example%% 

\section{Distributed Arrays}
\index{array!distributed}
\index{distributed array}
\index{\Xcd{DistArray}}
\index{DistArray}

Distributed arrays, instances of \xcd`DistArray[T]`, are very much like
\xcd`Array`s, except that they distribute information among multiple
\xcd`Place`s according to a \xcd`Dist` value passed in as a constructor
argument.  For example, the following code creates a distributed array holding
a thousand cells, each initialized to 0.0, distributed via a block
distribution over all places.
%~~gen ^^^ Arrays360
% package Arrays.Distarrays.basic.example;
% class Example {
% def example() {
%~~vis
\begin{xten}
val R <: Region = 1..1000;
val D <: Dist = Dist.makeBlock(R);
val da <: DistArray[Float] = DistArray.make[Float](D, (Point(1))=>0.0f);
\end{xten}
%~~siv
%}}
%~~neg




\section{Distributed Array Construction}\label{ArrayInitializer}
\index{distributed array!creation}
\index{\Xcd{DistArray}!creation}
\index{DistArray!creation}

\xcd`DistArray`s are instantiated by invoking one of the \xcd`make` factory
methods of the \xcd`DistArray` class.
A \xcd`DistArray` creation 
must take either an \xcd`Int` as an argument or a \xcd`Dist`. In the first
case,  a distributed array is created over the distribution \xcd`[0:N-1]->here`;
in the second over the given distribution. 

A distributed array creation operation may also specify an initializer
function.
The function is applied in parallel
at all points in the domain of the distribution. The
construction operation terminates locally only when the \xcd`DistArray` has been
fully created and initialized (at all places in the range of the
distribution).

For instance:
%~~gen ^^^ Arrays370
% package Arrays.DistArray.Construction.FeralWolf;
% class Example {
% def example() {
%~~vis
\begin{xten}
val data : DistArray[Int]
    = DistArray.make[Int](1..1000->here, ([i]:Point(1)) => i);
val blocked = Dist.makeBlock((1..1000)*(1..1000));
val data2 : DistArray[Int]
    = DistArray.make[Int](blocked, ([i,j]:Point(2)) => i*j);
\end{xten}
%~~siv
% }  }
%~~neg


{}\noindent 
The first declaration stores in \xcd`data` a reference to a mutable
distributed array with \xcd`1000` elements each of which is located in the
same place as the array. The element at \Xcd{[i]} is initialized to its index
\xcd`i`. 

The second declaration stores in \xcd`data2` a reference to a mutable
two-dimensional distributed array, whose coordinates both range from 1 to
1000, distributed in blocks over all \xcd`Place`s, 
initialized with \xcd`i*j`
at point \xcd`[i,j]`.

%%WHY-THIS-EXAMPLE%% In the following 
%%WHY-THIS-EXAMPLE%% %~x~gen
%%WHY-THIS-EXAMPLE%% % package Arrays.DistArrays.FlistArrays.GlistArrays;
%%WHY-THIS-EXAMPLE%% %~x~vis
%%WHY-THIS-EXAMPLE%% \begin{xten}
%%WHY-THIS-EXAMPLE%% val D1:Dist(1) = Dist.makeBlock(1..100);
%%WHY-THIS-EXAMPLE%% val D2:Dist(2) = Dist.makeBlock((1..100)*(-1..1));
%%WHY-THIS-EXAMPLE%% val ints : Array[Int]
%%WHY-THIS-EXAMPLE%%     = Array.make[Int](1000, ((i):Point) => i*i);
%%WHY-THIS-EXAMPLE%% val floats1 : Array[Float]
%%WHY-THIS-EXAMPLE%%     = Array.make[Float](D1, ((i):Point) => i*i as Float);
%%WHY-THIS-EXAMPLE%% val floats2 : Array[Float]
%%WHY-THIS-EXAMPLE%%    = Array.make[Float](D2, ((i,j):Point) => i+j as Float);;
%%WHY-THIS-EXAMPLE%% \end{xten}
%%WHY-THIS-EXAMPLE%% %~x~siv
%%WHY-THIS-EXAMPLE%% %
%%WHY-THIS-EXAMPLE%% %~x~neg


\section{Operations on Arrays and Distributed Arrays}

Arrays and distributed arrays share many operations.
In the following, let \xcd`a` be an array with base type T, and \xcd`da` be an
array with distribution \xcd`D` and base type \xcd`T`.




\subsection{Element operations}\index{array!access}
The value of \xcd`a` at a point \xcd`p` in its region of definition is
%~~exp~~`~~`~~a:Array[Int](3), p:Point(3) ~~ ^^^ Arrays380
obtained by using the indexing operation \xcd`a(p)`. 
The value of \xcd`da` at \xcd`p` is similarly
%~~exp~~`~~`~~da:DistArray[Int](3), p:Point(3) ~~ ^^^ Arrays390
\xcd`da(p)`
This operation
may be used on the left hand side of an assignment operation to update
the value: 
%~~stmt~~`~~`~~a:Array[Int](3), p:Point(3), t:Int ~~ ^^^ Arrays400
\xcd`a(p)=t;`
and 
%~~stmt~~`~~`~~da:DistArray[Int](3), p:Point(3), t:Int ~~ ^^^ Arrays410
\xcd`da(p)=t;`
The operator assignments, \xcd`a(i) += e` and so on,  are also
available. 

It is a runtime error to use either \xcd`da(p)` or \xcd`da(p)=v` at a place
other than \xcd`da.dist(p)`, \viz{} at the place that the element exists. 

%%HUH%%  For distributed array variables, the right-hand-side of an assignment must
%%HUH%%  have the same distribution \xcd`D` as an array being assigned. This
%%HUH%%  assignment involves
%%HUH%%  control communication between the sites hosting \xcd`D`. Each
%%HUH%%  site performs the assignment(s) of array components locally. The
%%HUH%%  assignment terminates when assignment has terminated at all
%%HUH%%  sites hosting \xcd`D`.

\subsection{Constant promotion}\label{ConstantArray}
\index{array!constant promotion}

For a region \xcd`R` and a value \xcd`v` of type \xcd`T`, the expression 
%~~genexp~~`~~`~~T~~R:Region, v:T ~~ ^^^ Arrays420
\xcd`new Array[T](R, v)` 
produces an array on region \xcd`R` initialized with value \xcd`v`
Similarly, 
for a distribution \xcd`D` and a value \xcd`v` of
type \xcd`T` the expression 
%~~genexp~~`~~`~~T ~~D:Dist, v:T ~~ ^^^ Arrays430
\xcd`DistArray.make[T](D, (Point(D.rank))=>v)`
constructs a distributed array with
distribution \xcd`D` and base type \xcd`T` initialized with \xcd`v`
at every point.

Note that \xcd`Array`s are constructed by constructor calls, but
\xcd`DistArrays` are constructed by calls to the factory methods
\xcd`DistArray.make`. This is because \xcd`Array`s are fairly simple objects,
but \xcd`DistArray`s may be implemented by different classes for different
distributions. The use of the factory method gives the library writer the
freedom to select appropriate implementations.


\subsection{Restriction of an array}\index{array!restriction}

Let \xcd`R` be a sub-region of \xcd`da.region`. Then 
%~~exp~~`~~`~~da:DistArray[Int](3), p:Point(3), R: Region(da.rank) ~~ ^^^ Arrays440
\xcd`da | R`
represents the sub-\xcd`DistArray` of \xcd`da` on the region \xcd`R`.
That is, \xcd`da | R` has the same values as \xcd`da` when subscripted by a
%~~exp~~`~~`~~R:Region, da: DistArray[Int]{da.region.rank == R.rank} ~~ ^^^ Arrays450
point in region \xcd`R && da.region`, and is undefined elsewhere.
`
Recall that a rich set of operators are available on distributions
(\Sref{XtenDistributions}) to obtain sub-distributions
(e.g. restricting to a sub-region, to a specific place etc).

%%GONE-AWAY%%  \subsection{Assembling an array}
%%GONE-AWAY%%  Let \xcd`da1,da2` be distributed arrays of the same base type \xcd`T` defined over
%%GONE-AWAY%%  distributions \xcd`D1` and \xcd`D2` respectively. 
%%GONE-AWAY%%  \paragraph{Assembling arrays over disjoint regions}\index{array!union!disjoint}
%%GONE-AWAY%%  
%%GONE-AWAY%%  
%%GONE-AWAY%%  If \xcd`D1` and \xcd`D2` are disjoint then the expression 
%%GONE-AWAY%%  %~x~genexp~~`~~`~~T ~~ da1: Array[T], da2: Array[T](da1.rank)~~
%%GONE-AWAY%%  \xcd`da1 || da2` denotes the unique array of base type \xcd`T` defined over the
%%GONE-AWAY%%  distribution \xcd`D1 || D2` such that its value at point \xcd`p` is
%%GONE-AWAY%%  \xcd`a1(p)` if \xcd`p` lies in \xcd`D1` and \xcd`a2(p)`
%%GONE-AWAY%%  otherwise. This array is a reference (value) array if \xcd`a1` is.
%%GONE-AWAY%%  
%%GONE-AWAY%%  \paragraph{Overlaying an array on another}\index{array!union!asymmetric}
%%GONE-AWAY%%  The expression
%%GONE-AWAY%%  \xcd`a1.overlay(a2)` (read: the array \xcd`a1` {\em overlaid with} \xcd`a2`)
%%GONE-AWAY%%  represents an array whose underlying region is the union of that of
%%GONE-AWAY%%  \xcd`a1` and \xcd`a2` and whose distribution maps each point \xcd`p`
%%GONE-AWAY%%  in this region to \xcd`D2(p)` if that is defined and to \xcd`D1(p)`
%%GONE-AWAY%%  otherwise. The value \xcd`a1.overlay(a2)(p)` is \xcd`a2(p)` if it is defined and \xcd`a1(p)` otherwise.
%%GONE-AWAY%%  
%%GONE-AWAY%%  This array is a reference (value) array if \xcd`a1` is.
%%GONE-AWAY%%  
%%GONE-AWAY%%  The expression \xcd`a1.update(a2)` updates the array \xcd`a1` in place
%%GONE-AWAY%%  with the result of \xcd`a1.overlay(a2)`.
%%GONE-AWAY%%  
%%GONE-AWAY%%  \oldtodo{Define Flooding of arrays}
%%GONE-AWAY%%  
%%GONE-AWAY%%  \oldtodo{Wrapping an array}
%%GONE-AWAY%%  
%%GONE-AWAY%%  \oldtodo{Extending an array in a given direction.}
%%GONE-AWAY%%  
\subsection{Operations on Whole Arrays}

\paragraph{Pointwise operations}\label{ArrayPointwise}\index{array!pointwise operations}
The unary \xcd`map` operation applies a function to each element of
a distributed or non-distributed array, returning a new distributed array with
the same distribution, or a non-distributed array with the same region.
For example, the following produces an array of cubes: 
%~~gen ^^^ Arrays460
%package Arrays_pointwise_a;
%class Example{
%def example() {
%~~vis
\begin{xten}
val A = new Array[Int](1..10, (p:Point(1))=>p(0) );
// A = 1,2,3,4,5,6,7,8,9,10
val cube = (i:Int) => i*i*i;
val B = A.map(cube);
// B = 1,8,27,64,216,343,512,729,1000
\end{xten}
%~~siv
%} } 
%~~neg

A variant operation lets you specify the array \Xcd{B} into which the result
will be stored.   
%~~gen ^^^ Arrays470
%package Arrays.map_with_result;
%class Example{
%def example() {
%~~vis
\begin{xten}
val A = new Array[Int](1..10, (p:Point(1))=>p(0) );
// A = 1,2,3,4,5,6,7,8,9,10
val cube = (i:Int) => i*i*i;
val B = new Array[Int](A.region); // B = 0,0,0,0,0,0,0,0,0,0
A.map(B, cube);
// B = 1,8,27,64,216,343,512,729,1000
\end{xten}
%~~siv
%} } 
%~~neg
\noindent
This is convenient if you have an already-allocated array lying around unused.
In particular, it can be used if you don't need \Xcd{A} afterwards and want to
reuse its space:
%~~gen ^^^ Arrays480
%package Arrays.map_reusing_space;
%class Example{
%def example() {
%~~vis
\begin{xten}
val A = new Array[Int](1..10, (p:Point(1))=>p(0) );
// A = 1,2,3,4,5,6,7,8,9,10
val cube = (i:Int) => i*i*i;
A.map(A, cube);
// A = 1,8,27,64,216,343,512,729,1000
\end{xten}
%~~siv
%} } 
%~~neg


The binary \xcd`map` operation takes a binary function and
another
array over the same region or distributed array over the same  distribution,
and applies the function 
pointwise to corresponding elements of the two arrays, returning
a new array or distributed array of the same shape.
The following code adds two distributed arrays: 
%~~gen ^^^ Arrays490
% package Arrays.Pointwise.Pointless.Map2;
% class Example{
%~~vis
\begin{xten}
static def add(da:DistArray[Int], db: DistArray[Int]{da.dist==db.dist})
    = da.map(db, Int.+);
\end{xten}
%~~siv
%}
%~~neg



\paragraph{Reductions}\label{ArrayReductions}\index{array!reductions}

Let \xcd`f` be a function of type \xcd`(T,T)=>T`.  Let
\xcd`a` be an array over base type \xcd`T`.
Let \xcd`unit` be a value of type \xcd`T`.
Then the
%~~genexp~~`~~`~~ T ~~ f:(T,T)=>T, a : Array[T], unit:T ~~ ^^^ Arrays500
operation \xcd`a.reduce(f, unit)` returns a value of type \xcd`T` obtained
by combining all the elements of \xcd`a` by use of  \xcd`f` in some unspecified order
(perhaps in parallel).   
The following code gives one method which 
meets the definition of \Xcd{reduce},
having a running total \Xcd{r}, and accumulating each value \xcd`a(p)` into it
using \Xcd{f} in turn.  (This code is simply given as an example; \Xcd{Array}
has this operation defined already.)
%~~gen ^^^ Arrays510
%package Arrays.Reductions.And.Eliminations;
% class Example {
%~~vis
\begin{xten}
def oneWayToReduce[T](a:Array[T], f:(T,T)=>T, unit:T):T {
  var r : T = unit;
  for(p in a.region) r = f(r, a(p));
  return r;
}
\end{xten}
%~~siv
%}
%~~neg


For example,  the following sums an array of integers.  \Xcd{f} is addition,
and \Xcd{unit} is zero.  
%~~gen ^^^ Arrays520
% package Arrays.Reductions.And.Emulsions;
% class Example {
% def example() {
%~~vis
\begin{xten}
val a = [1,2,3,4];
val sum = a.reduce(Int.+, 0); 
assert(sum == 10); // 10 == 1+2+3+4
\end{xten}
%~~siv
%}}
%~~neg

Other orders of evaluation, degrees of parallelism, and applications of
\Xcd{f(x,unit)} and \xcd`f(unit,x)`are also correct.
In order to guarantee that the result is precisely
determined, the  function \xcd`f` should be associative and
commutative, and the value \xcd`unit` should satisfy
\xcd`f(unit,x)` \xcd`==` \xcd`x` \xcd`==` \xcd`f(x,unit)`
for all \Xcd{x:T}.  




\xcd`DistArray`s have the same operation.
This operation involves communication between the places over which
the \xcd`DistArray` is distributed. The \Xten{} implementation guarantees that
only one value of type \xcd`T` is communicated from a place as part of
this reduction process.

\paragraph{Scans}\label{ArrayScans}\index{array!scans}


Let \xcd`f:(T,T)=>T`, \xcd`unit:T`, and \xcd`a` be an \xcd`Array[T]` or
\xcd`DistArray[T]`.  Then \xcd`a.scan(f,unit)` is the array or distributed
array of type \xcd`T` whose {$i$}th element in canonical order is the
reduction by \xcd`f` with unit \xcd`unit` of the first {$i$} elements of
\xcd`a`. 


This operation involves communication between the places over which the
distributed array is distributed. The \Xten{} implementation will endeavour to
minimize the communication between places to implement this operation.

Other operations on arrays, distributed arrays, and the related classes may be
found in the \xcd`x10.array` package.
