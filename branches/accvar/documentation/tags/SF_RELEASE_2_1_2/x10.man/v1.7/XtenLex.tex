\chapter{Lexical structure}

In general, \Xten{} follows \java{} rules \cite[Chapter 3]{jls2} for
lexical structure.

Lexically a program consists of a stream of white space, comments,
identifiers, keywords, literals, separators and operators.

\paragraph{Whitespace}
% Whitespace \index{whitespace} follows \java{} rules \cite[Chapter 3.6]{jls2}.
ASCII space, horizontal tab (HT), form feed (FF) and line
terminators constitute white space.

\paragraph{Comments}
% Comments \index{comments} follows \java{} rules
% \cite[Chapter 3.7]{jls2}. 
All text included within the ASCII characters ``\xcd"/*"'' and
``\xcd"*/"'' is
considered a comment and ignored; nested comments are not
allowed.  All text from the ASCII characters
``\xcd"//"'' to the end of line is considered a comment and is ignored.

\paragraph{Identifiers}

Identifiers \index{identifier} are defined as in \java.
Identifiers consist of a single letter followed by zero or more
letters or digits.
Letters are defined as the characters for which the \java{}
method \xcd"Character.isJavaIdentifierStart" returns true.
Digits are defined as the ASCII characters \xcd"0" through \xcd"9".

\paragraph{Keywords}
\Xten{} reserves the following keywords:
\begin{xten}
abstract        any             as              async
at              ateach          atomic          await
break           case            catch           class
clocked         const           continue        current
def             default         do              else
extends         extern          final           finally
finish          for             foreach         future
goto            has             here            if
implements      import          in              instanceof
interface       local           native          new
next            nonblocking     or              package
private         protected       property        public
return          safe            self            shared
static          super           switch          this
throw           throws          try             type
val             value           var             when
while
\end{xten}
Note that the primitive types are not considered keywords.
The keyword \xcd{goto} is reserved, but not used.

\paragraph{Literals}\label{Literals}\index{literals}

Literals are either integers, floating point numbers, booleans,
characters, strings, and \xcd"null".
\XtenCurrVer{} defines literal syntax in the same way as \java{} does.
Unsigned 32-bit integers are suffixed with
\xcd{U} or \xcd{u}.
Signed 64-bit integers are suffixed with
\xcd{L} or \xcd{l}.
Unsigned 64-bit integers are suffixed with
any of \xcd{LU}, \xcd{Lu}, \xcd{UL}, \xcd{Ul},
\xcd{lU}, \xcd{lu}, \xcd{uL}, or \xcd{ul}.

\paragraph{Separators}
\Xten{} has the following separators and delimiters:
\begin{xten}
( )  { }  [ ]  ;  ,  .
\end{xten}

\paragraph{Operators}
\Xten{} has the following operators:
\begin{xten}
==  !=  <   >   <=  >=
&&  ||  &   |   ^
<<  >>  >>>
+   -   *   /   %
++  --  !   ~
&=  |=  ^=
<<= >>= >>>
+=  -=  *=  /=  %=
=   ?   :   =>  ->
<:  :>  @   ..
\end{xten}




