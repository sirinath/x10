\chapter{Expressions}\label{XtenExpressions}\index{expressions}

\Xten{} has a rich expression language.
Evaluating an expression produces a value, or, in a few cases, no value. 
Expression evaluation may have side effects, such as change of the value of a 
\xcd`var` variable or a data structure, allocation of new values, or throwing
an exception. 

Evaluation is performed left to right, wherever possible.  For example, in 
%~~exp~~`~~`~~f:() => ((Int,Int)=>Int), a:()=>Int, b:()=>Int ~~
\xcd`f()(a(),b())`, \xcd`f()` is evaluated, then \xcd`a()`, then \xcd`b()`,
andd then the application.  

\section{Literals}

Literals denote fixed values of built-in types. 
The syntax for literals is given in \Sref{Literals}. 

The type that \Xten{} gives a literal often includes its value. \Eg, \xcd`1`
is of type \xcd`Int{self==1}`, and \xcd`true` is of type
\xcd`Boolean{self==true}`.

\section{\Xcd{this}}

\begin{grammar}
ThisExpression \: \xcd"this" \\
\| ClassName \xcd"." \xcd"this" \\
\end{grammar}

The expression \xcd"this" is a  local \xcd`val` containing a reference
to an instance of the lexically enclosing class.
It may be used only within the body of an instance method, a
constructor, or in the initializer of a instance field -- that is, the places
where there is an instance of the class under consideration.

Within an inner class, \xcd"this" may be qualified with the
name of a lexically enclosing class.  In this case, it
represents an instance of that enclosing class.  
For example, \xcd`Outer` is a class containing \xcd`Inner`.  Each instance of
\xcd`Inner` has a reference \xcd`outer` to the \xcd`Outer` involved in its
creation, which is acquired by use of \xcd`Outer.this`.
%~~gen
% package exp.vexp.pexp.lexp.shexp;
%~~vis
\begin{xten}
class Outer {
  val inner : Inner! = new Inner();
  class Inner {
    val outer : Outer = Outer.this;
  }
  def alwaysTrue() = (this == inner.outer);
}
\end{xten}
%~~siv
%
%~~neg

The type of a \xcd"this" expression is the
innermost enclosing class, or the qualifying class,
constrained by the class invariant and the
method guard, if any.

The \xcd"this" expression may also be used within constraints in
a class or interface header (the class invariant and
\xcd"extends" and \xcd"implements" clauses).  Here, the type of
\xcd"this" is restricted so that only properties declared in the
class header itself, and specifically not any members declared in the class
body or in supertypes, are accessible through \xcd"this".

\section{Local variables}

\begin{grammar}
LocalExpression \: Identifier \\
\end{grammar}

A local variable expression consists simply of the name of the local variable,
field of the current object, formal parameter in scope, etc. It evaluates to
the value of the local variable. \xcd`n` in the second line below is a local
variable expression. 
%~~gen 
% package exp.loc.al.varia.ble; 
% class Example {
% def example() { 
%~~vis
\begin{xten}
val n = 22;
val m = n + 56;
\end{xten}
%~~siv
%} }
%~~neg



\section{Field access}
\label{FieldAccess}


\begin{grammar}
FieldExpression \: Expression \xcd"." Identifier \\
                \| \xcd"super" \xcd"." Identifier \\
                \| ClassName \xcd"." Identifier \\
                \| ClassName \xcd"." \xcd"super" \xcd"." Identifier \\
\end{grammar}

A field of an object instance may be  accessed
with a field access expression.

The type of the access is the declared type of the field with the
actual target substituted for \xcd"this" in the type. 
% If the actual
%target is not a final access path (\Sref{FinalAccessPath}),
%an anonymous path is substituted for \xcd"this".

The field accessed is selected from the fields and value properties
of the static type of the target and its superclasses.

If the field target is given by the keyword \xcd"super", the target's type is
the superclass of the enclosing class.  This form is used to access fields of
the parent class shadowed by same-named fields of the current class.

If the field target is \xcd`Cls.super`, then the target's type is \xcd`Cls`,
which must be an ancestor class of the enclosing class.  This (admittedly
obscure) form is used to access fields of an ancestor class which are shadowed
by same-named fields of some more recent ancestor.  The following example
illustrates all four cases:

%~~gen
% package exp.re.ssio.ns.fiel.dacc.ess;
%~~vis
\begin{xten}
class Uncle {
  public static global val f = 1;
}
class Parent {
  public global val f = 2;
}
class Ego extends Parent {
  public global val f = 3;
  class Child extends Ego {
     public global val f = 4;
     def expDotId() = this.f; // 4
     def superDotId() = super.f; // 3
     def classNameDotId() = Uncle.f; // 1;
     def cnDotSuperDotId() = Ego.super.f; // 2
  }
}
\end{xten}
%~~siv
%
%~~neg


If the field target is \xcd"null", a \xcd"NullPointerException"
is thrown.

If the field target is a class name, a static field is selected.

It is illegal to access  a field that is not visible from
the current context.
It is illegal to access a non-static field
through a static field access expression.

\section{Function Literals}
Function literals are described in \Sref{Functions}.

\section{Calls}
\label{Call}
\label{MethodInvocation}
\label{MethodInvocationSubstitution}

\begin{grammar}
MethodCall \: TypeName \xcd"." Identifier TypeArguments\opt \xcd"(" ArgumentList\opt \xcd")" \\
           \| \xcd"super" \xcd"." Identifier TypeArguments\opt \xcd"(" ArgumentList\opt \xcd")" \\
           \| ClassName \xcd"." \xcd"super" \xcd"." Identifier TypeArguments\opt \xcd"(" ArgumentList\opt \xcd")" \\
Call \: Primary TypeArguments\opt \xcd"(" ArgumentList\opt \xcd")" \\
TypeArguments \: \xcd"[" Type ( \xcd"," Type )\star \xcd"]" \\
\end{grammar}

A \grammarrule{MethodCall} may be to either a \xcd"static" or an 
instance method.  A \grammarrule{Call} may invoke either a method
or a closure.  

The syntax is ambiguous; the target must be type-checked to determine if it is
the name of a method or if it refers to a field containing a closure. It is a
static error if a call may resolve to both a closure call or to a method call.
%~~gen
% package expres.sio.nsca.lls;
%~~vis
\begin{xten}
class Callsome {
  static val closure = () => 1;
  static def method () = 2;
  static val closureEvaluated = Callsome.closure();
  static val methodEvaluated = Callsome.method();
}
\end{xten}
%~~siv
%
%~~neg
However, adding a static method called \xcd`closure` makes \xcd`Callsome.closure()`
ambiguous: it could be a call to the closure, or to the static method: 

%~~gen
% package expres.sio.nsca.lls.twoooo;
% class Callsome {static val closure = () => 1; static def method () = 2; static val methodEvaluated = Callsome.method();
%~~vis
\begin{xten}
  static def closure () = 3;
  // ERROR: static errory = Callsome.closure();
\end{xten}
%~~siv
% }
%~~neg

A closure call \xcdmath"e($\dots$)" is shorthand for a method call
\xcdmath"e.apply($\dots$)". 

Method selection rules are similar to that of \java{}.\bard{Explain}
For a call with no explicit type arguments, a method with 
no parameters is considered more specific than a method with one or more
type parameters that would have to be inferred.

It is a static error if a method's \grammarrule{Guard} is not satisfied by the
caller. 

\section{Assignment}\index{assignment}\label{AssignmentStatement}

\begin{grammar}
Expression \: Assignment \\
Assignment \: SimpleAssignment \\
           \| OpAssignment \\
SimpleAssignment \: LeftHandSide \xcd"=" Expression \\
OpAssignment \: LeftHandSide \xcd"+=" Expression \\
             \: LeftHandSide \xcd"-=" Expression \\
             \: LeftHandSide \xcd"*=" Expression \\
             \: LeftHandSide \xcd"/=" Expression \\
             \: LeftHandSide \xcd"%=" Expression \\
             \: LeftHandSide \xcd"&=" Expression \\
             \: LeftHandSide \xcd"|=" Expression \\
             \: LeftHandSide \xcd"^=" Expression \\
             \: LeftHandSide \xcd"<<=" Expression \\
             \: LeftHandSide \xcd">>=" Expression \\
             \: LeftHandSide \xcd">>>=" Expression \\
LeftHandSide \: Identifier \\
             \| Primary \xcd"." Identifier \\
             \| Primary \xcd"(" Expression \xcd")" \\
\end{grammar}



The assignment expression \xcd"x = e" assigns a value given by
expression \xcd"e"
to a variable \xcd"x".  
Most often, \xcd`x` is a mutable (\xcd`var` variable).  The same syntax is
used for delayed initialization of a \xcd`val`, but \xcd`val`s can only be
initialized once.
%~~gen
% package express.ions.ass.ignment;
% class Example {
% static def exasmple() {
%~~vis
\begin{xten}
  var x : Int;
  val y : Int;
  x = 1;
  y = 2; // Correct; initializes y
  x = 3; 
  // Incorrect: y = 4;
\end{xten}
%~~siv
% } } 
%~~neg


There are three syntactic forms of
assignment: 
\begin{enumerate}
\item \xcd`x = e;`, assigning to a local variable, formal parameter, field of
      \xcd`this`, etc. 
\item \xcd`x.f = e;`, assigning to a field of an object.
\item \xcdmath`a(i$_1$,$\ldots$,i$_n$) = v;`, where {$n \ge 0$}, assigning to
      an element of an array or some other such structure. This is syntactic
      sugar for a method call: \xcdmath`a.set(v,i$_1$,$\ldots$,i$_n$)`.
      Naturally, it is a static error if no suitable \xcd`set` method exists
      for \xcd`a`.
\end{enumerate}

For a binary operator $\diamond$, the $\diamond$-assignment expression
\xcdmath"x $\diamond$= e" combines the current value of \xcd`x` with the value
of \xcd`e` by {$\diamond$}, and stores the result back into \xcd`x`.  
\xcd`i += 2`, for example, adds 2 to \xcd`i`. For variables and fields, 
\xcdmath"x $\diamond$= e" behaves just like \xcdmath"x = x $\diamond$ e". 

The subscripting forms of \xcdmath"a(i) $\diamond$= b" are slightly subtle.
Subexpressions of \xcd`a` and \xcd`i` are only evaluated once.  However,
\xcd`a(i)` and \xcd`a(i)=c` are each executed once---in particular, there is
one call to \xcd`a.apply(i)` and one to \xcd`a.set(i,c)`, the desugared forms
of \xcd`a(i)` and \xcd`a(i)=c`.  If subscripting is implemented strangely for
the class of \xcd`a`, the behavior is {\em not} necessarily updating a single
storage location. Specifically, \xcd`A()(I()) += B()` is tantamount to: 
%~~gen
% package expressions.stupid.addab;
% class Example {
% def example(A:()=>Rail[Int]!, I: () => Int, B: () => Int ) {
%~~vis
\begin{xten}
{
  val aa = A();  // Evaluate A() once
  val ii = I();  // Evaluate I() once
  val bb = B();  // Evaluate B() once
  val tmp = aa(ii) + bb; // read aa(ii)
  aa(ii) = tmp;  // write sum back to aa(ii)
}
\end{xten}
%~~siv
%}}
%~~neg

\limitation{+= does not currently meet this specification.}




\section{Increment and decrement}


The operators \xcd"++" and \xcd"--" increment and decrement
a variable, respectively.  
\xcd`x++` and \xcd`++x` both increment \xcd`x`, just as the statement 
\xcd`x += 1` would, and similarly for \xcd`--`.  

The difference between the two is the return value.  
\xcd`++x` returns the {\em new} value of \xcd`x`, after incrementing.
\xcd`x++` returns the {\em old} value of \xcd`, before incrementing.`

\limitation{This currently only works for numeric types.}

\section{Numeric Operations}
\label{XtenPromotions}
\index{promotion}
\index{numeric promotion}

Numeric types (\xcd`Byte`, \xcd`Short`, \xcd`Int`, \xcd`Long`, \xcd`Float`,
\xcd`Double`, and unsigned variants of fixed-point types) are normal X10
structs, though most of their methods are implemented via native code. They
obey the same general rules as other X10 structs. For example, numeric
operations are defined by \xcd`operator` definitions, the same way you could
for any struct.

Promoting a numeric value to a longer numeric type preserves the sign of the
value.  For example, \xcd`(255 as UByte) as UInt` is 255.  

\subsection{Conversions and coercions}

Specifically, each numeric type can be converted or coerced into each other
numeric type, perhaps with loss of accuracy.
%~~gen
% package exp.ress.io.ns.numeric.conversions;
% class ExampleOfConversionAndStuff {
% def example() {
%~~vis
\begin{xten}
val f : (Int)=>Boolean = (Int) => true; 
val n : Byte = 123 as Byte; // explicit 
val ok = f(n); // implicit
\end{xten}
%~~siv
% } }
%~~neg



\subsection{Unary plus and unary minus}

The unary \xcd`+` operation on numbers is an identity function.
The unary \xcd`-` operation on numbers is a negation function.
On unsigned numbers, these are two's-complement.  For example, 
\xcd`-(0x0F as UByte)` is 
\xcd`(0xF1 as UByte)`.
\bard{UInts and such are closed under negation -- the negative of a UInt is
done binarily.  }



\section{Bitwise complement}

The unary \xcd"~" operator, only defined on integral types, complements each
bit in its operand.  

\section{Binary arithmetic operations} 

The binary arithmetic operators perform the familiar binary arithmetic
operations: \xcd`+` adds, \xcd`-` subtracts, \xcd`*` multiplies, 
\xcd`/` divides, and \xcd`%`
computes remainder.

On integers, the operands are coerced to the longer of their two types, and
then operated upon.  
Floating point operations are determined by the IEEE 754
standard. 
The integer \xcd"/" and \xcd"%" throw a \xcd"DivideByZeroException"
if the right operand is zero.

\section{Binary shift operations}

The operands of the binary shift operations must be of integral type.
The type of the result is the type of the left operand.

If the promoted type of the left operand is \xcd"Int",
the right operand is masked with \xcd"0x1f" using the bitwise
AND (\xcd"&") operator, giving a number {$\le$} the number of bits in an
\xcd`Int`. 
If the promoted type of the left operand is \xcd"Long",
the right operand is masked with \xcd"0x3f" using the bitwise
AND (\xcd"&") operator, giving a number {$\le$} the number of bits in a
\xcd`Long`. 

The \xcd"<<" operator left-shifts the left operand by the number of
bits given by the right operand.
The \xcd">>" operator right-shifts the left operand by the number of
bits given by the right operand.  The result is sign extended;
that is, if the right operand is $k$,
the most significant $k$ bits of the result are set to the most
significant bit of the operand.

The \xcd">>>" operator right-shifts the left operand by the number of
bits given by the right operand.  The result is not sign extended;
that is, if the right operand is $k$,
the most significant $k$ bits of the result are set to \xcd"0".
This operation is deprecated, and may be removed in a later version of the
language. 

\section{Binary bitwise operations}

The binary bitwise operations operate on integral types, which are promoted to
the longer of the two types.
The \xcd"&" operator  performs the bitwise AND of the promoted operands.
The \xcd"|" operator  performs the bitwise inclusive OR of the promoted operands.
The \xcd"^" operator  performs the bitwise exclusive OR of the promoted operands.

\section{String concatenation}

The \xcd"+"  operator is used for string concatenation 
 as well as addition.
If either operand is of static type \xcd"x10.lang.String",
 the other operand is converted to a \xcd"String" , if needed,
  and  the two strings  are concatenated.
 String conversion of a non-\xcd"null" value is  performed by invoking the
 \xcd"toString()" method of the value.
  If the value is \xcd"null", the value is converted to 
  \xcd'"null"'.

The type of the result is \xcd"String".

 For example, 
%~~exp~~`~~`~~ ~~
      \xcd`"one " + 2 + here` 
      evaluates to something like \xcd`one 2(Place 0)`.  

\section{Logical negation}

The operand of the  unary \xcd"!" operator 
must be of type \xcd"x10.lang.Boolean".
The type of the result is \xcd"Boolean".
If the value of the operand is \xcd"true", the result is \xcd"false"; if
if the value of the operand  is \xcd"false", the result is \xcd"true".

\section{Boolean logical operations}

Operands of the binary boolean logical operators must be of type \xcd"Boolean".
The type of the result is \xcd"Boolean"

The \xcd"&" operator  evaluates to \xcd"true" if both of its
operands evaluate to \xcd"true"; otherwise, the operator
evaluates to \xcd"false".

The \xcd"|" operator  evaluates to \xcd"false" if both of its
operands evaluate to \xcd"false"; otherwise, the operator
evaluates to \xcd"true".

\section{Boolean conditional operations}

Operands of the binary boolean conditional operators must be of type
\xcd"Boolean". 
The type of the result is \xcd"Boolean"

The \xcd"&&" operator  evaluates to \xcd"true" if both of its
operands evaluate to \xcd"true"; otherwise, the operator
evaluates to \xcd"false".
Unlike the logical operator \xcd"&",
if the first operand is \xcd"false",
the second operand is not evaluated.

The \xcd"||" operator  evaluates to \xcd"false" if both of its
operands evaluate to \xcd"false"; otherwise, the operator
evaluates to \xcd"true".
Unlike the logical operator \xcd"||",
if the first operand is \xcd"true",
the second operand is not evaluated.

\section{Relational operations} 

The relational operations compare numbers, producing \xcd`Boolean` results.  

The \xcd"<" operator evaluates to \xcd"true" if the left operand is
less than the right.
The \xcd"<=" operator evaluates to \xcd"true" if the left operand is
less than or equal to the right.
The \xcd">" operator evaluates to \xcd"true" if the left operand is
greater than the right.
The \xcd">=" operator evaluates to \xcd"true" if the left operand is
greater than or equal to the right.

Floating point comparison is determined by the IEEE 754
standard.  Thus,
if either operand is NaN, the result is \xcd"false".
Negative zero and positive zero are considered to be equal.
All finite values are less than positive infinity and greater
than negative infinity.

\section{Conditional expressions}
\label{Conditional}

\begin{grammar}
ConditionalExpression \: Expression
                \xcd"?" Expression
                \xcd":" Expression 
\end{grammar}

A conditional expression evaluates its first subexpression (the
condition); if \xcd"true"
the second subexpression (the consequent) is evaluated; otherwise,
the third subexpression (the alternative) is evaluated.

The type of the condition must be \xcd"Boolean".
The type of the conditional expression is some common 
ancestor (as constrained by \Sref{LCA}) of the types of the consequent and the
alternative. 

For example, 
%~~exp~~`~~`~~a:Int,b:Int ~~
\xcd`a == b ? 1 : 2`
evaluates to \xcd`1` if \xcd`a` and \xcd`b` are the same, and \xcd`2` if they
are different.   As the type of \xcd`1` is \xcd`Int{self==1}` and of \xcd`2`
is \xcd`Int{self==2}`, the type of the conditional expression has the form
\xcd`Int{c}`, where \xcd`self==1` and \xcd`self==2` both imply \xcd`c`.  For
example, it might be \xcd`Int{true}` -- or perhaps it might be 
\xcd`Int{self != 8}`. Note that this term has no most accurate type in the X10
type system.

\section{Stable equality}
\label{StableEquality}
\index{==}\index{!=}

\begin{grammar}
EqualityExpression \: Expression \xcd"==" Expression \\
\| Expression \xcd"!=" Expression \\
\end{grammar}

The \xcd"==" and \xcd"!=" operators provide a fundamental, though
non-abstract, notion of equality.  \xcd`a==b` is true if the values of \xcd`a`
and \xcd`b` are extremely identical, in a sense defined shortly.  \xcd`a != b`
is true iff \xcd`a==b` is false.

\begin{itemize}
\item If \xcd`a` and \xcd`b` are values of object type, then \xcd`a==b` holds
      if \xcd`a` and \xcd`b` are the same object.
\item If one operand is \xcd`null`, then \xcd`a==b` holds iff the other is
      also \xcd`null`.
\item If the operands both have struct type, then they must be structurally equal;
that is, they must be instances of the same struct
and all their fields or components must be \xcd"==". 
\item The definition of equality for function types is specified in
      \Sref{FunctionEquality}.
\item If the operands have numeric types, they are coerced into the larger of
      the two types and then compared for numeric equality.
\end{itemize}

The predicates \xcd"==" and \xcd"!=" may not be overridden by the programmer.
Note that \xcd`a==b` is a form of \emph{stable equality}; that is, the result of
the equality operation is not affected by the mutable state of the program,
after evaluation of \xcd`a` and \xcd`b`. 


\section{Allocation}
\label{ClassCreation}

\begin{grammar}
NewExpression \: \xcd"new" ClassName TypeArguments\opt \xcd"(" ArgumentList\opt \xcd")"
        ClassBody\opt \\
  \| \xcd"new" InterfaceName TypeArguments\opt \xcd"(" ArgumentList\opt \xcd")"
        ClassBody
\end{grammar}

An allocation expression creates a new instance of a class and
invokes a constructor of the class.
The expression designates the class name and passes
type and value arguments to the constructor.

The allocation expression may have an optional class body.
In this case, an anonymous subclass of the given class is
allocated.   An anonymous class allocation may also specify a
single super-interface rather than a superclass; the superclass
of the anonymous class is \xcd"x10.lang.Object".

If the class is anonymous---that is, if a class body is
provided---then the constructor is selected from the superclass.
The constructor to invoke is selected using the same rules as
for method invocation (\Sref{MethodInvocation}).

The type of an allocation expression
is the return type of the constructor invoked, with appropriate
substitutions  of actual arguments for formal parameters, as
specified in \Sref{MethodInvocationSubstitution}.

It is illegal to allocate an instance of an \xcd"abstract" class.
It is illegal to allocate an instance of a class or to invoke a
constructor that is not visible at
the allocation expression.

Note that instantiating a struct type uses function application syntax, not
\xcd`new`.  As structs do not have subclassing, there is no need or
possibility of a {\em ClassBody}.


\section{Casts}\label{ClassCast}\index{classcast}

The cast operation may be used to cast an expression to a given type:

\begin{grammar}
UnaryExpression \: CastExpression \\
CastExpression \: UnaryExpression \xcd"as" Type \\
\end{grammar}

The result of this operation is a value of the given type if the cast
is permissible at run time, and either a compile-time error or a runtime
exception 
(\xcd`x10.lang.TypeCastException`) if it is not.  

When evaluating \xcd`E as T{c}`, first the value of \xcd`E` is converted to
type \xcd`T` (which may fail), and then the constraint \xcd`{c}` is checked. 



\begin{itemize}
\item If \xcd`T` is a primitive type, then \xcd`E`'s value is converted to type
      \xcd`T` according to the rules of
      \Sref{sec:effects-of-explicit-numeric-coercions}. 
      
\item If \xcd`T` is a class, then the first half of the cast succeeds if the
      run-time value of \xcd`E` is an instance of class \xcd`T`, or of a
      subclass 

\item If \xcd`T` is an interface, then the first half of the cast succeeds if
      the run-time value of \xcd`E` is an instance of a class implementing
      \xcd`T`. 

\item If \xcd`T` is a struct type, then the first half of the cast succeeds if
      the run-time value of \xcd`E` is an instance of \xcd`T`.  

\item If \xcd`T` is a function type, then the first half of the cast succeeds
      if the run-time value of \xcd`X` is a function of precisely that type.
\end{itemize}

If the first half of the cast succeeds, the second half -- the constraint
\xcd`{c}` -- must be checked.  In general this will be done at runtime, though
in special cases it can be checked at compile time.   For example, 
\xcd`n as Int{self != w}` succeeds if \xcd`n != w` --- even if \xcd`w` is a value
read from input, and thus not determined at compile time.

The compiler may forbid casts that it knows cannot possibly work. If there is
no way for the value of \xcd`E` to be of type \xcd`T{c}`, then 
\xcd`E as T{c}` can result in a static error, rather than a runtime error.  
For example, \xcd`1 as Int{self==2}` may fail to compile, because the compiler
knows that \xcd`1`, which has type \xcd`Int{self==1}`, cannot possibly be of
type \xcd`Int{self==2}`. 


%BB% \bard{This section need serious whomping.  The Java mention needs to go.  The
%BB% rules for coercions are given in \Sref{sec:effects-of-explicit-numeric-coercions}.
%BB% If the \xcd`Type` has a constraint, the constraint will be checked at runtime. 
%BB% We need to give examples. 
%BB% }
%BB% 
%BB% Type conversion is checked according to the
%BB% rules of the \java{} language (e.g., \cite[\S 5.5]{jls2}).
%BB% For constrained types, both the base
%BB% type and the constraint are checked.
%BB% If the
%BB% value cannot be cast to the appropriate type, a
%BB% \xcd"ClassCastException"
%BB% is thrown. 



% {\bf Conversions of numeric values}
% {\bf Can't do (a as T) if a can't be a T.}


%If the value cannot be cast to the
%appropriate place type a \xcd"BadPlaceException" is thrown. 

% Any attempt to cast an expression of a reference type to a value type
% (or vice versa) results in a compile-time error. Some casts---such as
% those that seek to cast a value of a subtype to a supertype---are
% known to succeed at compile-time. Such casts should not cause extra
% computational overhead at run time.

\section{\Xcd{instanceof}}
\label{instanceOf}
\index{\Xcd{instanceof}}

\Xten{} permits types to be used in an in instanceof expression
to determine whether an object is an instance of the given type:

\begin{grammar}
RelationalExpression \: RelationalExpression \xcd"instanceof" Type
\end{grammar}

In the above expression, \grammarrule{Type} is any type. At run time, the
result of this operator is \xcd"true" if the
\grammarrule{RelationalExpression} can be coerced to \grammarrule{Type}
without a \xcd"TypeCastException" being thrown or static error occurring.
Otherwise the result is \xcd"false". This determination may involve checking
that the constraint, if any, associated with the type is true for the given
expression.

%~~exp~~`~~`~~x:Int~~
For example, \xcd`3 instanceof Int{self==x}` is an overly-complicated way of
saying \xcd`3==x`.

\section{Subtyping expressions}

\begin{grammar}
SubtypingExpression \: Expression \xcd"<:" Expression \\
                    \| Expression \xcd":>" Expression \\
                    \| Expression \xcd"==" Expression \\
\end{grammar}

The subtyping expression \xcdmath"T$_1$ <: T$_2$" evaluates to \xcd"true"
\xcdmath"T$_1$" is a subtype of \xcdmath"T$_2$".

The expression \xcdmath"T$_1$ :> T$_2$" evaluates to \xcd"true"
\xcdmath"T$_2$" is a subtype of \xcdmath"T$_1$".

The expression \xcdmath"T$_1$ == T$_2$"
evaluates to  \xcd"true" \xcdmath"T$_1$" is a subtype of \xcdmath"T$_2$" and
if \xcdmath"T$_2$" is a subtype of \xcdmath"T$_1$".

Subtyping expressions are particularly useful in giving constraints on generic
types.  \xcd`x10.util.Ordered[T]` is an interface whose values can be compared
with values of type \xcd`T`. 
In particular, \xcd`T <: x10.util.Ordered[T]` is
true if values of type \xcd`T` can be compared to other values of type
\xcd`T`.  So, if we wish to define a generic class \xcd`OrderedList[T]`, of
lists whose elements are kept in the right order, we need the elements to be
ordered.  This is phrased as a constraint on \xcd`T`: 
%~~gen
% package expre.ssi.onsfgua.rde.dq.uantification;
%~~vis
\begin{xten}
class OrderedList[T]{T <: x10.util.Ordered[T]} {
  // ...
}
\end{xten}
%~~siv
%
%~~neg




\section{Contains expressions}

\begin{grammar}
ContainsExpression \: Expression \xcd"in" Expression \\
\end{grammar}

The expression \xcd"p in r" tests if a value \xcd"p" is in a collection
\xcd"r"; it evaluates to \xcd"r.contains(p)".
The collection \xcd"r"
must be of type \xcd"Collection[T]" and the value \xcd"p" must
be of type \xcd"T".

\section{Rail constructors}
\label{RailConstructors}

\begin{grammar}
RailConstructor \: \xcd"[" Expressions \xcd"]" \\
Expressions \: Expression ( \xcd"," Expression )\star \\
\end{grammar}

The rail constructor \xcdmath"[a$_0$, $\dots$, a$_{k-1}$]"
creates an instance of \xcd"ValRail" with length $k$, 
whose $i$th element is
\xcdmath"a$_i$".  The element type of the rail is a common ancestor of the
types of the \xcdmath"a$_i$"'s, as per \Sref{LCA}.
%~~gen
% package ex.pre.ssio.nsandrailconstructors;
% class Example {
% def example() {
%~~vis
\begin{xten}
val a : ValRail[Int] = [1,3,5];
val b : ValRail[Any] = [1, a, "please"];
\end{xten}
%~~siv
% } } 
%~~neg

Since arrays are subtypes of \xcd"(Point) => T",
rail constructors can be passed into the \xcd"Array" and
\xcd"ValArray" constructors as initializer functions.

Rail constructors of type \xcd"ValRail[Int]" and length \xcd"n" 
may be implicitly converted to type \xcd"Point{rank==n}".
Rail constructors of type \xcd"ValRail[Region]" and length \xcd"n" 
may be implicitly converted to type \xcd"Region{rank==n}".

%~~gen
% package ex.pre.ssio.nsandrailconstructors;
% class Exympyl {
% def example() {
%~~vis
\begin{xten}
val a : Point{rank==4} = [1,2,3,4];
val b : Region{rank==2} = [ -1 .. 1, -2 .. 2];
\end{xten}
%~~siv
% } } 
%~~neg


\section{Coercions and conversions}
\label{XtenConversions}
\label{User-definedCoercions}
\index{conversions}\index{coercions}

\XtenCurrVer{} supports the following coercions and conversions.

\subsection{Coercions}

A {\em coercion} does not change object identity; a coerced object may
be explicitly coerced back to its original type through a cast. A {\em
  conversion} may change object identity if the type being converted
to is not the same as the type converted from. \Xten{} permits
user-defined conversions (\Sref{sec:user-defined-conversions}).

\paragraph{Subsumption coercion.}
A subtype may be implicitly coerced to any supertype.
\index{coercions!subsumption coercion}

\paragraph{Explicit coercion (casting with \xcd"as")}
An object of any class may be explicitly coerced to any other
class type using the \xcd"as" operation.  If \xcd`Child <: Person` and
\xcd`rhys:Child`, then 
%~~gen
% package Types.Coercions;
%  class Person {}
%  class Child extends Person{} 
%  class Exampllllle { 
%    def example(rhys:Child) =
%~~vis
\begin{xten}
  rhys as Person
\end{xten}
%~~siv
%;}
%~~neg
is an expression of type \xcd`Person`.  

If the value coerced is not an instance of the target type,
a \xcd"ClassCastException" is thrown.  Casting to a constrained
type may require a run-time check that the constraint is
satisfied.
\index{coercions!explicit coercion}
\index{casting}
\index{\Xcd{as}}

\limitation{It is currently a static error, rather than the specified
\xcd`ClassCastException`, when the cast is statically determinable to be
impossible.}

\paragraph{Effects of explicit numeric coercion}
\label{sec:effects-of-explicit-numeric-coercions}

Coercing a number of one type to another type gives the best approximation of
the number in the result type, or a suitable disaster value if no
approximation is good enough.  

\begin{itemize}
\item Casting a number to a {\em wider} numeric type is safe and effective,
      and can be done by an implicit conversion as well as an explicit
%~~exp~~`~~`~~ ~~
      coercion.  For example, \xcd`4 as Long` produces the \xcd`Long` value of
      4. 
\item Casting a floating-point value to an integer value truncates the digits
      after the decimal point, thereby rounding the number towards zero.  
%~~exp~~`~~`~~
      \xcd`54.321 as Int` is \xcd`54`, and 
%~~exp~~`~~`~~ ~~
      \xcd`-54.321 as Int` is \xcd`-54`.
      If the floating-point value is too large to represent as that kind of
      integer, the coercion returns the largest or smallest value of that type
      instead: \xcd`1e110 as Int` is 
      \xcd`Int.MAX_VALUE`, \xcd`2147483647`. 

\item Casting a \xcd`Double` to a \xcd`Float` normally truncates digits: 
%~~exp~~`~~`~~ ~~
      \xcd`0.12345678901234567890 as Float` is \xcd`0.12345679f`.  This can
      turn a nonzero \xcd`Double` into \xcd`0.0f`, the zero of type
      \xcd`Float`: 
%~~exp~~`~~`~~ ~~
      \xcd`1e-100 as Float` is \xcd`0.0f`.  Since 
      \xcd`Double`s can be as large as about \xcd`1.79E308` and \xcd`Float`s
      can only be as large as about \xcd`3.4E38f`, a large \xcd`Double` will
      be converted to the special \xcd`Float` value of \xcd`Infinity`: 
%~~exp~~`~~`~~ ~~
      \xcd`1e100 as Float` is \xcd`Infinity`.
\item Integers are coerced to smaller integer types by truncating the
      high-order bits. If the value of the large integer fits into the smaller
      integer's range, this gives the same number in the smaller type: 
%~~exp~~`~~`~~ ~~
      \xcd`12 as Byte` is the \xcd`Byte`-sized 12, 
%~~exp~~`~~`~~ ~~
      \xcd`-12 as Byte` is -12. 
      However, if the larger integer {\em doesn't} fit in the smaller type,
%~~exp~~`~~`~~ ~~
      the numeric value and even the sign can change: \xcd`254 as Byte` is
      \xcd`Byte`-sized \xcd`-2`.  


\end{itemize}

\subsection{Conversions}

\paragraph{Widening numeric conversion.}
A numeric type may be implicitly converted to a wider numeric type. In
particular, an implicit conversion may be performed between a numeric
type and a type to its right, below:

\begin{xten}
Byte < Short < Int < Long < Float < Double
\end{xten}

\index{conversions!widening conversions}
\index{conversions!numeric conversions}

\paragraph{String conversion.}
Any object that is an operand of the binary
\xcd"+" operator may
be converted to \xcd"String" if the other operand is a \xcd"String".
A conversion to \xcd"String" is performed by invoking the \xcd"toString()"
method of the object.

\index{conversions!string conversion}

\paragraph{User defined conversions.}\label{sec:user-defined-conversions}
\index{conversions!user defined}

The user may define conversion operators from type \Xcd{A} {\em to} a
container type \Xcd{B} by specifying a method on \Xcd{B} as follows:

\begin{xten}
  public static operator (r: A): T = ... 
\end{xten}

The return type \Xcd{T} should be a subtype of \Xcd{B}. The return
type need not be specified explicitly; it will be computed in the
usual fashion if it is not. However, it is good practice for the
programmer to specify the return type for such operators explicitly.

For instance, the code for \Xcd{x10.lang.Point} contains:

\begin{xten}
  public static global safe operator (r: Rail[int])
     : Point(r.length) = make(r);
\end{xten}

The compiler looks for such operators on the container type \Xcd{B}
when it encounters an expression of the form \Xcd{r as B} (where
\Xcd{r} is of type \Xcd{A}). If it finds such a method, it sets the
type of the expression \Xcd{r as B} to be the return type of the
method. Thus the type of \Xcd{r as B} is guaranteed to be some subtype
of \Xcd{B}.

\begin{example}
Consider the following code:  



%~~stmt~~\begin{xten}~~\end{xten}~~ ~~
\begin{xten}
val p  = [2, 2, 2, 2, 2] as Point;
val q = [1, 1, 1, 1, 1] as Point;
val a = p - q;    
\end{xten}
This code fragment compiles successfully, given the above operator definition. 
The type of \Xcd{p} is inferred to be \Xcd{Point(5)} (i.e.{} the type 
%~~type~~`~~`~~ ~~
\xcd`Point{self.rank==5}`.
Similarly for \Xcd{q}. Hence the application of the operator ``\Xcd{-}'' is legal (it requires both arguments to have the same rank). The type of \Xcd{a} is computed as \Xcd{Point(5)}.
\end{example}
