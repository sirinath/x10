Tue Dec 21 06:47:22 2004
// This file x10/doc/fun.txt in the Purdue CVS

\section{The {\cf method} datatype}

We introduce the datatype:

\begin{x10}
   fun<T1,...,Tk,T>  
\end{x10}

\noindent which is the type of functions that take k arguments of the
corresponding type and yield a result of type T. Values of this type
may be stored in fields and array elements, passed as arguments,
returned from methods etc.

There are two ways of creating values at this type.  One way is to use
a {\cf fun} expression:
\begin{x10}
     fun T (ArgListopt) {Stm}  
\end{x10}

\noindent where {\cf ArgListopt} specifies the type and formal
arguments to the function, and {\cf Stm} must satisfy the conditions
for the body of a method of the given return type. The formal
arguments are to be considered bound in {\cf Stm}. {\cf Stm} may
reference variables bound in an outer scope only if those variables
are declared {\tt final}.

Second, we permit function literals. For any instance method or
operator {\cf m} defined on a class {\cf C}, with arguments of type
{\cf T1, \ldots ,Tk} and return type {\cf T}, the expression

\begin{x10}
     C#m  
\end{x10}
\noindent denotes a value of type {\cf fun<C, T1, \ldots , Tk, T>}. 

The only operation available on a value of type {\tt fun<T1, \ldots ,Tk,T>} is apply:

\begin{x10}
    f(t1,\ldots,tk)  
\end{x10}

\noindent The arguments must be of type {\cf T1,\ldots ,Tk}
respectively, and the type of the expression is {\cf
T}. Operationally, if {\cf f} is the value
\begin{x10}
   fun T (T1 v1, ..., Tk vk) {S}  
\end{x10}
\noindent then {\cf f(t1, \ldots, tk)} evaluates to {\cf S}, with {\cf
v1, \ldots ,vk} substituted by {\cf t1, \ldots, tk} respectively. If
{\cf f} is the value {\cf C\#m}, then {\cf f(t1, \ldots, tk)}
evaluates to {\cf t1.m(t2, \ldots ,tk)}.

Note that the function datatype is a first-class datatype: type
constructors such as nullable, future, array type constructors can be
applied to them. Thus one may have nullable functions, futures of
functions, arrays of functions etc.

\subsection{Implementation}

The {\cf fun(T1,...,Tk, T)} datatype may be implemented by the interface:
\begin{x10}
    public interface fun<T1,...,Tk, T> {
        T apply(T1 t1,..., Tk tk);
    }
\end{x10}
\noindent for every value of {\cf k}. The value
\begin{x10}
    fun T (T1 t1, ..., Tk tk) {S}
\end{x10}
may be translated to
\begin{x10}
    new fun<T1,...,Tk, T>() {
       public T apply(T1 t1,..., Tk tk) {
         S
       }
    }  
\end{x10}

The value {\cf C\#m} may be translated to

\begin{x10}
    new fun<C,T1,...,Tk, T>() {
       public T apply(C o, T1 t1,..., Tk tk) {
         return o.m(t1,...,tk);
       }
    }
\end{x10}

Application of a function {\cf f} to a tuple of arguments {\cf
(t1,\ldots, tn)} ) is performed by invoking the {\cf apply} function:

\begin{x10}
  f.apply(t1,...,tn);
\end{x10}

\subsection{Related language decisions}

\paragraph{Array constructors as functions.}

The introduction of the function datatype allows a simplification of
array constructors. They may now be specified as:

\begin{x10}
    new ArrayType Fun  
\end{x10}

\noindent where {\cf Fun} is a function which takes points as
arguments (over the region defined by {\cf ArrayType}) and returns
values of Base, the base type of the array. For instance

\begin{x10}
   new int[0:N] fun(region(0:N) i) { return i; }  
\end{x10}

Note that the function may be passed in as a parameter:
\begin{x10}
    public int[] makeArray( fun(region, int) init, int n) {
         return new int[0:n] init;
    }
\end{x10}

\paragraph{Scan and reduce operations.}

The class {\cf x10.lang.system} provides certain static methods on arrays:

\begin{x10}
  public extern static <T> T reduce(T[] a, fun<T,T,T> f, T identity);
  public extern static <T, distribution D> T[D] scan(T[D] a, fun<T,T,T> f, T identity);
  public extern static <T, distribution D> T[D] lift(T[D] a, fun<T,T> f);
\end{x10}

The first evaluates to a value of type {\tt T} obtained by applying
{\tt f} pairwise to all elements of the array {\tt a} in some
unspecified order.  (Therefore, for repeatability, it is necessary
that {\tt f} be associative and commutative, and {\tt identity}
satisfy the property {\tt f(a,identity)=a=f(identity,a)}.) The second
returns an array whose {\cf i}th element is obtained by reducing the
{\cf i}th sub-array of {\cf a} (in canonical order). The third returns
the array obtained by applying the operation {\cf f} on each element
of {\cf a}.

Thus a programmer may write:
\begin{x10}
    reduce(v, fun int (int x, int y) { return x+y;}, 0)  // reduce an int array
    reduce(v, int#'+', 0)  // same as above.
    lift(v, fun int (int x) {return x+x;}) // double the array.  
\end{x10}

\paragraph{Iterators as function}

Suppose the datatype {\cf fun<D, R>} is such that {\cf D} is statically
known to be finite. For instance {\cf D} may be a $k$-dimensional range.
Let {\cf f} be a value of such a type. Then the statement

\begin{x10}
   foreach f;
\end{x10}
\noindent executes {\cf f} asynchronously for each element {\cf d} in
the domain:
\begin{x10}
    async { f(d); }  
\end{x10}
Similarly the statement:
\begin{x10}
    for f;
\end{x10}
executes {\cf f} for each element {\cf d} in the domain in canonical order.
The statement
\begin{x10}
    ateach f;
\end{x10}
Similarly for ateach. 

This syntax has the advantage over the for/foreach/ateach syntax in
0.32 that the iterator may be stored in a datstructure, read from a
datastructure or passed in as an argument to a method invocation. For
instance

    public void iterate( fun(region, int) iterator) {
           foreach iterator;
    }
       


