\documentclass[10pt,a4paper]{article}

%% For typesetting theorems and some math symbols.
\usepackage{amssymb}
\usepackage{amsthm}

\usepackage{fullpage}

\title{Featherweight X10}

\author{}

\date{}




\usepackage{xspace}

% Macros for R^nRS.

\def\@makechapterhead#1{%
  \vspace*{50\p@}%
  {\parindent \z@ \raggedright \normalfont
    \ifnum \c@secnumdepth >\m@ne
        \huge\bfseries \thechapter \space\space\space
        \nobreak
    \fi
    \interlinepenalty\@M
    \Huge \bfseries #1\par\nobreak
    \vskip 40\p@
  }}


\makeatletter

\newcommand{\topnewpage}{\@topnewpage}

% Chapters, sections, etc.

\newcommand{\vest}{}
\newcommand{\dotsfoo}{$\ldots\,$}

\newcommand{\sharpfoo}[1]{{\tt\##1}}
\newcommand{\schfalse}{\sharpfoo{f}}
\newcommand{\schtrue}{\sharpfoo{t}}

\newcommand{\singlequote}{{\tt'}}  %\char19
\newcommand{\doublequote}{{\tt"}}
\newcommand{\backquote}{{\tt\char18}}
\newcommand{\backwhack}{{\tt\char`\\}}
\newcommand{\atsign}{{\tt\char`\@}}
\newcommand{\sharpsign}{{\tt\#}}
\newcommand{\verticalbar}{{\tt|}}

\newcommand{\coerce}{\discretionary{->}{}{->}}

% Knuth's \in sucks big boulders
\def\elem{\hbox{\raise.13ex\hbox{$\scriptstyle\in$}}}

\newcommand{\meta}[1]{{\noindent\hbox{\rm$\langle$#1$\rangle$}}}
\let\hyper=\meta
\newcommand{\hyperi}[1]{\hyper{#1$_1$}}
\newcommand{\hyperii}[1]{\hyper{#1$_2$}}
\newcommand{\hyperj}[1]{\hyper{#1$_i$}}
\newcommand{\hypern}[1]{\hyper{#1$_n$}}
\newcommand{\var}[1]{\noindent\hbox{\it{}#1\/}}  % Careful, is \/ always the right thing?
\newcommand{\vari}[1]{\var{#1$_1$}}
\newcommand{\varii}[1]{\var{#1$_2$}}
\newcommand{\variii}[1]{\var{#1$_3$}}
\newcommand{\variv}[1]{\var{#1$_4$}}
\newcommand{\varj}[1]{\var{#1$_j$}}
\newcommand{\varn}[1]{\var{#1$_n$}}

\newcommand{\vr}[1]{{\noindent\hbox{$#1$\/}}}  % Careful, is \/ always the right thing?
\newcommand{\vri}[1]{\vr{#1_1}}
\newcommand{\vrii}[1]{\vr{#1_2}}
\newcommand{\vriii}[1]{\vr{#1_3}}
\newcommand{\vriv}[1]{\vr{#1_4}}
\newcommand{\vrv}[1]{\vr{#1_5}}
\newcommand{\vrj}[1]{\vr{#1_j}}
\newcommand{\vrn}[1]{\vr{#1_n}}


\newcommand{\defining}[1]{\mainindex{#1}{\em #1}}
\newcommand{\ide}[1]{{\schindex{#1}\frenchspacing\tt{#1}}}

\newcommand{\lambdaexp}{{\cf lambda} expression}
\newcommand{\Lambdaexp}{{\cf Lambda} expression}
\newcommand{\callcc}{{\tt call-with-current-continuation}}

% \reallyindex{SORTKEY}{HEADCS}{TYPE}
% writes (index-entry "SORTKEY" "HEADCS" TYPE PAGENUMBER)
% which becomes  \item \HEADCS{SORTKEY} mainpagenumber ; auxpagenumber ...

\global\def\reallyindex#1#2#3{%
\write\@indexfile{"#1" "#2" #3 \thepage}}

\newcommand{\mainschindex}[1]{\label{#1}\reallyindex{#1}{tt}{main}}
\newcommand{\mainindex}[1]{\reallyindex{#1@{\rm #1}{main}}}
\newcommand{\schindex}[1]{\reallyindex{#1}{tt}{aux}}
\newcommand{\sharpindex}[1]{\reallyindex{#1}{sharpfoo}{aux}}
%vj%\renewcommand{\index}[1]{\reallyindex{#1}{rm}{aux}}

\newcommand{\domain}[1]{#1}
\newcommand{\nodomain}[1]{}
%\newcommand{\todo}[1]{{\rm$[\![$!!~#1$]\!]$}}
\newcommand{\todo}[1]{}

% \frobq will make quote and backquote look nicer.
\def\frobqcats{%\catcode`\'=13 %\catcode`\{=13{}\catcode`\}=13{}
\catcode`\`=13{}}
{\frobqcats
\gdef\frobqdefs{%\def'{\singlequote}
\def`{\backquote}}}%\def\{{\char`\{}\def\}{\char`\}}
\def\frobq{\frobqcats\frobqdefs}

% \cf = code font
% Unfortunately, \cf \cf won't work at all, so don't even attempt to
% next constructions which use them...
\newcommand{\cf}{\frenchspacing\tt}

% Same as \obeycr, but doesn't do a \@gobblecr.
{\catcode`\^^M=13 \gdef\myobeycr{\catcode`\^^M=13 \def^^M{\\}}%
\gdef\restorecr{\catcode`\^^M=5 }}

{\catcode`\^^I=13 \gdef\obeytabs{\catcode`\^^I=13 \def^^I{\hbox{\hskip 4em}}}}

{\obeyspaces\gdef {\hbox{\hskip0.5em}}}

\gdef\gobblecr{\@gobblecr}

\def\setupcode{\@makeother\^}

% Scheme example environment
% At 11 points, one column, these are about 56 characters wide.
% That's 32 characters to the left of the => and about 20 to the right.

\newenvironment{x10noindent}{
  % Commands for scheme examples
  \newcommand{\ev}{\>\>\evalsto}
  \newcommand{\lev}{\\\>\evalsto}
  \newcommand{\unspecified}{{\em{}unspecified}}
  \newcommand{\scherror}{{\em{}error}}
  \setupcode
  \small \cf \obeytabs \obeyspaces \myobeycr
  \begin{tabbing}%
\qquad\=\hspace*{5em}\=\hspace*{9em}\=\kill%   was 16em
\gobblecr}{\unskip\end{tabbing}}

%\newenvironment{scheme}{\begin{schemenoindent}\+\kill}{\end{schemenoindent}}
\newenvironment{x10}{
  % Commands for scheme examples
  \newcommand{\ev}{\>\>\evalsto}
  \newcommand{\lev}{\\\>\evalsto}
  \renewcommand{\em}{\rmfamily\itshape}
  \newcommand{\unspecified}{{\em{}unspecified}}
  \newcommand{\scherror}{{\em{}error}}
  \setupcode
  \small \cf \obeyspaces \myobeycr
  \footnotesize
  \begin{tabbing}%
\qquad\=\hspace*{5em}\=\hspace*{9em}\=\+\kill%   was 16em
\gobblecr}{\unskip\end{tabbing}\normalsize}

\newcommand{\evalsto}{$\Longrightarrow$}

% Manual entries

\newenvironment{entry}[1]{
  \vspace{3.1ex plus .5ex minus .3ex}\noindent#1%
\unpenalty\nopagebreak}{\vspace{0ex plus 1ex minus 1ex}}

\newcommand{\exprtype}{syntax}

% Primitive prototype
\newcommand{\pproto}[2]{\unskip%
\hbox{\cf\spaceskip=0.5em#1}\hfill\penalty 0%
\hbox{ }\nobreak\hfill\hbox{\rm #2}\break}

% Parenthesized prototype
\newcommand{\proto}[3]{\pproto{(\mainschindex{#1}\hbox{#1}{\it#2\/})}{#3}}

% Variable prototype
\newcommand{\vproto}[2]{\mainschindex{#1}\pproto{#1}{#2}}

% Extending an existing definition (\proto without the index entry)
\newcommand{\rproto}[3]{\pproto{(\hbox{#1}{\it#2\/})}{#3}}

% Grammar environment

\newenvironment{grammar}{
  \def\:{& \goesto{} &}
  \def\|{& $\vert$& }
  \def\opt{$^?$\ }
  \def\star{$^*$\ }
  \def\plus{$^+$\ }
  \em
  \begin{tabular}{rcl}
  }{\unskip\end{tabular}}

%\newcommand{\unsection}{\unskip}
\newcommand{\unsection}{{\vskip -2ex}}

% Commands for grammars
\newcommand{\arbno}[1]{#1\hbox{\rm*}}  
\newcommand{\atleastone}[1]{#1\hbox{$^+$}}

\newcommand{\goesto}{{\normalfont{::=}}}

% mark modifications (for the grammar) From Igor Pechtchanski/Watson/IBM@IBMUS
\newlength{\modwidth}\setlength{\modwidth}{0.005in}
\newlength{\modskip}\setlength{\modskip}{.4em}
\newlength{\@modheight}
\newlength{\@modpos}
\providecommand{\markmod}[1]{%
  \setlength{\@modheight}{#1}%
  \addtolength{\@modheight}{-0.06in}%
  \setlength{\@modpos}{\linewidth}%
  \addtolength{\@modpos}{0.285in}%         Magic
  \addtolength{\@modpos}{\modwidth}%
  \addtolength{\@modpos}{\modskip}%
  \marginpar{\vspace{-\@modheight}%
             \hspace{-\@modpos}%
             \rule{\modwidth}{#1}}%
}

% The index

\def\theindex{%\@restonecoltrue\if@twocolumn\@restonecolfalse\fi
%\columnseprule \z@
%!! \columnsep 35pt
\clearpage
\@topnewpage[
    \centerline{\large\bf\uppercase{Alphabetic index of definitions of concepts,}}
    \centerline{\large\bf\uppercase{keywords, and procedures}}
    \vskip 1ex \bigskip]
    \markboth{Index}{Index}
    \addcontentsline{toc}{chapter}{Alphabetic index of 
 definitions of concepts, keywords, and procedures}
    \bgroup %\small
    \parindent\z@
    \parskip\z@ plus .1pt\relax\let\item\@idxitem}

\def\@idxitem{\par\hangindent 40pt}

\def\subitem{\par\hangindent 40pt \hspace*{20pt}}

\def\subsubitem{\par\hangindent 40pt \hspace*{30pt}}

\def\endtheindex{%\if@restonecol\onecolumn\else\clearpage\fi
\egroup}

\def\indexspace{\par \vskip 10pt plus 5pt minus 3pt\relax}

\makeatother

%\newcommand{\Xten}{{\sf X10}}
%\newcommand{\XtenCurrVer}{{\sf X10 v1.7}}
%\newcommand{\java}{{\sf Java}}
%\newcommand{\Java}{{\sf Java}}

\newcommand{\Xten}{X10}
\newcommand{\XtenCurrVer}{\Xten{} v1.7}
\newcommand{\Java}{Java}
\newcommand{\java}{\Java{}}

\newcommand{\futureext}[1]{{\em \paragraph{Future Extensions.}#1}}
\newcommand{\tbd}{} % marker for things to be done later.
\newcommand{\limitation}[1]{{\em Limitation: #1}} % marker for things to be done later.


\newcommand\grammarrule[1]{\emph{#1}}

% Rationale

\newenvironment{rationale}{%
\bgroup\noindent{\sc Rationale:}\space}{%
\egroup}

% Notes

\newenvironment{note}{%
\bgroup\noindent{\sc Note:}\space}{%
\egroup}

\newenvironment{staticrule*}{%
\bgroup\noindent{\textsc{Static Semantics Rule:}\space}}{%
\egroup}

\newenvironment{staticrule}[1]{%
\bgroup\noindent{\textsc{Static Semantics Rule} (#1):\space}}{%
\egroup}

\newcommand\Sref[1]{\S\ref{#1}}
\newcommand\figref[1]{Figure~\ref{#1}}
\newcommand\tabref[1]{Table~\ref{#1}}
\newcommand\exref[1]{Example~\ref{#1}}

\newcommand\eat[1]{}



\begin{document}


\maketitle


\lstset{language=java,basicstyle=\ttfamily\small}

%\chapter{Featherweight Ownership and Immutability Generic Java}
\section{Introduction}


Main lemma:
(i)~Progress and Preservation.
(ii)~After a location is cooked, val fields are never assigned;
    Before a location is cooked, val (and var) fields are never read.

Conclusion:
val fields have a single unique value that is read by all threads.
Like in Java, we will need a barrier whenever the constructor of some cooker is finished.
However, Java has the problem that if this escapes, then even this barrier does not guarantee immutability.
In contrast, in X10, we do make such guarantee.
We could also have a different implementation technique, where we flush only the newly created objects (no need to flush the entire memory).



We begin with some definitions.

FX10 program consists of class declarations followed by the program's expression.

An expression~$\he$ is called \emph{closed} (denoted $\closed{\he}$) if $\he$ does not contain \proto nor
    free variables (but it may contain \cooked or other locations).
For example, the expression~$\code{new Foo<l>()}$ is closed, but $\code{new Foo<proto>()}$ is not closed.
Similarly, we define a closed type.

Given a field $\code{A}~\code{C'}~\hf$ and a method $\code{U m(}\ol{\hV}~\ol{\hx}\code{) K \lb~return e;~\rb}$ in class~\code{C"},
    then for any subclass \code{C} of \code{C"}
    we define:
\beqst
%    \code{\mtype{}(m,C)} &\Fdef \ol{\hV}\rightarrow\hU \\
    \code{\mtype{}(m,C<K>)} &\Fdef  [\hK/\proto](\ol{\hV}\rightarrow\hU)\\
    \code{\mbody{}(m,C)} &\Fdef  \he\\
    \code{\mcooker{}(m,C)} &\Fdef  \hK\\
    \code{\fclass{}(f,C)} &\Fdef  \code{C'}\\
    \code{\isVar{}(f,C)} &\Fdef  (\code{A}=\code{var})\\
\eeq
Function~$\fields{}$ returns the fields of a class, i.e.,~$\fields{}(\hC)=\ol{\hf}$.

Function~$\cooker{\hK}$ receives a proto set~$P$ and a cooker~$\hK ::= \proto ~|~ \cooked ~|~ {\hl}$,
    and return whether it is cooked or proto:
\beqst
\cooker{\hK} \Fdef
\begin{cases}
\code{cooked} & \hK=\code{cooked} \\
\code{proto} & \hK=\code{proto} \\
\code{cooked} & \hK=\code{l}\text{~~and~~}\hl \not \in P \\
\code{proto} & \hK=\code{l}\text{~~and~~}\hl \in P \\
\end{cases}
\eeq

Given an expression~\he, we define~$R(\he)$ to be the set
    of all ongoing constructors in~\he, i.e., all locations in subexpressions~\code{e;return l}.
Formally,  $R(\he) \Fdef  \{ \hl ~|~ \code{return l} \in \he \}$.
%Formally,
%\[
%R(\he) =
%\begin{cases}
%    R(\code{e'}) \cup \{ l \} & \text{if~}\he=(\code{e';return l}) \\
%    R(\code{e'}) & \text{if~}\he=(\code{e'.f}) \\
%    R(\code{e'}) \cup R(\code{e"}) & \text{if~}\he=(\code{e'.f=e"}) \\
%    \bigcup_{\code{e'}\in\ol{\he'}} R(\code{e'}) & \text{if~}\he=(\code{new T}\hparen{\ol{\code{e'}}}) \\
%    R(\code{e"}) \cup (\bigcup_{\code{e'}\in\ol{\code{e'}}} R(\ol{\code{e'}})) & \text{if~}\he=(\code{e".m}\hparen{\ol{\code{e'}}}) \\
%    \end{cases}
%\]

Function~$\NPE(\he)$ returns whether an expression will throw a \code{null}-pointer exception:
\beqs{NPE}
\NPE(\he) & \Fdef \NPE(\he,\code{false}) \\
\NPE(\he,\code{b})& \Fdef
\begin{cases}
    \NPE(\code{e'},\code{false}) & \text{if~}\he=(\code{e';return l}) \\
    \NPE(\code{e'},\code{true}) & \text{if~}\he=(\code{e'.f}) \\
    \NPE(\code{e"},\code{true}) \vee \NPE(\code{e'},\code{false}) & \text{if~}\he=(\code{e".f=e'}) \\
    \vee \NPE(\ol{\code{e'}},\code{false}) & \text{if~}\he=(\code{new T}\hparen{\ol{\code{e'}}}) \\
    \NPE(\code{e'},\code{true}) \vee \NPE(\ol{\code{e"}},\code{false}) & \text{if~}\he=(\code{e'.m}\hparen{\ol{\code{e"}}}) \\
    \code{b} & \text{if~}\he=(\code{null}) \\
    \code{false} & \text{if~}\he=(\code{l}) \\
    \end{cases}\\
&\text{An alternative definition (though longer) that takes into account the order of evaluation: (which one is better?)}\\
\NPE(\he) & \Fdef
\begin{cases}
    \NPE(\code{e'}) & \text{if~}\he=(\code{e';return l}) \\
    \code{true} & \text{if~}\he=(\code{null.f}) \\
    \NPE(\code{e`}) & \text{if~}\he=(\code{e`.f}) \\
    \code{true} & \text{if~}\he=(\code{null.f=e'}) \\
    \NPE(\code{e'}) & \text{if~}\he=(\code{l.f=e'}) \\
    \NPE(\code{e"}) & \text{if~}\he=(\code{e".f=e'}) \\
    \NPE(\code{e'}) & \text{if~}\he=(\code{new T}\hparen{\ol{v},\code{e"},\ol{\code{e'}}}) \\
    \code{true} & \text{if~}\he=(\code{null.m}\hparen{\ol{\code{e'}}}) \\
    \NPE(\code{e"}) & \text{if~}\he=(\code{l.m}\hparen{\ol{v},\code{e"},\ol{\code{e'}}}) \\
    \NPE(\code{e"}) & \text{if~}\he=(\code{e".m}\hparen{\ol{\code{e'}}}) \\
    \code{false} & \text{if~}\he=(\code{v}) \\
    \end{cases}
    \gap \code{e"}\neq\hv
    \gap \code{e`}\neq\code{null}\\
\eeq

Summary of syntax used:
The comma operator ($,$) represents disjoint union.
Environment $\Gamma$ maps variables to types, i.e., $\Gamma(\hx)=\hT$.

A \emph{location}~\hl is a pointer to an object on the heap.
An \emph{object} has the form~$\code{C<l>}\hparen{\ol{\hv}}$, where~\hC is a class,~$\hl$ is the object's cooker, and~$\ol{\hv}$ are the values of the object's fields.
A \emph{heap}~$H$ maps locations to objects, i.e., $H(\hl)=\code{C<l'>}\hparen{\ol{\hv}}$.
The proto-set $P \in \dom(H)$ is a set of locations whose constructor has not finished yet.

We define two relations over cookers:
    (i)~$\Pequals{\hK}{\code{K'}}$ meaning that the two cookers are equivalent, and
    (ii)~$\Ppoints{\hK}{\code{K'}}$ meaning that an object with cooker~$\hK$ can point to an object with cooker~$\code{K'}$.
\beqst
\Pequals{\hK}{\code{K'}} & \Fdef \hK=\code{K'} \text{~~or~~} \cooker{\hK}=\cooker{\code{K'}}=\code{cooked}\\
\Ppoints{\hK}{\code{K'}} & \Fdef \hK=\code{K'} \text{~~or~~} \cooker{\code{K'}}=\code{cooked}\\
\eeq

A heap~$H$ is well-typed for~$P$, written~$P \vdash H$, if it satisfies:
    (i)~there is a linear order~$\Tprec$ over~$\dom{}(H)$ such that for every location~\hl, $H(\hl)=\code{C<l'>}\hparen{\ol{\hv}}$,
        we have~$\hl' \Tprec \hl$,
        and
    (ii)~for each object~$\code{C<l>}\hparen{\ol{\hv}}$ with fields~$\fields{}(\hC)=\ol{\hf}$, and for each non-null field~$\hv_i\neq\code{null}$,
        we have that~$H(\hv_i) = \code{C'<l'>}\hparen{\ldots}$, $\code{C'} \st \fclass(\hf_i,\hC)$, and~$\Ppoints{\hl}{\code{l'}}$.

Summary of judgements:
\beqst
\cooker{\hK} & \quad \text{Returns either \code{cooked} or \code{proto}}\\
\closed{\he} & \quad \text{Expression \he is closed}\\
\NPE(\he) & \quad \text{Expression \he is about to throw a \code{null}-pointer exception}\\
\hC \st \code{C'} & \quad \text{Class \hC is a subclass of \code{C'}}\\
P \vdash \hT \st \code{T'} & \quad \text{Type \hT is a subtype of \code{T'}} \\
\Ppoints{\hK}{\code{K'}} & \quad \text{An object with cooker \hK can point to another with cooker \code{K'}}\\
\Pequals{\hK}{\code{K'}} & \quad \text{Cooker \hK is equivalent to \code{K'}}\\
\Gamma,H,P \vdash \he : \hT & \quad \text{Expression \he has type \code{T}}\\
P \vdash H,\he \rightsquigarrow H',\code{e'} & \quad \text{Expression \he reduces (in one step) to \code{e'}, and heap~$H$ reduces to~$H'$}\\
P \vdash H & \quad \text{Heap $H$ is well-typed for~$P$}\\
\eeq

\begin{smaller}

\begin{figure*}[htpb!]
\begin{center}
\begin{tabular}{|l|l|}
\hline

$\hK ::= \proto ~|~ \cooked ~|~ \textbf{\hl}$ & cooKer. \\

$\code{T} ::= \code{C<K>}$ & Type. \\

$\code{A} ::= \code{var}~|~\code{val}$ & Assignable (\code{var}) or final (\code{val}) field. \\

$\code{F} ::= \code{A}~\hT~\hf\texttt{;}$ & Field declaration. \\

$\hM ::= \code{T} ~ \hm\hparen{\ol{\code{T}} ~ \ol{\hx}}~\hK ~ \lb\ \hreturn ~ \he\texttt{;}~\rb$
& Method declaration. \\

$\hL ::= \hclass ~ \hC\code{~extends~C'} \lb\ \ol{\code{F}}~\ol{\hM}~\rb$
& cLass declaration. \\


$\hv ::= \code{null} ~|~ \textbf{\hl} $
& Values: either \code{null} or a location~\hl. \\


% No cast: \hparen{\hT} ~ \he ~|~
$\he ::= \hv ~|~ \hx ~|~ \he.\hf ~|~ \he.\hf = \he ~|~ \he.\hm\hparen{\ol{\he}} ~|~ \hnew ~ \hT\hparen{\ol{\he}}  ~|~ \textbf{\he\code{;return l}}$
& Expressions. \\ %: values, parameters, field access\&assignment, invocation, \code{new} start\&finish

\hline
\end{tabular}
\end{center}
\caption{FX10 Syntax. Class declarations in FX10 cannot contain locations~\hl (marked with a boldface).
    Such locations are created during the reduction process (see \RULE{R-New} in \Ref{Figure}{reduction}).}
\label{Figure:syntax}
\end{figure*}


\begin{figure*}[!bt]
\begin{center}
\begin{tabular}{|c|}
\hline


$\typerule{
}{
  \Gamma \vdash \hT \st \hT
}$
~\RULE{(S1)}\quad

$\typerule{
  \Gamma \vdash \code{S} \st \hT
   \gap
  \Gamma \vdash \hT \st \hU
}{
  \Gamma \vdash \code{S} \st \hU
}$
~\RULE{(S2)}\quad
$\typerule{
  \code{class C}_1\code{~extends C}_2^\hK
}{
  \Gamma \vdash \hC_1^\hK \st \code{C}_2^\hK
}$
~\RULE{(S3)}

\\

$\typerule{
  \hl \in \Gamma[K]
}{
  \Gamma \vdash \hl \st \proto
}$
~\RULE{(S4)}\quad
$\typerule{
  \hl \not \in \Gamma[K]
}{
  \Gamma \vdash \hl \st \cooked
}$
~\RULE{(S5)}\quad
$\typerule{
  \hl \not \in \Gamma[K]
}{
  \Gamma \vdash \cooked \st \hl
}$
~\RULE{(S6)}\quad

$\typerule{
    \Gamma \vdash \code{K} \st \code{K}'
}{
  \Gamma \vdash \hC^\code{K} \st \hC^{\code{K}'}
}$
~\RULE{(S7)}\quad
\\


$\typerule{
  \Gamma \vdash \code{T} \st \code{T}'
}{
  \Gamma \vdash \code{T} \Pst \code{T}'
}$
~\RULE{(S8)}\quad
$\typerule{
  \hl \not \in \Gamma[K]
}{
  \Gamma \vdash \code{C}^\hl \Pst \code{C}^\hK
}$
~\RULE{(S9)}\quad
\\

\hline
\end{tabular}
\end{center}
\caption{FX10 Subtyping Rules. The subtyping relation is $\st$, whereas $\Pst$ is the pointing relation (anything can point to a cooked object).}
\label{Figure:subtyping}
\end{figure*}


\begin{figure*}[t]
\begin{center}
\begin{tabular}{|c|}
\hline
$\typerule{
  \Gamma,P \cup \{ \hl \} \vdash \he:\hT
}{
  \Gamma,P \vdash \code{e;return l} : \Gamma(\hl)
}$
\quad \RULE{(T-return)}
\\\\

$\typerule{
\hK'=
\begin{cases}
\bot & \hK=\cooked \\
\hK & \text{otherwise} \\
\end{cases}
    \gap
  \mtype{}(\code{build},\code{C<K'>})=\ol{\code{T}}\rightarrow\code{Object}
    \gap
  \Gamma,P \vdash \ol{\he}:\ol{\code{V}}
    \gap
  \Gamma,P \vdash \ol{\code{V}} \st \ol{\code{T}}
}{
  \Gamma,P \vdash \code{new C<K>(}\ol{\he}\code{)} : \code{C<K>}
}$
\quad \RULE{(T-New)}\\

$\typerule{
}{
  \Gamma,P \vdash \hx : \Gamma(\hx)
}$
\quad \RULE{(T-Var)}
\qquad
$\typerule{
}{
  \Gamma,P \vdash \code{null} : \hT
}$
\quad \RULE{(T-null)}
\qquad
$\typerule{
}{
  \Gamma,P \vdash \hl : \Gamma(\hl)
}$
\quad \RULE{(T-Location)}\\\\

$\typerule{
  \Gamma,P \vdash \he:\code{C<K>}
    \gap
  \isCooked(\hK)
    \gap
  \ftype{}(\hf,\hC)=\code{C'}
}{
  \Gamma,P \vdash \he.\hf : \code{C'<K>}
}$
\quad \RULE{(T-Field-Access)}\\\\


$\typerule{
  \Gamma,P \vdash \he:\code{C<K>}
    \gap
  \ftype{}(\hf,\hC)=\code{C'}
    \gap
  \Gamma,P \vdash \code{e'}:\code{T'}
    \gap
  \Gamma,P \vdash \code{T'} \Pst \code{C'<K>}
    \\
  \big(\isProto(\hK) \text{~~or~~} \isVar{}(\hf,\hC)\big)
}{
  \Gamma,P \vdash \he.\hf = \code{e'} : \code{T'}
}$
\quad \RULE{(T-Field-Assignment)}\\\\

$\typerule{
  \Gamma,P \vdash \he':\code{C<K>}
    \gap
  \mtype{}(\hm,\code{C<K>})=\ol{\code{T}}\rightarrow\code{U}
    \gap
  \Gamma,P \vdash \ol{\he}:\ol{\code{V}}
    \gap
  \Gamma,P \vdash \ol{\code{V}} \st \ol{\code{T}}
    \\
  \isCooked(\hK)=\isCooked(\mproto{}(\hm,\code{C}))
}{
  \Gamma,P \vdash \he'\code{.m(}\ol{\he}\code{)} : \code{U}
}$
\quad \RULE{(T-Invoke)}\\


\hline
\end{tabular}
\end{center}
\caption{FX10 Expression Typing Rules.}
\label{Figure:expressions}
\end{figure*}


\begin{figure*}[t]
\begin{center}
\begin{tabular}{|c|}
\hline

$\typerule{
  \hl \not \in \dom(H)
    \gap
  \code{l'} =
    \begin{cases}
    \hl & \text{if~}\isCooked(\hK) \\
    \hK & \text{otherwise} \\
    \end{cases}
}{
  P \vdash H,\code{new C<K>}\hparen{\ol{\hv}} \rightarrow H[\hl \mapsto \code{C<l'>}\hparen{\ol{\code{null}}}],\hl\code{.build}\hparen{\ol{\hv}}\code{;return l}
}$
\quad \RULE{(R-New)}\\\\

$\typerule{
  H[\hl] = \code{C<K>}\hparen{\ol{\hv}}
    \gap
  \fields{}(\hC)=\ol{\hf}
}{
  P \vdash H,\hl.\hf_i \rightarrow H,\hv_i
}$
\quad \RULE{(R-Field-Access)}
\\\\

$\typerule{
  H[\hl] = \code{C<K>}\hparen{\ol{\hv}}
    \gap
  \fields{}(\hC)=\ol{\hf}
}{
  P \vdash H,\hl.\hf_i = \hv' \rightarrow H[\hl \mapsto \code{C<K>}\hparen{[\hv'/\hv_i]\ol{\hv}}],\hv'
}$
\quad \RULE{(R-Field-Assignment)}\\\\


$\typerule{
}{
  P \vdash H,\code{v;return l} \rightarrow H,\hl
}$
\quad \RULE{(R-return)}
\gap

$\typerule{
  H[\hl] = \code{C<K>}\hparen{\ldots}
    \gap
  \mbody{}(\hm,\code{C})=\ol{\hx}.\he'
}{
  P \vdash H,\hl\code{.m(}\ol{\hv}\code{)} \rightarrow H, [\ol{\hv}/\ol{\hx}, \hl/\this, \hl/\proto]\he'
}$
\quad \RULE{(R-Invoke)}\\\\



$\typerule{
  P \cup \{\hl\} \vdash H,\he \rightarrow H',\code{e'}
}{
  P \vdash H,\code{e;return l} \rightarrow H',\code{e';return l}
}$
\quad \RULE{(R-c1)}
\gap

$\typerule{
  P \vdash H,\he \rightarrow H',\code{e'}
}{
  P \vdash H,\code{e.f} \rightarrow H',\code{e'.f}
}$
\quad \RULE{(R-c2)}
\\\\

$\typerule{
  P \vdash H,\he \rightarrow H',\code{e'}
}{
  P \vdash H,\code{e.f=e"} \rightarrow H',\code{e'.f=e"}
}$
\quad \RULE{(R-c3)}
\gap

$\typerule{
  P \vdash H,\he \rightarrow H',\code{e'}
}{
  P \vdash H,\code{l.f=e} \rightarrow H',\code{l.f=e'}
}$
\quad \RULE{(R-c4)}
\\\\

$\typerule{
  P \vdash H,\he \rightarrow H',\code{e'}
}{
  P \vdash H,\code{new C<K>}\hparen{\ol{\hv},\he,\ol{\code{e"}}} \rightarrow H',\code{new C<K>}\hparen{\ol{\hv},\code{e'},\ol{\code{e"}}}
}$
\quad \RULE{(R-c5)}
\\\\


$\typerule{
  P \vdash H,\he \rightarrow H',\code{e'}
}{
  P \vdash H,\he\code{.m(}\ol{\code{e"}}\code{)} \rightarrow H',\code{e'}\code{.m(}\ol{\code{e"}}\code{)}
}$
\quad \RULE{(R-c6)}
\gap

$\typerule{
  P \vdash H,\he \rightarrow H',\code{e'}
}{
  P \vdash H,\code{l.m(}\ol{\hv},\he,\ol{\code{e"}}\code{)} \rightarrow H',\code{l.m(}\ol{\hv},\code{e'},\ol{\code{e"}}\code{)}
}$
\quad \RULE{(R-c7)}
\gap

\\
\hline
\end{tabular}
\end{center}
\caption{FX10 Reduction Rules. The congruence rules have the initial \RULE{R-c}.}
\label{Figure:reduction}
\end{figure*}

\end{smaller}


Next we describe the syntax (\Ref{Figure}{syntax}),
    subtyping rules (\Ref{Figure}{subtyping}),
    expression typing rules (\Ref{Figure}{expressions}),
    and reduction rules (\Ref{Figure}{reduction}).

\section{Syntax}
Obviously, class declarations cannot contain locations.

\section{Subtyping}


\section{Typing}
\paragraph{Method typing}
If \proto appears in $\mtype{}(\hm,\hC)$ then $\mcooker{}(\hm,\hC)=\proto$.

An overriding method must keep the same $\mtype$ and $\mcooker$.

In class~\hC, when typing a method:
        $\code{U} ~ \hm\hparen{\ol{\code{V}} ~ \ol{\hx}} ~ \hK~ \lb\ \hreturn ~ \he\texttt{;} \rb$\\
        we use an environment~$\Gamma=\{\ol{\hx}:\ol{\code{T}}, \this:\code{C<K>}\}$,
        and we must prove that~$\Gamma,\emptyset,\emptyset \vdash \he:\code{S}$
        and~$\emptyset \vdash \code{S} \st \code{U}$.

\paragraph{Expression typing}
See \Ref{Figure}{expressions}.


\section{Reduction}
See \Ref{Figure}{reduction}.

\begin{Theorem}[preservation]
  \textbf{(Progress and Preservation)}
    For every expression~$\he$, heap~$H$, and proto-set~$P$,
        there exists~$H'$ and~$\he'$
        such that
        \[
        \begin{cases}
        & \he \neq \hv\\
        & \NPE(\he)=\code{false}\\
        & \closed{\he}\\
        & P \cup R(\he) \vdash H\\
        & \emptyset,H,P \vdash \he : \hT\\
        \end{cases}
        \Longrightarrow
        \begin{cases}
        & P \vdash H,\he \rightarrow H',\he'\\
        &\closed{\he'}\\
        &P \cup R(\he') \vdash H'\\
        &\emptyset,H',P \vdash \he':\hT\\
        \end{cases}
        \]
\end{Theorem}


(ii)~After a location is cooked, val fields are never assigned;
    Before a location is cooked, val (and var) fields are never read.


\end{document}
