\chapter{Interfaces}
\label{XtenInterfaces}\index{interface}

An interface specifies signatures for zero or more public methods, properties,
\xcd`static val`s, 
classes, structs, interfaces, types
and an invariant. 

The following puny example illustrates all these features: 
% TODO Well, it would if there weren't a compiler bug in the way.
%~~gen ^^^Interfaces_static_val
% package Interfaces_static_val;
% NOCOMPILE
%~~vis
\begin{xten}
interface Pushable(text:String, prio:Int){prio != 0} {
  def push(): void;
  static val MAX_PRIO = 100;
  abstract class Pushedness{}
  struct Pushy{}
  interface Pushing{}
  static type Shove = Int;
}
class MessageButton(text:String, prio:Int){
  implements Pushable{self.prio==Pushable.MAX_PRIO} {
  public def push() { 
    x10.io.Console.OUT.println(text + " pushed");
  }
}
\end{xten}
%~~siv
%
%~~neg
\noindent
\xcd`Pushable` defines two properties, a method, and a static value.  It also
establishes an invariant, that \xcd`prio != 0`. 
\xcd`MessageButton` implements a constrained version of \xcd`Pushable`,
\viz\ one with maximum priority.  It also has \xcd`Pushable`'s properties.  It
defines the \xcd`push()` method given in the interface, as a \xcd`public`
method---interface methods are implicitly \xcd`public`.

A concrete type---a class or struct---can {\em implement} an interface,
typically by having all the methods and properties that the interface
requires.

A variable may be declared to be of interface type.  Such a variable has all
the properties and methods declared (directly or indirectly) by the interface;
nothing else is statically available.  Values of any concrete type which
implement the interface may be stored in the variable.  

\begin{ex}
The following code puts two quite different objects into the variable
\xcd`star`, both of which satisfy the interface \xcd`Star`.
%~~gen ^^^ Interfaces6l3f
% package Interfaces6l3f;
%~~vis
\begin{xten}
interface Star { def rise():void; }
class AlphaCentauri implements Star {
   public def rise() {}
}
class ElvisPresley implements Star {
   public def rise() {}
}
class Example {
   static def example() {
      var star : Star;
      star = new AlphaCentauri();
      star.rise();
      star = new ElvisPresley();
      star.rise();
   }
}
\end{xten}
%~~siv
%
%~~neg
\end{ex}
An interface may extend several interfaces, giving
X10 a large fraction of the power of multiple inheritance at a tiny fraction
of the cost.

\begin{ex}
%~~gen ^^^ Interfaces6g4u
% package Interfaces6g4u;
%~~vis
\begin{xten}
interface Star{}
interface Dog{}
class Sirius implements Dog, Star{}
class Lassie implements Dog, Star{}
\end{xten}
%~~siv
%
%~~neg
\end{ex}


\section{Interface Syntax}

\label{DepType:Interface}

%##(NormalInterfaceDecl TypeParamsWithVariance Properties PropertyList WhereClause ExtendsInterfaces InterfaceBody InterfaceMemberDecl
\begin{bbgrammar}
%(FROM #(prod:NormalInterfaceDecl)#)
 NormalInterfaceDecl \: Mods\opt \xcd"interface" Id TypeParamsWithVariance\opt Properties\opt WhereClause\opt ExtendsInterfaces\opt InterfaceBody & (\ref{prod:NormalInterfaceDecl}) \\
%(FROM #(prod:TypeParamsWithVariance)#)
TypeParamsWithVariance \: \xcd"[" TypeParamWithVarianceList \xcd"]" & (\ref{prod:TypeParamsWithVariance}) \\
%(FROM #(prod:Properties)#)
          Properties \: \xcd"(" PropertyList \xcd")" & (\ref{prod:Properties}) \\
%(FROM #(prod:PropertyList)#)
        PropertyList \: Property & (\ref{prod:PropertyList}) \\
                    \| PropertyList \xcd"," Property \\
%(FROM #(prod:WhereClause)#)
         WhereClause \: DepParams & (\ref{prod:WhereClause}) \\
%(FROM #(prod:ExtendsInterfaces)#)
   ExtendsInterfaces \: \xcd"extends" Type & (\ref{prod:ExtendsInterfaces}) \\
                    \| ExtendsInterfaces \xcd"," Type \\
%(FROM #(prod:InterfaceBody)#)
       InterfaceBody \: \xcd"{" InterfaceMemberDecls\opt \xcd"}" & (\ref{prod:InterfaceBody}) \\
%(FROM #(prod:InterfaceMemberDecl)#)
 InterfaceMemberDecl \: MethodDecl & (\ref{prod:InterfaceMemberDecl}) \\
                    \| PropertyMethodDecl \\
                    \| FieldDecl \\
                    \| ClassDecl \\
                    \| InterfaceDecl \\
                    \| TypeDefDecl \\
                    \| \xcd";" \\
\end{bbgrammar}
%##)


\noindent
The invariant associated with an interface is the conjunction of the
invariants associated with its superinterfaces and the invariant
defined at the interface. 

%%WRONG%% \begin{staticrule*}
%%WRONG%% The compiler declares an error if this constraint
%%WRONG%% is not consistent.  
%%WRONG%% \end{staticrule*}

Each interface implicitly defines a nullary getter method
\xcd"def p(): T" for each property \xcd"p: T". The interface may not have
another definition of a method \xcd`p()`. 



A class \xcd"C"  implements an interface \xcd"I" if
\begin{itemize}
\item \xcd`I`, or a subtype of \xcd`I`, appears in the \xcd`implements` list
      of \xcd`C`, 
\item \xcd`C`'s properties include all the properties of \xcd"I",
%%IMPLIED%% \item \xcd`C`'s class invariant $\mathit{inv}($\xcd"C"$)$ implies
%%IMPLIED%% $\mathit{inv}($\xcd"I"$)$.
\item Each method \xcd`m` defined by \xcd`I` is also a method of \xcd`C` --
      with the {\em  \xcd`public`} modifier added.   These methods may be
      \xcd`abstract` if \xcd`C` is \xcd`abstract`.
\end{itemize}


If \xcd`C` implements \xcd`I`, then the class invariant
(\Sref{DepType:ClassGuardDef}) for \xcd`C`,   $\mathit{inv}($\xcd"C"$)$, implies
the class invariant for \xcd`I`, $\mathit{inv}($\xcd"I"$)$.  That is, if the
interface \xcd`I` specifies some requirement, then every class \xcd`C` that
implements it satisfies that requirement.

\section{Access to Members}

All interface members are \xcd`public`, whether or not they are declared
public.  There is little purpose to non-public methods of an interface; they
would specify that implementing classes and structs have methods that cannot
be seen.

\section{Field Definitions}
\index{interface!field definition in}

An interface may declare a \xcd`val` field, with a value.  This field is implicitly
\xcd`public static val`: 
%~~gen ^^^ Interfaces10
% package Interface.Field;
%~~vis
\begin{xten}
interface KnowsPi {
  PI = 3.14159265358;
}
\end{xten}
%~~siv
%
%~~neg

Classes and structs implementing such an interface get the interface's fields as
\xcd`public static` fields.  Unlike properties and methods, there is no need
for the implementing class to declare them. 
%~~gen ^^^ Interfaces20
% package Interface.Field.Two;
% interface KnowsPi {PI = 3.14159265358;}
%~~vis
\begin{xten}
class Circle implements KnowsPi {
  static def area(r:Double) = PI * r * r;
}
class UsesPi {
  def circumf(r:Double) = 2 * r * KnowsPi.PI;
}
\end{xten}
%~~siv
%
%~~neg

\subsection{Fine Points of Fields}

If two parent interfaces give different fields of the same name, 
those fields must be referred to by qualified names.
%~~gen ^^^ Interface_field_name_collision
% 
%~~vis
\begin{xten}
interface E1 {static val a = 1;}
interface E2 {static val a = 2;}
interface E3 extends E1, E2{}
class Example implements E3 {
  def example() = E1.a + E2.a;
}
\end{xten}
%~~siv
%
%~~neg

If the {\em same} field \xcd`a` is inherited through many paths, there is no need to
disambiguate it:
%~~gen ^^^ Interfaces_multi
% package Interfaces.Mult.Inher.Field;
%~~vis
\begin{xten}
interface I1 { static val a = 1;} 
interface I2 extends I1 {}
interface I3 extends I1 {}
interface I4 extends I2,I3 {}
class Example implements I4 {
  def example() = a;
}
\end{xten}
%~~siv
%
%~~neg

The initializer of a field in an interface may be any expression.  It is
evaluated under the same rules as a \xcd`static` field of a class. 

\begin{ex}
In this example, a local class (\Sref{sect:LocalClasses}) \xcd`B` is defined,
with an inner interface \xcd`I`.  The field \xcd`V` of \xcd`I` uses a variable
\xcd`n` which is global to \xcd`B`.   In this case it is a truly baroque way
to bind a \xcd`val`, but other uses are nontrivial.

%~~gen ^^^ Interfaces3l4a
% package Interfaces3l4a;
%~~vis
\begin{xten}
class A {
  static def example(n: Int) {
    class B {
      interface I { val V = n*n; }
    }
   return B.I.V + 1;
  }
}
\end{xten}
%~~siv
% class Hook{ def run() { return A.example(5) == 26; }}
%~~neg
\end{ex}

\section{Generic Interfaces}

Interfaces, like classes and structs, can have type parameters.  
The discussion of generics in \Sref{TypeParameters} applies to interfaces,
without modification.

\begin{ex}
%~~gen ^^^ Interfaces7n1z
% package Interfaces7n1z;
%~~vis
\begin{xten}
interface ListOfFuns[T,U] extends x10.util.List[(T)=>U] {}
\end{xten}
%~~siv
%
%~~neg

\end{ex}

\section{Interface Inheritance}

The {\em direct superinterfaces} of a non-generic interface \xcd`I` are the interfaces
(if any) mentioned in the \xcd`extends` clause of \xcd`I`'s definition.
If \xcd`I`  is generic, the direct superinterfaces are of an instantiation of
\xcd`I` are the corresponding instantiations of those interfaces.
A {\em superinterface} of \xcd`I` is either \xcd`I` itself, or a direct
superinterface of a superinterface of \xcd`I`, and similarly for generic
interfaces.    

\xcd`I` inherits the members of all of its superinterfaces. Any class or
struct that has \xcd`I` in its \xcd`implements` clause also implements all of
\xcd`I`'s superinterfaces. 






\section{Members of an Interface}

The members of an interface \xcd`I` are the union of the following sets: 
\begin{enumerate}
\item All of the members appearing in \xcd`I`'s declaration;
\item All the members of its direct super-interfaces, except those which are
      hidden (\Sref{sect:Hiding}) by \xcd`I`
\item The members of \xcd`Any`.
\end{enumerate}

Overriding for instances is defined as for classes, \Sref{MethodOverload}
