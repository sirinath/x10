\chapter{Lexical structure}

In general, \Xten{} follows \java{} rules \cite[Chapter 3]{jls2} for
lexical structure.

Lexically a program consists of a stream of white space, comments,
identifiers, keywords, literals, separators and operators.

\paragraph{Whitespace}
% Whitespace \index{whitespace} follows \java{} rules \cite[Chapter 3.6]{jls2}.
ASCII space, horizontal tab (HT), form feed (FF) and line
terminators constitute white space.

\paragraph{Comments}
% Comments \index{comments} follows \java{} rules
% \cite[Chapter 3.7]{jls2}. 
All text included within the ASCII characters ``\xcd"/*"'' and
``\xcd"*/"'' is
considered a comment and ignored; nested comments are not
allowed.  All text from the ASCII characters
``\xcd"//"'' to the end of line is considered a comment and is ignored.

\paragraph{Identifiers}

Identifiers \index{identifier} are defined as in \java.
Identifiers consist of a single letter followed by zero or more
letters or digits.
Letters are defined as the characters for which the \java{}
method \xcd"Character.isJavaIdentifierStart" returns true.
Digits are defined as the ASCII characters \xcd"0" through \xcd"9".

\paragraph{Keywords}
\Xten{} reserves the following keywords:
\begin{xten}
abstract        any             as              async
at              ateach          atomic          await
break           case            catch           class
clocked         continue        current
def             default         do              else
extends         extern          final           finally
finish          for             foreach         future
global
goto            has             here            if             
implements      import          in
instanceof      interface
native          new             next            nonblocking     
offer           offers
or              package         pinned          private        
protected       property        public          return
safe            self            sequential      shared
static
super           switch          this            throw
throws          to              try             type
val             value           var             when
while
\end{xten}
Note that the primitive types are not considered keywords.
The keyword \xcd{goto} is reserved, but not used.

\paragraph{Literals}\label{Literals}\index{literals}

Briefly, \XtenCurrVer{} uses fairly standard syntax for its literals:
integers, unsigned integers, floating point numbers, booleans, 
characters, strings, and \xcd"null".  The most exotic points are (1) unsigned
numbers are marked by a \xcd`u` and cannot have a sign; (2) \xcd`true` and
\xcd`false` are the literals for the booleans; and (3) floating point numbers
are \xcd`Double` unless marked with an \xcd`f` for \xcd`Float`. 

Less briefly, we use the following abbreviations: 
\begin{displaymath}
\begin{array}{rcll}
\delta &=& \mbox{one or more decimal digits}\\
\delta_8 &=& \mbox{one or more octal digits}\\
\delta_{16} &=& \mbox{one or more hexadecimal digits, using \xcd`a`-\xcd`f`
for 10-15}\\
\iota &=& \delta 
        \mathbin{|} {\tt 0} \delta_8 
        \mathbin{|} {\tt 0x} \delta_{16}
        \mathbin{|} {\tt 0X} \delta_{16}
\\
\sigma &=& \mbox{optional \xcd`+` or \xcd`-`}\\
\beta &=& \delta 
          \mathbin{|} \delta {\tt .}
          \mathbin{|} \delta {\tt .} \delta
          \mathbin{|}  {\tt .} \delta \\
\xi &=& ({\tt e } \mathbin{|} {\tt E})
         \sigma
         \delta \\
\phi &=& \beta \xi
\end{array}
\end{displaymath}

\begin{itemize}

\item \xcd`true` and \xcd`false` are the \xcd`Boolean` literals.

\item \xcd`null` is a literal for the null value.  It has type
      \xcd`Any{self==null}`. 

\item \xcd`Int` literals have the form {$\sigma\iota$}; \eg, \xcd`123`,
      \xcd`-321` are decimal \xcd`Int`s, \xcd`0123` and \xcd`-0321` are octal
      \xcd`Int`s, and \xcd`0x123`, \xcd`-0X321`,  \xcd`0xBED`, and \xcd`0XEBEC` are
      hexadecimal \xcd`Int`s.  

\item \xcd`Long` literals have the form {$\sigma\iota{\tt l}$} or
      {$\sigma\iota{\tt L}$}. \Eg, \xcd`1234567890L`  and \xcd`0xBAGEL` are \xcd`Long` literals. 

\item \xcd`UInt` literals have the form {$\iota{\tt u}$} or {$\iota {\tt U}$}.
      \Eg, \xcd`123u`, \xcd`0123u`, and \xcd`0xBEAU` are \xcd`UInt` literals.

\item \xcd`ULong` literals have the form {$\iota {\tt ul}$} or {$\iota {\tt
      lu}$}, or capital versions of those.  For example, 
      \xcd`123ul`, \xcd`0124567012ul`,  \xcd`0xFLU`, \xcd`OXba1eful`, and \xcd`0xDecafC0ffeefUL` are \xcd`ULong`
      literals. 

\item \xcd`Float` literals have the form {$\sigma \phi {\tt f}$} or  {$\sigma
      \phi {\tt F}$}.  Note that the floating-point marker letter \xcd`f` is
      required: unmarked floating-point-looking literals are \xcd`Double`. 
      \Eg, \xcd`1f`, \xcd`6.023E+32f`, \xcd`6.626068E-34F` are \xcd`Float`
      literals. 

\item \xcd`Double` literals have the form {$\sigma \phi$}\footnote{Except that
      literals like \xcd`1` 
      which match both {$\iota$} and {$\phi$} are counted as
      integers, not \xcd`Double`; \xcd`Double`s require a decimal
      point, an exponent, or the \xcd`d` marker.
      }, {$\sigma \phi {\tt
      D}$}, and {$\sigma \phi {\tt d}$}.  
      \Eg, \xcd`0.0`, \xcd`0e100`, \xcd`229792458d`, and \xcd`314159265e-8`
      are \xcd`Double` literals.

\item \xcd`Char` literals have one of the following forms: 
      \begin{itemize}
      \item \xcd`'`{\it c}\xcd`'` where {\em c} is any printing ASCII
            character other than 
            \xcd`\` or \xcd`'`, representing the character {\em c} itself; 
            \eg, \xcd`'!'`;
      \item \xcd`'\b'`, representing backspace;
      \item \xcd`'\t'`, representing tab;
      \item \xcd`'\n'`, representing newline;
      \item \xcd`'\f'`, representing form feed;
      \item \xcd`'\r'`, representing return;
      \item \xcd`'\''`, representing single-quote;
      \item \xcd`'\"'`, representing double-quote;
      \item \xcd`'\\'`, representing backslash;
      \item \xcd`'\`{\em dd}\xcd`'`, where {\em dd} is one or more octal
            digits, representing the one-byte character numbered {\em dd}; it
            is an error if {\em dd}{$>255$}.      
      \end{itemize}

\item \xcd`String` literals consist of a double-quote \xcd`"`, followed by
      zero or more of the contents of a \xcd`Char` literal, followed by
      another double quote.  \Eg, \xcd`"hi!"`, \xcd`""`.

\item There are no literals of type \xcd`Byte` or \xcd`UByte`.

\end{itemize}



\paragraph{Separators}
\Xten{} has the following separators and delimiters:
\begin{xten}
( )  { }  [ ]  ;  ,  .
\end{xten}

\paragraph{Operators}
\Xten{} has the following operators:
\begin{xten}
==  !=  <   >   <=  >=
&&  ||  &   |   ^
<<  >>  >>>
+   -   *   /   %
++  --  !   ~
&=  |=  ^=
<<= >>= >>>=
+=  -=  *=  /=  %=
=   ?   :   =>  ->
<:  :>  @   ..
\end{xten}




