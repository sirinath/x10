X10 is an object oriented programming language with a sophisticated
    type system (constraints, class invariants, non-erased generics, closures)
    and concurrency constructs (asynchronous activities, multiple places, global references).
Object initialization is a cross-cutting concern that interacts with all these features
    in delicate ways that may cause type-, runtime-, and security- errors.
This paper discusses possible designs for object initialization,
    and the ``hardhat" design chosen and implemented in X10 version 2.1.
Our implementation includes a
    fixed-point inter-procedural (intra-class)
    data-flow analysis
    that infers, for each method called during initialization,
    the set of fields that are read and
    those that are asynchronously and synchronously assigned.
%Finally, we present a case study in which a large collection of java classes
%    were converted to X10,
%    and discuss the consequences of having a hardhat design.
