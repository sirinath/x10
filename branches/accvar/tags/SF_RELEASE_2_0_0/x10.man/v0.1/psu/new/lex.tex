% Lexical structure

%%\vfill\eject
\chapter{Lexical conventions}

This section gives an informal account of some of the lexical
conventions used in writing Scheme programs.  For a formal syntax of
Scheme, see section~\ref{BNF}.

\vest Upper and lower case forms of a letter are never distinguished
except within character and string constants.  For example, {\cf Foo} is
the same identifier as {\cf FOO}, and {\tt\#x1AB} is the same number as
{\tt\#X1ab}.

\section{Identifiers}
\label{syntaxsection}

Most identifiers\mainindex{identifier} allowed by other programming
languages are also acceptable to Scheme.  The precise rules for
forming identifiers vary among implementations of Scheme, but in all
implementations a sequence of letters, digits, and ``extended
alphabetic characters'' that begins with a character that cannot begin
a number is an identifier.  In addition, \ide{+}, \ide{-}, and
\ide{...} are identifiers.  Implementations that extend the syntax
of identifiers shall not introduce ambiguities with other parts of the
grammar; in particular nothing that can be generated by
\meta{number} (see~\ref{numbersyntax}) shall be interpreted as an
identifier.

Here are some examples of identifiers:

\begin{scheme}
lambda                   q
list->vector             soup
{+}                        V17a
<=?                      a34kTMNs
the-word-recursion-has-many-meanings%
\end{scheme}

%  _ = 5F   & = 26   ~ = 7E   ^ = 5E

Extended alphabetic characters may be used within identifiers as if
they were letters.  The following are extended alphabetic characters:

\begin{scheme}
+ - . * / < = > !\ ?\ :\ \$ \% \char"5F{} \char"26{} \char"7E{} \char"5E %
\end{scheme}

See~\ref{extendedalphas} for a formal syntax of identifiers.

\vest Identifiers have several uses within Scheme programs:
\begin{itemize}
\item Certain identifiers are reserved for use as syntactic keywords
(see below).\index{syntactic keyword}\index{keyword}
%  (This does not preclude their use as identifiers
%as well, although in certain situations ambiguities can result if this is done.)

\item Any identifier that is not a syntactic keyword may be used as a
variable\index{variable} (see~\ref{variablesection}).

\item When an identifier appears as a literal or within a literal
(see~\ref{quote}), it is being used to denote a {\em symbol}
(see~\ref{symbolsection}).

\end{itemize}

\label{keywordsection}
The following identifiers are syntactic keywords, and shall not be used
as variables:

\begin{scheme}
=>           else          or
and          if            quasiquote
begin        lambda        quote
case         let           set!
cond         let*          unquote
define       letrec        unquote-splicing
do%
\end{scheme}


\section{Whitespace and comments}

\defining{Whitespace} characters are spaces and newlines.
(Implementations typically provide additional whitespace characters such
as tab or page break.)  Whitespace is used for improved readability and
as necessary to separate tokens from each other, a token being an
indivisible lexical unit such as an identifier or number, but is
otherwise insignificant.  Whitespace may occur between any two tokens,
but not within a token.  Whitespace may also occur inside a string,
where it is significant.

A semicolon ({\tt;}) indicates the start of a
comment.\mainindex{comment}\mainschindex{;}  The comment continues to the
end of the line on which the semicolon appears.  Comments are invisible
to Scheme, but the end of the line is visible as whitespace.  This
prevents a comment from appearing in the middle of an identifier or
number.

\begin{scheme}
;;; The FACT procedure computes the factorial
;;; of a non-negative integer.
(define fact
  (lambda (n)
    (if (= n 0)
        1        ;Base case: return 1
        (* n (fact (- n 1))))))%
\end{scheme}


\section{Other notations}

\todo{Rewrite?}

For a description of the notations used for numbers, see~\ref{numbersection}.

\begin{list}{}{\advance\labelwidth-\labelsep
    \let\makelabel\cphmakelabel}

\item[{\tt+ - .\ }]
\hspace*{\fill}\newline
These are used in numbers, and may also occur anywhere in an identifier
except as the first character.  A delimited plus or minus sign by itself
is also an identifier.
A delimited period (not occurring within a number or identifier) is used
in the notation for pairs (see~\ref{listsection}), and to indicate a
rest-parameter in a  formal parameter list (see~\ref{lambda}).
A delimited sequence of three successive periods is also an identifier.

\item[\tt( )]
\hspace*{\fill}\newline
Parentheses are used for grouping and to notate lists
(see~\ref{listsection}).

\item[\singlequote]
The single quote character is used to indicate literal data (see~\ref{quote}).

\item[\backquote]
The backquote character is used to indicate almost-constant
data (see~\ref{quasiquote}).

\item[\tt, ,@]
\hspace*{\fill}\newline
The character comma and the sequence comma at-sign are used in conjunction
with backquote (see~\ref{quasiquote}).

\item[\tt"]
The double quote character is used to delimit strings (see~\ref{stringsection}).

\item[\backwhack]
Backslash is used in the syntax for character constants
(see~\ref{charactersection}) and as an escape character within string
constants (see~\ref{stringsection}).

\setbox0\hbox{\tt \char"5B{} \char"5D{} \char"7B{} \char"7D}
\item[\copy0]
\hspace*{\fill}\newline
Left and right square brackets and curly braces
are reserved for possible future extensions to the language.

\item[\sharpsign] Sharp sign is used for a variety of purposes depending on
the character that immediately follows it:

\item[\schtrue{} \schfalse{}]
\hspace*{\fill}\newline
These are the boolean constants (see~\ref{booleansection}).

\item[\sharpsign\backwhack]
This introduces a character constant (see~\ref{charactersection}).

\item[\sharpsign\tt(]
This introduces a vector constant (see~\ref{vectorsection}).  Vector constants
are terminated by~{\tt)}~.

\item[{\tt\#e \#i \#b \#o \#d \#x}]
\hspace*{\fill}\newline
These are used in the notation for numbers (see~\ref{numbernotations}).

\end{list}
