\chapter{Types}
\label{XtenTypes}\index{types}

{}\Xten{} is a {\em strongly typed} object language: every variable
and expression has a type that is known at compile-time. Further,
\Xten{} has a {\em unified} type system: all data items created at
runtime are {\em objects} (\S~\ref{XtenObjects}. Types limit the
values that variables can hold, and specify the places at which these
values lie.

{}\Xten{} supports two kinds of objects, {\em reference objects} and
{\em value objects}.  Reference objects are instances of {\em
reference classes} (\S~\ref{ReferenceClasses}). They may contain
mutable fields and must stay resident in the place in which they were
created. Value objects are instances of {\em value classes}
(\S~\ref{ValueClasses}). They are immutable and may be freely copied
from place to place. Either reference or value objects may be 
{\em scalar} (instances of a non-array class) or {\em aggregate} (instances
of arrays).

Correspondingly, \Xten{} has two kinds of types. A {\em reference
type} consists of a {\em data type}, which is a set of values, and a
{\em place type} which specifies the place at which the value
resides. A {\em value type} simply consists of a data type, since
\Xten{} automatically copies value objects from one place to
another. Types are constructed through the application of {\em type
constructors} (\S~\ref{TypeConstructors}).

Types are used in variable declarations, casts, object creation, array
creation, class literals and {\cf instanceof} expressions.\footnote{In
order to allow this version of the language to focus on the core new
ideas, \XtenCurrVer{} does not have user-definable classloaders,
though there is no technical reason why they could not have been
added.}

A variable is a storage location (\S~\ref{XtenVariables}). All
variables are initialized with a value and cannot be observed without
a value. Variables whose value may not be changed after initialization
are called {\em final variables} (or sometimes {\em constants}). The
programmer indicates that a variable is final by using the annotation
{\tt final} in the variable declaration and using the {\tt =} operator
to set its value. Variables whose value may change are said to be {\em
mutable}. The programmer indicates that a variable is mutable by
prefixing it with {\tt mutable} and using the {\tt :=} operator to set
its value.


\section{Type constructors}\index{type constructors}\label{TypeConstructors}

{}\Xten{} specifies five {\em type constructors}.  A type constructor
takes one or more types as arguments and produces a type.

Some of these constructors -- the {\em generic} constructors -- take
{\em type parameters}\label{TypeParameter}\index{type!parameters}. Type
parameters are of two kinds, {\em reference type
parameters}\index{type parameters!reference} and {\em value type
parameters}\index{type parameters!value}.

\begin{x10}
{\em TypeParameter ::}
  value {\em{}DataTypeName} [{\em{}TypeBound}]
  [reference] {\em{}DataTypeName}`@'{\em{}PlaceTypeName} 
               [{\em{}TypeBound}]
\end{x10}

\noindent A {\cf\em PlaceTypeName} is the name of some place
variable currently in scope, or the literal {\cf here} (standing for the
``current'' place) or the literal {\cf threadlocal}. The role of place types is discussed in  more detail in \S~\ref{PlaceTypes}.

Type bounds are defined as in \java{} \cite{gjspec} and place an upper
bound on the types that may be substituted in for the type
parameter. \Xten{} uses ``\_'' instead of ``?'' as an anonymous type
parameter.


\paragraph{Interface declarations.}
The {\em interface declaration} (\S~\ref{XtenInterfaces}) takes as
argument one or more interfaces (the {\em extended} interfaces), one
or more type parameters and the definition of constants and method
signatures and the name of the defined interface. An interface

Each such declaration introduces a data type constructor.
\begin{x10}
{\em InterfaceType ::}
  value {\em{}InterfaceTypeName} [{\em{}TypeParameters}]
  reference {\em{}InterfaceTypeName} [{\em{}TypeParameters}]
\end{x10}

The constructor takes as argument the name of the interface optionally
prefixed by {\cf value} or {\cf reference} (if neither is specified
{\cf reference} is assumed) followed by values for its type arguments
(if any) in angle brackets (e.g.{} {\cf value Field<double>}).  It
returns a data type provided that the actual type arguments satisfy
the associated bounds.  Semantically, the data type is the set of all
objects which are instances of classes which are value or reference
classes (based on the qualifier to the type) and which implement the
interface obtained from the interface declaration by substituting the
actual arguments for the formals.

\paragraph{Reference class declarations.}\label{ReferenceTypes}
The {\em reference class declaration} (\S~\ref{ReferenceClasses})
takes as argument a reference class (the {\em extended class}), one or
more interfaces (the {\em implemented interfaces}), one or more type
parameters, the definition of fields, methods and inner types, and
returns a class of the named type.  

It may be used to construct a reference data type by supplying values
for its type arguments in the same way as an interface (e.g.{} {\cf
Cell<float>}). Semantically, the data type is the set of all objects
which are instances of (subclasses of) the class obtained from the
class declaration by substituting the actual arguments for the
formals.  A reference data type can be augmented to a reference type
by adding place information, e.g. {\cf Cell<float>@P}.

\paragraph{Value class declarations.}
The {\em value class declaration} (\S~\ref{ReferenceClasses}) is
similar to the reference class declaration except that it must extend
either a value class or a reference class that has no mutable
fields. 
It may be used to construct a value type in the same way as a
reference class declaration can be used to construct a reference
type. (Note that a value type does not require a place type.)

\paragraph{Array type constructor.}

{}\XtenCurrVer{} does not have array class declarations
(\S~\ref{XtenArrays}). This means that user cannot define new array
class types. Instead arrays are created as instances of array types
constructed through the application of {\em array type constructors}
(\S~\ref{XtenArrays}).

The array type constructor takes as argument a type (the {\em
base type}), a distribution (\S~\ref{XtenDistributions}), and
optionally either the keyword {\cf reference} or {\cf value}
(the default is reference):
\begin{x10}
{\em{}ArrayType::}
 {\em{}BaseType} value `['[{\em{}Region}]`]'
 {\em{}BaseType} [reference] `['[{\em{}Distribution}]`]'
\end{x10}

\todo{Specify that a missing distribution means that the distribution will 
be supplied dynamically.}

The keyword {\cf value} indicates that the resulting type is a value
array data type all of whose components are final; the keyword
``reference'' indicates that the resulting type is a reference array
data type and the components of the array are mutable.  For instance,
{\cf int [(32,64)-> P]} is the data type of mutable arrays of {\cf
32x64} variables, each containing an {\cf int}, and each located at
{\cf P} (see \S~\ref{XtenDistributions}).  To obtain a reference type,
one must specify where the array itself is located; thus {\cf (int
[(32,64)->P])@Q} is the type of array objects located at {\cf Q} where
the array components themslves are at {\cf P} as discussed above.

Note that a distribution can be multidimensional, arrays can be
nested, value arrays of reference base types can be constructed, as
can reference arrays of value base type.  Indeed, value arrays of
reference components (where the components themselves may be arrays)
are often useful in programs that desire to share only the bottom
layers of the array while allowing the top layers to be copied to the
referencing places.

\paragraph{Nullable data type constructor.}
The {\em nullary type constructor} (\S~\ref{NullableTypeConstructor})
takes as argument a base data type and returns a new data type which
has the same values as the original one and a value denoted by the
literal {\cf null}, in case it did not already have this value.

\begin{x10}
{\em NullaryDataType ::}
   nullable \em{BaseDataType}
\end{x10}

This type constructor may be applied to value or reference, scalar or
array types.

The type {\tt nullable C} has the same visibility as the type {\tt
C}. It is always top-level (i.e., not nested and not inner). It has
the same members as {\tt C}. It implements the same interfaces as {\tt
C} along with the interface {\tt x10.lang.Nullable}.

The type {\tt nullable C} can be implicitly cast to the type {\tt T}
only if {\tt T} is equal to a type {\tt nullable D} and {\tt C} can be
implicitly cast to {\tt D}.




\section{Variables}\label{XtenVariables}\index{variables}

A variable of a reference data type {\tt reference R} where {\tt R} is
the name of an interface (possibly with type arguments) always holds a
reference to an instance of a class implementing the interface {\tt R}.

A variable of a reference data type {\tt R} where {\tt R} is the name
of a reference class (possibly with type arguments) always holds a
reference to an instance of the class {\tt R} or a class that is a
subclass of {\tt R}. 

A variable of a reference array data type {\tt R [D]} is always an
array which has as many variables as the size of the region underlying
the distribution {\cf D}. These variables are distributed across
places as specified by {\cf D} and have the type {\tt R}.

A variable of a nullary (reference or value) data type {\tt ?T} always
holds either the value (named by) {\tt null} or a value of type {\tt
T} (these cases are not mutually exclusive).

A variable of a value data type {\tt value R} where {\tt R} is the
name of an interface (possibly with type arguments) always holds
either a reference to an instance of a class implementing {\tt R} or
an instance of a class implementing {\tt R}. No program can
distinguish between the two cases.

A variable of a value data type {\tt R} where {\tt R} is the name of a
value class (possibly with type arguments) always holds a reference to
an instance of {\tt R} (or a class that is a subclass of {\tt R}) or
an instance of {\tt R} (or a class that is a subclass of {\tt R}). No
program can distinguish between the two cases.

A variable of a value array data type {\tt V value [R]} is always an
array which has as many variables as the size of the region {\tt R}.
Each of these variables is immutable and has the type {\tt V}.

As in \java, \Xten{} supports seven kinds of variables: {\em class
variables} (static variables), {\em instance variables} (the instance
fields of a class), {\em array components}, {\em method parameters},
{\em constructor parameters}, {\em exception-handler parameters} and
{\em local variables}. See \cite[\S 4.5.3]{jls2}.

\subsection{Final variables}\label{FinalVariable}\index{variable!final}\index{final variable}
A final variable satisfies two conditions: 
\begin{itemize}
\item it can be assigned to at most once, 
\item it must be assigned to before use. 
\end{itemize}

\Xten{} follows \java{} language rules in this respect \cite[\S
4.5.4,8.3.1.2,16]{jls2}. Briefly, the compiler must undertake a
specific analysis to statically guarantee the two properties above.

\todo{Check if this analysis needs to be revisited.}

\subsection{Initial values of variables}
\label{NullaryConstructor}\index{nullary constructor}
Every variable declared at a type must always contain a value of that type.

Every class variable, instance variable or array component variable is
initialized with a default value when it is created.  A variable
declared at a nullary type is always initialized with {\cf null}. For
a variable declared at a scalar class type it must be the case that a
nullary constructor for that class is visible at the site of the
declaration; the variable is initialized with the value returned by
invoking this constructor. For a variable declared at an array type it
must be the case that the base type is either nullable or a class type
with a nullary constructor visible at the site of the declaration. The
variable is then initialized with an array defined over the smallest
region and default distribution consistent with its declaration and
with each component of the array initialized to {\cf null} or the
result of invoking the nullary constructor.

Each method and constructor parameter is initialized to the
corresponding argument value provided by the invoker of the method. An
exception-handling parameter is initialized to the object thrown by
the exception. A local variable must be explicitly given a value by
initialization or assignment, in a way that the compiler can verify
using the rules for definite assignment \cite[\S~16]{jls2}.

Each class {\cf C} has an explicitly or implicitly defined nullary
constructor. If {\cf C} does not have an explicit nullary
constructor, it is a compile-time error if the class has a field at (a)~a
scalar type that is a {\cf class} whose nullary constructor is not
visible in {\cf C} or is an {\cf interface}, or (b)~at an array type whose
base type is a {\cf class} whose nullary constructor is not visible in
{\cf C} or is an {\cf interface}.

Otherwise a {\cf public} nullary constructor is created by
default. This constructor initializes each field of the class (if any)
as if it were a variable of the declared type of the field, as
described above.


\section{Objects}\label{XtenObjects}\index{Object}

An object is an instance of a scalar class or an array type. It is
created by using a class instance creation expression
(\S~\ref{ClassCreation}) or an array creation (\S~\ref{ArrayInitializer})
expression, such as an array initializer. An object that is an
instance of a reference (value) type is called a {\em reference} ({\em
value}) {\em object}. In \XtenCurrVer{} a reference object stays
resident at the place at which it was created for its entire lifetime.

{}\Xten{} has no operation to dispose of a reference.  Instead the
collection of all objects across all places is globally garbage
collected.

{}\Xten{} objects do not have any synchronization information (e.g.{}
a lock) associated with them. Thus the methods on {\tt
java.lang.Object} for waiting/synchronizing/notification etc are not
available in \Xten. Instead the programmer should use atomic sections
(S~\ref{AtomicSections}) for mutual exclusion and clocks
(S~\ref{XtenClocks}) for sequencing multiple parallel operations.

A reference object may have many references, stored in fields of
objects or components of arrays. A change to an object made through
one reference is visible through another reference. \Xten{} mandates
that all accesses to mutable objects shared between multiple
activities must occur in an atomic section (\S\ref{AtomicSections}).

Note that the creation of a remote async activity
(\S~\ref{AsyncActivity}) {\cf A} at {\cf P} may cause the automatic creation of
references to remote objects at {\cf P}. (A reference to a remote
object is called a {\em remote object reference}, to a local object a
{\em local object reference}.)  For instance {\cf A} may be created
with a reference to an object at {\cf P} held in a variable referenced
by the statement in {\cf A}.  Similarly the return of a value by a
{\cf future} may cause the automatic creation of a remote object
reference, incurring some communication cost.  An {}\Xten{}
implementation should try to ensure that the creation of a second or
subsequent reference to the same remote object at a given place does
not incur any (additional) communication cost.

A reference to an object may carry with it the values of final fields
of the object. The programmer is guaranteed that the implementation
will not incur the cost of communicating the values of final fields of
an object from the place where it is hosted to any other place more
than once for each target place, even for reference objects.

{}\Xten{} does not have an operation (such as Pascal's ``dereference''
operation) which returns an object given a reference to the
object. Rather, most operations on object references are transparently
performed on the bound object, as indicated below. The operations on
objects and object references include:
\begin{itemize}

{}\item Field access (\S~\ref{FieldAccess}). An activity holding a
reference to a reference object may perform this operation only if the
object is local.  An activity holding a reference to a value object
may perform this operation regardless of the location of the object
(since value objects can be copied freely from place to place). In
this case the cost of cost of copying the field from the place where
the object was created to the referencing place will be incurred at
most once per referencing place, according to the rule for final
fields discussed above.

\item Method invocation (\S~\ref{MethodInvocation}). 
An activity holding a reference to a reference object may perform this
operation only if the object is local.  An activity holding a
reference to a value object may perform this operation regardless of
the location of the object (since value objects can be copied
freely). The \Xten{} implementation must guarantee that the cost of
copying enough relevant state of the value object to enable this
method invocation to succeed is incurred at most once for each value
object per place.

{}\item Casting (\S~\ref{ClassCast}).  An activity can perform this
operation on local or remote objects, and does not incur any
communication cost.

{}\item {\cf instanceof} operator (\S~\ref{instanceOf}). 
An activity can perform this
operation on local or remote objects, and does not incur any
communication cost.

\item The stable equality operator {\cf ==} and {\cf !=} (\S~\ref{StableEquality}). An activity can perform these operations on local or remote objects,
and does not incur any communication cost.

% \item The ternary conditional operator {\cf ?:}
\end{itemize}

\section{Built-in types}
The package {\tt x10.lang} provides a number of built-in class and
interface declarations that can be used to construct types.
 
For instance several value types are provided that encapsulate
abstractions (such as fixed point and floating point arithmetic)
commonly implemented in hardware by modern computers.

\begin{figure}
Built-in interfaces:

\begin{x10}
Field            FixedPoint          FloatingPoint      
SignedFixedPoint UnsignedFixedPoint
\end{x10}

Built-in reference types:
\begin{x10}
Object Reference
\end{x10}

Built-in value types:
\begin{x10}
boolean byte          char  complex<Field> 
double  doubledouble  float int
long    longlong      short string
ubyte   ushort        value

place   distribution  region clock
\end{x10}
\caption{The contents of the {\tt x10.lang} package.}\label{XtenLangPackage}
\end{figure}
Please consult \cite{XtenLibrary} for more details.  \todo{Ensure that
there are no language dependencies.}  

\subsection{Place constraints}
\label{PlaceTypes}
\label{PlaceType}
\index{place types}
\label{DepType:PlaceType}\index{placetype}

An \Xten{} computation spans multiple places
(\Sref{XtenPlaces}). Each place constains data and activities that
operate on that data.  \XtenCurrVer{} does not permit the dynamic
creation of a place. Each \Xten{} computation is initiated with a
fixed number of places, as determined by a configuration parameter.
In this section we discuss how the programmer may supply place type
information, thereby allowing the compiler to check data locality,
i.e., that data items being accessed in an atomic section are local.

\begin{grammar}
PlaceConstraint     \: \xcd"!" Place\opt \\
Place              \:   Expression \\
\end{grammar}

Because of the importance of places in the \Xten{} design, special
syntactic support is provided for constrained types involving places.

All \Xten{} classes extend the class
\xcd"x10.lang.Object", which defines a property
\xcd"home" of type
\xcd"Place".

If a constrained reference type \xcd"T" has an \xcd"!p" suffix,
the constraint for \xcd"T" is implicitly assumed to contain the clause
\xcd"self.home==p"; that is,
\xcd"C{c}!p" is equivalent to \xcd"C{self.home==p && c}".

The place \xcd"p" may be ommitted. It defaults to \xcd"this" 
for types in field declarations, and to \xcd"here" elsewhere.


% The place specifier \xcd"any" specifies that the object can be
% located anywhere.  Thus, the location is unconstrained; that is,
% \xcd"C{c}!any" is equivalent to \xcd"C{c}".

% XXX ARRAY
%The place specifier \xcd"current" on an array base type
%specifies that an object with that type at point \xcd"p"
%in the array 
%is located at \xcd"dist(p)".  The \xcd"current" specifier can be
%used only with array types.





\section{Conversions and Promotions}\label{XtenConversions}\label{XtenPromotions}\index{conversions}\index{promotions}

{}\XtenCurrVer{} supports \java's conversions and promotions
(identity, widening, narrowing, value set, assignment, method
invocation, string, casting conversions and numeric promotions)
appropriately modified to support \Xten's built-in numeric classes
rather than \java's primitive numeric types.

This decision may be revisited in future version of the language in
favor of a streamlined proposal for allowing user-defined
specification of conversions and promotions for value types, as part
of the syntax for user-defined operators.
