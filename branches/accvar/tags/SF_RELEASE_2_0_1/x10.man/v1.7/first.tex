% First page

\thispagestyle{empty}

% \todo{"another" report?}

\title{Report on the Experimental Language \Xten \\
\large Version \integerversion}
\ifdraft
\author{\textsc{Draft} --- \today \\
\\
Vijay Saraswat \\
Nathaniel Nystrom \\
\\
Please send comments to \\
Nathaniel Nystrom at \texttt{nystrom@us.ibm.com}}
\else
\author{
Vijay Saraswat \\
Nathaniel Nystrom \\
\\
Please send comments to \\
Nathaniel Nystrom at \texttt{nystrom@us.ibm.com}}
\fi
\date{\today}

\maketitle

\newcommand\authorsc[1]{#1}
%\newcommand\authorsc[1]{\textsc{#1}}

This report provides a description of the programming
language \Xten. \Xten{} is a single-inheritance class-based object-oriented
(OO) programming language designed for high-performance, high-productivity
computing on high-end computers supporting $\approx 10^5$ hardware threads
and $\approx 10^{15}$ operations per second. 

{}\Xten{} is based on state-of-the-art object-oriented programming
languages and deviates from them only as necessary to support its
design goals. The language is intended to have a simple and clear
semantics and be readily accessible to mainstream OO programmers. It
is intended to support a wide variety of concurrent programming
idioms.
%, incuding data parallelism, task parallelism, pipelining.
%producer/consumer and divide and conquer.

%We expect to revise this document in the light of experience gained in implementing
%and using this language.

The \Xten{} design team consists of
\authorsc{Ganesh Bikshandi}, 
\authorsc{David Cunningham},
\authorsc{Robert Fuhrer},
\authorsc{David Grove},
\authorsc{Sreedhar Kodali}, 
\authorsc{Bruce Lucas},
\authorsc{Nathaniel Nystrom},
\authorsc{Igor Peshansky}, 
\authorsc{Vijay Saraswat},
\authorsc{Sayantan Sur}, 
\authorsc{Olivier Tardieu},
\authorsc{Pradeep Varma}, and
\authorsc{Krishna Nandivada Venkata}.

Past members include
\authorsc{David Bacon}, 
\authorsc{Raj Barik}, 
\authorsc{Bob Blainey}, 
\authorsc{Philippe Charles}, 
\authorsc{Perry Cheng}, 
\authorsc{Christopher Donawa}, 
\authorsc{Julian Dolby}, 
\authorsc{Kemal Ebcio\u{g}lu},
\authorsc{Patrick Gallop}, 
\authorsc{Christian Grothoff}, 
\authorsc{Allan Kielstra}, 
\authorsc{Sriram Krishnamoorthy}, 
\authorsc{Vivek Sarkar},
\authorsc{Armando Solar-Lezama},  
\authorsc{S. Alexander Spoon}, 
\authorsc{Christoph von Praun},
\authorsc{Jan Vitek}, and
\authorsc{Tong Wen}.

For extended discussions and support we would like to thank: 
Gheorghe Almasi,
Robert Blackmore,
Robert Callahan, 
Calin Cascaval, 
Norman Cohen, 
Elmootaz Elnozahy, 
John Field,
Kevin Gildea,
Chulho Kim,
Orren Krieger, 
Doug Lea, 
John McCalpin, 
Paul McKenney, 
Andrew Myers,
Ram Rajamony,
R.K.~Shyamasundar, 
Filip Pizlo, 
V.T.~Rajan, 
Frank Tip, Mandana Vaziri, and Hanhong Xue.

We thank Jonathan Rhees and William Clinger with help in obtaining the
\LaTeX{} style file and macros used in producing the Scheme report,
on which this document is based. We acknowledge the influence of
the $\mbox{\Java}^{\mbox{\textsc{tm}}}$ Language
Specification \cite{jls2}.
%document, as evidenced by the numerous citations in the text.

This document revises Version 1.5 of the Report, released in
June 2007.  It documents the language corresponding to Version
1.7 of the implementation, which is currently under development.

Earlier implementations benefited from significant contributions by
\authorsc{Raj Barik}, 
\authorsc{Philippe Charles}, 
\authorsc{Christopher Donawa}, 
\authorsc{Robert Fuhrer},
\authorsc{Christian Grothoff},
\authorsc{Nathaniel Nystrom},  
\authorsc{Igor Peshansky},  
\authorsc{Vijay Saraswat},
\authorsc{Vivek Sarkar}, 
\authorsc{Olivier Tardieu},  
\authorsc{Pradeep Varma}, 
\authorsc{Krishna Nandivada Venkata}, and
\authorsc{Christoph von Praun}.)
\authorsc{Tong Wen} has written many application programs
in \Xten{}. \authorsc{Guojing Cong} has helped in the
development of many applications.
The implementation of generics in \Xten{} was influenced by the
implementation of PolyJ~\cite{polyj} by Andrew Myers and Michael Clarkson.

