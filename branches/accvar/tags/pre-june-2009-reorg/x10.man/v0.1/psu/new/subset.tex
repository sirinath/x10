\chapter{Number System Subsets}

(This appendix is not a part of P1178/D4, Draft Standard for the
Scheme Programming Language, but is included for information only.)

\vest The Scheme number system (see~\ref{numbersection}) is typically
implemented using several internal representations.  A complete
implementation of Scheme numbers is both large and complex.  For this
reason, it may be undesirable to implement the entire functionality
and the Standard specifically allows implementations to provide
subsets of the full functionality.  This appendix is a guide to Scheme
implementors suggesting reasonable and consistent subsets.

\vest The appendix is organized into three parts.  The first part
describes a minimal subset, known as the \tupe{fixnum} subset.
Following this there are two sections discussing two distinct classes
of extensions: \tupe{exact} and \tupe{inexact} arithmetic.  Either or
both may be compatibly added to the \tupe{fixnum} base.  The addition
of any other numeric representation transforms simple \tupe{fixnum}
arithmetic into a more complicated ``generic'' system.  This is
usually accompanied by an additional performance cost associated with
the determination (usually at run-time) of the representation used for
the arguments.

\section{Minimal Subset}

\vest We believe that the minimum practical Scheme implementation
includes signed \tupe{exact} \tupe{integer}s of a fixed limited
precision.  (Even more restrictive subsets of the integers, e.g. the
omission of negative integers, are consistent with this Standard.
Such subsets are significantly less useful and we discourage their
implementation.)  In such an implementation, when the mathematically
expected result of a computation is not representable as an integer
within the appropriate range, the computation is considered to have
violated an implementation restriction.  In general when such a
situation occurs, the implementation may choose either to coerce the
result to an inexact number or to report the violation; in this case,
since there are no inexact numbers, the implementation must report the
violation.

\vest This representation choice is traditionally known as
{\em fixnum}\index{fixnum}, and has two primary advantages.  First, the
precision is usually chosen to correspond to the natural word size of
the machine, thus allowing the implementation to use the built-in
integer arithmetic hardware.  (This advantage is not fully realizable
on many machines because of the need to test for overflow.)  Second,
the size of the \tupe{fixnum} representation is typically chosen so
that fixnums need not be allocated in the heap.

\section{Exact Arithmetic}

\vest The first (obvious) extension is to eliminate the precision
restriction on \tupe{integer}s.  The traditional way is to implement
{\em bignum}\index{bignum} arithmetic, also known as extended (or infinite)
precision arithmetic (see~\cite{Knuth}).  Implementation of these
operations, along with the coercion between \tupe{fixnum} and
\tupe{bignum} representations, is a time-consuming and exacting task;
but most implementations have found the benefits outweigh the costs.
The algorithms are rather slow, and space for these numbers is
allocated in the heap.

\vest The next logical extension is the implementation of \tupe{exact}
\tupe{rational} arithmetic, sometimes referred to as \defining{ratnum}
arithmetic; the advantages of \tupe{exact} \tupe{rational} arithmetic
are significant, and given \tupe{bignum} arithmetic, the
implementation cost is small.  While it is possible to provide an
implementation of \tupe{exact} \tupe{rational} numbers based on a
\tupe{fixnum} (rather than \tupe{bignum}) representation, experience
has shown that the limitations of such implementations are unintuitive
and consequently hard to use.

% \vest Note that the performance of \tupe{rational}s depends heavily on
% the performance of the \ide{gcd} algorithms to maintain the rational
% number in lowest terms.  This Standard does not mandate whether the
% \ide{gcd} is taken when a \tupe{rational} number is constructed or
% whether it is performed at the time that the \ide{numerator} and
% \ide{denominator} procedures are called.

\vest If the implementation provides \tupe{rational} numbers, it should
supply the following procedures in addition to those required
in~\ref{numbersection}:

\begin{scheme}
/
numerator
denominator
rationalize
\end{scheme}

\vest The final extension to exact arithmetic is to implement
\tupe{exact} \tupe{real} arithmetic.  This is usually not done despite
the existence of at least one example implementation~\cite{Vuillemin}.
Probably this is because of the relative inefficiency of the
algorithms (compared to those for \tupe{inexact} \tupe{real} numbers),
and the lack of hardware support for them.

\section{Inexact Arithmetic}

\vest The usual representation employed to implement \tupe{inexact}
\tupe{real} numbers is a well-established one: floating-point.  We
strongly urge implementors using this representation to adhere to the
IEEE 32-bit and 64-bit floating-point standards.  A simple
implementation might support only one format (preferably the 64-bit
format) while a more elaborate one might support multiple formats,
carefully observing the restrictions of this Standard
(see~\ref{numbersection}).

\vest An interesting alternative to floating-point is \tupe{rational}
numbers whose denominator never exceeds a pre-determined maximum value
but whose numerator is unbounded.  This conveniently allows the
representation of arbitrarily large numbers but has a fixed smallest
increment (and hence a fixed smallest representable positive value).
The algorithms for this representation are virtually identical to
those used for \tupe{exact} \tupe{rational} arithmetic, and they have
mathematically acceptable behavior under error conditions.
% [Actually these are much more dense than fixed point -- CPH]
% They are
% an extension of the more widely known ``fixed point'' number
% representation.

\vest We do not recommend implementation of more restrictive forms of
\tupe{inexact} arithmetic, such as \tupe{inexact} \tupe{integer}
arithmetic, because \tupe{inexact} arithmetic is primarily useful for
approximations to the real number system, and the more restrictive
forms are not very useful.  Also, a good implementation of
\tupe{inexact} \tupe{real} arithmetic will to a large extent subsume
the other representations.

\vest If \tupe{inexact} \tupe{real} numbers are implemented, the
following operations should be supported in addition to those required
in~\ref{numbersection}:

\begin{scheme}
/            numerator     denominator
rationalize  exp           log
sin          cos           tan
asin         acos          atan
sqrt         expt
exact->inexact
inexact->exact%
\end{scheme}

\vest The final step in the number tree is to \tupe{complex}
arithmetic.  The usual implementation of these numbers is as pairs of
\tupe{real} numbers in a simple rectangular coordinate representation
($x+iy$).  We suggest implementors provide \tupe{complex} arithmetic
only if \tupe{real} arithmetic is also provided.  Without a reasonable
approximation to the real numbers the usefulness of \tupe{complex}
numbers is suspect and it is preferable to omit them entirely.

\vest The operations that should be provided to support complex numbers
(in addition to those of \tupe{inexact} \tupe{real} arithmetic) are:

\begin{scheme}
make-rectangular
make-polar
real-part
imag-part
magnitude
angle
\end{scheme}
