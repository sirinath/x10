\chapter{Lexical structure}

In general, \Xten{} follows \java{} rules \cite[Chapter 3]{jls2} for
lexical structure.

Lexically a program consists of a stream of white space, comments,
identifiers, keywords, literals, separators and operators.

\paragraph{Whitespace}
Whitespace \index{whitespace} follows \java{} rules
\cite[Chapter 3.6]{jls2}. ASCII space, horizontal tab (HT), form feed (FF) and line terminators constitute white space.


\paragraph{Comments}
Comments \index{comments} follows \java{} rules
\cite[Chapter 3.7]{jls2}. 
All text included within the ASCII characters ``/*'' and ``*/'' is
considered a comment and ignored. All text from the ASCII character
``//'' to the end of line is considered a comment and ignored.

\paragraph{Identifiers}
Identifiers \index{identifier} are defined as in \java.

\paragraph{Keywords}
\Xten{} reserves the following keywords \index{keyword} from \java:
\begin{x10}
abstract   break  case       catch
class      const  continue   default    
do         else   extends    final
finally    for    goto       if            
implements import instanceof interface
native     new    package    private      
protected  public return     static
super      switch this       throw
throws     try    void       while
\end{x10}
(Note that the primitive types are no longer considered keywords.)

\Xten{}  introduces the following keywords:
\begin{x10}
activitylocal async     ateach     atomic   
await         clocked   current    foreach
finish        future    here       next 
nullable      or        placelocal reference
value        when
\end{x10}

\paragraph{Literals}\label{Literals}\index{literals}

\XtenCurrVer{} defines literal syntax in the same way as \java{} does.

\paragraph{Separators}
\Xten{} has the following separators:
\begin{x10}
(	)	\{	\}	[  ]	; ,	.
\end{x10}

\paragraph{Operators}
\Xten{} has the following operators:
{\footnotesize
\begin{verbatim}
  =>  <    !   ~   ?   :   ==  <=  
  >=  !=   &&  ||  ++  --  +   - 
  *   /    &   |   ^   %   <<  >> 
  >>> +=   -=  *=  /=  &=  |=  ^=  
  %=  <<=  >>= >>> =  ->
\end{verbatim}}




