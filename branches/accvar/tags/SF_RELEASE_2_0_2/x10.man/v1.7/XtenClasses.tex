\chapter{Classes}
\label{XtenClasses}\index{class}

The {\em class declaration} has
a list of type \params,
value properties, 
a constraint (the {\em class invariant}, a single superclass,
one or more interfaces, and a class body containing the
the definition of
fields, methods, and member types.
Each such declaration introduces a class
type (\Sref{ReferenceTypes}).

\begin{grammar}
NormalClassDeclaration \:
      ClassModifiers\opt \xcd"class" Identifier  \\
   && TypeParameterList\opt PropertyList\opt Guard\opt \\
   && Super\opt Interfaces\opt ClassBody \\
\\
TypeParameterList     \:  \xcd"[" TypeParameters \xcd"]" \\
TypeParameters        \:  TypeParameter ( \xcd"," Typearameter )\star \\
TypeParameter         \:  Variance\opt Annotation\star Identifier     \\
Variance \: \xcd"+" \\
         && \xcd"-" \\
\\
PropertyList     \:  \xcd"(" Properties \xcd")" \\
Properties       \:  Property ( \xcd"," Property )\star \\
Property         \:  Annotation\star \xcd"val"\opt Identifier \xcd":" Type \\
\\
Super \: \xcd"extends" ClassType \\
Interfaces \: \xcd"implements" InterfaceType ( \xcd"," InterfaceType)\star \\
\\
ClassBody \: ClassMember\star \\
ClassMember \: ClassDeclaration \\
            \| InterfaceDeclaration \\
            \| FieldDeclaration \\
            \| MethodDeclaration \\
            \| ConstructorDeclaration \\
\end{grammar}

A type parameter declaration is given by an optional variance
tag and an identifier.
A type parameter must be
bound to a concrete type when an instance of the class is created.


A value property has a name and a type.   Value properties
are accessible in the same way as \xcd"public" \xcd"final"
fields.

\begin{staticrule*}
It is a compile-time error for a class
defining a value property \xcd"x: T" to have an ancestor class that defines
a value property with the name \xcd"x".  
\end{staticrule*}

Each class \xcd"C" defining a property \xcd"x: T" implicitly has a field

\begin{xten}
public val x : T;
\end{xten} 

\noindent and a getter method

\begin{xten}
public final def x(): T { return x; }
\end{xten}

\noindent Each interface \xcd"I" defining a property \xcd"x: T"
implicitly has a getter method

\begin{xten}
public def x(): T;
\end{xten}

\begin{staticrule*}
It is a compile-time error for a class or
interface defining a property \xcd"x :T" to have an existing method with
the signature \xcd"x(): T".
\end{staticrule*}

Properties are used to build dependent types from classes, as
described in \Sref{DepType:DepType}.

\label{ClassGuard}

The \grammarrule{Guard} in a class or interface declaration specifies an
explicit condition on the properties of the type, and is discussed further
in \Sref{DepType:Guard}.

\begin{staticrule*}
     Every constructor for a class defining
   properties \xcdmath"x$_1$: T$_1$, $\ldots$, x$_n$: T$_n$" must ensure that each of the fields
   corresponding to the properties is definitely initialized
   (cf. requirement on initialization of final fields in Java) before the
   constructor returns.
\end{staticrule*}

Type \params{}
are used to define generic classes and
interfaces, as described in \Sref{Generics}.

Classes are structured in a single-inheritance code
hierarchy, may implement multiple interfaces, may have static and
instance fields, may have static and instance methods, may have
constructors, may have static and instance initializers, may have
static and instance inner classes and interfaces. \Xten{} does not
permit mutable static state, so the role of static methods and
initializers is quite limited. Instead programmers should use
singleton classes to carry mutable static state.

Method signatures may specify checked exceptions. Method definitions
may be overridden by subclasses; the overriding definition may have a
declared return type that is a subclass of the return type of the
definition being overridden. Multiple methods with the same name but
different signatures may be provided on a class (ad hoc
polymorphism). The public/private/protected/package-protected access
modification framework may be used.

\oldtodo{Add the new rule for preventing leakage of this from a constructor.}

Because of its different concurrency model, \Xten{} does not support
\xcd"transient" and \xcd"volatile" field modifiers.

\oldtodo{Figure out class modifiers. Figure out which new ones need to be added to support IEEE FP.}

\section{Reference classes}\index{class!reference class}\label{ReferenceClasses}
A reference class is declared with the optional keyword \xcd"reference" preceding \xcd"class" in a class declaration. Reference
class declarations may be used to construct reference types
(\Sref{ReferenceTypes}). Reference classes may have mutable
fields. Instances of a reference class are always created in a fixed
place and in \XtenCurrVer{} stay there for the lifetime of the
object. (Future versions of \Xten{} may support object migration.)
Variables declared at a reference type always store a reference to the
object, regardless of whether the object is local or remote.

\section{Value classes}\index{class!value class}\label{ValueClasses}

{}\Xten{} singles out a certain set of classes for additional
support. A class is said to be {\em stateless} if all of its fields
are declared to be \xcd"final" (\Sref{FinalVariable}), otherwise it
is {\em stateful}. (\Xten{} has syntax for specifying an array class
with final fields, unlike \java{}.) A {\em stateless (stateful)
object} is an instance of a stateless (stateful) class.

{}\Xten{} allows the programmer to signify that a class (and all its
descendents) are stateless. Such a class is called a {\em value
class}.  The programmer specifies a value class by prefixing the
modifier \xcd"value" before the keyword \xcd"class" in a class
declaration.  (A class not declared to be a value class will be called
a {\em reference class}.)  Each instance field of a value class is
treated as \xcd"final". It is legal (but neither required nor recommended)
for fields in a value class to be declared final. For brevity, the \Xten{}
compiler allows the programmer to omit the keyword \xcd"class" after
\xcd" value" in a value class declaration.


\begin{grammar}
ValueClassDeclaration \:
      ClassModifiers\opt \xcd"value" \xcd"class"\opt Identifier  \\
   && TypePropertyList\opt PropertyList\opt Guard\opt \\
   && Super\opt Interfaces\opt ValueClassBody \\
\end{grammar}


The \xcd"Box" type constructor (\Sref{BoxType}) can
be used to declare variables whose value may be \xcd"null" or a value
type.

Stable equality for value types is defined through a deep walk,
bottoming out in fields of reference types (\Sref{StableEquality}).

\begin{staticrule*}
It is a compile-time error for a value class to inherit from a
stateful class or for a reference class to inherit from a value
class. All fields of a value class are implicitly declared \xcd"final".
\end{staticrule*}

\subsection{Representation}

Since value objects do not contain any updatable locations, they can
be freely copied from place to place. An implementation may use
copying techniques even within a place to implement value types,
rather than references. This is transparent to the programmer.

More explicitly, \Xten{} guarantees that an implementation must always
behave as if a variable of a reference type takes up as much space as
needed to store a reference that is either null or is bound to an
object allocated on the (appropriate) heap. However, \Xten{} makes no
such guarantees about the representation of a variable of value
type. The implementation is free to behave as if the value is stored
``inline'', allocated on the heap (and a reference stored in the
variable) or use any other scheme (such as structure-sharing) it may
deem appropriate. Indeed, an implementation may even dynamically
change the representation of an object of a value type, or dynamically
use different representations for different instances (that is,
implement automatic box/unboxing of values).

Implementations are strongly encouraged to implement value types as
space-efficiently as possible (e.g., inlining them or passing them in
registers, as appropriate).  Implementations are expected to cache
values of remote final value variables by default. If a value is
large, the programmer may wish to consider spawning a remote activity
(at the place the value was created) rather than referencing the
containing variable (thus forcing it to be cached).

\oldtodo{Need to figure out whether we should let the programmer be
aware of lazy pull vs full-value push of value objects. This is the
idea of introducing a *-annotation. Need to make a decision on
this. Could leave this for 0.7.}

\begin{example}
A functional \xcd"LinkedList" program may be written as follows:


\begin{xten}
value LinkedList { 
  val first: Object;
  val rest: LinkedList;
  public def this(first: Object) {
     this(first, null);
  }
  public def this(first: Object, rest: LinkedList) {
    this.first = first;
    this.rest = rest;
  }
  public def first(): Object {
    return first;
  }
  public def rest(): LinkedList {
    return rest;
  } 
  public def append(l: LinkedList): LinkedList {
    return (this.rest == null) 
        ? new LinkedList(this.first, l) 
        : this.rest.append(l);
  }
}
\end{xten}

Similarly, a \xcd"Complex" class may be implemented as follows:
\begin{xten}
value Complex { 
  re: Double;
  im: Double;
  public def this(re: Double, im: Double) {
     this.re=re;
     this.im=im;
  }
  public def add(other: Complex): Complex {
    return new Complex(this.re+other.re,
                       this.im+other.im);
  }
  public def mult(other: Complex): Complex {
    return new Complex(this.re^2-other.re^2,
                       2*this.im*other.im);
  }
  ...
}
\end{xten}
\end{example}

\section{Type invariants}
\index{type invariants}
\index{guards}

There is a general recipe for constructing a list of parameters or
properties \xcdmath"x$_1$: T$_1${c$_1$}, $\dots$, x$_k$: T$_k${c$_k$}" that must satisfy a given
(satisfiable) constraint \xcd"c". 

\begin{xtenmath}
class Foo(x$_1$: T1{x$_2$: T$_2$; $\dots$; x$_k$: T$_k$; c},
          x$_2$: T2{x$_3$: T$_3$; $\dots$; x$_k$: T$_k$; c},
          $\dots$
          x$_k$: T$_k${c}) {
  $\dots$
}
\end{xtenmath}

The first type \xcdmath"x$_1$: T$_1${x$_2$: T$_2$; $\dots$; x$_k$: T$_k$; c}" is consistent iff
\xcdmath"$\exists$x$_1$: T$_1$, x$_2$: T$_2$, $\dots$, x$_k$: T$_k$. c" is consistent. The second is
consistent iff
\begin{xtenmath}
$\forall$x$_1$: T$_1${x$_2$: T$_2$; $\dots$; x$_k$: T$_k$; c}
$\exists$x$_2$: T$_2$. $\exists$x$_3$: T$_3$, $\dots$, x$_k$: T$_k$. c
\end{xtenmath}
\noindent But this is always true. Similarly for the conditions for the other
properties.

Thus logically every satisfiable constraint \xcd"c" on a list of parameters
\xcdmath"x$_1$", \dots, \xcdmath"x$_k$"
can be expressed using the dependent types of 
\xcdmath"x$_i$", provided
that the constraint language is rich enough to permit existential
quantifiers.

Nevertheless we will find it convenient to permit the programmer to
explicitly specify a depclause after the list of properties, thus:
\begin{xten}
class Point(i: Int, j: Int) { ... }
class Line(start: Point, end: Point){end != start}
  = { ... }
class Triangle (a: Line, b: Line, c: Line)
        {a.end == b.start && b.end == c.start &&
         c.end == a.start} = { ... }
class SolvableQuad(a: Int, b: Int, c: Int)
                   {a*x*x+b*x+c==0} = { ... }
class Circle (r: Int, x: Int, y: Int)
              {r > 0 && r*r==x*x+y*y} = { ... }
class NonEmptyList extends List{n > 0} {...}
\end{xten}

Consider the definition of the class \xcd"Line". This may be thought of as
saying: the class \xcd"Line" has two fields, \xcd"start: Point" and
\xcd"end: Point".
Further, every instance of \xcd"Line" must satisfy the constraint that
\xcd"end != start". Similarly for the other class definitions. 

In the general case, the production for \grammarrule{NormalClassDeclaration}
specifies that the list of properties may be followed by a
\grammarrule{Guard}.

\begin{grammar}
NormalClassDeclaration \:
      ClassModifiers\opt \xcd"class" Identifier  \\
   && TypeParameterList\opt PropertyList\opt Guard\opt \\
   && Extends\opt Interfaces\opt ClassBody \\
\\
NormalInterfaceDeclaration \:
      InterfaceModifiers\opt \xcd"interface" Identifier  \\
   && TypeParameterList\opt PropertyList\opt Guard\opt \\
   && ExtendsInterfaces\opt InterfaceBody \\
\end{grammar}

All the properties in the list, together with inherited properties,
may appear in the \grammarrule{Guard}. A guard \xcd"c" with
value property list \xcdmath"x$_1$: T$_1$, $\dots$, x$_n$: T$_n$"
for a class \xcd"C" is said to be consistent if each of the \xcdmath"T$_i$" are
consistent and the constraint
\begin{xtenmath}
$\exists$x$_1$: T$_1$, $\dots$, x$_n$: T$_n$, self: C. c
\end{xtenmath}
\noindent is valid (always true).

\section{Class definitions}

Consider a class definition
\begin{xtenmath}
$\mbox{\emph{ClassModifiers}}^{\mbox{?}}$
class C(x$_1$: P$_1$, $\dots$, x$_n$: P$_n$) extends D{d}
   implements I$_1${c$_1$}, $\dots$, I$_k${c$_k$}
$\mbox{\emph{ClassBody}}$
\end{xtenmath}

Each of the following static semantics rules must be satisfied:

\begin{staticrule}{Int-implements}
The type invariant \xcdmath"$\mathit{inv}$(C)" of \xcd"C" must entail
\xcdmath"c$_i$[this/self]" for each $i$ in $\{1, \dots, k\}$
\end{staticrule}

\begin{staticrule}{Super-extends}
The return type \xcd"c" of each constructor in \grammarrule{ClassBody}
must entail \xcd"d".
\end{staticrule}

\section{Constructor definitions}

A constructor for a class \xcd"C" is guaranteed to return an object of the
class on successful termination. This object must satisfy i(C), the
class invariant associated with \xcd"C" (\Sref{DepType:TypeInvariant}).
However,
often the objects returned by a constructor may satisfy {\em stronger}
properties than the class invariant. \Xten{}'s dependent type system
permits these extra properties to be asserted with the constructor in
the form of a constrained type (the ``return type'' of the constructor):

\begin{grammar}
ConstructorDeclarator \:
  \xcd"def" \xcd"this" TypeParameterList\opt \xcd"(" FormalParameterList\opt \xcd")" \\
  && ReturnType\opt Guard\opt Throws\opt \\
ReturnType    \: \xcd":" Type \\
Guard   \: "{" DepExpression "}" \\
Throws    \: \xcd"throws" ExceptionType  ( \xcd"," ExceptionType )\star \\
ExceptionType \: ClassBaseType Annotation\star \\
\end{grammar}

\label{ConstructorGuard}

The parameter list for the constructor
may specify a \emph{guard} that is to be satisfied by the parameters
to the list.

\begin{example}
Here is another example.
\begin{xten}
public class List[T](n: Int{n >= 0}) {
    protected head: Box[T];
    protected tail: List[T](n-1);
    public def this(o: T, t: List[T]) : List[T](t.n+1) = {
        n = t.n+1;
        tail = t;
        head = o;
    }
    public def this() : List[T](0) = {
        n = 0;
        head = null;
        tail = null;
    }
    ...
}
\end{xten}
The second constructor returns a \xcd"List" that is guaranteed to have
\xcd"n==0";
the first constructor is guaranteed to return a List with \xcd"n>0"
(in fact, \xcd"n==t.n+1", where the argument to the constructor is \xcd"t"). 
This is recorded by the programmer in the constrained type associated with the
constructor.
\end{example}

\begin{staticrule}{Super-invoke}
   Let \xcd"C" be a class with properties
   \xcdmath"p$_1$: P$_1$, $\dots$, p$_n$: P$_n$", invariant \xcd"c"
   extending the constrained type \xcd"D{d}" (where \xcd"D" is the name of a class).

   For every constructor in \xcd"C" the compiler checks that the call to
   super invokes a constructor for \xcd"D" whose return type is strong enough
   to entail \xcd"d". Specifically, if the call to super is of the form 
     \xcdmath"super(e$_1$, $\dots$, e$_k$)"
   and the static type of each expression \xcdmath"e$_i$" is
   \xcdmath"S$_i$", and the invocation
   is statically resolved to a constructor
\xcdmath"def this(x$_1$: T$_1$, $\dots$, x$_k$: T$_k$){c}: D{d$_1$}"
   then it must be the case that 
\begin{xtenmath}
x$_1$: S$_1$, $\dots$, x$_i$: S$_i$ $\vdash$ x$_i$: T$_i$  (for $i \in \{1, \dots, k\}$)
x$_1$: S$_1$, $\dots$, x$_k$: S$_k$ $\vdash$ c  
d$_1$[a/self] && x$_1$: S$_1$ ... && x$_k$: S$_k$ $\vdash$ d[a/self]      
\end{xtenmath}
\noindent where \xcd"a" is a constant that does not appear in 
\xcdmath"x$_1$: S$_1$ $\wedge$ ... $\wedge$ x$_k$: S$_k$".
  
\end{staticrule}

\begin{staticrule}{Constructor return}
   The compiler checks that every constructor for \xcd"C" ensures that
   the properties \xcdmath"p$_1$,..., p$_n$" are initialized with values which satisfy
   \xcdmath"t(C)", and its own return type \xcd"c'" as follows.  In each constructor, the
   compiler checks that the static types \xcdmath"T$_i$" of the expressions \xcdmath"e$_i$"
   assigned to \xcdmath"p$_i$" are such that the following is
   true:
\begin{xtenmath}
p$_1$: T$_1$, $\dots$, p$_n$: T$_n$ $\vdash$ t(C) $\wedge$ c'     
\end{xtenmath}
\end{staticrule}
(Note that for the assignment of \xcdmath"e$_i$" to \xcdmath"p$_i$"
to be type-correct it must be the
    case that \xcdmath"p$_i$: T$_i$ $\wedge$ p$_i$: P$_i$".) 


\begin{staticrule}{Constructor invocation}
The compiler must check that every invocation \xcdmath"C(e$_1$, $\dots$, e$_n$)" to a
constructor is type correct: each argument \xcdmath"e$_i$" must have a static type
that is a subtype of the declared type \xcdmath"T$_i$" for the $i$th
argument of the
constructor, and the conjunction of static types of the argument must
entail the \grammarrule{Guard} in the parameter list of the constructor.
\end{staticrule}

\section{Field definitions}

Not every instance of a class needs to have every field defined on the
class. In Java-like languages this is ensured by conditionally setting
fields to a default value, such as \xcd"null", in those instances where the
fields are not needed.  

Consider the class \xcd"List" used earlier.  Here all instances of \xcd"List"
returned by the second constructor do not need the fields \xcd"value" and
\xcd"tail"; their value is set to null.

\label{FieldGuard}

\Xten{} permits a much cleaner solution that does not require default
values such as null to be stored in such fields. \Xten{} permits fields to
be {\em guarded} with a constraint.  The field is accessible
only if the \emph{guard} constraint is satisified.

\begin{grammar}
FieldDeclaration  \:
   FieldModifiers\opt \xcd"val" VariableDeclarators \xcd";" \\
   \|
   FieldModifiers\opt \xcd"var" VariableDeclarators \xcd";" \\
VariableDeclarators \:
        VariableDeclarator ( \xcd"," VariableDeclarator )\star \\
VariableDeclarator \:
   Identifier ( Constraint )\opt ( \xcd":" Type )\opt ( \xcd"=" Expression )\opt \\
\end{grammar}

It is illegal for code to access a guarded field through a reference
whose static type does not satisfy the associated guard, even
implicitly (i.e., through an implicit \xcd"this"). Rather the source
program should contain an explict cast, e.g., \xcd"me: C{c} = this as C{c}".

\begin{staticrule*}
Let \xcd"f" be a field defined in class
\xcd"C" with guard \xcd"c".  The compiler declares an error if
field \xcd"f" is accessed through a reference \xcd"o" whose static
type is not a subtype of \xcd"C{c}".
\end{staticrule*}

\begin{example}

We may now rewrite the List example:
\begin{xten}
public class List(n: Int{n>=0}) {
  protected val head{n>0}: Object;
  protected val tail{n>0}: List(n-1);
  public def this(o: Object, t: List): List(t.n+1) {
     property(t.n+1);
     head=o;
     tail=t;
  }
  public def this(): List(0) {
     property(0);
  }
  ...
}
\end{xten}

The fields \xcd"value" and \xcd"tail" do not exist for instances of the class
\xcd"List(0)".
\end{example}

It is a compile-time error for a class to have two fields of the same
name, even if their guards are different. A class \xcd"C" with a field
named \xcd"f" is said to {\em hide} a field in a superclass named \xcd"f".

\begin{staticrule*}
     A class may not declare two fields with the same name.
\end{staticrule*}

To avoid an ambiguity, it is a static error for a class to
declare a field with a function type (\Sref{FunctionTypes}) with
the same name and signature  as a method of the same class.

\subsection{Field hiding}

A subclass that defines a field \xcd"f" hides any field \xcd"f"
declared in a superclass, regardless of their types.  The
superclass field \xcd"f" may be accessed within the body of
the subclass via the reference \xcd"super.f".

\eat{
The definition of field hiding does not take
\grammar{Guard} into
account. Suppose a class \xcd"C" has a field

\begin{xten}
var f{c}: Foo;
\end{xten}
\noindent and a subclass \xcd"D" of \xcd"C" has a field
\begin{xten}
var f{d}: Fum;
\end{xten}

We will say that \xcd"D.f" hides \xcd"C.f", {\em regardless} of the
constraints \xcd"c" and \xcd"d". This is in keeping with \Java, and
permits a naive implementation which always allocates space for a
conditional field.

\begin{rationale}
It might seem attractive to require that \xcd"D.f"
hides \xcd"C.f" only if \xcd"d" implies \xcd"c". This would seem
to necessitate a rather complex implementation structure for classes,
requiring some kind of a heterogenous translation for
constrained types of \xcd"C"
and \xcd"D". This bears further investigation.
\end{rationale}
}

\section{Method definitions}

\Xten{} permits guarded method definitions, similar to guarded
field definitions. Additionally, the parameter list for a method may
contain a \grammarrule{Guard}.

\begin{grammar}
MethodDeclaration \: MethodHeader \xcd";" \\
                  \| MethodHeader \xcd"=" ClosureBody \\
MethodHeader \:  
  MethodModifiers\opt \xcd"def" Identifier TypeParameters\opt \\
&& \xcd"(" 
  FormalParameterList\opt \xcd")" Guard\opt \\
  && ReturnType\opt Throws\opt \\
\end{grammar}

In the formal parameter list, variables may be declared with
\xcd"val" or \xcd"var".  If neither is specified, the variable
is \xcd"val".

\label{MethodGuard}

The guard (specified by \grammarrule{Guard})
specifies a constraint \xcd"c" on the
properties of the class \xcd"C" on which the method is being defined. The
method exists only for those instances of \xcd"C" which satisfy \xcd"c".  It is
illegal for code to invoke the method on objects whose static type is
not a subtype of \xcd"C{c}".

\begin{staticrule*}
    The compiler checks that every method invocation
    \xcdmath"o.m(e$_1$, $\dots$, e$_n$)"
    for a method is type correct. Each each argument
    \xcdmath"e$_i$" must have a
    static type \xcdmath"S$_i$" that is a subtype of the declared type
    \xcdmath"T$_i$" for the $i$th
    argument of the method, and the conjunction of static types
    of the arguments must entail the guard in the parameter list
    of the method.

    The compiler checks that in every method invocation
    \xcdmath"o.m(e$_1$, $\dots$, e$_n$)"
    the static type of \xcd"o", \xcd"S", is a subtype of \xcd"C{c}", where the method
    is defined in class \xcd"C" and the guard for \xcd"m" is equivalent to
    \xcd"c".

    Finally, if the declared return type of the method is
    \xcd"D{d}", the
    return type computed for the call is
    \xcdmath"D{a: S; x$_1$: S$_1$; $\dots$; x$_n$: S$_n$; d[a/this]}",
    where \xcd"a" is a new
    variable that does not occur in
    \xcdmath"d, S, S$_1$, $\dots$, S$_n$", and
    \xcdmath"x$_1$, $\dots$, x$_n$" are the formal
    parameters of the method.
\end{staticrule*}

\begin{example}
Consider the program:
\begin{xten}
public class List(n: Int{n>=0}) {
  protected val head{n>0}: Object;
  protected val tail{n>0}: List(n-1);
  public def this(o: Object, t: List): List(t.n+1) = {
     property(t.n+1);
     head=o;
     tail=t;
  }
  public def this(): List(0) = {
     property(0);
  }
  public def append(l: List): List{self.n==this.n+l.n} = {
      return (n==0)? l
         : new List(head, tail.append(l)); 
  }
  public def nth(k: Int{1 <= k && k <= n}){n > 0}: Object = {
      return k==1 ? head : tail.nth(k-1);
  }
}
\end{xten}

The following code fragment
\begin{xten}
u: List{self.n==3} = ...
t: List{self.n==x} = ...;
s: List{self.n==x+3} = t.append(u);
\end{xten}
\noindent will typecheck. The type of the expression \xcd"t.append(u)" is 
\begin{xten}
List{a: List{self.n==x}; 
     l: List{self.n==3}; self.n==a.n+l.n}  
\end{xten}
\noindent which is equivalent to:
\begin{xten}
List{self.n==x+3}
\end{xten}
\end{example}

The method body is either an expression, a block of statements,
or a block ending with an expression.

\subsection{Property methods}

A method declared with the modifier \xcd"property" may be used
in constraints.  A property method declared in a class must have
a body and must not be \xcd"void".  The body of the method must
consist of only a single \xcd"return" statement or a single
expression.  It is a static error of the expression cannot be
represented in the constraint system.

Property methods in classes are implicitly \xcd"final"; they cannot be
overridden.

A property method definition may omit the formal parameters and
the \xcd"def" keyword.  That is, the following are equivalent:

\begin{xten}
property def rail(): boolean = rect && onePlace == here && zeroBased;
property rail: boolean = rect && onePlace == here && zeroBased;
\end{xten}

\subsection{Method overloading, overriding, hiding, shadowing and obscuring}
\label{MethodOverload}

The definitions of method overloading, overriding, hiding, shadowing
and obscuring in \Xten{} are the same as in \Java, modulo the following
considerations motivated by type parameters and dependent types.

Two or more methods of a class or interface may have the same
name if they have a different number of type parameters, or
they have value parameters of different types.

The definition of a method declaration \xcdmath"m$_1$" ``having the same signature
as'' a method declaration \xcdmath"m$_2$" involves identity of types. Two \Xten{} types
are defined to be identical iff they are equivalent (\Sref{DepType:Equivalence}).
Two methods are said to have {\em the same signature} if (a)
they have the same number of type parameters, (b) they have the
same number of formal (value) parameters, and (c) for each formal parameter
their types are equivalent. It is a compile-time error for there
to be two methods with the same name and same signature in a class
(either defined in that class or in a superclass).

\begin{staticrule*}
  A class \xcd"C" may not have two declarations for a method named \xcd"m"---either
  defined at \xcd"C" or inherited:
\begin{xtenmath}
def m[X$_1$, $\dots$, X$_m$](v$_1$: T$_1${t$_1$}, $\dots$, v$_n$: T$_n${t$_n$}){tc}: T {...}
def m[X$_1$, $\dots$, X$_m$](v$_1$: S$_1${s$_1$}, $\dots$, v$_n$: S$_n${s$_n$}){sc}: S {...}
\end{xtenmath}
\noindent
if it is the case that the types \xcd"C{tc}", \xcdmath"T$_1${t$_1$}",
\dots, \xcdmath"T$_n${t$_n$}" are
equivalent to the types \xcdmath"C{sc}, S$_1${t$_1$}, $\dots$, T$_n${t$_n$}"
respectively.
\end{staticrule*}

In addition, the guard of a overriding method must be 
no stronger than the guard of the overridden method.   This
ensures that any virtual call to the method
satisfies the guard of the callee.

\begin{staticrule*}
  If a class \xcd"C" overrides a method of a class or interface
  \xcd"B", the guard of the method in \xcd"B" must entail
  the guard of the method in \xcd"C".
\end{staticrule*}

A class \xcd"C" inherits from its direct superclass and superinterfaces all
their methods visible according to the access restriction modifiers
public/private/protected/(package) of the superclass/superinterfaces
that are not hidden or overridden. A method \xcdmath"M$_1$" in a class
\xcd"C" overrides
a method \xcdmath"M$_2$" in a superclass \xcd"D" if
\xcdmath"M$_1$" and \xcdmath"M$_2$" have the same signature.
Methods are overriden on a signature-by-signature basis.

A method invocation \xcdmath"o.m(e$_1$, $\dots$, e$_n$)"
is said to have the {\em static signature}
\xcdmath"<T, T$_1$, $\dots$, T$_n$>" where \xcd"T" is the static type of
\xcd"o", and
\xcdmath"T$_1$",
\dots,
\xcdmath"T$_n$"
are the static types of \xcdmath"e$_1$", \dots, \xcdmath"e$_n$",
respectively.  As in
\Java, it must be the case that the compiler can determine a single
method defined on \xcd"T" with argument type
\xcdmath"T$_1$", \dots \xcdmath"T$_n$"; otherwise, a
compile-time error is declared. However, unlike \Java, the \Xten{} type \xcd"T"
may be a dependent type \xcd"C{c}". Therefore, given a class definition for
\xcd"C" we must determine which methods of \xcd"C" are available at a type
\xcd"C{c}". But the answer to this question is clear: exactly those methods
defined on \xcd"C" are available at the type \xcd"C{c}"
whose guard \xcd"d" is implied by \xcd"c".


\begin{example}
  Consider the definitions:
\begin{xten}
class Point(i: Int, j: Int) {...}
class Line(s: Point, e: Point{self != i}) {
  // m1: Both points lie in the right half of the plane
  def draw(){s.i>= 0 && e.i >= 0} = {...}
  // m2: Both points lie on the y-axis
  def draw(){s.i== 0 && e.i == 0} = {...}
  // m3: Both points lie in the top half of the plane
  def draw(){s.j>= 0 && e.j >= 0} = {...}
  // m4: The general method
  def draw() = {...}
} 
\end{xten} 
\noindent  Three different implementations are given for the
\xcd"draw" method, one
  for the case in which the line lies in the right half of the plane,
  one for the case that the line lies on the y-axis and the third for
  the case that the line lies in the top half of the plane.


\noindent  Consider the invocation
\begin{xten}
m: Line{s.i < 0} = ...
m.draw();
\end{xten}

\noindent  This generates a compile time error because there is no applicable
  method definition.

\noindent  Consider the invocation

\begin{xten}
m: Line{s.i>=0 && s.j>=0 && e.i>=0 && e.j>=0} = ...
m.draw();
\end{xten}

\noindent  This generates a compile time error because both
\xcd"m1" and \xcd"m3" are applicable.

\noindent  Consider the invocation
\begin{xten}
m: Line{s.i>=0 && s.j>=0 && e.i>=0} = ...
m.draw();
\end{xten}
  This does not generate any compile-time error since only m1 is
  applicable. 
\end{example}


In the last example, notice that at runtime \xcd"m1" will be invoked
(assuming \xcd"m" contains an instance of the \xcd"Line" class, and not some
subclass of \xcd"Line" that overrides this method). This will be the case
even if \xcd"m" satisfies at runtime the stronger conditions for \xcd"m2" (i.e.,
\xcd"s.i==0 && e.i==0"). That is, dynamic method lookup will not take into
account the  ``strongest'' constraint that the receiver may
satisfy, i.e.,
its ``strongest constrained type''. 

\begin{rationale}
  The design decision that dynamic method lookup should ignore
  dependent type information was made to keep the design and the
  implementation simple and to ensure that serious errors such as
  method invocation errors are captured at compile-time.
 
  Consider the above example and the invocation
\begin{xten}
m: Line = ...
m.draw();    
\end{xten}


   Statically the compiler will not report an error because m4 is the
   only method that is applicable. However, if dynamic method lookup
   were to use constrained types then we would face the problem that if m is a
   line that lives in the upper right quadrant then both \xcd"m2"
   and \xcd"m3"
   are applicable and one does not override the other. Hence we must
   report an error dynamically.

   As discussed above, the programmer can write code with \xcd"instanceof"
   and class casts that perform any application-appropriate
   discrimination.  
\end{rationale}

\subsection{Method annotations}

\subsubsection{\Xcd{atomic} annotation}

A method may be declared \xcd"atomic".

\begin{grammar}
  MethodModifier \: \xcd"atomic"  
\end{grammar}

Such a method is treated as if the statement in its body is wrapped 
implicitly in an \xcd"atomic" statement.

\subsubsection{\Xcd{local} annotation}\label{LocalAnnotation}\index{local!\xcd"local"}

A method may be declared \xcd"local".

\begin{grammar}
  MethodModifier \: \xcd"local"  
\end{grammar}

By declaring a method \xcd"local" the programmer asserts that while
executing this method an activity will only access local memory.

The compiler implements the following rules to guarantee this property.

Let \xcd"o" be any expression occurring in the body of the
method. Assume its static datatype is \xcd"F". 

\begin{itemize}
\item Local methods can only be overridden by local methods. 

\item If the body of the method contains any field access \xcd"o.e", then
the static placetype of \xcd"o" must be \xcd"here". 

The programmer can always ensure that this condition is satisfied
(albeit at the risk of introducing a runtime exception) by replacing
each field access \xcd"o.e" with \xcd"(o as F!here).e".

\item If the body of the method contains any assignments to fields
(e.g. \xcd"o.e Op= t", or \xcd"Op o.e" or \xcd"o.e Op") then the
static placetype of \xcd"o" must be \xcd"here".

The programmer can always ensure that this condition is satisfied by
replacing \xcd"o.e Op= t" by \xcd"o1.e Op=t" and preceding it (in the
same basic block) with the local variable declaration \xcd"o1: F!here = o as F!here" (for some new local variable \xcd"o1"). Similarly for
\xcd"Op o.e" and \xcd"o.e Op".

\item Recall that the static placetype of an array access \xcd"o(e)"
is \xcd"o.dist(e)". Therefore, any read/write array access
\xcd"o(e)" must be guarded by the condition \xcd"o.dist(e) == here".  (Since  \xcd"e" may have side-effects, the compiler must
ensure that the place check uses the value returned by the same
expression evaluation that is used to access the array element.)

\item If the body of the method contains any method invocation
\xcdmath"o.m(t$_1$,$\dots$,t$_k$)" then the method invoked must be local. Additionally,
the static place type of \xcd"o" must be \xcd"here". 
As above, the programmer can always ensure the second
condition is satisfied by writing such a method invocation
as \xcdmath"(o as F!here).m(t$_1$,$\dots$,t$_k$)".
\end{itemize}

Note that reads/writes to local variables or method parameters are
always local, hence the compiler does not have to check any extra
conditions.

A method declared \xcd"atomic" is automatically declared
to be \xcd"local".
