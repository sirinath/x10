%III. Applied constrained calculii. (3 pages)
%
%For each example below, formal static and dynamic semantics rules for
%new constructs extension over the core CFJ. Subject-reduction and
%type-soundness theorems. Proofs to be found in fuller version of
%paper.
%
%(a) arrays, region, distributions -- type safe implies no arrayoutofbounds
%exceptions, only ClassCastExceptions (when dynamic checks fail).
%
%Use Satish's conditional constraints example.
%-- emphasize what is new over DML. 
%
%(b) places, concurrency -- place types.
%
%(c) ownership types, alias control.
%

The following section presents examples using several different
constraint systems.

In the following we will use the shorthand $\tt C(\bar{t}:c)$ for the
type $\tt C(:\bar{f}=\bar{t},c)$ where the declaration of the class
{\tt C} is $\tt \class\ C(\bar{\tt T}\ \bar{\tt f}:c)\ldots$.  Also,
we abbreviate $\tt C(\bar{t}:\true)$ as $\tt C(\bar{t})$.  Finally, we
will also have need to use the shorthand ${\tt C}_1(\bar{t}_:{\tt
c}_1)\& \ldots {\tt C}_k(\bar{\tt t}_k:{\tt c}_k)$ for the type
${\tt C}_1(:\bar{\tt f}_1=\bar{\tt t}_1, \ldots, \bar{\tt
f}_k=\bar{\tt t}_k,{\tt c}_1,\ldots,{\tt c}_k$ 
provided that the ${\tt C}_i$ form a subtype chain
and the declared fields of ${\tt C}_i$ are ${\tt f}_i$.

Constraints naturally allow for partial specification
(e.g. inequalities) or incomplete specification (no constraint on a
variable) with the same simple syntax. In the example below,
the type of {\tt a} does not place any constraint on the second
dimension of {\tt a}, but this dimension can be used in other
types (e.g., the return type).
{\footnotesize
\begin{verbatim}
  class Matrix(int m, int n) {
    Matrix(m,a.n) mul(Matrix(:m=this.n) a) {...}
    ...
  }
\end{verbatim}}

Constraints also naturally permit the expression of existential types:
{\footnotesize
\begin{verbatim}
  class List(int length) { 
    List(:self.length <= length) filter(Comparator k) {...} 
    ...
  }
\end{verbatim}}
\noindent
Here, the length of the list returned by the "filter" method is 
unknown, but is bound by the length of the original list.

\subsection{X10}

\subsection{Self types and binary methods}

Self types~\cite{bsg95,bfp-ecoop97-match} can be implemented
using a {\tt klass} property on objects.  The {\tt klass}
property represents the run-time class of the object.
Self types can be used to solve the binary method problem \cite{bruce-binary}.

In the example below, the {\tt Set} interface has a "union" method
whose argument must be of the same class as {\tt this}.
\noindent This enables the {\tt IntSet} class's {\tt union}
method to access the {\tt bits} field of its argument {\tt s}.
{\footnotesize
\begin{verbatim}
  interface Set(:Class klass) {
    Set(this.klass) union(Set(this.klass) s);
  }
  class IntSet(:Class klass) implements Set(klass) {
    long bits;

    IntSet(IntSet.class)() { property(IntSet.class); }

    IntSet(IntSet.class)(int(:0 <= self, self <= 63) i) {
      property(IntSet.class);
      bits = 1 << i; }

    Set(this.klass) union(Set(this.klass) s) {
      IntSet(this.klass) r = new IntSet(this.klass);
      r.bits = this.bits | s.bits;
      return r; }
  }
\end{verbatim}}
\noindent
The key to ensuring that this code type-checks is the
\rn{T-constr}
rule.
With a constraint system ${\cal C}_{\mathsf{klass}}$ aware of
the {\tt klass} property, the rule 
\rn{T-var} is used to subsume an expression of type
${\tt Set(this.class)}$ to type ${\tt IntSet(this.class)}$
when {\tt this} is known to be an {\tt IntSet}:
{\footnotesize
\[
\from{\begin{array}{c}
{\tt IntSet}~{\tt this}, {\tt Set}({\tt this.klass})~{\tt s}
        \vdash {\tt Set}({\tt this.klass})~{\tt s} \\
{\tt IntSet}~{\tt this}, {\tt Set}({\tt this.klass})~{\tt s}
        \vdash_{{\cal C}_{\mathsf{klass}}} {\tt IntSet}({\tt this.klass})~{\tt s} \\
\end{array}}
\infer{
{\tt IntSet}~{\tt this}, {\tt Set}({\tt this.klass})~{\tt s}
        \vdash {\tt IntSet}({\tt this.klass})~{\tt s}}
\]}

\subsection{AVL trees and red--black trees}

AVL trees can be modeled so that the data structure invariant is
enforced statically.

{\footnotesize
\begin{verbatim}
class AVLList(int(:self >= 0) height) {...}
class Leaf(Object key) extends AVLList(0) {...}
class Node(Object key, AVLList l, AVLList r
         : int d=l.height-r.height; -1 <= d, d <= 1) 
    extends AVLList(max(l.height,r.height)+1) { ... }
\end{verbatim}}

Red/black trees may be modeled similarly. Such trees have the
invariant that (a) all leaves are black, (b) each non-leaf node has
the same number of black nodes on every path to a leaf (the black
height), (c) the immediate children of every red node are black.
{\footnotesize
\begin{verbatim}
class Tree(int blackHeight) {...}
class Leaf extends Tree(0) { int value; ...}
class Node(boolean isBlack, 
           Tree(:this.isBlack || isBlack) l, 
           Tree(:this.isBlack || isBlack,
                 blackHeight=l.blackHeight) r)
    extends Tree(l.blackHeight+1) { int value; ... }
\end{verbatim}}

\subsection{Bounds checks}

Xi and Pfenning proposed using dependent types for eliminating
array bounds checks~\cite{xi98array}.

In CFJ, an array of type {\tt T[]} indexed by (signed) integers
can be modeled as a class with the following
signature:\footnote{For this example, we assume the language supports generics.}
\begin{verbatim}
class Array<T>(int(:self >= 0) length) {
  T get(int(:0 <= self, self < this.length) i);
  void set(int(:0 <= self, self < this.length) i, T v);
}
\end{verbatim}

Constraint system based on Presburger arithmetic:
\begin{verbatim}
a ::= b < b | b = b
b ::= t | n | b*b | b+b
\end{verbatim}

Some code that iterates over an array (sugaring {\tt get} and {\tt set}):
\begin{verbatim}
double dot(double[] x, double[] y
         : x.length = y.length) {
  double r = 0.; 
  for (int(:self >= 0, self < x.length)
       i = 0; i < x.length; i++) {
    r += x[i] * y[i];
  }
  return r;
}
\end{verbatim}

\eat{
Another one:
\begin{verbatim}
double[](:length = x.length) saxpy(double a, double[] x, double[] y : x.length = y.length) {
    double[](:length = x.length) result = new double[x.length];
    for (int(:self >= 0, self < x.length) i = 0; i < x.length; i++) {
        result[i] = a * x[i] + y[i];
    }
    return result;
}
\end{verbatim}
}


\eat{
\subsection{Binary search}

An informal study by Jon Bentley~\cite{programming-pearls}
found that x\% of professional programmers attending in a course
could not correctly implement binary search.

Dependent types can help here by adding the invariants to the
index types.

\subsection{Quicksort}

\begin{verbatim}
int(:left <= self & self <= right)
partition(T[] array, int left, int right, int pivotIndex : left <= pivotIndex & pivotIndex <= right) {
     T pivotValue = array[pivotIndex];

     // Move pivot to end
     swap(array, pivotIndex, right);

     int(:left <= self & self <= right) storeIndex;
     storeIndex = left;
     for (int(:left <= self & self <= right-1) i = left; i < right; i++) {
         if (array[i] <= pivotValue) {
             swap(array, storeIndex, i);
             storeIndex++;
         }
     }

     // Move pivot to its final place
     swap(array, right, storeIndex)
     return storeIndex;
}

void swap(T[] array,
          int(:0 <= self & self < array.length i,
          int(:0 <= self & self < array.length j) {
    T tmp = array[i];
    array[i] = array[j];
    array[j] = tmp;
}

void quicksort(T[] array, int left, int right : left <= right) {
    if (left < right) {
         // select a pivot index
         int(:left <= self & self <= right) pivotIndex = (left + right) / 2;
         pivotNewIndex = partition(array, left, right, pivotIndex)
         quicksort(array, left, pivotNewIndex-1)
         quicksort(array, pivotNewIndex+1, right)
    }
}
\end{verbatim}
}

\subsection{Nullable types}

A constraint system that supports disequalities can be used to
enforce a non-null invariant on reference types.
A non-null type {\tt T} can be written simply as {\tt T(:self != null)}.

\subsection{A distributed binary tree}
This example is due to Satish Chandra. We wish to specify a balanced
distributed tree with the property that its right child is always at
the same place as its parent, and once the left child is at the same
place then the entire subtree is at that place:

{\footnotesize
\begin{verbatim}
class Tree(boolean localLeft,
  Tree(: this.localLeft => (loc=here,self.localLeft)) left, 
  Tree(: loc=here) right) extends Object { ...}
\end{verbatim}}

\subsection{Places}

% \begin{figure}
% \begin{tightcode}
\quad\num{1}/**
\quad\num{2}
\quad\num{3} * This class implements the notion of places in X10. The maximum
\quad\num{4} * number of places is determined by a configuration parameter
\quad\num{5} * (MAX\_PLACES). Each place is indexed by a nat, from 0 to MAX\_PLACES;
\quad\num{6} * thus there are MAX\_PLACES+1 places. This ensures that there is
\quad\num{7} * always at least 1 place, the 0'th place.
\quad\num{8}
\quad\num{9} * We use a dependent parameter to ensure that the compiler can track
\quad\num{10} * indices for places.
\quad\num{11} *
\quad\num{12} * Note that place(i), for i <= MAX\_PLACES, can now be used as a non-empty type.
\quad\num{13} * Thus it is possible to run an async at another place, without using arays---
\quad\num{14} * just use async(place(i)) \{\ldots{}\} for an appropriate i.
\quad\num{15}
\quad\num{16} * @author Christoph von Praun
\quad\num{17} * @author vj
\quad\num{18} */
\quad\num{19}
\quad\num{20}package x10.lang;
\quad\num{21}
\quad\num{22}import x10.util.List;
\quad\num{23}import x10.util.Set;
\quad\num{24}
\quad\num{25}public value class place (nat i : i <= MAX\_PLACES)\{
\quad\num{26}
\quad\num{27}    /** The number of places in this run of the system. Set on
\quad\num{28}     * initialization, through the command line/init parameters file.
\quad\num{29}     */
\quad\num{30}    config nat MAX\_PLACES;
\quad\num{31}
\quad\num{32}    \emph{// Create this array at the very beginning.}
\quad\num{33}    private constant place value [] myPlaces = new place[MAX\_PLACES+1] fun place (int i) \{
\quad\num{34}	return new place( i )(); \};
\quad\num{35}
\quad\num{36}    /** The last place in this program execution.
\quad\num{37}     */
\quad\num{38}    public static final place LAST\_PLACE = myPlaces[MAX\_PLACES];
\quad\num{39}
\quad\num{40}    /** The first place in this program execution.
\quad\num{41}     */
\quad\num{42}    public static final place FIRST\_PLACE = myPlaces[0];
\quad\num{43}    public static final Set<place> places = makeSet( MAX\_PLACES );
\quad\num{44}
\quad\num{45}    /** Returns the set of places from first place to last place.
\quad\num{46}     */
\quad\num{47}    public static Set<place> makeSet( nat lastPlace ) \{
\quad\num{48}	Set<place> result = new Set<place>();
\quad\num{49}	for ( int i : 0 .. lastPlace ) \{
\quad\num{50}	    result.add( myPlaces[i] );
\quad\num{51}	\}
\quad\num{52}	return result;
\quad\num{53}    \}
\quad\num{54}
\quad\num{55}    /**  Return the current place for this activity.
\quad\num{56}     */
\quad\num{57}    public static place here() \{
\quad\num{58}	return activity.currentActivity().place();
\quad\num{59}    \}
\quad\num{60}
\quad\num{61}    /** Returns the next place, using modular arithmetic. Thus the
\quad\num{62}     * next place for the last place is the first place.
\quad\num{63}     */
\quad\num{64}    public place(i+1 % MAX\_PLACES) next()  \{ return next( 1 ); \}
\quad\num{65}
\quad\num{66}    /** Returns the previous place, using modular arithmetic. Thus the
\quad\num{67}     * previous place for the first place is the last place.
\quad\num{68}     */
\quad\num{69}    public place(i-1 % MAX\_PLACES) prev()  \{ return next( -1 ); \}
\quad\num{70}
\quad\num{71}    /** Returns the k'th next place, using modular arithmetic. k may
\quad\num{72}     * be negative.
\quad\num{73}     */
\quad\num{74}    public place(i+k % MAX\_PLACES) next( int k ) \{
\quad\num{75}	return places[ (i + k) % MAX\_PLACES];
\quad\num{76}    \}
\quad\num{77}
\quad\num{78}    /**  Is this the first place?
\quad\num{79}     */
\quad\num{80}    public boolean isFirst() \{ return i==0; \}
\quad\num{81}
\quad\num{82}    /** Is this the last place?
\quad\num{83}     */
\quad\num{84}    public boolean isLast() \{ return i==MAX\_PLACES; \}
\quad\num{85}\}
\end{tightcode}

% \end{figure}

\subsection{$k$-dimensional regions}

% \begin{figure}
% \begin{tightcode}
\quad\num{1}package x10.lang;
\quad\num{2}
\quad\num{3}/** A region represents a k-dimensional space of points. A region is a
\quad\num{4} * dependent class, with the value parameter specifying the dimension
\quad\num{5} * of the region.
\quad\num{6} * @author vj
\quad\num{7} * @date 12/24/2004
\quad\num{8} */
\quad\num{9}
\quad\num{10}public final value class region( int dimension : dimension >= 0 )  \{
\quad\num{11}
\quad\num{12}    /** Construct a 1-dimensional region, if low <= high. Otherwise
\quad\num{13}     * through a MalformedRegionException.
\quad\num{14}     */
\quad\num{15}    extern public region (: dimension==1) (int low, int high)
\quad\num{16}        throws MalformedRegionException;
\quad\num{17}
\quad\num{18}    /** Construct a region, using the list of region(1)'s passed as
\quad\num{19}     * arguments to the constructor.
\quad\num{20}     */
\quad\num{21}    extern public region( List(dimension)<region(1)> regions );
\quad\num{22}
\quad\num{23}    /** Throws IndexOutOfBoundException if i > dimension. Returns the
\quad\num{24}        region(1) associated with the i'th dimension of this otherwise.
\quad\num{25}     */
\quad\num{26}    extern public region(1) dimension( int i )
\quad\num{27}        throws IndexOutOfBoundException;
\quad\num{28}
\quad\num{29}
\quad\num{30}    /** Returns true iff the region contains every point between two
\quad\num{31}     * points in the region.
\quad\num{32}     */
\quad\num{33}    extern public boolean isConvex();
\quad\num{34}
\quad\num{35}    /** Return the low bound for a 1-dimensional region.
\quad\num{36}     */
\quad\num{37}    extern public (:dimension=1) int low();
\quad\num{38}
\quad\num{39}    /** Return the high bound for a 1-dimensional region.
\quad\num{40}     */
\quad\num{41}    extern public (:dimension=1) int high();
\quad\num{42}
\quad\num{43}    /** Return the next element for a 1-dimensional region, if any.
\quad\num{44}     */
\quad\num{45}    extern public (:dimension=1) int next( int current )
\quad\num{46}        throws IndexOutOfBoundException;
\quad\num{47}
\quad\num{48}    extern public region(dimension) union( region(dimension) r);
\quad\num{49}    extern public region(dimension) intersection( region(dimension) r);
\quad\num{50}    extern public region(dimension) difference( region(dimension) r);
\quad\num{51}    extern public region(dimension) convexHull();
\quad\num{52}
\quad\num{53}    /**
\quad\num{54}       Returns true iff this is a superset of r.
\quad\num{55}     */
\quad\num{56}    extern public boolean contains( region(dimension) r);
\quad\num{57}    /**
\quad\num{58}       Returns true iff this is disjoint from r.
\quad\num{59}     */
\quad\num{60}    extern public boolean disjoint( region(dimension) r);
\quad\num{61}
\quad\num{62}    /** Returns true iff the set of points in r and this are equal.
\quad\num{63}     */
\quad\num{64}    public boolean equal( region(dimension) r) \{
\quad\num{65}        return this.contains(r) && r.contains(this);
\quad\num{66}    \}
\quad\num{67}
\quad\num{68}    \emph{// Static methods follow.}
\quad\num{69}
\quad\num{70}    public static region(2) upperTriangular(int size) \{
\quad\num{71}        return upperTriangular(2)( size );
\quad\num{72}    \}
\quad\num{73}    public static region(2) lowerTriangular(int size) \{
\quad\num{74}        return lowerTriangular(2)( size );
\quad\num{75}    \}
\quad\num{76}    public static region(2) banded(int size, int width) \{
\quad\num{77}        return banded(2)( size );
\quad\num{78}    \}
\quad\num{79}
\quad\num{80}    /** Return an \code\{upperTriangular\} region for a dim-dimensional
\quad\num{81}     * space of size \code\{size\} in each dimension.
\quad\num{82}     */
\quad\num{83}    extern public static (int dim) region(dim) upperTriangular(int size);
\quad\num{84}
\quad\num{85}    /** Return a lowerTriangular region for a dim-dimensional space of
\quad\num{86}     * size \code\{size\} in each dimension.
\quad\num{87}     */
\quad\num{88}    extern public static (int dim) region(dim) lowerTriangular(int size);
\quad\num{89}
\quad\num{90}    /** Return a banded region of width \{\code width\} for a
\quad\num{91}     * dim-dimensional space of size \{\code size\} in each dimension.
\quad\num{92}     */
\quad\num{93}    extern public static (int dim) region(dim) banded(int size, int width);
\quad\num{94}
\quad\num{95}
\quad\num{96}\}
\end{tightcode}

% \end{figure}

\subsection{Point}


% \begin{figure}
% \begin{tightcode}
\quad\num{1}package x10.lang;
\quad\num{2}
\quad\num{3}public final class point( region region ) \{
\quad\num{4}    parameter int dimension = region.dimension;
\quad\num{5}    \emph{// an array of the given size.}
\quad\num{6}    int[dimension] val;
\quad\num{7}
\quad\num{8}    /** Create a point with the given values in each dimension.
\quad\num{9}     */
\quad\num{10}    public point( int[dimension] val ) \{
\quad\num{11}        this.val = val;
\quad\num{12}    \}
\quad\num{13}
\quad\num{14}    /** Return the value of this point on the i'th dimension.
\quad\num{15}     */
\quad\num{16}    public int valAt( int i) throws IndexOutOfBoundException \{
\quad\num{17}        if (i < 1 || i > dimension) throw new IndexOutOfBoundException();
\quad\num{18}        return val[i];
\quad\num{19}    \}
\quad\num{20}
\quad\num{21}    /** Return the next point in the given region on this given
\quad\num{22}     * dimension, if any.
\quad\num{23}     */
\quad\num{24}    public void inc( int i )
\quad\num{25}        throws IndexOutOfBoundException, MalformedRegionException \{
\quad\num{26}        int val = valAt(i);
\quad\num{27}        val[i] = region.dimension(i).next( val );
\quad\num{28}    \}
\quad\num{29}
\quad\num{30}    /** Return true iff the point is on the upper boundary of the i'th
\quad\num{31}     * dimension.
\quad\num{32}     */
\quad\num{33}    public boolean onUpperBoundary(int i)
\quad\num{34}        throws IndexOutOfBoundException \{
\quad\num{35}        int val = valAt(i);
\quad\num{36}        return val == region.dimension(i).high();
\quad\num{37}    \}
\quad\num{38}
\quad\num{39}    /** Return true iff the point is on the lower boundary of the i'th
\quad\num{40}     * dimension.
\quad\num{41}     */
\quad\num{42}    public boolean onLowerBoundary(int i)
\quad\num{43}        throws IndexOutOfBoundException \{
\quad\num{44}        int val = valAt(i);
\quad\num{45}        return val == region.dimension(i).low();
\quad\num{46}    \}
\quad\num{47}\}
\quad\num{48}
\end{tightcode}

% \end{figure}

\subsection{Distribution}


% \begin{figure}
% \begin{tightcode}
\quad\num{1}package x10.lang;
\quad\num{2}
\quad\num{3}/** A distribution is a mapping from a given region to a set of
\quad\num{4} * places. It takes as parameter the region over which the mapping is
\quad\num{5} * defined. The dimensionality of the distribution is the same as the
\quad\num{6} * dimensionality of the underlying region.
\quad\num{7}
\quad\num{8}   @author vj
\quad\num{9}   @date 12/24/2004
\quad\num{10} */
\quad\num{11}
\quad\num{12}public final value class distribution( region region ) \{
\quad\num{13}    /** The parameter dimension may be used in constructing types derived
\quad\num{14}     * from the class distribution. For instance,
\quad\num{15}     * distribution(dimension=k) is the type of all k-dimensional
\quad\num{16}     * distributions.
\quad\num{17}     */
\quad\num{18}    parameter int dimension = region.dimension;
\quad\num{19}
\quad\num{20}    /** places is the range of the distribution. Guranteed that if a
\quad\num{21}     * place P is in this set then for some point p in region,
\quad\num{22}     * this.valueAt(p)==P.
\quad\num{23}     */
\quad\num{24}    public final Set<place> places; \emph{// consider making this a parameter?}
\quad\num{25}
\quad\num{26}    /** Returns the place to which the point p in region is mapped.
\quad\num{27}     */
\quad\num{28}    extern public place valueAt(point(region) p);
\quad\num{29}
\quad\num{30}    /** Returns the region mapped by this distribution to the place P.
\quad\num{31}        The value returned is a subset of this.region.
\quad\num{32}     */
\quad\num{33}    extern public region(dimension) restriction( place P );
\quad\num{34}
\quad\num{35}    /** Returns the distribution obtained by range-restricting this to Ps.
\quad\num{36}        The region of the distribution returned is contained in this.region.
\quad\num{37}     */
\quad\num{38}    extern public distribution(:this.region.contains(region))
\quad\num{39}        restriction( Set<place> Ps );
\quad\num{40}
\quad\num{41}    /** Returns a new distribution obtained by restricting this to the
\quad\num{42}     * domain region.intersection(R), where parameter R is a region
\quad\num{43}     * with the same dimension.
\quad\num{44}     */
\quad\num{45}    extern public (region(dimension) R) distribution(region.intersection(R))
\quad\num{46}        restriction();
\quad\num{47}
\quad\num{48}    /** Returns the restriction of this to the domain region.difference(R),
\quad\num{49}        where parameter R is a region with the same dimension.
\quad\num{50}     */
\quad\num{51}    extern public (region(dimension) R) distribution(region.difference(R))
\quad\num{52}        difference();
\quad\num{53}
\quad\num{54}    /** Takes as parameter a distribution D defined over a region
\quad\num{55}        disjoint from this. Returns a distribution defined over a
\quad\num{56}        region which is the union of this.region and D.region.
\quad\num{57}        This distribution must assume the value of D over D.region
\quad\num{58}        and this over this.region.
\quad\num{59}
\quad\num{60}        @seealso distribution.asymmetricUnion.
\quad\num{61}     */
\quad\num{62}    extern public (distribution(:region.disjoint(this.region) &&
\quad\num{63}                                dimension=this.dimension) D)
\quad\num{64}        distribution(region.union(D.region)) union();
\quad\num{65}
\quad\num{66}    /** Returns a distribution defined on region.union(R): it takes on
\quad\num{67}        this.valueAt(p) for all points p in region, and D.valueAt(p) for all
\quad\num{68}        points in R.difference(region).
\quad\num{69}     */
\quad\num{70}    extern public (region(dimension) R) distribution(region.union(R))
\quad\num{71}        asymmetricUnion( distribution(R) D);
\quad\num{72}
\quad\num{73}    /** Return a distribution on region.setMinus(R) which takes on the
\quad\num{74}     * same value at each point in its domain as this. R is passed as
\quad\num{75}     * a parameter; this allows the type of the return value to be
\quad\num{76}     * parametric in R.
\quad\num{77}     */
\quad\num{78}    extern public (region(dimension) R) distribution(region.setMinus(R))
\quad\num{79}        setMinus();
\quad\num{80}
\quad\num{81}    /** Return true iff the given distribution D, which must be over a
\quad\num{82}     * region of the same dimension as this, is defined over a subset
\quad\num{83}     * of this.region and agrees with it at each point.
\quad\num{84}     */
\quad\num{85}    extern public (region(dimension) r)
\quad\num{86}        boolean subDistribution( distribution(r) D);
\quad\num{87}
\quad\num{88}    /** Returns true iff this and d map each point in their common
\quad\num{89}     * domain to the same place.
\quad\num{90}     */
\quad\num{91}    public boolean equal( distribution( region ) d ) \{
\quad\num{92}        return this.subDistribution(region)(d)
\quad\num{93}            && d.subDistribution(region)(this);
\quad\num{94}    \}
\quad\num{95}
\quad\num{96}    /** Returns the unique 1-dimensional distribution U over the region 1..k,
\quad\num{97}     * (where k is the cardinality of Q) which maps the point [i] to the
\quad\num{98}     * i'th element in Q in canonical place-order.
\quad\num{99}     */
\quad\num{100}    extern public static distribution(:dimension=1) unique( Set<place> Q );
\quad\num{101}
\quad\num{102}    /** Returns the constant distribution which maps every point in its
\quad\num{103}        region to the given place P.
\quad\num{104}    */
\quad\num{105}    extern public static (region R) distribution(R) constant( place P );
\quad\num{106}
\quad\num{107}    /** Returns the block distribution over the given region, and over
\quad\num{108}     * place.MAX\_PLACES places.
\quad\num{109}     */
\quad\num{110}    public static (region R) distribution(R) block() \{
\quad\num{111}        return this.block(R)(place.places);
\quad\num{112}    \}
\quad\num{113}
\quad\num{114}    /** Returns the block distribution over the given region and the
\quad\num{115}     * given set of places. Chunks of the region are distributed over
\quad\num{116}     * s, in canonical order.
\quad\num{117}     */
\quad\num{118}    extern public static (region R) distribution(R) block( Set<place> s);
\quad\num{119}
\quad\num{120}
\quad\num{121}    /** Returns the cyclic distribution over the given region, and over
\quad\num{122}     * all places.
\quad\num{123}     */
\quad\num{124}    public static (region R) distribution(R) cyclic() \{
\quad\num{125}        return this.cyclic(R)(place.places);
\quad\num{126}    \}
\quad\num{127}
\quad\num{128}    extern public static (region R) distribution(R) cyclic( Set<place> s);
\quad\num{129}
\quad\num{130}    /** Returns the block-cyclic distribution over the given region, and over
\quad\num{131}     * place.MAX\_PLACES places. Exception thrown if blockSize < 1.
\quad\num{132}     */
\quad\num{133}    extern public static (region R)
\quad\num{134}        distribution(R) blockCyclic( int blockSize)
\quad\num{135}        throws MalformedRegionException;
\quad\num{136}
\quad\num{137}    /** Returns a distribution which assigns a random place in the
\quad\num{138}     * given set of places to each point in the region.
\quad\num{139}     */
\quad\num{140}    extern public static (region R) distribution(R) random();
\quad\num{141}
\quad\num{142}    /** Returns a distribution which assigns some arbitrary place in
\quad\num{143}     * the given set of places to each point in the region. There are
\quad\num{144}     * no guarantees on this assignment, e.g. all points may be
\quad\num{145}     * assigned to the same place.
\quad\num{146}     */
\quad\num{147}    extern public static (region R) distribution(R) arbitrary();
\quad\num{148}
\quad\num{149}\}
\quad\num{150}
\end{tightcode}

% \end{figure}

\subsection{Arrays}

Following ZPL~\cite{ZPL}, arrays in X10
are defined over sets of $n$-dimensional {\em index points}
called {\em regions}~\cite{gps06-arrays}.
For instance, the region {\tt [0:200,1:100]} specifies a
collection of two-dimensional points {\tt (i,j)} with {\tt i}
ranging from {\tt 0} to {\tt 200} and {\tt j} ranging from
{\tt 1} to {\tt 100}.

Constrained types ensure array bounds
violations do not occur.
An array access type-checks if the index point can be statically
determined to be in the region over which the array is defined.

Region constraints have the following syntax:

\begin{tabular}{rrcl}
  (atom)   &{\tt a} &::=& ${\tt r} \subseteq {\tt r}$ \\
  (region) &{\tt r} &::=& ${\tt t} \bnf [{\tt b}_1:{\tt
  d}_1,\ldots,{\tt b}_k:{\tt d}_k] \bnf {}$  \\
           &        && ${\tt r} | {\tt r} \bnf {\tt r} \mbox{\tt \&} {\tt r} \bnf {\tt r} - {\tt r} \bnf {\tt r} + {\tt p}$ \\
  (point)  &{\tt p} &::=& ${\tt t} \bnf [{\tt b}_1,\ldots,{\tt b}_k]$ \\
(integer)&{\tt b},{\tt d} &::=& ${\tt t} \bnf {\tt n}$ \\
\end{tabular}

Constraints include subset constraints between regions.
Regions used in constraints are either constraint terms,
region constants, union ({\tt |}), intersection ({\tt \&}), and
set difference ($-$), or regions where each point is
shifted by a point ({\tt p}).

\eat{
Arrays have the following properties:
distribution rank region rect rail onePlace zeroBased

Regions:
rank rect zeroBased

Distributions:
rank region rect onePlace zeroBased

Points:
rank
}

\begin{figure*}
\begin{tightcode}
\quad\num{1}point NORTH = new point(1,0);
\quad\num{2}point WEST  = new point(0,1);
\quad\num{3}void sor(double omega, double[.] G, int iter) \{
\quad\num{4}  region(:self=G.region) outer = G.region;
\quad\num{5}  region(:self$\subseteq$outer) inner =
\quad\num{6}    outer & (outer-NORTH) & (outer+NORTH)
\quad\num{7}          & (outer-WEST)  & (outer+WEST);
\quad\num{8}  region d0 = inner.project(0);
\quad\num{9}  region d1 = inner.project(1);
\quad\num{10}  if (d1.size() == 0) return;
\quad\num{11}  int d1min = d1.min()[0];
\quad\num{12}  int d1max = d1.max()[0];
\quad\num{13}  for (point[off] : [1:iter*2]) \{
\quad\num{14}    int red\_black = off % 2;
\quad\num{15}    finish foreach (point[i]: d0) \{
\quad\num{16}      if (i % 2 == red\_black) \{
\quad\num{17}        for (point ij: inner & [i:i,d1min:d1max]) \{
\quad\num{18}          G[ij] = omega / 4.
\quad\num{19}                * (G[ij-NORTH] + G[ij+NORTH]
\quad\num{20}                 + G[ij-WEST]  + G[ij+WEST])
\quad\num{21}                * (1. - omega) * G[ij];
\quad\num{22}        \}
\quad\num{23}      \}
\quad\num{24}    \}
\quad\num{25}  \}
\quad\num{26}\}
\end{tightcode}

\caption{Successive over-relaxation with regions}
\label{fig:sor}
\end{figure*}

For example, the code in Figure~\ref{fig:sor} performs a successive
over-relaxation~\cite{sor} of an $n \times n$ matrix {\tt G}.
The type-checker establishes that the {\tt region}
property of the point {\tt ij} (line 16) is
{\tt inner \& [i,i:d1min,d1max]}, and that this
region is a subset of {\tt outer}, the region of the array {\tt G}.

%\begin{figure}
%\subsection {Array} 

\begin{verbatim}

template<int N>
class Point
{
 private:
 const int values_[N]; 
};

class point<1>
{
   const int i_;
};

class point<2>
{
   const int i_;
   const int j_;
};

class point<3>
{
   const int i_;
   const int j_;
   const int k_;
};


template<int RANK>
class Region
{
  public:
 
  int linearIndex (const Point<RANK>& x) const;

  bool isEqual (const Region<RANK>& r) const;

}; 


template<int RANK>
class ConvexRegion : private Region<RANK>
{
  public:
 
  Region (int size[RANK], int stride[RANK], const Point<RANK>& origin);

  int linearIndex (const Point<RANK>& x) const;

  bool isEqual (const Region<RANK>& r) const;

  private:
  
  int size_[RANK];

  Point<RANK> origin; 	

  int stride_ [RANK];

  int linearStep_ [RANK];
 

  //for regions that are formed by restriction operation
  int globalLinearStep_ [RANK]; 	

}; 

template<int RANK>
class ArbitRegion : private Region<RANK>
{
  public:
 
  int linearIndex (const Point<RANK>& x) const;

  bool isEqual (const Region<RANK>& r) const;

  private:
  
  Region<Rank>* regions_;
  int numRegions_; 

}; 


template<int RANK>
Region<RANK> intersect (const Region<RANK>& r1, const Region<RANK>& r2);

template<int RANK>
Region<RANK> union (const Region<RANK>& r1, const Region<RANK>& r2);

template<int RANK>
Region<RANK> difference (const Region<RANK>& r1, const Region<RANK>& r2);


template <int RANK>
class Distribution
{
public:

protected

   Region<RANK> region;
  
};

template <int RANK>
class ConstDistribution : private Distribution<RANK>
{

public:

  
};

template <int RANK>
class BlockCyclicDistribution : private Distribution<RANK>
{
public:
  
};

template <int RANK>
class UniqueDistribution : private Distribution<RANK>
{
public:
  
};


template <typename T, int RANK>
class Array 
{
	Array (Distribution<RANK> dist);
	
	Array (Array<T, RANK>& A, Region<Rank>& R);

	Array<T, RANK>* clone();

	Distribution<RANK> dist() const;
		
        T getScalarAt (const Point<RANK>& P) const;
	
	T getScalarAt (int n) const;

        void putScalarAt (const Point<RANK>& P, T val);
	
	void putScalarAt (int n, T val);
	
	~Array(); 
 				
	private:

	T* data_;

	protected:
	
        Distribution <RANK> dist_;	
}; 

// Useful for casting a scalar to array

template <typename T, int RANK>
class UnitArray : public Array <T, RANK>
{
   UnitArray (int value) : Array();
	
   T& operator[] (const Point<RANK>& P);
   
   Array<T, RANK>& operator[] (const Region<RANK>& R);

   private:
 
   T value_; //the same value is replicated in the array
	     //upon a write, create a new data_ for this array
};

//initialization routines
template <typename T, int RANK, typename CONST_INIT>
void initialize (Array<T, RANK>& arr, CONST_INIT op);

template <typename T, int RANK, typename POINT_INIT> (check if this is valid)
void initialize (Array<T, RANK>& arr, POINT_INIT<RANK> op);

//pointwise routines for standard operators
template <typename T, int RANK>
void iterate (Array<T, RANK>& arr, order_t order, op_t op);

template <typename T, int RANK, int N>
void iterate (Array<T, RANK> (&args) [N], order_t order, op_t (&op)[N]);

//pointwise routines for "lift"ed operators
template <typename T, int RANK, typename SCALAR_OP>
void iterate (Array<T, RANK>& arr, order_t order, SCALAR_OP op);

template <typename T, int RANK, int N, typename SCALAR_OP>
void iterate (Array<T, RANK> (&args) [N], order_t order,  SCALAR_OP op);

//reduce
template <typename T, int RANK-1>
void reduce (Array<T, RANK> &arg, int dim, op_t op);

//scan
template <typename T, int RANK-1>
void scan (Array<T, RANK> &arg, int dim, op_t op);


//restriction
Array<T, RANK>& restriction (const Distribution<RANK>& R);	


//assembling

Array<T, RANK>& assemble (const Array<T, RANK>& a1, const Array<T, RANK>& a2);	

Array<T, RANK>& overlay (const Array<T, RANK>& a1, const Array<T, RANK>& a2);	

Array<T, RANK>& update (const Array<T, RANK>& a1, const Array<T, RANK>& a2);	

// How are value arrays and reference arrays reflected in the design?
//     o In the library it is always reference; the compiler should create copies for value arrays.

\end{verbatim}
%\end{figure}

\subsection{Clocks}

Clock types

\subsection{Capabilities}

Capablities (from Radha and Vijay's paper on neighborhoods)

\subsection{Ownership types}

\begin{figure}
\begin{tightcode}
\quad\num{1}class Owned(Owned owner) \{ \}
\quad\num{2}
\quad\num{3}class List(Owned valOwner
\quad\num{4}         : owner contains valOwner)
\quad\num{5}    extends Owned \{
\quad\num{6}  Owned(valOwner) head;
\quad\num{7}  List(this, valOwner) tail;
\quad\num{8}
\quad\num{9}  List(:owner=o, valOwner=v, o contains v)
\quad\num{10}      (Owned o, Owned v: o contains v) \{
\quad\num{11}    super(o);
\quad\num{12}    property(v);
\quad\num{13}  \}
\quad\num{14}
\quad\num{15}  List(this, valOwner) expose() \{
\quad\num{16}    return tail;
\quad\num{17}  \}
\quad\num{18}
\quad\num{19}  \ldots{}
\quad\num{20}\}
\end{tightcode}

\caption{Ownership types}
\label{fig:ownership}
\end{figure}

Figure~\ref{fig:ownership} shows
 a fragment of {\tt List} class, 
demonstrating how ownership
types~\cite{ownership-types} can be encoded in CFJ.
Each {\tt Owned} object has an {\tt owner} property.
Objects also have properties used as owner parameters.
The {\tt List} class has a property {\tt valOwner} that is
instantiated with the owner of the values in the list, stored
in the {\tt head} field of each element.
The {\tt tail} of the list is owned by the list object itself.

To enforce the ``owners as dominators'' property, the owner of
the values {\tt valOwner} must be contained within the owner
of the list itself; that is, {\tt valOwner} must be {\tt owner}
or {\tt valOwner}'s owner must be contained in {\tt owner}.
This is captured by the constraint {\tt owner contains valOwner}.

The {\tt expose} method incorrectly leaks the
list's {\tt tail} pointer.
The constraint system catches this XXX.

\subsection{Discussion}

\paragraph{Control-flow.}
Tricky to encode.  Need something like "pc" label~\cite{jif}.

\paragraph{Type state.}
Type state depends on the mutable state of the 
objects.  Cannot do in this framework.

Dependent types are of use in annotations~\cite{ns07-x10anno}.
