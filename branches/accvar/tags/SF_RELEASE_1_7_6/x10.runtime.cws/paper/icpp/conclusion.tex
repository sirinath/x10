\section{Related work}\label{sec:related}
%\paragraph{Related work}
Adaptive batching bears some similarities to the steal-half algorithm
of Shavit et al, and its variants. Both approaches attempt to cope
with non-hierarchical workloads for graph problems. In the steal-half
algorithm, each node is queued as its own task; and thieves take half
(or some other percentage) of the nodes available per steal
attempt. In contrast, in our approach, the tasks are pre-batched, so
only one batch is stolen at a time. This can substantially reduce
queue overhead, contention and data movement costs, but comes with
potential disadvantages because nodes cannot be stolen while they are
being batched, and batches cannot be re-split.  For example, our
approach does not allow for a subset of the nodes from a stolen batch
to themselves be re-stolen by other threads (as does
steal-half). However, queue-sensing adaptation makes consequent
impediments to global progress highly unlikely.  Because we adaptively
choose batch sizes so that there are always (during steady state
processing) some nodes available to be stolen from each active thread,
imbalanced progress by any one of them has little impact on the
ability of others to find and steal new work.  Additionally, by
relating batching rules to sequential processing thresholds needed for
any work-stealing program, our approach supports simpler empirically
guided performance tuning.
\section{Conclusion}\label{sec:concl}
%\paragraph{Conclusion}
In this paper we have shown how several graph algorithms can be
expressed concisely and elegantly in \Xten. These algorithms rely
heavily on support for fine-grained concurrency. The \Xten{} runtime
(\XWS) implements fine-grained concurrency through an enhanced
work-stealing scheduler. Specifically the scheduler supports
improperly nested tasks, detection of global termination, and phased
work-stealing.  We measure the performance of spanning tree algorithms
implemented with pseudo-depth-first search and breadth-first search on
two multicore systems. We also present a strategy to adaptively control the granularity of parallel tasks in the work-stealing scheme. We show that the \XWS{} programs scale and
exhibit performance comparable with hand-written C programs.

\paragraph{Acknowledgements} 
We thank Raj Barik for his contributions to the implementation of the
C++ version of \XWS. We thank the rest of the \Xten{} team for many
discussions of these issues. This material is based upon work
supported by the Defense Advanced Research Projects Agency under its
Agreement No.  HR0011-07-9-0002.
