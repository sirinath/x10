%\extrapart{Bibliography and references}

% My reference for proper reference format is:
%    Mary-Claire van Leunen.
%    {\em A Handbook for Scholars.}
%    Knopf, 1978.
% I think the references list would look better in ``open'' format,
% i.e. with the three blocks for each entry appearing on separate
% lines.  I used the compressed format for SIGPLAN in the interest of
% space.  In open format, when a block runs over one line,
% continuation lines should be indented; this could probably be done
% using some flavor of latex list environment.  Maybe the right thing
% to do in the long run would be to convert to Bibtex, which probably
% does the right thing, since it was implemented by one of van
% Leunen's colleagues at DEC SRC.
%  -- Jonathan

% This is just a personal remark on your question on the RRRS:
% The language CUCH (Curry-Church) was implemented by 1964 and 
% is a practical version of the lambda-calculus (call-by-name).
% One reference you may find in Formal Language Description Languages
% for Computer Programming T.~B.~Steele, 1965 (or so).
%  -- Matthias Felleisen


\begin{thebibliography}{99}

\bibitem{SICP}
Harold Abelson and Gerald Jay Sussman with Julie Sussman.
{\em Structure and Interpretation of Computer Programs.}
MIT Press, Cambridge, 1985.

\bibitem{readfloat}
William Clinger.
How to read floating point numbers accurately.
In {\em Proceedings of the 1990 ACM SIGPLAN Conference on Programming
  Language Design and Implementation}.  Forthcoming.

University of Oregon Technical Report CIS-TR-90-01.

\bibitem{RRRS}
William Clinger, editor.
The revised revised report on Scheme, or an uncommon Lisp.
MIT Artificial Intelligence Memo 848, August 1985.
Also published as Computer Science Department Technical Report 174,
  Indiana University, June 1985.

\bibitem{R4RS}
William Clinger and Jonathan Rees, editors.
The revised$^4$ report on the algorithmic language Scheme.
University of Oregon Technical Report CIS-TR-90-02.

\bibitem{Scheme311}
Carol Fessenden, William Clinger, Daniel P.~Friedman, and Christopher Haynes.
Scheme 311 version 4 reference manual.
Indiana University Computer Science Technical Report 137, February 1983.
Superceded by~\cite{Scheme84}.

\bibitem{Scheme84}
D.~Friedman, C.~Haynes, E.~Kohlbecker, and M.~Wand.
Scheme 84 interim reference manual.
Indiana University Computer Science Technical Report 153, January 1985.

\bibitem{CFractions}
G.~H.~Hardy and E.~M.~Wright.
{\em An Introduction to the Theory of Numbers.} 5th ed.
Oxford University Press, New York NY, 1979.

\bibitem{Haskell}
Paul Hudak and Philip Wadler, editors.
Report on the Functional Programming Language Haskell.
Yale University Research Report YALEU/DCS/RR-666, December 1988.

\bibitem{IEEE}
{\em IEEE Standard 754-1985.  IEEE Standard for Binary Floating-Point
Arithmetic.}  IEEE, New York, 1985.

\bibitem{Knuth}
Donald E. Knuth.
The Art of Computer Programming, volume 2: Seminumerical Algorithms.
Addison-Wesley, Reading MA, 1969.

\bibitem{Landin65}
Peter Landin.
A correspondence between Algol 60 and Church's lambda notation: Part I.
{\em Communications of the ACM} 8(2):89--101, February 1965.

\bibitem{Matula68}
David W. Matula.
In-and-Out Conversions.
{\em Communications of the ACM} 11(1):47--50, January 1968.

\bibitem{Matula70}
David W. Matula.
A Formalization of Floating-Point Numeric Base Conversion.
{\em IEEE Transactions on Computers} C-19, 8:681-692, August 1970.

\bibitem{MITScheme}
MIT Department of Electrical Engineering and Computer Science.
Scheme manual, seventh edition.
September 1984.

\bibitem{Penfield81}
Paul Penfield, Jr.
Principal values and branch cuts in complex APL.
In {\em APL '81 Conference Proceedings,} pages 248--256.
ACM SIGAPL, San Francisco, September 1981.
Proceedings published as {\em APL Quote Quad} 12(1), ACM, September 1981.

\bibitem{Pitman83}
Kent M.~Pitman.
The revised MacLisp manual (Saturday evening edition).
MIT Laboratory for Computer Science Technical Report 295, May 1983.

\bibitem{Rees82}
Jonathan A.~Rees and Norman I.~Adams IV.
T: A dialect of Lisp or, lambda: The ultimate software tool.
In {\em Conference Record of the 1982 ACM Symposium on Lisp and
  Functional Programming}, pages 114--122.

\bibitem{R3RS}
Jonathan Rees and William Clinger, editors.
The revised$^3$ report on the algorithmic language Scheme.
In {\em ACM SIGPLAN Notices} 21(12), ACM, December 1986.

\bibitem{Reynolds72}
John Reynolds.
Definitional interpreters for higher order programming languages.
In {\em ACM Conference Proceedings}, pages 717--740.
ACM, \todo{month?}~1972.

\bibitem{Rabbit}
Guy Lewis Steele Jr.
Rabbit: a compiler for Scheme.
MIT Artificial Intelligence Laboratory Technical Report 474, May 1978.

\bibitem{CLtL}
Guy Lewis Steele Jr.
{\em Common Lisp: The Language.}
Digital Press, Burlington MA, 1984.

\bibitem{CLtL2}
Guy Lewis Steele Jr.
{\em Common Lisp: The Language.} 2d ed.
Digital Press, Bedford MA, 1990.

\bibitem{Scheme78}
Guy Lewis Steele Jr.~and Gerald Jay Sussman.
The revised report on Scheme, a dialect of Lisp.
MIT Artificial Intelligence Memo 452, January 1978.

\bibitem{Heuristic}
Guy Lewis Steele Jr.~and Jon L White.
How to Print Floating-Point Numbers Accurately.
In {\em Proceedings of the 1990 ACM SIGPLAN Conference on Programming
  Language Design and Implementation}.  Forthcoming.

\bibitem{Stoy77}
Joseph E.~Stoy.
{\em Denotational Semantics: The Scott-Strachey Approach to
  Programming Language Theory.}
MIT Press, Cambridge, 1977.

\bibitem{Scheme75}
Gerald Jay Sussman and Guy Lewis Steele Jr.
Scheme: an interpreter for extended lambda calculus.
MIT Artificial Intelligence Memo 349, December 1975.

\bibitem{Vuillemin}
Jean Vuillemin.
Exact real computer arithmetic with continued fractions.
In {\em Proceedings of the 1988 ACM Conference on Lisp and
  Functional Programming}, pages 14--27.

\end{thebibliography}
