\section{Clocked types}\index{types!clocked}

We allow types to specify clocks, via a {\cf clocked(c)} modifier,
e.g.{}

\begin{x10}
  clocked(c) int r;
\end{x10}

This declaration asserts that {\cf r} is accessible
(readable/writable) only by those statements that are clocked on {\cf
c}. Thus propagation of the clock provides some control over the
``visibility'' of {\cf r}.

The declaration 

\begin{x10}
  clocked(c) final int l = r;
\end{x10}

\noindent asserts additionally that in each clock instant {\cf l} is final, 
i.e.{} the value of {\cf l} may be reset at the beginning of each phase
of {\tt c} but stays constant during the phase.

This statement terminates when the computation of {\tt r} has
terminated and the assignment has been performed.

\todo{Generalize the syntax so that aggregate variables can be clocked with an aggregate clock of the same shape.}

\subsection{Clocked assignment}\index{assignment!clocked}
We expect that most arrays containing application data will be
declared to be {\cf clocked final}. We support this very powerful type
declaration by providing a new statement:
{\footnotesize
\begin{verbatim}
  next(c) l = r; 
\end{verbatim}}


\noindent 
for a variable $l$ declared to be clocked on $c$. The statement
assigns $r$ to the {\em next} value of $l$. There may be multiple such
assignments before the clock advances. The last such assignment
specifies the value of the variable that will be visible after the
clock has advanced.  If the variable is {\cf clocked final} it is
guaranteed that {\em all} readers of the variable throughout this
phase will see the value $r$.

The expression {\tt r} is implicitly treated as {\tt now(c) r}. That
is, the clock {\tt c} will not advance until the computation of {\tt r} has
terminated.

