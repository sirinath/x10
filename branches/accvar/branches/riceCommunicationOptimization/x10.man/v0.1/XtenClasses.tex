\chapter{Classes}
\label{XtenClasses}\index{class}

{}\Xten{} classes are essentially the same as \java{} classes \cite[\S 8]{jls2}. Classes
are structured in a single-inheritance code hierarchy, may implement
multiple interfaces, may have static and instance fields, may have
static and instance methods, may have constructors, may have static
and instance initializers, may have static and instance inner classes
and interfaces. Method signatures may specify checked
exceptions. Method definitions may be overridden by subclasses; the
overriding definition may have a declared return type that is a
subclass of the return type of the definition being
overridden. Multiple methods with the same name but different
signatures may be provided on a class (ad hoc polymorphism). The
public/private/protected/package-protected access modification
framework may be used.

\todo{Add the new rule for preventing leakage of this from a constructor.}

Because of its different concurrency model, \Xten{} does not support
{\tt transient} and {\tt volatile} field modifiers.

\todo{Figure out class modifiers. Figure out which new ones need to be added to support IEEE FP.}

\section{Reference classes}\index{class!reference class}\label{ReferenceClasses}

A reference class is declared with the optional keyword {\tt
reference} preceding {\tt class} in a class declaration. Reference
class declarations may be used to construct reference types
(\S~\ref{ReferenceTypes}). Reference classes may have mutable
fields. Instances of a reference class are always created in a fixed
place and in \XtenCurrVer{} stay there for the lifetime of the
object. (Future versions of \Xten{} may support object migration.)
Variables declared at a reference type always store a reference to the
object, regardless of whether the object is local or remote.


\section{Value classes}\index{class!value class}\label{ValueClasses}

{}\Xten{} singles out a certain set of classes for additional
support. A class is said to be {\em stateless} if all of its fields
are declared to be {\cf final} (\S~\ref{FinalVariable}), otherwise it
is {\em stateful}. (\Xten{} has syntax for specifying an array class
with final fields, unlike \java{}.) A {\em stateless (stateful)
object} is an instance of a stateless (stateful) class.

{}\Xten{} allows the programmer to signify that a class (and all its
descendents) are stateless. Such a class is called a {\em value
class}.  The programmer specifies a value class by prefixing the
modifier {\cf value} before the keyword {\cf class} in a class
declaration.  (A class not declared to be a value class will be called
a {\em reference class}.)  Each instance field of a value class is
treated as {\cf final}. It is legal (but neither required nor recommended)
for fields in a value class to be declared final. It is a compile-time
error for a value class to inherit from a stateful class or for a
reference class to inherit from a value class. For brevity, the \Xten{}
compiler allows the programmer to omit the keyword class after value
in a value class declaration.

The {\cf nullable} type-constructor (\S~\ref{NullableTypeConstructor}) can
be used to declare variables whose value may be {\cf null} or a value
type.

{}\Xten{} provides a built in definition for {\cf .equals()} for a
value type, namely stable equality ({\tt ==},
\S~\ref{StableEquality}). The programmer is free to override {\cf
.equals} with his/her own definition. (The behavior of {\tt ==} cannot
be overridden however.)

Value types also support user-defined operators and constructors. 

\subsection{Representation}

Since value objects do not contain any updatable locations, they can
be freely copied from place to place. An implementation may use
copying techniques even within a place to implement value types,
rather than references. This is transparent to the programmer.

More explicitly, \Xten{} guarantees that an implementation must always
behave as if a variable of a reference type takes up as much space as
needed to store a reference that is either null or is bound to an
object allocated on the (appropriate) heap. However, \Xten{} makes no
such guarantees about the representation of a variable of value
type. The implementation is free to behave as if the value is stored
``inline'', allocated on the heap (and a reference stored in the
variable) or use any other scheme (such as structure-sharing) it may
deem appropriate. Indeed, an implementation may even dynamically
change the representation of an object of a value type, or dynamically
use different representations for different instances (that is,
implement automatic box/unboxing of values).

Implementations are strongly encouraged to implement value types as
space-efficiently as possible (e.g.{} inlining them or passing them in
registers, as appropriate).  Implementations are expected to cache
values of remote final value variables by default. If a value is
large, the programmer may wish to consider spawning a remote activity
(at the place the value was created) rather than referencing the
containing variable (thus forcing it to be cached).

\todo{ Need to figure out whether we should let the programmer be
aware of lazy pull vs full-value push of value objects. This is the
idea of introducing a *-annotation. Need to make a decision on
this. Could leave this for 0.7.}

\subsection{Example}
\begin{x10}
 value LinkedList <Node> \{ 
  Node first;
  nullable LinkedList<Node> rest;
  public
     LinkedList(Node first) \{
     this(first, null);
  \}
  public
    LinkedList(Node first,  nullable LinkedList<Node> rest) \{
    this.first = first;
    this.rest = rest;
  \}
  public 
    Node first() \{
    return first;
  \}
  public 
    nullable LinkedList<Node> rest() \{
    return rest;
  \} 
  public
    void append(LinkedList<Node> l) \{
    return (this.rest == null) 
        ? new LinkedList<Node>(this.first, l) 
        : this.rest.append(l);
  \}
  public 
    LinkedList<Node> reverse(LinkedList<Node> l) \{ 
    return 
      rest == null 
      ? new LinkedList<Node>(first, l)  
      : rest.reverse(new LinkedList<Node>(first, l));
   \}
  public
    LinkedList<Node> reverse() \{
    return (rest == null) 
        ? this 
        : rest.reverse(new LinkedList<Node>(first));
  \}
\}
\end{x10}

\section{Ranges}\label{XtenRanges}\index{ranges}

\XtenCurrVer{} supports extremely simple forms of {\tt enums}, namely {\em ranges}. A range is a set of {\em points}.

For two constants or final variables {\tt a} and {\tt b} of a fixed
point type (e.g.{} {\cf long}), and a third constant or final variable
{\tt k} of the same type, the literal {\tt a..b:k} (read: ``from {\tt
a} to {\tt b} with stride {\tt k}'') represents the full range of
values from {\tt a} to {\tt b} (inclusive), starting with {\tt a} and
taking strides of length {\tt k}. That is, the range contains all the
numbers {\tt a}, {\tt a+k}, {\tt a+2*k}, \ldots {\tt a+n*k} which are
less than or equal to {\tt b}. The canonical order for elements of a
range is the canonical order for the underlying fixed point type.

The notation {\tt a..b} is read as shorthand for {\tt a..b:1}.

\Xten{} allows ranges to be named:
\begin{x10}
   range   E = 0..100;
\end{x10}

Such ranges may be used in a {\tt for}/{\tt foreach} loop
(\S~\ref{ForLoop}) or in a region declaration.


\todo{Merge in JDK 1.5 enums.}




