\chapter{Types}
\label{XtenTypes}\index{types}

{}\Xten{} is a {\em strongly typed} object-oriented language: every variable
and expression has a type that is known at compile-time. Further,
\Xten{} has a {\em unified} type system: all data items created at
run time are {\em objects} (\Sref{XtenObjects}). Types limit the
values that variables can hold and specify the places at which these
values lie.

{}\Xten{} supports two kinds of objects, {\em reference objects} and
{\em value objects}.  Reference objects are instances of {\em
reference classes} (\Sref{ReferenceClasses}). They may contain
mutable fields and must stay resident in the place in which they were
created. Value objects are instances of {\em value classes}
(\Sref{ValueClasses}). They are immutable and may be freely copied
from place to place. Either reference or value objects may be 
{\em scalar} (instances of a non-array class) or {\em aggregate} (instances
of arrays).
Only variables of reference types may be assigned \xcd"null".

Types are used in variable declarations, explicit coercions and
conversions, object creation, array
creation, class literals and \xcd"instanceof" expressions.\footnote{In
order to allow this version of the language to focus on the core new
ideas, \XtenCurrVer{} does not have user-definable class loaders,
though there is no technical reason why they could not have been
added.}

Types in \Xten{} are specified through declarations and through
type constructors, described in the remainder of the chapter:

\begin{itemize}
\item A class declaration defines a \emph{class type}
(\Sref{ReferenceTypes}), which
may be either a reference class or a value class.
\item An interface declaration defines an \emph{interface type}
(\Sref{InterfaceTypes}).
\item The \emph{boxed type} constructor produces a reference
type from a value type (\Sref{BoxedTypes}).
\iftypeparams\else
\item Classes and interface have \emph{type properties}, which
are instance members and introduce \emph{path types}
(\Sref{PathTypes}).
A class or interface with one or more type properties is a
\emph{generic class} or \emph{generic interface}
(\Sref{Generics}).
\fi
\iftypeparams
\item Classes and interface have \emph{type parameters}.
A class or interface with one or more type parameters is a
\emph{generic class} or \emph{generic interface}
(\Sref{Generics}).
\fi
\item New type constructors may be defined with \emph{type
definitions} (\Sref{TypeDefs}).
\item Methods, constructors, closures, and type definitions may
have \emph{type parameters}, which are instantiated with
concrete types at invocation (\Sref{TypeParameters}).
\item \emph{Function type} constructors are used to define
function types; closures and method selectors have function
type (\Sref{FunctionTypes}).
\item A \emph{constrained type} constrains the properties of
a base type (\Sref{ConstrainedTypes}).
\item Types may be marked with user-defined annotations.
\emph{Annotated types} (\Sref{AnnotatedTypes})
may be processed by compiler plugins.
\end{itemize}

\begin{grammar}
Type \: FunctionType \\
    \| ConstrainedType  \\

FunctionType \: TypeParameters\opt \xcd"(" Formals\opt \xcd")"
Constraint\opt Throws\opt \xcd"=>" Type \\
TypeParameters \: \xcd"[" TypeParameter ( \xcd"," TypeParameter )\star \xcd"]" \\
TypeParameter \: Identifier \\
Throws \: \xcd"throws" ( TypeName \xcd"," TypeName )\star \\

ConstrainedType \: Annotation\star BaseType Constraint\opt
PlaceConstraint\opt \\

BaseType \: ClassBaseType \\
     \| InterfaceBaseType \\
     \| PathType \\
     \| \xcd"(" Type \xcd")" \\

ClassType \: Annotation\star ClassBaseType Constraint\opt
PlaceConstraint\opt \\
InterfaceType \: Annotation\star InterfaceBaseType Constraint\opt
PlaceConstraint\opt \\

PathType \: Expression \xcd"." Identifier \\

Annotation \: \xcd"@" InterfaceBaseType Constraint\opt \\

ClassOrInterfaceType \: ClassType \\ \| InterfaceType \\
ClassBaseType \: TypeName \\
InterfaceBaseType \: TypeName \\
\end{grammar}

% \section{Type definitions and type constructors}
% 
% Types in \Xten{} are specified through declarations and through
% type constructors:

% \paragraph{Class types.}

\section{Classes and interfaces}
\label{ReferenceTypes}

\subsection{Class types}

\index{types!class types}
\index{class}
\index{class declaration}
\index{declaration!class declaration}
\index{declaration!reference class declaration}

        A {\em class declaration} (\Sref{XtenClasses})
        introduces a 
        {\em class type}
        containing all instances of the class.

        Class instances are created via
        constructor calls.  Class instances have fields and
        methods, type members, and
\iftypeparams
        value properties bound at construction time.
\else
        value and type properties bound at construction time.
\fi
        In addition, classes have static
        members: constant fields, type definitions, and member
        classes and member interfaces.

\iftypeparams
        A class with type parameters is {\em generic}.  A class
        type is legal only if all of its parameters are instantiated
        on concrete types.
\fi

\Xten{} does
not
permit mutable static state, so the role of static methods and
initializers is quite limited. Instead programmers should use
singleton classes to carry mutable static state.

\index{\Xcd{Object}}
\index{\Xcd{x10.lang.Object}}

        Classes
        are structured in a
        single-inheritance hierarchy.  All reference classes
        extend the class \xcd"x10.lang.Ref".
        All value classes extend the class \xcd"x10.lang.Value".
        Classes are declared to extend a single
        superclass (except for \xcd"Ref" and \xcd"Value", which extend
        no other class),
        All classes implement the
        interface \xcd"x10.lang.Object"
        and zero or more other interfaces.

\xcd"Object" has the following signature:

\begin{xten}
package x10.lang;

public interface Object {
    public def toString(): String;
    public def equals(Object): Boolean;
    public def hashCode(): Int;
    public def className(): String;
}
\end{xten}


\index{class!reference class}
\index{reference class type}
\index{types!value class types}
\index{class!value class}
\index{value class declaration}
\index{declaration!value class declaration}
\index{\Xcd{Ref}}
\index{\Xcd{x10.lang.Ref}}
\index{\Xcd{Value}}
\index{\Xcd{x10.lang.Value}}

        Classes may be declared to be either reference classes,
        or value classes.
A \emph{value class type}
is a class type in which all fields of the class are final.
%
Variables of reference class type may be \xcd"null"; 
variables of value class type may not be \xcd"null".
%
        All reference classes extend the
        \xcd"x10.lang.Ref" class; all value classes
        extend the \xcd"x10.lang.Value" class.

        It is a static error if a value class extends
        \xcd"Ref" (or any subclass of \xcd"Ref").
        It is a static error if a reference class extends
        \xcd"Value" (or any subclass of \xcd"Value").

\xcd"Ref" and \xcd"Value" have the following signatures:

\begin{xten}
package x10.lang;

public class Ref(location: Place) {
    public Ref() { property(here); }
}

public class Value { }
\end{xten}


\index{class!reference class}
\index{reference class type}
\index{types!value class types}
\index{class!value class}
\index{value class declaration}
\index{declaration!value class declaration}
\index{\Xcd{Ref}}
\index{\Xcd{x10.lang.Ref}}
\index{\Xcd{Value}}
\index{\Xcd{x10.lang.Value}}

\subsection{Interface types}
\label{InterfaceTypes}

\index{types!interface types}
\index{interface}
\index{interface declaration}
\index{declaration!interface declaration}

        An {\em interface declaration} (\Sref{XtenInterfaces})
        defines an {\em interface type}, which specifies a set of
        methods, type members, and properties to be implemented by any class
        declared to implement the interface.  Interfaces also
        have static members: constant fields, type definitions,
        and member classes and interfaces.

        An interface may extend multiple interfaces.
        All interfaces extend \xcd"x10.lang.Object".

Classes may be declared to implement 
multiple interfaces. 
Semantically, the interface type is the set of all objects that are
instances of (value or reference) classes that implement the
interface. A class implements an interface if it is declared to
and if
it implements all the methods and properties defined in the interface.

\subsection{Value properties}

\index{properties!value properties}

        Classes and interfaces may have {\em value properties},
        public final fields bound to object creation.  
%
        For example, the following code declares a class named
        \xcd"Point" with properties \xcd"x" and \xcd"y"
        and a \xcd"move" method.  The properties are bound
        using the \xcd"property" statement in the constructor.

\begin{xten}
class Point(x: Int, y: Int) {
  def this(x: Int, y: Int) { property(x, y); }
  def move(dx: Int, dy: Int) = new Point(x+dx, y+dy);
}
\end{xten}

        The value properties of a class or interface may be
        constrained with a boolean expression.  The type
        \xcd"Point{x==0}" is the set of all points whose \xcd"x"
        property is \xcd"0".


\iftypeparams
\else

\subsection{Type properties}
\label{TypeProperties}

\index{types!type properties}
\index{properties!type properties}

        Classes and interfaces may have {\em type properties}, bound
        to concrete types on object creation.
        A class or interface type
        with one or more type properties is called a {\em
        generic type}.

The following code declares a class \xcd"Cell" with a type
property named \xcd"X".
\begin{xten}
class Cell[X] {
    var x: X;
    def this(x: X) { this.x = x; }
    def get(): X = x;
    def set(x: X) = { this.x = x: }
}
\end{xten}
The class has a mutable field \xcd"x", initialized by the
constructor, and has a getter method and a setter method.

This example shows that type properties are in many ways similar to
type parameters as provided in languages such as Java and Scala.
Type properties are types in their own right:
they may be used in any context a type may be used,
including in \xcd"instanceof" and cast expressions.

\label{PathTypes}
\index{types!path types}

However, the key semantic distinction between type properties
and type parameters is that type properties are instance
members.
% vj: fixed the text that follows Wed Jun 18 02:01:11 2008
        Type properties introduce {\em path types}.
        Let \xcd"o" be the object obtained by evaluating an expression
	\xcd"e" of type \xcd"T", where 
        \xcd"T" is a class or interface type declared with
        a type property \xcd"X". Then \xcd"e.X" is an expression
	which evaluates to the type \xcd"T" that \xcd"X" was bound to when 
	\xcd"o" was created. 
Thus, for an expression \xcd"e" of type \xcd"Cell" (\xcd"e" must be a
final access path, see below), \xcd"e.X" is a type and \xcd"e.get()"
returns a value of that type.

Within the body of a class, a property name \xcd"X" resolves
to \xcd"this.X" (or to \xcd"C.this.X" if \xcd"X" is a property of
an enclosing class \xcd"C"), just as value property names are
resolved.
\fi

\iftypeparams
\else
\subsection{Final access paths}
\label{FinalAccessPath}
\index{final access path}

        To ensure soundness of the type system, the expression \xcd"e"
	used in a path type \xcd"e.X" must be a \emph{final access
	path}, either a final local variable or formal parameter
	(including the special variables \xcd"this" and \xcd"self"),
	or a final field or value property of a final access path.
\fi

\section{Type parameters}
\label{TypeParameters}

\index{types!type parameters}
\index{methods!parametrized methods}
\index{constructors!parametrized constructors}
\index{closures!parametrized closures}
\index{parametrized methods}
\index{parametrized constructors}
\index{parametrized closures}

\iftypeparams
        A class, interface, method, or closure may have type parameters
        whose scope is the signature and body of the declaring
        class, interface, method,
        or closure.
\else
        A method, constructor, or closure may have type parameters
        whose scope is the signature and body of the declaring method,
        constructor or closure.
\fi
        Similarly, a type definition may have
        type parameters that scope over the body of the type
        definition.

        Type parameters may be constrained by a {\em guard} on the declaration
        (\Sref{ClassGuard}, \Sref{TypeDefGuard},
\iftypeparams\else
        \Sref{ConstructorGuard},
\fi
        \Sref{MethodGuard},\Sref{ClosureGuard}).
\iftypeparams
        The type parameters of classes and interfaces must be
        bound to concrete types (possibly to a type parameter)
        for the type to be legal; thus \xcd"List[int]" and
        \xcd"List[C]" are legal types, but \xcd"List" alone is
        not.
        The type parameters of
        methods and closures
\else
        The type parameters of
        methods, constructors, and closures,
\fi
        must be bound to
        concrete types at invocation.
        Parametrized type definitions specify new type
        constructors; the type parameters of a type definition must be bound
        to yield a type.

\iftypeparams

\subsection{Generic types}
\label{Generics}
\index{types!generic types}
        A {\em generic class} is a class
        declared with one or more type parameters.  Generic
        classes can be instantiated by instantiating the type
        parameters of the base type.

Consider the following declaration of a \xcd"Cell" class.
\begin{xten}
class Cell[X] {
    var x: X;
    def this(x: X) { this.x = x; }
    def get(): X = x;
    def set(x: X) = { this.x = x: }
}
\end{xten}

This declares a class \xcd"Cell" with a type parameter \xcd"X".
\xcd"Cell" may be used as a type by instantiating \xcd"X".

\xcd"Cell[int]" is the type of all \xcd"Cell" containing an
\xcd"int".  The \xcd"get" method returns an \xcd"int"; the
\xcd"set" method takes an \xcd"int" as argument.  Note that
\xcd"Cell" alone is not a legal type because the parameter is
not bound.

Parameters may be declares as invariant, covariant, or
contravariant.  The \xcd"X" parameter of \xcd"Cell" above is
invariant.  
Consider the following classes:
\begin{xten}
class Get[+X] {
    var x: X;
    def this(x: X) { this.x = x; }
    def get(): X = x;
}

class Set[-X] {
    var x: X;
    def this(x: X) { this.x = x; }
    def set(x: X) = { this.x = x: }
}
\end{xten}
The \xcd"X" parameter of the \xcd"Get" class is covariant;
the \xcd"X" parameter of the \xcd"Set" class is contravariant.

Given types \xcd"S" and \xcd"T".
\begin{itemize}
\item
If the parameter of \xcd"Get" is covariant, then
\xcd"Get[S]" is a subtype of \xcd"Get[T]" if
\xcd"S" is a {\em subtype} of \xcd"T".

\item
If the parameter of \xcd"Set" is covariant, then
\xcd"Set[S]" is a subtype of \xcd"Set[T]" if
\xcd"S" is a {\em supertype} of \xcd"T".

\item
If the parameter of \xcd"Cell" is invariant, then
\xcd"Cell[S]" is a subtype of \xcd"Cell[T]" if
\xcd"S" is a {\em equal} to \xcd"T".
\end{itemize}

\fi



\section{Type definitions}
\label{TypeDefs}

\index{types!type definitions}
\index{declarations!type definitions}
\index{types!path types}

        Types may be defined through {\em type definitions}.
        A type definition is a type alias or a type-valued function
        that maps value and type parameters to another type.
        A type definition may appear as a static or instance
        class or interface
        member or in a statement block.

        Type definitions may have zero or more type parameters
        and value parameters.
The general form of a type definition is:

\begin{xtenmath}
type X[Y$_1$, $\dots$, Y$_m$](x$_1$: T$_1$, $\dots$ x$_n$: T$_n$){c} = U;
\end{xtenmath}

\xcdmath"Y$_1$", \dots, \xcdmath"Y$_m$"
are the type parameters of \xcd"X";
\xcdmath"x$_1$", \dots, \xcdmath"x$_n$"
are the value parameters of \xcd"X" with types
\xcdmath"T$_1$", \dots, \xcdmath"T$_n$".
The constraint \xcd"c" is called the \emph{guard} of
\xcd"X".  Finally, \xcd"U" is the defining type of \xcd"X".
\label{TypeDefGuard}

The expresion \xcdmath"X[S$_1$, $\dots$, S$_m$](e$_1$, $\dots$, e$_n$)"
denotes a type whenever 
\xcdmath"S$_1$", \dots, \xcdmath"S$_m$" are type arguments and 
\xcdmath"e$_1$", \dots, \xcdmath"e$_n$" are value arguments with
types \xcdmath"T$_1$", \dots, \xcdmath"T$_n$"
such that the guard \xcd"c" holds. The type denoted is
the defining type \xcd"U" with the appropriate substitutions applied:
\xcdmath"U[S$_1$/Y$_1$,$\dots$,S$_m$/Y$_m$,e$_1$/x$_1$,$\dots$,e$_n$/x$_n$)".

        If the type definition 
        is a static member of a class or
        interface \xcd"C", then the type definition
        defines the type constructor \xcd"C.X"
and
\xcdmath"C.X[S$_1$, $\dots$, S$_m$](e$_1$, $\dots$, e$_n$)"
is a type.

        If the type definition 
        is an instance member of a class or
        interface \xcd"C", then, for a final access path expression \xcd"e" of type \xcd"C",
        the type definition
        defines the type constructor \xcd"e.X"
and
\xcdmath"e.X[S$_1$, $\dots$, S$_m$](e$_1$, $\dots$, e$_n$)"
is a type.

        If the type definition 
        appears in a statement block, then it
        defines the type constructor \xcd"X", visible in the
        remainder of the block,
and
\xcdmath"X[S$_1$, $\dots$, S$_m$](e$_1$, $\dots$, e$_n$)"
is a type.

\section{Path types and path type constructors}

\index{types!path types}
\index{final access path}

\iftypeparams
Instance member type definitions declare \emph{path types}.
\else
Type properties and
instance member type definitions 
declare \emph{path types}.
\fi
If the type definition is parametrized, the definition specifies
a type constructor; instantiating the parameters yields a type.
A path type (constructor)
is of the form \xcd"p.X", and consists of a final access path
\xcd"p", and a type
member
\iftypeparams\else
or type property
\fi
\xcd"X".
The final access path \xcd"p" must be either \xcd"this", \xcd"self", a final
local variable, or a final field or value property whose target
is itself a final access path.  The path must type-check 
as a normal expression in the scope in which it appears.
The type member
\iftypeparams
or property
\fi
\xcd"X" must be an instance member of the type of the
path \xcd"p".

The name resolution rules for \xcd"X" are identical to the name
resolution rules for fields and local variables.
If an identifier \xcd"X" resolves to a type member or property
of an enclosing class or interface \xcd"T", then \xcd"X" is equivalent to
the path type \xcd"T.this.X".

\iftypeparams
\subsection{Final access paths}
\label{FinalAccessPath}
\index{final access path}

        To ensure soundness of the type system, the expression \xcd"e"
	used in a path type \xcd"e.X" must be a \emph{final access
	path}, either a final local variable or formal parameter
	(including the special variables \xcd"this" and \xcd"self"),
	or a final field or value property of a final access path.
\fi

\section{Constrained types}
\label{ConstrainedTypes}
\label{DepType:DepType}
\label{DepTypes}

\index{dependent types}
\index{constrained types}
\index{generic types}
\index{types!constrained types}
\index{types!dependent types}
\index{types!generic types}

        Given a type \xcd"T", a {\em constrained type}
        \xcd"T{e}" may be
        constructed by constraining its
        \iftypeparams\else
        type and value
        \fi
        properties with a
        boolean expression \xcd"e".

        \xcd"T{e}" is a {\em dependent type}, that is, a type
        dependent on values.
        The type \xcd"T" is called the
        {\em base type} and \xcd"e" is called the {\em constraint}.
        For reference types, the constraint may
        specify the places at which the object resides.

For brevity, the constraint may be omitted and
interpreted as \xcd"true".
\if 0
\iftypeparams
A type may also be constrained using argument lists.
The syntax 
\xcdmath"C(e$_1$,$\dots$,e$_n$)" is sugar for
\xcdmath"C{x$_1$==e$_1$,$\dots$,x$_n$==e$_n$}"
where the \xcdmath"x$_i$" are the
value properties of \xcd"C".  
If the list of arguments is empty, it may be omitted.
\else
A type may also be constrained using lists of type arguments and
value arguments.
The syntax 
\xcdmath"C[T$_1$,$\dots$,T$_m$](e$_1$,$\dots$,e$_n$)" is sugar for
\xcdmath"C{X$_1$==T$_1$,$\dots$,X$_m$==T$_m$,x$_1$==e$_1$,$\dots$,x$_n$==e$_n$}"
where \xcdmath"X$_i$" are the type properties and \xcdmath"x$_i$" are the
value properties of \xcd"C".  
If either list of properties is empty, it may be omitted.
\fi

\iftypeparams\else
In this shortened syntax, a type \xcd"T" used may also be annotated
with
a \emph{use-site variance tag}, either \xcd"+" or \xcd"-":
if \xcd"X" is a type property, then
the syntax \xcd"C[+T]" is sugar \xcd"C{X<:T}" and
\xcd"C[-T]" is sugar \xcd"C{X:>T}"; of course,
\xcd"C[T]" is sugar \xcd"C{X==T}".
\fi
\fi

Constraints on properties induce a natural subtyping relationship:
\xcd"C{c}" is a subtype of
\xcd"D{d}" if \xcd"C" is a subclass of \xcd"D" and
\xcd"c" entails \xcd"d".

Type parameters cannot be constrained.

\subsection{Constraints}

\def\withmath#1{\relax\ifmmode#1\else{$#1$}\fi}
\def\LL#1{\withmath{\lbrack\!\lbrack #1\rbrack\!\rbrack}}

Expressions used as constraints are restricted by the
constraint system in use to ensure that the constraints can
be solved at compile time.  The \Xten{} compiler allows compiler
plugins to be installed to extend the constraint language and
the
constraint system.
Constraints must be of type \xcd"Boolean".
The
compiler supports the following constraint syntax.

\iftypeparams

\begin{grammar}
Constraint \: ValueArguments     Guard\opt \\
           \| ValueArguments\opt Guard     \\
           \\
ValueArguments   \:  \xcd"(" ArgumentList\opt \xcd")" \\
ArgumentList     \:  Expression ( \xcd"," Expression )\star \\
Guard            \: \xcd"{" DepExpression \xcd"}" \\
DepExpression    \: ( Formal \xcd";" )\star ArgumentList \\
\end{grammar}

\else

\begin{grammar}
Constraint \: TypeArguments     ValueArguments\opt Guard\opt \\
           \| TypeArguments\opt ValueArguments     Guard\opt \\
           \| TypeArguments\opt ValueArguments\opt Guard     \\
           \\
TypeArguments    \:  \xcd"[" Type ( \xcd"," Type )\star \xcd"]" \\
ValueArguments   \:  \xcd"(" ArgumentList\opt \xcd")" \\
ArgumentList     \:  Expression ( \xcd"," Expression )\star \\
Guard            \: \xcd"{" DepExpression \xcd"}" \\
DepExpression    \: ( Formal \xcd";" )\star ArgumentList \\
\end{grammar}

\fi

The default \Xten{} constraint system supports equality expressions
(\xcd"=="), conjunction (\xcd"&&"), and subtyping and supertyping (\xcd"<:" and
\xcd":>") expressions over constants, final access paths, and
types, and existential quantification over typed variables.

\emph{
Subsequent implementations are intended to support boolean algebra,
arithmetic, relational algebra, etc., to permit types over regions and
distributions. We envision this as a major step towards removing most,
if not all, dynamic array bounds and place checks from \Xten{}.
}


\iftypeparams
\else

\subsection{Generic types}
\label{Generics}
\index{types!generic types}
        A {\em generic class} is a class
        declared with one or more type properties.  Generic
        classes can be instantiated by constraining the type
        properties of the base type.

As with value properties, type properties may be constrained
by predicates to produce new types.
Type properties are constrained using type equality and
subtyping expressions.

Consider the following declaration of a \xcd"Cell" class.
\begin{xten}
class Cell[X] {
    var x: X;
    def this(x: X) { this.x = x; }
    def get(): X = x;
    def set(x: X) = { this.x = x: }
}
\end{xten}
The following are legal types that constrain \xcd"Cell".

\begin{itemize}
\item \xcd"Cell".  This type has no constraints on \xcd"X".
Any type that constrains \xcd"X", including the three below,
is a subtype of \xcd"Cell".  The type \xcd"Cell" is equivalent to
\xcd"Cell{true}".
%
For a \xcd"Cell" \xcd"c", the return type of the \xcd"get" method 
is \xcd"c.X".  
Since the property \xcd"X" is unconstrained,  
the caller can only assign the return value of \xcd"get"
to a variable of type \xcd"c.X" or of type \xcd"Object".
In the following code, \xcd"y" cannot be passed to the \xcd"set" method
because, since \xcd"X" is not bounded,
it cannot be determined statically if \xcd"Object" is a
subtype of \xcd"c.X".
\begin{xten}
val x: c.X = c.get();
val y: Object = c.get();
c.set(x); // legal
c.set(y); // illegal
\end{xten}

\item \xcd"Cell{X==Float}" or \xcd"Cell[Float]".
The type property \xcd"X" is bound to \xcd"Float".
Assuming \xcd"c" has this type, the following code is legal:
\begin{xten}
val x: Float = c.get();
c.set(1.0);
\end{xten}
The type of \xcd"c.get()" is \xcd"c.X", which is equivalent to
\xcd"Float".
Similarly, the \xcd"set" method takes a \xcd"Float" as argument.

\item \xcd"Cell{X<:Int}" or \xcd"Cell[+Int]".
This type constrains \xcd"X" to be a subtype of \xcd"Int".
All instances of this type must bind \xcd"X" to a subtype of \xcd"Int".
The following expressions have this type:
\begin{xten}
new Cell[Int](1);
new Cell[Int{self==3}](3);
\end{xten}
The cell in the first expression may contain any \xcd"Int".
The cell in the second expression may contain only \xcd"3".
%
If \xcd"c" has the type \xcd"Cell{X<:Int}",
then \xcd"c.get()" has type \xcd"c.X", which is an unknown but
fixed subtype of \xcd"Int".  The \xcd"set" method of \xcd"c" can
only be called with an object of type \xcd"c.X".

\item \xcd"Cell{X:>String}" or \xcd"Cell[-String]".  This type bounds the type property
\xcd"X"
from below.  The \xcd"set" method may be called with any
supertype of \xcd"String"; the return type of the \xcd"get"
method is known to be a
supertype of \xcd"String" (and implicitly a subtype of \xcd"Object").
\end{itemize}

\fi



\subsection{Place constraints}
\label{PlaceTypes}
\label{PlaceType}
\index{place types}
\label{DepType:PlaceType}\index{placetype}

An \Xten{} computation spans multiple places
(\Sref{XtenPlaces}). Each place constains data and activities that
operate on that data.  \XtenCurrVer{} does not permit the dynamic
creation of a place. Each \Xten{} computation is initiated with a
fixed number of places, as determined by a configuration parameter.
In this section we discuss how the programmer may supply place type
information, thereby allowing the compiler to check data locality,
i.e., that data items being accessed in an atomic section are local.

\begin{grammar}
PlaceConstraint     \: \xcd"!" Place\opt \\
Place              \:   Expression \\
\end{grammar}

Because of the importance of places in the \Xten{} design, special
syntactic support is provided for constrained types involving places.

All \Xten{} classes extend the class
\xcd"x10.lang.Object", which defines a property
\xcd"home" of type
\xcd"Place".

If a constrained reference type \xcd"T" has an \xcd"!p" suffix,
the constraint for \xcd"T" is implicitly assumed to contain the clause
\xcd"self.home==p"; that is,
\xcd"C{c}!p" is equivalent to \xcd"C{self.home==p && c}".

The place \xcd"p" may be ommitted. It defaults to \xcd"this" 
for types in field declarations, and to \xcd"here" elsewhere.


% The place specifier \xcd"any" specifies that the object can be
% located anywhere.  Thus, the location is unconstrained; that is,
% \xcd"C{c}!any" is equivalent to \xcd"C{c}".

% XXX ARRAY
%The place specifier \xcd"current" on an array base type
%specifies that an object with that type at point \xcd"p"
%in the array 
%is located at \xcd"dist(p)".  The \xcd"current" specifier can be
%used only with array types.




\subsection{Constraint semantics}

\begin{staticrule}{Variable occurrence}
In a dependent type \xcd"T" = \xcd"C{c}", the only variables that may
occur in \xcd"c" are (a)
\xcd"self", (b) properties visible at \xcd"T", (c) final local variables, final
method parameters or final constructor parameters visible at \xcd"T", (d)
final fields visible at \xcd"T"'s lexical place in the source program.  
\end{staticrule}

\begin{staticrule}{Restrictions on \xcd"this"}
  The special variable \xcd"this" may be used in a dependent clause for a type \xcd"T"
  only if (a)~\xcd"T" occurs in a property declaration for a
  class, (b)~\xcd"T"
  occurs in an instance method, (c)~\xcd"T" occurs in an
  instance field, (d)~\xcd"T"
  occurs in an instance initializer.

  In particular, \xcd"this" may not be used in types that occur in a static
  context, or in the arguments, body or return type of a constructor or
  in the extends or implements clauses of class and interface
  definitions.  In these contexts, the object that \xcd"this" would
  correspond to is not defined.
\end{staticrule}

\begin{staticrule}{Variable visibility}
  If a type \xcd"T" occurs in a field, method or constructor
  declaration, then all variables used in \xcd"T" must have at least the
  same visibility as the declaration.  The relation ``at least the same
  visibility as'' is given by the transitive closure of:

\begin{xten}
public > protected > package > private
\end{xten}

All inherited properties of a type \xcd"T" are visible in the property
list of \xcd"T", and the body of \xcd"T".

\end{staticrule}

In general, variables (i.e., local variables, parameters,
properties, fields) are visible at
\xcd"T" if they are defined before \xcd"T" in the program. This rule applies to
types in property lists as well as parameter lists (for methods and
constructors).
A formal parameter is visible in the types of all other formal
parameters of the same method, constructor, or type definition,
as well as in the method or constructor body itself.
Properties are accessible via their containing object--\xcd"this"
within the body of their class declaration.  The special
variable \xcd"this" is in scope at each property
declaration, constructor signatures and bodies, instance method signatures
and bodies,
and instance field signatures and initializers, but not in scope
at \xcd"static" method or field declarations or \xcd"static"
initializers.  

We permit variable declarations \xcd"v: T" where \xcd"T" is obtained
from a dependent type \xcd"C{c}" by replacing one or more occurrences
of \xcd"self" in \xcd"c" by \xcd"v". (If such a declaration \xcd"v: T"
is type-correct, it must be the case that the variable \xcd"v" is not
visible at the type \xcd"T". Hence we can always recover the
underlying dependent type \xcd"C{c}" by replacing all occurrences of \xcd"v"
in the constraint of \xcd"T" by \xcd"self".)

For instance, \xcd"v: Int{v > 0}" is shorthand for \xcd"v: Int{self > 0}".

\begin{staticrule}{Constraint type}
The type of a constraint \xcd"c" must be \xcd"Boolean".  
\end{staticrule}

A variable occurring in the constraint \xcd"c" of a dependent type, other than
\xcd"self" or a property of \xcd"self", is said to be a {\em
parameter} of \xcd"c".\label{DepType:Parameter} \index{parameter}

An instance \xcd"o" of \xcd"C" is said to be of type \xcd"C{c}"
(or: {\em belong to}
\xcd"C{c}") if the predicate \xcd"c" evaluates to \xcd"true" in the current lexical
environment, augmented with the binding \xcd"self" $\mapsto$ \xcd"o". We shall
use the function \LL{\mbox{\Xcd{C\{c\}}}} to denote the set of
objects that belong to \xcd"C{c}". 

\subsection{Type invariants}\label{DepType:TypeInvariant}\index{Type invariant}
\label{DepType:Guard}

A class or interface declaration may specify a {\em guard},
an invariant on all
instances of that type.
All the properties of the type, together with inherited properties,
may appear in the class invariant.
\iftypeparams
The class invariant may also constrain the class's type parameters.
\fi
A guard \xcd"c" with
\iftypeparams\else
type property list \xcdmath"X$_1$, $\dots$, X$_k$" and
\fi
value property list \xcdmath"x$_1$: T$_1$, $\dots$, x$_n$: T$_n$"
for a class \xcd"C" is said to be consistent if each of the \xcdmath"T$_i$" are
consistent and the constraint
\iftypeparams
\begin{xtenmath}
$\exists$x$_1$: T$_1$, $\dots$, x$_n$: T$_n$, self: C. c
\end{xtenmath}
\else
\begin{xtenmath}
$\exists$X$_1$, $\dots$, X$_k$, x$_1$: T$_1$, $\dots$, x$_n$: T$_n$, self: C. c
\end{xtenmath}
\fi
\noindent is valid (always true).

With every defined class or interface \xcd"T" we associate a {\em type
invariant} $\mathit{inv}($\xcd"T"$)$ as follows. The type
invariant associated with \xcd"x10.lang.Object" is 
\xcd"true".

The type invariant associated with any interface \xcd"I" that extends
interfaces \xcdmath"I$_1$, $\dots$, I$_k$" and defines properties
\xcdmath"x$_1$: P$_1$, $\dots$, x$_n$: P$_n$" and
specifies a guard \xcd"c" is given by:

\begin{xtenmath}
$\mathit{inv}$(I$_1$) && $\dots$ && $\mathit{inv}$(I$_k$)
    && self.x$_1$: P$_1$ && $\dots$ && self.x$_n$: P$_n$ && c  
\end{xtenmath}

Similarly the type invariant associated with any class \xcd"C" that
implements interfaces \xcdmath"I$_1$, $\dots$, I$_k$",
extends class \xcd"D" and defines properties
\xcdmath"x$_1$: P$_1$, $\dots$, x$_n$: P$_n$" and
specifies a guard \xcd"c" is
given by:
\begin{xtenmath}
$\mathit{inv}$(D) && $\mathit{inv}$(I$_1$) && $\dots$ && $\mathit{inv}$(I$_k$)
    && self.x$_1$: P$_1$ && $\dots$ && self.x$_n$: P$_n$ && c  
\end{xtenmath}

It is required that the
type invariant associated with a class entail the type invariants of
each interface that it implements.

It is guaranteed that for any variable \xcd"v" of
type \xcd"T{c}" (where \xcd"T" is an interface name or a class name) the only
objects \xcd"o" that may be stored in \xcd"v" are such that \xcd"o" satisfies
$\mathit{inv}(\mbox{\xcd"T"}[\mbox{\xcd"o"}/\mbox{\xcd"this"}])
\wedge \mbox{\xcd"c"}[\mbox{\xcd"o"}/\mbox{\xcd"self"}]$.

\subsection{Consistency of dependent types}\label{DepType:Consistency}\index{dependent type,consistency}

A dependent type \xcd"C{c}" may contain zero or more parameters. We require
that a type never be empty---so that it is possible for a variable of
the type to contain a value. This is accomplished by requiring that
the constraint \xcd"c" must be satisfiable {\em regardless} of the value assumed
by parameters to the constraint (if any). Formally, consider a type
\xcd"T" = \xcd"C{c}", with the variables
\xcdmath"f$_1$: F$_1$, $\dots$, f$_k$: F$_k$"
free in \xcd"c".  Let 
\xcdmath"$S$ = {f$_1$: F$_1$, $\dots$, f$_k$: F$_k$, f$_{k+1}$: F$_{k+1}$, $\dots$, f$_n$: F$_n$}"
be the smallest set of
declarations containing
\xcdmath"f$_1$: F$_1$, $\dots$, f$_k$: F$_k$"
and closed under the rule:
\xcd"f: F" in $S$ if a reference to variable \xcd"f" (which
is declared as \xcd"f: F") occurs in a type in $S$.

(NOTE: The syntax rules for the language ensure that $S$ is always
finite. The type for a variable \xcd"v" cannot reference a variable whose
type depends on \xcd"v".)

We say that \xcd"T" = \xcd"C{c}" is {\em parametrically consistent} (in brief:
{\em consistent}) if:

\begin{itemize}
\item Each type \xcdmath"F$_1$, $\dots$, F$_n$" is (recursively) parametrically consistent, and
\item It can be established that
\xcdmath"$\forall$f$_1$: F$_1$, $\dots$, f$_n$: F$_n$. $\exists$self: C. c && $\mathit{inv}$(C)".
\end{itemize}

\noindent
where \xcdmath"$\mathit{inv}$(C)" is the invariant associated
with the type \xcd"C" (\Sref{DepType:TypeInvariant}).  Note by
definition of $S$ the formula above has no free variables.

\begin{staticrule*}
For a declaration \xcd"v: T" to be type-correct, \xcd"T" must be parametrically
consistent. The compiler issues an error if it cannot determine
the type is parametrically consistent.
\end{staticrule*}

\begin{example}

A class that represents a line has two distinct points:

\begin{xten}
class Line(start: Point, 
           end: Point{self != this.start}) {...}      
\end{xten}
\end{example}

One can use dependent type to define other closed geometric figures as well.

\begin{example}
Here is an example:
\begin{xten}
class Point(x: Int, y: Int) {...}
\end{xten}


To see that the declaration \xcd"end: Point{self != start}" is
parametrically consistent, note that the following formula is valid:
\begin{xtenmath}
$\forall$this: Line. $\exists$self: Point. self != this.start  
\end{xtenmath}
\noindent since the set of all \xcd"Point"s has more than one element.
\end{example}

\begin{example}
A triangle has three lines sharing three vertices.
\begin{xten}
class Triangle 
 (a: Line, 
  b: Line{a.end == b.start}, 
  c: Line{b.end == c.start && c.end == a.start}) 
 { ... }
\end{xten}


Given \xcd"a: Line", the type \xcd"b: Line{a.end == b.start}" is consistent,
and
given the two, the type \xcd"c: Line{b.end == c.start, c.end == a.start}"
is consistent.

%%Similarly:
%%
%%   // A class with properties a, b,c,x satisfying the 
%%   // given constraints.
%%   class SolvableQuad(a: Int, b: Int, 
%%                      c: Int{b*b - 4*a*c >= 0},
%%                      x: Int{a*x*x + b*x + c==0}) { 
%%     ...
%%   }
%%
%%  // A class with properties r, x, and y satisfying
%%  // the conditions for (x,y) to lie on a circle with center (0,0)
%%  // and radius r.
%%   class Circle (r: Int{r > 0},
%%                 x: Int{r*r - x*x >= 0},
%%                 y: Int{y*y == r*r -x*x}) { 
%%   ...
%%   }
\end{example}

\section{Function types}
\label{FunctionTypes}



\index{types!function types}

        Function types are defined via the \xcd"=>" type
        constructor.  Closures (\Sref{Closures}) and method
        selectors (\Sref{MethodSelectors}) are of function type.
        The general form of a function type is:
\begin{xtenmath}
[X$_1$, $\dots$, X$_m$](x$_1$: T$_1$, $\dots$, x$_n$: T$_n$){c} => T
        throws S$_1$, $\dots$, S$_k$
\end{xtenmath}
        This
        is the type of functions that take type parameters
        \xcdmath"X$_i$"
        and value parameters
        \xcdmath"x$_i$"
        of types
        \xcdmath"T$_i$"
        such that the guard \xcd"c" holds
        and returns a value of type \xcd"T" or throws
        exceptions of 
        types S$_i$.

The
\iftypeparams
type and value parameters
\else
type and value properties
\fi
are in scope throughout the function signature---they may be
used
in the types of other formal parameters and in the return type.
Value parameters names (but not types) may be omitted if
they are not used. 
Like other types, function types may be constrained.

\begin{grammar}
FunctionType \: TypeParameters\opt \xcd"(" Formals\opt \xcd")" Constraint\opt
\xcd"=>" Type Throws\opt \\
TypeParameters \: \xcd"[" TypeParameter ( \xcd"," TypeParameter
)\star \xcd"]" \\
TypeParameter \: Identifier \\
Formals \: Formal ( \xcd"," Formal )\star \\
\end{grammar}

\subsection{Functions are value objects}
\label{FunctionsAreObjects}

Functions in \Xten{} are value objects.  
The function type
\xcdmath"[X$_1$, $\dots$, X$_k$](x$_1$: T$_1$ $\dots$, x$_n$: T$_n$) => S"
may be considered equivalent to an interface type
with a method:
\begin{xtenmath}
def apply[X$_1$, $\dots$, X$_k$](x$_1$: Y$_1$, $\dots$, x$_n$: Y$_n$): Z;
\end{xtenmath}
Classes and interfaces may implement or extend a function type
by defining the \xcd"apply" method.  Since each function type is 
anonymous, a class or interface may implement or extend
more than one function type as long as the \xcd"apply" methods
do not have the same signature.

As objects, closure body may refer to
\xcd"this", which is a reference to the current function,
and use it to invoke the closure recursively.

\section{Boxed types}
\label{BoxedTypes}
\label{BoxType}

\index{types!boxed type}
\index{\Xcd{Box}}

The reference type \xcd"x10.lang.Box[T]" is a \emph{boxed type},
and allows a value type \xcd"T" to be used at a
reference type.
%
If \xcd"T" is a value type, \xcd"Box[T]" is a reference type
with the same methods and fields as \xcd"T".
If \xcd"T" is a reference type, \xcd"Box[T]" is equivalent to
\xcd"T".

Fields and methods of \xcd"T" may be accessed through an
expression of type \xcd"Box[T]".
If \xcd"T" implements an interface \xcd"I", then
\xcd"Box[T]" also implements \xcd"I". 
%% It used to say 
%% Note that if \xcd"T" is a subtype of \xcd"S",
%% then \xcd"Box[T]" and \xcd"Box[S]" are unrelated
%% But what have we said before that would allow the reader to infer this?

If \xcd"T" is a subtype of \xcd"S", then \xcd"Box[T]" and \xcd"Box[S]"
are unrelated.

There are implicit coercions from value objects to boxed value
objects (\Sref{XtenConversions}).

% The type \xcd"Box[T]" is a subtype of \xcd"T".

The \xcd"to" expression may be used to convert to and from a
boxed type:

\begin{xten}
val x = new Box[Int](0); // box using constructor call
val y = x to Int;        // unbox using to-operation
val z = y to Box;        // box using to-operation
\end{xten}

Assignment of a value type \xcd"T" into a variable of compatible reference type
(either \xcd"Box[T]", \xcd"Ref", \xcd"Object", or an interface that is
implemented by \xcd"T")
automatically boxes the value.  Assignment of a boxed type
\xcd"Box[T]" into
a variable of value type \xcd"T" automatically unboxes the value.

\eat{
Access to a field of a value of boxed type
automatically unboxes the value and accesses the field
of the unboxed value; that is, if \xcd"x: Box[T]",
then \xcd"x.f" is equivalent to \xcd"(x to T).f".

A method invocation on a value of boxed type
automatically unboxes the value and 
invokes the method on the unboxed value; that is, if \xcd"x: Box[T]",
then \xcd"x.m()" is equivalent to \xcd"(x to T).m()".
}


\section{Annotated types}
\label{AnnotatedTypes}

\index{types!annotated types}
\index{annotations!type annotations}

        Any \Xten{} type may be annotated with zero or more
        user-defined \emph{type annotations}
        (\Sref{XtenAnnotations}).  

        Annotations are defined as (constrained) interface types and are
        processed by compiler plugins, which may interpret the
        annotation symbolically.

        A type \xcd"T" is annotated by with interface types
        \xcdmath"A$_1$", \dots,
        \xcdmath"A$_n$"
        using the syntax
        \xcdmath"@A$_1$ $\dots$ @A$_n$ T".

\section{Subtyping and type equivalence}\label{DepType:Equivalence}
\index{type equivalence}
\index{subtyping}

Subtyping is relation between types.  It is the
reflexive, transitive 
closure of the {\em direct subtyping} relation, defined as
follows.

\paragraph{Class types.}  A class type is a direct subtype of
any
class it is declared to extend.  A class type is direct subtype
of any interfaces it is declared to implement.

\paragraph{Interface types.}  An interface type is a direct
subtype of any interfaces it is declared to extend.

\paragraph{Boxed types.}
The type \xcd"Box[T]" is a subtype of any interface implemented
by \xcd"T".

\paragraph{Function types.}

Function types are covariant on their return type and
contravariant on their argument types.
For instance,
a function type
\xcd"(S1) => T1" 
is a subtype of another function type
\xcd"(S2) => T2" 
if \xcd"S2" is a subtype of \xcd"S1"
and \xcd"T1" is a subtype of \xcd"T2".

\paragraph{Constrained types.}

Two dependent types \xcd"C{c}" and \xcd"C{d}" are said to be {\em equivalent} if 
\xcd"c" is true whenever \xcd"d" is, and vice versa. Thus, 
$\LL{\mbox{\Xcd{C\{c\}}}} = \LL{\mbox{\Xcd{C\{d\}}}}$.

Note that two dependent type that are syntactically different may be
equivalent. For instance, \xcd"Int{self >= 0}" and
\xcd"Int{self == 0 || self > 0}" are equivalent though they are syntactically
distinct. The \Java{} type system is essentially a nominal system---two
types are the same if and only if they have the same name. The \Xten{}
type system extends the nominal type system of \Java{} to permit
constraint-based equivalence.

A dependent type \xcd"C{c}" is a subtype of a type \xcd"C{d}" if
\xcd"c" implies \xcd"d".  In such a case we have
$\LL{\mbox{\Xcd{C\{c\}}}}$ is a
subset of $\LL{\mbox{\Xcd{C\{d\}}}}$. All dependent types
defined on a class \xcd"C"
refine the unconstrained class type \xcd"C"; \xcd"C" is
equivalent to \xcd"C{true}".

\paragraph{Path types.}

A path type \xcd"p.X" is a subtype of a type \xcd"T" if
\xcd"p" has type \xcd"C{c}" (where \xcd"C" has a type member
\xcd"X") and \xcd"c" implies that \xcd"self.X" is a subtype of \xcd"T".

\paragraph{Type parameters.}

A type parameter \xcd"X" of a class or interface \xcd"C"
is a subtype of a type \xcd"T" if
the 
class invariant of \xcd"C" implies that \xcd"X" is a subtype of \xcd"T".
Similarly, \xcd"T" is a subtype of parameter \xcd"X" if the
class invariant implies the relationship.

A type parameter \xcd"X" of a method
\iftypeparams\else
or constructor
\fi
\xcd"m"
is a subtype of a type \xcd"T" if
the 
guard of \xcd"m" implies that \xcd"X" is a subtype of \xcd"T".
Similarly, \xcd"T" is a subtype of parameter \xcd"X" if the
guard implies the relationship.


\section{Least common ancestor of types}
\label{LCA}

To compute the type of conditional expressions
(\Sref{Conditional}),
and of rail constructors
(\Sref{RailConstructors}), the least common ancestor of types
must be computed.

The least common ancestor of two  types
\xcdmath"T$_1$" and \xcdmath"T$_2$"
is the
unique most-specific type
that is a supertype of both
\xcdmath"T$_1$" and \xcdmath"T$_2$".

If the most-specific type is not unique (which can happen when
\xcdmath"T$_1$" and \xcdmath"T$_2$" both implement two
or more incomparable interfaces), then
least common ancestor type is \xcd"x10.lang.Object".

\section{Coercions and conversions}
\label{XtenConversions}
\label{User-definedCoercions}
\index{conversions}\index{coercions}

\XtenCurrVer{} supports the following coercions and conversions

\subsection{Coercions}

A {\em coercion} does not change object identity; a coerced object may
be explicitly coerced back to its original type through a cast. A {\em
  conversion} may change object identity if the type being converted
to is not the same as the type converted from. \Xten{} permits
user-defined conversions (\Sref{sec:user-defined-conversions}).

\paragraph{Subsumption coercion.}
A subtype may be implicitly coerced to any supertype.
\index{coercions!subsumption coercion}

\paragraph{Explicit coercion (casting with \xcd"as")}
A reference type may be explicitly coerced to any other
reference type using the \xcd"as" operation.
If the value coerced is not an instance of the target type,
a \xcd"ClassCastException" is thrown.  Casting to a constrained
type may require a run-time check that the constraint is
satisfied.
\index{coercions!explicit coercion}
\index{casting}
\index{\Xcd{as}}

\subsection{Conversions}

\paragraph{Narrowing conversion.}
A value class may be explicitly converted to any superclass
using the \xcd"as" operation.

\index{conversions!narrowing conversions}

\paragraph{Widening numeric conversion.}
A numeric type may be implicitly converted to a wider numeric type. In
particular, an implicit conversion may be performed between a numeric
type and a type to its right, below:

\begin{xten}
Byte < Short < Int < Long < Float < Double
\end{xten}

\index{conversions!widening conversions}
\index{conversions!numeric conversions}

\paragraph{String conversion.}
Any object that is an operand of the binary
\xcd"+" operator may
be converted to \xcd"String" if the other operand is a \xcd"String".
A conversion to \xcd"String" is performed by invoking the \xcd"toString()"
method of the object.

\index{conversions!string conversion}

\paragraph{User defined conversions.}\label{sec:user-defined-conversions}
\index{conversions!user defined}

The user may define conversion operators from type \Xcd{A} {\em to} a
container type \Xcd{B} by specifying a method on \Xcd{B} as follows:

\begin{xten}
  public static operator (r: A): T = ... 
\end{xten}

The return type \Xcd{T} should be a subtype of \Xcd{B}. The return
type need not be specified explicitly; it will be computed in the
usual fashion if it is not. However, it is good practice for the
programmer to specify the return type for such operators explicitly.

For instance, the code for \Xcd{x10.lang.Point} contains:

\begin{xten}
  public static global safe operator (r: Rail[int])
     : Point(r.length) = make(r);
\end{xten}

The compiler looks for such operators on the container type \Xcd{B}
when it encounters an expression of the form \Xcd{r as B} (where
\Xcd{r} is of type \Xcd{A}). If it finds such a method, it sets the
type of the expression \Xcd{r as B} to be the return type of the
method. Thus the type of \Xcd{r as B} is guaranteed to be some subtype
of \Xcd{B}.

\begin{example}
Consider the following code:  
\begin{xten}
val p  = [2, 2, 2, 2, 2] as Point;
val q = [1, 1, 1, 1, 1] as Point;
val a = p - q;    
\end{xten}
This code fragment compiles successfully, given the above operator definition. 
The type of \Xcd{p} is inferred to be \Xcd{Point(5)} (i.e.{} the type 
\xcd"Point{self.rank==5}".
Similarly for \Xcd{q}. Hence the application of the operator ``\Xcd{-}'' is legal (it requires both arguments to have the same rank). The type of \Xcd{a} is computed as \Xcd{Point(5)}.
\end{example}


%\subsection{Syntactic abbreviations}\label{DepType:SyntaxAbbrev}


\section{Built-in types}

The package \xcd"x10.lang" provides a number of built-in class and
interface declarations that can be used to construct types.

\subsection{The interface \Xcd{Object}}
\label{Object}\index{\Xcd{Object}}\index{\Xcd{x10.lang.Object}}

The interface \xcd"x10.lang.Object" is the supertype of all
reference and value classes.
A variable of this type can hold a reference to an instance of any
reference type, including boxed value types.

\begin{xten}
package x10.lang;
public interface Object {
  def toString(): String {...}
  def equals(o: Object): Boolean {...}
  def hashCode(): Int {...}
}
\end{xten}

The method \xcd"equals" and \xcd"hashCode" are useful in hash tables,
and are defined as in \java. The default implementation of \xcd"equals"
is stable equality, \Sref{StableEquality}. This method may be overridden
in a (value or reference) subclass.

\subsection{The class \Xcd{Ref}}
\label{Ref}\index{\Xcd{Ref}}\index{\Xcd{x10.lang.Ref}}

The class \xcd"x10.lang.Ref" is the superclass of all reference
classes.  A variable of this type can hold a reference to an
instance of any reference type, including boxed value types.

\subsection{The class \Xcd{Value}}
\label{Value}\index{\Xcd{Value}}\index{\Xcd{x10.lang.Value}}

The value class \xcd"x10.lang.Value" is the superclass of by all value
classes.
A variable of this type can hold an instance of a value type.

\subsection{The class \Xcd{String}}
\label{String}\index{\Xcd{String}}\index{\Xcd{x10.lang.String}}

All strings in \Xten{} are instances of the value class
\xcd"x10.lang.String".
A string object is immutable,
and has a concatenation operator (\xcd"+") available on it.

\subsection{Primitive value classes}

Several value types are provided that encapsulate
abstractions (such as fixed point and floating point arithmetic)
commonly implemented in hardware by modern computers.

\begin{xten}
package x10.lang;

public value Boolean { }

public abstract value Number { }
public abstract value Integer extends Number { }

public value Byte extends Integer  { }
public value Short extends Integer { }
public value Int extends Integer   { }
public value Char extends Integer  { }
public value Long extends Integer  { }

public value Float extends Number  { }
public value Double extends Number { }
\end{xten}

A program may contain literals
that stand for specific instances of these classes. The syntax
for literals is the same as for \java{} (\Sref{Literals}).

Binary and unary operations produce new instances of these
classes.

Values of one numeric class may be coerced into values of
another value class via the \xcd"to" operation.

\futureext{
\Xten{} may provide mechanisms in the future to permit the programmer
to specify how a specific value class is to be mapped to special
hardware operations (e.g., along the lines of
\cite{kava}). Similarly, mechanisms may be provided to permit the user
to specify new syntax for literals.
}

\section{Array Type Constructors}
\label{ArrayypeConstructors}\index{array types}

{}\XtenCurrVer{} does not have array class declarations
(\S~\ref{XtenArrays}). This means that user cannot define new array
class types. Instead arrays are created as instances of array types
constructed through the application of {\em array type constructors}
(\S~\ref{XtenArrays}).

The array type constructor takes as argument a type (the {\em base
type}), an optional distribution (\S~\ref{XtenDistributions}), and
optionally either the keyword {\cf reference} or {\cf value} (the
default is reference):
\begin{x10}
18    ArrayType ::= Type [ ]
438   ArrayType ::= X10ArrayType
439   X10ArrayType ::= Type [ . ]
440     | Type reference [ . ]
441     | Type value [ . ]
442     | Type [ DepParameterExpr ]
443     | Type reference [ DepParameterExpr ]
444     | Type value [ DepParameterExpr ]
\end{x10}

The array type {\cf Type[ ] } is the type of all arrays of base
type {\tt Type} defined over the distribution {\tt 0:N -> here}
for some positive integer {\tt N}.

The qualifier {\tt value} ({\tt reference}) specifies that the array
is a {\tt value}({\tt reference}) array. The array elements of a {\tt
value} array are all {\tt final}.\footnote{Note that the base type of a {\tt value} array can be a value class or a reference class, just as the 
type of a {\tt final} variable can be a value class or a reference class.
}If the qualifier is not specified,
the array is a {\tt reference} array.

The array type {\cf Type reference [.]} is the type of all (reference)
arrays of basetype {\tt Type}. Such an array can take on any
distribution, over any region. Similarly, {\cf Type value [.]} is the
type of all value arrays of basetype {\tt Type}.

\XtenCurrVer{} also allows a distribution to be specified between {\tt
[} and {\tt ]}. The distribution must be an expression of type
{\tt distribution} (e.g.{} a {\tt final} variable) whose
value does not depend on the value of any mutable variable.

Future extensions to \Xten{} will support a more general syntax for
arrays which allows for the specification of dependent types, 
e.g.{} {\tt double[:rank 3]}, the type of all arrays of 
{\tt double} of rank {\tt 3}.



\subsection{Future types}

The class \xcd"x10.lang.Future[T]"
is the type of all \xcd"future" expressions.
The type represents a value which when forced will return a value of type
\xcd"T". The class makes available the following methods:

\begin{xten}
package x10.lang;
public class Future[T] implements () => T {
  public def apply(): T = force();
  public def force(): T = ...;
  public def forced(): Boolean = ...;
}
\end{xten}
  


\section{Type inference}
\label{TypeInference}
\index{types!inference}
\index{type inference}

\XtenCurrVer{} supports limited local type inference, permitting
variable types and return types to be elided.
It is a static error if an omitted type cannot be inferred or
uniquely determined.

\subsection{Variable declarations}

The type of a variable declaration can be omitted if the
declaration has an initializer.  The inferred type of the
variable is the computed type of the initializer.

\subsection{Return types}

The return type of a method can be omitted if the method has a
body (i.e., is not \xcd"abstract" or \xcd"extern").  The
inferred return type is the computed type of the body.

The return type of a closure can be omitted.
The inferred return type is the computed type of the body.

The return type of a constructor can be omitted if the
constructor has a body (i.e., is not \xcd"extern").
The inferred return type if the enclosing class type with
properties bound to the arguments in the constructor's \xcd"property"
statement, if any, or to the unconstrained class type.

The inferred type of a method or closure body is the least common ancestor
of the types of the expressions in \xcd"return" statements
in the body.  If the method does not return a value, the
inferred type is \xcd"Void".

\subsection{Type arguments}

A call to a polymorphic method, closure, or constructor may omit the
explicit type arguments.  If the method has a type parameter
\xcd"T", the type argument corresponding to \xcd"T" is inferred
to be the least common ancestor of the types of any formal
parameters of type \xcd"T".

Consider the following method:
\begin{xten}
def choose[T](a: T, b: T): T { ... }
\end{xten}
%
Given \xcd"Set[T] <: Collection[T]"
and \xcd"List[T] <: Collection[T]",
in the following snippet, the algorithm should infer the type
\xcd"Collection" for \xcd"x".
\begin{xten}
intSet: Set[Int];
stringList: List[String];
val x = choose(intSet, stringList);
\end{xten}
%
And in this snippet, the algorithm should infer the type
\xcd"Collection[Int]" for \xcd"y".
\begin{xten}
intSet: Set[Int];
intList: List[Int];
val y = choose(intSet, intList);
\end{xten}
%
Finally, in this snippet, the algorithm should infer the type
\xcd"Collection{T <: Number}" for \xcd"z".
\begin{xten}
intSet: Set[Int];
numList: List{T <: Number};
val z = choose(intSet, numList);
\end{xten}

Now, consider the following example:
\begin{xten}
def union[T](a: Set[T], b: Set[T]) : Set[T];
\end{xten}
The union method cannot be called with just arguments of type \xcd"Set".
\begin{xten}
set1: Set;
set2: Set;
val a = union(set1, set2);
\end{xten}
This is illegal because the type system cannot demonstrate that
\xcd"set1.T" and \xcd"set2.T" are equal.
The following, however, is acceptable:
\begin{xten}
set1: Set;
set2: Set[set1.T];
val a = union(set1, set2);
\end{xten}

However, unlike \xcd"union" above,
the following method can be called with an argument of type \xcd"Set"
because there are no constraints on \xcd"T":
\begin{xten}
def unmodifiableSet[T](set: Set[T]): Set[T];
\end{xten}

Using desugared syntax, the method is
equivalent to: 
\begin{xten}
def unmodifiableSet[T](set: Set{self.T==T}): Set{self.T==T};
\end{xten}
Any \xcd"Set" can be passed in: for an argument \xcd"e", the method
is instantiated on \xcd"e.T".
%
Note that if this method were defined as:
\begin{xten}
def unmodifiableSet(set: Set): Set;
\end{xten}
then the connection between the element types of the
argument and of the return types would be broken.
However, one could use the following signature to keep the
connection, without introducing a method type parameter. 
\begin{xten}
def unmodifiableSet(set: Set): Set[set.T];
\end{xten}

