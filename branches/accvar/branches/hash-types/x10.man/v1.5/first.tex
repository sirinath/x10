% First page

\thispagestyle{empty}

% \todo{"another" report?}

\topnewpage[{
\begin{center}   
{\huge\bf Report on the Experimental Language \Xten{}}
\vskip 1ex
$$
\begin{tabular}{l@{\extracolsep{.5in}}lll}
\multicolumn{4}{c}{\sc  Version 1.1}\\
\multicolumn{4}{c}{\sc Please send comments to 
V\authorsc{IJAY} S\authorsc{ARASWAT} at 
{\tt vsaraswa@us.ibm.com}}\\
%\multicolumn{4}{c}{({\sc IBM Confidential})}

%\ldots
\end{tabular}
$$
\vskip 2ex
% {\it Dedicated to the Memory of APL} % vj
{\bf Jun 30, 2007}
\vskip 2.6ex
\end{center}


}]


\chapter*{Summary}
This draft report provides an initial description of the programming
language \Xten. \Xten{} is a single-inheritance class-based object-oriented
(OO) programming language designed for high-performance, high-productivity
computing on high-end computers supporting $\approx 10^5$ hardware threads
and $\approx 10^{15}$ operations per second. 

{}\Xten{} is based on state-of-the-art object-oriented programming
languages and deviates from them only as necessary to support its
design goals. The language is intended to have a simple and clear
semantics and be readily accessible to mainstream OO programmers. It
is intended to support a wide variety of concurrent programming
idioms.
%, incuding data parallelism, task parallelism, pipelining.
%producer/consumer and divide and conquer.

%We expect to revise this document in the light of experience gained in implementing
%and using this language.

The \Xten{} design team consists of
D\authorsc{AVID} B\authorsc{ACON}, 
R\authorsc{AJ} B\authorsc{ARIK}, 
G\authorsc{ANESH} B\authorsc{IKSHANDI}, 
B\authorsc{OB} B\authorsc{LAINEY}, 
P\authorsc{HILIPPE} C\authorsc{HARLES}, 
P\authorsc{ERRY} C\authorsc{HENG}, 
C\authorsc{HRISTOPHER} D\authorsc{ONAWA}, 
J\authorsc{ULIAN} D\authorsc{OLBY}, 
K\authorsc{EMAL} E\authorsc{BCIO\u{G}LU},
R\authorsc{OBERT} F\authorsc{UHRER},
P\authorsc{ATRICK} G\authorsc{ALLOP}, 
C\authorsc{HRISTIAN} G\authorsc{ROTHOFF}, 
A\authorsc{LLAN} K\authorsc{IELSTRA}, 
S\authorsc{REEDHAR} K\authorsc{ODALI}, 
S\authorsc{RIRAM} K\authorsc{RISHNAMOORTHY}, 
N\authorsc{ATHANIEL} N\authorsc{YSTROM}, 
I\authorsc{IGOR} P\authorsc{ESHANSKY}, 
V\authorsc{IJAY} S\authorsc{ARASWAT} (contact author), 
V\authorsc{IVEK} S\authorsc{ARKAR},
A\authorsc{RMANDO} S\authorsc{OLAR-LEZAMA},  
C\authorsc{HRISTOPH von} P\authorsc{RAUN},
P\authorsc{RADEEP} V\authorsc{ARMA},
K\authorsc{RISHNA} V\authorsc{ENKATA},
J\authorsc{AN} V\authorsc{ITEK}, and
T\authorsc{ONG} W\authorsc{EN}.

For extended discussions and support we would like to thank: 
Robert Callahan, Calin
Cascaval, Norman Cohen, Elmootaz Elnozahy, John Field, Bob Fuhrer,
Orren Krieger, Doug Lea, John McCalpin, Paul McKenney, Ram Rajamony,
R.K.~Shyamasundar, Filip Pizlo, V.T.~Rajan, Frank Tip, and Mandana Vaziri.

We thank Jonathan Rhees and William Clinger with help in obtaining the
\LaTeX{} style file and macros used in producing the Scheme report,
after which this document is based. We acknowledge the influence of
the $\mbox{\Java}^{\mbox{TM}}$ Language Specification \cite{jls2}
document, as evidenced by the numerous citations in the text.

This document revises Version {\cf 1.01} of the Report, released in
December 2006. It documents the language corresponding to the first
revision of the first version of the implementation.  This
revision was done by
R\authorsc{AJ} B\authorsc{ARIK}, 
P\authorsc{HILIPPE} C\authorsc{HARLES}, 
C\authorsc{HRISTOPHER} D\authorsc{ONAWA}, 
R\authorsc{OBERT} F\authorsc{UHRER},
N\authorsc{ATHANIEL} N\authorsc{YSTROM},  
V\authorsc{IJAY} S\authorsc{ARASWAT},
V\authorsc{IVEK} S\authorsc{ARKAR},
P\authorsc{RADEEP} V\authorsc{ARMA} and
K\authorsc{RISHNA} V\authorsc{ENKATA}.
(Earlier implementations benefited from significant contributions by
C\authorsc{HRISTIAN} G\authorsc{ROTHOFF} and 
C\authorsc{HRISTOPH von} P\authorsc{RAUN}.)
T\authorsc{ONG} W\authorsc{EN} has written many application programs
in \Xten{}. G\authorsc{UOJING} C\authorsc{ONG} has helped in the
development of many applications.


%\vfill
%\begin{center}
%{\large \bf
%*** DRAFT*** \\
%%August 31, 1989
%\today
%}\end{center}

\vfill
\eject


\chapter*{Contents}
\addvspace{3.5pt}                  % don't shrink this gap
\renewcommand{\tocshrink}{-3.5pt}  % value determined experimentally
{\footnotesize
\tableofcontents
}

\vfill
\eject


