\Xten{} is a class-based object-oriented language
that provides both dependent and generic types.
The language has a sequential core similar to Java or Scala, but
also
constructs for concurrency and distribution.

A key feature that interacts with generics is that the type
system provides both reference types and value types.
An instance of a reference type is an object on the
heap.  All reference types are subclasses of \xcd"Object".
Variables of reference type may be \xcd"null".
In contrast, an instance of a value type might be represented in
unboxed form on the call stack
and can never be \xcd"null".
Ideally, the design should support instantiation of generics on both 
reference and value types.

This section describes the design of generics for \Xten,
including several alternative designs.  These alternatives
demonstrate the expressiveness of constrained kinds.

%\footnote{We plan
%to support traits in a future version of the language.}
\todo{emph interactions with data constraints}

\subsection{Type constraints}

To permit genericity, variables \Xcd{X} must be admitted over types.
The choice of type variables is discussed below.  We assume here
that classes have a means of introducing new type variables
either as type parameters or as type members.
For instance,
the class \xcd"List" introduces a type variable \xcd"X"
representing the list's element type.

\Xten already supports constraints over values, so it is natural
to extend these to constraints over types.
Here, we ask: how should type variables be constrained?

Constraints occur in several places in the \Xten syntax.  They
are of course permitted in constrained types \xcd"C{c}".
Constraints may also be used as \emph{class invariants}, 
which are constraints on the class's properties and other
final variables in scope.
The class invariant must be established by the class's
constructor and subsequently holds for all instances of the
class.

Methods and constructors
may also have constraints, or \emph{guards}, on their parameters.
A guard must be satisfied by the caller of the method and
will hold throughout its body.  Type constraints used in the
method guard restrict the types of the arguments or of the method
receiver.  

\subsubsection{Nominal subtyping and equality constraints}

In class-based OO languages such as Java,
types are equipped with a partial
order (the \emph{subtyping} order) generated from the user program
through the ``\Xcd{extends}'' relationship.  
This motivates a very natural constraint system on types.  For a type
variable \Xcd{X} we should be able to assert the constraint
\Xcd{X}~$\extends$~\Xcd{T}: a valuation (mapping from variables to types) realizes
this constraint if it maps \Xcd{X} to a type that extends \Xcd{T}.
Constraints on types can specify either subtype (\xcd"<="),
supertype (\xcd">="), or equality bounds (\xcd"==").

Using subtyping constraints in the class invariant provides a
means to bound the type variables introduced by the class
declaration.  Constraints in constrained types
\xcd"C{c}", can bound
type-valued members of the base type \xcd"C".

These constraints also can be used in method guards.
This feature is similar to optional methods in CLU~\cite{clu} and to generalized type constraints in C$\sharp$~\cite{emir06}.
For instance, given a list of \xcd"T", one could define a
method \xcd"print" with a guard that requires that \xcd"T" be a
subtype of \xcd"Printable":
\begin{xtenmathnoindent}
  def print(){T <= Printable} {
    head.print();
    tail.print();
  }
\end{xtenmathnoindent}
This constraint ensures that the \xcd"head" field of type
\xcd"T" has a \xcd"print()" method.

\subsubsection{Structural constraints}

One should also
be able to require that a type have a
particular member---a field with a given name and type, or a method
with a given name and signature.
We introduce the constraints 
\Xcd{T} \Xcd{has} \Xcd{f:T} and \Xcd{T} \Xcd{has}
\Xcdmath{m($\tbar{x}\ty\tbar{S}$):T} to express this.
These
constraints allow one to define an alternative version of the
\xcd"print" methods above  as:
\begin{xtenmathnoindent}
  def print(){T has print(): void} {
    head.print();
    tail.print();
  }
\end{xtenmathnoindent}
Rather than restricting the actual receiver to lists whose
element type implements \xcd"Printable", with structural
constraints, any list whose element type has a \xcd"print"
method may be used.
\eat{
This feature makes it easier to integrate
third-party libraries, where interface names might not be
compatible.
}

Structural constraints on types are found in many languages.
For instance,
Haskell supports type
classes~\cite{haskell,haskell-type-classes}.
%
%ML's module system allows modules to be constrained by
%structural signatures~\cite{ml}.
In Modula-3, type equivalence is structural
rather than nominal as in object-oriented languages of the C
family (e.g., C++, Java, and \Xten{}).
Unity~\cite{malayeriIntegrating08}
is a Java-like language with both nominal and structural subtyping.

In the class invariant, a structural constraint can bound the
class's type variables, similarly to 
the language PolyJ~\cite{java-popl97}, which allows type
parameters to be
bounded using
structural \emph{where clauses}~\cite{where-clauses}.
For example, a sorted list class
could be written as follows in PolyJ:
{
\begin{xtennoindent}
class SortedList[T] where T {int compareTo(T)} {
  void add(T x) {... x.compareTo(y) ...}
  ...
}
\end{xtennoindent}}
The \xcd"where" clause states that the type parameter
\xcd"T" must have a
method \xcd"compareTo" with the given signature.
\xcd"SortedList" can be instantiated on any type
that provides the method.
With nominal bounds, \xcd"SortedList" could only require that
its parameters implement an interface such as \xcd"Comparable".

\subsubsection{Default values}

Recall that \Xten's type system provides both reference types and value
types.
In languages like Java with
primitive types, every type has a default value---\xcd"null" for
reference types, \xcd"false" or \xcd"0" for primitive
types---used to initialize arrays of that type.
In \Xten, some types do not have an obvious default value.  For example,
\xcd"int{self>0}" does not contain the value \xcd"0".
Consequently, a useful constraint is \xcd"T has default", which
holds if the type \xcd"T" has a default value.

\subsubsection{{\tt instanceof} constraints}

Lastly, we consider constraints of the form \xcd"x" \xcd"instanceof" \xcd"T".
By relating types and values in a single constraint, 
these constraints provide considerable expressive power.
For instance, 
consider the class declaration:
\begin{xtennoindent}
class C {
  def equals[T](x:T) {this instanceof T} = (this==x);
}
\end{xtennoindent}
The \xcd"equals" method can be called with any object
that is a supertype of \xcd"C".

\xcd"instanceof" constraints can be used to build intersection
types, e.g., \xcd"Object{self instanceof A,self instanceof B}"
should be a subtype of \xcd"A" and \xcd"B"..

\subsection{Type variables}
\label{sec:type-properties}
\label{sec:variance}

Languages such as Java~\cite{Java3} and
Scala~\cite{scala} introduce \emph{type parameters} on classes
and methods.
An alternative approach, used by BETA~\cite{beta}, 
Scala~\cite{scala}, and other languages is, to use type members.
In \Xten, one can
generalize properties to include type-valued properties:
A \emph{type property}
is a final object member initialized at construction time with a
concrete type.  

\subsubsection{Type properties}

\label{sec:usability}
\label{sec:parameters-vs-fields}

Like normal value properties, type properties
can be used in constrained types through the variable \xcd"self".
%
This immediately suggests use-site variance
constraints~\cite{unifying-genericity,variant-parametric-types}
on type properties.
The type of a list of integers, say, can be written as
\xcd"List{self.T==int}".  
Nominal subtyping constraints, then, may be used to
provide use-site variance constraints.
%
Consider the following subtypes of \xcd"List" with type property
\xcd"T":
\begin{itemize}
\item \xcd"List".  This type has no constraints on the type
property \xcd"T".
Any type that constrains \xcd"T"
is a subtype of \xcd"List".
\eat{
The type \xcd"List" is equivalent to \xcd"List{true}".
%
For a \xcd"List" \xcd"v", the return type of the \xcd"get" method
is \xcd"v.T".
Since the property \xcd"T" is unconstrained,
the caller can only assign the return value of \xcd"get"
to a variable of type \xcd"v.T".
}

\item \xcd"List{T==float}".
The type property \xcd"T" is bound to \xcd"float".
For a final expression \xcd"v" of this type,
\xcd"v.T" and \xcd"float" are equivalent types and can be used
interchangeably.

\item \xcdmath"List{T$\extends$Collection}".
This type constrains \xcd"T" to be a subtype of \xcd"Collection".
All instances of this type must bind \xcd"T" to a subtype of
\xcd"Collection"; for example \xcd"List[Set]" (i.e.,
\xcd"List{T==Set}") is a subtype of
\xcdmath"List{T$\extends$Collection}" because \xcd"T==Set" entails
\xcdmath"T$\extends$Collection".

\item \xcdmath"List{T$\super$String}".  This type bounds the type property
\xcd"T"
from below. 
\end{itemize}

While expressive,
type properties have a number of usability issues.
The key difference between type parameters and type properties
is that type properties are
instance \emph{members} bound during object construction.  Type
properties are thus accessible through expressions---\xcd"e.T" is
a legal type (if \xcd"e" is final)---and are inherited by subclasses.
These features give type properties more expressive power than
type parameters, as we shall describe below; however, because they 
provide similar functionality with often subtle distinctions,
type properties can be difficult to use, especially for novices,
and require more care in the design.
For instance,
since type properties are inherited,
the language design needs
to account for ambiguities introduced when the same name is
used for different type properties declared in or inherited into a class.

\eat{
Inheriting type properties may also lead to confusion
As an example, in the following hypothetical code extended with
type properties (declared as normal properties with the ``type''
\xcd"*"),
\xcd"HashMap"  inherits the properties \xcd"K" and \xcd"V" from
\xcd"AbstractMap".
\begin{xten}
class AbstractMap(K:*, V:*) {
  abstract def get(K): V;
  abstract def put(K, V): V;
}

class HashMap implements Map {
  def get(k: K): V = ...;
  def put(k: K, v: V): V = ...;
}
\end{xten}
A user more familiar with type parameters might declare
\xcd"HashMap" as follows:
\begin{xten}
class HashMap(K:*,V:*) implements Map(K,V) {
  def get(k: K): V = ...;
  def put(k: K, v: V): V = ...;
}
\end{xten}
This declaration would introduce a new pair of type properties
named \xcd"K" and
\xcd"V" that shadow the inherited properties.
A na{\"\i}ve implementation of type properties would store run-time
type information for all four properties in each instance
of \xcd"HashMap".
}

\paragraph{\normalfont\bf\em Virtual types.}

Type properties provide expressive power much like 
\emph{virtual
types}~\cite{beta,mp89-virtual-classes,ernst06-virtual};
moreover, they can also
be constrained at the use-site,
can be refined on a per-object basis without explicit subclassing,
and can be refined contravariantly as well as covariantly.

Thorup~\cite{thorup97}
proposed adding genericity to Java using virtual types.  For example,
a generic \xcd"List" class can be written as follows:
{
\begin{xten}
abstract class List {
  abstract typedef T;
  T get(int i) { ... }
}
\end{xten}}
\noindent
The virtual type \xcd"T" is unbound in \xcd"List", but 
can be refined by binding \xcd"T" in a subclass:
{
\begin{xten}
abstract class NumberList extends List {
  abstract typedef T as Number;
}
class IntList extends NumberList {
  final typedef T as Integer;
}
\end{xten}}
\noindent
Only classes where \xcd"T" is final bound, such as \xcd"IntList",
can be non-abstract.  Scala~\cite{scala} supports abstract types
and virtual types in a similar way.
%
The analogous definition of 
\xcd"List" in \Xten{} using type properties is as follows:
{
\begin{xten}
class List(T:*) {
  def get(i: int): T { ... }
}
\end{xten}}

\noindent
Unlike the virtual-type version,
the \Xten{} version of \xcd"List" is not abstract;
\xcd"T" need not be instantiated by a subclass because it can be
instantiated (constrained) on a per-object basis.
Rather than declaring subclasses of \xcd"List",
one uses the constrained subtypes
\xcdmath"List{T$\extends$Number}" and \xcd"List{T==Integer}".

Type properties can also be refined contravariantly.
For instance, one can write the type \xcdmath"List{T$\super$Integer}".

\paragraph{Self types.}

Type properties can also be used to support a form of self
type~\cite{bruce-binary,bsg95}.
%
Self types can be implemented by introducing a
type property \Xcd{type} to the root of the class hierarchy,
\Xcd{Object}:
\begin{xtenmath}
class Object(type:*){type <= Object} { $\dots$ }
\end{xtenmath}

\noindent
For any final path expression \Xcd{p}, the type
$\Xcd{p}.\Xcd{type}$ represents all instances of the fixed,
but statically unknown, run-time class referred to by \Xcd{p}.
Scala's path-dependent types~\cite{scala} and J\&'s
dependent classes~\cite{nqm06}
take a similar approach.

Self types are achieved by
constraining types so that if a path expression \Xcd{p}
has type \Xcd{C}, then
$\Xcd{p}.\Xcd{type} \subtype \Xcd{C}$.
In particular, one can add the class invariant
$\Xcd{this}.\Xcd{type} \subtype \Xcd{C}$ to every class \Xcd{C}.
This invariant ensures that
$\Xcd{this}.\Xcd{type}$ is a subtype
of the lexically enclosing class.

The property must be initialized to the given class, so, without
further language support, one must create an instance of
\xcd"Object" with \xcd"new" \xcd"Object(Object)" to initialize
the \xcd"type" property.

\subsubsection{Type parameters}

Most OO languages provide genericity through type parameters on
classes and methods.  The development of a nominal OO type
system with type parameters is now standard (cf.  FGJ~\cite{FJ}).

Scala~\cite{scala} supports definition-site variance
annotations:
a parameter may be declared in-, co-, or
contravariant.
If the parameter \xcd"X" of a class \xcd"C" is covariant,
then \xcd"S" a subtype of
\xcd"T" implies  \xcd"C[S]" is a subtype of \xcd"C[T]".
Similarly, if \xcd"X" is contravariant,
\xcd"C[T]" is a subtype of \xcd"C[S]".
Invariant parameters are the default; a covariant parameter is
declared by prepending ``\xcd"+"'' to the parameter name in the
class header; a contravariant parameter is declared by
prepending ``\xcd"-"''.  The usage of variant parameter types in
the body of their class must be
restricted to ensure the subtyping relation holds.

Java, by contrast, supports use-site variance through wildcards.
This has a number of usability problems~\cite{wildcards-are-evil},
which also occur with constrained type properties, above.

\subsection{Overloading and dispatch}

The next question to address is the overloading semantics for
methods with constraints on formal parameters and with method
guards.  This issue was considered in non-generic \Xten but was
revisited in light of type constraints.

One option is to ignore constraints when checking for
overloading.  This means that \xcd"m(int{self==0})" and
\xcd"m(int{self==1})", for instance, are considered to have the
same signature; if both occur within the same class, a
compile-time error occurs.

Another option is to allow the
overloading: methods are resolved at compile-time, based on the
constraints.  It is an error if a call could resolve to more
than one method.  One question is whether to rule out overlapping
methods (e.g., \xcd"m(int{self>=0})" and \xcd"m(int{self==1})"),
or to permit them and have the caller resolve any
ambiguities.

\todo{Use type constraints, not value constraints}

Allowing the overloading on constraints can also complicate method
overriding by introducing partial overrides.
Consider:
\begin{xtennoindent}
  class A {
    def m(x:int{self<=0}) = ...; // 1
    def m(x:int{self>=0}) = ...; // 2
  }
  class B {
    def m(x:int{self==0}) = ...; // 3
  }
\end{xtennoindent}
\noindent
\xcd"B.m"'s constraint on \xcd"x" partially overrides the
constraint on both \xcd"m" methods of \xcd"A".  Given a variable
\xcd"b" of type \xcd"B": \xcd"b.m(-1)" invokes \xcd"A.m" (method 1),
\xcd"b.m(0)" invokes \xcd"B.m" (method 3), and \xcd"b.m(1)"
invokes \xcd"A.m" (method 2).  Clients of \xcd"B" could get
confused about which method gets invoked.  One option is to
require that when a method with a given name is overridden, all
other methods with that name should be overridden as well.

Finally, one could support a form of predicate
dispatch~\cite{jpred}, selecting the method to invoke by
\emph{dynamically} evaluating the method guard. 
With type constraints and predicate dispatch, multi-method
dispatch can be implemented.  \todo{example}

\subsection{Implementation}

Finally, we turn to the implementation of generics.
To implement a generic class \xcd"C[X]" one can either generate a single 
class for \xcd"C" in the target language (homogeneous translation)
or generate one class per instantiation
\xcdmath"C[T$_1$]", \dots,
\xcdmath"C[T$_k$]" (heterogeneous translation).
The former approach reduces the amount of generated code; the
latter enables specialization based on the type arguments to
\xcd"C".  Hybrid approaches are possible as well.

Java's approach is to erase type parameters and to use the homogeneous
translation.  Erasure admits more dynamic errors because
it permits, for instance, a \xcd"C<A>" to be cast to \xcd"C<B>".
Retrieving a field of static type \xcd"B" could cause a run-time
type error when an \xcd"A" is returned instead.
The homogeneous translation is aided by a restriction that type
parameters cannot be instantiated on primitive types and by
using nominal subtyping bounds on types.
These restrictions ensure type parameters can be represented
with their type bound, or \xcd"Object" if unbounded.
Since, in \Xten, it should be possible to instantiate a generic
type on both value types and reference types, a homogeneous translation
must box value types so that both kinds of types have the same
representation.

PolyJ~\cite{java-popl97} supports structural bounds and uses a
homogeneous translation with adapter objects to allow generic
code to invoke methods on values of its type parameters.

Other languages, such as C++, use a heterogeneous
translation, specializing the generic class for each
instantiation.
C$\sharp$,
NextGen~\cite{nextgen}, and
Fortress~\cite{fortress} takes this approach as well, reducing
(static) code bloat by instantiating generic classes at run time.

A compromise approach is to specialize for only a few parameter
types, for example the primitive types, but to use a homogeneous
translation otherwise.

Representing type variables at run-time allows the language
to support run-time casts to generic types,
including possibly types instantiated on constrained types.

With
non-generic types, a cast such as
\xcd"r"~\xcd"as"~\xcd"Region{rank==k}" can be implemented by
checking the run-time class of the value being
cast---\xcd"r"~\xcd"instanceof"~\xcd"Region"---and then
evaluating the constraint---\xcd"r.rank==k".
%
However, the issue is more subtle with generic casts.
For instance, to do
\xcd"A"~\xcd"as"~\xcd"Array[int{self>=0}]"
one must check at run time that the concrete type used to instantiate
the \xcd"Array"'s type parameter is equivalent to
\xcd"int{self>=0}".  This check could involve a run-time
entailment check, 
breaking the phase distinction between
compile time and run time for constraint solving.

One approach is to restrict the language 
to rule out casts to type parameters 
and to generic types with subtyping constraints, ensuring that
entailment checks are not needed at run time.
Alternatively, 
the constraint solver could be embedded into the runtime system.
However, this
solution can result in inefficient run-time casts
if entailment checking for the given constraint system is expensive.
Finally, one can simply erase the constraints from the run-time
type information, preserving the base type.  As with Java's
erasure semantics, this approach is prone to run-time type
errors.

\subsection{X10 design decisions}

Given these considerations, the \Xten makes the following choices:
\begin{itemize}
\item \Xten supports subtyping and equality constraints on types
\item \Xten does not support structural bounds, but may do so in
the future.  \Xten has closures with structural subtyping, which
can be used in many of the cases structural type bounds would be
used.
\item Classes have type parameters with definition-site variance
rather than type properties with use-site variance annotations.
Properties are just too unfamiliar.
Usability outweighs expressive power. 
\item Run-time type information is preserved, but constraints
are not.  
\end{itemize}


\subsection{Related work}
\label{sec:related}
Constraint-based type systems, dependent types, and generic types
have been well-studied in the literature.

\paragraph{Constraint-based type systems.}

The use of constraints for type inference and subtyping has a history
going back to Mitchell~\cite{mitchell84} and by
Reynolds~\cite{reynolds85}.  These and subsequent systems are based on
constraints over types, but not over values.  Trifonov and
Smith~\cite{trifonov96} proposed a type system in which types are
refined using subtyping constraints.
Pottier~\cite{pottier96simplifying} presents a constraint-based type
system for an ML-like language with subtyping.  These developments
lead to \hmx~\cite{sulzmann97type}, a constraint-based framework for
Hindley--Milner-style type systems.  The framework is parametrized on
the specific constraint system $X$; instantiating $X$ yields
extensions of the HM type system.  Constraints in \hmx{} are over
types, not values. The \hmx{} approach is an important precursor to
our constrained types approach. The principal difference is that
\hmx{} applies to functional languages and does not integrate
dependent types.

%
Sulzmann and Stuckey~\cite{sulzmann-hmx-clpx} showed that the
type inference algorithm for \hmx can be encoded as a
constraint logic program parametrized by the constraint system
$X$. This is very much in spirit with our approach.
Constrained types open the door to {\em user-defined}
predicates and functions, effectively permitting the user to enrich
$\cal C$ (hence the power of the compile-time type-checker) by
developing application-specific constraints using a constraint
programming language such as CLP($\cal C$) \cite{clp} or the richer
RCC($\cal C$) \cite{DBLP:conf/fsttcs/JagadeesanNS05}.

\paragraph{Dependent types.}

Dependent type
systems~\cite{xi99dependent,calc-constructions,epigram,cayenne}
parametrize types on values.  Refinement type
systems~\cite{refinement-types,conditional-types,jones94,sized-types,flanagan-popl06,flanagan-fool06,liquid-types},
introduced by Freeman and Pfenning~\cite{refinement-types}, are dependent type
systems that extend a base type system through constraints on values.  These
systems do not treat value and type constraints uniformly.

Our work is closely related to DML, \cite{xi99dependent}, an
extension of ML with dependent types. DML is also built
parametrically on a constraint solver. Types are refinement types;
they do not affect the operational semantics and erasing the
constraints yields a legal DML program.  This differs from generic constrained
types, where erasure of subtyping constraints can prevent the program from
type-checking.
DML does not permit any run-time checking of constraints
(dynamic casts).

The most obvious distinction between DML and constrained types
lies in the target
domain: DML is designed for functional programming
whereas constrained types are designed for imperative, concurrent
object-oriented languages. 
But there are several other
crucial differences as well.

DML achieves its separation between compile-time and run-time processing
by not permitting program
variables to be used in types. Instead, a parallel set of (universally
or existentially quantified) ``index'' variables are
introduced.
%
Second, DML permits only variables of basic index sorts known to
the constraint solver (e.g., \Xcd{bool}, \Xcd{int}, \Xcd{nat}) to
occur in types. In contrast, constrained types permit program
variables at any type to occur in constrained types. As with DML
only operations specified by the constraint system are permitted in
types. However, these operations always include field selection and
equality on object references.  Note that DML-style constraints are easily
encoded in constrained types.

% {\em Conditional
% types}~\cite{conditional-types} extend refinement types to
% encode control-flow information in the types.
% %
% Jones introduced {\em qualified types}, which permit
% types to be constrained by a finite set of
% predicates~\cite{jones94}.
% %
% {\em Sized types}~\cite{sized-types}
% annotate types with the sizes of recursive data structures.
% Sizes are linear functions of size variables.
% Size inference is performed using a constraint solver for
% Presburger arithmetic~\cite{omega}.
% % constraints on types, support primitive recursion only

% Index objects must be pure.
% Singleton types int(n).
% ML$^{\Pi}_0$:
% Refinement of the ML type system: does not affect the
% operational semantics.  Can erase to ML$_0$.

% Jay and Sekanina 1996: array bounds checking based on shape
% types.

% Ada dependent types.
% Ada has constrained array definitions.  A constraint
% \cite{ada-ref-man}.  Not clear if they're dependent.  Are
% there other dependent types?  Generics are dependent?

        % Used for array bounds by Morrisett et al (I think--need
        % to find paper)

% Singleton types~\cite{aspinall-singletons}.

Logically qualified types, or liquid types~\cite{liquid-types},
permit types in a base Hindley--Milner-style type system to be refined with
conjunctions of logical qualifiers.  The subtyping relation is similar to
\Xten{}'s, that is, two liquid types are in the subtyping relation if their base
types are and if one type's qualifier implies the other's.
The Hindley--Milner type
inference algorithm is used to infer base types; these types are used as templates for inference of the liquid types.
The types of certain expressions are over-approximated to ensure inference
is decidable.
To improve precision of the inference algorithm, and hence
to reduce the annotation burden on the programmer, 
the type system is path sensitive.

Hybrid type-checking~\cite{flanagan-popl06,flanagan-fool06}
introduced another refinement type system.
While typing is undecidable, dynamic checks are inserted into
the program when necessary if the type-checker (which
includes a constraint solver) cannot determine
type safety statically.
In \FXG{}, dynamic type checks, including tests of dependent
constraints, are inserted only at explicit casts or
\Xcd{instanceof} expressions; constraint solving is performed at compile time.

% Where clauses for F-bounded polymorphism~\cite{where-clauses}
% Bounded quantification: Cardelli and Wegner.  Bound T with T'
% In F-bounded polymorphism~\cite{f-bounds}, type variables are bounded by a function of 
% the type variable. 
% Not dependent types.

Concoqtion~\cite{concoqtion} extends types in OCaml~\cite{ocaml}
with constraints written as Coq~\cite{coq} rules.
While the types are expressive, supporting the full generality
of the Coq language, proofs must be
provided to satisfy the type checker.
\Xten{} supports only constraints that can be checked by a
constraint solver during compilation.
Concoqtion encodes OCaml types and value to allow reasoning in
the Coq formulae; however, there is an impedance mismatch
caused by the differing syntax, representation, and behavior
of OCaml versus Coq.

\eat{
Cayenne~\cite{cayenne} is a Haskell-like language with fully dependent types.
There is no distinction between static and dynamic types.
Type-checking is undecidable.
There is no notion of datatype refinement as in DML.

Epigram~\cite{epigram,epigram-matter}
is a dependently typed functional programming language based on
a type theory with inductive families.
Epigram does not have a phase distinction between values and
types.
}

\eat{
$\lambda^{\sf Con}$ is a lambda calculus with assertions.
Findler, Felleisen, Contracts for higher-order functions (ICFP02)

  example: int[> 9]

contracts are either simple predicates or function contracts.
defined by (define/contract ...)

enforced at run-time.
}

% Jif~\cite{jif,jflow} is an extension of Java in which
% types are labeled with security policies enforced by the
% compiler.

\eat{
ESC/Java~\cite{esc-java}
allow programmers to write object invariants and pre- and
post-conditions that are enforced statically
by the compiler using an automated theorem prover.
Static checking is undecidable and, in the presence of loops,
is unsound (but still useful) unless the programmer supplies loop invariants.
ESC/Java can enforce invariants on mutable state.
}

% and Spec$\sharp$~\cite{specsharp}

\eat{
Pluggable and optional type systems were proposed by
Bracha~\cite{bracha04-pluggable} and provide another means of
specifying refinement types.
Type annotations, implemented in compiler plugins, serve only to
reject programs statically that might otherwise have dynamic
type errors.
CQual~\cite{foster-popl02} extends C with user-defined type
qualifiers.  These
qualifiers may be flow-sensitive and may be inferred. 
CQual supports only a fixed set of typing rules
for all qualifiers.
In contrast, the {\em semantic type qualifiers} of
Chin, Markstrum, and Millstein~\cite{chin05-qualifiers}
allow programmers to define typing rules for qualifiers
in a meta language that allows type-checking rules to be
specified declaratively.
JavaCOP~\cite{javacop-oopsla06} is a pluggable type system
framework for Java.  Annotations are defined in a meta language
that allows type-checking rules to be specified declaratively.
JSR 308~\cite{jsr308} is a proposal for adding user-defined type qualifiers
to Java.
}

% Holt, Cordy, the Turing programming language

% Ou, Tan, Mandelbaum, Walker, Dynamic typing with dependent types
% Separate dependent and simple parts of the program.
% Statically type the dependent parts.
% Dynamic checks when passing values into dependent part.

\paragraph{Genericity.}

Genericity in object-oriented languages is usually
supported through
type parametrization.

A number of proposals 
for adding genericity to Java quickly followed
the initial release of
the language~\cite{GJ,Pizza,java-popl97,thorup97,allen03}.
GJ~\cite{GJ} implements invariant type
parameters via type erasure.
PolyJ~\cite{java-popl97} supports run-time representation of types
via adapter objects, and also permits instantiation of
parameters on primitive types and structural parameter bounds.
Viroli and Natali~\cite{reflective-generics,type-passing-generics}
also support
a run-time representation of types, using Java's reflection API.
NextGen~\cite{nextgen,allen03} was implemented using run-time 
instantiation.
\Xten{}'s generics have a hybrid implementation, adopting PolyJ's
adapter object approach for dependent types and for 
type introspection and using NextGen's run-time
instantiation approach for greater efficiency.
% MixGen~\cite{allen04} extends NextGen with mixins.

\csharp also supports generic types via run-time instantiation in the
CLR~\cite{csharp-generics}.  Type parameters may be declared
with definition-site variance tags.
Generalized type constraints were proposed for
\csharp~\cite{emir06}.  Methods can be annotated with subtyping
constraints that must be satisfied to invoke the method.
Generic \Xten{} supports these constraints, as well as constraints
on values, with method and constructor where clauses.

\eat{
\FXG{} does not support \emph{bivariance}~\cite{variant-parametric-types}; a
class \xcd"C" is bivariant in a type property \xcd"X" if \xcd"C{self.X==S}" is
a subtype of \xcd"C{self.X==T}" for any \xcd"S" and \xcd"T".  Bivariance is
useful for writing code in which the property \xcd"X" is ignored.  One can
achieve  this effect in \FXG{} simply by leaving \xcd"X" unconstrained.
}

\eat{
Parametric types with use-site variance are related to existential types:
\xcd"C<+T>" corresponds to the bounded existential $\exists\tcd{X<:T}.C<X>$;
\xcd"C<-T>" corresponds to the bounded existential $\exists\tcd{X:>T}.C<X>$;
\xcd"C<*>" corresponds to the unbounded existential $\exists\tcd{X}.C<X>$.
\FXG{} has a similar correspondence:
\xcd"C{X<:T}" corresponds to the bounded existential \xcdmath"C\{\exists\tcd{self}:C.self.X<:T\}";
\xcd"C{X:>T}" corresponds to the bounded existential \xcdmath"C\{\exists\tcd{self}.C<X\}";
\xcd"C" corresponds to the unbounded existential \xcdmath"C\{\exists\tcd{self}.C<X\}".
}


