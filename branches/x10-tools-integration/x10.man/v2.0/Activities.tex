\chapter{Activities}\label{XtenActivities}

An {\em activity} is a statement being executed, independently, with its own
local variables; it may be thought of as a very light-weight thread. An
\Xten{} computation may have many concurrent {\em activities} executing at any
give time.  All X10 code runs as part of an activity; when an X10 program is
started, the \xcd`main` method is invoked in an activity, called the {\em root
activity}.\index{root
activity}


Activities coordinate their execution by various control and data structures.
For example, `
%~~stmt~~`~~`~~x:Int, var y:Int ~~
\xcd`await (x==0);` blocks the current activity until some other activity
sets \xcd`x` to zero.  However, activities determine the places at which they
may be blocked and resumed, by \xcd`await` and similar constructs.  There are
no means by which one activity can arbitrarily interrupt, block, or resume
another, no method  \xcd`activity.interrupt()`.

An activity may be {\em running}, {\em blocked} on some condition or {\em
terminated}. If terminated, it is terminated in the same way that its
statement is: in particular, if the statement terminates abruptly, the
activity terminates abruptly for the same reason.
(\Sref{ExceptionModel}).

Activities can be long-running entities with a good deal of local state.  In
particular they can involve recursive method calls (and therefore have runtime
stacks).  However, activities can also be short-running light-weight entities,
\eg, it is reasonable to have an activity that simply increments a variable.

An activity may asynchronously and in parallel launch activities at
other places.  Every activity save the initial \xcd`main` activity is spawned
by another.  Thus, at any instant, the activities in a program form a tree.

X10 uses this tree in crucial ways.  
First is the distinction 
between {\em local} termination and {\em global}
termination of a statement. The execution of a statement by an
activity is said to terminate locally when the activity has finished
all its computation. (For instance the
creation of an asynchronous activity terminates locally when the
activity has been created.)  It is said to terminate globally when it
has terminated locally and all activities that it may have spawned at
any place have, recursively, terminated globally.
For example, consider: 
%~~gen
% package Activites.Are.For.Whacktivities;
% class Example {
% def example( s1:() => Void, s2 : () => Void ) {
%~~vis
\begin{xten}
async {s1();}
async {s2();}
\end{xten}
%~~siv
% } } 
%~~neg
The primary activity spawns two child activities and then terminates locally,
very quickly.  The child activities may take arbitrary amounts of time to
terminate (and may spawn grandchildren).  When \xcd`s1()`, \xcd`s2()`, and
all their descendants terminate locally, then the primary activity terminates
globally. 

The program as a whole terminates when the root activity terminates globally.
In particular, X10 does not permit the creation of 
daemon threads---threads that outlive the lifetime of the root
activity.  We say that an \Xten{} computation is {\em rooted}
(\Sref{initial-computation}).

\futureext{ We may permit the initial activity to be a daemon activity
to permit reactive computations, such as webservers, that may not
terminate.}

\section{The \Xten{} rooted exception model}
\label{ExceptionModel}
\index{Exception!model}

The rooted nature of \Xten{} computations permits the definition of a
{\em rooted exception model.} In multi-threaded programming languages
there is a natural parent-child relationship between a thread and a
thread that it spawns. Typically the parent thread continues execution
in parallel with the child thread. Therefore the parent thread cannot
serve to catch any exceptions thrown by the child thread. 

The presence of a root activity and the concept of global termination permits
\Xten{} to adopt a more powerful exception model. In any state of the
computation, say that an activity $A$ is {\em a root of} an activity $B$ if
$A$ is an ancestor of $B$ and $A$ is blocked at a statement (such as the
\xcd"finish" statement \Sref{finish}) awaiting the termination of $B$ (and
possibly other activities). For every \Xten{} computation, the \emph{root-of}
relation is guaranteed to be a tree. The root of the tree is the root activity
of the entire computation. If $A$ is the nearest root of $B$, the path from
$A$ to $B$ is called the {\em activation path} for the activity.\footnote{Note
  that depending on the state of the computation the activation path may
  traverse activities that are running, blocked or terminated.}

We may now state the exception model for \Xten.  An uncaught exception
propagates up the activation path to its nearest root activity, where
it may be handled locally or propagated up the \emph{root-of} tree when
the activity terminates (based on the semantics of the statement being
executed by the activity).\footnote{In \XtenCurrVer{} the \xcd"finish"
statement is the only statement that marks its activity as a root
activity. Future versions of the language may introduce more such
statements.}  In Java, exceptions may be overlooked because there is no good
place to put a \xcd`try`-\xcd`catch` block; this is never the case in X10.

\section{\xcd`at`: Place changing}\label{AtStatement}

An activity may change place using the \xcd"at" statement or
\xcd"at" expression:

\begin{grammar}
Statement \: AtStatement \\
AtStatement \: \xcd"at" PlaceExpressionSingleList Statement \\
Expression \: AtExpression \\
AtExpression \: \xcd"at" PlaceExpressionSingleList ClosureBody 
\end{grammar}

The statement \xcd"at (p) S" executes the statement \xcd"S"
synchronously at a place described by \xcd"p".
The expression \xcd"at (p) E" executes the statement \xcd"E"
synchronously at place \xcd"p", returning the result to the
originating place.  



\Eg, if \xcd`obj` is an object with a non-\xcd`global`
method \xcd`meth()`, the general way to call the method from anywhere is 
%~~gen
% package Activities.Are.Wonderful.And.Delicious;
% class Methbearer {
%   def meth() {}
%   def useIt(x:Methbearer) {
%~~vis
\begin{xten}
at(x.home) x.meth();
\end{xten}
%~~siv
%}}
%~~neg
\noindent 
Or, if you want to capture the result of the method call, use the
\xcd`at`-expression:  
%~~gen
% package Activities.Are.Cute.And.Bubbly;
% class Methbearer {
%   def meth():String = "Hi there!";
%   def useIt(x:Methbearer) {
%~~vis
\begin{xten}
val res = at(x.home)x.meth();
\end{xten}
%~~siv
%}}
%~~neg

\xcd`p` may be an expression of type \xcd`Place`, in which case its value is
used as the place to execute the body: 
%~~gen
% package Activities.At.A.Standstill;
% class Example {
% def example(ob: Object, S: ()=>Void) {
%~~vis
\begin{xten}
   at (here.next()) S();
   at (ob.home) S();
\end{xten}
%~~siv
% } } 
%~~neg
\noindent
Or, \xcd`p` may be an object, in which case the body is executed at
\xcd`p.home`: 
%~~gen
% package Activities.At.A.Run;
% class Example {
% def example(ob: Object, S: ()=>Void) {
%~~vis
\begin{xten}
   at (ob) S();
   // Same as at(ob.home) S();
\end{xten}
%~~siv
% } } 
%~~neg
Structs and functions are global values, without home places, so they cannot
be used as the place expression for \xcd`at`.  



\xcd`at(p)S` does {\em not} start a new activity.  It should be thought of as
transporting the current activity to \xcd`p`, running \xcd`S` there, and then
transporting it back.    If you want to start a new activity, use \xcd`async`;
if you want to start a new activity at \xcd`p`, use 
\xcd`at(p) async S`.  




As a consequence of this, \xcd`S` may contain constructs which only make sense
within a single activity.  For example, 
%~~gen
% package Activities.For.Fun.And.Profit;
% import x10.util.*;
% class Thing {
% def nice():Boolean = true;
% def poke(){}
% static def example(things:List[Thing]!) {
%~~vis
\begin{xten}
  for(x in things) 
    if (at(x.home) x.nice()) 
        return x;
\end{xten}
%~~siv
% return null;
%}}
%~~neg
returns the first nice thing in a collection.   If we had used 
\xcd`async at(x.home)`, this would not be allowed; 
you can't \xcd`return` from an
\xcd`async`. 



\section{\xcd`async`: Spawning an activity}\label{AsynchronousActivity}\label{AsyncActivity}

Asynchronous activities serve as a single abstraction for supporting a
wide range of concurrency constructs such as message passing, threads,
DMA, streaming, data prefetching. (In general, asynchronous operations
are better suited for supporting scalability than synchronous
operations.)

An activity is created by executing the \xcd`async` statement: 

\begin{grammar}
Statement \: AsyncStatement \\
AsyncStatement \: \xcd"async" PlaceExpressionSingleList\opt Statement \\
PlaceExpressionSingleList \: \xcd"(" PlaceExpression \xcd")" \\
PlaceExpression \: Expression 
\end{grammar} 

The basic form of \xcd`async` is \xcd`async S`, which starts a new activity
located \xcd`here` executing \xcd`S`.    The form 
\xcd`async (p) S` is simply shorthand for \xcd`async at(p) S`, covering the
common case where the activity is to be executed at some other place. 

Note that the array subscript expression \xcd"a(i)", when used as a place
expression evaluates to \xcd"a(i).home", \viz. the home of the contents of
\xcd`a(i)` In general, this is not the same place that the array cell itself
is, \viz. \xcd"a.dist(i)". Accesses to \xcd"a(i)" should typically be guarded
by the place expression \xcd"a.dist(i)": 
%~~gen
% package Activities.For.Tactivities.At.Some.Home;
% class Example {
% def example(a:DistArray[Int](1), i:Int) {
%~~vis
\begin{xten}
async (a.dist(i)) {
  a(i) += 1;
}
\end{xten}
%~~siv
%} } 
%~~neg

\bard{The followingin para is under investigation:}
In many cases the compiler may infer the unique place at which the
statement is to be executed by an analysis of the types of the
variables occurring in the statement. (The place must be such that the
statement can be executed safely, without generating a
\xcd"BadPlaceException".) In such cases the programmer may omit the
place designator; the compiler will throw an error if it cannot
determine the unique designated place.\footnote{\XtenCurrVer{} does
not specify a particular algorithm; this will be fixed in future
versions.}

An activity $A$ executes the statement \xcd"async (P) S" by launching
a new activity $B$ at place \xcd`P` (or \xcd`P.home` if \xcd`P` is of an
object type), to execute \xcd`S`. The statement terminates locally as soon as $B$ is
launched.  The activation path for $B$ is that of $A$ augmented by the
information that {$A$} is the parent of {$B$}. 
$B$
terminates normally when $S$ terminates normally.  It terminates
abruptly if $S$ throws an uncaught exception. The exception is
propagated to $A$ if $A$ is a root activity (see \Sref{finish}),
otherwise it is propagated through $A$ to $A$'s root activity. Note that while
{$A$} is running, exceptions thrown by activities it has already
spawned may propagate through it up to its root activity, without {$A$} noticing.

Multiple activities launched by a single activity at another place are not
ordered in any way. They are added to the set of activities at the target
place and will be executed based on the local scheduler's decisions.
If some particular sequencing of events is needed, \xcd`when`, \xcd`atomic`,
\xcd`await`, \xcd`finish`, clocks, and other X10 constructs can be used.
\Xten{} implementations are not required to have fair schedulers,
though every implementation should make a best faith effort to ensure
that every activity eventually gets a chance to make forward progress.

\begin{staticrule*}
The statement in the body of an \xcd"async" is subject to the
restriction that it must be acceptable as the body of a \xcd"void"
method for an anonymous inner class declared at that point in the code,
which throws no checked exceptions. As such, it may reference
variables in lexically enclosing scopes (including \xcd"clock"
variables, \Sref{XtenClocks}) provided that such variables are
(implicitly or explicitly) \xcd"val".
\end{staticrule*}

\section{Finish}\index{finish}\label{finish}
The statement \xcd"finish S" converts global termination to local
termination and introduces a root activity.   It executes \xcd`S`, and then
waits for all activities spawned by \xcd`S`, directly or indirectly, to
finish. It also collects exceptions thrown by those activities.

\begin{grammar}
Statement \: FinishStatement \\
FinishStatement \: \xcd"finish" Statement 
\end{grammar}

An activity $A$ executes \xcd"finish S" by executing \xcd"S".  The
execution of \xcd"S" may spawn other asynchronous activities (here or
at other places).  Uncaught exceptions thrown or propagated by any
activity spawned by \xcd"S" are accumulated at \xcd"finish S".
\xcd"finish S" terminates locally when all activities spawned by
\xcd"S" terminate globally (either abruptly or normally). If \xcd"S"
terminates normally, then \xcd"finish S" terminates normally and $A$
continues execution with the next statement after \xcd"finish S".  If
\xcd"S" terminates abruptly, then \xcd"finish S" terminates abruptly
and throws a single exception, \Xcd{x10.lang.MultipleExceptions}
formed from the collection of exceptions accumulated at \xcd"finish S".

Thus a \xcd"finish S" statement serves as a collection point for
uncaught exceptions generated during the execution of \xcd"S".

Note that repeatedly \xcd"finish"ing a statement has little effect after
the first \xcd"finish": \xcd"finish finish S" is indistinguishable
from \xcd"finish S" if \xcd`S` throws no exceptions.  (If \xcd`S` throws
exceptions, \xcd`finish S` wraps them in one layer of 
\xcd`MultipleExceptions` and \xcd`finish finish S` in two layers.)

%%OLIVIER-DENIES%% \paragraph{Interaction with clocks.}\label{sec:finish:clock-rule}
%%OLIVIER-DENIES%% 
%%OLIVIER-DENIES%% \xcd"finish S" interacts with clocks (\Sref{XtenClocks}). 
%%OLIVIER-DENIES%% While executing \xcd"S", an activity must not spawn any \xcd"clocked"
%%OLIVIER-DENIES%% asyncs. (Asyncs spawned during the execution of \xcd"S" may spawn
%%OLIVIER-DENIES%% clocked asyncs.) A
%%OLIVIER-DENIES%% \xcd"ClockUseException"\index{clock!ClockUseException} is thrown
%%OLIVIER-DENIES%% if (and when) this condition is violated.
%%OLIVIER-DENIES%% 
%%OLIVIER-DENIES%% This is necessary to prevent deadlocks.  In the following invalid code 
%%OLIVIER-DENIES%% %~s~gen
%%OLIVIER-DENIES%% % package Activities.Finish.Hates.Clocks;
%%OLIVIER-DENIES%% % class Example{
%%OLIVIER-DENIES%% % def example() {
%%OLIVIER-DENIES%% %~s~vis
%%OLIVIER-DENIES%% \begin{xten}
%%OLIVIER-DENIES%% val c:Clock = Clock.make();
%%OLIVIER-DENIES%% async clocked(c) {                // (A) 
%%OLIVIER-DENIES%%       finish async clocked(c) {   // (B) INVALID
%%OLIVIER-DENIES%%             next;                 // (Bnext)
%%OLIVIER-DENIES%%       }
%%OLIVIER-DENIES%%       next;                       // (Anext)
%%OLIVIER-DENIES%% }
%%OLIVIER-DENIES%% \end{xten}
%%OLIVIER-DENIES%% %~s~siv
%%OLIVIER-DENIES%% % } } 
%%OLIVIER-DENIES%% %~s~neg
%%OLIVIER-DENIES%% \xcd`(A)`, first of all, waits for the \xcd`finish` containing \xcd`(B)` to
%%OLIVIER-DENIES%% finish.  
%%OLIVIER-DENIES%% \xcd`(B)` will execute its \xcd`next` at \xcd`(Bnext)`, and then wait for all
%%OLIVIER-DENIES%% other activities registered on \xcd`c` to execute their \xcd`next`s.
%%OLIVIER-DENIES%% However, \xcd`(A)` is registered on \xcd`c`.  So, \xcd`(B)` cannot finish
%%OLIVIER-DENIES%% until \xcd`(A)` has proceeded to \xcd`(Anext)`, and \xcd`(A)` cannot proceed
%%OLIVIER-DENIES%% until \xcd`(B)` finishes. Thus, this causes deadlock.
%%OLIVIER-DENIES%% 
%%OLIVIER-DENIES%% 
%%OLIVIER-DENIES%% 
%%OLIVIER-DENIES%% In \XtenCurrVer{} this condition is checked dynamically; future
%%OLIVIER-DENIES%% versions of the language will introduce type qualifiers which permit
%%OLIVIER-DENIES%% this condition to be checked statically.
%%OLIVIER-DENIES%% 
%%OLIVIER-DENIES%% \futureext{
%%OLIVIER-DENIES%% The semantics of \xcd"finish S" is conjunctive; it terminates when all
%%OLIVIER-DENIES%% the activities created during the execution of \xcd"S" (recursively)
%%OLIVIER-DENIES%% terminate. In many situations (e.g., nondeterministic search) it is
%%OLIVIER-DENIES%% natural to require a statement to terminate when any {\em one} of the
%%OLIVIER-DENIES%% activities it has spawned succeeds. The other activities may then be
%%OLIVIER-DENIES%% safely aborted. Future versions of the language may introduce a
%%OLIVIER-DENIES%% \xcd"finishone S" construct to support such speculative or nondeterministic
%%OLIVIER-DENIES%% computation.
%%OLIVIER-DENIES%% }
%%OLIVIER-DENIES%% 

\section{Initial activity}\label{initial-computation}\index{initial activity}

An \Xten{} computation is initiated from the command line on the
presentation of a classname \xcd"C". The class must have a
\xcd"public static def main(a: Rail[String]):Void" method, otherwise an
exception is thrown
and the computation terminates.  The single statement
\begin{xten}
finish async (Place.FIRST_PLACE) {
  C.main(s);
}
\end{xten} 
\noindent is executed where \xcd"s" is an Rail of strings created
from the command line arguments. This single activity is the root activity
for the entire computation. (See \Sref{XtenPlaces} for a discussion of
places.)

%% Say something about configuration information? 

\section{Foreach statements}\index{\Xcd{foreach}}\label{foreach-section}


\begin{grammar}
Statement \: ForEachStatement \\
ForEachStatement \: 
      \xcd"foreach" \xcd"(" Formal \xcd"in" Expression \xcd")"
          Statement 
\end{grammar}


The \xcd"foreach" statement is a parallel version of the enhanced \xcd"for"
statement (\Sref{ForAllLoop}). \xcd`for(x in C)S` executes \xcd`S` {\em
  sequentially}, with everything happening \xcd`here`. \xcd`foreach(x in C)S`
executes \xcd`S` for each iteration of the loop {\em in parallel}, located at
\xcd`x.home`. It is thus equivalent to:
\begin{xten}
foreach (x in C)
  async at (x.home) S
\end{xten}

As a common and useful special case, \xcd`C` may be a \xcd`Dist` or an
\xcd`Array`.  For both of these, \xcd`foreach(x in C)S` is treated just like 
\xcd`foreach(x in C.region)S`.  \xcd`x` ranges over the \xcd`Point`s of the
region.  Each activity that \xcd`foreach` starts is located at \xcd`here` --
the same place that the \xcd`foreach` statement itself is executing.  (If you
want to start an activity at the place where the array element \xcd`C(p)` is
located, use \xcd`ateach` (\Sref{ateach-section}) instead of \xcd`foreach`.)

Exceptions thrown by \xcd`S`, like other exceptions in \xcd`async`s, are
propagated to the root activity of the \xcd`foreach`.  



\section{Ateach statements}\index{\Xcd{ateach}}\label{ateach-section}

\begin{grammar}
Statement \: AtEachStatement \\
AtEachStatement \:
      \xcd"ateach" \xcd"(" Formal \xcd"in" Expression \xcd")"
         Statement 
\end{grammar}

The \xcd"ateach" statement is similar to the \xcd"foreach"
statement, but it spawns activites at each place of a distribution. 
In \xcd`ateach(p in D) S`, 
\xcd`D` must be either of type \xcd"Dist" or of type
\xcd`DistArray[T]`, 
and \xcd`p` will be of type \xcd"Point".

This statement differs from \xcd"foreach" only in
that each activity is spawned at the place specified by the
distribution for the point. That is, if \xcd`D` is a \xcd`Dist`, 
\xcd"ateach(p in D) S" could be implemented as:
\begin{xten}
foreach (p in D.region) 
  async (D(p)) S(p)
\end{xten}

However, the compiler may implement it more efficiently to avoid extraneous
communications.  In particular, it is recommended that \xcd`ateach(p in D)S`
be implemented as the following code, which coordinates with each place of
\xcd`D` just once, rather than once per element of \xcd`D` at that place: 

%~~gen
% package Activities.Activities.Activities;
% class EquivCode {
% static def S(p:Place) {}
% static def example(D:Dist) {
%~~vis
\begin{xten}
foreach (p in D.places()) at (p) {
    foreach (pt in D|here) {
        S(p);
    }
}
\end{xten}
%~~siv
%}} 
%~~neg

If \xcd`e` is an \xcd`DistArray[T]`, then \xcd`ateach (p in e)S` is identical to
\xcd`ateach(p in e.dist)S`; the iteration is over the array's underlying
distribution.   
The code below is a common and generally efficient way to work with the
elements of a distributed array:
%~~gen
%package Activities.For.Fnu.And.Pforit;
%class Example[T]{
% global def dealWith(T):Void = {}
% def idiom(A:DistArray[T]){
%~~vis
\begin{xten}
ateach(p in A) 
  dealWith(A(p));
\end{xten}
%~~siv
%}}
%~~neg




\section{Futures}\label{XtenFutures}

\Xten{} provides syntactic support for {\em asynchronous expressions}, also
known as futures.  Futures let you start a subcomputation that will return a
value, and let it proceed asynchronously.  When you need its value, apply
\xcd`force()` to the future, which will wait until that computation is done
(if necessary).  Then, calling the future as a nullary function retrieves the
value of the computation.
%~~gen
% package Activities.Futures.Exmaple.NotATreeAnymore;
% class Futu {
%   public static def main(argv:Rail[String]!):Void {
%~~vis
\begin{xten}
    val future_a = future "expression that takes a long time";
    val b = "Some long computation happens here";
    future_a.force();
    val a = future_a(); // get its value
\end{xten}
%~~siv
%  }}
%~~neg




\begin{grammar}
Primary \: FutureExpression \\
FutureExpression \:
  \xcd"future" PlaceExpressionSingleList\opt ClosureBody
\end{grammar} 



In more detail, in an expression \xcd"future (Q) e", the place
expression \xcd"Q" is treated as in an \xcd"async" statement. \xcd"e"
is an expression of some type \xcd"T". \xcd"e" may reference only
those variables in the enclosing lexical environment which are
declared to be \xcd"val".

If the type of \xcd"e" is \xcd"T" then the type of
\xcd"future (Q) e" is \xcd"Future[T]".  This 
type \xcd"Future[T]" is defined as if by:
\begin{xten}
package x10.lang;
public interface Future[T] implements () => T {
  public def apply(): T;
  global def forced(): Boolean;
  global def force(): T;
}
\end{xten}

Evaluation of \xcd"future (Q) e" terminates locally with the creation
of a value \xcd"f" of type \xcd"Future[T]".  This value may be
stored in objects, passed as arguments to methods, returned from
method invocation etc. 

At any point, the method \xcd"forced" may be invoked on \xcd"f". This
method returns without blocking, with the value \xcd"true" if the
asynchronous evaluation of \xcd"e" has terminated globally and with
the value \xcd"false" if it has not.

\xcd"Future[T]" is a subtype of the function type \xcd"() => T".
Invoking---\emph{forcing}---the future \xcd"f" blocks until the
asynchronous evaluation of \xcd"e" has terminated globally. If the
evaluation terminates successfully with value \xcd"v", then the method
invocation returns \xcd"v". If the evaluation terminates abruptly with
exception \xcd"z", then the method throws exception \xcd"z". Multiple
invocations of the function (by this or any other activity) do not
result in multiple evaluations of \xcd"e". The results of the first
evaluation are stored in the future \xcd"f" and used to respond to all
queries.




\section{At expressions}

\begin{grammar}
Expression \: \xcd"at" \xcd"(" Expression \xcd")" Expression
\end{grammar}

An \Xcd{at} expression evaluates an expression synchronously at the
given place and returns its value.  If \xcd`fld` is a non-global field of 
\xcd`ob`, then \xcd`ob.fld` can be read (from anywhere) by: 
%~~gen
% package Activities.AtExpressions;
% class Example {
%   val fld = 1;
%   static def example(val ob: Example) {
%~~vis
\begin{xten}
val f = at(ob) ob.fld;
\end{xten}
%~~siv
% } } 
%~~neg





The expression evaluation may spawn asynchronous activities. The \Xcd{at}
expression will return without waiting for those activities to terminate. That
is, \Xcd{at} does not have built-in \Xcd{finish} semantics.

\section{Shared variables}
\label{Shared}

\limitation{Shared variables are not currently implemented.}

A {\em shared local variable} is declared with the annotation \xcd"shared".
It can be accessed within any control construct in its scope, including
\Xcd{async}, \Xcd{at}, \Xcd{future} and closures.

Note that the lifetime of some of these constructs may outlast the
lifetime of the scope -- requiring the implementation to allocate them
outside the current stack frame.

\section{Atomic blocks}\label{AtomicBlocks}\index{atomic blocks}
Languages such as \java{} use low-level synchronization locks to allow
multiple interacting threads to coordinate the mutation of shared
data. \Xten{} eschews locks in favor of a very simple high-level
construct, the {\em atomic block}.

A programmer may use atomic blocks to guarantee that invariants of
shared data-structures are maintained even as they are being accessed
simultaneously by multiple activities running in the same place.  

For example, consider a class \xcd`Redund[T]`, which encapsulates a list
\xcd`list` and, (redundantly) keeps the size of the list in a second field
\xcd`size`.  Then \xcd`r:Redund[T]` has the invariant 
\xcd`r.list.size() == r.size`, which must be true at any point that there are
no method calls on \xcd`r` active.

If the \xcd`add` method on \xcd`Redund` (which adds an element to the list) 
were defined as: 
%~~gen
% package Activities.Atomic.Redund.One;
% import x10.util.*;
% class Redund[T] {
%   val list = new ArrayList[T]();
%   var size : Int = 0;
%~~vis
\begin{xten}
def add(x:T) { // Incorrect
  this.list.add(x);
  this.size = this.size + 1;
}
\end{xten}
%~~siv
%}
%~~neg
Then two activities simultaneously adding elements to the same \xcd`r` could break the
invariant.  Suppose that \xcd`r` starts out empty.  Let the first activity
perform the \xcd`list.add`, and compute \xcd`this.size+1`, which is 1, but not store it
back into \xcd`this.size` yet.  
(At this point, \xcd`r.list.size()==1` and \xcd`r.size==0`; the invariant
expression is false, but, as the first call to \xcd`r.add()` is active, the
invariant does not need to be true -- it only needs to be true when the
call finishes.)
Now, let the second activity do its call to
\xcd`add` to completion, which finishes with \xcd`r.size==1`.  
(As before, the invariant expression is false, but a call to \xcd`r.add()` is
still active, so the invariant need not be true.)
Finally, let
the first activity finish, which assigns the \xcd`1` computed before back into
\xcd`this.size`.  At the end, there are two elements in \xcd`r.list`, but
\xcd`r.size==1`. Since there are no calls to \xcd`r.add()` active, the
invariant must be true, but it is not.

In this case, the invariant can be maintained by making the increment atomic.
Doing so forbids that sequence of events; the \xcd`atomic` block cannot be
stopped partway.  
%~~gen
% package Activities.Atomic.Redund.Two;
% import x10.util.*;
% class Redund[T] {
%   val list = new ArrayList[T]();
%   var size : Int = 0;
%~~vis
\begin{xten}
def add(x:T) { 
  this.list.add(x);
  atomic { this.size = this.size + 1; }
}
\end{xten}
%~~siv
%}
%~~neg



\subsection{Unconditional atomic blocks}
The simplest form of an atomic block is the {\em unconditional
atomic block}:

\begin{grammar}
Statement \: AtomicStatement \\
AtomicStatement \: \xcd"atomic"  Statement \\
MethodModifier \: \xcd"atomic" \\
\end{grammar}

For the sake of efficient implementation \XtenCurrVer{} requires
that the atomic block be {\em analyzable}, that is, the set of
locations that are read and written by the \grammarrule{BlockStatement} are
bounded and determined statically.\footnote{A static bound is a constant
that depends only on the program text, and is independent 
of any runtime parameters. }
The exact algorithm to be used by
the compiler to perform this analysis will be specified in future
versions of the language.
\tbd{}

Such a statement is executed by an activity as if in a single step
during which all other concurrent activities in the same place are
blocked. If execution of the statement may throw an exception, it is
the programmer's responsibility to wrap the atomic block within a
\xcd"try"/{\xcd"finally" clause and include undo code in the finally
clause. Thus the \xcd"atomic" statement only guarantees atomicity on
successful execution, not on a faulty execution.


We allow methods of an object to be annotated with \xcd"atomic". Such
a method is taken to stand for a method whose body is wrapped within an
\xcd"atomic" statement.

Atomic blocks are closely related to non-blocking synchronization
constructs \cite{herlihy91waitfree}, and can be used to implement 
non-blocking concurrent algorithms.

\begin{staticrule*}
In \xcd"atomic S", \xcd"S" may include calls to \xcd`safe` methods, and use of
sequential control structures.

It may {\em not} include an \xcd"async" activity (such as creation
of a \Xcd{future}).

It may {\em not} include any statement that may potentially block at
runtime (\eg, \xcd"when", \xcd"force" operations, \xcd"next"
operations on clocks, \xcd"finish"). 

All locations accessed in an atomic block must statically satisfy the
{\em locality condition}: they must belong to the place of the current
activity.\index{locality condition}\label{LocalityCondition} 

\end{staticrule*}


The compiler checks for this condition by checking whether the statement
could be the body of a \xcd"void" method annotated with \xcd"safe" at
that point in the code (\Sref{SafeAnnotation}).

\paragraph{Consequences.}
Note an important property of an (unconditional) atomic block:

\begin{eqnarray}
 \mbox{\xcd"atomic \{s1; atomic s2\}"} &=& \mbox{\xcd"atomic \{s1; s2\}"}
\end{eqnarray}

Atomic blocks do not introduce deadlocks.    They may exhibit all the bad
behavior of sequential programs, including throwing exceptions and running
forever, but they are guaranteed not to deadlock.


\subsubsection{Example}

The following class method implements a (generic) compare and swap (CAS) operation:


%~~gen
% package Activities.And.Protein;
% class CASSizer{
%~~vis
\begin{xten}
var target:Object = null;
public atomic def CAS(old: Object, new: Object): Boolean {
   if (target.equals(old)) {
     target = new;
     return true;
   }
   return false;
}
\end{xten}
%~~siv
%}
%~~neg

\subsection{Conditional atomic blocks}

Conditional atomic blocks allow the activity to wait for some condition to be
satisfied before executing an atomic block. For example, consider a
\xcd`Redund` class holding a list \xcd`r.list` and, redundantly, its length
\xcd`r.size`.  A \xcd`pop` operation will delay until the \xcd`Redund` is
nonempty, and then remove an element and update the length.  
%~~gen
% package Activities.Condato.Example.Not.A.Tree;
% import x10.util.*;
% class Redund[T] {
% val list = new ArrayList[T]();
% var size : Int = 0;
%~~vis
\begin{xten}
def pop():T {
  var ret : T;
  when(size>0) {
    ret = list.removeAt(0);
    size --;
    }
  return ret;
}
\end{xten}
%~~siv
% }
%~~neg


The execution of the test is atomic with the execution of the block.  This is
important; it means that no other activity can sneak in and make the condition
be false before the block is executed.  In this example, two \xcd`pop`s
executing on a list with one element would work properly. Without the
conditional atomic block -- even doing the decrement atomically -- one call to
\xcd`pop` could pass the \xcd`size>0` guard; then the other call could run to
completion (removing the only element of the list); then, when the first call
proceeds, its \xcd`removeAt` will fail.  

Note that \xcd`if` would not work here.  
\xcd`if(size>0) atomic{size--; return list.removeAt(0);}` allows another
activity to act between the test and the atomic block.  
And 
\xcd`atomic{ if(size>0) {size--; ret = list.removeAt(0);}}` 
does not wait for \xcd`size>0` to become true.


Conditional atomic blocks are of the form:

\begin{grammar}
Statement \:  WhenStatement \\
WhenStatement \:  \xcd"when" \xcd"(" Expression \xcd")" Statement \\
            \| WhenStatement \xcd"or" \xcd"(" Expression \xcd")" Statement 
\end{grammar}

In such a statement the one or more expressions are called {\em
guards} and must be \xcd"Boolean" expressions. The statements are the
corresponding {\em guarded statements}.  

An activity executing such a statement suspends until such time as any
one of the guards is true in the current state. In that state, the
statement corresponding to the first guard that is true is executed.
The checking of the guards and the execution of the corresponding
guarded statement is done atomically. 

\Xten{} does not guarantee that a conditional atomic block
will execute if its condition holds only intermittently. For, based on
the vagaries of the scheduler, the precise instant at which a
condition holds may be missed. Therefore the programmer is advised to
ensure that conditions being tested by conditional atomic blocks are
eventually stable, \ie, they will continue to hold until the block
executes (the action in the body of the block may cause the condition
to not hold any more).

%%Fourth, \Xten{} does not guarantees only {\em weak fairness} when executing
%%conditional atomic blocks. Let $c$ be the guard of some conditional
%%atomic block $A$. $A$ is required to make forward progress only if
%%$c$ is {\em eventually stable}. That is, any execution $s_1, s_2,
%%\ldots$ of the program is considered illegal only if there is a $j$
%%such that $c$ holds in all states $s_k$ for $k > j$ and in which $A$
%%does not execute. Specifically, if the system executes in such a way
%%that $c$ holds only intermmitently (that is, for some state in which
%%$c$ holds there is always a later state in which $c$ does not hold),
%%$A$ is not required to be executed (though it may be executed).

\begin{rationale}
The guarantee provided by \xcd"wait"/\xcd"notify" in \java{} is no
stronger. Indeed conditional atomic blocks may be thought of as a
replacement for \java's wait/notify functionality.
\end{rationale} 

We note two common abbreviations. The statement \xcd"when (true) S" is
behaviorally identical to \xcd"atomic S": it never suspends. Second,
\xcd"when (c) {;}" may be abbreviated to \xcd"await(c);"---it
simply indicates that the thread must await the occurrence of a
certain condition before proceeding.  Finally note that a \xcd"when"
statement with multiple branches is behaviorally identical to a
\xcd"when" statement with a single branch that checks the disjunction of
the condition of each branch, and whose body contains an
\xcd"if"/\xcd"then"/\xcd"else" checking each of the branch conditions.

\begin{staticrule*}
For the sake of efficient implementation certain restrictions are
placed on the guards and statements in a conditional atomic
block. 
\end{staticrule*}

Guards are statically required not to have side-effects, not to spawn
asynchronous activities (as for the \xcd`sequential` qualifier on methods) and
to have a statically determinable upper bound on their execution (as for the
\xcd`nonblocking` qualifier on methods).

The body of a \xcd"when" statement must satisfy the conditions
for the body of an \xcd"atomic" block.

Note that this implies that guarded statements are required to be {\em
flat}, that is, they may not contain conditional atomic blocks. (The
implementation of nested conditional atomic blocks may require
sophisticated operational techniques such as rollbacks.)


\begin{example}
The following class shows how to implement a bounded buffer of size
$1$ in \Xten{} for repeated communication between a sender and a
receiver.  The call \xcd`buf.send(ob)` waits until the buffer has space, and
then puts \xcd`ob` into it.  Dually, \xcd`buf.receive()` waits until the
buffer has something in it, and then returns that thing.


%~~gen
% package Activities;
%~~vis
\begin{xten}
class OneBuffer[T] {
  var datum: T;
  var filled: Boolean = false;
  public def send(v: T) {
    when (!filled) {
      this.datum = v;
      this.filled = true;
    }
  }
  public def receive(): T {
    when (filled) {
      v: T = datum;
      filled = false;
      return v;
    }
  }
}
\end{xten}
%~~siv
%
%~~neg
\end{example}

