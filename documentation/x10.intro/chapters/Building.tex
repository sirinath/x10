\chapter{Building X10}
The X10 runtime has several build-time configuration options, which enable
specific communications protocols, and reduce error checking within the X10
libraries at runtime for better performance.  There is a similar set of options
available when compiling your own program with the x10c++ compiler.  When and
how to use these is detailed here.

\section{Network transport implementations}
For communications, the X10 runtime has the following transport implementations:
\begin{itemize}
\item Sockets: Uses TCP/IP sockets to support multiple places on one or more hosts.
This is the default implementation, and is the only option when using managed
X10.  The sockets transport uses SSH to launch the binaries across the network.
If you have a simple cluster of machines that support Ethernet and SSH, the
sockets transport is a good choice.
\item Standalone: Supports multiple places on a single host, using shared memory
between places. Standalone has high bandwidth, but limited message sizes and
only supports places running on a single machine.
\item MPI: An implementation of X10RT on top of MPI-2. This supports all the
hardware that your MPI implementation supports, such as Infiniband and
Ethernet, and should be used for systems where MPI is preferred.  It does not
(currently) use MPI's collective implementations, but instead uses our own
collective implementations.
\item PAMI: An IBM communications API that comes with the IBM Parallel Environment:
http://www-03.ibm.com/systems/software/parallel/index.html.  PAMI supports
high-end networks such HFI (Host Fabric Interface), BlueGene, Infiniband,
shared memory, and also Ethernet.  The PAMI implementation uses PAMI's
collectives, and is X10's best-performing transport.  If your system has the IBM
Parallel Environment installed, you'll want to use PAMI.
\end{itemize}

The X10 runtime will always be built with Sockets and Standalone
transports, as no special libraries need to be available on the system to compile them.  But
if you wish to make use of MPI or PAMI in your program, you must build the X10
runtime from source, specifying that you want to build in support for one or
both of these transports.  Similarly, when building your own program, you can
choose which transport to use and other options via arguments to x10c++.

\section{Building the X10 runtime}
The X10 runtime is built using ant (http://ant.apache.org/).  When you
are satisfied that your program is operating correctly, for best performance in
your programs, build the X10 runtime with the {\tt optimize} and
{\tt NO\_CHECKS} options turned on.  These will reduce the error checking within
the X10 library classes at runtime.  If you want to enable support for PAMI or
MPI transports, turn on the {\tt X10RT\_PAMI} and/or {\tt X10RT\_MPI} options. 
For example: ``{\tt ant -DX10RT\_PAMI=true -Doptimize=true -DNO\_CHECKS=true
dist}''

\section{Building your program}
You use the x10c++ compiler for building your own programs.  The X10 compiler
takes arguments to enable runtime optimizations, reduced error checking, and
transport selection of your program source, at build-time.  Use the {\tt -O}
{\tt -NO\_CHECKS}, and {\tt -STATIC\_CHECKS} flags for the fastest performance,
after you are satisfied with the correctness of your program.  To choose a
network transport other than sockets, specify {\tt pami}, {\tt mpi}, or {\tt
standalone} as the value of the {\tt -x10rt} flag.  For example: 
``{\tt x10c++ -O -NO\_CHECKS -STATIC\_CHECKS -x10rt pami YourProgram.x10}''.
