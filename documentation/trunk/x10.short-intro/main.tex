\documentclass[a4paper]{article}

%% For typesetting theorems and some math symbols.
\usepackage{amssymb}
\usepackage{amsthm}

\usepackage{fullpage}

\usepackage{relsize}
\usepackage{amsmath}
\usepackage{url}


% http://en.wikibooks.org/wiki/LaTeX/Packages/Listings
\usepackage{listings}

\usepackage[ruled]{algorithm} % [plain]
\usepackage[noend]{algorithmic} % [noend]
\renewcommand\algorithmiccomment[1]{// \textit{#1}} %


% For fancy end of line formatting.
\usepackage{microtype}


% For smaller font.
\usepackage{pslatex}

\usepackage{xspace}
\usepackage{graphicx}


\newcommand{\normalcode}[1]{\texttt{\textup{#1}}}
\def\codesmaller{\small}
\newcommand{\myCOMMENT}[1]{\COMMENT{\small #1}}
\newcommand{\code}[1]{\texttt{\textup{\codesmaller #1}}}

\title{A Brief Introduction to X10}

\author{Yoav Zibin and Nathaniel Clinger and Vijay Saraswat}

\date{}


\begin{document}


\maketitle


\lstset{language=java,basicstyle=\ttfamily\small}


\begin{abstract}
X10 is an object oriented (OO) programming language with a sophisticated
    type system (constraints, class invariants, non-erased generics, closures)
    and concurrency constructs (asynchronous activities, multiple places, global references).
This paper introduces people to X10, and in particular,
    its motivation, goals,
    contributions, design, domains of applicability, and availability.
X10 is a Java-style OO language with additional constructs that support concurrent and distributed programming.
This paper is intended for people familiar with Java that wish to learn how to program in X10
    and use these additional concurrency constructs in order to take advantage of scale out and of heterogeneity.
Furthermore, the paper describe the design goals of X10 and illustrate how these were achieved
    using simple program fragments.
\end{abstract}

\section{Introduction and Motivation}
Java currently lacks the ability to deal with peta-bytes of data
    that do not fit into main memory of a single machine (or \emph{place} in X10 jargon).
Moreover, Java cannot take advantage of GPUs and FPGAs that offer superior
    computational power over the normal CPUs.
An X10 program can scale out to millions of places, that can store peta-bytes of data.
This introduction paper describes the motivation and design behind X10,
    and explains how to write programs that scale out.

We will first present a simple \code{HelloWorld} program in X10,
    and then transform it into a distributed \code{HelloWholeWorld} program.

\begin{lstlisting}
public class HelloWorld {
  public static def main(args:Array[String]):void {
     Console.OUT.println("First argument: " + args(0));
  }
}
\end{lstlisting}
X10 is a Java-style object oriented (OO) language.
A program in X10 consists of classes that may
    declare fields and methods.
As in Java, the start of execution is the static \code{main} method
    that has a single parameter \code{args} which is an array the program arguments.
The program prints the first such argument.

We will now explain the differences in this program syntax between X10 and Java.
(i)~The return type in X10 is specified after the method parameters, and it may be automatically inferred and therefore omitted.
E.g., \code{:void} could have been omitted and because there are no \code{return} statements the compiler infers that the return type is \code{void}.
(ii)~Arrays in X10 and Java are different in many ways.
In Java, arrays are one-dimensional, indexed by an integer (\code{args[0]}), and use the special syntax \code{String[]}.
In X10, arrays are multi-dimensional, indexed by integers or a point (\code{args(0)}), and use the normal syntax for a generic class \code{Array[String]}.
(iii)~Generics in Java (e.g., \code{ArrayList<String>}) are erased at runtime, whereas generics in X10 are reified which means
    that the generic type parameters are kept at runtime.
This incurs a runtime overhead but it increases type-safety, security, and flexibility.
(iv)~The out stream in Java is \code{System.out} whereas in X10 it is \code{Console.OUT}.

We will now present a distributed and concurrent program in X10,
    that prints the first argument in all places.

\begin{lstlisting}
public class HelloWholeWorld {
  public static def main(args:Array[String]):void {
     finish
        for (p in Place.places())
          async
            at (p)
              Console.OUT.println("(At " + p + ") First argument: " + args(0));
  }
}
\end{lstlisting}

This program demonstrates the two most basic X10 concepts: \emph{places} and \emph{activities}.
Places enable distributed programming by using the \code{at} construct,
    whereas activities enable concurrent programming by using the \code{async} and \code{finish} constructs.

A \emph{place} in X10 may represent a GPU, a unique processor, or even a different machine.
From the programmer perspective, the running program has a single global address space.
The places partition that global address space: no two places have any storage
in common. (Therefore, X10 is a PGAS language: partitioned global address space.)
But, since there is a single global address space, an activity at one place may
    refer directly to storage at another (using distributed arrays or global references).
The \code{at (somePlace) STATEMENT} construct is used to shift the execution to \code{somePlace}.
Specifically, we copy all the data required for executing \code{STATEMENT} to \code{somePlace},
    and then start executing \code{STATEMENT} at \code{somePlace}.
In our example, we copy both \code{p} and \code{args} to place \code{p}.
\code{Place.places()} is a collection of all places,
    and we use an advanced \code{for} loop to iterate over all these places.

An \emph{activity} in X10 can be thought of as a lightweight thread in Java.
Activities in X10 can be short lived (e.g., simply summing two variables),
    or long lived (e.g., for the entire duration of the program).
An activity is created using \code{async S}, and we wait for activities to finish using \code{finish S}.
An activity \emph{locally terminated} after it finished executing \code{S}.
An activity \emph{globally terminated} after it locally terminated and
    after all activities spawned in \code{S} have (recursively) globally terminated.
The \code{finish S} construct waits until all activities spawned in \code{S} have globally terminated.

\subsection{How to parallelize a sequential program?}

\begin{lstlisting}
public class Fib {
  public static def serialFib(n:Int):Int {
    if (n<=1) return n;
    val fib1 = serialFib(n-1);
    val fib2 = serialFib(n-2);
    return fib1+fib2;
  }
  public static def concurrentFib(n:Int):Int {
    if (n<=1) return n;
    val fib1:Int;
    val fib2:Int;
    finish {
      async
        fib1 = concurrentFib(n-1);
      fib2 = concurrentFib(n-2);
    }
    return fib1+fib2;
  }
  public static def distributedFib(n:Int):Int {
    if (n<=1) return n;
    val fib1:Int;
    val fib2:Int;
    finish {
      async
        at (here.next())
          fib1 = concurrentFib(n-1);
      fib2 = concurrentFib(n-2);
    }
    return fib1+fib2;
  }
  public static def loadBalancingFib(n:Int):Int {
    todo: how to use UTS?
  }
}
\end{lstlisting}

Local variables are defined by using \code{val}, meaning the local is final value or immutable
    and cannot be reassigned.
Final locals have a single value during their lifetime, i.e., they may not be re-assigned.
In contrast, \code{var} means the local is a variable that can be re-assigned.

\subsection{X10 motivation}
X10 is an evolution of Java for concurrency and heterogeneity.
Programs written in X10 can scale to millions of places while achieving linear speedups.
Furthermore, these places can be very different, e.g., one place can be a CPU and another a GPU.

X10 focuses on high programmer productivity while maintaining high performance.
X10 is a high-level language with a rich type-system that ensures type- and memory-safety.
By leverages 5+ years of R\&D funded by DARPA/HPCS, X10 is high performance,
    even relative to MPI programs written in C.
%X10's goal is to deliver scalable performance competitive with C+MPI.

X10 has the following features:
\begin{description}
  \item[Modern OO language]
    X10 supports all the OO features found in Java,
        including classes, inner classes, interface, inheritance, overloading, etc.
    OO programming facilitates building libraries and frameworks that promotes reuse.
    Additionally, X10 has a richer type-system that allows the programmer
        to specify constraints over types, e.g., \code{Int\{self!=0\}}.
    Other novel X10 features include closures, structs, and reified generics.
  \item[Interoperability with Java]
    X10 code can be compiled into Java classes and run on any JVM.
    In this mode, X10 has full interoperability with Java,
        i.e., your X10 code can make calls into Java libraries,
        and your Java code can make calls into X10 code.
  \item[Fine-grained concurrency]
    Using \code{async} and \code{finish}, the programmer can specify
        fine-grained concurrency.
    X10 also has \code{atomic} blocks for mutual exclusion of activities,
        and \code{when} conditional and clocks for synchronization between activities.
    X10 runtime has an efficient novel work-stealing~\cite{} implementation
        that avoids the overhead of creating threads for short-lived activities.
  \item[Distribute computation]
    Given a large-scale cluster, we wish to distribute the computation in an efficient way.
    X10 gives the programmer complete control over how to distribute data and computation.
    There are also libraries for efficient load-balancing between places.
  \item[Heterogeneity]
    Heterogeneity nowadays is ubiquitous, and computation can be done on a CPU, GPU, FPGA, or a commodity cluster.
    X10 has a single programming model for computation offload.
    The programmer does not need to learn a new model, e.g., in order to use the GPU or
        a cluster special broadcasting feature.
  \item[Safety, correctness and productivity]
    Writing concurrent and distributed code is tricky and error-prone.
    Programming in X10 eliminates many sources of errors found in other languages,
        e.g., memory management errors of C++, initialization errors in Java,
        and subtle concurrency bugs due to data-races (leading to non-determinate code)
        and deadlocks.
    X10 supports various annotations that allow writing deadlock-free and determinate code.
\end{description}

X10 can be deployed on a multitude of target environments,
    ranging from high-end large clustered systems (BlueGene, P7IH),
    and medium-scale commodity systems (~1000 cores and ~1 terabyte main memory),
    all the way down to developer laptops (whether they run Linux, Mac, or Windows).
Developing X10 programs is easy using an eclipse-based IDE, with an integrated debugger.

Next we go into the various X10 features that increase productivity and performance.

\subsection{Closures}
Closures are in X10 are first-class values that represent functions.
For example, \code{(x:Int)=>x+1} is a function that receives an integer and returns an integer,
    thus its type is \code{(Int)=>Int}.
Closures are useful for array manipulation (mapping/scanning/reducing values of an array),
    e.g., suppose you have an array of grades (\code{Array[Double]}) and you want to give all students
        an increase by some factor:
\begin{lstlisting}
def giveFactor(gradesArray:Array[Double], factor:Double) {
  val newGrades = gradesArray.map( (x:Double) => x*factor );
  return newGrades;
}
\end{lstlisting}
Note how the closure can refer to variables in the outside scope (\code{factor}),
    but only if they are immutable (\code{val} and not \code{var}).


\subsection{Structs}
Java has 8 primitive types (\code{boolean}, \code{char}, \code{byte}, \code{short}, \code{int}, \code{long}, \code{float}, \code{double}).
X10 does not have primitive types, but instead it uses \emph{structs}.
A struct has more restrictions than classes, but it is more efficient for small objects.
A struct is defined in a similar way to a class but it has 2 restrictions:
    (i)~it cannot extend or be extended (it can implement interfaces),
    and (ii)~all its fields must be \code{val}.
\begin{lstlisting}
struct RGB {
  val r:Byte;
  val g:Byte;
  val b:Byte;
  def this(r:Byte,g:Byte,b:Byte) { // constructor
    this.r = r;
    this.g = g;
    this.b = b;
  }
}
\end{lstlisting}
You create a struct value like creating an object, by calling a constructor \code{new RGB(1,2)};
    in the case of a struct the keyword \code{new} is optional.
A struct value is header-less, i.e., it does not carry any type information.
Therefore, an \code{RGB} value will require only 3 bytes (if you ignore alignment issues).
It is also inlinable, i.e., it does not require an extra level of indirection like normal objects.
% no reference to a struct value,
Thus, a \code{Array[RGB]} will not store pointers to \code{RGB} objects, but instead these objects will be inlined
    directly in the array, thus saving the extra level of indirection of a lot of extra storage space.
Structs do not have object-identity, and therefore they are compared by value and not by reference.
Specifically, the compiler will automatically create an \code{equals} method in \code{RGB} that
    checks that all fields are equal.

If a struct is upcasted to an interface, then the compiler automatically boxes it and adds the required type-information.
For example, \code{(new RGB(1,2)) as Any}.


\subsection{Type inference}
Method return types.
Constructor return types
Types for initialized val�s can be omitted


\subsection{Constrained types}
Matrix example.
Types can specify constraints between values
E.g. Matrix{self.I==M,self.J==N} is an M x N matrix.
Can only be multiplied with an N1 x O matrix if N1==N: def mult(a:Matrix{self.I==this.J}):Matrix{self.I==this.I, self.J==a.J}
Constraints can be used as class invariants and method guards


\subsection{Type definitions}
Type
type Matrix(M,N)=Matrix{self.M==M,self.N==N}

\subsection{Reified generics}

\section{Implementation}
Two back-ends: Java and C++.
Compilation flow - include the image on slide 7.

\end{document}
