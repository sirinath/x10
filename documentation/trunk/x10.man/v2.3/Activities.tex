\chapter{Activities}\label{XtenActivities}
\index{activity}


An {\em activity} is a statement being executed, independently, with its own
local variables; it may be thought of as a very light-weight thread. An
\Xten{} computation may have many concurrent {activities} executing at any
give time.  All X10 code runs as part of an activity; when an X10 program is
started, the \xcd`main` method is invoked in an activity, called the {\em root
activity}.\index{root
activity}


Activities coordinate their execution by various control and data structures.
For example, 
%~~stmt~~`~~`~~x:Int, var y:Int ~~ ^^^ Activities10
\xcd`when(x==0);` blocks the current activity until some other activity
sets \xcd`x` to zero.  However, activities determine the places at which they
may be blocked and resumed, by \xcd`when` and similar constructs.  There are
no means by which one activity can arbitrarily interrupt, block, or resume
another.


\index{activity!running}
\index{activity!blocked}
\index{activity!terminated}
An activity may be {\em running}, {\em blocked} on some condition or {\em
terminated}. If it is terminated, it is terminated in the same way that its
statement is: in particular, if the statement terminates abruptly, the
activity terminates abruptly for the same reason.
(\Sref{ExceptionModel}).

Activities can be long-running entities with a good deal of local state.  In
particular they can involve recursive method calls (and therefore have runtime
stacks).  However, activities can also be short-running light-weight entities,
\eg, it is reasonable to have an activity that simply increments a variable.

An activity may asynchronously and in parallel launch activities at
other places.  Every activity except the initial \xcd`main` activity is spawned
by another.  Thus, at any instant, the activities in a program form a tree.

\index{termination}
\index{termination!local}
\index{termination!global}
X10 uses this tree in crucial ways.  
First is the distinction 
between {\em local} termination and {\em global}
termination of a statement. The execution of a statement by an
activity is said to terminate locally when the activity has finished
all its computation. (For instance the
creation of an asynchronous activity terminates locally when the
activity has been created.)  It is said to terminate globally when it
has terminated locally and all activities that it may have spawned at
any place have, recursively, terminated globally.
For example, consider: 
%~~gen ^^^ Activities20
% package Activites.Are.For.Whacktivities;
% class Example {
% def example( s1:() => void, s2 : () => void ) {
%~~vis
\begin{xten}
async {s1();}
async {s2();}
\end{xten}
%~~siv
% } } 
%~~neg
The primary activity spawns two child activities and then terminates locally,
very quickly.  The child activities may take arbitrary amounts of time to
terminate (and may spawn grandchildren).  When \xcd`s1()`, \xcd`s2()`, and
all their descendants terminate locally, then the primary activity terminates
globally. 

The program as a whole terminates when the root activity terminates globally.
In particular, X10 does not permit the creation of 
daemon threads---threads that outlive the lifetime of the root
activity.  We say that an \Xten{} computation is {\em rooted}
(\Sref{initial-computation}).

\futureext{ We may permit the initial activity to be a daemon activity
to permit reactive computations, such as webservers, that may not
terminate.}

\section{The \Xten{} rooted exception model}
\label{ExceptionModel}
\index{exception!model}
\index{exception!rooted}
\index{exception}


The rooted nature of \Xten{} computations permits the definition of a
{\em rooted exception model.} In multi-threaded programming languages
there is a natural parent-child relationship between a thread and a
thread that it spawns. Typically the parent thread continues execution
in parallel with the child thread. Therefore the parent thread cannot
serve to catch any exceptions thrown by the child thread. 

The presence of a root activity and the concept of global termination permits
\Xten{} to adopt a more powerful exception model. In any state of the
computation, say that an activity $A$ is {\em a root of} an activity $B$ if
$A$ is an ancestor of $B$ and $A$ is blocked at a statement (such as the
\xcd"finish" statement \Sref{finish}) awaiting the termination of $B$ (and
possibly other activities). For every \Xten{} computation, the \emph{root-of}
relation is guaranteed to be a tree. The root of the tree is the root activity
of the entire computation. If $A$ is the nearest root of $B$, the path from
$A$ to $B$ is called the {\em activation path} for the activity.\footnote{Note
  that depending on the state of the computation the activation path may
  traverse activities that are running, blocked or terminated.}

We may now state the exception model for \Xten.  An uncaught exception
propagates up the activation path to its nearest root activity, where
it may be handled locally or propagated up the \emph{root-of} tree when
the activity terminates (based on the semantics of the statement being
executed by the activity).\footnote{In \XtenCurrVer{} the \xcd"finish"
statement is the only statement that marks its activity as a root
activity. Future versions of the language may introduce more such
statements.}  
There is always a good place to put a \xcd`try`-\xcd`catch` block to catch
exceptions thrown by an asynchronous activity.

\section{{\tt async}: Spawning an activity}\label{AsynchronousActivity}\label{AsyncActivity}
\index{async}
\index{activity!creating}

Asynchronous activities serve as a single abstraction for supporting a
wide range of concurrency constructs such as message passing, threads,
DMA, streaming, and data prefetching. (In general, asynchronous operations
are better suited for supporting scalability than synchronous
operations.)

An activity is created by executing the \xcd`async` statement: 

%##(AsyncStatement ClockedClause
\begin{bbgrammar}
%(FROM #(prod:AsyncStmt)#)
           AsyncStmt \: \xcd"async" ClockedClause\opt Stmt & (\ref{prod:AsyncStmt}) \\
                     \| \xcd"clocked" \xcd"async" Stmt \\
%(FROM #(prod:ClockedClause)#)
       ClockedClause \: \xcd"clocked" Arguments & (\ref{prod:ClockedClause}) \\
\end{bbgrammar}
%##)


The basic form of \xcd`async` is \xcd`async S`, which starts a new activity
located \xcd`here` executing \xcd`S`.   (For the clocked form, see
\Sref{ClockedFinish}.)  

Multiple activities launched by a single activity at another place are not
ordered in any way. They are added to the set of activities at the target
place and will be executed based on the local scheduler's decisions.
If some particular sequencing of events is needed, \xcd`when`, \xcd`atomic`,
\xcd`finish`, clocks, and other X10 constructs can be used.
\Xten{} implementations are not required to have fair schedulers,
though every implementation should make a best faith effort to ensure
that every activity eventually gets a chance to make forward progress.


The statement in the body of an \xcd"async" is subject to the
restriction that it must be acceptable as the body of a \xcd"void"
method for an anonymous inner class declared at that point in the code. For
example, it may reference \xcd`val` variables in lexically enclosing scopes,
but not \xcd`var` variables.  Similarly, it cannot \xcd`break` or
\xcd`continue` surrounding loops.


\section{Finish}\index{finish}\label{finish}
The statement \xcd"finish S" converts global termination to local
termination.

%##(FinishStatement
\begin{bbgrammar}
%(FROM #(prod:FinishStmt)#)
          FinishStmt \: \xcd"finish" Stmt & (\ref{prod:FinishStmt}) \\
                     \| \xcd"clocked" \xcd"finish" Stmt \\
\end{bbgrammar}
%##)

An activity $A$ executes \xcd"finish S" by executing \xcd"S" and
then waiting for all activities spawned by \xcd`S` (directly or
indirectly, here or at other places) to terminate. An activity may
terminate normally, or abruptly, i.e. by throwing an exception.
All exceptions thrown by spawned activities are caught and
accumulated. 

\xcd"finish S" terminates locally when all activities spawned by
\xcd"S" terminate globally (either abruptly or normally). If \xcd"S"
terminates normally, then \xcd"finish S" terminates normally and $A$
continues execution with the next statement after \xcd"finish S".  If
\xcd"S" or one of the activities spawned by it terminate abruptly,
then \xcd"finish S" terminates abruptly and throws a single exception,
of type \Xcd{x10.lang.MultipleExceptions}, formed from the collection of
exceptions accumulated at \xcd"finish S".

Thus \xcd"finish S" statement serves as a collection point for
uncaught exceptions generated during the execution of \xcd"S".

Note that repeatedly \xcd"finish"ing a statement has little effect after
the first \xcd"finish": \xcd"finish finish S" is indistinguishable
from \xcd"finish S" if \xcd`S` terminates normally.  If \xcd`S` throws
exceptions, \xcd`finish S` collects the exceptions and wraps them in a 
\xcd`MultipleExceptions`, whereas \xcd`finish finish S` does the same, and
then puts that \xcd`MultipleExceptions` inside of a second
\xcd`MultipleExceptions`. 


\section{Initial activity}\label{initial-computation}
\index{initial activity}
\index{activity!initial}

An \Xten{} computation is initiated from the command line on the
presentation of a class or struct name \xcd"C". The container must have a 
\xcd`main` method: 
\begin{xtenmath}
public static def main(a: Array[String](1)):void
\end{xtenmath}
method, 
or a 
\begin{xtenmath}
public static def main(a: Array[String]):void
\end{xtenmath}
method, 
otherwise an exception is thrown
and the computation terminates.  The single statement
\begin{xten}
finish async at (Place.FIRST_PLACE) {
  C.main(s);
}
\end{xten} 
\noindent is executed where \xcd"s" is a one-dimensional \xcd`Array` of
strings created 
from the command line arguments. This single activity is the root activity
for the entire computation. (See \Sref{XtenPlaces} for a discussion of
places.)


\section{Ateach statements}\index{\Xcd{ateach}}\label{ateach-section}
\index{ateach}
\deprecated{} The \xcd`ateach` construct is deprecated.

%##(AtEachStatement LoopIndexDeclarator LoopIndex 
\begin{bbgrammar}
%(FROM #(prod:AtEachStmt)#)
          AtEachStmt \: \xcd"ateach" \xcd"(" LoopIndex \xcd"in" Exp \xcd")" ClockedClause\opt Stmt & (\ref{prod:AtEachStmt}) \\
                     \| \xcd"ateach" \xcd"(" Exp \xcd")" Stmt \\
%(FROM #(prod:LoopIndexDeclr)#)
      LoopIndexDeclr \: Id HasResultType\opt & (\ref{prod:LoopIndexDeclr}) \\
                     \| \xcd"[" IdList \xcd"]" HasResultType\opt \\
                     \| Id \xcd"[" IdList \xcd"]" HasResultType\opt \\
%(FROM #(prod:LoopIndex)#)
           LoopIndex \: Mods\opt LoopIndexDeclr & (\ref{prod:LoopIndex}) \\
                     \| Mods\opt VarKeyword LoopIndexDeclr \\
\end{bbgrammar}
%##)
In \xcd`ateach(p in D) S`, \xcd`D` must be either of type \xcd"Dist"
(see \Sref{XtenDistributions}) or of type \xcd`DistArray[T]` (see
\Sref{XtenArrays}), and \xcd`p` will be of type \xcd"Point" (see
\Sref{point-syntax}). If \xcd`D` is an \xcd`DistArray[T]`, then
\xcd`ateach (p in D)S` is identical to 
\xcd`ateach(p in D.dist)S`; the iteration is over the array's underlying
distribution.   

Instead of writing \xcd`ateach (p in D) S` the programmer could write 
%%AT-COPY%% %~~gen ^^^ Activities80
%%AT-COPY%% % package Activities.Activities.Activities;
%%AT-COPY%% % class EquivCode {
%%AT-COPY%% % static def S(pt:Point) {}
%%AT-COPY%% % static def example(D:Dist) {
%%AT-COPY%% %~~vis
%%AT-COPY%% \begin{xten}
%%AT-COPY%% for (place in D.places()) async at (place; p, D, *) {
%%AT-COPY%%     for (p in D|here) {
%%AT-COPY%%         S(p);
%%AT-COPY%%     }
%%AT-COPY%% }
%%AT-COPY%% \end{xten}
%%AT-COPY%% %~~siv
%%AT-COPY%% %}} 
%%AT-COPY%% %~~neg

%~~gen ^^^ Activities80
% package Activities.Activities.Activities;
% class EquivCode {
% static def S(pt:Point) {}
% static def example(D:Dist) {
%~~vis
\begin{xten}
for (place in D.places()) at (place) async {
    for (p in D|here) async {
        S(p);
    }
}
\end{xten}
%~~siv
%}} 
%~~neg
For each point \xcd`p` in \xcd`D`, statement \xcd`S` is executed concurrently 
at place \xcd`D(p)`.

\xcd`break` and \xcd`continue` statements may not be applied to \xcd`ateach`.

\section{\xcd`var`s and Activities}

X10 restricts the use of local \xcd`var` variables in activities, to make
programs more deterministic. Specifically, a local \xcd`var` variable \xcd`x`
defined outside of \xcd`async S` cannot appear inside \xcd`async S` unless
there is a \xcd`finish` surrounding \xcd`async S` with the definition of
\xcd`x` outside of it.

\begin{ex}
The following code is fine; the definition of \xcd`result` appears outside of
the \xcd`finish` block: 
%~~gen ^^^ Activities6f6s
% package Activities6f6s;
% class Example { static def example() { 
%~~vis
\begin{xten}
var result : Int = 0;
finish { 
  async result = 1;
}
assert result == 1;
\end{xten}
%~~siv
%} } 
% class Hook{ def run() {Example.example(); return true;}}
%~~neg

This code is deterministic: the \xcd`async` will finish before the
\xcd`assert` starts, and the \xcd`assert`'s test will be true.

However, without the
\xcd`finish`, it would be wrong, and would not compile in X10.  If it were
allowed to compile, the activity might finish or might not finish before the
\xcd`println`, and the program would not be deterministic.
\end{ex}


\section{Atomic blocks}\label{AtomicBlocks}
\index{atomic}

X10's \xcd`atomic` blocks provide a high-level construct for coordinating
the mutation of shared data. 
A programmer may use atomic blocks to guarantee that invariants of
shared data-structures are maintained even as they are being accessed
simultaneously by multiple activities running in the same place.  

An X10 program in which all accesses (both reads and writes) of shared
variables appear in \xcd`atomic` or \xcd`when` blocks is guaranteed to use all
shared variables atomically.  Equivalently, 
if two accesses to some shared variable \xcd`v` could collide at runtime, and
one is in an atomic block, then the other must be in an atomic block as well
to guarantee atomicity of the accesses to \xcd`v`. 
If some accesses to shared variables are not
protected by \xcd`atomic` or \xcd`when`, then race conditions or deadlocks may
occur.  

In particular, atomic sections at the same place are atomic with respect to
each other. They may not be atomic with respect to non-atomic code, or with
respect to atomic sections at different places.

X10 guarantees that atomic sections at the same place are mutually exclusive.
That is, if one activity $A$ at a given place $p$ is executing an atomic
section, then no other activity $B$ at $p$ will also be executing an atomic
section. If such a $B$ attempts to execute an \xcd`atomic` or \xcd`when`
command, it will be blocked until $A$ finishes executing its atomic section.  



%##(AtomicStatement WhenStatement
\begin{bbgrammar}
%(FROM #(prod:AtomicStmt)#)
          AtomicStmt \: \xcd"atomic" Stmt & (\ref{prod:AtomicStmt}) \\
%(FROM #(prod:WhenStmt)#)
            WhenStmt \: \xcd"when" \xcd"(" Exp \xcd")" Stmt & (\ref{prod:WhenStmt}) \\
\end{bbgrammar}
%##)

\begin{ex}
Consider a class \xcd`Redund[T]`, which encapsulates a list
\xcd`list` and, (redundantly) keeps the size of the list in a second field
\xcd`size`.  Then \xcd`r:Redund[T]` has the invariant 
\xcd`r.list.size() == r.size`, which must be true at any point at which
no method calls on \xcd`r` are active.

If the \xcd`add` method on \xcd`Redund` (which adds an element to the list) 
were defined as: 
%~~gen ^^^ Activities90
% package Activities.Atomic.Redund.One;
% import x10.util.*;
% class Redund[T] {
%   val list = new ArrayList[T]();
%   var size : Int = 0;
%~~vis
\begin{xten}
def add(x:T) { // Incorrect
  this.list.add(x);
  this.size = this.size + 1;
}
\end{xten}
%~~siv
%}
%~~neg
Then two activities simultaneously adding elements to the same \xcd`r` could break the
invariant.  Suppose that \xcd`r` starts out empty.  Let the first activity
perform the \xcd`list.add`, and compute \xcd`this.size+1`, which is 1, but not store it
back into \xcd`this.size` yet.  
(At this point, \xcd`r.list.size()==1` and \xcd`r.size==0`; the invariant
expression is false, but, as the first call to \xcd`r.add()` is active, the
invariant does not need to be true -- it only needs to be true when the
call finishes.)
Now, let the second activity do its call to
\xcd`add` to completion, which finishes with \xcd`r.size==1`.  
(As before, the invariant expression is false, but a call to \xcd`r.add()` is
still active, so the invariant need not be true.)
Finally, let
the first activity finish, which assigns the \xcd`1` computed before back into
\xcd`this.size`.  At the end, there are two elements in \xcd`r.list`, but
\xcd`r.size==1`. Since there are no calls to \xcd`r.add()` active, the
invariant is required to be true, but it is not.

In this case, the invariant can be maintained by making the increment atomic.
Doing so forbids that sequence of events; the \xcd`atomic` block cannot be
stopped partway.  
%~~gen ^^^ Activities100
% package Activities.Atomic.Redund.Two;
% import x10.util.*;
% class Redund[T] {
%   val list = new ArrayList[T]();
%   var size : Int = 0;
%~~vis
\begin{xten}
def add(x:T) { 
  atomic {
    this.list.add(x);
    this.size = this.size + 1; 
  }
}
\end{xten}
%~~siv
%}
%~~neg
\end{ex}

\subsection{Unconditional atomic blocks}
The simplest form of an atomic block is the {\em unconditional
atomic block}: \xcd`atomic S`.
When \xcd`atomic S` is executing at some place \xcd`p`, no other activity at
\xcd`p` may enter an atomic block.  
So, other activities may continue, even at the same place, but code protected
by atomic blocks is not subject to interference from other code in atomic
blocks. 

If execution of the statement may throw an exception, it is
the programmer's responsibility to wrap the atomic block within a
\xcd"try"/\xcd"finally" clause and include undo code in the finally
clause. Thus the \xcd"atomic" statement only guarantees atomicity on
successful execution, not on a faulty execution.




Atomic blocks are closely related to non-blocking synchronization
constructs \cite{herlihy91waitfree}, and can be used to implement 
non-blocking concurrent algorithms.

Code executed inside of \xcd`atomic S` and \xcd`when(E)S` is subject
to certain restrictions. A violation of these restrictions causes an 
\xcd`IllegalOperationException` to be thrown at the point of the violation.

\begin{itemize}
\item \xcd`S` may not spawn another activity.
\item \xcd`S` may not use any blocking statements; \xcd`when`, \xcd`next`,
      \xcd`finish`.  (The use of a nested \xcd`atomic` is permitted.)
\item \xcd`S` may not \xcd`force()` a \xcd`Future`. 
\item \xcd`S` may not use \xcd`at` expressions.
\end{itemize}




Note an important property of an (unconditional) atomic block:
\begin{eqnarray}
\mbox{\xcd"atomic \{s1; atomic s2\}"} &=& \mbox{\xcd"atomic \{s1; s2\}"}
\end{eqnarray}

Atomic blocks do not introduce deadlocks.    They may exhibit all the bad
behavior of sequential programs, including throwing exceptions and running
forever, but they are guaranteed not to deadlock.



\begin{ex}
The following class method implements a (generic) compare and swap (CAS) operation:


%~~gen ^^^ Activities110
% package Activities.And.Protein;
% class CASSizer{
%~~vis
\begin{xten}
var target:Any = null;
public atomic def CAS(old1: Any, y: Any):Boolean {
   if (target.equals(old1)) {
     target = y;
     return true;
   }
   return false;
}
\end{xten}
%~~siv
%}
%~~neg
\end{ex}


\subsection{Conditional atomic blocks}
\index{atomic!conditional}
\index{when}



Conditional atomic blocks allow the activity to wait for some condition to be
satisfied before executing an atomic block. For example, consider a
\xcd`Redund` class holding a list \xcd`r.list` and, redundantly, its length
\xcd`r.size`.  A \xcd`pop` operation will delay until the \xcd`Redund` is
nonempty, and then remove an element and update the length.  
%~~gen ^^^ Activities120
% package Activities.Condato.Example.Not.A.Tree;
% import x10.util.*;
% class Redund[T] {
% val list = new ArrayList[T]();
% var size : Int = 0;
%~~vis
\begin{xten}
def pop():T {
  var ret : T;
  when(size>0) {
    ret = list.removeAt(0);
    size --;
    }
  return ret;
}
\end{xten}
%~~siv
% }
%~~neg


The execution of the test is atomic with the execution of the block.  This is
important; it means that no other activity can sneak in and make the condition
be false after the test was seen to be true, but 
before the block is executed.  In this example, two \xcd`pop`s
executing on a list with one element would work properly. Without the
conditional atomic block -- even doing the decrement atomically -- one call to
\xcd`pop` could pass the \xcd`size>0` guard; then the other call could run to
completion (removing the only element of the list); then, when the first call
proceeds, its \xcd`removeAt` will fail.  

Note that \xcd`if` would not work here.  
\begin{xtenmath}
if(size>0) atomic{size--; return list.removeAt(0);}
\end{xtenmath} 
allows another
activity to act between the test and the atomic block.  
And 
\begin{xtenmath}
atomic{ if(size>0) {size--; ret = list.removeAt(0);}}
\end{xtenmath}
does not wait for \xcd`size>0` to become true.


Conditional atomic blocks are of the form \xcd`when(b)S`; 
\xcd`b` is called the {\em guard}, and \xcd`S` the {\em body}.

An activity executing such a statement suspends until such time as the  guard
is true in the current state. In that state, the 
body is executed. 
The checking of the guards and the execution of the corresponding
guarded statement is done atomically. 

\Xten{} does not guarantee that a conditional atomic block
will execute if its condition holds only intermittently. For, based on
the vagaries of the scheduler, the precise instant at which a
condition holds may be missed. Therefore the programmer is advised to
ensure that conditions being tested by conditional atomic blocks are
eventually stable, \ie, they will continue to hold until the block
executes (the action in the body of the block may cause the condition
to not hold any more).





The statement \xcd"when (true) S" is
behaviorally identical to \xcd"atomic S": it never suspends.

The body \xcd`S` of \xcd`when(b)S` is subject to the same restrictions that
the body of \xcd`atomic S` is.  The guard is subject to the same restrictions
as well.  Furthermore, guards should not have side effects.


Note that this implies that guarded statements are required to be {\em
flat}, that is, they may not contain conditional atomic blocks. (The
implementation of nested conditional atomic blocks may require
sophisticated operational techniques such as rollbacks.)


\begin{ex}
The following class shows how to implement a bounded buffer of size
$1$ in \Xten{} for repeated communication between a sender and a
receiver.  The call \xcd`buf.send(ob)` waits until the buffer has space, and
then puts \xcd`ob` into it.  Dually, \xcd`buf.receive()` waits until the
buffer has something in it, and then returns that thing.


%~~gen ^^^ Activities130
% package Activities;
%~~vis
\begin{xten}
class OneBuffer[T] {
  var datum: T;
  def this(t:T) { this.datum = t; this.filled = true; }
  var filled: Boolean;
  public def send(v: T) {
    when (!filled) {
      this.datum = v;
      this.filled = true;
    }
  }
  public def receive(): T {
    when (filled) {
      v: T = datum;
      filled = false;
      return v;
    }
  }
}
\end{xten}
%~~siv
%
%~~neg
\end{ex}

\subsubsection{When  {\tt when} is Tested}
\index{when!timing}

Suppose that activity {$A$} is blocked waiting on \xcd`when(e)S`, because
\xcd`e` is \xcd`false`.    If some other activity {$B$} changes the state 
in an \xcd`atomic` section 
in a
way that makes \xcd`e` become \xcd`true`, then either: 
\begin{itemize}
\item {$A$} will eventually execute \xcd`S`, or
\item Some activity {$C \ne A$} will cause \xcd`e` to become \xcd`false` again.
\end{itemize}
In particular, if no other activity ever falsifies \xcd`e`, then {$A$} will,
eventually, discover that \xcd`e` evaluates to \xcd`true` and run \xcd`S`.

Two caveats are worth noting: 
\begin{itemize}
\item X10 has no guarantees of fairness or liveness.
\item X10 only makes guarantees about state changes {\em in atomic sections}
      alerting \xcd`when`s.  State changes outside of atomic sections might
      not cause {$A$} to re-evaluate \xcd`e`.  
\end{itemize}

\begin{ex}
The method \xcd`good` below will always terminate.  In particular, if the
\xcd`when` statement is allowed to run first and block on \xcd`c()`, the
\xcd`atomic` will alert it that \xcd`c()` has changed. 

The method \xcd`bad` has
no such guarantee: it might terminate if the compiler and scheduler are in a
generous mood, or the \xcd`when` might wait forever to be told that \xcd`c()`
is now true.   Without an \xcd`atomic`, the \xcd`when` statement might not be
notified about the change in \xcd`c()`.  
%~~gen ^^^ Activities5m7u
% package Activities5m7u;
% class Example {
%~~vis
\begin{xten}
static def good() {
  val c = new Cell[Boolean](false);
  async {
    atomic {c() = true;}
  }
  when( c() ); 
}
static def bad() {
  val c = new Cell[Boolean](false);
  async {
    c() = true;
  }
  when( c() ); 
}
\end{xten}
%~~siv
% }
%~~neg


\end{ex}



\section{Use of Atomic Blocks}

The semantics of atomicity is chosen as a compromise between programming
simplicity and efficient implementation.  Unlike some possible definitions of
``atomic'', atomic blocks do not provide absolute atomicity.  

Atomic blocks are atomic with respect to {\em each other}.
%~~gen ^^^ Activities4c2r
% package Activities4c2r;
% class Example {
% def example() {
%~~vis
\begin{xten}
var n : Int = 0;
finish {
  async atomic n = n + 1; //(a)
  async atomic n = n + 2; //(b)
}
\end{xten}
%~~siv
%}}
%~~neg
This program has only two possible interleavings: either \xcd`(a)` entirely
precedes \xcd`(b)` or \xcd`(b)` entirely precedes \xcd`(a)`.  Both end up with
\xcd`n==3`. 


However, atomic blocks are not atomic with respect to non-atomic code.  It we
remove the  \xcd`atomic`s on \xcd`(a)`, we get far messier semantics.
%~~gen ^^^ Activities5u4q
% package Activities5u4q;
% class Example {
% def example() {
%~~vis
\begin{xten}
var n : Int = 0;
finish {
  // LEGAL BUT UNWISE 
  async n = n + 1;          //(a)
  async atomic n = n + 2;   //(b)
}
\end{xten}
%~~siv
%}}
%~~neg

If X10 had absolute atomic semantics, this program would be guaranteed to
treat the atomic increment as a single statement.  This would permit three
interleavings: the two possible from the fully atomic program, or a third one
with the events:  \xcd`(a)`'s read of \xcd`0` from \xcd`n`, the entirety of
\xcd`(b)`, and then \xcd`(a)`'s write of \xcd`0+1` back to \xcd`n`.  This
interleaving results in \xcd`n==1`. So, with absolute atomic semantics,
\xcd`n==1` or \xcd`n==3` are the possible results.

However, X10's semantics are weaker than that.  Atomic statements are atomic
with respect to each other --- but there is no guarantee about how they
interact with non-atomic statements at all.  They might even break up the
atomicity of an \xcd`atomic` block.
In particular, the following
fourth interleaving is possible: \xcd`(a)`'s read of \xcd`0` from \xcd`n`, 
\xcd`(b)`'s read of \xcd`0` from \xcd`n`, \xcd`(a)`'s write of \xcd`1` to
\xcd`n`, and \xcd`(b)`'s write of \xcd`2` to \xcd`n`.   Thus, \xcd`n==2` is
permissible as a result in X10.

X10's semantics permit more efficient implementation than absolute atomicity.
Absolute atomicity would, in principle, require all activities at place
\xcd`p` to stop whenever one of them enters an atomic section, which would
seriously curtail concurrency.  X10 simply requires that, when one activity is
in an atomic section, that other activities stop {\em when they are trying to
enter an atomic section} --- which is to say, they can continue computing on
their own all they like.  The difference can be substantial, both in execution
time and possible behaviors.

However, X10's semantics do impose a certain burden on the programmer.  A
sufficient rule of thumb is that, if {\em any} access to a
variable is done in an atomic section, then {\em all} accesses to it must be
in atomic sections.  

Atomic sections are a powerful and convenient general solution.  Classes in
the package 
\xcd`x10.util.concurrent` may be more efficient and more convenient in
particular cases.  For example, an \xcd`AtomicInteger` provides an atomic
integer cell, with atomic get, set, compare-and-set, and add operations.
Each \xcd`AtomicInteger` takes care of its own locking.  Accesses to one 
\xcd`AtomicInteger` {$a$} only block activities which try to access {$a$} ---
not others, not even if they are using different \xcd`AtomicInteger`s or even
\xcd`atomic` blocks.
