\chapter{Rails and Arrays}\label{XtenArrays}\index{array}

\section{Overview}

Indexable memory is a fundamental abstraction provided by a
programming language. To enable the programmer to best balance
performance and flexibility, \Xten{} provides a layered implementation
of indexable memory. The foundation of all indexable storage in
\Xten{} is \xcd{x10.lang.Rail}, an intrinsic one-dimensional array
analogous to the built-in arrays provided by languages such as C or
Java. On top of \xcd{Rail}, more sophisticated array abstractions can
be constructed completely as user-defined \Xten{} classes. Two such
families of user-defined array abstractions are included in the
\Xten{} core class libraries in the \xcd{x10.array} and
\xcd{x10.regionarray} packages.  Both families of arrays provide both
local (single place) and distributed (multi-place) arrays.

The next section specifies \xcd{Rail}.  Subsequent sections outline
the \xcd{x10.array} and \xcd{x10.regionarray} packages.  As discussed,
in more detail below, \xcd{Rail} and the classes of \xcd{x10.array}
provide significantly higher performance operations than the
corresponding classes of \xcd{x10.regionarray}. Therefore we recommend
that programmers only use the more general arrays of
\xcd{x10.regionarray} where the increased flexibility justifies the
redudced performance. We also encourage programmers to use the classes
of \xcd{x10.array} as an example of how to define high-performance
array abstractions in \Xten{} and use them as templates for defining
additional high-performance array abstractions (for example
column-major arrays as in Fortran or 1-based arrays as in MATLAB).

\section{Rails}\label{XtenRails}\index{Rail}

The \xcd{Rail} class provides a basic abstraction of fixed-sized
indexed storage.  If \xcd{r} is a \xcd{Rail[T]}, then \xcd{r} contains
\xcd{r.size} elements of type \xcd{T} that may be accessed using the
\xcd{Long} values \xcd{0} to \xcd{r.size-1} as indices.  All accesses
to the elements of a \xcd{Rail} are checked: attempting to use an
index that is less than \xcd{0} or greater than \xcd{r.size-1} will
result in an \xcd{ArrayIndexOutOfBoundsException} being thrown.

As shown in the example below, instances of \xcd{Rail[T]} may be created
using one of several constructors that initialize the data elements to
the zero-value of \xcd{T}, a single initial value, or to a different
initial value for each element of the \xcd{Rail}.

%~~gen ^^^ Rail10
% package Arrays.Rails.Rail10;
% class Example {
% def example() {
%~~vis
\begin{xten}
// A zero-initialized Rail of 10 doubles
val r1 = new Rail[Double](10);

// A Rail of 10 doubles, all initialized to pi
val r2 = new Rail[Double](10, Math.PI);

// A Rail of 10 doubles, r3(i) == i*pi
val r3 = new Rail[Double](10, (i:long)=>i*Math.PI);
\end{xten}
%~~siv
%} } 
%~~neg

As described in more detail in section~\ref{sect:RailCtors}, \Xten{} also includes a
shorthand form for \xcd{Rail} construction: simply put brackets around
a list of expressions.

%~~gen ^^^ Rail20
% package Arrays.Rails.Rail20;
% class Example{
% def example() {
%~~vis
\begin{xten}
// A Rail[Long] containing the first 5 primes
val r1 = [2,3,5,7,11];

// A Rail[Double] such that r2(i) == i*pi
val r2 = [Math.PI, 2*Math.PI, 3*Math.PI, 4*Math.PI];
\end{xten}
%~~siv
% } } 
%~~neg

Accessing and updating single elements of a \xcd{Rail} is doing using
application syntax. For example,
%~~stmt~~`~~`~~a:Rail[Long], i:Long ~~ ^^^ Rail30
\xcd`a(i)=a(i+1);`
sets the \xcd`i`th element of \xcd`a` to the value of the \xcd`i+1`
element of \xcd`a`.  If \xcd`T` supports the \xcd`+` or \xcd`-`
operation, then the usual pre/post increment/decrement operations are
also available on individual array elements.  For example, 
%~~exp~~`~~`~~a:Rail[Long], i:Long ~~ ^^^ Rail40
\xcd`a(i)++`
is equivalent to
%~~exp~~`~~`~~a:Rail[Long], i:Long ~~ ^^^ Rail50
\xcd`a(i) = a(i)+1`

Iteration over the elements of a \xcd{Rail} can be accomplished using
several equivalent idioms.  Furthermore, via some modest compiler
support, each of these for loops will actually result in identical
generated code.  In the examples below, \xcd`r` is a \xcd{Rail[Long]}
and \xcd`sum` is a local variable of type \xcd`Long`.
%~~gen ^^^ Rail60
% package Arrays.Rails.Rail60;
% class Example{
% def example(r:Rail[Long]) {
% var sum:long = 0;
%~~vis
\begin{xten}
// A classic C-style for loop
for (var i:long=0; i<r.size; i++) {
    sum += r(i);
}

// Iterate over the LongRange 0..(r.size-1)
for (i in 0..(r.size-1)) {
    sum += r(i);
}

// Get the LongRange to iterate over from r
for (i in r.range()) {
    sum += r(i);
}

// Directly iterate over the values of r
for (v in r) {
    sum += v;
} 
\end{xten}
%~~siv
% } } 
%~~neg

Basic bulk operations such as clearing (setting to zero), filling with
a single value, and copying are provided by methods of the \xcd{Rail}
class.  Efficient copying of Rails across places is supported via the
combination of the \xcd{x10.lang.GlobalRail} struct\footnote{a specialized
  version of \xcd{x10.lang.GlobalRef} that includes the \xcd{size} of the
  referenced \xcd{Rail} to enable bounds checking} and the
\xcd{asyncCopy} method of \xcd{Rail}. 

Additional complex bulk operations on \xcd{Rail} such as sorting,
searching, map, and reduce are provided by the
\xcd{x10.util.RailUtils} class.

When implementing higher-level data structures that use \xcd{Rail} as
their backing storage, there may be significant performance advantage
in performing unsafe operations on \xcd{Rail}.  For example, when
building a multi-dimensional \xcd{Array} class, the \xcd{Array} level
bounds checking and initialization logic make the \xcd{Rail} level
operations redundant. To support such scenarios the class
\xcd{x10.lang.Unsafe} provides methods to allocate uninitialized
\xcd{Rail} objects and to access the elements of a \xcd{Rail} without
bounds checking.  These unsafe extensions should be used judiciously,
as improper use can result in memory safety violations that would not
be possible in pure \Xten{} code. 

\section{x10.array: Simple Arrays}\label{XtenSimpleArray}

Classes in the \xcd{x10.array} package provide high-performance
implementations of both local and distributed multi-dimension arrays.
The array implementations in this package are restricted to the case
of rectangular, dense, zero-based index spaces.  By making this
restriction, the indexing calculations for both single-place and
multi-place arrays can be optimized, resulting in an array
implementation that should obtain equivalent performance to the
corresponding array abstractions in C or Fortran. More general index
spaces are supported by the classes in the \xcd{x10.regionarray}
package (see Section~\ref{XtenRegionArray}.

The three main concepts of this package: iteration spaces, arrays, and
distributed arrays are outlined below.  All three concepts are
implemented as simple class hierarchies with an abstract superclass
and a collection of concrete final subclasses that contain the
performance-critical operations.  Client code using the abstractions
can be written (with lower performance) using the abstract APIs
provided by the superclass, but performance sensitive code should be
written using the more specific type of the concrete subclasses.  This
enables compile time optimization and inlining of key operations,
resulting in optimal code sequences.

\subsection{Points}\label{point-syntax}
\index{point}
\index{point!syntax}

Both kinds of arrays are indexed by \xcd`Point`s, which are
$n$-dimensional tuples of integers.  The \xcd`rank` property of a
point gives its dimensionality.  Points can be constructed from
integers, or \xcd`Rail[Long] by the \xcd`Point.make` factory methods:
%~~gen ^^^ArraysPointsExample1
% package Arrays.Points.Example1;
% class Example1 {
% def example1() {
%~~vis
\begin{xten}
val origin_1 : Point{rank==1} = Point.make(0);
val origin_2 : Point{rank==2} = Point.make(0,0);
val origin_5 : Point = Point.make(new Rail[Long](5));
\end{xten}
%~~siv
% } } 
%~~neg

There is an implicit conversion from \xcd`Rail[Long]` to 
\xcd`Point`, giving
a convenient syntax for constructing points: 

%~~gen ^^^ Arrays30
% package Arrays.Points.Example2;
% class Example{
% def example() {
%~~vis
\begin{xten}
val p : Point = [1,2,3];
val q : Point{rank==5} = [1,2,3,4,5];
val r : Point(3) = [11,22,33];
\end{xten}
%~~siv
% } } 
%~~neg

The coordinates of a point are available by function application, or, if you
prefer, by subscripting; \xcd`p(i)` is the
\xcd`i`th coordinate of the point \xcd`p`. 
\xcdmath`Point($n$)` is a \Xcd{type}-defined shorthand  for 
\xcdmath`Point{rank==$n$}`.

\subsection{IterationSpace}
An \xcd{IterationSpace} is a generalization of \xcd{LongRange} to
multiple dimensions.  The \xcd{rank} property of the
\xcd{IterationSpace} corresponds to its dimensionality.  An
\xcd{IterationSpace} represents an ordered collection of \xcd{Points}
of the same \xcd{rank} as the \xcd{IterationSpace}.  The primary
purpose of an \xcd{IterationSpace} is to represent the indices of an
\xcd{Array} or \xcd{DistArray}.  

\subsection{Array}

The abstract \xcd{Array} class provides rank-generic operations for
single place multi-dimensional arrays.   Its concrete subclasses
\xcd{Array_1}, \xcd{Array_2}, etc. provide rank-specific operations
such as efficient element access. The APIs of the classes are designed
to be a natural extension of the \xcd{Rail} API to multiple
dimensions.  In most usage scenarios, programmers should
operate using the types of the concrete subclasses, not of \xcd{Array}
itself. 

The example below illustrates the construction and indexing operations
of rank 2 arrays using a simple matrix multiply kernel where \xcd'N'
is a \xcd{Long}.

%~~gen ^^^ SimpleArrays10
% package Arrays.SimpleArrays.Example1;
% import x10.array.*;
% class Example{
% def example(N:long) {
%~~vis
\begin{xten}
val a = new Array_2[double](N, N, (i:long,j:long)=>(i*j) as double);
val b = new Array_2[double](N, N, (i:long,j:long)=>(i-j) as double);
val c = new Array_2[double](N, N, (i:long,j:long)=>(i+j) as double);

for (i in 0..(N-1))
  for (j in 0..(N-1))
    for (k in 0..(N-1))
      a(i,j) += b(i,k)*c(k,j);
\end{xten}
%~~siv
% } } 
%~~neg

Similarly to \xcd{Rail}, iteration over the elements of a \xcd{Array} 
can be accomplished using
several equivalent idioms.  Furthermore, via some modest compiler
support, each of these for loops will actually result in identical
generated code.  In the examples below, \xcd`a` is a \xcd{Array_2[Long]}
and \xcd`sum` is a local variable of type \xcd`Long`.
%~~gen ^^^ SimpleArrays20
% package Arrays.SimpleArrays.Example2;
% import x10.array.*;
% class Example{
% def example(a:Array_2[Long]) {
% var sum:long = 0;
%~~vis
\begin{xten}
// A classic C-style for loop
for (var i:long=0; i<a.numElems_1; i++) {
  for (var j:long=0; j<a.numElems_2; j++) {
    sum += a(i,j);
  }
}

// Iterate over the indices of a using Point destructuring 
for ([i,j] in a.indices()) {
    sum += a(i,j);
}

// Directly iterate over the values of a
for (v in a) {
    sum += v;
} 
\end{xten}
%~~siv
% } } 
%~~neg

An additional idiom which iterates over the \xcd{Point}s in the
\xcd{IterationSpace} of \xcd{a} without destructuring the \xcd{Point}s
is also supported, but should be avoid in the current implementation of
\Xten{} as the compiler does not optimize away all of the overheads
associated with explicit use of \xcd{Point} objects.
%~~gen ^^^ SimpleArrays30
% package Arrays.SimpleArrays.Example3;
% import x10.array.*;
% class Example{
% def example(a:Array_2[Long]) {
% var sum:long = 0;
%~~vis
\begin{xten}
// Iterate over the indices of a using Points
for (p in a.indices()) {
    sum += a(p);
}
\end{xten}
%~~siv
% } } 
%~~neg

\subsection{DistArray}

The abstract class \xcd{DistArray} and its concrete subclasses
represent an extension of the \xcd{Array} API to multiple places. 
In \Xten{} version 2.4.0 \xcd{DistArray}s are still a work in
progress.  Only three concrete implementations are available:
\xcd{DistArray_Unique} which has one data element per \xcd{Place},
\xcd{DistArray_Block_1} which block distributes a rank-1 array across
the \xcd{Place}s in its \xcd{PlaceGroup}, and
\xcd{DistArray_BlockBlock_2} which distributes blocks of a rank-2
array across the \xcd{Place}s in its \xcd{PlaceGroup}.

The API of \xcd{DistArray} is preliminary in this release of \Xten{}
we anticipate extending it with more operations and supporting a wider
range of ranks and distributions in future release of \Xten{}.

\section{x10.regionarray: Flexible Arrays}\label{XtenRegionArray}

Classes in the \Xcd{x10.regionarray} package provide the most general and 
flexible array abstraction that support mapping arbitrary multi-dimensional
index spaces to data elements. Although they are significantly more
flexible than \Xcd{Rail}s or the classes of the \Xcd{x10.array}
package, this flexibility does carry with it an expectation of lower
runtime performance. 

\Xcd{Array}s provide indexed access to data at a single \Xcd{Place}, {\em via}
\Xcd{Point}s---indices of any dimensionality. \Xcd{DistArray}s is similar, but
spreads the data across multiple \xcd`Place`s, {\em via} \Xcd{Dist}s.  

\subsection{Regions}\label{XtenRegions}\index{region}
\index{region!syntax}

A {\em region} is a set of points of the same rank.  {}\Xten{}
provides a built-in class, \xcd`x10.regionarray.Region`, to allow the
creation of new regions and to perform operations on regions. 
Each region \xcd`R` has a property \xcd`R.rank`, giving the dimensionality of
all the points in it.

\begin{ex}
%~~gen ^^^ Arrays40
% package Arrays40;
% import x10.regionarray.*;
% class Example {
% static def example() {
%~~vis
\begin{xten}
val MAX_HEIGHT=20;
val Null = Region.makeUnit(); //Empty 0-dimensional region
val R1 = Region.make(1, 100); // Region 1..100
val R2 = Region.make(1..100);  // Region 1..100
val R3 = Region.make(0..99, -1..MAX_HEIGHT);
val R4 = Region.makeUpperTriangular(10);
val R5 = R4 && R3; // intersection of two regions
\end{xten}
%~~siv
% } } 
%~~neg

The \xcd`LongRange` value \xcd`1..100` can be used to construct
the one-dimensional \xcd`Region` consisting of the points
$\{$\xcdmath`[m]`, \dots, \xcdmath`[n]`$\}$
\xcd`Region` by using the \xcd`Region.make` factory method.  
\xcd`LongRange`s are useful in building up regions, especially rectangular regions.  
\end{ex}

By a special dispensation, the compiler knows that, if \xcd`r : Region(m)` and
\xcd`s : Region(n)`, then \xcd`r*s : Region(m+n)`.  (The X10 type system
ordinarily could not specify the sum; the best it could do 
would be \xcd`r*s : Region`, with the rank of the region unknown.)  This
feature allows more convenient use of arrays; in particular, one does not need
to keep track of ranks nearly so much.

Various built-in regions are provided through  factory
methods on \xcd`Region`.  
\begin{itemize}
%~~exp~~"~~"~~ n:Long ~~ import x10.regionarray.*; ^^^Arrays3s5h
\item \xcd"Region.makeEmpty(n)" returns an empty region of rank \xcd"n".
%~~exp~~"~~"~~ n:Long ~~ import x10.regionarray.*; ^^^Arrays3x4j
\item \xcd"Region.makeFull(n)" returns the region containing all points of
      rank \xcd"n".  
%~~exp~~"~~"~~ ~~ import x10.regionarray.*; ^^^Arrays7l3d
\item \xcd"Region.makeUnit()" returns the region of rank 0 containing the
      unique point of rank 0.  It is useful as the identity for Cartesian
      product of regions.
%~~exp~~"~~"~~ normal:Point, k:Long ~~ import x10.regionarray.*; ^^^Arrays3l7z
\item \xcd"Region.makeHalfspace(normal, k)",
      where \xcd`normal` is a \xcd`Point` and \xcd`k` an \xcd`Long`, 
      returns the unbounded
      half-space of rank \xcd"normal.rank", consisting of all points \xcd"p"
      satisfying the vector inequality \xcdmath`p$\cdot$normal $\le$ k`.
%~~exp~~"~~"~~ min:Rail[Long], max:Rail[Long] ~~ import x10.regionarray.*; ^^^Arrays3i3n
\item \xcd"Region.makeRectangular(min, max)", 
      where \xcd"min" and \xcd"max"
      are rank-1 length-\xcd`n` integer arrays, returns a
      \xcd"Region(n)" equal to: 
      \xcdmath`[min(0) .. max(0), $\ldots$, min(n-1)..max(n-1)]`.
%~~exp~~"~~"~~ size: int, a: int, b: int~~ import x10.regionarray.*; ^^^Arrays2f2y
\item \xcd"Region.makeBanded(size, a, b)" constructs the
      banded \xcd"Region(2)" of size \xcd"size", with \Xcd{a} bands above
      and \Xcd{b} bands below the diagonal.
%~~exp~~"~~"~~size:Long ~~ import x10.regionarray.*; ^^^Arrays5s3q
\item \xcd"Region.makeBanded(size)" constructs the banded \Xcd{Region(2)} with
      just the main diagonal.
%~~exp~~`~~`~~N:Long ~~ import x10.regionarray.*; ^^^Arrays5s3qtri
\item \xcd`Region.makeUpperTriangular(N)` returns a region corresponding
to the non-zero indices in an upper-triangular \xcd`N x N` matrix.
%~~exp~~`~~`~~N:Long ~~ import x10.regionarray.*; ^^^Arrays5s3qlowertri
\item \xcd`Region.makeLowerTriangular(N)` returns a region corresponding
to the non-zero indices in a lower-triangular \xcd`N x N` matrix.
\item 
  If \xcd`R` is a region, and \xcd`p` a Point of the same rank, then 
%~~exp~~`~~`~~R:Region, p:Point(R.rank) ~~ import x10.regionarray.*; ^^^ Arrays50
  \xcd`R+p` is \xcd`R` translated forwards by 
  \xcd`p` -- the region whose
%~~exp~~`~~`~~r:Point, p:Point(r.rank) ~~ import x10.regionarray.*; ^^^ Arrays60
  points are \xcd`r+p` 
  for each \xcd`r` in \xcd`R`.
\item 
  If \xcd`R` is a region, and \xcd`p` a Point of the same rank, then 
%~~exp~~`~~`~~R:Region, p:Point(R.rank) ~~ import x10.regionarray.*; ^^^ Arrays70
  \xcd`R-p` is \xcd`R` translated backwards by 
  \xcd`p` -- the region whose
%~~exp~~`~~`~~r:Point, p:Point(r.rank) ~~ import x10.regionarray.*; ^^^ Arrays80
  points are \xcd`r-p` 
  for each \xcd`r` in \xcd`R`.

\end{itemize}

All the points in a region are ordered canonically by the
lexicographic total order. Thus the points of the region \xcd`(1..2)*(1..2)`
are ordered as 
\begin{xten}
(1,1), (1,2), (2,1), (2,2)
\end{xten}
Sequential iteration statements such as \xcd`for` (\Sref{ForAllLoop})
iterate over the points in a region in the canonical order.

A region is said to be {\em rectangular}\index{region!convex} if it is of
the form \xcdmath`(T$_1$ * $\cdots$ * T$_k$)` for some set of intervals
\xcdmath`T$_i = $ l$_i$ .. h$_i$ `. 
In particular an \xcd`LongRange` turned into a \xcd`Region` is rectangular: 
%~~exp~~`~~`~~ ~~ import x10.regionarray.*; ^^^Arrays3x4z
\xcd`Region.make(1..10)`.
Such a
region satisfies the property that if two points $p_1$ and $p_3$ are
in the region, then so is every point $p_2$ between them (that is, it is {\em convex}). 
(Banded and triangular regions are not rectangular.)
The operation
%~~exp~~`~~`~~R:Region ~~ import x10.regionarray.*; ^^^ Arrays90
\xcd`R.boundingBox()` gives the smallest rectangular region containing
\xcd`R`.

\subsubsection{Operations on regions}
\index{region!operations}

Let \xcd`R` be a region. A {\em sub-region} is a subset of \xcd"R".
\index{region!sub-region}

Let \xcdmath`R1` and \xcdmath`R2` be two regions whose types establish that
they are of the same rank. Let \xcdmath`S` be another region; its rank is
irrelevant. 

\xcdmath`R1 && R2` is the intersection of \xcdmath`R1` and
\xcdmath`R2`, \viz, the region containing all points which are in both
\Xcd{R1} and \Xcd{R2}.  \index{region!intersection}
%~~exp~~`~~`~~ ~~ import x10.regionarray.*; ^^^ Arrays100
For example, \xcd`Region.make(1..10) && Region.make(2..20)` is \Xcd{2..10}.


\xcdmath`R1 * S` is the Cartesian product of \xcdmath`R1` and
\xcdmath`S`,  formed by pairing each point in \xcdmath`R1` with every  point in \xcdmath`S`.
\index{region!product}
%~~exp~~`~~`~~ ~~ import x10.regionarray.*; ^^^ Arrays110
Thus, \xcd`Region.make(1..2)*Region.make(3..4)*Region.make(5..6)`
is the region of rank \Xcd{3} containing the eight points with coordinates
\xcd`[1,3,5]`, \xcd`[1,3,6]`, \xcd`[1,4,5]`, \xcd`[1,4,6]`,
\xcd`[2,3,5]`, \xcd`[2,3,6]`, \xcd`[2,4,5]`, \xcd`[2,4,6]`.


For a region \xcdmath`R` and point \xcdmath`p` of the same rank,
%~~exp~~`~~`~~R:Region, p:Point{p.rank==R.rank} ~~ import x10.regionarray.*; ^^^ Arrays120
\xcd`R+p` 
and
%~~exp~~`~~`~~R:Region, p:Point{p.rank==R.rank} ~~ import x10.regionarray.*; ^^^ Arrays130
\xcd`R-p` 
represent the translation of the region
forward 
and backward 
by \xcdmath`p`. That is, \Xcd{R+p} is the set of points
\Xcd{p+q} for all \Xcd{q} in \Xcd{R}, and \Xcd{R-p} is the set of \Xcd{q-p}.

More \Xcd{Region} methods are described in the API documentation.

\subsection{Arrays}
\index{array}

Arrays are organized data, arranged so that the data can be accessed with subscripts.
An \xcd`Array[T]` \Xcd{A} has a \Xcd{Region} \Xcd{A.region}, specifying which
\Xcd{Point}s are in \Xcd{A}.  For each point \Xcd{p} in \Xcd{A.region},
\Xcd{A(p)} is the datum of type \Xcd{T} associated with \Xcd{p}.  X10
implementations should 
attempt to store \xcd`Array`s efficiently, and to make array element accesses
quick---\eg, avoiding constructing \Xcd{Point}s when unnecessary.

This generalizes the concepts of arrays appearing in many other programming
languages.  A \Xcd{Point} may have any number of coordinates, so an
\xcd`Array` can have, in effect, any number of integer subscripts.  

\begin{ex}Indeed, it is possible to write code that works on \Xcd{Array}s regardless 
of dimension.  For example, to add one \Xcd{Array[Long]} \Xcd{src} into another
\Xcd{dest}, 
%~~gen ^^^ Arrays140
%package Arrays.Arrays.Arrays.Example;
%import x10.regionarray.*;
% class Example{
%~~vis
\begin{xten}
static def addInto(src: Array[Long], dest:Array[Long])
  {src.region == dest.region}
  {
    for (p in src.region) 
       dest(p) += src(p);
  }
\end{xten}
%~~siv
%}
% class Hook{
%   def run() { 
%     val a = new Array[Long](3, [1,2,3]);
%     val b = new Array[Long](a.region, (p:Point(1)) => 10*a(p) );
%     Example.addInto(a, b);
%     return b(0) == 11 && b(1) == 22 && b(2) == 33;
% }}
%~~neg
\noindent
Since \Xcd{p} is a \Xcd{Point}, it can hold as many coordinates as are
necessary for the arrays \Xcd{src} and \Xcd{dest}.
\end{ex}

The basic operation on arrays is subscripting: if \Xcd{A} is an \Xcd{Array[T]}
and \Xcd{p} a point with the same rank as \xcd`A.region`, then
%~~exp~~`~~`~~A:Array[Long], p:Point{self.rank == A.region.rank} ~~ import x10.regionarray.*; ^^^ Arrays150
\xcd`A(p)`
is the value of type \Xcd{T} associated with point \Xcd{p}.
This is the same operation as function application
(\Sref{sect:FunctionApplication}); arrays implement function types, and can be
used as functions.

Array elements can be changed by assignment. If \Xcd{t:T}, 
%~~gen ^^^ Arrays160
%package Arrays.Arrays.Subscripting.Is.From.Mars;
%import x10.regionarray.*; 
%class Example{
%def example[T](A:Array[T], p: Point{rank == A.region.rank}, t:T){
%~~vis
\begin{xten}
A(p) = t;
\end{xten}
%~~siv
%} } 
%~~neg
modifies the value associated with \Xcd{p} to be \Xcd{t}, and leaves all other
values in \Xcd{A} unchanged.

An \Xcd{Array[T]} named \Xcd{a} has: 
\begin{itemize}
%~~exp~~`~~`~~a:Array[Long] ~~ import x10.regionarray.*; ^^^ Arrays170
\item \xcd`a.region`: the \Xcd{Region} upon which \Xcd{a} is defined.
%~~exp~~`~~`~~a:Array[Long] ~~ import x10.regionarray.*; ^^^ Arrays180
\item \xcd`a.size`: the number of elements in \Xcd{a}.
%~~exp~~`~~`~~a:Array[Long] ~~ import x10.regionarray.*; ^^^ Arrays190
\item \xcd`a.rank`, the rank of the points usable to subscript \Xcd{a}. 
      \xcd`a.rank` is a cached copy of 
      \Xcd{a.region.rank}.
\end{itemize}

\subsubsection{Array Constructors}
\index{array!constructor}

To construct an array whose elements all have the same value \Xcd{init}, call
\Xcd{new Array[T](R, init)}. 
For example, an array of a thousand \xcd`"oh!"`s can be made by:
%~~exp~~`~~`~~ ~~ import x10.regionarray.*; ^^^ Arrays200
\xcd`new Array[String](1000, "oh!")`.


To construct and initialize an array, call the two-argument constructor. 
\Xcd{new Array[T](R, f)} constructs an array of elements of type \Xcd{T} on
region \Xcd{R}, with \Xcd{a(p)} initialized to \Xcd{f(p)} for each point
\Xcd{p} in \Xcd{R}.  \Xcd{f} must be a function taking a point of rank
\Xcd{R.rank} to a value of type \Xcd{T}.  

\begin{ex}
One way to construct the array \xcd`[11, 22, 33]` is with an array constructor
%~~exp~~`~~`~~ ~~ import x10.regionarray.*; ^^^ Arrays210
\xcd`new Array[Long](3, (i:long)=>11*i)`. 
To construct a multiplication table, call
%~~exp~~`~~`~~ ~~ import x10.regionarray.*; ^^^ Arrays220
\xcd`new Array[Long](Region.make(0..9, 0..9), (p:Point(2)) => p(0)*p(1))`.
\end{ex}

Other constructors are available; see the API documentation and
\Sref{sect:RailCtors}. 

\subsubsection{Array Operations}
\index{array!operations on}

The basic operation on \Xcd{Array}s is subscripting.  If \Xcd{a:Array[T]} and 
\xcd`p:Point{rank == a.rank}`, then \Xcd{a(p)} is the value of type \Xcd{T}
appearing at position \Xcd{p} in \Xcd{a}.    The syntax is identical to
function application, and, indeed, arrays may be used as functions.
\Xcd{a(p)} may be assigned to, as well, by the usual assignment syntax
%~~exp~~`~~`~~a:Array[Long], p:Point{rank == a.rank}, t:Long ~~ import x10.regionarray.*; ^^^ Arrays230
\xcd`a(p)=t`.
(This uses the application and setting syntactic sugar, as given in \Sref{set-and-apply}.)

Sometimes it is more convenient to subscript by longs.  Arrays of rank 1-4
can, in fact, be accessed by longs: 
%~~gen ^^^ Arrays240
%package Arrays240;
%import x10.regionarray.*;
%class Example{
%static def example(){
%~~vis
\begin{xten}
val A1 = new Array[Long](10, 0);
A1(4) = A1(4) + 1;
val A4 = new Array[Long](Region.make(1..2, 1..3, 1..4, 1..5), 0);
A4(2,3,4,5) = A4(1,1,1,1)+1;
\end{xten}
%~~siv
% assert A1(4) == 1 && A4(2,3,4,5) == 1;
%}}
% class Hook{ def run() {Example.example(); return true;}}
%~~neg



Iteration over an \Xcd{Array} is defined, and produces the \Xcd{Point}s of the
array's region.  If you want to use the values in the array, you have to
subscript it.  For example, you could take the logarithm of every element of an
\Xcd{Array[Double]} by: 
%~~gen ^^^ Arrays250
%package Arrays250;
%import x10.regionarray.*;
%class Example{
%static def example(a:Array[Double]) {
%~~vis
\begin{xten}
for (p in a) a(p) = Math.log(a(p));
\end{xten}
%~~siv
%}}
% class Hook{ def run() { val a = new Array[Double](2, [1.0,2.0]); Example.example(a); return a(0)==Math.log(1.0) && a(1)==Math.log(2.0); }}

%~~neg

The method \xcd`a.values()` can be used to enumerate all the values of an \xcd`Array[T]` array \xcd`a`.
%~~gen ^^^ Arrays251
%package Arrays251;
%import x10.regionarray.*;
%class Example{
%~~vis
\begin{xten}
static def allNonNegatives(a:Array[Double]):Boolean {
 for (v in a.values()) if (v < 0.0D) return false;
 return true;
}
\end{xten}
%~~siv
%}
% class Hook{ def run() { val a = new Array[Double](2, [1.0,2.0]); return Example.allNonNegatives(a);  }}
%~~neg


\subsection{Distributions}\label{XtenDistributions}
\index{distribution}

Distributed arrays are spread across multiple \xcd`Place`s.  
A {\em distribution}, a mapping from a region to a set of places, 
describes where each element of a distributed array is kept.
Distributions are embodied by the class \Xcd{x10.regionarray.Dist} and its
subclasses. 
The {\em rank} of a distribution is the rank of the underlying region, and
thus the rank of every point that the distribution applies to.


\begin{ex}
%~~gen ^^^ Arrays260
%package Arrays.Dist_example_a;
%import x10.regionarray.*;
% class Example{
% def example() {
%~~vis
\begin{xten}
val R  <: Region = Region.make(1..100);
val D1 <: Dist = Dist.makeBlock(R);
val D2 <: Dist = Dist.makeConstant(R, here);
\end{xten}
%~~siv
% } } 
%~~neg

\xcd`D1` distributes the region \xcd`R` in blocks, with a set of consecutive
points at each place, as evenly as possible.  \xcd`D2` maps all the points in
\xcd`R` to \xcd`here`.  
\end{ex}

Let \xcd`D` be a distribution. 
%~~exp~~`~~`~~D:Dist ~~ import x10.regionarray.*; ^^^ Arrays270
\xcd`D.region` 
denotes the underlying
region. 
Given a point \xcd`p`, the expression
%~~exp~~`~~`~~ D:Dist, p:Point{p.rank == D.rank}~~ import x10.regionarray.*; ^^^ Arrays280
\xcd`D(p)` represents the application of \xcd`D` to \xcd`p`, that is,
the place that \xcd`p` is mapped to by \xcd`D`. The evaluation of the
expression \xcd`D(p)` throws an \xcd`ArrayIndexOutofBoundsException`
if \xcd`p` does not lie in the underlying region.


\subsubsection{{\tt PlaceGroup}s}

A \xcd`PlaceGroup` represents an ordered set of \xcd`Place`s.
\xcd`PlaceGroup`s exist for performance and scaleability: they are more
efficient, in certain critical places, than general collections of
\xcd`Place`. \xcd`PlaceGroup` implements \xcd`Sequence[Place]`, and thus
provides familiar operations -- \xcd`pg.size()` for the number of places,
\xcd`pg.iterator()` to iterate over them, etc.  

\xcd`PlaceGroup` is an abstract class.  The concrete class
\xcd`SparsePlaceGroup` is intended for a small group of places. 
%~~exp~~`~~`~~ somePlace:Place ~~ ^^^Arrays1j6q
\xcd`new SparsePlaceGroup(somePlace)` is a good \xcd`PlaceGroup` containing
one place.  
%~~exp~~`~~`~~ seqPlaces: Rail[Place] ~~ ^^^Arrays9g6f
\xcd`new SparsePlaceGroup(seqPlaces)`
constructs a sparse place group from a Rail of places.

\subsubsection{Operations returning distributions}
\index{distribution!operations}



Let \xcd`R` be a region, \xcd`Q` 
a \xcd`PlaceGroup`, and \xcd`P` a place.

\paragraph{Unique distribution} \index{distribution!unique}
%~~exp~~`~~`~~Q:PlaceGroup ~~ import x10.regionarray.*; ^^^ Arrays290
The distribution \xcd`Dist.makeUnique(Q)` is the unique distribution from the
region \xcd`Region.make(1..k)` to \xcd`Q` mapping each point \xcd`i` to
\xcd`pi`.


\paragraph{Constant distributions.} \index{distribution!constant}
%~~exp~~`~~`~~R:Region, P:Place ~~ import x10.regionarray.*; ^^^ Arrays300
The distribution \xcd`Dist.makeConstant(R,P)` maps every point in region
\xcd`R` to place \xcd`P`.  
%~~exp~~`~~`~~R:Region ~~ import x10.regionarray.*; ^^^Arrays9n5n
The special case \xcd`Dist.makeConstant(R)` maps every point in \xcd`R` to
\xcd`here`. 

\paragraph{Block distributions.}\index{distribution!block}
%~~exp~~`~~`~~R:Region ~~ import x10.regionarray.*; ^^^ Arrays320
The distribution \xcd`Dist.makeBlock(R)` distributes the elements of \xcd`R`,
in approximately-even blocks, over all the places available to the program. 
There are other \xcd`Dist.makeBlock` methods capable of controlling the
distribution and the set of places used; see the API documentation.


\paragraph{Domain Restriction.} \index{distribution!restriction!region}

If \xcd`D` is a distribution and \xcd`R` is a sub-region of {\cf
%~~exp~~`~~`~~D:Dist,R :Region{R.rank==D.rank} ~~ import x10.regionarray.*; ^^^ Arrays330
D.region}, then \xcd`D | R` represents the restriction of \xcd`D` to
\xcd`R`---that is, the distribution that takes each point \xcd`p` in \xcd`R`
to 
%~~exp~~`~~`~~D:Dist, p:Point{p.rank==D.rank} ~~ import x10.regionarray.*; ^^^ Arrays340
\xcd`D(p)`, 
but doesn't apply to any points but those in \xcd`R`.

\paragraph{Range Restriction.}\index{distribution!restriction!range}

If \xcd`D` is a distribution and \xcd`P` a place expression, the term
%~~exp~~`~~`~~ D:Dist, P:Place~~ import x10.regionarray.*; ^^^ Arrays350
\xcd`D | P` 
denotes the sub-distribution of \xcd`D` defined over all the
points in the region of \xcd`D` mapped to \xcd`P`.

Note that \xcd`D | here` does not necessarily contain adjacent points
in \xcd`D.region`. For instance, if \xcd`D` is a cyclic distribution,
\xcd`D | here` will typically contain points that differ by the number of
places. 
An implementation may find a
way to still represent them in contiguous memory, \eg, using an arithmetic
function to map from the region index to an index 
into the array.


\subsection{Distributed Arrays}
\index{array!distributed}
\index{distributed array}
\index{\Xcd{DistArray}}
\index{DistArray}

Distributed arrays, instances of \xcd`DistArray[T]`, are very much like
\xcd`Array`s, except that they distribute information among multiple
\xcd`Place`s according to a \xcd`Dist` value passed in as a constructor
argument.  

\begin{ex}The following code creates a distributed array holding
a thousand cells, each initialized to 0.0, distributed via a block
distribution over all places.
%~~gen ^^^ Arrays360
% package Arrays.Distarrays.basic.example;
% import x10.regionarray.*;
% class Example {
% def example() {
%~~vis
\begin{xten}
val R <: Region = Region.make(1..1000);
val D <: Dist = Dist.makeBlock(R);
val da <: DistArray[Float] 
       = DistArray.make[Float](D, (Point(1))=>0.0f);
\end{xten}
%~~siv
%}}
%~~neg
\end{ex}



\subsection{Distributed Array Construction}\label{ArrayInitializer}
\index{distributed array!creation}
\index{\Xcd{DistArray}!creation}
\index{DistArray!creation}

\xcd`DistArray`s are instantiated by invoking one of the \xcd`make` factory
methods of the \xcd`DistArray` class.
A \xcd`DistArray` creation 
must take either an \xcd`Long` as an argument or a \xcd`Dist`. In the first
case,  a distributed array is created over the distribution 
%~~exp~~`~~`~~N:Long ~~ import x10.regionarray.*; ^^^Arrays1s6g
\xcd`Dist.makeConstant(Region.make(0, N-1),here)`;
in the second over the given distribution. 

\begin{ex}A distributed array creation operation may also specify an initializer
function.
The function is applied in parallel
at all points in the domain of the distribution. The
construction operation terminates locally only when the \xcd`DistArray` has been
fully created and initialized (at all places in the range of the
distribution).

For instance:
%~~gen ^^^ Arrays370
% package Arrays.DistArray.Construction.FeralWolf;
% import x10.regionarray.*;
% class Example {
% def example() {
%~~vis
\begin{xten}
val ident = ([i]:Point(1)) => i;
val data : DistArray[Long]
    = DistArray.make[Long](Dist.makeConstant(Region.make(1, 9)), ident);
val blk = Dist.makeBlock(Region.make(1..9, 1..9));
val data2 : DistArray[Long]
    = DistArray.make[Long](blk, ([i,j]:Point(2)) => i*j);
\end{xten}
%~~siv
% }  }
%~~neg

{}\noindent 
The first declaration stores in \xcd`data` a reference to a mutable
distributed array with \xcd`9` elements each of which is located in the
same place as the array. The element at \Xcd{[i]} is initialized to its index
\xcd`i`. 

The second declaration stores in \xcd`data2` a reference to a mutable
two-dimensional distributed array, whose coordinates both range from 1 to
9, distributed in blocks over all \xcd`Place`s, 
initialized with \xcd`i*j`
at point \xcd`[i,j]`.
\end{ex}

\subsection{Operations on Arrays and Distributed Arrays}

Arrays and distributed arrays share many operations.
In the following, let \xcd`a` be an array with base type T, and \xcd`da` be an
array with distribution \xcd`D` and base type \xcd`T`.

\subsubsection{Element operations}\index{array!access}
The value of \xcd`a` at a point \xcd`p` in its region of definition is
%~~exp~~`~~`~~a:Array[Long](3), p:Point(3) ~~ import x10.regionarray.*; ^^^ Arrays380
obtained by using the indexing operation \xcd`a(p)`. 
The value of \xcd`da` at \xcd`p` is similarly
%~~exp~~`~~`~~da:DistArray[Long](3), p:Point(3) ~~ import x10.regionarray.*; ^^^ Arrays390
\xcd`da(p)`.
This operation
may be used on the left hand side of an assignment operation to update
the value: 
%~~stmt~~`~~`~~a:Array[Long](3), p:Point(3), t:Long ~~ import x10.regionarray.*; ^^^ Arrays400
\xcd`a(p)=t;`
and 
%~~stmt~~`~~`~~da:DistArray[Long](3), p:Point(3), t:Long ~~ import x10.regionarray.*; ^^^ Arrays410
\xcd`da(p)=t;`
The operator assignments, \xcd`a(i) += e` and so on,  are also
available. 

It is a runtime error to 
access arrays, with \xcd`da(p)` or \xcd`da(p)=v`, at a place
other than \xcd`da.dist(p)`, \viz{} at the place that the element exists. 


\subsubsection{Arrays of Single Values}\label{ConstantArray}
\index{array!constant promotion}

For a region \xcd`R` and a value \xcd`v` of type \xcd`T`, the expression 
%~~genexp~~`~~`~~T~~R:Region{self!=null}, v:T ~~ import x10.regionarray.*; ^^^ Arrays420
\xcd`new Array[T](R, v)` 
produces an array on region \xcd`R` initialized with value \xcd`v`.
Similarly, 
for a distribution \xcd`D` and a value \xcd`v` of
type \xcd`T` the expression 
\begin{xtenmath}
DistArray.make[T](D, (Point(D.rank))=>v)
\end{xtenmath}
constructs a distributed array with
distribution \xcd`D` and base type \xcd`T` initialized with \xcd`v`
at every point.

Note that \xcd`Array`s are constructed by constructor calls, but
\xcd`DistArrays` are constructed by calls to the factory methods
\xcd`DistArray.make`. This is because \xcd`Array`s are fairly simple objects,
but \xcd`DistArray`s may be implemented by different classes for different
distributions. The use of the factory method gives the library writer the
freedom to select appropriate implementations.


\subsubsection{Restriction of an array}\index{array!restriction}

Let \xcd`R` be a sub-region of \xcd`da.region`. Then 
%~~exp~~`~~`~~da:DistArray[Long](3), p:Point(3), R: Region(da.rank) ~~ import x10.regionarray.*; ^^^ Arrays440
\xcd`da | R`
represents the sub-\xcd`DistArray` of \xcd`da` on the region \xcd`R`.
That is, \xcd`da | R` has the same values as \xcd`da` when subscripted by a
%~~exp~~`~~`~~R:Region, da: DistArray[Long]{da.region.rank == R.rank} ~~ import x10.regionarray.*; ^^^ Arrays450
point in region \xcd`R && da.region`, and is undefined elsewhere.

Recall that a rich set of operators are available on distributions
(\Sref{XtenDistributions}) to obtain sub-distributions
(e.g. restricting to a sub-region, to a specific place etc).


\subsubsection{Operations on Whole Arrays}

\paragraph{Pointwise operations}\label{ArrayPointwise}\index{array!pointwise operations}
The unary \xcd`map` operation applies a function to each element of
a distributed or non-distributed array, returning a new distributed array with
the same distribution, or a non-distributed array with the same region.

The following produces an array of cubes: 
%~~gen ^^^ Arrays460
%package Arrays_pointwise_a;
%import x10.regionarray.*;
%class Example{
%static def example() {
%~~vis
\begin{xten}
val A = new Array[Long](11, (i:long)=>i);
assert A(3) == 3 && A(4) == 4 && A(10) == 10; 
val cube = (i:Long) => i*i*i;
val B = A.map(cube);
assert B(3) == 27 && B(4) == 64 && B(10) == 1000; 
\end{xten}
%~~siv
%} } 
% class Hook{ def run() {Example.example(); return true;}}
%~~neg

A variant operation lets you specify the array \Xcd{B} into which the result
will be stored, 
%~~gen ^^^ Arrays470
%package Arrays.map_with_result;
%import x10.regionarray.*;
%class Example{
%static def example() {
%~~vis
\begin{xten}
val A = new Array[Long](11, (i:long)=>i);
assert A(3) == 3 && A(4) == 4 && A(10) == 10; 
val cube = (i:Long) => i*i*i;
val B = new Array[Long](A.region); // B = 0,0,0,0,0,0,0,0,0,0,0
A.map(B, cube);
assert B(3) == 27 && B(4) == 64 && B(10) == 1000; 
\end{xten}
%~~siv
%} } 
% class Hook{ def run() {Example.example(); return true;}}
%~~neg
\noindent
This is convenient if you have an already-allocated array lying around unused.
In particular, it can be used if you don't need \Xcd{A} afterwards and want to
reuse its space:
%~~gen ^^^ Arrays480
%package Arrays.map_reusing_space;
%import x10.regionarray.*;
%class Example{
%static def example() {
%~~vis
\begin{xten}
val A = new Array[Long](11, (i:long)=>i);
assert A(3) == 3 && A(4) == 4 && A(10) == 10; 
val cube = (i:Long) => i*i*i;
A.map(A, cube);
assert A(3) == 27 && A(4) == 64 && A(10) == 1000; 
\end{xten}
%~~siv
%} } 
% class Hook{ def run() {Example.example(); return true;}}
%~~neg


The binary \xcd`map` operation takes a binary function and
another
array over the same region or distributed array over the same  distribution,
and applies the function 
pointwise to corresponding elements of the two arrays, returning
a new array or distributed array of the same shape.
The following code adds two distributed arrays: 
%~~gen ^^^ Arrays490
% package Arrays.Pointwise.Pointless.Map2;
% import x10.regionarray.*;
% class Example{
%~~vis
\begin{xten}
static def add(da:DistArray[Long], db: DistArray[Long])
    {da.dist==db.dist}
    = da.map(db, (a:Long,b:Long)=>a+b);
\end{xten}
%~~siv
%}
%~~neg



\paragraph{Reductions}\label{ArrayReductions}\index{array!reductions}

Let \xcd`f` be a function of type \xcd`(T,T)=>T`.  Let
\xcd`a` be an array over base type \xcd`T`.
Let \xcd`unit` be a value of type \xcd`T`.
Then the
%~~genexp~~`~~`~~ T ~~ f:(T,T)=>T, a : Array[T], unit:T ~~ import x10.regionarray.*; ^^^ Arrays500
operation \xcd`a.reduce(f, unit)` returns a value of type \xcd`T` obtained
by combining all the elements of \xcd`a` by use of  \xcd`f` in some unspecified order
(perhaps in parallel).   
The following code gives one method which 
meets the definition of \Xcd{reduce},
having a running total \Xcd{r}, and accumulating each value \xcd`a(p)` into it
using \Xcd{f} in turn.  (This code is simply given as an example; \Xcd{Array}
has this operation defined already.)
%~~gen ^^^ Arrays510
%package Arrays.Reductions.And.Eliminations;
%import x10.regionarray.*;
% class Example {
%~~vis
\begin{xten}
def oneWayToReduce[T](a:Array[T], f:(T,T)=>T, unit:T):T {
  var r : T = unit;
  for(p in a.region) r = f(r, a(p));
  return r;
}
\end{xten}
%~~siv
%}
%~~neg


For example,  the following sums an array of longs.  \Xcd{f} is addition,
and \Xcd{unit} is zero.  
%~~gen ^^^ Arrays520
% package Arrays.Reductions.And.Emulsions;
%import x10.regionarray.*;
% class Example {
% static def example() {
%~~vis
\begin{xten}
val a = new Array[Long](4, (i:long)=>i+1);
val sum = a.reduce((a:Long,b:Long)=>a+b, 0); 
assert(sum == 10); // 10 == 1+2+3+4
\end{xten}
%~~siv
%}}
% class Hook{ def run() {Example.example(); return true;}}
%~~neg

Other orders of evaluation, degrees of parallelism, and applications of
\Xcd{f(x,unit)} and \xcd`f(unit,x)`are also correct.
In order to guarantee that the result is precisely
determined, the  function \xcd`f` should be associative and
commutative, and the value \xcd`unit` should satisfy
\xcd`f(unit,x)` \xcd`==` \xcd`x` \xcd`==` \xcd`f(x,unit)`
for all \Xcd{x:T}.  




\xcd`DistArray`s have the same operation.
This operation involves communication between the places over which
the \xcd`DistArray` is distributed. The \Xten{} implementation guarantees that
only one value of type \xcd`T` is communicated from a place as part of
this reduction process.

\paragraph{Scans}\label{ArrayScans}\index{array!scans}


Let \xcd`f:(T,T)=>T`, \xcd`unit:T`, and \xcd`a` be an \xcd`Array[T]` or
\xcd`DistArray[T]`.  Then \xcd`a.scan(f,unit)` is the array or distributed
array of type \xcd`T` whose {$i$}th element in canonical order is the
reduction by \xcd`f` with unit \xcd`unit` of the first {$i$} elements of
\xcd`a`. 


This operation involves communication between the places over which the
distributed array is distributed. The \Xten{} implementation will endeavour to
minimize the communication between places to implement this operation.

Other operations on arrays, distributed arrays, and the related classes may be
found in the \xcd`x10.regionarray` package.
