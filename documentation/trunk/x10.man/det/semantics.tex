Featherweight X10 (FX10) is a formal calculus for X10 intended to  complement Featherweight Java
(FJ).  It models imperative aspects of X10 including the concurrency
constructs \hfinish{} and \hasync{}.


\paragraph{Overview of formalism}
\Subsection{Syntax}

\begin{figure}[htpb!]
\begin{center}
\begin{tabular}{|l|l|}
\hline

$\hP ::= \ol{\hL},\hS$ & Program. \\

$\hL ::= \hclass ~ \hC~\hextends~\hD~\lb~\ol{\hF};~\ol{\hM}~\rb$
& cLass declaration. \\

$\hF ::= \hvar\,\hf:\hC$
& Field declaration. \\

$\hM ::= \hG\ \hdef\ \hm(\ol{\hx}:\ol{\hC}):\hC\{\hS\}$
& Method declaration. \\

$\hG ::= \code{@NonEscaping} ~~|~~ \code{@NoThisAccess}$
& Method modifier. \\

$\hp ::= \hl ~~|~~ \hx$
& Path. \\ %(location or parameter)

$\he ::=  \hp.\hf  ~|~ \hnew{\hC}$
& Expressions. \\ %: locations, parameters, field access\&assignment,  %invocation, \code{new}

$\hS ::=  \hp.\hf = \hp; ~|~ \hp.\hm(\ol{\hp});  ~|~ \hval{\hx}{\he}{\hS}$ &\\
$~~~~|~ \finish{\hS}~|~ \async{\hS} ~|~ \hS~\hS$
& Statements. \\ %: locations, parameters, field access\&assignment, invocation, \code{new}

\hline
\end{tabular}
\end{center}
\caption{FX10 Syntax.
    The terminals are locations (\hl), parameters and \hthis (\hx), field name (\hf), method name (\hm), class name (\hB,\hC,\hD,\hObject),
        and keywords (\hhnew, \hfinish, \hasync, \code{val}).
    The program source code cannot contain locations (\hl), because locations are only created during execution/reduction in \RULE{R-New} of \Ref{Figure}{reduction}.
    }
\label{Figure:syntax}
\end{figure}

\Ref{Figure}{syntax} shows the syntax of FX10.
%(\Ref{Section}{val} will later add the \hval and \hvar field modifiers.)
Expression~$\hval{\hx}{\he}{\hS}$ evaluates $\he$, assigns it to a
new variable $\hx$, and then evaluates \hS. The scope of \hx{} is \hS.

The syntax is similar to the real X10 syntax with the following difference:
%Non-escaping methods are marked with \code{@NonEscaping}, such methods
%can be invoked on raw objects (and can be used to initialize them).
FX10 does not have constructors; instead, an object is initialized by assigning to its fields or
    by calling
    non-escaping methods.

\Subsection{Reduction}
A {\em heap}~$H$ is a mapping from a given set of locations to {\em
  objects}. An object is a pair $C(F)$ where $C$ is a class (the exact
class of the object), and $F$ is a partial map from the fields of $C$
to locations.
We say the object~\hl is {\em total/cooked} (written~$\cooked_H(\hl)$)
    if its map is total, i.e.,~$H(\hl)=\hC(F) \gap \dom(F)=\fields(\hC)$.

We say that a heap~$H$ {\em satisfies} $\phi$ (written~$H \vdash \phi$)
    if the plus assertions in~$\phi$ (ignoring the minus assertions) are true in~$H$,
    i.e., if~$\phi \vdash +\hl$ then~$\hl$ is cooked in~$H$
    and if~$\phi \vdash +\hl.\hf$ then~$H(\hl)=\hC(F)$ and~$F(\hf)$ is cooked in~$H$.


%An {\em annotation} $N$ for a heap $H$ maps each $l \in \dom(H)$ to a
%possibly empty set of fields $a(H(l))$ of the class of $H(l)$ disjoint
%from $\dom(H(l))$. (These are the fields currently being
%asynchronously initialized.) The logic of initialization described
%above is clearly sound for the obvious interpretation of formulas over
%annotated heaps. For future reference, we say that that a heap $H$
%{\em satisfies} $\phi$ if there is some annotation $N$ (and some
%valuation $v$ assigning locations in $\dom(H)$ to free variables of
%$\phi$) such that $\phi$ evaluates to true.

The reduction relation is described in
Figure~\ref{Figure:reduction}. An S-configuration is of the form
$\hS,H$ where \hS{} is a statement and $H$ is a heap (representing a
computation which is to execute $\hS$ in the heap $H$), or $H$
(representing terminated computation). An
E-configuration is of the form $\he,H$ and represents the
computation which is to evaluate $\he$ in the configuration $H$. The
set of {\em values} is the set of locations; hence E-configurations of
the form $\hl,H$ are terminal.

Two transition relations $\reduce{}$ are defined, one over S-configurations and
the other over E-configurations.
For $X$ a partial function, we use the notation $X[v \mapsto
  e]$ to represent the partial function which is the same as $X$
except that it maps $v$ to $e$.
The rules defining these relations are
standard.
The only minor novelty is in how \hasync{} is defined. The
critical rule is the last rule in~\RULE{(R-Step)}  -- it specifies the
``asynchronous'' nature of \hasync{} by permitting \hS{} to make a step
even if it is preceded by $\async{\hS_1}$.
%
The rule~\RULE{(R-New)} returns a new location that is bound to a new
object that is an instance of \hC{} with none of its fields initialized.
%
The rule~\RULE{(R-Access)} ensures that the field is initialized before it is
read ($\hf_i$ is contained in $\ol{\hf}$).

\begin{figure*}[t]
\begin{center}
\begin{tabular}{|c|}
\hline
$\typerule{
 \hS,H \reduce H'
}{
  \begin{array}{l}
    \finish{\hS},H \reduce H'\\
    \async{\hS},H \reduce H'\\
    \hS~\hS', H \reduce \hS',H'\\
  \end{array}
}$~\RULE{(R-Term)}
~
$\typerule{
 \hS,H \reduce \hS', H'
}{
  \begin{array}{l}
    \finish{\hS},H \reduce \finish{\hS'},H'\\
    \async{\hS},H \reduce \async{\hS'}, H'\\
    \hS~\hS_1, H \reduce \hS'~\hS_1,H'\\
    \async{\hS_1}~\hS, H \reduce \async{\hS_1}~\hS', H'\\
  \end{array}
}$~\RULE{(R-Step)}
~
$\typerule{
  \he,H \reduce \hl,H'
}{
  \hval{\hx}{\he}{\hS},H \reduce \hS[\hl/\hx], H'
}$~\RULE{(R-Val)}
\\\\

$\typerule{
    \hl' \not \in \dom(H)
}{
  \hnew{\hC},H \reduce \hl',H[ \hl' \mapsto \hC()]
}$~\RULE{(R-New)}
\quad
$\typerule{
    H(\hl')=\hC(\ldots)
        \gap
    \mbody{}(\hm,\hC)=\ol{\hx}.\hS
}{
  \hl'.\hm(\ol{\hl}),H \reduce \hS[\ol{\hl}/\ol{\hx},\hl'/\hthis],H
}$~\RULE{(R-Invoke)}
\quad

\\\\

$\typerule{
    H(\hl)=\hC(\ol{\hf}\mapsto\ol{\hl'})
}{
  \hl.\hf_i,H \reduce \hl_i',H
}$~\RULE{(R-Access)}
\quad
$\typerule{
    H(\hl)=\hC(F) \gap \cooked_H(\hl')
}{
  \hl.\hf=\hl',H \reduce H[ \hl \mapsto \hC(F[ \hf \mapsto \hl'])]
}$~\RULE{(R-Assign)}
\\
\hline
\end{tabular}
\end{center}
\caption{FX10 Reduction Rules ($\hS,H \reducesto \hS',H' ~|~H'$ and~$\he,H \reduce \hl,H'$).}
\label{Figure:reduction}
\end{figure*}


\Subsection{Results}

We say a heap $H$ is {\em correctly cooked} (written~$\vdash H$)
    if a field can point only to cooked objects, i.e.,~for every object $o=C(F)$ in the range of $H$ and
    every field $f \in \dom(F)$ it is the case that every object~$\hl=H(F(f))$ is cooked ($\cooked_H(\hl)$).
We shall only consider correctly cooked heaps (valid
programs will only produce correctly cooked heaps).
As the program is executed, the heap monotonically becomes more and more cooked.
Formally,~$H'$ is \emph{more cooked} than~$H$ (written~$H' \vdash H$)
    if for every~$\hl \in \dom(H)$, we have~$H(\hl)=\hC(F)$,~$H'(\hl)=\hC(F')$, and~$\dom(F)\subseteq\dom(F')$.

A {\em heap typing} $\Gamma$ is a mapping from locations to classes. $H$ is
said to be typed by $\Gamma$ if for each $l\in\dom(H)$, the class of $H(l)$
is a subclass of $\Gamma(l)$.  Since our treatment separates out effects
from types, and the treatment of types is standard, we shall assume
that all programs and heaps are typed.

A statement~$\hS$ is \emph{closed} (written~$\vdash \hS$) if it does not contain any free variables.
We say that $\hS$ is \emph{annotable} if there exists $\phi,\psi$ such that
$\phi~\hS~\psi$ can be established.\footnote{An example of a statement that is
{\em not} annotable is $\hval{\hx}{\hnew
  \hC}{\hval{\hy}{\hx.\hf}{\hz.\hg=\hy}}$ where $\hC{}$ has a field
\hf. This attempts to read a field of a variable initialized with a
brand-new object.}

We say that a program $\hP=\ol{\hL}\hS$ is {\em proper} if it is
well-typed and each method in \hL{} can be decorated with pre-post assertions
$(\phi,\psi)$, and \hS is annotable.
The decorations must satisfy the property that under the
assumption that every method satisfies its assertion (this is for use
in recursive calls) we can establish for every method
$\hdef~\hm(\ol{\hx}:\ol{\hC})\{\hS\}$ with assertion $(\phi,\psi)$ that
it is the case that the free variables of $\phi,\psi$ are contained in
$\this,\ol{\hx}$, and that $\phi~\hS~\psi$.

We prove the following theorems. In all these theorems the background
program \hP{} is assumed to be proper. The first theorem is analogous
to subject-reduction for typing systems.
\begin{Theorem}{\textbf{Preservation}}
%\label{Theorem:Preservation}
Let $\phi~\hS~\psi$, $\vdash \hS$, $\vdash H$, $H \vdash \phi$.
(a) If $\hS,H \reducesto H'$ then $\vdash H'$, $H' \vdash H$, $H' \vdash +\psi$.
(b) If $\hS,H \reducesto \hS',H'$ then $\vdash \hS'$, $\vdash H'$, $H' \vdash H$,
there exists $\phi',\psi'$ such
that $H' \vdash \phi'$, $\phi'~\hS'~\psi'$, $\phi' \vdash \phi$, $\psi' \vdash \psi$.
\end{Theorem}

\begin{Theorem}{\textbf{Progress}}
%\label{Theorem:Progress}
Let $\phi~\hS~\psi$, $\vdash \hS$, $\vdash H$, $H \vdash \phi$.
The configuration $\hS,H$ is not stuck.
\end{Theorem}

For proofs, please see associated technical report.

Because our reduction rules only allow reads from initialized fields,
a corollary is that a field can only be read after it was assigned,
and an attempt to read a field will always succeed.
