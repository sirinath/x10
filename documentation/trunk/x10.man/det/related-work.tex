\label{sec:related}

\subsection{Programming Models}
Determinism in parallel programming is a very active area of research.
Guava \cite{guava} introduces restrictions on shared memory Java
programs that ensure no data-races primarily by distinguishing
monitors (all access is synchronized) from values (immutable) and
objects (private to a thread). However Guava is not safe since Guava
programs may use \code{wait}/\code{notify} for arbitrary concurrent
signalling and hence may not executable with a sequential
schedule. The Revisions programming model \cite{Revisions} guarantees
determinism by isolating asynchronous tasks but merging their writes
determinately. However, the model explicitly does not require that
a sequential schedule be valid (c.f. Figure~1 in \cite{Revisions}).

%\cite{Steele:1989:MAP:96709.96731} introduced the idea that

DPJ develops the ``determinacy-by-default'' slogan using a static
type-and-effects system to establish commutativity of concurrent
actions.  The deterministic fragment of DPJ is safe according to the
definition above. Safe X10 offers a much richer concurrency model
which guarantees the safety of common idioms such as accumulators and
cyclic tasks (clocks) without relying on effects annotations. The
lightweight effects mechanism in X10 can be extended to support a much
richer effects framework (along the lines of DPJ) using X10's
constrained type system.  We leave this as future work.

The SafeJava language \cite{Rinard04safejava:a} is unfortunately not safe
according to our definition, even though it guarantees determinacy and
deadlock freedom, using ownership types, unique pointers and partially
ordered lock levels. Again, a sequential scheduler is not admissible
for the model.

Some data-flow synchronization based languages and frameworks (e.g.{}
Kahn style process networks \cite{kahn1974semantics}, concurrent
constraint programming \cite{saraswat1993concurrent}, \cite{SHIM}) are guaranteed
determinate but not safe according to our definition since they do not
permit sequential schedules. Indeed they permit the possibility of
deadlock. (The notion of safety is also not quite relevant since these
frameworks do not support shared mutable variables.)

Synchronous programming languages like Esterel are completely
deterministic, but meant for embedded targets. An Esterel program
executes in clock steps and the outputs are conceptually synchronous
with its inputs.  It is a finite state language that is easy to verify
formally. An Esterel program is susceptible to
causalities. Causalities are similar to deadlocks, but can be easily
detected at compile-time.  The problem with synchronous models is that
they do not perform well. To out knowledge, most Esterel compilers
generate sequential code and not concurrent.

SHIM~\cite{edwards2005shim2,tardieu2006scheduling-independent} is also
a deterministic concurrent programming language, but SHIM programs may
deadlocks and allows only a single task to write at any
phase. StreamIt~\cite{thies2001streamit} example is a synchronous
dataflow language that provides determinism. It has simple static
verification techniques for deadlock and buffer-overflow.  However,
StreamIt is a strict subset of SHIM and StreamIt's design limits it to
a small class of streaming applications.

In contrast,
Cilk~\cite{blumofe1995cilk} is a non-deterministic language that it covers a larger
class of applications. It is C based
and the programmer must explicitly ask for parallelism using
the \emph{spawn} and the \emph{sync} constructs.
Cilk permits data races.
%\figref{non-det}, for example,
%is a non-deterministic concurrent program in Cilk.
Explicit techniques~\cite{cheng1998detecting} are
required for checking data races in Cilk programs.



\subsection{Determinizing Tools}
Determinizing run-times support coarse-grained fork-join concurrency
by maintaining a different copy of memory for each activity and
merging them determinately at finish points (\cite{grace},
\cite{dmp}, \cite{kendo},\cite{determinator}). Safe
X10 can run on such systems in principle, but does not require them.
To execute Safe X10, such systems need to support fine-grained
asynchrony (with some form of work-stealing or fork-joining
scheduler), clocks and accumulators.

Kendo is a purely software system that deterministically multi-threads
concurrent applications.  Kendo~\cite{olszewski2009kendo} ensures a
deterministic order of all lock acquisitions for a given program
input. Kendo comes with three shortcomings. It operates completely at runtime,
and there is considerable performance penalty. Secondly, if
we have the sequence \emph{lock(A); lock (B)} in one thread and
\emph{lock(B); lock(A)} in another thread, a deterministic ordering of
locks may still deadlock. Thirdly, the tool operates only when
shared data is protected by locks.

Software Transactional Memory (STM)~\cite{shavit1995software}
  is an alternative to locks: a thread completes modifications to
shared memory without regard for what other threads might be doing. At the end of the transaction,
it validates and commits if the validation was successful, otherwise it rolls back and re-executes
the transaction. STM mechanisms avoid races but do not solve the non-determinism problem.

Berger's Grace\cite{berger2009grace} is a run-time tool
that is based on STM.
If there is a conflict during commit, the threads are committed in
a particular sequential order.
The problem with Grace is that it incurs a lot of run-time
overhead.
%This paper partially solves this overhead problem
%by addressing the issue at compile-time and
%thereby reducing a considerable amount of run-time overhead.
%
Determinator\cite{aviram2010efficient} is another tool
that allows parallel processes to execute as long as they do not share
resources. If they do share resources and the accesses are unsafe, then
the operating throws an exception (a page fault).
%
Cored-Det~\cite{bergan2010coreDet}, based on DMP~\cite{devietti2009dmp}
uses a deterministic token that is passed
among all threads.  A thread to modify a shared variable must first
wait for the token and for all threads to block on that
token. DMP is hardware based.
Although, deadlocks may be avoided, we believe this setting is
non-distributed because it forces all threads to synchronize and
therefore leads to a considerable performance penalty. In Safe X10,
only threads that share a particular channel must synchronize
on that channel; other threads can run independently.

Deterministic replay
systems~\cite{choi1998deterministic,altekar2009odr} facilitate
debugging of concurrent programs to produce repeatable behavior. They
are based on record/replay systems. The system replays a specific
behavior (such as thread interleaving) of a concurrent program based
on records. The primary purpose of replay systems is debugging; they
do not guarantee determinism.  They incur a high runtime overhead and
are input dependent.  For every new input, a new set of records is
generally maintained.

Like replay systems, Burmin and Sen~\cite{Burnim2009asserting} provide a framework for
checking determinism for multi-threaded programs. Their tool does not
introduce deadlocks, but their tool does not guarantee determinism
because it is merely a testing tool that checks the execution trace
with previously executed traces to see if the values match.
Our goal is to guarantee determinism at compile time -- given a program,
it will generate the same output for a given input.



%\subsection{Type Systems and Verifiers}
%Finally, type and effect systems like DPJ~\cite{bocchino2009type}
% have been designed for deterministic parallel programming to see if
%memory locations overlap. Our technique is more explicit.
%In general, type systems require the programmer to manually annotate the program. Our model can also be implemented using annotations in existing
%programming languages - we in fact annotated the X10 programming language.

%Martin Vechev's tool \cite{vechev2011automatic}
%finds determinacy bugs in loops that run parallel bodies. It analyzes
%array references and indices to ensure that there are no read-write and
