%% Includes treatment of clocked variables.
Featherweight X10 (FX10) is a formal calculus for X10 intended to  complement Featherweight Java
(FJ).  It models imperative aspects of X10 including the concurrency
constructs \hfinish{}, \hasync{}, clocks and accumulators.


\paragraph{Overview of formalism}
\Subsection{Syntax}
There are three predefined classes in FX10:
    \code{Object},
    \code{Acc} (representing accumulators),
    and \code{ClockedAcc} (representing clocked accumulators).
X10 has an effect system that guarantees that
    \hasync{} only accesses immutable state, clocked values and accumulators.
The effect system also ensures that clocked values can only be assigned once every phase,
    however clocked accumulators can be assigned multiple times.
FX10 only models the runtime behavior without the effect system.
Therefore, in order to guarantee determinacy without an effect system,
    FX10 does not model field assignment nor clocked values (only clocked accumulators).
Objects are still mutable because accumulators and clocked accumulators
    are mutable.



\Ref{Figure}{syntax} shows the abstract syntax of FX10.
A constructor in FX10 assigns its arguments to the object's fields.

$\epsilon$ is the empty statement. Sequencing associates to the left.
%\code{async}
%and \code{finish} bind tighter than statement sequencing, so
%$\code{async}~\hS~\hS'$ is the sequencing of $\code{async}~\hS$ and
%$\hS'$.
The block syntax in source code is just \code{\{\hS\}},
however at run-time the block is augmented with (the initially empty) set of locations
$\pi$ representing the objects registered on the block,
     which are those that were created in that block.

%(\Ref{Section}{val} will later add the \hval and \hvar field modifiers.)
Statement~$\hval{\hx}{\he}{\hS}$ evaluates $\he$, assigns it to a
new variable $\hx$, and then evaluates \hS. The scope of \hx{} is \hS.

Statement $\hp_1 \leftarrow \hp_2$ is legal only when $\hp_1$ is an accumulator,
    and it accumulates $\hp_2$ into $\hp_1$ (given the reduction operator that was supplied when the accumulator was created).
Statement $\hp_1() = \hp_2$ is legal only when $\hp_1$ is a clocked value,
    and it writes $\hp_2$ into the next value of $\hp_1$.
Expression $\hp()$ is legal only when $\hp$ is an accumulator or a clocked value.
%    for a clocked-value it checks that $\hp$ is in $\pi$,
%    and for an accumulator it checks at runtime that it was executed in the block that created $\hp$
%    and it waits until all activities created in the block have terminated or got stuck.


\Subsection{Reduction}
A {\em heap}~$H$ is a mapping from a given set of locations to {\em
  objects}. An object is a pair $\hC(\ol{\hl})$ where $\hC$ is a class (the exact
class of the object), and $\ol{\hl}$ are the values in the fields.

The reduction relation is described in
Figure~\ref{Figure:reduction}. An {\em S-configuration} is of the form
$\hS,H$ where \hS{} is a statement and $H$ is a heap (representing a
computation which is to execute $\hS$ in the heap $H$), or $H$
(representing a successfully terminated computation), or $\err$
representing a computation that terminated in an error. An
E-configuration is of the form $\he,H$ and represents the
computation which is to evaluate $\he$ in the configuration $H$. The
set of {\em values} is the set of locations; hence E-configurations of
the form $\hl,H$ are terminal.

Two transition relations $\preduce{}$ are defined, one over
S-configurations and the other over E-configurations. Here $\pi$ is a
set of locations which can currently be asynchronously accessed.  Thus
each transition is performed in a context that knows about the current
set of capabilities.  For $X$ a partial function, we use the notation
$X[v \mapsto e]$ to represent the partial function which is the same
as $X$ except that it maps $v$ to $e$.

The rules for termination, step, val, new, invoke, and access
are standard.  The only minor novelty is in how \hasync{} is
defined. The critical rule is the last rule in~\RULE{(R-Step)} -- it
specifies the ``asynchronous'' nature of \hasync{} by permitting \hS{}
to make a step even if it is preceded by $\async{\hS_1}$. Further,
each block records its set of synchronous capabilities. When
descending into the body, the block's own capabiliites are added to
those obtained from the environment.

%
The rule~\RULE{(R-New)} returns a new location that is bound to a new
object that is an instance of \hC{} with its fields set to the argument;
    the new object is registered with the block.
(It is enough to register only clocked-values and accumulators, but
    for simplicity we register all newly created objects.)
Given a set of registered objects $\ol{\hl}$ in a heap $H$,
    we define
    $\hAcc(\ol{\hl})$ to return only the accumulators in~$\ol{\hl}$.
 %   and (ii)~$\fclocked(\ol{\hl})$ only the clocked-values.


%todo: give an example of registration and legality.

%
The rule~\RULE{(R-Access)} ensures that the field is initialized before it is
read ($\hf_i$ is contained in $\ol{\hf}$).
%
The rule~\RULE{(R-Acc-A-R)} permits the accumulator to be read in a
block provided that the current set of capabilities permit it (if not,
an error is thrown), and provided that the only statements prior to
the read are re is no nested async prior to the read are clocked
asyncs that are stuck at an \hadvance.

The rule~\RULE{(R-Acc-W)} updates the current contents of the
accumulator provided that the current set of capabilities permit
asynchronous access to the accumulator (if not, an erorr is thrown).

The rule~\RULE{(R-Advance)} permits a \code{clocked finish} to advance
only if all the top-level \code{clocked asyncs} in the scope of the
\code{clocked finish} and before an \code{advance} are stuck at an
\code{advance}. A \code{clocked finish} can also advance if all the
the top-level \code{clocked asyncs} in its scope are stuck. In this
case, the statement in the body of the \code{clocked finish} has
terminated, and left behind only possibly clocked \code{async}. This
rule corresponds to the notion that the body of a \code{clocked
  finish} deregisters itself from the clock on local termination.
Note that in both these rules un-clocked \code{aync} may exist in the
scope of the \code{clocked finish}; they do not come in the way of the
\code{clocked finish} advancing.

%\begin{figure}[htpb!]
\begin{figure*}[t]
\begin{center}
\begin{tabular}{|l|l|}
\hline

$\hP ::= \ol{\hL},\hS$ & Program. \\

$\hL ::= \hclass ~ \hC~\hextends~\hD~\lb~\ol{\hF};~\ol{\hM}~\rb$
& cLass declaration. \\

$\hF ::= \hvar\,\hf:\hC$
& Field declaration. \\

$\hM ::= \hdef\ \hm(\ol{\hx}:\ol{\hC}):\hC\{\hS\}$
& Method declaration. \\

$\hp ::= \hl ~~|~~ \hx$
& Path. \\ %(location or parameter)

$\he ::=  \hp.\hf  ~|~ \hnew{\hC}(\ol{\hp}) ~|~ \hp()$
& Expressions. \\ %: locations, parameters, field access,  %invocation, \code{new}
$\hB ::= \blockt{\pi}{\hS}$
& Blocks. \\
$\hS ::=  \epsilon ~|~ \hadvance; ~|~ \hp.\hm(\ol{\hp});  ~|~ \valt{\hx}{\he}$ &\\
$~~~~|~ \hp \leftarrow \hp$ &\\
$~~~~|~ \hp() = \hp$ &\\
$~~~~|~ \pclocked~\finishasync{\hB}$&\\
$~~~~|~ \hS~\hS$
& Statements. \\ %: locations, parameters, field access, invocation, \code{new}

\hline
\end{tabular}
\end{center}
FX10 Syntax.
    The terminals are locations (\hl), parameters and \hthis (\hx), field name (\hf), method name (\hm), class name (\hB,\hC,\hD,\hObject),
        and keywords (\hhnew, \hfinish, \hasync, \code{val}).
    The program source code cannot contain locations (\hl), because locations are only created during execution/reduction in \RULE{R-New} of \Ref{Figure}{reduction}.

\label{Figure:syntax}
%\end{figure}

%\begin{figure*}[t]

\begin{center}
\begin{tabular}{|c|}
\hline
\\\\
$\typerule{
}{
 \epsilon,\cH \preduce \cH
}$~\RULE{(R-Epsilon)}
~
$\typerule{
 \hS,\cH \preduce \hS',\cH' ~|~ \cH' ~|~ \err
}{
  \begin{array}{rcl}
    \blockt{\ol{\hl}}{\hS},\cH&\ptreduce &\blockt{\ol{\hl}}{\hS'},\cH'
    ~|~ \cH' ~|~ \err \gap \pi=\pi',\ol{\hl} \\
    \hS~\hS_1, \cH &\preduce& \hS'~\hS_1,\cH' ~|~ \hS_1,\cH' ~|~ \err\\
  \end{array}
}$~\RULE{(R-Trans)}
\\\\
$\typerule{
 \hB,\cH \preduce \hB',\cH' ~|~ \cH' ~|~ \err
}{
    \async{\hB},\cH \ptreduce  \async{\hB'},\cH'  ~|~ \cH' ~|~ \err \gap \pi=\hAcc(\pi')\\
    \finish{}{\hB},\cH \ptreduce \finish{}{\hB'},\cH'  ~|~ \cH' ~|~ \err \gap \pi=\hAcc(\pi')\\
    \hclocked~\async{\hB},\cH \preduce \hclocked~\async{\hB'},\cH'  ~|~ \cH' ~|~ \err \\
   \blockt{\ol{\hl}}{\hS_1~\hclocked~\finish{\hB} \hS},\cH \ptreduce
  \blockt{\ol{\hl}}{\hS_1~\hclocked~\finish{\hB'} \hS},\cH' ~|~ \blockt{\ol{\hl}}{\hS_1 \hS},\cH' ~|~ \err \gap \pi=\hAcc(\pi') \cup \ol{\hl} \gap \hS_1\stuck
}$~\RULE{(R-Trans-B)}
\\\\
$\typerule{
  \hS,\cH \preduce \hS', \cH'
}{
  \pclocked~\async{\hB}~\hS, \cH \preduce \pclocked~\async{\hB}~\hS', \cH'\\
}$~\RULE{(R-Async)}
\\\\
$\typerule{
  \he,\cH \reduce \hl,\cH'
}{
  \hval{\hx}{\he}{\hS},\cH \preduce \hS[\hl/\hx], \cH'\\
 % \valt{\hx}{\he},\cH \preduce \cH'
}$~\RULE{(R-Val)}
~
$\typerule{
    \cH(\hl')=\hC(\ldots)
        \gap
    \mbody{}(\hm,\hC)=\ol{\hx}.\hS
}{
  \hl'.\hm(\ol{\hl}),\cH \preduce \hS[\ol{\hl}/\ol{\hx},\hl'/\hthis],\cH
}$~\RULE{(R-Invoke)}
\\\\
$\typerule{
    \cH(\hl)=\hC(\ol{\hf}\mapsto\ol{\hl'})\gap\gap
\hg=\hf_i ~|~\hg\not\in\ol{\hf}
}{
  \hl.\hg,\cH \preduce \hl_i',\cH ~|~ \err
}$~\RULE{(R-Access)}
\\\\
$\typerule{
    \cH(\hl)=\hClockedAcc(\hr,\ho,\he) \gap \hl\in \pi ~|~ \hl\not\in\pi
}{
  \hl()=\hl',\cH \preduce \cH[ \hl \mapsto \hClockedAcc(\hr,\ho,\hr(\he,\hl'))] ~|~ \err
}$~\RULE{(R-Clocked-A)}
\quad
$\typerule{
  \cH(\hl)=\hClockedAcc(\hr,\ho,\he) \gap \hl\in\pi ~|~ \hl\not\in\pi
}{
  \hval{\hx}{\hl()}{\hS},\cH \preduce \hS[\ho/\hx], \cH ~|~ \err\\
}$~\RULE{(R-Clocked-R)}
\\\\
$\typerule{}{
    \epsilon \stuck
}$~\RULE{(R-Q-CA)}
~
$\typerule{\hB\stuck}{
    \hclocked~\async~\hB\stuck
}$~\RULE{(R-Q-CA)}
~
$\typerule{\hS\stuck}{
   \blockt{\ol{\hl}}{\hS}\stuck\\
   \blockt{\ol{\hl}}{\hS~\xadvance~\hS'}\stuck\\
}$~\RULE{(R-Q-B)}
~
$\typerule{
  \hS_1 \stuck \gap \hS_2 \stuck
}{
   \hS_1~\hS_2 \stuck
}$~\RULE{(R-Q-S)}
\\\\
$\typerule{
\hS_1\stuck \gap\cH(\ha)=\hAcc(\hr,\hv) \gap  \gap \ha\in \ol{\hl} ~|~
\ha\not\in \ol{\hl}
}{
\blockt{\ol{\hl}}{\hS_1~\hval{\hx}{\ha()}~\hS},\cH \preduce
\blockt{\ol{\hl}}{\hS_1~\hS[\hv/\hx]}, \cH' ~|~ \err
}$~\RULE{(R-Acc-R)}
~
$\typerule{
  \cH(\ha)=\hAcc(\hr,\hv)\gap \hw=\hr(\hv,\hp)  \gap\gap \ha\in \pi~|~ \ha\not\in\pi
}{
  \ha \leftarrow \hp,\cH \preduce \cH[\ha \mapsto \hAcc(\hr,\hw)] ~|~ \err
}$~\RULE{(R-Acc-W)}
\\\\
$\typerule{
    \hS_1\stuck\gap\hl' \not \in \dom(\cH)
}{
\blockt{\ol{\hm}}{\hS_1~\valt{\hx}{\hnew{\hC(\ol{\hl})}}~\hS},\cH \preduce
   \blockt{\ol{\hm},\hl}{\hS_1~\hS[\hl'/\hx]},\cH[ \hl' \mapsto \hC(\ol{\hl})] \\
%\blockt{\pi_1}{\acc{\hx}{\hnew{\hAcc(\hr,\hz)}}},\cH \preduce
%   \cH[ \hl \mapsto \hAcc(\hr,\hz)]
}$~\RULE{(R-New)}
~
$\typerule{
\hS_1\stuck \gap\gap \hS,\cH \preduce \hS',\cH'
}{
   \hS_1~\hS,\cH \preduce \hS_1~\hS',\cH'
}$~\RULE{(R-Q)}
\\\\
$\typerule{
  \hB \ureduce \hB'
}{
    \hclocked~\async{\hB} \ureduce \hclocked~\async{\hB'} \\
}$~\RULE{(R-Adv-S)}
~
$\typerule{}{
  \hclocked~\finishasync{\hB} \ureduce  \\
\quad\quad\quad \hclocked~\finishasync{\hB}\\
    \xadvance~\hS \ureduce \hS\\
}$~\RULE{(R-Adv-A,CAF)}
\\\\

$\typerule{
  \hS \ureduce \hS'
}{
   \blockt{\ol{\hl}}{\hS}\ureduce \blockt{\ol{\hl}}{\hS'}
}$~\RULE{(R-Adv-B)}
~
$\typerule{
    \hS_1 \ureduce \hS_1'\ \     \hS_2 \ureduce \hS_2'
}{
  \hS_1~\hS_2\ureduce \hS_1'~\hS_2'
}$~\RULE{(R-Adv-S)}
~
$\typerule{
    \hS_1 \stuck \gap \hB \ureduce \hB' \gap H'=H[\ol{\hl}\mapsto \hswitch~H[\ol{\hl}]]
}{
  \begin{array}{l}
  \blockt{\ol{\hl}}{\hS_1~\hclocked~\finish{\hB} \hS},\cH\preduce \\
\quad\quad\quad \blockt{\ol{\hl}}{\hS_1~\hclocked~\finish{\hB'} \hS},\cH'
  \end{array}
}$~\RULE{(R-Adv)}
\\\\
\hline
\end{tabular}
\end{center}


\caption{FX10 Syntax and Reduction Rules ($\hS,\cH \preduce \hS',\cH' ~|~\cH'$ and~$\he,\cH \preduce \hl,\cH'$).}
\label{Figure:reduction}
\end{figure*}


\Subsection{Results}
