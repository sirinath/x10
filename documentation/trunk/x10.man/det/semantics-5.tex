%% Includes treatment of clocked variables.
Featherweight X10 (FX10) is a formal calculus for X10 intended to  complement Featherweight Java
(FJ).  It models imperative aspects of X10 including the concurrency
constructs \hfinish{} and \hasync{}.


\paragraph{Overview of formalism}
\Subsection{Syntax}


\Ref{Figure}{syntax} shows the abstract syntax of FX10. 
$\epsilon$ is the empty statement. Sequencing associates to the left.
%\code{async}
%and \code{finish} bind tighter than statement sequencing, so 
%$\code{async}~\hS~\hS'$ is the sequencing of $\code{async}~\hS$ and
%$\hS'$. 
The block syntax in source code is just \code{\{\hS\}},
however at run-time the block is augmented with (the initially empty) set of locations
$\pi$ representing the write capabilities of the block.

%(\Ref{Section}{val} will later add the \hval and \hvar field modifiers.)
Expression~$\hval{\hx}{\he}{\hS}$ evaluates $\he$, assigns it to a
new variable $\hx$, and then evaluates \hS. The scope of \hx{} is \hS.

The syntax is similar to the real X10 syntax with the following difference:
%Non-escaping methods are marked with \code{@NonEscaping}, such methods
%can be invoked on raw objects (and can be used to initialize them).
FX10 does not have constructors; instead, an object is initialized by
assigning to its fields. X10 uses \code{acc} and not \code{val} for new
accumulator declarations. These are minor simplifications intended to
streamline the formal presentation.

\Subsection{Reduction}
A {\em heap}~$H$ is a mapping from a given set of locations to {\em
  objects}. An object is a pair $C(F)$ where $C$ is a class (the exact
class of the object), and $F$ is a partial map from the fields of $C$
to locations.
%We say the object~\hl is {\em total/cooked} (written~$\cooked_H(\hl)$)
%    if its map is total, i.e.,~$H(\hl)=\hC(F) \gap \dom(F)=\fields(\hC)$.

%We say that a heap~$H$ {\em satisfies} $\phi$ (written~$H \vdash \phi$)
%    if the plus assertions in~$\phi$ (ignoring the minus assertions) are true in~$H$,
%    i.e., if~$\phi \vdash +\hl$ then~$\hl$ is cooked in~$H$
%    and if~$\phi \vdash +\hl.\hf$ then~$H(\hl)=\hC(F)$ and~$F(\hf)$ is cooked in~$H$.


%An {\em annotation} $N$ for a heap $H$ maps each $l \in \dom(H)$ to a
%possibly empty set of fields $a(H(l))$ of the class of $H(l)$ disjoint
%from $\dom(H(l))$. (These are the fields currently being
%asynchronously initialized.) The logic of initialization described
%above is clearly sound for the obvious interpretation of formulas over
%annotated heaps. For future reference, we say that that a heap $H$
%{\em satisfies} $\phi$ if there is some annotation $N$ (and some
%valuation $v$ assigning locations in $\dom(H)$ to free variables of
%$\phi$) such that $\phi$ evaluates to true.

The reduction relation is described in
Figure~\ref{Figure:reduction}. An {\em S-configuration} is of the form
$\hS,H$ where \hS{} is a statement and $H$ is a heap (representing a
computation which is to execute $\hS$ in the heap $H$), or $H$
(representing a successfully terminated computation), or $\err$
representing a computation that terminated in an error. An
E-configuration is of the form $\he,H$ and represents the
computation which is to evaluate $\he$ in the configuration $H$. The
set of {\em values} is the set of locations; hence E-configurations of
the form $\hl,H$ are terminal.

Two transition relations $\preduce{}$ are defined, one over
S-configurations and the other over E-configurations. Here $\pi$ is a
set of locations which can currently be asynchronously accessed.  Thus
each transition is performed in a context that knows about the current
set of capabilities.  For $X$ a partial function, we use the notation
$X[v \mapsto e]$ to represent the partial function which is the same
as $X$ except that it maps $v$ to $e$.

The rules for termination, step, val, new, invoke, access and assign
are standard.  The only minor novelty is in how \hasync{} is
defined. The critical rule is the last rule in~\RULE{(R-Step)} -- it
specifies the ``asynchronous'' nature of \hasync{} by permitting \hS{}
to make a step even if it is preceded by $\async{\hS_1}$. Further,
each block records its set of synchronous capabilities. When
descending into the body, the block's own capabiliites are added to
those obtained from the environment.

%
The rule~\RULE{(R-New)} returns a new location that is bound to a new
object that is an instance of \hC{} with none of its fields initialized.
%
The rule~\RULE{(R-Access)} ensures that the field is initialized before it is
read ($\hf_i$ is contained in $\ol{\hf}$).
%
The rule~\RULE{(R-Acc-N)} adds the new location bound to the
accumulator as a synchronous access capability to the current async.
%
The rule~\RULE{(R-Acc-A-R)} permits the accumulator to be read in a
block provided that the current set of capabilities permit it (if not,
an error is thrown), and provided that the only statements prior to
the read are re is no nested async prior to the read are clocked
asyncs that are stuck at an \hadvance.

The rule~\RULE{(R-Acc-W)} updates the current contents of the
accumulator provided that the current set of capabilities permit
asynchronous access to the accumulator (if not, an erorr is thrown).

The rule~\RULE{(R-Advance)} permits a \code{clocked finish} to advance
only if all the top-level \code{clocked asyncs} in the scope of the
\code{clocked finish} and before an \code{advance} are stuck at an
\code{advance}. A \code{clocked finish} can also advance if all the
the top-level \code{clocked asyncs} in its scope are stuck. In this
case, the statement in the body of hte \code{clocked finish} has
terminated, and left behind only possibly clocked \code{async}. This
rule corresponds to the notion that the body of a \code{clocked
  finish} deregisters itself from the clock on local termination.
Note that in both these rules unlocked \code{aync} may exist in the
scope of the \code{clocked finish}; they do not come in the way of the
\code{clocked finish} advancing.

%\begin{figure}[htpb!]
\begin{figure*}[t]
\begin{center}
\begin{tabular}{|l|l|}
\hline

$\hP ::= \ol{\hL},\hS$ & Program. \\

$\hL ::= \hclass ~ \hC~\hextends~\hD~\lb~\ol{\hF};~\ol{\hM}~\rb$
& cLass declaration. \\

$\hF ::= \hvar\,\hf:\hC$
& Field declaration. \\

$\hM ::= \hdef\ \hm(\ol{\hx}:\ol{\hC}):\hC\{\hS\}$
& Method declaration. \\

$\hp ::= \hl ~~|~~ \hx$
& Path. \\ %(location or parameter)

$\he ::=  \hp.\hf  ~|~ \hnew{\hC} ~|~ \ha()$
& Expressions. \\ %: locations, parameters, field access\&assignment,  %invocation, \code{new}
$\hB ::= \blockt{\pi}{\hS}$
& Blocks. \\
$\hS ::=  \epsilon ~|~\hp.\hf = \hp; ~|~ \hp.\hm(\ol{\hp});  ~|~ \valt{\hx}{\he}$ &\\
$~~~~|~\valt{\hx}{\hnew{\hAcc(\hp,\hp)}} ~|~ \ha \leftarrow \hp$ &\\
$~~~~|~\valt{\hx}{\hnew{\hClocked(\hp,\hp)}} ~|~ \ha() = \hp$ &\\
$~~~~|~ \pclocked~\finish{\ol{\hp}}{\hB}$&\\
$~~~~|~ \pclocked~\async{\hB}~|~ \hS~\hS$
& Statements. \\ %: locations, parameters, field access\&assignment, invocation, \code{new}

\hline
\end{tabular}
\end{center}
FX10 Syntax.
    The terminals are locations (\hl), parameters and \hthis (\hx), field name (\hf), method name (\hm), class name (\hB,\hC,\hD,\hObject),
        and keywords (\hhnew, \hfinish, \hasync, \code{val}).
    The program source code cannot contain locations (\hl), because locations are only created during execution/reduction in \RULE{R-New} of \Ref{Figure}{reduction}.

\label{Figure:syntax}
%\end{figure}

%\begin{figure*}[t]

\begin{center}
\begin{tabular}{|c|}
\hline
\\\\
$\typerule{
}{
 \epsilon,\cH \preduce \cH
}$~\RULE{(R-Epsilon)}
~
$\typerule{
 \hS,\cH \preduce \hS',\cH' ~|~ \cH' ~|~ \err
}{
  \begin{array}{rcl}
    \blockt{\ol{\hl}}{\hS},\cH&\ptreduce &\blockt{\ol{\hl}}{\hS'},\cH' 
    ~|~ \cH' ~|~ \err \gap \pi=\pi'_1,\ol{\hl};\pi'_2\\
    \hS~\hS_1, \cH &\preduce& \hS'~\hS_1,\cH' ~|~ \hS_1,\cH' ~|~ \err
  \end{array}
}$~\RULE{(R-Trans)}
\\\\
$\typerule{
 \hB,\cH \preduce \hB',\cH' ~|~ \cH' ~|~ \err
}{
    \pclocked~\async{\hB},\cH \preduce 
   \pclocked~\async{\hB'},\cH'  ~|~ \cH' ~|~ \err \\
    \pclocked~\finish{\ol{\hl}}{\hB},\cH \ptreduce 
   \pclocked~\finish{\ol{\hl}}{\hB'},\cH'  ~|~ \cH' ~|~ \err \gap\gap \pi=\pi_1';\pi_2',\ol{\hl}\\
}$~\RULE{(R-Trans-B)}
\\\\
$\typerule{
  \hS,\cH \preduce \hS', \cH'
}{
  \pclocked~\async{\hB}~\hS, \cH \preduce \pclocked~\async{\hB}~\hS', \cH'\\
}$~\RULE{(R-Async)}
\\\\
$\typerule{
  \he,\cH \reduce \hl,\cH'
}{
  \hval{\hx}{\he}{\hS},\cH \preduce \hS[\hl/\hx], \cH'\\
 % \valt{\hx}{\he},\cH \preduce \cH'
}$~\RULE{(R-Val)}
~
$\typerule{
    \hl' \not \in \dom(\cH)
}{
  \hnew{\hC},\cH \preduce \hl',\cH[ \hl' \mapsto \hC()]
}$~\RULE{(R-New)}
~
$\typerule{
    \cH(\hl')=\hC(\ldots)
        \gap
    \mbody{}(\hm,\hC)=\ol{\hx}.\hS
}{
  \hl'.\hm(\ol{\hl}),\cH \preduce \hS[\ol{\hl}/\ol{\hx},\hl'/\hthis],\cH
}$~\RULE{(R-Invoke)}
\\\\
$\typerule{
    \cH(\hl)=\hC(\ol{\hf}\mapsto\ol{\hl'})\gap\gap 
\hg=\hf_i ~|~\hg\not\in\ol{\hf}
}{
  \hl.\hg,\cH \preduce \hl_i',\cH ~|~ \err
}$~\RULE{(R-Access)}
\quad
$\typerule{
    \cH(\hl)=\hC(F) 
}{
  \hl.\hf=\hl',\cH \preduce \cH[ \hl \mapsto \hC(F[ \hf \mapsto \hl'])]
}$~\RULE{(R-Assign)}
\\\\
$\typerule{
    \cH(\hl)=\hClocked(\ho,\he) \gap \hl\in \pi_2 ~|~ \hl\not\in\pi_2
}{
  \hl()=\hl',\cH \preduce \cH[ \hl \mapsto \hClocked(\ho,\hl')] ~|~ \err
}$~\RULE{(R-Clocked-A)}
\quad
$\typerule{
  \cH(\hl)=\hClocked(\ho,\he) \gap \hl\in\pi_2 ~|~ \hl\not\in\pi_2
}{
  \hval{\hx}{\hl()}{\hS},\cH \preduce \hS[\ho/\hx], \cH ~|~ \err\\
}$~\RULE{(R-Clocked-R)}
\\\\
$\typerule{
    \hl \not \in \dom(\cH)
}{
\blockt{\pi_1}{\acc{\hx}{\hnew{\hAcc(\hr,\hz)}}~\hS},\cH \preduce 
   \blockt{\pi_1,l}{\hS[\hl/\hx]},\cH[ \hl \mapsto \hAcc(\hr,\hz)] \\
%\blockt{\pi_1}{\acc{\hx}{\hnew{\hAcc(\hr,\hz)}}},\cH \preduce 
%   \cH[ \hl \mapsto \hAcc(\hr,\hz)]
}$~\RULE{(R-Acc-N)}
\\\\
$\typerule{}{
    \epsilon \stuck
}$~\RULE{(R-Q-CA)}
~
$\typerule{B\stuck}{
    \hclocked~\async~B\stuck
}$~\RULE{(R-Q-CA)}
~
$\typerule{\hS\stuck}{
   \blockt{\ol{\hl}}{\hS}\stuck\\
   \blockt{\ol{\hl}}{\hS~\xadvance~\hS'}\stuck\\
}$~\RULE{(R-Q-B)}
~
$\typerule{
  \hS_1 \stuck \gap \hS_2 \stuck
}{
   \hS_1~\hS_2 \stuck
}$~\RULE{(R-Q-S)}
\\\\
$\typerule{
\cH(\ha)=\hAcc(\hr,\hv) \gap S\stuck \gap \gap    \ha\in \ol{\hl} ~|~
\ha\not\in \ol{\hl}
}{
\blockt{\ol{\hl}}{\hS~\hval{\hx}{\ha()}~\hS_1},\cH \preduce 
\blockt{\ol{\hl}}{\hS~\hS_1[\hv/\hx]}, \cH' ~|~ \err
}$~\RULE{(R-Acc-R)}
~
$\typerule{
  \cH(\ha)=\hAcc(\hr,\hv)\gap \hw=\hr(\hv,\hp)  \gap\gap \ha\in \pi_1~|~ \ha\not\in\pi_1
}{
  \ha \leftarrow \hp,\cH \preduce \cH[\ha \mapsto \hAcc(\hr,\hw)] ~|~ \err
}$~\RULE{(R-Acc-W)}
\\\\
$\typerule{
\hS_1\stuck \gap\gap \hS,\cH \preduce \hS',\cH'
}{
   \hS_1~\hS,\cH \preduce \hS_1~\hS',\cH'
}$~\RULE{(R-Q)}
\\\\
$\typerule{
  \hB \ureduce \hB'
}{
    \hclocked~\async{\hB} \ureduce \hclocked~\async{\hB'} \\
}$~\RULE{(R-Adv-S)}
~
$\typerule{}{
  \hclocked~\finishasync{\hB} \ureduce  \\
\quad\quad\quad \hclocked~\finishasync{\hB}\\
    \xadvance~\hS \ureduce \hS\\
}$~\RULE{(R-Adv-A,CAF)}
\\\\

$\typerule{
  \hS \ureduce \hS'
}{
   \blockt{\ol{\hl}}{\hS}\ureduce \blockt{\ol{\hl}}{\hS'}
}$~\RULE{(R-Adv-B)}
~
$\typerule{
    \hS_1 \ureduce \hS_1'\ \     \hS_2 \ureduce \hS_2'
}{
  \hS_1~\hS_2\ureduce \hS_1'~\hS_2'
}$~\RULE{(R-Adv-S)}
~
$\typerule{
    \hB \ureduce \hB' \gap H'=H[\ol{\hl}\mapsto H[\ol{\hl}]']
}{
  \begin{array}{l}
    \hclocked~\finish{\ol{\hl}}{\hB},\cH\preduce \\
\quad\quad\quad \hclocked~\finish{\ol{\hl}}{\hB'},\cH'
  \end{array}
}$~\RULE{(R-Adv)}
\\\\
\hline
\end{tabular}
\end{center}


\caption{FX10 Syntax and Reduction Rules ($\hS,\cH \preduce \hS',\cH' ~|~\cH'$ and~$\he,\cH \preduce \hl,\cH'$).}
\label{Figure:reduction}
\end{figure*}


\Subsection{Results}
