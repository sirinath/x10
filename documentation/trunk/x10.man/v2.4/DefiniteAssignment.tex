\chapter{Definite Assignment}
\label{sect:DefiniteAssignment}
\index{definite assignment}
\index{assignment!definite}
\index{definitely assigned}
\index{definitely not assigned}

X10 requires, reasonably enough, that every variable be set before it is read.
Sometimes this is easy, as when a variable is declared and assigned together: 
%~~gen ^^^ DefiniteAssignment4x1u
% package DefiniteAssignment4x1u;
% class Example {
% def example() {
%~~vis
\begin{xten}
  var x : Int = 0;
  assert x == 0;
\end{xten}
%~~siv
%}}
%~~neg
However, it is convenient to allow programs to make decisions before
initializing variables.
%~~gen ^^^ DefiniteAssignment4u7z
% package DefiniteAssignment4u7z;
% class Example {
%~~vis
\begin{xten}
def example(a:Int, b:Int) {
  val max:Int;
  //ERROR: assert max==max; // can't read 'max'
  if (a > b) max = a;
  else max = b;
  assert max >= a && max >= b;
}
\end{xten}
%~~siv
%}
%~~neg
This is particularly useful for \xcd`val` variables.  \xcd`var`s could be
initialized to a default value and then reassigned with the right value.
\xcd`val`s must be initialized once and cannot be changed, so they must be
initialized with the correct value. 

However, one must be careful -- and the X10 compiler enforces this care.
Without the \xcd`else` clause, the preceding code might not give \xcd`max` a
value by the \xcd`assert`.  

This leads to the concept of {\em definite assignment} \cite{jls2}.
A variable is definitely assigned at a point in code if, no matter how that
point in code is reached, the variable has been assigned to.  In X10,
variables must be definitely assigned before they can be read.


As X10 requires that \xcd`val` variables {\em not} be initialized
twice,  we need the dual concept as well.  A variable is {\em definitely
unassigned} at a point in code if it cannot have been assigned there.  For
example, immediately after \xcd`val x:Int`, \xcd`x` is definitely unassigned. 

Finally, we need the concept of {\em singly} and {\em multiply assigned}.
A variable is singly assigned in a block if it is assigned precisely
once; it is multiply assigned if it could possibly be assigned more than once.  
\xcd`var`s can  multiply assigned as desired. \xcd`val`s must be singly
assigned.  For example, the code \xcd`x = 1; x = 2;` is perfectly fine if
\xcd`x` is a \xcd`var`, but incorrect (even in a constructor) if \xcd`x` is a
\xcd`val`.  

At some points in code, a variable might be neither definitely assigned nor
definitely unassigned.    Such states are not always useful.  
%~~gen ^^^ DefiniteAssignment4f5z
% package DefiniteAssignment4f5z;
% class Example {
% 
%~~vis
\begin{xten}
def example(flag : Boolean) {
  var x : Int;
  if (flag) x = 1;
  // x is neither def. assigned nor unassigned.
  x = 2; 
  // x is def. assigned.
\end{xten}
%~~siv
% } } 
%~~neg
This shows that we cannot simply define ``definitely unassigned'' as ``not
definitely assigned''.   If \xcd`x` had been a \xcd`val` rather than a
\xcd`var`, the previous example would not be allowed.    

Unfortunately, a completely accurate definition of ``definitely assigned''
or ``definitely unassigned'' is undecidable -- impossible for the compiler to
determine.  So, X10 takes a {\em conservative approximation} of these
concepts.  If X10's definition says that \xcd`x` is definitely assigned (or
definitely unassigned), then it will be assigned (or not assigned) in every
execution of the program.  

However, there are programs which X10's algorithm says are incorrect, but
which actually would behave properly if they were executed.   In the following
example, \xcd`flag` is either \xcd`true` or \xcd`false`, and in either case
\xcd`x` will be initialized.  However, X10's analysis does not understand this
--- thought it {\em would} understand if the example were coded with an
\xcd`if-else` rather than a pair of \xcd`if`s.  So, after the two \xcd`if`
statements, \xcd`x` is not definitely assigned, and thus the \xcd`assert`
statement, which reads it, is forbidden.  
%~~gen ^^^ DefiniteAssignment3x6i
% package DefiniteAssignment3x6i;
% class Example{ 
%~~vis
\begin{xten}
def example(flag:Boolean) {
  var x : Int;
  if (flag) x = 1;
  if (!flag) x = 2;
  // ERROR: assert x < 3;
}
\end{xten}
%~~siv
%}
%~~neg

\section{Asynchronous Definite Assignment}


Local variables and instance fields allow {\em asynchronous assignment}. A local
variable can be assigned in an \xcd`async` statement, and, when the
\xcd`async` is \xcd`finish`ed, the variable is definitely assigned.  

\begin{ex}
%~~gen ^^^ DefiniteAssignment4a6n
% package DefiniteAssignment4a6n;
% class Example {
% def example() {
%~~vis
\begin{xten}
val a : Int;
finish {
  async {
    a = 1;
  } 
  // a is not definitely assigned here
}
// a is definitely assigned after 'finish'
assert a==1; 
\end{xten}
%~~siv
%} } 
%~~neg
\end{ex}

This concept supports a core X10 programming idiom.  A \xcd`val` variable may
be initialized asynchronously, thereby providing a means for returning a value
from an \xcd`async` to be used after the enclosing \xcd`finish`.  

\section{Characteristics of Definite Assignment}

The properties ``definitely assigned'', ``singly assigned'', and
``definitely unassigned'' are computed by a conservative approximation of
X10's evaluation rules.

The precise details are up to the implementation. 
Many basic cases must be handled accurately; \eg, \xcd`x=1;` definitely and
singly assigns \xcd`x`.  

However, in more complicated cases, a conforming X10 may mark as invalid 
some code which, when executed, would actually be correct.  
For example, the following
program fragment will always result in \xcd`x` being definitely and singly
assigned:  
\begin{xten}
val x : Int;
var b : Boolean = mysterious();
if (b) {
   x = cryptic();
}
if (!b) { 
   x = unknown();
}
\end{xten}
However, most conservative approximations of program execution won't mark
\xcd`x` as properly initialized, though it is.   For \xcd`x` to be properly
initialized, precisely one of the 
two assignments to \xcd`x` must be executed. If \xcd`b` were true initially,
it would still be true after the call to \xcd`cryptic()` --- since methods
cannot modify their caller's local variables -- and so the first but not the
second assignment would happen. If \xcd`b` were false initially, it would
still be false when \xcd`!b` is tested, and so the second but not the first
assignment would happen.  Either way, \xcd`x` is definitely and singly assigned.

However, for a slightly different program, this analysis would be wrong. \Eg,
if  \xcd`b` were a field of \xcd`this` rather than a local variable,
\xcd`cryptic()` could change \xcd`b`; if \xcd`b` were true initially, both
assignments might happen, which is incorrect for a \xcd`val`.  

This sort of reasoning is beyond  most conservative approximation algorithms.
(Indeed, many do not bother checking that \xcd`!b` late in the program is the
opposite of \xcd`b` earlier.)
Algorithms that pay attention to such details and subtleties tend to be
fairly expensive, which would lead to very slow compilation for X10 -- for the
sake of obscure cases.

X10's analysis provides at least the following guarantees. We describe them in
terms of a statement \xcd`S` performing some collection of possible numbers of
assignments to variables --- on a scale of ``0'', ``1'', and ``many''. For
example, \xcd`if (b) x=1; else {x=1;x=2;y=2;}` might assign to \xcd`x` one or
many times, and might assign to \xcd`y` zero or one time. Hence, after it,
\xcd`x` is definitely assigned and may be multiply assigned, and \xcd`y` is
neither definitely assigned nor definitely unassigned.  

These descriptions are combined in natural ways.  For example, if \xcd`R` says
that \xcd`x` will be assigned 0 or 1 times, and \xcd`S` says it will be
assigned precisely once, then \xcd`R;S` will assign it one or many times.  If
only one or \xcd`R` or \xcd`S` will occur, as from \xcd`if (b) R; else S;`, 
then \xcd`x` may be assigned 0 or 1 times. 

This information is sufficient for the tests X10 makes.  If \xcd`x` can is
assigned one or many times in \xcd`S`, it is definitely assigned.  It is an
error if 
\xcd`x` is ever read at a point where it have been assigned zero times.  It is
an error if a \xcd`val` may be assigned many times.


We do not guarantee that any particular X10 compiler uses this algorithm;
indeed, as of the time of writing, the X10 compiler uses a somewhat more
precise one.  However, any conformant X10 compiler must provide results which
are at least as accurate as this analysis.




\subsubsection{Assignment: {\tt x = e}}   

\xcd`x = e` assigns to \xcd`x`, in addition to whatever assignments
\xcd`e` makes.   For example, if \xcd`this.setX(y)` sets a field \xcd`x` to
\xcd`y` and returns \xcd`y`, then \xcd`x = this.setX(y)` definitely and
multiply assigns \xcd`x`.  

\subsubsection{{\tt async} and {\tt finish}}

By itself, \xcd`async S` provides few guarantees.  After \xcd`async{x=1;}`
finishes, we know that there is a separate activity which will, when the
scheduler lets it, set \xcd`x` to \xcd`1`.  We do not know that anything has
happened yet.

However, if there is a \xcd`finish` around the \xcd`async`, the situation is
clearer.  After \xcd`finish{ async{ x=1; }}`, \xcd`x` has definitely been
assigned.  

In general, if an \xcd`async S` appears in the body of a \xcd`finish` in a way
that guarantees that it will be executed, then, after the \xcd`finish`, the
assignments made by \xcd`S` will have occurred.  For example, if \xcd`S`
definitely assigns to \xcd`x`, and the body of the \xcd`finish` guarantees
that \xcd`async S` will be executed, then \xcd`finish{...async S...}`
definitely assigns \xcd`x`.



\subsubsection{{\tt if} and {\tt switch}}

When \xcd`if(E) S else T` finishes, it will have performed the assignments of
\xcd`E`, together with those of either \xcd`S` or \xcd`T` but not both.  For
example, \xcd`if (b) x=1; else x=2;` definitely assigns \xcd`x`,
but \xcd`if (b) x=1;` does not.

{\tt switch} is more complex, but follows the same principles as \xcd`if`.
For example, \xcd`switch(E){case 1: A; break; case 2: B; default: C;}`  
performs the assignments of \xcd`E`, and those of precisely one of \xcd`A`, or
\xcd`B;C`, or \xcd`C`.  Note that case \xcd`2` falls through to the default
case, so it performs the same assignments as \xcd`B;C`.

\subsubsection{Sequencing}

When \xcd`R;S` finishes, it will have performed the assignments of \xcd`R` and
those of \xcd`S`.  For example, \xcd`x=1;y=2;` definitely assigns \xcd`x` and
\xcd`y`, and \xcd`x=1;x=2;` multiply assigns \xcd`x`. 


\subsubsection{Loops}

\xcd`while(E)S` performs the assignments of \xcd`E` one or more times, and
those of \xcd`S` zero or more times.  For example, if \xcd`while(b()){x=1;}`
might assign to \xcd`x` zero, one, or many times.  
\xcd`do S while(E)` performs the assignments of \xcd`E` one or more times, and
those of \xcd`S` one or more times. 

\xcd`for(A;B;C)D` performs the assignments of \xcd`A` once, those of \xcd`B`
one or more times, and those of \xcd`C` and \xcd`D` one or more times.
\xcd`for(x in E)S` performs the assignments of \xcd`E` once and those of
\xcd`S` zero or more times.  

Loops are of very little value for providing definite assignments, since X10
does not in general know how many times they will be executed. 

\xcd`continue` and \xcd`break` inside of a loop are hard to describe in simple
terms.  They may be conservatively assumed to cause the loop give no
information about the variables assigned inside of it.
For example, the analysis may conservatively conclude that 
\xcd`do{ x = 1; if (true) break; } while(true)` may 
assign to \xcd`x` zero, one, or many times, overlooking the more precise fact
that it is assigned once.  




\subsubsection{Method Calls}

A method call \xcd`E.m(A,B)` performs the assignments of \xcd`E`, \xcd`A`, and
\xcd`B` once each, and also those of \xcd`m`.  This implies that X10 must be
aware of the possible assignments performed by each method.


If X10 has complete information about \xcd`m` (as when \xcd`m` is a
\xcd`private` or \xcd`final` method), this is straightforward.  When such
information is fundamentally impossible to acquire, as when \xcd`m` is a
non-final method invocation, X10 has no choice but to assume that \xcd`m`
might do anything that a method can do.    (For this reason, the only methods
that can be called from within a constructor on a raw --
incompletely-constructed -- object) are the \xcd`private` and
\xcd`final` ones.)  
\begin{itemize}
\item \xcd`m` cannot assign to local fields of the caller; methods have no
      such power.
\item \xcd`m` can assign to \xcd`var` fields of \xcd`this` freely. 
\item \xcd`m` cannot initialize \xcd`val` fields of \xcd`this`.  (But see
      \Sref{sect:call-another-ctor}; when one constructor calls another as the
      first statement of its body, the other constructor can initialize
      v\xcd`val` fields. This is a constructor call, not a method call.) 
\end{itemize}

Recall that every container must be fully initialized upon exit
from its constructor.  
X10 places certain restrictions on which methods can be called from a
constructor; see \Sref{sect:nonescaping}.  One of these restrictions is that
methods called before object initialization is complete must be \xcd`final` or
\xcd`private` --- and hence, available for static analysis.  So, when checking
field initialization, X10 will ensure: 
\begin{enumerate}
\item Each \xcd`val` field is initialized before it is read.   
      A method that does not read a \xcd`val` field \xcd`f` {\em may} be
      called before \xcd`f` is initialized; a method that reads \xcd`f` must
      not be called until \xcd`f` is initialized.        
      For example, 
      a constructor may have the form:
%~~gen ^^^ DefiniteAssignment4x6k
% package DefiniteAssignment4x6k;
%~~vis
\begin{xten}
class C {
  val f : Int;
  val g : String;
  def this() {
     f = fless();
     g = useF();
  }
  private def fless() = "f not used here".length();
  private def useF() = "f=" + this.f;
}
\end{xten}
%~~siv
%
%~~neg

\item \xcd`var` fields require a deeper analysis.  Consider a \xcd`var`
      field \xcd`var x:T`  without initializer.  If \xcd`T` has a default
      value, \xcd`x` may be read inside of a constructor before it is
      otherwise written, and it will 
      have its default value.   

      If \xcd`T` has no default value, an analysis
      like that used for \xcd`val`s must be performed to determine that
      \xcd`x` is initialized before it is used.  The situation is 
      more complex than for \xcd`val`s, however, because a method can assign to
      \xcd`x` as well read from it.  The X10 compiler computes a conservative
      approximation of which methods
      read and write which \xcd`var` fields. (Doing this carefully 
      requires finding a solution of a set of equations over sets of
      variables, with each callable method having equations describing what it
      reads and writes.)    

\end{enumerate}


\subsubsection{{\tt at} 
%and \xcd`athome`
}

%%AT-COPY%% \xcd`at(E)S` performs the assignments of \xcd`E`. Within \xcd`S`, only those
%%AT-COPY%% assignments to variables \xcd`x` from the surrounding environment which take
%%AT-COPY%% place within a suitable \xcd`athome(x)R` are counted. 
%%AT-COPY%% 
%%AT-COPY%% \begin{ex}
%%AT-COPY%% In the following code, the outer variable named \xcd`a` is definitely assigned
%%AT-COPY%% once, by the assignment \xcd`a = 3;`.  The inner variable (also named \xcd`a`)
%%AT-COPY%% is definitely multiply assigned 
%%AT-COPY%% by the two assignments \xcd`a = 1;` and \xcd`a = 2;` 
%%AT-COPY%% between the \xcd`at` and the \xcd`athome`.  
%%AT-COPY%% 
%%AT-COPY%% %~~gen ^^^ DefiniteAssignment3n5q
%%AT-COPY%% % package DefiniteAssignment3n5q;
%%AT-COPY%% % KNOWNFAIL-at
%%AT-COPY%% % class DefAss { def defass() { 
%%AT-COPY%% %~~vis
%%AT-COPY%% \begin{xten}
%%AT-COPY%% var a : Int;
%%AT-COPY%% at(here.next(); var a : Int = a) {
%%AT-COPY%%   a = 1;
%%AT-COPY%%   a = 2; 
%%AT-COPY%%   athome(a) a = 3;
%%AT-COPY%% }
%%AT-COPY%% \end{xten}
%%AT-COPY%% %~~siv
%%AT-COPY%% % } } 
%%AT-COPY%% %~~neg
%%AT-COPY%% 
%%AT-COPY%% 
%%AT-COPY%% \end{ex}
%%AT-COPY%% 

\xcd`at(p)S` cannot perform any assignments.  \xcd`this` cannot be read or
written by an \xcd`at`-statement.

\subsubsection{{\tt atomic}}

\xcd`atomic S` performs the assignments of \xcd`S`, 
and \xcd`when(E)S` performs those of \xcd`E` and \xcd`S`.  

\subsubsection{{\tt try}}

\xcd`try S catch(x:T1) E1 ... catch(x:Tn) En  finally F` 
performs some or all of the assignments of \xcd`S`, plus all the assignments
of zero or one of the \xcd`E`'s, plus those of \xcd`F`.  
For example,
\begin{xten}
try {
  x = boomy();
  x = 0;
}
catch(e:Boom) { y = 1; }
finally { z = 1; }
\end{xten}
\noindent 
assigns \xcd`x` zero, one, or many times\footnote{A more precise
analysis could discover that \xcd`x` cannot be initialized only once.}, 
assigns \xcd`y` zero or one time, and assigns \xcd`z` exactly once.

\subsubsection{Expression Statements}

Expression statements \xcd`E;`, and other statements that execute an
expression and do something innocuous with it (local variable declaration and
\xcd`assert`) have the same effects as \xcd`E`. 

\subsubsection{{\tt return}, {\tt throw}}

Statements that do not finish normally, such as \xcd`return` and \xcd`throw`,
don't initialize anything (though the computation of the return or thrown
value may).    They also terminate a line of computation.  For example, 
\xcd`if(b) {x=1; return;}  x=2;` definitely and singly assigns \xcd`x`.  
