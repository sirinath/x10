\chapter{Definite Assignment}
\label{sect:DefiniteAssignment}
\index{definite assignment}
\index{assignment!definite}
\index{definitely assigned}
\index{definitely not assigned}

X10 requires, reasonably enough, that every variable be set before it is read.
Sometimes this is easy, as when a variable is declared and assigned together: 
%~~gen ^^^ DefiniteAssignment4x1u
% package DefiniteAssignment4x1u;
% class Example {
% def example() {
%~~vis
\begin{xten}
  var x : Int = 0;
  assert x == 0;
\end{xten}
%~~siv
%}}
%~~neg
However, it is convenient to allow programs to make decisions before
initializing variables.
%~~gen ^^^ DefiniteAssignment4u7z
% package DefiniteAssignment4u7z;
% class Example {
%~~vis
\begin{xten}
def example(a:Int, b:Int) {
  val max:Int;
  // max cannot be read here.
  if (a > b) max = a;
  else max = b;
  assert max >= a && max >= b;
}
\end{xten}
%~~siv
%}
%~~neg
This is particularly useful for \xcd`val` variables.  \xcd`var`s could be
initialized to a default value and then reassigned with the right value, but
\xcd`val`s must be initialized once correctly and cannot be changed. 

However, one must be careful -- and the X10 compiler enforces this care.
Without the \xcd`else` clause, the preceding code might not give \xcd`max` a
value by the \xcd`assert`.  

This leads to the concept of {\em definite assignment} \cite{Javasomething}.
A variable is definitely assigned at a point in code if, no matter how that
point in code is reached, the variable has been assigned to.  In X10,
variables must be definitely assigned before they can be read.


As X10 requires that \xcd`val` variables {\em not} be initialized
twice,  we need the dual concept as well.  A variable is {\em definitely
unassigned} at a point in code if it cannot have been assigned there.  For
example, immediately after \xcd`val x:Int`, \xcd`x` is definitely unassigned.  

At some points in code, a variable might be neither definitely assigned nor
definitely unassigned.    Such states are not always useful.  
%~~gen ^^^ DefiniteAssignment4f5z
% package DefiniteAssignment4f5z;
% class Example {
% 
%~~vis
\begin{xten}
def example(flag : Boolean) {
  var x : Int;
  if (flag) x = 1;
  // x is neither def. assigned nor unassigned.
  x = 2; 
  // x is def. assigned.
\end{xten}
%~~siv
% } } 
%~~neg
This shows that we cannot simply define ``definitely unassigned'' as ``not
definitely assigned''.   If \xcd`x` had been a \xcd`val` rather than a
\xcd`var`, the previous example would not be allowed.  

Unfortunately, a completely accurate definition of ``definitely assigned''
or ``definitely unassigned'' is undecidable -- impossible for the compiler to
determine.  So, X10 takes a {\em conservative approximation} of these
concepts.  If X10's definition says that \xcd`x` is definitely assigned (or
definitely unassigned), then it will be assigned (or not assigned) in every
execution of the program.  

However, there are programs which X10's algorithm says are incorrect, but
which actually would behave properly if they were executed.   In the following
example, \xcd`flag` is either \xcd`true` or \xcd`false`, and in either case
\xcd`x` will be initialized.  However, X10's analysis does not understand this
--- thought it {\em would} understand if the example were coded with an
\xcd`if-else` rather than a pair of \xcd`if`s.  So, after the two \xcd`if`
statements, \xcd`x` is not definitely assigned, and thus the \xcd`assert`
statement, which reads it, is forbidden.  
%~~gen ^^^ DefiniteAssignment3x6i
% package DefiniteAssignment3x6i;
% class Example{ 
%~~vis
\begin{xten}
def example(flag:Boolean) {
  var x : Int;
  if (flag) x = 1;
  if (!flag) x = 2;
  // Not Allowed: assert x < 3;
}
\end{xten}
%~~siv
%}
%~~neg

\section{Technical Details of Definite Assignment}

Definite assignedness for local variables is determined by one algorithm;
definited assignedness for fields by a different one.  

Local variables (but not fields) allow {\em asynchronous assignment}. A local
variable can be assigned in an \xcd`async` statement, and, when the
\xcd`async` is \xcd`finish`ed, the variable is definitely assigned.  
\begin{ex}
%~~gen ^^^ DefiniteAssignment4a6n
% package DefiniteAssignment4a6n;
% class Example {
% def example() {
%~~vis
\begin{xten}
val a : Int;
finish {
  async {
    a = 1;
  } 
  // a is not definitely assigned here
}
assert a==1;
\end{xten}
%~~siv
%} } 
%~~neg

\end{ex}
