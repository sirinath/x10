
\chapter{Variables}\label{XtenVariables}\index{variable}

%%OLDA variable is a storage location.  \Xten{} supports seven kinds of
%%OLDvariables: constant {\em class variables} (static variables), {\em
%%OLD  instance variables} (the instance fields of a class), {\em array
%%OLD  components}, {\em method parameters}, {\em constructor parameters},
%%OLD{\em exception-handler parameters} and {\em local variables}.

A {\em variable} is an X10 identifier associated with a value within some
context. Variable bindings have these essential properties:
\begin{itemize}
\item {\bf Type:} What sorts of values can be bound to the identifier;
\item {\bf Scope:} The region of code in which the identifier is associated
      with the entity;
\item {\bf Lifetime:} The interval of time in which the identifier is
      associated with the entity.
\item {\bf Visibility:} Which parts of the program can read or manipulate the
      value through the variable.
\end{itemize}



X10 has many varieties of variables, used for a number of purposes. 
\begin{itemize}
\item Class variables, also known as the static fields of a class, which hold
      their values for the lifetime of the class.  
\item Instance variables, which hold their values for the lifetime of an
      object;
\item Array elements, which are not individually named and hold their values
      for the lifetime of an array;
\item Formal parameters to methods, functions, and constructors, which hold
      their values for the duration of method (etc.) invocation;
\item Local variables, which hold their values for the duration of execution
      of a block.
\item Exception-handler parameters, which hold their values for the execution
      of the exception being handled. 
\end{itemize}
A few other kinds of things are called variables for historical reasons; \eg,
type parameters are often called type variables, despite not being variables
in this sense because they do not refer to X10 values.  Other named entities,
such as classes and methods, are not called variables.  However, all
name-to-whatever bindings enjoy similar concepts of scope and visibility.  

\begin{ex}
In the following example, 
\xcd`n` is an instance variable, and \xcd`nxt` is a
local variable defined within the method \xcd`bump`.\footnote{This code is
unnecessarily turgid for the sake of the example.  One would generally write
\xcd`public def bump() = ++n;`.   }
%~~gen ^^^ Vars10
% package Vars.For.Squares;
%~~vis
\begin{xten}
class Counter {
  private var n : Int = 0;
  public def bump() : Int {
    val nxt = n+1;
    n = nxt;
    return nxt;
    }
}
\end{xten}
%~~siv
% class Hook{ def run() { val c = new Counter(); val d = new Counter();
%   assert c.bump() == 1;  
%   assert c.bump() == 2;  
%   assert c.bump() == 3;  
%   assert c.bump() == 4;  
%   assert d.bump() == 1;  
%   assert c.bump() == 5;  
%   return true;
% } }
%~~neg
Both variables have type \xcd`Int` (or
perhaps something more specific).    The scope of \xcd`n` is the body of
\xcd`Counter`; the scope of \xcd`nxt` is the body of \xcd`bump`.  The
lifetime of \xcd`n` is the lifetime of the \xcd`Counter` object holding it;
the lifetime of \xcd`nxt` is the duration of the call to \xcd`bump`. Neither
variable can be seen from outside of its scope.
\end{ex}
\label{exploded-syntax}
\label{VariableDeclarations}
\index{variable declaration}


Variables whose value may not be changed after initialization are said to be
{\em immutable}, or {\em constants} (\Sref{FinalVariables}), or simply
\xcd`val` variables. Variables whose value may change are {\em mutable} or
simply \xcd`var` variables. \xcd`var` variables are declared by the \xcd`var`
keyword. \xcd`val` variables may be declared by the \xcd`val` keyword; when a
variable declaration does not include either \xcd`var` or \xcd`val`, it is
considered \xcd`val`. 

A variable---even a \xcd`val` -- can be declared in one statement, and then
initialized later on.  It must be initialized before it can be used
(\Sref{sect:DefiniteAssignment}).  


\begin{ex}
The following example illustrates many of the variations on variable
declaration: 
%~~gen ^^^ Vars20
%package Vars.For.Bears.In.Chairs;
%class VarExample{
%static def example() {
%~~vis
\begin{xten}
val a : Int = 0;               // Full 'val' syntax
b : Int = 0;                   // 'val' implied
val c = 0;                     // Type inferred
var d : Int = 0;               // Full 'var' syntax
var e : Int;                   // Not initialized
var f : Int{self != 100} = 0;  // Constrained type
val g : Int;                   // Init. deferred
if (a > b) g = 1; else g = 2;  // Init. done here.
\end{xten}
%~~siv
%}}
%~~neg
\end{ex}





\section{Immutable variables}
\label{FinalVariables}
\index{variable!immutable}
\index{immutable variable}
\index{variable!val}
\index{val}

%##(LocalVariableDeclStatement LocalVariableDecl
\begin{bbgrammar}
%(FROM #(prod:LocalVariableDeclStatement)#)
LocalVariableDeclStatement \: LocalVariableDecl \xcd";" & (\ref{prod:LocalVariableDeclStatement}) \\
%(FROM #(prod:LocalVariableDecl)#)
   LocalVariableDecl \: Mods\opt VarKeyword VariableDeclarators & (\ref{prod:LocalVariableDecl}) \\
                    \| Mods\opt VariableDeclaratorsWithType \\
                    \| Mods\opt VarKeyword FormalDeclarators \\
\end{bbgrammar}
%##)

An immutable (\xcd`val`) variable can be given a value (by initialization or
assignment) at 
most once, and must be given a value before it is used.  Usually this is
achieved by declaring and initializing the variable in a single statement, 
such as \Xcd{val x = 3}, with syntax 
(\ref{prod:LocalVariableDecl}) using the {\it VariableDeclarators} or {\it
VariableDeclaratorsWithType} alternatives.

\begin{ex}
After these declarations, \xcd`a` and \xcd`b` cannot be assigned to further,
or even redeclared:  
%~~gen ^^^ Vars30
% package Vars.In.Snares;
% class ABitTedious{
% def example() {
%~~vis
\begin{xten}
val a : Int = 10;
val b = (a+1)*(a-1);
// ERROR: a = 11;  // vals cannot be assigned to.
// ERROR: val a = 11; // no redeclaration.
\end{xten}
%~~siv
%}}
%~~neg

\end{ex}

In two special cases, the declaration and assignment are separate.  One 
case is how constructors give values to \xcd`val` fields of objects.  In this
case, production (\ref{prod:LocalVariableDecl}) is taken, with the {\it
FormalDeclarators} option, such as  \Xcd{var n:Int;}.  

\begin{ex} The
\xcd`Example` class has an immutable field \xcd`n`, which is given different
values depending on which constructor was called. \xcd`n` can't be given its
value by initialization when it is declared, since it is not knowable which
constructor is called at that point.  
%~~gen ^^^ Vars40
% package Vars.For.Cares;
%~~vis
\begin{xten}
class Example {
  val n : Int; // not initialized here
  def this() { n = 1; }
  def this(dummy:Boolean) { n = 2;}
}
\end{xten}
%~~siv
%
%~~neg
\end{ex}


The other case of separating declaration and assignment is in function
and method call, described in \Sref{sect:formal-parameters}.  The formal
parameters are bound to the corresponding actual 
parameters, but the binding does not happen until the function is called.  

\begin{ex}
In
the code below, \xcd`x` is initialized to \xcd`3` in the first call and
\xcd`4` in the second.
%~~gen ^^^ Vars50
%package Vars.For.Swears;
%class Examplement {
%static def whatever() {
%~~vis
\begin{xten}
val sq = (x:Int) => x*x;
x10.io.Console.OUT.println("3 squared = " + sq(3));
x10.io.Console.OUT.println("4 squared = " + sq(4));
\end{xten}
%~~siv
%}}
%~~neg
\end{ex}




%%IMMUTABLE%% An immutable variable satisfies two conditions: 
%%IMMUTABLE%% \begin{itemize}
%%IMMUTABLE%% \item it can be assigned to at most once, 
%%IMMUTABLE%% \item it must be assigned to before use. 
%%IMMUTABLE%% \end{itemize}
%%IMMUTABLE%% 
%%IMMUTABLE%% \Xten{} follows \java{} language rules in this respect \cite[\S
%%IMMUTABLE%% 4.5.4,8.3.1.2,16]{jls2}. Briefly, the compiler must undertake a
%%IMMUTABLE%% specific analysis to statically guarantee the two properties above.
%%IMMUTABLE%% 
%%IMMUTABLE%% Immutable local variables and fields are defined by the \xcd"val"
%%IMMUTABLE%% keyword.  Elements of value arrays are also immutable.
%%IMMUTABLE%% 
%%IMMUTABLE%% \oldtodo{Check if this analysis needs to be revisited.}

\section{Initial values of variables}
\label{NullaryConstructor}\index{nullary constructor}
\index{initial value}
\index{initialization}


Every assignment, binding, or initialization to a variable of type \xcd`T{c}`
must be an instance of type \xcd`T` satisfying the constraint \xcd`{c}`.
Variables must be given a value before they are used. This may be done by
initialization -- giving a variable a value as part
of its declaration. 

\begin{eg}
These variables are all initialized: 
%~~gen ^^^ Vars60
%package Vars.For.Bears;
%class VarsForBears{
%def check() {
%~~vis
\begin{xten}
  val immut : Int = 3;
  var mutab : Int = immut;
  val use = immut + mutab;
\end{xten}
%~~siv
%}}
%~~neg
\end{eg}

Or, a variable may be given a value by an assignment.  \xcd`var` variables may
be assigned to repeatedly.  \xcd`val` variables may only be assigned once; the
compiler will ensure that they are assigned before they are used.

\begin{eg}
The variables in the following example are given their initial values by
assignment.  Note that they could not be used before those assignments,
nor could \xcd`immu` be assigned repeatedly.
%~~gen ^^^ Vars70
%package Vars.For.Stars;
%abstract class VarsForStars{
% abstract def cointoss(): Boolean;
% abstract def println(Any):void;
%def check() {
%~~vis
\begin{xten}
  var muta : Int;
  // ERROR:  println(muta);
  muta = 4;
  val use2A = muta * 10;
  val immu : Int;
  // ERROR: println(immu);
  if (cointoss())   {immu = 1;}
  else              {immu = use2A;}
  val use2B = immu * 10;
  // ERROR: immu = 5;
\end{xten}
%~~siv
%}}
%~~neg
\end{eg}

Every class variable must be initialized before it is read, through
the execution of an explicit initializer. Every
instance variable must be initialized before it is read, through the
execution of an explicit or implicit initializer or a constructor.
Implicit initializers initialize \xcd`var`s to the default values of their
types (\Sref{DefaultValues}). Variables of types which do not have default
values are not implicitly initialized.



Each method and constructor parameter is initialized to the
corresponding argument value provided by the invoker of the method. An
exception-handling parameter is initialized to the object thrown by
the exception. A local variable must be explicitly given a value by
initialization or assignment, in a way that the compiler can verify
using the rules for definite assignment \Sref{sect:DefiniteAssignment}.


\section{Destructuring syntax}
\index{variable declarator!destructuring}
\index{destructuring}
\Xten{} permits a \emph{destructuring} syntax for local variable
declarations with explicit initializers,  and for formal parameters, of type \xcd`Point`, \Sref{point-syntax}. 
A point is a sequence of zero or more \xcd`Int`-valued coordinates.  
It is often useful to get at the coordinates directly, in variables. 

%##(VariableDeclarator
\begin{bbgrammar}
%(FROM #(prod:VariableDeclarator)#)
  VariableDeclarator \: Id HasResultType\opt \xcd"=" VariableInitializer & (\ref{prod:VariableDeclarator}) \\
                    \| \xcd"[" IdList \xcd"]" HasResultType\opt \xcd"=" VariableInitializer \\
                    \| Id \xcd"[" IdList \xcd"]" HasResultType\opt \xcd"=" VariableInitializer \\
\end{bbgrammar}
%##)

The syntax \xcdmath"val [a$_1$, $\ldots$, a$_n$] = e;" declares {$n$}
\xcd`Int` variables, bound to the first {$n$} components of the \xcd`Point` value of
\xcd`e`; it is an error if \xcd`e` is not a \xcd`Point` with at least {$n$} components.
The syntax \xcdmath"val p[a$_1$, $\ldots$, a$_n$] = e;"  is similar, but also
declares the variable \xcd`p` as \xcdmath"Point".  


\begin{ex}
The following code makes an anonymous point with one coordinate \xcd`11`, and
binds \xcd`i` to \xcd`11`.  Then it makes a point with coordinates \xcd`22`
and \xcd`33`, binds \xcd`p` to that point, and \xcd`j` and \xcd`k` to \xcd`22`
and \xcd`33` respectively.
%~~gen ^^^ Vars80
% package Vars.For.Glares;
% class DestructuringEx1 {
% def whyJustForLocals() {
%~~vis
\begin{xten}
val [i] : Point = Point.make(11);
val p[j,k] = Point.make(22,33);
val q[l,m] = [44,55]; // coerces an array to a point.
\end{xten}
%~~siv
%}}
%~~neg
 
%%A useful idiom for iterating over a range of numbers is: 
%%%~x~gen ^^^ Vars90
%%%package Vars.For.Swear.Bears;
%%% class ForBear {
%%% def forbear() {
%%%~x~vis
%%\begin{xten}
%%var sum : Int = 0;
%%for ([i] in 1..100) sum += i;
%%\end{xten}
%%%~x~siv
%%% ; } } 
%%%~x~neg
%%\noindent
%%The brackets in \xcd`[i]` introduce destructuring, making X10 treat \xcd`i`
%%as an \xcd`Int`; without them, it would be a \xcd`Point`.  

\end{ex}


\section{Formal parameters}
\label{sect:formal-parameters}
\index{formal parameter}
\index{parameter}


Formal parameters are the variables which hold values transmitted into a
method or function.  
They are always declared with a type.  (Type inference is not
available, because there is no single expression to deduce a type from.)
The variable name can be omitted if it is not to be used in the
scope of the declaration, as in the type of the method 
\xcd`static def main(Array[String]):void` executed at the start of a program that
does not use its command-line arguments.

\xcd`var` and \xcd`val` behave just as they do for local
variables, \Sref{local-variables}.  In particular, the following \xcd`inc`
method is allowed, but, unlike some languages, does {\em not} increment its
actual parameter.  \xcd`inc(j)` creates a new local 
variable \xcd`i` for the method call, initializes \xcd`i` with the value of
\xcd`j`, increments \xcd`i`, and then returns.  \xcd`j` is never changed.
%~~gen ^^^ Vars100
% package Vars.For.Squares.Of.Mares;
% class Ink {
%~~vis
\begin{xten}
static def inc(var i:Int) { i += 1; }
\end{xten}
%~~siv
%}
%~~neg


\section{Local variables and Type Inference}
\label{local-variables}
\index{variable!local}
\index{local variable}
Local variables are declared in a limited scope, and, dynamically, keep their
values only for so long as the scope is being executed.  They may be \xcd`var`
or \xcd`val`.  
They may have 
initializer expressions: \xcd`var i:Int = 1;` introduces 
a variable \xcd`i` and initializes it to 1.
If the variable is immutable
(\xcd"val")
the type may be omitted and
inferred from the initializer type (\Sref{TypeInference}).

The variable declaration \xcd`val x:T=e;` confirms that \xcd`e`'s value is of
type \xcd`T`, and then introduces the variable \xcd`x` with type \xcd`T`.  For
example, consider a class \xcd`Tub` with a property \xcd`p`.
%~~gen ^^^ Vars_Tub
% package Vars.Local;
%~~vis
\begin{xten}
class Tub(p:Int){
  def this(pp:Int):Tub{self.p==pp} {property(pp);}
  def example() {
    val t : Tub = new Tub(3);
  }
}
\end{xten}
%~~siv
%
%~~neg
\noindent
produces a variable \xcd`t` of type \xcd`Tub`, even though the expression
\xcd`new Tub(3)` produces a value of type \xcd`Tub{self.p==3}` -- that is, a
\xcd`Tub`  whose \xcd`p` field is 3.  This can be inconvenient when the
constraint information is required.

\index{\Xcd{<:}}
\index{documentation type declaration}
Including type information in variable declarations is generally good
programming practice: it explains to both the compiler and human readers
something of the intent of the variable.  However, including types in 
\xcd`val t:T=e` can obliterate helpful information.  So, X10 allows a {\em
documentation type declaration}, written \xcd`val t <: T = e`.  This 
has the same effect as \xcd`val t = e`, giving \xcd`t` the full type inferred
from \xcd`e`; but it also confirms statically that that type is at least
\xcd`T`.  

\begin{ex}The following gives \xcd`t` the type \xcd`Tub{self.p==3}` as
desired.  However, a similar declaration with an inappropriate type will fail
to compile.
%~~gen ^^^ Vars_Var_Bounded
% package Vars.Local.not.the.express.plz;
% class Tub(p:Int){
%   def this(pp:Int):Tub{self.p==pp} {property(pp);}
%   def example() {
%     val t : Tub = new Tub(3);
%   }
% }
% class TubBounded{
% def example() {
%~~vis
\begin{xten}
   val t <: Tub = new Tub(3);
   // ERROR: val u <: Int = new Tub(3);
\end{xten}
%~~siv
%}}
%~~neg

\end{ex}



\section{Fields}
\index{field}
\index{object!field}
\index{struct!field}
\index{class!field}

%##(FieldDeclarators FieldDecl FieldDeclarator HasResultType FieldKeyword Mod
\begin{bbgrammar}
%(FROM #(prod:FieldDeclarators)#)
    FieldDeclarators \: FieldDeclarator & (\ref{prod:FieldDeclarators}) \\
                    \| FieldDeclarators \xcd"," FieldDeclarator \\
%(FROM #(prod:FieldDecl)#)
           FieldDecl \: Mods\opt FieldKeyword FieldDeclarators \xcd";" & (\ref{prod:FieldDecl}) \\
                    \| Mods\opt FieldDeclarators \xcd";" \\
%(FROM #(prod:FieldDeclarator)#)
     FieldDeclarator \: Id HasResultType & (\ref{prod:FieldDeclarator}) \\
                    \| Id HasResultType\opt \xcd"=" VariableInitializer \\
%(FROM #(prod:HasResultType)#)
       HasResultType \: \xcd":" Type & (\ref{prod:HasResultType}) \\
                    \| \xcd"<:" Type \\
%(FROM #(prod:FieldKeyword)#)
        FieldKeyword \: \xcd"val" & (\ref{prod:FieldKeyword}) \\
                    \| \xcd"var" \\
%(FROM #(prod:Mod)#)
                 Mod \: \xcd"abstract" & (\ref{prod:Mod}) \\
                    \| Annotation \\
                    \| \xcd"atomic" \\
                    \| \xcd"final" \\
                    \| \xcd"native" \\
                    \| \xcd"private" \\
                    \| \xcd"protected" \\
                    \| \xcd"public" \\
                    \| \xcd"static" \\
                    \| \xcd"transient" \\
                    \| \xcd"clocked" \\
\end{bbgrammar}
%##)

Like most other kinds of variables in X10, 
the fields of an object can be either \xcd`val` or \xcd`var`. 
Fields can be \xcd`static` (\Sref{FieldDefinitions}).
Field declarations may have optional
initializer expressions, as for local variables, \Sref{local-variables}.
\xcd`var` fields without an initializer are initialized with the default value
of their type. \xcd`val` fields without an initializer must be initialized by
each constructor.


For \xcd`val` fields, as for \xcd`val` local variables, the type may be
omitted and inferred from the initializer type (\Sref{TypeInference}).
\xcd`var` fields, like \xcd`var` local variables, must be declared with a type.



%%GRAM%% \begin{grammar}
%%GRAM%% FieldDeclaration
%%GRAM%%         \: FieldModifier\star \xcd"var" FieldDeclaratorsWithType \\&& ( \xcd"," FieldDeclaratorsWithType )\star \\
%%GRAM%%         \| FieldModifier\star \xcd"val" FieldDeclarators \\&& ( \xcd"," FieldDeclarators )\star \\
%%GRAM%%         \| FieldModifier\star FieldDeclaratorsWithType \\&& ( \xcd"," FieldDeclaratorsWithType )\star \\
%%GRAM%% FieldDeclarators
%%GRAM%%         \: FieldDeclaratorsWithType \\
%%GRAM%%         \: FieldDeclaratorWithInit \\
%%GRAM%% FieldDeclaratorId
%%GRAM%%         \: Identifier  \\
%%GRAM%% FieldDeclaratorWithInit
%%GRAM%%         \: FieldDeclaratorId Init \\
%%GRAM%%         \| FieldDeclaratorId ResultType Init \\
%%GRAM%% FieldDeclaratorsWithType
%%GRAM%%         \: FieldDeclaratorId ( \xcd"," FieldDeclaratorId )\star ResultType \\
%%GRAM%% FieldModifier \: Annotation \\
%%GRAM%%                 \| \xcd"static" \\ \| \xcd`property` \\ \| \xcd`global` \\
%%GRAM%% \end{grammar}
%%GRAM%% 
%%GRAM%% 

%%ACC%%  \section{Accumulator Variables}
%%ACC%%  
%%ACC%%  Accumulator variables allow the accumulation of partial results to produce a
%%ACC%%  final result.  For example, an accumulator variable could compute a running
%%ACC%%  sum, product, maximum, or minimum of a collection of numbers.  In particular,
%%ACC%%  many concurrent activites can accumulate safely into the {\em same} local
%%ACC%%  variable, without need for \Xcd{atomic} blocks or other explicit coordination.  
%%ACC%%  
%%ACC%%  An accumulator variable is associated with a {\em reducer}, which explains how
%%ACC%%  new partial values are accumulated.
%%ACC%%  
%%ACC%%  \subsection{Reducers}
%%ACC%%  
%%ACC%%  A notion of accumulation has two aspects: 
%%ACC%%  \begin{enumerate}
%%ACC%%  \item A {\bf zero} value, which is the initial value of the accumulator,
%%ACC%%        before any partial results have been included.  When accumulating a sum,
%%ACC%%        the zero value is \Xcd{0}; when accumulating a product, it is \Xcd{1}.
%%ACC%%  \item A {\bf combining function}, explaining how to combine two partial
%%ACC%%        accumulations into a whole one.  When accumulating a sum, partial sums
%%ACC%%        should be added together; for a product, they should be multiplied.  
%%ACC%%  \end{enumerate}
%%ACC%%  
%%ACC%%  In X10, this is represented as a value of type
%%ACC%%  \Xcd{x10.lang.Reducer[T]}: 
%%ACC%%  %~acc~gen
%%ACC%%  %package Vars.Notx10lang.Reducerererer;
%%ACC%%  %~acc~vis
%%ACC%%  \begin{xten}
%%ACC%%  struct Reducer[T](zero:T, apply: (T,T)=>T){}
%%ACC%%  \end{xten}
%%ACC%%  %~acc~siv
%%ACC%%  %
%%ACC%%  %~acc~neg
%%ACC%%  \noindent 
%%ACC%%  If \Xcd{r:Reducer[T]}, then \Xcd{r.zero} is the zero element, and
%%ACC%%  \Xcd{r(a,b)} --- which can also be written \Xcd{r.apply(a,b)} --- is the
%%ACC%%  combination of \Xcd{a} and \Xcd{b}.
%%ACC%%  
%%ACC%%  For example, the reducers for adding and multiplying integers are: 
%%ACC%%  %~acc~gen
%%ACC%%  %package Vars.Notx10lang.Reducererererererer;
%%ACC%%  %struct Reducer[T](zero:T, apply: (T,T)=>T){}
%%ACC%%  %class Example{
%%ACC%%  %~acc~vis
%%ACC%%  \begin{xten}
%%ACC%%  val summer = Reducer[Int](0, Int.+);
%%ACC%%  val producter = Reducer[Int](1, Int.*);
%%ACC%%  \end{xten}
%%ACC%%  %~acc~siv
%%ACC%%  %}
%%ACC%%  %~acc~neg
%%ACC%%  
%%ACC%%  
%%ACC%%  Reduction is guaranteed to be deterministic if the reducer is {\em
%%ACC%%  Abelian},\footnote{This term is borrowed from abstract algebra, where such a
%%ACC%%  reducer, together with its type, forms an Abelian monoid.}
%%ACC%%  that is, 
%%ACC%%  \begin{enumerate}
%%ACC%%  \item \Xcd{r.apply} is pure; that is, has no side effects;
%%ACC%%  \item \Xcd{r.apply} is commutative; that is, \Xcd{r(a,b) == r(b,a)} for all
%%ACC%%        inputs \Xcd{a} and \Xcd{b};
%%ACC%%  \item \Xcd{r.apply} is associative; that is, 
%%ACC%%        \Xcd{r(a,r(b,c)) == r(r(a,b),c)} for all \Xcd{a}, \Xcd{b}, and \Xcd{c}.
%%ACC%%  \item \Xcd{r.zero} is the identity element for \Xcd{r.apply}; that is, 
%%ACC%%        \Xcd{r(a, r.zero) == a}
%%ACC%%        for all \Xcd{a}.
%%ACC%%  \end{enumerate}
%%ACC%%  
%%ACC%%  
%%ACC%%  
%%ACC%%  
%%ACC%%  \Xcd{summer} and \Xcd{producter} satisfy all these conditions, and give
%%ACC%%  determinate reductions. The compiler does not require or check these, though.
%%ACC%%  
%%ACC%%  
%%ACC%%  \subsection{Accumulators}
%%ACC%%  
%%ACC%%  If \Xcd{r} is a  value of type \Xcd{Reducer[T]}, then an accumulator of type
%%ACC%%  \Xcd{T} using \Xcd{r} is declared as:
%%ACC%%  %~accTODO~gen
%%ACC%%  % package Vars.Accumulators.Basic.Little.Idea;
%%ACC%%  % class C[T]{
%%ACC%%  % static def example (r:Reducer[T]) {
%%ACC%%  %~accTODO~vis
%%ACC%%  \begin{xten}
%%ACC%%  acc(r) x : T;
%%ACC%%  acc(r) y; 
%%ACC%%  \end{xten}
%%ACC%%  %~accTODO~siv
%%ACC%%  %
%%ACC%%  %~accTODO~neg
%%ACC%%  The type declaration \Xcd{T} is optional; if specified, it must be the same
%%ACC%%  type that the reducer \Xcd{r} uses.
%%ACC%%  
%%ACC%%  \subsection{Sequential Use of Accumulators}
%%ACC%%  
%%ACC%%  The sequential use of accumulator variables is straightforward, and could be
%%ACC%%  done as easily without accumulators.  (The power of accumulators is in their
%%ACC%%  concurrent use, \Sref{ConcurrentUseOfAccumulators}.)
%%ACC%%  
%%ACC%%  A variable declared as \Xcd{acc(r) x:T;} is initialized to \Xcd{r.zero}.  
%%ACC%%  
%%ACC%%  Assignment of values of \Xcd{acc} variables has nonstandard semantics.
%%ACC%%  \Xcd{x = v;} causes the value \Xcd{r(v,x)} to be stored in \Xcd{x} --- in
%%ACC%%  particular, {\em not} the value of \Xcd{v}.
%%ACC%%  
%%ACC%%  Reading a value from an accumulator retrieves the current accumulation.
%%ACC%%  
%%ACC%%  For example, the sum and product of a list \Xcd{L} of integers can be computed
%%ACC%%  by: 
%%ACC%%  %~accTODO~gen
%%ACC%%  %package Vars.Accumulators.Are.For.Bisimulators;
%%ACC%%  % import java.util.*;
%%ACC%%  % class Example{
%%ACC%%  % static def example(L: List[Int]) {
%%ACC%%  %~accTODO~vis
%%ACC%%  \begin{xten}
%%ACC%%  val summer = Reducer[Int](0, Int.+);
%%ACC%%  val producter = Reducer[Int](1, Int.*);
%%ACC%%  acc(summer) sum;
%%ACC%%  acc(producter) prod;
%%ACC%%  for (x in L) {
%%ACC%%    sum = x;
%%ACC%%    prod = x;
%%ACC%%  }
%%ACC%%  x10.io.Console.OUT.println("Sum = " + sum + "; Product = " + prod);
%%ACC%%  \end{xten}
%%ACC%%  %~accTODO~siv
%%ACC%%  %
%%ACC%%  %~accTODO~neg
%%ACC%%  
%%ACC%%  
%%ACC%%  
%%ACC%%  \subsection{Concurrent Use of Accumulators}
%%ACC%%  \label{ConcurrentUseOfAccumulators}
%%ACC%%  \index{accumulator!and activities}
%%ACC%%  
%%ACC%%  Accumulator variables are restricted and synchronized in ways that make them
%%ACC%%  ideally suited for concurrent accumulation of data.   The {\em governing
%%ACC%%  activity} of an accumulator is the activity in which the \Xcd{acc} variable is
%%ACC%%  declared.  
%%ACC%%  
%%ACC%%  \begin{enumerate}
%%ACC%%  \item The governing activity can read the accumulator at any point that it has
%%ACC%%        no running sub-activities.  
%%ACC%%  \item Any activity that has lexical access to the accumulator can write to it.  
%%ACC%%        All
%%ACC%%        writes are performed atomically, without need for \Xcd{atomic} or other
%%ACC%%        concurrency control.
%%ACC%%  \end{enumerate}
%%ACC%%  
%%ACC%%  If the reducer is Abelian, this guarantees that \Xcd{acc} variables cannot
%%ACC%%  cause race conditions; the result of such a computation is determinate,
%%ACC%%  independent of the scheduling of activities. Read-read conflicts are
%%ACC%%  impossible, as only a single activity, the governing activity, can read the
%%ACC%%  \Xcd{acc} variable. Read-write conflicts are impossible, as reads are only
%%ACC%%  allowed at points where the only activity which can refer to the \Xcd{acc}
%%ACC%%  variable is the governing activity. Two activities may try to write the
%%ACC%%  \Xcd{acc} variable at the same time. The writes are performed atomically, so
%%ACC%%  they behave as if they happened in some (arbitrary) order---and, because the
%%ACC%%  reducer is Abelian, the order of writes doesn't matter.
%%ACC%%  
%%ACC%%  If the reducer is not Abelian---\eg, it is accumulating a string result by
%%ACC%%  concatenating a lot of partial strings together---the result is indeterminate.
%%ACC%%  However, because the accumulator operations are atomic, it will be the result
%%ACC%%  of {\em some} combination of the individual elements by the reduction
%%ACC%%  operation, \eg, the concatenation of the partial strings in {\em some} order.  
%%ACC%%  
%%ACC%%  
%%ACC%%  
%%ACC%%  For example, the following code computes triangle numbers {$\sum_{i=1}^{n}i$}
%%ACC%%  concurrently.\footnote{This program is highly inefficient. Even ignoring the
%%ACC%%    constant-time formula {$\sum_{i=1}^{n}i = \frac{n(n+1)}{2}$}, this program
%%ACC%%    incurs the cost of starting {$n$} activities and coordinating {$n$} accesses
%%ACC%%    to the accumulator. Accumulator variables are of most value in multi-place,
%%ACC%%    multi-core computations.}
%%ACC%%  
%%ACC%%  
%%ACC%%  %~accTODO~gen
%%ACC%%  %package Vars.Accumulator.Concurrency.Example;
%%ACC%%  %class Example{
%%ACC%%  %
%%ACC%%  %~accTODO~vis
%%ACC%%  \begin{xten}
%%ACC%%  def triangle(n:Int) {
%%ACC%%    val summer = Reducer[Int](0, Int.+);
%%ACC%%    acc(summer) sum; 
%%ACC%%    finish {
%%ACC%%      for([i] in 1..n) async {
%%ACC%%        sum = i;  // (A)
%%ACC%%      }
%%ACC%%      // (C)
%%ACC%%    }
%%ACC%%    return sum; // (B)
%%ACC%%  }
%%ACC%%  \end{xten}
%%ACC%%  %~accTODO~siv
%%ACC%%  %}
%%ACC%%  %~accTODO~neg
%%ACC%%  
%%ACC%%  The governing activity of the \Xcd{acc} variable \Xcd{sum} is the activity
%%ACC%%  including the body of \Xcd{triangle}.  It starts up \Xcd{n} sub-activities,
%%ACC%%  each of which adds one value to \Xcd{sum} at point \Xcd{(A)}.  Note that these
%%ACC%%  activities cannot {\em read} the value of \Xcd{sum}---only the governing
%%ACC%%  activity can do that---but they can update it.  
%%ACC%%  
%%ACC%%  At point \Xcd{(B)}, \Xcd{triangle} returns the value in \Xcd{sum}. It is
%%ACC%%  clear, from the \Xcd{finish} statement, that all sub-activities started by the
%%ACC%%  governing process have finished at this point. X10 forbids reading of
%%ACC%%  \Xcd{sum}, even by the governing process, at point \Xcd{(C)}, since
%%ACC%%  sub-activities writing into it could still be active when the governing
%%ACC%%  activity reaches this point.  The \Xcd{return sum;} statement could not be
%%ACC%%  moved to \Xcd{(C)}, which is good, because the program would be wrong if it
%%ACC%%  were there.
%%ACC%%  
%%ACC%%  
%%ACC%%  
%%ACC%%  
%%ACC%%  \subsubsection{Accumulators and Places}
%%ACC%%  \index{accumulator!and places} Activity variables can be read and written from
%%ACC%%  any place, without need for \Xcd{GlobalRef}s. We may spread the previous
%%ACC%%  computation out among all the available processors by simply sticking in an
%%ACC%%  \Xcd{at(...)} statement at point \Xcd{(D)}, without need for other
%%ACC%%  restructuring of the program.
%%ACC%%  
%%ACC%%  %~accTODO~gen
%%ACC%%  %package Vars.Accumulator.Concurrency.Example.Multiplacey;
%%ACC%%  %class Example{
%%ACC%%  %~accTODO~vis
%%ACC%%  \begin{xten}
%%ACC%%  def triangle(n:Int) {
%%ACC%%    val summer = Reducer[Int](0, Int.+);
%%ACC%%    acc(summer) sum; 
%%ACC%%    finish {
%%ACC%%      for([i] in 1..n) async 
%%ACC%%        at(Places.place(i % Places.MAX_PLACES) { //(D)
%%ACC%%          sum = i;  // (A)
%%ACC%%      }
%%ACC%%    }
%%ACC%%    return sum; // (B)
%%ACC%%  }
%%ACC%%  \end{xten}
%%ACC%%  %~accTODO~siv
%%ACC%%  %}
%%ACC%%  %~accTODO~neg
%%ACC%%  
%%ACC%%  \subsubsection{Accumulator Parameters}
%%ACC%%  \index{accumulator variables!as parameters}
%%ACC%%  \index{parameters!accumulator}
%%ACC%%  
%%ACC%%  Accumulators can be passed to methods and closures, by giving the keyword 
%%ACC%%  \Xcd{acc} instead of \Xcd{var} or \Xcd{val}.  Reducers are not specified; each
%%ACC%%  accumulator comes with its own reducer.  However, the type \Xcd{T} of the
%%ACC%%  accumulator {\em is} required.
%%ACC%%  
%%ACC%%  For example, the following method takes a list of numbers, and accumulates
%%ACC%%  those that are divisible by 2 in \Xcd{evens}, and those that are divisible by
%%ACC%%  3 in \Xcd{triples}: 
%%ACC%%  %~accTODO~gen
%%ACC%%  %package Vars.accumulators.parameters.oscillators.convulsitors.proximators;
%%ACC%%  %import x10.util.*;
%%ACC%%  %class Whatever {
%%ACC%%  %~accTODO~vis
%%ACC%%  \begin{xten}
%%ACC%%  static def split23(L:List[Int], acc evens:Int, acc triples:Int) {
%%ACC%%    for(n in L) {
%%ACC%%       if (n % 2 == 0) evens = n;
%%ACC%%       if (n % 3 == 0) triples = n;
%%ACC%%    }
%%ACC%%  }
%%ACC%%  static val summer = Reducer[Int](0, Int.+);
%%ACC%%  static val producter = Reducer[Int](1, Int.*);
%%ACC%%  static def sumEvenPlusProdTriple(L:List[Int]) {
%%ACC%%    acc(summer) sumEven;
%%ACC%%    acc(producter) prodTriple;
%%ACC%%    split23(L, sumEven, prodTriple);
%%ACC%%    return sumEven + prodTriple;
%%ACC%%  }
%%ACC%%  \end{xten}
%%ACC%%  %~accTODO~siv
%%ACC%%  %}
%%ACC%%  %~accTODO~neg
%%ACC%%  
%%ACC%%  \subsection{Indexed Accumulators}
%%ACC%%  \index{accumulator!indexed}
%%ACC%%  \index{accumulator!array}
%%ACC%%  
%%ACC%%  
%%ACC%%  \noo{Define this!}
%%ACC%%  
%%ACC%%  %~accTODO~gen
%%ACC%%  % package Vars.Indexed.Accumulators;
%%ACC%%  %~accTODO~vis
%%ACC%%  \begin{xten}
%%ACC%%  class BoolAccum implements SelfAccumulator[Boolean, Int] {
%%ACC%%    var sumTrue = 0, sumFalse = 0;
%%ACC%%    def update(k:Boolean, v:Int) { 
%%ACC%%       if (k) sumTrue += k; else sumFalse += k;
%%ACC%%    }
%%ACC%%    def update(ks:Array[Boolean]{rail}, vs:Array[Int]{ks.size == vs.size}) {
%%ACC%%       for([i] in ks.region) update(ks(i), vs(i));  }
%%ACC%%    
%%ACC%%  }
%%ACC%%  \end{xten}
%%ACC%%  %~accTODO~siv
%%ACC%%  %
%%ACC%%  %~accTODO~neg
