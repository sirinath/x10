\subsection{Object Initialization}
\label{ObjectInitialization}
\index{initialization}
\index{constructor}
\index{object!constructor}
\index{struct!constructor}

\noo{Confirm this chapter with the paper}

X10 does object initialization safely.  It avoids certain bad things which
trouble some other languages:
\begin{enumerate}
\item Use of a field before the field has been initialized.
\item A program reading two different values from a \xcd`val` field of a
      container. 
\item \Xcd{this} escaping from a constructor, which can cause problems as
      noted below. 

\end{enumerate}

It should be unsurprising that fields must not be used before they are
initialized. At best, it is uncertain what value will be in them, as in
\Xcd{x} below. Worse, the value might not even be an allowable value; \Xcd{y},
declared to be nonzero in the following example, might be zero before it is
initialized.
\begin{xten}
// Not correct X10
class ThisIsWrong {
  val x : Int;
  val y : Int{y != 0};
  def this() {
    x10.io.Console.OUT.println("x=" + x + "; y=" + y);
    x = 1; y = 2;
  }
}
\end{xten}

One particularly insidious way to read uninitialized fields is to allow
\Xcd{this} to escape from a constructor. For example, the constructor could
put \Xcd{this} into a data structure before initializing it, and another
activity could read it from the data structure and look at its fields:
\begin{xten}
class Wrong {
  val shouldBe8 : Int;
  static Cell[Wrong] wrongCell = new Cell[Wrong]();
  static def doItWrong() {
     finish {
       async { new Wrong(); } // (A)
       assert( wrongCell().shouldBe8 == 8); // (B)
     }
  }
  def this() {
     wrongCell.set(this); // (C) - ILLEGAL
     this.shouldBe8 = 8; // (D)
  }
}
\end{xten}
\noindent
In this example, the underconstructed \Xcd{Wrong} object is leaked into a
storage cell at line \Xcd{(C)}, and then initialized.  The \Xcd{doItWrong}
method constructs a new \Xcd{Wrong} object, and looks at the \Xcd{Wrong}
object in the storage cell to check on its \Xcd{shouldBe8} field.  One
possible order of events is the following:
\begin{enumerate}
\item \Xcd{doItWrong()} is called.
\item \Xcd{(A)} is started.  Space for a new \Xcd{Wrong} object is allocated.
      Its \Xcd{shouldBe8} field, not yet initialized, contains some garbage
      value.
\item \Xcd{(C)} is executed, as part of the process of constructing a new
      \Xcd{Wrong} object.  The new, uninitialized object is stored in
      \Xcd{wrongCell}.
\item Now, the initialization activity is paused, and execution of the main activity
      proceeds from \Xcd{(B)}.
\item The value in \Xcd{wrongCell} is retrieved, and is \Xcd{shouldBe8} field
      is read.  This field contains garbage, and the assertion fails.
\item Now let the initialization activity proceed with \Xcd{(D)},
      initializing \Xcd{shouldBe8} --- too late.
\end{enumerate}

The \xcd`at` statement (\Sref{AtStatement}) introduces the potential for
escape as well. The following class prints an uninitialized value: 
%~~gen ^^^ ThisEscapingViaAt_MustFailCompile
% package ObjInit_at;
% NOCOMPILE
%~~vis
\begin{xten}
class Example {
  val a: Int;
  def this() { 
    at(here.next()) {
      // Recall that 'this' is a copy of 'this' outside 'at'.
      Console.OUT.println("this.a = " + this.a);
    }
    this.a = 1;
  }
}
\end{xten}
%~~siv
%
%~~neg


X10 must protect against such possibilities.  The rules explaining how
constructors can be written are somewhat intricate; they are designed to allow
as much programming as possible without leading to potential problems.
Ultimately, they simply are elaborations of the fundamental principles that
uninitialized fields must never be read, and \Xcd{this} must never be leaked.

%%RAW%% \subsection{Raw and Cooked Objects}
%%RAW%% \index{raw}
%%RAW%% \index{cooked}
%%RAW%% 
%%RAW%% An object is {\em raw} before its constructor ends, and {\em cooked} after its
%%RAW%% constructor ends. Note that, when an object is cooked, all its subobjects are
%%RAW%% cooked.  
%%RAW%% 



\subsection{Constructors and Non-Escaping Methods}
\index{non-escaping}
\label{sect:nonescaping}

In general, constructors must not be allowed to call methods with \Xcd{this} as
an argument or receiver. Such calls could leak references to \Xcd{this},
either directly from a call to \Xcd{cell.set(this)}, or indirectly because
\Xcd{toString} leaks \Xcd{this}, and the concatenation
\Xcd`"Escaper = "+this` calls \Xcd{toString}.\footnote{This is abominable behavior for
\Xcd{toString}, but it cannot be prevented -- save by a scheme such as we
present in this section.}
%~WRONG~gen
%package ObjectInit.CtorAndNonEscaping.One;
%~WRONG~vis
\begin{xten}
// This code violates this chapter's constraints
// and thus will not compile in X10.
class Escaper {
  static val Cell[Escaper] cell = new Cell[Escaper]();
  def toString() {
    cell.set(this);
    return "Evil!";
  }
  def this() {
    cell.set(this);
    x10.io.Console.OUT.println("Escaper = " + this);
  }
}
\end{xten}
%~WRONG~siv
%
%~WRONG~neg

However, it is convenient to be able to call methods from constructors; {\em
e.g.}, a class might have eleven constructors whose common behavior is best
described by three methods.
Under certain stringent conditions, it {\em is}
safe to call a method: the method called must not leak references to
\Xcd{this}, and must not read \Xcd{val}s or \Xcd{var}s which might not have
been assigned.

So, X10 performs a static dataflow analysis, sufficient to guarantee that
method calls in constructors are safe.  This analysis requires having access
to or guarantees about all the code that could possibly be called.  This can
be accomplished in two ways:
\begin{enumerate}
\item Ensuring that only code from the class itself can be called, by
      forbidding overriding of
      methods called from the constructor: they can be marked \Xcd{final} or
      \Xcd{private}, or the whole class can be \Xcd{final}.
\item Marking the methods called from the constructor by
      \xcd`@NonEscaping`.
\end{enumerate}

\subsubsection{Non-Escaping Methods}
\index{method!non-escaping}
\index{method!implicitly non-escaping}
\index{method!NonEscaping}
\index{implicitly non-escaping}
\index{non-escaping}
\index{non-escaping!implicitly}
\index{NonEscaping}


A method may be annotated with \xcd`@NonEscaping`.  This
imposes several restrictions on the method body, and on all methods overriding
it.  However, it is the only way that a method can be called from
constructors.  The
\Xcd{@NonEscaping} annotation makes explicit all the X10 compiler's needs for
constructor-safety.

A method can, however, be safe to call from constructors without being marked
\Xcd{@NonEscaping}. We call such methods {\em implicitly non-escaping}.
Implicitly non-escaping methods need to obey the same constraints on
\Xcd{this}, \Xcd{super}, and variable usage as \Xcd{@NonEscaping} methods. An
implicitly non-escaping method {\em could} be marked as
\xcd`@NonEscaping` for some list of variables; the compiler, in
effect, infers the annotation. In addition, implicitly non-escaping methods
must be \Xcd{private} or \Xcd{final} or members of a \Xcd{final} class; this
corresponds to the hereditary nature of \Xcd{@NonEscaping} (by forbidding
inheritance of implicitly non-escaping methods).

We say that a method is {\em non-escaping} if it is either implicitly
non-escaping, or annotated \Xcd{@NonEscaping}.

The first requirement on non-escaping methods is that they do not allow
\Xcd{this} to escape. Inside of their bodies, \Xcd{this} and \Xcd{super} may
only be used for field access and assignment, and as the receiver of
non-escaping methods.

Finally, if a method \Xcd{m} in class \Xcd{C} is marked
\xcd`@NonEscaping`, then every method which overrides \Xcd{m} in any
subclass of \Xcd{C} must be annotated with precisely the same annotation,
\xcd`@NonEscaping`, as well.

The following example uses most of the possible variations (leaving out
\Xcd{final} class).  \Xcd{aplomb()} explicitly forbids reading any field but
\Xcd{a}. \Xcd{boric()} is called after \Xcd{a} and \Xcd{b} are set, but
\Xcd{c} is not.
The \xcd`@NonEscaping` annotation on \xcd`boric()` is optional, but the
compiler will print a warning if it is left out.
\Xcd{cajoled()} is only called after all fields are set, so it
can read anything; its annotation, too, is not required.   \Xcd{SeeAlso} is able to override \Xcd{aplomb()}, because
\Xcd{aplomb()} is \xcd`@NonEscaping("a")`; it cannot override the final method
\Xcd{boric()} or the private one \Xcd{cajoled()}.  
The method \Xcd{aplomb()} can be overridden, but it is crucial that the 
subclass's version, 
\Xcd{SeeAlso.aplomb()}, be 
declared \xcd`@NonEscaping("a")`, just like \Xcd{C2.aplomb()}.
%~~gen ^^^ ObjectInitialization10
%package ObjInit.C2;
%~~vis
\begin{xten}
import x10.compiler.*;

final class C2 {
  protected val a:Int, b:Int, c:Int;
  protected var x:Int, y:Int, z:Int;
  def this() {
    a = 1;
    this.aplomb();
    b = 2;
    this.boric();
    c = 3;
    this.cajoled();
  }
  @NonEscaping def aplomb() {
    x = a;
    // this.boric(); // not allowed; boric reads b.
    // z = b; // not allowed -- only 'a' can be read here
  }
  @NonEscaping final def boric() {
    y = b;
    this.aplomb(); // allowed; 
       // a is definitely set before boric is called
    // z = c; // not allowed; c is not definitely written
  }
  @NonEscaping private def cajoled() {
    z = c;
  }
}

\end{xten}
%~~siv
%
%~~neg



\subsection{Fine Structure of Constructors}
\label{SFineStructCtors}

The code of a constructor consists of four segments, three of them optional
and one of them implicit.
\begin{enumerate}
\item The first segment is an optional call to \Xcd{this(...)} or
      \Xcd{super(...)}.  If this is supplied, it must be the first statement
      of the constructor.  If it is not supplied, the compiler treats it as a
      nullary super-call \Xcd{super()};
\item If the class or struct has properties, there must be a single
      \Xcd{property(...)} command in the constructor.  Every execution path
      through the constructor must go through this \Xcd{property(...)} command
      precisely once.   The second segment of the constructor is the code
      following the first segment, up to and including the \Xcd{property()}
      statement.

      If the class or struct has no properties, the \Xcd{property()} call must
      be omitted. If it is present, the second segment is defined as before.  If
      it is absent, the second segment is empty.
\item The third segment is automatically generated.  Fields with initializers
      are initialized immediately after the \Xcd{property} statement.
      In the following example, \Xcd{b} is initialized to \Xcd{y*9000} in
      segment three.  The initialization makes sense and does the right
      thing; \Xcd{b} will be \Xcd{y*9000} for every \Xcd{Overdone} object.
      (This would not be possible if field initializers were processed
      earlier, before properties were set.)
\item The fourth segment is the remainder of the constructor body.
\end{enumerate}

The segments in the following code are shown in the comments.
%~~gen ^^^ ObjectInitialization20
% package ObjectInitialization.ShowingSegments;
%~~vis
\begin{xten}
class Overlord(x:Int) {
  def this(x:Int) { property(x); }
}//Overlord
class Overdone(y:Int) extends Overlord  {
  val a : Int;
  val b =  y * 9000;
  def this(r:Int) {
    super(r);                      // (1)
    x10.io.Console.OUT.println(r); // (2)
    val rp1 = r+1;
    property(rp1);                 // (2)
    // field initializations here  // (3)
    a = r + 2;                     // (4)
  }
}//Overdone
\end{xten}
%~~siv
%
%~~neg

The rules of what is allowed in the three segments are different, though
unsurprising.  For example, properties of the current class can only be read
in segment 3 or 4---naturally, because they are set at the end of segment 2.

\subsubsection{Initialization and Inner Classses}
\index{constructor!inner classes in}

Constructors of inner classes are tantamount to method calls on \Xcd{this}.
For example, the constructor for Inner {\bf is} acceptable.  It does not leak
\Xcd{this}.  It leaks \Xcd{Outer.this}, which is an utterly different object.
So, the call to \Xcd{this.new Inner()} in the \Xcd{Outer} constructor {\em
is} illegal.  It would leak \Xcd{this}.  There is no special rule in effect
preventing this; a constructor call of an inner class is no
different from a method as far as leaking is concerned.
%~~gen ^^^ ObjectInitialization30
% package ObjInit.InnerClass; 
% NOTEST
%~~vis
\begin{xten}
class Outer {
  static val leak : Cell[Outer] = new Cell[Outer](null);
  class Inner {
     def this() {Outer.leak.set(Outer.this);}
  }
  def /*Outer*/this() {
     //ERROR: val inner = this.new Inner();
  }
}
\end{xten}
%~~siv
%
%~~neg



\subsubsection{Initialization and Closures}
\index{constructor!closure in}

Closures in constructors may not refer to \xcd`this`.  They may not even refer
to fields of \xcd`this` that have been initialized.   For example, the
closure \xcd`bad1` is not allowed because it refers to \xcd`this`; \xcd`bad2`
is not allowed because it mentions \xcd`a` --- which is, of course, identical
to \xcd`this.a`. 

%%-deleted-%% valid if they were invoked (or inlined) at the
%%-deleted-%%place of creation. For example, \Xcd{closure} below is acceptable because it
%%-deleted-%%only refers to fields defined at the point it was written.  \Xcd{badClosure}
%%-deleted-%%would not be acceptable, because it refers to fields that were not defined at
%%-deleted-%%that point, although they were defined later.
%~~gen ^^^ ObjectInitialization40
% package ObjectInitialization.Closures; 
%~~vis
\begin{xten}
class C {
  val a:Int;
  def this() {
    this.a = 1;
    //ERROR: val bad1 = () => this; 
    //ERROR: val bad2 = () => a*10;
  }
}
\end{xten}
%~~siv
%
%~~neg


\subsection{Definite Initialization in Constructors}


An instance field \Xcd{var x:T}, when \Xcd{T} has a default value, need not be
explicitly initialized.  In this case, \Xcd{x} will be initialized to the
default value of type \Xcd{T}.  For example, a \Xcd{Score} object will have
its \Xcd{currently} field initialized to zero, below:
%~~gen ^^^ ObjectInitialization50
% package ObjectInit.DefaultInit;
%~~vis
\begin{xten}
class Score {
  public var currently : Int;
}
\end{xten}
%~~siv
%
%~~neg

All other sorts of instance fields do need to be initialized before they can
be used.  \Xcd{val} fields must be initialized, even if their type has a
default value.  It would be silly to have a field \Xcd{val z : Int} that was
always given default value of \Xcd{0} and, since it is \Xcd{val}, can never be
changed.  \Xcd{var} fields whose type has no default value must be initialized
as well, such as \xcd`var y : Int{y != 0}`, since it cannot be assigned a
sensible initial value.

The fundamental principles are:
\begin{enumerate}
\item \Xcd{val} fields must be assigned precisely once in each constructor on every
possible execution path.
\item \Xcd{var} fields of defaultless type must be
assigned at least once on every possible execution path, but may be assigned
more than once.
\item No variable may be read before it is guaranteed to have been
assigned.
\item Initialization may be by field initialization expressions (\Xcd{val x :
      Int = y+z}), or by uninitialized fields \Xcd{val x : Int;} plus
an initializing assignment \Xcd{x = y+z}.  Recall that field initialization
expressions are performed after the \Xcd{property} statement, in segment 3 in
the terminology of \Sref{SFineStructCtors}.
\end{enumerate}



\subsection{Summary of Restrictions on Classes and Constructors}

The following table tells whether a given feature is (yes), is not (no) or is
with some conditions (note) allowed in a given context.   For example, a
property method is allowed with the type of another property, as long as it
only mentions the preceding properties. The first column of the table gives
examples, by line of the following code body.

\begin{tabular}{||l|l|c|c|c|c|c|c||}
\hline
~
  & {\bf Example}
  & {\bf Prop.}
  & {\bf {\tt \small self==this}(1)}
  & {\bf Prop.Meth.}
  & {\bf {\tt this}}
  & {\bf {fields}}
\\\hline
Type of property
  & (A)
  & %?properties
    yes (2)
  & no %? self==this
  & no %? property methods
  & no %? this may be used
  & no %? fields may be used
\\\hline
Class Invariant
  & (B)
  & yes %?properties
  & yes %? self==this
  & yes %? property methods
  & yes %? this may be used
  & no %? fields may be used
\\\hline
Supertype (3)
  & (C), (D)
  & yes%?properties
  & yes%? self==this
  & yes%? property methods
  & no%? this may be used
  & no%? fields may be used
\\\hline
Property Method Body
  & (E)
  & yes %?properties
  & yes %? self==this
  & yes %? property methods
  & yes %? this may be used
  & no %? fields may be used
\\\hline

Static field (4)
  & (F) (G)
  & no %?properties
  & no %? self==this
  & no %? property methods
  & no %? this may be used
  & no %? fields may be used
\\\hline

Instance field (5)
  & (H), (I)
  & yes %?properties
  & yes %? self==this
  & yes %? property methods
  & yes %? this may be used
  & yes %? fields may be used
\\\hline

Constructor arg. type
  & (J)
  & no %?properties
  & no %? self==this
  & no  %? property methods
  & no %? this may be used
  & no %? fields may be used
\\\hline

Constructor guard
  & (K)
  & no %?properties
  & no %? self==this
  & no %? property methods
  & no %? this may be used
  & no %? fields may be used
\\\hline

Constructor ret. type
  & (L)
  & yes %?properties
  & yes %? self==this
  & yes %? property methods
  & yes %? this may be used
  & yes %? fields may be used
\\\hline

Constructor segment 1
  & (M)
  & no%?properties
  & yes%? self==this
  & no%? property methods
  & no%? this may be used
  & no%? fields may be used
\\\hline


Constructor segment 2
  & (N)
  & no%?properties
  & yes%? self==this
  & no%? property methods
  & no%? this may be used
  & no%? fields may be used
\\\hline

Constructor segment 4
  & (O)
  & yes%?properties
  & yes%? self==this
  & yes%? property methods
  & yes%? this may be used
  & yes%? fields may be used
\\\hline

Methods
  & (P)
  & yes %?properties
  & yes %? self==this
  & yes %? property methods
  & yes %? this may be used
  & yes %? fields may be used
\\\hline



\iffalse
place
  & (pos)
  & %?properties
  & %? self==this
  & %? property methods
  & %? this may be used
  & %? fields may be used
\\\hline
\fi
\end{tabular}

Details:

\begin{itemize}
\item (1) {Top-level {\tt self} only.}
\item (2) {The type of the {$i^{th}$} property may only mention
                 properties {$1$} through {$i$}.}
\item (3) Super-interfaces follow the same rules as supertypes.
\item (4) The same rules apply to types and initializers.
\end{itemize}



The example indices refer to the following code:
%~~gen ^^^ ObjectInitialization60
% package ObjectInit.GrandExample;
% class Supertype[T]{}
% interface SuperInterface[T]{}
%~~vis
\begin{xten}
class Example (
   prop : Int,
   proq : Int{prop != proq},                    // (A)
   pror : Int
   )
   {prop != 0}                                  // (B)
   extends Supertype[Int{self != prop}]         // (C)
   implements SuperInterface[Int{self != prop}] // (D)
{
   property def propmeth() = (prop == pror);    // (E)
   static staticField
      : Cell[Int{self != 0}]                    // (F)
      = new Cell[Int{self != 0}](1);            // (G)
   var instanceField
      : Int {self != prop}                      // (H)
      = (prop + 1) as Int{self != prop};        // (I)
   def this(
      a : Int{a != 0},
      b : Int{b != a}                           // (J)
      )
      {a != b}                                  // (K)
      : Example{self.prop == a && self.proq==b} // (L)
   {
      super();                                  // (M)
      property(a,b,a);                          // (N)
      // fields initialized here
      instanceField = b as Int{self != prop};   // (O)
   }

   def someMethod() =
        prop + staticField + instanceField;     // (P)
}
\end{xten}
%~~siv
%
%~~neg
