\chapter{Expressions}\label{XtenExpressions}\index{expression}

\Xten{} has a rich expression language.
Evaluating an expression produces a value, or, in a few cases, no value. 
Expression evaluation may have side effects, such as change of the value of a 
\xcd`var` variable or a data structure, allocation of new values, or throwing
an exception. 



\section{Literals}
\index{literal}

Literals denote fixed values of built-in types. 
The syntax for literals is given in \Sref{Literals}. 

The type that \Xten{} gives a literal often includes its value. \Eg, \xcd`1`
is of type \xcd`Int{self==1}`, and \xcd`true` is of type
\xcd`Boolean{self==true}`.

\section{{\tt this}}
\index{this}
\index{\Xcd{this}}

\begin{bbgrammar}
%(FROM #(prod:Primary)#)
             Primary \: \xcd"this" (\ref{prod:Primary}) \\
                    \| \xcd"this" \\
                    \| ClassName \xcd"." \xcd"this" \\
\end{bbgrammar}


The expression \xcd"this" is a  local \xcd`val` containing a reference
to an instance of the lexically enclosing class.
It may be used only within the body of an instance method, a
constructor, or in the initializer of a instance field -- that is, the places
where there is an instance of the class under consideration.

Within an inner class, \xcd"this" may be qualified with the
name of a lexically enclosing class.  In this case, it
represents an instance of that enclosing class.  


\begin{ex}
\xcd`Outer` is a class containing \xcd`Inner`.  Each instance of
\xcd`Inner` has a reference \xcd`Outer.this` to the \xcd`Outer` involved in its
creation.  \xcd`Inner` has access to the fields of \xcd`Outer.this`, as seen
in the \xcd`outerThree` and \xcd`alwaysTrue` methods.  Note that \xcd`Inner`
has its own \xcd`three` field, which is different from and not even the same
type as \xcd`Outer.this.three`. 
%~~gen ^^^ Expressions10
% package exp.vexp.pexp.lexp.shexp; 
% NOTEST
%~~vis 
\begin{xten}
class Outer {
  val three = 3;
  class Inner {
     val three = "THREE";
     def outerThree() = Outer.this.three;
     def alwaysTrue() = outerThree() == 3;
  }
}
\end{xten}
%~~siv
%
%~~neg
\end{ex}

The type of a \xcd"this" expression is the
innermost enclosing class, or the qualifying class,
constrained by the class invariant and the
method guard, if any.

The \xcd"this" expression may also be used within constraints in
a class or interface header (the class invariant and
\xcd"extends" and \xcd"implements" clauses).  Here, the type of
\xcd"this" is restricted so that only properties declared in the
class header itself, and specifically not any members declared in the class
body or in supertypes, are accessible through \xcd"this".

\section{Local variables}

%##(Id
\begin{bbgrammar}
%(FROM #(prod:Id)#)
                  Id \: identifier & (\ref{prod:Id}) \\
\end{bbgrammar}
%##)

A local variable expression consists simply of the name of the local variable,
field of the current object, formal parameter in scope, etc. It evaluates to
the value of the local variable. 


\begin{ex}
\xcd`n` in the second line below is a local
variable expression.  The \xcd`n` in the first line is not; it is part of a
local variable declaration.
%~~gen  ^^^ Expressions20
% package exp.loc.al.varia.ble; 
% class Example {
% def example() { 
%~~vis
\begin{xten}
val n = 22;
val m = n + 56;
\end{xten}
%~~siv
%} }
%~~neg

\end{ex}

\section{Field access}
\label{FieldAccess}
\index{field!access to}

%##(FieldAccess
\begin{bbgrammar}
%(FROM #(prod:FieldAccess)#)
         FieldAccess \: Primary \xcd"." Id & (\ref{prod:FieldAccess}) \\
                    \| \xcd"super" \xcd"." Id \\
                    \| ClassName \xcd"." \xcd"super"  \xcd"." Id \\
                    \| Primary \xcd"." \xcd"class"  \\
                    \| \xcd"super" \xcd"." \xcd"class"  \\
                    \| ClassName \xcd"." \xcd"super"  \xcd"." \xcd"class"  \\
\end{bbgrammar}
%##)

A field of an object instance may be  accessed
with a field access expression.

The type of the access is the declared type of the field with the
actual target substituted for \xcd"this" in the type. 

\begin{ex}
The declaration of \xcd`b` below has a constraint involving \xcd`this`.  
The use of an instance of it, \xcd`f.b`, has the same constraint involving
\xcd`f` instead of \xcd`this`, as required.
%~~gen ^^^ Expressions5s7v
% package Expressions5s7v;
%~~vis
\begin{xten}
class Fielded {
  public val a : Int = 1;
  public val b : Int{this.a == b} = this.a;
  static def example() {
    val f : Fielded = new Fielded();
    val fb : Int{fb == f.a} = f.b;
  }
}
\end{xten}
%~~siv
%
%~~neg

\end{ex}
% If the actual
%target is not a final access path (\Sref{FinalAccessPath}),
%an anonymous path is substituted for \xcd"this".

The field accessed is selected from the fields and value properties
of the static type of the target and its superclasses.

If the field target is given by the keyword \xcd"super", the target's type is
the superclass of the enclosing class.  This form is used to access fields of
the parent class shadowed by same-named fields of the current class.

If the field target is \xcd`Cls.super`, then the target's type is \xcd`Cls`,
which must be an  enclosing class.  This (admittedly
obscure) form is used to access fields of an ancestor class which are shadowed
by same-named fields of some more recent ancestor.  

\begin{ex}
This illustrates all four cases of field access.
%~~gen ^^^ Expressions30
% package exp.re.ssio.ns.fiel.dacc.ess;
% NOTEST
%~~vis
\begin{xten}
class Uncle {
  public static val f = 1;
}
class Parent {
  public val f = 2;
}
class Ego extends Parent {
  public val f = 3;
  class Child extends Ego {
     public val f = 4;
     def classNameDotId() =  Uncle.f;     // 1
     def cnDotSuperDotId() = Ego.super.f; // 2
     def superDotId() =      super.f;     // 3
     def expDotId() =        this.f;      // 4
  }
}
\end{xten}
%~~siv
%
%~~neg
\end{ex}

If the field target is \xcd"null", a \xcd"NullPointerException"
is thrown.
If the field target is a class name, a static field is selected.
It is illegal to access  a field that is not visible from
the current context.
It is illegal to access a non-static field
through a static field access expression.  However, it is legal to access a
static field through a non-static reference.

\section{Function Literals}
Function literals are described in \Sref{Functions}.

\section{Calls}
\label{Call}
\label{MethodInvocation}
\label{MethodInvocationSubstitution}
\index{invocation}
\index{call}
\index{invocation!method}
\index{call!method}
\index{invocation!function}
\index{call!function}
\index{method!calling}
\index{method!invoking}


%##(MethodInvocation ArgumentList
\begin{bbgrammar}
%(FROM #(prod:MethodInvocation)#)
    MethodInvocation \: MethodPrimaryPrefix \xcd"(" ArgumentList\opt \xcd")" & (\ref{prod:MethodInvocation}) \\
                    \| MethodSuperPrefix \xcd"(" ArgumentList\opt \xcd")" \\
                    \| MethodClassNameSuperPrefix \xcd"(" ArgumentList\opt \xcd")" \\
                    \| MethodName TypeArguments\opt \xcd"(" ArgumentList\opt \xcd")" \\
                    \| Primary \xcd"." Id TypeArguments\opt \xcd"(" ArgumentList\opt \xcd")" \\
                    \| \xcd"super" \xcd"." Id TypeArguments\opt \xcd"(" ArgumentList\opt \xcd")" \\
                    \| ClassName \xcd"." \xcd"super"  \xcd"." Id TypeArguments\opt \xcd"(" ArgumentList\opt \xcd")" \\
                    \| Primary TypeArguments\opt \xcd"(" ArgumentList\opt \xcd")" \\
%(FROM #(prod:ArgumentList)#)
        ArgumentList \: Exp & (\ref{prod:ArgumentList}) \\
                    \| ArgumentList \xcd"," Exp \\
\end{bbgrammar}
%##)


A \grammarrule{MethodInvocation} may be to either a \xcd"static" method, an
instance method, or a closure.

The syntax for method invocations is ambiguous. \xcd`ob.m()` could either be
the invocation of a method named \xcd`m` on object \xcd`ob`, or the
application of a function held in a field \xcd`ob.m`. The target \xcd`ob` must
be type-checked to determine which of these it is.  It is a static error if
both cases are possible after type checking.

\begin{ex}
%~~gen ^^^ Expressions40
% package expres.sio.nsca.lls;
%~~vis
\begin{xten}
class Callsome {
  static val closure : () => Int = () => 1;
  static def method()            = 2;
  static def example() {
     assert Callsome.closure() == 1;
     assert Callsome.method()  == 2;
  } 
}
\end{xten}
%~~siv
% class Hook{ def run() { Callsome.example(); return true; } }
%~~neg
However, adding a static method [mis]named \xcd`closure` makes
\xcd`Callsome.closure()` 
ambiguous: it could be a call to the closure, or to the static method: 
%~~gen ^^^ Expressions50
% package expres.sio.nsca.lls.twoooo;
% class Callsome {static val closure = () => 1; static def method () = 2; static val methodEvaluated = Callsome.method();
%~~vis
\begin{xten}
  static def closure () = 3;
  // ERROR: static errory = Callsome.closure();
\end{xten}
%~~siv
% }
%~~neg
\end{ex}

The application form \xcd`e(f,g)`, when \xcd`e` evaluates to an object or
struct, invokes the application \xcd`operator`, 
defined in the form 
%~~gen ^^^ Expressions2x1f
% package Expressions2x1f;
% class Example[F,G] {
%~~vis
\begin{xten}
public operator this(f:F, g:G) = "value";
\end{xten}
%~~siv
%  }
%~~neg


Method selection rules are given in \Sref{sect:MethodResolution}.

It is a static error if a method's \grammarrule{Guard} is not statically
satisfied by the 
caller.  

\begin{ex}
In this example, a \xcd`DivideBy` object provides the service of dividing
numbers by \xcd`denom` --- so long as \xcd`denom` is not zero. 
In the \xcd`example` method, \xcd`this.div(100)`  is not allowed; there is no
guarantee that \xcd`denom != 0`.  Casting \xcd`this` to a type 
whose constraint implies \xcd`denom != 0` permits the method call.
%~~gen ^^^ Expressions60
%package Expressions.Calls.Guarded.By.Walls;
% KNOWNFAIL
%~~vis
\begin{xten}
class DivideBy(denom:Int) {
  def div(numer:Int){denom != 0} = numer / denom;
  def example() {
     val thisCast = (this as DivideBy{self.denom != 0});
     thisCast.div(100);
     //ERROR: this.div(100); 
  }
}
\end{xten}
%~~siv
% class Hook{ def run() { (new DivideBy(1)).example(); return true; } }
%~~neg
\end{ex}

\section{Assignment}\index{assignment}\label{AssignmentStatement}

%##(Assignment LeftHandSide AssignmentOperator
\begin{bbgrammar}
%(FROM #(prod:Assignment)#)
          Assignment \: LeftHandSide AssignmentOperator AssignmentExp & (\ref{prod:Assignment}) \\
                    \| ExpName  \xcd"(" ArgumentList\opt \xcd")" AssignmentOperator AssignmentExp \\
                    \| Primary  \xcd"(" ArgumentList\opt \xcd")" AssignmentOperator AssignmentExp \\
%(FROM #(prod:LeftHandSide)#)
        LeftHandSide \: ExpName & (\ref{prod:LeftHandSide}) \\
                    \| FieldAccess \\
%(FROM #(prod:AssignmentOperator)#)
  AssignmentOperator \: \xcd"=" & (\ref{prod:AssignmentOperator}) \\
                    \| \xcd"*=" \\
                    \| \xcd"/=" \\
                    \| \xcd"%=" \\
                    \| \xcd"+=" \\
                    \| \xcd"-=" \\
                    \| \xcd"<<=" \\
                    \| \xcd">>=" \\
                    \| \xcd">>>=" \\
                    \| \xcd"&=" \\
                    \| \xcd"^=" \\
                    \| \xcd"|=" \\
\end{bbgrammar}
%##)



The assignment expression \xcd"x = e" assigns a value given by
expression \xcd"e"
to a variable \xcd"x".  
Most often, \xcd`x` is mutable, a \xcd`var` variable.  The same syntax is
used for delayed initialization of a \xcd`val`, but \xcd`val`s can only be
initialized once.
%~~gen ^^^ Expressions70
% package express.ions.ass.ignment;
% class Example {
% static def exasmple() {
%~~vis
\begin{xten}
  var x : Int;
  val y : Int;
  x = 1;
  y = 2; // Correct; initializes y
  x = 3; 
  // ERROR: y = 4;
\end{xten}
%~~siv
% } } 
%~~neg


There are three syntactic forms of
assignment: 
\begin{enumerate}
\item \xcd`x = e;`, assigning to a local variable, formal parameter, field of
      \xcd`this`, etc. 
\item \xcd`x.f = e;`, assigning to a field of an object.
\item \xcdmath`a(i$_1$,$\ldots$,i$_n$) = v;`, where {$n \ge 0$}, assigning to
      an element of an array or some other such structure. This is an operator
      call (\Sref{sect:operators}).  For well-behaved classes it works like
      array assignment, mutatis mutandis, but there is no actual guarantee,
      and the compiler makes no assumptions about how this works for arbitrary \xcd`a`.
      Naturally, it is a static error if no suitable assignment operator
      for \xcd`a`.
\end{enumerate}

For a binary operator $\diamond$, the $\diamond$-assignment expression
\xcdmath"x $\diamond$= e" combines the current value of \xcd`x` with the value
of \xcd`e` by {$\diamond$}, and stores the result back into \xcd`x`.  
\xcd`i += 2`, for example, adds 2 to \xcd`i`. For variables and fields, 
\xcdmath"x $\diamond$= e" behaves just like \xcdmath"x = x $\diamond$ e". 

The subscripting forms of \xcdmath"a(i) $\diamond$= b" are slightly subtle.
Subexpressions of \xcd`a` and \xcd`i` are only evaluated once.  However,
\xcd`a(i)` and \xcd`a(i)=c` are each executed once---in particular, there is
one call to the application operator, and one to the assignment operator.
If subscripting is implemented strangely for
the class of \xcd`a`, the behavior is {\em not} necessarily updating a single
storage location. Specifically, \xcd`A()(I()) += B()` is tantamount to: 
%~~gen ^^^ Expressions80
% package expressions.stupid.addab;
% class Example {
% def example(A:()=>Rail[Int], I: () => Int, B: () => Int ) {
%~~vis
\begin{xten}
{
  val aa = A();  // Evaluate A() once
  val ii = I();  // Evaluate I() once
  val bb = B();  // Evaluate B() once
  val tmp = aa(ii) + bb; // read aa(ii)
  aa(ii) = tmp;  // write sum back to aa(ii)
}
\end{xten}
%~~siv
%}}
%~~neg

\limitation{+= does not currently meet this specification.}




\section{Increment and decrement}
\index{increment}
\index{decrement}
\index{\Xcd{++}}
\index{\Xcd{--}}


The operators \xcd"++" and \xcd"--" increment and decrement
a variable, respectively.  
\xcd`x++` and \xcd`++x` both increment \xcd`x`, just as the statement 
\xcd`x += 1` would, and similarly for \xcd`--`.  

The difference between the two is the return value.  
\xcd`++x` and \xcd`--x` return the {\em new} value of \xcd`x`, after
incrementing or decrementing.
\xcd`x++` and \xcd`x--` return the {\em old} value of \xcd`x`, before
incrementing or decrementing.


\limitation{This currently only works for numeric types.}

\section{Numeric Operations}
\label{XtenPromotions}
\index{promotion}
\index{numeric promotion}
\index{numeric operations}
\index{operation!numeric}

Numeric types (\xcd`Byte`, \xcd`Short`, \xcd`Int`, \xcd`Long`, \xcd`Float`,
\xcd`Double`, \xcd`Complex`, and unsigned variants of fixed-point types) are normal X10
structs, though most of their methods are implemented via native code. They
obey the same general rules as other X10 structs. For example, numeric
operations, coercions, and conversions are defined by \xcd`operator` definitions, the same way you could
for any struct.

Promoting a numeric value to a longer numeric type preserves the sign of the
value.  For example, \xcd`(255 as UByte) as UInt` is 255. 

Most of these operations can be defined on user-defined types as well.  While
it is good practice to keep such operations consistent with the numeric
operations whenever possible, the compiler neither enforces nor assumes any
particular semantics of user-defined operations. 

\subsection{Conversions and coercions}

Specifically, each numeric type can be converted or coerced into each other
numeric type, perhaps with loss of accuracy.
%~~gen ^^^ Expressions90
% package exp.ress.io.ns.numeric.conversions;
% class ExampleOfConversionAndStuff {
% def example() {
%~~vis
\begin{xten}
val n : Byte = 123 as Byte; // explicit 
val f : (Int)=>Boolean = (Int) => true; 
val ok = f(n); // implicit
\end{xten}
%~~siv
% } }
%~~neg



\subsection{Unary plus and unary minus}

The unary \xcd`+` operation on numbers is an identity function.
The unary \xcd`-` operation on numbers is a negation function.
On unsigned numbers, these are two's-complement.  For example, 
\xcd`-(0x0F as UByte)` is 
\xcd`(0xF1 as UByte)`.
\bard{UInts and such are closed under negation -- the negative of a UInt is
done binarily.  }



\section{Bitwise complement}

The unary \xcd"~" operator, only defined on integral types, complements each
bit in its operand.  

\section{Binary arithmetic operations} 

The binary arithmetic operators perform the familiar binary arithmetic
operations: \xcd`+` adds, \xcd`-` subtracts, \xcd`*` multiplies, 
\xcd`/` divides, and \xcd`%`
computes remainder.

On integers, the operands are coerced to the longer of their two types, and
then operated upon.  
Floating point operations are determined by the IEEE 754
standard. 
The integer \xcd"/" and \xcd"%" throw an exception 
if the right operand is zero.



\section{Binary shift operations}

The operands of the binary shift operations must be of integral type.
The type of the result is the type of the left operand.
The right operand, describing a number of bits, must be unsigned: 
%~~exp~~`~~`~~ x:Int ~~ ^^^Expressions1l4m
\xcd`x << 1U`.  


If the promoted type of the left operand is \xcd"Int",
the right operand is masked with \xcd"0x1f" using the bitwise
AND (\xcd"&") operator, giving a number at most the number of bits in an
\xcd`Int`. 
If the promoted type of the left operand is \xcd"Long",
the right operand is masked with \xcd"0x3f" using the bitwise
AND (\xcd"&") operator, giving a number at most the number of bits in a
\xcd`Long`. 

The \xcd"<<" operator left-shifts the left operand by the number of
bits given by the right operand.
The \xcd">>" operator right-shifts the left operand by the number of
bits given by the right operand.  The result is sign extended;
that is, if the right operand is $k$,
the most significant $k$ bits of the result are set to the most
significant bit of the operand.

The \xcd">>>" operator right-shifts the left operand by the number of
bits given by the right operand.  The result is not sign extended;
that is, if the right operand is $k$,
the most significant $k$ bits of the result are set to \xcd"0".
This operation is deprecated, and may be removed in a later version of the
language. 


\section{Binary bitwise operations}

The binary bitwise operations operate on integral types, which are promoted to
the longer of the two types.
The \xcd"&" operator  performs the bitwise AND of the promoted operands.
The \xcd"|" operator  performs the bitwise inclusive OR of the promoted operands.
The \xcd"^" operator  performs the bitwise exclusive OR of the promoted operands.

\section{String concatenation}
\index{string!concatenation}

The \xcd"+"  operator is used for string concatenation 
 as well as addition.
If either operand is of static type \xcd"x10.lang.String",
 the other operand is converted to a \xcd"String" , if needed,
  and  the two strings  are concatenated.
 String conversion of a non-\xcd"null" value is  performed by invoking the
 \xcd"toString()" method of the value.
  If the value is \xcd"null", the value is converted to 
  \xcd'"null"'.

The type of the result is \xcd"String".

 For example, 
%~~exp~~`~~`~~ ~~ ^^^ Expressions100
      \xcd`"one " + 2 + here` 
      evaluates to  \xcd`one 2(Place 0)`.  

\section{Logical negation}

The operand of the  unary \xcd"!" operator 
must be of type \xcd"x10.lang.Boolean".
The type of the result is \xcd"Boolean".
If the value of the operand is \xcd"true", the result is \xcd"false"; if
if the value of the operand  is \xcd"false", the result is \xcd"true".

\section{Boolean logical operations}

Operands of the binary boolean logical operators must be of type \xcd"Boolean".
The type of the result is \xcd"Boolean"

The \xcd"&" operator  evaluates to \xcd"true" if both of its
operands evaluate to \xcd"true"; otherwise, the operator
evaluates to \xcd"false".

The \xcd"|" operator  evaluates to \xcd"false" if both of its
operands evaluate to \xcd"false"; otherwise, the operator
evaluates to \xcd"true".

\section{Boolean conditional operations}

Operands of the binary boolean conditional operators must be of type
\xcd"Boolean". 
The type of the result is \xcd"Boolean"

The \xcd"&&" operator  evaluates to \xcd"true" if both of its
operands evaluate to \xcd"true"; otherwise, the operator
evaluates to \xcd"false".
Unlike the logical operator \xcd"&",
if the first operand is \xcd"false",
the second operand is not evaluated.

The \xcd"||" operator  evaluates to \xcd"false" if both of its
operands evaluate to \xcd"false"; otherwise, the operator
evaluates to \xcd"true".
Unlike the logical operator \xcd"||",
if the first operand is \xcd"true",
the second operand is not evaluated.

\section{Relational operations} 

The relational operations on numeric types compare numbers, producing
\xcd`Boolean` results.

The \xcd"<" operator evaluates to \xcd"true" if the left operand is
less than the right.
The \xcd"<=" operator evaluates to \xcd"true" if the left operand is
less than or equal to the right.
The \xcd">" operator evaluates to \xcd"true" if the left operand is
greater than the right.
The \xcd">=" operator evaluates to \xcd"true" if the left operand is
greater than or equal to the right.

Floating point comparison is determined by the IEEE 754
standard.  Thus,
if either operand is NaN, the result is \xcd"false".
Negative zero and positive zero are considered to be equal.
All finite values are less than positive infinity and greater
than negative infinity.



\section{Conditional expressions}
\index{\Xcd{? :}}
\index{conditional expression}
\index{expression!conditional}
\label{Conditional}

%##(ConditionalExp
\begin{bbgrammar}
%(FROM #(prod:ConditionalExp)#)
      ConditionalExp \: ConditionalOrExp & (\ref{prod:ConditionalExp}) \\
                    \| ClosureExp \\
                    \| AtExp \\
                    \| FinishExp \\
                    \| ConditionalOrExp \xcd"?" Exp \xcd":" ConditionalExp \\
\end{bbgrammar}
%##)

A conditional expression evaluates its first subexpression (the
condition); if \xcd"true"
the second subexpression (the consequent) is evaluated; otherwise,
the third subexpression (the alternative) is evaluated.

The type of the condition must be \xcd"Boolean".
The type of the conditional expression is some common 
ancestor (as constrained by \Sref{LCA}) of the types of the consequent and the
alternative. 

\begin{ex}
%~~exp~~`~~`~~a:Int,b:Int ~~ ^^^ Expressions110
\xcd`a == b ? 1 : 2`
evaluates to \xcd`1` if \xcd`a` and \xcd`b` are the same, and \xcd`2` if they
are different.   As the type of \xcd`1` is \xcd`Int{self==1}` and of \xcd`2`
is \xcd`Int{self==2}`, the type of the conditional expression has the form
\xcd`Int{c}`, where \xcd`self==1` and \xcd`self==2` both imply \xcd`c`.  For
example, it might be \xcd`Int{true}` -- or perhaps it might be 
\xcd`Int{self != 8}`. Note that this term has no most accurate type in the X10
type system.
\end{ex}

The subexpression not selected is not evaluated.

\begin{ex}
The following use of the conditional expression prevents division by zero.  If
\xcd`den==0`, the division is not performed at all.
%~~gen ^^^ Expressions4t3m
% package Expressions4t3m;
% class Hook {
% static def example(num:Int, den:Int ) =
%~~vis
\begin{xten}
(den == 0) ? 0 : num/den
\end{xten}
%~~siv
%; 
% def run() { 
%   return example(1,0) == 0 && example(6,3) == 2;
% } }
%~~neg

Similarly, the following code performs a method call if \xcd`op` is non-null,
and avoids the null pointer error if it is null.  Defensive coding like this
is quite common when working with possibly-null objects.
%~~gen ^^^ Expressions6o2b
% package Expressions6o2b;
% class Hook { 
% static def example(ob:Object) = 
%~~vis
\begin{xten}
(ob == null) ? null : ob.toString();
\end{xten}
%~~siv
%def run() {
%  return example(null) == null && example("yes").equals("yes"); 
% } } 
%~~neg



\end{ex}

\section{Stable equality}
\label{StableEquality}
\index{\Xcd{==}}
\index{equality}

\begin{bbgrammar}
 EqualityExp    \: RelationalExp & (\ref{prod:EqualityExp})\\%<FROM #(prod:EqualityExp)#
    \| EqualityExp \xcd"==" RelationalExp\\
    \| EqualityExp \xcd"!=" RelationalExp\\
    \| Type  \xcd"==" Type \\
\end{bbgrammar}


The \xcd"==" and \xcd"!=" operators provide a fundamental, though
non-abstract, notion of equality.  \xcd`a==b` is true if the values of \xcd`a`
and \xcd`b` are extremely identical.

\begin{itemize}
\item If \xcd`a` and \xcd`b` are values of object type, then \xcd`a==b` holds
      if \xcd`a` and \xcd`b` are the same object.
\item If one operand is \xcd`null`, then \xcd`a==b` holds iff the other is
      also \xcd`null`.
\item If the operands both have struct type, then they must be structurally equal;
that is, they must be instances of the same struct
and all their fields or components must be \xcd"==". 
\item The definition of equality for function types is specified in
      \Sref{FunctionEquality}.
\item If the operands have numeric types, they are coerced into the larger of
      the two types (see \Sref{WideningConversions}) and then compared for numeric equality.
\end{itemize}

\xcd`a != b`
is true iff \xcd`a==b` is false.

The predicates \xcd"==" and \xcd"!=" may not be overridden by the programmer.

\xcd`==` provides a {\em stable} notion of equality.  If two values are
\xcd`==` at any time, they remain \xcd`==` forevermore, regardless of what
happens to the mutable state of the program. 

\begin{ex}
Regardless of the values and types of \xcd`a` and \xcd`b`, 
or the behavior of \xcd`any_code_at_all` (which may, indeed, be
any code at all---not just a method call), the value of 
\xcd`a==b` does not change: 
%~~gen ^^^ Expressions1i5k
% package Expressions1i5k;
% class Example{ 
% def example( something: ()=>Int, something_else: ()=>Int,
%   any_code_at_all: () => Int) {
%~~vis
\begin{xten}
val a = something();
val b = something_else();
val eq1 = (a == b);
any_code_at_all();
val eq2 = (a == b);
assert eq1 == eq2;
\end{xten}
%~~siv
%} } 
%~~neg
\end{ex}



\section{Allocation}
\label{ClassCreation}
\index{new}
\index{allocation}
\index{class!instantation}
\index{class!construction}
\index{struct!instantation}
\index{struct!construction}
\index{instantation}

%##(ClassInstCreationExp
\begin{bbgrammar}
%(FROM #(prod:ClassInstCreationExp)#)
ClassInstCreationExp \: \xcd"new" TypeName TypeArguments\opt \xcd"(" ArgumentList\opt \xcd")" ClassBody\opt & (\ref{prod:ClassInstCreationExp}) \\
                    \| \xcd"new" TypeName \xcd"[" Type \xcd"]" \xcd"[" ArgumentList\opt \xcd"]" \\
                    \| Primary \xcd"." \xcd"new" Id TypeArguments\opt \xcd"(" ArgumentList\opt \xcd")" ClassBody\opt \\
                    \| AmbiguousName \xcd"." \xcd"new" Id TypeArguments\opt \xcd"(" ArgumentList\opt \xcd")" ClassBody\opt \\
\end{bbgrammar}
%##)

An allocation expression creates a new instance of a class and
invokes a constructor of the class.
The expression designates the class name and passes
type and value arguments to the constructor.

The allocation expression may have an optional class body.
In this case, an anonymous subclass of the given class is
allocated.   An anonymous class allocation may also specify a
single super-interface rather than a superclass; the superclass
of the anonymous class is \xcd"x10.lang.Object".

If the class is anonymous---that is, if a class body is
provided---then the constructor is selected from the superclass.
The constructor to invoke is selected using the same rules as
for method invocation (\Sref{MethodInvocation}).

The type of an allocation expression
is the return type of the constructor invoked, with appropriate
substitutions  of actual arguments for formal parameters, as
specified in \Sref{MethodInvocationSubstitution}.

It is illegal to allocate an instance of an \xcd"abstract" class.
The usual visibility rules apply to allocations: 
it is illegal to allocate an instance of a class or to invoke a
constructor that is not visible at
the allocation expression.

Note that instantiating a struct type can use function application syntax; 
\xcd`new` is optional.  As structs do not have subclassing, there is no need or
possibility of a {\em ClassBody}.


\section{Casts}\label{ClassCast}\index{cast}
\index{type conversion}

The cast operation may be used to cast an expression to a given type:

%##(CastExp
\begin{bbgrammar}
%(FROM #(prod:CastExp)#)
             CastExp \: Primary & (\ref{prod:CastExp}) \\
                    \| ExpName \\
                    \| CastExp \xcd"as" Type \\
\end{bbgrammar}
%##)

The result of this operation is a value of the given type if the cast
is permissible at run time, and either a compile-time error or a runtime
exception 
(\xcd`x10.lang.TypeCastException`) if it is not.  

When evaluating \xcd`E as T{c}`, first the value of \xcd`E` is converted to
type \xcd`T` (which may fail), and then the constraint \xcd`{c}` is checked. 



\begin{itemize}
\item If \xcd`T` is a primitive type, then \xcd`E`'s value is converted to type
      \xcd`T` according to the rules of
      \Sref{sec:effects-of-explicit-numeric-coercions}. 
      
\item If \xcd`T` is a class, then the first half of the cast succeeds if the
      run-time value of \xcd`E` is an instance of class \xcd`T`, or of a
      subclass. 

\item If \xcd`T` is an interface, then the first half of the cast succeeds if
      the run-time value of \xcd`E` is an instance of a class implementing
      \xcd`T`. 

\item If \xcd`T` is a struct type, then the first half of the cast succeeds if
      the run-time value of \xcd`E` is an instance of \xcd`T`.  

\item If \xcd`T` is a function type, then the first half of the cast succeeds
      if the run-time value of \xcd`X` is a function of that type, or a
      subtype of it.
\end{itemize}

If the first half of the cast succeeds, the second half -- the constraint
\xcd`{c}` -- must be checked.  In general this will be done at runtime, though
in special cases it can be checked at compile time.   For example, 
\xcd`n as Int{self != w}` succeeds if \xcd`n != w` --- even if \xcd`w` is a value
read from input, and thus not determined at compile time.

The compiler may forbid casts that it knows cannot possibly work. If there is
no way for the value of \xcd`E` to be of type \xcd`T{c}`, then 
\xcd`E as T{c}` can result in a static error, rather than a runtime error.  
For example, \xcd`1 as Int{self==2}` may fail to compile, because the compiler
knows that \xcd`1`, which has type \xcd`Int{self==1}`, cannot possibly be of
type \xcd`Int{self==2}`. 


%BB% \bard{This section need serious whomping.  The Java mention needs to go.  The
%BB% rules for coercions are given in \Sref{sec:effects-of-explicit-numeric-coercions}.
%BB% If the \xcd`Type` has a constraint, the constraint will be checked at runtime. 
%BB% We need to give examples. 
%BB% }
%BB% 
%BB% Type conversion is checked according to the
%BB% rules of the \java{} language (e.g., \cite[\S 5.5]{jls2}).
%BB% For constrained types, both the base
%BB% type and the constraint are checked.
%BB% If the
%BB% value cannot be cast to the appropriate type, a
%BB% \xcd"ClassCastException"
%BB% is thrown. 



% {\bf Conversions of numeric values}
% {\bf Can't do (a as T) if a can't be a T.}


%If the value cannot be cast to the
%appropriate place type a \xcd"BadPlaceException" is thrown. 

% Any attempt to cast an expression of a reference type to a value type
% (or vice versa) results in a compile-time error. Some casts---such as
% those that seek to cast a value of a subtype to a supertype---are
% known to succeed at compile-time. Such casts should not cause extra
% computational overhead at run time.

\section{\Xcd{instanceof}}
\label{instanceOf}
\index{\Xcd{instanceof}}
\index{instanceof}

\Xten{} permits types to be used in an in instanceof expression
to determine whether an object is an instance of the given type:

%##(RelationalExp
\begin{bbgrammar}
%(FROM #(prod:RelationalExp)#)
       RelationalExp \: RangeExp & (\ref{prod:RelationalExp}) \\
                    \| SubtypeConstraint \\
                    \| RelationalExp \xcd"<" RangeExp \\
                    \| RelationalExp \xcd">" RangeExp \\
                    \| RelationalExp \xcd"<=" RangeExp \\
                    \| RelationalExp \xcd">=" RangeExp \\
                    \| RelationalExp \xcd"instanceof" Type \\
                    \| RelationalExp \xcd"in" ShiftExp \\
\end{bbgrammar}
%##)

In the above expression, \grammarrule{Type} is any type. At run time, the
result of this operator is \xcd"true" if the
\grammarrule{RelationalExpression} can be coerced to \grammarrule{Type}
without a \xcd"TypeCastException" being thrown or static error occurring.
Otherwise the result is \xcd"false". This determination may involve checking
that the constraint, if any, associated with the type is true for the given
expression.

%~~exp~~`~~`~~x:Int~~ ^^^ Expressions120
For example, \xcd`3 instanceof Int{self==x}` is an overly-complicated way of
saying \xcd`3==x`.


However, it is a static error if \xcd`e` cannot possibly be an instance of
\xcd`C{c}`; the compiler will reject \xcd`1 instanceof Int{self == 2}` because
\xcd`1` can never satisfy \xcd`Int{self == 2}`. Similarly, \Xcd{1 instanceof
String} is a static error, rather than an expression always returning false. 

\limitationx
X10 does not currently handle \xcd`instanceof` of generics in the way you
%~NO~exp~~`~~`~~r:Array[Int](1) ~~
might expect.  For example, \xcd`r instanceof Array[Int{self != 0}]` does
not test that every element of \xcd`r` is non-zero; instead, the compiler
rejects it.


\section{Subtyping expressions}
\index{\Xcd{<:}}
\index{\Xcd{:>}}
\index{subtype!test}


%##(SubtypeConstraint
\begin{bbgrammar}
%(FROM #(prod:SubtypeConstraint)#)
   SubtypeConstraint \: Type  \xcd"<:" Type  & (\ref{prod:SubtypeConstraint}) \\
                    \| Type  \xcd":>" Type  \\
\end{bbgrammar}
%##)

The subtyping expression \xcdmath"T$_1$ <: T$_2$" evaluates to \xcd"true" if
\xcdmath"T$_1$" is a subtype of \xcdmath"T$_2$".

The expression \xcdmath"T$_1$ :> T$_2$" evaluates to \xcd"true" if
\xcdmath"T$_2$" is a subtype of \xcdmath"T$_1$".

The expression \xcdmath"T$_1$ == T$_2$"
evaluates to  \xcd"true" if 
\xcdmath"T$_1$" is a subtype of \xcdmath"T$_2$" and
if \xcdmath"T$_2$" is a subtype of \xcdmath"T$_1$".

\begin{ex}
Subtyping expressions are particularly useful in giving constraints on generic
types.  \xcd`x10.util.Ordered[T]` is an interface whose values can be compared
with values of type \xcd`T`. 
In particular, \xcd`T <: x10.util.Ordered[T]` is
true if values of type \xcd`T` can be compared to other values of type
\xcd`T`.  So, if we wish to define a generic class \xcd`OrderedList[T]`, of
lists whose elements are kept in the right order, we need the elements to be
ordered.  This is phrased as a constraint on \xcd`T`: 
%~~gen ^^^ Expressions130
% package expre.ssi.onsfgua.rde.dq.uantification;
%~~vis
\begin{xten}
class OrderedList[T]{T <: x10.util.Ordered[T]} {
  // ...
}
\end{xten}
%~~siv
%
%~~neg
\end{ex}


\section{Contains expressions}
\index{in}

\begin{bbgrammar}
       RelationalExp \:RelationalExp \xcd"in" ShiftExp & (\ref{prod:RelationalExp}) \\
\end{bbgrammar}

\xcd`in` is a binary operator, definable in \Sref{sect:operators}.  It is
conventionally used for checking containment.

\begin{ex}
The built-in type \xcd`Region` provides \xcd`in`, testing whether a
\xcd`Point` is in the region: 
%~~gen ^^^ Expressions6d2z
% package Expressions6d2z;
% class Hook { def run() {
%~~vis
\begin{xten}
assert [3] in 1..10;
assert !([10] in 1..3);
\end{xten}
%~~siv
% return true;
%}}
%~~neg

Other types can provide them as well:
%~~gen ^^^ Expressions3c4m
% package Expressions3c4m;
%~~vis
\begin{xten}
class Cont {
   operator this in (Int) = true;
   operator (String) in this = false;
   static operator (Cont) in (b:Boolean) = b;
   static def example() {
      val c:Cont = new Cont();
      assert c in 4 && !("odd" in c) && (c in true);
   }
}
\end{xten}
%~~siv
%class Hook{ def run() { Cont.example(); return true; } }
%~~neg


\end{ex}

\section{Array Constructors}
\label{sect:ArrayCtors}
\index{array!construction}
\index{array!literal}


\begin{bbgrammar}
             Primary \: 
                    \xcd"[" ArgumentList\opt \xcd"]" \\
                    \| ClassInstCreationExp \\

ClassInstCreationExp \: \xcd"new" TypeName TypeArguments\opt \xcd"(" ArgumentList\opt \xcd")" ClassBody\opt & (\ref{prod:ClassInstCreationExp}) \\
                    \| \xcd"new" TypeName \xcd"[" Type \xcd"]" \xcd"[" ArgumentList\opt \xcd"]" \\
                    \| Primary \xcd"." \xcd"new" Id TypeArguments\opt \xcd"(" ArgumentList\opt \xcd")" ClassBody\opt \\
                    \| AmbiguousName \xcd"." \xcd"new" Id TypeArguments\opt \xcd"(" ArgumentList\opt \xcd")" ClassBody\opt \\
\end{bbgrammar}
%##)

X10 includes short syntactic forms for constructing one-dimensional arrays.
The shortest form is to enclose some expressions in brackets: 
%~~gen ^^^ Expressions140
% package Expressions.ArrayCtor.Primo;
% class Example {
% def example() {
%~~vis
\begin{xten}
val ints <: Array[Int](1) = [1,3,7,21];
\end{xten}
%~~siv
%}}
%~~neg

The expression \xcdmath"[e$_1$, $\ldots$, e$_n$]" produces an \Xcd{n}-element
\xcd`Array[T](1)`, where \xcd`T` is the computed common supertype (\Sref{LCA}) of the {\bf
base types} of the expressions  \xcdmath"e$_i$". 

\begin{ex}
The type of
\xcd`[0,1,2]` is \Xcd{Array[Int](1)}.    
More importantly, the type of 
\xcd`[0]` is also \xcd`Array[Int](1)`.  It is {\em not} 
\xcd`Array[Int{self==0}](1)`, even though all the elements are all 
of type \xcd`Int{self==0}`.  This is subtle but important. There are many
functions that take \xcd`Array[Int](1)`s, such as conversions to \xcd`Point`.
These functions do {\em not} take
\xcd`Array[Int{self==0}]`'s.

(Suppose that the function took \xcd`a:Array[Int](1)` and did 
the operation \xcd`a(i)=100`.   This operation is perfectly fine for
an \xcd`Array[Int](1)`, which is all the compiler knows about \xcd`a`.  
However, it is invalid for an \xcd`Array[Int{self==0}](1)`, because it assigns
a non-zero value to an element of the array, violating the type constraint
which says that all the elements are zero.  So, \xcd`Array[Int{self==0}](1)`
is not and must not be a subtype of \xcd`Array[Int](1)` --- the two types are simply unrelated.
%~~type~~`~~`~~ ~~ ^^^ Expressions150
Since there are far more uses for \xcd`Array[Int](1)` than
%~~type~~`~~`~~ ~~ ^^^ Expressions160
\xcd`Array[Int{self==0}](1)`, the compiler produces the former.)
\end{ex}

The expression \xcdmath"new Array[T][e$_1$, $\ldots$, e$_n$]"
produces 
an \xcd`n`-element \xcd`Array[T](1)`, whose elements are the values
\xcdmath"e$_i$".  It is a static error if any \xcdmath"e$_i$" cannot be
coerced to \xcd`T`.


\begin{ex}
Occasionally one does actually need \xcd`Array[Int{self==0}](1)`, 
or, say, \xcd`Array[Eel{self != null}](1)`, an array of non-null \xcd`Eel`s.  
For these cases, X10 provides an array constructor which does allow
specification of the element type: \xcd`new Array[T][e1...en]`.  Each
element \xcd`ei` must be of type \xcd`T`.  The resulting array is of type
\xcd`Array[T](1)`.  
%~~gen ^^^ Expressions170
%package Expressions.ArrayCtor.Details;
%class Eel{}
%class Example{
%def example(){
%~~vis
\begin{xten}
val zero <: Array[Int{self == 0}](1) 
          = new Array[Int{self == 0}][0];
val non1 <: Array[Int{self != 1}](1) 
          = new Array[Int{self != 1}][0];
val eels <: Array[Eel{self != null}](1) 
          = new Array[Eel{self != null}][ new Eel() ];
\end{xten}
%~~siv
%}}
%~~neg
\end{ex}


\section{Coercions and conversions}
\label{XtenConversions}
\label{User-definedCoercions}
\index{conversion}\index{coercion}
\index{type!conversion}\index{type!coercion}

\XtenCurrVer{} supports the following coercions and conversions.

\subsection{Coercions}

%##(CastExp
\begin{bbgrammar}
%(FROM #(prod:CastExp)#)
             CastExp \: CastExp \xcd"as" Type \\
\end{bbgrammar}
%##)


A {\em coercion} does not change object identity; a coerced object may
be explicitly coerced back to its original type through a cast. A {\em
  conversion} may change object identity if the type being converted
to is not the same as the type converted from. \Xten{} permits
user-defined conversions (\Sref{sec:user-defined-conversions}).

\paragraph{Subsumption coercion.}
A value of a subtype may be implicitly coerced to any supertype.  
\index{coercion!subsumption}

\begin{ex}
If \xcd`Child <: Person` and \xcd`val rhys:Child`, then \xcd`rhys` may be used
in any context that expects a \xcd`Person`.  For example, 
%~~gen ^^^ Expressions7f1h
% package Expressions7f1h;
% class Person{}
% class Child extends Person {}
%~~vis
\begin{xten}
class Example {
  def greet(Person) = "Hi!";
  def example(rhys: Child) {
     greet(rhys);
  }
}
\end{xten}
%~~siv
%
%~~neg

Similarly, \xcd`2` (whose innate type is \xcd`Int{self==2}`)
is usable in a context requiring a non-zero integer
(\xcd`Int{self != 0}`).  
\end{ex}

\paragraph{Explicit Coercion (Casting with \xcd"as")}

All classes and interfaces allow the use of the \xcd`as` operator for explicit
type coercion.  
Any class or
interface may be cast to any interface.  
Any interface may be cast to
any class.  Also, any interface can be cast to a struct that implements
(directly or indirectly) that interface.

\begin{ex}
In the following code, a \xcd`Person` is cast to \xcd`Childlike`.  There is
nothing in the class definition of \xcd`Person` that suggests that a
\xcd`Person` can be \xcd`Childlike`.  However, the \xcd`Person` in question,
\xcd`p`, is actually a \xcd`HappyChild` --- a subclass of \xcd`Person` --- and
is, in fact, \xcd`Childlike`.  

Similarly, the \xcd`Childlike` value \xcd`cl` is cast to \xcd`Happy`.  Though
these two interfaces are unrelated, the value of \xcd`cl` is, in fact,
\xcd`Happy`.  And the \xcd`Happy` value \xcd`hc` is cast to the class
\xcd`Child`, though there is no relationship between the two, but the actual
value is a \xcd`HappyChild`, and thus the cast is correct at runtime.

\xcd`Cyborg` is a struct rather than a class.  So, it cannot have substructs,
and all the interfaces of all \xcd`Cyborg`s are known: a \xcd`Cyborg` is
\xcd`Personable`, but not \xcd`Childlike` or \xcd`Happy`.  So, it is correct
and meaningful to cast \xcd`r` to \xcd`Personable`.  There is no way that a
cast to \xcd`Childlike` could succeed, so \xcd`r as Childlike` is a static error.

%~~gen ^^^ Expressions180
% package Types.Coercions;
%~~vis
\begin{xten}
interface Personable {}
class Person implements Personable {}
interface Childlike extends Personable {}
class Child extends Person implements Childlike {}
struct Cyborg implements Personable {}
interface Happy {}
class HappyChild extends Child implements Happy {}
class Example {
  static def example() {
    var p : Person = new HappyChild();
    val cl : Childlike = p as Childlike; // class -> interface
    val hc : Happy = cl as Happy; //        interface -> interface
    val ch : Child = hc as Child; //        interface -> class
    var r : Cyborg = Cyborg();
    val rl : Personable = r as Personable; 
    // ERROR: r as Childlike
  }
}
\end{xten}
%~~siv
% class Hook{ def run(){ Example.example(); return true; } }
%~~neg




\end{ex}


If the value coerced is not an instance of the target type,
and no coercion operators that can convert it to that type are defined, 
a \xcd"ClassCastException" is thrown.  Casting to a constrained
type may require a run-time check that the constraint is
satisfied.
\index{coercion!explicit}
\index{cast}
\index{\Xcd{as}}

\limitation{It is currently a static error, rather than the specified
\xcd`ClassCastException`, when the cast is statically determinable to be
impossible.}



\paragraph{Effects of explicit numeric coercion}
\label{sec:effects-of-explicit-numeric-coercions}

Coercing a number of one type to another type gives the best approximation of
the number in the result type, or a suitable disaster value if no
approximation is good enough.  

\begin{itemize}
\item Casting a number to a {\em wider} numeric type is safe and effective,
      and can be done by an implicit conversion as well as an explicit
%~~exp~~`~~`~~ ~~ ^^^ Expressions190
      coercion.  For example, \xcd`4 as Long` produces the \xcd`Long` value of
      4. 
\item Casting a floating-point value to an integer value truncates the digits
      after the decimal point, thereby rounding the number towards zero.  
%~~exp~~`~~`~~ ^^^ Expressions200
      \xcd`54.321 as Int` is \xcd`54`, and 
%~~exp~~`~~`~~ ~~ ^^^ Expressions210
      \xcd`-54.321 as Int` is \xcd`-54`.
      If the floating-point value is too large to represent as that kind of
      integer, the coercion returns the largest or smallest value of that type
      instead: \xcd`1e110 as Int` is 
      \xcd`Int.MAX_VALUE`, \viz{} \xcd`2147483647`. 

\item Casting a \xcd`Double` to a \xcd`Float` normally truncates binary digits: 
%~~exp~~`~~`~~ ~~ ^^^ Expressions220
      \xcd`0.12345678901234567890 as Float` is approximately \xcd`0.12345679f`.  This can
      turn a nonzero \xcd`Double` into \xcd`0.0f`, the zero of type
      \xcd`Float`: 
%~~exp~~`~~`~~ ~~ ^^^ Expressions230
      \xcd`1e-100 as Float` is \xcd`0.0f`.  Since 
      \xcd`Double`s can be as large as about \xcd`1.79E308` and \xcd`Float`s
      can only be as large as about \xcd`3.4E38f`, a large \xcd`Double` will
      be converted to the special \xcd`Float` value of \xcd`Infinity`: 
%~~exp~~`~~`~~ ~~ ^^^ Expressions240
      \xcd`1e100 as Float` is \xcd`Infinity`.
\item Integers are coerced to smaller integer types by truncating the
      high-order bits. If the value of the large integer fits into the smaller
      integer's range, this gives the same number in the smaller type: 
%~~exp~~`~~`~~ ~~ ^^^ Expressions250
      \xcd`12 as Byte` is the \xcd`Byte`-sized 12, 
%~~exp~~`~~`~~ ~~ ^^^ Expressions260
      \xcd`-12 as Byte` is -12. 
      However, if the larger integer {\em doesn't} fit in the smaller type,
%~~exp~~`~~`~~ ~~ ^^^ Expressions270
      the numeric value and even the sign can change: \xcd`254 as Byte` is
      the \xcd`Byte`sized \xcd`-2y`.  

\item Casting an unsigned integer type to a signed integer type of the same
      size (\eg, \xcd`UInt` to \xcd`Int`) preserves 2's-complement bit pattern
      (\eg,  
      \xcd`UInt.MAX_VALUE as Int == -1`.   Casting an unsigned integer type to
      a signed integer type of a different size is equivalent to first casting
      to an unsigned integer type of the target size, and then casting to a
      signed integer type.

\item Casting a signed integer type to an unsigned one is similar.  

\end{itemize}

\subsubsection{User-defined Coercions}
\index{coercion!user-defined}

Users may define coercions from arbitrary types into the container type
\xcd`B`, and coercions from \xcd`B` to arbitrary types, by providing
\xcd`static operator` definitions for the \xcd`as` operator in the definition of
\xcd`B`.  

\begin{ex}

%~~gen ^^^ Expressions2j7z
% package Expressions2j7z;
% KNOWNFAIL
%~~vis
\begin{xten}
class Bee {
  public static operator (x:Bee) as Int = 1;
  public static operator (x:Int) as Bee = new Bee();
  def example() {
    val b:Bee = 2 as Bee; 
    assert (b as Int) == 1;
  }
}
\end{xten}
%~~siv
%
%~~neg


\end{ex}



\subsection{Conversions}
\index{conversion}
\index{type!conversion}

\paragraph{Widening numeric conversion.}
\label{WideningConversions}
A numeric type may be implicitly converted to a wider numeric type. In
particular, an implicit conversion may be performed between a numeric
type and a type to its right, below:

\begin{xten}
Byte < Short < Int < Long < Float < Double
UByte < UShort < UInt < ULong
\end{xten}

Furthermore, an unsigned integer type may be implicitly coerced a signed type
large 
enough to hold any value of the type: \xcd`UByte` to \xcd`Short`, \xcd`UShort`
to \xcd`Int`, \xcd`UInt` to \xcd`Long`.  There are no implicit conversions
from signed to unsigned numbers, since they cannot treat negatives properly.

There are no implicit conversions in cases when overflow is possible.  For
example, there is no implicit conversion between \xcd`Int` and \xcd`UInt`.  If
it is necessary to convert between these types, use \xcd`n as Int` or 
\xcd`n as UInt`, generally with a test to ensure that the value will fit and
code to handle the case in which it does not.  


\index{conversion!widening}
\index{conversion!numeric}

\paragraph{String conversion.}
Any value that is an operand of the binary
\xcd"+" operator may
be converted to \xcd"String" if the other operand is a \xcd"String".
A conversion to \xcd"String" is performed by invoking the \xcd"toString()"
method.

\index{conversion!string}

\paragraph{User defined conversions.}\label{sec:user-defined-conversions}
\index{conversion!user-defined}

The user may define implicit conversion operators from type \Xcd{A} {\em to} a
container type \Xcd{B} by specifying an operator in \Xcd{B}'s definition of the form:

\begin{xten}
  public static operator (r: A): T = ... 
\end{xten}

The return type \Xcd{T} should be a subtype of \Xcd{B}. The return
type need not be specified explicitly; it will be computed in the
usual fashion if it is not. However, it is good practice for the
programmer to specify the return type for such operators explicitly.
The return type can be more specific than simply \xcd`B`, for cases when there
is more information available.


\begin{ex}
The code for \Xcd{x10.lang.Point} contains a conversion from 
one-dimensional \xcd`Array`s of integers to \xcd`Point`s of the same length: 
\begin{xten}
  public operator (r: Array[Int](1)): Point(r.length) = make(r);
\end{xten}
This conversion is used whenever an array of integers appears in a 
context that requires a \xcd`Point`, such as subscripting. Note 
that \xcd`a` requires a \xcd`Point` of rank 2 as a subscript, and that 
a two-element \xcd`Array` (like \xcd`[2,4]`) is converted to a 
\xcd`Point(2)`.
%~~gen ^^^ Expressions4f4y
% package Expressions4f4y;
% class Example { def example() {
%~~vis
\begin{xten}
val a = new Array[String]((2..3) * (4..5), "hi!");
a([2,4]) = "converted!";
\end{xten}
%~~siv
%} } 
%~~neg


\end{ex}

\section{Parenthesized Expressions}

If \xcd`E` is any expression, \xcd`(E)` is an expression which, when
evaluated, produces the same result as \xcd`E`.   

\begin{ex}
The main use of parentheses is to write complex expressions for which the 
standard precedence order of operations is not appropriate: \xcd`1+2*3` is 7,
but \xcd`(1+2)*3` is 9.  

Similarly, but perhaps less familiarly, 
parentheses can disambiguate other expressions.  In the following code, 
\xcd`funny.f` is a field-selection expression, and so \xcd`(funny.f)()` means
``select the \xcd`f` field from \xcd`funny`, and evaluate it''.  However, 
\xcd`funny.f()` means ``evaluate the \xcd`f` method on object \xcd`funny`.''  
%~~gen ^^^ Expressions3f6f
% package Expressions3f6f;
%~~vis
\begin{xten}
class Funny {
  def f () = 1;
  val f = () => 2;
  static def example() {
    val funny = new Funny();
    assert funny.f() == 1;
    assert (funny.f)() == 2;
  }
}
\end{xten}
%~~siv
% class Hook{ def run() { Funny.example(); return true; }}
%~~neg


\end{ex}

Note that this does {\em
not} mean that \xcd`E` and \xcd`(E)` are identical in all respects; for
example, if \xcd`i` is an \xcd`Int` variable, \xcd`i++` increments \xcd`i`,
but \xcd`(i)++` is not allowed.    \xcd`++` is an assignment; it operates on
variables, not merely values, and \xcd`(i)` is simply an expression whose {\em
value} is the same as that of \xcd`i`. 
