\chapter{Places}\label{XtenPlaces}\index{places}

An \Xten{} place is a repository for data and activities, corresponding
loosely to a process or a processor. Places induce a concept of ``local''. The
activities running in a place may access data items located at that place with
the efficiency of on-chip access. Accesses to remote places may take orders of
magnitude longer. X10's system of places is designed to make this obvious.
Programmers are aware of the places of their data, and know when they are
incurring communication costs, but the actual operation to do so is easy. It's
not hard to use non-local data; it's simply hard to to do so accidentally.

The set of places available to a computation is determined at the time that
the program is started, and remains fixed through the run of the program. See
the {\tt README} documentation on how to set command line and configuration
options to set the number of places.)

Places are first-class values in X10, as instances of the built-in struct,
\xcd"x10.lang.Place".   \xcd`Place` provides a number of useful ways to
query places, such as \xcd`Place.places`, a \xcd`ValRail` of all the places
available to the current run of the program.

Every object \xcd`ob` created by the program has a {\em home place}
\xcd`ob.home`. Accesses to \xcd`ob` from activities located at \xcd`ob.home`
are privileged.  Mutable fields can only be changed from \xcd`ob.home`, and
normal fields and methods can only be accessed from there. 

However, objects can be referred to from anywhere.  Places other than
\xcd`ob.home` may have {\em remote references} to \xcd`ob`.  Remote references
convey fewer privileges than local ones, but they are far from useless.
Fields and methods can be defined to be \xcd`global`, usable from anywhere. 


\section{The Structure of Places}

Places are numbered 0 through \xcd`Places.MAX_PLACES`, stored in the field
\xcd`pl.id`.  The \xcd`ValRail` \xcd`Place.places` contains the places of the
program, in numeric order. 
The program starts by executing a \xcd`main` method at
\xcd`Place.FIRST_PLACE`, which is \xcd`Place.places(0)`; see
\Sref{initial-computation}. 

Operations on places include \xcd`pl.next()`, which gives the next entry
(looping around) in \xcd`Place.places` and its opposite \xcd`pl.prev()`. In
particular, \xcd`here.next()` means ``a place other than \xcd`here`'', except
in single-place executions.
%~~exp~~`~~`~~pl:Place~~
There are also a number of tests, like \xcd`pl.isSPE()` and 
%~~exp~~`~~`~~pl:Place ~~
\xcd`pl.isCUDA()`, which test for particular kinds of processors.

Future versions of the language may permit user-definable
places, and the ability to dynamically create places.

Place expressions \index{place expression} (\viz, expressions of type
\xcd`Place`), such as \xcd`here` 
%~~exp~~`~~`~~ ob:Object~~
and \xcd`ob.home`, are used in \xcd`at` and
\xcd`asynch` statements.  



\section{\Xcd{here}}\index{here}\label{Here}

The variable \xcd"here" is always bound to the place at which the current
computation is running, in the same way that \xcd`this` is always bound to the
instance of the current class (for non-static code), or \xcd`self` is bound to
the instance of the type currently being constrained.  
\xcd`here` may denote different places in the same method body, due to
place-shifting operations. In the following code, \xcd`here` has one value at
\xcd`h0`, and a different one at \xcd`h1`. 
%~~gen
% package places.are.For.Graces;
% class Example {
% def example() {
%~~vis
\begin{xten}
val h0 = here;
at (here.next()) {
  val h1 = here; 
  assert (h0 != h1);
}
\end{xten}
%~~siv
%} } 
%~~neg
\noindent
(Similar examples show that \xcd`self` and \xcd`this` have the same behavior:
\xcd`self` can be modified by constrained types appearing inside of type
constraints, and \xcd`this` by inner classes.)


\begin{grammar}
Expression \: \xcd"here" \\
\end{grammar}

\begin{example}
For example, the following method looks through a collection of \xcd`Thing`s
for ones which belong in the current place \xcd`here`, and deals with the
things which do.  Note that every object \xcd`thing` has a property
\xcd`thing.home` giving its home location.
%~~gen
%package Places.Are.For.Graces;;
%abstract class Thing {}
%class DoMine {
%  static def dealWith(Thing) {}	
%~~vis
\begin{xten}
  public static def dealWithLocal(things: Rail[Thing]!) {
     for(thing in things) {
    	 if (thing.home == here) 
            dealWith(thing);
     }	  
  }
\end{xten}
%~~siv
%}
%~~neg



\end{example}

\xcd`here` is frequently used in constraints, quite often of the form
\xcd`ob.home == here`. Such constraints are necessary to check that a
non-global method can be called on \xcd`ob`: 


%TODO~~gen
% package Places.Are.For.Aces.Of.Spaces;
% class Thing {
% def nonGlobalMethod():Void{}
% static def example() {
%TODO~~vis
\begin{xten}
val ob : Thing{self.home == here} = new Thing();
ob.nonGlobalMethod();
\end{xten}
%TODO~~siv
%}}
%TODO~~neg

This idiom is so common and useful that the constraint
\xcd`T{self.home==here}` can be abbreviated as \xcd`T!`: 

%~~gen
% package Places.Are.For.Aces.Of.Graces;
% class Thing {
% def nonGlobalMethod():Void{}
% static def example() {
%~~vis
\begin{xten}
val ob : Thing! = new Thing();
ob.nonGlobalMethod();
\end{xten}
%~~siv
%}}
%~~neg


\limitationx {\em In the current implementation, sometimes \xcd^T{self.home==here}^
does not work, though \Xcd{T!} does.}

