\chapter{Lexical structure}


Lexically a program consists of a stream of white space, comments,
identifiers, keywords, literals, separators and operators, all of them
composed of ASCII characters. 

\paragraph{Whitespace}
\index{white space}
% Whitespace \index{whitespace} follows \java{} rules \cite[Chapter 3.6]{jls2}.
ASCII space, horizontal tab (HT), form feed (FF) and line
terminators constitute white space.

\paragraph{Comments}
\index{comment}
% Comments \index{comments} follows \java{} rules
% \cite[Chapter 3.7]{jls2}. 
All text included within the ASCII characters ``\xcd"/*"'' and
``\xcd"*/"'' is
considered a comment and ignored; nested comments are not
allowed.  All text from the ASCII characters
``\xcd"//"'' to the end of line is considered a comment and is ignored.

\paragraph{Identifiers}
\index{identifier}
\index{variable name}

Identifiers consist of a single letter followed by zero or more
letters or digits.
The letters are the ASCII characters \xcd`a` through \xcd`z`, \xcd`A` through
\xcd`Z`, and \xcd`_`.
Digits are defined as the ASCII characters \xcd"0" through \xcd"9". Case is
significant; \xcd`a` and \xcd`A` are distinct identifiers, \xcd`as` is a
keyword, but \xcd`As` and \xcd`AS` are identifiers.

\paragraph{Keywords}
\index{keywords}
\Xten{} reserves the following keywords:
\begin{xten}
abstract       false          offers         transient      
as             final          operator       true           
assert         finally        package        try            
async          finish         private        var            
ateach         for            property       when           
break          goto           protected      while          
case           if             public         at             
catch          implements     return         atomic         
class          import         self           await          
continue       in             static         clocked        
def            instanceof     struct         here           
default        interface      super          next           
do             native         switch         offer          
else           new            this           resume         
extends        null           throw          type           
\end{xten}
Note that the primitive types are not considered keywords.

\paragraph{Literals}\label{Literals}\index{literal}

Briefly, \XtenCurrVer{} uses fairly standard syntax for its literals:
integers, unsigned integers, floating point numbers, booleans, 
characters, strings, and \xcd"null".  The most exotic points are (1) unsigned
numbers are marked by a \xcd`u` and cannot have a sign; (2) \xcd`true` and
\xcd`false` are the literals for the booleans; and (3) floating point numbers
are \xcd`Double` unless marked with an \xcd`f` for \xcd`Float`. 

Less briefly, we use the following abbreviations: 
\begin{displaymath}
\begin{array}{rcll}
d &=& \mbox{one or more decimal digits}\\
d_8 &=& \mbox{one or more octal digits}\\
d_{16} &=& \mbox{one or more hexadecimal digits, using \xcd`a`-\xcd`f`
for 10-15}\\
i &=& d 
        \mathbin{|} {\tt 0} d_8 
        \mathbin{|} {\tt 0x} d_{16}
        \mathbin{|} {\tt 0X} d_{16}
\\
s &=& \mbox{optional \xcd`+` or \xcd`-`}\\
b &=& d 
          \mathbin{|} d {\tt .}
          \mathbin{|} d {\tt .} d
          \mathbin{|}  {\tt .} d \\
x &=& ({\tt e } \mathbin{|} {\tt E})
         s
         d \\
f &=& b x
\end{array}
\end{displaymath}

\begin{itemize}

\item \xcd`true` and \xcd`false` are the \xcd`Boolean` literals. \index{Boolean!literal}\index{literal!Boolean}

\item \xcd`null` is a literal for the null value.  It has type
      \xcd`Any{self==null}`. \index{null} \index{object!literal}

\item \index{Int!literal}\index{literal!integer}
\xcd`Int` literals have the form {$si$}; \eg, \xcd`123`,
      \xcd`-321` are decimal \xcd`Int`s, \xcd`0123` and \xcd`-0321` are octal
      \xcd`Int`s, and \xcd`0x123`, \xcd`-0X321`,  \xcd`0xBED`, and \xcd`0XEBEC` are
      hexadecimal \xcd`Int`s.  

\item \xcd`Long` literals have the form {$si{\tt l}$} or
      {$si{\tt L}$}. \Eg, \xcd`1234567890L`  and \xcd`0xBAGEL` are \xcd`Long` literals. 

\item \xcd`UInt` literals have the form {$i{\tt u}$} or {$i {\tt U}$}.
      \Eg, \xcd`123u`, \xcd`0123u`, and \xcd`0xBEAU` are \xcd`UInt` literals.

\item \xcd`ULong` literals have the form {$i {\tt ul}$} or {$i {\tt
      lu}$}, or capital versions of those.  For example, 
      \xcd`123ul`, \xcd`0124567012ul`,  \xcd`0xFLU`, \xcd`OXba1eful`, and \xcd`0xDecafC0ffeefUL` are \xcd`ULong`
      literals. 

\item \xcd`Short` literals have the form {$si{\tt s}$} or
      {$si{\tt S}$}. \Eg,  414S, \xcd`OxACES` and \xcd`7001s` are short
      literals. 

\item \xcd`UShort` literals  form {$i {\tt us}$} or {$i {\tt
      su}$}, or capital versions of those.  For example, \xcd`609US`, 
      \xcd`107us`, and \xcd`OxBeaus` are unsigned short literals.

\item \xcd`Byte` literals have the form  {$si{\tt y}$} or
      {$si{\tt Y}$}.  (The letter \xcd`B` cannot be used for bytes, as it is
      a hexadecimal digit.)  \xcd`50Y` and \xcd`OxBABY` are byte literals.

\item \xcd`UByte` literals have the form {$i {\tt uy}$} or {$i {\tt yu}$}, or
      capitalized versions of those.  For example, \xcd`9uy` and \xcd`OxBUY`
      are \xcd`UByte` literals.
      


\item \xcd`Float` literals have the form {$s f {\tt f}$} or  {$s
\index{float!literal}
\index{literal!float}
      f {\tt F}$}.  Note that the floating-point marker letter \xcd`f` is
      required: unmarked floating-point-looking literals are \xcd`Double`. 
      \Eg, \xcd`1f`, \xcd`6.023E+32f`, \xcd`6.626068E-34F` are \xcd`Float`
      literals. 

\item \xcd`Double` literals have the form {$s f$}\footnote{Except that
\index{double!literal}
\index{literal!double}
      literals like \xcd`1` 
      which match both {$i$} and {$f$} are counted as
      integers, not \xcd`Double`; \xcd`Double`s require a decimal
      point, an exponent, or the \xcd`d` marker.
      }, {$s f {\tt
      D}$}, and {$s f {\tt d}$}.  
      \Eg, \xcd`0.0`, \xcd`0e100`, \xcd`1.3D`,  \xcd`229792458d`, and \xcd`314159265e-8`
      are \xcd`Double` literals.

\item 
\index{char!literal}
\index{literal!char}
\xcd`Char` literals have one of the following forms: 
      \begin{itemize}
      \item \xcd`'`{\it c}\xcd`'` where {\em c} is any printing ASCII
            character other than 
            \xcd`\` or \xcd`'`, representing the character {\em c} itself; 
            \eg, \xcd`'!'`;
      \item \xcd`'\b'`, representing backspace;
      \item \xcd`'\t'`, representing tab;
      \item \xcd`'\n'`, representing newline;
      \item \xcd`'\f'`, representing form feed;
      \item \xcd`'\r'`, representing return;
      \item \xcd`'\''`, representing single-quote;
      \item \xcd`'\"'`, representing double-quote;
      \item \xcd`'\\'`, representing backslash;
      \item \xcd`'\`{\em dd}\xcd`'`, where {\em dd} is one or more octal
            digits, representing the one-byte character numbered {\em dd}; it
            is an error if {\em dd}{$>0377$}.      
      \end{itemize}

\item
\index{string!literal} 
\index{literal!string}
\xcd`String` literals consist of a double-quote \xcd`"`, followed by
      zero or more of the contents of a \xcd`Char` literal, followed by
      another double quote.  \Eg, \xcd`"hi!"`, \xcd`""`.


\end{itemize}



\paragraph{Separators}
\Xten{} has the following separators and delimiters:
\begin{xten}
( )  { }  [ ]  ;  ,  .
\end{xten}

\paragraph{Operators}
\index{operator}
\Xten{} has the following operator,  type constructor, and miscellaneous symbols.  (\xcd`?` and
\xcd`:` comprise a single ternary operator, but are written separately.)
\begin{xten}
==  !=  <   >   <=  >=
&&  ||  &   |   ^
<<  >>  >>>
+   -   *   /   %
++  --  !   ~
&=  |=  ^=
<<= >>= >>>=
+=  -=  *=  /=  %=
=   ?   :  =>  ->
<:  :>  @   ..
\end{xten}




