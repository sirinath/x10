FX10 is a formal calculus for X10 intended to  complement Featherweight Java
(FJ).  It models imperative aspects of X10 including the concurrency
constructs \finish{} and \async{}. 
FX10 models the heart of the field initialization problem:
    a field can be read only after it  is definitely assigned. 

%    and that final (\code{val}) fields can be assigned exactly once (in \Ref{Section}{val}).

The basic idea behind the formalization is very straightforward. We
break up the formalization into two distinct but interacting
subsystems, a {\em type-system}\Ref{Section}{Type system} and an {\em
  effects} system \Ref{Section}{Effect system}. The type
system is completely standard -- think the system of FJ, adapted to
the richer constructs of FX10.

The effect system is built on a very simple {\em logic of
  initialization assertions}.  The primitive formula $+\hx$
($+\hp.\hf$) asserts that the variable $\hx$ (the field $\hf$ of
$\hp$) is definitely initialized with a cooked object, and the formula
$-\hp.\hf$ asserts that $\hp.\hf$ is being initialized by a concurrent
activity (and hence $\hp.\hf$ will be definitely initialized once an
enclosing $\finish$ is crossed).\footnote{There is no need for the
  formula $-\hx$ since local variables declared within an $\async$
  cannot be accessed outside it.}  An {\em initialization formula}
$\phi$ or $\psi$ is simply a conjunction of such formulas $\phi \wedge
\psi$ or an existential quantification $\exists{}~\hx.\phi$.
An {\em
  effects assertion} $\phi~S~\psi$ (for $S$ a statement) is read as a
partial correctness assertion: when executed in a heap that satisfies
the constraint $\phi$, $S$ will on termination result in a heap that
satisfies $\psi$.  Since we do not model \code{null}, our
formalization can be particularly simple: variables once initialized
stay initialized, hence $S$ will also satisfy $\phi$.

Another feature of our approach is that unlike Masked
Types~\cite{XinQi:2009}, the source program syntax does not permit the
specification of initialization assertions. Instead we use a standard
least fixed point computation to automatically decorate each method
$\hdef~\hm(\ol{\hx}:\ol{\hC})\{\hS\}$ with pairs $(\phi, \psi)$ (in the free
variables $\hthis, \ol{\hx}$) such that under the assumption that all
methods satisfy their corresponding assertion we can show that
$\phi~\hS~\psi$.\footnote{Note that this approach permits a formal $\hx$ to a
  method to be ``uncooked'' ($\phi$ does not entail $+\hx$ or
  $+\hx.\hf$ for any field $\hf$) or partially cooked ($\phi$ does not
  entail $+\hx$ but may entail $+\hx.\hf$ for some fields $\hf$). As a
  result of the method invocation the formal may become more cooked.}
This computation must be sensitive to the semantics
of method overriding, that is a method with decoration $(\phi,\psi)$
can only be over-ridden by a method with decoration $(\phi',\psi')$
that is ``at least as strong as'' $(\phi,\psi)$ (viz, it must be the
case that ($\phi \vdash \phi'$ and $\psi' \vdash \psi$). Further, if
the method is not marked \code{@NonEscaping}, then $\phi$ is required
to entail $+\hthis$ (that is, \code{this} is cooked), and if it marked
\code{@NoThisAccess} then $\phi,\psi$ cannot have \code{this} free.

\comment{{\sc We also have to check that if it is marked
    \code{@NonEscaping} then it does not escape.}}

By not permitting the user to specify initialization assertions we
make the source language much simpler than \cite{XinQi:2009} and
usable by most programmers. The down side is that some initialization
idioms, such as cyclic initialization, are not expressible.

For reasons of space we do not include the (standard) details behind
the decoration of methods with initialization assertions. We also omit
many extensions (such as generics, interfaces, constraints, casting,
inner classes, overloading, co-variant return types, private/final
locals, field initializers etc.) necessary to establish the soundness
of initialization for the full language discussed in the first half of
the paper. FX10 also does not model places because the language design
decision to only permit cooked objects to cross places means that the
rules for \code{at} are routine.

We use the usual notation of~$\ol{\hx}$ to represent a vector or set of~$\hx_1, \ldots,\hx_n$.
A program~\hP is a pair of class declarations~$\ol{\hL}$ (that is assumed to be global information)
    and a statement \hS.
%Like in masked-types, there is no \hnull value in the language, because none is needed for object initialization.
%Phrased differently,
%    FX10 guarantees that a field is read only after assigned, so no need to initialize it with \hnull.
%In the formalism, an object is represented as a mapping from initialized fields to their values
%    (so initially the mapping is empty because no field is initialized).

\paragraph{Overview of formalism}
\Ref{Section}{Syntax} presents the syntax of FX10.
% todo
\Ref{Section}{Type system} briefly describes the type
system. \Ref{Section}{Effect system} defines the flow-sensitive rules for
    ($\phi~\hS~\psi$), while
    \Ref{Section}{Reduction} gives the reduction rules for statements
    and expressions ($\hS,\hH \reducesto \hS',\hH' ~|~\hH'$
    and~$\he,\hH \reduce \hl,\hH'$). Finally \Ref{Section}{Results}
    presents the main formal result (soundness theorem).
%Finally, \Ref{Section}{val} extends the formalism with \hval and \hvar fields.


\Subsection{Syntax}

\begin{figure}[htpb!]
\begin{center}
\begin{tabular}{|l|l|}
\hline

$\hP ::= \ol{\hL},\hS$ & Program. \\

$\hL ::= \hclass ~ \hC~\hextends~\hD~\lb~\ol{\hF};~\ol{\hM}~\rb$
& cLass declaration. \\

$\hF ::= \hvar\,\hf:\hC$
& Field declaration. \\

$\hM ::= \hG\ \hdef\ \hm(\ol{\hx}:\ol{\hC}):\hC\{\hS\}$
& Method declaration. \\

$\hG ::= \code{@NonEscaping} ~~|~~ \code{@NoThisAccess}$
& Method modifier. \\

$\hp ::= \hl ~~|~~ \hx$
& Path. \\ %(location or parameter)

$\he ::=  \hp.\hf  ~|~ \hnew{\hC}$
& Expressions. \\ %: locations, parameters, field access\&assignment,  %invocation, \code{new}

$\hS ::=  \hp.\hf = \hp; ~|~ \hp.\hm(\ol{\hp});  ~|~ \hval{\hx}{\he}{\hS}$ &\\
$~~~~|~ \hfinish~\hS~|~ \hasync~\hS ~|~ \hS\ \hS$
& Statements. \\ %: locations, parameters, field access\&assignment, invocation, \code{new}

\hline
\end{tabular}
\end{center}
\caption{FX10 Syntax.
    The terminals are locations (\hl), parameters and \hthis (\hx), field name (\hf), method name (\hm), class name (\hB,\hC,\hD,\hObject),
        and keywords (\hescaping, \nonescaping, \hhnew, \finish, \async, \code{val}).
    The program source code cannot contain locations (\hl), because locations are only created during execution/reduction in \RULE{R-New} of \Ref{Figure}{reduction}.
    }
\label{Figure:syntax}
\end{figure}

\Ref{Figure}{syntax} shows the syntax of FX10.
%(\Ref{Section}{val} will later add the \hval and \hvar field modifiers.)
Expression~$\hval{\hx}{\he}{\hS}$ evaluates $\he$, assigns it to a
new variable $\hx$, and then evaluates \hS. The scope of \he{} is \hS.

The syntax is similar to X10 real syntax with the following difference:
%Non-escaping methods are marked with \code{@NonEscaping}, such methods
%can be invoked on raw objects (and can be used to initialize them).
FX10 does not have constructors; instead, an object is initialized by assigning to its fields or
    by calling
    non-escaping methods.

%To simplify presentation, we allow only paths~$\hp$
%    (variables \hx at
%    compile time, or heap locations \hl at run time) to appear in field access, assignment and method call.
%This does not restrict expressiveness, because of \code{val} expressions.

\Subsection{Type system}
%\begin{lstlisting}
%class D extends Object { f:Object; }
%val d:D = new D; d.f = new Object
%\end{lstlisting}
%Then, the type of \code{d} is \code{D} where field \code{f} is uninitialized
%    (which we formally write below as $\code{D}\myinit{\hf}{\hf}$).
%Therefore, reading \code{d.f} would be illegal, but writing to \code{d.f} would update the type of \code{d}
%    to be fully initialized.

The type system for FX10 checks that every parameter and variable has
a type (a type is the name of a class), and that a variable of type
\hC can be assigned only expressions whose type is a subclass of \hC,
and can only be the receiver of invocations of methods defined in
\hC. The type system is formalized along the lines of FJ. No
complications are introduced by the extra features of FX10 --
assignable fields, local variable declarations, \hfinish{} and
\hasync{}. We omit details for lack of space and because they are
completely routine.

In the rest of this section we shall assume that the program being
considered $\hL,\hS$ is well-typed.

\Subsection{Effect system}
\begin{figure*}[t]
\begin{center}
\begin{tabular}{|c|}
\hline

$\typerule{
 \phi \vdash +\hp.\hf \gap \phi, +\hx~\hS~\psi
}{
 \phi~\hval{\hx}{\hp.\hf}{\hS}~ \exists \hx.\psi
}$~\RULE{(T-Access)}
\quad
$\typerule{
  \phi~\hS~\psi
}{
 \phi~ \hval{\hx}{\hnew{\hC()}}{\hS}~\exists \hx.\psi
}$~\RULE{(T-New)}
\quad
$\typerule{
  \phi \vdash +\hq
}{
 \phi ~ \hp.\hf=\hq ~ + \hp.\hf
}$~\RULE{(T-Assign)}
\\\\

$\typerule{
    \phi~\hS~\psi
}{
  \begin{array}{l}
    \phi~\finish\,\hS~ +\psi \\
    \phi~\async\,\hS~ -\psi
  \end{array}
}$~\RULE{(T-Finish,Async)}

\quad
$\typerule{
  \phi~\hS_1~\psi_1
        \gap
    \phi,\psi_1~\hS_2~\psi_2
}{
  \phi~\hS_1;\hS_2~\psi_1,\psi_2
}$~\RULE{(T-Seq)}
\quad

$\typerule{
\hm(\ol{\hx}):: \phi' \Rightarrow \psi' \gap \phi \vdash \phi'[\hp/\hthis,\ol{\hp}/\ol{\hx}]
}{
\phi~\hp.\hm(\ol{\hp})~\psi'[\hp/\hthis,\ol{\hp}/\ol{\hx}]
}$~\RULE{(T-Invoke)}\\

\hline
\end{tabular}
\end{center}
\caption{FX10 Effect System ($\phi~\hS~\psi$)}
\label{Figure:effects}
\end{figure*}

We use a simple logic of initialization for our basic assertions. 
This is an intuitionistic logic over the 
primitive formulas $+\hx$ (the variable or parameter $\hx$ is
initialized), $+\hp.\hf$ (the field $\hp.\hf$ is initialized), and
$-\hp.\hf$ (the field $\hp.\hf$ is being concurrently
initialized). We are only concerned with conjunctions and existential
quantifications over these formulas:
$$
 \phi,\psi {::=}  \htrue ~|~ +\hx ~|~ +\hp.\hf ~|~ -\hp.\hf ~|~ \phi \wedge \psi
 ~|~ \exists \hx.\phi
$$

The notion of substitution on formulas $\phi\char`[\ol{\hx}/\ol{\hz}\char`]$ is
specified in a standard fashion.

The inference relation is the usual intuitionistic implication over
these formulas, and the following additional proof rules: 
(1)~$\phi\vdash +\hp.\hf$ if $\phi\vdash +\hp$; and (2)~if
the exact class of $\hx$ is $\hC$, and $\hC$ has the fields $\ol{\hf}$,
then $\phi\vdash +\hp$ if $\phi\vdash \hp.\hf_i$, for each $i$. 

A {\em heap} is a mapping from a given set of locations to {\em
  objects}. An object is a pair $C(u)$ where $C$ is a class (the exact
class of the object), and $u$ is a partial map from the fields of $C$
to locations. We say the object is {\em total} if its map is total.

An {\em annotation} $N$ for a heap $H$ maps each $l \in \dom(H)$ to a
possibly empty set of fields $a(H(l))$ of the class of $H(l)$ disjoint
from $\dom(H(l))$. (These are the fields currently being
asynchronously initialized.) The logic of initialization described
above is clearly sound for the obvious interpretation of formulas over
annotated heaps. For future reference, we say that that a heap $H$
{\em satisfies} $\phi$ if there is some annotation $N$ (and some
valuation $v$ assigning locations in $\dom(H)$ to free variables of
$\phi$) such that $\phi$ evaluates to true.

The proof rules for the judgement $\phi~\hS~\psi$ are given in
Figure~\ref{Figure:effects}. They use two syntactic operations on initialization
formulas defined as follows. 
$+\psi$ is defined inductively as follows: 
$+\htrue=\htrue$, 
$++\hx=+\hx$, 
$+\pm\hp.\hf=+\hp.\hf$, 
$+(\phi \wedge\psi)=(+\phi)\wedge(+\psi)$
and $+\exists x.\phi=\exists x.+\phi$. 
$-\psi$ is defined similarly:
$-\htrue=\htrue$, 
$-+\hx=\htrue$, 
$-\pm\hp.\hf=-\hp.\hf$, 
$-(\phi \wedge\psi)=(-\phi)\wedge(-\psi)$
and $-\exists x.\phi=\exists x.-\phi$.\footnote{It turns out that
  expressions of the form $-+\hx$ never arise since \async{} is a
  scoping construct, hence a local variable declared within it never
  ``crosses'' out.}

The rule~\RULE{(T-Access)} can be read as asserting: if $\phi$ entails
the field $\hp.\hf$ is initialized, and with the assumption $\phi$
(together with $+\hx$ which states that $\hx$ is initialized to a
cooked object), we can establish that execution of $\hS$ satisfies the assertion
$\psi$ then we can establish that execution of 
$\hval{\hx}{\hp.\hf}{\hS}$ in 
(a heap satisfying) $\phi$ establishes $\exists \hx.\psi$. Here we
must take care to project $\hx$ out of $\psi$ since $\hx$ is not
accessible outside its scope $\hS$. 
The rule ~\RULE{(T-New)} can be read in a similar way except that when
executing $\hS$ we can make no assumption that $\hx$ is initialized,
since it has been initialized with a raw object (none of its fields
are initialized). Subsequent assignments to the fields of $\hx$ will
introduce effects recording that those fields have been initialized.
The rule~\RULE{(T-Assign)} checks that \hq{} is initialized to a
cooked object and then asserts that $\hp.\hf$ is initialized to a
cooked object.
The rule~\RULE{(T-Finish)} can be understood as recording that after a
\finish{} has been ``crossed'' all asynchronous initializations $\psi$
can be considered to have been performed $\phi$. Conversely, 
the rule~\RULE{(T-Async)} states that any initializations must be
considered asynchronous to the surrounding context. 
The rule~\RULE{(T-Seq)} is a slight variation of the stadard rule for
sequential composition that permits $\phi$ to be used in the
antecedent of $\hS_2$, exploiting monotonicity of effects. Note the
effects recorded for $\hS_1;\hS_2$ are a conjunction of the effects
recorded for $\hS_1$ and $\hS_2$.  
The rule~\RULE{(T-Invoke)} is routine.

As an example, consider the following classes. Assertions are provided
in-line. 
\begin{lstlisting}
class A extends Object {
  var f:A; var g:A; var h:A;
  @NonEscaping def build(a:A) {
    // inferred decoration: phi => psi
    // phi= +this.g, +a
    // psi= -this.h, +this.f
    // checks phi implies +this.g
    val x = this.g;
    // adds -this.h to psi
    async this.h=x;
    finish {
      // checks phi implies +a
      // adds -this.f to psi
      async this.f=a;
    }
    // adds +this.f to psi
  }
}
class B extends A { f3:A; }
\end{lstlisting}


\Subsection{Reduction}

The reduction relation is described in
Figure~\ref{Figure:reduction}. An S-configuration is of the form
$\hS,\hH$ where \hS{} is a statement and $\hH$ a heap (representing a
computation which is to execute $\hS$ in the heap $\hH$), or $\hH$
(representing terminated computation). An
E-configuration is of the form $\he,\hH$ and represents the
computation which is to evaluate $\he$ in the configuration $\hH$. The
set of {\em values} is the set of locations; hence E-configurations of
the form $\hl,\hH$ are terminaal. 

Two transition relations $\reduce{}$ are defined, one over S-configurations and
the other over E-configurations. 
For $X$ a partial function, we use the notation $X[v \mapsto
  e]$ to represent the partial function which is the same as $X$
except that it maps $v$ to $e$.
The rules defining these relations are 
standard. 
The only minor novelty is in how \async{} is defined. The
critical rule is the last rule in~\RULE{(R-Step)}  -- it specifies the 
``asynchronous'' nature of \async{} by permitting \hS{} to make a step
even if it is preceded by $\async{} \hS_1$.  
%
The rule ~\RULE{(R-New)} returns a new location that is bound to a new
object that is an instance of \hC{} with none of its fields initialized.
%
The rule 
~\RULE{(R-Access)} ensures that the field is initialized before it is
read ($\hf_i$ is contained in $\ol{\hf}$).

\begin{figure*}[t]
\begin{center}
\begin{tabular}{|c|}
\hline
$\typerule{
 \hS,\hH \reduce \hH'
}{
  \begin{array}{l}
    \finish~\hS,\hH \reduce \hH'\\
    \async~\hS,\hH \reduce \hH'\\
    \hS;\hS', \hH \reduce \hS',\hH'\\
  \end{array}
}$~\RULE{(R-Term)}
~
$\typerule{
 \hS,\hH \reduce \hS', \hH'
}{
  \begin{array}{l}
    \finish~\hS,\hH \reduce \finish~\hS',\hH'\\
    \async~\hS,\hH \reduce \async~\hS', \hH'\\
    \hS;\hS_1, \hH \reduce \hS';\hS_1,\hH'\\
    \async~\hS_1;\hS, \hH \reduce \async~\hS_1;\hS', \hH'\\
  \end{array}
}$~\RULE{(R-Step)}
~
$\typerule{
  \he,\hH \reduce \hl,\hH'
}{
  \hval{\hx}{\he}{\hS},\hH \reduce \hS[\hl/\hx], \hH'
}$~\RULE{(R-Val)}
\\\\

$\typerule{
    \hl' \not \in \dom(\hH)
}{
  \hnew{\hC},\hH \reduce \hl',\hH[ \hl' \mapsto \hC()]
}$~\RULE{(R-New)}
\quad
$\typerule{
    \hH(\hl')=\hC(\ldots)
        \gap
    \mbody{}(\hm,\hC)=\ol{\hx}.\hS
}{
  \hl'.\hm(\ol{\hl}),\hH \reduce \hS[\ol{\hl}/\ol{\hx},\hl'/\hthis],\hH
}$~\RULE{(R-Invoke)}
\quad

\\\\

$\typerule{
    \hH(\hl)=\hC(\ol{\hf}\mapsto\ol{\hl'})
}{
  \hl.\hf_i,\hH \reduce \hl_i',\hH
}$~\RULE{(R-Access)}
\quad
$\typerule{
    \hH(\hl)=\hC(\hF)
}{
  \hl.\hf=\hl',\hH \reduce \hH[ \hl \mapsto \hC(\hF[ \hf \mapsto \hl'])]
}$~\RULE{(R-Assign)}
\\
\hline
\end{tabular}
\end{center}
\caption{FX10 Reduction Rules ($\hS,\hH \reducesto \hS',\hH' ~|~\hH'$ and~$\he,\hH \reduce \hl,\hH'$).}
\label{Figure:reduction}
\end{figure*}


\Subsection{Results}

We say a heap $H$ is {\em cooked} if a field can point only to cooked
objects, i.e.{} for every object $o=C(u)$ in the range of $H$ and
every field $f \in \dom(u)$ it is the case that $u(f)\in \dom (H)$ and
$H(u(f))$ is a total object. We shall only consider cooked heaps (valid
programs will only produce cooked heaps).

A {\em heap typing} $T$ is a mapping from locations to classes. $H$ is
said to be typed by $T$ if for each $l\in\dom(H)$, the class of $H(l)$
is a subclass of $T(l)$.  Since our treatment separates out effects
from types, and the treatment of types is standard, we shall assume
that all programs and heaps are typed.

We say that a program \hP{} is {\em proper} if it is well-typed using
(an appropriate variation of) FJ rules, and each method can be
decorated with pre-post assertions $(\phi,\psi)$. The decorations must
satisfy the property that under the assumption that every method
satisfies its assertion (this is for use in recursive calls) we can
establish for every method $\hdef\hm(\ol{\hx}:\ol{\hC})\{\hS\}$ with
assertion $(\phi,\psi)$ that it is the case that the free variables of
$\phi,\psi$ are contained in $\this,\ol{\hx}$, and that
$\phi~\hS~\psi$.

Consider a program $\hP,\hS$, where \hP is proper.  Then in a cooked,
typed heap, \hS can always progress (or terminate):

\begin{Theorem}[preservation]
  \textbf{(Progress and Preservation)} For every closed
  statement~$\hS$, assertion $\phi~\hS~\psi$, and a cooked, typed
  heap~$H$ satisfying $\phi$: either there exists a cooked, typed heap $\hH'$
  satisfying $\psi$ such that $\hS,\hH \reducesto \hH'$; or there
  exists an $\hS',\phi'$ such that $\phi'~\hS'~\psi$ and there exists
  a cooked, typed $\hH'$ satisfying $\phi'$ such that $\hS,\hH \reducesto
  \hS',\hH'$.

\end{Theorem}
\begin{proof}
See our technical report.
\end{proof}
Because our reduction rules only allow reads from initialized fields,
a corollary is that a field can only be read after it was assigned.
