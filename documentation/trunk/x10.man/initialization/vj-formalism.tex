FX10 is similar to Featherweight Java (FJ), however it is both imperative (using a heap) and it
    models X10 specific constructs such as \finish and \async.
FX10 models the heart of the field initialization problem:
    a field can be read only after it  was definitely assigned.
%    and that final (\code{val}) fields can be assigned exactly once (in \Ref{Section}{val}).
FX10 type-checking rules are flow-sensitive and are similar to those in Masked Types~\cite{XinQi:2009},
    with two major differences:
    (i)~simplicity,
    (ii)~concurrency support.
FX10 is simpler than masked types, but also less expressive, e.g., FX10 cannot express cyclic initialization.
The simplicity stems from the fact that masked are not visible in the program itself.
Instead, masks are used only during effects-checking to describe partially initialized objects.
In addition, FX10 models concurrency constructs such as \finish and \async (that do not exist in masked types).

FX10 only model the core X10 relevant to the field initialization problem,
    so it does not include many X10 features such as
    generics, interfaces, constraints, casting, inner classes, overloading, co-variant return type, private/final,
    locals, field initializers, etc.
FX10 also does not model places because the only raw object is \this,
        and a raw \this cannot be captured by an \code{at},
        thus only cooked objects can cross places.
Finally, instead of inferring the method return type and what fields a method reads and writes,
    everything is explicit in the syntax.
Each non-escaping method has a method modifier~$\nonescaping(R;S;A)$, where
            $R$ is the set of fields that can be read by the method,
                $S$ are the fields that must be definitely-assigned by the method,
                and $A$ are the fields that must be definitely-asynchronously-assigned by the method
                (see \Ref{Figure}{Read-Write-Order}).

We use the usual notation of~$\ol{\hx}$ to represent a vector or set of~$\hx_1, \ldots,\hx_n$.
A program~\hP is a pair of class declarations~$\ol{\hL}$ (that is assumed to be global information)
    and an expression \he.
Like in masked-types, there is no \hnull value in the language, because none is needed for object initialization.
Phrased differently,
    FX10 guarantees that a field is read only after assigned, so no need to initialize it with \hnull.
In the formalism, an object is represented as a mapping from initialized fields to their values
    (so initially the mapping is empty because no field is initialized).

\paragraph{Overview of formalism}
\Ref{Section}{Syntax} presents the syntax of FX10.
\Ref{Section}{Typing} defines various helper functions and shows the flow-sensitive typing rules
    (e.g., $\Gdash \he:\hT, \Gamma'$ denotes that expression~\he has type \hT in environment $\Gamma$, and $\Gamma'$ is the new environment after executing \he).
\Ref{Section}{Reduction} gives the reduction rules ($\he,H \reduce \he',H'$) and our soundness theorem.
%Finally, \Ref{Section}{val} extends the formalism with \hval and \hvar fields.


\Subsection{Syntax}


\begin{figure}[htpb!]
\begin{center}
\begin{tabular}{|l|l|}
\hline

$\hP ::= \ol{\hL},\hS$ & Program. \\

$\hL ::= \hclass ~ \hC~\hextends~\hD~\lb~\ol{\hF};~\ol{\hM}~\rb$
& cLass declaration. \\

$\hF ::= \mbox{var}\,\hf:\hC$
& Field declaration. \\

$\hM ::= \hG\ \mbox{def}\ \hm(\ol{\hx}:\ol{\hC}):\hC\{\hS\}$
& Method declaration. \\

$\hG ::= \hescaping ~~|~~ \nonescaping$
& Method modifier. \\

$\hp ::= \hl ~~|~~ \hx$
& Path. \\ %(location or parameter)

$\he ::=  \hp.\hf  ~|~ \hnew{\hC}$ 
& Expressions. \\ %: locations, parameters, field access\&assignment,  %invocation, \code{new}

$\hS ::=  \hp.\hf = \hp; ~|~ \hp.\hm(\ol{\hp});  ~|~ \hval{\hx}{\he};$ &\\
$~~~~|~ \hfinish~\hS~|~ \hasync~\hS ~|~ \hS\ \hS$
& Statements. \\ %: locations, parameters, field access\&assignment, invocation, \code{new}

\hline
\end{tabular}
\end{center}
\caption{FX10 Syntax.
    The terminals are locations (\hl), parameters and \this (\hx), field name (\hf), method name (\hm), class name (\hB,\hC,\hD,\hObject),
        and keywords (\hescaping, \nonescaping, \hhnew, \finish, \async, \code{val}).
    The program source code cannot contain locations (\hl), because locations are only created during execution/reduction in \RULE{R-New} of \Ref{Figure}{reduction}.
    }
\label{Figure:syntax}
\end{figure}


\Ref{Figure}{syntax} shows the syntax of FX10.
%(\Ref{Section}{val} will later add the \hval and \hvar field modifiers.)
Expression~$\hval{\hx}{\he}{\he'}$ means to evaluate first \he then to evaluate and return~$\he'$.
Expression~$\hasync~\he;\he'$ means to evaluate \he concurrently (in a different thread) and return $\he'$.

The syntax is similar to X10 real syntax with the following difference:
Escaping methods are marked with \hescaping
    because \this may escape (e.g., passing \this as an argument to another method).
The receiver of escaping methods must be a cooked object,
    whereas non-escaping methods can be called on a raw object (in order to initialize it).
FX10 does not have constructors; instead, an object is initialized by assigning to its fields or
    by calling
    non-escaping methods.
%To simplify presentation, we allow only paths~$\hp$
%    (variables \hx at
%    compile time, or heap locations \hl at run time) to appear in field access, assignment and method call.
%This does not restrict expressiveness, because of \code{val} expressions.

\Subsection{Typing}
\begin{lstlisting}
class D extends Object { f:Object; }
val d:D = new D; d.f = new Object
\end{lstlisting}
Then, the type of \code{d} is \code{D} where field \code{f} is uninitialized
    (which we formally write below as $\code{D}\myinit{\hf}{\hf}$).
Therefore, reading \code{d.f} would be illegal, but writing to \code{d.f} would update the type of \code{d}
    to be fully initialized.


Consider the following classes:
\begin{lstlisting}
class A extends Object {
  var f:A; var g:A; var h:A;
  nonescaping(g;f;f,h) build(a:A):A =
    val x = this;
    async this.h=g;
    val q = this;
    finish {
      val y = g;
      async this.f=a;
      val r =  a;
      val w=this; 
      val z = this;
    }
}
class B extends A { f3:A; }
\end{lstlisting}


\emph{Subclassing} is exactly as in FJ:
    the transitive closure of the \hextends relation,
    i.e., $\hC \st \hD$ if \hC transitively extends $\hD$.
(We also call it \emph{subtyping}, though types in FX10 also have masks,
    but those are ignored for subtyping purposes.)
%\emph{Subtyping} is defined as follows:
%    $\hC\myinit{\ol{\hf_s}}{\ol{\hf_a}} \st \hC'\myinit{\ol{\hf'_s}}{\ol{\hf'_a}}$
%    iff
%    $\hC \st \hC'$ and~$\ol{\hf_s} \subseteq \ol{\hf'_s}$ and~$\ol{\hf_a} \subseteq \ol{\hf'_a}$.
%Therefore, a cooked object can be used anywhere a raw object is expected
%    (e.g., $\hC\mycooked \st \hC\myinit{{\hf}}{{\hf}}$).

Similarly to FJ, we define:
(i)~$\fields(\hC)=\ol{\hf}:\ol{\hD}$ returns all fields of \hC (both those declared by \hC and recursively inherited from its superclass),
    and~$\ftype{}(\hf_i,\hC) =\hD_i$.
(ii)~$\mtype(\hm,\hC) = \ol{\hB}\mapsto\hD$ returns the type of method \hm in class \hC.
(iii)~$\mbody(\hm,\hC) = \ol{\hx}.\he$ returns the method body of \hm in class \hC.
Because methods have a modifier (\hG) in FX10, we also define:
(iv)~$\mmodifier(\hm,\hC) = \hG$ returns the method modifier of \hm in class \hC.


We use a simple logic of initialization for our basic assertions. We
consider primitive formulas $+\hx$ (the variable or parameter $\hx$ is
initialized), $+\hp.\hf$ (the field $\hp.hf$ is initialized), and
$-\hp.\hf$ (the field $\hp.\hf$ is being conciurrently initialized):

$$
 \phi,\psi {::=} \epsilon ~|~ +\hx ~|~ +\hp.\hf ~|~ -\hp.\hf \alt \phi,\psi
$$

The inference rules over formulas are straightforward: $\phi \vdash d$
if $d \in \phi$; $\phi\vdash +\hp.\hf$ if $\phi\vdash +\hp$; and if
the exact type of $\hx$ is $\hC$, and $\hC$ has the fields $\ol{\hf}$,
then $\phi\vdash +\hp$ if $\phi\vdash \hp.\hf_i$, for each $i$.

\begin{figure*}[t]
\begin{center}
\begin{tabular}{|c|}
\hline


$\typerule{
 \phi \vdash +p.f \gap \phi, +x~\hS~\psi
}{
 \phi \hval{\hx}{\hp.\hf_i}{\hS}~\psi
}$~\RULE{(T-Access)}
\quad
$\typerule{
  \phi~\hS~\psi
}{
 \phi~ \hval{\hx}{\hnew{\hC}}{\hS}~\psi
}$~\RULE{(T-New)}
\quad
$\typerule{
  \phi \vdash + q
}{
 \phi ~ p.\hf=q ~ + p.\hf
}$~\RULE{(T-Assign)}
\\\\

$\typerule{
    \phi~\hS~\psi
}{
  \begin{array}{l}
    \phi~\finish\,\hS~ +\psi \\
    \phi~\async\,\hS~ -\psi
  \end{array}
}$~\RULE{(T-Finish,Async)}

\quad
$\typerule{
  \phi~\hS_1~\psi_1
        \gap
    \phi,\psi_1~\hS_2~\psi_2
}{
  \phi~\hS_1;\hS_2~\psi_1,\psi_2
}$~\RULE{(T-Seq)}
\quad

$\typerule{
m(\ol{\hx}):: \phi' \Rightarrow \psi' \gap \phi \vdash \phi'[\hp/this,\ol{\hp}/\ol{\hx}]
}{
\phi~\hp.m(\ol{\hp})~\psi'[\hp/this,\ol{\hp}/\ol{\hx}]
}$~\RULE{(T-Invoke)}\\

\hline
\end{tabular}
\end{center}
\caption{FX10 Typing Rules ($\phi~\hS\psi$)}
\label{Figure:expression-typing}
\end{figure*}


We check that method declarations are ok in two parts (the two parts are separated below by a blank line).
The first part ensures that the type of the method body is a subtype
    of the return type, and that the body entails the method modifier.
The second part ensures that an overriding method
    has the same signature as the overridden one,
    and that non-escaping methods are final (cannot be overridden). %the method modifier is stronger in the subclass. %(whether it is \hescaping or if it has the 3 sets specified).
As in FJ, a class is ok if all its methods are ok.

\beqst %{method-ok}
\typerule{
  t  =
        \begin{cases}
        \hC\mycooked & \hG=\hescaping \\
        \hC\myinit{A}{A}  & \hG=\nonescaping(R;W_s;W_a) \\
            & ~~\fields{}(\hC)=\ol{\hf}:\ol{\hB} \\
            & ~~A=(\ol{\hf} \setminus R),* \\
        \end{cases}\\
  \Gamma = \{ \ol{\hx}:\ol{\hD\mycooked},\this:t \} \\
  \Gdash \he : \hD'\mycooked,\Gamma' \gap \Delta = \Gamma' - \Gamma \gap
    \hD' \st \hD \\
  \code{if }\hG=\nonescaping(R;W_s;W_a) ~\code{then }  \Delta(\this) = \initsets{\ol{f_s}}{\ol{f_a}} \\
  %todo
    ~\hand W_s \subseteq W_a  \hand \ol{f_s} \supseteq W_s \hand \ol{f_a} \supseteq W_a \\\\
  \hclass~\hC~\hextends~\hC'~\lb~\ldots~\rb\\
  \code{if }\mtype{}(\hm,\hC')=\ol{\hD'} \mapsto \hD'\\
    ~~\code{then } \ol{\hD'}\equiv\ol{\hD} \hand \hD'\equiv\hD \hand \\
    \mmodifier{}(\hm,\hC)=\mmodifier{}(\hm,\hC')=\hescaping
}{
  \hG ~ \hm(\ol{\hx}:\ol{\hD}):\hD=\he ~~\OK~\IN~\hC
}
\eeq



\Subsection{Reduction}

\begin{figure*}[t]
\begin{center}
\begin{tabular}{|c|}
\hline
$\typerule{
 \hS,\hH \reduce \hH'
}{
  \begin{array}{l}
    \finish~\hS,\hH \reduce \hH'\\
    \async~\hS,\hH \reduce \hH'\\
    \hS;\hS', \hH \reduce \hS',\hH'\\
  \end{array}
}$~\RULE{(R-Term)}
~
$\typerule{
 \hS,\hH \reduce \hS', \hH'
}{
  \begin{array}{l}
    \finish~\hS,\hH \reduce \finish~\hS',\hH'\\
    \async~\hS,\hH \reduce \async~\hS', \hH'\\
    \hS;\hS_1, \hH \reduce \hS';\hS_1,\hH'\\
    \async~\hS_1;\hS, \hH \reduce \async~\hS_1;\hS', \hH'\\
  \end{array}
}$~\RULE{(R-Step)}
~
$\typerule{
  \he,\hH \reduce \hl,\hH'
}{
  \hval{\hx}{\he}{\hS},\hH \reduce \hS[\hl/\hx], \hH'
}$~\RULE{(R-Val)}
\\\\

$\typerule{
    \hl' \not \in \dom(\hH)
}{
  \hnew{\hC},\hH \reduce \hl',\hH[ \hl' \mapsto \hC()]
}$~\RULE{(R-New)}
\quad
$\typerule{
    \hH(\hl')=\hC(\ldots)
        \gap
    \mbody{}(\hm,\hC)=\ol{\hx}.\hS
}{
  \hl'.\hm(\ol{\hl}),\hH \reduce \hS[\ol{\hl}/\ol{\hx},\hl'/\this],\hH
}$~\RULE{(R-Invoke)}
\quad


\\\\

$\typerule{
    \hH(\hl)=\hC(\ol{\hf}\mapsto\ol{\hl'})
}{
  \hl.\hf_i,\hH \reduce \hl_i',\hH
}$~\RULE{(R-Access)}
\quad
$\typerule{
    \hH(\hl)=\hC(\hF)
        \gap
    \hF' = \hF[ \hf \mapsto \hl"]
}{
  \hl.\hf=\hl',\hH \reduce \hl',\hH[ \hl \mapsto \hC(\hF')]
}$~\RULE{(R-Assign)}
\\
\hline
\end{tabular}
\end{center}
\caption{FX10 Reduction Rules ($\hH,\hS \reducesto \hH',\hS' ~|~\hH'$).}
\label{Figure:reduction}
\end{figure*}

An object~$\ho = \code{C}(\ol{\hf}\mapsto\ol{\hl'})$ is an instance of some class~\hC
    where fields~$\ol{\hf}$ has been initialized to locations~$\ol{\hl'}$.
A heap~$H$ is a mapping from locations~\hl to objects~$\ho$.
A heap-typing~$\Gamma_H$ maps locations to their type,
    i.e., if~$H[\hl] = \code{C}(\ol{\hf}\mapsto\ol{\hl'})$,
        $\fields{}(\hC)=\ol{\hf'}:\ol{\hD}$, and~$A=\ol{\hf'}\setminus\ol{\hf}$,
    then~$\Gamma_H[\hl] = \hC\myinit{A}{A}$.
A heap~$H$ is \emph{well-typed} iff
    each field location is a subtype of the declared field type,
    i.e., for every location in the heap~$\hl \in \dom(H)$,
        where $H[\hl] = \code{C}(\ol{\hf}\mapsto\ol{\hl'})$
        and
        for every field~$\hf_i$, we have that~$H[\hl_i'] = \hC'(\ldots)$ and
        $\hC' \st \ftype{}(\hf_i,\hC)$.


An expression~\he is called \emph{closed} if it does not contain
    any free variables (i.e., it does not contain method parameters \hx nor~\this).


Consider a program $\hP,\he$, where~\he is closed and \hP is well-typed.
Then in a well-typed heap, \he can always be reduced to a location.
Because our reduction rules only allow reads from initialized fields,
    then a corollary is that a field can only be read after it was assigned.


\begin{Theorem}[preservation]
  \textbf{(Progress and Preservation)}
    For every closed expression~$\he \neq \hl$, and a well-typed heap~$H$,
        if $\Gamma_{H} \vdash \he : \hC\myinit{\ol{\hf_s}}{\ol{\hf_a}}, \Gamma'$,
        then there exists~$H',\he',\hC',\Delta$ such that
        (i)~$H,\he \reducesto H',\he'$,
        (ii)~$\Gamma_{H'} \vdash \he':\hC'\myinit{\ol{\hf'_s}}{\ol{\hf'_a}}, \Gamma"$,
        and~$\hC' \st \hC$,
        and~$\ol{\hf'_s} \subseteq \ol{\hf_s}$,
        and~$\ol{\hf'_a} \subseteq \ol{\hf_a}$,
        and~$\Gamma" = \Gamma'+\Delta$,
        (iii)~$H'$ is well-typed,
        (iv)~$\he'$ is closed.
\end{Theorem}
\begin{proof}
See our technical report.
\end{proof}
