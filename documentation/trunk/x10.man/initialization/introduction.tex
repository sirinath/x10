Constructing an object in a safe way is not easy:
    it is well known that dynamic dispatching
    or leaking \this during object construction
    is error-prone~\cite{Dean:1996,Seo:2007:SBD:1522565.1522587,Gil:2009:WRS:1615184.1615216},
    and various type systems and verifiers have been proposed to
    handle safe object initialization~\cite{Hubert:2010:ESO:1888881.1888890,Zibin:2010:OIG:1869459.1869509,Fahndrich:2007:EOI:1297027.1297052,XinQi:2009}.
As languages become more and more complex,
    new pitfalls are created due to the interactions between
    language features.

X10 is an object oriented programming language with a sophisticated
    type system (constraints, class invariants, non-erased generics, closures)
    and concurrency constructs (asynchronous activities, multiple places, global references).
This paper shows that object initialization is a cross-cutting concern
    that interact with other features in the language.
We discuss several language designs that restrict these interactions,
    and explain why we chose the \emph{hardhat} design for X10.

{Hardhat} was termed in~\cite{Gil:2009:WRS:1615184.1615216}
    and it describes a design that prohibits dynamic dispatching or leaking \this during construction.
A hardhat design limits the user
    but also protects her from future bugs.
X10's hardhat design is even stricter due to additional language features
    such as concurrency, places, and closures.

On the other end of the spectrum,
    Java and C\# allow
    dynamic dispatching and leaking \this
    and still maintain type- and runtime- safety
    by relying on the fact that every type has a default zero value
    (whether that zero is 0, false, or \code{null}),
    and all fields are zero-initialized before the constructor begins.
As a consequence,
    a half-baked object can leak before all its fields are set. %\cite{Seo:2007:SBD:1522565.1522587} - reading uninitialized field references
Phrased differently,
    when reading a final field, one can read the default value initially and later read a different value.
Another source of subtle bugs is due to the synchronization barrier
    at the end of a constructor~\cite{JSR133}
    after which all assignments to final fields are guaranteed to be written.
The programmer is warned (in the documentation only!)
    that objects with final fields are thread-safe only if
    \this does not escape its constructor.
%Before JSR 133~\cite{JSR133},
%    immutable objects could have different values in different threads
%    if synchronization was not used properly.
Finally, if the type-system is augmented, for example, with non-null types, then
    a default value no longer exists,
    which leads to complicated type-systems for initialization~\cite{Fahndrich:2007:EOI:1297027.1297052,XinQi:2009}.

\mbox{C++}, as usual, gives you enough rope to hang yourself.
Fields are not zero-initialized, and therefore if \this leaks,
    one can read an uninitialized field.
Moreover, method calls are statically bound during construction,
    which may result in an exception at runtime
    if one tries to invoke a virtual method of an abstract class (see \Ref{Figure}{Dynamic-dispatch} below).
(Determining whether this happens is an intractable problem~\cite{Gil:1998:CTA:646155.679689}.)


The remainder of this introduction presents examples in X10
    that slowly add language features and describes their interaction with
    object initialization.
The examples below show X10 code, and the lines marked with \code{//err}
    match errors issued by the compiler according to the hardhat design that
    will be described in \Ref{Section}{designs}.

\subsection{Constructors and inheritance}
%Methods in X10 are denoted by \code{def}, and \code{val} denotes a final field.
Object initialization begins by invoking a constructor, denoted by \code{def this()}.

Consider \Ref{Figure}{Escaping-this}.
The first leak causes a problem because field \code{a} was not assigned yet.
However, leaking is still a problem even after all fields have been assigned,
    because fields in a subclass might not have been initialized yet.

\begin{figure}
\begin{lstlisting}
class A {
  val a:Int;
  def this() {
    LeakIt.foo(this); //err
    a = 1;
    LeakIt.foo(this); //err
  }
}
class B extends A {
  val b:Int;
  def this() {
    super();
    b = 2;
  }
}
\end{lstlisting}
\caption{Escaping \this example.
    X10: a raw \this cannot escape.}
\label{Figure:Escaping-this}
\end{figure}

In X10, \this cannot leak.

%\begin{figure}
%\begin{lstlisting}
%class A {
%  static val INSTANCES = new ArrayList[A]();
%  def this() { // constructor
%    // initialize A's fields
%    INSTANCES.add(this);
%  }
%}
%\end{lstlisting}
%\caption{Static-container paradigm that leaks \this.}
%\label{Figure:Static-container-paradigm}
%\end{figure}

\subsection{Dynamic dispatch}


\begin{figure}
\begin{lstlisting}
abstract class A {
  def this() {
    init(); //err
  }
  abstract def init():void;
}
class B extends A {
  val b:Int;
  def this(i:Int) {
    super(i);
    b = 3;
  }
  def init():void {
    val x = b;
  }
}
\end{lstlisting}
\caption{Dynamic dispatch example.
    X10: a raw \this can be the receiver only for private or final methods.}
\label{Figure:Dynamic-dispatch}
\end{figure}


In X10, you can call only private or final methods
    during construction,
    therefore there is no dynamic dispatching during construction.

In C++ it gives an error that you called a pure virtual function.

In Java you get the default value of b, not 3.
This is undesired in Java, and Java tries to catch some of these cases by
    forbidding a call to a member function before the super object was initialized.
For example, the following is illegal in Java (and X10) if \code{f} is a member function:
\begin{lstlisting}
class B extends A { B() {super(f()); }}
\end{lstlisting}



\subsection{Exceptions}


\begin{figure}
\begin{lstlisting}
try {super();} catch(Throwable e){}
\end{lstlisting}
\caption{Exceptions example.
    X10: if the construction of an object finished normally, then all constructor
        calls have finished normally.}
\label{Figure:Exceptions}
\end{figure}

This is also checked in Java, that prohibits any statement before a super call.
Failure to implement this verification properly led to a famous attack~\cite{Dean:1996}


\subsection{Concurrency}
Explain async+finish.
If the finish was removed, then it would be an error ...

\begin{figure}
\begin{lstlisting}
class A {
  val a:Int;
  val b:Int;
  def this() {
    finish {
      async a = 1;
    }
    async b = 2; //err
  }
}
\end{lstlisting}
\caption{Asynchronously assigned fields example.
    X10: all field assignments must finish when the constructor finishes.}
\label{Figure:Asynchronously-init}
\end{figure}


\subsection{Multiple Places}
Places require serialization and deserialization (both custom and automatic) across "at".

\begin{figure}
\begin{lstlisting}
class A {
  val a:Int;
  def this() {
    // Execute at another place
    at (here.next())
      this.a = 1; //err
  }
}
\end{lstlisting}
\caption{Multi-place initialization example.
    X10: a raw \this cannot cross to another place.}
\label{Figure:Multi-place}
\end{figure}


\subsection{Global references}
Special GlobalRef class: let \this escape and it creates a cycle.
We allow a raw \this to escape iff
* field is private with a field initializer.
* cannot be used with a raw \this receiver.

\begin{figure}
\begin{lstlisting}
class A {
  private val root =
   new GlobalRef(this);
  def me() = root();
}
class B extends A {
  def this() {
    val alias = me(); //err
  }
}
\end{lstlisting}
\caption{\code{GlobalRef} example.
    X10: ...}
\label{Figure:GlobalRef}
\end{figure}

If \code{me()} was prefixed with
\code{@NonEscaping public final}
then accessing \code{root} would be an error.
%Cannot use 'root' because a GlobalRef[...](this) cannot be used in a field initializer, constructor, or methods called from a constructor.

\subsection{Constraints}
Constraints and default values.
The following types do not have a default value:
\code{Int\lb self!=0\rb}
\code{String\lb self!=null\rb}

Therefore the fields of an object cannot be zero-initialized in X10.

\begin{figure}
\begin{lstlisting}
class A {
  var a:Int{self!=0}; //err
}
\end{lstlisting}
\caption{No default value example.
    X10: you must assign to \code{var} fields without a default value.}
\label{Figure:Constraints}
\end{figure}


\subsection{Properties}
Properties are final values that can be used in constraints,
    e.g., \code{Array} has a \code{size} property,
    so an array of size 2 can be expressed as: \code{Array\lb self.size==2\rb}.

Properties are final fields that are initialized before all other fields.


\begin{figure}
\begin{lstlisting}
class A(a:Int) {
  def this() {
    property(42);
  }
}
class B(b:Int) {b!=0} extends A {
  val f1 = a+b;
  val f2:Int;
  def this() {
    super();
    f2 = f1; // err
    property(3);
  }
}
\end{lstlisting}
\caption{Properties example.
    X10: properties must be initialized before fields.}
\label{Figure:Properties}
\end{figure}


%\subsection{Class Invariant}
% Should I talk about interface and abstract property methods? Doesn't relate to initialization...
The \emph{class invariant} may refer only to properties of the class,
    and it must be satisfied after the property call in every constructor.


\subsection{Closures}

\begin{figure}
\begin{lstlisting}
class A {
  val a = 3;
  def this() {
    val local = this.a;
    val closure1 =
      ()=>local;
    val closure2 =
      ()=>this.a; //err
    at (here.next())
      closure2();
  }
}
\end{lstlisting}
\caption{Closure capture \this example.
    X10: a closure cannot capture a raw \this.}
\label{Figure:Closures}
\end{figure}


\subsection{Inner classes}
\begin{figure}
\begin{lstlisting}
class A {
  class Inner {
    val b = a;
  }
  val inner =
    this.new Inner(); //err
  val a = 3;
}
\end{lstlisting}
\caption{\this escapes as an outer instance.
    X10: a raw \this cannot be a receiver of \code{new}.}
\label{Figure:InnerClass}
\end{figure}


\subsection{Generics}
\Ref{Figure}{Constraints} showed that a \code{var} must be assigned if
    it does not contain the zero value.
For generics, we added a \code{haszero} type condition that requires a type parameter to have the zero value.

\begin{figure}
\begin{lstlisting}
class A[T] {
  var a:T; //err
}
class B[T] {T haszero} {
  var a:T;
}
class Usage {
  var b1:B[Int];
  var b1:B[String];
  var b1:B[Int{self!=0}]; //err
}
\end{lstlisting}
\caption{\code{haszero} type condition.
    X10: statically checks a type has the zero value.}
\label{Figure:Generics}
\end{figure}


examples for \code{Array}

 \code{Zero.get[T]()}


\subsection{Static initialization}
todo
