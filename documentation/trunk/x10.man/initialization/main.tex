\documentclass[9pt]{sigplanconf}


\conferenceinfo{X10'11,} {June 4, 2011, San Jose, California, USA.}
\CopyrightYear{2011}
%\copyrightdata{}

% \stepcounter Doesn't work in a caption!
\newcounter{RuleCounter}
\stepcounter{RuleCounter}
\newcommand{\userule}[1]{\arabic{#1}}
\newcommand{\definerule}[1]{\newcounter{#1}\addtocounter{#1}{\arabic{RuleCounter}}\stepcounter{RuleCounter}}
\newcommand{\myrule}[2]{\textbf{Rule #1:} #2.}
\newcommand{\removeGlobalRef}[1]{}

\usepackage{xspace}

% Macros for R^nRS.

\def\@makechapterhead#1{%
  \vspace*{50\p@}%
  {\parindent \z@ \raggedright \normalfont
    \ifnum \c@secnumdepth >\m@ne
        \huge\bfseries \thechapter \space\space\space
        \nobreak
    \fi
    \interlinepenalty\@M
    \Huge \bfseries #1\par\nobreak
    \vskip 40\p@
  }}


\makeatletter

\newcommand{\topnewpage}{\@topnewpage}

% Chapters, sections, etc.

\newcommand{\vest}{}
\newcommand{\dotsfoo}{$\ldots\,$}

\newcommand{\sharpfoo}[1]{{\tt\##1}}
\newcommand{\schfalse}{\sharpfoo{f}}
\newcommand{\schtrue}{\sharpfoo{t}}

\newcommand{\singlequote}{{\tt'}}  %\char19
\newcommand{\doublequote}{{\tt"}}
\newcommand{\backquote}{{\tt\char18}}
\newcommand{\backwhack}{{\tt\char`\\}}
\newcommand{\atsign}{{\tt\char`\@}}
\newcommand{\sharpsign}{{\tt\#}}
\newcommand{\verticalbar}{{\tt|}}

\newcommand{\coerce}{\discretionary{->}{}{->}}

% Knuth's \in sucks big boulders
\def\elem{\hbox{\raise.13ex\hbox{$\scriptstyle\in$}}}

\newcommand{\meta}[1]{{\noindent\hbox{\rm$\langle$#1$\rangle$}}}
\let\hyper=\meta
\newcommand{\hyperi}[1]{\hyper{#1$_1$}}
\newcommand{\hyperii}[1]{\hyper{#1$_2$}}
\newcommand{\hyperj}[1]{\hyper{#1$_i$}}
\newcommand{\hypern}[1]{\hyper{#1$_n$}}
\newcommand{\var}[1]{\noindent\hbox{\it{}#1\/}}  % Careful, is \/ always the right thing?
\newcommand{\vari}[1]{\var{#1$_1$}}
\newcommand{\varii}[1]{\var{#1$_2$}}
\newcommand{\variii}[1]{\var{#1$_3$}}
\newcommand{\variv}[1]{\var{#1$_4$}}
\newcommand{\varj}[1]{\var{#1$_j$}}
\newcommand{\varn}[1]{\var{#1$_n$}}

\newcommand{\vr}[1]{{\noindent\hbox{$#1$\/}}}  % Careful, is \/ always the right thing?
\newcommand{\vri}[1]{\vr{#1_1}}
\newcommand{\vrii}[1]{\vr{#1_2}}
\newcommand{\vriii}[1]{\vr{#1_3}}
\newcommand{\vriv}[1]{\vr{#1_4}}
\newcommand{\vrv}[1]{\vr{#1_5}}
\newcommand{\vrj}[1]{\vr{#1_j}}
\newcommand{\vrn}[1]{\vr{#1_n}}


\newcommand{\defining}[1]{\mainindex{#1}{\em #1}}
\newcommand{\ide}[1]{{\schindex{#1}\frenchspacing\tt{#1}}}

\newcommand{\lambdaexp}{{\cf lambda} expression}
\newcommand{\Lambdaexp}{{\cf Lambda} expression}
\newcommand{\callcc}{{\tt call-with-current-continuation}}

% \reallyindex{SORTKEY}{HEADCS}{TYPE}
% writes (index-entry "SORTKEY" "HEADCS" TYPE PAGENUMBER)
% which becomes  \item \HEADCS{SORTKEY} mainpagenumber ; auxpagenumber ...

\global\def\reallyindex#1#2#3{%
\write\@indexfile{"#1" "#2" #3 \thepage}}

\newcommand{\mainschindex}[1]{\label{#1}\reallyindex{#1}{tt}{main}}
\newcommand{\mainindex}[1]{\reallyindex{#1@{\rm #1}{main}}}
\newcommand{\schindex}[1]{\reallyindex{#1}{tt}{aux}}
\newcommand{\sharpindex}[1]{\reallyindex{#1}{sharpfoo}{aux}}
%vj%\renewcommand{\index}[1]{\reallyindex{#1}{rm}{aux}}

\newcommand{\domain}[1]{#1}
\newcommand{\nodomain}[1]{}
%\newcommand{\todo}[1]{{\rm$[\![$!!~#1$]\!]$}}
\newcommand{\todo}[1]{}

% \frobq will make quote and backquote look nicer.
\def\frobqcats{%\catcode`\'=13 %\catcode`\{=13{}\catcode`\}=13{}
\catcode`\`=13{}}
{\frobqcats
\gdef\frobqdefs{%\def'{\singlequote}
\def`{\backquote}}}%\def\{{\char`\{}\def\}{\char`\}}
\def\frobq{\frobqcats\frobqdefs}

% \cf = code font
% Unfortunately, \cf \cf won't work at all, so don't even attempt to
% next constructions which use them...
\newcommand{\cf}{\frenchspacing\tt}

% Same as \obeycr, but doesn't do a \@gobblecr.
{\catcode`\^^M=13 \gdef\myobeycr{\catcode`\^^M=13 \def^^M{\\}}%
\gdef\restorecr{\catcode`\^^M=5 }}

{\catcode`\^^I=13 \gdef\obeytabs{\catcode`\^^I=13 \def^^I{\hbox{\hskip 4em}}}}

{\obeyspaces\gdef {\hbox{\hskip0.5em}}}

\gdef\gobblecr{\@gobblecr}

\def\setupcode{\@makeother\^}

% Scheme example environment
% At 11 points, one column, these are about 56 characters wide.
% That's 32 characters to the left of the => and about 20 to the right.

\newenvironment{x10noindent}{
  % Commands for scheme examples
  \newcommand{\ev}{\>\>\evalsto}
  \newcommand{\lev}{\\\>\evalsto}
  \newcommand{\unspecified}{{\em{}unspecified}}
  \newcommand{\scherror}{{\em{}error}}
  \setupcode
  \small \cf \obeytabs \obeyspaces \myobeycr
  \begin{tabbing}%
\qquad\=\hspace*{5em}\=\hspace*{9em}\=\kill%   was 16em
\gobblecr}{\unskip\end{tabbing}}

%\newenvironment{scheme}{\begin{schemenoindent}\+\kill}{\end{schemenoindent}}
\newenvironment{x10}{
  % Commands for scheme examples
  \newcommand{\ev}{\>\>\evalsto}
  \newcommand{\lev}{\\\>\evalsto}
  \renewcommand{\em}{\rmfamily\itshape}
  \newcommand{\unspecified}{{\em{}unspecified}}
  \newcommand{\scherror}{{\em{}error}}
  \setupcode
  \small \cf \obeyspaces \myobeycr
  \footnotesize
  \begin{tabbing}%
\qquad\=\hspace*{5em}\=\hspace*{9em}\=\+\kill%   was 16em
\gobblecr}{\unskip\end{tabbing}\normalsize}

\newcommand{\evalsto}{$\Longrightarrow$}

% Manual entries

\newenvironment{entry}[1]{
  \vspace{3.1ex plus .5ex minus .3ex}\noindent#1%
\unpenalty\nopagebreak}{\vspace{0ex plus 1ex minus 1ex}}

\newcommand{\exprtype}{syntax}

% Primitive prototype
\newcommand{\pproto}[2]{\unskip%
\hbox{\cf\spaceskip=0.5em#1}\hfill\penalty 0%
\hbox{ }\nobreak\hfill\hbox{\rm #2}\break}

% Parenthesized prototype
\newcommand{\proto}[3]{\pproto{(\mainschindex{#1}\hbox{#1}{\it#2\/})}{#3}}

% Variable prototype
\newcommand{\vproto}[2]{\mainschindex{#1}\pproto{#1}{#2}}

% Extending an existing definition (\proto without the index entry)
\newcommand{\rproto}[3]{\pproto{(\hbox{#1}{\it#2\/})}{#3}}

% Grammar environment

\newenvironment{grammar}{
  \def\:{& \goesto{} &}
  \def\|{& $\vert$& }
  \def\opt{$^?$\ }
  \def\star{$^*$\ }
  \def\plus{$^+$\ }
  \em
  \begin{tabular}{rcl}
  }{\unskip\end{tabular}}

%\newcommand{\unsection}{\unskip}
\newcommand{\unsection}{{\vskip -2ex}}

% Commands for grammars
\newcommand{\arbno}[1]{#1\hbox{\rm*}}  
\newcommand{\atleastone}[1]{#1\hbox{$^+$}}

\newcommand{\goesto}{{\normalfont{::=}}}

% mark modifications (for the grammar) From Igor Pechtchanski/Watson/IBM@IBMUS
\newlength{\modwidth}\setlength{\modwidth}{0.005in}
\newlength{\modskip}\setlength{\modskip}{.4em}
\newlength{\@modheight}
\newlength{\@modpos}
\providecommand{\markmod}[1]{%
  \setlength{\@modheight}{#1}%
  \addtolength{\@modheight}{-0.06in}%
  \setlength{\@modpos}{\linewidth}%
  \addtolength{\@modpos}{0.285in}%         Magic
  \addtolength{\@modpos}{\modwidth}%
  \addtolength{\@modpos}{\modskip}%
  \marginpar{\vspace{-\@modheight}%
             \hspace{-\@modpos}%
             \rule{\modwidth}{#1}}%
}

% The index

\def\theindex{%\@restonecoltrue\if@twocolumn\@restonecolfalse\fi
%\columnseprule \z@
%!! \columnsep 35pt
\clearpage
\@topnewpage[
    \centerline{\large\bf\uppercase{Alphabetic index of definitions of concepts,}}
    \centerline{\large\bf\uppercase{keywords, and procedures}}
    \vskip 1ex \bigskip]
    \markboth{Index}{Index}
    \addcontentsline{toc}{chapter}{Alphabetic index of 
 definitions of concepts, keywords, and procedures}
    \bgroup %\small
    \parindent\z@
    \parskip\z@ plus .1pt\relax\let\item\@idxitem}

\def\@idxitem{\par\hangindent 40pt}

\def\subitem{\par\hangindent 40pt \hspace*{20pt}}

\def\subsubitem{\par\hangindent 40pt \hspace*{30pt}}

\def\endtheindex{%\if@restonecol\onecolumn\else\clearpage\fi
\egroup}

\def\indexspace{\par \vskip 10pt plus 5pt minus 3pt\relax}

\makeatother

%\newcommand{\Xten}{{\sf X10}}
%\newcommand{\XtenCurrVer}{{\sf X10 v1.7}}
%\newcommand{\java}{{\sf Java}}
%\newcommand{\Java}{{\sf Java}}

\newcommand{\Xten}{X10}
\newcommand{\XtenCurrVer}{\Xten{} v1.7}
\newcommand{\Java}{Java}
\newcommand{\java}{\Java{}}

\newcommand{\futureext}[1]{{\em \paragraph{Future Extensions.}#1}}
\newcommand{\tbd}{} % marker for things to be done later.
\newcommand{\limitation}[1]{{\em Limitation: #1}} % marker for things to be done later.


\newcommand\grammarrule[1]{\emph{#1}}

% Rationale

\newenvironment{rationale}{%
\bgroup\noindent{\sc Rationale:}\space}{%
\egroup}

% Notes

\newenvironment{note}{%
\bgroup\noindent{\sc Note:}\space}{%
\egroup}

\newenvironment{staticrule*}{%
\bgroup\noindent{\textsc{Static Semantics Rule:}\space}}{%
\egroup}

\newenvironment{staticrule}[1]{%
\bgroup\noindent{\textsc{Static Semantics Rule} (#1):\space}}{%
\egroup}

\newcommand\Sref[1]{\S\ref{#1}}
\newcommand\figref[1]{Figure~\ref{#1}}
\newcommand\tabref[1]{Table~\ref{#1}}
\newcommand\exref[1]{Example~\ref{#1}}

\newcommand\eat[1]{}


\begin{document}


\lstset{language=java,basicstyle=\ttfamily\small}


\title{Object Initialization in X10}


\authorinfo{Yoav Zibin \and David Cunningham \and Igor Peshansky \and Vijay Saraswat}
           {IBM research in TJ Watson}
           {yzibin$~|~$dcunnin$~|~$igorp$~|~$vsaraswa@us.ibm.com}


\maketitle


\begin{abstract}
Modern object-oriented languages such as \Xten require a rich framework
for types capable of expressing value-dependency, type-dependency
and supporting pluggable, application-specific extensions.

In earlier work, we presented the framework of \emph{constrained
types} for concurrent, object-oriented languages, parametrized by
an underlying constraint system $\cal X$. Constraint systems are a
very expressive framework for partial information. Types are viewed
as formulas \Xcd{C\{c\}} where \Xcd{C} is the name of a class
or an interface and \Xcd{c} is a constraint in $\cal X$ on the
immutable instance state of \Xcd{C} (the \emph{properties}).
Many (value-)dependent type systems for object-oriented languages
can be viewed as constrained types.

This paper extends the constrained types approach to handle
\emph{type-dependency} (``genericity''). The key idea is to extend
the constraint system to include predicates over types such as
\Xcd{X} is a subtype of \Xcd{T}.  Generic types are supported
by introducing type parameters and permitting programs to impose
constraints on such parameters.

To illustrate the underlying theory, we develop a formal family of
programming languages with a common set of sound type-checking rules
parameterized on a constraint system $\cal X$.  By varying $\cal X$
and extending the type system, we obtain languages with the power
of \FJ, \FGJ, and languages that provide dependent types, structural
subtyping, and constraints that relate values and types.  The core
of the \Xten language is a concrete instantiation of the framework.  We
describe the design and implementation of \Xten, which is available
for download at \texttt{x10-lang.org}.


\end{abstract}

%\category{D.3.3}{Programming Languages}{Language Constructs and Features}
%\category{D.1.5}{Programming Techniques}{Object-oriented Programming}

%\terms
%Asynchronous, Initialization, Types, X10

% Keywords are not required in the paper itself - only in the submission system's meta data.
% \keywords
% Immutability, Ownership, Java


\Section[introduction]{Introduction}

\section{Introduction}
\label{s:intr}

 Graph theoretic problems arise in several traditional and emerging scientific disciplines such as VLSI design, optimization, databases, and computational biology. There are plenty of theoretically fast parallel algorithms, for example, work-time optimal PRAM algorithms, for graph problems; however, in
 practice few parallel implementations beat the best sequential implementations for arbitrary, sparse
 graphs. The mismatch between theory and practice suggests a large gap between algorithmic model and the actual architecture. We observe that the gap is increasing as new diversified architectures emerge. Elegant solutions seem hard to come by from even combined efforts of algorithmic and architectural improvement. What is lacking is an effient way of mapping fine-grained parallelism expressed by the algorithm to target architectures with good performance. X10 is a new parallel programming language that provides expressive programming constructs and efficient runtime support that effectively helps reduce the gap between theory and practice in solving graph problems. In this paper we show that with X10 the fine-grained parallelism for a graph problem can be expressed much easier at a high algorithmic level, and the X10 program, compared with native C implementation, is much simpler and more elegant, and achieves comparable, and sometimes, even better performance. 

 The challenges of solving large-scale graph problems on current and emerging systems come from the irregular and combinatorial nature of the problem. Many of the important real world graphs, for example, internet topology, social interaction network, transfortation network, protein-protein interaction network, and etc., exhibit a ``small-world'' nature, and can be modeled as the so-called ``scale-free'' graph. There is no known efficient technique to partion such graph, which makes it hard to solve on distributed-memory systems. Also compared with the well-known sequential algorithms, for example, depth-first search (DFS) or breadth-first search (BFS) for the spanning tree problem, the parallel graph algorithms take exotic approaches such as ``graft-and-shortcut''. In the absence of efficient scheduling support of parallel activities, fine-grained parallelism incurs large overhead on current systems and oftentimes do not show practical parallel performance advantage. Lastly, graph algorithms tend to be load/store intensive compared with other scientific problems. For example,  They put great pressure on the memory subsystem. The problem obviously gets worse on distributed-memory architectures if necessary task management and memory affinity scheduling are not provided.  
 
 There are several features of X10 that make it extremely helpful in soving large-scale graph problems. X10 provides a shared virtual address space that obviates the need to partition a graph and issue message passing commands explicitly to access remote data. The irregular nature of the graph is also the reason why no SSCA benchmark has been implemented in MPI. X10 provides a wide range of constructs that are de. X10 has a lot of balancing.

 Our target architecure is a cluster of symmetric multiprocessor(SMP) nodes. Each SMP node may further comprise of chip multiprocessors (CMPs).  SMPs and CMPs are becoming very powerful and common place. Most of the high performance
computers are clusters of SMPs and/or CMPs. It is important to solve for them.
It is important to show flexibility but also good support of  PRAM algorithms for graph problems can be emulated much easier and more. 

The problem we consider if the spanning tree problem. It is notoriously hard to achieve good parallel performance.  Several good ones, we show X10 support that can do better. 

 The rest of the paper is organized as follows. Sections~\ref{s:design} describes algorithm design with the X10 language.
 Section~\ref{s:runtime} presents the workstealing runtime support for load-balancing in X10, and compare with other runtime systems, for example, CILK. 
 Section~\ref{s:results} provides our experimental results on current main-stream SMPs.
 In Section~\ref{s:concl} we conclude and give future work. 
 Throughout the paper, we
 use $n$ and $m$ to denote the number of vertices and the number of
 edges of an input graph $G=(V,E)$, respectively. 
  




\Section[rules]{X10 Initialization Rules}
X10 is an advanced object-oriented language with a complex type-system
    and concurrency constructs.
This section describes how object initialization interacts with X10 features.
We begin with object-oriented features found in mainstream languages,
    such as constructors, inheritance, dynamic dispatch, exceptions, and inner classes.
We then proceed to X10's type-system features,
    such as constraints, properties, class invariants, closures, (non-erased) generics, and structs,
followed by the parallel features of X10 for writing concurrent code (\code{finish} and \code{async}),
    and distributed code (\code{at}\removeGlobalRef{ and global references}).
Next we describe the inter-procedural data-flow analysis that ensures that
    a field is read only after it has been assigned.
Finally, we summarize the virtues and attributes of initialization in X10.


\subsection{Constructors and inheritance}
Inheritance is the first feature that interacts with initialization:
    when class \code{B} inherits from \code{A}
    then every instance of \code{B} has a sub-object that is like an instance of \code{A}.
When we initialize an instance of \code{B}, we must first initialize its \code{A} sub-object.
We do this in X10 by forcing the constructors of \code{B} to make a super call,
    i.e., call a constructor of \code{A}
    (either explicitly or implicitly).



\begin{figure}
\vspace{-0.2cm}\begin{lstlisting}
class A {
  val a:Int;
  def this() {
    LeakIt.foo(this); //err
    this.a = 1;
    val me = this; //err
    LeakIt.foo(me);
    this.m2(); // so m2 is implicitly non-escaping
  }
  // permitted to escape
  final def m1() {
    LeakIt.foo(this);
  }
  // implicitly non-escaping because of this.m2()
  final def m2() {
    LeakIt.foo(this); //err
  }
  // explicitly non-escaping
  @NonEscaping final def m3() {
    LeakIt.foo(this); //err
  }
}
class B extends A {
  val b:Int;
  def this() {
    super(); this.b = 2; super.m3();
  }
}
\end{lstlisting}\vspace{-0.2cm}
\definerule{Rule3}
\definerule{Rule4}
\caption{Escaping \this example.
    \textbf{Definition of \emph{raw}:}
    {\this and \code{super} are \emph{raw} in {non-escaping} methods and in field initializers}.
    \textbf{Definition of \emph{non-escaping}:}
        {A method is \emph{non-escaping} if it is a constructor,
            or annotated with \code{@NonEscaping} or \code{@NoThisAccess},
            or a method that is called on a raw \this receiver}.
    \myrule{\arabic{Rule3}}{A raw \this or \code{super} cannot escape or be aliased}
    \myrule{\arabic{Rule4}}{A call on a raw \code{super} is allowed only for a \code{@NonEscaping} method}
    (\code{\textbf{final}} and \code{@NoThisAccess} are related
        to dynamic dispatch as shown in \Ref{Figure}{Dynamic-dispatch}.)}
\label{Figure:Escaping-this}
\end{figure}

\Ref{Figure}{Escaping-this} shows X10 code that demonstrates the interaction
    between inheritance and initialization,
    and explains by example why leaking \this during construction can cause bugs.
In all the examples, all errors issued by the X10 compiler are marked with \code{//err}
    (and if there is no such mark then the code is correct).

We say that an object is \emph{raw} (also called partially initialized) before its constructor ends,
    and afterward it is \emph{cooked} (also called fully initialized).
Note that when an object is cooked, all its sub-objects must be cooked as well.
X10 prohibits any aliasing or leaking of \this during construction,
    therefore only \this or \code{super} can be raw (any other variable is definitely cooked).

Object initialization begins by invoking a constructor,
    denoted by the method definition \code{def this()}.
The first leak would cause a problem because field \code{a} was not assigned yet.
However, even after all the fields of \code{A} have been assigned,
    leaking is still a problem
    because fields in a subclass (field \code{b}) have not yet been initialized.
Note that leaking is not a problem if \this is not raw, e.g., in \code{m1()}.

We begin with two definitions:
    (i)~when an object is \emph{raw}, and
    (ii)~when a method is \emph{non-escaping}.
(i)~Variables \this and \code{super} are raw
    during the object's construction,
    i.e., in field initializers and in {non-escaping} methods
    (methods that cannot escape or leak \this).
(ii)~Obviously constructors are non-escaping,
    but you can also annotate methods \emph{explicitly} as \code{@NonEscaping},
    or they can be inferred to be \emph{implicitly} non-escaping
    if they are called on a raw \this receiver.

For example, \code{m2} is \emph{implicitly} non-escaping (and therefore cannot leak \this)
    because of the call to \code{m2}
    in the constructor.
The user could also mark \code{m2} \emph{explicitly} as non-escaping by using the annotation
    \code{@NonEscaping}.
(Like in Java, \code{@} is used for annotations in X10.)
We recommend explicitly marking non-escaping methods as \code{@NonEscaping} to show intent,
    as done on method \code{m3}.
Without this annotation the call \code{super.m3()} in \code{B} would be illegal,
    due to rule~\userule{Rule4}.
(We could infer that \code{m3} must be non-escaping,
    but that would cause a dependency from a subclass to a superclass,
    which is not natural for people used to separate compilation.)
Finally, we note that all errors in this example are due to rule~\userule{Rule3}
    that prevents leaking a raw \this or \code{super}.



\vspace{-0.2cm}
\subsection{Dynamic dispatch}
Dynamic dispatch interacts with initialization by transferring control to the subclass
    before the superclass completed its initialization.
\Ref{Figure}{Dynamic-dispatch} demonstrates why dynamic dispatch is error-prone during construction:
    calling \code{m1} in \code{A} would dynamically dispatch
    to the implementation in \code{B}
    that would read the default value.



\begin{figure}
\vspace{-0.2cm}\begin{lstlisting}
abstract class A {
  val a1:Int, a2:Int;
  def this() {
    this.a1 = m1(); //err1
    this.a2 = m2();
  }
  abstract def m1():Int;
  @NoThisAccess abstract def m2():Int;
}
class B extends A {
  var b:Int = 3; // non-final field
  def m1() {
    val x = super.a1;
    val y = this.b;
    return 1;
  }
  @NoThisAccess def m2() {
    val x = super.a1; //err2
    val y = this.b; //err3
    return 2;
  }
}
\end{lstlisting}\vspace{-0.2cm}
\definerule{Rule5}
\definerule{Rule6}
\caption{Dynamic dispatch example.
    \myrule{\arabic{Rule5}}{A non-escaping method must be private or final, unless it has \code{@NoThisAccess}}
    \myrule{\arabic{Rule6}}{A method with \code{@NoThisAccess} cannot access \this or \code{super} (neither read nor write its fields)}
    }
\label{Figure:Dynamic-dispatch}
\end{figure}


X10 prevents dynamic dispatch by requiring that non-escaping methods
    must be private or final
    (so overriding is impossible).
For example, \code{\itshape err1} is caused by rule~\userule{Rule5}
    because \code{m1} is neither private nor final nor \code{@NoThisAccess}.

However, sometimes dynamic dispatch is required during construction.
For example, if a subclass needs to refine initialization
    of the superclass's fields.
Such refinement cannot have any access to \this, and therefore
    such methods must be marked with \code{@NoThisAccess}.
For example, \code{\itshape err2} and \code{\itshape err3} are caused by rule~\userule{Rule6} that prohibits access \this or \code{super}
    when using \code{@NoThisAccess}.
\code{@NoThisAccess} prohibits any access to \this,
    however, one could still access the method parameters.
(If the subclass needs to read a certain field of the superclass that was previously assigned,
    then that field can be passed as an argument.) % to the \code{@NoThisAccess} method


In C++, the call to \code{m1} is legal,
    but at runtime
    methods are statically bound,
    so you will get a crash trying to call a pure virtual function.
In Java, the call to \code{m1} is also legal,
    but at runtime
    methods are dynamically bound,
    so the implementation of \code{m1} in \code{B}
    will read the default values of \code{a1} and \code{b}.
%This behavior is undesired in Java,
%    and Java discourages it by trying to catch statically most of these cases.
%For example, Java prohibits calls to member functions before the super object was initialized,
%    as this example shows (which is also illegal in X10):
%\vspace{-0.2cm}\begin{lstlisting}
%class B extends A { B() {super(f()); }}
%\end{lstlisting}\vspace{-0.2cm}




\subsection{Exceptions}
Constructing an object may not always end normally,
    e.g., building a date object from an illegal date string should throw an exception.
Exceptions combined with inheritance interact with initialization in the following way:
    a cooked object must have cooked sub-objects,
    therefore if a constructor ends normally (thus returning a cooked object)
    then all preceding constructor calls (either \code{super(\ldots)} or \code{this(\ldots)})
    must end normally as well.
Phrased differently, in a constructor it should not be possible to
    recover from an exception thrown by a this or super constructor call.
This is one of the reason why a constructor call must be the first statement in Java;
    failure to verify this led to a famous security attack~\cite{Dean:1996}.

\begin{figure}
\vspace{-0.2cm}\begin{lstlisting}
class B extends A {
  def this() {
    try { super(); } catch(e:Throwable){} //err
  }
}
\end{lstlisting}\vspace{-0.2cm}
\definerule{Rule7}
\caption{Exceptions example:
    if a constructor ends normally (without throwing an exception),
        then all preceding constructor calls ended normally as well.
    \myrule{\arabic{Rule7}}{If a constructor does not call \code{super(\ldots)} or \code{this(\ldots)},
        then an implicit \code{super()} is added at the beginning of the constructor;
        the first statement in a constructor is a constructor call (either \code{super(\ldots)} or \code{this(\ldots)});
        a constructor call may only appear as the first statement in a constructor
        }
    }
\label{Figure:Exceptions}
\end{figure}


\Ref{Figure}{Exceptions} shows that it is an error to try to recover from an exception thrown
    by a constructor call;
    the reason for the error is rule~\userule{Rule7} that requires the first statement to be \code{super()}.


\Subsection[Inner]{Inner classes}
Inner classes usually read the outer instance's fields during construction,
    e.g., a list iterator would read the list's header node.
Therefore, X10 requires that the outer instance is cooked,
    and prohibits creating an inner instance when the receiver is a raw \this.


\begin{figure}
\vspace{-0.2cm}\begin{lstlisting}
class Outer {
  val a:Int;
  def this() {
    // Outer.this is raw
    Outer.this. new Inner(); //err
    new Nested(); // ok
    a = 3;
  }
  class Inner {
    def this() {
      // Inner.this is raw, but
      // Outer.this is cooked
      val x = Outer.this.a;
    }
  }
  static class Nested {}
}
\end{lstlisting}\vspace{-0.2cm}
\definerule{Rule8}
\caption{Inner class example: the outer instance is always cooked.
    \myrule{\arabic{Rule8}}{a raw \this cannot be the receiver of \code{new}}
    }
\label{Figure:InnerClass}
\end{figure}

\Ref{Figure}{InnerClass} shows it is an error in X10 to create an inner instance
    if the outer is raw (from rule~\userule{Rule8}),
    but it is ok to create an instance of a static nested class,
    because it has no outer instance.

In fact, it is possible to view this rule as a special case to the rule that
    prohibits leaking a raw \this
    (because when you create an inner instance on a raw \this receiver,
    you created an alias,
    and now you have two raw objects: \code{Inner.this} and \code{Outer.this}).
We wish to keep the invariant that only one \this can be raw.

%To reduce complexity in the following subsections,
In our rules, we assume that there is a single \this reference,
    because we can convert all inner, anonymous and local classes into
    static nested classes
    by passing the outer instance and all other captured variables
    explicitly as arguments to the constructor.






We now turn our attention to X10's sophisticated type-system
    features not found in main-stream languages:
    constraints, properties, class invariants, closures, (non-erased) generics, and structs.

\subsection{Constraints and default/zero values}
X10 supports constrained types using the syntax \code{T\lb{}c\rb},
    where \code{c} is a boolean expression that can use final variables in scope,
    literals, properties (described below),
    the special keyword \code{self}
    that denotes the type itself,
    field access, equality (\code{==}) and disequality (\code{!=}).
There are plans to add arithmetic inequality (\code{<}, \code{<=})
    to X10 in the future,
    and one can plug in any constraint solver into the~X10~compiler.

As a consequence of constrained types,
    some types do not have a default value, e.g., \code{Int\lb self!=0\rb}.
Therefore, in X10, the fields of an object cannot be zero-initialized as done in Java.
Furthermore, in Java, a non-final field does not have to be assigned in a constructor
    because it is assumed to have an implicit zero initializer.
X10 follows the same principle, and a non-final field is implicitly given a zero initializer
    \emph{if its type has-zero}.
\Ref{Figure}{Constraints} defines when a type \emph{has-zero},
    and gives examples of types without zero.
Note that \code{i0} has to be assigned because it is a final field (\code{val}),
    as opposed to \code{i1} which is non-final (\code{var}).

\begin{figure}
\vspace{-0.2cm}\begin{lstlisting}
class A {
  val i0:Int; //err
  var i1:Int;
  var i2:Int{self!=0}; //err
  var i3:Int{self!=0} = 3;
  var i4:Int{self==42}; //err
  var s1:String;
  var s2:String{self!=null}; //err
  var b1:Boolean;
  var b2:Boolean{self==true}; //err
}
\end{lstlisting}\vspace{-0.2cm}
\definerule{Rule10}
\caption{Default value example.
    \textbf{Definition of \emph{has-zero}:}
        {A type \emph{has-zero} if it contains the zero value
            (which is either \code{null}, \code{false}, 0, or
                zero in all fields for user-defined structs)
            or if it is a type parameter guarded with \code{haszero} (see \Ref{Section}{Generics-and-Structs}).}
    \myrule{\arabic{Rule10}}{A \code{var} field that lacks a field initializer and whose type has-zero,
        is implicitly given a zero initializer}
    }
\label{Figure:Constraints}
\end{figure}


\subsection{Properties and the class invariant}
Properties are final fields that can be used in constraints,
    e.g., \code{Array} has a \code{size} property,
    so an array of size 2 can be expressed as: \code{Array\lb self.size==2\rb}.
The differences between a property and a final field are both syntactic and semantic,
    as seen in class \code{A} of \Ref{Figure}{Properties}.
Syntactically, properties are defined after the class name,
    must have a type and cannot have an initializer,
    and must be initialized in a constructor using a property call statement written as \code{property(\ldots)}.
Semantically, properties are initialized before all other fields,
    and they can be used in constraints with the prefix \code{self}.


\begin{figure}
\vspace{-0.2cm}\begin{lstlisting}
class A(a:Int) {
  def this(x:Int) {
    property(x);
  }
}
class B(b:Int) {b==a} extends A {
  val f1 = a+b, f2:Int, f3:A{this.a==self.a};
  def this(x:Int) {
    super(x);
    val i1 = super.a;
    val i2 = this.b; //err
    val i3 = this.f1; //err
    f2 = 2; //err
    property(x);
    f3 = new A(this.a);
  }
}
\end{lstlisting}\vspace{-0.2cm}
\definerule{Rule17}
\definerule{Rule18}
\definerule{Rule19}
\caption{Properties and class invariant example:
        properties (\code{a} and \code{b})
        are final fields that are initialized before all other fields
        using a property call (\code{property(\ldots);} statement).
    {If a class does not define any properties, then
        an implicit \code{property()} is added
        after the (implicit or explicit) \code{super(\ldots)}.}
    {Field initializers are executed in their declaration order
        after the (implicit or explicit) property call.}
    \myrule{\arabic{Rule17}}{If a constructor does not call \code{this(\ldots)},
        then it must have exactly one
        property call, and it must be unconditionally executed
        (unless the constructor throws an exception)}
    \myrule{\arabic{Rule18}}{The class invariant must be satisfied after the property call}
    \myrule{\arabic{Rule19}}{The super fields can only be accessed after \code{super(\ldots)},
        and the fields of \this can only be accessed after \code{property(\ldots)}}
    }
\label{Figure:Properties}
\end{figure}


When using the prefix \this, you can access both properties and other final fields.
The difference between \this and \code{self} is
    shown in field \code{f3} in \Ref{Figure}{Properties}:
    \code{this.a} refers to the property \code{a} stored in \this,
    whereas \code{self.a} refers to \code{a} stored in the object to which \code{f3} refers.
(In the constructor, we indeed see that we assign to \code{f3} a new instance of \code{A}
    whose \code{a} property is equal to \code{this.a}.)

% Should I talk about interface and abstract property methods? Doesn't relate to initialization...

Properties must be initialized before other fields because
    field initializers and field types can refer to properties (see initializer for \code{f1} and the type of \code{f3}).
The superclass's fields can be accessed after the super call,
    and the other fields after the property call;
    field initializers are executed after the property call.

The \emph{class invariant} (\code{\lb{}b==a\rb} in \Ref{Figure}{Properties})
    may refer only to properties,
    and it must be satisfied after the property call (rule~\userule{Rule18}).
%For example, \code{new B(1,1)} is ok,
%    but \code{new B(1,2)} is rejected.



\subsection{Closures}
Closures are functions that can refer to final variables in the enclosing scope,
    e.g., they can refer to final method parameters, locals, and \this.
When a closure refers to a variable, we say that the variable is \emph{captured} by the closure,
    and the variable is thus stored in the closure object.
Closures interact with initialization when they capture \this during construction.

\begin{figure}
%    LeakIt.foo(closure1);
\vspace{-0.2cm}\begin{lstlisting}
class A {
  var a:Int = 3;
  def this() {
    val closure1 = ()=>this.a; //err
    at(here.next()) closure1();
    val local_a = this.a;
    val closure2 = ()=>local_a;
  }
}
\end{lstlisting}\vspace{-0.2cm}
\definerule{Rule20}
\caption{Closures example.
    \myrule{\arabic{Rule20}}{A closure cannot capture a raw \this}
    }
\label{Figure:Closures}
\end{figure}


\Ref{Figure}{Closures} shows why it is prohibited to capture a raw \this in a closure:
    that closure can later escape to another place which will serialize all captured variables
    (including the raw \this, which should not be serialized, see \Ref{Section}{Multiple-Places}).
The work-around for using a field in a closure is to define a local that will refer only to the field (which is definitely cooked)
    and capture the local instead of the field as done in \code{closure2}.



\subsection{Generics and Structs}
\label{Section:Generics-and-Structs}
\emph{Structs} in X10 are header-less inlinable objects
    that cannot inherit code (i.e., they can \emph{implement} interfaces, but cannot \emph{extend} anything).
Therefore an instance of a struct type has a known size and can be inlined in a containing object.
Java's primitive types (\code{int}, \code{byte}, etc) are represented as structs in X10.
Structs, as opposed to classes, do not contain the value \code{null}.

\emph{Generics} in X10 are reified, i.e, not erased as in Java.
For example, a \code{Box[T]} has a single field of type \code{T},
    and instances of \code{Box[Byte]} and \code{Box[Double]}
    have the same size in Java but different sizes in X10.
%an instance of type \code{Box[Byte]} would have a different size than one of type \code{Box[Byte]}
Although generics are not a new concept,
    the combination of generics and the lack of default values
    leads to new pitfalls.
For example, in Java and C\#, it is possible to define an equivalent to

~~~~~~~\code{class A[T] \lb{} var a:T; \rb}\\
However, this is illegal in X10 because we cannot be sure that \code{T} has-zero (see \Ref{Figure}{Constraints}),
    e.g., if the user instantiates \code{A[Int\lb{}self!=0\rb]} then field \code{a} cannot be assigned a zero value
    without violating type-safety.
Therefore X10 has a type predicate written \code{X haszero} that evaluates to true if type \code{X} has-zero.
Using \code{haszero} in a constraint (e.g., in a class invariant or a method guard),
    makes it possible to guarantee that a type-parameter will be instantiated with a type that has-zero.

\begin{figure}
\vspace{-0.2cm}\begin{lstlisting}
class B[T] {T haszero} {
  var f1:T;
  val f2 = Zero.get[T]();
}
struct WithZeroValue(x:Int,y:Int) {}
struct WithoutZeroValue(x:Int{self!=0}) {}
class Usage {
  var b1:B[Int];
  var b2:B[Int{self!=0}]; //err
  var b3:B[WithZeroValue];
  var b4:B[WithoutZeroValue]; //err
}
\end{lstlisting}\vspace{-0.2cm}
\definerule{Rule21}
    \caption{\code{haszero} type predicate example.
    \myrule{\arabic{Rule21}}{A type must be consistent, i.e., it cannot contradict the environment; the environment includes final variables in scope, method guards, and class invariants.}
    }
\label{Figure:Generics}
\end{figure}



\Ref{Figure}{Generics} shows an example of a generic class \code{B[T]}
    that constrains the type-parameter \code{T} to always have a zero value.
According to rule~\userule{Rule10}, field \code{f1} has an implicit zero field initializer.
It is also possible to write the initializer explicitly (as done in field \code{f2}) by using the static method \code{Zero.get[X]()}
    (that is guarded by \code{X haszero}).
Next we see two struct definitions:
    the first has two properties that has-zero,
    and the second has a property that does not have zero.
According to the definition of has-zero in \Ref{Figure}{Constraints},
    a struct has-zero if all its fields has-zero,
    therefore \code{WithZeroValue haszero} is true, but
    \code{WithoutZeroValue haszero} is false.
Finally, we see an example of usages of \code{B[T]},
    where two usages are legal and two are illegal
    (see rule~\userule{Rule21}).







We now turn our attention to the parallel features of X10 for
    concurrent programming (\code{finish} and \code{async})
    and distributed programming (\code{at}\removeGlobalRef{ and global references}).
\Ref{Section}{Parallelism} already explained how parallel code is written in X10,
    and what are the common pitfalls of initialization in parallel code.
Next we present the rules that prevent these pitfalls.

\subsection{Concurrent programming and Initialization}


\begin{figure}
\vspace{-0.2cm}\begin{lstlisting}
class A {
  var f1:Int; // note: var field
  val f2:Int; // note: val field
  val f3:Int;
  def this() {//err: f2 was not definitely assigned
    async f1 = 1; async f2 = 2;
    finish { async f3 = 3; }
  }
}
\end{lstlisting}\vspace{-0.2cm}
\definerule{Rule13}
\definerule{Rule13a}
\caption{Concurrency in initialization example: asynchronously assign to a field.
    \myrule{\arabic{Rule13}}{A constructor must finish assigning to all fields at least once}
    \myrule{\arabic{Rule13a}}{A final field can be assigned at most once}
    %\myrule{\arabic{Rule13}}{All field assignments must finish when the constructor ends}
    }
\label{Figure:Asynchronously-init}
\end{figure}


\Ref{Figure}{Asynchronously-init} shows how to asynchronously assign to fields.
Recall that we wish to guarantee that one can never read an uninitialized field,
    therefore rule~\userule{Rule13} ensures that all fields are assigned at least once.

All three fields in \code{A} are asynchronously assigned,
    however, only \code{f2} is not definitely assigned at the end of the constructor.
Assigning to \code{f3} has an enclosing \code{finish}, so
    it is definitely assigned.
Field \code{f1} is also definitely assigned, because it is non-final
    so from rule~\userule{Rule10} it has an implicit zero field initializer.
However, field \code{f2} is final so it does not have an implicit field initializer.
Moreover, \code{f2} is only asynchronously assigned,
    and the constructor does not have to wait for this assignment to finish,
    thus violating rule~\userule{Rule13}.
(The exact data-flow analysis to enforce rule~\userule{Rule13} is described in
    \Ref{Section}{Read-write-rules}.)
Rule~\userule{Rule13a} is the same as in Java: a final field is assigned \emph{at most} once
    (and, combined with rule~\userule{Rule13}, we know it is assigned \emph{exactly} once).



\begin{figure}
\vspace{-0.2cm}\begin{lstlisting}
class A {
  val f:Int;
  def this() { //err: f was not definitely assigned
    // Execute at another place
    at (here.next())
      this.f = 1; //err: this escaped
  }
}
\end{lstlisting}\vspace{-0.2cm}
\definerule{Rule14}
% todo: Support remote-initialization (at and at back to init a field of this)
\caption{Distributed initialization example. %only cooked objects can be copied.
    \myrule{\arabic{Rule14}}{a raw \this cannot be captured by an \code{at}}
    }
\label{Figure:Multi-place}
\end{figure}

\subsection{Distributed programming and Initialization}
\label{Section:Multiple-Places}
X10 programs can be executed on a distributed system with multiple places
    that have no shared memory.
Objects are copied from one place to another when captured by an \code{at}.
Copying is done by first serializing the object into a buffer,
    sending the buffer to the other place, and then de-serializing the buffer at the other place.
As in Java, one can write custom serialization code in X10 by implementing the \code{CustomSerialization} interface,
    and defining the method \code{serialize():SerialData} and the constructor \code{this(data:SerialData)}.


\Ref{Figure}{Multi-place} shows a common pitfall
    where a raw \this escapes to another place,
    and the field assignment would have been done on a copy of \this.
We wish to de-serialize only cooked objects,
    and therefore rule~\userule{Rule14} prohibits \this to be captured by an \code{at}.
Consequently, we also report that field \code{f} was not definitely assigned.


%\vspace{-0.2cm}\begin{lstlisting}
%class A {
%  var i:Int;
%  val distArray = DistArray.make( ..., (Point)=>this.i); // "this" is serialized to another place before it is cooked
%}
%\end{lstlisting}\vspace{-0.2cm}

\removeGlobalRef{
\subsection{Global references}
\label{Section:Global-references}
A distributed data-structures is dispersed over multiple places,
    and it is convenient to have pointers from one place to an object in another place.
These cross-places pointers are called \emph{global references}.
A global reference has two fields: \code{object} that points to some object,
    and \code{home} which is the place where the global reference was created.
When a global reference is serialized, we serialize its \code{home} and the value of the \emph{pointer} to the \code{object}
    (we do not serialize the \code{object}).
%Phrased differently, serializing an object will recursively serialize all its fields;
%    the recursion ends when there are no fields or with global references.
%Retrieving the object is only allowed at place \code{home}.
%    i.e., the apply method returns the root and is guarded by \code{home==here}.
For example, suppose that \code{o} is some object.
Then, when a box pointing to \code{o} is serialized, then \code{o} is recursively serialized.
However, when a global reference pointing to \code{o} is serialized, then only the pointer to \code{o} is serialized (not \code{o} itself).
%  assert r()==o;
%def copyExample(o:Any, p:Place) {
%\vspace{-0.2cm}\begin{lstlisting}
%  val box = new Box(o);
%  at (here.next()) { // copies box and o
%    val x = box;
%  }
%  val r = new GlobalRef(o);
%  at (here.next()) { // copies r but not o
%    val x = r;
%  }
%\end{lstlisting}\vspace{-0.2cm}

% No room to talk about custom serialization:
% Places require serialization and deserialization (both custom and automatic) across "at".

A \emph{global object} has mutable state in a single place~\code{p}
    and methods that can be called from any place that mutate state in~\code{p}.
The common \emph{global object idiom}
    uses a global reference
    to point to a single mutable object.
\Ref{Figure}{GlobalRef} shows a global counter object
    that has mutable state (\code{count}) in a single place,
    but any place can increment the counter by incrementing \code{count} at that single place,
    which is \code{root.home}.
Note how \code{root()} returns the referent of the global reference.
(The call \code{root()} is guarded by \code{root.home==here}, which is statically verified in this code.)


%  def me() = root();
%}
%class B extends A {
%  def this() {
%    val alias = me(); //err
%  }
%}
%If \code{me()} was prefixed with
%\code{@NonEscaping public final}
%then accessing \code{root} would be an error.
%Cannot use 'root' because a GlobalRef[\ldots](this) cannot be used in a field initializer, constructor, or methods called from a constructor.

\begin{figure}
\vspace{-0.2cm}\begin{lstlisting}
class GlobalCounter {
  private val root = new GlobalRef(this);
  transient var count:Int;
  def this() {
    val aliasToThis = this.root(); //err
  }
  def inc() {
    return at(root.home) root().count++;
  }
}
\end{lstlisting}\vspace{-0.2cm}
\definerule{TransientRule}
\caption{A global counter example.
    \textbf{Revision of~\userule{Rule3}:}
        A raw \this can only escape to a \code{GlobalRef} constructor in a private field initializer;
            that field cannot be read via a raw \this receiver.
    \myrule{\arabic{TransientRule}}{The type of a \code{transient} field must satisfy \code{haszero}}
    }
\label{Figure:GlobalRef}
\end{figure}

Note that a raw \this leaked into the constructor of \code{GlobalRef}
    which is a violation of rule~\userule{Rule3}.
Because this idiom is common in distributed programming,
    we relaxed this rule and allow \this to escape but only into a \code{GlobalRef} object
    with a severe restriction:
    it must happen in a private field initializer that is not read during construction.
For example, \code{root} cannot be read in the constructor
    because it contains an alias to \code{this} that might escape.

As in Java, the \code{transient} keyword marks a field that should not be serialized.
Upon de-serialization, such a field has the zero value.
Therefore, the type of a transient field must have the zero value (rule~\userule{TransientRule}).

We finally note that the global object idiom is error-prone because
    it is easy to forget to use \code{root()} before accessing a mutable field.
There is an RFC that suggests an annotation that will automatically convert a class to a global class
    using this pattern.


}


\subsection{Read and write of fields}
\label{Section:Read-write-rules}


\begin{figure}
\vspace{-0.2cm}\begin{lstlisting}
class A {
  val a:Int;
  def this() {
    readA(); //err1
    finish {
      async {
        a = 1; // assigned={a}
        readA();
      } // asyncAssigned={a}
      readA(); //err2
    } // assigned={a}
    readA();
  }
  private def readA() { // reads={a}
    val x = a;
  }
}
class B {
  var i:Int{self!=0}, j:Int{self!=0};
  def this() {
    finish {
     asyncWriteI(); // asyncAssigned={i}
    } // assigned={i}
    writeJ();// assigned={i,j}
    readIJ();
  }
  private def asyncWriteI() { // asyncAssigned={i}
    async i=1;
  }
  private def writeJ() { // reads={i} assigned={j}
    if (i==1) writeJ(); else this.j = 1;
  }
  private def readIJ() { // reads={i,j}
    val x = this.i+this.j;
  }
}
\end{lstlisting}\vspace{-0.2cm}
\definerule{Rule11}
\definerule{Rule11b}
\caption{Read-Write order for fields.
    We infer for each method three sets:
        (i)~fields it reads (i.e., these fields must be assigned before the method is called),
        (ii)~fields it assigns,
        (iii)~fields it assigns asynchronously.
    The data-flow maintains these three sets before and after each statement;
        \code{\itshape assigned} becomes \code{\itshape asyncAssigned} after an \code{async},
        and \code{\itshape asyncAssigned} becomes \code{\itshape assigned} after a \code{finish}.
    In this example, we omitted empty sets.
    \myrule{\arabic{Rule11}}{A field may be read only if it is definitely-assigned}
    \myrule{\arabic{Rule11b}}{A final field may be written only if it is definitely-unassigned}
    }
\label{Figure:Read-Write-Order}
\end{figure}

We now present a data-flow analysis for guaranteeing
    that a field is read only after it was written,
    and that a final field is assigned exactly once.
Java performs an \emph{intra}-procedural data-flow analysis in \emph{constructors} to calculate
    when a \emph{final} field is definitely-assigned and definitely-unassigned.
In contrast, X10 performs an \emph{inter}-procedural fixed-point data-flow analysis
    in all \emph{non-escaping methods} (and constructors) to calculate
    when {a} field (\emph{both final and non-final}) is
    definitely-assigned, \emph{definitely-asynchronously-assigned}, and definitely-unassigned.
The details are explained using examples (\Ref{Figure}{Read-Write-Order}) by comparison with Java;
    the full analysis is described in X10's language specification.

X10, like Java, allows \emph{writing} to a final field only when it is definitely-\emph{unassigned},
    and it allows \emph{reading} from a final field only when it is definitely-\emph{assigned}.
X10 also has the same read restriction on non-final fields
    (recall that rule~\userule{Rule10} adds a field initializer if the field's type has-zero).


Consider first only final fields.
    They are easier to type-check because they can only be assigned in constructors.
X10 extends Java rules,
    by calculating for each non-escaping method \code{m} the set of final fields it reads,
    and calling \code{m} is legal only if these fields have been definitely assigned.
For example, in class \code{A}, method \code{readA} reads field \code{a}
    and therefore cannot be called before \code{a} is assigned (e.g., \code{\itshape err1}).
Note that Java does not perform this check, and it is legal to call \code{readA}
    which will return the zero value of \code{a}.
X10 also adds the notion of \emph{definitely-asynchronously-assigned}
    which means a field was definitely-assigned within an \code{async}
    (so it cannot be read, e.g., \code{\itshape err2}),
    but after an enclosing \code{finish} it will become definitely-assigned
    (so it can be read).
The flow maintains three sets:
    \code{\itshape reads}, \code{\itshape assigned}, and \code{\itshape asyncAssigned}.
If a method reads an uninitialized field, then we add it to its \code{\itshape reads} set;
    however, if a constructor reads an uninitialized field, then it is an error.
Phrased differently, the \code{\itshape reads} set of a constructor must be empty.

Now consider non-final fields.
    They can be assigned and read in methods,
        thus requiring a fixed-point algorithm.
For example, consider method \code{writeJ}.
Initially, \code{\itshape reads} is empty,
    while \code{\itshape assigned} and \code{\itshape asyncAssigned} are the entire set of fields.
In the first iteration, we add \code{i} to \code{\itshape reads},
    and when we join the two branches of the \code{if},
    \code{\itshape assigned} is decreased to only \code{j}.
The fixed-point calculation, in every iteration, increases \code{\itshape reads}
    and decreases \code{\itshape assigned} and \code{\itshape asyncAssigned},
    until a fixed-point is reached.


\subsection{Static initialization}
Unlike Java, X10 does not support dynamic class loading,
    and all static fields in X10 are final.
Thus, initialization of static fields is a one-time phase %, denoted the static-init phase,
    that is done before the \code{main} method is executed.
Reading a static field in this phase \emph{waits} until the field is initialized,
    which may lead to dead-lock.
However, in practice, deadlock is rare,
    and usually found quickly the first time a program is executed.
%The exact details can be found in the language specification.

\subsection{Virtues and attributes of initialization in X10}
We assume there is a single \this variable, because all nested classes can be converted to static,
    as described in \Ref{Section}{Inner}.
Therefore, initialization in X10 has the following attributes:
(i)~\this (and its alias \code{super}) is the only accessible raw object in scope (rule~\userule{Rule3}),
(ii)~only cooked objects cross places (rule~\userule{Rule14}),
(iii)~only \code{@NoThisAccess} methods can be dynamically dispatched during construction (rule~\userule{Rule5}),
(iv)~all final field assignments finish by the time the constructor ends (rule~\userule{Rule13}),
(v)~it is not possible to read an uninitialized field (rule~\userule{Rule11}), and
(vi)~reading a final field always results in the same value (rule~\userule{Rule11b} combined with attribute~(v)).


\vspace{-0.3cm}
\Section[designs]{Alternative Initialization Designs}


\subsection{Default values design}
every type has a default value.
(when reading a val field, you may get the default value or its final value.)

\subsection{proto design}
A proto modifier for locals
allowed cyclic immutable objects.

\subsection{Hardhat design}
\label{Section:hardhat}

Discuss why we didn't allow aliasing of \this (to avoid an alias analysis).

A method is non-escaping, if it is called from a constructor or from another non-escaping method.
A non-escaping method must be private, final, or in a final class.
Constructors and non-escaping methods cannot leak \this.


- "this" cannot escape, no dynamic dispatching, etc.
- Dataflow for definite-assignment for locals (similarity to Java, with the extension for async-initialization)
- Dataflow for definite-assignment for fields:
  + @NoThisAccess and @NonEscaping
  + The fixed-point algorithm to infer the sets of fields that are written, async-written, and read, on each private method that is transitively called from a ctor or field-init. Maybe also add a nice proof :)


Property design (what would happen if we use val fields instead of properties.)

Fields are partition into two: property fields and normal fields. Property fields are assigned together before any other field.
\begin{lstlisting}
class A(x:Int) {
  val y:Int{self!=x};
  val z = x*x;
}
\end{lstlisting}
Vs.
Fields are ordered, and all ctors need to assign in the same order as the fields declaration.
\begin{lstlisting}
class A {
  val x:Int;
  val y:Int{self!=x};
  val z = x*x;
}
\end{lstlisting}



what can you read before (inclusive), and what after.


Hardhat claims:
* \this is the only accessible raw object (there could be several raw objects in the heap but only \this is accessible). Reason: \this cannot be aliased or leaked.
* only cooked objects cross places. Reason: \this is the only non-cooked object and it cannot have any aliases and it cannot cross an \code{at}.
* there is no dynamic dispatch during construction. Reason: only calling private or final methods.
* all field writes finish by the time the ctor ends. Reason: data flow ensures that any write within an async has an enclosing finish.
* one cannot read an uninitialized field. Reason: reading from \this is ok by data-flow, any other read is from a cooked object.


%\Section[implementation]{Implementation}
%implementation design, overheads, some measurements, etc.

outlines our implementation within the X10 compiler using the polyglot framework,
    the compilation time overhead of checking these initialization rules,
    and the annotation overhead in our X10 code base.


Due to page limitation, we mainly focused on the formal effect system for POPL,
but we can easily add an empirical evaluation section that describes some test cases (where minor code refactoring was needed) and shows the annotation burden.
X10 has only two possible method annotations: @NonEscaping, @NoThisAccess.
Methods transitively called from a constructor are implicitly non-escaping (but the compiler issues a warning that they should be marked as @NonEscaping).
SPECjbb and M3R are closed-source whereas the rest is open-source and publicly available at x10-lang.org

------------------------------------------------------------------------------
Programs:           XRX SPECjbb     M3R UTS Other
\# of lines          27153   14603       71682   2765    155345
\# of files          257 63      294 14  2283
\# of constructors       276 267     401 23  1297
\# of methods            2216    2475        2831    124 8273
\# of non-escaping methods   8   38      34  3   83
\# of @NonEscaping       7   7       13  1   62
\# of @NoThisAccess      1   0       1   0   12
------------------------------------------------------------------------------
XRX: X10 Runtime (and libraries)
SPECjbb: SPECjbb from 2005 converted to X10
M3R: Map-reduce in X10
UTS: Global load balancing
Other: Programmer guide examples, test suite, issues, samples
------------------------------------------------------------------------------

As can be seen, the annotations burden is minor.

Asynchronous initialization was not used in our applications because they pre-date this feature.
(It is used in our examples and tests 50 times.)
However, it is a useful pattern, especially for local variables.
More importantly, the analysis prevents bugs such as:
val res:Int;
finish {
  async {
    res = doCalculation();
  }
  // WRONG to use res here
}
// OK to use res here

Here are two examples for the use of annotations:
1) In Any.x10 we have:
@NoThisAccess def typeName():String
Method typeName is overridden in subclasses to return a constant string (all structs automatically override this method).
This annotation allows typeName() to be called even during construction.
2) In HashMap.x10, after we added the strict initializations rules, we had to refactor put and rehash methods into:
public def put(k: K, v: V) = putInternal(k,v);
@NonEscaping protected final def putInternal(k: K, v: V) {...}
(Similarly, we have rehash() and rehashInternal())
The reason is that putInternal is called from the deserialization constructor:
def this(x:SerialData) { ... putInternal(...) ... }
And we still want subclasses to be able to override the "put" method.



Compilation time:   XRX SPECjbb     M3R UTS Other
total           65241   78952       254020  72205   548547
Fields          156 1649        3330    1272    2862
Locals          32  51      117 33  126


Implementation lines of code:
CheckEscapingThis: 951
InitChecker: 805
Polyglot dataflow framework: DataFlow 1309


%\Section[case-study]{Case Study}
%a lot of Java code was recently translated to X10, and Java is less strict regarding initialization. How did affect the translation?


\Section[related-work]{Related Work}
\label{sec:related}

\subsection{Programming Models} 
Determinism in parallel programming is a very active area of research.
Guava \cite{guava} introduces restrictions on shared memory Java
programs that ensure no data-races primarily by distinguishing
monitors (all access is synchronized) from values (immutable) and
objects (private to a thread). However Guava is not safe since Guava
programs may use \code{wait}/\code{notify} for arbitrary concurrent
signalling and hence may not executable with a sequential
schedule. The Revisions programming model \cite{Revisions} guarantees
determinism by isolating asynchronous tasks but merging their writes
determinately. However, the model explicitly does not require that
a sequential schedule be valid (c.f. Figure~1 in \cite{Revisions}).

%\cite{Steele:1989:MAP:96709.96731} introduced the idea that

DPJ develops the ``determinacy-by-default'' slogan using a static
type-and-effects system to establish commutativity of concurrent
actions.  The deterministic fragment of DPJ is safe according to the
definition above. Safe X10 offers a much richer concurrency model
which guarantees the safety of common idioms such as accumulators and
cyclic tasks (clocks) without relying on effects annotations. The
lightweight effects mechanism in X10 can be extended to support a much
richer effects framework (along the lines of DPJ) using X10's
constrained type system.  We leave this as future work.

The SafeJava language \cite{Rinard04safejava:a} is unfortunately not safe
according to our definition, even though it guarantees determinacy and
deadlock freedom, using ownership types, unique pointers and partially
ordered lock levels. Again, a sequential scheduler is not admissible
for the model.

Some data-flow synchronization based languages and frameworks (e.g.{}
Kahn style process networks \cite{kahn1974semantics}, concurrent
constraint programming \cite{saraswat1993concurrent}, \cite{SHIM}) are guaranteed
determinate but not safe according to our definition since they do not
permit sequential schedules. Indeed they permit the possibility of
deadlock. (The notion of safety is also not quite relevant since these
frameworks do not support shared mutable variables.)

Synchronous programming languages like Esterel are completely
deterministic, but meant for embedded targets. An Esterel program
executes in clock steps and the outputs are conceptually synchronous
with its inputs.  It is a finite state language that is easy to verify
formally. An Esterel program is susceptible to
causalities. Causalities are similar to deadlocks, but can be easily
detected at compile-time.  The problem with synchronous models is that
they do not perform well. To out knowledge, most Esterel compilers
generate sequential code and not concurrent.
 
SHIM~\cite{edwards2005shim2,tardieu2006scheduling-independent} is also
a deterministic concurrent programming language, but SHIM programs may
deadlocks.  Secondly, SHIM allows only a single task to write at any
phase; we allow multiply writes.

Apart from SHIM, there are a few programming models and languages that
provide explicit determinism. StreamIt~\cite{thies2001streamit}, for
example is a synchronous dataflow language that provides
determinism. It has simple static verification techniques for deadlock
and buffer-overflow.  However, StreamIt is a strict subset of SHIM and
StreamIt's design limits it to a small class of streaming
applications.
 


In contrast, 
Cilk~\cite{blumofe1995cilk} is a non-deterministic language that it covers a larger
class of applications. It is C based
and the programmer must explicitly ask for parallelism using 
the \emph{spawn} and the \emph{sync} constructs. 
Cilk permits data races.
%\figref{non-det}, for example,
%is a non-deterministic concurrent program in Cilk. 
Explicit techniques~\cite{cheng1998detecting} are
required for checking data races in Cilk programs.  




\subsection{Determinizing Tools} 
Determinizing run-times support coarse-grained fork-join concurrency
by maintaining a different copy of memory for each activity and
merging them determinately at finish points (\cite{grace},
\cite{dmp}, \cite{kendo},\cite{determinator}). Safe
X10 can run on such systems in principle, but does not require them.
To execute Safe X10, such systems need to support fine-grained
asynchrony (with some form of work-stealing or fork-joining
scheduler), clocks and accumulators.

Kendo is a purely software system that deterministically multi-threads
concurrent applications.  Kendo~\cite{olszewski2009kendo} ensures a
deterministic order of all lock acquisitions for a given program
input.

Kendo comes with three shortcomings. It operates completely at runtime,
and there is considerable performance penalty. Secondly, if
we have the sequence \emph{lock(A); lock (B)} in one thread and
\emph{lock(B); lock(A)} in another thread, a deterministic ordering of
locks may still deadlock. Thirdly, the tool operates only when
shared data is protected by locks.

Software Transactional Memory (STM)~\cite{shavit1995software}
  is an alternative to locks: a thread completes modifications to 
shared memory without regard for what other threads might be doing. At the end of the transaction,
it validates and commits if the validation was successful, otherwise it rolls back and re-executes
the transaction. STM mechanisms avoid races but do not solve the non-determinism problem.

Berger's Grace\cite{berger2009grace} is a run-time tool
that is based on STM. 
If there is a conflict during commit, the threads are committed in
a particular sequential order (determined by the order
The problem with Grace is that it incurs a lot of run-time
overhead. This dissertation  partially solves this overhead problem
by addressing the issue at compile-time and
thereby reducing a considerable amount of run-time overhead.

Like Grace, Determinator\cite{aviram2010efficient} is another tool
that allows parallel processes to execute as long as they do not share 
resources. If they do share resources and the accesses are unsafe, then
the operating throws an exception (a page fault). 

Cored-Det~\cite{bergan2010coreDet}, based on DMP~\cite{devietti2009dmp} 
uses a deterministic token that is passed
among all threads.  A thread to modify a shared variable must first
wait for the token and for all threads to block on that
token. DMP is hardware based. 
Although, deadlocks may be avoided, we believe this setting is
non-distributed because it forces all threads to synchronize and
therefore leads to a considerable performance penalty. In Safe X10,
only threads that share a particular channel must synchronize
on that channel; other threads can run independently.

 Deterministic replay systems~\cite{choi1998deterministic,altekar2009odr} facilitate debugging of concurrent programs to produce
repeatable behavior. They are based on record/replay systems. The system
replays a specific behavior (such as thread interleaving) of a concurrent
program based on records. The primary purpose of replay systems 
is debugging; they do not guarantee determinism. 
They incur a high runtime overhead and are input dependent.
For every new input, a new set of records is generally maintained.

Like replay systems, Burmin and Sen~\cite{Burnim2009asserting} provide a framework for
checking determinism for multi-threaded programs. Their tool does not
introduce deadlocks, but their tool does not guarantee determinism
because it is merely a testing tool that checks the execution trace
with previously executed traces to see if the values match. Our
goal is to guarantee determinism at compile time -- given a program,
it will generate the same output for a given input.



%\subsection{Type Systems and Verifiers} 
%Finally, type and effect systems like DPJ~\cite{bocchino2009type} 
% have been designed for deterministic parallel programming to see if
%memory locations overlap. Our technique is more explicit. 
%In general, type systems require the programmer to manually annotate the program. Our model can also be implemented using annotations in existing
%programming languages - we in fact annotated the X10 programming language.

%Martin Vechev's tool \cite{vechev2011automatic}
%finds determinacy bugs in loops that run parallel bodies. It analyzes
%array references and indices to ensure that there are no read-write and 



\Section[conclusion]{Conclusion}
We show that many determinate concurrent programs can be written in
\Xten{} using determinate, deadlock-free constructs, so that they are
determinate by design.


\acks
This material is based in part on work supported by the Defense Advanced Research Projects Agency under its Agreement No. HR0011-07-9-0002.


\bibliographystyle{abbrv} %plain plainnat abbrvnat abbrv
\bibliography{x10-init}

\end{document}


Tested my serialization hypotheses using this Java code:
        ObjectOutput out = new ObjectOutputStream(new FileOutputStream("filename.ser"));
        out.writeObject("C:\\cygwin\\home\\Yoav\\intellij\\sourceforge\\x10.tests,C:\\cygwin\\home\\Yoav\\intellij\\sourceforge\\x10.dist\\samples,C:\\cygwin\\home\\Yoav\\intellij\\sourceforge\\x10.runtime\\src-x10"+"C:\\cygwin\\home\\Yoav\\intellij\\sourceforge\\x10.tests,C:\\cygwin\\home\\Yoav\\intellij\\sourceforge\\x10.dist\\samples,C:\\cygwin\\home\\Yoav\\intellij\\sourceforge\\x10.runtime\\src-x10"+"C:\\cygwin\\home\\Yoav\\intellij\\sourceforge\\x10.tests,C:\\cygwin\\home\\Yoav\\intellij\\sourceforge\\x10.dist\\samples,C:\\cygwin\\home\\Yoav\\intellij\\sourceforge\\x10.runtime\\src-x10"+"C:\\cygwin\\home\\Yoav\\intellij\\sourceforge\\x10.tests,C:\\cygwin\\home\\Yoav\\intellij\\sourceforge\\x10.dist\\samples,C:\\cygwin\\home\\Yoav\\intellij\\sourceforge\\x10.runtime\\src-x10"+"C:\\cygwin\\home\\Yoav\\intellij\\sourceforge\\x10.tests,C:\\cygwin\\home\\Yoav\\intellij\\sourceforge\\x10.dist\\samples,C:\\cygwin\\home\\Yoav\\intellij\\sourceforge\\x10.runtime\\src-x10"+"C:\\cygwin\\home\\Yoav\\intellij\\sourceforge\\x10.tests,C:\\cygwin\\home\\Yoav\\intellij\\sourceforge\\x10.dist\\samples,C:\\cygwin\\home\\Yoav\\intellij\\sourceforge\\x10.runtime\\src-x10"+"C:\\cygwin\\home\\Yoav\\intellij\\sourceforge\\x10.tests,C:\\cygwin\\home\\Yoav\\intellij\\sourceforge\\x10.dist\\samples,C:\\cygwin\\home\\Yoav\\intellij\\sourceforge\\x10.runtime\\src-x10"+"C:\\cygwin\\home\\Yoav\\intellij\\sourceforge\\x10.tests,C:\\cygwin\\home\\Yoav\\intellij\\sourceforge\\x10.dist\\samples,C:\\cygwin\\home\\Yoav\\intellij\\sourceforge\\x10.runtime\\src-x10".substring(1));
        out.close(); // 2KB!!
        ObjectOutput out2 = new ObjectOutputStream(new FileOutputStream("filename2.ser"));
        out2.writeObject("C".substring(1));
        out2.close(); // 7bytes!


\subsection{Alternative property design}

Property design (what would happen if we use val fields instead of properties.)

Fields are partition into two: property fields and normal fields. Property fields are assigned together before any other field.
\begin{lstlisting}
class A(x:Int) {
  val y:Int{self!=x};
  val z = x*x;
}
\end{lstlisting}
Vs.
Fields are ordered, and all ctors need to assign in the same order as the fields declaration.
\begin{lstlisting}
class A {
  val x:Int;
  val y:Int{self!=x};
  val z = x*x;
}
\end{lstlisting}



\subsection{Static initialization}
X10 does not support dynamic class loading as opposed to Java,
    and all static fields in X10 are final.
Thus, initialization of static fields is a one-time phase, denoted the static-init phase,
    that is done before the \code{main} method is executed.

During the static-init phase we must finish writing to all static fields,
    and reading a static field \emph{waits} until the field is initialized
    (i.e., the current activity/thread blocks if the field was not written to,
    and it resumes after another activity writes to it).
Obviously, this may lead to deadlock as demonstrated by \Ref{Figure}{Static-init}.
However, in practice, deadlock is rare,
    and usually found quickly the first time a program is executed.

\begin{figure}
\begin{lstlisting}
class A {
  static val a:Int = B.b;
}
class B {
  static val b:Int = A.a;
}
\end{lstlisting}
\caption{Static initialization example:
    the program will deadlock at run-time
    during the static-init phase (before the \code{main} method is executed).
    }
\label{Figure:Static-init}
\end{figure}


\subsection{Memory model and constructor barrier}

todo: discuss final in Java and String class.
(people think that removing final may only hurt performance, but it may be semantic changing.)

type safety and the weak memory model:
* in Java if you don't use final fields correctly or leak this, you will simply see default values (you don't lose type-safety)
* in X10 it could break type safety (if we don't put a barrier at the end of a ctor).

class Box[T] {T haszero} {
  var value:T;
}
class A {
  static val box = new Box[A]();

  var f:Int{self!=0} = 1;
}

var a = new A();
a.f = 2;
A.box.value = a;

Suppose another activity reads A.box.value.
Should the writes to
a.f
and
A.box.value
be ordered? (I don't think we should order them without losing performance)

Therefore, X10 needs a synchronization barrier at the end of a ctor that guarantees that all writes to the fields (both VAL and VAR) of the object has finished before the handle is returned.
(This is different from Java that only promises this for final fields. And the barrier also happens again after deserialization - requiring this weird freeze operation and you could freeze a final field again even after the ctor ended due to deserialization.)


Bowen proposes this transformation in order to inline ctors:
class A {
  static val box = new Box[A]();

  var f:Int{self!=0} = 0;
}


var a = new A();
a.f = 1;
// INSERT BARRIER EXPLICITLY HERE
a.f = 2;
A.box.value = a;
