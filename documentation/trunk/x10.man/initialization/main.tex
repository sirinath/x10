\documentclass[9pt]{sigplanconf}


\conferenceinfo{X10'11,} {June 4, 2011, San Jose, California, USA.}
\CopyrightYear{2011}
%\copyrightdata{}

% Doesn't work in a caption!
%\newcounter{RuleCounter}
%\stepcounter{RuleCounter}
% \arabic{RuleCounter}
\newcommand{\myrule}[2]{\textbf{Rule #1:} #2.}

\usepackage{xspace}

% Macros for R^nRS.

\def\@makechapterhead#1{%
  \vspace*{50\p@}%
  {\parindent \z@ \raggedright \normalfont
    \ifnum \c@secnumdepth >\m@ne
        \huge\bfseries \thechapter \space\space\space
        \nobreak
    \fi
    \interlinepenalty\@M
    \Huge \bfseries #1\par\nobreak
    \vskip 40\p@
  }}


\makeatletter

\newcommand{\topnewpage}{\@topnewpage}

% Chapters, sections, etc.

\newcommand{\vest}{}
\newcommand{\dotsfoo}{$\ldots\,$}

\newcommand{\sharpfoo}[1]{{\tt\##1}}
\newcommand{\schfalse}{\sharpfoo{f}}
\newcommand{\schtrue}{\sharpfoo{t}}

\newcommand{\singlequote}{{\tt'}}  %\char19
\newcommand{\doublequote}{{\tt"}}
\newcommand{\backquote}{{\tt\char18}}
\newcommand{\backwhack}{{\tt\char`\\}}
\newcommand{\atsign}{{\tt\char`\@}}
\newcommand{\sharpsign}{{\tt\#}}
\newcommand{\verticalbar}{{\tt|}}

\newcommand{\coerce}{\discretionary{->}{}{->}}

% Knuth's \in sucks big boulders
\def\elem{\hbox{\raise.13ex\hbox{$\scriptstyle\in$}}}

\newcommand{\meta}[1]{{\noindent\hbox{\rm$\langle$#1$\rangle$}}}
\let\hyper=\meta
\newcommand{\hyperi}[1]{\hyper{#1$_1$}}
\newcommand{\hyperii}[1]{\hyper{#1$_2$}}
\newcommand{\hyperj}[1]{\hyper{#1$_i$}}
\newcommand{\hypern}[1]{\hyper{#1$_n$}}
\newcommand{\var}[1]{\noindent\hbox{\it{}#1\/}}  % Careful, is \/ always the right thing?
\newcommand{\vari}[1]{\var{#1$_1$}}
\newcommand{\varii}[1]{\var{#1$_2$}}
\newcommand{\variii}[1]{\var{#1$_3$}}
\newcommand{\variv}[1]{\var{#1$_4$}}
\newcommand{\varj}[1]{\var{#1$_j$}}
\newcommand{\varn}[1]{\var{#1$_n$}}

\newcommand{\vr}[1]{{\noindent\hbox{$#1$\/}}}  % Careful, is \/ always the right thing?
\newcommand{\vri}[1]{\vr{#1_1}}
\newcommand{\vrii}[1]{\vr{#1_2}}
\newcommand{\vriii}[1]{\vr{#1_3}}
\newcommand{\vriv}[1]{\vr{#1_4}}
\newcommand{\vrv}[1]{\vr{#1_5}}
\newcommand{\vrj}[1]{\vr{#1_j}}
\newcommand{\vrn}[1]{\vr{#1_n}}


\newcommand{\defining}[1]{\mainindex{#1}{\em #1}}
\newcommand{\ide}[1]{{\schindex{#1}\frenchspacing\tt{#1}}}

\newcommand{\lambdaexp}{{\cf lambda} expression}
\newcommand{\Lambdaexp}{{\cf Lambda} expression}
\newcommand{\callcc}{{\tt call-with-current-continuation}}

% \reallyindex{SORTKEY}{HEADCS}{TYPE}
% writes (index-entry "SORTKEY" "HEADCS" TYPE PAGENUMBER)
% which becomes  \item \HEADCS{SORTKEY} mainpagenumber ; auxpagenumber ...

\global\def\reallyindex#1#2#3{%
\write\@indexfile{"#1" "#2" #3 \thepage}}

\newcommand{\mainschindex}[1]{\label{#1}\reallyindex{#1}{tt}{main}}
\newcommand{\mainindex}[1]{\reallyindex{#1@{\rm #1}{main}}}
\newcommand{\schindex}[1]{\reallyindex{#1}{tt}{aux}}
\newcommand{\sharpindex}[1]{\reallyindex{#1}{sharpfoo}{aux}}
%vj%\renewcommand{\index}[1]{\reallyindex{#1}{rm}{aux}}

\newcommand{\domain}[1]{#1}
\newcommand{\nodomain}[1]{}
%\newcommand{\todo}[1]{{\rm$[\![$!!~#1$]\!]$}}
\newcommand{\todo}[1]{}

% \frobq will make quote and backquote look nicer.
\def\frobqcats{%\catcode`\'=13 %\catcode`\{=13{}\catcode`\}=13{}
\catcode`\`=13{}}
{\frobqcats
\gdef\frobqdefs{%\def'{\singlequote}
\def`{\backquote}}}%\def\{{\char`\{}\def\}{\char`\}}
\def\frobq{\frobqcats\frobqdefs}

% \cf = code font
% Unfortunately, \cf \cf won't work at all, so don't even attempt to
% next constructions which use them...
\newcommand{\cf}{\frenchspacing\tt}

% Same as \obeycr, but doesn't do a \@gobblecr.
{\catcode`\^^M=13 \gdef\myobeycr{\catcode`\^^M=13 \def^^M{\\}}%
\gdef\restorecr{\catcode`\^^M=5 }}

{\catcode`\^^I=13 \gdef\obeytabs{\catcode`\^^I=13 \def^^I{\hbox{\hskip 4em}}}}

{\obeyspaces\gdef {\hbox{\hskip0.5em}}}

\gdef\gobblecr{\@gobblecr}

\def\setupcode{\@makeother\^}

% Scheme example environment
% At 11 points, one column, these are about 56 characters wide.
% That's 32 characters to the left of the => and about 20 to the right.

\newenvironment{x10noindent}{
  % Commands for scheme examples
  \newcommand{\ev}{\>\>\evalsto}
  \newcommand{\lev}{\\\>\evalsto}
  \newcommand{\unspecified}{{\em{}unspecified}}
  \newcommand{\scherror}{{\em{}error}}
  \setupcode
  \small \cf \obeytabs \obeyspaces \myobeycr
  \begin{tabbing}%
\qquad\=\hspace*{5em}\=\hspace*{9em}\=\kill%   was 16em
\gobblecr}{\unskip\end{tabbing}}

%\newenvironment{scheme}{\begin{schemenoindent}\+\kill}{\end{schemenoindent}}
\newenvironment{x10}{
  % Commands for scheme examples
  \newcommand{\ev}{\>\>\evalsto}
  \newcommand{\lev}{\\\>\evalsto}
  \renewcommand{\em}{\rmfamily\itshape}
  \newcommand{\unspecified}{{\em{}unspecified}}
  \newcommand{\scherror}{{\em{}error}}
  \setupcode
  \small \cf \obeyspaces \myobeycr
  \footnotesize
  \begin{tabbing}%
\qquad\=\hspace*{5em}\=\hspace*{9em}\=\+\kill%   was 16em
\gobblecr}{\unskip\end{tabbing}\normalsize}

\newcommand{\evalsto}{$\Longrightarrow$}

% Manual entries

\newenvironment{entry}[1]{
  \vspace{3.1ex plus .5ex minus .3ex}\noindent#1%
\unpenalty\nopagebreak}{\vspace{0ex plus 1ex minus 1ex}}

\newcommand{\exprtype}{syntax}

% Primitive prototype
\newcommand{\pproto}[2]{\unskip%
\hbox{\cf\spaceskip=0.5em#1}\hfill\penalty 0%
\hbox{ }\nobreak\hfill\hbox{\rm #2}\break}

% Parenthesized prototype
\newcommand{\proto}[3]{\pproto{(\mainschindex{#1}\hbox{#1}{\it#2\/})}{#3}}

% Variable prototype
\newcommand{\vproto}[2]{\mainschindex{#1}\pproto{#1}{#2}}

% Extending an existing definition (\proto without the index entry)
\newcommand{\rproto}[3]{\pproto{(\hbox{#1}{\it#2\/})}{#3}}

% Grammar environment

\newenvironment{grammar}{
  \def\:{& \goesto{} &}
  \def\|{& $\vert$& }
  \def\opt{$^?$\ }
  \def\star{$^*$\ }
  \def\plus{$^+$\ }
  \em
  \begin{tabular}{rcl}
  }{\unskip\end{tabular}}

%\newcommand{\unsection}{\unskip}
\newcommand{\unsection}{{\vskip -2ex}}

% Commands for grammars
\newcommand{\arbno}[1]{#1\hbox{\rm*}}  
\newcommand{\atleastone}[1]{#1\hbox{$^+$}}

\newcommand{\goesto}{{\normalfont{::=}}}

% mark modifications (for the grammar) From Igor Pechtchanski/Watson/IBM@IBMUS
\newlength{\modwidth}\setlength{\modwidth}{0.005in}
\newlength{\modskip}\setlength{\modskip}{.4em}
\newlength{\@modheight}
\newlength{\@modpos}
\providecommand{\markmod}[1]{%
  \setlength{\@modheight}{#1}%
  \addtolength{\@modheight}{-0.06in}%
  \setlength{\@modpos}{\linewidth}%
  \addtolength{\@modpos}{0.285in}%         Magic
  \addtolength{\@modpos}{\modwidth}%
  \addtolength{\@modpos}{\modskip}%
  \marginpar{\vspace{-\@modheight}%
             \hspace{-\@modpos}%
             \rule{\modwidth}{#1}}%
}

% The index

\def\theindex{%\@restonecoltrue\if@twocolumn\@restonecolfalse\fi
%\columnseprule \z@
%!! \columnsep 35pt
\clearpage
\@topnewpage[
    \centerline{\large\bf\uppercase{Alphabetic index of definitions of concepts,}}
    \centerline{\large\bf\uppercase{keywords, and procedures}}
    \vskip 1ex \bigskip]
    \markboth{Index}{Index}
    \addcontentsline{toc}{chapter}{Alphabetic index of 
 definitions of concepts, keywords, and procedures}
    \bgroup %\small
    \parindent\z@
    \parskip\z@ plus .1pt\relax\let\item\@idxitem}

\def\@idxitem{\par\hangindent 40pt}

\def\subitem{\par\hangindent 40pt \hspace*{20pt}}

\def\subsubitem{\par\hangindent 40pt \hspace*{30pt}}

\def\endtheindex{%\if@restonecol\onecolumn\else\clearpage\fi
\egroup}

\def\indexspace{\par \vskip 10pt plus 5pt minus 3pt\relax}

\makeatother

%\newcommand{\Xten}{{\sf X10}}
%\newcommand{\XtenCurrVer}{{\sf X10 v1.7}}
%\newcommand{\java}{{\sf Java}}
%\newcommand{\Java}{{\sf Java}}

\newcommand{\Xten}{X10}
\newcommand{\XtenCurrVer}{\Xten{} v1.7}
\newcommand{\Java}{Java}
\newcommand{\java}{\Java{}}

\newcommand{\futureext}[1]{{\em \paragraph{Future Extensions.}#1}}
\newcommand{\tbd}{} % marker for things to be done later.
\newcommand{\limitation}[1]{{\em Limitation: #1}} % marker for things to be done later.


\newcommand\grammarrule[1]{\emph{#1}}

% Rationale

\newenvironment{rationale}{%
\bgroup\noindent{\sc Rationale:}\space}{%
\egroup}

% Notes

\newenvironment{note}{%
\bgroup\noindent{\sc Note:}\space}{%
\egroup}

\newenvironment{staticrule*}{%
\bgroup\noindent{\textsc{Static Semantics Rule:}\space}}{%
\egroup}

\newenvironment{staticrule}[1]{%
\bgroup\noindent{\textsc{Static Semantics Rule} (#1):\space}}{%
\egroup}

\newcommand\Sref[1]{\S\ref{#1}}
\newcommand\figref[1]{Figure~\ref{#1}}
\newcommand\tabref[1]{Table~\ref{#1}}
\newcommand\exref[1]{Example~\ref{#1}}

\newcommand\eat[1]{}


\begin{document}


\lstset{language=java,basicstyle=\ttfamily\small}


\title{Object Initialization in X10}


\authorinfo{Yoav Zibin \and Igor Peshansky \and David Cunningham \and Vijay Saraswat}
           {IBM research in TJ Watson}
           {yzibin$|$igorp$|$dcunnin$|$vsaraswa@us.ibm.com}


\maketitle


\begin{abstract}
Modern object-oriented languages such as \Xten require a rich framework
for types capable of expressing value-dependency, type-dependency
and supporting pluggable, application-specific extensions.

In earlier work, we presented the framework of \emph{constrained
types} for concurrent, object-oriented languages, parametrized by
an underlying constraint system $\cal X$. Constraint systems are a
very expressive framework for partial information. Types are viewed
as formulas \Xcd{C\{c\}} where \Xcd{C} is the name of a class
or an interface and \Xcd{c} is a constraint in $\cal X$ on the
immutable instance state of \Xcd{C} (the \emph{properties}).
Many (value-)dependent type systems for object-oriented languages
can be viewed as constrained types.

This paper extends the constrained types approach to handle
\emph{type-dependency} (``genericity''). The key idea is to extend
the constraint system to include predicates over types such as
\Xcd{X} is a subtype of \Xcd{T}.  Generic types are supported
by introducing type parameters and permitting programs to impose
constraints on such parameters.

To illustrate the underlying theory, we develop a formal family of
programming languages with a common set of sound type-checking rules
parameterized on a constraint system $\cal X$.  By varying $\cal X$
and extending the type system, we obtain languages with the power
of \FJ, \FGJ, and languages that provide dependent types, structural
subtyping, and constraints that relate values and types.  The core
of the \Xten language is a concrete instantiation of the framework.  We
describe the design and implementation of \Xten, which is available
for download at \texttt{x10-lang.org}.


\end{abstract}

\category{D.3.3}{Programming Languages}{Language Constructs and Features}
\category{D.1.5}{Programming Techniques}{Object-oriented Programming}

\terms
Asynchronous, Initialization, Types, X10

% Keywords are not required in the paper itself - only in the submission system's meta data.
% \keywords
% Immutability, Ownership, Java


\Section[introduction]{Introduction}

\section{Introduction}
\label{s:intr}

 Graph theoretic problems arise in several traditional and emerging scientific disciplines such as VLSI design, optimization, databases, and computational biology. There are plenty of theoretically fast parallel algorithms, for example, work-time optimal PRAM algorithms, for graph problems; however, in
 practice few parallel implementations beat the best sequential implementations for arbitrary, sparse
 graphs. The mismatch between theory and practice suggests a large gap between algorithmic model and the actual architecture. We observe that the gap is increasing as new diversified architectures emerge. Elegant solutions seem hard to come by from even combined efforts of algorithmic and architectural improvement. What is lacking is an effient way of mapping fine-grained parallelism expressed by the algorithm to target architectures with good performance. X10 is a new parallel programming language that provides expressive programming constructs and efficient runtime support that effectively helps reduce the gap between theory and practice in solving graph problems. In this paper we show that with X10 the fine-grained parallelism for a graph problem can be expressed much easier at a high algorithmic level, and the X10 program, compared with native C implementation, is much simpler and more elegant, and achieves comparable, and sometimes, even better performance. 

 The challenges of solving large-scale graph problems on current and emerging systems come from the irregular and combinatorial nature of the problem. Many of the important real world graphs, for example, internet topology, social interaction network, transfortation network, protein-protein interaction network, and etc., exhibit a ``small-world'' nature, and can be modeled as the so-called ``scale-free'' graph. There is no known efficient technique to partion such graph, which makes it hard to solve on distributed-memory systems. Also compared with the well-known sequential algorithms, for example, depth-first search (DFS) or breadth-first search (BFS) for the spanning tree problem, the parallel graph algorithms take exotic approaches such as ``graft-and-shortcut''. In the absence of efficient scheduling support of parallel activities, fine-grained parallelism incurs large overhead on current systems and oftentimes do not show practical parallel performance advantage. Lastly, graph algorithms tend to be load/store intensive compared with other scientific problems. For example,  They put great pressure on the memory subsystem. The problem obviously gets worse on distributed-memory architectures if necessary task management and memory affinity scheduling are not provided.  
 
 There are several features of X10 that make it extremely helpful in soving large-scale graph problems. X10 provides a shared virtual address space that obviates the need to partition a graph and issue message passing commands explicitly to access remote data. The irregular nature of the graph is also the reason why no SSCA benchmark has been implemented in MPI. X10 provides a wide range of constructs that are de. X10 has a lot of balancing.

 Our target architecure is a cluster of symmetric multiprocessor(SMP) nodes. Each SMP node may further comprise of chip multiprocessors (CMPs).  SMPs and CMPs are becoming very powerful and common place. Most of the high performance
computers are clusters of SMPs and/or CMPs. It is important to solve for them.
It is important to show flexibility but also good support of  PRAM algorithms for graph problems can be emulated much easier and more. 

The problem we consider if the spanning tree problem. It is notoriously hard to achieve good parallel performance.  Several good ones, we show X10 support that can do better. 

 The rest of the paper is organized as follows. Sections~\ref{s:design} describes algorithm design with the X10 language.
 Section~\ref{s:runtime} presents the workstealing runtime support for load-balancing in X10, and compare with other runtime systems, for example, CILK. 
 Section~\ref{s:results} provides our experimental results on current main-stream SMPs.
 In Section~\ref{s:concl} we conclude and give future work. 
 Throughout the paper, we
 use $n$ and $m$ to denote the number of vertices and the number of
 edges of an input graph $G=(V,E)$, respectively. 
  




\Section[designs]{Object Initialization Designs}


\subsection{Default values design}
every type has a default value.
(when reading a val field, you may get the default value or its final value.)

\subsection{proto design}
A proto modifier for locals
allowed cyclic immutable objects.

\subsection{Hardhat design}
\label{Section:hardhat}

Discuss why we didn't allow aliasing of \this (to avoid an alias analysis).

A method is non-escaping, if it is called from a constructor or from another non-escaping method.
A non-escaping method must be private, final, or in a final class.
Constructors and non-escaping methods cannot leak \this.


- "this" cannot escape, no dynamic dispatching, etc.
- Dataflow for definite-assignment for locals (similarity to Java, with the extension for async-initialization)
- Dataflow for definite-assignment for fields:
  + @NoThisAccess and @NonEscaping
  + The fixed-point algorithm to infer the sets of fields that are written, async-written, and read, on each private method that is transitively called from a ctor or field-init. Maybe also add a nice proof :)


Property design (what would happen if we use val fields instead of properties.)

Fields are partition into two: property fields and normal fields. Property fields are assigned together before any other field.
\begin{lstlisting}
class A(x:Int) {
  val y:Int{self!=x};
  val z = x*x;
}
\end{lstlisting}
Vs.
Fields are ordered, and all ctors need to assign in the same order as the fields declaration.
\begin{lstlisting}
class A {
  val x:Int;
  val y:Int{self!=x};
  val z = x*x;
}
\end{lstlisting}



what can you read before (inclusive), and what after.


Hardhat claims:
* \this is the only accessible raw object (there could be several raw objects in the heap but only \this is accessible). Reason: \this cannot be aliased or leaked.
* only cooked objects cross places. Reason: \this is the only non-cooked object and it cannot have any aliases and it cannot cross an \code{at}.
* there is no dynamic dispatch during construction. Reason: only calling private or final methods.
* all field writes finish by the time the ctor ends. Reason: data flow ensures that any write within an async has an enclosing finish.
* one cannot read an uninitialized field. Reason: reading from \this is ok by data-flow, any other read is from a cooked object.


\Section[implementation]{Hardhat implementation}
implementation design, overheads, some measurements, etc.

outlines our implementation within the X10 compiler using the polyglot framework,
    the compilation time overhead of checking these initialization rules,
    and the annotation overhead in our X10 code base.


Due to page limitation, we mainly focused on the formal effect system for POPL,
but we can easily add an empirical evaluation section that describes some test cases (where minor code refactoring was needed) and shows the annotation burden.
X10 has only two possible method annotations: @NonEscaping, @NoThisAccess.
Methods transitively called from a constructor are implicitly non-escaping (but the compiler issues a warning that they should be marked as @NonEscaping).
SPECjbb and M3R are closed-source whereas the rest is open-source and publicly available at x10-lang.org

------------------------------------------------------------------------------
Programs:           XRX SPECjbb     M3R UTS Other
\# of lines          27153   14603       71682   2765    155345
\# of files          257 63      294 14  2283
\# of constructors       276 267     401 23  1297
\# of methods            2216    2475        2831    124 8273
\# of non-escaping methods   8   38      34  3   83
\# of @NonEscaping       7   7       13  1   62
\# of @NoThisAccess      1   0       1   0   12
------------------------------------------------------------------------------
XRX: X10 Runtime (and libraries)
SPECjbb: SPECjbb from 2005 converted to X10
M3R: Map-reduce in X10
UTS: Global load balancing
Other: Programmer guide examples, test suite, issues, samples
------------------------------------------------------------------------------

As can be seen, the annotations burden is minor.

Asynchronous initialization was not used in our applications because they pre-date this feature.
(It is used in our examples and tests 50 times.)
However, it is a useful pattern, especially for local variables.
More importantly, the analysis prevents bugs such as:
val res:Int;
finish {
  async {
    res = doCalculation();
  }
  // WRONG to use res here
}
// OK to use res here

Here are two examples for the use of annotations:
1) In Any.x10 we have:
@NoThisAccess def typeName():String
Method typeName is overridden in subclasses to return a constant string (all structs automatically override this method).
This annotation allows typeName() to be called even during construction.
2) In HashMap.x10, after we added the strict initializations rules, we had to refactor put and rehash methods into:
public def put(k: K, v: V) = putInternal(k,v);
@NonEscaping protected final def putInternal(k: K, v: V) {...}
(Similarly, we have rehash() and rehashInternal())
The reason is that putInternal is called from the deserialization constructor:
def this(x:SerialData) { ... putInternal(...) ... }
And we still want subclasses to be able to override the "put" method.



Compilation time:   XRX SPECjbb     M3R UTS Other
total           65241   78952       254020  72205   548547
Fields          156 1649        3330    1272    2862
Locals          32  51      117 33  126


Implementation lines of code:
CheckEscapingThis: 951
InitChecker: 805
Polyglot dataflow framework: DataFlow 1309


\Section[case-study]{Case Study}
a lot of Java code was recently translated to X10, and Java is less strict regarding initialization. How did affect the translation?


\Section[related-work]{Related Work}
\label{sec:related}

\subsection{Programming Models} 
Determinism in parallel programming is a very active area of research.
Guava \cite{guava} introduces restrictions on shared memory Java
programs that ensure no data-races primarily by distinguishing
monitors (all access is synchronized) from values (immutable) and
objects (private to a thread). However Guava is not safe since Guava
programs may use \code{wait}/\code{notify} for arbitrary concurrent
signalling and hence may not executable with a sequential
schedule. The Revisions programming model \cite{Revisions} guarantees
determinism by isolating asynchronous tasks but merging their writes
determinately. However, the model explicitly does not require that
a sequential schedule be valid (c.f. Figure~1 in \cite{Revisions}).

%\cite{Steele:1989:MAP:96709.96731} introduced the idea that

DPJ develops the ``determinacy-by-default'' slogan using a static
type-and-effects system to establish commutativity of concurrent
actions.  The deterministic fragment of DPJ is safe according to the
definition above. Safe X10 offers a much richer concurrency model
which guarantees the safety of common idioms such as accumulators and
cyclic tasks (clocks) without relying on effects annotations. The
lightweight effects mechanism in X10 can be extended to support a much
richer effects framework (along the lines of DPJ) using X10's
constrained type system.  We leave this as future work.

The SafeJava language \cite{Rinard04safejava:a} is unfortunately not safe
according to our definition, even though it guarantees determinacy and
deadlock freedom, using ownership types, unique pointers and partially
ordered lock levels. Again, a sequential scheduler is not admissible
for the model.

Some data-flow synchronization based languages and frameworks (e.g.{}
Kahn style process networks \cite{kahn1974semantics}, concurrent
constraint programming \cite{saraswat1993concurrent}, \cite{SHIM}) are guaranteed
determinate but not safe according to our definition since they do not
permit sequential schedules. Indeed they permit the possibility of
deadlock. (The notion of safety is also not quite relevant since these
frameworks do not support shared mutable variables.)

Synchronous programming languages like Esterel are completely
deterministic, but meant for embedded targets. An Esterel program
executes in clock steps and the outputs are conceptually synchronous
with its inputs.  It is a finite state language that is easy to verify
formally. An Esterel program is susceptible to
causalities. Causalities are similar to deadlocks, but can be easily
detected at compile-time.  The problem with synchronous models is that
they do not perform well. To out knowledge, most Esterel compilers
generate sequential code and not concurrent.
 
SHIM~\cite{edwards2005shim2,tardieu2006scheduling-independent} is also
a deterministic concurrent programming language, but SHIM programs may
deadlocks.  Secondly, SHIM allows only a single task to write at any
phase; we allow multiply writes.

Apart from SHIM, there are a few programming models and languages that
provide explicit determinism. StreamIt~\cite{thies2001streamit}, for
example is a synchronous dataflow language that provides
determinism. It has simple static verification techniques for deadlock
and buffer-overflow.  However, StreamIt is a strict subset of SHIM and
StreamIt's design limits it to a small class of streaming
applications.
 


In contrast, 
Cilk~\cite{blumofe1995cilk} is a non-deterministic language that it covers a larger
class of applications. It is C based
and the programmer must explicitly ask for parallelism using 
the \emph{spawn} and the \emph{sync} constructs. 
Cilk permits data races.
%\figref{non-det}, for example,
%is a non-deterministic concurrent program in Cilk. 
Explicit techniques~\cite{cheng1998detecting} are
required for checking data races in Cilk programs.  




\subsection{Determinizing Tools} 
Determinizing run-times support coarse-grained fork-join concurrency
by maintaining a different copy of memory for each activity and
merging them determinately at finish points (\cite{grace},
\cite{dmp}, \cite{kendo},\cite{determinator}). Safe
X10 can run on such systems in principle, but does not require them.
To execute Safe X10, such systems need to support fine-grained
asynchrony (with some form of work-stealing or fork-joining
scheduler), clocks and accumulators.

Kendo is a purely software system that deterministically multi-threads
concurrent applications.  Kendo~\cite{olszewski2009kendo} ensures a
deterministic order of all lock acquisitions for a given program
input.

Kendo comes with three shortcomings. It operates completely at runtime,
and there is considerable performance penalty. Secondly, if
we have the sequence \emph{lock(A); lock (B)} in one thread and
\emph{lock(B); lock(A)} in another thread, a deterministic ordering of
locks may still deadlock. Thirdly, the tool operates only when
shared data is protected by locks.

Software Transactional Memory (STM)~\cite{shavit1995software}
  is an alternative to locks: a thread completes modifications to 
shared memory without regard for what other threads might be doing. At the end of the transaction,
it validates and commits if the validation was successful, otherwise it rolls back and re-executes
the transaction. STM mechanisms avoid races but do not solve the non-determinism problem.

Berger's Grace\cite{berger2009grace} is a run-time tool
that is based on STM. 
If there is a conflict during commit, the threads are committed in
a particular sequential order (determined by the order
The problem with Grace is that it incurs a lot of run-time
overhead. This dissertation  partially solves this overhead problem
by addressing the issue at compile-time and
thereby reducing a considerable amount of run-time overhead.

Like Grace, Determinator\cite{aviram2010efficient} is another tool
that allows parallel processes to execute as long as they do not share 
resources. If they do share resources and the accesses are unsafe, then
the operating throws an exception (a page fault). 

Cored-Det~\cite{bergan2010coreDet}, based on DMP~\cite{devietti2009dmp} 
uses a deterministic token that is passed
among all threads.  A thread to modify a shared variable must first
wait for the token and for all threads to block on that
token. DMP is hardware based. 
Although, deadlocks may be avoided, we believe this setting is
non-distributed because it forces all threads to synchronize and
therefore leads to a considerable performance penalty. In Safe X10,
only threads that share a particular channel must synchronize
on that channel; other threads can run independently.

 Deterministic replay systems~\cite{choi1998deterministic,altekar2009odr} facilitate debugging of concurrent programs to produce
repeatable behavior. They are based on record/replay systems. The system
replays a specific behavior (such as thread interleaving) of a concurrent
program based on records. The primary purpose of replay systems 
is debugging; they do not guarantee determinism. 
They incur a high runtime overhead and are input dependent.
For every new input, a new set of records is generally maintained.

Like replay systems, Burmin and Sen~\cite{Burnim2009asserting} provide a framework for
checking determinism for multi-threaded programs. Their tool does not
introduce deadlocks, but their tool does not guarantee determinism
because it is merely a testing tool that checks the execution trace
with previously executed traces to see if the values match. Our
goal is to guarantee determinism at compile time -- given a program,
it will generate the same output for a given input.



%\subsection{Type Systems and Verifiers} 
%Finally, type and effect systems like DPJ~\cite{bocchino2009type} 
% have been designed for deterministic parallel programming to see if
%memory locations overlap. Our technique is more explicit. 
%In general, type systems require the programmer to manually annotate the program. Our model can also be implemented using annotations in existing
%programming languages - we in fact annotated the X10 programming language.

%Martin Vechev's tool \cite{vechev2011automatic}
%finds determinacy bugs in loops that run parallel bodies. It analyzes
%array references and indices to ensure that there are no read-write and 



\Section[conclusion]{Conclusion}
We show that many determinate concurrent programs can be written in
\Xten{} using determinate, deadlock-free constructs, so that they are
determinate by design.



\bibliographystyle{plainnat}
\bibliography{x10-init}

\end{document}
