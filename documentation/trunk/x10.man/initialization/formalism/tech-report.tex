\documentclass[9pt,a4paper]{article}

%% For typesetting theorems and some math symbols.
\usepackage{amssymb}
\usepackage{amsthm}

\usepackage{fullpage}

\title{Initialization in X10 - Technical Report}

\author{Yoav Zibin \texttt{yoav.zibin@gmail.com}}

\date{}




\usepackage{xspace}

% Macros for R^nRS.

\def\@makechapterhead#1{%
  \vspace*{50\p@}%
  {\parindent \z@ \raggedright \normalfont
    \ifnum \c@secnumdepth >\m@ne
        \huge\bfseries \thechapter \space\space\space
        \nobreak
    \fi
    \interlinepenalty\@M
    \Huge \bfseries #1\par\nobreak
    \vskip 40\p@
  }}


\makeatletter

\newcommand{\topnewpage}{\@topnewpage}

% Chapters, sections, etc.

\newcommand{\vest}{}
\newcommand{\dotsfoo}{$\ldots\,$}

\newcommand{\sharpfoo}[1]{{\tt\##1}}
\newcommand{\schfalse}{\sharpfoo{f}}
\newcommand{\schtrue}{\sharpfoo{t}}

\newcommand{\singlequote}{{\tt'}}  %\char19
\newcommand{\doublequote}{{\tt"}}
\newcommand{\backquote}{{\tt\char18}}
\newcommand{\backwhack}{{\tt\char`\\}}
\newcommand{\atsign}{{\tt\char`\@}}
\newcommand{\sharpsign}{{\tt\#}}
\newcommand{\verticalbar}{{\tt|}}

\newcommand{\coerce}{\discretionary{->}{}{->}}

% Knuth's \in sucks big boulders
\def\elem{\hbox{\raise.13ex\hbox{$\scriptstyle\in$}}}

\newcommand{\meta}[1]{{\noindent\hbox{\rm$\langle$#1$\rangle$}}}
\let\hyper=\meta
\newcommand{\hyperi}[1]{\hyper{#1$_1$}}
\newcommand{\hyperii}[1]{\hyper{#1$_2$}}
\newcommand{\hyperj}[1]{\hyper{#1$_i$}}
\newcommand{\hypern}[1]{\hyper{#1$_n$}}
\newcommand{\var}[1]{\noindent\hbox{\it{}#1\/}}  % Careful, is \/ always the right thing?
\newcommand{\vari}[1]{\var{#1$_1$}}
\newcommand{\varii}[1]{\var{#1$_2$}}
\newcommand{\variii}[1]{\var{#1$_3$}}
\newcommand{\variv}[1]{\var{#1$_4$}}
\newcommand{\varj}[1]{\var{#1$_j$}}
\newcommand{\varn}[1]{\var{#1$_n$}}

\newcommand{\vr}[1]{{\noindent\hbox{$#1$\/}}}  % Careful, is \/ always the right thing?
\newcommand{\vri}[1]{\vr{#1_1}}
\newcommand{\vrii}[1]{\vr{#1_2}}
\newcommand{\vriii}[1]{\vr{#1_3}}
\newcommand{\vriv}[1]{\vr{#1_4}}
\newcommand{\vrv}[1]{\vr{#1_5}}
\newcommand{\vrj}[1]{\vr{#1_j}}
\newcommand{\vrn}[1]{\vr{#1_n}}


\newcommand{\defining}[1]{\mainindex{#1}{\em #1}}
\newcommand{\ide}[1]{{\schindex{#1}\frenchspacing\tt{#1}}}

\newcommand{\lambdaexp}{{\cf lambda} expression}
\newcommand{\Lambdaexp}{{\cf Lambda} expression}
\newcommand{\callcc}{{\tt call-with-current-continuation}}

% \reallyindex{SORTKEY}{HEADCS}{TYPE}
% writes (index-entry "SORTKEY" "HEADCS" TYPE PAGENUMBER)
% which becomes  \item \HEADCS{SORTKEY} mainpagenumber ; auxpagenumber ...

\global\def\reallyindex#1#2#3{%
\write\@indexfile{"#1" "#2" #3 \thepage}}

\newcommand{\mainschindex}[1]{\label{#1}\reallyindex{#1}{tt}{main}}
\newcommand{\mainindex}[1]{\reallyindex{#1@{\rm #1}{main}}}
\newcommand{\schindex}[1]{\reallyindex{#1}{tt}{aux}}
\newcommand{\sharpindex}[1]{\reallyindex{#1}{sharpfoo}{aux}}
%vj%\renewcommand{\index}[1]{\reallyindex{#1}{rm}{aux}}

\newcommand{\domain}[1]{#1}
\newcommand{\nodomain}[1]{}
%\newcommand{\todo}[1]{{\rm$[\![$!!~#1$]\!]$}}
\newcommand{\todo}[1]{}

% \frobq will make quote and backquote look nicer.
\def\frobqcats{%\catcode`\'=13 %\catcode`\{=13{}\catcode`\}=13{}
\catcode`\`=13{}}
{\frobqcats
\gdef\frobqdefs{%\def'{\singlequote}
\def`{\backquote}}}%\def\{{\char`\{}\def\}{\char`\}}
\def\frobq{\frobqcats\frobqdefs}

% \cf = code font
% Unfortunately, \cf \cf won't work at all, so don't even attempt to
% next constructions which use them...
\newcommand{\cf}{\frenchspacing\tt}

% Same as \obeycr, but doesn't do a \@gobblecr.
{\catcode`\^^M=13 \gdef\myobeycr{\catcode`\^^M=13 \def^^M{\\}}%
\gdef\restorecr{\catcode`\^^M=5 }}

{\catcode`\^^I=13 \gdef\obeytabs{\catcode`\^^I=13 \def^^I{\hbox{\hskip 4em}}}}

{\obeyspaces\gdef {\hbox{\hskip0.5em}}}

\gdef\gobblecr{\@gobblecr}

\def\setupcode{\@makeother\^}

% Scheme example environment
% At 11 points, one column, these are about 56 characters wide.
% That's 32 characters to the left of the => and about 20 to the right.

\newenvironment{x10noindent}{
  % Commands for scheme examples
  \newcommand{\ev}{\>\>\evalsto}
  \newcommand{\lev}{\\\>\evalsto}
  \newcommand{\unspecified}{{\em{}unspecified}}
  \newcommand{\scherror}{{\em{}error}}
  \setupcode
  \small \cf \obeytabs \obeyspaces \myobeycr
  \begin{tabbing}%
\qquad\=\hspace*{5em}\=\hspace*{9em}\=\kill%   was 16em
\gobblecr}{\unskip\end{tabbing}}

%\newenvironment{scheme}{\begin{schemenoindent}\+\kill}{\end{schemenoindent}}
\newenvironment{x10}{
  % Commands for scheme examples
  \newcommand{\ev}{\>\>\evalsto}
  \newcommand{\lev}{\\\>\evalsto}
  \renewcommand{\em}{\rmfamily\itshape}
  \newcommand{\unspecified}{{\em{}unspecified}}
  \newcommand{\scherror}{{\em{}error}}
  \setupcode
  \small \cf \obeyspaces \myobeycr
  \footnotesize
  \begin{tabbing}%
\qquad\=\hspace*{5em}\=\hspace*{9em}\=\+\kill%   was 16em
\gobblecr}{\unskip\end{tabbing}\normalsize}

\newcommand{\evalsto}{$\Longrightarrow$}

% Manual entries

\newenvironment{entry}[1]{
  \vspace{3.1ex plus .5ex minus .3ex}\noindent#1%
\unpenalty\nopagebreak}{\vspace{0ex plus 1ex minus 1ex}}

\newcommand{\exprtype}{syntax}

% Primitive prototype
\newcommand{\pproto}[2]{\unskip%
\hbox{\cf\spaceskip=0.5em#1}\hfill\penalty 0%
\hbox{ }\nobreak\hfill\hbox{\rm #2}\break}

% Parenthesized prototype
\newcommand{\proto}[3]{\pproto{(\mainschindex{#1}\hbox{#1}{\it#2\/})}{#3}}

% Variable prototype
\newcommand{\vproto}[2]{\mainschindex{#1}\pproto{#1}{#2}}

% Extending an existing definition (\proto without the index entry)
\newcommand{\rproto}[3]{\pproto{(\hbox{#1}{\it#2\/})}{#3}}

% Grammar environment

\newenvironment{grammar}{
  \def\:{& \goesto{} &}
  \def\|{& $\vert$& }
  \def\opt{$^?$\ }
  \def\star{$^*$\ }
  \def\plus{$^+$\ }
  \em
  \begin{tabular}{rcl}
  }{\unskip\end{tabular}}

%\newcommand{\unsection}{\unskip}
\newcommand{\unsection}{{\vskip -2ex}}

% Commands for grammars
\newcommand{\arbno}[1]{#1\hbox{\rm*}}  
\newcommand{\atleastone}[1]{#1\hbox{$^+$}}

\newcommand{\goesto}{{\normalfont{::=}}}

% mark modifications (for the grammar) From Igor Pechtchanski/Watson/IBM@IBMUS
\newlength{\modwidth}\setlength{\modwidth}{0.005in}
\newlength{\modskip}\setlength{\modskip}{.4em}
\newlength{\@modheight}
\newlength{\@modpos}
\providecommand{\markmod}[1]{%
  \setlength{\@modheight}{#1}%
  \addtolength{\@modheight}{-0.06in}%
  \setlength{\@modpos}{\linewidth}%
  \addtolength{\@modpos}{0.285in}%         Magic
  \addtolength{\@modpos}{\modwidth}%
  \addtolength{\@modpos}{\modskip}%
  \marginpar{\vspace{-\@modheight}%
             \hspace{-\@modpos}%
             \rule{\modwidth}{#1}}%
}

% The index

\def\theindex{%\@restonecoltrue\if@twocolumn\@restonecolfalse\fi
%\columnseprule \z@
%!! \columnsep 35pt
\clearpage
\@topnewpage[
    \centerline{\large\bf\uppercase{Alphabetic index of definitions of concepts,}}
    \centerline{\large\bf\uppercase{keywords, and procedures}}
    \vskip 1ex \bigskip]
    \markboth{Index}{Index}
    \addcontentsline{toc}{chapter}{Alphabetic index of 
 definitions of concepts, keywords, and procedures}
    \bgroup %\small
    \parindent\z@
    \parskip\z@ plus .1pt\relax\let\item\@idxitem}

\def\@idxitem{\par\hangindent 40pt}

\def\subitem{\par\hangindent 40pt \hspace*{20pt}}

\def\subsubitem{\par\hangindent 40pt \hspace*{30pt}}

\def\endtheindex{%\if@restonecol\onecolumn\else\clearpage\fi
\egroup}

\def\indexspace{\par \vskip 10pt plus 5pt minus 3pt\relax}

\makeatother

%\newcommand{\Xten}{{\sf X10}}
%\newcommand{\XtenCurrVer}{{\sf X10 v1.7}}
%\newcommand{\java}{{\sf Java}}
%\newcommand{\Java}{{\sf Java}}

\newcommand{\Xten}{X10}
\newcommand{\XtenCurrVer}{\Xten{} v1.7}
\newcommand{\Java}{Java}
\newcommand{\java}{\Java{}}

\newcommand{\futureext}[1]{{\em \paragraph{Future Extensions.}#1}}
\newcommand{\tbd}{} % marker for things to be done later.
\newcommand{\limitation}[1]{{\em Limitation: #1}} % marker for things to be done later.


\newcommand\grammarrule[1]{\emph{#1}}

% Rationale

\newenvironment{rationale}{%
\bgroup\noindent{\sc Rationale:}\space}{%
\egroup}

% Notes

\newenvironment{note}{%
\bgroup\noindent{\sc Note:}\space}{%
\egroup}

\newenvironment{staticrule*}{%
\bgroup\noindent{\textsc{Static Semantics Rule:}\space}}{%
\egroup}

\newenvironment{staticrule}[1]{%
\bgroup\noindent{\textsc{Static Semantics Rule} (#1):\space}}{%
\egroup}

\newcommand\Sref[1]{\S\ref{#1}}
\newcommand\figref[1]{Figure~\ref{#1}}
\newcommand\tabref[1]{Table~\ref{#1}}
\newcommand\exref[1]{Example~\ref{#1}}

\newcommand\eat[1]{}

\mynewcommand{\async}{\code{async}}
\mynewcommand{\finish}{\code{finish}}
\mynewcommand{\R}{\code{R}}
\mynewcommand{\SW}{\code{SW}}
\mynewcommand{\AW}{\code{AW}}
\mynewcommand{\Read}{\code{Read}}
\mynewcommand{\SyncWrite}{\code{SyncWrite}}
\mynewcommand{\AsyncWrite}{\code{AsyncWrite}}

\mynewcommand{\hasync}{\code{async}}
\mynewcommand{\hfinish}{\code{finish}}
\mynewcommand{\hR}{\code{R}}
\mynewcommand{\hSW}{\code{SW}}
\mynewcommand{\hAW}{\code{AW}}
\mynewcommand{\hObject}{\code{Object}}

\mynewcommand{\hFM}{\code{FM}}
\mynewcommand{\hMM}{\code{MM}}
\mynewcommand{\hP}{\code{P}}

\mynewcommand{\hvar}{\code{var}}
\mynewcommand{\hval}{\code{val}}
\mynewcommand{\hescaping}{\code{escaping}}
\mynewcommand{\hextends}{\code{extends}}



\begin{document}


\maketitle


\lstset{language=java,basicstyle=\ttfamily\small}

\section{Introduction}
This technical report formalizes the hardhat initialization rules in X10
    using \emph{Featherweight X10} (FX10).
Read first the paper ``Object Initialization in X10" to understand
    the terminology (raw and cooked objects),
    the initialization rules of X10,
    and their connection with X10's concurrent and distributed constructs (\finish, \async, and \code{at}).

FX10 is similar to Featherweight Java (FJ), however it is both imperative (using a heap) and it
    models X10 specific constructs such as \finish and \async.
FX10 also models field initialization including the fact that one can read a field only after it was definitely assigned,
    and that final (\code{val}) fields can be assigned exactly once.

FX10 does \emph{not} model other aspects of X10 such as:
\begin{description}
  \item[Places] X10 can run code in multiple places, and the \code{at} keyword is used to execute code in a different place.
    Because the only raw object is \this,
        and a raw \this cannot be captured by an \code{at},
        then only cooked objects cross places.
    Therefore, there is no initialization issues with \code{at} and FX10 does not model it.
  \item[Inference]
    X10 uses inference in various places:
        (i)~it infers the type of a final field with an initializer,
        (ii)~it infers method return types,
        (iii)~using an inter-procedural dataflow analysis it infers
            3 sets for each non-escaping method: \R, \SW, \AW.
            \R is the set of fields that can be read by the method,
                \SW are the fields that must be definitely-assigned by the method,
                and \AW are the fields that must be definitely-asynchronously-assigned by the method.

    FX10 does not model inference, and instead this information is explicitly presented.
  \item[null]
    FX10 guarantees that all fields are assigned when the object becomes cooked,
        and that fields are read only after written to.
    Therefore, \hnull is no longer needed.

  \item[Misc]
    Generics, constraints, casting, inner classes, overloading, co-variant return type, private/final, etc.
\end{description}

We will next describe the syntax.
Judgements:

Typing: Expression~\he has type \hC in environment $\Gamma$:
$\Gdash \he:\hC$.
Helper methods, like \R, \AW, and \SW.
Reduction:
$\he,H \reduce \he',H'$.

\section{Syntax}


\begin{figure}[htpb!]
\begin{center}
\begin{tabular}{|l|l|}
\hline

$\hP ::= \ol{\hL}$ & Program. \\

$\hL ::= \hclass ~ \hC~\hextends~\hD~\lb~\ol{\hFM} ~ \ol{\hf}:\ol{\hC};~~\ctor(\ol{\hx}:\ol{\hC}) \lb~\super(\ol{\he});\he; \rb ~~\ol{\hM}~\rb$
& cLass declaration. \\

$\hM ::= \hMM ~ \hm(\ol{\hx}:\ol{\hC}):\hC=\he;$
& Method declaration. \\

$\hFM ::= \hval~|~\hvar$
& Field Modifier. \\

$\hMM ::= \hescaping ~~|~~ \Read(\ol{\hf})~~\SyncWrite(\ol{\hf})~~\AsyncWrite(\ol{\hf})$
& Method Modifier. \\

$\he ::= \hl ~|~ \hx ~|~ \he.\hf ~|~ \he.\hf = \he ~|~ \he.\hm(\ol{\he}) ~|~ \hnew ~ \hC(\ol{\he})~|~ \hfinish~\he~|~ \hasync~\he;\he$
& Expressions. \\ %: locations, parameters, field access\&assignment, invocation, \code{new}

\hline
\end{tabular}
\end{center}
\caption{FX10 Syntax.
    The terminals are locations (\hl), parameters and \this (\hx), field name (\hf), method name (\hm), class name (\hB,\hC,\hD,\hObject),
        and keywords (\hescaping, \R, \SW, \AW, \hnew, \finish, \async, \hval, \hvar, \ctor).
    The program source code cannot contain locations (\hl), because locations are only created during execution/reduction in \RULE{R-New} of \Ref{Figure}{reduction}.
    }
\label{Figure:syntax}
\end{figure}


\section{Typing}
{Subtyping} is exactly as in FJ:
    the transitive closure of the \hextends relation.
That is, $\hC \st \hD$ iff \hC transitively extends $\hD$ (including the case that $\hC=\hD$.

Similarly, we define:
(i)~$\fields(\hC)=\ol{\hD} \ol{\hf}$ returns all fields of \hC (both those declared by \hC and recursively inherited from its superclass),
(ii)~$\mtype(\hm,\hC) = \ol{\hB}\mapsto\hD$ returns the type of method \hm in class \hC,
(iii)~$\mbody(\hm,\hC) = \ol{\hx}.\he$ returns the method body of \hm in class \hC.
Because methods have a modifier (\hMM) in FX10, we also define:
(iv)~$\mmodifier(\hm,\hC) = \hMM$ returns the method modifier of \hm in class \hC.
Fields also have a modifier (\hFM) so we define:
(v)~$\fmodifier(\hf,\hC) = \hFM$ returns the field modifier of \hf in class \hC (either \hval or \hvar).

Finally, because FX10 have constructors, we define:
(vi)~$\ctortype{}(\hC) = \ol{\hB}\mapsto\hC$ returns the type of the constructor in \hC,
(vii)~$\ctorbody{}(\hC) = \ol{\hx}.\hsuper(\ol{\he});\he$ returns the body of the constructor.
(Note that constructors do not have a modifier, because they implicitly must write to all fields and cannot read any.)



\begin{figure*}[t]
\begin{center}
\begin{tabular}{|c|}
\hline
$\typerule{
    \Gdash \he : \hC
}{
  \Gdash \finish~\he~:~\hC
}$~\RULE{(T-finish)}
\quad
$\typerule{
    \Gdash \he : \hC
        \gap
    \Gdash \he' : \hC'
}{
  \Gdash \async~\he;~\he'~:~\hC'
}$~\RULE{(T-async)}\\\\


$\typerule{
}{
  \Gdash \hl : \Gamma(\hl)
}$~\RULE{(T-location)}
\quad
$\typerule{
}{
  \Gdash \hx : \Gamma(\hx)
}$~\RULE{(T-parameter)}\\\\


$\typerule{
  \Gdash \he_0:\hC_0
    \gap
  \fields{}(\hC_0)=\ol{\hf}:\ol{\hC}
}{
  \Gdash \he_0.\hf_i : \hC_i
}$~\RULE{(T-Field-Access)}\\\\

$\typerule{
  \Gdash \he_0:\hC_0
    \gap
  \fields{}(\hC_0)=\ol{\hf}:\ol{\hC}
    \gap
  \Gdash \he':\hC'
    \gap
  \Gdash \hC' \st \hC_i
}{
  \Gdash \he_0.\hf_i = \he' : \hC'
}$~\RULE{(T-Field-Assign)}\\\\

$\typerule{
  \Gdash \he_0:\hC_0
    \gap
  \mtype{}(\hm,\hC_0)=\ol{\hD}\mapsto\hC
    \gap
  \Gdash \ol{\he}:\ol{\hC}
    \gap
  \Gdash \ol{\hC} \st \ol{\hD}
}{
  \Gdash \he_0.\hm(\ol{\he}) : \hC
}$~\RULE{(T-Invoke)}\\\\

$\typerule{
  \ctortype{}(\hC)=\ol{\hD}\mapsto\hC
    \gap
  \Gdash \ol{\he}:\ol{\hC}
    \gap
  \Gdash \ol{\hC} \st \ol{\hD}
}{
  \Gdash \hnew ~ \hC(\ol{\he}) : \hC
}$~\RULE{(T-New)}\\\\


\hline
\end{tabular}
\end{center}
\caption{FX10 Expression Typing Rules ($\Gdash \he:\hC$).
    Rules \RULE{(T-finish)} and \RULE{(T-async)} handle the new constructs in FX10,
        while the other rules are identical to those in FJ.}
\label{Figure:expressions}
\end{figure*}


A method cannot assign to any \hval fields (no matter what the target is):
\beqs{methodVal}
    &\methodVal([\he_0,\ldots,\he_n])= \methodVal(\he_0) \code{~and~} \ldots \code{~and~}  \methodVal(\he_n)\\
    &\methodVal(\he")=
        \begin{cases}
        \htrue & \he" \equiv \hl \\
        \htrue & \he" \equiv \hx \\
        \methodVal(\he) & \he" \equiv \he.\hf \\
        \methodVal([\he,\he']) \code{~and~} \hFM=\hvar & \he" \equiv \he.\hf = \he' \gap \Gdash \he:\hC \gap \fmodifier{}(\hf,\hC)=\hFM \\
        \methodVal([\he',\ol{\he}]) & \he" \equiv \he'.\hm(\ol{\he}) \\
        \methodVal([\ol{\he}]) & \he" \equiv \hnew ~ \hC(\ol{\he}) \\
        \methodVal(\he) & \he" \equiv \hfinish~\he \\
        \methodVal([\he,\he']) & \he" \equiv \hasync~\he;\he' \\
        \end{cases}
%$\hl ~|~ \hx ~|~ \he.\hf ~|~ \he.\hf = \he ~|~ \he.\hm(\ol{\he}) ~|~ \hnew ~ \hC(\ol{\he})~|~ \hfinish~\he~|~ \hasync~\he;\he$
\eeq

A constructor can assign to a \hval field at most once and only if the target is \this:
\beqs{ctorVal}
    &\ctorVal([\he_0,\ldots,\he_n],F)= \ctorVal(\he_0,F) \code{~and~} \ctorVal(\he_1,\AW([\he_0]) \cup F) \code{~and~} \ldots \code{~and~}  \ctorVal(\he_n,\AW([\he_0,\ldots,\he_{n-1}]) \cup F)\\
    &\ctorVal(\he",F)=
        \begin{cases}
        \htrue & \he" \equiv \hl \\
        \htrue & \he" \equiv \hx \\
        \ctorVal(\he,F) & \he" \equiv \he.\hf \\
        \ctorVal([\he,\he'],F \cup \{ \hf \}) \code{~and~} \\(\hFM=\hvar \code{~or~} (\he=\this \code{~and~} \hf \not \in F)) & \he" \equiv \he.\hf = \he' \gap \Gdash \he:\hC \gap \fmodifier{}(\hf,\hC)=\hFM \\
        \ctorVal([\he',\ol{\he}],F) & \he" \equiv \he'.\hm(\ol{\he}) \\
        \ctorVal([\ol{\he}],F) & \he" \equiv \hnew ~ \hC(\ol{\he}) \\
        \ctorVal(\he,F) & \he" \equiv \hfinish~\he \\
        \ctorVal([\he,\he'],F) & \he" \equiv \hasync~\he;\he' \\
        \end{cases}
%$\hl ~|~ \hx ~|~ \he.\hf ~|~ \he.\hf = \he ~|~ \he.\hm(\ol{\he}) ~|~ \hnew ~ \hC(\ol{\he})~|~ \hfinish~\he~|~ \hasync~\he;\he$
\eeq

Like in FJ, we check that method declarations are ok by ensuring that the type of the method body is a subtype
    of the return type, and that an overriding method has the same signature.
Unlike FJ, we also check that the method modifier (whether it is \hescaping or if it has the 3 sets specified).
Finally, we need to check that we do not assign to a \hval field ($\methodVal(\he)$).
Note that~\eq{method-ok} requires that~$\Read(\hm,\hC) \subseteq \Read(\hm,\hC')$
    so if \hm is escaping, then $\hm'$ must be escaping.

\beqst %{method-ok}
\typerule{
  \Gamma = \{ \ol{\hx}:\ol{\hD},\this:\hC \} \\
  \hclass~\hC~\hextends~\hC'~\lb~\ldots~\rb\\
  \Gdash \he : \hD' \gap \hD' \st \hD \\
  \methodVal(\he) \\
  \code{if }\hMM=\Read(R)~\SyncWrite(W_s)~\AsyncWrite(W_a)\\
    ~~\code{then } \R(\he,R) \code{~and~} W_s \subseteq W_a  \code{~and~} \SW(\he) \supseteq W_s \code{~and~} \AW(\he) \supseteq W_a\\
  \code{if }\mtype{}(\hm,\hC')=\ol{\hD'} \mapsto \hD'\\
    ~~\code{then } \ol{\hD'}\equiv\ol{\hD} \code{~and~} \hD'\equiv\hD \code{~and~} \\
  \code{if }\mmodifier{}(\hm,\hC')=\hMM'\\
    ~~\code{then } \hMM=\hescaping \Rightarrow \hMM'=\hescaping \code{~and~} \\
    \gap \hMM=\Read(R)~\SyncWrite(W_s)~\AsyncWrite(W_a) \Rightarrow\\
        (\hMM'=\hescaping  \code{~or~}
        (\hMM'=\Read(R')~\SyncWrite(W'_s)~\AsyncWrite(W'_a) \code{~and~}
            (R \subseteq R' \code{~and~} W_s \supseteq W'_s \code{~and~} W_a \supseteq W'_a))
        \\
}{
  \hMM ~ \hm(\ol{\hx}:\ol{\hD}):\hD=\he ~~\OK~\IN~\hC
}~\RULE{(T-Method)}
\eeq

Similarly, we need to check that constructor declarations are ok by checking the \hsuper call,
    that all fields are definitely-assigned and non are read before written,
    and that \hval fields are treated correctly (assigned at most once).
Obviously, the constructor of \hObject has no parameters.

\beqst %{ctor-ok}
\typerule{
  \Gamma = \{ \ol{\hx}:\ol{\hD},\this:\hC \} \\
  \hclass~\hC~\hextends~\hC'~\lb~\ldots~\rb\\
  F_s = \fields{}(\hC') \gap F_a = \fields{}(\hC') \gap F_d = F_a \setminus F_s \\
  \Gdash \he' : \hB' \\
  \Gdash \ol{\he} : \ol{\hB} \gap
    \ctortype{}(\hC')=\ol{\hB"} \mapsto \hC'
    \gap \Gdash \ol{\hB} \st \ol{\hB"} \\
  \ctorVal([\ol{\he},\he'],F_s) \\
  \R([\ol{\he}],\emptyset) \code{~and~}
  \R(\he',F_s) \code{~and~}
    \SW(\he') \supseteq F_d \\
}{
  \ctor(\ol{\hx}:\ol{\hD}) \lb~\super(\ol{\he});\he'; \rb ~~\OK~\IN~\hC
}\gap \RULE{(T-Ctor)}
\eeq


As in FJ, a class is ok if its constructor \RULE{(T-Ctor)} and all its methods \RULE{(T-Method)}are ok.


\section{Reduction}
Heap is well-typed.
Progress and Preservation

If a program is well-typed, then undefined values never appear and the expression is never stuck.


H, \he closed
valCheck(\he,H)  (we have everything in H - which val fields are already assigned)



\begin{Theorem}[preservation]
  \textbf{(Progress and Preservation)}
    For every closed expression~$\he \neq \hl$, $K$, and $H$,
        if $K,\Gamma_{H} \vdash \he : \hC$ and $H$ is well-typed for~$K \cup K(\he)$,
        then there exists~$H',\he',\hC'$ such that
        (i)~$K \vdash H,\he \reducesto H',\he'$,
        (ii)~$K,\Gamma_{H'} \vdash \he':\hC'$,
        and~$K,\Gamma_{H'} \vdash \hC' \st \hC$,
        (iii)~$H'$ is well-typed for~$K \cup K(\he')$,
        (iv)~\hC, \hC', and $\he'$ are closed.
\end{Theorem}
\begin{proof}
\ldots
\end{proof}


\end{document}
