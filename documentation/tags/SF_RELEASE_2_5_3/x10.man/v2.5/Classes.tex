

\chapter{Classes}
\label{XtenClasses}\index{class}
\label{ReferenceClasses}





\section{Principles of X10 Objects}\label{XtenObjects}\index{object}
\index{class}

\subsection{Basic Design}

Objects are instances of classes: the most common and most powerful sort of
value in X10.  The other kinds of values, structs and functions, are more
specialized.%, better in some circumstances but not in all.

Classes are structured in a forest of single-inheritance code
hierarchies. Like C++, but unlike Java, there is no single root
class (e.g.{} \Xcd{java.lang.Object}) that all classes inherit from.  Classes
may have any or all of these features: 
\begin{itemize}
\item Implementing any number of interfaces;
\item Static and instance \xcd`val` fields; 
\item Instance \xcd`var` fields; 
\item Static and instance methods;
\item Constructors;
\item Properties;
\item Static and instance nested containers.
\item Static type definitions
\end{itemize}


\Xten{} objects (unlike Java objects) do not have locks associated with them.
Programmers may use atomic blocks (\Sref{AtomicBlocks}) for mutual
exclusion and clocks (\Sref{XtenClocks}) for sequencing multiple parallel
operations.

An object exists in a single location: the place that it was created.  One
place cannot use or even directly refer to an object in a different place.   A
special type, \Xcd{GlobalRef[T]}, allows explicit cross-place references. 

The basic operations on objects are:
\begin{itemize}

\item Construction (\Sref{ObjectInitialization}).  Objects are created, 
      their \xcd`var` and \xcd`val` fields initialized, and other invariants
      established.

\item Field access (\Sref{FieldAccess}). 
The static, instance, and property fields of an object can be retrieved; \xcd`var` fields
can be set.  

\item Method invocation (\Sref{MethodInvocation}).  
Static, instance, and property methods of an object can be invoked.

\item Casting (\Sref{ClassCast}) and instance testing with \xcd`instanceof`
(\Sref{instanceOf}) Objects can be cast or type-tested.  

\item The equality operators \xcd"==" and \xcd"!=".  
Objects can be compared for equality with the \Xcd{==} operation.  This checks
object {\em identity}: two objects are \Xcd{==} iff they are the same object.

\end{itemize}

  

\subsection{Class Declaration Syntax}
\label{sect:ClassDeclSyntax}

The {\em class declaration} has a list of type parameters, a list of
properties, a constraint (the {\em class invariant}), zero or one superclass,
zero or more interfaces that it implements, and a class body containing the
the definition of fields, properties, methods, and member types. Each such
declaration introduces a class type (\Sref{ReferenceTypes}).

%##(ClassDecln TypeParamsI TypeParamIList Properties PropertyList Property Guard Super Interfaces InterfaceTypeList ClassBody ClassMemberDeclns ClassMemberDecln
\begin{bbgrammar}
%(FROM #(prod:ClassDecln)#)
          ClassDecln \: Mods\opt \xcd"class" Id TypeParamsI\opt Properties\opt Guard\opt Super\opt Interfaces\opt ClassBody & (\ref{prod:ClassDecln}) \\
%(FROM #(prod:TypeParamsI)#)
         TypeParamsI \: \xcd"[" TypeParamIList \xcd"]" & (\ref{prod:TypeParamsI}) \\
%(FROM #(prod:TypeParamIList)#)
      TypeParamIList \: TypeParam & (\ref{prod:TypeParamIList}) \\
                     \| TypeParamIList \xcd"," TypeParam \\
                     \| TypeParamIList \xcd"," \\
%(FROM #(prod:Properties)#)
          Properties \: \xcd"(" PropList \xcd")" & (\ref{prod:Properties}) \\
%(FROM #(prod:PropList)#)
            PropList \: Prop & (\ref{prod:PropList}) \\
                     \| PropList \xcd"," Prop \\
%(FROM #(prod:Prop)#)
                Prop \: Annotations\opt Id ResultType & (\ref{prod:Prop}) \\
%(FROM #(prod:Guard)#)
               Guard \: DepParams & (\ref{prod:Guard}) \\
%(FROM #(prod:Super)#)
               Super \: \xcd"extends" ClassType & (\ref{prod:Super}) \\
%(FROM #(prod:Interfaces)#)
          Interfaces \: \xcd"implements" InterfaceTypeList & (\ref{prod:Interfaces}) \\
%(FROM #(prod:InterfaceTypeList)#)
   InterfaceTypeList \: Type & (\ref{prod:InterfaceTypeList}) \\
                     \| InterfaceTypeList \xcd"," Type \\
%(FROM #(prod:ClassBody)#)
           ClassBody \: \xcd"{" ClassMemberDeclns\opt \xcd"}" & (\ref{prod:ClassBody}) \\
%(FROM #(prod:ClassMemberDeclns)#)
   ClassMemberDeclns \: ClassMemberDecln & (\ref{prod:ClassMemberDeclns}) \\
                     \| ClassMemberDeclns ClassMemberDecln \\
%(FROM #(prod:ClassMemberDecln)#)
    ClassMemberDecln \: InterfaceMemberDecln & (\ref{prod:ClassMemberDecln}) \\
                     \| CtorDecln \\
\end{bbgrammar}
%##)




\section{Fields}
\label{FieldDefinitions}
\index{object!field}
\index{field}

Objects may have {\em instance fields}, or simply {\em fields} (called
``instance variables'' in C++ and Smalltalk, and ``slots'' in CLOS): places to
store data that is pertinent to the object.  Fields, like variables, may be
mutable (\xcd`var`) or immutable (\xcd`val`).  

A class may have {\em static fields}, which store data pertinent to the
entire class of objects.  See \Sref{StaticInitialization} for more
information. 
Because of its emphasis on safe concurrency, \Xten{} requires static
fields to be immutable (\xcd`val`). 

No two fields of the same class may have the same name.  A field may have the
same name as a method, although for fields of functional type there is a
subtlety (\Sref{sect:disambiguations}).  

\subsection{Field Initialization}
\index{field!initialization}
\index{initialization!of field}

Fields may be given values via {\em field initialization expressions}:
\xcd`val f1 = E;` and \xcd`var f2 : Long = F;`. Other fields of \xcd`this` may
be referenced, but only those that {\em precede} the field being initialized.


\begin{ex}The following is correct, but would not be if the fields were
reversed:

%~~gen ^^^ Classes10
%package Classes_field_init_expr_a;
%~~vis
\begin{xten}
class Fld{
  val a = 1;
  val b = 2+a;
}
\end{xten}
%~~siv
% class Hook{ def run() {
%   val f = new Fld();
%   assert f.a == 1 && f.b == 3;
%   return true;}}
%~~neg
\end{ex}

\subsection{Field hiding}
\label{sect:FieldHiding}
\index{field!hiding}


A subclass that defines a field \xcd"f" hides any field \xcd"f"
declared in a superclass, regardless of their types.  The
superclass field \xcd"f" may be accessed within the body of
the subclass via the reference \xcd"super.f".

With inner classes, it is occasionally necessary to 
write \xcd`Cls.super.f` to get at a hidden field \xcd`f` of an outer class
\xcd`Cls`. 

\begin{ex}
The \xcd`f` field in \xcd`Sub` hides the \xcd`f` field in \xcd`Super`
The \xcd`superf` method provides access to the \xcd`f` field in \xcd`Super`.
%~~gen ^^^ Classes20
% package classes.fields.primus;
%~~vis
\begin{xten}
class Super{ 
  public val f = 1; 
}
class Sub extends Super {
  val f = true;
  def superf() : Long = super.f; // 1
}
\end{xten}
%~~siv
% class Hook { def run() { 
%   val sub = new Sub();
%   assert sub.f == true;
%   assert sub.superf() == 1;
%   return true;} }
%~~neg
\end{ex}

\begin{ex}
Hidden fields of outer classes can be accessed by suitable forms: 
%~~gen ^^^ Classes30
% package classes.fields.secundus; 
% // NOTEST
%~~vis
\begin{xten}
class A {
   val f = 3;
}
class B extends A {
   val f = 4;
   class C extends B {
      // C is both a subclass and inner class of B
      val f = 5;
       def example() {
         assert f         == 5 : "field of C";
         assert super.f   == 4 : "field of superclass";
         assert B.this.f  == 4 : "field of outer instance";
         assert B.super.f == 3 : "super.f of outer instance";
       }
    }
}
\end{xten}
%~~siv
% class Hook { def run() { ((new B()).new C()).example(); return true; } }
%~~neg
\end{ex}

\subsection{Field qualifiers}
\label{FieldQualifier}
\index{qualifier!field}
\index{field!qualifier}

The behavior of a field may be changed by a field qualifier, such as
\xcd`static` or \xcd`transient`.  


\subsubsection{\Xcd{static} qualifier}
\index{field!static}

A \xcd`val` field may be declared to be {\em static}, as described in
\Sref{FieldDefinitions}. 

\subsubsection{\Xcd{transient} Qualifier}
\label{TransientFields}
\index{transient}
\index{field!transient}

A field may be declared to be {\em transient}.  Transient fields are excluded
from the deep copying that happens when information is sent from place to
place in an \Xcd{at} statement.    The value of a transient field of a copied
object is the default value of its type, regardless of the value of the field
in the original.  If the type of a field has no
default value, it cannot be marked \Xcd{transient}.

%%AT-COPY%% %~~gen ^^^ Classes40
%%AT-COPY%% % package Classes.Transient.Example;
%%AT-COPY%% % KNOWNFAIL-at
%%AT-COPY%% %~~vis
%%AT-COPY%% \begin{xten}
%%AT-COPY%% class Trans { 
%%AT-COPY%%    val copied = "copied";
%%AT-COPY%%    transient var transy : String = "a very long string";
%%AT-COPY%%    def example() {
%%AT-COPY%%       at (here; this) { // causes copying of 'this'
%%AT-COPY%%          assert(this.copied.equals("copied"));
%%AT-COPY%%          assert(this.transy == null);
%%AT-COPY%%       }
%%AT-COPY%%    }
%%AT-COPY%% }
%%AT-COPY%% \end{xten}
%%AT-COPY%% %~~siv
%%AT-COPY%% % class Hook{ def run() {(new Example()).example(); return true;}}
%%AT-COPY%% %~~neg
%%AT-COPY%% 

%~~gen ^^^ Classes40
% package Classes.Transient.Example;
% 
%~~vis
\begin{xten}
class Trans { 
   val copied = "copied";
   transient var transy : String = "a very long string";
   def example() {
      at (here) { // causes copying of 'this'
         assert(this.copied.equals("copied"));
         assert(this.transy == null);
      }
   }
}
\end{xten}
%~~siv
% class Hook{ def run() {(new Trans()).example(); return true;}}
%~~neg


\section{Properties}
\label{PropertiesInClasses}
\index{property}

The properties of an object (or struct) are a restricted form of public
\xcd`val` fields.\footnote{In many cases, a 
\xcd`val` field can be upgraded to a \xcd`property`, which 
entails no compile-time or runtime cost.  Some cannot be, \eg, in cases where
cyclic structures of \xcd`val` fields are required.} 
For example,  every array has a \xcd`rank` telling
how many subscripts it takes.  User-defined classes can have whatever
properties are desired. 

Properties differ from public \xcd`val` fields in a few ways: 
\begin{enumerate}
\item Property references are allowed on \xcd`self` in constraints:
      \xcd`self.prop`.  Field references are not.
\item Properties are in scope for all instance initialization expressions.
      \xcd`val` fields are not.
\item The graph of values reachable from a given object by following only
      property links is acyclic.  Conversely, it is possible (and routine) for
      two objects to point to each other with \xcd`val` fields.
\item Properties are declared in the class header; \xcd`val` fields are
      defined in the class body.
\item Properties are set in constructors by a \xcd`property` statement.
      \xcd`val` fields are set by assignment.
\end{enumerate}



Properties are defined in parentheses, after the name of the class.  They are
given values by the \xcd`property` command in constructors.

\begin{ex}
\xcd`Proper` has a single property, \xcd`t`.  \xcd`new Proper(4)` creates a
\xcd`Proper` object with \xcd`t==4`. 
%~~gen ^^^ Classes50
% package Classes.Toss.Freedom.Disk2;
%~~vis
\begin{xten}
class Proper(t:Long) {
  def this(t:Long) {property(t);}
}
\end{xten}
%~~siv
% class Hook{ def run() {
%   val p = new Proper(4);
%   return p.t == 4;
% } } 
%~~neg

\end{ex}


It is a static error for a class
defining a property \xcd"x: T" to have a subclass class that defines
a property or a field with the name \xcd"x".


A property \xcd`x:T` induces a field with the same name and type, 
as if defined with: 
%~~gen ^^^ Classes60
% package Classes.For.Masses.Of.NevermindTheRest;
% class Exampll[T] {
%~~vis
\begin{xten}
public val x : T;
\end{xten} 
%~~siv
% def this(y:T) { x=y; }
% }
%~~neg

\index{property!initialization}
Properties are initialized in a constructor by the invocation of a special \Xcd{property}
statement. The requirement to use the \xcd`property` statement means that all properties
must be given values at the same time: a container either has its properties
or it does not.
\begin{xten}
property(e1,..., en);
\end{xten}
The number and types of arguments to the \Xcd{property} statement must match
the number and types of the properties in the class declaration, in order.  
Every constructor of a class with properties must invoke \xcd`property(...)`
precisely once; it is a static error if X10 cannot prove that this holds.



By construction, the graph whose nodes are values and whose edges are
properties is acyclic.  \Eg, there cannot be values \xcd`a` and \xcd`b` with
properties \xcd`c` and \xcd`d` such that \xcd`a.c == b` and \xcd`b.d == a`.

\begin{ex}
%~~gen ^^^ Classes7h2f
% package Classes7h2f;
%~~vis
\begin{xten}
class Proper(a:Long, b:String) {
  def this(a:Long, b:String) {
      property(a, b);
  }
  def this(z:Long) {
      val theA = z+5;
      val theB = "X"+z;
      property(theA, theB);
  }
  static def example() {
      val p = new Proper(1, "one");
      assert p.a == 1 && p.b.equals("one");
      val q = new Proper(10);
      assert q.a == 15 && q.b.equals("X10");
  }
}
\end{xten}
%~~siv
% class Hook{ def run() {Proper.example(); return true;}}
%~~neg
\end{ex}

\subsection{Properties and Field Initialization}

Fields with explicit initializers are evaluated immediately after the
\xcd`property` command, and all properties are in scope when initializers are
evaluated.  

\begin{ex}
Class \xcd`Init` initializes the field \xcd`a` to be a rail of \xcd`n`
elements, where \xcd`n` is a property.    
When \xcd`new Init(4)` is executed, the constructor first sets \xcd`n` to
\xcd`4` via the \xcd`property` statement, and then initializes \xcd`a` to a
4-element rail.

However, \xcd`Outit` uses a field rather than a property for \xcd`n`.  
If the \xcd`ERROR` line were present, it would not compile.  \xcd`n` has not
been definitely assigned (\Sref{sect:DefiniteAssignment}) at this point, and
\xcd`n` has not been given its value, so \xcd`a` cannot be computed.  
(If one insisted that \xcd`n` be a property, \xcd`a` would have to be
initialized in the constructor, rather than by an initialization expression.)
%~~gen ^^^ Classes9c9r
% package Classes9c9r;
%~~vis
\begin{xten}
class Init(n:Long) {
  val a = new Rail[String](n, "");
  def this(n:Long) { property(n); }
}
class Outit {
  val n : Long;
  //ERROR: val a = new Rail[String](n, "");
  def this(m:Long) { this.n = m; }
}
\end{xten}
%~~siv
%
%~~neg


\end{ex}

\subsection{Properties and Fields}

A container with a property named \xcd`p`, or a nullary property method named
\xcd`p()`, cannot have a field named \xcd`p` --- either defined in that
container, or inherited from a superclass.

\subsection{Acyclicity of Properties}
\index{properties!acyclic}

X10 has certain restrictions that, ultimately, require that properties are
simpler than their containers.  For example, \xcd`class A(a:A){}` is not
allowed.  
Formally, this requirement is that there is  a total order $\preceq$ 
on all classes and
structs such that, if $A$ extends $B$, then $A \prec B$, and
if $A$ has a property of type $B$, then $A \prec B$, where $A \prec B$ means
$A \preceq B$ and $A \ne B$.   
For example, the preceding class \xcd`A` is ruled out because we would need
\xcd`A`$\prec$\xcd`A`, which violates the definition of $\prec$.
The programmer need not (and cannot) specify
$\preceq$, and rarely need worry about its existence.  

Similarly, 
the type of a property may not simply be a type parameter.  
For example, \xcd`class A[X](x:X){}` is illegal.





\section{Methods}
\label{sect:Methods}
\index{method}
\index{signature}
\index{method!signature}
\index{method!instance}
\index{method!static}

As is common in object-oriented languages, objects can have {\em methods}, of
two sorts.  {\em Static methods} are functions, conceptually associated with a
class and defined in its namespace.  {\em Instance methods} are parameterized
code bodies associated with an instance of the class, which execute with
convenient access to that instance's fields. 

Each method has a {\em signature}, telling what arguments it accepts, what
type it returns, and what precondition it requires. Method definitions may be
overridden by subclasses; the overriding definition may have a declared return
type that is a subtype of the return type of the definition being overridden.
Multiple methods with the same name but different signatures may be provided
\index{overloading}
\index{polymorphism}
on a class (called ``overloading'' or ``ad hoc polymorphism''). Methods may be
declared \Xcd{public}, \Xcd{private}, \Xcd{protected}, or given default package-level access
rights.

%##(MethMods MethodDeclaration TypeParams Formals FormalList HasResultType MethodBody BinOpDecln PrefixOpDecln ApplyOpDecln ConversionOpDecln
\begin{bbgrammar}
%(FROM #(prod:MethMods)#)
            MethMods \: Mods\opt & (\ref{prod:MethMods}) \\
                     \| MethMods \xcd"property"  \\
                     \| MethMods Mod \\
%(FROM #(prod:MethodDecln)#)
         MethodDecln \: MethMods \xcd"def" Id TypeParams\opt Formals Guard\opt Throws\opt HasResultType\opt MethodBody & (\ref{prod:MethodDecln}) \\
                     \| BinOpDecln \\
                     \| PrefixOpDecln \\
                     \| ApplyOpDecln \\
                     \| SetOpDecln \\
                     \| ConversionOpDecln \\
%(FROM #(prod:TypeParams)#)
          TypeParams \: \xcd"[" TypeParamList \xcd"]" & (\ref{prod:TypeParams}) \\
%(FROM #(prod:Formals)#)
             Formals \: \xcd"(" FormalList\opt \xcd")" & (\ref{prod:Formals}) \\
%(FROM #(prod:FormalList)#)
          FormalList \: Formal & (\ref{prod:FormalList}) \\
                     \| FormalList \xcd"," Formal \\
%(FROM #(prod:Throws)#)
             Throws \: \xcd"throws" ThrowList & (\ref{prod:Throws}) \\
%(FROM #(prod:ThrowsList)#)
          ThrowsList \: Type & (\ref{prod:ThrowsList}) \\
                     \| ThrowsList \xcd"," Type \\
%(FROM #(prod:HasResultType)#)
       HasResultType \: ResultType & (\ref{prod:HasResultType}) \\
                     \| \xcd"<:" Type \\
%(FROM #(prod:MethodBody)#)
          MethodBody \: \xcd"=" LastExp \xcd";" & (\ref{prod:MethodBody}) \\
                     \| Annotations\opt Block \\
                     \| \xcd";" \\
%(FROM #(prod:BinOpDecln)#)
          BinOpDecln \: MethMods \xcd"operator" TypeParams\opt \xcd"(" Formal  \xcd")" BinOp \xcd"(" Formal  \xcd")" Guard\opt HasResultType\opt MethodBody & (\ref{prod:BinOpDecln}) \\
                     \| MethMods \xcd"operator" TypeParams\opt \xcd"this" BinOp \xcd"(" Formal  \xcd")" Guard\opt HasResultType\opt MethodBody \\
                     \| MethMods \xcd"operator" TypeParams\opt \xcd"(" Formal  \xcd")" BinOp \xcd"this" Guard\opt HasResultType\opt MethodBody \\
%(FROM #(prod:PrefixOpDecln)#)
       PrefixOpDecln \: MethMods \xcd"operator" TypeParams\opt PrefixOp \xcd"(" Formal  \xcd")" Guard\opt HasResultType\opt MethodBody & (\ref{prod:PrefixOpDecln}) \\
                     \| MethMods \xcd"operator" TypeParams\opt PrefixOp \xcd"this" Guard\opt HasResultType\opt MethodBody \\
%(FROM #(prod:ApplyOpDecln)#)
        ApplyOpDecln \: MethMods \xcd"operator" \xcd"this" TypeParams\opt Formals Guard\opt HasResultType\opt MethodBody & (\ref{prod:ApplyOpDecln}) \\
%(FROM #(prod:ConversionOpDecln)#)
   ConversionOpDecln \: ExplConvOpDecln & (\ref{prod:ConversionOpDecln}) \\
                     \| ImplConvOpDecln \\
\end{bbgrammar}
%##)


\index{parameter!var}
\index{parameter!val}
A formal parameter may have a \xcd"val" or \xcd"var"
% , or \Xcd{ref}
modifier; \xcd`val` is the default.
The body of the method is executed in an environment in which 
each formal parameter corresponds to a local variable (\xcd`var` iff the
formal parameter is \xcd`var`)
and is initialized with the value of the actual parameter.

\subsection{Forms of Method Definition}

There are several syntactic forms for definining methods.   The forms that
include a block, such as \xcd`def m(){S}`, allow an arbitrary block.  These
forms can define a \xcd`void` method, which does not return a value. 

The
forms that include an expression, such as \xcd`def m()=E`, require a
syntactically and semantically valid expression.   These forms cannot define a
\xcd`void` method, because expressions cannot be \xcd`void`.  

There are no other semantic differences between the two forms. 

\subsection{Method Return Types}

A method with an explicit return type returns values of that type.
A method without an
explicit return type is given a return type by type inference.
A {\em call} to a method has type given by substituting information about the
actual \xcd`val` parameters for the formals.

\begin{ex}

In the example below, \xcd`met1` has an explicit return type \xcd`Ret{n==a}`.
\xcd`met2` does not, so its return type is computed, also to be
\xcd`Ret{n==a}`, because that's what the implicitly-defined constructor 
returns.

\xcd`use3` requires that its argument have \xcd`n==3`.  
\xcd`example` shows that both \xcd`met1` and \xcd`met2` can be used to produce
such an object.  In both cases, the actual argument \xcd`3` is substituted for
the formal argument \xcd`a` in the return type expression for the method
\xcd`Ret{n==a}`, giving the type \xcd`Ret{n==3}` as required by \xcd`use3`.

%~~gen ^^^ Classes9q2w
% package Classes9q2w;
%~~vis
\begin{xten}
class Ret(n:Long) {
  static def met1(a:Long) : Ret{n==a} = new Ret(a);
  static def met2(a:Long)             = new Ret(a);
  static def use3(Ret{n==3}) {}
  static def example() {
     use3(met1(3));
     use3(met2(3));
  }  
}
\end{xten}
%~~siv
%
%~~neg


\end{ex}


\subsection{Throws Clause}
The \xcd`throws` clause indicates what checked exceptions may be
raised during the execution of the method and are not handled by
\xcd'catch' blocks within the method.  If a checked exception may
escape from the method, then it must be by a subtype of one of the
types listed in the \xcd`throws` clause of the method.   Checked
exceptions are defined to be any subclass of
\xcd{x10.lang.CheckedThrowable} that are not also subclasses of
either \xcd{x10.lang.Exception} or \xcd{x10.lang.Error}. 

If a method is implementing an interface or overriding a superclass
method the set of types represented by its \xcd'throws' clause must by
a (potentially improper) subset of the types of the \xcd'throws'
clause of the method it is overriding. 

\subsection{Final Methods}
\index{final}
\index{method!final}
An instance method may be given the \xcd`final` qualifier.  \xcd`final`
methods may not be overridden.

\subsection{Generic Instance Methods}
\index{method!generic instance}

\limitationx{}
In X10, an instance method may be generic: 
%~~gen ^^^ Classes1b7z
% package Classes1b7z;
% NOTEST
%~~vis
\begin{xten}
class Example {
  def example[T](t:T) = "I like " + t;
}
\end{xten}
%~~siv
%
%~~neg

However, the C++ back end does not currently support generic virtual instance
methods like \xcd`example`.  It does allow generic instance methods which are
\xcd`final` or \xcd`private`, and it does allow generic static methods.  


\subsection{Method Guards}
\label{MethodGuard}
\index{method!guard}
\index{guard!on method}

Often, a method will only make sense to invoke under certain
statically-determinable conditions.  These conditions may be expressed as a
guard on the method.

\begin{ex}
For example, \xcd`example(x)` is only
well-defined when \xcd`x != null`, as \xcd`null.toString()` throws a null
pointer exception, and returns nothing: 
%~~gen ^^^ Classes80
% package Classes.methodwithconstraintthingie;
% 
%~~vis
\begin{xten}
class Example {
   var f : String = "";
   def setF(x:Any){x != null} : void {
      this.f = x.toString();
   }
}
\end{xten}
%~~siv
%
%~~neg
\noindent
(We could have used a constrained type \xcd`Any{self!=null}` for \xcd`x`
instead; in
most cases it is a matter of personal preference or convenience of expression
which one to use.) 
\end{ex}


The requirement of having a method guard 
is that callers must demonstrate to
the X10
compiler and/or runtime that the guard is satisfied.  
With the \xcd`STATIC_CHECKS` compiler option in force (\Sref{sect:Callstyle}), this is
checked at compile time, and there is no runtime cost.
Indeed, this code can be more efficient than usual, as it is statically provable that \xcd`x != null`.

When \xcd`STATIC_CHECKS` is not in force, dynamic checks are generated as
needed; method guards are checked at runtime. This is potentially more
expensive, but may be more convenient. 

\begin{ex}
The following code fragment contains a line which will not compile 
with \xcd`STATIC_CHECKS` on (assuming the guarded \xcd`example` method above).  (X10's type system does not attempt to propagate 
information from \xcd`if`s.)  It will compile with \xcd`STATIC_CHECKS` off,
but it may insert an extra \xcd`null`-test for \xcd`x`.  
%~~gen ^^^ Classes90
% package Classes.methodguardnadacastthingie;
%//OPTIONS: -STATIC_CHECKS
% class Example {var f : String = ""; def example(x:Any){x != null} {this.f = x.toString();}}
% class Eyample {
%~~vis
\begin{xten}
  def exam(e:Example, x:Any) {
    if (x != null) 
       e.example(x as Any{x != null});
       // If STATIC_CHECKS is in force: 
       // ERROR: if (x != null) e.example(x); 
  }
\end{xten}
%~~siv
%}
%~~neg
\end{ex}


The guard \xcd`{c}` 
in a guarded method 
\xcd`def m(){c} = E;`
specifies a constraint \xcd"c" on the
properties of the class \xcd"C" on which the method is being defined. The
method, in effect, only exists  for those instances of \xcd"C" which satisfy
\xcd"c".  It is 
illegal for code to invoke the method on objects whose static type is
not a subtype of \xcd"C{c}".

Specifically: 
    the compiler checks that every method invocation
    \xcdmath"o.m(e$_1$, $\dots$, e$_n$)"
    is type correct. Each argument
    \xcdmath"e$_i$" must have a
    static type \xcdmath"S$_i$" that is a subtype of the declared type
    \xcdmath"T$_i$" for the $i$th
    argument of the method, and the conjunction of the constraints on the
    static types 
    of the arguments must entail the guard in the parameter list
    of the method.

    The compiler checks that in every method invocation
    \xcdmath"o.m(e$_1$, $\dots$, e$_n$)"
    the static type of \xcd"o", \xcd"S", is a subtype of \xcd"C{c}", where the method
    is defined in class \xcd"C" and the guard for \xcd"m" is equivalent to
    \xcd"c".

    Finally, if the declared return type of the method is
    \xcd"D{d}", the
    return type computed for the call is
    \xcdmath"D{a: S; x$_1$: S$_1$; $\dots$; x$_n$: S$_n$; d[a/this]}",
    where \xcd"a" is a new
    variable that does not occur in
    \xcdmath"d, S, S$_1$, $\dots$, S$_n$", and
    \xcdmath"x$_1$, $\dots$, x$_n$" are the formal
    parameters of the method.


\limitation{
Using a reference to an outer class, \xcd`Outer.this`, in a constraint, is not supported.
}


\subsection{Property methods}
\index{method!property}
\index{property method}

%##(PropertyMethodDeclaration
\begin{bbgrammar}
%(FROM #(prod:PropMethodDecln)#)
     PropMethodDecln \: MethMods Id TypeParams\opt Formals Guard\opt Throws\opt HasResultType\opt MethodBody & (\ref{prod:PropMethodDecln}) \\
                     \| MethMods Id Guard\opt HasResultType\opt MethodBody \\
\end{bbgrammar}
%##)

Property methods are methods that can be evaluated in constraints, as
properties can.   They provide a means of abstraction over properties; \eg,
interfaces can specify property methods that implementing containers must
provide, but, just as they cannot specify ordinary fields, they cannot specify
property fields.   Property methods are very limited in computing power: they
must obey the same restrictions as constraint expressions.  In particular,
they cannot have side effects, or even much code in their bodies.


\begin{ex}
The \xcd`eq()` method below tells if the \xcd`x` and \xcd`y`
properties are equal; the \xcd`is(z)` method tells if they are both equal to
\xcd`z`.  
The \xcd`eq` and \xcd`is` property methods are used in types in the
\xcd`example` method.
%~~gen ^^^ Classes100
%package Classes.PropertyMethods;
%~~vis
\begin{xten}
class Example(x:Long, y:Long) {
   def this(x:Long, y:Long) { property(x,y); }
   property eq() = (x==y);
   property is(z:Long) = x==z && y==z;
   def example( a : Example{eq()}, b : Example{is(3)} ) {}
}
\end{xten}
%~~siv
%
%~~neg
\end{ex}

A property method declared in a class must have
a body and must not be \xcd"void".  The body of the method must
consist of only a single \xcd"return" statement with an expression,  or a single
expression.  It is a static error if the expression cannot be
represented in the constraint system.   Property methods may be \xcd`abstract`
in \xcd`abstract` classes, and may be specified in interfaces, but are
implicitly \xcd`final` in 
non-\xcd`abstract` classes. 

The expression may contain invocations of other property methods.  The
compiler ensures that there are no circularities in property methods, so
property method evaluations always terminate.

Property methods in classes are implicitly \xcd"final"; they cannot be
overridden.  It is a static error if a superclass has a property method with a
given signature, and a subclass has a method or property method with the same
signature.   It is a static error if a superclass has a property with some
name \xcd`p`, and a subclass has a nullary method of any kind (instance,
static, or property) also named \xcd`p`. 



A nullary property method definition may omit 
the \xcd"def" keyword.  That is, the following are equivalent:

%~~gen ^^^ Classes110
% package classes.waifsome1;
% class Waif(rect:Boolean, onePlace:Place, zeroBased:Boolean) {
%~~vis
\begin{xten}
property def rail(): Boolean = 
   rect && onePlace == here && zeroBased;
\end{xten}
%~~siv
%}
%~~neg
and
%~~gen ^^^ Classes120
% package classes.waifsome2;
% class Waif(rect:Boolean, onePlace:Place, zeroBased:Boolean) {
%~~vis
\begin{xten}
property rail(): Boolean = 
   rect && onePlace == here && zeroBased;
\end{xten}
%~~siv
%}
%~~neg

Similarly, nullary property methods can be inspected in constraints without
\xcd`()`. If \xcd`ob`'s type has a property \xcd`p`, then \xcd`ob.p` is that
property. Otherwise, if it has a nullary property method \xcd`p()`, \xcd`ob.p`
is equivalent to \xcd`ob.p()`. As a consequence, if the type provides both a
property \xcd`p` and a nullary method \xcd`p()`, then the property can be
accessed as \xcd`ob.p` and the method as \xcd`ob.p()`.\footnote{This only
applies to nullary property methods, not nullary instance methods.  Nullary
property methods perform limited computations, have no side effects, and
always return the same value, since
they have to be expressed in the constraint sublanguage.  In this sense, a
nullary property method does not behave hugely different from a property.
Indeed, a compilation scheme which cached the value of the property method
would all but erase the distinction.  Other methods may
have more behavior, \eg, side effects, so we keep the \xcd`()` to make it
clear that a method call is potentially complex.
}

%~~longexp~~`~~` ^^^ Classes130
% package classes.not.weasels;
% class Waif(rect:Boolean, onePlace:Place, zeroBased:Boolean) {
%   def this(rect:Boolean, onePlace:Place, zeroBased:Boolean) 
%          :Waif{self.rect==rect, self.onePlace==onePlace, self.zeroBased==zeroBased}
%          {property(rect, onePlace, zeroBased);}
%   property rail(): Boolean = rect && onePlace == here && zeroBased;
%   static def zoink() {
%      val w : Waif{
%~~vis
\xcd`w.rail`, with either definition above, 
% }= new Waif(true, here, true);
% }}
%~~pxegnol
is equivalent to 
%~~longexp~~`~~` ^^^ Classes140
% package classes.not.ferrets;
% class Waif(rect:Boolean, onePlace:Place, zeroBased:Boolean) {
%   def this(rect:Boolean, onePlace:Place, zeroBased:Boolean) 
%          :Waif{self.rect==rect, self.onePlace==onePlace, self.zeroBased==zeroBased}
%          {property(rect, onePlace, zeroBased);}
%   property rail(): Boolean = rect && onePlace == here && zeroBased;
%   static def zoink() {
%      val w : Waif{
%~~vis
\xcd`w.rail()`
% }= new Waif(true, here, true);
% }}
%~~pxegnol


\subsubsection{Limitation of Property Methods}

\limitationx{} 
Currently, X10 forbids the use of property methods which have all the
following features: 
\begin{itemize}
\item they are abstract, and
\item they have one or more arguments, and
\item they appear as subterms in constraints.
\end{itemize}
Any two of these features may be combined, but the three together may not be. 

\begin{ex} 
The constraint in \xcd`example1` is concrete, not abstract.  The constraint in
\xcd`example2` is nullary, and has no arguments.  The constraint in
\xcd`example3` appears at the top level, rather than as a subterm ({\em cf.}
the equality expressions \xcd`A==B` in the other examples).    However,
\xcd`example4` combines all three features, and is not allowed.
%~~gen ^^^ Classes7a5j
% package Classes7a5j;
% // If example4() compiles, then the limitation in Classes7a5j's section is
% // gone, so delete the whole subsection from the spec.
%~~vis
\begin{xten}
class Cls {
  property concrete(a:Long) = 7;
}
interface Inf {
  property nullary(): Long;
  property topLevel(z:Long):Boolean;
  property allThree(z:Long):Long;
}
class Example{
  def example1(Cls{self.concrete(3)==7}) = 1;
  def example2(Inf{self.nullary()==7})   = 2;
  def example3(Inf{self.topLevel(3)})    = 3;
  //ERROR: def example4(Inf{self.allThree(3)==7}) = "fails";
}
\end{xten}
%~~siv
%
%~~neg
\end{ex}


\subsection{Method overloading, overriding, hiding, shadowing and obscuring}
\label{MethodOverload}
\index{method!overloading}



The definitions of method overloading, overriding, hiding, shadowing and
obscuring in \Xten{} are familiar from languages such as Java, modulo the
following considerations motivated by type parameters and dependent types.



Two or more methods of a class or interface may have the same
name if they have a different number of type parameters, or
they have formal parameters of different constraint-erased types (in some instantiation of the
generic parameters). 



\begin{ex}
The following overloading of \xcd`m` is unproblematic.
%~~gen ^^^ Classes150
% package Classes.Mful;
%~~vis
\begin{xten}
class Mful{
   def m() = 1;
   def m[T]() = 2;
   def m(x:Long) = 3;
   def m[T](x:Long) = 4;
}
\end{xten}
%~~siv
%
%~~neg
\end{ex}


A class definition may include methods which are ambiguous in {\em some}
generic instantiation. (It is a compile-time error if the methods are
ambiguous in {\em every} generic instantiation, but excluding class
definitions which are are ambiguous in {\em some} instantiation would exclude
useful cases.)  It is a compile-time error to {\em use} an ambiguous method
call. 

\begin{ex}
The following class definition is acceptable.  However, the marked method
calls are ambiguous, and hence not acceptable.
%~~gen ^^^ Classes4d5e
% package Classes4d5e;
%~~vis
\begin{xten}
class Two[T,U]{
  def m(x:T)=1;
  def m(x:Long)=2;
  def m[X](x:X)=3;
  def m(x:U)=4;
  static def example() {
    val t12 = new Two[Long, Any]();
    // ERROR: t12.m(2);
    val t13  = new Two[String, Any]();
    t13.m("ferret");
    val t14 = new Two[Boolean,Boolean]();
    // ERROR: t14.m(true);
  }
}
\end{xten}
%~~siv
% // SKIP_NATIVE_X10: XTENLANG-1191 Native X10 does not support non-final generic instance methods
%~~neg
\noindent
The call \xcd`t12.m(2)` could refer to either the \xcd`1` or \xcd`2`
definition of \xcd`m`, so it is not allowed.   
The call \xcd`t14.m(true)` could refer to either the \xcd`1` or \xcd`4`
definition, so it, too, is not allowed.

The call \xcd`t13.m("ferret")` refers only to the \xcd`1` definition.  If
the \xcd`1` definition were absent, type argument inference would make it
refer to the \xcd`3` definition.  However, X10 will choose a fully-specified
call if there is one, before trying type inference, so this call unambiguously
refers to \xcd`1`.
\end{ex}


\XtenCurrVer{} does not permit overloading based on constraints. That is, the
following is {\em not} legal, although either method definition individually
is legal:
\begin{xten}
   def n(x:Long){x==1} = "one";
   def n(x:Long){x!=1} = "not";
\end{xten}




The definition of a method declaration \xcdmath"m$_1$" ``having the same signature
as'' a method declaration \xcdmath"m$_2$" involves identity of types. 



The {\em constraint erasure} of a type \xcdmath"T", 
\xcdmath"$ce$(T)",
is obtained by removing all the constraints outside of functions in \xcd`T`,
specificially: 
\begin{eqnarray}
ce({\tt T}) &=& {\tt T} \mbox{ if \xcd`T` is a container or interface}\\
ce({\tt T\{c\}}) &=& ce({\tt T})\\
ce({\tt T[S}_1{\tt,}\ldots{\tt,S}_n{\tt ]})
  &=&
ce({\tt T}){\tt [} ce({\tt S}_1){\tt,}\ldots{\tt,} ce({\tt S}_n){\tt ]}\\
ce({\tt (S}_1{\tt,}\ldots{\tt,S}_n{\tt ) => T})
  &=&
{\tt }{\tt (} ce({\tt S}_1){\tt,}\ldots{\tt,} ce({\tt S}_n){\tt ) => } 
ce({\tt T})
\end{eqnarray}



 Two methods are said to have {\em erasure equivalent signatures} if (a) they have the
 same number of type parameters, 
(b) they have the same number of formal (value) parameters, and (c)
for each formal parameter the constraint erasure of its types are erasure equivalent.
It is a 
compile-time error for there to be two methods with the same name and
erasure equivalent signatures in a class (either defined in that class or in a
superclass), unless the signatures are identical (without erasures) and one of the methods is
defined in a superclass (in which case the superclass's method is overridden
by the subclass's, and the subclass's method's return type must be a subtype of
the superclass's method's return type).  

 



In addition, the guard of an overridden method
must entail
the guard of the overriding method.   This
ensures that any virtual call to the method
satisfies the guard of the callee.

\begin{ex}
In the following example, the call to \xcd`s.recip(3)` in \xcd`example()`
will invoke \xcd`Sub.recip(n)`.  The call is legitimate because
\xcd`Super.recip`'s guard, \xcd`n != 0`, is satisfied by \xcd`3`.  
The guard on \xcd`Sub.recip(n)` is simply
\xcd`true`, which is also satisfied.  However, if we had used the \xcd`ERROR`
line's definition, the guard on \xcd`Sub.recip(n)` would be \xcd`n != 0, n != 3`, which
is not satisfied by \xcd`3`, so -- despite the call statically type-checking
-- at runtime the call would violate its guard and (in this case) throw an exception.
%~~gen ^^^ Classes5l3r
% package Classes5l3r;
% // FOR-ERR-CASE-DELETE: def recip(m:Long){true} = 1.0/m;
%~~vis
\begin{xten}
class Super {
  def recip(n:Long){n != 0} = 1.0/n;
}
class Sub extends Super{
  //ERROR: def recip(n:Long){n != 0, n != 3} = 1.0/(n * (n-3));
  def recip(m:Long){true} = 1.0/m;
}
class Example{
  static def example() {
     val s : Super = new Sub();
     s.recip(3);
  }
}
\end{xten}
%~~siv
%
%~~neg

\end{ex}


  If a class \xcd"C" overrides a method of a class or interface
  \xcd"B", the guard of the method in \xcd"B" must entail
  the guard of the method in \xcd"C".


A class \xcd"C" inherits from its direct superclass and superinterfaces all
their methods visible according to the access modifiers
of the superclass/superinterfaces that are not hidden or overridden. A method \xcdmath"M$_1$" in a class
\xcd"C" overrides
a method \xcdmath"M$_2$" in a superclass \xcd"D" if
\xcdmath"M$_1$" and \xcdmath"M$_2$" have erasedly equivalent signatures.
Methods are overriden on a signature-by-signature basis.  It is a compile-time
error if an instance method overrides a static method.  (But is it permitted
for an instance {\em field} to hide a static {\em field}; that's hiding
(\Sref{sect:FieldHiding}), not 
overriding, and hence totally different.)

\section{Constructors}
\label{sect:constructors}
\index{constructor}

Instances of classes are created by the \xcd`new` expression: \\
%##(ClassInstCreationExp
\begin{bbgrammar}
%(FROM #(prod:ObCreationExp)#)
       ObCreationExp \: \xcd"new" TypeName TypeArgs\opt \xcd"(" ArgumentList\opt \xcd")" ClassBody\opt & (\ref{prod:ObCreationExp}) \\
                     \| Primary \xcd"." \xcd"new" Id TypeArgs\opt \xcd"(" ArgumentList\opt \xcd")" ClassBody\opt \\
                     \| FullyQualifiedName \xcd"." \xcd"new" Id TypeArgs\opt \xcd"(" ArgumentList\opt \xcd")" ClassBody\opt \\
\end{bbgrammar}
%##)

This constructs a new object, and calls some code, called a {\em constructor},
to initialize the newly-created object properly.

Constructors are defined like methods, except that they must be named \xcd`this`
and ordinary methods may not be.    The content of a constructor body has
certain capabilities (\eg, \xcd`val` fields of the object may be initialized)
and certain restrictions (\eg, most methods cannot be called); see
\Sref{ObjectInitialization} for the details.

\begin{ex}

The following class provides two constructors.  The unary constructor 
\xcd`def this(b : Long)` allows initialization of the \xcd`a` field to an 
arbitrary value.  The nullary constructor \xcd`def this()` gives it a default
value of 10.  The \xcd`example` method illustrates both of these calls.


%~~gen ^^^ ClassesCtor10
% package ClassesCtor10;
%~~vis
\begin{xten}
class C {
  public val a : Long;
  def this(b : Long) { a = b; } 
  def this()        { a = 10; }
  static def example() {
     val two = new C(2);
     assert two.a == 2;
     val ten = new C(); 
     assert ten.a == 10;
  }
}
\end{xten}
%~~siv
% class Hook{ def run() {C.example(); return true;}}
%~~neg
\end{ex}


\subsection{Automatic Generation of Constructors}
\index{constructor!generated}

Classes that have no constructors written in the class declaration are
automatically given a constructor which sets the class properties and does
nothing else. If this automatically-generated constructor is not valid (\eg,
if the class has \xcd`val` fields that need to be initialized in a
constructor), the class has no constructor, which is a static error.

\begin{ex}
The following class has no explicit constructor.
Its implicit constructor is 
\xcd`def this(x:Long){property(x);}`
This implicit constructor is valid, and so is the class. 
%~~gen ^^^ ClassesCtor20
% package ClassesCtor20;
%~~vis
\begin{xten}
class C(x:Long) {
  static def example() {
    val c : C = new C(4);
    assert c.x == 4;
  }
}
\end{xten}
%~~siv
% class Hook{ def run() {C.example(); return true;}}
%~~neg
\noindent 


The following class has the same default constructor.  However, that
constructor does not initialize \xcd`d`, and thus is invalid.  This 
class does not compile; it needs an explicit constructor.
%~~gen ^^^ ClassCtor30_MustFailCompile
% NOCOMPILE
%~~vis
\begin{xten}
// THIS CODE DOES NOT COMPILE
class Cfail(x:Long) {
  val d: Long;
  static def example() {
    val wrong = new Cfail(40);
  }
}
\end{xten}
%~~siv
%
%~~neg


\end{ex}

\subsection{Calling Other Constructors}
\label{sect:call-another-ctor}

The {\em first} statement of a constructor body may be a call of the form 
\xcd`this(a,b,c)` or \xcd`super(a,b,c)`.  The former will execute the body of
the matching constructor of the current class; the latter, of the superclass. 
This allows a measure of abstraction in constructor definitions; one may be
defined in terms of another.

\begin{ex}
The following class has two constructors.  \xcd`new Ctors(123)` constructs a
new \xcd`Ctors` object with parameter 123.  \xcd`new Ctors()` constructs one
whose parameter has a default value of 100: 
%~~gen ^^^ Classes5q6q
% package Classes5q6q;
%~~vis
\begin{xten}
class Ctors {
  public val a : Long;
  def this(a:Long) { this.a = a; }
  def this()      { this(100);  }
}
\end{xten}
%~~siv
%class Hook{ def run() {
% val x = new Ctors(10); assert x.a == 10;
% val y = new Ctors(); assert y.a == 100;
% return true;}}
%~~neg
\end{ex}

In the case of a class which implements \xcd`operator ()` 
--- or any other constructor and application with the same signature --- 
this can be ambiguous.  If \xcd`this()` appears as the first statement of a
constructor body, it could, in principle, mean either a constructor call or an
operator evaluation.   This ambiguity is resolved so that \xcd`this()` always
means the constructor invocation.  If, for some reason, it is necessary to
invoke an application operator as the first meaningful statement of a
constructor, write the target of the application as \xcd`(this)`, \eg,
\xcd`(this)(a,b);`. 

\subsection{Return Type of Constructor}

A constructor for class \xcd`C` may have a return type \xcd`C{c}`.  The return
type specifies a constraint on the kind of object returned.  It cannot change
its {\em class} --- a constructor for class \xcd`C` always returns an instance
of class \xcd`C`.  
If no explicit return type is specified, the constructor's return type is
inferred.

\begin{ex}
The constructor \xcd`(A)` below, having no explicit return type, 
has its return type inferred.  
\xcd`n` is set by the \xcd`property` statement to \xcd`1`, so the return type
is inferred as \xcd`Ret{self.n==1}.`
The constructor \xcd`(B)` has \xcd`Ret{n==self.n}` as an explicit return type.
The \xcd`example()` code shows both of these in action.

%~~gen ^^^ Classes1v9a
% package Classes1v9a;
%~~vis
\begin{xten}
class Ret(n:Long) {
   def this()    { property(1); }     // (A)
   def this(n:Long) : Ret{n==self.n} { // (B)
      property(n);
   }
   static def typeIs[T](x:T){}
   static def example() {
     typeIs[Ret{self.n==1}](new Ret());  // uses (A)
     typeIs[Ret{self.n==3}](new Ret(3)); // uses (B)
   }
}
\end{xten}
%~~siv
%
%~~neg


\end{ex}

\section{Static initialization}
\label{StaticInitialization}
\index{initialization!static} 
Static fields in \Xten{} are immutable and are guaranteed to be
initialized before they are accessed. Static fields are initialized on
a per-Place basis; thus an activity that reads a static field in two
different Places may read different values for the content of the
field in each Place.  Static fields are not eagerly initialized, thus
if a particular static field is not accessed in a given Place then the
initializer expression for that field may not be evaluated in that
Place.

When an activity running in a Place \Xcd{P} attempts to read a static
field \Xcd{F} that has not yet been initialized in \Xcd{P}, then the
activity will evaluate the initializer expression for \Xcd{F} and
store the resulting value in \Xcd{F}. It is guaranteed that at most
one activity in each Place will attempt to evaluate the initializer
expression for a given static field.  If a second activity attempts to
read \Xcd{F} while the first activity is still executing the
initializer expression the second activity will be suspended until the
first activity finishes evaluating the initializer and stores the
resulting value in \Xcd{F}.

The initializer expression may directly or indirectly read other
static fields in the program.  If there is a cycle in the field
initialization dependency graph for a set of static fields, then any
activities accessing those fields may deadlock, which in turn may
result in the program deadlocking.\footnote{The current \Xten{}
  runtime does not dynamically detect this situation. Future versions
  of \Xten may be able to detect this and convert such a deadlock into the
  throwing of an \Xcd{ExceptionInInitializer} exception.}.

If an exception is raised during the evaluation of a static field's
initializer expression, then the field is deemed uninitializable in
that Place and any subsequent attempt to access the static field's
value by another activity in the Place will also result in an
exception being raised.\footnote{The implementation will make a best
  effort attempt to present stack trace information about the original
  cause of the exception in all subsequent raised exceptions}.  Failure
to initialize a field in one Place does not impact the initialization
status of the field in other Places.

\subsection{Compatability with Prior Versions of \Xten{}}
Previous versions of \Xten{} eagerly initialized all static fields in
the program at Place 0 and serialized the resulting values to all
other Places before beginning execution of the user main function.  It
is possible to simulate these serialization semantics for specific
static fields under the lazy per-Place initialization semantics
by using the idiom below:

\begin{xten}
// Pre X10 2.3 code
// expr evaluated once in Place 0 and resulting value 
// serialized to all other places
public static val x:T = expr;

// X10 2.3 code when T haszero is false
private static val x_holder:Cell[T] = 
    (here == Place.FIRST_PLACE) ? new Cell[T](expr): null;
public static val x:T = at (Place.FIRST_PLACE) x_holder();

// simpler X10 2.3 code when T haszero is true
private static val x_holder:T = 
    (here == Place.FIRST_PLACE) ? expr : Zero.get[T]();
public static val x:T = at (Place.FIRST_PLACE) x_holder;

\end{xten}

A slightly more complex variant of the above idiom in which the
initializer expression for the public field conditionally does the \xcd{at}
only when not executed at \xcd{Place.FIRST_PLACE} can be used to
obtain exactly the same serialization behavior as the pre \Xten{} v2.3
semantics.  When necessary, eager initialization for specific static fields
can be simulated by reading the static fields in \xcd{main} before
executing the rest of the program.

\section{User-Defined Operators}
\label{sect:operators}
\index{operator}
\index{operator!user-defined}

%##(MethodDeclaration
\begin{bbgrammar}
%(FROM #(prod:MethodDecln)#)
         MethodDecln \: MethMods \xcd"def" Id TypeParams\opt Formals Guard\opt Throws\opt HasResultType\opt MethodBody & (\ref{prod:MethodDecln}) \\
                     \| BinOpDecln \\
                     \| PrefixOpDecln \\
                     \| ApplyOpDecln \\
                     \| SetOpDecln \\
                     \| ConversionOpDecln \\
\end{bbgrammar}
%##)


It is often convenient to have methods named by symbols rather than words.
For example, suppose that we wish to define a \xcd`Poly` class of
polynomials -- for the sake of illustration, single-variable polynomials with
\xcd`Long` coefficients.  It would be very nice to be able to manipulate these
polynomials by the usual operations: \xcd`+` to add, \xcd`*` to multiply,
\xcd`-` to subtract, and \xcd`p(x)` to compute the value of the polynomial at
argument \xcd`x`.  We would like to write code thus: 
%~~gen ^^^ Classes160
% package Classes.In.Poly101;
% // Integral-coefficient polynomials of one variable.
% class Poly {
%   public val coeff : Rail[Long];
%   public def this(coeff: Rail[Long]) { this.coeff = coeff;}
%   public def degree() = (coeff.size-1);
%   public def a(i:Long) = (i<0 || i>this.degree()) ? 0L : coeff(i);
%
%   public static operator (c : Long) as Poly = new Poly([c as Long]);
%
%   public operator this(x:Long) {
%     val d = this.degree();
%     var s : Long = this.a(d);
%     for( i in 1L .. this.degree() ) {
%        s = x * s + a(d-i);
%     }
%     return s;
%   }
%
%   public operator this + (p:Poly) =  new Poly(
%      new Rail[Long](
%         Math.max(this.coeff.size, p.coeff.size),
%         (i:Long) => this.a(i) + p.a(i)
%      ));
%   public operator this - (p:Poly) = this + (-1)*p;
%
%   public operator this * (p:Poly) = new Poly(
%      new Rail[Long](
%        this.degree() + p.degree() + 1,
%        (k:Long) => sumDeg(k, this, p)
%        )
%      );
%
%
%   public operator (n : Long) + this = (n as Poly) + this;
%   public operator this + (n : Long) = (n as Poly) + this;
%
%   public operator (n : Long) - this = (n as Poly) + (-1) * this;
%   public operator this - (n : Long) = ((-n) as Poly) + this;
%
%   public operator (n : Long) * this = new Poly(
%      new Rail[Long](
%        this.degree()+1,
%        (k:Long) => n * this.a(k)
%      ));
%   private static def sumDeg(k:Long, a:Poly, b:Poly) {
%      var s : Long = 0;
%      for( i in 0L .. k ) s += a.a(i) * b.a(k-i);
%        // x10.io.Console.OUT.println("sumdeg(" + k + "," + a + "," + b + ")=" + s);
%      return s;
%      };
%   public final def toString() {
%      var allZeroSoFar : Boolean = true;
%      var s : String ="";
%      for( i in 0L..this.degree() ) {
%        val ai = this.a(i);
%        if (ai == 0L) continue;
%        if (allZeroSoFar) {
%           allZeroSoFar = false;
%           s = term(ai, i);
%        }
%        else
%           s +=
%              (ai > 0 ? " + " : " - ")
%             +term(ai, i);
%      }
%      if (allZeroSoFar) s = "0";
%      return s;
%   }
%   private final def term(ai: Long, n:Long) {
%      val xpow = (n==0L) ? "" : (n==1L) ? "x" : "x^" + n ;
%      return (ai == 1L) ? xpow : "" + Math.abs(ai) + xpow;
%   }
%
%   public static def Main(ss:Rail[String]):void {main(ss);};
%


%~~vis
\begin{xten}
  public static def main(Rail[String]):void {
     val X = new Poly([0L,1L]);
     val t <: Poly = 7 * X + 6 * X * X * X; 
     val u <: Poly = 3 + 5*X - 7*X*X;
     val v <: Poly = t * u - 1;
     for( i in -3 .. 3) {
       x10.io.Console.OUT.println(
         "" + i + "	X:" + X(i) + "	t:" + t(i) 
         + "	u:" + u(i) + "	v:" + v(i)
         );
     }
  }

\end{xten}
%~~siv
%}
%~~neg

Writing the same code with method calls, while possible, is far less elegant: 
%~~gen ^^^ Classes170

%package Classes.In.Remedial.Poly101;
% // Integer-coefficient polynomials of one variable.
% class UglyPoly {
%   public val coeff : Rail[Long];
%   public def this(coeff: Rail[Long]) { this.coeff = coeff;}
%   public def degree() = coeff.size-1;
%   public  def  a(i:Long) = (i<0 || i>this.degree()) ? 0L : coeff(i);
%
%   public static operator (c : Long) as UglyPoly = new UglyPoly([c as Long]);
%
%   public def apply(x:Long) {
%     val d = this.degree();
%     var s : Long = this.a(d);
%     for( i in 1L .. this.degree() ) {
%        s = x * s + a(d-i);
%     }
%     return s;
%   }
%
%   public operator this + (p:UglyPoly) =  new UglyPoly(
%      new Rail[Long](
%         Math.max(this.coeff.size, p.coeff.size),
%         (i:Long) => this.a(i) + p.a(i)
%      ));
%   public operator this - (p:UglyPoly) = this + (-1)*p;
%
%   public operator this * (p:UglyPoly) = new UglyPoly(
%      new Rail[Long](
%        this.degree() + p.degree() + 1,
%        (k:Long) => sumDeg(k, this, p)
%        )
%      );
%
%
%   public operator (n : Long) + this = (n as UglyPoly) + this;
%   public operator this + (n : Long) = (n as UglyPoly) + this;
%
%   public operator (n : Long) - this = (n as UglyPoly) + (-1) * this;
%   public operator this - (n : Long) = ((-n) as UglyPoly) + this;
%
%   public operator (n : Long) * this = new UglyPoly(
%      new Rail[Long](
%        this.degree()+1,
%        (k:Long) => n * this.a(k)
%      ));
%   private static def sumDeg(k:Long, a:UglyPoly, b:UglyPoly) {
%      var s : Long = 0;
%      for( i in 0L .. k ) s += a.a(i) * b.a(k-i);
%        // x10.io.Console.OUT.println("sumdeg(" + k + "," + a + "," + b + ")=" + s);
%      return s;
%      };
%   public final def toString() {
%      var allZeroSoFar : Boolean = true;
%      var s : String ="";
%      for( i in 0L..this.degree() ) {
%        val ai = this.a(i);
%        if (ai == 0L) continue;
%        if (allZeroSoFar) {
%           allZeroSoFar = false;
%           s = term(ai, i);
%        }
%        else
%           s +=
%              (ai > 0 ? " + " : " - ")
%             +term(ai, i);
%      }
%      if (allZeroSoFar) s = "0";
%      return s;
%   }
%   private final def term(ai: Long, n:Long) {
%      val xpow = (n==0L) ? "" : (n==1L) ? "x" : "x^" + n ;
%      return (ai == 1L) ? xpow : "" + Math.abs(ai) + xpow;
%   }
%
%   def mult(p:UglyPoly) : UglyPoly = this * p;
%   def mult(n:Long)     : UglyPoly = n * this;
%   def plus(p:UglyPoly) : UglyPoly = this + p;
%   def plus(n:Long)     : UglyPoly = n + this;
%   def minus(p:UglyPoly): UglyPoly = this - p;
%   def minus(n:Long)    : UglyPoly = this - n;
%   static def const(n:Long): UglyPoly = n as UglyPoly;
%
%~~vis
\begin{xten}
  public static def uglymain() {
     val X = new UglyPoly([0L,1L]);
     val t <: UglyPoly 
           = X.mult(7).plus(
               X.mult(X).mult(X).mult(6));  
     val u <: UglyPoly 
           = const(3).plus(
               X.mult(5)).minus(X.mult(X).mult(7));
     val v <: UglyPoly = t.mult(u).minus(1);
     for( i in -3 .. 3) {
       x10.io.Console.OUT.println(
         "" + i + "	X:" + X.apply(i) + "	t:" + t.apply(i) 
          + "	u:" + u.apply(i) + "	v:" + v.apply(i)
         );
     }
  }
\end{xten}
%~~siv
%}
%~~neg

The operator-using code can be written in X10, though a few variations are
necessary to handle such exotic cases as \xcd`1+X`.



Most X10 operators can be given definitions.\footnote{Indeed, even for the
standard types, these operators are defined in the library.  Not even as basic
an operation as integer addition is built into the language.  Conversely, if
you define a full-featured numeric type, it will have most of the privileges that
the standard ones enjoy.  The missing priveleges are (1) literals; 
(2) \xcd`*` won't track ranks, as it does for \xcd`Region`s; 
(3) \xcd`&&` and \xcd`||` won't short-circuit, as they do for \xcd`Boolean`s, 
(4) the built-in notion of equality \xcd`a==b` will only coincide with
the programmible notion \xcd`a.equals(b)`, as they do for most library types,
if coded that way; and (5) it is 
impossible to define an 
operation like \xcd`String.+` which converts both its left and right arguments
from any type.  For example, a \xcd`Polar` type might
have many representations for the origin, as radius 0 and any angle; these
will be \xcd`equals()`, but will not be \xcd`==`; whereas for the standard
\xcd`Complex` type, the two equalities are the same.}  (However, \xcd`&&` and
\xcd`||` 
are only short-circuiting for \xcd`Boolean` expressions; user-defined versions
of these operators have no special execution behavior.)

The user-definable operations are (in order of precedence): \\
\begin{tabular}{l}
implicit type coercions\\
postfix \xcd`()`\\
\xcd`as T`\\
these unary operators:  \xcd`- + ! ~ | & / ^ * %`\\
\xcd`..`\\
\xcd`*      /       %      **`\\
\xcd`+` \xcd`     -` \\
\xcd`<<     >>      >>>    ->     <-     >-      -<      !`\\
\xcd`>      ` \xcd`>=     ` \xcd`<     ` \xcd`<=     ` 
\xcd`~      !~`\\
\xcd`&` \\
\xcd`^` \\
\xcd`|` \\
\xcd`&&` \\
\xcd`||` \\
\end{tabular}

Several of these operators have no standard meaning on any library type, and
are included purely for programmer convenience.  


Many operators may be defined either in \xcd`static` or instance forms.  Those
defined in instance form are dynamically dispatched, just like an instance
method.  Those defined in static form are statically dispatched, just like a
static method.  Operators are scoped like methods; static operators are scoped
like static methods.

\begin{ex}
%~~gen ^^^ Classes6a1j
% package oifClasses6a1j;
% class Whatever {
% 
%~~vis
\begin{xten}
static class Trace(n:Long){
  public static operator !(f:Trace) 
      = new Trace(10 * f.n + 1);
  public operator -this = new Trace (10 * this.n + 2);
}
static class Brace extends Trace{
  def this(n:Long) { super(n); }
  public operator -this = new Brace (10 * this.n + 3);
  static def example() {
     val t = new Trace(1);
     assert (!t).n == 11;
     assert (-t).n == 12 && (-t instanceof Trace);
     val b = new Brace(1);
     assert (!b).n == 11;
     assert (-b).n == 13 && (-b instanceof Brace);
  }
}

\end{xten}
%~~siv
% // And checking the unambiguous syntax while I'm here...
% //static class Glook { def checky(t:Trace) { 
% //   Trace.operator !(t);
% //   t.operator -();
% //} }
% }
%~~neg
\end{ex}

%%OP%% Operators may be invoked by unambiguous syntax, loosely akin to a
%%OP%% fully-qualified name. For example, \xcd`!t` above may be invoked as
%%OP%% \xcd`Trace.operator !(t)`. This unambiguous syntax may be used even if there
%%OP%% are several \xcd`!` operators that could apply to \xcd`t`, rendering the
%%OP%% convenient short form \xcd`!t` unavailable in some context.




\subsection{Binary Operators}

Binary operators, illustrated by \xcd`+`, may be defined statically in a
container \xcd`A` as:
\begin{xten}
static operator (b:B) + (c:C) = ...;
\end{xten}
%%OP%% In this case it may be invoked as \xcd`A.operator +(b,c)`.
Or, it may be defined as  as an instance operator by one of the forms:
\begin{xten}
operator this + (b:B) = ...;
operator (b:B) + this = ...;
\end{xten}
%%OP%% and be invoked as 
%%OP%% \xcd`a.operator +(b)`
%%OP%% and as 
%%OP%% \xcd`a.operator ()+(b)` 
%%OP%% respectively.

\begin{ex}

Defining the sum \xcd`P+Q` of two polynomials looks much like a method
definition.  It uses the \xcd`operator` keyword instead of \xcd`def`, and
\xcd`this` appears in the definition in the place that a \xcd`Poly` would
appear in a use of the operator.  So, 
\xcd`operator this + (p:Poly)` explains how to add \xcd`this` to a
\xcd`Poly` value.
%~~gen ^^^ Classes180
% package Classes.In.Poly102;
%~~vis
\begin{xten}
class Poly {
  public val coeff : Rail[Long];
  public def this(coeff: Rail[Long]) { 
    this.coeff = coeff;}
  public def degree() = coeff.size-1;
  public def  a(i:Long) 
    = (i<0 || i>this.degree()) ? 0L : coeff(i);
  public operator this + (p:Poly) =  new Poly(
     new Rail[Long](
        Math.max(this.coeff.size, p.coeff.size),
        (i:Long) => this.a(i) + p.a(i)
     )); 
  // ... 
\end{xten}
%~~siv
%   public operator (n : Long) + this = new Poly([n as Long]) + this;
%   public operator this + (n : Long) = new Poly([n as Long]) + this;
% 
%   def makeSureItWorks() {
%      val x = new Poly([0L,1L]);
%      val p <: Poly = x+x+x;
%      val q <: Poly = 1+x;
%      val r <: Poly = x+1;
%   }
%     
% }
%~~neg


The sum of a polynomial and an integer, \xcd`P+3`, looks like
an overloaded method definition.  
%~~gen ^^^ Classes190
% package Classes.In.Poly103;
% class Poly {
%   public val coeff : Rail[Long];
%   public def this(coeff: Rail[Long]) { this.coeff = coeff;}
%   public def degree() = coeff.size-1;
%   public def  a(i:Long) = (i<0 || i>this.degree()) ? 0L : coeff(i);
% 
%   public operator this + (p:Poly) =  new Poly(
%      new Rail[Long](
%         Math.max(this.coeff.size, p.coeff.size),
%         (i:Long) => this.a(i) + p.a(i)
%      ));
%    public operator (n : Long) + this = new Poly([n as Long]) + this;
%~~vis
\begin{xten}
   public operator this + (n : Long) 
          = new Poly([n as Long]) + this;
\end{xten}
%~~siv
% 
%   def makeSureItWorks() {
%      val x = new Poly([0L,1L]);
%      val p <: Poly = x+x+x;
%      val q <: Poly = 1+x;
%      val r <: Poly = x+1;
%   }
%     
% }
%~~neg


However, we want to allow the sum of an integer and a polynomial as well:
\xcd`3+P`.  It would be quite inconvenient to have to define this as a method
on \xcd`Long`; changing \xcd`Long` is far outside of normal coding.  So, we
allow it as a method on \xcd`Poly` as well.


%~~gen ^^^ Classes200
% package Classes.In.Poly104o;
% class Poly {
%   public val coeff : Rail[Long];
%   public def this(coeff: Rail[Long]) { this.coeff = coeff;}
%   public def degree() = coeff.size-1;
%   public def  a(i:Long) = (i<0 || i>this.degree()) ? 0L : coeff(i);
% 
%   public operator this + (p:Poly) =  new Poly(
%      new Rail[Long](
%         Math.max(this.coeff.size, p.coeff.size),
%         (i:Long) => this.a(i) + p.a(i)
%      ));
%~~vis
\begin{xten}
   public operator (n : Long) + this 
          = new Poly([n as Long]) + this;
\end{xten}
%~~siv
% 
%   public operator this + (n : Long) = new Poly([n as Long]) + this;
%   def makeSureItWorks() {
%      val x = new Poly([0L,1L]);
%      val p <: Poly = x+x+x;
%      val q <: Poly = 1+x;
%      val r <: Poly = x+1;
%   }
%     
% }
%~~neg

Furthermore, it is sometimes convenient to express a binary operation as a
static method on a class. 
The definition for the sum of two
\xcd`Poly`s could have been written:
%~~gen ^^^ Classes210
% package Classes.In.Poly105;
% class Poly {
%   public val coeff : Rail[Long];
%   public def this(coeff: Rail[Long]) { this.coeff = coeff;}
%   public def degree() = coeff.size-1;
%   public def  a(i:Long) = (i<0 || i>this.degree()) ? 0L : coeff(i);
%~~vis
\begin{xten}
  public static operator (p:Poly) + (q:Poly) =  new Poly(
     new Rail[Long](
        Math.max(q.coeff.size, p.coeff.size),
        (i:Long) => q.a(i) + p.a(i)
     ));
\end{xten}
%~~siv
%
%   public operator (n : Long) + this = new Poly([n as Long]) + this;
%   public operator this + (n : Long) = new Poly([n as Long]) + this;
% 
%   def makeSureItWorks() {
%      val x = new Poly([0L,1L]);
%      val p <: Poly = x+x+x;
%      val q <: Poly = 1n+x;
%      val r <: Poly = x+1n;
%      val s <: Poly = 1+x;
%      val t <: Poly = x+1;
%   }
%     
% }
%~~neg

\end{ex}

When X10 attempts to typecheck a binary operator expression like \xcd`P+Q`, it
first typechecks \xcd`P` and \xcd`Q`. Then, it looks for operator declarations
for \xcd`+` in the types of \xcd`P` and \xcd`Q`. If there are none, it is a
static error. If there is precisely one, that one will be used. If there are
several, X10 looks for a {\em best-matching} operation, \viz{} one which does
not require the operands to be converted to another type. For example,
\xcd`operator this + (n:Long)` and \xcd`operator this + (n:Int)` both apply to
\xcd`p+1n`, because \xcd`1n` can be converted from an \xcd`Int` to a \xcd`Long`.
However, the \xcd`Int` version will be chosen because it does not require a
conversion. If even the best-matching operation is not uniquely determined,
the compiler will report a static error.


\subsection{Unary Operators}

Unary operators,  illustrated by \xcd`!`, may be defined statically in
container 
\xcd`A` as 
\begin{xten}
static operator !(x:A) = ...;
\end{xten}
or as instance operators by: 
\begin{xten}
operator !this = ...;
\end{xten}

%%OP%% A statically-defined unary operator \xcd`!` may be invoked on \xcd`a:A` as 
%%OP%% \xcd`A.operator !(a)`.  An instance operator may be invoked as
%%OP%% \xcd`a.operator !()`.  

The rules for typechecking a unary operation are the same as for methods; the
complexities of binary operations are not needed.

\begin{ex}
The operator to negate a polynomial is: 

%~~gen ^^^ Classes220
% package Classes.In.Poly106;
% class Poly {
%   public val coeff : Rail[Long];
%   public def this(coeff: Rail[Long]) { this.coeff = coeff;}
%   public def degree() = coeff.size-1;
%   public def  a(i:Long) = (i<0 || i>this.degree()) ? 0L : coeff(i);
%~~vis
\begin{xten}
  public operator - this = new Poly(
    new Rail[Long](coeff.size, (i:Long) => -coeff(i))
    );
\end{xten}
%~~siv
%   def makeSureItWorks() {
%      val x = new Poly([0L,1L]);
%      val p <: Poly = -x;
%   }
% }
%~~neg



\end{ex}


\subsection{Type Conversions}
\label{sect:type-conv}
\index{type conversion!user-defined}


Explicit type conversions, \xcd`e as A`, can be defined as operators on
class \xcd`A`, or on the container type of \xcd`e`.  These must be static
operators.  

To define an operator in \xcd`class A` (or \xcd`struct A)` converting values
of type \xcd`B` into type \xcd`A`, use the syntax: 
\begin{xten}
static operator (x:B) as ? {c} = ... 
\end{xten}
The \xcd`?` indicates the containing type \xcd`A`.  
The guard clause \xcd`{c}` may be omitted.



\begin{ex}
%~~gen ^^^ Classes230
% package Classes_explicit_type_conversions_a;
%~~vis
\begin{xten}
class Poly {
  public val coeff : Rail[Long];
  public def this(coeff: Rail[Long]) { this.coeff = coeff;}
  public static operator (a:Long) as ? = new Poly([a as Long]);
  public static def main(Rail[String]):void {
     val three : Poly = 3L as Poly;
  }
}
\end{xten}
%~~siv
%
%~~neg
\end{ex}
The \xcd`?` may be given a bound, such as \xcd`as ? <: Caster`, if desired.
  

There is little difference between an explicit conversion \xcd`e as T` and a
method call \xcd`e.asT()`.  The explicit conversion does say undeniably what
the result type will be.  However, as described in \Sref{sect:ambig-cast},
sometimes the built-in meaning of \xcd`as` as a cast overrides the
user-defined explicit conversion.  

Explicit casts are most suitable for cases
which resemble the use of explicit casts among the arithmetic types, where, 
for example, \xcd`1.0 as Int` is a way to turn a floating-point number into the
corresponding integer.  
While there is nothing in X10 which
requires it, \xcd`e as T` has the connotation that it gives a good
approximation of \xcd`e` in type \xcd`T`, just as \xcd`1` is a good
(indeed, perfect) approximation of \xcd`1.0` in type \xcd`Int`.  

\subsection{Implicit Type Coercions}
\label{sect:ImplicitCoercion}
\index{type conversion!implicit}

An implicit type conversion from \xcd`U`  to \xcd`T` may be specified in
container \xcd`T`.  
The syntax for it is: 
\begin{xten}
static operator (u:U) : T = e;
\end{xten}
%%OP%% which may be invoked by the unambiguous syntax 
%%OP%% \xcd`T.operator[T](u)` or \xcd`U.operator[T](u)`.
%%OP%% 



Implicit coercions are used automatically by the compiler on method calls 
(\Sref{sect:MethodResolution}) and assignments (\Sref{AssignmentStatement}).
Implicit coercions may be used when a value of type \xcd`T` appears in a
context expecting a value of type \xcd`U`.  If \xcd`T <: U`, no implicit
coercion is needed; \eg, a method \xcd`m` expecting an \xcd`Long` argument may 
be called as \xcd`m(3)`, with an argument of type \xcd`Long{self==3}`, which is
a subtype of \xcd`m`'s argument type \xcd`Long`. 
However, if it is not the case that \xcd`T <: U`, but there is an implicit
coercion from \xcd`T` to \xcd`U` defined in container \xcd`U`, then this
implicit coercion will be applied.

\begin{ex}
We can define an implicit coercion from \xcd`Long` to \xcd`Poly`,
and avoid having to define the sum of an integer and a polynomial
as many special cases.  In the following example, we only define \xcd`+` on
two polynomials.  The
calculation \xcd`1+x` coerces \xcd`1` to a polynomial and uses polynomial
addition to add it to \xcd`x`.

%~~gen ^^^ Classes240
% package Classes.And.Implicit.Coercions;
% class Poly {
%   public val coeff : Rail[Long];
%   public def this(coeff: Rail[Long]) { this.coeff = coeff;}
%   public def degree() = (coeff.size-1);
%   public def  a(i:Long) = (i<0 || i>this.degree()) ? 0L : coeff(i);
%   public final def toString() {
%      var allZeroSoFar : Boolean = true;
%      var s : String ="";
%      for( i in 0L..this.degree() ) {
%        val ai = this.a(i);
%        if (ai == 0L) continue;
%        if (allZeroSoFar) {
%           allZeroSoFar = false;
%           s = term(ai, i);
%        }
%        else 
%           s += 
%              (ai > 0 ? " + " : " - ")
%             +term(ai, i);
%      }
%      if (allZeroSoFar) s = "0";
%      return s;
%   }
%   private final def term(ai: Long, n:Long) {
%      val xpow = (n==0L) ? "" : (n==1L) ? "x" : "x^" + n ;
%      return (ai == 1L) ? xpow : "" + Math.abs(ai) + xpow;
%   }

%~~vis
\begin{xten}
  public static operator (c : Long) : Poly 
     = new Poly([c as Long]);

  public static operator (p:Poly) + (q:Poly) = new Poly(
      new Rail[Long](
        Math.max(p.coeff.size, q.coeff.size),
        (i:Long) => p.a(i) + q.a(i)
     ));

  public static def main(Rail[String]):void {
     val x = new Poly([0L,1L]);
     x10.io.Console.OUT.println("1+x=" + (1L+x));
  }
\end{xten}
%~~siv
%}
%~~neg
\end{ex}



\subsection{Assignment and Application Operators}
\index{assignment operator}
\index{application operator}
\index{()}
\index{()=}
\label{set-and-apply}
X10 allows types to implement the subscripting / function application
operator, and indexed assignment.  The \xcd`Array`-like classes take advantage
of both of these in \xcd`a(i) = a(i) + 1`.  

\xcd`a(b,c,d)`
is an operator call, to an operator defined with 
\xcd`public operator this(b:B, c:C, d:D)`.  It may be overloaded.
For
example, an ordered dictionary structure could allow subscripting by numbers
with \xcd`public operator this(i:Long)`, and by strings with 
\xcd`public operator this(s:String)`.  


\xcd`a(i,j)=b` is an \xcd`operator` as well, with zero or more indices
\xcd`i,j`.  It may also be overloaded. 

The update operations \xcd`a(i) += b` 
(for all binary operators in place of \xcd`+`)
are defined to be the same as the
corresponding \xcd`a(i) = a(i) + b`. This applies for all arities of
arguments, and all types, and all binary operations. Of course to use this,
the \xcd`+`, application and assignment \xcd`operator`s must be defined.


\begin{ex}

The \xcd`Oddvec` class of somewhat peculiar vectors illustrates this.

\xcd`a()` returns a string representation of the oddvec, which ordinarily
would 
0be done by \xcd`toString()` instead.  
\xcd`a(i)` sensibly picks out one of the three
coordinates of \xcd`a`.
\xcd`a()=b` sets all the coordinates of \xcd`a` to \xcd`b`.
\xcd`a(i)=b` assigns to one of the
coordinates.  \xcd`a(i,j)=b` assigns different values to \xcd`a(i)` and
\xcd`a(j)`.  

%~~gen ^^^ Classes250
% package Classes.Assignments1_oddvec;
%~~vis
\begin{xten}
class Oddvec {
  var v : Rail[Long] = new Rail[Long](3);
  public operator this () = 
      "(" + v(0) + "," + v(1) + "," + v(2) + ")";
  public operator this () = (newval: Long) { 
    for(p in v.range) v(p) = newval;
  }
  public operator this(i:Long) = v(i);
  public operator this(i:Long, j:Long) = [v(i),v(j)];
  public operator this(i:Long) = (newval:Long) 
      {v(i) = newval;}
  public operator this(i:Long, j:Long) = (newval:Long) 
      { v(i) = newval; v(j) = newval+1;} 
  public def example() {
    this(1) = 6;   assert this(1) == 6;
    this(1) += 7;  assert this(1) == 13;
  }
\end{xten}
%~~siv
% }
%  class Hook { def run() {
%     val a = new Oddvec();
%     assert a().equals("(0,0,0)");
%     a() = 1;
%     assert a().equals("(1,1,1)");
%     a(1) = 4;
%     assert a().equals("(1,4,1)");
%     a(0,2) = 5;
%     assert a().equals("(5,4,6)");
%     return true;
%   }
% }
%~~neg

\end{ex}

\section{Class Guards and Invariants}\label{DepType:ClassGuard}
\index{type invariants}
\index{class invariants}
\index{invariant!type}
\index{invariant!class}
\index{guard}


Classes (and structs and interfaces) may specify a {\em class guard}, a
constraint which must hold on all values of the class.    In the following
example, a \xcd`Line` is defined by two distinct \xcd`Pt`s\footnote{We use \xcd`Pt`
to avoid any possible confusion with the built-in class \xcd`Point`.}
%~~gen ^^^ Classes260
% package classes.guards.invariants.glurp;
%~~vis
\begin{xten}
class Pt(x:Long, y:Long){}
class Line(a:Pt, b:Pt){a != b} {}
\end{xten}
%~~siv
%
%~~neg

In most cases the class guard could be phrased as a type constraint on a property of
the class instead, if preferred.  Arguably, a symmetric constraint like two
points being different is better expressed as a class guard, rather than
asymmetrically as a constraint on one type: 
%~~gen ^^^ Classes270
% package classes.guards.invariants.glurp2;
% class Pt(x:Long, y:Long){}
%~~vis
\begin{xten}
class Line(a:Pt, b:Pt{a != b}) {}
\end{xten}
%~~siv
%
%~~neg



\label{DepType:TypeInvariant}
\index{class invariant}
\index{invariant!class}
\index{class!invariant}
\label{DepType:ClassGuardDef}



With every container  or interface \xcd"T" we associate a {\em type
invariant} $\mathit{inv}($\xcd"T"$)$, which describes the guarantees on the
properties of values of type \xcd`T`.  

Every value of \xcd`T` satisfies $\mathit{inv}($\xcd"T"$)$ at all times.  This
is somewhat stronger than the concept of type invariant in most languages
(which only requires that the invariant holds when no method calls are
active).  X10 invariants only concern properties, which are immutable; thus,
once established, they cannot be falsified.

The type
invariant associated with \xcd"x10.lang.Any"
is 
\xcd"true".

The type invariant associated with any interface or struct \xcd"I" that extends
interfaces \xcdmath"I$_1$, $\dots$, I$_k$" and defines properties
\xcdmath"x$_1$: P$_1$, $\dots$, x$_n$: P$_n$" and
specifies a guard \xcd"c" is given by:

\begin{xtenmath}
$\mathit{inv}$(I$_1$) && $\dots$ && $\mathit{inv}$(I$_k$) &&
self.x$_1$ instanceof P$_1$ &&  $\dots$ &&  self.x$_n$ instanceof P$_n$ 
&& c  
\end{xtenmath}

Similarly the type invariant associated with any class \xcd"C" that
implements interfaces \xcdmath"I$_1$, $\dots$, I$_k$",
extends class \xcd"D" and defines properties
\xcdmath"x$_1$: P$_1$, $\dots$, x$_n$: P$_n$" and
specifies a guard \xcd"c" is
given by the same thing with the invariant of the superclass \xcd`D` conjoined:
\begin{xtenmath}
$\mathit{inv}$(I$_1$) && $\dots$ && $\mathit{inv}$(I$_k$) 
&& self.x$_1$ instanceof P$_1$ &&  $\dots$ &&  self.x$_n$ instanceof P$_n$ 
&& c  
&& $\mathit{inv}$(D)
\end{xtenmath}


Note that the type invariant associated with a class entails the type
invariants of each interface that it implements (directly or indirectly), and
the type invariant of each ancestor class.
It is guaranteed that for any variable \xcd"v" of
type \xcd"T{c}" (where \xcd"T" is an interface name or a class name) the only
objects \xcd"o" that may be stored in \xcd"v" are such that \xcd"o" satisfies
$\mathit{inv}(\mbox{\xcd"T"}[\mbox{\xcd"o"}/\mbox{\xcd"this"}])
\wedge \mbox{\xcd"c"}[\mbox{\xcd"o"}/\mbox{\xcd"self"}]$.



\subsection{Invariants for {\tt implements} and {\tt extends} clauses}\label{DepType:Implements}
\label{DepType:Extends}
\index{type-checking!implements clause}
\index{type-checking!extends clause}
\index{implements}
\index{extends}
Consider a class definition
\begin{xtenmath}
$\mbox{\emph{ClassModifiers}}^{\mbox{?}}$
class C(x$_1$: P$_1$, $\dots$, x$_n$: P$_n$){c} extends D{d}
   implements I$_1${c$_1$}, $\dots$, I$_k${c$_k$}
$\mbox{\emph{ClassBody}}$
\end{xtenmath}

These two rules must be satisfied:


\begin{itemize}

\item 
The type invariant \xcdmath"$\mathit{inv}$(C)" of \xcd"C" must entail
\xcdmath"c$_i$[this/self]" for each $i$ in $\{1, \dots, k\}$


\item The return type \xcd"c" of each constructor in a class \xcd`C`
must entail the invariant \xcdmath"$\mathit{inv}$(C)".
\end{itemize}

\subsection{Timing of Invariant Checks}

\index{invariant!checked}

The invariants for a container are checked immediately after the
\xcd`property` statement in the container's constructor. 
This is the earliest that the invariant could possibly be checked. 
Recall that an invariant 
can mention the properties of the container (which are set, forever, at that
point in the code), but cannot mention the \xcd`val`
or \xcd`var` fields (which might not be set at that point), or \xcd`this`
(which might not have been fully initialized).  

If X10 can prove that the invariant always holds given the \xcd`property`
statement and other known information, it may omit the actual check.




\subsection{Invariants and constructor definitions}
\index{invariant!and constructor}
\index{constructor!and invariant}

A constructor for a class \xcd"C" is guaranteed to return an object of the
class on successful termination. This object must satisfy  \xcdmath"$\mathit{inv}$(C)", the
class invariant associated with \xcd"C" (\Sref{DepType:TypeInvariant}).
However,
often the objects returned by a constructor may satisfy {\em stronger}
properties than the class invariant. \Xten{}'s dependent type system
permits these extra properties to be asserted with the constructor in
the form of a constrained type (the ``return type'' of the constructor):

%##(CtorDeclaration
\begin{bbgrammar}
%(FROM #(prod:CtorDecln)#)
           CtorDecln \: Mods\opt \xcd"def" \xcd"this" TypeParams\opt Formals Guard\opt HasResultType\opt CtorBody & (\ref{prod:CtorDecln}) \\
\end{bbgrammar}
%##)

\label{ConstructorGuard}

The parameter list for the constructor
may specify a \emph{guard} that is to be satisfied by the parameters
to the list.

\begin{ex}
%%TODO--rewrite this
Here is another example, constructed as a simplified 
version of \Xcd{x10.regionarray.Region}.  The \xcd`mockUnion` method 
has the type, though not the value, that a true \xcd`union` method would have.

%~~gen ^^^ Classes280
%package Classes.SimplifiedRegion;
%import x10.regionarray.*;
%~~vis
\begin{xten}
class MyRegion(rank:Long) {
  static type MyRegion(n:Long)=MyRegion{rank==n};
  def this(r:Long):MyRegion(r) {
    property(r);
  }
  def this(diag:Rail[Long]):MyRegion(diag.size){ 
    property(diag.size);
  }
  def mockUnion(r:MyRegion(rank)):MyRegion(rank) = this;
  def example() {
    val R1 : MyRegion(3L) = new MyRegion([4,4,4 as Long]); 
    val R2 : MyRegion(3L) = new MyRegion([5,4,1]); 
    val R3 = R1.mockUnion(R2); // inferred type MyRegion(3)
  }
}
\end{xten}
%~~siv
%
%~~neg
The first constructor returns the empty region of rank \Xcd{r}.  The
second constructor takes a \Xcd{Rail[Long]} of arbitrary length
\Xcd{n} and returns a \Xcd{MyRegion(n)} (intended to represent the set
of points in the rectangular parallelopiped between the origin and the
\Xcd{diag}.)

The code in \xcd`example` typechecks, and \xcd`R3`'s type is inferred as
\xcd`MyRegion(3)`.  


\end{ex}

   Let \xcd"C" be a class with properties
   \xcdmath"p$_1$: P$_1$, $\dots$, p$_n$: P$_n$", and invariant \xcd"c"
   extending the constrained type \xcd"D{d}" (where \xcd"D" is the name of a
   class).



   For every constructor in \xcd"C" the compiler checks that the call to
   super invokes a constructor for \xcd"D" whose return type is strong enough
   to entail \xcd"d". Specifically, if the call to super is of the form 
     \xcdmath"super(e$_1$, $\dots$, e$_k$)"
   and the static type of each expression \xcdmath"e$_i$" is
   \xcdmath"S$_i$", and the invocation
   is statically resolved to a constructor
\xcdmath"def this(x$_1$: T$_1$, $\dots$, x$_k$: T$_k$){c}: D{d$_1$}"
   then it must be the case that 
\begin{xtenmath}
x$_1$: S$_1$, $\dots$, x$_i$: S$_i$ entails x$_i$: T$_i$  (for $i \in \{1, \dots, k\}$)
x$_1$: S$_1$, $\dots$, x$_k$: S$_k$ entails c  
d$_1$[a/self], x$_1$: S$_1$, ..., x$_k$: S$_k$ entails d[a/self]      
\end{xtenmath}
\noindent where \xcd"a" is a constant that does not appear in 
\xcdmath"x$_1$: S$_1$ $\wedge$ ... $\wedge$ x$_k$: S$_k$".

   The compiler checks that every constructor for \xcd"C" ensures that
   the properties \xcdmath"p$_1$,..., p$_n$" are initialized with values which satisfy
   $\mathit{inv}($\xcd"T"$)$, and its own return type \xcd"c'" as follows.  In each constructor, the
   compiler checks that the static types \xcdmath"T$_i$" of the expressions \xcdmath"e$_i$"
   assigned to \xcdmath"p$_i$" are such that the following is
   true:
\begin{xtenmath}
p$_1$: T$_1$, $\dots$, p$_n$: T$_n$ entails $\mathit{inv}($T$)$ $\wedge$ c'     
\end{xtenmath}

(Note that for the assignment of \xcdmath"e$_i$" to \xcdmath"p$_i$"
to be type-correct it must be the
    case that \xcdmath"p$_i$: T$_i$ $\wedge$ p$_i$: P$_i$".) 



The compiler must check that every invocation \xcdmath"C(e$_1$, $\dots$, e$_n$)" to a
constructor is type correct: each argument \xcdmath"e$_i$" must have a static type
that is a subtype of the declared type \xcdmath"T$_i$" for the $i$th
argument of the
constructor, and the conjunction of static types of the argument must
entail the constraint in the parameter list of the constructor.

\section{Generic Classes}

Classes, like other units, can be generic.  They can be parameterized by
types.  The parameter types are used just like ordinary types inside the body
of the generic class -- with a few exceptions.  

\begin{ex}
A \xcd`Colorized[T]` holds a thing of type \xcd`T`, and a string which is intended to represent
its color.  Any type can be used for \xcd`T`; the \xcd`example` method shows
\xcd`Long` and \xcd`Boolean`.  The \xcd`thing()` method retrieves the thing;
note that its return type is the generic type variable \xcd`T`.  X10 is aware 
that \xcd`colLong.thing()` is an \xcd`Long` and \xcd`colTrue.thing()` is a
\xcd`Boolean`, and uses those typings in \xcd`example`. 
%~~gen ^^^ Classes6d9u
% package Classes6d9u;
%~~vis
\begin{xten}
class Colorized[T] {
  private var thing:T; 
  private var color:String; 
  def this(thing:T, color:String) {
     this.thing = thing;
     this.color = color;
  }
  public def thing():T = thing;
  public def color():String = color;  
  public static def example() {
    val colLong  : Colorized[Long] 
                = new Colorized[Long](3, "green");
    assert colLong.thing() == 3 
                && colLong.color().equals("green");
    val colTrue : Colorized[Boolean] 
                = new Colorized[Boolean](true, "blue");
    assert colTrue.thing() 
                && colTrue.color().equals("blue");
  }
}
\end{xten}
%~~siv
%class Hook{ def run() {Colorized.example(); return true;}}
%~~neg


\end{ex}



\subsection{Use of Generics}

An unconstrained type variable \Xcd{X} can be instantiated by any type. All
the operations of \Xcd{Any} are available on a 
variable of type \Xcd{X}. Additionally, variables of type
\Xcd{X} may be used with \Xcd{==, !=}, in \Xcd{instanceof}, and casts.  

If a type variable is constrained, the operations implied by its constraint
are available as well.

\begin{ex}
The interface \xcd`Named` describes entities which know their own name.  The
class \xcd`NameMap[T]` is a specialized map which stores and retrieves
\xcd`Named` entities by name.  The call \xcd`t.name()` in \xcd`put()` is only
valid because the constraint \xcd`{T <: Named}` implies that \xcd`T` is a
subtype of \xcd`Named`, and hence provides all the operations of \xcd`Named`. 
%~~gen ^^^ Types6e6x
% package Types6e6x;
% import x10.util.*;
%~~vis
\begin{xten}
interface Named { def name():String; }
class NameMap[T]{T <: Named, T haszero} {
   val m = new HashMap[String, T]();
   def put(t:T) { m.put(t.name(), t); }
   def get(s:String):T = m.getOrThrow(s);
}
\end{xten}
%~~siv
%
%~~neg


\end{ex}





\section{Object Initialization}
\label{ObjectInitialization}
\index{initialization}
\index{constructor}
\index{object!constructor}
\index{struct!constructor}

% \noo{Confirm this chapter with the paper}

X10 does object initialization safely.  It avoids certain bad things which
trouble some other languages:
\begin{enumerate}
\item Use of a field before the field has been initialized.
\item A program reading two different values from a \xcd`val` field of a
      container. 
\item \Xcd{this} escaping from a constructor, which can cause problems as
      noted below. 

\end{enumerate}

It should be unsurprising that fields must not be used before they are
initialized. At best, it is uncertain what value will be in them, as in
\Xcd{x} below. Worse, the value might not even be an allowable value; \Xcd{y},
declared to be nonzero in the following example, might be zero before it is
initialized.
\begin{xten}
// Not correct X10
class ThisIsWrong {
  val x : Long;
  val y : Long{y != 0};
  def this() {
    x10.io.Console.OUT.println("x=" + x + "; y=" + y);
    x = 1; y = 2;
  }
}
\end{xten}

One particularly insidious way to read uninitialized fields is to allow
\Xcd{this} to escape from a constructor. For example, the constructor could
put \Xcd{this} into a data structure before initializing it, and another
activity could read it from the data structure and look at its fields:
\begin{xten}
class Wrong {
  val shouldBe8 : Long;
  static Cell[Wrong] wrongCell = new Cell[Wrong]();
  static def doItWrong() {
     finish {
       async { new Wrong(); } // (A)
       assert( wrongCell().shouldBe8 == 8); // (B)
     }
  }
  def this() {
     wrongCell.set(this); // (C) - ILLEGAL
     this.shouldBe8 = 8; // (D)
  }
}
\end{xten}
\noindent
In this example, the underconstructed \Xcd{Wrong} object is leaked into a
storage cell at line \Xcd{(C)}, and then initialized.  The \Xcd{doItWrong}
method constructs a new \Xcd{Wrong} object, and looks at the \Xcd{Wrong}
object in the storage cell to check on its \Xcd{shouldBe8} field.  One
possible order of events is the following:
\begin{enumerate}
\item \Xcd{doItWrong()} is called.
\item \Xcd{(A)} is started.  Space for a new \Xcd{Wrong} object is allocated.
      Its \Xcd{shouldBe8} field, not yet initialized, contains some garbage
      value.
\item \Xcd{(C)} is executed, as part of the process of constructing a new
      \Xcd{Wrong} object.  The new, uninitialized object is stored in
      \Xcd{wrongCell}.
\item Now, the initialization activity is paused, and execution of the main activity
      proceeds from \Xcd{(B)}.
\item The value in \Xcd{wrongCell} is retrieved, and is \Xcd{shouldBe8} field
      is read.  This field contains garbage, and the assertion fails.
\item Now let the initialization activity proceed with \Xcd{(D)},
      initializing \Xcd{shouldBe8} --- too late.
\end{enumerate}

The \xcd`at` statement (\Sref{AtStatement}) introduces the potential for
escape as well. The following class prints an uninitialized value: 
%~~gen ^^^ ThisEscapingViaAt_MustFailCompile
% package ObjInit_at;
% NOCOMPILE
%~~vis
\begin{xten}
// This code violates this chapter's constraints
// and thus will not compile in X10.
class Example {
  val a: Long;
  def this() { 
    at(here.next()) {
      // Recall that 'this' is a copy of 'this' outside 'at'.
      Console.OUT.println("this.a = " + this.a);
    }
    this.a = 1;
  }
}
\end{xten}
%~~siv
%
%~~neg


X10 must protect against such possibilities.  The rules explaining how
constructors can be written are somewhat intricate; they are designed to allow
as much programming as possible without leading to potential problems.
Ultimately, they simply are elaborations of the fundamental principles that
uninitialized fields must never be read, and \Xcd{this} must never be leaked.

%%RAW%% \subsection{Raw and Cooked Objects}
%%RAW%% \index{raw}
%%RAW%% \index{cooked}
%%RAW%% 
%%RAW%% An object is {\em raw} before its constructor ends, and {\em cooked} after its
%%RAW%% constructor ends. Note that, when an object is cooked, all its subobjects are
%%RAW%% cooked.  
%%RAW%% 



\subsection{Constructors and Non-Escaping Methods}
\index{non-escaping}
\label{sect:nonescaping}

In general, constructors must not be allowed to call methods with \Xcd{this} as
an argument or receiver. Such calls could leak references to \Xcd{this},
either directly from a call to \Xcd{cell.set(this)}, or indirectly because
\Xcd{toString} leaks \Xcd{this}, and the concatenation
\Xcd`"Escaper = "+this` calls \Xcd{toString}.\footnote{This is abominable behavior for
\Xcd{toString}, but it cannot be prevented -- save by a scheme such as we
present in this section.}
%~WRONG~gen
%package ObjectInit.CtorAndNonEscaping.One;
%~WRONG~vis
\begin{xten}
// This code violates this chapter's constraints
// and thus will not compile in X10.
class Escaper {
  static val Cell[Escaper] cell = new Cell[Escaper]();
  def toString() {
    cell.set(this);
    return "Evil!";
  }
  def this() {
    cell.set(this);
    x10.io.Console.OUT.println("Escaper = " + this);
  }
}
\end{xten}
%~WRONG~siv
%
%~WRONG~neg

However, it is convenient to be able to call methods from constructors; {\em
e.g.}, a class might have eleven constructors whose common behavior is best
described by three methods.
Under certain stringent conditions, it {\em is}
safe to call a method: the method called must not leak references to
\Xcd{this}, and must not read \Xcd{val}s or \Xcd{var}s which might not have
been assigned.

So, X10 performs a static dataflow analysis, sufficient to guarantee that
method calls in constructors are safe.  This analysis requires having access
to or guarantees about all the code that could possibly be called.  This can
be accomplished in two ways:
\begin{enumerate}
\item Ensuring that only code from the class itself can be called, by
      forbidding overriding of
      methods called from the constructor: they can be marked \Xcd{final} or
      \Xcd{private}, or the whole class can be \Xcd{final}.
\item Marking the methods called from the constructor by
      \xcd`@NonEscaping` or \xcd`@NoThisAccess`
\end{enumerate}

\subsubsection{Non-Escaping Methods}
\index{method!non-escaping}
\index{method!implicitly non-escaping}
\index{method!NonEscaping}
\index{implicitly non-escaping}
\index{non-escaping}
\index{non-escaping!implicitly}
\index{NonEscaping}


A method may be annotated with \xcd`@NonEscaping`.  This
imposes several restrictions on the method body, and on all methods overriding
it.  However, it is the only way that a method can be called from
constructors.  The
\Xcd{@NonEscaping} annotation makes explicit all the X10 compiler's needs for
constructor-safety.

A method can, however, be safe to call from constructors without being marked
\Xcd{@NonEscaping}. We call such methods {\em implicitly non-escaping}.
Implicitly non-escaping methods need to obey the same constraints on
\Xcd{this}, \Xcd{super}, and variable usage as \Xcd{@NonEscaping} methods. An
implicitly non-escaping method {\em could} be marked as
\xcd`@NonEscaping`; the compiler, in
effect, infers the annotation. In addition, all non-escaping methods
must be \Xcd{private} or \Xcd{final} or members of a \Xcd{final} class; this
corresponds to the hereditary nature of \Xcd{@NonEscaping} (by forbidding
inheritance of implicitly non-escaping methods).

We say that a method is {\em non-escaping} if it is either implicitly
non-escaping, or annotated \Xcd{@NonEscaping}.

The first requirement on non-escaping methods is that they do not allow
\Xcd{this} to escape. Inside of their bodies, \Xcd{this} and \Xcd{super} may
only be used for field access and assignment, and as the receiver of
non-escaping methods.


The following example uses the possible variations.  \Xcd{aplomb()} 
explicitly forbids reading any field but
\Xcd{a}. \Xcd{boric()} is called after \Xcd{a} and \Xcd{b} are set, but
\Xcd{c} is not.
The \xcd`@NonEscaping` annotation on \xcd`boric()` is optional, but the
compiler will print a warning if it is left out.
\Xcd{cajoled()} is only called after all fields are set, so it
can read anything; its annotation, too, is not required.   \Xcd{SeeAlso} is able to override \Xcd{aplomb()}, because
\Xcd{aplomb()} is \xcd`@NonEscaping`; it cannot override the final method
\Xcd{boric()} or the private one \Xcd{cajoled()}.  
%~~gen ^^^ ObjectInitialization10
%package ObjInit.C2;
%~~vis
\begin{xten}
import x10.compiler.*;

final class C2 {
  protected val a:Long; protected val b:Long; protected val c:Long;
  protected var x:Long; protected var y:Long; protected var z:Long;
  def this() {
    a = 1;
    this.aplomb();
    b = 2;
    this.boric();
    c = 3;
    this.cajoled();
  }
  @NonEscaping def aplomb() {
    x = a;
    // this.boric(); // not allowed; boric reads b.
    // z = b; // not allowed -- only 'a' can be read here
  }
  @NonEscaping final def boric() {
    y = b;
    this.aplomb(); // allowed; 
       // a is definitely set before boric is called
    // z = c; // not allowed; c is not definitely written
  }
  @NonEscaping private def cajoled() {
    z = c;
  }
}

\end{xten}
%~~siv
%
%~~neg

\subsubsection{NoThisAccess Methods}

A method may be annotated \xcd`@NoThisAccess`.  \xcd`@NoThisAccess` methods
may be called from constructors, and they may be overridden in subclasses.
However, they may not refer to \xcd`this` in any way -- in particular, they
cannot refer to fields of \xcd`this`, nor to \xcd`super`.

\begin{ex}

The class \xcd`IDed` has an \xcd`Float`-valued \xcd`id` field.  The method
\xcd`count()` is used to initialize the \xcd`id`.  For \xcd`IDed` objects,
the \xcd`id` is the count of \xcd`IDed`s created with the same parity of its
\xcd`kind`.   Note that \xcd`count()` does not refer to \xcd`this`, though
it does refer to a \xcd`static` field \xcd`counts`. 

The subclass \xcd`SubIDed` has \xcd`id`s that depend on \xcd`kind%3`
as well as the parity of \xcd`kind`.  It overrides the \xcd`count()`
method.  The body of \xcd`count()` still cannot refer to \xcd`this`.
Nor can it refer to \xcd`super` (which is \xcd`self` under another name).
This precludes the use of a \xcd`super` call.  This is why we have separated
the body of \xcd`count` out as the static method \xcd`kind2count` -- without
that, we would have had to duplicate its body, as we could not call 
\xcd`super.count(kind)` in a \xcd`NoThisAccess` method, as is shown by 
the \xcd`ERROR` line \xcd`(A)`. 

Note that \xcd`NoThisAccess` is in \xcd`x10.compiler` and must be imported,
and that the overriding method \xcd`SubIDed.count` must be declared
\xcd`@NoThisAccess` as well as the overridden method.
Line \xcd`(B)` is not allowed because \xcd`code` is a field of \xcd`this`, 
and field accesses are forbidden.   Line \xcd`(C)` references \xcd`this`
directly, which, of course, is forbidden by \xcd`@NoThisAccess`.  


%~~gen ^^^ ObjectInitialization7p2v
% package ObjectInitialization7p2v;
%~~vis
\begin{xten}
import x10.compiler.*;
class UseNoThisAccess {
  static class IDed {
    protected static val counts = [0 as Long,0];
    protected var code : Long;
    val id: Float;
    public def this(kind:Long) { 
      code = kind;
      this.id = this.count(kind); 
    }
    protected static def kind2count(kind:Long) = ++counts(kind % 2);
    @NoThisAccess def count(kind:Long) : Float = kind2count(kind);
  }
  static class SubIDed extends IDed {
    protected static val subcounts = [0 as Long, 0, 0];
    public static val all = new x10.util.ArrayList[SubIDed]();
    public def this(kind:Long) { 
       super(kind); 
    }
    @NoThisAccess
    def count(kind:Long) : Float {
       val subcount <: Long = ++subcounts(kind % 3);
       val supercount <: Float = kind2count(kind);
       //ERROR: val badSuperCount = super.count(kind); //(A)
       //ERROR: code = kind;                           //(B)
       //ERROR: all.add(this);                         //(C)
       return  supercount + 1.0f / subcount;
    }
  }
}
\end{xten}
%~~siv
%
%~~neg


\end{ex}

\subsection{Fine Structure of Constructors}
\label{SFineStructCtors}

The code of a constructor consists of four segments, three of them optional
and one of them implicit.
\begin{enumerate}
\item The first segment is an optional call to \Xcd{this(...)} or
      \Xcd{super(...)}.  If this is supplied, it must be the first statement
      of the constructor.  If it is not supplied, the compiler treats it as a
      nullary super-call \Xcd{super()};
\item If the class or struct has properties, there must be a single
      \Xcd{property(...)} command in the constructor, or a \xcd`this(...)`
      constructor call.  Every execution path
      through the constructor must go through this \Xcd{property(...)} command
      precisely once.   The second segment of the constructor is the code
      following the first segment, up to and including the \Xcd{property()}
      statement.

      If the class or struct has no properties, the \Xcd{property()} call must
      be omitted. If it is present, the second segment is defined as before.  If
      it is absent, the second segment is empty.
\item The third segment is automatically generated.  Fields with initializers
      are initialized immediately after the \Xcd{property} statement.
      In the following example, \Xcd{b} is initialized to \Xcd{y*9000} in
      segment three.  The initialization makes sense and does the right
      thing; \Xcd{b} will be \Xcd{y*9000} for every \Xcd{Overdone} object.
      (This would not be possible if field initializers were processed
      earlier, before properties were set.)
\item The fourth segment is the remainder of the constructor body.
\end{enumerate}

The segments in the following code are shown in the comments.
%~~gen ^^^ ObjectInitialization20
% package ObjectInitialization.ShowingSegments;
%~~vis
\begin{xten}
class Overlord(x:Long) {
  def this(x:Long) { property(x); }
}//Overlord
class Overdone(y:Long) extends Overlord  {
  val a : Long;
  val b =  y * 9000;
  def this(r:Long) {
    super(r);                      // (1)
    x10.io.Console.OUT.println(r); // (2)
    val rp1 = r+1;
    property(rp1);                 // (2)
    // field initializations here  // (3)
    a = r + 2 + b;                 // (4)
  }
  def this() {
    this(10);                      // (1), (2), (3)
    val x = a + b;                 // (4)
  }
}//Overdone
\end{xten}
%~~siv
%
%~~neg

The rules of what is allowed in the three segments are different, though
unsurprising.  For example, properties of the current class can only be read
in segment 3 or 4---naturally, because they are set at the end of segment 2.

\subsubsection{Initialization and Inner Classses}
\index{constructor!inner classes in}

Constructors of inner classes are tantamount to method calls on \Xcd{this}.
For example, the constructor for Inner {\bf is} acceptable.  It does not leak
\Xcd{this}.  It leaks \Xcd{Outer.this}, which is an utterly different object.
So, the call to \Xcd{this.new Inner()} in the \Xcd{Outer} constructor {\em
is} illegal.  It would leak \Xcd{this}.  There is no special rule in effect
preventing this; a constructor call of an inner class is no
different from a method as far as leaking is concerned.
%~~gen ^^^ ObjectInitialization30
% package ObjInit.InnerClass; 
% // NOTEST-packaging-issue
%~~vis
\begin{xten}
class Outer {
  static val leak : Cell[Outer] = new Cell[Outer](null);
  class Inner {
     def this() {Outer.leak.set(Outer.this);}
  }
  def /*Outer*/this() {
     //ERROR: val inner = this.new Inner();
  }
}
\end{xten}
%~~siv
%
%~~neg



\subsubsection{Initialization and Closures}
\index{constructor!closure in}

Closures in constructors may not refer to \xcd`this`.  They may not even refer
to fields of \xcd`this` that have been initialized.   For example, the
closure \xcd`bad1` is not allowed because it refers to \xcd`this`; \xcd`bad2`
is not allowed because it mentions \xcd`a` --- which is, of course, identical
to \xcd`this.a`. 

%%-deleted-%% valid if they were invoked (or inlined) at the
%%-deleted-%%place of creation. For example, \Xcd{closure} below is acceptable because it
%%-deleted-%%only refers to fields defined at the point it was written.  \Xcd{badClosure}
%%-deleted-%%would not be acceptable, because it refers to fields that were not defined at
%%-deleted-%%that point, although they were defined later.
%~~gen ^^^ ObjectInitialization40
% package ObjectInitialization.Closures; 
%~~vis
\begin{xten}
class C {
  val a:Long;
  def this() {
    this.a = 1;
    //ERROR: val bad1 = () => this; 
    //ERROR: val bad2 = () => a*10;
  }
}
\end{xten}
%~~siv
%
%~~neg


\subsection{Definite Initialization in Constructors}


An instance field \Xcd{var x:T}, when \Xcd{T} has a default value, need not be
explicitly initialized.  In this case, \Xcd{x} will be initialized to the
default value of type \Xcd{T}.  For example, a \Xcd{Score} object will have
its \Xcd{currently} field initialized to zero, below:
%~~gen ^^^ ObjectInitialization50
% package ObjectInit.DefaultInit;
%~~vis
\begin{xten}
class Score {
  public var currently : Long;
}
\end{xten}
%~~siv
%
%~~neg

All other sorts of instance fields do need to be initialized before they can
be used.  \Xcd{val} fields must be initialized in the constructor, even if
their type has a 
default value.  It would be silly to have a field \Xcd{val z : Long} that was
always given default value of \Xcd{0} and, since it is \Xcd{val}, can never be
changed.  \Xcd{var} fields whose type has no default value must be initialized
as well, such as \xcd`var y : Long{y != 0}`, since it cannot be assigned a
sensible initial value.

The fundamental principles are:
\begin{enumerate}
\item \Xcd{val} fields must be assigned precisely once in each constructor on every
possible execution path.
\item \Xcd{var} fields of defaultless type must be
assigned at least once on every possible execution path, but may be assigned
more than once.
\item No variable may be read before it is guaranteed to have been
assigned.
\item Initialization may be by field initialization expressions (\Xcd{val x :
      Long = y+z}), or by uninitialized fields \Xcd{val x : Long;} plus
an initializing assignment \Xcd{x = y+z}.  Recall that field initialization
expressions are performed after the \Xcd{property} statement, in segment 3 in
the terminology of \Sref{SFineStructCtors}.
\end{enumerate}



\subsection{Summary of Restrictions on Classes and Constructors}

The following table tells whether a given feature is (yes), is not (no) or is
with some conditions (note) allowed in a given context.   For example, a
property method is allowed with the type of another property, as long as it
only mentions the preceding properties. The first column of the table gives
examples, by line of the following code body.

\begin{tabular}{||l|l|c|c|c|c|c|c||}
\hline
~
  & {\bf Example}
  & {\bf Prop.}
  & {\bf {\tt \small self==this}(1)}
  & {\bf Prop.Meth.}
  & {\bf {\tt this}}
  & {\bf {fields}}
\\\hline
Type of property
  & (A)
  & %?properties
    yes (2)
  & no %? self==this
  & no %? property methods
  & no %? this may be used
  & no %? fields may be used
\\\hline
Class Invariant
  & (B)
  & yes %?properties
  & yes %? self==this
  & yes %? property methods
  & yes %? this may be used
  & no %? fields may be used
\\\hline
Supertype (3)
  & (C), (D)
  & yes%?properties
  & yes%? self==this
  & yes%? property methods
  & no%? this may be used
  & no%? fields may be used
\\\hline
Property Method Body
  & (E)
  & yes %?properties
  & yes %? self==this
  & yes %? property methods
  & yes %? this may be used
  & no %? fields may be used
\\\hline

Static field (4)
  & (F) (G)
  & no %?properties
  & no %? self==this
  & no %? property methods
  & no %? this may be used
  & no %? fields may be used
\\\hline

Instance field (5)
  & (H), (I)
  & yes %?properties
  & yes %? self==this
  & yes %? property methods
  & yes %? this may be used
  & yes %? fields may be used
\\\hline

Constructor arg. type
  & (J)
  & no %?properties
  & no %? self==this
  & no  %? property methods
  & no %? this may be used
  & no %? fields may be used
\\\hline

Constructor guard
  & (K)
  & no %?properties
  & no %? self==this
  & no %? property methods
  & no %? this may be used
  & no %? fields may be used
\\\hline

Constructor ret. type
  & (L)
  & yes %?properties
  & yes %? self==this
  & yes %? property methods
  & yes %? this may be used
  & yes %? fields may be used
\\\hline

Constructor segment 1
  & (M)
  & no%?properties
  & yes%? self==this
  & no%? property methods
  & no%? this may be used
  & no%? fields may be used
\\\hline


Constructor segment 2
  & (N)
  & no%?properties
  & yes%? self==this
  & no%? property methods
  & no%? this may be used
  & no%? fields may be used
\\\hline

Constructor segment 4
  & (O)
  & yes%?properties
  & yes%? self==this
  & yes%? property methods
  & yes%? this may be used
  & yes%? fields may be used
\\\hline

Methods
  & (P)
  & yes %?properties
  & yes %? self==this
  & yes %? property methods
  & yes %? this may be used
  & yes %? fields may be used
\\\hline



\iffalse
place
  & (pos)
  & %?properties
  & %? self==this
  & %? property methods
  & %? this may be used
  & %? fields may be used
\\\hline
\fi
\end{tabular}

Details:

\begin{itemize}
\item (1) {Top-level {\tt self} only.}
\item (2) {The type of the {$i^{th}$} property may only mention
                 properties {$1$} through {$i$}.}
\item (3) Super-interfaces follow the same rules as supertypes.
\item (4) The same rules apply to types and initializers.
\end{itemize}



The example indices refer to the following code:
%~~gen ^^^ ObjectInitialization60
% package ObjectInit.GrandExample;
% class Supertype[T]{}
% interface SuperInterface[T]{}
%~~vis
\begin{xten}
class Example (
   prop : Long,
   proq : Long{prop != proq},                    // (A)
   pror : Long
   )
   {prop != 0}                                  // (B)
   extends Supertype[Long{self != prop}]         // (C)
   implements SuperInterface[Long{self != prop}] // (D)
{
   property def propmeth() = (prop == pror);    // (E)
   static staticField
      : Cell[Long{self != 0}]                    // (F)
      = new Cell[Long{self != 0}](1);            // (G)
   var instanceField
      : Long {self != prop}                      // (H)
      = (prop + 1) as Long{self != prop};        // (I)
   def this(
      a : Long{a != 0},
      b : Long{b != a}                           // (J)
      )
      {a != b}                                  // (K)
      : Example{self.prop == a && self.proq==b} // (L)
   {
      super();                                  // (M)
      property(a,b,a);                          // (N)
      // fields initialized here
      instanceField = b as Long{self != prop};   // (O)
   }

   def someMethod() =
        prop + staticField() + instanceField;   // (P)
}
\end{xten}
%~~siv
%
%~~neg

\section{Method Resolution}
\index{method!resolution}
\index{method!which one will get called}
\label{sect:MethodResolution}

Method resolution is the problem of determining, statically, which method (or
constructor or operator)
should be invoked, when there are several choices that could be invoked.  For
example, the following class has two overloaded \xcd`zap` methods, one taking
an \Xcd{Any}, and the other a \Xcd{Resolve}.  Method resolution will figure
out that the call \Xcd{zap(1..4)} should call \xcd`zap(Any)`, and
\Xcd{zap(new Resolve())} should call \xcd`zap(Resolve)`.  

\begin{ex}
%~~gen ^^^ MethodResolution10
%package MethodResolution.yousayyouwantaresolution;
% // This depends on https://jira.codehaus.org/browse/XTENLANG-2696
%~~vis
\begin{xten}
class Res {
  public static interface Surface {}
  public static interface Deface {}

  public static class Ace implements Surface {
    public static operator (Boolean) : Ace = new Ace();
    public static operator (Place) : Ace = new Ace();
  }
  public static class Face implements Surface, Deface{}

  public static class A {}
  public static class B extends A {}
  public static class C extends B {}

  def m(x:A) = 0;
  def m(x:Long) = 1;
  def m(x:Boolean) = 2;
  def m(x:Surface) = 3;
  def m(x:Deface) = 4; 

  def example() {
     assert m(100) == 1 : "Long"; 
     assert m(new C()) == 0 : "C";
     // An Ace is a Surface, unambiguous best choice
     assert m(new Ace()) == 3 : "Ace";
     // ERROR: m(new Face());

     // The match must be exact.
     // ERROR: assert m(here) == 3 : "Place";

     // Boolean could be handled directly, or by 
     // implicit coercion Boolean -> Ace.
     // Direct matches always win.
     assert m(true) == 2 : "Boolean"; 
  }
\end{xten}
%~~siv
%  public static def main(argv:Rail[String]) {(new Res()).example(); Console.OUT.println("That's all!");}
% public def claim() { val ace : Ace = here; assert m(ace)==3; }
% }
% class Hook{ def run(){ (new Res()).example(); return true;} }
%~~neg

In the \xcd`"Long"` line, there is a very close match.  \xcd`100` is an
\xcd`Long`.  In fact, \xcd`100` is an \xcd`Long{self==100}`, so even in this
case the type of the actual parameter is not {\em precisely} equal to the type
of the method.

In the \xcd`"C"` line of the example, \xcd`new C()` is an instance of \xcd`C`,
which is a subtype of \xcd`A`, so the \xcd`A` method applies.  No other method
does, and so the \xcd`A` method will be invoked.

Similarly, in the \xcd`"Ace"` line, the \xcd`Ace` class implements
\xcd`Surface`, and so \xcd`new Ace()` matches the \xcd`Surface` method. 

However, a \xcd`Face` is both a \xcd`Surface` and a \xcd`Deface`, so there is
no unique best match for the invocation \xcd`m(new Face())`.  This invocation
would be forbidden, and a compile-time error issued.


The match must be exact.  There is an implicit coercion 
from \xcd`Place` to \xcd`Ace`, and \xcd`Ace` implements \xcd`Surface`, so the
code
\begin{xten}
val ace : Ace = here;
assert m(ace) == 3;
\end{xten}
works, by using the \xcd`Surface` form of \xcd`m`.  But doing it in one step
requires a deeper search than X10 performs\footnote{In general this search is
unbounded, so X10 can't perform it.}, and is not allowed.


For \xcd`m(true)`, both the \xcd`Boolean` and, with the implicit coercion,
\xcd`Ace` methods could apply.  Since the \xcd`Boolean` method applies
directly, and the \xcd`Ace` method requires an implicit coercion, this call
resolves to the \xcd`Boolean` method, without an error.

\end{ex}


The basic concept of method resolution is:
\begin{enumerate}
\item List all the methods that could possibly be used, inferring generic
      types but not performing implicit coercions.    
\item If one possible method is more specific than all the others, that one 
      is the desired method.
\item If there are two or more methods neither of which is more specific than
      the others, then the method invocation is ambiguous.  Method resolution
      fails and reports an error.
\item Otherwise, no possible methods were found without implicit coercions.
      Try the preceding steps again, but with coercions allowed: zero or one
      implicit coercion for each argument.  If a single
      most specific method is found with coercions, it is the desired method.
      If there are several, the invocation is ambiguous and erronious.
\item If no methods were found even with coercions, then the method invocation
      is undetermined.  Method resolution fails and reports an error.
\end{enumerate}

After method resolution is done, there is a validation phase that checks the
legality of the call, based on the \xcd`STATIC_CHECKS` compiler flag.  
With \xcd`STATIC_CHECKS`, the method's constraints must be satisfied; that is,
they must be entailed (\Sref{SemanticsOfConstraints}) by the information in
force at the point of the call.  With \xcd`DYNAMIC_CHECKS`, if the constraint
is not entailed at that point, a dynamic check is inserted to make sure that
it is true at runtime.

\noindent
In the presence of X10's highly-detailed type system, some subtleties arise. 
One point, at least, is {\em not} subtle. The same procedure is used, {\em
mutatis mutandis} for method, constructor, and operator resolution.  



\subsection{Space of Methods}

X10 allows some constructs, particularly \xcd`operator`s, to be defined in a
number of ways, and invoked in a number of ways. This section specifies which
forms of definition could correspond to a given definiendum.
%%OP%% , and (redundantly)
%%OP%% the syntax for invoking that definition unambiguously.  

Method invocations \xcd`a.m(b)`, where \xcd`a` is an expression, can be either
of the following forms.  There may be any number of arguments.
\begin{itemize}
\item An instance method on \xcd`a`, of the form \xcd`def m(B)`.
%%OP%% , so that the   invocation is \xcd`a.m(b)`;
\item A static method on \xcd`a`'s class, of the form \xcd`static def m(B)`.
%%OP%%       so that the invocation is \xcd`A.m(b)`.
\end{itemize}

The meaning of an invocation of the form \xcd`m(b)`, with any number of
arguments, depends slightly on its context.  Inside of a constraint, it might
mean \xcd`self.m(b)`.  Outside of a constraint, there is no \xcd`self`
defined, so it can't mean that.  The first of these that applies will be
chosen. 
\begin{enumerate}
\item Invoke a method on \xcd`this`, \viz{} \xcd`this.m(b)`.  Inside a
      constraint, it may also invoke a property method on \xcd`self`, \viz.
      \xcd`self.m(b)`.  It is an error if both \xcd`this.m(b)` and
      \xcd`self.m(b)` are possible.
\item Invoke a function named \xcd`m` in a local or field.
\item Construct a structure named \xcd`m`.
\end{enumerate}

Static method invocations, \xcd`A.m(b)`, where \xcd`A` is a container name,
can only be static.  There may be any number of arguments.
\begin{itemize}
\item A static method on \xcd`A`, of the form \xcd`static def m(B)`.
%%OP%%       the invocation is \xcd`A.m(b)`; 
\end{itemize}


Constructor invocations, \xcd`new A(b)`, must invoke constructors. There may
be any number of arguments. 
\begin{itemize}
\item A constructor on \xcd`A`, of the form \xcd`def this(B)`.
%%OP%% , so that the
%%OP%%       invocation is \xcd`new A(b)`.
\end{itemize}


A unary operator \xcdmath"$\star$ a" may be defined as: 
\begin{itemize}
\item An instance operator on \xcd`A`, defined as 
      \xcdmath"operator $\star$ this()".
%%OP%%       so that the invocation is 
%%OP%%       \xcdmath"a.operator $\star$()"; or
\item A static operator on \xcd`A`, defined as 
      \xcdmath"operator $\star$(a:A)".
%%OP%%       so that the invocation is 
%%OP%%       \xcdmath"A.operator $\star$(a)"
\end{itemize}

A binary operator \xcdmath"a $\star$ b" may be defined as: 
\begin{itemize}
\item An instance operator on \xcd`A`, defined as 
      \xcdmath"operator this $\star$(b:B)";
%%OP%%       so that the invocation is \xcdmath"a.operator $\star$(b)", 
or
\item A right-hand operator on \xcd`B`, defined as
      \xcdmath"operator (a:A) $\star$ this"; or
%%OP%%       so that the invocation is \xcdmath"b.operator ()$\star$(b)"

\item A static operator on \xcd`A`, defined as
      \xcdmath"operator (a:A) $\star$ (b:B)", 
%%OP%%       so that the invocation is \xcdmath"A.operator $\star$(a,b)"
; or
\item A static operator on \xcd`B`, if \xcd`A` and \xcd`B` are different
      classes, defined as
      \xcdmath"operator (a:A) $\star$ (b:B)"
%%OP%% , so that the invocation is 
%%OP%%       \xcdmath"B.operator $\star$(a,b)".
\end{itemize}
\noindent
If none of those resolve to a method, then either operand may be implicitly
coerced to the
other.  If one of the following two situations obtains, it will be done; if
both, the expression causes a static error.
\begin{itemize}
\item An implicit coercion from \xcd`A` to \xcd`B`, and 
      an operator \xcdmath"B $\star$ B" can be used, by 
      coercing \xcd`a` to be of type \xcd`B`, and then using \xcd`B`'s
      $\star$.  
\item An implicit coercion from \xcd`B` to \xcd`A`, and 
      an operator \xcdmath"A $\star$ A" can be used,
      coercing \xcd`b` to be of type \xcd`A`, and then using \xcd`A`'s
      $\star$.  
\end{itemize}

An application \xcd`a(b)`, for any number of arguments, can come from a number
of things. 
\begin{itemize}
\item an application operator on \xcd`a`, defined as \xcd`operator this(b:B)`;
%%OP%% , so that the 
%%OP%% invocation is \xcd`a.operator()(b)`
\item If \xcd`a` is an identifier, \xcd`a(b)` can also be a method invocation
      equivalent to \xcd`this.a(b)`, which  invokes \xcd`a` as
      either an instance or static method on \xcd`this`
\item If \xcd`a` is a qualified identifier, \xcd`a(b)` can also be an
      invocation of a struct constructor.
\end{itemize}


An indexed assignment, \xcd`a(b)=c`, for any number of \xcd`b`'s, can only
come from an indexed assignment definition: 
\begin{itemize}
\item \xcd`operator this(b:B)=(c:C) {...}`
%%OP%%       so that the invocation is \xcd`a.operator()=(b,c)`.
\end{itemize}

An implicit coercion, in 
which a value \xcd`a:A` is used in a context which requires a value of some
other non-subtype \xcd`B`, 
can only come from implicit coercion operation defined on
\xcd`B`: 
\begin{itemize}
\item an implicit coercion in \xcd`B`:
      \xcd`static operator (a:A):B`;
%%OP%%       so that the coercion is \xcd`B.operator[B](a)`;
\end{itemize}

An explicit conversion \xcd`a as B` can come from an explicit conversion
operator, or an implicit coercion operator.  X10 tries two things, in order,
only checking 2 if 1 fails: 
\begin{enumerate}
\item An \xcd`as` operator in \xcd`B`: 
      \xcdmath"static operator (a:A) as ?";
%%OP%%       so that the conversion is \xcd`B.operator as[B](a)`

\item or, failing that, an implicit coercion in \xcd`B`:
      \xcd`static operator (a:A):B`.
%%OP%% , so that the conversion is 
%%OP%%       \xcd`B.operator[B](a)`;

\end{enumerate}



\subsection{Possible Methods}

This section describes what it means for a method to be a {\em possible}
resolution of a method invocation.  



Generics introduce several subtleties, especially with the inference of
generic types. 
For the purposes of method resolution, all that matters about a method,
constructor, or operator \xcd`M` --- we use the word ``method'' to include all
three choices for this section --- is its signature, plus which method it is.
So, a typical \xcd`M` might look like 
\xcdmath"def m[G$_1$,$\ldots$, G$_g$](x$_1$:T$_1$,$\ldots$, x$_f$:T$_f$){c} =...".  The code body \xcd`...` is irrelevant for the purpose of whether a
given method call means \xcd`M` or not, so we ignore it for this section.

All that matters about a method definition, for the purposes of method
resolution, is: 
\begin{enumerate}
\item The method name \xcd`m`;
\item The generic type parameters of the method \xcd`m`,  \xcdmath"G$_1$,$\ldots$, G$_g$".  If there
      are no generic type parameters, {$g=0$}.  
\item The types \xcdmath"x$_1$:T$_1$,$\ldots$, x$_f$:T$_f$" of the formal parameters.  If
      there are no formal parameters, {$f=0$}. In the case of an instance
      method, the receiver will be the first formal parameter.\footnote{The
      variable names are relevant because one formal can be mentioned in a
      later type, or even a constraint: {\tt def f(a:Long, b:Point\{rank==a\})=...}.}
\item A {\em unique identifier} \xcd`id`, sufficient to tell the compiler
      which method body is intended.  A file name and position in that file
      would suffice.  The details of the identifier are not relevant.
\end{enumerate}

For the purposes of understanding method resolution, we assume that all the
actual parameters of an invocation are simply variables: \xcd`x1.meth(x2,x3)`.
This is done routinely by the compiler in any case; the code 
\xcd`tbl(i).meth(true, a+1)` would be treated roughly as 
\begin{xten}
val x1 = tbl(i);
val x2 = true;
val x3 = a+1;
x1.meth(x2,x3);
\end{xten}

All that matters about an invocation \xcd`I` is: 
\begin{enumerate}
\item The method name \xcdmath"m$'$";
\item The generic type parameters \xcdmath"G$'_1$,$\ldots$, G$'_g$".  If there
      are no generic type parameters, {$g=0$}.  
\item The names and types \xcdmath"x$_1$:T$'_1$,$\ldots$, x$_f$:T$'_f$" of the
      actual parameters.
      If
      there are no actual parameters, {$f=0$}. In the case of an instance
      method, the receiver is the first actual parameter.
\end{enumerate}

The signature of the method resolution procedure is: 
\xcd`resolve(invo : Invocation, context: Set[Method]) : MethodID`.  
Given a particular invocation and the set \xcd`context` of all methods
which could be called at that point of code, method resolution either returns
the unique identifier of the method that should be called, or (conceptually)
throws an exception if the call cannot be resolved.

The procedure for computing \xcd`resolve(invo, context)` is: 
\begin{enumerate}
\item Eliminate from \xcd`context` those methods which are not {\em
      acceptable}; \viz, those whose name, type parameters, and formal parameters
      do not suitably match \xcd`invo`.  In more detail:
      \begin{itemize}
      \item The method name \xcd`m` must simply equal the invocation name \xcdmath"m$'$";
      \item X10 infers type parameters, by an algorithm given in \Sref{TypeParamInfer}.
      \item The method's type parameters are bound to the invocation's for the
            remainder of the acceptability test.
      \item The actual parameter types must be subtypes of the formal
            parameter types, or be coercible to such subtypes.  Parameter $i$
            is a subtype if \xcdmath"T$'_i$ <: T$_i$".  It is implicitly
            coercible to a subtype if either it is a subtype, or if there is
            an implicit coercion operator 
            defined from \xcdmath"T$'_i$" to some type \xcd`U`, and 
            \xcdmath"U <: T$_i$". \index{method resolution!implicit coercions
            and} \index{implicit coercion}\index{coercion}.  If coercions are
            used to resolve the method, they will be called on the arguments
            before the method is invoked.
            
      \end{itemize}
\item Eliminate from \xcd`context` those methods which are not {\em
      available}; \viz, those which cannot be called due to visibility
      constraints, such as methods from other classes marked \xcd`private`.
      The remaining methods are both acceptable and available; they might be
      the one that is intended.
\item If the method invocation is a \xcd`super` invocation appearing in class
      \xcd`Cl`, methods of \xcd`Cl` and its subclasses are considered
      unavailable as well.
      
\item From the remaining methods, find the unique \xcd`ms` which is more specific than all the
      others, \viz, for which \xcd`specific(ms,mo) = true` for all other
      methods \xcd`mo`.
      The specificity test \xcd`specific` is given next.
      \begin{itemize}
      \item If there is a unique such \xcd`ms`, then
            \xcd`resolve(invo,context)` returns the \xcd`id` of \xcd`ms`.  
      \item If there is not a unique such \xcd`ms`, then \xcd`resolve` reports
            an error.
      \end{itemize}

\end{enumerate}

The subsidiary procedure \xcd`specific(m1, m2)` determines whether method
\xcd`m1` is equally or more specific than \xcd`m2`.  \xcd`specific` is not a
total order: is is possible for each one to be considered more specific than
the other, or either to be more specific.  \xcd`specific` is computed as: 
\begin{enumerate}
\item Construct an invocation \xcd`invo1` based on \xcd`m1`: 
      \begin{itemize}
      \item \xcd`invo1`'s method name is \xcd`m1`'s method name;
      \item \xcd`invo1`'s generic parameters are those of \xcd`m1`--- simply
            some type variables.
      \item \xcd`invo1`'s parameters are those of \xcd`m1`.
      \end{itemize}
\item If \xcd`m2` is acceptable for the invocation \xcd`invo1`,
      \xcd`specific(m1,m2)` returns true; 
\item Construct an invocation \xcd`invo2p`, which is \xcd`invo1` with the
      generic parameters erased.  Let \xcd`invo2` be \xcd`invo2p` with generic
      parameters as inferred by X10's type inference algorithm.  If type
      inference fails, \xcd`specific(m1,m2)` returns false.
\item If \xcd`m2` is acceptable for the invocation \xcd`invo2`,
      \xcd`specific(m1,m2)` returns true; 
\item Otherwise, \xcd`specific(m1,m2)` returns false.
\end{enumerate}

\subsection{Field Resolution}

An identifier \xcd`p` can refer to a number of things.  The rules are somewhat
different inside and outside of a constraint.

Outside of a constraint, the compiler chooses
the first one from the following list which applies: 
\begin{enumerate}
\item A local variable named \xcd`p`.
\item A field of \xcd`this`, \viz{} \xcd`this.p`.
\item A nullary property method, \xcd`this.p()`
\item A member type named \xcd`p`.
\item A package named \xcd`p`.
\end{enumerate}

Inside of a constraint, the rules are slightly different, because \xcd`self`
is available, and packages cannot be used per se.
\begin{enumerate}
\item A local variable named \xcd`p`.
\item A property of \xcd`this` or of \xcd`self`, \viz{} \xcd`this.p` or
      \xcd`self.p`.  If both are available, report an error.
\item A nullary property method, \xcd`this.p()`
\item A member type named \xcd`p`.
\end{enumerate}

\subsection{Other Disambiguations}
\label{sect:disambiguations}

It is possible to have a field of the same name as a method.
Indeed, it is a common pattern to have private field and a public
method of the same name to access it:
\begin{ex}
%~~gen ^^^ MethodResolution_disamb_a
%package MethodResolution_disamb_a;
%~~vis
\begin{xten}
class Xhaver {
  private var x: Long = 0;
  public def x() = x;
  public def bumpX() { x ++; }
}
\end{xten}
%~~siv
%
%~~neg
\end{ex}

\begin{ex}
However, this can lead to syntactic ambiguity in the case where the field
\Xcd{f} of object \xcd`a` is a
function, rail, array, list, or the like, and where \xcd`a` 
has a method also named
\xcd`f`.  The term \Xcd{a.f(b)} could either mean ``call method \xcd`f` of \xcd`a` upon
\xcd`b`'', or ``apply the function \xcd`a.f` to argument \xcd`b`''.  

%~~gen  ^^^ MethodResolution_disamb_b
%package MethodResolution_disamb_b;
%NOCOMPILE
%~~vis
\begin{xten}
class Ambig {
  public val f : (Long)=>Long =  (x:Long) => x*x;
  public def f(y:int) = y+1;
  public def example() {
      val v = this.f(10);
      // is v 100, or 11?
  }
}
\end{xten}
%~~siv
%
%~~neg
\end{ex}

In the case where a syntactic form \xcdmath"E.m(F$_1$, $\ldots$, F$_n$)" could
be resolved as either a method call, or the application of a field \xcd`E.m`
to some arguments, it will be treated as a method call.  
The application of \xcd`E.m` to some arguments can be specified by adding
parentheses:  \xcdmath"(E.m)(F$_1$, $\ldots$, F$_n$)".

\begin{ex}

%~~gen ^^^ MethodResolution_disamb_c
%package MethodResolution_disamb_c;
%NOCOMPILE
%~~vis
\begin{xten}
class Disambig {
  public val f : (Long)=>Long =  (x:Long) => x*x;
  public def f(y:int) = y+1;
  public def example() {
      assert(  this.f(10)  == 11  );
      assert( (this.f)(10) == 100 );
  }
}
\end{xten}
%~~siv
%
%~~neg

\end{ex}

Similarly, it is possible to have a method with the same name as a struct, say
\xcd`ambig`, giving an ambiguity as to whether \xcd`ambig()` is a struct
constructor invocation or a method invocation.  This ambiguity is resolved by
treating it as a method invocation.  If the constructor invocation is desired,
it can be achieved by including the optional \xcd`new`.  That is, 
\xcd`new ambig()` is struct constructor invocation; \xcd`ambig()` is a 
method invocation.

\section{Static Nested Classes}
\label{StaticNestedClasses}
\index{class!static nested}
\index{class!nested}
\index{static nested class}

One class (or struct or interface) may be nested within another.  The simplest
way to do this is as a \xcd`static` nested class, written by putting one class
definition at top level inside another, with the inner one having a
\xcd`static` modifier.  
For most purposes, a static nested class behaves like a top-level class.
However, a static nested class has access to private static
fields and methods of its containing class.  

Nested interfaces and static structs are permitted as well.

%~~gen ^^^ InnerClasses10
% package Classes.StaticNested; 
% NOTEST
%~~vis
\begin{xten}
class Outer {
  private static val priv = 1;
  private static def special(n:Long) = n*n;
  public static class StaticNested {
     static def reveal(n:Long) = special(n) + priv;
  }
}
\end{xten}
%~~siv
%
%~~neg

\section{Inner Classes}
\label{InnerClasses}
\index{class!inner}
\index{inner class}


Non-static nested classes are called {\em inner classes}. An inner class
instance can be thought of as a very elaborate member of an object --- one
with a full class structure of its own.   The crucial characteristic of an
inner class instance is that it has an implicit reference to an instance of
its containing class.  

\begin{ex}
This feature is particularly useful when an instance of the inner class makes
no sense without reference to an instance of the outer, and is closely tied to
it.  For example, consider a range class, describing a span of integers {$m$}
to {$n$}, and an iterator over the range.  The iterator might as well have
access to the range object, and there is little point to discussing
iterators-over-ranges without discussing ranges as well.
In the following example, the inner class \xcd`RangeIter` iterates over the
enclosing \xcd`Range`.  

It has its own private cursor field \xcd`n`, telling
where it is in the iteration; different iterations over the same \xcd`Range`
can exist, and will each have their own cursor.
It is perhaps unwise to use the name \xcd`n` for a field of the inner class,
since it is also a field of the outer class, but it is legal.  (It can happen
by accident as well -- \eg, if a programmer were to add a field \xcd`n` to a
superclass of the  outer class, the inner class would still work.)
It does not even
interfere with the inner class's ability to refer to the outer class's \xcd`n`
field: the cursor initialization 
refers to the \xcd`Range`'s lower bound through a fully qualified name
\xcd`Range.this.n`.
The initialization of its \xcd`n` field refers to the outer class's \xcd`m` field, which is
not shadowed and can be referred to directly, as \xcd`m`.


%~~gen ^^^ InnerClasses20
% package Classes.InnerClasses_a; 
% NOTEST
%~~vis
\begin{xten}
class Range(m:Long, n:Long) implements Iterable[Long]{
  public def iterator ()  = new RangeIter();
  private class RangeIter implements Iterator[Long] {
     private var n : Long = m;
     public def hasNext() = n <= Range.this.n;
     public def next() = n++;
  }
  public static def main(argv:Rail[String]) {
    val r = new Range(3,5);
    for(i in r) Console.OUT.println("i=" + i);
  }
}
\end{xten}
%~~siv
%
%~~neg
\end{ex}

An inner class has full access to the members of its enclosing class, both
static and instance.  In particular, it can access \xcd`private` information,
just as methods of the enclosing class can.  

An inner class can have its own members.  
Inside instance methods of an inner class, \xcd`this` refers to the instance
of the {\em inner} class.  The instance of the outer class can be accessed as
{\em Outer}\xcd`.this` (where {\em Outer} is the name of the outer class).
If, for some dire reason, it is necessary to have an inner class within an inner
class, the innermost class can refer to the \xcd`this` of either outer class
by using its name.

An inner class can inherit from any class in scope,
with no special restrictions. \xcd`super` inside an inner class refers to the
inner class's superclass. If it is necessary to refer to the outer classes's
superclass, use a qualified name of the form {\em Outer}\xcd`.super`.

The members of inner classes must be instance members.  They cannot be static
members.  Classes, interfaces, static methods, static fields, and typedefs are
not allowed as members of inner classes. 
The same restriction applies to local classes (\Sref{sect:LocalClasses}).

\index{inner class!extending}
Consider
an inner class \xcd`IC1` of some outer class \xcd`OC1`, being extended by 
another class \xcd`IC2`. However, since an \xcd`IC1` only exists as a
dependent of an \xcd`OC1`, each \xcd`IC2` must be associated with an \xcd`OC1`
--- or a subtype thereof --- as well.   So, \xcd`IC2` must be an inner class
of either \xcd`OC1` or some subclass \xcd`OC2 <: OC1`.

\begin{ex}For example, one often extends an
inner class when one extends its outer class: 
%~~gen ^^^ InnerClasses30
% package Classes.Innerclasses.Are.For.Innermasses;
%~~vis
\begin{xten}
class OC1 {
   class IC1 {}
}
class OC2 extends OC1 {
   class IC2 extends IC1 {} 
}
\end{xten}
%~~siv
%
%~~neg
\end{ex}


The hiding of method names has one fine point.  If an inner class defines a
method named \xcd`doit`, then {\em all} methods named \xcd`doit` from the
outer class are hidden --- even if they have different argument types than the
one defined in the inner class.
They are still accessible via
\xcd`Outer.this.doit()`, but not simply via \xcd`doit()`.  The following code
is correct, but would not be correct if the ERROR line were uncommented.

%~~gen ^^^ InnerClasses40
% package Classes.Innerclasses.StupidOverloading; 
% NOTEST
%~~vis
\begin{xten}
class Outer {
  def doit() {}
  def doit(String) {}
  class Inner { 
     def doit(Boolean, Outer) {}
     def example() {
        doit(true, Outer.this);
        Outer.this.doit();
        //ERROR: doit("fails");
     }
  }
}
\end{xten}
%~~siv
%
%~~neg


\subsection{Constructors and Inner Classes}
\label{sect:InnerClassCtor}
\index{inner class!constructor}

If \xcd`IC` is an inner class of \xcd`OC`, then instance code in the body of
\xcd`OC` can create instances of \xcd`IC` simply by calling a constructor
\xcd`new IC(...)`: 
%~~gen ^^^ InnerClasses50
% package Classes.Innerclasses.Constructors.Easy;
%~~vis
\begin{xten}
class OC {
  class IC {}
  def method(){
    val ic = new IC();
  }
}
\end{xten}
%~~siv
%
%~~neg

Instances of \xcd`IC` can be constructed from elsewhere as well.  Since every
instance of \xcd`IC` is associated with an instance of \xcd`OC`, an \xcd`OC`
must be supplied to the \xcd`IC` constructor.  The syntax for doing so is: 
\xcd`oc.new IC()`.  For example: 
%~~gen ^^^ InnerClasses60
% package Classes.Inner_a; 
% NOTEST
% /*NONSTATIC*/
%~~vis
\begin{xten}
class OC {
  class IC {}
  static val oc1 = new OC();
  static val oc2 = new OC();
  static val ic1 = oc1.new IC();
  static val ic2 = oc2.new IC();
}
class Elsewhere{
  def method(oc : OC) {
    val ic = oc.new IC();
  }
}
\end{xten}
%~~siv
%
%~~neg


\section{Local Classes}
\label{sect:LocalClasses}

Classes can be defined and instantiated in the middle of methods and other
code blocks.
A local class in a static method is a static class; a local class in an
instance method is an inner class.
 Local classes are local to the block in which they are defined.
They have access to almost everything defined at that point in the method; the
one exception is that they cannot use \xcd`var` variables. Local classes
cannot be \xcd`public`, \xcd`protected`, or \xcd`private`, because they are
only visible from within the block of declaration. They cannot be
\xcd`static`.

\begin{ex}
The following example illustrates the use of a local class \xcd`Local`, 
defined inside the body of method \xcd`m()`. 
%~~gen ^^^ InnerClasses5p9v
% package InnerClasses5p9v;
% NOTEST
%~~vis
\begin{xten}
class Outer {
  val a = 1;
  def m() {
    val a = -2; 
    val b = 2;
    class Local {
      val a = 3;
      def m() = 100*Outer.this.a + 10*b + a; 
    }
    val l : Local = new Local();
    assert l.m() == 123;
  }//end of m()
}
\end{xten}
%~~siv
% class Hook{ def run() {
%   val o <: Outer = new Outer();
%   o.m();
%   return true;
% } }
%~~neg
Note that the middle \xcd`a`,
whose value is \xcd`-2`, is not accessible inside of \xcd`Local`; it is
shadowed by \xcd`Local`'s \xcd`a` field.  \xcd`Outer`'s \xcd`a` is also
shadowed, but the notation \xcd`Outer.this` gives a reference to the enclosing
\xcd`Outer` object.  There is no corresponding notation to access shadowed local
variables from the enclosing block; if you need to get them, rename the fields
of \xcd`Local`.    
\end{ex}


The members of inner classes must be instance members.  They cannot be static
members.  Classes, interfaces, static methods, static fields, and typedefs are
not allowed as members of local classes. 
The same restriction applies to inner classes (\Sref{InnerClasses}). 





\section{Anonymous Classes}
\index{class!anonymous}
\index{anonymous class}

It is possible to define a new local class and instantiate it as part of an
expression.  The new class can extend an existing class or interface.  Its body
can include all of the usual members of a local class. It can refer to any
identifiers available at that point in the expression --- except for \xcd`var`
variables.  An anonymous class in a static context is a static inner class.

Anonymous classes are useful when you want to package several pieces of
behavior together (a single piece of behavior can often be expressed as a
function, which is syntactically lighter-weight), or if you want to extend and
vary an extant class without going through the trouble of actually defining a
whole new class.

The syntax for an anonymous class is a constructor call followed immediately
by a braced class body: \xcd`new C(1){def foo()=2;}`.

\begin{ex}In the following minimalist example, the abstract class \xcd`Choice`
encapsulates a decision.   A \xcd`Choice` has a \xcd`yes()` and a \xcd`no()`
method.  The \xcd`choose(b)` method will invoke one of the two.  \xcd`Choice`s
also have names.

The \xcd`main()` method creates a specific \xcd`Choice`.  \xcd`c` is not a
immediate instance of \xcd`Choice` --- as an abstract class, \xcd`Choice` has
no immediate instances. \xcd`c` is an instance of an anonymous class which
inherits from \xcd`Choice`, but supplies \xcd`yes()` and \xcd`no()` methods.
These methods modify the contents of the \xcd`Cell[Long]` \xcd`n`.  (Note that,
as \xcd`n` is a local variable, it would take a few lines more coding to
extract \xcd`c`'s class, name it, and make it an inner class.)  The call to
\xcd`c.choose(true)`  will call \xcd`c.yes()`, incrementing \xcd`n()`, in a
rather roundabout manner.

%~~gen ^^^ InnerClasses70
% package ClassInnnerclassAnonclassOw; 
%~~vis
\begin{xten}
abstract class Choice(name: String) {
  def this(name:String) {property(name);}
  def choose(b:Boolean) { 
     if (b) this.yes(); else this.no(); }
  abstract def yes():void;
  abstract def no():void;
}

class Example {
  static def main(Rail[String]) {
    val n = new Cell[Long](0);
    val c = new Choice("Inc Or Dec") {
      def yes() { n() += 1; }
      def no()  { n() -= 1; }
      };
    c.choose(true);
    Console.OUT.println("n=" + n());
  }
}

\end{xten}
%~~siv
%
%~~neg
\end{ex}

Anonymous classes have many of the features of classes in general.  A few
features are unavailable because they don't make sense.

\begin{itemize}

\item Anonymous classes don't have constructors.  Since they don't have names,
      there's no way a constructor could get called in the ordinary way.
      Instead, the \xcd`new C(...)` expression must match a constructor of the
      parent class \xcd`C`, which will be called to initialize the
      newly-created object of the anonymous class.

\item The \xcd`public`,
      \xcd`private`, and \xcd`protected`  modifiers don't make sense for
      anonymous classes:  
      Anonymous classes, being anonymous,
      cannot be referenced at all, so references to them can't be public,
      private, or protected.

\item Anonymous classes cannot be \xcd`abstract`.  Since they only exist in
      combination with a constructor call, they must be constructable.  The
      parent class of the anonymous class may be abstract, or may be an
      interface; in this case, the anonymous class must provide all the
      methods that the parent demands.

\item Anonymous classes cannot have explicit \xcd`extends` or \xcd`implements`
      clauses; there's no place in the syntax for them. They have a single
      parent and that is that. 
\end{itemize}
