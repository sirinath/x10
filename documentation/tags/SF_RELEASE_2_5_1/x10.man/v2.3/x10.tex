%\documentclass[10pt,twoside,twocolumn,notitlepage]{report}
%\documentclass[12pt,twoside,notitlepage]{report}
\documentclass[10pt,twoside,notitlepage]{report}
\usepackage{tex/x10}
\usepackage{tex/tenv}
\def\Hat{{\tt \char`\^}}
\usepackage{url}
\usepackage{times}
\usepackage{tex/txtt}
\usepackage{ifpdf}
\usepackage{tocloft}
\usepackage{tex/bcprules}
\usepackage{xspace}
\usepackage{makeidx}

\newif\ifdraft
%\drafttrue
\draftfalse

\pagestyle{headings}
\showboxdepth=0
\makeindex

\usepackage{tex/commands}

\usepackage[
pdfauthor={Vijay Saraswat, Bard Bloom, Igor Peshansky, Olivier Tardieu, and David Grove},
pdftitle={X10 Language Specification},
pdfcreator={pdftex},
pdfkeywords={X10},
linkcolor=blue,
citecolor=blue,
urlcolor=blue
]{hyperref}

\ifpdf
          \pdfinfo {
              /Author   (Vijay Saraswat, Bard Bloom, Igor Peshansky, Olivier Tardieu, and David Grove)
              /Title    (X10 Language Specification)
              /Keywords (X10)
              /Subject  ()
              /Creator  (TeX)
              /Producer (PDFLaTeX)
          }
\fi

\def\headertitle{The \XtenCurrVer{} Report}
\def\integerversion{2.3}

% Sizes and dimensions

%\topmargin -.375in       %    Nominal distance from top of page to top of
                         %    box containing running head.
%\headsep 15pt            %    Space between running head and text.

%\textheight 9.0in        % Height of text (including footnotes and figures, 
                         % excluding running head and foot).

%\textwidth 5.5in         % Width of text line.
\columnsep 15pt          % Space between columns 
\columnseprule 0pt       % Width of rule between columns.

\parskip 5pt plus 2pt minus 2pt % Extra vertical space between paragraphs.
\parindent 0pt                  % Width of paragraph indentation.
%\topsep 0pt plus 2pt            % Extra vertical space, in addition to 
                                % \parskip, added above and below list and
                                % paragraphing environments.


\newif\iftwocolumn

\makeatletter
\twocolumnfalse
\if@twocolumn
\twocolumntrue
\fi
\makeatother

\iftwocolumn

\oddsidemargin  0in    % Left margin on odd-numbered pages.
\evensidemargin 0in    % Left margin on even-numbered pages.

\else

\oddsidemargin  .5in    % Left margin on odd-numbered pages.
\evensidemargin .5in    % Left margin on even-numbered pages.

\fi


\newtenv{example}{Example}[section]
\newtenv{planned}{Planned}[section]

\begin{document}

% \section{Work In Progress}
% \begin{itemize}
%     \item Rewrite first chapter
%     \item Describe library classes, including such fundamentals as Any and String
%     \item Examples for covariant/contravariant generics are wrong -- use Nate's examples
%     \item Describe local classes.
%     \item Reduce the use of \xcd`self` in constraints.
%     \item Copy sections of grammar to relevant sections of text.
%     \item Do something about 4.12.3
% \end{itemize}
% 
% {\bf Feedback:} 
% To help us the most, we would appreciate comments in one of these formats: 
% \begin{itemize}
% \item An annotated copy of the PDF document, if it's convenient.  Acrobat
%       Writer can produce helpful highlighting and sticky notes.  If you don't
%       use Acrobat Writer, don't fuss.
% \item Text comments.  Since the document is still being edited, page numbers
%       are going to be useless as pointers to the text.  If possible, we'd like
%       pointers to sections by number and title: {\em In 12.1, ``Empty
%       Statement'', please discuss side effects and performance implications
%       for this construct''}  If it's a long section, giving us a couple words
%       we can grep for would help too.
% \end{itemize}
% 
% Thank you very much!




% \parindent 0pt %!! 15pt                    % Width of paragraph indentation.

%\hfil {\bf 7 Feb 2005}
%\hfil \today{}

% First page

\thispagestyle{empty}

% \todo{"another" report?}

\topnewpage[{
\begin{center}   
{\huge\bf Report on the Experimental Language \Xten{}}
\vskip 1ex
$$
\begin{tabular}{l@{\extracolsep{.5in}}lll}
\multicolumn{4}{c}{\sc  Version 1.1}\\
\multicolumn{4}{c}{\sc Please send comments to 
V\authorsc{IJAY} S\authorsc{ARASWAT} at 
{\tt vsaraswa@us.ibm.com}}\\
%\multicolumn{4}{c}{({\sc IBM Confidential})}

%\ldots
\end{tabular}
$$
\vskip 2ex
% {\it Dedicated to the Memory of APL} % vj
{\bf Jun 30, 2007}
\vskip 2.6ex
\end{center}


}]


\chapter*{Summary}
This draft report provides an initial description of the programming
language \Xten. \Xten{} is a single-inheritance class-based object-oriented
(OO) programming language designed for high-performance, high-productivity
computing on high-end computers supporting $\approx 10^5$ hardware threads
and $\approx 10^{15}$ operations per second. 

{}\Xten{} is based on state-of-the-art object-oriented programming
languages and deviates from them only as necessary to support its
design goals. The language is intended to have a simple and clear
semantics and be readily accessible to mainstream OO programmers. It
is intended to support a wide variety of concurrent programming
idioms.
%, incuding data parallelism, task parallelism, pipelining.
%producer/consumer and divide and conquer.

%We expect to revise this document in the light of experience gained in implementing
%and using this language.

The \Xten{} design team consists of
D\authorsc{AVID} B\authorsc{ACON}, 
R\authorsc{AJ} B\authorsc{ARIK}, 
G\authorsc{ANESH} B\authorsc{IKSHANDI}, 
B\authorsc{OB} B\authorsc{LAINEY}, 
P\authorsc{HILIPPE} C\authorsc{HARLES}, 
P\authorsc{ERRY} C\authorsc{HENG}, 
C\authorsc{HRISTOPHER} D\authorsc{ONAWA}, 
J\authorsc{ULIAN} D\authorsc{OLBY}, 
K\authorsc{EMAL} E\authorsc{BCIO\u{G}LU},
R\authorsc{OBERT} F\authorsc{UHRER},
P\authorsc{ATRICK} G\authorsc{ALLOP}, 
C\authorsc{HRISTIAN} G\authorsc{ROTHOFF}, 
A\authorsc{LLAN} K\authorsc{IELSTRA}, 
S\authorsc{REEDHAR} K\authorsc{ODALI}, 
S\authorsc{RIRAM} K\authorsc{RISHNAMOORTHY}, 
N\authorsc{ATHANIEL} N\authorsc{YSTROM}, 
I\authorsc{IGOR} P\authorsc{ESHANSKY}, 
V\authorsc{IJAY} S\authorsc{ARASWAT} (contact author), 
V\authorsc{IVEK} S\authorsc{ARKAR},
A\authorsc{RMANDO} S\authorsc{OLAR-LEZAMA},  
C\authorsc{HRISTOPH von} P\authorsc{RAUN},
P\authorsc{RADEEP} V\authorsc{ARMA},
K\authorsc{RISHNA} V\authorsc{ENKATA},
J\authorsc{AN} V\authorsc{ITEK}, and
T\authorsc{ONG} W\authorsc{EN}.

For extended discussions and support we would like to thank: 
Robert Callahan, Calin
Cascaval, Norman Cohen, Elmootaz Elnozahy, John Field, Bob Fuhrer,
Orren Krieger, Doug Lea, John McCalpin, Paul McKenney, Ram Rajamony,
R.K.~Shyamasundar, Filip Pizlo, V.T.~Rajan, Frank Tip, and Mandana Vaziri.

We thank Jonathan Rhees and William Clinger with help in obtaining the
\LaTeX{} style file and macros used in producing the Scheme report,
after which this document is based. We acknowledge the influence of
the $\mbox{\Java}^{\mbox{TM}}$ Language Specification \cite{jls2}
document, as evidenced by the numerous citations in the text.

This document revises Version {\cf 1.01} of the Report, released in
December 2006. It documents the language corresponding to the first
revision of the first version of the implementation.  This
revision was done by
R\authorsc{AJ} B\authorsc{ARIK}, 
P\authorsc{HILIPPE} C\authorsc{HARLES}, 
C\authorsc{HRISTOPHER} D\authorsc{ONAWA}, 
R\authorsc{OBERT} F\authorsc{UHRER},
N\authorsc{ATHANIEL} N\authorsc{YSTROM},  
V\authorsc{IJAY} S\authorsc{ARASWAT},
V\authorsc{IVEK} S\authorsc{ARKAR},
P\authorsc{RADEEP} V\authorsc{ARMA} and
K\authorsc{RISHNA} V\authorsc{ENKATA}.
(Earlier implementations benefited from significant contributions by
C\authorsc{HRISTIAN} G\authorsc{ROTHOFF} and 
C\authorsc{HRISTOPH von} P\authorsc{RAUN}.)
T\authorsc{ONG} W\authorsc{EN} has written many application programs
in \Xten{}. G\authorsc{UOJING} C\authorsc{ONG} has helped in the
development of many applications.


%\vfill
%\begin{center}
%{\large \bf
%*** DRAFT*** \\
%%August 31, 1989
%\today
%}\end{center}

\vfill
\eject


\chapter*{Contents}
\addvspace{3.5pt}                  % don't shrink this gap
\renewcommand{\tocshrink}{-3.5pt}  % value determined experimentally
{\footnotesize
\tableofcontents
}

\vfill
\eject


 

\clearpage

{\parskip 0pt
\addtolength{\cftsecnumwidth}{0.5em}
\addtolength{\cftsubsecnumwidth}{0.5em}
%\addtolength{\cftsecindent}{0.5em}
\addtolength{\cftsubsecindent}{0.5em}
\tableofcontents
}


\chapter{Introduction}

\subsection*{Background}
Larger computational problems require more powerful computers capable of
performing a larger number of operations per second. The era of
increasing performance by simply increasing clocking frequency now
seems to be behind us. It is becoming increasingly difficult
to mange chip power and heat.  Instead, computer
designers are starting to look at {\em scale out} systems in which the
system's computational capacity is increased by adding additional
nodes of comparable power to existing nodes, and connecting nodes with
a high-speed communication network.

A central problem with scale out systems is a definition of the {\em
memory model}, that is, a model of the interaction between shared
memory and  simultaneous (read, write) operations on that
memory by multiple processors. The traditional ``one operation at a
time, to completion'' model that underlies Lamport's notion of {\em
sequential consistency} (SC) proves too expensive to implement in
hardware, at scale. Various models of {\em relaxed consistency} have
proven too difficult for programmers to work with.  

One response to this problem has been to move to a {\em fragmented
memory model}. Multiple processors are made to interact via a
relatively language-neutral message-passing format such as MPI
\cite{mpi}. This model has enjoyed some success: several
high-performance applications have been written in this
style. Unfortunately, this model leads to a {\em loss of programmer
productivity}: the message-passing format is integrated into the host
language by means of an application-programming interface (API), the
programmer must explicitly represent and manage the interaction
between multiple processes and choreograph their data exchange; large
data-structures (such as distributed arrays, graphs, hash-tables) that
are conceptually unitary must be thought of as fragmented across
different nodes; all processors must generally execute the same code
(in an SPMD fashion) etc.

One response to this problem has been the advent of the {\em
partitioned global address space} (PGAS) model underlying languages
such as UPC, Titanium and Co-Array Fortran \cite{pgas,titanium}. These
languages permit the programmer to think of a single computation
running across multiple processors, sharing a common address
space. All data resides at some processors, which is said to have {\em
affinity} to the data.  Each processor may operate directly on the
data it contains but must use some indirect mechanism to access or
update data at other processors. Some kind of global {\em barriers}
are used to ensure that processors remain roughly in lock-step.

\Xten{} is a modern object-oriented programming language
in the PGAS family. The fundamental goal of \Xten{} is to enable
scalable, high-performance, high-productivity transformational
programming for high-end computers---for traditional numerical
computation workloads (such as weather simulation, molecular dynamics,
particle transport problems etc) as well as commercial server
workloads.

\Xten{} is based on state-of-the-art object-oriented
programming ideas primarily to take advantage of their proven
flexibility and ease-of-use for a wide spectrum of programming
problems. \Xten{} takes advantage of several years of research (e.g.,
in the context of the Java Grande forum,
\cite{moreira00java,kava}) on how to adapt such languages to the context of
high-performance numerical computing. Thus \Xten{} provides support
for user-defined {\em struct types} (such as \xcd"Int", \xcd"Float",
\xcd"Complex" etc), supports a very
flexible form of multi-dimensional arrays (based on ideas in ZPL
\cite{zpl}) and supports IEEE-standard floating point arithmetic.
Some capabilities for supporting operator overloading are also provided.

{}\Xten{} introduces a flexible treatment of concurrency, distribution
and locality, within an integrated type system. \Xten{} extends the
PGAS model with {\em asynchrony} (yielding the {\em APGAS} programming
model). {}\Xten{} introduces {\em places} as an abstraction for a
computational context with a locally synchronous view of shared
memory. An \Xten{} computation runs over a large collection of places.
Each place hosts some data and runs one or more {\em
activities}. Activities are extremely lightweight threads of
execution. An activity may synchronously (and {\em atomically}) use
one or more memory locations in the place in which it resides,
leveraging current symmetric multiprocessor (SMP) technology.  
To access or update memory at other places, it must 
spawn activities asynchronously (either explicitly or implicitly).
\Xten{} provides weaker ordering guarantees for
inter-place data access, enabling applications to scale.  {\em
Immutable} data needs no consistency management and may be freely
copied by the implementation between places.  One or more {\em clocks}
may be used to order activities running in multiple
places.  Arrays may be distributed across multiple
places. Arrays support parallel collective operations. A novel
exception flow model ensures that exceptions thrown by asynchronous
activities can be caught at a suitable parent activity.  The type
system tracks which memory accesses are local. The programmer may
introduce place casts which verify the access is local at run time.
Linking with native code is supported.

\XtenCurrVer builds on v1.7 to support the following features: {\em
  structs} (i.e., ``header-less'', inlinable objects), type rules for
preventing escape of \xcd{this} from a constructor, 
the introduction of a global object model, permitting user-specified
(immutable) fields to be replicated with the object reference.
\xcd{value} classes are no longer supported; their functionality is
accomplished by using structs or global fields and methods.


Several representative idioms for concurrency and communication have
already found pleasant expression in \Xten. We intend to develop
several full-scale applications to get better experience with the
language, and revisit the design in the light of this experience.


\chapter{Overview of \Xten}

\Xten{} is a statically typed object-oriented language, extending a sequential
core language with \emph{places}, \emph{activities}, \emph{clocks},
(distributed, multi-dimensional) \emph{arrays} and \emph{struct} types. All
these changes are motivated by the desire to use the new language for
high-end, high-performance, high-productivity computing.

\section{Object-oriented features}

The sequential core of \Xten{} is a {\em container-based} object-oriented language
similar to \java{} and C++, and more recent languages such as Scala.  
Programmers write \Xten{} code by defining containers for data and behavior
called 
\emph{classes}
(\Sref{XtenClasses}) and
\emph{structs}
(\Sref{XtenStructs}), 
often abstracted as 
\emph{interfaces}
(\Sref{XtenInterfaces}).
X10 provides inheritance and subtyping in fairly traditional ways. 

\begin{ex}

\xcd`Normed` describes entities with a \xcd`norm()` method. \xcd`Normed` is
intended to be used for entities with a position in some coordinate system,
and \xcd`norm()` gives the distance between the entity and the origin. A
\xcd`Slider` is an object which can be moved around on a line; a
\xcd`PlanePoint` is a fixed position in a plane. Both \xcd`Slider`s and
\xcd`PlanePoint`s have a sensible \xcd`norm()` method, and implement
\xcd`Normed`.

%~~gen ^^^ Overview10
% package Overview;
%~~vis
\begin{xten}
interface Normed {
  def norm():Double;
}
class Slider implements Normed {
  var x : Double = 0;
  public def norm() = Math.abs(x);
  public def move(dx:Double) { x += dx; }
}
struct PlanePoint implements Normed {
  val x : Double; val y:Double;
  public def this(x:Double, y:Double) {
    this.x = x; this.y = y;
  }
  public def norm() = Math.sqrt(x*x+y*y);
}
\end{xten}
%~~siv
%
%~~neg
\end{ex}

\paragraph{Interfaces}

An \Xten{} interface specifies a collection of abstract methods; \xcd`Normed`
specifies just \xcd`norm()`. Classes and
structs can be specified to {\em implement} interfaces, as \xcd`Slider` and
\xcd`PlanePoint` implement \xcd`Normed`, and, when they do so, must provide
all the methods that the interface demands.

Interfaces are
purely abstract. Every value of type \xcd`Normed` must be an instance of some
class like \xcd`Slider` or some struct like \xcd`PlanePoint` which implements
\xcd`Normed`; no value can be \xcd`Normed` and nothing else. 


\paragraph{Classes and Structs}



There are two kinds of containers: \emph{classes}
(\Sref{ReferenceClasses}) and \emph{structs} (\Sref{Structs}). Containers hold
data in {\em fields}, and give concrete implementations of 
methods, as \xcd`Slider` and \xcd`PlainPoint` above.

Classes are organized in a single-inheritance tree: a class may have only a
single parent class, though it may implement many interfaces and have many
subclasses. Classes may have mutable fields, as \xcd`Slider` does.

In contrast, structs are headerless values, lacking the internal organs
which give objects their intricate behavior.  This makes them less powerful
than objects (\eg, structs cannot inherit methods, though objects can), but also
cheaper (\eg, they can be inlined, and they require less space than objects).  
Structs are immutable, though their fields may be immutably set to objects
which are themselves mutable.  They behave like objects in all ways consistent
with these limitations; \eg, while they cannot {\em inherit} methods, they can
have them -- as \xcd`PlanePoint` does.

\Xten{} has no primitive classes per se. However, the standard library
\xcd"x10.lang" supplies structs and objects \xcd"Boolean", \xcd"Byte",
\xcd"Short", \xcd"Char", \xcd"Int", \xcd"Long", \xcd"Float", \xcd"Double",
\xcd"Complex" and \xcd"String". The user may defined additional arithmetic
structs using the facilities of the language.



\paragraph{Functions.}

X10 provides functions (\Sref{Closures}) to allow code to be used
as values.  Functions are first-class data: they can be stored in lists,
passed between activities, and so on.  \xcd`square`, below, is a function
which squares an \xcd`Long`.  \xcd`of4` takes an \xcd`Long`-to-\xcd`Long`
function and applies it to the number \xcd`4`.  So, \xcd`fourSquared` computes
\xcd`of4(square)`, which is \xcd`square(4)`, which is 16, in a fairly
complicated way.
%~~gen ^^^ Overview20
% package Overview.of.Functions.one;
% class Whatever{
% def chkplz() {
%~~vis
\begin{xten}
  val square = (i:Long) => i*i;
  val of4 = (f: (Long)=>Long) => f(4);
  val fourSquared = of4(square);
\end{xten}
%~~siv
%}}
%~~neg



Functions are used extensively in X10
programs.  For example, a common way to construct and initialize an \xcd`Rail[Long]` --
that is, a fixed-length one-dimensional array of numbers, like an \xcd`long[]` in \java{} -- is to
pass two arguments to a factory method: the first argument being the length of
the rail, and the second being a function which computes the initial value of
the \xcd`i`{$^{th}$} element.  The following code constructs a 1-dimensional
rail 
initialized to the squares of 0,1,...,9: \xcd`r(0) == 0`, \xcd`r(5)==25`, etc. 
%~~gen ^^^ Overview30
% package Overview.of.Functions.two;
% class Whatevermore {
%  def plzchk(){
%    val square = (i:Long) => (i*i);
%~~vis
\begin{xten}
  val r : Rail[Long] = new Rail[Long](10, square);
\end{xten}
%~~siv
%}}
%~~neg








\paragraph{Constrained Types}

X10 containers may declare {\em properties}, which are fields bound immutably
at the creation of the container.  The static analysis system understands
properties, and can work with them logically.   


For example, an implementation of matrices \xcd`Mat` might have the numbers of
rows and columns as properties.  A little bit of care in definitions allows
the definition of a \xcd`+` operation that works on matrices of the same
shape, and \xcd`*` that works on matrices with appropriately matching shapes.
%~~gen ^^^ Overview40
%package Overview.Mat2;
%~~vis
\begin{xten}
abstract class Mat(rows:Long, cols:Long) {
 static type Mat(r:Long, c:Long) = Mat{rows==r&&cols==c};
 abstract operator this + (y:Mat(this.rows,this.cols))
                 :Mat(this.rows, this.cols);
 abstract operator this * (y:Mat) {this.cols == y.rows} 
                 :Mat(this.rows, y.cols);
\end{xten}
%~~siv
%  static def makeMat(r:Long,c:Long) : Mat(r,c) = null;
%  static def example(a:Long, b:Long, c:Long) {
%    val axb1 : Mat(a,b) = makeMat(a,b);
%    val axb2 : Mat(a,b) = makeMat(a,b);
%    val bxc  : Mat(b,c) = makeMat(b,c);
%    val axc  : Mat(a,c) = (axb1 +axb2) * bxc;
%  }
%}
%~~neg



The following code typechecks (assuming that \xcd`makeMat(m,n)` is a function
which creates an \xcdmath"m$\times$n" matrix).
However, an attempt to compute \xcd`axb1 + bxc` or
\xcd`bxc * axb1` would result in a compile-time type error:
%~~gen ^^^ Overview50
%package Overview.Mat1;
%//OPTIONS: -STATIC_CHECKS
%abstract class Mat(rows:Long, cols:Long) {
%  static type Mat(r:Long, c:Long) = Mat{rows==r&&cols==c};
%  public def this(r:Long, c:Long) : Mat(r,c) = {property(r,c);}
%  static def makeMat(r:Long,c:Long) : Mat(r,c) = null;
%  abstract  operator this + (y:Mat(this.rows,this.cols)):Mat(this.rows, this.cols);
%  abstract  operator this * (y:Mat) {this.cols == y.rows} : Mat(this.rows, y.cols);
%~~vis
\begin{xten}
  static def example(a:Long, b:Long, c:Long) {
    val axb1 : Mat(a,b) = makeMat(a,b);
    val axb2 : Mat(a,b) = makeMat(a,b);
    val bxc  : Mat(b,c) = makeMat(b,c);
    val axc  : Mat(a,c) = (axb1 +axb2) * bxc;
    //ERROR: val wrong1 = axb1 + bxc;
    //ERROR: val wrong2 = bxc * axb1;
  }

\end{xten}
%~~siv
%}
%~~neg

The ``little bit of care'' shows off many of the features of constrained
types.    
The \xcd`(rows:Long, cols:Long)` in the class definition declares two
properties, \xcd`rows` and \xcd`cols`.\footnote{The class is officially declared
abstract to allow for multiple implementations, like sparse and band matrices,
but in fact is abstract to avoid having to write the actual definitions of
\xcd`+` and \xcd`*`.}  

A constrained type looks like \xcd`Mat{rows==r && cols==c}`: a type
name, followed by a Boolean expression in braces.  
The \xcd`type` declaration on the second line makes
\xcd`Mat(r,c)` be a synonym for \xcd`Mat{rows==r && cols==c}`,
allowing for compact types in many places.

Functions can return constrained types.  
The \xcd`makeMat(r,c)` method returns a \xcd`Mat(r,c)` -- a matrix whose shape
is given by the arguments to the method.    In
particular, constructors can have constrained return types to provide specific
information about the constructed values.

The arguments of methods can have type constraints as well.  The 
\xcd`operator this +` line lets \xcd`A+B` add two matrices.  The type of the
second argument \xcd`y` is constrained to have the same number of rows and
columns as the first argument \xcd`this`. Attempts to add mismatched matrices
will be flagged as type errors at compilation.

At times it is more convenient to put the constraint on the method as a whole,
as seen in the \xcd`operator this *` line. Unlike for \xcd`+`, there is no
need to constrain both dimensions; we simply need to check that the columns of
the left factor match the rows of the right. This constraint is written in
\xcd`{...}` after the argument list.  The shape of the result is computed from
the shapes of the arguments.

And that is all that is necessary for a user-defined class of matrices to have
shape-checking for matrix addition and multiplication.  The \xcd`example`
method compiles under those definitions.








\paragraph{Generic types}

Containers may have type parameters, permitting the definition of
{\em generic types}.  Type parameters may be instantiated by any X10 type.  It
is thus possible to make a list of integers \xcd`List[Long]`, a list of
non-zero integers \xcd`List[Long{self != 0}]`, or a list of people
\xcd`List[Person]`.  In the definition of \xcd`List`, \xcd`T` is a type
parameter; it can be instantiated with any type.
%~~gen ^^^ Overview60
%~~vis
\begin{xten}
class List[T] {
    var head: T;
    var tail: List[T];
    def this(h: T, t: List[T]) { head = h; tail = t; }
    def add(x: T) {
        if (this.tail == null)
            this.tail = new List[T](x, null);
        else
            this.tail.add(x);
    }
}
\end{xten}
%~~siv
%~~neg
The constructor (\xcd"def this") initializes the fields of the new object.
The \xcd"add" method appends an element to the list.
\xcd"List" is a generic type.  When  instances of \xcd"List" are
allocated, the type \param{} \xcd"T" must be bound to a concrete
type.  \xcd"List[Long]" is the type of lists of element type
\xcd"Long", \xcd"List[List[String]]" is the type of lists whose elements are
themselves lists of string, and so on.

%%BARD-HERE

\section{The sequential core of X10}

The sequential aspects of X10 are mostly familiar from C and its progeny.
\Xten{} enjoys the familiar control flow constructs: \xcd"if" statements,
\xcd"while" loops, \xcd"for" loops, \xcd"switch" statements, \xcd`throw` to
raise exceptions and \xcd`try...catch` to handle them, and so on.

X10 has both implicit coercions and explicit conversions, and both can be
defined on user-defined types.  Explicit conversions are written with the
\xcd`as` operation: \xcd`n as Long`.  The types can be constrained: 
%~~exp~~`~~`~~n:Long~~ ^^^ Overview70
\xcd`n as Long{self != 0}` converts \xcd`n` to a non-zero integer, and throws a
runtime exception if its value as an integer is zero.

\section{Places and activities}

The full power of X10 starts to emerge with concurrency.
An \Xten{} program is intended to run on a wide range of computers,
from uniprocessors to large clusters of parallel processors supporting
millions of concurrent operations. To support this scale, \Xten{}
introduces the central concept of \emph{place} (\Sref{XtenPlaces}).
A place can be thought of as a virtual shared-memory multi-processor:
a computational unit with a finite (though perhaps changing) number of
hardware threads and a bounded amount of shared memory, uniformly
accessible by all threads.



An \Xten{} computation acts on \emph{values}(\Sref{XtenObjects}) through
the execution of lightweight threads called
\emph{activities}(\Sref{XtenActivities}). 
An {\em object}
 has a small, statically fixed set of fields, each of
which has a distinct name. A scalar object is located at a single place and
stays at that place throughout its lifetime. An \emph{aggregate} object has
many fields (the number may be known only when the object is created),
uniformly accessed through an index (\eg, an integer) and may be distributed
across many places. The distribution of an aggregate object remains unchanged
throughout the computation, thought different aggregates may be distributed
differently. Objects are garbage-collected when no longer useable; there are
no operations in the language to allow a programmer to explicitly release
memory.

{}\Xten{} has a \emph{unified} or \emph{global address space}. This means that
an activity can reference objects at other places. However, an activity may
synchronously access data items only in the current place, the place in which
it is running. It may atomically update one or more data items, but only in
the current place.   If it becomes necessary to read or modify an object at
some other place \xcd`q`, the {\em place-shifting} operation \xcd`at(q;F)` can
be used, to move part of the activity to \xcd`q`.  \xcd`F` is a specification
of what information will be sent to \xcd`q` for use by that part of the
computation. 
It is easy to compute
across multiple places, but the expensive operations (\eg, those which require
communication) are readily visible in the code. 

\paragraph{Atomic blocks.}

X10 has a control construct \xcd"atomic S" where \xcd"S" is a statement with
certain restrictions. \xcd`S` will be executed atomically, without
interruption by other activities. This is a common primitive used in
concurrent algorithms, though rarely provided in this degree of generality by
concurrent programming languages.

More powerfully -- and more expensively -- X10 allows conditional atomic
blocks, \xcd`when(B)S`, which are executed atomically at some point when
\xcd`B` is true.  Conditional atomic blocks are one of the strongest
primitives used in concurrent algorithms, and one of the least-often
available. 

\paragraph{Asynchronous activities.}

An asynchronous activity is created by a statement \xcd"async S", which starts
up a new activity running \xcd`S`.  It does not wait for the new activity to
finish; there is a separate statement (\xcd`finish`) to do that.


\section{Distributed heap management}

\Xten{} is the language for parallel and distributed computing, which is based on the APGAS (Asynchronous Partitioned Global Address Space) programming model. In (A)PGAS, the address space is partitioned into multiple semi-spaces. The semi-space is called \emph{place} in \Xten{}. In Managed \Xten{} (\Xten{} on \java{} VMs), a place is represented as a single \java{} VM and the semi-space is mapped to the heap of the \java{} VM.

\Xten{} supports garbage collection. Objects in a local heap (local objects) are collected with (local) garbage collection and there is no way to explicitly free them. The reference to local objects is called \emph{local reference}.

In addition, \Xten{} has another type of reference called \emph{remote reference}. Unlike local reference, remote reference can reference objects at both local and remote places.

With remote reference, an activity (something like thread, it runs on a place at a time but it can move itself to different places) can access objects at a remote place (remote objects) when the activity has moved to the remote place. The place where an object is created is the home place of the object and it does not change for the lifetime.

To guarantee an activity can access remote objects at their home place, the objects with remote reference are protected from (local) garbage collection at their home place even if they have no local reference. Objects can be garbage collected only when they have neither local nor remote reference. The garbage collection that takes care of remote reference is called distributed garbage collection and it is supported in Managed \Xten{}.

Distributed garbage collection in Managed \Xten{} \cite{KawachiyaX1012} tracks the lifetime of remote reference with reference counting. When the local garbage collection at a remote place detects the remote reference is no longer needed at the place, the count is decremented. When the count becomes zero, the local garbage collection at the home place is ready to collect the referenced object in the ordinary way.

This mechanism works in most cases, but when there is unbalance in heap allocation rate between places, there is a risk of out of memory error at a frequently allocating place. This is because remote reference from infrequently allocating (i.e. infrequently garbage collected) places could retain remotely referenced objects longer than needed.

To avoid the out of memory error even with unbalanced heap allocation rate, there is a way to explicitly release remote reference.

A single call of \xcd`PlaceLocalHandle.destroy()` (\xcd`PlaceLocalHandle` is an \Xten{} type that bundles multiple remote references to the objects at different places) releases all remote references immediately, thus the local garbage collection at each place becomes ready to collect the referenced object in the ordinary way. It can be called at the point where the all objects referenced by the handle are no longer needed to be accessible with the handle. Local reference to the object at each place won't be affected.



\section{Clocks}
The MPI style of coordinating the activity of multiple processes with
a single barrier is not suitable for the dynamic network of heterogeneous
activities in an \Xten{} computation.  
X10 allows multiple barriers in a form that supports determinate,
deadlock-free parallel computation, via the \xcd`Clock` type.

A single \xcd`Clock` represents a computation that occurs in phases.
At any given time, an activity is {\em registered} with zero or more clocks.
The X10 statement \xcd`next` tells all of an activity's registered clocks that
the activity has finished the current phase, and causes it to wait for the
next phase.  Other operations allow waiting on a single clock, starting
new clocks or new activities registered on an extant clock, and so on. 

%%INTRO-CLOCK%  Activities may use clocks to repeatedly detect quiescence of arbitrary
%%INTRO-CLOCK%  programmer-specified, data-dependent set of activities. Each activity
%%INTRO-CLOCK%  is spawned with a known set of clocks and may dynamically create new
%%INTRO-CLOCK%  clocks. At any given time an activity is \emph{registered} with zero or
%%INTRO-CLOCK%  more clocks. It may register newly created activities with a clock,
%%INTRO-CLOCK%  un-register itself with a clock, suspend on a clock or require that a
%%INTRO-CLOCK%  statement (possibly involving execution of new async activities) be
%%INTRO-CLOCK%  executed to completion before the clock can advance.  At any given
%%INTRO-CLOCK%  step of the execution a clock is in a given phase. It advances to the
%%INTRO-CLOCK%  next phase only when all its registered activities have \emph{quiesced}
%%INTRO-CLOCK%  (by executing a \xcd"next" operation on the clock).
%%INTRO-CLOCK%  When a clock advances, all its activities may now resume execution.
%%INTRO-CLOCK%  

Clocks act as {barriers} for a dynamically varying collection of activities.
They generalize the barriers found in MPI style program in that an activity
may use multiple clocks simultaneously. Yet programs using clocks properly are
guaranteed not to suffer from deadlock.

%%HERE

\section{Arrays, regions and distributions}

X10 provides \xcd`DistArray`s, {\em distributed arrays}, which spread data
across many places. An underlying \xcd`Dist` object provides the {\em
distribution}, telling which elements of the \xcd`DistArray` go in which
place. \xcd`Dist` uses subsidiary \xcd`Region` objects to abstract over the
shape and even the dimensionality of arrays.
Specialized X10 control statements such as \xcd`ateach` provide efficient
parallel iteration over distributed arrays.


\section{Annotations}

\Xten{} supports annotations on classes and interfaces, methods
and constructors,
variables, types, expressions and statements.
These annotations may be processed by compiler plugins.

\section{Translating MPI programs to \Xten{}}

While \Xten{} permits considerably greater flexibility in writing
distributed programs and data structures than MPI, it is instructive
to examine how to translate MPI programs to \Xten.

Each separate MPI process can be translated into an \Xten{}
place. Async activities may be used to read and write variables
located at different processes. A single clock may be used for barrier
synchronization between multiple MPI processes. \Xten{} collective
operations may be used to implement MPI collective operations.
\Xten{} is more general than MPI in (a)~not requiring synchronization
between two processes in order to enable one to read and write the
other's values, (b)~permitting the use of high-level atomic blocks
within a process to obtain mutual exclusion between multiple
activities running in the same node (c)~permitting the use of multiple
clocks to combine the expression of different physics (e.g.,
computations modeling blood coagulation together with computations
involving the flow of blood), (d)~not requiring an SPMD style of
computation.


%\note{Relaxed exception model}
\section{Summary and future work}
\subsection{Design for scalability}
\Xten{} is designed for scalability, by encouraging working with local data,
and limiting the ability of events at one place to delay those at another. For
example, an activity may atomically access only multiple locations in the
current place. Unconditional atomic blocks are dynamically guaranteed to be
non-blocking, and may be implemented using non-blocking techniques that avoid
mutual exclusion bottlenecks. 
%TODO: yoav says: ``no idea what [the following] means''
Data-flow synchronization permits point-to-point
coordination between reader/writer activities, obviating the need for
barrier-based or lock-based synchronization in many cases.

\subsection{Design for productivity}
\Xten{} is designed for productivity.

\paragraph{Safety and correctness.}



Programs written in \Xten{} are guaranteed to be statically
\emph{type safe}, \emph{memory safe} and \emph{pointer safe},
with certain exceptions given in \Sref{sect:LimitationOfStrictTyping}.

Static type safety guarantees that every location contains only values whose
dynamic type agrees with the location's static type. The compiler allows a
choice of how to handle method calls. In strict mode, method calls are
statically checked to be permitted by the static types of operands. In lax
mode, dynamic checks are inserted when calls may or may not be correct,
providing weaker static correctness guarantees but more programming
convenience. 

Memory safety guarantees that an object may only access memory within its
representation, and other objects it has a reference to. \Xten{} does not
permit 
pointer arithmetic, and bound-checks array accesses dynamically if necessary.
\Xten{} uses garbage collection to collect objects no longer referenced by any
activity. \Xten{} guarantees that no object can retain a reference to an
object whose memory has been reclaimed. Further, \Xten{} guarantees that every
location is initialized at run time before it is read, and every value read
from a word of memory has previously been written into that word.

%XXX
%Pointer safety guarantees that a null pointer exception cannot be
%thrown by an operation on a value of a non-nullable type.

%Because places are reflected in the type system, static type safety
%also implies \emph{place safety}. All operations that need to be performed
%locally are, in fact, performed locally.  All data which is declared to be
%stored locally are, in fact, stored locally.

\Xten{} programs that use only the common, specified clock idioms and unconditional atomic
blocks are guaranteed not to deadlock. Unconditional atomic blocks
are non-blocking, hence cannot introduce deadlocks.
Many concurrent programs can be shown to be determinate (hence
race-free) statically.

\paragraph{Integration.}
A key issue for any new programming language is how well it can be
integrated with existing (external) languages, system environments,
libraries and tools.

%TODO: Yoav asks ``you mean interop''?
We believe that \Xten{}, like \java{}, will be able to support a large
number of libraries and tools. An area where we expect future versions
of \Xten{} to improve on \java{} like languages is \emph{native
integration} (\Sref{NativeCode}). Specifically, \Xten{} will 
permit multi-dimensional local arrays to be operated on natively by
native code.

\subsection{Conclusion}
{}\Xten{} is considerably higher-level than thread-based languages in
that it supports dynamically spawning lightweight activities, the
use of atomic operations for mutual exclusion, and the use of clocks
for repeated quiescence detection.

Yet it is much more concrete than languages like HPF in that it forces
the programmer to explicitly deal with distribution of data
objects. In this the language reflects the designers' belief that
issues of locality and distribution cannot be hidden from the
programmer of high-performance code in high-end computing.  A
performance model that distinguishes between computation and
communication must be made explicit and transparent.\footnote{In this
\Xten{} is similar to more modern languages such as ZPL \cite{zpl}.}
At the same time we believe that the place-based type system and
support for generic programming will allow the \Xten{} programmer to
be highly productive; many of the tedious details of
distribution-specific code can be handled in a generic fashion.

\chapter{Lexical structure}

In general, \Xten{} follows \java{} rules \cite[Chapter 3]{jls2} for
lexical structure.

Lexically a program consists of a stream of white space, comments,
identifiers, keywords, literals, separators and operators.

\paragraph{Whitespace}
% Whitespace \index{whitespace} follows \java{} rules \cite[Chapter 3.6]{jls2}.
ASCII space, horizontal tab (HT), form feed (FF) and line
terminators constitute white space.

\paragraph{Comments}
% Comments \index{comments} follows \java{} rules
% \cite[Chapter 3.7]{jls2}. 
All text included within the ASCII characters ``\xcd"/*"'' and
``\xcd"*/"'' is
considered a comment and ignored; nested comments are not
allowed.  All text from the ASCII characters
``\xcd"//"'' to the end of line is considered a comment and is ignored.

\paragraph{Identifiers}

Identifiers consist of a single letter followed by zero or more
letters or digits.
Letters are defined as the characters for which the \java{}
method \xcd"Character.isJavaIdentifierStart" returns true.
Digits are defined as the ASCII characters \xcd"0" through \xcd"9".

\paragraph{Keywords}
\Xten{} reserves the following keywords:
\begin{xten}
abstract       do             in             public         
as             else           instanceof     return         
assert         extends        interface      self           
async          false          native         static         
ateach         final          new            struct         
break          finally        null           super          
case           finish         offers         switch         
catch          for            operator       this           
class          goto           package        throw          
continue       if             private        transient      
def            implements     property       true           
default        import         protected      try            
\end{xten}
Note that the primitive types are not considered keywords.

\paragraph{Literals}\label{Literals}\index{literals}

Briefly, \XtenCurrVer{} uses fairly standard syntax for its literals:
integers, unsigned integers, floating point numbers, booleans, 
characters, strings, and \xcd"null".  The most exotic points are (1) unsigned
numbers are marked by a \xcd`u` and cannot have a sign; (2) \xcd`true` and
\xcd`false` are the literals for the booleans; and (3) floating point numbers
are \xcd`Double` unless marked with an \xcd`f` for \xcd`Float`. 

Less briefly, we use the following abbreviations: 
\begin{displaymath}
\begin{array}{rcll}
d &=& \mbox{one or more decimal digits}\\
d_8 &=& \mbox{one or more octal digits}\\
d_{16} &=& \mbox{one or more hexadecimal digits, using \xcd`a`-\xcd`f`
for 10-15}\\
i &=& d 
        \mathbin{|} {\tt 0} d_8 
        \mathbin{|} {\tt 0x} d_{16}
        \mathbin{|} {\tt 0X} d_{16}
\\
s &=& \mbox{optional \xcd`+` or \xcd`-`}\\
b &=& d 
          \mathbin{|} d {\tt .}
          \mathbin{|} d {\tt .} d
          \mathbin{|}  {\tt .} d \\
x &=& ({\tt e } \mathbin{|} {\tt E})
         s
         d \\
f &=& b x
\end{array}
\end{displaymath}

\begin{itemize}

\item \xcd`true` and \xcd`false` are the \xcd`Boolean` literals.

\item \xcd`null` is a literal for the null value.  It has type
      \xcd`Any{self==null}`. 

\item \xcd`Int` literals have the form {$si$}; \eg, \xcd`123`,
      \xcd`-321` are decimal \xcd`Int`s, \xcd`0123` and \xcd`-0321` are octal
      \xcd`Int`s, and \xcd`0x123`, \xcd`-0X321`,  \xcd`0xBED`, and \xcd`0XEBEC` are
      hexadecimal \xcd`Int`s.  

\item \xcd`Long` literals have the form {$si{\tt l}$} or
      {$si{\tt L}$}. \Eg, \xcd`1234567890L`  and \xcd`0xBAGEL` are \xcd`Long` literals. 

\item \xcd`UInt` literals have the form {$i{\tt u}$} or {$i {\tt U}$}.
      \Eg, \xcd`123u`, \xcd`0123u`, and \xcd`0xBEAU` are \xcd`UInt` literals.

\item \xcd`ULong` literals have the form {$i {\tt ul}$} or {$i {\tt
      lu}$}, or capital versions of those.  For example, 
      \xcd`123ul`, \xcd`0124567012ul`,  \xcd`0xFLU`, \xcd`OXba1eful`, and \xcd`0xDecafC0ffeefUL` are \xcd`ULong`
      literals. 

\item \xcd`Float` literals have the form {$s f {\tt f}$} or  {$s
      f {\tt F}$}.  Note that the floating-point marker letter \xcd`f` is
      required: unmarked floating-point-looking literals are \xcd`Double`. 
      \Eg, \xcd`1f`, \xcd`6.023E+32f`, \xcd`6.626068E-34F` are \xcd`Float`
      literals. 

\item \xcd`Double` literals have the form {$s f$}\footnote{Except that
      literals like \xcd`1` 
      which match both {$i$} and {$f$} are counted as
      integers, not \xcd`Double`; \xcd`Double`s require a decimal
      point, an exponent, or the \xcd`d` marker.
      }, {$s f {\tt
      D}$}, and {$s f {\tt d}$}.  
      \Eg, \xcd`0.0`, \xcd`0e100`, \xcd`229792458d`, and \xcd`314159265e-8`
      are \xcd`Double` literals.

\item \xcd`Char` literals have one of the following forms: 
      \begin{itemize}
      \item \xcd`'`{\it c}\xcd`'` where {\em c} is any printing ASCII
            character other than 
            \xcd`\` or \xcd`'`, representing the character {\em c} itself; 
            \eg, \xcd`'!'`;
      \item \xcd`'\b'`, representing backspace;
      \item \xcd`'\t'`, representing tab;
      \item \xcd`'\n'`, representing newline;
      \item \xcd`'\f'`, representing form feed;
      \item \xcd`'\r'`, representing return;
      \item \xcd`'\''`, representing single-quote;
      \item \xcd`'\"'`, representing double-quote;
      \item \xcd`'\\'`, representing backslash;
      \item \xcd`'\`{\em dd}\xcd`'`, where {\em dd} is one or more octal
            digits, representing the one-byte character numbered {\em dd}; it
            is an error if {\em dd}{$>255$}.      
      \end{itemize}

\item \xcd`String` literals consist of a double-quote \xcd`"`, followed by
      zero or more of the contents of a \xcd`Char` literal, followed by
      another double quote.  \Eg, \xcd`"hi!"`, \xcd`""`.

\item There are no literals of type \xcd`Byte`, \xcd`UByte`, \xcd`Short`, or
      \xcd`UShort`.  

\end{itemize}



\paragraph{Separators}
\Xten{} has the following separators and delimiters:
\begin{xten}
( )  { }  [ ]  ;  ,  .
\end{xten}

\paragraph{Operators}
\Xten{} has the following operators:
\begin{xten}
==  !=  <   >   <=  >=
&&  ||  &   |   ^
<<  >>  >>>
+   -   *   /   %
++  --  !   ~
&=  |=  ^=
<<= >>= >>>=
+=  -=  *=  /=  %=
=   ?   :   =>  ->
<:  :>  @   ..
\end{xten}





\chapter{Types}
\label{XtenTypes}\index{types}

{}\Xten{} is a {\em strongly typed} object-oriented language: every
variable and expression has a type that is known at compile-time.
Types limit the values that variables can hold and specify the places
at which these values can lie.


{}\Xten{} supports three kinds of runtime entities, {\em objects},
{\em structs}, and {\em functions}. Objects are instances of {\em
  classes} (\Sref{ReferenceClasses}). They may contain mutable fields
and stay resident in the place in which they were
created. 
Objects are said to be {\em boxed} in that variables of a
class type are implemented through a single memory location that
contains a reference to the memory containing the declared state of
the object (and other meta-information such as the list of methods of
the object). Thus objects are represented through an extra level of
indirection. A consequence of this flexibility is that every class
type contains the value \Xcd{null} corresponding to the invalid
reference. \Xcd{null} is often useful as a default value. Further, two
objects may be compared for identity (\Xcd{==}) in constant time by
simply containing references to the memory used to represent the
objects.

Structs are instances of {\em struct types} (\Sref{StructClasses}). They are a
restricted variant of classes, lacking meta-information; this makes them less
flexible, but in many cases more efficient. When it is semantically
meaningful, converting a class into a struct or vice-versa is quite easy.
Structs are immutable and may be freely copied from place to place. Further,
they may be allocated inline, using only as much memory as necessary to hold
and align the fields of the struct.

Functions, called closures or lambda-expressions in other languages, are
instances of {\em function types|} {\Sref{Functions}). Functions can refer to
%~~exp~~`~~`~~y:Int ~~
variables from the surrounding environment; \eg, \xcd`(x:Int)=>x*y` is a unary
integer function which multiplies its argument by the variable \xcd`y` from
the surrounding block.  
Functions may be freely copied from place to place and may be repeatedly
applied to a set of arguments.

These runtime entities are classified by {\em types}. Types are used in
variable declarations, explicit coercions and conversions, object creation,
array creation, class literals, static state and method accessors, and
\xcd"instanceof" expressions.

The basic relationship between values and types is {\em instantiation}. For
example, \xcd`1` is an instance of type of integers, \xcd`Int`. It is also an
instance of type of all entities \xcd`Any`, and of type of nonzero integers
\xcd`Int{self != 0}`, and many others.

The basic relationship between types is {\em subtyping}: \xcd`T <: U` holds if
every instance of \xcd`T` is also an instance of \xcd`U`. Two important kinds
of subtyping are {\em subclassing} and {\em strengthening}.  Subclassing is a
familiar notion from object-oriented programming.  In a class
hierarchy with classes \xcd`Animal` and \xcd`Cat` arranged in the usual way,
every \xcd`Cat` is an \xcd`Animal`, so \xcd`Cat <: Animal` by subclassing.  
Strengthening is an equally familiar notion from logic.   The instances of
\xcd`Int{self != 0}` are all elements of \xcd`Int{true}` as well, because
\xcd`self != 0` logically implies \xcd`true`; so 
\xcd`Int{self != 0} <: Int{true} == Int` by strengthening.  X10 uses both
notions of subtyping.




\subsection*{The Grammar of Types}

Types are described by the following grammar: 
\bard{Is this still correct?}
\begin{grammar}
Type \: FunctionType \\
    \| ConstrainedType  \\

FunctionType \: TypeParameters\opt \xcd"(" Formals\opt \xcd")"
Constraint\opt Throws\opt \xcd"=>" Type \\
TypeParameters \: \xcd"[" TypeParameter ( \xcd"," TypeParameter )\star \xcd"]" \\
TypeParameter \: Identifier \\
Throws \: \xcd"throws" TypeName ( \xcd"," TypeName )\star \\

ConstrainedType \: Annotation\star BaseType Constraint\opt
PlaceConstraint\opt \\

BaseType \: ClassBaseType \\
     \| InterfaceBaseType \\
     \| PathType \\
     \| \xcd"(" Type \xcd")" \\

ClassType \: Annotation\star ClassBaseType Constraint\opt
PlaceConstraint\opt \\
InterfaceType \: Annotation\star InterfaceBaseType Constraint\opt
PlaceConstraint\opt \\

PathType \: Expression \xcd"." Identifier \\

Annotation \: \xcd"@" InterfaceBaseType Constraint\opt \\

ClassOrInterfaceType \: ClassType \\ \| InterfaceType \\
ClassBaseType \: TypeName \\
InterfaceBaseType \: TypeName \\
\end{grammar}

% \section{Type definitions and type constructors}
% 
% Types in \Xten{} are specified through declarations and through
% type constructors:

% \paragraph{Class types.}

\section{\xcd`Any`}

It is quite convenient to have a type which all values are instances of; that
is, a supertype of all types.\footnote{Java, for one, suffers a number of
  inconveniences because some built-in types like \xcd`int` and \xcd`char`
  aren't subtypes of anything else.}  X10's universal supertype is the
  interface \xcd`Any`. 


\begin{xten}
package x10.lang;
public interface Any {
  property def home():Place;
  property def at(p:Object):Boolean;
  property def at(p:Place):Boolean;
  global safe def toString():String;
  global safe def typeName():String;
  global safe def equals(Any):Boolean;
  global safe def hashCode():Int;
}
\end{xten}

\xcd`Any` provides a handful of essential methods that make sense and are
useful for everything.\footnote{The behavioral annotation \xcd`property` is
  explained in \Sref{properties}; \xcd`safe` in \Sref{SafeAnnotation}, and
  \xcd`global` in \Sref{GlobalAnnotation}.} \xcd`a.toString()` produces a
string representation of \xcd`a`, and \xcd`a.typeName()` the string
representation of its type; both are useful for debugging.  \xcd`aequals(b)`
is the programmer-overridable equality test, and \xcd`a.hashCode()` an integer
useful for hashing.  \xcd`at()` and \xcd`home()` are used in multi-place
computing. 



\section{Classes and interfaces}
\label{ReferenceTypes}

\subsection{Class types}

\index{types!class types}
\index{class}
\index{class declaration}
\index{declaration!class declaration}
\index{declaration!reference class declaration}

A {\em class declaration} (\Sref{XtenClasses}) introduces a {\em class type}
containing all instances of the class.  The \xcd`Position` class below
could describe the position of a slider control, for example.

%~~gen
% package Types.By.Cripes.Classes;
%~~vis
\begin{xten}
class Position {
  private var x : Int = 0;
  public def move(dx:Int) { x += dx; }
  public def pos() : Int = x;
}
\end{xten}
%~~siv
%
%~~neg

Class instances, also called objects, are created via constructor calls. Class
instances have fields and methods, type members, and value properties bound at
construction time. In addition, classes have static members: constant fields,
type definitions, and member classes and member interfaces.

A class with type parameters is {\em generic}. A class type is instantiatable
only if all of its parameters are instantiated on concrete types.  The
\xcd`Cell[T]` class provides a container capable of holding a value of type
\xcd`T`, or being empty.

%~~gen
% package Types.For.Gripes.Of.Wesley.Snipes;
%~~vis
\begin{xten}
class Cell[T] {
  var empty : Boolean = true;
  var contents : T;
  public def putIn(t:T) { 
    contents = t; empty = false; 
  }
  public def emptyOut() { empty = true; }
  public def isEmpty() = empty;
  public def getOut():T throws Exception {
     if (empty) throw new Exception("Empty!");
     return contents ;
  }
}
\end{xten}
%~~siv
%
%~~neg


\Xten{} does not permit mutable static state. A fundamental principle of the
X10 model of computation is that all mutable state be local to some place
(\Sref{XtenPlaces}), and, as static variables are globally available, they
cannot be mutable. When mutable global state is necessary, programmers should
use singleton classes, putting the state in an object and using place-shifting
commands (\Sref{AtStatement}) and atomicity (\Sref{AtomicBlocks}) as necessary
to mutate it safely.

\index{\Xcd{Object}}
\index{\Xcd{x10.lang.Object}}

Classes are structured in a single-inheritance hierarchy. All classes extend
the class \xcd"x10.lang.Object", directly or indirectly. Each class other than
\xcd`Object` extends a single parent class.  \xcd`Object` provides no behavior
of its own, beyond that required by \xcd`Any`.

\index{class!reference class}
\index{reference class type}
\index{\Xcd{Object}}
\index{\Xcd{x10.lang.Object}}


\index{null}
\bard{We've got to say this better.}
Variables of class type may contain the value \xcd"null". 

\subsection{Interface types}
\label{InterfaceTypes}

\index{types!interface types}
\index{interface}
\index{interface declaration}
\index{declaration!interface declaration}

An {\em interface declaration} (\Sref{XtenInterfaces}) defines an {\em
interface type}, specifying a set of methods, type members, and
properties which must be provided by any class declared to implement the
interface. 


Interfaces can also have static members: constant fields, type definitions,
and member classes and interfaces.  However, interfaces cannot specify that
implementing classes must provide static members.

An interface may extend multiple interfaces.  
%~~gen
%package Types.For.Snipes.Interfaces;
%~~vis
\begin{xten}
interface Named {
  def name():String;
}
interface Mobile {
  def move(howFar:Int):Void;
}
interface Person extends Named, Mobile {}
interface NamedPoint extends Named, Mobile{} 
\end{xten}
%~~siv
%
%~~neg


Classes may be declared to implement multiple interfaces.
Semantically, the interface type is the set of all objects that are
instances of classes that implement the interface. A class implements
an interface if it is declared to and if it implements all the methods
and properties defined in the interface.  For example, \xcd`KimThePoint`
implements \xcd`Person`, and hence \xcd`Named` and \xcd`Mobile`.  It would be
a static error if \xcd`KimThePoint` had no \xcd`name` method.

%~~gen
%interface Named {
%   def name():String;
% }
% interface Mobile {
%   def move(howFar:Int):Void;
% }
% interface Person extends Named, Mobile {}
% interface NamedPoint extends Named, Mobile{} 
%~~vis
\begin{xten}
class KimThePoint implements Person {
   var pos : Int = 0;
   public def name() = "Kim (" + pos + ")";
   public def move(dPos:Int) { pos += dPos; }
}
\end{xten}
%~~siv
%
%~~neg


\subsection{Properties}
\index{properties}
\label{properties}

Classes, interfaces, and structs may have {\em properties}, public \xcd`val` instance
fields bound on object creation. For example, the following code declares a
class named \xcd"Coords" with properties \xcd"x" and \xcd"y" and a \xcd"move"
method. The properties are bound using the \xcd"property" statement in the
constructor.

%~~gen
%package not.x10.lang;
%~~vis
\begin{xten}
class Coords(x: Int, y: Int) {
  def this(x: Int, y: Int) : Int{this.x==x, this.y==y} 
    = { property(x, y); }
  def move(dx: Int, dy: Int) = new Coords(x+dx, y+dy);
}
\end{xten}
%~~siv
%~~neg

Properties, unlike other public \xcd`val` fields, can be used  
at compile time in {\em constraints}. This allows us
to specify subtypes based on properties, by appending a boolean expression to
the type. For example, the type \xcd"Coords{x==0}" is the set of all points
whose \xcd"x" property is \xcd"0".  Details of this substantial topic are
found in \Sref{ConstrainedTypes}.



\section{Type parameters and Generic Types}
\label{TypeParameters}

\index{types!type parameters}
\index{methods!parametrized methods}
\index{constructors!parametrized constructors}
\index{closures!parametrized closures}
\label{Generics}
\index{types!generic types}

A class, interface, method, closure, or type definition  may have type
parameters.  Type parameters can be used as types, and will be bound to types
on instantiation.  For example, a generic stack class may be defined as 
\xcd`Stack[T]{...}`.  Stacks can hold values of any type; \eg, 
%~~type~~`~~`~~ ~~class Stack[T]{}
\xcd`Stack[Int]` is a stack of integers, and 
%~~type~~`~~`~~ ~~class Stack[T]{}
\xcd`Stack[Point{self!=null}]`is a stack of non-null \xcd`Point`s.
Generics {\em must} be instantiated when they are used: \xcd`Stack`, by
itself, is not a valid type.
Type parameters may be constrained by a guard on the declaration
(\Sref{ClassGuard}, \Sref{TypeDefGuard},
\Sref{MethodGuard},\Sref{ClosureGuard}).

\index{types!concrete types}
\index{concrete type}
A {\em generic type} is a class, struct,  interface, or type declared with one or
more type parameters.  When instantiated with concrete (\viz, non-generic)
types for its parameters, a generic type becomes a concrete type and can be
used like any other type. For example,
\xcd`Stack` is a generic type, 
%~~type~~`~~`~~ ~~class Stack[T]{}
\xcd`Stack[Int]` is a concrete type, and can be used as one: 
%~~stmt~~`~~`~~ ~~class Stack[T]{}
\xcd`var stack : Stack[Int];`


A \xcd`Cell[T]` is a generic object, capable of holding a value of type
\xcd`T`.  For example, a \xcd`Cell[Int]` can hold an \xcd`Int`, and a
\xcd`Cell[Cell[Int]{self!=0}]` can hold a \xcd`Cell[Int]` which in turn can
only hold non-zero numbers.  \xcd`Cell`s are actually useful in situations
where values must be bound immutably for one reason, but need to be mutable.
%~~gen
% package ch4;
%~~vis
\begin{xten}
class Cell[T] {
    var x: T;
    def this(x: T) { this.x = x; }
    def get(): T = x;
    def set(x: T) = { this.x = x; }
}
\end{xten}
%~~siv
%~~neg


\xcd"Cell[Int]" is the type of \xcd`Int`-holding cells.  
The \xcd"get" method on a \xcd`Cell[Int]` returns an \xcd"Int"; the
\xcd"set" method takes an \xcd"Int" as argument.  Note that
\xcd"Cell" alone is not a legal type because the parameter is
not bound.

\subsection{Variance of Type Parameters}
\index{covariant}
\index{contravariant}
\index{invariant}
\index{type parameter!covariant}
\index{type parameter!contravariant}
\index{type parameter!invariant}

Consider classes \xcd`Person :> Child`.  Every child is a person, but there
are people who are not children.  What is the relationship between
\xcd`Cell[Person]` and \xcd`Cell[Child]`?  

\subsubsection{Why Variance Is Necessary}

In this case, \xcd`Cell[Person]` and \xcd`Cell[Child]` should be unrelated.  
If we had \xcd`Cell[Person] :> Cell[Child]`, the following code would let us
assign a \xcd`old` (a \xcd`Person` but not a \xcd`Child`) to a
variable \xcd`young` of type \xcd`Child`, thereby breaking the type system: 
\begin{xten}
// INCORRECTLY assuming Cell[Person] :> Cell[Child]
val cc : Cell[Child] = new Cell[Child]();
val cp : Cell[Person] = cc; // legal upcast
cp.set(old);       // legal since old : Person
val young : Child = cc.get(); 
\end{xten}

Similarly, if \xcd`Cell[Person] <: Cell[Child]`: 
\begin{xten}
// INCORRECTLY assuming Cell[Person] <: Cell[Child]
val cp : Cell[Person] = new Cell[Person];
val cc : Cell[Child] = cp; // legal upcast
val cp.set(old); 
val young : Child = cc.get();
\end{xten}

So, there cannot be a subtyping relationship in either direction between the
two. And indeed, neither of these programs passes the X10 typechecker.


\subsubsection{Legitimate Variance}

The \xcd`Cell[Person]`-vs-\xcd`Cell[Child]` problems occur because it is
possible to both store and retrieve values from the same object. However,
entities with only one of the two capabilities {\em can} sensibly have some
subtyping relations. Furthermore, both sorts of entity are useful. An entity
which can store values but not retrieve them can nonetheless summarize them.
An object which can retrieve values but not store values can be constructed
with an initial value, providing a read-only cell.

So, X10 provides {\em variance} to support these options.  Type parameters
may be defined in one of three forms.  
\begin{enumerate}
\item {\em invariant}: Given a definition \xcd`class C[T]{...}`, \xcd`C[Person]` and
      \xcd`C[Child]` are unrelated classes; neither is a subclass of the
      other.
\item {\em covariant}: Given a definition \xcd`class C[+T]{...}` (the \xcd`+` indicates
      covariance), \xcd`C[Person] :> C[Child]`.  This is appropriate when
      \xcd`C` allows retrieving values but not setting them.
\item {\em contravariant}: Given a definition \xcd`class C[-T]{...}` (the \xcd`-` indicates
      contravariance), \xcd`C[Person] <: C[Child]`.  This is appropriate when
      \xcd`C` allows storing values but not retrieving them.
\end{enumerate}


The \xcd"T" parameter of \xcd"Cell" above is
invariant.  

A typical example of covariance is \xcd`Get`.  As the \xcd`example()` method
shows, a \xcd`Get[T]` must be constructed with its value, and will return that
value whenever desired.
%~~gen
% package ch4;
%~~vis
\begin{xten}
class Get[+T] {
  var x: T;
  def this(x: T) { this.x = x; }
  def get(): T = x;
  static def example() {
     val g : Get[Int]! = new Get[Int](31);
     val n : Int = g.get();
     x10.io.Console.OUT.print("It's " + n);
     x10.io.Console.OUT.print("It's still " + g.get());
  }
}
\end{xten}
%~~siv
%~~neg


A typical example of contravariance is \xcd`Set`.  As the \xcd`example()`
method shows,  a variety of objects\footnote{Objects but no structs.  If we
had wanted structs too, we could have used a \xcd`Cell[Any]`.}  can be put into a
\xcd`Set[Object]`.  While the object itself cannot be retrieved, some summary
information about it -- in this case, its \xcd`typeName` -- can be.  
%~~gen
% package ch4;
%~~vis
\begin{xten}
class Set[-T] {
  var x: T;
  def this(x: T) { this.x = x; }
  def set(x: T) = { this.x = x; } 
  def summary(): String = this.x.typeName();
  static def example() {
    val s : Set[Object]! = new Set[Object](new Throwable());
    s.summary(); // == "x10.lang.Throwable"
    s.set("A String");
    s.summary(); // == "x10.lang.String";
  }    
}
\end{xten}
%~~siv
%
%~~neg


Given types \xcd"S" and \xcd"T": 
\begin{itemize}
\item
If the parameter of \xcd"Get" is covariant, then
\xcd"Get[S]" is a subtype of \xcd"Get[T]" if
\xcd"S" is a {\em subtype} of \xcd"T".

\item
If the parameter of \xcd"Set" is contravariant, then
\xcd"Set[S]" is a subtype of \xcd"Set[T]" if
\xcd"S" is a {\em supertype} of \xcd"T".

\item
If the parameter of \xcd"Cell" is invariant, then
\xcd"Cell[S]" is a subtype of \xcd"Cell[T]" if
\xcd"S" is a {\em equal} to \xcd"T".
\end{itemize}


In order to make types marked as covariant and contravariant semantically
sound, X10 performs extra checks.  
A covariant type parameter is permitted to appear only in covariant type positions,
and a contravariant type parameter in contravariant positions. 
\begin{itemize}
\item The return type of a method is a covariant position.
\item The argument types of a method are contravariant positions.
\item Whether a type argument position of a generic class, interface or struct type \Xcd{C}
is covariant or contravariant is determined by the \Xcd{+} or \Xcd{-} annotation
at that position in the declaration of \Xcd{C}.
\end{itemize}

There are similar restrictions on use of covariant and contravariant values. 
\bard{Get them!  What are they?}


\section{Type definitions}
\label{TypeDefs}

\index{types!type definitions}
\index{declarations!type definitions}

\section{Type definitions}

With value arguments, type arguments, and constraints, the
syntax for \Xten{} types can often be verbose;
\Xten{} therefore provides {\em type definitions}
to allow aliases to be defined for types.
Type definitions have the following syntax:

\begin{grammar}
TypeDefinition \: 
                \xcd"type"~Identifier
                           ( \xcd"[" TypeParameters \xcd"]" )\opt \\
                        && ( \xcd"(" Formals \xcd")" )\opt
                            Constraint\opt \xcd"=" Type \\
\end{grammar}

\noindent
A type definition can be thought of as a type-valued function,
mapping type parameters and value parameters to a concrete type.
%
The following examples are legal type definitions:
\begin{xten}
type StringSet = Set[String];
type MapToList[K,V] = Map[K,List[V]];
type Nat = Int{self>=0};
type Int(x: Int) = Int{self==x};
type Int(lo: Int, hi: Int) = Int{lo <= self, self <= hi};
\end{xten}

As the two definitions of \xcd"Int" demonstrate, type definitions may 
be overloaded: a type definition with a different number of type
parameters or with different types of value
parameters---according to the method overloading rules
(\Sref{MethodOverload}) define distinct types.

Type definitions may appear as class members or in the body of a
method, constructor, or initializer.  Type definitions that are
members of a class are \xcd"static"; type properties can be used
for non-static type definitions.

Type definitions are applicative, not generative; that is, they
are define aliases for types and do not introduce new types.
Thus, the following code is legal:
\begin{xten}
type A = Int;
type B = String;
type C = String;
a: A = 3;
b: B = new C("Hi");
c: C = b + ", Mom!";
\end{xten}
A type defined by a type definition
has the same constructors as its defining type; however, a
constructor may not be invoked using a given type definition
name if the constructor return type is not a subtype of the
defined type.

All type definitions are members of their enclosing package or
class.  A compilation unit may have one or more type definitions
or class or interface declarations with the same name, as long
as the types are unique by overloading.




\section{Constrained types}
\label{ConstrainedTypes}
\label{DepType:DepType}
\label{DepTypes}

\index{dependent types}
\index{constrained types}
\index{generic types}
\index{types!constrained types}
\index{types!dependent types}
\index{types!generic types}


Basic types, like \xcd`Int` and \xcd`List[String]`, provide useful
descriptions of data.  Indeed, most typed programming languages get by with no
more specific descriptions.

However, there are a lot of things that one frequently wants to say about
data.  One might want to know that a \xcd`String` variable is not \xcd`null`,
or that a matrix is square, or that one matrix has the same number of columns
that another has rows (so they can be multiplied).  In the multicore setting,
one might wish to know that two values are located at the same processor.

In most languages, there is simply no way to say these things statically.
Programmers must made do with comments, \xcd`assert` statements, and dynamic
tests.  X10 can do better, with {\em constraints} on types (and methods and
other things).

A constraint is a boolean expression \xcd`e` attached to a basic type \xcd`T`,
written \xcd`T{e}`.  (Only a limited selection of boolean expressions is
available.)  The values of type \xcd`T{e}` are the values of \xcd`T` for which
\xcd`e` is true.  For example: 

\begin{itemize}
%~~type~~`~~`~~ ~~
\item \xcd`String{self != null}` is the type of non-null strings.  \xcd`self`
      is a special variable available only in constraints; it refers to the
      datum being constrained.   
\item If \xcd`Matrix` has properties \xcd`rows` and \xcd`cols`, 
%~~type~~`~~`~~ ~~class Matrix(rows:Int,cols:Int){}
      \xcd`Matrix{rows == cols}` is the type of square matrices.
\item One way to say that \xcd`a` has the same number of columns that \xcd`b`
      has rows (so that \xcd`a*b` is a valid matrix product), one could say: 
%~~gen
% package Types.cripes.whered.you.get.those.gripes;
% class Matrix(rows:Int, cols:Int){
% public static def someMatrix(): Matrix = null;
% public static def example(){
%~~vis
\begin{xten}
  val a : Matrix = someMatrix() ;
  var b : Matrix{b.rows == a.cols} ;
\end{xten}
%~~siv
%}}
%~~neg

\item One way to say that objects \xcd`c` and \xcd`d` are located at the same
      place is: 
%~~gen
% package Types.flipes.knipes.shipes.wipes;
% class Exampler {
% static def someObject(): Object = null;
% static def example() {
%~~vis
\begin{xten}
  val a : Object = someObject();
  var b : Object{a.home == b.home};
\end{xten}
%~~siv
%}}
%~~neg
\end{itemize}



%%BARD-HERE


Given a type \xcd"T", a {\em constrained type} \xcd"T{e}" may be constructed
by constraining its properties with a boolean expression of a limited sort
\xcd"e". The values of \xcd`T{e}` are those values of type \xcd`T` for which
\xcd`e` evaluates to \xcd`true`.  \Eg, \xcd`Point` has a property
\xcd`rank:Int`.  If \xcd`p : Point`, \xcd`p` may have any \xcd`rank`.
\xcd`Point{rank == 3}` is the point type constrained to only those values
whose \xcd`rank` property is 3.   

A common use of constrained types is to explain where objects are located.
Every object has a \xcd`home` property. If \xcd`Person` is a type of people,
then \xcd`Person{home==here}` is the type of people whose data is stored at
the current location.  As explained in \Sref{XtenPlaces}, certain operations
can only be performed at an object's home, so having this expressible as a
type is crucial.


\xcd"T{e}" is a {\em dependent type}, that is, a type dependent on values. The
type \xcd"T" is called the {\em base type} and \xcd"e" is called the {\em
constraint}. 

For brevity, the constraint may be omitted and
interpreted as \xcd"true".

Constraints may refer to values in the local environment: 
%~~gen
% class ConstraintsMayReferToValues {
% def thoseValues() {
%~~vis
\begin{xten}
     val n = 1;
     var p : Point{rank == n};
\end{xten}
%~~siv
%}}
%~~neg
Indeed, there is technically no need for a constraint to refer to the
properties of its type; it can refer entirely to the environment, thus: 
%~~gen
% class ConstraintsMayReferToValuesTwo {
% def thoseValues() {
%~~vis
\begin{xten}
     val m = 1;
     val n = 2;
     var p : Point{m != n};
\end{xten}
%~~siv
%}}
%~~neg

Constraints on properties induce a natural subtyping relationship:
\xcd"C{c}" is a subtype of
\xcd"D{d}" if \xcd"C" is a subclass of \xcd"D" and
\xcd"c" entails \xcd"d".

Type parameters cannot be constrained.

\subsection{Constraints}

\def\withmath#1{\relax\ifmmode#1\else{$#1$}\fi}
\def\LL#1{\withmath{\lbrack\!\lbrack #1\rbrack\!\rbrack}}

Expressions used as constraints are restricted by the constraint
system in use to ensure that the constraints can be solved at compile
time.  The \Xten{} compiler allows compiler plugins to be installed to
extend the constraint language and the constraint system.  Constraints
must be of type \xcd"Boolean".  The compiler supports the following
constraint syntax.


\begin{grammar}
Constraint \: ValueArguments     Guard\opt \\
           \| ValueArguments\opt Guard     \\
           \\
ValueArguments   \:  \xcd"(" ArgumentList\opt \xcd")" \\
ArgumentList     \:  Expression ( \xcd"," Expression )\star \\
Guard            \: \xcd"{" DepExpression \xcd"}" \\
DepExpression    \: ( Formal \xcd";" )\star ArgumentList \\
\end{grammar}

In \XtenCurrVer{} value constraints may be equalities (\xcd"=="),
disequalities (\xcd"!=") and conjunctions thereof.  The terms over
which these constraints are specified include literals and
(accessible, immutable) variables and fields, property methods, and the special
constants {\tt here}, {\tt self}, and {\tt this}. Additionally, place
types are permitted (\Sref{PlaceTypes}).

\index{self}
When constraining a value of type \xcd`T`, \xcd`self` refers to the object of
type \xcd`T` which is being constrained.  For example, \xcd`Int{self == 4}` is
the type of \xcd`Int`s which are equal to 4 -- the best possible description
of \xcd`4`, and a very difficult type to express without using \xcd`self`.  


Type constraints may be subtyping and supertyping (\xcd"<:" and
\xcd":>") expressions over types.

The static constraint checker approximates computational reality in some
cases.  For example, it assumes that built-in types are infinite. This is a
good approximation for \xcd`Int`.  It is a poor approximation for \xcd`Boolean`,
as the checker believes that \xcd`a != b && a != c && b != c` is satisfiable
over \xcd`Boolean`, which it is not.  However, the checker is always correct
when computing the truth or falsehood of a constraint.


% //, and existential quantification over typed variables.

\emph{
Subsequent implementations are intended to support boolean algebra,
arithmetic, relational algebra, etc., to permit types over regions and
distributions. We envision this as a major step towards removing most,
if not all, dynamic array bounds and place checks from \Xten{}.
}


\subsubsection{Acyclicity restriction}

To ensure that type-checking is decidable, we
require that property graphs be acyclic.
That is, it should not be the case at runtime that
a set of objects can be created such that the
graph formed by taking objects as nodes and adding an edge from $m$ to
$n$ if $m$ has a property whose value is $n$ has a cycle in it.

Currently this restriction is not checked by the compiler. Future
versions of the compiler will check this restriction by introducing
rules on escaping of \Xcd{this} (\Sref{protorules}) before the invocation of
\Xcd{property} calls.


\subsection{Place constraints}
\label{PlaceTypes}
\label{PlaceType}
\index{place types}
\label{DepType:PlaceType}\index{placetype}

An \Xten{} computation spans multiple places
(\Sref{XtenPlaces}). Each place constains data and activities that
operate on that data.  \XtenCurrVer{} does not permit the dynamic
creation of a place. Each \Xten{} computation is initiated with a
fixed number of places, as determined by a configuration parameter.
In this section we discuss how the programmer may supply place type
information, thereby allowing the compiler to check data locality,
i.e., that data items being accessed in an atomic section are local.

\begin{grammar}
PlaceConstraint     \: \xcd"!" Place\opt \\
Place              \:   Expression \\
\end{grammar}

Because of the importance of places in the \Xten{} design, special
syntactic support is provided for constrained types involving places.

All \Xten{} classes extend the class
\xcd"x10.lang.Object", which defines a property
\xcd"home" of type
\xcd"Place".

If a constrained reference type \xcd"T" has an \xcd"!p" suffix,
the constraint for \xcd"T" is implicitly assumed to contain the clause
\xcd"self.home==p"; that is,
\xcd"C{c}!p" is equivalent to \xcd"C{self.home==p && c}".

The place \xcd"p" may be ommitted. It defaults to \xcd"this" 
for types in field declarations, and to \xcd"here" elsewhere.


% The place specifier \xcd"any" specifies that the object can be
% located anywhere.  Thus, the location is unconstrained; that is,
% \xcd"C{c}!any" is equivalent to \xcd"C{c}".

% XXX ARRAY
%The place specifier \xcd"current" on an array base type
%specifies that an object with that type at point \xcd"p"
%in the array 
%is located at \xcd"dist(p)".  The \xcd"current" specifier can be
%used only with array types.




\subsection{Constraint semantics}

\begin{staticrule}{Variable occurrence}
In a dependent type \xcd"T" = \xcd"C{c}", the only variables that may
occur in \xcd"c" are (a)
\xcd"self", (b) properties visible at \xcd"T", (c)  local \xcd`val`s, \xcd`val`
method parameters or \xcd`val` constructor parameters visible at \xcd"T", (d)
\xcd`val` fields visible at \xcd"T"'s lexical place in the source program.  
\end{staticrule}

\begin{staticrule}{Restrictions on \xcd"this"}
  The special variable \xcd"this" may be used in a dependent clause for a type \xcd"T"
  only if \xcd`this` may be used in an expression at that point.  \Viz, if 
  \xcd"this" occurs in (a) a property declaration for a
  class, (b) an instance method, (c) an
  instance field, or (d) an instance initializer.

  In particular, \xcd"this" may not be used in types that occur in a static
  context, or in the arguments, body or return type of a constructor or
  in the extends or implements clauses of class and interface
  definitions.  In these contexts, the object that \xcd"this" would
  correspond to is not defined.
\end{staticrule}

\begin{staticrule}{Variable visibility}
  If a type \xcd"T" occurs in a field, method or constructor
  declaration, then all variables used in \xcd"T" must have at least the
  same visibility as the declaration.  The relation ``at least the same
  visibility as'' is given by the transitive closure of:

\begin{xten}
public > protected > package > private
\end{xten}

All inherited properties of a type \xcd"T" are visible in the property
list of \xcd"T", and the body of \xcd"T".

\end{staticrule}

In general, variables (i.e., local variables, parameters,
properties, fields) are visible at
\xcd"T" if they are defined before \xcd"T" in the program. This rule applies to
types in property lists as well as parameter lists (for methods and
constructors).
A formal parameter is visible in the types of all other formal
parameters of the same method, constructor, or type definition,
as well as in the method or constructor body itself.
Properties are accessible via their containing object--\xcd"this"
within the body of their class declaration.  The special
variable \xcd"this" is in scope at each property
declaration, constructor signatures and bodies, instance method signatures
and bodies,
and instance field signatures and initializers, but not in scope
at \xcd"static" method or field declarations or \xcd"static"
initializers.  

We permit variable declarations \xcd"v: T" where \xcd"T" is obtained
from a dependent type \xcd"C{c}" by replacing one or more occurrences
of \xcd"self" in \xcd"c" by \xcd"v". (If such a declaration \xcd"v: T"
is type-correct, it must be the case that the variable \xcd"v" is not
visible at the type \xcd"T". Hence we can always recover the
underlying dependent type \xcd"C{c}" by replacing all occurrences of \xcd"v"
in the constraint of \xcd"T" by \xcd"self".)

For instance, \xcd"v: Int{v == 0}" is shorthand for \xcd"v: Int{self == 0}".

\begin{staticrule}{Constraint type}
The type of a constraint \xcd"c" must be \xcd"Boolean".  
\end{staticrule}

A variable occurring in the constraint \xcd"c" of a dependent type, other than
\xcd"self" or a property of \xcd"self", is said to be a {\em
parameter} of \xcd"c".\label{DepType:Parameter} \index{parameter}

An instance \xcd"o" of \xcd"C" is said to be of type \xcd"C{c}"
(or: {\em belong to}
\xcd"C{c}") if the predicate \xcd"c" evaluates to \xcd"true" in the current lexical
environment, augmented with the binding \xcd"self" $\mapsto$ \xcd"o". We shall
use the function \LL{\mbox{\Xcd{C\{c\}}}} to denote the set of
objects that belong to \xcd"C{c}". 


\subsection{Consistency of dependent types}\label{DepType:Consistency}\index{dependent type,consistency}

A dependent type \xcd"C{c}" may contain zero or more parameters. We require
that a type never be empty---so that it is possible for a variable of
the type to contain a value. This is accomplished by requiring that
the constraint \xcd"c" must be satisfiable {\em regardless} of the value assumed
by parameters to the constraint (if any). Formally, consider a type
\xcd"T" = \xcd"C{c}", with the variables
\xcdmath"f$_1$: F$_1$, $\dots$, f$_k$: F$_k$"
free in \xcd"c".  Let 
\xcdmath"$S$ = {f$_1$: F$_1$, $\dots$, f$_k$: F$_k$, f$_{k+1}$: F$_{k+1}$, $\dots$, f$_n$: F$_n$}"
be the smallest set of
declarations containing
\xcdmath"f$_1$: F$_1$, $\dots$, f$_k$: F$_k$"
and closed under the rule:
\xcd"f: F" in $S$ if a reference to variable \xcd"f" (which
is declared as \xcd"f: F") occurs in a type in $S$.

(NOTE: The syntax rules for the language ensure that $S$ is always
finite. The type for a variable \xcd"v" cannot reference a variable whose
type depends on \xcd"v".)

We say that \xcd"T" = \xcd"C{c}" is {\em parametrically consistent} (in brief:
{\em consistent}) if:

\begin{itemize}
\item Each type \xcdmath"F$_1$, $\dots$, F$_n$" is (recursively) parametrically consistent, and
\item It can be established that
\xcdmath"$\forall$f$_1$: F$_1$, $\dots$, f$_n$: F$_n$. $\exists$self: C. c && $\mathit{inv}$(C)".
\end{itemize}

\noindent
where \xcdmath"$\mathit{inv}$(C)" is the invariant associated
with the type \xcd"C" (\Sref{DepType:TypeInvariant}).  Note by
definition of $S$ the formula above has no free variables.

\begin{staticrule*}
For a declaration \xcd"v: T" to be type-correct, \xcd"T" must be parametrically
consistent. The compiler issues an error if it cannot determine
the type is parametrically consistent.
\end{staticrule*}

\begin{example}

A class that represents a line has two distinct points:\footnote{We call them
\xcd`Position` to avoid confusion with the built-in class \xcd`Point`}

%~~gen
% 
%~~vis
\begin{xten}
class Position(x: Int, y: Int) {
   def this(x:Int,y:Int){property(x,y);}
   }
class Line(start: Position, 
           end: Position{self != start}) {}
\end{xten}
\end{example}
%~~siv
%~~neg


\begin{example}
One can use dependent type to define other closed geometric figures as well.

To see that the declaration \xcd"end: Position{self != start}" is
parametrically consistent, note that the following formula is valid:
\begin{xtenmath}
$\forall$this: Line. $\exists$self: Position. self != this.start  
\end{xtenmath}
\noindent since the set of all \xcd"Position"s has more than one element.
\end{example}

\begin{example}
A triangle has three lines sharing three vertices.

%~~gen
%package triangleExample;
% class Position(x: Int, y: Int) {
%    def this(x:Int,y:Int){property(x,y);}
%    }
% class Line(start: Position, 
%            end: Position{self != start}) {}
% 
%~~vis
\begin{xten}
class Triangle 
 (a: Line, 
  b: Line{a.end == b.start}, 
  c: Line{b.end == c.start && c.end == a.start}) 
 {
   def this(a:Line,
            b: Line{a.end == b.start}, 
            c: Line{b.end == c.start && c.end == a.start}) 
   {property(a,b,c);}
 }
\end{xten}
%~~siv
%
%~~neg

Given \xcd"a: Line", the type \xcd"b: Line{a.end == b.start}" is consistent,
and
given the two, the type \xcd"c: Line{b.end == c.start, c.end == a.start}"
is consistent.

%%Similarly:
%%
%%   // A class with properties a, b,c,x satisfying the 
%%   // given constraints.
%%   class SolvableQuad(a: Int, b: Int, 
%%                      c: Int{b*b - 4*a*c >= 0},
%%                      x: Int{a*x*x + b*x + c==0}) { 
%%     ...
%%   }
%%
%%  // A class with properties r, x, and y satisfying
%%  // the conditions for (x,y) to lie on a circle with center (0,0)
%%  // and radius r.
%%   class Circle (r: Int{r > 0},
%%                 x: Int{r*r - x*x >= 0},
%%                 y: Int{y*y == r*r -x*x}) { 
%%   ...
%%   }
\end{example}

\section{Function types}
\label{FunctionTypes}
\label{FunctionType}
\index{function!types}
\index{types!function types}

        Function types are defined via the \xcd"=>" type
        constructor.  Closures (\Sref{Closures}) and method
        selectors (\Sref{MethodSelectors}) are of function type.
        The general form of a function type is:
\begin{xtenmath}
(x$_1$: T$_1$, $\dots$, x$_n$: T$_n$){c} => T
        throws S$_1$, $\dots$, S$_k$
\end{xtenmath}
        This
        is the type of functions that take 
        value parameters
        \xcdmath"x$_i$"
        of types
        \xcdmath"T$_i$"
        such that the guard \xcd"c" holds
        and returns a value of type \xcd"T" or throws
        exceptions of 
        types S$_i$.

The value parameters are in scope throughout the function
signature---they may be used in the types of other formal parameters
and in the return type.  Value parameters names  may be
omitted if they are not used.  The guard specifies a condition that 
must hold for an application to be well-typed.

\begin{grammar}
FunctionType \: TypeParameters\opt \xcd"(" Formals\opt \xcd")" Constraint\opt
\xcd"=>" Type Throws\opt \\
TypeParameters \: \xcd"[" TypeParameter ( \xcd"," TypeParameter
)\star \xcd"]" \\
TypeParameter \: Identifier \\
Formals \: Formal ( \xcd"," Formal )\star \\
\end{grammar}


For every sequence of types \xcd"T1,..., Tn,T", and \xcd"n" distinct variables
\xcd"x1,...,xn" and constraint \xcd"c", the expression
\xcd"(x1:T1,...,xn:Tn){c}=>T" is a \emph{function type}. It stands for
 the set of all functions \xcd"f" which can be applied in a place \xcd"p" to a
 list of values \xcd"(v1,...,vn)" provided that the constraint
 \xcd"c[v1,...,vn,p/x1,...,xn,here]" is true, and which returns a value of
 type \xcd"T[v1,...vn/x1,...,xn]". When \xcd"c" is true, the clause \xcd"{c}" can be
 omitted. When \xcd"x1,...,xn" do not occur in \xcd"c" or \xcd"T", they can be
 omitted. Thus the type \xcd"(T1,...,Tn)=>T" is actually shorthand for
 \xcd"(x1:T1,...,xn:Tn){true}=>T", for some variables \xcd"x1,...,xn".


Juxtaposition is used to express function application: the expression
\xcd"f(a1,..,an)" expresses the application of a function \xcd"f" to the argument
list \xcd"a1,...,an".

\index{Exception!unchecked}
Note that function invocation may throw unchecked exceptions. 

A function type is covariant in its result type and contravariant in
each of its argument types. That is, let 
\xcd"S1,...,Sn,S,T1,...Tn,T" be any
types satisfying \xcd"Si <: Ti" and \xcd"S <: T". Then
\xcd"(x1:T1,...,xn:Tn){c}=>S" is a subtype of
\xcd"(x1:S1,...,xn:Sn){c}=>T".


A value \xcd"f" of a function type \xcd"(x1:T1,...,xn:Tn){c}=>T" also
has all the methods of \Xcd{Any} associated with it (see \Sref{FunctionAnyMethods}).


A function type \xcd"F=(x1:T1,...,xn:Tn){c}=>T" can be used as the declared type of local variables, parameters, loop variables, return types of methods and in \xcd"_� instanceof F" and \xcd"_ as F" expressions. 


A class or struct definition may use a function type \xcd"F" in its
implements clause; this declares an abstract method 
\xcd"def apply(x1:T1,...,xn:Tn){c}:T" on that class. Similarly, an interface
definition may specify a function type "F" in its "extends" clause. A
class or struct implementing such an interface implicitly defines an
abstract method \xcd"def apply(x1:T1,..,xn:Tn){c}:T". Expressions of such
a struct, class or interface type can be assigned to variables of type
\xcd"F" and can be applied via juxtaposition to an argument list of the
right type.


Thus, objects and structs in \Xten{} may behave like functions. 

A function type \xcd"F" is not a class type in that it does not extend any
type or implement any interfaces, or support equality tests. \xcd"F" cannot be extended by any type. It
is not an interface type in that it is not a subtype of
\xcd"x10.lang.Object". (Values of type \xcd"F" cannot be assigned to variables of
type \xcd"x10.lang.Object".) It is not a struct type in that it has no
defined fields and hence no notion of structural equality.

\xcd"null" is a legal value for a function type. 


\section{Annotated types}
\label{AnnotatedTypes}

\index{types!annotated types}
\index{annotations!type annotations}

        Any \Xten{} type may be annotated with zero or more
        user-defined \emph{type annotations}
        (\Sref{XtenAnnotations}).  

        Annotations are defined as (constrained) interface types and are
        processed by compiler plugins, which may interpret the
        annotation symbolically.

        A type \xcd"T" is annotated by interface types
        \xcdmath"A$_1$", \dots,
        \xcdmath"A$_n$"
        using the syntax
        \xcdmath"@A$_1$ $\dots$ @A$_n$ T".

\section{Subtyping and type equivalence}\label{DepType:Equivalence}
\index{type equivalence}
\index{subtyping}

Subtyping is relation between types.  It is the
reflexive, transitive 
closure of the {\em direct subtyping} relation, defined as
follows.

\paragraph{Class types.}  A class type is a direct subtype of
any
class it is declared to extend.  A class type is direct subtype
of any interfaces it is declared to implement.

\paragraph{Interface types.}  An interface type is a direct
subtype of any interfaces it is declared to extend.

\paragraph{Function types.}

Function types are covariant on their return type and
contravariant on their argument types.
For instance,
a function type
\xcd"(S1) => T1" 
is a subtype of another function type
\xcd"(S2) => T2" 
if \xcd"S2" is a subtype of \xcd"S1"
and \xcd"T1" is a subtype of \xcd"T2".

\paragraph{Constrained types.}

Two dependent types \xcd"C{c}" and \xcd"C{d}" are said to be {\em equivalent} if 
\xcd"c" is true whenever \xcd"d" is, and vice versa. Thus, 
$\LL{\mbox{\Xcd{C\{c\}}}} = \LL{\mbox{\Xcd{C\{d\}}}}$.

Note that two dependent type that are syntactically different may be
equivalent. For instance, \xcd"Int{self >= 0}" and
\xcd"Int{self == 0 || self > 0}" are equivalent though they are syntactically
distinct. The \Java{} type system is essentially a nominal system---two
types are the same if and only if they have the same name. The \Xten{}
type system extends the nominal type system of \Java{} to permit
constraint-based equivalence.

A dependent type \xcd"C{c}" is a subtype of a type \xcd"C{d}" if
\xcd"c" implies \xcd"d".  When this subtyping relationship holds, 
$\LL{\mbox{\Xcd{C\{c\}}}}$ is a
subset of $\LL{\mbox{\Xcd{C\{d\}}}}$. All dependent types
defined on a class \xcd"C"
refine the unconstrained class type \xcd"C"; \xcd"C" is
equivalent to \xcd"C{true}".

\paragraph{Type parameters.}

A type parameter \xcd"X" of a class or interface \xcd"C"
is a subtype of a type \xcd"T" if
the 
class invariant of \xcd"C" implies that \xcd"X" is a subtype of \xcd"T".
Similarly, \xcd"T" is a subtype of parameter \xcd"X" if the
class invariant implies the relationship.

A type parameter \xcd"X" of a method
\xcd"m"
is a subtype of a type \xcd"T" if
the 
guard of \xcd"m" implies that \xcd"X" is a subtype of \xcd"T".
Similarly, \xcd"T" is a subtype of parameter \xcd"X" if the
guard implies the relationship.


\section{Least common ancestor of types}
\label{LCA}

To compute the type of conditional expressions
(\Sref{Conditional}),
and of rail constructors
(\Sref{RailConstructors}), the least common ancestor of types
must be computed.

The least common ancestor of two  types
\xcdmath"T$_1$" and \xcdmath"T$_2$"
is the
unique most-specific type
that is a supertype of both
\xcdmath"T$_1$" and \xcdmath"T$_2$".

If the most-specific type is not unique (which can happen when
\xcdmath"T$_1$" and \xcdmath"T$_2$" both implement two
or more incomparable interfaces), then
least common ancestor type is \xcd"x10.lang.Any".

\section{Coercions and conversions}
\label{XtenConversions}
\label{User-definedCoercions}
\index{conversions}\index{coercions}

\XtenCurrVer{} supports the following coercions and conversions

\subsection{Coercions}

A {\em coercion} does not change object identity; a coerced object may
be explicitly coerced back to its original type through a cast. A {\em
  conversion} may change object identity if the type being converted
to is not the same as the type converted from. \Xten{} permits
user-defined conversions (\Sref{sec:user-defined-conversions}).

\paragraph{Subsumption coercion.}
A subtype may be implicitly coerced to any supertype.
\index{coercions!subsumption coercion}

\paragraph{Explicit coercion (casting with \xcd"as")}
A reference type may be explicitly coerced to any other
reference type using the \xcd"as" operation.
If the value coerced is not an instance of the target type,
a \xcd"ClassCastException" is thrown.  Casting to a constrained
type may require a run-time check that the constraint is
satisfied.
\index{coercions!explicit coercion}
\index{casting}
\index{\Xcd{as}}

\subsection{Conversions}

\paragraph{Narrowing conversion.}
A value class may be explicitly converted to any superclass
using the \xcd"as" operation.

\index{conversions!narrowing conversions}

\paragraph{Widening numeric conversion.}
A numeric type may be implicitly converted to a wider numeric type. In
particular, an implicit conversion may be performed between a numeric
type and a type to its right, below:

\begin{xten}
Byte < Short < Int < Long < Float < Double
\end{xten}

\index{conversions!widening conversions}
\index{conversions!numeric conversions}

\paragraph{String conversion.}
Any object that is an operand of the binary
\xcd"+" operator may
be converted to \xcd"String" if the other operand is a \xcd"String".
A conversion to \xcd"String" is performed by invoking the \xcd"toString()"
method of the object.

\index{conversions!string conversion}

\paragraph{User defined conversions.}\label{sec:user-defined-conversions}
\index{conversions!user defined}

The user may define conversion operators from type \Xcd{A} {\em to} a
container type \Xcd{B} by specifying a method on \Xcd{B} as follows:

\begin{xten}
  public static operator (r: A): T = ... 
\end{xten}

The return type \Xcd{T} should be a subtype of \Xcd{B}. The return
type need not be specified explicitly; it will be computed in the
usual fashion if it is not. However, it is good practice for the
programmer to specify the return type for such operators explicitly.

For instance, the code for \Xcd{x10.lang.Point} contains:

\begin{xten}
  public static global safe operator (r: Rail[int])
     : Point(r.length) = make(r);
\end{xten}

The compiler looks for such operators on the container type \Xcd{B}
when it encounters an expression of the form \Xcd{r as B} (where
\Xcd{r} is of type \Xcd{A}). If it finds such a method, it sets the
type of the expression \Xcd{r as B} to be the return type of the
method. Thus the type of \Xcd{r as B} is guaranteed to be some subtype
of \Xcd{B}.

\begin{example}
Consider the following code:  
\begin{xten}
val p  = [2, 2, 2, 2, 2] as Point;
val q = [1, 1, 1, 1, 1] as Point;
val a = p - q;    
\end{xten}
This code fragment compiles successfully, given the above operator definition. 
The type of \Xcd{p} is inferred to be \Xcd{Point(5)} (i.e.{} the type 
\xcd"Point{self.rank==5}".
Similarly for \Xcd{q}. Hence the application of the operator ``\Xcd{-}'' is legal (it requires both arguments to have the same rank). The type of \Xcd{a} is computed as \Xcd{Point(5)}.
\end{example}


%\subsection{Syntactic abbreviations}\label{DepType:SyntaxAbbrev}

\section{Built-in types}

The package \xcd"x10.lang" provides a number of built-in class and
interface declarations that can be used to construct types.

\subsection{The class \Xcd{Object}}
\label{Object}
\index{\Xcd{Object}}
\index{\Xcd{x10.lang.Object}}

The class \xcd"x10.lang.Object" is the supertype of all classes.
A variable of this type can hold a reference to any object.
The code for this class (with annotations removed) is:
\begin{xten}
public class Object (home: Place) 
     implements Any
{
    public native def this();
    public property def home() = home;
    public property def at(p:Place) = home==p;
    public property def at(r:Object) = home==r.home;
    public global safe native def toString() : String;
    public global safe native def typeName() : String;
    public global safe def equals(x:Any) = this == x;
    public global safe native def hashCode():Int;
}
\end{xten}

\subsection{The class \Xcd{String}}
\label{String}\index{\Xcd{String}}\index{\Xcd{x10.lang.String}}

Strings in \Xten{} are instances of the class \xcd"x10.lang.String", and are
all immutable.
Strings are one of the few types with literals, rather than simply
      constructors.  String literals are the familiar \xcd`"`-delimited
      strings, like \xcd`"this"` and \xcd`"that"`.

Every X10 value has a \xcd`String` print representation, given by
      \xcd`whatever.toString()`.   
All values can be implicitly converted to strings by the concatenation
      operation \xcd`+`, which calls their \xcd`toString()` methods if they
      are not strings already.  For example, 
%~~exp~~`~~`~~ ~~
      \xcd`"one " + 2 + here` 
      evaluates to something like \xcd`one 2(Place 0)`.  



\section{Array Type Constructors}
\label{ArrayypeConstructors}\index{array types}

{}\XtenCurrVer{} does not have array class declarations
(\S~\ref{XtenArrays}). This means that user cannot define new array
class types. Instead arrays are created as instances of array types
constructed through the application of {\em array type constructors}
(\S~\ref{XtenArrays}).

The array type constructor takes as argument a type (the {\em base
type}), an optional distribution (\S~\ref{XtenDistributions}), and
optionally either the keyword {\cf reference} or {\cf value} (the
default is reference):
\begin{x10}
18    ArrayType ::= Type [ ]
438   ArrayType ::= X10ArrayType
439   X10ArrayType ::= Type [ . ]
440     | Type reference [ . ]
441     | Type value [ . ]
442     | Type [ DepParameterExpr ]
443     | Type reference [ DepParameterExpr ]
444     | Type value [ DepParameterExpr ]
\end{x10}

The array type {\cf Type[ ] } is the type of all arrays of base
type {\tt Type} defined over the distribution {\tt 0:N -> here}
for some positive integer {\tt N}.

The qualifier {\tt value} ({\tt reference}) specifies that the array
is a {\tt value}({\tt reference}) array. The array elements of a {\tt
value} array are all {\tt final}.\footnote{Note that the base type of a {\tt value} array can be a value class or a reference class, just as the 
type of a {\tt final} variable can be a value class or a reference class.
}If the qualifier is not specified,
the array is a {\tt reference} array.

The array type {\cf Type reference [.]} is the type of all (reference)
arrays of basetype {\tt Type}. Such an array can take on any
distribution, over any region. Similarly, {\cf Type value [.]} is the
type of all value arrays of basetype {\tt Type}.

\XtenCurrVer{} also allows a distribution to be specified between {\tt
[} and {\tt ]}. The distribution must be an expression of type
{\tt distribution} (e.g.{} a {\tt final} variable) whose
value does not depend on the value of any mutable variable.

Future extensions to \Xten{} will support a more general syntax for
arrays which allows for the specification of dependent types, 
e.g.{} {\tt double[:rank 3]}, the type of all arrays of 
{\tt double} of rank {\tt 3}.



\subsection{Future types}

The class \xcd"x10.lang.Future[T]"
is the type of all \xcd"future" expressions.
The type represents a value which when forced will return a value of type
\xcd"T". The class makes available the following methods:

\begin{xten}
package x10.lang;
public class Future[T] implements () => T {
  public def apply(): T = force();
  public def force(): T = ...;
  public def forced(): Boolean = ...;
}
\end{xten}
  


\section{Type inference}
\label{TypeInference}
\index{types!inference}
\index{type inference}

\XtenCurrVer{} supports limited local type inference, permitting
variable types and return types to be elided.
It is a static error if an omitted type cannot be inferred or
uniquely determined.

\subsection{Variable declarations}

The type of a variable declaration can be omitted if the
declaration has an initializer.  The inferred type of the
variable is the computed type of the initializer.

\subsection{Return types}

The return type of a method can be omitted if the method has a
body (i.e., is not \xcd"abstract" or \xcd"extern").  The
inferred return type is the computed type of the body.

The return type of a closure can be omitted.
The inferred return type is the computed type of the body.

The return type of a constructor can be omitted if the
constructor has a body (i.e., is not \xcd"extern").
The inferred return type is the enclosing class type with
properties bound to the arguments in the constructor's \xcd"property"
statement, if any, or to the unconstrained class type.

\index{Void}
The inferred type of a method or closure body is the least common ancestor
of the types of the expressions in \xcd"return" statements
in the body.  If the method does not return a value, the
inferred type is \xcd"Void".

\subsection{Type arguments}

A call to a polymorphic method %, closure, or constructor 
may omit the
explicit type arguments.  If the method has a type parameter
\xcd"T", the type argument corresponding to \xcd"T" is inferred
to be the least common ancestor of the types of any formal
parameters of type \xcd"T".

%TODO--check this!
Consider the following method:
\begin{xten}
def choose[T](a: T, b: T): T { ... }
\end{xten}
%
Given \xcd"Set[T] <: Collection[T]", 
\xcd"List[T] <: Collection[T]",
and \xcd"SubClass <: SuperClass",
in the following snippet, the algorithm will infer the type
\xcd"Collection[Any]" for \xcd"x".
\begin{xten}
def m(intSet: Set[Int], stringList: List[String]) {
  val x = choose(intSet, stringList);
...
}
\end{xten}
%
And in this snippet, the algorithm should infer the type
\xcd"Collection[Int]" for \xcd"y".
\begin{xten}
def m(intSet: Set[Int], intList: List[Int]) {
  val y = choose(intSet, intList);
  ...
}
\end{xten}
%
Finally, in this snippet, the algorithm should infer the type
\xcd"Collection{T <: SuperClass}" for \xcd"z".
\begin{xten}
def m(intSet: Set[SubClass], numList: List{T <: SuperClass}) {
  val z = choose(intSet, numList);
  ...
}
\end{xten}

	

\chapter{Variables}\label{XtenVariables}\index{variable}

%%OLDA variable is a storage location.  \Xten{} supports seven kinds of
%%OLDvariables: constant {\em class variables} (static variables), {\em
%%OLD  instance variables} (the instance fields of a class), {\em array
%%OLD  components}, {\em method parameters}, {\em constructor parameters},
%%OLD{\em exception-handler parameters} and {\em local variables}.

A {\em variable} is an X10 identifier associated with a value within some
context. Variable bindings have these essential properties:
\begin{itemize}
\item {\bf Type:} What sorts of values can be bound to the identifier;
\item {\bf Scope:} The region of code in which the identifier is associated
      with the entity;
\item {\bf Lifetime:} The interval of time in which the identifier is
      associated with the entity.
\item {\bf Visibility:} Which parts of the program can read or manipulate the
      value through the variable.
\end{itemize}



X10 has many varieties of variables, used for a number of purposes. They will
be described in more detail in this chapter.  
\begin{itemize}
\item Class variables, also known as the static fields of a class, which hold
      their values for the lifetime of the class.  
\item Instance variables, which hold their values for the lifetime of an
      object;
\item Array elements, which are not individually named and hold their values
      for the lifetime of an array;
\item Formal parameters to methods, functions, and constructors, which hold
      their values for the duration of method (etc.) invocation;
\item Local variables, which hold their values for the duration of execution
      of a block.
\item Exception-handler parameters, which hold their values for the execution
      of the exception being handled. 
\end{itemize}
A few other kinds of things are called variables for historical reasons; \eg,
type parameters are often called type variables, despite not being variables
in this sense because they do not refer to X10 values.  Other named entities,
such as classes and methods, are not called variables.  However, all
name-to-whatever bindings enjoy similar concepts of scope and visibility.  

In the following example, \xcd`n` is an instance variable, and \xcd`nxt` is a
local variable defined within the method \xcd`bump`.\footnote{This code is
unnecessarily turgid for the sake of the example.  One would generally write
\xcd`public def bump() = ++n;`.   }
%~~gen
% package Vars.For.Squares;
%~~vis
\begin{xten}
class Counter {
  private var n : Int = 0;
  public def bump() : Int {
    val nxt = n+1;
    n = nxt;
    return nxt;
    }
}
\end{xten}
%~~siv
%
%~~neg
Both variables have type \xcd`Int` (or
perhaps something more specific).    The scope of \xcd`n` is the body of
\xcd`Counter`; the scope of \xcd`nxt` is the body of \xcd`bump`.  The
lifetime of \xcd`n` is the lifetime of the \xcd`Counter` object holding it;
the lifetime of \xcd`nxt` is the duration of the call to \xcd`bump`. Neither
variable can be seen from outside of its scope.

\label{exploded-syntax}
\label{VariableDeclarations}
\index{variable declaration}


Variables whose value may not be changed after initialization are said to be
{\em immutable}, or {\em constants} (\Sref{FinalVariables}), or simply
\xcd`val` variables. Variables whose value may change are {\em mutable} or
simply \xcd`var` variables. \xcd`var` variables are declared by the \xcd`var`
keyword. \xcd`val` variables may be declared by the \xcd`val` keyword; when a
variable declaration does not include either \xcd`var` or \xcd`val`, it is
considered \xcd`val`. 


%~~gen
%package Vars.For.Bears.In.Chairs;
%class VarExample{
%static def example() {
%~~vis
\begin{xten}
val a : Int = 0;               // Full 'val' syntax
b : Int = 0;                   // 'val' implied
val c = 0;                     // Type inferred
var d : Int = 0;               // Full 'var' syntax
var e : Int;                   // Not initialized
var f : Int{self != 100} = 0;  // Constrained type
\end{xten}
%~~siv
%}}
%~~neg







\section{Immutable variables}
\label{FinalVariables}
\index{variable!immutable}
\index{immutable variable}
\index{variable!val}
\index{val}

Immutable (\xcd`val`) variables can be given values (by initialization or assignment) at
most once, and must be given values before they are used.  Usually this is
achieved by declaring and initializing the variable in a single statement.
%~~gen
% package Vars.In.Snares;
% class ABitTedious{
% def example() {
%~~vis
\begin{xten}
val a : Int = 10;
val b = (a+1)*(a-1);
\end{xten}
%~~siv
%}}
%~~neg
\xcd`a` and \xcd`b` cannot be assigned to further.

In other cases, the declaration and assignment are separate.  One such
case is how constructors give values to \xcd`val` fields of objects.  The
\xcd`Example` class has an immutable field \xcd`n`, which is given different
values depending on which constructor was called. \xcd`n` can't be given its
value by initialization when it is declared, since it is not knowable which
constructor is called at that point.  
%~~gen
% package Vars.For.Cares;
%~~vis
\begin{xten}
class Example {
  val n : Int; // not initialized here
  def this() { n = 1; }
  def this(dummy:Boolean) { n = 2;}
}
\end{xten}
%~~siv
%
%~~neg

Another common case of separating declaration and assignment is in function
and method call.  The formal parameters are bound to the corresponding actual
parameters, but the binding does not happen until the function is called.  In
the code below, \xcd`x` is initialized to \xcd`3` in the first call and
\xcd`4` in the second.
%~~gen
%package Vars.For.Swears;
%class Examplement {
%static def whatever() {
%~~vis
\begin{xten}
val sq = (x:Int) => x*x;
x10.io.Console.OUT.println("3 squared = " + sq(3));
x10.io.Console.OUT.println("4 squared = " + sq(4));
\end{xten}
%~~siv
%}}
%~~neg





%%IMMUTABLE%% An immutable variable satisfies two conditions: 
%%IMMUTABLE%% \begin{itemize}
%%IMMUTABLE%% \item it can be assigned to at most once, 
%%IMMUTABLE%% \item it must be assigned to before use. 
%%IMMUTABLE%% \end{itemize}
%%IMMUTABLE%% 
%%IMMUTABLE%% \Xten{} follows \java{} language rules in this respect \cite[\S
%%IMMUTABLE%% 4.5.4,8.3.1.2,16]{jls2}. Briefly, the compiler must undertake a
%%IMMUTABLE%% specific analysis to statically guarantee the two properties above.
%%IMMUTABLE%% 
%%IMMUTABLE%% Immutable local variables and fields are defined by the \xcd"val"
%%IMMUTABLE%% keyword.  Elements of value arrays are also immutable.
%%IMMUTABLE%% 
%%IMMUTABLE%% \oldtodo{Check if this analysis needs to be revisited.}

\section{Initial values of variables}
\label{NullaryConstructor}\index{nullary constructor}
\index{initial value}
\index{initialization}


Every assignment, binding, or initialization to a variable of type \xcd`T{c}`
must be an instance of type \xcd`T` satisfying the constraint \xcd`{c}`.
Variables must be given a value before they are used. This may be done by
initialization, which is the only way for immutable (\xcd`val`) variables and
one option for mutable (\xcd`var`) ones: 

%~~gen
%package Vars.For.Bears;
%class VarsForBears{
%def check() {
%~~vis
\begin{xten}
  val immut : Int = 3;
  var mutab : Int = immut;
  val use = immut + mutab;
\end{xten}
%~~siv
%}}
%~~neg
Or, for mutable variables, it may be done by a later assignment.  

%~~gen
%package Vars.For.Stars;
%class VarsForStars{
%def check() {
%~~vis
\begin{xten}
  var muta2 : Int;
  muta2 = 4;
  val use = muta2 * 10;
\end{xten}
%~~siv
%}}
%~~neg


Every class variable must be initialized before it is read, through
the execution of an explicit initializer. Every
instance variable must be initialized before it is read, through the
execution of an explicit or implicit initializer or a constructor.
Implicit initializers initialize \xcd`var`s to the default values of their
types (\Sref{DefaultValues}). Variables of types which do not have default
values are not implicitly initialized.



Each method and constructor parameter is initialized to the
corresponding argument value provided by the invoker of the method. An
exception-handling parameter is initialized to the object thrown by
the exception. A local variable must be explicitly given a value by
initialization or assignment, in a way that the compiler can verify
using the rules for definite assignment \cite[\S~16]{jls2}.


\section{Destructuring syntax}
\index{variable declarator!destructuring}
\index{destructuring}
\Xten{} permits a \emph{destructuring} syntax for local variable
declarations with explicit initializers,  and for formal parameters, of type \xcd`Point`, \Sref{point-syntax}.
(Future versions of X10 may allow destructuring of other types as well.) 
A point is a sequence of {$r \ge 0$} \xcd`Int`-valued coordinates.  
It is often useful to get at the coordinates directly, in variables. 

The following code makes an anonymous point with one coordinate \xcd`11`, and
binds \xcd`i` to \xcd`11`.  Then it makes a point with coordinates \xcd`22`
and \xcd`33`, binds \xcd`p` to that point, and \xcd`j` and \xcd`k` to \xcd`22`
and \xcd`33` respectively.
%~~gen
% package Vars.For.Glares;
% class DestructuringEx1 {
% def whyJustForLocals() {
%~~vis
\begin{xten}
val [i] : Point = Point.make(11);
val p[j,k] = Point.make(22,33);
val q[l,m] = [44,55]; // coerces an array to a point.
\end{xten}
%~~siv
%}}
%~~neg

A useful idiom for iterating over a range of numbers is: 
%~~gen
%package Vars.For.Bears;
% class ForBear {
% def forbear() {
%~~vis
\begin{xten}
var sum : Int = 0;
for ([i] in 1..100) sum += i;
\end{xten}
%~~siv
% ; } } 
%~~neg
\noindent
The brackets in \xcd`[i]` introduce destructuring, making X10 treat \xcd`i`
as an \xcd`Int`; without them, it would be a \xcd`Point`.  

In general, a pattern of the form \xcdmath"[i$_1$,$\ldots$,i$_n$]" matches a
point with {$n$} coordinates, binding \xcdmath"i$_j$" to coordinate {$j$}.  
A pattern of the form \xcdmath"p[i$_1$,$\ldots$,i$_n$]" does the same,  but
also binds \xcd`p` to the point.

\section{Formal parameters}
\index{formal parameter}
\index{parameter}

\begin{bbgrammar}
 FormalParams    \: \xcd"(" FormalParamList\opt \xcd")" & (\ref{prod:FormalParams})\\%<FROM #(prod:FormalParams)#
 FormalParamList    \: FormalParam & (\ref{prod:FormalParamList})\\%<FROM #(prod:FormalParamList)#
    \| FormalParamList \xcd"," FormalParam\\
 FormalParam    \: Mods\opt FormalDeclarator & (\ref{prod:FormalParam})\\%<FROM #(prod:FormalParam)#
    \| Mods\opt VarKeyword FormalDeclarator\\
    \| Type\\
 FormalDeclarators    \: FormalDeclarator & (\ref{prod:FormalDeclarators})\\%<FROM #(prod:FormalDeclarators)#
    \| FormalDeclarators \xcd"," FormalDeclarator\\
 FormalDeclarator    \: Id ResultType & (\ref{prod:FormalDeclarator})\\%<FROM #(prod:FormalDeclarator)#
    \| \xcd"[" IdList \xcd"]" ResultType\\
    \| Id \xcd"[" IdList \xcd"]" ResultType\\
 ResultType    \: \xcd":" Type & (\ref{prod:ResultType})\\%<FROM #(prod:ResultType)#
\end{bbgrammar}

Formal parameters are the variables which hold values transmitted into a
method or function.  
They are always declared with a type.  (Type inference is not
available, because there is no single expression to deduce a type from.)
The variable name can be omitted if it is not to be used in the
scope of the declaration, as in the type of the method 
\xcd`static def main(Array[String]):void` executed at the start of a program that
does not use its command-line arguments.

\xcd`var` and \xcd`val` behave just as they do for local
variables, \Sref{local-variables}.  In particular, the following \xcd`inc`
method is allowed, but, unlike some languages, does {\em not} increment its
actual parameter.  \xcd`inc(j)` creates a new local 
variable \xcd`i` for the method call, initializes \xcd`i` with the value of
\xcd`j`, increments \xcd`i`, and then returns.  \xcd`j` is never changed.
%~~gen
% package Vars.For.Squares.Of.Mares;
% class Ink {
%~~vis
\begin{xten}
static def inc(var i:Int) { i += 1; }
\end{xten}
%~~siv
%}
%~~neg


\section{Local variables}\label{local-variables}
\index{variable!local}
\index{local variable}
Local variables are declared in a limited scope, and, dynamically, keep their
values only for so long as the scope is being executed.  They may be \xcd`var`
or \xcd`val`.  
They may have 
initializer expressions: \xcd`var i:Int = 1;` introduces 
a variable \xcd`i` and initializes it to 1.
If the variable is immutable
(\xcd"val")
the type may be omitted and
inferred from the initializer type (\Sref{TypeInference}).

The variable declaration \xcd`val x:T=e;` confirms that \xcd`e`'s value is of
type \xcd`T`, and then introduces the variable \xcd`x` with type \xcd`T`.  For
example, consider a class Tub with a property \xcd`p`.
%~~gen
% package Vars.Local;
%~~vis
\begin{xten}
class Tub(p:Int){
  def this(pp:Int):Tub{self.p==pp} {property(pp);}
  def example() {
    val t : Tub = new Tub(3);
  }
}
\end{xten}
%~~siv
%
%~~neg
\noindent
produces a variable \xcd`t` of type \xcd`Tub`, even though the expression
\xcd`new Tub(3)` produces a value of type \xcd`Tub{self.p==3}` -- that is, a
\xcd`Tub`  whose \xcd`p` field is 3.  This can be inconvenient when the
constraint information is required.

\index{\Xcd{<:}}
Including type information in variable declarations is generally good
programming practice: it explains to both the compiler and human readers
something of the intent of the variable.  However, including types in 
\xcd`val t:T=e` can obliterate helpful information.  So, X10 allows a {\em
documentation type declaration}, written \xcd`val t <: T = e`.  This 
has the same effect as \xcd`val t = e`, giving \xcd`t` the full type inferred
from \xcd`e`; but it also confirms statically that that type is at least
\xcd`T`.  For example, the following gives \xcd`t` the type \xcd`Tub{self.p==3}` as
desired: 
%~~gen
% package Vars.Local;
% class TubBounded{
% def example() {
%~~vis
\begin{xten}
   val t <: Tub = new Tub(3);
\end{xten}
%~~siv
%}}
%~~neg
\noindent
However, replacing \xcd`Tub` by \xcd`Int` would result in a compilation error. 

Variables do not need to be initialized at the time of definition -- not even
\xcd`val`s. They must be initialized by the time of use, and \xcd`val`s may
only be assigned to once. The X10 compiler performs static checks guaranteeing
this restriction. The following is correct, albeit obtuse: 
%~~gen
%package Vars.Local;
% class NotInitVal {
%~~vis
\begin{xten}
static def main(r: Array[String](1)):void {
  val a : Int;
  a = r.size;
  val b : String;
  if (a == 5) b = "five?"; else b = "" + a + " args"; 
  // ... 
\end{xten}
%~~siv
%} }
%~~neg



\section{Fields}
\index{field}
\index{object!field}
\index{struct!field}
\index{class!field}

\begin{bbgrammar}
 FieldDeclarators    \: FieldDeclarator & (\ref{prod:FieldDeclarators})\\%<FROM #(prod:FieldDeclarators)#
    \| FieldDeclarators \xcd"," FieldDeclarator\\
 FieldDecl    \: Mods\opt FieldKeyword FieldDeclarators \xcd";" & (\ref{prod:FieldDecl})\\%<FROM #(prod:FieldDecl)#
    \| Mods\opt FieldDeclarators \xcd";"\\
 FieldDeclarator    \: Id HasResultType & (\ref{prod:FieldDeclarator})\\%<FROM #(prod:FieldDeclarator)#
    \| Id HasResultType\opt \xcd"=" VariableInitializer\\
 HasResultType    \: \xcd":" Type & (\ref{prod:HasResultType})\\%<FROM #(prod:HasResultType)#
    \| \xcd"<:" Type\\
 FieldKeyword    \: \xcd"val" & (\ref{prod:FieldKeyword})\\%<FROM #(prod:FieldKeyword)#
    \| \xcd"var"\\
 Mod    \: \xcd"abstract" & (\ref{prod:Mod})\\%<FROM #(prod:Mod)#
    \| Annotation\\
    \| \xcd"atomic"\\
    \| \xcd"final"\\
    \| \xcd"native"\\
    \| \xcd"private"\\
    \| \xcd"protected"\\
    \| \xcd"public"\\
    \| \xcd"static"\\
    \| \xcd"transient"\\
    \| \xcd"clocked"\\

\end{bbgrammar}

Like most other kinds of variables in X10, 
the fields of an object can be either \xcd`val` or \xcd`var`. 
Fields can be \xcd`static`,\xcd`global`, or \xcd`property`; see
\Sref{FieldDefinitions} and \Sref{PropertiesInClasses}.
Field declarations may have optional
initializer expressions, as for local variables, \Sref{local-variables}.
\xcd`var` fields without an initializer are initialized with the default value
of their type. \xcd`val` fields without an initializer must be initialized by
each constructor.


For \xcd`val` fields, as for \xcd`val` local variables, the type may be
omitted and inferred from the initializer type (\Sref{TypeInference}).
\xcd`var` files, like \xcd`var` local variables, must be declared with a type.



%%GRAM%% \begin{grammar}
%%GRAM%% FieldDeclaration
%%GRAM%%         \: FieldModifier\star \xcd"var" FieldDeclaratorsWithType \\&& ( \xcd"," FieldDeclaratorsWithType )\star \\
%%GRAM%%         \| FieldModifier\star \xcd"val" FieldDeclarators \\&& ( \xcd"," FieldDeclarators )\star \\
%%GRAM%%         \| FieldModifier\star FieldDeclaratorsWithType \\&& ( \xcd"," FieldDeclaratorsWithType )\star \\
%%GRAM%% FieldDeclarators
%%GRAM%%         \: FieldDeclaratorsWithType \\
%%GRAM%%         \: FieldDeclaratorWithInit \\
%%GRAM%% FieldDeclaratorId
%%GRAM%%         \: Identifier  \\
%%GRAM%% FieldDeclaratorWithInit
%%GRAM%%         \: FieldDeclaratorId Init \\
%%GRAM%%         \| FieldDeclaratorId ResultType Init \\
%%GRAM%% FieldDeclaratorsWithType
%%GRAM%%         \: FieldDeclaratorId ( \xcd"," FieldDeclaratorId )\star ResultType \\
%%GRAM%% FieldModifier \: Annotation \\
%%GRAM%%                 \| \xcd"static" \\ \| \xcd`property` \\ \| \xcd`global` \\
%%GRAM%% \end{grammar}
%%GRAM%% 
%%GRAM%% 

%%ACC%%  \section{Accumulator Variables}
%%ACC%%  
%%ACC%%  Accumulator variables allow the accumulation of partial results to produce a
%%ACC%%  final result.  For example, an accumulator variable could compute a running
%%ACC%%  sum, product, maximum, or minimum of a collection of numbers.  In particular,
%%ACC%%  many concurrent activites can accumulate safely into the {\em same} local
%%ACC%%  variable, without need for \Xcd{atomic} blocks or other explicit coordination.  
%%ACC%%  
%%ACC%%  An accumulator variable is associated with a {\em reducer}, which explains how
%%ACC%%  new partial values are accumulated.
%%ACC%%  
%%ACC%%  \subsection{Reducers}
%%ACC%%  
%%ACC%%  A notion of accumulation has two aspects: 
%%ACC%%  \begin{enumerate}
%%ACC%%  \item A {\bf zero} value, which is the initial value of the accumulator,
%%ACC%%        before any partial results have been included.  When accumulating a sum,
%%ACC%%        the zero value is \Xcd{0}; when accumulating a product, it is \Xcd{1}.
%%ACC%%  \item A {\bf combining function}, explaining how to combine two partial
%%ACC%%        accumulations into a whole one.  When accumulating a sum, partial sums
%%ACC%%        should be added together; for a product, they should be multiplied.  
%%ACC%%  \end{enumerate}
%%ACC%%  
%%ACC%%  In X10, this is represented as a value of type
%%ACC%%  \Xcd{x10.lang.Reducer[T]}: 
%%ACC%%  %~acc~gen
%%ACC%%  %package Vars.Notx10lang.Reducerererer;
%%ACC%%  %~acc~vis
%%ACC%%  \begin{xten}
%%ACC%%  struct Reducer[T](zero:T, apply: (T,T)=>T){}
%%ACC%%  \end{xten}
%%ACC%%  %~acc~siv
%%ACC%%  %
%%ACC%%  %~acc~neg
%%ACC%%  \noindent 
%%ACC%%  If \Xcd{r:Reducer[T]}, then \Xcd{r.zero} is the zero element, and
%%ACC%%  \Xcd{r(a,b)} --- which can also be written \Xcd{r.apply(a,b)} --- is the
%%ACC%%  combination of \Xcd{a} and \Xcd{b}.
%%ACC%%  
%%ACC%%  For example, the reducers for adding and multiplying integers are: 
%%ACC%%  %~acc~gen
%%ACC%%  %package Vars.Notx10lang.Reducererererererer;
%%ACC%%  %struct Reducer[T](zero:T, apply: (T,T)=>T){}
%%ACC%%  %class Example{
%%ACC%%  %~acc~vis
%%ACC%%  \begin{xten}
%%ACC%%  val summer = Reducer[Int](0, Int.+);
%%ACC%%  val producter = Reducer[Int](1, Int.*);
%%ACC%%  \end{xten}
%%ACC%%  %~acc~siv
%%ACC%%  %}
%%ACC%%  %~acc~neg
%%ACC%%  
%%ACC%%  
%%ACC%%  Reduction is guaranteed to be deterministic if the reducer is {\em
%%ACC%%  Abelian},\footnote{This term is borrowed from abstract algebra, where such a
%%ACC%%  reducer, together with its type, forms an Abelian monoid.}
%%ACC%%  that is, 
%%ACC%%  \begin{enumerate}
%%ACC%%  \item \Xcd{r.apply} is pure; that is, has no side effects;
%%ACC%%  \item \Xcd{r.apply} is commutative; that is, \Xcd{r(a,b) == r(b,a)} for all
%%ACC%%        inputs \Xcd{a} and \Xcd{b};
%%ACC%%  \item \Xcd{r.apply} is associative; that is, 
%%ACC%%        \Xcd{r(a,r(b,c)) == r(r(a,b),c)} for all \Xcd{a}, \Xcd{b}, and \Xcd{c}.
%%ACC%%  \item \Xcd{r.zero} is the identity element for \Xcd{r.apply}; that is, 
%%ACC%%        \Xcd{r(a, r.zero) == a}
%%ACC%%        for all \Xcd{a}.
%%ACC%%  \end{enumerate}
%%ACC%%  
%%ACC%%  
%%ACC%%  
%%ACC%%  
%%ACC%%  \Xcd{summer} and \Xcd{producter} satisfy all these conditions, and give
%%ACC%%  determinate reductions. The compiler does not require or check these, though.
%%ACC%%  
%%ACC%%  
%%ACC%%  \subsection{Accumulators}
%%ACC%%  
%%ACC%%  If \Xcd{r} is a  value of type \Xcd{Reducer[T]}, then an accumulator of type
%%ACC%%  \Xcd{T} using \Xcd{r} is declared as:
%%ACC%%  %~accTODO~gen
%%ACC%%  % package Vars.Accumulators.Basic.Little.Idea;
%%ACC%%  % class C[T]{
%%ACC%%  % static def example (r:Reducer[T]) {
%%ACC%%  %~accTODO~vis
%%ACC%%  \begin{xten}
%%ACC%%  acc(r) x : T;
%%ACC%%  acc(r) y; 
%%ACC%%  \end{xten}
%%ACC%%  %~accTODO~siv
%%ACC%%  %
%%ACC%%  %~accTODO~neg
%%ACC%%  The type declaration \Xcd{T} is optional; if specified, it must be the same
%%ACC%%  type that the reducer \Xcd{r} uses.
%%ACC%%  
%%ACC%%  \subsection{Sequential Use of Accumulators}
%%ACC%%  
%%ACC%%  The sequential use of accumulator variables is straightforward, and could be
%%ACC%%  done as easily without accumulators.  (The power of accumulators is in their
%%ACC%%  concurrent use, \Sref{ConcurrentUseOfAccumulators}.)
%%ACC%%  
%%ACC%%  A variable declared as \Xcd{acc(r) x:T;} is initialized to \Xcd{r.zero}.  
%%ACC%%  
%%ACC%%  Assignment of values of \Xcd{acc} variables has nonstandard semantics.
%%ACC%%  \Xcd{x = v;} causes the value \Xcd{r(v,x)} to be stored in \Xcd{x} --- in
%%ACC%%  particular, {\em not} the value of \Xcd{v}.
%%ACC%%  
%%ACC%%  Reading a value from an accumulator retrieves the current accumulation.
%%ACC%%  
%%ACC%%  For example, the sum and product of a list \Xcd{L} of integers can be computed
%%ACC%%  by: 
%%ACC%%  %~accTODO~gen
%%ACC%%  %package Vars.Accumulators.Are.For.Bisimulators;
%%ACC%%  % import java.util.*;
%%ACC%%  % class Example{
%%ACC%%  % static def example(L: List[Int]) {
%%ACC%%  %~accTODO~vis
%%ACC%%  \begin{xten}
%%ACC%%  val summer = Reducer[Int](0, Int.+);
%%ACC%%  val producter = Reducer[Int](1, Int.*);
%%ACC%%  acc(summer) sum;
%%ACC%%  acc(producter) prod;
%%ACC%%  for (x in L) {
%%ACC%%    sum = x;
%%ACC%%    prod = x;
%%ACC%%  }
%%ACC%%  x10.io.Console.OUT.println("Sum = " + sum + "; Product = " + prod);
%%ACC%%  \end{xten}
%%ACC%%  %~accTODO~siv
%%ACC%%  %
%%ACC%%  %~accTODO~neg
%%ACC%%  
%%ACC%%  
%%ACC%%  
%%ACC%%  \subsection{Concurrent Use of Accumulators}
%%ACC%%  \label{ConcurrentUseOfAccumulators}
%%ACC%%  \index{accumulator!and activities}
%%ACC%%  
%%ACC%%  Accumulator variables are restricted and synchronized in ways that make them
%%ACC%%  ideally suited for concurrent accumulation of data.   The {\em governing
%%ACC%%  activity} of an accumulator is the activity in which the \Xcd{acc} variable is
%%ACC%%  declared.  
%%ACC%%  
%%ACC%%  \begin{enumerate}
%%ACC%%  \item The governing activity can read the accumulator at any point that it has
%%ACC%%        no running sub-activities.  
%%ACC%%  \item Any activity that has lexical access to the accumulator can write to it.  
%%ACC%%        All
%%ACC%%        writes are performed atomically, without need for \Xcd{atomic} or other
%%ACC%%        concurrency control.
%%ACC%%  \end{enumerate}
%%ACC%%  
%%ACC%%  If the reducer is Abelian, this guarantees that \Xcd{acc} variables cannot
%%ACC%%  cause race conditions; the result of such a computation is determinate,
%%ACC%%  independent of the scheduling of activities. Read-read conflicts are
%%ACC%%  impossible, as only a single activity, the governing activity, can read the
%%ACC%%  \Xcd{acc} variable. Read-write conflicts are impossible, as reads are only
%%ACC%%  allowed at points where the only activity which can refer to the \Xcd{acc}
%%ACC%%  variable is the governing activity. Two activities may try to write the
%%ACC%%  \Xcd{acc} variable at the same time. The writes are performed atomically, so
%%ACC%%  they behave as if they happened in some (arbitrary) order---and, because the
%%ACC%%  reducer is Abelian, the order of writes doesn't matter.
%%ACC%%  
%%ACC%%  If the reducer is not Abelian---\eg, it is accumulating a string result by
%%ACC%%  concatenating a lot of partial strings together---the result is indeterminate.
%%ACC%%  However, because the accumulator operations are atomic, it will be the result
%%ACC%%  of {\em some} combination of the individual elements by the reduction
%%ACC%%  operation, \eg, the concatenation of the partial strings in {\em some} order.  
%%ACC%%  
%%ACC%%  
%%ACC%%  
%%ACC%%  For example, the following code computes triangle numbers {$\sum_{i=1}^{n}i$}
%%ACC%%  concurrently.\footnote{This program is highly inefficient. Even ignoring the
%%ACC%%    constant-time formula {$\sum_{i=1}^{n}i = \frac{n(n+1)}{2}$}, this program
%%ACC%%    incurs the cost of starting {$n$} activities and coordinating {$n$} accesses
%%ACC%%    to the accumulator. Accumulator variables are of most value in multi-place,
%%ACC%%    multi-core computations.}
%%ACC%%  
%%ACC%%  
%%ACC%%  %~accTODO~gen
%%ACC%%  %package Vars.Accumulator.Concurrency.Example;
%%ACC%%  %class Example{
%%ACC%%  %
%%ACC%%  %~accTODO~vis
%%ACC%%  \begin{xten}
%%ACC%%  def triangle(n:Int) {
%%ACC%%    val summer = Reducer[Int](0, Int.+);
%%ACC%%    acc(summer) sum; 
%%ACC%%    finish {
%%ACC%%      for([i] in 1..n) async {
%%ACC%%        sum = i;  // (A)
%%ACC%%      }
%%ACC%%      // (C)
%%ACC%%    }
%%ACC%%    return sum; // (B)
%%ACC%%  }
%%ACC%%  \end{xten}
%%ACC%%  %~accTODO~siv
%%ACC%%  %}
%%ACC%%  %~accTODO~neg
%%ACC%%  
%%ACC%%  The governing activity of the \Xcd{acc} variable \Xcd{sum} is the activity
%%ACC%%  including the body of \Xcd{triangle}.  It starts up \Xcd{n} sub-activities,
%%ACC%%  each of which adds one value to \Xcd{sum} at point \Xcd{(A)}.  Note that these
%%ACC%%  activities cannot {\em read} the value of \Xcd{sum}---only the governing
%%ACC%%  activity can do that---but they can update it.  
%%ACC%%  
%%ACC%%  At point \Xcd{(B)}, \Xcd{triangle} returns the value in \Xcd{sum}. It is
%%ACC%%  clear, from the \Xcd{finish} statement, that all sub-activities started by the
%%ACC%%  governing process have finished at this point. X10 forbids reading of
%%ACC%%  \Xcd{sum}, even by the governing process, at point \Xcd{(C)}, since
%%ACC%%  sub-activities writing into it could still be active when the governing
%%ACC%%  activity reaches this point.  The \Xcd{return sum;} statement could not be
%%ACC%%  moved to \Xcd{(C)}, which is good, because the program would be wrong if it
%%ACC%%  were there.
%%ACC%%  
%%ACC%%  
%%ACC%%  
%%ACC%%  
%%ACC%%  \subsubsection{Accumulators and Places}
%%ACC%%  \index{accumulator!and places} Activity variables can be read and written from
%%ACC%%  any place, without need for \Xcd{GlobalRef}s. We may spread the previous
%%ACC%%  computation out among all the available processors by simply sticking in an
%%ACC%%  \Xcd{at(...)} statement at point \Xcd{(D)}, without need for other
%%ACC%%  restructuring of the program.
%%ACC%%  
%%ACC%%  %~accTODO~gen
%%ACC%%  %package Vars.Accumulator.Concurrency.Example.Multiplacey;
%%ACC%%  %class Example{
%%ACC%%  %~accTODO~vis
%%ACC%%  \begin{xten}
%%ACC%%  def triangle(n:Int) {
%%ACC%%    val summer = Reducer[Int](0, Int.+);
%%ACC%%    acc(summer) sum; 
%%ACC%%    finish {
%%ACC%%      for([i] in 1..n) async 
%%ACC%%        at(Places.place(i % Places.MAX_PLACES) { //(D)
%%ACC%%          sum = i;  // (A)
%%ACC%%      }
%%ACC%%    }
%%ACC%%    return sum; // (B)
%%ACC%%  }
%%ACC%%  \end{xten}
%%ACC%%  %~accTODO~siv
%%ACC%%  %}
%%ACC%%  %~accTODO~neg
%%ACC%%  
%%ACC%%  \subsubsection{Accumulator Parameters}
%%ACC%%  \index{accumulator variables!as parameters}
%%ACC%%  \index{parameters!accumulator}
%%ACC%%  
%%ACC%%  Accumulators can be passed to methods and closures, by giving the keyword 
%%ACC%%  \Xcd{acc} instead of \Xcd{var} or \Xcd{val}.  Reducers are not specified; each
%%ACC%%  accumulator comes with its own reducer.  However, the type \Xcd{T} of the
%%ACC%%  accumulator {\em is} required.
%%ACC%%  
%%ACC%%  For example, the following method takes a list of numbers, and accumulates
%%ACC%%  those that are divisible by 2 in \Xcd{evens}, and those that are divisible by
%%ACC%%  3 in \Xcd{triples}: 
%%ACC%%  %~accTODO~gen
%%ACC%%  %package Vars.accumulators.parameters.oscillators.convulsitors.proximators;
%%ACC%%  %import x10.util.*;
%%ACC%%  %class Whatever {
%%ACC%%  %~accTODO~vis
%%ACC%%  \begin{xten}
%%ACC%%  static def split23(L:List[Int], acc evens:Int, acc triples:Int) {
%%ACC%%    for(n in L) {
%%ACC%%       if (n % 2 == 0) evens = n;
%%ACC%%       if (n % 3 == 0) triples = n;
%%ACC%%    }
%%ACC%%  }
%%ACC%%  static val summer = Reducer[Int](0, Int.+);
%%ACC%%  static val producter = Reducer[Int](1, Int.*);
%%ACC%%  static def sumEvenPlusProdTriple(L:List[Int]) {
%%ACC%%    acc(summer) sumEven;
%%ACC%%    acc(producter) prodTriple;
%%ACC%%    split23(L, sumEven, prodTriple);
%%ACC%%    return sumEven + prodTriple;
%%ACC%%  }
%%ACC%%  \end{xten}
%%ACC%%  %~accTODO~siv
%%ACC%%  %}
%%ACC%%  %~accTODO~neg
%%ACC%%  
%%ACC%%  \subsection{Indexed Accumulators}
%%ACC%%  \index{accumulator!indexed}
%%ACC%%  \index{accumulator!array}
%%ACC%%  
%%ACC%%  
%%ACC%%  \noo{Define this!}
%%ACC%%  
%%ACC%%  %~accTODO~gen
%%ACC%%  % package Vars.Indexed.Accumulators;
%%ACC%%  %~accTODO~vis
%%ACC%%  \begin{xten}
%%ACC%%  class BoolAccum implements SelfAccumulator[Boolean, Int] {
%%ACC%%    var sumTrue = 0, sumFalse = 0;
%%ACC%%    def update(k:Boolean, v:Int) { 
%%ACC%%       if (k) sumTrue += k; else sumFalse += k;
%%ACC%%    }
%%ACC%%    def update(ks:Array[Boolean]{rail}, vs:Array[Int]{ks.size == vs.size}) {
%%ACC%%       for([i] in ks.region) update(ks(i), vs(i));  }
%%ACC%%    
%%ACC%%  }
%%ACC%%  \end{xten}
%%ACC%%  %~accTODO~siv
%%ACC%%  %
%%ACC%%  %~accTODO~neg

\chapter{Names and packages}
\label{packages} \index{name}\index{package}

\section{Names}

An X10 program consists largely of giving names to entities, and then
manipulating the entities by their names. The entities involved may be
compile-time constructs, like packages, types and classes, or run-time
constructs, like numbers and strings and objects.  

X10 names can be {\em simple names}, which look like identifiers: \xcd`vj`,
\xcd`x10`, \xcd`AndSoOn`. Or, they can be {\em qualified names}, which are
sequences of two or more identifiers separated by dots: \xcd`x10.lang.String`, 
\xcd`somePack.someType`, \xcd`a.b.c.d.e.f`.   Some entities have only simple
names; some have both simple and qualified names.

Every declaration that introduces a name has a {\em scope}: the region of the
program in which the named entity can be referred to by a simple name.  
In some cases, entities may be referred to by qualified names outside of their
scope.  \Eg, a \xcd`public` class \xcd`C` defined in package \xcd`p` can be
referred to by the simple name \xcd`C` inside of \xcd`p`, or by the qualified
name \xcd`p.C` from anywhere.  

Many sorts of entities have {\em members}.  Packages have classes, structs,
and interfaces as members.  Those, in turn, have fields, methods, types, and
so forth as members.  The member \xcd`x` of an entity named \xcd`E` (as a
simple or qualified name) has the name \xcd`E.x`; it may also have other
names.  

\subsection{Shadowing}
\index{shadowing}
\index{namespace}

One declaration $d$ may {\em shadow} another declaration $d'$ in part of the
scope of $d'$, if $d$ and $d'$ declare variables with the same simple name $n$.
When $d$ shadows $d'$, a use of $n$ might refer to $d$'s $n$ (unless some
$d''$ in turn shadows $d$), but will never refer to $d'$'s $n$.

X10 has four namespaces:
\begin{itemize}
\item {\bf Types:} for classes, interfaces, structs, and defined types.
\item {\bf Values:} for \xcd`val`- and \xcd`var`-bound variables; fields;
      and formal parameters of all sorts.
\item {\bf Methods:} for methods of classes, interfaces, and structs.
\item {\bf Packages:} for packages.
\end{itemize}

A declaration $d$ in one namespace, binding a name $n$ to an entity $e$,
shadows all other declarations of that name $n$ in scope at the point where
$d$ is declared. This shadowing is in effect for the entire scope of $d$.  
Declarations in different namespaces do not shadow each other.
Thus, a local variable declaration may shadow a field declaration, but not a
class declaration.

Declarations which only introduce qualified names --- in X10, this is only
package declarations --- cannot shadow anything.

The rules for shadowing of imported names are given in \Sref{sect:ImportDecl}.

\subsection{Hiding}
\index{hiding}
\label{sect:Hiding}

Shadowing is ubiquituous in X10. Another, and considerably rarer, way that one
definition of a given simpl ename can render another definition of the same
name unavailable is {\em hiding}. If a class \xcd`Super` defines a field named
\xcd`x`, and a subclass \xcd`Sub` of \xcd`Super` also defines a field named
\xcd`x`, then, for \xcd`Sub`s, references to the \xcd`x` field get \xcd`Sub`'s
\xcd`x` rather than \xcd`Super`'s. In this case, \xcd`Super`'s \xcd`x` is said
to be {\em hidden}.

Hiding is technically different from shadowing, because hiding applies in more
circumstances: a use of class \xcd`Sub`, such as \xcd`sub.x`, may involve
hiding of name \xcd`x`, though it could not involve shadowing of \xcd`x`
because \xcd`x` is need not be declared as a name at that point.

\subsection{Obscuring}
\index{obscuring}
\label{sect:Obscuring}

The third way in which a definition of a simple name may become unavailable is
{\em obscuring}. This well-named concept says that, if \xcd`n` can be
interpreted as two or more of: a variable, a type, and a package, then it will
be interpreted as a variable if that is possible, or a type if it cannot be
interpreted as a variable. In this case, the unavailable interpretations are
{\em obscured}. 

\begin{ex}
In the \xcd`example` method of the following code, both a struct and a local
variable are named \xcd`eg`.  Following the obscuring rules, The call
\xcd`eg.ow()` in the first \xcd`assert` uses the variable rather than the struct.  
As the second \xcd`assert` demonstrates, the struct can be accessed through
its fully-qualified name.   Note that none of this would have happened if the
coder had followed the convention that structs have capitalized names,
\xcd`Eg`, and variables have lower-case ones, \xcd`eg`. 

%~~gen ^^^ Packages5t5g
% NOTEST
%~~vis
\begin{xten}
package obscuring;
struct eg {
   static def ow()= 1;
   static struct Bite {
      def ow() = 2;
   }
   def example() {
       val eg = Bite();
       assert eg.ow() == 2;
       assert obscuring.eg.ow() == 1;
     }
}

\end{xten}
%~~siv
% class Hook{ def run() { (eg()).example(); return true; } }
%~~neg

\end{ex}

Due to obscuring, it may be impossible to refer to a type or a package via a
simple name in some circumstances.  Obscuring does not block qualified names.



\subsection{Ambiguity and Disambiguation}

Neither simple nor qualified names are necessarily unique.  There can be, in
general, many entities that have the same name.  This is perfectly ordinary,
and, when done well, considered good programming practice.   Various forms of
{\em disambiguation} are used to tell which entity is meant by a given name;
\eg, methods with the same name may be disambiguated by the types of their
arguments (\Sref{sect:MethodResolution}).

\begin{ex}
In the following example, there are three static methods with 
qualified name \xcd`DisambEx.Example.m`; they can be disambiguated by their
different arguments.   Inside the body of the third, the simple name \xcd`i`
refers to both the \xcd`Int` formal of \xcd`m`, and to the static method 
\xcd`DisambEx.Example.i`.  
%~~gen ^^^ Packages9e6r
%~~vis
\begin{xten}
package DisambEx; 
class Example {
  static def m() = 1;
  static def m(Boolean) = 2;
  static def i() = 3;
  static def m(i:Int) {
    if (i > 10) {
      return i() + 1;
    }
    return i;
  }
  static def example() {
    assert m() == 1;
    assert m(true) == 2;
    assert m(20) == 4;
  }
}
\end{xten}
%~~siv
% class Hook{ def run() { Example.example(); return true; } }
%~~neg
\end{ex}



\section{Access Control}
\index{public}\index{protected}\index{private}

X10 allows programmers {\em access control}, that is, the ability to determine
statically where identifiers of most sorts are visible.  In particular, X10
allows {\em information hiding}, wherein certain data can be accessed from
only limited parts of the program. 

There are four access control modes: 
\xcd"public" , \xcd"protected", \xcd"private"
and uninflected package-specific scopes, much like those of Java. 
Most things can be public or private; a few things (\eg, class members) can
also be protected or package-scoped.  

Accessibility of one X10 entity (package, container, member, etc.) from within
a package or container is defined as follows: 
\begin{itemize}
\item Packages are always accessible.
\item If a container \xcd`C` is public, and, if it is inside of another
      container \xcd`D`,
      container \xcd`D` is accessible, then \xcd`C` is accessible.  
\item A member \xcd`m` of a container \xcd`C` is accessible from within
      another  \xcd`E`
      if \xcd`C` is
      accessible, and: 
      \begin{itemize}
      \item \xcd`m` is declared \xcd`public`; or
      \item \xcd`C` is an interface; or
      \item \xcd`m` is declared \xcd`protected`, and either the access is from
            within the same package that \xcd`C` is defined in, or from within
            the body of a subclass of \xcd`C` (but see
            \Sref{sect:protected-details} for some fine points); or
      \item \xcd`m` is declared \xcd`private`, and the access is from within
            the top-level class which contains the definition of \xcd`C` ---
            which may be \xcd`C` itself, or, if \xcd`C` is a nested container, an
            outer class around \xcd`C`; or
      \item \xcd`m` has no explicit class declaration (hence using the
            implicit ``package''-level access control), and the access occurs
            from the same package that \xcd`C` is declared in.
      \end{itemize}
\end{itemize}

\subsection{Details of \xcd`protected`}
\label{sect:protected-details}

\xcd`protected` access has a few fine points. 
Within the body of a subclass \xcd`D` of the class \xcd`C` containing
the definition of a protected member \xcd`m`, 

\begin{itemize}

\item An access \xcd`e.fld` to a field, or \xcd`e.m(...)` to a method, is
      permitted precisely when the type of \xcd`e` is either \xcd`D` or a
      subtype of \xcd`D`.  
For example, the access to \xcd`that.f` in the following code is acceptable, but
the access to \xcd`xhax.f` is not.  
%~~gen ^^^ Packages9q4y
% package Packages9q4y;
%~~vis
\begin{xten}
class C {
  protected var f : Int = 0;
}
class X extends C {}
class D extends C {
  def usef(that:D, xhax:X) {
     this.f += that.f; 
     // ERROR: this.f += xhax.f;
}
\end{xten}
%~~siv
%
%~~neg

\limitation{Some X10 compilers improperly allow access to {\tt xhax} -- as,
indeed, some Java compilers do, despite Java having the analogous rule.
However, X10 allows all permitted accesses, so the workaround is trivial.}

\item An access through a qualified name \xcd`Q.N` is permitted precisely when
      the type of \xcd`Q` is \xcd`D` or a subtype of \xcd`D`. 

\end{itemize}

Qualified access to a protected constructor is subtle.  Let \xcd`C` be a class
with a \xcd`protected` constructor $c$, and let \xcd`S` be the innermost
class containing a use $u$ of $c$.  There are three cases: 

\begin{itemize}
\item Super superclass construction invocations, \xcd`super(...)` or
      \xcd`E.super(...)`, are permitted.
\item Anonymous class instance creations, \xcd`new C(...){...}`
      and \xcd`E.new C(...){...}`, are
      permitted.
\item No other accesses are permitted. 
\end{itemize}

\section{Packages}

A package is a named collection of top-level type declarations, \viz, class,
interface, and struct declarations. Package names are sequences of
identifiers, like \xcd`x10.lang` and \xcd`com.ibm.museum`. The multiple names
are simply a convenience, though there is a tenuous relationship between
packages \xcd`a`, \xcd`a.b`, and \xcd`a.c`.   Packages can be accessed by
name from anywhere: a package may contain private elements, but may not itself
be private. 

Packages and protection modifiers determine which top-level names can be used
where. Only the \xcd`public` members of package \xcd`pack.age` can be accessed
outside of \xcd`pack.age` itself.  
%~~gen~~Stimulus ^^^ Stimulus
% NOTEST 
%~~vis
\begin{xten}
package pack.age;
class Deal {
  public def make() {}
}
public class Stimulus {
  private def taxCut() = true;
  protected def benefits() = true;
  public def jobCreation() = true;
  /*package*/ def jumpstart() = true;
}
\end{xten}
%~~siv
% 
%
%~~neg

The class \xcd`Stimulus` can be referred to from anywhere outside of
\xcd`pack.age` by its full name of \xcd`pack.age.Stimulus`, or can be imported
and referred to simply as \xcd`Stimulus`.  The public \xcd`jobCreation()`
method of a \xcd`Stimulus` can be referred to from anywhere as well; the other
methods have smaller visibility.  The non-\xcd`public` class \xcd`Deal` cannot
be used from outside of \xcd`pack.age`.  



\subsection{Name Collisions}

It is a static error for a package to have two members with the same name. For
example, package \xcd`pack.age` cannot define two classes both named
\xcd`Crash`, nor a class and an interface with that name.

Furthermore, \xcd`pack.age` cannot define a member \xcd`Crash` if there is
another package named \xcd`pack.age.Crash`, nor vice-versa. (This prohibition
is the only actual relationship between the two packages.)  This prevents the
ambiguity of whether \xcd`pack.age.Crash` refers to the class or the package.  
Note that the naming convention that package names are lower-case and package
members are capitalized prevents such collisions.


\section{{\tt import} Declarations}
\label{sect:ImportDecl}
\index{import}

Any public member of a package can be referred to from anywhere through a
fully-qualified name: \xcd`pack.age.Stimulus`.    

Often, this is too awkward.  X10 has two ways to allow code outside of a class
to refer to the class by its short name (\xcd`Stimulus`): single-type imports
and on-demand imports.   

Imports of either kind appear at the start of the file, immediately after the
\xcd`package` directive if there is one; their scope is the whole file.

\subsection{Single-Type Import}

The declaration \xcd`import ` {\em TypeName} \xcd`;` imports a single type
into the current namespace.  The type it imports must be a fully-qualified
name of an extant type, and it must either be in the same package (in which
case the \xcd`import` is redundant) or be declared \xcd`public`.  

Furthermore, when importing \xcd`pack.age.T`, there must not be another type
named \xcd`T` at that point: neither a  \xcd`T` declared in \xcd`pack.age`,
nor a \xcd`inst.ant.T` imported from some other package.

The declaration \xcd`import E.n;`, appearing in file $f$ of a package named
\xcd`P`, shadows the following types named \xcd`n` when they appear in $f$: 
\begin{itemize}
\item Top-level types named \xcd`n` appearing in other files of \xcd`P`, and 
\item Types named \xcd`n` imported by automatic imports
      (\Sref{sect:AutomaticImport}) in $f$.
\end{itemize}
\noindent


\subsection{Automatic Import}
\label{sect:AutomaticImport}

The automatic import \xcd`import pack.age.*;`, loosely, imports all the public
members of \xcd`pack.age`.  In fact, it does so somewhat carefully, avoiding
certain errors that could occur if it were done naively.  Types defined in the
current package, and those imported by single-type imports, shadow those
imported by automatic imports.   Names automatically imported never shadow any
other names.



\subsection{Implicit Imports}

The packages \xcd`x10.lang` and \xcd`x10.array` are automatically imported in all files
without need for further specification.

%%BARD-HERE



\section{Conventions on Type Names}

%##(TypeName PackageName
\begin{bbgrammar}
%(FROM #(prod:TypeName)#)
            TypeName \: Id & (\ref{prod:TypeName}) \\
                    \| TypeName \xcd"." Id \\
%(FROM #(prod:PackageName)#)
         PackageName \: Id & (\ref{prod:PackageName}) \\
                    \| PackageName \xcd"." Id \\
\end{bbgrammar}
%##)


While not enforced by the compiler, classes and interfaces
in the \Xten{} library follow the following naming conventions.
Names of types---including classes,
type parameters, and types specified by type definitions---are in
CamelCase and begin with an uppercase letter.  (Type variables are often
single capital letters, such as \xcd`T`.)
For backward
compatibility with languages such as C and \java{}, type
definitions are provided to allow primitive types
such as \xcd"int" and \xcd"boolean" to be written in lowercase.
Names of methods, fields, value properties, and packages are in camelCase and
begin with a lowercase letter. 
Names of \xcd"static val" fields are in all uppercase with words
separated by \xcd"_"'s.

\chapter{Interfaces}
\label{XtenInterfaces}\index{interface}

An interface specifies signatures for zero or more public methods, property
methods,
\xcd`static val`s, 
classes, structs, interfaces, types
and an invariant. 

The following puny example illustrates all these features: 
% TODO Well, it would if there weren't a compiler bug in the way.
%~~gen ^^^Interfaces_static_val
% package Interfaces_static_val;
% 
%~~vis
\begin{xten}
interface Pushable{prio() != 0} {
  def push(): void;
  static val MAX_PRIO = 100;
  abstract class Pushedness{}
  struct Pushy{}
  interface Pushing{}
  static type Shove = Int;
  property text():String;
  property prio():Int;
}
class MessageButton(text:String)
  implements Pushable{self.prio()==Pushable.MAX_PRIO} {
  public def push() { 
    x10.io.Console.OUT.println(text + " pushed");
  }
  public property text() = text;
  public property prio() = Pushable.MAX_PRIO;
}
\end{xten}
%~~siv
%
%~~neg
\noindent
\xcd`Pushable` defines two property methods, one normal method, and a static
value.  It also 
establishes an invariant, that \xcd`prio() != 0`. 
\xcd`MessageButton` implements a constrained version of \xcd`Pushable`,
\viz\ one with maximum priority.  It
defines the \xcd`push()` method given in the interface, as a \xcd`public`
method---interface methods are implicitly \xcd`public`.

\limitation{X10 may not always detect that type invariants of interfaces are
satisfied, even when they obviously are.}

A container---a class or struct---can {\em implement} an interface,
typically by having all the methods and property methods that the interface
requires, and by providing a suitable \xcd`implements` clause in its definition.

A variable may be declared to be of interface type.  Such a variable has all
the property and normal methods declared (directly or indirectly) by the
interface; 
nothing else is statically available.  Values of any concrete type which
implement the interface may be stored in the variable.  

\begin{ex}
The following code puts two quite different objects into the variable
\xcd`star`, both of which satisfy the interface \xcd`Star`.
%~~gen ^^^ Interfaces6l3f
% package Interfaces6l3f;
%~~vis
\begin{xten}
interface Star { def rise():void; }
class AlphaCentauri implements Star {
   public def rise() {}
}
class ElvisPresley implements Star {
   public def rise() {}
}
class Example {
   static def example() {
      var star : Star;
      star = new AlphaCentauri();
      star.rise();
      star = new ElvisPresley();
      star.rise();
   }
}
\end{xten}
%~~siv
%
%~~neg
\end{ex}
An interface may extend several interfaces, giving
X10 a large fraction of the power of multiple inheritance at a tiny fraction
of the cost.

\begin{ex}
%~~gen ^^^ Interfaces6g4u
% package Interfaces6g4u;
%~~vis
\begin{xten}
interface Star{}
interface Dog{}
class Sirius implements Dog, Star{}
class Lassie implements Dog, Star{}
\end{xten}
%~~siv
%
%~~neg
\end{ex}


\section{Interface Syntax}

\label{DepType:Interface}

%##(NormalInterfaceDecl TypeParamsI TypeParamI Guard ExtendsInterfaces InterfaceBody InterfaceMemberDecl
\begin{bbgrammar}
%(FROM #(prod:NormalInterfaceDecl)#)
 NormalInterfaceDecl \: Mods\opt \xcd"interface" Id TypeParamsI\opt Properties\opt Guard\opt ExtendsInterfaces\opt InterfaceBody & (\ref{prod:NormalInterfaceDecl}) \\
%(FROM #(prod:TypeParamsI)#)
         TypeParamsI \: \xcd"[" TypeParamIList \xcd"]" & (\ref{prod:TypeParamsI}) \\
%(FROM #(prod:TypeParamI)#)
          TypeParamI \: Id & (\ref{prod:TypeParamI}) \\
                     \| \xcd"+" Id \\
                     \| \xcd"-" Id \\
%(FROM #(prod:Guard)#)
               Guard \: DepParams & (\ref{prod:Guard}) \\
%(FROM #(prod:ExtendsInterfaces)#)
   ExtendsInterfaces \: \xcd"extends" Type & (\ref{prod:ExtendsInterfaces}) \\
                     \| ExtendsInterfaces \xcd"," Type \\
%(FROM #(prod:InterfaceBody)#)
       InterfaceBody \: \xcd"{" InterfaceMemberDecls\opt \xcd"}" & (\ref{prod:InterfaceBody}) \\
%(FROM #(prod:InterfaceMemberDecl)#)
 InterfaceMemberDecl \: MethodDecl & (\ref{prod:InterfaceMemberDecl}) \\
                     \| PropertyMethodDecl \\
                     \| FieldDecl \\
                     \| ClassDecl \\
                     \| InterfaceDecl \\
                     \| TypeDefDecl \\
                     \| \xcd";" \\
\end{bbgrammar}
%##)


\noindent
The invariant associated with an interface is the conjunction of the
invariants associated with its superinterfaces and the invariant
defined at the interface. 



A class \xcd"C"  implements an interface \xcd"I" if
\begin{itemize}
\item \xcd`I`, or a subtype of \xcd`I`, appears in the \xcd`implements` list
      of \xcd`C`, 
\item \xcd`C`'s property methods include all the property methods  of \xcd"I",
\item Each method \xcd`m` defined by \xcd`I` is also a method of \xcd`C` --
      with the {\em  \xcd`public`} modifier added.   These methods may be
      \xcd`abstract` if \xcd`C` is \xcd`abstract`.
\end{itemize}


If \xcd`C` implements \xcd`I`, then the class invariant
(\Sref{DepType:ClassGuardDef}) for \xcd`C`,   $\mathit{inv}($\xcd"C"$)$, implies
the class invariant for \xcd`I`, $\mathit{inv}($\xcd"I"$)$.  That is, if the
interface \xcd`I` specifies some requirement, then every class \xcd`C` that
implements it satisfies that requirement.

\section{Access to Members}

All interface members are \xcd`public`, whether or not they are declared
public.  There is little purpose to non-public methods of an interface; they
would specify that implementing classes and structs have methods that cannot
be seen.

\section{Property Methods}

An interface may declare \xcd`property` methods.  All non-\xcd`abstract`
containers implementing such an interface must provide all the \xcd`property`
methods specified.  

\section{Field Definitions}
\index{interface!field definition in}

An interface may declare a \xcd`val` field, with a value.  This field is implicitly
\xcd`public static val`.  In particular, it is {\em not} an instance field. 
%~~gen ^^^ Interfaces10
% package Interface.Field;
%~~vis
\begin{xten}
interface KnowsPi {
  PI = 3.14159265358;
}
\end{xten}
%~~siv
%
%~~neg

Classes and structs implementing such an interface get the interface's fields as
\xcd`public static` fields.  Unlike  methods, there is no need
for the implementing class to declare them. 
%~~gen ^^^ Interfaces20
% package Interface.Field.Two;
% interface KnowsPi {PI = 3.14159265358;}
%~~vis
\begin{xten}
class Circle implements KnowsPi {
  static def area(r:Double) = PI * r * r;
}
class UsesPi {
  def circumf(r:Double) = 2 * r * KnowsPi.PI;
}
\end{xten}
%~~siv
%
%~~neg

\subsection{Fine Points of Fields}

If two parent interfaces give different static fields of the same name, 
those fields must be referred to by qualified names.
%~~gen ^^^ Interface_field_name_collision
% 
%~~vis
\begin{xten}
interface E1 {static val a = 1;}
interface E2 {static val a = 2;}
interface E3 extends E1, E2{}
class Example implements E3 {
  def example() = E1.a + E2.a;
}
\end{xten}
%~~siv
%
%~~neg

If the {\em same} field \xcd`a` is inherited through many paths, there is no need to
disambiguate it:
%~~gen ^^^ Interfaces_multi
% package Interfaces.Mult.Inher.Field;
%~~vis
\begin{xten}
interface I1 { static val a = 1;} 
interface I2 extends I1 {}
interface I3 extends I1 {}
interface I4 extends I2,I3 {}
class Example implements I4 {
  def example() = a;
}
\end{xten}
%~~siv
%
%~~neg

The initializer of a field in an interface may be any expression.  It is
evaluated under the same rules as a \xcd`static` field of a class. 

\begin{ex}
In this example, a local class (\Sref{sect:LocalClasses}) \xcd`B` is defined,
with an inner interface \xcd`I`.  The field \xcd`V` of \xcd`I` uses a variable
\xcd`n` which is global to \xcd`B`.   In this case it is a truly baroque way
to bind a \xcd`val`, but other uses are nontrivial.

%~~gen ^^^ Interfaces3l4a
% package Interfaces3l4a;
%~~vis
\begin{xten}
class TheOne {
  static val ONE = 1;
  interface WelshOrFrench {
    val UN = 1;
  }
  static class Onesome implements WelshOrFrench {
    static def example() {
      assert UN == ONE;
    }
  }
}
\end{xten}
%~~siv
% class Hook{ def run() {TheOne.Onesome.example(); return true;}}
%~~neg
\end{ex}

\section{Generic Interfaces}

Interfaces, like classes and structs, can have type parameters.  
The discussion of generics in \Sref{TypeParameters} applies to interfaces,
without modification.

\begin{ex}
%~~gen ^^^ Interfaces7n1z
% package Interfaces7n1z;
%~~vis
\begin{xten}
interface ListOfFuns[T,U] extends x10.util.List[(T)=>U] {}
\end{xten}
%~~siv
%
%~~neg

\end{ex}

\section{Interface Inheritance}

The {\em direct superinterfaces} of a non-generic interface \xcd`I` are the interfaces
(if any) mentioned in the \xcd`extends` clause of \xcd`I`'s definition.
If \xcd`I`  is generic, the direct superinterfaces are of an instantiation of
\xcd`I` are the corresponding instantiations of those interfaces.
A {\em superinterface} of \xcd`I` is either \xcd`I` itself, or a direct
superinterface of a superinterface of \xcd`I`, and similarly for generic
interfaces.    

\xcd`I` inherits the members of all of its superinterfaces. Any class or
struct that has \xcd`I` in its \xcd`implements` clause also implements all of
\xcd`I`'s superinterfaces. 






\section{Members of an Interface}

The members of an interface \xcd`I` are the union of the following sets: 
\begin{enumerate}
\item All of the members appearing in \xcd`I`'s declaration;
\item All the members of its direct super-interfaces, except those which are
      hidden (\Sref{sect:Hiding}) by \xcd`I`
\item The members of \xcd`Any`.
\end{enumerate}

Overriding for instances is defined as for classes, \Sref{MethodOverload}

\chapter{Classes}
\label{XtenClasses}\index{class}
\label{ReferenceClasses}

\section{Principles of X10 Objects}\label{XtenObjects}\index{object}
\index{class}

\subsection{Basic Design}

Objects are instances of classes: the most common and most powerful sort of
value in X10.  The other kinds of values, structs and functions, are more
specialized, better in some circumstances but not in all.
\xcd"x10.lang.Object" is the most general class; all other classes inherit
from it, directly or indirectly. 


Classes are structured in a single-inheritance code
hierarchy.   They may have any or all of these features: 
\begin{itemize}
\item Implementing any number of interfaces;
\item Static and instance \xcd`val` fields; 
\item Instance \xcd`var` fields; 
\item Static and instance methods;
\item Constructors;
\item Properties;
\item Static and instance nested containers.
\end{itemize}


\Xten{} objects do not have locks associated with them.
Programmers should use atomic blocks (\Sref{AtomicBlocks}) for mutual
exclusion and clocks (\Sref{XtenClocks}) for sequencing multiple parallel
operations.

An object exists in a single location: the place that it was created.  One
place cannot use or even directly refer to an object in a different place.   A
special type, \Xcd{GlobalRef[T]}, allows explicit cross-place references. 

The basic operations on objects are:
\begin{itemize}

{}\item Field access (\Sref{FieldAccess}). 
The static and instance fields of an object can be retrieved; \xcd`var` fields
can be set.  

{}\item Method invocation (\Sref{MethodInvocation}).  
Static and instance methods of an object can be invoked.

{}\item Casting (\Sref{ClassCast}) and instance testing with \xcd`instanceof`
(\Sref{instanceOf}) Objects can be cast or type-tested.  

\item The equality operators \xcd"==" and \xcd"!="
Objects can be compared for equality with the \Xcd{==} operation.  This checks
object {\em identity}: two objects are \Xcd{==} iff they are the same object.

\end{itemize}

  
 
\subsection{Class Declaration Syntax}

The {\em class declaration} has a list of type parameters, a list of
properties, a constraint (the {\em class invariant}), a single superclass,
zero or more interfaces that it implements, and a class body containing the
the definition of fields, properties, methods, and member types. Each such
declaration introduces a class type (\Sref{ReferenceTypes}).

%##(NormalClassDecl TypeParamsWithVariance TypeParamWithVarianceList Properties PropertyList Property WhereClause Super Interfaces InterfaceTypeList ClassBody ClassBodyDecls ClassMemberDecl
\begin{bbgrammar}
%(FROM #(prod:NormalClassDecl)#)
     NormalClassDecl \: Mods\opt \xcd"class" Id TypeParamsWithVariance\opt Properties\opt WhereClause\opt Super\opt Interfaces\opt ClassBody & (\ref{prod:NormalClassDecl}) \\
%(FROM #(prod:TypeParamsWithVariance)#)
TypeParamsWithVariance \: \xcd"[" TypeParamWithVarianceList \xcd"]" & (\ref{prod:TypeParamsWithVariance}) \\
%(FROM #(prod:TypeParamWithVarianceList)#)
TypeParamWithVarianceList \: TypeParamWithVariance & (\ref{prod:TypeParamWithVarianceList}) \\
                    \| TypeParamWithVarianceList \xcd"," TypeParamWithVariance \\
%(FROM #(prod:Properties)#)
          Properties \: \xcd"(" PropertyList \xcd")" & (\ref{prod:Properties}) \\
%(FROM #(prod:PropertyList)#)
        PropertyList \: Property & (\ref{prod:PropertyList}) \\
                    \| PropertyList \xcd"," Property \\
%(FROM #(prod:Property)#)
            Property \: Annotations\opt Id ResultType & (\ref{prod:Property}) \\
%(FROM #(prod:WhereClause)#)
         WhereClause \: DepParams & (\ref{prod:WhereClause}) \\
%(FROM #(prod:Super)#)
               Super \: \xcd"extends" ClassType & (\ref{prod:Super}) \\
%(FROM #(prod:Interfaces)#)
          Interfaces \: \xcd"implements" InterfaceTypeList & (\ref{prod:Interfaces}) \\
%(FROM #(prod:InterfaceTypeList)#)
   InterfaceTypeList \: Type & (\ref{prod:InterfaceTypeList}) \\
                    \| InterfaceTypeList \xcd"," Type \\
%(FROM #(prod:ClassBody)#)
           ClassBody \: \xcd"{" ClassBodyDecls\opt \xcd"}" & (\ref{prod:ClassBody}) \\
%(FROM #(prod:ClassBodyDecls)#)
      ClassBodyDecls \: ClassBodyDecl & (\ref{prod:ClassBodyDecls}) \\
                    \| ClassBodyDecls ClassBodyDecl \\
%(FROM #(prod:ClassMemberDecl)#)
     ClassMemberDecl \: FieldDecl & (\ref{prod:ClassMemberDecl}) \\
                    \| MethodDecl \\
                    \| PropertyMethodDecl \\
                    \| TypeDefDecl \\
                    \| ClassDecl \\
                    \| InterfaceDecl \\
                    \| \xcd";" \\
\end{bbgrammar}
%##)




\section{Fields}
\label{FieldDefinitions}
\index{object!field}
\index{field}

Objects may have {\em instance fields}, or simply {\em fields} (called
``instance variables'' in C++ and Smalltalk, and ``slots'' in CLOS): places to
store data that is pertinent to the object.  Fields, like variables, may be
mutable (\xcd`var`) or immutable (\xcd`val`).  

Class may have {\em static fields}, which store data pertinent to the
entire class of objects.  
See \Sref{StaticInitialization} for more information.

No two fields of the same class may have the same name.  A field may have the
same name as a method, although for fields of functional type there is a
subtlety (\Sref{sect:disambiguations}).  

\subsection{Field Initialization}
\index{field!initialization}
\index{initialization!of field}

Fields may be given values via {\em field initialization expressions}:
\xcd`val f1 = E;` and \xcd`var f2 : Int = F;`. Other fields of \xcd`this` may
be referenced, but only those that {\em precede} the field being initialized.
For example, the following is correct, but would not be if the fields were
reversed:

%~~gen ^^^ Classes10
%package Classes_field_init_expr_a;
%~~vis
\begin{xten}
class Fld{
  val a = 1;
  val b = 2+a;
}
\end{xten}
%~~siv
%
%~~neg


\subsection{Field hiding}
\index{hiding}
\index{field|hiding}


A subclass that defines a field \xcd"f" hides any field \xcd"f"
declared in a superclass, regardless of their types.  The
superclass field \xcd"f" may be accessed within the body of
the subclass via the reference \xcd"super.f".

With inner classes, it is occasionally necessary to 
write \xcd`Cls.super.f` to get at a hidden field \xcd`f` of an outer class
\xcd`Cls`. 

\begin{ex}
The \xcd`f` field in \xcd`Sub` hides the \xcd`f` field in \xcd`Super`
The \xcd`superf`` method provides access to the \xcd`f` field in \xcd`Super`.
%~~gen ^^^ Classes20
% package classes.fields.primus;
%~~vis
\begin{xten}
class Super{ 
  public val f = 1; 
}
class Sub extends Super {
  val f = true;
  def superf() : Int = super.f; // 1
}
\end{xten}
%~~siv
% class Hook { def run() { return (new Sub()).superf() == 1; }} 
%~~neg
\end{ex}

%~~gen ^^^ Classes30
% package classes.fields.secundus; 
% NOTEST
%~~vis
\begin{xten}
class A {
   val f = 3;
}
class B extends A {
   val f = 4;
   class C extends B {
      // C is both a subclass and inner class of B
      val f = 5;
       def example() {
         assert f         == 5 : "field of C";
         assert super.f   == 4 : "field of superclass";
         assert B.this.f  == 4 : "field of outer instance";
         assert B.super.f == 3 : "super.f of outer instance";
       }
    }
}
\end{xten}
%~~siv
% class Hook { def run() { ((new B()).new C()).example(); return true; } }
%~~neg


\subsection{Field qualifiers}
\label{FieldQualifier}
\index{qualifier!field}
\index{field!qualifier}

The behavior of a field may be changed by a field qualifier, such as
\xcd`static` or \xcd`transient`.  


\subsubsection{\Xcd{static} qualifier}
\index{field!static}

A \xcd`val` field may be declared to be {\em static}, as described in
\Sref{FieldDefinitions}. 

\subsubsection{\Xcd{transient} Qualifier}
\label{TransientFields}
\index{transient}
\index{field!transient}

A field may be declared to be {\em transient}.  Transient fields are excluded
from the deep copying that happens when information is sent from place to
place in an \Xcd{at} statement.    The value of a transient field of a copied
object is the default value of its type, regardless of the value of the field
in the original.  If the type of a field has no
default value, it cannot be marked \Xcd{transient}.
%~~gen ^^^ Classes40
% package Classes.Transient.Example;
% KNOWNFAIL-at
%~~vis
\begin{xten}
class Trans { 
   val copied = "copied";
   transient var transy : String = "a very long string";
   def example() {
      at (here; this) { // causes copying of 'this'
         assert(this.copied.equals("copied"));
         assert(this.transy == null);
      }
   }
}
\end{xten}
%~~siv
%
%~~neg


\section{Properties}
\label{PropertiesInClasses}
\index{property}

The properties of an object (or struct) are  public \xcd`val` fields
usable at compile time in constraints.\footnote{In many cases, a 
\xcd`val` field can be upgraded to a \xcd`property`, which 
entails no compile-time or runtime cost.  Some cannot be, \eg, in cases where
cyclic structures of \xcd`val` fields are required.} 
For example,  every array has a \xcd`rank` telling
how many subscripts it takes.  User-defined classes can have whatever
properties are desired. 

Properties are defined in parentheses, after the name of the class.  They are
given values by the \xcd`property` command in constructors.

%~~gen ^^^ Classes50
% package Classes.Toss.Freedom.Disk2;
%~~vis
\begin{xten}
class Proper(t:Int) {
  def this(t:Int) {property(t);}
}
\end{xten}
%~~siv
%
%~~neg




It is a static error for a class
defining a property \xcd"x: T" to have a subclass class that defines
a property or a field with the name \xcd"x".


A property \xcd`x:T` induces a field with the same name and type, 
as if defined with: 
%~~gen ^^^ Classes60
% package Classes.For.Masses.Of.NevermindTheRest;
% class Exampll[T] {
%~~vis
\begin{xten}
public val x : T;
\end{xten} 
%~~siv
% def this(y:T) { x=y; }
% }
%~~neg
\noindent It also defines a nullary getter method, 
%~~gen ^^^ Classes70
% package Classes_nullary_getter_a;
% class Exampllll[T] {
% public val x : T;
% def this(y:T) { x=y; }
%~~vis
\begin{xten}
public final def x()=x;
\end{xten}
%~~siv
%}
%~~neg





\index{property!initialization}
Properties are initialized by the invocation of a special \Xcd{property}
statement: 
\begin{xten}
property(e1,..., en);
\end{xten}
The number and types of arguments to the \Xcd{property} statement must match
the number and types of the properties in the class declaration.  
Every constructor of a class with properties must invoke \xcd`property(...)`
precisely once; it is a static error if X10 cannot prove that this holds.
The requirement to use the \xcd`property` statement means that all properties
must be given values at the same time.  

By construction, the graph whose nodes are values and whose edges are
properties is acyclic.  \Eg, there cannot be values \xcd`a` and \xcd`b` with
properties \xcd`c` and \xcd`d` such that \xcd`a.c == b` and \xcd`b.d == a`.


\subsection{Properties and Fields}

A container with a property named \xcd`p`, or a nullary property method named
\xcd`p()`, cannot have a field named \xcd`p` --- either defined in that
container, or inherited from a superclass.

\subsection{Acyclicity of Properties}
\index{properties!acyclic}

X10 has certain restrictions that, ultimately, require that properties are
simpler than their containers.  For example, \xcd`class A(a:A){}` is not
allowed.  
Formally, this requirement is that there is  a total order $\preceq$ 
on all classes and
structs such that, if $A$ extends $B$, then $A \prec B$, and
if $A$ has a property of type $B$, then $A \prec B$, where $A \prec B$ means
$A \preceq B$ and $A \ne B$.   
For example, the preceding class \xcd`A` is ruled out because we would need
\xcd`A`$\prec$\xcd`A`, which violates the definition of $\prec$.
The programmer need not (and cannot) specify
$\preceq$, and rarely need worry about its existence.  

Similarly, 
the type of a property may not simply be a type parameter.  
For example, \xcd`class A[X](x:X){}` is illegal.





\section{Methods}
\index{method}
\index{signature}
\index{method!signature}
\index{method!instance}
\index{method!static}

As is common in object-oriented languages, objects can have {\em methods}, of
two sorts.  {\em Static methods} are functions, conceptually associated with a
class and defined in its namespace.  {\em Instance methods} are parameterized
code bodies associated with an instance of the class, which execute with
convenient access to that instance's fields. 

Each method has a {\em signature}, telling what arguments it accepts, what
type it returns, and what precondition it requires. Method definitions may be
overridden by subclasses; the overriding definition may have a declared return
type that is a subtype of the return type of the definition being overridden.
Multiple methods with the same name but different signatures may be provided
\index{overloading}
\index{polymorphism}
on a class (called ``overloading'' or ``ad hoc polymorphism''). Methods may be
declared \Xcd{public}, \Xcd{private}, \Xcd{protected}, or given default package-level access
rights.

%##(MethMods MethodDecl TypeParams FormalParams FormalParamList HasResultType MethodBody
\begin{bbgrammar}
%(FROM #(prod:MethMods)#)
            MethMods \: Mods\opt & (\ref{prod:MethMods}) \\
                    \| MethMods \xcd"property"  \\
                    \| MethMods Mod \\
%(FROM #(prod:MethodDecl)#)
          MethodDecl \: MethMods \xcd"def" Id TypeParams\opt FormalParams WhereClause\opt HasResultType\opt Offers\opt MethodBody & (\ref{prod:MethodDecl}) \\
                    \| MethMods \xcd"operator" TypeParams\opt \xcd"(" FormalParam  \xcd")" BinOp \xcd"(" FormalParam  \xcd")" WhereClause\opt HasResultType\opt Offers\opt MethodBody \\
                    \| MethMods \xcd"operator" TypeParams\opt PrefixOp \xcd"(" FormalParam  \xcd")" WhereClause\opt HasResultType\opt Offers\opt MethodBody \\
                    \| MethMods \xcd"operator" TypeParams\opt \xcd"this" BinOp \xcd"(" FormalParam  \xcd")" WhereClause\opt HasResultType\opt Offers\opt MethodBody \\
                    \| MethMods \xcd"operator" TypeParams\opt \xcd"(" FormalParam  \xcd")" BinOp \xcd"this" WhereClause\opt HasResultType\opt Offers\opt MethodBody \\
                    \| MethMods \xcd"operator" TypeParams\opt PrefixOp \xcd"this" WhereClause\opt HasResultType\opt Offers\opt MethodBody \\
                    \| MethMods \xcd"operator" \xcd"this" TypeParams\opt FormalParams WhereClause\opt HasResultType\opt Offers\opt MethodBody \\
                    \| MethMods \xcd"operator" \xcd"this" TypeParams\opt FormalParams \xcd"=" \xcd"(" FormalParam  \xcd")" WhereClause\opt HasResultType\opt Offers\opt MethodBody \\
                    \| MethMods \xcd"operator" TypeParams\opt \xcd"(" FormalParam  \xcd")" \xcd"as" Type WhereClause\opt Offers\opt MethodBody \\
                    \| MethMods \xcd"operator" TypeParams\opt \xcd"(" FormalParam  \xcd")" \xcd"as" \xcd"?" WhereClause\opt HasResultType\opt Offers\opt MethodBody \\
                    \| MethMods \xcd"operator" TypeParams\opt \xcd"(" FormalParam  \xcd")" WhereClause\opt HasResultType\opt Offers\opt MethodBody \\
%(FROM #(prod:TypeParams)#)
          TypeParams \: \xcd"[" TypeParamList \xcd"]" & (\ref{prod:TypeParams}) \\
%(FROM #(prod:FormalParams)#)
        FormalParams \: \xcd"(" FormalParamList\opt \xcd")" & (\ref{prod:FormalParams}) \\
%(FROM #(prod:FormalParamList)#)
     FormalParamList \: FormalParam & (\ref{prod:FormalParamList}) \\
                    \| FormalParamList \xcd"," FormalParam \\
%(FROM #(prod:HasResultType)#)
       HasResultType \: \xcd":" Type & (\ref{prod:HasResultType}) \\
                    \| \xcd"<:" Type \\
%(FROM #(prod:MethodBody)#)
          MethodBody \: \xcd"=" LastExp \xcd";" & (\ref{prod:MethodBody}) \\
                    \| \xcd"=" Annotations\opt \xcd"{" BlockStatements\opt LastExp \xcd"}" \\
                    \| \xcd"=" Annotations\opt Block \\
                    \| Annotations\opt Block \\
                    \| \xcd";" \\
\end{bbgrammar}
%##)


\index{parameter!var}
\index{parameter!val}
A formal parameter may have a \xcd"val" or \xcd"var"
% , or \Xcd{ref}
modifier; \xcd`val` is the default.
The body of the method is executed in an environment in which 
each formal parameter corresponds to a local variable (\xcd`var` iff the
formal parameter is \xcd`var`)
and is initialized with the value of the actual parameter.

\subsection{Generic Instance Methods}
\index{method!generic instance}

\limitationx{}
In X10, an instance method may be generic: 
%~~gen ^^^ Classes1b7z
% package Classes1b7z;
% NOTEST
%~~vis
\begin{xten}
class Example {
  def example[T](t:T) = "I like " + t;
}
\end{xten}
%~~siv
%
%~~neg

However, the C++ back end does not currently support generic virtual instance
methods like \xcd`example`.  It does allow generic instance methods which are
\xcd`final` or \xcd`private`, and it does allow generic static methods.  


\subsection{Method Guards}
\label{MethodGuard}
\index{method!guard}
\index{guard!on method}

Often, a method will only make sense to invoke under certain
statically-determinable conditions.  For example, \xcd`example(x)` is only
well-defined when \xcd`x != null`, as \xcd`null.toString()` throws a null
pointer exception: 
%~~gen ^^^ Classes80
% package Classes.methodwithconstraintthingie;
% 
%~~vis
\begin{xten}
class Example {
   var f : String = "";
   def example(x:Object){x != null} = {
      this.f = x.toString();
   }
}
\end{xten}
%~~siv
%
%~~neg
\noindent
(We could have used a constrained type \xcd`Object{self!=null}` for \xcd`x`
instead; in
most cases it is a matter of personal preference or convenience of expression
which one to use.) 

The requirement of having a method guard is that callers must demonstrate to
the X10
compiler that the guard is satisfied.  (As usual with static constraint
checking, there is no runtime cost.  Indeed, this code can be more efficient
than usual, as it is statically provable that \xcd`x != null`.)
This may require a cast: 
%~~gen ^^^ Classes90
% package Classes.methodguardnadacastthingie;
% 
% class Example {var f : String = ""; def example(x:Object){x != null} = {this.f = x.toString();}}
% class Eyample {
%~~vis
\begin{xten}
  def exam(e:Example, x:Object) {
    if (x != null) 
       e.example(x as Object{x != null});
    // WRONG: if (x != null) e.example(x);
  }
\end{xten}
%~~siv
%}
%~~neg



The guard \xcd`{c}` 
in a guarded method 
\xcd`def m(){c} = E;`
specifies a constraint \xcd"c" on the
properties of the class \xcd"C" on which the method is being defined. The
method, in effect, only exists  for those instances of \xcd"C" which satisfy
\xcd"c".  It is 
illegal for code to invoke the method on objects whose static type is
not a subtype of \xcd"C{c}".

Specifically: 
    the compiler checks that every method invocation
    \xcdmath"o.m(e$_1$, $\dots$, e$_n$)"
    is type correct. Each argument
    \xcdmath"e$_i$" must have a
    static type \xcdmath"S$_i$" that is a subtype of the declared type
    \xcdmath"T$_i$" for the $i$th
    argument of the method, and the conjunction of the constraints on the
    static types 
    of the arguments must entail the guard in the parameter list
    of the method.

    The compiler checks that in every method invocation
    \xcdmath"o.m(e$_1$, $\dots$, e$_n$)"
    the static type of \xcd"o", \xcd"S", is a subtype of \xcd"C{c}", where the method
    is defined in class \xcd"C" and the guard for \xcd"m" is equivalent to
    \xcd"c".

    Finally, if the declared return type of the method is
    \xcd"D{d}", the
    return type computed for the call is
    \xcdmath"D{a: S; x$_1$: S$_1$; $\dots$; x$_n$: S$_n$; d[a/this]}",
    where \xcd"a" is a new
    variable that does not occur in
    \xcdmath"d, S, S$_1$, $\dots$, S$_n$", and
    \xcdmath"x$_1$, $\dots$, x$_n$" are the formal
    parameters of the method.


\limitation{
Using a reference to an outer class, \xcd`Outer.this`, in a constraint, is not supported.
}


\subsection{Property methods}
\index{method!property}
\index{property method}

Property methods are methods that can be evaluated in constraints.  
For example, the \xcd`eq()` method below tells if the \xcd`x` and \xcd`y`
properties are equal; the \xcd`is(z)` method tells if they are both equal to
\xcd`z`.  These can be used in constraints, as illustrated in the
\xcd`example()` method.
%~~gen ^^^ Classes100
%package Classes.PropertyMethods;
%~~vis
\begin{xten}
class Example(x:Int, y:Int) {
   def this(x:Int, y:Int) { property(x,y); }
   property eq() = (x==y);
   property is(z:Int) = x==z && y==z;
   def example( a : Example{eq()}, b : Example{is(3)} ) {}
}
\end{xten}
%~~siv
%
%~~neg


A method declared with the modifier \xcd"property" may be used
in constraints.  A property method declared in a class must have
a body and must not be \xcd"void".  The body of the method must
consist of only a single \xcd"return" statement or a single
expression.  It is a static error if the expression cannot be
represented in the constraint system.   Property methods may be \xcd`abstract`
in \xcd`abstract` classes, but are implicitly \xcd`final` in
non-\xcd`abstract` classes. 

The expression may contain invocations of other property methods. It is the
responsibility of the programmer to ensure that the evaluation of a property
terminates at compile-time, otherwise the type-checker will not terminate and
the program will fail to compile in a potentially most unfortunate way.

Property methods in classes are implicitly \xcd"final"; they cannot be
overridden.  It is a static error if a superclass has a property method with a
given signature, and a subclass has a method or property method with the same
signature.   It is a static error if a superclass has a property with some
name \xcd`p`, and a subclass has a nullary method of any kind (instance,
static, or property) also named \xcd`p`. 

% We should fix it so that a property p (or a property method p())  in the super class  cannot be shadowed by a field p in a subclass, or overridden by a method p() in the subclass.


A nullary property method definition may omit the formal parameters and
the \xcd"def" keyword.  That is, the following are equivalent:



%~~gen ^^^ Classes110
% package classes.waifsome1;
% class Waif(rect:Boolean, onePlace:Place, zeroBased:Boolean) {
%~~vis
\begin{xten}
property def rail(): Boolean = rect && onePlace == here && zeroBased;
\end{xten}
%~~siv
%}
%~~neg
and
%~~gen ^^^ Classes120
% package classes.waifsome2;
% class Waif(rect:Boolean, onePlace:Place, zeroBased:Boolean) {
%~~vis
\begin{xten}
property rail(): Boolean = rect && onePlace == here && zeroBased;
\end{xten}
%~~siv
%}
%~~neg

Similarly, nullary property methods can be inspected in constraints without
\xcd`()`. If \xcd`ob`'s type has a property \xcd`p`, then \xcd`ob.p` is that
property. Otherwise, if it has a nullary property method \xcd`p()`, \xcd`ob.p`
is equivalent to \xcd`ob.p()`. As a consequence, if the type provides both a
property \xcd`p` and a nullary method \xcd`p()`, then the property can be
accessed as \xcd`ob.p` and the method as \xcd`ob.p()`.\footnote{This only
applies to nullary property methods, not nullary instance methods.  Nullary
property methods perform limited computations, have no side effects, and
always return the same value, since
they have to be expressed in the constraint sublanguage.  In this sense, a
nullary property method does not behave hugely different from a property.
indeed, a compilation scheme which cached the value of the property method
would all but erase the distinction.  Other methods may
have more behavior, \eg, side effects, so we keep the \xcd`()` to make it
clear that a method call is potentially large.
}

%~~longexp~~`~~` ^^^ Classes130
% package classes.not.weasels;
% class Waif(rect:Boolean, onePlace:Place, zeroBased:Boolean) {
%   def this(rect:Boolean, onePlace:Place, zeroBased:Boolean) 
%          :Waif{self.rect==rect, self.onePlace==onePlace, self.zeroBased==zeroBased}
%          = {property(rect, onePlace, zeroBased);}
%   property rail(): Boolean = rect && onePlace == here && zeroBased;
%   static def zoink() {
%      val w : Waif{
%~~vis
\xcd`w.rail`, with either definition above, 
% }= new Waif(true, here, true);
% }}
%~~pxegnol
is equivalent to 
%~~longexp~~`~~` ^^^ Classes140
% package classes.not.ferrets;
% class Waif(rect:Boolean, onePlace:Place, zeroBased:Boolean) {
%   def this(rect:Boolean, onePlace:Place, zeroBased:Boolean) 
%          :Waif{self.rect==rect, self.onePlace==onePlace, self.zeroBased==zeroBased}
%          = {property(rect, onePlace, zeroBased);}
%   property rail(): Boolean = rect && onePlace == here && zeroBased;
%   static def zoink() {
%      val w : Waif{
%~~vis
\xcd`w.rail()`
% }= new Waif(true, here, true);
% }}
%~~pxegnol




\subsection{Method overloading, overriding, hiding, shadowing and obscuring}
\label{MethodOverload}
\index{method!overloading}



The definitions of method overloading, overriding, hiding, shadowing and
obscuring in \Xten{} are familiar from languages such as Java, modulo the
following considerations motivated by type parameters and dependent types.



Two or more methods of a class or interface may have the same
name if they have a different number of type parameters, or
they have formal parameters of different types.  \Eg, the following is legal: 

%~~gen ^^^ Classes150
% package Classes.Mful;
%~~vis
\begin{xten}
class Mful{
   def m() = 1;
   def m[T]() = 2;
   def m(x:Int) = 3;
   def m[T](x:Int) = 4;
}
\end{xten}
%~~siv
%
%~~neg

\XtenCurrVer{} does not permit overloading based on constraints. That is, the
following is {\em not} legal, although either method definition individually
is legal:
\begin{xten}
   def n(x:Int){x==1} = "one";
   def n(x:Int){x!=1} = "not";
\end{xten}


The definition of a method declaration \xcdmath"m$_1$" ``having the same signature
as'' a method declaration \xcdmath"m$_2$" involves identity of types. 

\noo{The following few paragraphs need to be stared at carefully.  I think
they are contradictory and/or wrong.}

The {\em constraint erasure} of a type \xcdmath"T" is defined as follows.
The constraint erasure of  (a)~a class, interface or struct type \xcdmath"T" is 
\xcdmath"T"; (b)~a type \xcdmath"T{c}" is the constraint erasure of 
\xcdmath"T"; (b)~a type \xcdmath"T[S$_1$,$\ldots$,S$_n$]" 
is \xcdmath"T'[S$_1$',$\ldots$,S$_n$']" where each primed type is the erasure of 
the corresponding unprimed type.
 Two methods are said to have {\em equivalent signatures} if (a) they have the
 same number of type parameters, 
(b) they have the same number of formal (value) parameters, and (c)
for each formal parameter the constraint erasure of its types are equivalent. It is a
compile-time error for there to be two methods with the same name and
equivalent signatures in a class (either defined in that class or in a
superclass), unless the signatures are identical and one of the methods is
defined in a superclass (in which case the superclass's method is overridden
by the subclass's).

 

A class \xcd"C" may not have two non-identical but equivalent 
declarations for
a method named 
\xcd"m"---either 
defined at \xcd"C" or inherited:
\begin{xtenmath}
def m[X$_1$, $\dots$, X$_m$](v$_1$: T$_1$, $\dots$, v$_n$: T$_n$){tc}: T {...}
def m[X$_1$, $\dots$, X$_m$](v$_1$: S$_1$, $\dots$, v$_n$: S$_n$){sc}: S {...}
\end{xtenmath}
\noindent
if it is the case that the constraint erasures of the types \xcdmath"T$_1$",
\dots, \xcdmath"T$_n$" are
equivalent to the constraint erasures of the types \xcdmath"S$_1$, $\dots$, T$_n$"
respectively.



In addition, the guard of a overriding method must be 
no stronger than the guard of the overridden method.   This
ensures that any virtual call to the method
satisfies the guard of the callee.


  If a class \xcd"C" overrides a method of a class or interface
  \xcd"B", the guard of the method in \xcd"B" must entail
  the guard of the method in \xcd"C".


A class \xcd"C" inherits from its direct superclass and superinterfaces all
their methods visible according to the access modifiers
of the superclass/superinterfaces that are not hidden or overridden. A method \xcdmath"M$_1$" in a class
\xcd"C" overrides
a method \xcdmath"M$_2$" in a superclass \xcd"D" if
\xcdmath"M$_1$" and \xcdmath"M$_2$" have the same signature with constraints erased.
Methods are overriden on a signature-by-signature basis.

\section{Constructors}
\index{constructor}

Instances of classes are created by the \xcd`new` expression: \\
%##(ClassInstCreationExp
\begin{bbgrammar}
%(FROM #(prod:ClassInstCreationExp)#)
ClassInstCreationExp \: \xcd"new" TypeName TypeArguments\opt \xcd"(" ArgumentList\opt \xcd")" ClassBody\opt & (\ref{prod:ClassInstCreationExp}) \\
                    \| \xcd"new" TypeName \xcd"[" Type \xcd"]" \xcd"[" ArgumentList\opt \xcd"]" \\
                    \| Primary \xcd"." \xcd"new" Id TypeArguments\opt \xcd"(" ArgumentList\opt \xcd")" ClassBody\opt \\
                    \| AmbiguousName \xcd"." \xcd"new" Id TypeArguments\opt \xcd"(" ArgumentList\opt \xcd")" ClassBody\opt \\
\end{bbgrammar}
%##)

This constructs a new object, and calls some code, called a {\em constructor},
to initialize the newly-created object properly.

Constructors are defined like methods, except that they are named \xcd`this`
and ordinary methods may not be.    The content of a constructor body has
certain capabilities (\eg, \xcd`val` fields of the object may be initialized)
and certain restrictions (\eg, most methods cannot be called); see
\Sref{ObjectInitialization} for the details.

\begin{ex}

The following class provides two constructors.  The unary constructor 
\xcd`def this(b : Int)` allows initialization of the \xcd`a` field to an 
arbitrary value.  The nullary constructor \xcd`def this()` gives it a default
value of 10.  The \xcd`example` method illustrates both of these calls.


%~~gen ^^^ ClassesCtor10
% package ClassesCtor10;
%~~vis
\begin{xten}
class C {
  public val a : Int;
  def this(b : Int) { a = b; } 
  def this()        { a = 10; }
  static def example() {
     val two = new C(2);
     assert two.a == 2;
     val ten = new C(); 
     assert ten.a == 10;
  }
}
\end{xten}
%~~siv
%
%~~neg
\end{ex}

\subsection{Automatic Generation of Constructors}
\index{constructor!generated}

Classes that have no constructors written in the class declaration are
automatically given a constructor which sets the class properties and does
nothing else. If this automatically-generated constructor is not valid (\eg,
if the class has \xcd`val` fields that need to be initialized in a
constructor), the class has no constructor, which is a static error.

\begin{ex}
The following class has no explicit constructor.
Its implicit constructor is 
\xcd`def this(x:Int){property(x);}`
This implicit constructor is valid, and so is the class. 
%~~gen ^^^ ClassesCtor20
% package ClassesCtor20;
%~~vis
\begin{xten}
class C(x:Int) {
  static def example() {
    val c : C = new C(4);
    assert c.x == 4;
  }
}
\end{xten}
%~~siv
%
%~~neg
\noindent 


The following class has the same default constructor.  However, that
constructor does not initialize \xcd`d`, and thus is invalid.  This 
class does not compile; it needs an explicit constructor.
%~~gen ^^^ ClassCtor30_MustFailCompile
% NOCOMPILE
%~~vis
\begin{xten}
// THIS CODE DOES NOT COMPILE
class Cfail(x:Int) {
  val d: Int;
  static def example() {
    val wrong = new Cfail(40);
  }
}
\end{xten}
%~~siv
%
%~~neg


\end{ex}

\subsection{Calling Other Constructors}

The {\em first} statement of a constructor body may be a call of the form 
\xcd`this(a,b,c)` or \xcd`super(a,b,c)`.  The former will execute the body of
the matching constructor of the current class; the latter, of the superclass. 
This allows a measure of abstraction in constructor definitions; one may be
defined in terms of another.

\begin{ex}
The following class has two constructors.  \xcd`new Ctors(123)` constructs a
new \xcd`Ctors` object with parameter 123.  \xcd`new Ctors()` constructs one
whose parameter has a default value of 100: 
%~~gen ^^^ Classes5q6q
% package Classes5q6q;
%~~vis
\begin{xten}
class Ctors {
  val a : Int;
  def this(a:Int) { this.a = a; }
  def this() {
    this(100);
  }
}
\end{xten}
%~~siv
%
%~~neg
\end{ex}

In the case of a class which implements \xcd`operator ()` 
--- or any other constructor and application with the same signature --- 
this can be ambiguous.  If \xcd`this()` appears as the first statement of a
constructor body, it could, in principle, mean either a constructor call or an
operator evaluation.   This ambiguity is resolved so that \xcd`this()` always
means the constructor invocation.  If, for some reason, it is necessary to
invoke an application operator early in a constructor, precede it with a dummy
statement, such as \xcd`if(false);`  

\section{Static initialization}
\label{StaticInitialization}
\index{initialization!static}
The \Xten{} runtime implements the following procedure to ensure
reliable initialization of the static state of classes.


Execution (of an entire X10 program) commences with a single thread executing
the 
\emph{initialization} phase of an \Xten{} computation at place \Xcd{0}. This
phase must complete successfully before the body of the \Xcd{main} method is
executed.

The initialization phase should be thought of as if it is implemented in
the following fashion. (The implementation may do something more
efficient as long as it is faithful to this semantics.)

\begin{xten}
Within the scope of a new finish
for every static field f of every class C 
   (with type T and initializer e):
async {
  val l = e; 
  ateach (Dist.makeUnique()) {
     assign l to the static f field of 
         the local C class object;
     mark the f field of the local C 
         class object as initialized;
  }
}
\end{xten}

During this phase, any read of a static field \Xcd{C.f} (where \Xcd{f} is of type \Xcd{T})
is replaced by a call to the method \Xcd{C.read\_f():T} defined on class \Xcd{C}
as follows

\begin{xten}
def read_f():T {
   when (initialized(C.f)){};
   return C.f;
}
\end{xten}
 

If all these activities terminate normally, all static fields have values of
their declared types, 
and the \Xcd{finish} terminates normally. If
any activity throws an exception, the \Xcd{finish} throws an
exception. Since no user code is executing which can catch exceptions
thrown by the finish, such exceptions are printed on the console, and
computation aborts.

If the activities deadlock, the implementation deadlocks.

In all cases, the main method is executed only once all static fields
have been initialized correctly.

Since static state is immutable and is replicated to all places via 
the initialization phase as described above, it can be accessed from
any place.



\section{User-Defined Operators}
\label{sect:operators}
\index{operator}
\index{operator!user-defined}

It is often convenient to have methods named by symbols rather than words.
For example, suppose that we wish to define a \xcd`Poly` class of
polynomials -- for the sake of illustration, single-variable polynomials with
\xcd`Int` coefficients.  It would be very nice to be able to manipulate these
polynomials by the usual operations: \xcd`+` to add, \xcd`*` to multiply,
\xcd`-` to subtract, and \xcd`p(x)` to compute the value of the polynomial at
argument \xcd`x`.  We would like to write code thus: 
%~~gen ^^^ Classes160
% package Classes.In.Poly101;
% // Integer-coefficient polynomials of one variable.
% class Poly {
%   public val coeff : Array[Int](1);
%   public def this(coeff: Array[Int](1)) { this.coeff = coeff;}
%   public def degree() = coeff.size-1;
%   public def a(i:Int) = (i<0 || i>this.degree()) ? 0 : coeff(i);
%
%   public static operator (c : Int) as Poly = new Poly([c]);
%
%   public operator this(x:Int) {
%     val d = this.degree();
%     var s : Int = this.a(d);
%     for( i in 1 .. this.degree() ) {
%        s = x * s + a(d-i);
%     }
%     return s;
%   }
%
%   public operator this + (p:Poly) =  new Poly(
%      new Array[Int](
%         Math.max(this.coeff.size, p.coeff.size),
%         (i:Int) => this.a(i) + p.a(i)
%      ));
%   public operator this - (p:Poly) = this + (-1)*p;
%
%   public operator this * (p:Poly) = new Poly(
%      new Array[Int](
%        this.degree() + p.degree() + 1,
%        (k:Int) => sumDeg(k, this, p)
%        )
%      );
%
%
%   public operator (n : Int) + this = (n as Poly) + this;
%   public operator this + (n : Int) = (n as Poly) + this;
%
%   public operator (n : Int) - this = (n as Poly) + (-1) * this;
%   public operator this - (n : Int) = ((-n) as Poly) + this;
%
%   public operator (n : Int) * this = new Poly(
%      new Array[Int](
%        this.degree()+1,
%        (k:Int) => n * this.a(k)
%      ));
%   private static def sumDeg(k:Int, a:Poly, b:Poly) {
%      var s : Int = 0;
%      for( i in 0 .. k ) s += a.a(i) * b.a(k-i);
%        // x10.io.Console.OUT.println("sumdeg(" + k + "," + a + "," + b + ")=" + s);
%      return s;
%      };
%   public final def toString() = {
%      var allZeroSoFar : Boolean = true;
%      var s : String ="";
%      for( i in 0..this.degree() ) {
%        val ai = this.a(i);
%        if (ai == 0) continue;
%        if (allZeroSoFar) {
%           allZeroSoFar = false;
%           s = term(ai, i);
%        }
%        else
%           s +=
%              (ai > 0 ? " + " : " - ")
%             +term(ai, i);
%      }
%      if (allZeroSoFar) s = "0";
%      return s;
%   }
%   private final def term(ai: Int, n:Int) = {
%      val xpow = (n==0) ? "" : (n==1) ? "x" : "x^" + n ;
%      return (ai == 1) ? xpow : "" + Math.abs(ai) + xpow;
%   }
%
%   public static def Main(ss:Array[String](1)) = main(ss);
%


%~~vis
\begin{xten}
  public static def main(Array[String](1)):void {
     val X = new Poly([0,1]);
     val t <: Poly = 7 * X + 6 * X * X * X; 
     val u <: Poly = 3 + 5*X - 7*X*X;
     val v <: Poly = t * u - 1;
     for( i in -3 .. 3) {
       x10.io.Console.OUT.println(
         "" + i + "	X:" + X(i) + "	t:" + t(i) 
         + "	u:" + u(i) + "	v:" + v(i)
         );
     }
  }

\end{xten}
%~~siv
%}
%~~neg

Writing the same code with method calls, while possible, is far less elegant: 
%~~gen ^^^ Classes170

%package Classes.In.Remedial.Poly101;
% // Integer-coefficient polynomials of one variable.
% class UglyPoly {
%   public val coeff : Array[Int](1);
%   public def this(coeff: Array[Int](1)) { this.coeff = coeff;}
%   public def degree() = coeff.size-1;
%   public  def  a(i:Int) = (i<0 || i>this.degree()) ? 0 : coeff(i);
%
%   public static operator (c : Int) as UglyPoly = new UglyPoly([c]);
%
%   public def apply(x:Int) {
%     val d = this.degree();
%     var s : Int = this.a(d);
%     for( i in 1 .. this.degree() ) {
%        s = x * s + a(d-i);
%     }
%     return s;
%   }
%
%   public operator this + (p:UglyPoly) =  new UglyPoly(
%      new Array[Int](
%         Math.max(this.coeff.size, p.coeff.size),
%         (i:Int) => this.a(i) + p.a(i)
%      ));
%   public operator this - (p:UglyPoly) = this + (-1)*p;
%
%   public operator this * (p:UglyPoly) = new UglyPoly(
%      new Array[Int](
%        this.degree() + p.degree() + 1,
%        (k:Int) => sumDeg(k, this, p)
%        )
%      );
%
%
%   public operator (n : Int) + this = (n as UglyPoly) + this;
%   public operator this + (n : Int) = (n as UglyPoly) + this;
%
%   public operator (n : Int) - this = (n as UglyPoly) + (-1) * this;
%   public operator this - (n : Int) = ((-n) as UglyPoly) + this;
%
%   public operator (n : Int) * this = new UglyPoly(
%      new Array[Int](
%        this.degree()+1,
%        (k:Int) => n * this.a(k)
%      ));
%   private static def sumDeg(k:Int, a:UglyPoly, b:UglyPoly) {
%      var s : Int = 0;
%      for( i in 0 .. k ) s += a.a(i) * b.a(k-i);
%        // x10.io.Console.OUT.println("sumdeg(" + k + "," + a + "," + b + ")=" + s);
%      return s;
%      };
%   public final def toString() = {
%      var allZeroSoFar : Boolean = true;
%      var s : String ="";
%      for( i in 0..this.degree() ) {
%        val ai = this.a(i);
%        if (ai == 0) continue;
%        if (allZeroSoFar) {
%           allZeroSoFar = false;
%           s = term(ai, i);
%        }
%        else
%           s +=
%              (ai > 0 ? " + " : " - ")
%             +term(ai, i);
%      }
%      if (allZeroSoFar) s = "0";
%      return s;
%   }
%   private final def term(ai: Int, n:Int) = {
%      val xpow = (n==0) ? "" : (n==1) ? "x" : "x^" + n ;
%      return (ai == 1) ? xpow : "" + Math.abs(ai) + xpow;
%   }
%
%   def mult(p:UglyPoly) = this * p;
%   def mult(n:Int) = n * this;
%   def plus(p:UglyPoly) = this + p;
%   def plus(n:Int) = n + this;
%   def minus(p:UglyPoly) = this - p;
%   def minus(n:Int) = this - n;
%   static def const(n:Int) = n as UglyPoly;
%
%
%~~vis
\begin{xten}
  public static def uglymain() {
     val X = new UglyPoly([0,1]);
     val t <: UglyPoly = X.mult(7).plus(
                          X.mult(X).mult(X).mult(6));  
     val u <: UglyPoly = const(3).plus(
                           X.mult(5)).minus(X.mult(X).mult(7));
     val v <: UglyPoly = t.mult(u).minus(1);
     for( i in -3 .. 3) {
       x10.io.Console.OUT.println(
         "" + i + "	X:" + X.apply(i) + "	t:" + t.apply(i) 
          + "	u:" + u.apply(i) + "	v:" + v.apply(i)
         );
     }
  }
\end{xten}
%~~siv
%}
%~~neg

The operator-using code can be written in X10, though a few variations are
necessary to handle such exotic cases as \xcd`1+X`.

%% HERE!!

Most X10 operators can be given definitions.\footnote{Indeed, even for the
standard types, these operators are defined in the library.  Not even as basic
an operation as integer addition is built into the language.  Conversely, if
you define a full-featured numeric type, it will have most of the privileges that
the standard ones enjoy.  The missing priveleges are (1) literals; (2) 
the \xcd`..` operator won't compute the \xcd`zeroBased` and \xcd`rail`
properties as it does for \xcd`Int` ranges; (3) \xcd`*` won't track ranks, as
it does for \xcd`Region`s; 
(4) \xcd`&&` and \xcd`||` won't short-circuit, as they do for \xcd`Boolean`s, 
 and (5) \xcd`a==b` will only coincide with
\xcd`a.equals(b)` if coded that way.  For example, a \xcd`Polar` type might
have many representations for the origin, as radius 0 and any angle; these
will be \xcd`equals()`, but will not be \xcd`==`.}  (However, \xcd`&&` and
\xcd`||` 
are only short-circuiting for \xcd`Boolean` expressions; user-defined versions
of these operators have no special execution behavior.)

The user-definable operations are (in order of precedence): \\
\begin{tabular}{l}
implicit type coercions\\
postfix \xcd`()`\\
\xcd`as T`\\
unary \xcd`-`, unary \xcd`+`, \xcd`!`, \xcd`~`\\
\xcd`..`\\
\xcd`*     `  \xcd`/     `  \xcd`%` \\
\xcd`+` \xcd`     -` \\
\xcd`<<    ` \xcd`>>    ` \xcd`>>>   ` \xcd`->` \\
\xcd`>     ` \xcd`>=    ` \xcd`<     ` \xcd`<=     ` 
\xcd`in     ` 
\xcd`&` \\
\xcd`^` \\
\xcd`|` \\
\xcd`&&` \\
\xcd`||` \\
\end{tabular}



\subsection{Binary Operators}

Defining the sum \xcd`P+Q` of two polynomials looks much like a method
definition.  It uses the \xcd`operator` keyword instead of \xcd`def`, and
\xcd`this` appears in the definition in the place that a \xcd`Poly` would
appear in a use of the operator.  So, 
\xcd`operator this + (p:Poly)` explains how to add \xcd`this` to a
\xcd`Poly` value.
%~~gen ^^^ Classes180
% package Classes.In.Poly102;
%~~vis
\begin{xten}
class Poly {
  public val coeff : Array[Int](1);
  public def this(coeff: Array[Int](1)) { this.coeff = coeff;}
  public def degree() = coeff.size-1;
  public def  a(i:Int) = (i<0 || i>this.degree()) ? 0 : coeff(i);

  public operator this + (p:Poly) =  new Poly(
     new Array[Int](
        Math.max(this.coeff.size, p.coeff.size),
        (i:Int) => this.a(i) + p.a(i)
     )); 
  // ... 
\end{xten}
%~~siv
%   public operator (n : Int) + this = new Poly([n]) + this;
%   public operator this + (n : Int) = new Poly([n]) + this;
% 
%   def makeSureItWorks() {
%      val x = new Poly([0,1]);
%      val p <: Poly = x+x+x;
%      val q <: Poly = 1+x;
%      val r <: Poly = x+1;
%   }
%     
% }
%~~neg


The sum of a polynomial and an integer, \xcd`P+3`, looks like
an overloaded method definition.  
%~~gen ^^^ Classes190
% package Classes.In.Poly103;
% class Poly {
%   public val coeff : Array[Int](1);
%   public def this(coeff: Array[Int](1)) { this.coeff = coeff;}
%   public def degree() = coeff.size-1;
%   public def  a(i:Int) = (i<0 || i>this.degree()) ? 0 : coeff(i);
% 
%   public operator this + (p:Poly) =  new Poly(
%      new Array[Int](
%         Math.max(this.coeff.size, p.coeff.size),
%         (i:Int) => this.a(i) + p.a(i)
%      ));
%    public operator (n : Int) + this = new Poly([n]) + this;
%~~vis
\begin{xten}
   public operator this + (n : Int) = new Poly([n]) + this;
\end{xten}
%~~siv
% 
%   def makeSureItWorks() {
%      val x = new Poly([0,1]);
%      val p <: Poly = x+x+x;
%      val q <: Poly = 1+x;
%      val r <: Poly = x+1;
%   }
%     
% }
%~~neg


However, we want to allow the sum of an integer and a polynomial as well:
\xcd`3+P`.  It would be quite inconvenient to have to define this as a method
on \xcd`Int`; changing \xcd`Int` is far outside of normal coding.  So, we
allow it as a method on \xcd`Poly` as well.


%~~gen ^^^ Classes200
% package Classes.In.Poly104o;
% class Poly {
%   public val coeff : Array[Int](1);
%   public def this(coeff: Array[Int](1)) { this.coeff = coeff;}
%   public def degree() = coeff.size-1;
%   public def  a(i:Int) = (i<0 || i>this.degree()) ? 0 : coeff(i);
% 
%   public operator this + (p:Poly) =  new Poly(
%      new Array[Int](
%         Math.max(this.coeff.size, p.coeff.size),
%         (i:Int) => this.a(i) + p.a(i)
%      ));
%~~vis
\begin{xten}
   public operator (n : Int) + this = new Poly([n]) + this;
\end{xten}
%~~siv
% 
%   public operator this + (n : Int) = new Poly([n]) + this;
%   def makeSureItWorks() {
%      val x = new Poly([0,1]);
%      val p <: Poly = x+x+x;
%      val q <: Poly = 1+x;
%      val r <: Poly = x+1;
%   }
%     
% }
%~~neg

Furthermore, it is sometimes convenient to express a binary operation as a
static method on a class. 
The definition for the sum of two
\xcd`Poly`s could have been written:
%~~gen ^^^ Classes210
% package Classes.In.Poly105;
% class Poly {
%   public val coeff : Array[Int](1);
%   public def this(coeff: Array[Int](1)) { this.coeff = coeff;}
%   public def degree() = coeff.size-1;
%   public def  a(i:Int) = (i<0 || i>this.degree()) ? 0 : coeff(i);
%~~vis
\begin{xten}
  public static operator (p:Poly) + (q:Poly) =  new Poly(
     new Array[Int](
        Math.max(q.coeff.size, p.coeff.size),
        (i:Int) => q.a(i) + p.a(i)
     ));
\end{xten}
%~~siv
%
%   public operator (n : Int) + this = new Poly([n]) + this;
%   public operator this + (n : Int) = new Poly([n]) + this;
% 
%   def makeSureItWorks() {
%      val x = new Poly([0,1]);
%      val p <: Poly = x+x+x;
%      val q <: Poly = 1+x;
%      val r <: Poly = x+1;
%   }
%     
% }
%~~neg

This requires the following grammar: \\
%##(MethodDecl
\begin{bbgrammar}
%(FROM #(prod:MethodDecl)#)
          MethodDecl \: MethMods \xcd"def" Id TypeParams\opt FormalParams WhereClause\opt HasResultType\opt Offers\opt MethodBody & (\ref{prod:MethodDecl}) \\
                    \| MethMods \xcd"operator" TypeParams\opt \xcd"(" FormalParam  \xcd")" BinOp \xcd"(" FormalParam  \xcd")" WhereClause\opt HasResultType\opt Offers\opt MethodBody \\
                    \| MethMods \xcd"operator" TypeParams\opt PrefixOp \xcd"(" FormalParam  \xcd")" WhereClause\opt HasResultType\opt Offers\opt MethodBody \\
                    \| MethMods \xcd"operator" TypeParams\opt \xcd"this" BinOp \xcd"(" FormalParam  \xcd")" WhereClause\opt HasResultType\opt Offers\opt MethodBody \\
                    \| MethMods \xcd"operator" TypeParams\opt \xcd"(" FormalParam  \xcd")" BinOp \xcd"this" WhereClause\opt HasResultType\opt Offers\opt MethodBody \\
                    \| MethMods \xcd"operator" TypeParams\opt PrefixOp \xcd"this" WhereClause\opt HasResultType\opt Offers\opt MethodBody \\
                    \| MethMods \xcd"operator" \xcd"this" TypeParams\opt FormalParams WhereClause\opt HasResultType\opt Offers\opt MethodBody \\
                    \| MethMods \xcd"operator" \xcd"this" TypeParams\opt FormalParams \xcd"=" \xcd"(" FormalParam  \xcd")" WhereClause\opt HasResultType\opt Offers\opt MethodBody \\
                    \| MethMods \xcd"operator" TypeParams\opt \xcd"(" FormalParam  \xcd")" \xcd"as" Type WhereClause\opt Offers\opt MethodBody \\
                    \| MethMods \xcd"operator" TypeParams\opt \xcd"(" FormalParam  \xcd")" \xcd"as" \xcd"?" WhereClause\opt HasResultType\opt Offers\opt MethodBody \\
                    \| MethMods \xcd"operator" TypeParams\opt \xcd"(" FormalParam  \xcd")" WhereClause\opt HasResultType\opt Offers\opt MethodBody \\
\end{bbgrammar}
%##)
When X10 attempts to typecheck a binary operator expression like \xcd`P+Q`, it
first typechecks \xcd`P` and \xcd`Q`. Then, it looks for operator declarations
for \xcd`+` in the types of \xcd`P` and \xcd`Q`. If there are none, it is a
static error. If there is precisely one, that one will be used. If there are
several, X10 looks for a {\em best-matching} operation, \viz{} one which does
not require the operands to be converted to another type. For example,
\xcd`operator this + (n:Long)` and \xcd`operator this + (n:Int)` both apply to
\xcd`p+1`, because \xcd`1` can be converted from an \xcd`Int` to a \xcd`Long`.
However, the \xcd`Int` version will be chosen because it does not require a
conversion. If even the best-matching operation is not uniquely determined,
the compiler will report a static error.

The main difference between expressing a binary operation as an instance
method (with a \xcd`this` in the definition) and a static one (no \xcd`this`)
is that instance methods don't apply any conversions, while static methods
attempt to convert both arguments. 




\subsection{Unary Operators}

Unary operators are defined in a similar way, with \xcd`this` appearing in the
\xcd`operator` definition where an actual value would occur in a unary
expression.  The operator to negate a polynomial is: 

%~~gen ^^^ Classes220
% package Classes.In.Poly106;
% class Poly {
%   public val coeff : Array[Int](1);
%   public def this(coeff: Array[Int](1)) { this.coeff = coeff;}
%   public def degree() = coeff.size-1;
%   public def  a(i:Int) = (i<0 || i>this.degree()) ? 0 : coeff(i);
%~~vis
\begin{xten}
  public operator - this = new Poly(
    new Array[Int](coeff.size, (i:Int) => -coeff(i))
    );
\end{xten}
%~~siv
%   def makeSureItWorks() {
%      val x = new Poly([0,1]);
%      val p <: Poly = -x;
%   }
% }
%~~neg

The rules for typechecking a unary operation are the same as for methods; the
complexities of binary operations are not needed.

\bard{List the operators which this works for, in precedence order}


\subsection{Type Conversions}
\index{type conversion!user-defined}

Explicit type conversions, \xcd`e as T{c}`, can be defined as operators on
class \xcd`T`.

%~~gen ^^^ Classes230
% package Classes_explicit_type_conversions_a;
%~~vis
\begin{xten}
class Poly {
  public val coeff : Array[Int](1);
  public def this(coeff: Array[Int](1)) { this.coeff = coeff;}
  public static operator (a:Int) as Poly = new Poly([a]);
  public static def main(Array[String](1)):void {
     val three : Poly = 3 as Poly;
  }
}
\end{xten}
%~~siv
%
%~~neg


Furthermore, \xcd`T` may be written as \xcd`?` in the definition of a type
conversion operator (and only there) to have it inferred from context: 

%~~gen ^^^ Classes9x1k
% package Classes9x1k;
%~~vis
\begin{xten}
class Caster(n:Int) {
  public static operator (a:Int) as ? = new Caster(a); 
  public static def example() {
    val c : Caster{n==3} = 3 as Caster{n==3};
  }
}
\end{xten}
%~~siv
%
%~~neg


The \xcd`?` may be given a bound, such as \xcd`as ? <: Caster`, if desired.

% TODO

%%TODO%%  You may define a type conversion to a constrained type, like \xcd`Poly` in
%%TODO%%  the previous example.   If you convert to a more specific constraint, X10 will use
%%TODO%%  the conversion, but insert a dynamic check to make sure that you have
%%TODO%%  satisfied the more specific constraint.  
%%TODO%%  For example: 
%%TODO%%  %~x~gen
%%TODO%%  %package Classes.And.Type.Conversions;
%%TODO%%  %~x~vis
%%TODO%%  \begin{xten}
%%TODO%%  class Uni(n:Int) {
%%TODO%%  
%%TODO%%    public def this(n:Int) : Uni{self.n==n} = {property(n);}
%%TODO%%    static operator (String) as Uni{self.n != 9} = new Uni(3);
%%TODO%%    public static def main(Array[String](1)):void {
%%TODO%%      val u = "" as Uni{self.n != 9 && self.n != 3};
%%TODO%%    }
%%TODO%%  }
%%TODO%%  \end{xten}
%%TODO%%  %~x~siv
%%TODO%%  %
%%TODO%%  %~x~neg
%%TODO%%  The string \xcd`""` is converted to \xcd`Uni{self.n != 9}` via the defined
%%TODO%%  conversion operator, and that value is checked against the remaining
%%TODO%%  constraints \xcd`{self.n != 3}` at runtime.  (In this case it will fail.)
%%TODO%%  
%%TODO%%  There may be many conversions from different types to \xcd`T`, but there may
%%TODO%%  be at most one conversion from any given type to \xcd`T`. 
%%TODO%%  
\bard{Syntax}

\subsection{Implicit Type Coercions}
\label{sect:ImplicitCoercion}
\index{type conversion!implicit}

You may also define {\em implicit} type coercions to \xcd`T{c}` as static
operators in class \xcd`T`.  The syntax for this is
\xcd`static operator (x:U) : T{c} = e`.
Implicit coercions are used automatically by the compiler on method calls 
(\Sref{sect:MethodResolution}) and assignments (\Sref{domedomedome}).



For example, we can define an implicit coercion from \xcd`Int` to \xcd`Poly`,
and avoid having to define the sum of an integer and a polynomial
as many special cases.  In the following example, we only define \xcd`+` on
two polynomials (using a \xcd`static` operator, so that implicit coercions
will be used -- they would not be for an instance method operator).  The
calculation \xcd`1+x` coerces \xcd`1` to a polynomial and uses polynomial
addition to add it to \xcd`x`.

%~~gen ^^^ Classes240
% package Classes.And.Implicit.Coercions;
% class Poly {
%   public val coeff : Array[Int](1);
%   public def this(coeff: Array[Int](1)) { this.coeff = coeff;}
%   public def degree() = coeff.size-1;
%   public def  a(i:Int) = (i<0 || i>this.degree()) ? 0 : coeff(i);
%   public final def toString() = {
%      var allZeroSoFar : Boolean = true;
%      var s : String ="";
%      for( i in 0..this.degree() ) {
%        val ai = this.a(i);
%        if (ai == 0) continue;
%        if (allZeroSoFar) {
%           allZeroSoFar = false;
%           s = term(ai, i);
%        }
%        else 
%           s += 
%              (ai > 0 ? " + " : " - ")
%             +term(ai, i);
%      }
%      if (allZeroSoFar) s = "0";
%      return s;
%   }
%   private final def term(ai: Int, n:Int) = {
%      val xpow = (n==0) ? "" : (n==1) ? "x" : "x^" + n ;
%      return (ai == 1) ? xpow : "" + Math.abs(ai) + xpow;
%   }

%~~vis
\begin{xten}
  public static operator (c : Int) : Poly = new Poly([c]);

  public static operator (p:Poly) + (q:Poly) = new Poly(
      new Array[Int](
        Math.max(p.coeff.size, q.coeff.size),
        (i:Int) => p.a(i) + q.a(i)
     ));

  public static def main(Array[String](1)):void {
     val x = new Poly([0,1]);
     x10.io.Console.OUT.println("1+x=" + (1+x));
  }
\end{xten}
%~~siv
%}
%~~neg

\bard{Syntax}

\subsection{Assignment and Application Operators}
\index{assignment operator}
\index{application operator}
\index{()}
\index{()=}
\label{set-and-apply}
X10 allows types to implement the subscripting / function application
operator, and indexed assignment.  The \xcd`Array`-like classes take advantage
of both of these in \xcd`a(i) = a(i) + 1`.  

\xcd`a(b,c,d)`
is an operator call, to an operator defined with 
\xcd`public operator this(b:B, c:C, d:D)`.  It may be overloaded.
For
example, an ordered dictionary structure could allow subscripting by numbers
with \xcd`public operator this(i:Int)`, and by string-valued keys with 
\xcd`public operator this(s:String)`.  


\xcd`a(i,j)=b` is an \xcd`operator` as well, with zero or more indices
\xcd`i,j`.  It may also be overloaded. 

The update operations \xcd`a(i) += b` are defined to be the same as the
corresponding \xcd`a(i) = a(i) + b`. This applies for all arities of
arguments, and all types, and all binary operations. Of course to use this,
both the application and assignment \xcd`operator`s must be defined.


\begin{ex}

The \xcd`Oddvec` class of somewhat peculiar vectors illustrates this.
\xcd`a()` returns a string representation of the oddvec, which probably should
be done by \xcd`toString()` instead.  
\xcd`a(i)` sensibly picks out one of the three
coordinates of \xcd`a`.
\xcd`a()=b` sets all the coordinates of \xcd`a` to \xcd`b`.
\xcd`a(i)=b` assigns to one of the
coordinates.  \xcd`a(i,j)=b` assigns different values to \xcd`a(i)` and
\xcd`a(j)`, purely for the sake of the example.

%~~gen ^^^ Classes250
% package Classes.Assignments1_oddvec;
%~~vis
\begin{xten}
class Oddvec {
  var v : Array[Int](1) = new Array[Int](3, (Int)=>0);
  public operator this () = 
      "(" + v(0) + "," + v(1) + "," + v(2) + ")";
  public operator this () = (newval: Int) { 
    for(p in v) v(p) = newval;
  }
  public operator this(i:Int) = v(i);
  public operator this(i:Int, j:Int) = [v(i),v(j)];
  public operator this(i:Int) = (newval:Int) = {v(i) = newval;}
  public operator this(i:Int, j:Int) = (newval:Int) = {
       v(i) = newval; v(j) = newval+1;} 
  public def example() {
    this(1) = 6;   assert this(1) == 6;
    this(1) += 7;  assert this(1) == 13;
  }
\end{xten}
%~~siv
% }
%  class Hook { def run() {
%     val a = new Oddvec();
%     assert a().equals("(0,0,0)");
%     a() = 1;
%     assert a().equals("(1,1,1)");
%     a(1) = 4;
%     assert a().equals("(1,4,1)");
%     a(0,2) = 5;
%     assert a().equals("(5,4,6)");
%     return true;
%   }
% }
%~~neg

\end{ex}

\section{Class Guards and Invariants}\label{DepType:ClassGuard}
\index{type invariants}
\index{class invariants}
\index{invariant!type}
\index{invariant!class}
\index{guard}


Classes (and structs and interfaces) may specify a {\em class guard}, a
constraint which must hold on all values of the class.    In the following
example, a \xcd`Line` is defined by two distinct \xcd`Pt`s\footnote{We use \xcd`Pt`
to avoid any possible confusion with the built-in class \xcd`Point`.}
%~~gen ^^^ Classes260
% package classes.guards.invariants.glurp;
%~~vis
\begin{xten}
class Pt(x:Int, y:Int){}
class Line(a:Pt, b:Pt){a != b} {}
\end{xten}
%~~siv
%
%~~neg

In most cases the class guard could be phrased as a type constraint on a property of
the class instead, if preferred.  Arguably, a symmetric constraint like two
points being different is better expressed as a class guard, rather than
asymmetrically as a constraint on one type: 
%~~gen ^^^ Classes270
% package classes.guards.invariants.glurp2;
% class Pt(x:Int, y:Int){}
%~~vis
\begin{xten}
class Line(a:Pt, b:Pt{a != b}) {}
\end{xten}
%~~siv
%
%~~neg



\label{DepType:TypeInvariant}
\index{class invariant}
\index{invariant!class}
\index{class!invariant}
\label{DepType:ClassGuardDef}



With every defined class, struct,  or interface \xcd"T" we associate a {\em type
invariant} $\mathit{inv}($\xcd"T"$)$, which describes the guarantees on the
properties of values of type \xcd`T`.  

Every value of \xcd`T` satisfies $\mathit{inv}($\xcd"T"$)$ at all times.  This
is somewhat stronger than the concept of type invariant in most languages
(which only requires that the invariant holds when no method calls are
active).  X10 invariants only concern properties, which are immutable; thus,
once established, they cannot be falsified.

The type
invariant associated with \xcd"x10.lang.Any"
is 
\xcd"true".

The type invariant associated with any interface or struct \xcd"I" that extends
interfaces \xcdmath"I$_1$, $\dots$, I$_k$" and defines properties
\xcdmath"x$_1$: P$_1$, $\dots$, x$_n$: P$_n$" and
specifies a guard \xcd"c" is given by:

\begin{xtenmath}
$\mathit{inv}$(I$_1$) && $\dots$ && $\mathit{inv}$(I$_k$) 
    && self.x$_1$ instanceof P$_1$ &&  $\dots$ &&  self.x$_n$ instanceof P$_n$ 
    && c  
\end{xtenmath}

Similarly the type invariant associated with any class \xcd"C" that
implements interfaces \xcdmath"I$_1$, $\dots$, I$_k$",
extends class \xcd"D" and defines properties
\xcdmath"x$_1$: P$_1$, $\dots$, x$_n$: P$_n$" and
specifies a guard \xcd"c" is
given by the same thing with the invariant of the superclass \xcd`D` conjoined:
\begin{xtenmath}
$\mathit{inv}$(I$_1$) && $\dots$ && $\mathit{inv}$(I$_k$) 
    && self.x$_1$ instanceof P$_1$ &&  $\dots$ &&  self.x$_n$ instanceof P$_n$ 
    && c  
    && $\mathit{inv}$(D)
\end{xtenmath}


Note that the type invariant associated with a class entails the type
invariants of each interface that it implements (directly or indirectly), and
the type invariant of each ancestor class.
It is guaranteed that for any variable \xcd"v" of
type \xcd"T{c}" (where \xcd"T" is an interface name or a class name) the only
objects \xcd"o" that may be stored in \xcd"v" are such that \xcd"o" satisfies
$\mathit{inv}(\mbox{\xcd"T"}[\mbox{\xcd"o"}/\mbox{\xcd"this"}])
\wedge \mbox{\xcd"c"}[\mbox{\xcd"o"}/\mbox{\xcd"self"}]$.



\subsection{Invariants for {\tt implements} and {\tt extends} clauses}\label{DepType:Implements}
\label{DepType:Extends}
\index{type-checking!implements clause}
\index{type-checking!extends clause}
\index{implements}
\index{extends}
Consider a class definition
\begin{xtenmath}
$\mbox{\emph{ClassModifiers}}^{\mbox{?}}$
class C(x$_1$: P$_1$, $\dots$, x$_n$: P$_n$) extends D{d}
   implements I$_1${c$_1$}, $\dots$, I$_k${c$_k$}
$\mbox{\emph{ClassBody}}$
\end{xtenmath}

Each of the following static semantics rules must be satisfied:

\noo{reformat this}

The type invariant \xcdmath"$\mathit{inv}$(C)" of \xcd"C" must entail
\xcdmath"c$_i$[this/self]" for each $i$ in $\{1, \dots, k\}$



The return type \xcd"c" of each constructor in a class \xcd`C`
must entail the invariant \xcdmath"$\mathit{inv}$(C)".


\subsection{Invariants and constructor definitions}
\index{invariant!and constructor}
\index{constructor!and invariant}

A constructor for a class \xcd"C" is guaranteed to return an object of the
class on successful termination. This object must satisfy  \xcdmath"$\mathit{inv}$(C)", the
class invariant associated with \xcd"C" (\Sref{DepType:TypeInvariant}).
However,
often the objects returned by a constructor may satisfy {\em stronger}
properties than the class invariant. \Xten{}'s dependent type system
permits these extra properties to be asserted with the constructor in
the form of a constrained type (the ``return type'' of the constructor):

%##(CtorDecl
\begin{bbgrammar}
%(FROM #(prod:CtorDecl)#)
            CtorDecl \: Mods\opt \xcd"def" \xcd"this" TypeParams\opt FormalParams WhereClause\opt HasResultType\opt  CtorBody & (\ref{prod:CtorDecl}) \\
\end{bbgrammar}
%##)

\label{ConstructorGuard}

The parameter list for the constructor
may specify a \emph{guard} that is to be satisfied by the parameters
to the list.

\begin{ex}
%%TODO--rewrite this
Here is another example, constructed as a simplified 
version of \Xcd{x10.array.Region}.  The \xcd`mockUnion` method 
has the type, though not the value, that a true \xcd`union` method would have.

%~~gen ^^^ Classes280
%package Classes.SimplifiedRegion;
%~~vis
\begin{xten}
class MyRegion(rank:Int) {
  static type MyRegion(n:Int)=MyRegion{rank==n};
  def this(r:Int):MyRegion(r) {
    property(r);
  }
  def this(diag:Array[Int](1)):MyRegion(diag.size){ 
    property(diag.size);
  }
  def mockUnion(r:MyRegion(rank)):MyRegion(rank) = this;
  def example() {
    val R1 : MyRegion(3) = new MyRegion([4,4,4]); 
    val R2 : MyRegion(3) = new MyRegion([5,4,1]); 
    val R3 = R1.mockUnion(R2); // inferred type MyRegion(3)
  }
}
\end{xten}
%~~siv
%
%~~neg
The first constructor returns the empty region of rank \Xcd{r}.  The
second constructor takes a \Xcd{Array[Int](1)} of arbitrary length
\Xcd{n} and returns a \Xcd{MyRegion(n)} (intended to represent the set
of points in the rectangular parallelopiped between the origin and the
\Xcd{diag}.)

The code in \xcd`example` typechecks, and \xcd`R3`'s type is inferred as
\xcd`MyRegion(3)`.  


\end{ex}

   Let \xcd"C" be a class with properties
   \xcdmath"p$_1$: P$_1$, $\dots$, p$_n$: P$_n$", and invariant \xcd"c"
   extending the constrained type \xcd"D{d}" (where \xcd"D" is the name of a
   class).



   For every constructor in \xcd"C" the compiler checks that the call to
   super invokes a constructor for \xcd"D" whose return type is strong enough
   to entail \xcd"d". Specifically, if the call to super is of the form 
     \xcdmath"super(e$_1$, $\dots$, e$_k$)"
   and the static type of each expression \xcdmath"e$_i$" is
   \xcdmath"S$_i$", and the invocation
   is statically resolved to a constructor
\xcdmath"def this(x$_1$: T$_1$, $\dots$, x$_k$: T$_k$){c}: D{d$_1$}"
   then it must be the case that 
\begin{xtenmath}
x$_1$: S$_1$, $\dots$, x$_i$: S$_i$ entails x$_i$: T$_i$  (for $i \in \{1, \dots, k\}$)
x$_1$: S$_1$, $\dots$, x$_k$: S$_k$ entails c  
d$_1$[a/self], x$_1$: S$_1$, ..., x$_k$: S$_k$ entails d[a/self]      
\end{xtenmath}
\noindent where \xcd"a" is a constant that does not appear in 
\xcdmath"x$_1$: S$_1$ $\wedge$ ... $\wedge$ x$_k$: S$_k$".

   The compiler checks that every constructor for \xcd"C" ensures that
   the properties \xcdmath"p$_1$,..., p$_n$" are initialized with values which satisfy
   $\mathit{inv}($\xcd"T"$)$, and its own return type \xcd"c'" as follows.  In each constructor, the
   compiler checks that the static types \xcdmath"T$_i$" of the expressions \xcdmath"e$_i$"
   assigned to \xcdmath"p$_i$" are such that the following is
   true:
\begin{xtenmath}
p$_1$: T$_1$, $\dots$, p$_n$: T$_n$ entails $\mathit{inv}($T$)$ $\wedge$ c'     
\end{xtenmath}

(Note that for the assignment of \xcdmath"e$_i$" to \xcdmath"p$_i$"
to be type-correct it must be the
    case that \xcdmath"p$_i$: T$_i$ $\wedge$ p$_i$: P$_i$".) 



The compiler must check that every invocation \xcdmath"C(e$_1$, $\dots$, e$_n$)" to a
constructor is type correct: each argument \xcdmath"e$_i$" must have a static type
that is a subtype of the declared type \xcdmath"T$_i$" for the $i$th
argument of the
constructor, and the conjunction of static types of the argument must
entail the constraint in the parameter list of the constructor.



\subsection{Object Initialization}
\label{ObjectInitialization}
\index{initialization}
\index{constructor}
\index{object!constructor}
\index{struct!constructor}


X10 does object initialization safely.  It avoids certain bad things which
trouble some other languages:
\begin{enumerate}
\item Use of a field before the field has been initialized.
\item A program reading two different values from a \xcd`val` field of a
      container. 
\item \Xcd{this} escaping from a constructor, which can cause problems as
      noted below. 

\end{enumerate}

It should be unsurprising that fields must not be used before they are
initialized. At best, it is uncertain what value will be in them, as in
\Xcd{x} below. Worse, the value might not even be an allowable value; \Xcd{y},
declared to be nonzero in the following example, might be zero before it is
initialized.
\begin{xten}
// Not correct X10
class ThisIsWrong {
  val x : Int;
  val y : Int{y != 0};
  def this() {
    x10.io.Console.OUT.println("x=" + x + "; y=" + y);
    x = 1; y = 2;
  }
}
\end{xten}

One particularly insidious way to read uninitialized fields is to allow
\Xcd{this} to escape from a constructor. For example, the constructor could
put \Xcd{this} into a data structure before initializing it, and another
activity could read it from the data structure and look at its fields:
\begin{xten}
class Wrong {
  val shouldBe8 : Int;
  static Cell[Wrong] wrongCell = new Cell[Wrong]();
  static def doItWrong() {
     finish {
       async { new Wrong(); } // (A)
       assert( wrongCell().shouldBe8 == 8); // (B)
     }
  }
  def this() {
     wrongCell.set(this); // (C) - ILLEGAL
     this.shouldBe8 = 8; // (D)
  }
}
\end{xten}
\noindent
In this example, the underconstructed \Xcd{Wrong} object is leaked into a
storage cell at line \Xcd{(C)}, and then initialized.  The \Xcd{doItWrong}
method constructs a new \Xcd{Wrong} object, and looks at the \Xcd{Wrong}
object in the storage cell to check on its \Xcd{shouldBe8} field.  One
possible order of events is the following:
\begin{enumerate}
\item \Xcd{doItWrong()} is called.
\item \Xcd{(A)} is started.  Space for a new \Xcd{Wrong} object is allocated.
      Its \Xcd{shouldBe8} field, not yet initialized, contains some garbage
      value.
\item \Xcd{(C)} is executed, as part of the process of constructing a new
      \Xcd{Wrong} object.  The new, uninitialized object is stored in
      \Xcd{wrongCell}.
\item Now, the initialization activity is paused, and execution of the main activity
      proceeds from \Xcd{(B)}.
\item The value in \Xcd{wrongCell} is retrieved, and is \Xcd{shouldBe8} field
      is read.  This field contains garbage, and the assertion fails.
\item Now let the initialization activity proceed with \Xcd{(D)},
      initializing \Xcd{shouldBe8} --- too late.
\end{enumerate}

The \xcd`at` statement (\Sref{AtStatement}) introduces the potential for
escape as well. The following class prints an uninitialized value: 
%~~gen ^^^ ThisEscapingViaAt_MustFailCompile
% package ObjInit_at;
% NOCOMPILE
%~~vis
\begin{xten}
class Example {
  val a: Int;
  def this() { 
    at(here.next()) {
      // Recall that 'this' is a copy of 'this' outside 'at'.
      Console.OUT.println("this.a = " + this.a);
    }
    this.a = 1;
  }
}
\end{xten}
%~~siv
%
%~~neg


X10 must protect against such possibilities.  The rules explaining how
constructors can be written are somewhat intricate; they are designed to allow
as much programming as possible without leading to potential problems.
Ultimately, they simply are elaborations of the fundamental principles that
uninitialized fields must never be read, and \Xcd{this} must never be leaked.

\subsection{Raw and Cooked Objects}
\index{raw}
\index{cooked}

An object is {\em raw} before its constructor ends, and {\em cooked} after its
constructor ends. Note that, when an object is cooked, all its subobjects are
cooked.  




\subsection{Constructors and NonEscaping Methods}
\index{non-escaping}
\label{sect:nonescaping}

In general, constructors must not be allowed to call methods with \Xcd{this} as
an argument or receiver. Such calls could leak references to \Xcd{this},
either directly from a call to \Xcd{cell.set(this)}, or indirectly because
\Xcd{toString} leaks \Xcd{this}, and the concatenation
\Xcd`"Escaper = "+this` calls \Xcd{toString}.\footnote{This is abominable behavior for
\Xcd{toString}, but it cannot be prevented -- save by a scheme such as we
present in this section.}
%~WRONG~gen
%package ObjectInit.CtorAndNonEscaping.One;
%~WRONG~vis
\begin{xten}
class Escaper {
  static val Cell[Escaper] cell = new Cell[Escaper]();
  def toString() {
    cell.set(this);
    return "Evil!";
  }
  def this() {
    cell.set(this);
    x10.io.Console.OUT.println("Escaper = " + this);
  }
}
\end{xten}
%~WRONG~siv
%
%~WRONG~neg

However, it is convenient to be able to call methods from constructors; {\em
e.g.}, a class might have eleven constructors whose common behavior is best
described by three methods.
Under certain stringent conditions, it {\em is}
safe to call a method: the method called must not leak references to
\Xcd{this}, and must not read \Xcd{val}s or \Xcd{var}s which might not have
been assigned.

So, X10 performs a static dataflow analysis, sufficient to guarantee that
method calls in constructors are safe.  This analysis requires having access
to or guarantees about all the code that could possibly be called.  This can
be accomplished in two ways:
\begin{enumerate}
\item Ensuring that only code from the class itself can be called, by
      forbidding overriding of
      methods called from the constructor: they can be marked \Xcd{final} or
      \Xcd{private}, or the whole class can be \Xcd{final}.
\item Marking the methods called from the constructor by
      \xcd`@NonEscaping`.
\end{enumerate}

\subsubsection{Non-Escaping Methods}
\index{method!non-escaping}
\index{method!implicitly non-escaping}
\index{method!NonEscaping}
\index{implicitly non-escaping}
\index{non-escaping}
\index{non-escaping!implicitly}
\index{NonEscaping}


A method may be annotated with \xcd`@NonEscaping`.  This
imposes several restrictions on the method body, and on all methods overriding
it.  However, it is the only way that a method can be called from
constructors.  The
\Xcd{@NonEscaping} annotation makes explicit all the X10 compiler's needs for
constructor-safety.

A method can, however, be safe to call from constructors without being marked
\Xcd{@NonEscaping}. We call such methods {\em implicitly non-escaping}.
Implicitly non-escaping methods need to obey the same constraints on
\Xcd{this}, \Xcd{super}, and variable usage as \Xcd{@NonEscaping} methods. An
implicitly non-escaping method {\em could} be marked as
\xcd`@NonEscaping` for some list of variables; the compiler, in
effect, infers the annotation. In addition, implicitly non-escaping methods
must be \Xcd{private} or \Xcd{final} or members of a \Xcd{final} class; this
corresponds to the hereditary nature of \Xcd{@NonEscaping} (by forbidding
inheritance of implicitly non-escaping methods).

We say that a method is {\em non-escaping} if it is either implicitly
non-escaping, or annotated \Xcd{@NonEscaping}.

The first requirement on non-escaping methods is that they do not allow
\Xcd{this} to escape. Inside of their bodies, \Xcd{this} and \Xcd{super} may
only be used for field access and assignment, and as the receiver of
non-escaping methods.

Finally, if a method \Xcd{m} in class \Xcd{C} is marked
\xcd`@NonEscaping`, then every method which overrides \Xcd{m} in any
subclass of \Xcd{C} must be annotated with precisely the same annotation,
\xcd`@NonEscaping`, as well.

The following example uses most of the possible variations (leaving out
\Xcd{final} class).  \Xcd{aplomb()} explicitly forbids reading any field but
\Xcd{a}. \Xcd{boric()} is called after \Xcd{a} and \Xcd{b} are set, but
\Xcd{c} is not.
The \xcd`@NonEscaping` annotation on \xcd`boric()` is optional, but the
compiler will print a warning if it is left out.
\Xcd{cajoled()} is only called after all fields are set, so it
can read anything; its annotation, too, is not required.   \Xcd{SeeAlso} is able to override \Xcd{aplomb()}, because
\Xcd{aplomb()} is \xcd`@NonEscaping("a")`; it cannot override the final method
\Xcd{boric()} or the private one \Xcd{cajoled()}.  Even for overriding
\Xcd{aplomb()}, it is crucial that \Xcd{SeeAlso.aplomb()} be
declared \xcd`@NonEscaping("a")`, just like \Xcd{C2.aplomb()}.
%~~gen ^^^ ObjectInitialization10
%package ObjInit.C2;
%~~vis
\begin{xten}
import x10.compiler.*;

final class C2 {
  protected val a:Int, b:Int, c:Int;
  protected var x:Int, y:Int, z:Int;
  def this() {
    a = 1;
    this.aplomb();
    b = 2;
    this.boric();
    c = 3;
    this.cajoled();
  }
  @NonEscaping def aplomb() {
    x = a;
    // this.boric(); // not allowed; boric reads b.
    // z = b; // not allowed -- only 'a' can be read here
  }
  @NonEscaping final def boric() {
    y = b;
    this.aplomb(); // allowed; a is definitely set before boric is called
    // z = c; // not allowed; c is not definitely written
  }
  @NonEscaping private def cajoled() {
    z = c;
  }
}

\end{xten}
%~~siv
%
%~~neg



\subsection{Fine Structure of Constructors}
\label{SFineStructCtors}

The code of a constructor consists of four segments, three of them optional
and one of them implicit.
\begin{enumerate}
\item The first segment is an optional call to \Xcd{this(...)} or
      \Xcd{super(...)}.  If this is supplied, it must be the first statement
      of the constructor.  If it is not supplied, the compiler treats it as a
      nullary super-call \Xcd{super()};
\item If the class or struct has properties, there must be a single
      \Xcd{property(...)} command in the constructor.  Every execution path
      through the constructor must go through this \Xcd{property(...)} command
      precisely once.   The second segment of the constructor is the code
      following the first segment, up to and including the \Xcd{property()}
      statement.

      If the class or struct has no properties, the \Xcd{property()} call must
      be omitted. If it is present, the second segment is defined as before.  If
      it is absent, the second segment is empty.
\item The third segment is automatically generated.  Fields with initializers
      are initialized immediately after the \Xcd{property} statement.
      In the following example, \Xcd{b} is initialized to \Xcd{y*9000} in
      segment three.  The initialization makes sense and does the right
      thing; \Xcd{b} will be \Xcd{y*9000} for every \Xcd{Overdone} object.
      (This would not be possible if field initializers were processed
      earlier, before properties were set.)
\item The fourth segment is the remainder of the constructor body.
\end{enumerate}

The segments in the following code are shown in the comments.
%~~gen ^^^ ObjectInitialization20
% package ObjectInitialization.ShowingSegments;
%~~vis
\begin{xten}
class Overlord(x:Int) {
  def this(x:Int) { property(x); }
}//Overlord
class Overdone(y:Int) extends Overlord  {
  val a : Int;
  val b =  y * 9000;
  def this(r:Int) {
    super(r);                      // (1)
    x10.io.Console.OUT.println(r); // (2)
    val rp1 = r+1;
    property(rp1);                 // (2)
    // field initializations here  // (3)
    a = r + 2;                     // (4)
  }
}//Overdone
\end{xten}
%~~siv
%
%~~neg

The rules of what is allowed in the three segments are different, though
unsurprising.  For example, properties of the current class can only be read
in segment 3 or 4---naturally, because they are set at the end of segment 2.

\subsubsection{Initialization and Inner Classses}
\index{constructor!inner classes in}

Constructors of inner classes are tantamount to method calls on \Xcd{this}.
For example, the constructor for Inner {\bf is} acceptable.  It does not leak
\Xcd{this}.  It leaks \Xcd{Outer.this}, which is an utterly different object.
So, the call to \Xcd{this.new Inner()} in the \Xcd{Outer} constructor {\em
is} illegal.  It would leak \Xcd{this}.  There is no special rule in effect
preventing this; a constructor call of an inner class is no
different from a method as far as leaking is concerned.
%~~gen ^^^ ObjectInitialization30
% package ObjInit.InnerClass; 
% NOTEST
%~~vis
\begin{xten}
class Outer {
  static val leak : Cell[Outer] = new Cell[Outer](null);
  class Inner {
     def this() {Outer.leak.set(Outer.this);}
  }
  def /*Outer*/this() {
     //ERROR: val inner = this.new Inner();
  }
}
\end{xten}
%~~siv
%
%~~neg



\subsubsection{Initialization and Closures}
\index{constructor!closure in}

Closures in constructors may not refer to \xcd`this`.  They may not even refer
to fields of \xcd`this` that have been initialized.   For example, the
closure \xcd`bad1` is not allowed because it refers to \xcd`this`; \xcd`bad2`
is not allowed because it mentions \xcd`a` --- which is, of course, identical
to \xcd`this.a`. 

%%-deleted-%% valid if they were invoked (or inlined) at the
%%-deleted-%%place of creation. For example, \Xcd{closure} below is acceptable because it
%%-deleted-%%only refers to fields defined at the point it was written.  \Xcd{badClosure}
%%-deleted-%%would not be acceptable, because it refers to fields that were not defined at
%%-deleted-%%that point, although they were defined later.
%~~gen ^^^ ObjectInitialization40
% package ObjectInitialization.Closures; 
%~~vis
\begin{xten}
class C {
  val a:Int;
  def this() {
    this.a = 1;
    //ERROR: val bad1 = () => this; 
    //ERROR: val bad2 = () => a*10;
  }
}
\end{xten}
%~~siv
%
%~~neg


\subsection{Definite Initialization in Constructors}


An instance field \Xcd{var x:T}, when \Xcd{T} has a default value, need not be
explicitly initialized.  In this case, \Xcd{x} will be initialized to the
default value of type \Xcd{T}.  For example, a \Xcd{Score} object will have
its \Xcd{currently} field initialized to zero, below:
%~~gen ^^^ ObjectInitialization50
% package ObjectInit.DefaultInit;
%~~vis
\begin{xten}
class Score {
  public var currently : Int;
}
\end{xten}
%~~siv
%
%~~neg

All other sorts of instance fields do need to be initialized before they can
be used.  \Xcd{val} fields must be initialized, even if their type has a
default value.  It would be silly to have a field \Xcd{val z : Int} that was
always given default value of \Xcd{0} and, since it is \Xcd{val}, can never be
changed.  \Xcd{var} fields whose type has no default value must be initialized
as well, such as \xcd`var y : Int{y != 0}`, since it cannot be assigned a
sensible initial value.

The fundamental principles are:
\begin{enumerate}
\item \Xcd{val} fields must be assigned precisely once in each constructor on every
possible execution path.
\item \Xcd{var} fields of defaultless type must be
assigned at least once on every possible execution path, but may be assigned
more than once.
\item No variable may be read before it is guaranteed to have been
assigned.
\item Initialization may be by field initialization expressions (\Xcd{val x :
      Int = y+z}), or by uninitialized fields \Xcd{val x : Int;} plus
an initializing assignment \Xcd{x = y+z}.  Recall that field initialization
expressions are performed after the \Xcd{property} statement, in segment 3 in
the terminology of \Sref{SFineStructCtors}.
\end{enumerate}



\subsection{Summary of Restrictions on Classes and Constructors}

The following table tells whether a given feature is (yes), is not (no) or is
with some conditions (note) allowed in a given context.   For example, a
property method is allowed with the type of another property, as long as it
only mentions the preceding properties. The first column of the table gives
examples, by line of the following code body.

\begin{tabular}{||l|l|c|c|c|c|c|c||}
\hline
~
  & {\bf Example}
  & {\bf Prop.}
  & {\bf {\tt \small self==this}(1)}
  & {\bf Prop.Meth.}
  & {\bf {\tt this}}
  & {\bf {fields}}
\\\hline
Type of property
  & (A)
  & %?properties
    yes (2)
  & no %? self==this
  & no %? property methods
  & no %? this may be used
  & no %? fields may be used
\\\hline
Class Invariant
  & (B)
  & yes %?properties
  & yes %? self==this
  & yes %? property methods
  & yes %? this may be used
  & no %? fields may be used
\\\hline
Supertype (3)
  & (C), (D)
  & yes%?properties
  & yes%? self==this
  & yes%? property methods
  & no%? this may be used
  & no%? fields may be used
\\\hline
Property Method Body
  & (E)
  & yes %?properties
  & yes %? self==this
  & yes %? property methods
  & yes %? this may be used
  & no %? fields may be used
\\\hline

Static field (4)
  & (F) (G)
  & no %?properties
  & no %? self==this
  & no %? property methods
  & no %? this may be used
  & no %? fields may be used
\\\hline

Instance field (5)
  & (H), (I)
  & yes %?properties
  & yes %? self==this
  & yes %? property methods
  & yes %? this may be used
  & yes %? fields may be used
\\\hline

Constructor arg. type
  & (J)
  & no %?properties
  & no %? self==this
  & no  %? property methods
  & no %? this may be used
  & no %? fields may be used
\\\hline

Constructor guard
  & (K)
  & no %?properties
  & no %? self==this
  & no %? property methods
  & no %? this may be used
  & no %? fields may be used
\\\hline

Constructor ret. type
  & (L)
  & yes %?properties
  & yes %? self==this
  & yes %? property methods
  & yes %? this may be used
  & yes %? fields may be used
\\\hline

Constructor segment 1
  & (M)
  & no%?properties
  & yes%? self==this
  & no%? property methods
  & no%? this may be used
  & no%? fields may be used
\\\hline


Constructor segment 2
  & (N)
  & no%?properties
  & yes%? self==this
  & no%? property methods
  & no%? this may be used
  & no%? fields may be used
\\\hline

Constructor segment 4
  & (O)
  & yes%?properties
  & yes%? self==this
  & yes%? property methods
  & yes%? this may be used
  & yes%? fields may be used
\\\hline

Methods
  & (P)
  & yes %?properties
  & yes %? self==this
  & yes %? property methods
  & yes %? this may be used
  & yes %? fields may be used
\\\hline



\iffalse
place
  & (pos)
  & %?properties
  & %? self==this
  & %? property methods
  & %? this may be used
  & %? fields may be used
\\\hline
\fi
\end{tabular}

Details:

\begin{itemize}
\item (1) {Top-level {\tt self} only.}
\item (2) {The type of the {$i^{th}$} property may only mention
                 properties {$1$} through {$i$}.}
\item (3) Super-interfaces follow the same rules as supertypes.
\item (4) The same rules apply to types and initializers.
\end{itemize}



The example indices refer to the following code:
%~~gen ^^^ ObjectInitialization60
% package ObjectInit.GrandExample;
% class Supertype[T]{}
% interface SuperInterface[T]{}
%~~vis
\begin{xten}
class Example (
   prop : Int,
   proq : Int{prop != proq},                    // (A)
   pror : Int
   )
   {prop != 0}                                  // (B)
   extends Supertype[Int{self != prop}]         // (C)
   implements SuperInterface[Int{self != prop}] // (D)
{
   property def propmeth() = (prop == pror);    // (E)
   static staticField
      : Cell[Int{self != 0}]                    // (F)
      = new Cell[Int{self != 0}](1);            // (G)
   var instanceField
      : Int {self != prop}                      // (H)
      = (prop + 1) as Int{self != prop};        // (I)
   def this(
      a : Int{a != 0},
      b : Int{b != a}                           // (J)
      )
      {a != b}                                  // (K)
      : Example{self.prop == a && self.proq==b} // (L)
   {
      super();                                  // (M)
      property(a,b,a);                          // (N)
      // fields initialized here
      instanceField = b as Int{self != prop};   // (O)
   }

   def someMethod() =
        prop + staticField + instanceField;     // (P)
}
\end{xten}
%~~siv
%
%~~neg


\section{Method Resolution}
\index{method!resolution}
\index{method!which one will get called}
\label{sect:MethodResolution}

Method resolution is the problem of determining, statically, which method (or
constructor or operator)
should be invoked, when there are several choices that could be invoked.  For
example, the following class has two overloaded \xcd`zap` methods, one taking
an \Xcd{Object}, and the other a \Xcd{Resolve}.  Method resolution will figure
out that the call \Xcd{zap(1..4)} should call \xcd`zap(Object)`, and
\Xcd{zap(new Resolve())} should call \xcd`zap(Resolve)`.  

%~~gen ^^^ MethodResolution10
%package MethodResolution.yousayyouwantaresolution;
%~~vis
\begin{xten}
class Resolve {
  static def zap(Object) = "Object";
  static def zap(Resolve) = "Resolve";
  public static def main(argv:Array[String](1)) {
    Console.OUT.println(zap(1..4));
    Console.OUT.println(zap(new Resolve()));
  }
}
\end{xten}
%~~siv
%
%~~neg

The basic concept of method resolution is quite straightforward: 
\begin{enumerate}
\item List all the methods that could possibly be used (counting implicit
      coercions); 
\item Pick the most specific one; 
\item Fail if there is not a unique most specific one.
\end{enumerate}
\noindent
In the presence of X10's highly-detailed type system, some subtleties arise. 
One point, at least, is {\em not} subtle. The same procedure is used, {\em
mutatis mutandis} for method, constructor, and operator resolution.  

Generics introduce several subtleties, especially with the inference of
generic types. 


For the purposes of method resolution, all that matters about a method,
constructor, or operator \xcd`M` --- we use the word ``method'' to include all
three choices for this section --- is its signature, plus which method it is.
So, a typical \xcd`M` might look like 
\xcdmath"def m[G$_1$,$\ldots$, G$_g$](x$_1$:T$_1$,$\ldots$, x$_f$:T$_f$){c} =...".  The code body \xcd`...` is irrelevant for the purpose of whether a
given method call means \xcd`M` or not, so we ignore it for this section.

All that matters about a method definition, for the purposes of method
resolution, is: 
\begin{enumerate}
\item The method name \xcd`m`;
\item The generic type parameters of the method \xcd`M`,  \xcdmath"G$_1$,$\ldots$, G$_g$".  If there
      are no generic type parameters, {$g=0$}.  
\item The types \xcdmath"x$_1$:T$_1$,$\ldots$, x$_f$:T$_f$" of the formal parameters.  If
      there are no formal parameters, {$f=0$}. In the case of an instance
      method, the receiver will be the first formal parameter.\footnote{The
      variable names are relevant because one formal can be mentioned in a
      later type, or even a constraint: {\tt def f(a:Int, b:Point\{rank==a\})=...}.}
\item The constraint \xcd`c` of the method \xcd`M`. If no constraint is specified, \xcd`c` is
      \xcd`true`. 
\item A {\em unique identifier} \xcd`id`, sufficient to tell the compiler
      which method body is intended.  A file name and position in that file
      would suffice.  The details of the identifier are not relevant.
\end{enumerate}

For the purposes of understanding method resolution, we assume that all the
actual parameters of an invocation are simply variables: \xcd`x1.meth(x2,x3)`.
This is done routinely by the compiler in any case; the code 
\xcd`tbl(i).meth(true, a+1)` would be treated roughly as 
\begin{xten}
val x1 = tbl(i);
val x2 = true;
val x3 = a+1;
x1.meth(x2,x3);
\end{xten}

All that matters about an invocation \xcd`I` is: 
\begin{enumerate}
\item The method name \xcdmath"m$'$";
\item The generic type parameters \xcdmath"G$'_1$,$\ldots$, G$'_g$".  If there
      are no generic type parameters, {$g=0$}.  
\item The names and types \xcdmath"x$_1$:T$'_1$,$\ldots$, x$_f$:T$'_f$" of the
      actual parameters.
      If
      there are no actual parameters, {$f=0$}. In the case of an instance
      method, the receiver is the first actual parameter.
\end{enumerate}

The signature of the method resolution procedure is: 
\xcd`resolve(invo : Invocation, context: Set[Method]) : MethodID`.  
Given a particular invocation and the set \xcd`context` of all methods
which could be called at that point of code, method resolution either returns
the unique identifier of the method that should be called, or (conceptually)
throws an exception if the call cannot be resolved.

The procedure for computing \xcd`resolve(invo, context)` is: 
\begin{enumerate}
\item Eliminate from \xcd`context` those methods which are not {\em
      acceptable}; \viz, those whose name, type parameters, formal parameters,
      and constraint do not suitably match \xcd`invo`.  In more detail:
      \begin{itemize}
      \item The method name \xcd`m` must simply equal the invocation name \xcdmath"m$'$";
      \item X10 infers type parameters, by an algorithm given in \Sref{TypeParamInfer}.
      \item The method's type parameters are bound to the invocation's for the
            remainder of the acceptability test.
      \item The actual parameter types must be subtypes of the formal
            parameter types, or be coercible to such subtypes.  Parameter $i$
            is a subtype if \xcdmath"T$'_i$ <: T$_i$".  It is implicitly
            coercible to a subtype if there is an implicit coercion operator
            defined from \xcdmath"T$'_i$" to some type \xcd`U`, and 
            \xcdmath"U <: T$_i$". \index{method resolution!implicit coercions
            and} \index{implicit coercion}\index{coercion}.  If coercions are
            used to resolve the method, they will be called on the arguments
            before the method is invoked.
            
      \item The formal constraint \xcd`c` must be satisfied in the invoking
            context. 
      \end{itemize}
\item Eliminate from \xcd`context` those methods which are not {\em
      available}; \viz, those which cannot be called due to visibility
      constraints, such as methods from other classes marked \xcd`private`.
      The remaining methods are both acceptable and available; they might be
      the one that is intended.
\item From the remaining methods, find the unique \xcd`ms` which is more specific than all the
      others, \viz, for which \xcd`specific(ms,mo) = true` for all other
      methods \xcd`mo`.
      The specificity test \xcd`specific` is given next.
      \begin{itemize}
      \item If there is a unique such \xcd`ms`, then
            \xcd`resolve(invo,context)` returns the \xcd`id` of \xcd`ms`.  
      \item If there is not a unique such \xcd`ms`, then \xcd`resolve` reports
            an error.
      \end{itemize}

\end{enumerate}

The subsidiary procedure \xcd`specific(m1, m2)` determines whether method
\xcd`m1` is equally or more specific than \xcd`m2`.  \xcd`specific` is not a
total order: is is possible for each one to be considered more specific than
the other, or either to be more specific.  \xcd`specific` is computed as: 
\begin{enumerate}
\item Construct an invocation \xcd`invo1` based on \xcd`m1`: 
      \begin{itemize}
      \item \xcd`invo1`'s method name is \xcd`m1`'s method name;
      \item \xcd`invo1`'s generic parameters are those of \xcd`m1`--- simply
            some type variables.
      \item \xcd`invo1`'s parameters are those of \xcd`m1`.
      \end{itemize}
\item If \xcd`m2` is acceptable for the invocation \xcd`invo1`,
      \xcd`specific(m1,m2)` returns true; 
\item Construct an invocation \xcd`invo2p`, which is \xcd`invo1` with the
      generic parameters erased.  Let \xcd`invo2` be \xcd`invo2p` with generic
      parameters as inferred by X10's type inference algorithm.  If type
      inference fails, \xcd`specific(m1,m2)` returns false.
\item If \xcd`m2` is acceptable for the invocation \xcd`invo2`,
      \xcd`specific(m1,m2)` returns true; 
\item Otherwise, \xcd`specific(m1,m2)` returns false.
\end{enumerate}

\subsection{Other Disambiguations}

It is possible to have a field of the same name as a method.
Indeed, it is a common pattern to have private field and a public
method of the same name to access it:
\begin{ex}
%~~gen ^^^ MethodResolution_disamb_a
%package MethodResolution_disamb_a;
%~~vis
\begin{xten}
class Xhaver {
  private var x: Int = 0;
  public def x() = x;
  public def bumpX() { x ++; }
}
\end{xten}
%~~siv
%
%~~neg
\end{ex}

\begin{ex}
However, this can lead to syntactic ambiguity in the case where the field
\Xcd{f} of object \xcd`a` is a
function, array, list, or the like, and where \xcd`a` has a method also named
\xcd`f`.  The term \Xcd{a.f(b)} could either mean ``call method \xcd`f` of \xcd`a` upon
\xcd`b`'', or ``apply the function \xcd`a.f` to argument \xcd`b`''.  

%~~gen  ^^^ MethodResolution_disamb_b
%package MethodResolution_disamb_b;
%NOCOMPILE
%~~vis
\begin{xten}
class Ambig {
  public val f : (Int)=>Int =  (x:Int) => x*x;
  public def f(y:int) = y+1;
  public def example() {
      val v = this.f(10);
      // is v 100, or 11?
  }
}
\end{xten}
%~~siv
%
%~~neg
\end{ex}

In the case where a syntactic form \xcdmath"E.m(F$_1$, $\ldots$, F$_n$)" could
be resolved as either a method call, or the application of a field \xcd`E.m`
to some arguments, it will be treated as a method call.  
The application of \xcd`E.m` to some arguments can be specified by adding
parentheses:  \xcdmath"(E.m)(F$_1$, $\ldots$, F$_n$)".

\begin{ex}

%~~gen ^^^ MethodResolution_disamb_c
%package MethodResolution_disamb_c;
%NOCOMPILE
%~~vis
\begin{xten}
class Disambig {
  public val f : (Int)=>Int =  (x:Int) => x*x;
  public def f(y:int) = y+1;
  public def example() {
      assert(  this.f(10)  == 11  );
      assert( (this.f)(10) == 100 );
  }
}
\end{xten}
%~~siv
%
%~~neg

\end{ex}


\subsection{Static Nested Classes}
\index{class!static nested}
\index{class!nested}
\index{static nested class}

One class (or struct or interface) may be nested within another.  The simplest
way to do this is as a \xcd`static` nested class. 
For most purposes, a static nested class behaves like a top-level class.
However, a static inner class has access to private static
fields and methods of its containing class.  

Nested interfaces and static structs are permitted as well.

%~~gen
% package Classes.StaticNested;
%~~vis
\begin{xten}
class Outer {
  private static val priv = 1;
  private static def special(n:Int) = n*n;
  public static class StaticNested {
     static def reveal(n:Int) = special(n) + priv;
  }
}
\end{xten}
%~~siv
%
%~~neg

\subsection{Inner Classes}
\index{class!inner}
\index{inner class}


Non-static nested classes are called {\em inner classes}. An inner class
instance can be thought of as a very elaborate member of an object --- one
with a full class structure of its own.   The crucial characteristic of an
inner class instance is that it has an implicit reference to an instance of
its containing class.  


This feature is particularly useful when an instance of the inner class makes
no sense without reference to an instance of the outer, and is closely tied to
it.  For example, consider a range class, describing a span of integers {$m$}
to {$n$}, and an iterator over the range.  The iterator might as well have
access to the range object, and there is little point to discussing
iterators-over-ranges without discussing ranges as well.
In the following example, the inner class \xcd`RangeIter` iterates over the
enclosing \xcd`Range`.  

It has its own private cursor field \xcd`n`, telling
where it is in the iteration; different iterations over the same \xcd`Range`
can exist, and will each have their own cursor.
It is perhaps unwise to use the name \xcd`n` for a field of the inner class,
since it is also a field of the outer class, but it is legal.  (It can happen
by accident as well -- \eg, if a programmer were to add a field \xcd`n` to a
superclass of the  outer class, the inner class would still work.)
It does not even
interfere with the inner class's ability to refer to the outer class's \xcd`n`
field: the cursor initialization 
refers to the \xcd`Range`'s lower bound through a fully qualified name
\xcd`Range.this.n`.
Its \xcd`hasNext()` method refers to the outer class's \xcd`m` field, which is
not shadowed.


%~~gen
% package Classes.InnerClasses.Range.Against.The.Machine;
%~~vis
\begin{xten}
class Range(m:Int, n:Int) implements Iterable[Int]{
  public def iterator ()  = new RangeIter();
  private class RangeIter implements Iterator[Int] {
     private var n : Int = m;
     public def hasNext() = n <= Range.this.n;
     public def next() = n++;
  }
  public static def main(argv:Array[String](1)) {
    val r = new Range(3,5);
    for(i in r) Console.OUT.println("i=" + i);
  }
}
\end{xten}
%~~siv
%
%~~neg

An inner class has full access to the members of its enclosing class, both
static and instance.  In particular, it can access \xcd`private` information,
just as methods of the enclosing class can.  

An inner class can have its own members.  
Inside instance methods of an inner class, \xcd`this` refers to the instance
of the {\em inner} class.  The instance of the outer class can be accessed as
{\em Outer}\xcd`.this` (where {\em Outer} is the name of the outer class).
If, for some dire reason, it is necessary to have an inner class within an inner
class, the innermost class can refer to the \xcd`this` of either outer class
by using its name.

An inner class can inherit from any class in scope,
with no special restrictions. \xcd`super` inside an inner class refers to the
inner class's superclass. If it is necessary to refer to the outer classes's
superclass, use a qualified name of the form {\em Outer}\xcd`.super`.

The only restriction placed on the members of inner classes is that static
fields of an inner class must be compile-time constant expressions. 

\index{inner class!extending}
An inner class \xcd`IC1` of some outer class \xcd`OC1` can be extended by
another class \xcd`IC2`. However, since an \xcd`IC1` only exists as a
dependent of an \xcd`OC1`, each \xcd`IC2` must be associated with an \xcd`OC1`
--- or a subtype thereof --- as well.   For example, one often extends an
inner class when one extends its outer class: 
%~~gen
% package Classes.Innerclasses.Are.For.Innermasses;
%~~vis
\begin{xten}
class OC1 {
   class IC1 {}
}
class OC2 extends OC1 {
   class IC2 extends IC1 {} 
}
\end{xten}
%~~siv
%
%~~neg

The hiding of method names has one fine point.  If an inner class defines a
method named \xcd`doit`, then {\em all} methods named \xcd`doit` from the
outer class are hidden --- even if they have different argument types than the
one defined in the inner class.
They are still accessible via
\xcd`Outer.this.doit()`, but not simply via \xcd`doit()`.  The following code
is correct, but would not be correct if the ERROR line were uncommented.

%~~gen
% package Classes.Innerclasses.StupidOverloading;
%~~vis
\begin{xten}
class Outer {
  def doit() {}
  def doit(String) {}
  class Inner { 
     def doit(Boolean, Outer) {}
     def example() {
        doit(true, Outer.this);
        Outer.this.doit();
        //ERROR: doit("fails");
     }
  }
}
\end{xten}
%~~siv
%
%~~neg


\subsubsection{Constructors and Inner Classes}
\index{inner class!constructor}

If \xcd`IC` is an inner class of \xcd`OC`, then instance code in the body of
\xcd`OC` can create instances of \xcd`IC` simply by calling a constructor
\xcd`new IC(...)`: 
%~~gen
% package Classes.Innerclasses.Constructors.Easy;
%~~vis
\begin{xten}
class OC {
  class IC {}
  def method(){
    val ic = new IC();
  }
}
\end{xten}
%~~siv
%
%~~neg

Instances of \xcd`IC` can be constructed from elsewhere as well.  Since every
instance of \xcd`IC` is associated with an instance of \xcd`OC`, an \xcd`OC`
must be supplied to the \xcd`IC` constructor.  The syntax for doing so is: 
\xcd`oc.new IC()`.  For example: 
%~~gen
% package Classes.Innerclasses.Constructors.Whythesnorkisthissocomplicated;
%~~vis
\begin{xten}
class OC {
  class IC {}
  static val oc1 = new OC();
  static val oc2 = new OC();
  static val ic1 = oc1.new IC();
  static val ic2 = oc2.new IC();
}
class Elsewhere{
  def method(oc : OC) {
    val ic = oc.new IC();
  }
}
\end{xten}
%~~siv
%
%~~neg




\noo{Local Classes}


%% vj Thu Sep 19 21:34:13 EDT 2013
% updated for v2.4 -- no change.
\chapter{Structs}
\label{XtenStructs}
\label{StructClasses}
\label{Structs}
\index{struct}


X10 objects are a powerful general-purpose programming tool. However, the
power must be paid for in space and time. In space, a typical object
implementation requires some extra memory for run-time class information, as
well as a pointer for each reference to the object. In time, a typical object
requires an extra indirection to read or write data, and some run-time
computation to figure out which method body to call.

For high-performance computing, this overhead may not be acceptable for all
objects. X10 provides structs, which are stripped-down objects. They are less
powerful than objects; in particular they lack inheritance and mutable fields.
Without inheritance, method calls do not need to do any lookup; they can be
implemented directly. Accordingly, structs can be implemented and used more
cheaply than objects, potentially avoiding the space and time overhead.
(Currently, the C++ back end avoids the overhead, but the Java back end
implements structs as Java objects and does not avoid it.)



Structs and classes are interoperable. Both can implement interfaces; in
particular, like all X10 values they implement \xcd`Any`.  Subroutines 
whose arguments are defined by interfaces can take both structs and classes.
(Some caution is necessary here: referring to a struct through an interface
requires overhead similar to that required for an object.)



In many cases structs can be converted to classes or classes to structs,
within the constraints of structs. If you start off defining a struct and
decide you need a class instead, the code change required is simply changing
the keyword \xcd`struct` to \xcd`class`. If you have a class that does not use
inheritance or mutable fields, it can be converted to a struct by changing its
keyword. Client code using the struct that was a class will need certain
changes: \eg, the \xcd`new` keyword must be added in constructor calls, and
structs (unlike classes) cannot be \xcd`null`.    





\section{Struct declaration}
\index{struct!declaration}

%##(StructDecln TypeParamsI Properties Guard Interfaces ClassBody
\begin{bbgrammar}
%(FROM #(prod:StructDecln)#)
         StructDecln \: Mods\opt \xcd"struct" Id TypeParamsI\opt Properties\opt Guard\opt Interfaces\opt ClassBody & (\ref{prod:StructDecln}) \\
%(FROM #(prod:TypeParamsI)#)
         TypeParamsI \: \xcd"[" TypeParamIList \xcd"]" & (\ref{prod:TypeParamsI}) \\
%(FROM #(prod:Properties)#)
          Properties \: \xcd"(" PropList \xcd")" & (\ref{prod:Properties}) \\
%(FROM #(prod:Guard)#)
               Guard \: DepParams & (\ref{prod:Guard}) \\
%(FROM #(prod:Interfaces)#)
          Interfaces \: \xcd"implements" InterfaceTypeList & (\ref{prod:Interfaces}) \\
%(FROM #(prod:ClassBody)#)
           ClassBody \: \xcd"{" ClassMemberDeclns\opt \xcd"}" & (\ref{prod:ClassBody}) \\
\end{bbgrammar}
%##)



All fields of a struct must be \xcd`val`.

A struct \Xcd{S} cannot contain a field of type \Xcd{S}, or a field of struct
type \Xcd{T} which, recursively, contains a field of type \Xcd{S}.  This
restriction is necessary to permit \xcd`S` to be implemented as a contiguous
block of memory of size equal to the sum of the sizes of its fields.  


Values of a struct \Xcd{C} type can be created by invoking a constructor
defined in \Xcd{C}.  Unlike for classes, the  \Xcd{new} keyword is optional
for struct constructors.  

\begin{ex}
Leaving out \xcd`new` can improve readability in some cases: 
%~~gen ^^^ Structs10
% package Structs.For.Muckts;
%~~vis
\begin{xten}
struct Polar(r:Double, theta:Double){
  def this(r:Double, theta:Double) {property(r,theta);}
  static val Origin = Polar(0,0);
  static val x0y1   = Polar(1, 3.14159/2);
  static val x1y0   = new Polar(1, 0); 
}
\end{xten}
%~~siv
%
%~~neg


When a struct and a method have the same name (often in violation of the X10
capitalization convention), 
\xcd`new` may be used to resolve to the struct's constructor.  
%~~gen ^^^ Structs2w3o
% package Structs2w3o;
%~~vis
\begin{xten}
struct Ambig(x:Long) {
  static def Ambig(x:Long) = "ambiguity please";
  static def example() { 
    val useMethod      = Ambig(1);
    val useConstructor = new Ambig(2);
  }
}
\end{xten}
%~~siv
%
%~~neg

\end{ex}

Structs support the same notions of generics, properties, and constrained
types that classes do.  

\begin{ex}

%~~gen ^^^ Structs6i5t
% package Structs6i5t;
%~~vis
\begin{xten}
struct Exam[T](nQuestions:Long){T <: Question} {
  public static interface Question {}
  // ... 
}
\end{xten}
%~~siv
%
%~~neg


\end{ex}

%%NOW_GONE%% \begin{ex}The \xcd`Pair` type below provides pairs
%%NOW_GONE%% of values; the \xcd`diag()` method applies only when the two elements of the
%%NOW_GONE%% pair are equal, and returns that common value: 
%%NOW_GONE%% %~x~gen ^^^ Structs20
%%NOW_GONE%% % package Structs20;
%%NOW_GONE%% %~x~vis
%%NOW_GONE%% \begin{xten}
%%NOW_GONE%% struct Pair[T,U](t:T, u:U) {
%%NOW_GONE%%   def this(t:T, u:U) { property(t,u); }
%%NOW_GONE%%   def diag(){T==U && t==u} = t;
%%NOW_GONE%% }
%%NOW_GONE%% \end{xten}
%%NOW_GONE%% %~x~siv
%%NOW_GONE%% % class Hook{ def run() {
%%NOW_GONE%% %   val p = Pair(3,3); 
%%NOW_GONE%% %   return p.diag() == 3;
%%NOW_GONE%% % }}
%%NOW_GONE%% %~x~neg
%%NOW_GONE%% \end{ex}

\section{Boxing of structs}
\index{auto-boxing!struct to interface}
\index{struct!auto-boxing}
\index{struct!casting to interface}
\label{auto-boxing} 
If a struct \Xcd{S} implements an interface \Xcd{I} (\eg, \Xcd{Any}),
a value \Xcd{v} of type \Xcd{S} can be assigned to a variable of type
\Xcd{I}. The implementation creates an object \Xcd{o} that is an
instance of an anonymous class implementing \Xcd{I} and containing
\Xcd{v}.  The result of invoking a method of \Xcd{I} on \Xcd{o} is the
same as invoking it on \Xcd{v}. This operation is termed {\em auto-boxing}.
It allows full interoperability of structs and objects---at the cost of losing
the extra efficiency of the structs when they are boxed.

In a generic class or struct obtained by instantiating a type parameter
\Xcd{T} with a struct \Xcd{S}, variables declared at type \Xcd{T} in the body
of the class are not boxed. They are implemented as if they were declared at
type \Xcd{S}.

\begin{ex}
The rail \xcd`aa` in the following example is a \xcd`Rail[Any]`.  It
initially holds two objects.  Then, its elements are replaced by two structs,
both of which are auto-boxed.  Note that no fussing is required to put an
integer into a \xcd`Rail[Any]`.  
However, a rail of structs, such as \xcd`ah`, holds {\em unboxed} structs
and does not incur boxing overhead.
%~~gen ^^^ Structs3q6l
% package Structs3q6l;
%~~vis
\begin{xten}
struct Horse(x:Long){
  static def example(){
    val aa : Rail[Any] = ["a String" as Any, "another one"];
    aa(0) = Horse(8);
    aa(1) = 13;
    val ah : Rail[Horse] = [Horse(7), Horse(13)];
  }
}
\end{xten}
%~~siv
%
%~~neg


\end{ex}

\section{Optional Implementation of {\tt Any} methods}
\label{StructAnyMethods}
\index{Any!structs}

Two
structs are equal (\Xcd{==}) if and only if their corresponding fields
are equal (\Xcd{==}). 

All structs implement \Xcd{x10.lang.Any}. 
Structs are required to implement the following methods from \xcd`Any`.  
Programmers need not provide them; X10 will produce them automatically if 
the program does not include them. 
\begin{xten}
  public def equals(Any):Boolean;
  public def hashCode():Int;
  public def typeName():String;
  public def toString():String;  
\end{xten}


A programmer who provides an explicit implementation
of \Xcd{equals(Any)} for a struct \Xcd{S} should also consider
supplying a definition for \Xcd{equals(S):Boolean}. This will often
yield better performance since the cost of an upcast to \Xcd{Any} and
then a downcast to \Xcd{S} can be avoided.

\section{Primitive Types}
\index{types!primitive}
\index{primitive types}
\index{Int}
\index{UInt}
\index{Long}
\index{ULong}
\index{Char}
\index{Byte}
\index{UByte}
\index{Boolean}
\index{Short}
\index{UShort}
\index{Float}
\index{Double}

Certain types that might be built in to other languages are in fact
implemented as structs in package \xcd`x10.lang` in X10. Their methods and
operations are often provided with \xcd`@Native` (\Sref{NativeCode}) rather
than X10 code, however. These types are:
\begin{xten}
Boolean, Char, Byte, Short, Int, Long
Float, Double, UByte, UShort, UInt, ULong
\end{xten}

\subsection{Signed and Unsigned Integers}
\index{types!unsigned}
\index{integers!unsigned}
\index{unsigned}

X10 has an unsigned integer type corresponding to each integer type:
\xcd`UInt` is an unsigned \xcd`Int`, and so on. These types can be used for
binary programming, or when an extra bit of precision for counters or other
non-negative numbers is needed in integer arithmetic. However, X10 does not
otherwise encourage the use of unsigned arithmetic.




 
%%WRONG%% \section{Generic programming with structs}
%%WRONG%% \index{struct!generic}
%%WRONG%% \index{generic!struct}
%%WRONG%% 
%%WRONG%% 
%%WRONG%% 
%%WRONG%% The programmer must be aware of the different interpretations of
%%WRONG%% equality (\Sref{StableEquality}) for structs and classes and ensure that the
%%WRONG%% code is correctly written for both cases. 
%%WRONG%% 
%%WRONG%% \index{isObject}
%%WRONG%% \index{isStruct}
%%WRONG%% \index{isFunction}
%%WRONG%% Three static methods on \xcd`x10.lang.System` 
%%WRONG%% allow you to tell what kind of value \xcd`x` is: object,
%%WRONG%% struct, or function.  
%%WRONG%% \xcd`System.isObject(x)` returns true if \xcd`x` is a value of \xcd`Object`
%%WRONG%% type, including \xcd`null`; \xcd`System.isStruct(x)` returns true if \xcd`x`
%%WRONG%% is a \xcd`struct`; \xcd`System.isFunction(x)` returns true if \xcd`x` is a
%%WRONG%% closure value.  Precisely one of these three functions returns true for any
%%WRONG%% X10 value \xcd`x`.  
%%WRONG%% 
%%WRONG%% \begin{xten}
%%WRONG%% val x:X = ...;
%%WRONG%% if (System.isObject(x)) { // x is a real object
%%WRONG%%    val x2 = x as Object; // this cast will always succeed.
%%WRONG%%    ...
%%WRONG%% } else if (System.isStruct(x)) { // x is a struct
%%WRONG%%    ...
%%WRONG%% } else {  
%%WRONG%%   assert System.isFunction(x);
%%WRONG%% }
%%WRONG%% \end{xten}
%%WRONG%%  
  
\section{Example structs}

\xcd`x10.lang.Complex` provides a detailed example of a practical struct,
suitable for use in a library.  For a shorter example, we define the
\xcd`Pair` struct.   A \xcd`Pair` packages
two values of possibly unrelated type together in a single value, \eg, to
return two values from a function.  

\xcd`divmod` computes the quotient and remainder of \xcdmath"a $\div$ b" (naively).
It returns both, packaged as a \xcd`Pair[UInt, UInt]`.  Note that the
constructor uses type inference, and that the quotient and remainder are
accessed through the \xcd`first` and \xcd`second` fields.
%~~gen ^^^ Structs30
% package Structs30Pair;
%~~vis
\begin{xten}
struct Pair[T,U] {
    public val first:T;
    public val second:U;
    public def this(first:T, second:U):Pair[T,U] {
        this.first = first;
        this.second = second;
    }
    public def toString() 
        = "(" + first + ", " + second + ")";
}
class Example {
  static def divmod(var a:UInt, b:UInt): Pair[UInt, UInt] {
     assert b > 0u;
     var q : UInt = 0un;
     while (a > b) {q += 1un; a -= b;}
     return Pair(q, a); 
  }
  static def example() {
     val qr = divmod(22un, 7un);
     assert qr.first == 3un && qr.second == 1un;
  }
}
\end{xten}
%~~siv
%class Hook{ def run() { Example.example(); return true; } } 
%~~neg

\section{Nested Structs}
\index{struct!static nested}
\index{static nested struct}

Static nested structs may be defined, essentially as static nested classes
except for making them structs
(\Sref{StaticNestedClasses}).  Inner structs may be defined, essentially as
inner classes except making them structs (\Sref{InnerClasses}).
\limitationx{} Nested structs must be currently be declared static.

\section{Default Values of Structs}
\label{sect:DefaultValuesOfStructs}


If all fields of a struct have default values, then the struct has a
default value, \viz, the struct whose fields are all set to their
default values.  If some field does not have a default value, neither
does the struct.

\begin{ex}

In the following code, the \xcd`Example` struct has a default value whose
\xcd`i` field is \xcd`0`.  If an \xcd`Example` is ever constructed by the
constructor, its \xcd`i` field will be \xcd`1`.  This program does a slightly
subtle dance to get ahold of a default \xcd`Example`, by having an instance
\xcd`var` (which, unlike most kinds of variables, does not need to get
initialized before use (though that exemption only applies if its type has a
default value)).   As the \xcd`assert` confirms, the default \xcd`Example`
does indeed have an \xcd`i` field of \xcd`0`.

%~~gen ^^^ Structs6r3w
% package Structs6r3w;
% 
%~~vis
\begin{xten}
class StructDefault {
  static struct Example {
    val i : Long;
    def this() { i = 1; }
  }
  var ex : Example; 
  static def example() {
     val ex = (new StructDefault()).ex;
     assert ex.i == 0;
  }
\end{xten}
%~~siv
% }
%  class Hook { def run() { StructDefault.example(); return true; } } 
%~~neg


\end{ex}


\section{Converting Between Classes And Structs}

Code written using structs can be modified to use classes, or vice versa.
Caution must be used in certain places. 

Class and struct {\em definitions} are syntactically nearly identical:
change the \xcd`class` keyword to \xcd`struct` or vice versa.  Of course,
certain important class features can't be used with structs, such as
inheritance and \xcd`var` fields. 

Converting code that {\em uses} the class or struct requires a certain amount
of caution.
Suppose, in particular, that we want to convert the class \xcd`Class2Struct`
to a struct, and \xcd`Struct2Class` to a class.
%~~gen ^^^ Structs40
%package Structs.Converting;
%~~vis
\begin{xten}
class Class2Struct {
  val a : Long;
  def this(a:Long) { this.a = a; }
  def m() = a;
}
struct Struct2Class { 
  val a : Long;
  def this(a:Long) { this.a = a; }
  def m() = a;
}
\end{xten}
%~~siv
%
%~~neg

\begin{enumerate}

\item Class constructors require the \xcd`new` keyword; struct constructors
      allow  it but do not require it.  
      \xcd`Struct2Class(3)` to will need to be converted to 
      \xcd`new Struct2Class(3)`.

\item Objects and structs have different notions of \xcd`==`.  
      For objects, \xcd`==` means ``same object''; for structs, it means
      ``same contents''. Before conversion, both \xcd`assert`s in the
      following program succeed.  After converting and fixing constructors,
      both of them fail.
%~~gen ^^^ Structs50
%package Structs.Converting.And.Conniving;
% class Class2Struct {
%   val a : Long;
%   def this(a:Long) { this.a = a; }
%   def m() = a;
% }
% struct Struct2Class { 
%   val a : Long;
%   def this(a:Long) { this.a = a; }
%   def m() = a;
% }
%class Example {
% static def Examplle() {
%~~vis
\begin{xten}
val a = new Class2Struct(2);
val b = new Class2Struct(2);
assert a != b;
val c = Struct2Class(3);
val d = Struct2Class(3);
assert c==d;
\end{xten}
%~~siv
%}}
%~~neg

\item Objects can be set to \xcd`null`.  Structs cannot.  

\item The rules for default values are quite different.  
The default value of an object type (if it exists) is \xcd`null`, which behaves quite
differently from an ordinary object of that type; \eg, you cannot call methods
on \xcd`null`, whereas you can on an ordinary object. The default value for
a struct type (if it exists) is a struct like any other of its type, and you
can call methods on it as for any other.


\end{enumerate}



\chapter{Functions}
\label{Functions}
\label{functions}
\index{functions}
\label{Closures}

\section{Overview}
Functions, the last of the three kinds of values in X10, encapsulate pieces of
code which can be applied to a vector of arguments to produce a value.
Functions, when applied, can do nearly anything that any other code could do:
fail to terminate, throw an exception, modify variables, spawn activities,
execute in several places, and so on. X10 functions are not mathematical
functions: the \xcd`f(1)` may return \xcd`true` on one call and \xcd`false` on
an immediately following call.

It is a limitation of \XtenCurrVer{} that functions do not support
type arguments. This limitation may be removed in future versions of
the language.

A \emph{function literal} \xcd"(x1:T1,..,xn:Tn){c}:T=>e" creates a function of
type\\ \xcd"(x1:T1,...,xn:Tn){c}=>T" (\Sref{FunctionType}).  For example, 
\xcd`(x:Int) => x*x` is a function literal describing the squaring function on
integers.   
\xcd`null` is also a function value.

\limitationx{} Function literals do not currently support guards. 

Function application is written \xcd`f(a,b,c)`, following common mathematical
usage. 
\index{Exception!unchecked}


The function body may be a block.  To compute integer squares by repeated
addition (inefficiently), one may write: 
%~~gen
% package Functions.Are.For.Spunctions;
% class Examplllll {
% static 
%~~vis
\begin{xten}
val sq: (Int) => Int 
      = (n:Int) => {
           var s : Int = 0;
           val abs_n = n < 0 ? -n : n;
           for ([i] in 1..abs_n) s += abs_n;
           s
        };
\end{xten}
%~~siv
%}
%~~neg




A function literal evaluates to a function entity {$\phi$}. When {$\phi$} is
applied to a suitable list of actual parameters \xcd`a1`-\xcd`an`, it
evaluates \xcd`e` with the formal parameters bound to the actual parameters.
So, the following are equivalent, where \xcd`e` is an expression involving
\xcd`x1` and \xcd`x2`\footnote{Strictly, there are a few other requirements;
  \eg, \xcd`result` must be a \xcd`var` of type \xcd`T` defined outside the
  outer block, the variables \xcd`a1` and \xcd`a2` had better not appear in
  \xcd`e`, and everything in sight had better typecheck properly.}

%~~gen
% package functions2.why.is.there.a.two;
% abstract class FunctionsTooManyFlippingFunctions[T, T1, T2]{
% abstract def arg1():T1;
% abstract def arg2():T2;
% def thing1(e:T) {var result:T;
%~~vis
\begin{xten}
{
  val f = (x1:T1,x2:T2){true}:T => e;
  val a1 : T1 = arg1();
  val a2 : T2 = arg2();
  result = f(a1,a2);
}
\end{xten}
%~~siv
%}}
%~~neg
and 
%~~gen
% package functions2.why.is.there.a.two.but.here.is.the.other.one;
% abstract class FunctionsTooManyFlippingFunctions[T, T1, T2]{
% abstract def arg1():T1;
% abstract def arg2():T2;
% def thing1(e:T) {var result:T;
%~~vis
\begin{xten}
{
  val a1 : T1 = arg1();
  val a2 : T2 = arg2();
  {
     val x1 : T1 = a1;
     val x2 : T2 = a2;
     result = e;
  }  
}
\end{xten}
%~~siv
%}}
%~~neg
\noindent
This doesn't quite work if the body is a statement rather than an expression.
A few language features are forbidden (\xcd`break` or \xcd`continue` of a loop
that surrounds the function literal) or mean something different (\xcd`return`
inside a function returns from the function). 





The \emph{method selector expression} \Xcd{e.m.(x1:T1,...,xn:Tn)} (\Sref{MethodSelectors})
permits the specification of the function underlying
the method \Xcd{m}, which takes arguments of type \Xcd{(x1:T1,..., xn:Tn)}.
Within this function, \Xcd{this} is bound to the result of evaluating \Xcd{e}.

Function types may be used in \Xcd{implements} clauses of class
definitions. Instances of such classes may be used as functions of the
given type.  Indeed, an object may behave like any (fixed) number of
functions, since the class it is an instance of may implement any
(fixed) number of function types.

%\section{Implementation Notes}
%\begin{itemize}
%
%\item Note that e.m.(T1,...,Tn) will evaluate e to create a
%  function. This function will be applied later to given
%  arguments. Thus this syntax can be used to evaluate the receiver of
%  a method call ahead of the actual invocation. The resulting function
%  can be used multiple times, of course.
%\end{itemize}


\section{Function Literals}
\index{literal!function}
\label{FunctionLiteral}

\Xten{} provides first-class, typed functions, including
\emph{closures}, \emph{operator functions}, and \emph{method
  selectors}.

\begin{grammar}
ClosureExpression \:
        \xcd"("
        Formals\opt
        \xcd")"
\\ &&
        Guard\opt
        ReturnType\opt
        \xcd"=>" ClosureBody \\
ClosureBody \:
        Expression \\
        \| \xcd"{" Statement\star \xcd"}" \\
        \| \xcd"{" Statement\star Expression \xcd"}" \\
\end{grammar}

Functions have zero or more formal parameters and an optional return type.
The body has the 
same syntax as a method body; it may be either an expression, a block
of statements, or a block terminated by an expression to return. In
particular, a value may be returned from the body of the function
using a return statement (\Sref{ReturnStatement}). 

The type of a
function is a function type (\Sref{FunctionType}).  In some cases the
return type \Xcd{T} is also optional and defaults to the type of the
body. If a formal \Xcd{xi} does not occur in any
\Xcd{Tj}, \Xcd{c}, \Xcd{T} or \Xcd{e}, the declaration \Xcd{xi:Ti} may
be replaced by just \Xcd{Ti}: \xcd`(Int)=>7` is the integer function returning
7 for all inputs.

\label{ClosureGuard}

As with methods, a function may declare a guard to
constrain the actual parameters with which it may be invoked.
The guard may refer to the type parameters, formal parameters,
and any \xcd`val`s in scope at the function expression.

The body of the function is evaluated when the function is
invoked by a call expression (\Sref{Call}), not at the function's
place in the program text.

As with methods, a function with return type \xcd"void" cannot
have a terminating expression. 
If the return type is omitted, it is inferred, as described in
\Sref{TypeInference}.
It is a static error if the return type cannot be inferred.  \Eg,
\xcd`(Int)=>null` is not well-defined; X10 does not know which type of
\xcd`null` is intended.  
%~~exp~~`~~`~~ ~~
But \xcd`(Int):Array[Double](1) => null` is legal.


\begin{example}
The following method takes a function parameter and uses it to
test each element of the list, returning the first matching
element.  It returns \xcd`absent` if no element matches.

%~~gen
% package functions2.oh.no;
% import x10.util.*;
% class Finder {
% static 
%~~vis
\begin{xten}

def find[T](f: (T) => Boolean, xs: List[T], absent:T): T = {
  for (x: T in xs)
    if (f(x)) return x;
  absent
  }
\end{xten}
%~~siv
% }
%~~neg

The method may be invoked thus:
%~~gen
% package functions2.oh.no.my.ears;
% import x10.util.*;
% class Finderator {
% static def find[T](f: (T) => Boolean, xs: x10.util.List[T], absent:T): T = {
%  for (x: T in xs)
%    if (f(x)) return x;
%  absent
%}
% static def checkery() {
%~~vis
\begin{xten}
xs: List[Int] = new ArrayList[Int]();
x: Int = find((x: Int) => x>0, xs, 0);
\end{xten}
%~~siv
%}}
%~~neg

\end{example}



\subsection{Outer variable access}

In a function
\xcdmath"(x$_1$: T$_1$, $\dots$, x$_n$: T$_n$){c} => { s }"
the types \xcdmath"T$_i$", the guard \xcd"c" and the body \xcd"s"
may access many, though not all, sorts of variables from outer scopes.  
Specifically, they can access: 
\begin{itemize}
\item All fields of the enclosing object and class;
\item All type parameters;
\item All \xcd`val` variables;
\end{itemize}
\noindent
\xcd`var` variables cannot be accessed.


The function body may refer to instances of enclosing classes using
the syntax \xcd"C.this", where \xcd"C" is the name of the
enclosing class.  \xcd`this` refers to the instance of the immediately
enclosing class, as usual.

For example, the following is legal.  
However, the commented-out line would not be legal.
Note that \xcd`a` is not a local \xcd`var` variable. It is a field of
\xcd`this`. A reference to \xcd`a` is simply short for \xcd`this.a`, which is
a use of a \xcd`val` variable (\xcd`this`).  
%~~gen
% package Functions.areLikeGrunctions.fromConjunctionJunctions;
%~~vis
\begin{xten}
class Lambda {
   var a : Int = 0;
   val b = 0;
   def m(var c : Int, val d : Int) {
      var e : Int = 0;
      val f : Int = 0;
      val closure = (var i: Int, val j: Int) => {
    	  return a + b + d + f + j + this.a + Lambda.this.a;
          // ILLEGAL: return c + e + i;
      };
      return closure;
   }
}
\end{xten}
%~~siv
%
%~~neg

%%SHARED%% 
%%SHARED%% 
%%SHARED%% Access to variables is not automatically atomic.  As
%%SHARED%% with any code that might mutate shared data concurrently, be sure to protect
%%SHARED%% references to mutable shared state with \xcd`atomic`. For example, the
%%SHARED%% following code returns a pair of closures which operate on the same shared
%%SHARED%% variable \xcd`a`, which are concurrency-safe---even if invoked many times
%%SHARED%% simultaneously. Without \xcd`atomic`, it would no longer be concurrency-safe.
%%SHARED%% 
%%SHARED%% 
%%SHARED%% %~s~gen
%%SHARED%% % package Functions2.Are.All.Too.Much;
%%SHARED%% % class Fun2Frivols {
%%SHARED%% %~s~vis
%%SHARED%% \begin{xten}
%%SHARED%%   def counters() {
%%SHARED%%       var a : Int = 0;
%%SHARED%%        return [
%%SHARED%%           () => {atomic a ++;},
%%SHARED%%           () => {atomic return a;}
%%SHARED%%           ];
%%SHARED%%    }
%%SHARED%% \end{xten}
%%SHARED%% %~s~siv
%%SHARED%% %}
%%SHARED%% %
%%SHARED%% %~s~neg


%SHARED% \begin{note}
%SHARED% The main activity may run in parallel with any
%SHARED% functions it creates. Hence even the read of an outer variable by the
%SHARED% body of a function may result in a race condition. Since functions are
%SHARED% first-class, the analysis of whether a function may execute in parallel
%SHARED% with the activity that created it may be difficult.
%SHARED% \end{note}

%% vj: This should be verified.
%\begin{note}
%The rule for accessing outer variables from function bodies
%should be the same as the rule for accessing outer variables from local
%or anonymous classes.
%\end{note}

\section{Method selectors}
\label{MethodSelectors}
\index{function!method selector}
\index{method!underlying function}

A method selector expression allows a method to be used as a
first-class function, without writing a function expression for it.
For example, consider a class \xcd`Span` defining ranges of integers.  

%~~gen
% package Functions2.Span;
%~~vis
\begin{xten}
class Span(low:Int, high:Int) {
  def this(low:Int, high:Int) {property(low,high);}
  def between(n:Int) = low <= n && n <= high;
  def example() {
    val digit = new Span(0,9);
    val isDigit : (Int) => Boolean = digit.between.(Int);
    if (isDigit(8)) Console.OUT.println("8 is!");
  }
}
\end{xten}
%~~siv
%
%~~neg
\noindent


In \xcd`example()`, 
%~~exp~~`~~`~~ digit:Span~~class Span(low:Int, high:Int) {def this(low:Int, high:Int) {property(low,high);} def between(n:Int) = low <= n && n <= high;}
\xcd`digit.between.(Int)` 
is a unary function testing whether its argument is between zero
and nine.  It could also be written 
%~~exp~~`~~`~~ digit:Span~~class Span(low:Int, high:Int) {def this(low:Int, high:Int) {property(low,high);} def between(n:Int) = low <= n && n <= high;}
\xcd`(n:Int) => digit.between(n)`.

%%GRAMMAR%% This is formalized thus:
%%GRAMMAR%% 
%%GRAMMAR%% \begin{grammar}
%%GRAMMAR%% MethodSelector \:
%%GRAMMAR%%         Primary \xcd"."
%%GRAMMAR%%         MethodName \xcd"."
%%GRAMMAR%%                 TypeParameters\opt \xcd"(" Formals\opt \xcd")" \\
%%GRAMMAR%%       \|
%%GRAMMAR%%         TypeName \xcd"."
%%GRAMMAR%%         MethodName \xcd"."
%%GRAMMAR%%                 TypeParameters\opt \xcd"(" Formals\opt \xcd")" \\
%%GRAMMAR%% \end{grammar}

The \emph{method selector expression} \Xcd{e.m.(T1,...,Tn)} is type
correct only if  the static type of \Xcd{e} is a
class or struct or interface \xcd`V` with a method
\Xcd{m(x1:T1,...xn:Tn)\{c\}:T} defined on it (for some
\Xcd{x1,...,xn,c,T)}. At runtime the evaluation of this expression
evaluates \Xcd{e} to a value \Xcd{v} and creates a function \Xcd{f}
which, when applied to an argument list \Xcd{(a1,...,an)} (of the right
type) yields the value obtained by evaluating \Xcd{v.m(a1,...,an)}.

Thus, the method selector

\begin{xtenmath}
e.m.(T$_1$, $\dots$, T$_n$)
\end{xtenmath}
\noindent behaves as if it were the function
\begin{xtenmath}
((v:V)=>
  (x$_1$: T$_1$, $\dots$, x$_n$: T$_n$){c} 
  => v.m(x$_1$, $\dots$, x$_n$))
(e)
\end{xtenmath}



Because of overloading, a method name is not sufficient to
uniquely identify a function for a given class.
One needs the argument type information as well.
The selector syntax (dot) is used to distinguish \xcd"e.m()" (a
method invocation on \xcd"e" of method named \xcd"m" with no arguments)
from \xcd"e.m.()"
(the function bound to the method). 

A static method provides a binding from a name to a function that is
independent of any instance of a class; rather it is associated with the
class itself. The static function selector
\xcdmath"T.m.(T$_1$, $\dots$, T$_n$)" denotes the
function bound to the static method named \xcd"m", with argument types
\xcdmath"(T$_1$, $\dots$, T$_n$)" for the type \xcd"T". The return type
of the function is specified by the declaration of \xcd"T.m".

There is no difference between using a function defined directly 
directly using the function syntax, or obtained via static or
instance function selectors.


\section{Operator functions}
\label{OperatorFunction}
\index{function!operator}
Every binary operator (e.g.,
\xcd"+",
\xcd"-",
\xcd"*",
\xcd"/",
\dots) has a family of functions, one for
each type on which the operator is defined. The function can be
selected using the ``\xcd`.`'' syntax:


\begin{xtenmath}
String.+             $\equiv$ (x: String, y: String): String => x + y
Long.-               $\equiv$ (x: Long, y: Long): Long => x - y
Float.-              $\equiv$ (x: Float, y: Float): Float => x - y
Boolean.&            $\equiv$ (x: Boolean, y: Boolean): Boolean => x & y
Int.<                $\equiv$ (x: Int, y: Int): Boolean => x < y
\end{xtenmath}

%~~gen
% package Functions.Operatorfunctionsgracklegrackle;
% class JustATest {
% val dummy = [String.+,
%  Long.-,
%  Float.-,
%  Boolean.&,
%  Int.<
%  ];
% }
%~~vis
\begin{xten}
\end{xten}
%~~siv
%
%~~neg


%%TODO -- fix commented-out lines!

%~~gen
% package Functions2.For.The.Lose;
% class TypecheckThatSillyExample {
%   def checker() {
%    val l1 : (String, String) => String = String.+;
%    val r1 : (String, String) => String = (x: String, y: String): String => x + y;
%    val l2 : (Long,Long) => Long = Long.-;
%    val r2 : (Long,Long) => Long = (x: Long, y: Long): Long => x - y;
%//var v1 : (Float,Float) => Float = Float.-(Float,Float) ;
%var v2 : (Float,Float) => Float = (x: Float, y: Float): Float => x - y;
%//var v3 : (Int) => Int =  Int.-(Int)     ;      ;
%var v4  : (Int) => Int  =  (x: Int): Int => -x;
%var v5 : (Boolean,Boolean) => Boolean = Boolean.&            ;
%var v6 : (Boolean,Boolean) => Boolean =  (x: Boolean, y: Boolean): Boolean => x & y;
%//var v7 : (Boolean) => Boolean = Boolean.!            ;
%var v8 : (Boolean) => Boolean =  (x: Boolean): Boolean => !x;
%//var v9 : (Int,Int) => Boolean = Int.<(Int,Int)       ;
%var v10: (Int,Int) => Boolean =  (x: Int, y: Int): Boolean => x < y;
%//var v11 : (Dist,Place)=>Dist = Dist.|(Place)        ;
%var v12 : (Dist,Place)=>Dist=  (d: Dist, p: Place): Dist => d | p;
%}
% }
%~~vis
%~~siv
%
%~~neg

Unary and binary promotion (\Sref{XtenPromotions}) is not performed
when invoking these
operations; instead, the operands are coerced individually via implicit
coercions (\Sref{XtenConversions}), as appropriate.


%%WE-NEVER-GOT-TO-IT%%  \begin{planned}
%%WE-NEVER-GOT-TO-IT%%  
%%WE-NEVER-GOT-TO-IT%%  {\bf The following is not implemented in version 2.0.3:}
%%WE-NEVER-GOT-TO-IT%%  
%%WE-NEVER-GOT-TO-IT%%  Additionally, for every expression \xcd"e" of a type \xcd"T" at which a binary
%%WE-NEVER-GOT-TO-IT%%  operator \xcd"OP" is defined, the expression \xcd"e.OP" or
%%WE-NEVER-GOT-TO-IT%%  \xcd"e.OP(T)" represents the function
%%WE-NEVER-GOT-TO-IT%%  defined by:
%%WE-NEVER-GOT-TO-IT%%  
%%WE-NEVER-GOT-TO-IT%%  \begin{xten}
%%WE-NEVER-GOT-TO-IT%%  (x: T): T => { e OP x }
%%WE-NEVER-GOT-TO-IT%%  \end{xten}
%%WE-NEVER-GOT-TO-IT%%  
%%WE-NEVER-GOT-TO-IT%%  \begin{grammar}
%%WE-NEVER-GOT-TO-IT%%  Primary \: Expr \xcd"." Operator \xcd"(" Formals\opt \xcd")" \\
%%WE-NEVER-GOT-TO-IT%%          \| Expr \xcd"." Operator \\
%%WE-NEVER-GOT-TO-IT%%  \end{grammar}
%%WE-NEVER-GOT-TO-IT%%  
%%WE-NEVER-GOT-TO-IT%%  %% For every expression \xcd"e" of a type \xcd"T" at which a unary
%%WE-NEVER-GOT-TO-IT%%  %%operator \xcd"OP" is defined, the expression \xcd"e.OP()"
%%WE-NEVER-GOT-TO-IT%%  %% represents the function defined by:
%%WE-NEVER-GOT-TO-IT%%  
%%WE-NEVER-GOT-TO-IT%%  %% \begin{xten}
%%WE-NEVER-GOT-TO-IT%%  %% (): T => { OP e }
%%WE-NEVER-GOT-TO-IT%%  %% \end{xten}
%%WE-NEVER-GOT-TO-IT%%  
%%WE-NEVER-GOT-TO-IT%%  For example,
%%WE-NEVER-GOT-TO-IT%%  one may write an expression that adds one to each member of a
%%WE-NEVER-GOT-TO-IT%%  list \xcd"xs" by:
%%WE-NEVER-GOT-TO-IT%%  
%%WE-NEVER-GOT-TO-IT%%  %%TODO -- when this topic works, make the example wwork too.
%%WE-NEVER-GOT-TO-IT%%  %~x~gen
%%WE-NEVER-GOT-TO-IT%%  % package Functions2.Wants.A.Dinner.Reservation;
%%WE-NEVER-GOT-TO-IT%%  % import x10.util.*;
%%WE-NEVER-GOT-TO-IT%%  % class Reservation {
%%WE-NEVER-GOT-TO-IT%%  % def smerp() {
%%WE-NEVER-GOT-TO-IT%%  %   val xs = new ArrayList[Int]();
%%WE-NEVER-GOT-TO-IT%%  %~x~vis
%%WE-NEVER-GOT-TO-IT%%  \begin{xten}
%%WE-NEVER-GOT-TO-IT%%  xs.map(1.+);
%%WE-NEVER-GOT-TO-IT%%  \end{xten}
%%WE-NEVER-GOT-TO-IT%%  %~x~siv
%%WE-NEVER-GOT-TO-IT%%  % }
%%WE-NEVER-GOT-TO-IT%%  % }
%%WE-NEVER-GOT-TO-IT%%  %
%%WE-NEVER-GOT-TO-IT%%  %~x~neg
%%WE-NEVER-GOT-TO-IT%%  \end{planned}
%%WE-NEVER-GOT-TO-IT%%  
%%WE-NEVER-GOT-TO-IT%%  
\section{Functions as objects of type \Xcd{Any}}
\label{FunctionAnyMethods}

\label{FunctionEquality}
\index{function!equality} \index{equality!function} Two functions \Xcd{f} and
\Xcd{g} are equal if both were obtained by the same evaluation of a function
literal.\footnote{A literal may occur in program text within a loop, and hence
  may be evaluated multiple times.} Further, it is guaranteed that if two
functions are equal then they refer to the same locations in the environment
and represent the same code, so their executions in an identical situation are
indistinguishable. (Specifically, if \xcd`f == g`, then \xcd`f(1)` can be
substituted for \xcd`g(1)` and the result will be identical. However, there is
no guarantee that \xcd`f(1)==g(1)` will evaluate to true. Indeed, there is no
guarantee that \xcd`f(1)==f(1)` will evaluate to true either, as \xcd`f` might
be a function which returns {$n$} on its {$n^{th}$} invocation. However,
\xcd`f(1)==f(1)` and \xcd`f(1)==g(1)` are interchangeable.)
\index{function!==}


Every function type implements all the methods of \Xcd{Any}.
\xcd`f.equals(g)` is equivalent to \xcd`f==g`.  \xcd`f.hashCode()`, 
\xcd`f.toString()`, and \xcd`f.typeName()` are implementation-dependent, but
respect \xcd`equals` and the basic contracts of \xcd`Any`. 

\index{function!equals}
\index{function!hashCode}
\index{function!toString}
\index{function!typeName}
\index{function!home}
\index{function!at(Place)}
\index{function!at(Object)}



\chapter{Expressions}\label{XtenExpressions}\index{expression}

\Xten{} has a rich expression language.
Evaluating an expression produces a value, or, in a few cases, no value. 
Expression evaluation may have side effects, such as change of the value of a 
\xcd`var` variable or a data structure, allocation of new values, or throwing
an exception. 



\section{Literals}
\index{literal}

Literals denote fixed values of built-in types. 
The syntax for literals is given in \Sref{Literals}. 

The type that \Xten{} gives a literal often includes its value. \Eg, \xcd`1`
is of type \xcd`Int{self==1}`, and \xcd`true` is of type
\xcd`Boolean{self==true}`.

\section{{\tt this}}
\index{this}
\index{\Xcd{this}}

%##(Primary
\begin{bbgrammar}
%(FROM #(prod:Primary)#)
             Primary \: \xcd"here" & (\ref{prod:Primary}) \\
                    \| \xcd"[" ArgumentList\opt \xcd"]" \\
                    \| Literal \\
                    \| \xcd"self" \\
                    \| \xcd"this" \\
                    \| ClassName \xcd"." \xcd"this" \\
                    \| \xcd"(" Exp \xcd")" \\
                    \| ClassInstCreationExp \\
                    \| FieldAccess \\
                    \| MethodInvocation \\
                    \| MethodSelection \\
                    \| OperatorFunction \\
\end{bbgrammar}
%##)

The expression \xcd"this" is a  local \xcd`val` containing a reference
to an instance of the lexically enclosing class.
It may be used only within the body of an instance method, a
constructor, or in the initializer of a instance field -- that is, the places
where there is an instance of the class under consideration.

Within an inner class, \xcd"this" may be qualified with the
name of a lexically enclosing class.  In this case, it
represents an instance of that enclosing class.  
For example, \xcd`Outer` is a class containing \xcd`Inner`.  Each instance of
\xcd`Inner` has a reference \xcd`Outer.this` to the \xcd`Outer` involved in its
creation.  \xcd`Inner` has access to the fields of \xcd`Outer.this`, as seen
in the \xcd`outerThree` and \xcd`alwaysTrue` methods.  Note that \xcd`Inner`
has its own \xcd`three` field, which is different from and not even the same
type as \xcd`Outer.this.three`. 
%~~gen ^^^ Expressions10
% package exp.vexp.pexp.lexp.shexp; 
% NOTEST
%~~vis 
\begin{xten}
class Outer {
  val three = 3;
  class Inner {
     val three = "THREE";
     def outerThree() = Outer.this.three;
     def alwaysTrue() = outerThree() == 3;
  }
}
\end{xten}
%~~siv
%
%~~neg

The type of a \xcd"this" expression is the
innermost enclosing class, or the qualifying class,
constrained by the class invariant and the
method guard, if any.

The \xcd"this" expression may also be used within constraints in
a class or interface header (the class invariant and
\xcd"extends" and \xcd"implements" clauses).  Here, the type of
\xcd"this" is restricted so that only properties declared in the
class header itself, and specifically not any members declared in the class
body or in supertypes, are accessible through \xcd"this".

\section{Local variables}

%##(Id
\begin{bbgrammar}
%(FROM #(prod:Id)#)
                  Id \: identifier & (\ref{prod:Id}) \\
\end{bbgrammar}
%##)

A local variable expression consists simply of the name of the local variable,
field of the current object, formal parameter in scope, etc. It evaluates to
the value of the local variable. \xcd`n` in the second line below is a local
variable expression. 
%~~gen  ^^^ Expressions20
% package exp.loc.al.varia.ble; 
% class Example {
% def example() { 
%~~vis
\begin{xten}
val n = 22;
val m = n + 56;
\end{xten}
%~~siv
%} }
%~~neg



\section{Field access}
\label{FieldAccess}
\index{field!access to}

%##(FieldAccess
\begin{bbgrammar}
%(FROM #(prod:FieldAccess)#)
         FieldAccess \: Primary \xcd"." Id & (\ref{prod:FieldAccess}) \\
                    \| \xcd"super" \xcd"." Id \\
                    \| ClassName \xcd"." \xcd"super"  \xcd"." Id \\
                    \| Primary \xcd"." \xcd"class"  \\
                    \| \xcd"super" \xcd"." \xcd"class"  \\
                    \| ClassName \xcd"." \xcd"super"  \xcd"." \xcd"class"  \\
\end{bbgrammar}
%##)

A field of an object instance may be  accessed
with a field access expression.

The type of the access is the declared type of the field with the
actual target substituted for \xcd"this" in the type. 
% If the actual
%target is not a final access path (\Sref{FinalAccessPath}),
%an anonymous path is substituted for \xcd"this".

The field accessed is selected from the fields and value properties
of the static type of the target and its superclasses.

If the field target is given by the keyword \xcd"super", the target's type is
the superclass of the enclosing class.  This form is used to access fields of
the parent class shadowed by same-named fields of the current class.

If the field target is \xcd`Cls.super`, then the target's type is \xcd`Cls`,
which must be an  enclosing class.  This (admittedly
obscure) form is used to access fields of an ancestor class which are shadowed
by same-named fields of some more recent ancestor.  The following example
illustrates all four cases:

%~~gen ^^^ Expressions30
% package exp.re.ssio.ns.fiel.dacc.ess;
% NOTEST
%~~vis
\begin{xten}
class Uncle {
  public static val f = 1;
}
class Parent {
  public val f = 2;
}
class Ego extends Parent {
  public val f = 3;
  class Child extends Ego {
     public val f = 4;
     def expDotId() = this.f; // 4
     def superDotId() = super.f; // 3
     def classNameDotId() = Uncle.f; // 1;
     def cnDotSuperDotId() = Ego.super.f; // 2
  }
}
\end{xten}
%~~siv
%
%~~neg


If the field target is \xcd"null", a \xcd"NullPointerException"
is thrown.

If the field target is a class name, a static field is selected.

It is illegal to access  a field that is not visible from
the current context.
It is illegal to access a non-static field
through a static field access expression.

\section{Function Literals}
Function literals are described in \Sref{Functions}.

\section{Calls}
\label{Call}
\label{MethodInvocation}
\label{MethodInvocationSubstitution}
\index{invocation}
\index{call}
\index{invocation!method}
\index{call!method}
\index{invocation!function}
\index{call!function}
\index{method!calling}
\index{method!invoking}


%##(MethodInvocation ArgumentList
\begin{bbgrammar}
%(FROM #(prod:MethodInvocation)#)
    MethodInvocation \: MethodPrimaryPrefix \xcd"(" ArgumentList\opt \xcd")" & (\ref{prod:MethodInvocation}) \\
                    \| MethodSuperPrefix \xcd"(" ArgumentList\opt \xcd")" \\
                    \| MethodClassNameSuperPrefix \xcd"(" ArgumentList\opt \xcd")" \\
                    \| MethodName TypeArguments\opt \xcd"(" ArgumentList\opt \xcd")" \\
                    \| Primary \xcd"." Id TypeArguments\opt \xcd"(" ArgumentList\opt \xcd")" \\
                    \| \xcd"super" \xcd"." Id TypeArguments\opt \xcd"(" ArgumentList\opt \xcd")" \\
                    \| ClassName \xcd"." \xcd"super"  \xcd"." Id TypeArguments\opt \xcd"(" ArgumentList\opt \xcd")" \\
                    \| Primary TypeArguments\opt \xcd"(" ArgumentList\opt \xcd")" \\
%(FROM #(prod:ArgumentList)#)
        ArgumentList \: Exp & (\ref{prod:ArgumentList}) \\
                    \| ArgumentList \xcd"," Exp \\
\end{bbgrammar}
%##)


A \grammarrule{MethodInvocation} may be to either a \xcd"static" method, an
instance method, or a closure.

The syntax is ambiguous; the target must be type-checked to determine if it is
the name of a method or if it refers to a field containing a closure. It is a
static error if a call may resolve to both a closure call or to a method call.
%~~gen ^^^ Expressions40
% package expres.sio.nsca.lls;
%~~vis
\begin{xten}
class Callsome {
  static val closure = () => 1;
  static def method () = 2;
  static val closureEvaluated = Callsome.closure();
  static val methodEvaluated = Callsome.method();
}
\end{xten}
%~~siv
%
%~~neg
However, adding a static method called \xcd`closure` makes \xcd`Callsome.closure()`
ambiguous: it could be a call to the closure, or to the static method: 

%~~gen ^^^ Expressions50
% package expres.sio.nsca.lls.twoooo;
% class Callsome {static val closure = () => 1; static def method () = 2; static val methodEvaluated = Callsome.method();
%~~vis
\begin{xten}
  static def closure () = 3;
  // ERROR: static errory = Callsome.closure();
\end{xten}
%~~siv
% }
%~~neg

A closure call \xcdmath"e($\dots$)" is shorthand for a method call
\xcdmath"e.apply($\dots$)". 

Method selection rules are given in \Sref{sect:MethodResolution}.

It is a static error if a method's \grammarrule{Guard} is not satisfied by the
caller.  For example: 
%~~gen ^^^ Expressions60
%package Expressions.Calls.Guarded.By.Walls;
%~~vis
\begin{xten}
class DivideBy(denom:Int) {
  def doIt(numer:Int){denom != 0} = numer / denom;
  def example() {
     //ERROR: denom might be zero: this.doIt(100); 
     (this as DivideBy{self.denom != 0}).doIt(100);
  }
}
\end{xten}
%~~siv
%
%~~neg


\section{Assignment}\index{assignment}\label{AssignmentStatement}

%##(Assignment LeftHandSide AssignmentOperator
\begin{bbgrammar}
%(FROM #(prod:Assignment)#)
          Assignment \: LeftHandSide AssignmentOperator AssignmentExp & (\ref{prod:Assignment}) \\
                    \| ExpName  \xcd"(" ArgumentList\opt \xcd")" AssignmentOperator AssignmentExp \\
                    \| Primary  \xcd"(" ArgumentList\opt \xcd")" AssignmentOperator AssignmentExp \\
%(FROM #(prod:LeftHandSide)#)
        LeftHandSide \: ExpName & (\ref{prod:LeftHandSide}) \\
                    \| FieldAccess \\
%(FROM #(prod:AssignmentOperator)#)
  AssignmentOperator \: \xcd"=" & (\ref{prod:AssignmentOperator}) \\
                    \| \xcd"*=" \\
                    \| \xcd"/=" \\
                    \| \xcd"%=" \\
                    \| \xcd"+=" \\
                    \| \xcd"-=" \\
                    \| \xcd"<<=" \\
                    \| \xcd">>=" \\
                    \| \xcd">>>=" \\
                    \| \xcd"&=" \\
                    \| \xcd"^=" \\
                    \| \xcd"|=" \\
\end{bbgrammar}
%##)



The assignment expression \xcd"x = e" assigns a value given by
expression \xcd"e"
to a variable \xcd"x".  
Most often, \xcd`x` is a mutable (\xcd`var` variable).  The same syntax is
used for delayed initialization of a \xcd`val`, but \xcd`val`s can only be
initialized once.
%~~gen ^^^ Expressions70
% package express.ions.ass.ignment;
% class Example {
% static def exasmple() {
%~~vis
\begin{xten}
  var x : Int;
  val y : Int;
  x = 1;
  y = 2; // Correct; initializes y
  x = 3; 
  // Incorrect: y = 4;
\end{xten}
%~~siv
% } } 
%~~neg


There are three syntactic forms of
assignment: 
\begin{enumerate}
\item \xcd`x = e;`, assigning to a local variable, formal parameter, field of
      \xcd`this`, etc. 
\item \xcd`x.f = e;`, assigning to a field of an object.
\item \xcdmath`a(i$_1$,$\ldots$,i$_n$) = v;`, where {$n \ge 0$}, assigning to
      an element of an array or some other such structure. This is syntactic
      sugar for a method call: \xcdmath`a.set(v,i$_1$,$\ldots$,i$_n$)`.
      Naturally, it is a static error if no suitable \xcd`set` method exists
      for \xcd`a`.
\end{enumerate}

For a binary operator $\diamond$, the $\diamond$-assignment expression
\xcdmath"x $\diamond$= e" combines the current value of \xcd`x` with the value
of \xcd`e` by {$\diamond$}, and stores the result back into \xcd`x`.  
\xcd`i += 2`, for example, adds 2 to \xcd`i`. For variables and fields, 
\xcdmath"x $\diamond$= e" behaves just like \xcdmath"x = x $\diamond$ e". 

The subscripting forms of \xcdmath"a(i) $\diamond$= b" are slightly subtle.
Subexpressions of \xcd`a` and \xcd`i` are only evaluated once.  However,
\xcd`a(i)` and \xcd`a(i)=c` are each executed once---in particular, there is
one call to \xcd`a.apply(i)` and one to \xcd`a.set(i,c)`, the desugared forms
of \xcd`a(i)` and \xcd`a(i)=c`.  If subscripting is implemented strangely for
the class of \xcd`a`, the behavior is {\em not} necessarily updating a single
storage location. Specifically, \xcd`A()(I()) += B()` is tantamount to: 
%~~gen ^^^ Expressions80
% package expressions.stupid.addab;
% class Example {
% def example(A:()=>Rail[Int], I: () => Int, B: () => Int ) {
%~~vis
\begin{xten}
{
  val aa = A();  // Evaluate A() once
  val ii = I();  // Evaluate I() once
  val bb = B();  // Evaluate B() once
  val tmp = aa(ii) + bb; // read aa(ii)
  aa(ii) = tmp;  // write sum back to aa(ii)
}
\end{xten}
%~~siv
%}}
%~~neg

\limitation{+= does not currently meet this specification.}




\section{Increment and decrement}
\index{increment}
\index{decrement}
\index{\Xcd{++}}
\index{\Xcd{--}}


The operators \xcd"++" and \xcd"--" increment and decrement
a variable, respectively.  
\xcd`x++` and \xcd`++x` both increment \xcd`x`, just as the statement 
\xcd`x += 1` would, and similarly for \xcd`--`.  

The difference between the two is the return value.  
\xcd`++x` returns the {\em new} value of \xcd`x`, after incrementing.
\xcd`x++` returns the {\em old} value of \xcd`, before incrementing.`

\limitation{This currently only works for numeric types.}

\section{Numeric Operations}
\label{XtenPromotions}
\index{promotion}
\index{numeric promotion}
\index{numeric operations}
\index{operation!numeric}

Numeric types (\xcd`Byte`, \xcd`Short`, \xcd`Int`, \xcd`Long`, \xcd`Float`,
\xcd`Double`, and unsigned variants of fixed-point types) are normal X10
structs, though most of their methods are implemented via native code. They
obey the same general rules as other X10 structs. For example, numeric
operations are defined by \xcd`operator` definitions, the same way you could
for any struct.

Promoting a numeric value to a longer numeric type preserves the sign of the
value.  For example, \xcd`(255 as UByte) as UInt` is 255.  

\subsection{Conversions and coercions}

Specifically, each numeric type can be converted or coerced into each other
numeric type, perhaps with loss of accuracy.
%~~gen ^^^ Expressions90
% package exp.ress.io.ns.numeric.conversions;
% class ExampleOfConversionAndStuff {
% def example() {
%~~vis
\begin{xten}
val n : Byte = 123 as Byte; // explicit 
val f : (Int)=>Boolean = (Int) => true; 
val ok = f(n); // implicit
\end{xten}
%~~siv
% } }
%~~neg



\subsection{Unary plus and unary minus}

The unary \xcd`+` operation on numbers is an identity function.
The unary \xcd`-` operation on numbers is a negation function.
On unsigned numbers, these are two's-complement.  For example, 
\xcd`-(0x0F as UByte)` is 
\xcd`(0xF1 as UByte)`.
\bard{UInts and such are closed under negation -- the negative of a UInt is
done binarily.  }



\section{Bitwise complement}

The unary \xcd"~" operator, only defined on integral types, complements each
bit in its operand.  

\section{Binary arithmetic operations} 

The binary arithmetic operators perform the familiar binary arithmetic
operations: \xcd`+` adds, \xcd`-` subtracts, \xcd`*` multiplies, 
\xcd`/` divides, and \xcd`%`
computes remainder.

On integers, the operands are coerced to the longer of their two types, and
then operated upon.  
Floating point operations are determined by the IEEE 754
standard. 
The integer \xcd"/" and \xcd"%" throw an exception 
if the right operand is zero.



\section{Binary shift operations}

The operands of the binary shift operations must be of integral type.
The type of the result is the type of the left operand.

If the promoted type of the left operand is \xcd"Int",
the right operand is masked with \xcd"0x1f" using the bitwise
AND (\xcd"&") operator, giving a number at most the number of bits in an
\xcd`Int`. 
If the promoted type of the left operand is \xcd"Long",
the right operand is masked with \xcd"0x3f" using the bitwise
AND (\xcd"&") operator, giving a number at most the number of bits in a
\xcd`Long`. 

The \xcd"<<" operator left-shifts the left operand by the number of
bits given by the right operand.
The \xcd">>" operator right-shifts the left operand by the number of
bits given by the right operand.  The result is sign extended;
that is, if the right operand is $k$,
the most significant $k$ bits of the result are set to the most
significant bit of the operand.

The \xcd">>>" operator right-shifts the left operand by the number of
bits given by the right operand.  The result is not sign extended;
that is, if the right operand is $k$,
the most significant $k$ bits of the result are set to \xcd"0".
This operation is deprecated, and may be removed in a later version of the
language. 

\section{Binary bitwise operations}

The binary bitwise operations operate on integral types, which are promoted to
the longer of the two types.
The \xcd"&" operator  performs the bitwise AND of the promoted operands.
The \xcd"|" operator  performs the bitwise inclusive OR of the promoted operands.
The \xcd"^" operator  performs the bitwise exclusive OR of the promoted operands.

\section{String concatenation}
\index{string!concatenation}

The \xcd"+"  operator is used for string concatenation 
 as well as addition.
If either operand is of static type \xcd"x10.lang.String",
 the other operand is converted to a \xcd"String" , if needed,
  and  the two strings  are concatenated.
 String conversion of a non-\xcd"null" value is  performed by invoking the
 \xcd"toString()" method of the value.
  If the value is \xcd"null", the value is converted to 
  \xcd'"null"'.

The type of the result is \xcd"String".

 For example, 
%~~exp~~`~~`~~ ~~ ^^^ Expressions100
      \xcd`"one " + 2 + here` 
      evaluates to  \xcd`one 2(Place 0)`.  

\section{Logical negation}

The operand of the  unary \xcd"!" operator 
must be of type \xcd"x10.lang.Boolean".
The type of the result is \xcd"Boolean".
If the value of the operand is \xcd"true", the result is \xcd"false"; if
if the value of the operand  is \xcd"false", the result is \xcd"true".

\section{Boolean logical operations}

Operands of the binary boolean logical operators must be of type \xcd"Boolean".
The type of the result is \xcd"Boolean"

The \xcd"&" operator  evaluates to \xcd"true" if both of its
operands evaluate to \xcd"true"; otherwise, the operator
evaluates to \xcd"false".

The \xcd"|" operator  evaluates to \xcd"false" if both of its
operands evaluate to \xcd"false"; otherwise, the operator
evaluates to \xcd"true".

\section{Boolean conditional operations}

Operands of the binary boolean conditional operators must be of type
\xcd"Boolean". 
The type of the result is \xcd"Boolean"

The \xcd"&&" operator  evaluates to \xcd"true" if both of its
operands evaluate to \xcd"true"; otherwise, the operator
evaluates to \xcd"false".
Unlike the logical operator \xcd"&",
if the first operand is \xcd"false",
the second operand is not evaluated.

The \xcd"||" operator  evaluates to \xcd"false" if both of its
operands evaluate to \xcd"false"; otherwise, the operator
evaluates to \xcd"true".
Unlike the logical operator \xcd"||",
if the first operand is \xcd"true",
the second operand is not evaluated.

\section{Relational operations} 

The relational operations compare numbers, producing \xcd`Boolean` results.  

The \xcd"<" operator evaluates to \xcd"true" if the left operand is
less than the right.
The \xcd"<=" operator evaluates to \xcd"true" if the left operand is
less than or equal to the right.
The \xcd">" operator evaluates to \xcd"true" if the left operand is
greater than the right.
The \xcd">=" operator evaluates to \xcd"true" if the left operand is
greater than or equal to the right.

Floating point comparison is determined by the IEEE 754
standard.  Thus,
if either operand is NaN, the result is \xcd"false".
Negative zero and positive zero are considered to be equal.
All finite values are less than positive infinity and greater
than negative infinity.

\section{Conditional expressions}
\index{\Xcd{? :}}
\index{conditional expression}
\index{expression!conditional}
\label{Conditional}

%##(ConditionalExp
\begin{bbgrammar}
%(FROM #(prod:ConditionalExp)#)
      ConditionalExp \: ConditionalOrExp & (\ref{prod:ConditionalExp}) \\
                    \| ClosureExp \\
                    \| AtExp \\
                    \| FinishExp \\
                    \| ConditionalOrExp \xcd"?" Exp \xcd":" ConditionalExp \\
\end{bbgrammar}
%##)

A conditional expression evaluates its first subexpression (the
condition); if \xcd"true"
the second subexpression (the consequent) is evaluated; otherwise,
the third subexpression (the alternative) is evaluated.

The type of the condition must be \xcd"Boolean".
The type of the conditional expression is some common 
ancestor (as constrained by \Sref{LCA}) of the types of the consequent and the
alternative. 

For example, 
%~~exp~~`~~`~~a:Int,b:Int ~~ ^^^ Expressions110
\xcd`a == b ? 1 : 2`
evaluates to \xcd`1` if \xcd`a` and \xcd`b` are the same, and \xcd`2` if they
are different.   As the type of \xcd`1` is \xcd`Int{self==1}` and of \xcd`2`
is \xcd`Int{self==2}`, the type of the conditional expression has the form
\xcd`Int{c}`, where \xcd`self==1` and \xcd`self==2` both imply \xcd`c`.  For
example, it might be \xcd`Int{true}` -- or perhaps it might be 
\xcd`Int{self != 8}`. Note that this term has no most accurate type in the X10
type system.

\section{Stable equality}
\label{StableEquality}
\index{\Xcd{==}}
\index{equality}

\begin{bbgrammar}
 EqualityExp    \: RelationalExp & (\ref{prod:EqualityExp})\\%<FROM #(prod:EqualityExp)#
    \| EqualityExp \xcd"==" RelationalExp\\
    \| EqualityExp \xcd"!=" RelationalExp\\
    \| Type  \xcd"==" Type \\
\end{bbgrammar}


The \xcd"==" and \xcd"!=" operators provide a fundamental, though
non-abstract, notion of equality.  \xcd`a==b` is true if the values of \xcd`a`
and \xcd`b` are extremely identical.

\begin{itemize}
\item If \xcd`a` and \xcd`b` are values of object type, then \xcd`a==b` holds
      if \xcd`a` and \xcd`b` are the same object.
\item If one operand is \xcd`null`, then \xcd`a==b` holds iff the other is
      also \xcd`null`.
\item If the operands both have struct type, then they must be structurally equal;
that is, they must be instances of the same struct
and all their fields or components must be \xcd"==". 
\item The definition of equality for function types is specified in
      \Sref{FunctionEquality}.
\item If the operands have numeric types, they are coerced into the larger of
      the two types (see \Sref{WideningConversions}) and then compared for numeric equality.
\end{itemize}

\xcd`a != b`
is true iff \xcd`a==b` is false.

The predicates \xcd"==" and \xcd"!=" may not be overridden by the programmer.
Note that \xcd`a==b` is a form of \emph{stable equality}; that is, the result of
the equality operation is not affected by the mutable state of the program,
after evaluation of \xcd`a` and \xcd`b`. 


\section{Allocation}
\label{ClassCreation}
\index{new}
\index{allocation}
\index{class!instantation}
\index{class!construction}
\index{struct!instantation}
\index{struct!construction}
\index{instantation}

%##(ClassInstCreationExp
\begin{bbgrammar}
%(FROM #(prod:ClassInstCreationExp)#)
ClassInstCreationExp \: \xcd"new" TypeName TypeArguments\opt \xcd"(" ArgumentList\opt \xcd")" ClassBody\opt & (\ref{prod:ClassInstCreationExp}) \\
                    \| \xcd"new" TypeName \xcd"[" Type \xcd"]" \xcd"[" ArgumentList\opt \xcd"]" \\
                    \| Primary \xcd"." \xcd"new" Id TypeArguments\opt \xcd"(" ArgumentList\opt \xcd")" ClassBody\opt \\
                    \| AmbiguousName \xcd"." \xcd"new" Id TypeArguments\opt \xcd"(" ArgumentList\opt \xcd")" ClassBody\opt \\
\end{bbgrammar}
%##)

An allocation expression creates a new instance of a class and
invokes a constructor of the class.
The expression designates the class name and passes
type and value arguments to the constructor.

The allocation expression may have an optional class body.
In this case, an anonymous subclass of the given class is
allocated.   An anonymous class allocation may also specify a
single super-interface rather than a superclass; the superclass
of the anonymous class is \xcd"x10.lang.Object".

If the class is anonymous---that is, if a class body is
provided---then the constructor is selected from the superclass.
The constructor to invoke is selected using the same rules as
for method invocation (\Sref{MethodInvocation}).

The type of an allocation expression
is the return type of the constructor invoked, with appropriate
substitutions  of actual arguments for formal parameters, as
specified in \Sref{MethodInvocationSubstitution}.

It is illegal to allocate an instance of an \xcd"abstract" class.
It is illegal to allocate an instance of a class or to invoke a
constructor that is not visible at
the allocation expression.

Note that instantiating a struct type uses function application syntax, not
\xcd`new`.  As structs do not have subclassing, there is no need or
possibility of a {\em ClassBody}.


\section{Casts}\label{ClassCast}\index{cast}
\index{type conversion}

The cast operation may be used to cast an expression to a given type:

%##(CastExp
\begin{bbgrammar}
%(FROM #(prod:CastExp)#)
             CastExp \: Primary & (\ref{prod:CastExp}) \\
                    \| ExpName \\
                    \| CastExp \xcd"as" Type \\
\end{bbgrammar}
%##)

The result of this operation is a value of the given type if the cast
is permissible at run time, and either a compile-time error or a runtime
exception 
(\xcd`x10.lang.TypeCastException`) if it is not.  

When evaluating \xcd`E as T{c}`, first the value of \xcd`E` is converted to
type \xcd`T` (which may fail), and then the constraint \xcd`{c}` is checked. 



\begin{itemize}
\item If \xcd`T` is a primitive type, then \xcd`E`'s value is converted to type
      \xcd`T` according to the rules of
      \Sref{sec:effects-of-explicit-numeric-coercions}. 
      
\item If \xcd`T` is a class, then the first half of the cast succeeds if the
      run-time value of \xcd`E` is an instance of class \xcd`T`, or of a
      subclass. 

\item If \xcd`T` is an interface, then the first half of the cast succeeds if
      the run-time value of \xcd`E` is an instance of a class implementing
      \xcd`T`. 

\item If \xcd`T` is a struct type, then the first half of the cast succeeds if
      the run-time value of \xcd`E` is an instance of \xcd`T`.  

\item If \xcd`T` is a function type, then the first half of the cast succeeds
      if the run-time value of \xcd`X` is a function of that type, or a
      subtype of it.
\end{itemize}

If the first half of the cast succeeds, the second half -- the constraint
\xcd`{c}` -- must be checked.  In general this will be done at runtime, though
in special cases it can be checked at compile time.   For example, 
\xcd`n as Int{self != w}` succeeds if \xcd`n != w` --- even if \xcd`w` is a value
read from input, and thus not determined at compile time.

The compiler may forbid casts that it knows cannot possibly work. If there is
no way for the value of \xcd`E` to be of type \xcd`T{c}`, then 
\xcd`E as T{c}` can result in a static error, rather than a runtime error.  
For example, \xcd`1 as Int{self==2}` may fail to compile, because the compiler
knows that \xcd`1`, which has type \xcd`Int{self==1}`, cannot possibly be of
type \xcd`Int{self==2}`. 


%BB% \bard{This section need serious whomping.  The Java mention needs to go.  The
%BB% rules for coercions are given in \Sref{sec:effects-of-explicit-numeric-coercions}.
%BB% If the \xcd`Type` has a constraint, the constraint will be checked at runtime. 
%BB% We need to give examples. 
%BB% }
%BB% 
%BB% Type conversion is checked according to the
%BB% rules of the \java{} language (e.g., \cite[\S 5.5]{jls2}).
%BB% For constrained types, both the base
%BB% type and the constraint are checked.
%BB% If the
%BB% value cannot be cast to the appropriate type, a
%BB% \xcd"ClassCastException"
%BB% is thrown. 



% {\bf Conversions of numeric values}
% {\bf Can't do (a as T) if a can't be a T.}


%If the value cannot be cast to the
%appropriate place type a \xcd"BadPlaceException" is thrown. 

% Any attempt to cast an expression of a reference type to a value type
% (or vice versa) results in a compile-time error. Some casts---such as
% those that seek to cast a value of a subtype to a supertype---are
% known to succeed at compile-time. Such casts should not cause extra
% computational overhead at run time.

\section{\Xcd{instanceof}}
\label{instanceOf}
\index{\Xcd{instanceof}}
\index{instanceof}

\Xten{} permits types to be used in an in instanceof expression
to determine whether an object is an instance of the given type:

%##(RelationalExp
\begin{bbgrammar}
%(FROM #(prod:RelationalExp)#)
       RelationalExp \: RangeExp & (\ref{prod:RelationalExp}) \\
                    \| SubtypeConstraint \\
                    \| RelationalExp \xcd"<" RangeExp \\
                    \| RelationalExp \xcd">" RangeExp \\
                    \| RelationalExp \xcd"<=" RangeExp \\
                    \| RelationalExp \xcd">=" RangeExp \\
                    \| RelationalExp \xcd"instanceof" Type \\
                    \| RelationalExp \xcd"in" ShiftExp \\
\end{bbgrammar}
%##)

In the above expression, \grammarrule{Type} is any type. At run time, the
result of this operator is \xcd"true" if the
\grammarrule{RelationalExpression} can be coerced to \grammarrule{Type}
without a \xcd"TypeCastException" being thrown or static error occurring.
Otherwise the result is \xcd"false". This determination may involve checking
that the constraint, if any, associated with the type is true for the given
expression.

%~~exp~~`~~`~~x:Int~~ ^^^ Expressions120
For example, \xcd`3 instanceof Int{self==x}` is an overly-complicated way of
saying \xcd`3==x`.


However, it is a static error if \xcd`e` cannot possibly be an instance of
\xcd`C{c}`; the compiler will reject \xcd`1 instanceof Int{self == 2}` because
\xcd`1` can never satisfy \xcd`Int{self == 2}`. Similarly, \Xcd{1 instanceof
String} is a static error, rather than an expression always returning false. 

\limitationx
X10 does not currently handle \xcd`instanceof` of generics in the way you
%~NO~exp~~`~~`~~r:Array[Int](1) ~~
might expect.  For example, \xcd`r instanceof Array[Int{self != 0}]` does
not test that every element of \xcd`r` is non-zero; instead, the compiler
rejects it.


\section{Subtyping expressions}
\index{\Xcd{<:}}
\index{\Xcd{:>}}
\index{subtype!test}


%##(SubtypeConstraint
\begin{bbgrammar}
%(FROM #(prod:SubtypeConstraint)#)
   SubtypeConstraint \: Type  \xcd"<:" Type  & (\ref{prod:SubtypeConstraint}) \\
                    \| Type  \xcd":>" Type  \\
\end{bbgrammar}
%##)

The subtyping expression \xcdmath"T$_1$ <: T$_2$" evaluates to \xcd"true"
\xcdmath"T$_1$" is a subtype of \xcdmath"T$_2$".

The expression \xcdmath"T$_1$ :> T$_2$" evaluates to \xcd"true"
\xcdmath"T$_2$" is a subtype of \xcdmath"T$_1$".

The expression \xcdmath"T$_1$ == T$_2$"
evaluates to  \xcd"true" \xcdmath"T$_1$" is a subtype of \xcdmath"T$_2$" and
if \xcdmath"T$_2$" is a subtype of \xcdmath"T$_1$".

Subtyping expressions are particularly useful in giving constraints on generic
types.  \xcd`x10.util.Ordered[T]` is an interface whose values can be compared
with values of type \xcd`T`. 
In particular, \xcd`T <: x10.util.Ordered[T]` is
true if values of type \xcd`T` can be compared to other values of type
\xcd`T`.  So, if we wish to define a generic class \xcd`OrderedList[T]`, of
lists whose elements are kept in the right order, we need the elements to be
ordered.  This is phrased as a constraint on \xcd`T`: 
%~~gen ^^^ Expressions130
% package expre.ssi.onsfgua.rde.dq.uantification;
%~~vis
\begin{xten}
class OrderedList[T]{T <: x10.util.Ordered[T]} {
  // ...
}
\end{xten}
%~~siv
%
%~~neg




\section{Contains expressions}
\index{in}

%##(RelationalExp
\begin{bbgrammar}
%(FROM #(prod:RelationalExp)#)
       RelationalExp \: RangeExp & (\ref{prod:RelationalExp}) \\
                    \| SubtypeConstraint \\
                    \| RelationalExp \xcd"<" RangeExp \\
                    \| RelationalExp \xcd">" RangeExp \\
                    \| RelationalExp \xcd"<=" RangeExp \\
                    \| RelationalExp \xcd">=" RangeExp \\
                    \| RelationalExp \xcd"instanceof" Type \\
                    \| RelationalExp \xcd"in" ShiftExp \\
\end{bbgrammar}
%##)

The expression \xcd"p in r" tests if a value \xcd"p" is in a collection
\xcd"r"; it evaluates to \xcd"r.contains(p)".
The collection \xcd"r"
must be of type \xcd"Collection[T]" and the value \xcd"p" must
be of type \xcd"T".

\section{Array Constructors}
\label{sect:ArrayCtors}
\index{array!construction}
\index{array!literal}

%##(Primary ClassInstCreationExp
\begin{bbgrammar}
%(FROM #(prod:Primary)#)
             Primary \: \xcd"here" & (\ref{prod:Primary}) \\
                    \| \xcd"[" ArgumentList\opt \xcd"]" \\
                    \| Literal \\
                    \| \xcd"self" \\
                    \| \xcd"this" \\
                    \| ClassName \xcd"." \xcd"this" \\
                    \| \xcd"(" Exp \xcd")" \\
                    \| ClassInstCreationExp \\
                    \| FieldAccess \\
                    \| MethodInvocation \\
                    \| MethodSelection \\
                    \| OperatorFunction \\
%(FROM #(prod:ClassInstCreationExp)#)
ClassInstCreationExp \: \xcd"new" TypeName TypeArguments\opt \xcd"(" ArgumentList\opt \xcd")" ClassBody\opt & (\ref{prod:ClassInstCreationExp}) \\
                    \| \xcd"new" TypeName \xcd"[" Type \xcd"]" \xcd"[" ArgumentList\opt \xcd"]" \\
                    \| Primary \xcd"." \xcd"new" Id TypeArguments\opt \xcd"(" ArgumentList\opt \xcd")" ClassBody\opt \\
                    \| AmbiguousName \xcd"." \xcd"new" Id TypeArguments\opt \xcd"(" ArgumentList\opt \xcd")" ClassBody\opt \\
\end{bbgrammar}
%##)

X10 includes short syntactic forms for constructing one-dimensional arrays.
The shortest form is to enclose some expressions in brackets: 
%~~gen ^^^ Expressions140
% package Expressions.ArrayCtor.Primo;
% class Example {
% def example() {
%~~vis
\begin{xten}
val ints <: Array[Int](1) = [1,3,7,21];
\end{xten}
%~~siv
%}}
%~~neg

The expression \Xcd{[e1,e2,e3, ..., en]} produces an \Xcd{n}-element
\xcd`Array[T](1)`, where \xcd`T` is the least common supertype of the {\bf
  base types} of the expressions \xcd`ei`. For example, the type of
\xcd`[0,1,2]` is \Xcd{Array[Int](1)}.    

More importantly, the type of 
\xcd`[0]` is also \xcd`Array[Int](1)`.  It is {\em not} 
\xcd`Array[Int{self==0}](1)`, even though all the elements are all 
of type \xcd`Int{self==0}`.  This is subtle but important. There are many
functions that take \xcd`Array[Int](1)`s, such as conversions to \xcd`Point`.
These functions do {\em not} take
\xcd`Array[Int{self==0}]`'s.

(Suppose that the function took \xcd`a:Array[Int](1)` and did 
the operation \xcd`a(i)=100`.   This operation is perfectly fine for
an \xcd`Array[Int](1)`, which is all the compiler knows about \xcd`a`.  
However, it is invalid for an \xcd`Array[Int{self==0}](1)`, because it assigns
a non-zero value to an element of the array, violating the type constraint
which says that all the elements are zero.  So, \xcd`Array[Int{self==0}](1)`
is not a subtype of \xcd`Array[Int](1)` --- the two types are simply unrelated.)
%~~type~~`~~`~~ ~~ ^^^ Expressions150
Since there are far more uses for \xcd`Array[Int](1)` than
%~~type~~`~~`~~ ~~ ^^^ Expressions160
\xcd`Array[Int{self==0}](1)`, the compiler produces the former.

Still, occasionally one does actually need \xcd`Array[Int{self==0}](1)`, 
or, say, \xcd`Array[Eel{self != null}](1)`, an array of non-null \xcd`Eel`s.  
For these cases, X10 provides an array constructor which does allow
specification of the element type: \xcd`new Array[T][e1...en]`.  Each
element \xcd`ei` must be of type \xcd`T`.  The resulting array is of type
\xcd`Array[T](1)`.  
%~~gen ^^^ Expressions170
%package Expressions.ArrayCtor.Details;
%class Eel{}
%class Example{
%def example(){
%~~vis
\begin{xten}
val zero <: Array[Int{self == 0}](1) = new Array[Int{self == 0}][0];
val non1 <: Array[Int{self != 1}](1) = new Array[Int{self != 1}][0];
val eels <: Array[Eel{self != null}](1) = 
    new Array[Eel{self != null}][ new Eel() ];
\end{xten}
%~~siv
%}}
%~~neg



%%OLD-RAIL%% 
%%OLD-RAIL%% 
%%OLD-RAIL%% \noo{This is now an Array ctor and the text needs revision}
%%OLD-RAIL%% \label{RailConstructors}
%%OLD-RAIL%% 
%%OLD-RAIL%% \begin{grammar}
%%OLD-RAIL%% RailConstructor \: \xcd"[" Expressions \xcd"]" \\
%%OLD-RAIL%% Expressions \: Expression ( \xcd"," Expression )\star \\
%%OLD-RAIL%% \end{grammar}
%%OLD-RAIL%% 
%%OLD-RAIL%% The rail constructor \xcdmath"[a$_0$, $\dots$, a$_{k-1}$]"
%%OLD-RAIL%% creates an instance of \xcd"ValRail" with length $k$, 
%%OLD-RAIL%% whose $i$th element is
%%OLD-RAIL%% \xcdmath"a$_i$".  The element type of the rail is a common ancestor of the
%%OLD-RAIL%% types of the \xcdmath"a$_i$"'s, as per \Sref{LCA}.
%%OLD-RAIL%% %~s~gen
%%OLD-RAIL%% % package ex.pre.ssio.nsandrailconstructors;
%%OLD-RAIL%% % class Example {
%%OLD-RAIL%% % def example() {
%%OLD-RAIL%% %~s~vis
%%OLD-RAIL%% \begin{xten}
%%OLD-RAIL%% val a <: Array[Int] = [1,3,5];
%%OLD-RAIL%% val b <: Array[Any] = [1, a, "please"];
%%OLD-RAIL%% \end{xten}
%%OLD-RAIL%% %~s~siv
%%OLD-RAIL%% % } } 
%%OLD-RAIL%% %~s~neg
%%OLD-RAIL%% 
%%OLD-RAIL%% Since arrays are subtypes of \xcd"(Point) => T",
%%OLD-RAIL%% rail constructors can be passed into the \xcd"Array" and
%%OLD-RAIL%% \xcd"ValArray" constructors as initializer functions.
%%OLD-RAIL%% 
%%OLD-RAIL%% Rail constructors of type \xcd"ValRail[Int]" and length \xcd"n" 
%%OLD-RAIL%% may be implicitly converted to type \xcd"Point{rank==n}".
%%OLD-RAIL%% Rail constructors of type \xcd"ValRail[Region]" and length \xcd"n" 
%%OLD-RAIL%% may be implicitly converted to type \xcd"Region{rank==n}".
%%OLD-RAIL%% 
%%OLD-RAIL%% %~s~gen
%%OLD-RAIL%% % package ex.pre.ssio.nsandrailconstructors;
%%OLD-RAIL%% % class Exympyl {
%%OLD-RAIL%% % def example() {
%%OLD-RAIL%% %~s~vis
%%OLD-RAIL%% \begin{xten}
%%OLD-RAIL%% val a : Point{rank==4} = [1,2,3,4];
%%OLD-RAIL%% val b : Region{rank==2} = (-1 .. 1) * (-2 .. 2);
%%OLD-RAIL%% \end{xten}
%%OLD-RAIL%% %~s~siv
%%OLD-RAIL%% % } } 
%%OLD-RAIL%% %~s~neg
%%OLD-RAIL%% 

\section{Coercions and conversions}
\label{XtenConversions}
\label{User-definedCoercions}
\index{conversion}\index{coercion}
\index{type!conversion}\index{type!coercion}

\XtenCurrVer{} supports the following coercions and conversions.

\subsection{Coercions}

%##(CastExp
\begin{bbgrammar}
%(FROM #(prod:CastExp)#)
             CastExp \: Primary & (\ref{prod:CastExp}) \\
                    \| ExpName \\
                    \| CastExp \xcd"as" Type \\
\end{bbgrammar}
%##)


A {\em coercion} does not change object identity; a coerced object may
be explicitly coerced back to its original type through a cast. A {\em
  conversion} may change object identity if the type being converted
to is not the same as the type converted from. \Xten{} permits
user-defined conversions (\Sref{sec:user-defined-conversions}).

\paragraph{Subsumption coercion.}
A subtype may be implicitly coerced to any supertype.
\index{coercion!subsumption}

\paragraph{Explicit coercion (casting with \xcd"as")}



An object of any class may be explicitly coerced to any other
class type using the \xcd"as" operation.  If \xcd`Child <: Person` and
\xcd`rhys:Child`, then 
%~~gen ^^^ Expressions180
% package Types.Coercions;
%  class Person {}
%  class Child extends Person{} 
%  class Exampllllle { 
%    def example(rhys:Child) =
%~~vis
\begin{xten}
  rhys as Person
\end{xten}
%~~siv
%;}
%~~neg
is an expression of type \xcd`Person`.  

If the value coerced is not an instance of the target type,
a \xcd"ClassCastException" is thrown.  Casting to a constrained
type may require a run-time check that the constraint is
satisfied.
\index{coercion!explicit}
\index{cast}
\index{\Xcd{as}}

\limitation{It is currently a static error, rather than the specified
\xcd`ClassCastException`, when the cast is statically determinable to be
impossible.}

\paragraph{Effects of explicit numeric coercion}
\label{sec:effects-of-explicit-numeric-coercions}

Coercing a number of one type to another type gives the best approximation of
the number in the result type, or a suitable disaster value if no
approximation is good enough.  

\begin{itemize}
\item Casting a number to a {\em wider} numeric type is safe and effective,
      and can be done by an implicit conversion as well as an explicit
%~~exp~~`~~`~~ ~~ ^^^ Expressions190
      coercion.  For example, \xcd`4 as Long` produces the \xcd`Long` value of
      4. 
\item Casting a floating-point value to an integer value truncates the digits
      after the decimal point, thereby rounding the number towards zero.  
%~~exp~~`~~`~~ ^^^ Expressions200
      \xcd`54.321 as Int` is \xcd`54`, and 
%~~exp~~`~~`~~ ~~ ^^^ Expressions210
      \xcd`-54.321 as Int` is \xcd`-54`.
      If the floating-point value is too large to represent as that kind of
      integer, the coercion returns the largest or smallest value of that type
      instead: \xcd`1e110 as Int` is 
      \xcd`Int.MAX_VALUE`, \xcd`2147483647`. 

\item Casting a \xcd`Double` to a \xcd`Float` normally truncates digits: 
%~~exp~~`~~`~~ ~~ ^^^ Expressions220
      \xcd`0.12345678901234567890 as Float` is \xcd`0.12345679f`.  This can
      turn a nonzero \xcd`Double` into \xcd`0.0f`, the zero of type
      \xcd`Float`: 
%~~exp~~`~~`~~ ~~ ^^^ Expressions230
      \xcd`1e-100 as Float` is \xcd`0.0f`.  Since 
      \xcd`Double`s can be as large as about \xcd`1.79E308` and \xcd`Float`s
      can only be as large as about \xcd`3.4E38f`, a large \xcd`Double` will
      be converted to the special \xcd`Float` value of \xcd`Infinity`: 
%~~exp~~`~~`~~ ~~ ^^^ Expressions240
      \xcd`1e100 as Float` is \xcd`Infinity`.
\item Integers are coerced to smaller integer types by truncating the
      high-order bits. If the value of the large integer fits into the smaller
      integer's range, this gives the same number in the smaller type: 
%~~exp~~`~~`~~ ~~ ^^^ Expressions250
      \xcd`12 as Byte` is the \xcd`Byte`-sized 12, 
%~~exp~~`~~`~~ ~~ ^^^ Expressions260
      \xcd`-12 as Byte` is -12. 
      However, if the larger integer {\em doesn't} fit in the smaller type,
%~~exp~~`~~`~~ ~~ ^^^ Expressions270
      the numeric value and even the sign can change: \xcd`254 as Byte` is
      \xcd`Byte`-sized \xcd`-2`.  


\end{itemize}

\subsection{Conversions}
\index{conversion}
\index{type!conversion}

\paragraph{Widening numeric conversion.}
\label{WideningConversions}
A numeric type may be implicitly converted to a wider numeric type. In
particular, an implicit conversion may be performed between a numeric
type and a type to its right, below:

\begin{xten}
Byte < Short < Int < Long < Float < Double
\end{xten}

\index{conversion!widening}
\index{conversion!numeric}

\paragraph{String conversion.}
Any value that is an operand of the binary
\xcd"+" operator may
be converted to \xcd"String" if the other operand is a \xcd"String".
A conversion to \xcd"String" is performed by invoking the \xcd"toString()"
method.

\index{conversion!string}

\paragraph{User defined conversions.}\label{sec:user-defined-conversions}
\index{conversion!user-defined}

The user may define conversion operators from type \Xcd{A} {\em to} a
container type \Xcd{B} by specifying a method on \Xcd{B} as follows:

\begin{xten}
  public static operator (r: A): T = ... 
\end{xten}

The return type \Xcd{T} should be a subtype of \Xcd{B}. The return
type need not be specified explicitly; it will be computed in the
usual fashion if it is not. However, it is good practice for the
programmer to specify the return type for such operators explicitly.

For instance, the code for \Xcd{x10.lang.Point} contains:

\begin{xten}
  public operator (r: Array[Int](1)): Point(r.length) = make(r);
\end{xten}

The compiler looks for such operators on the container type \Xcd{B}
when it encounters an expression of the form \Xcd{r as B} (where
\Xcd{r} is of type \Xcd{A}). If it finds such a method, it sets the
type of the expression \Xcd{r as B} to be the return type of the
method. Thus the type of \Xcd{r as B} is guaranteed to be some subtype
of \Xcd{B}.

\begin{example}
Consider the following code:  



%~~stmt~~\begin{xten}~~\end{xten}~~ ~~ ^^^ Expressions280
\begin{xten}
val p  = [2, 2, 2, 2, 2] as Point;
val q = [1, 1, 1, 1, 1] as Point;
val a = p - q;    
\end{xten}
This code fragment compiles successfully, given the above operator definition. 
The type of \Xcd{p} is inferred to be \Xcd{Point(5)} (i.e.{} the type 
%~~type~~`~~`~~ ~~ ^^^ Expressions290
\xcd`Point{self.rank==5}`.
Similarly for \Xcd{q}. Hence the application of the operator ``\Xcd{-}'' is legal (it requires both arguments to have the same rank). The type of \Xcd{a} is computed as \Xcd{Point(5)}.
\end{example}
	
\chapter{Statements}\label{XtenStatements}\index{statements}

This chapter describes the statements in the sequential core of
\Xten{}.  Statements involving concurrency and distribution
are described in \Sref{XtenActivities}.

\section{Empty statement}

\begin{grammar}
Statement \: \xcd";" \\
\end{grammar}

The empty statement \xcd";" does nothing.  It is useful when a
loop header is evaluated for its side effects.  For example,
the following code sums the elements of an array.
\begin{xten}
var sum: Int = 0;
for (i: Int = 0; i < a.length; i++, sum += a[i])
    ;
\end{xten}

\section{Local variable declaration}

\begin{grammar}
Statement \: LocalVariableDeclarationStatement \\
             LocalVariableDeclarationStatement \:
             LocalVariableDeclaration \xcd";" \\
\end{grammar}

The syntax of local variables declarations is described in
\Sref{VariableDeclarations}.

Local variables may be declared only within a block statement
(\Sref{Blocks}).
The scope of a local variable declaration is the 
statement itself and the subsequent statements in the block.

\section{Block statement}
\label{Blocks}

\begin{grammar}
Statement \: BlockStatement \\
BlockStatement \: \xcd"{" Statement\star \xcd"}" \\
\end{grammar}

A block statement consists of a sequence of statements delimited
by ``\xcd"{"'' and ``\xcd"}"''.  Statements are evaluated in
order.  The scope of local variables introduced within the block  
is the remainder of the block following the variable declaration.

\section{Expression statement}

\begin{grammar}
Statement \: ExpressionStatement \\
ExpressionStatement \: StatementExpression \xcd";" \\
StatementExpression \: Assignment \\
          \| Allocation \\
          \| Call \\
\end{grammar}

The expression statement evaluates an expression, ignoring the
result.  The expression must be either an assignment, an
allocation, or a call.

\section{Labeled statement}

\begin{grammar}
Statement \: LabeledStatement \\
LabeledStatement \: Identifier \xcd":" Statement \\
\end{grammar}

Statements may be labeled.  The label may be used as the target
of a break or continue statement.  The scope of a label is the
statement labeled.

\section{Break statement}

\begin{grammar}
Statement \: BreakStatement \\
BreakStatement \: \xcd"break" Identifier\opt \\
\end{grammar}

An unlabeled break statement exits the currently enclosing loop
or switch statement.

An labeled break statement exits the enclosing loop
or switch statement with the given label.

It is illegal to break out of a loop not defined in the current
method, constructor, initializer, or closure.

The following code searches for an element of a two-dimensional
array and breaks out of the loop when found:

\begin{xten}
var found: Boolean = false;
for (i: Int = 0; i < a.length; i++)
    for (j: Int = 0; j < a(i).length; j++)
        if (a(i)(j) == v) {
            found = true;
            break;
        }
\end{xten}

\section{Continue statement}

\begin{grammar}
Statement \: ContinueStatement \\
ContinueStatement \: \xcd"continue" Identifier\opt \\
\end{grammar}

An unlabeled continue statement branches to the top of the
currently enclosing loop.

An labeled break statement branches to the top of the enclosing loop
with the given label.

It is illegal to continue a loop not defined in the current
method, constructor, initializer, or closure.

\section{If statement}

\begin{grammar}
Statement \: IfThenStatement \\
          \| IfThenElseStatement \\
IfThenStatement \: \xcd"if" \xcd"(" Expression \xcd")" Statement \\
IfThenElseStatement \: \xcd"if" \xcd"(" Expression \xcd")" Statement \xcd"else" Statement \\
\end{grammar}

An if statement comes in two forms: with and without an else
clause.

The if-then statement evaluates a condition expression and 
evaluates the consequent expression if the condition is
\xcd"true".  If the 
condition is \xcd"false",
the if-then statement completes normally.

The if-then-else statement evaluates a condition expression and 
evaluates the consequent expression if the condition is
\xcd"true"; otherwise, the alternative statement is evaluated.

The condition must be of type \xcd"Boolean".

\section{Switch statement}

\begin{grammar}
Statement \: SwitchStatement \\
SwitchStatement \: \xcd"switch" \xcd"(" Expression \xcd")" \xcd"{" Case\plus \xcd"}" \\
Case \: \xcd"case" Expression \xcd":" Statement\star \\
     \| \xcd"default" \xcd":" Statement\star \\
\end{grammar}

A switch statement evaluates an index expression and then branches to
a case whose value equal to the value of the index expression.
If no such case exists, the switch branches to the 
\xcd"default" case, if any.

Statements in each case branch evaluated in sequence.  At the
end of the branch, normal control-flow falls through to the next case, if
any.  To prevent fall-through, a case branch may be exited using
a \xcd"break" statement.

The index expression must be of type \xcd"Int".

Case labels must be of type \xcd"Int" and must be compile-time
constants.  Case labels cannot be duplicated within the
\xcd"switch" statement.

\section{While statement}

\index{while@\xcd"while"}

\begin{grammar}
Statement \: WhileStatement \\
WhileStatement \: \xcd"while" \xcd"(" Expression \xcd")" Statement \\
\end{grammar}

A while statement evaluates a condition and executes a loop body
if \xcd"true".  If the loop body completes normally (either by reaching
the end or via a \xcd"continue" statement with the loop header
as target), the condition is reevaluated and the loop repeats if
\xcd"true".  If the condition is \xcd"false", the loop
exits.

The condition must be of type \xcd"Boolean".

\section{Do--while statement}

\index{do@\xcd"do"}

\begin{grammar}
Statement \: DoWhileStatement \\
DoWhileStatement \: \xcd"do" Statement \xcd"while" \xcd"(" Expression \xcd")" \xcd";" \\
\end{grammar}


A do-while statement executes a loop body, and then evaluates a
condition expression.  If \xcd"true", the loop repeats.
Otherwise, the loop exits.

The condition must be of type \xcd"Boolean".

\section{For statement}

\index{for@\xcd"for"}

\begin{grammar}
Statement \: ForStatement \\
          \| EnhancedForStatement \\
ForStatement \: \xcd"for" \xcd"("
        ForInit\opt \xcd";" Expression\opt \xcd";" ForUpdate\opt
        \xcd")" Statement \\
ForInit \:
        StatementExpression ( \xcd"," StatementExpression )\star
        \\
      \| LocalVariableDeclaration \\
EnhancedForStatement \: \xcd"for" \xcd"("
        Formal \xcd"in" Expression 
        \xcd")" Statement \\
\end{grammar}

\Xten{} provides two forms of for statement: a basic for
statement and an enhanced for statement.

A basic for statement consists of an initializer, a condition, an
iterator, and a body.  First, the initializer is evaluated.
The initializer may introduce local variables that are in scope
throughout the for statement.  An empty initializer is
permitted.
Next, the condition is evaluated.  If \xcd"true", the loop body
is executed; otherwise, the loop exits.
The condition may be omitted, in which case the condition is
considered \xcd"true".
If the loop completes normally (either by reaching the end
or via a \xcd"continue" statement with the loop header as
target),
the iterator is evaluated and then the condition is reevaluated
and the loop repeats if
\xcd"true".  If the condition is \xcd"false", the loop
exits.

The condition must be of type \xcd"Boolean".
The initializer and iterator are statements, not expressions
and so do not have types.

\label{ForAllLoop}

% XXX REGION

An enhanced for statement is used to iterate over a collection.
If the formal parameter is of type \xcd"T",
the collection expression must be of type \xcd"Iterable[T]".
Exploded
syntax may
be used for the formal parameter (\Sref{exploded-syntax}).
Each iteration of the loop
binds the parameter to another element of the collection.
If the parameter is final, it may not be assigned within the
loop body.

In a common case, the
the collection is intended to be of type
\xcd"Region" and the formal parameter is of type \xcd"Point".
Expressions \xcd"e" of type \xcd"Dist" and
\xcd"Array" are also accepted, and treated as if they were
\xcd"e.region".
If the collection is a region, the \xcd"for" statement
enumerates the points in the region in canonical order.



\section{Throw statement}
\index{throw}

\begin{grammar}
Statement \: ThrowStatement \\
ThrowStatement \: \xcd"throw" Expression \xcd";"
\end{grammar}

The \xcd"throw" statement throws an exception.  The exception
must be a subclass of the value class \xcd"x10.lang.Throwable". 
% null not allowed since a value class;
% If the exception is
% \xcd"null", a \xcd"NullPointerException" is thrown.

\begin{example}
The following statement checks if an index is in range and
throws an exception if not.
\begin{xten}
if (i < 0 || i > x.length)
    throw new IndexOutOfBoundsException();
\end{xten}
\end{example}

\section{Try--catch statement}

\begin{grammar}
Statement \: TryStatement \\
TryStatement \: \xcd"try" BlockStatement Catch\plus Finally\opt \\
             \| \xcd"try" BlockStatement Catch\star Finally \\
Catch \: \xcd"catch" \xcd"(" Formal \xcd")" BlockStatement \\
Finally \: \xcd"finally" BlockStatement \\
\end{grammar}

Exceptions are handled with a \xcd"try" statement.
A \xcd"try" statement consists of a \xcd"try" block, zero or more
\xcd"catch" blocks, and an optional \xcd"finally" block.

First, the \xcd"try" block is evaluated.  If the block throws an
exception, control transfers to the first matching \xcd"catch"
block, if any.  A \xcd"catch" matches if the value of the
exception thrown is a subclass of the \xcd"catch" block's formal
parameter type.

The \xcd"finally" block, if present, is evaluated on all normal
and exceptional control-flow paths from the \xcd"try" block.
If the \xcd"try" block completes normally
or via a \xcd"return", a \xcd"break", or a
\xcd"continue" statement, 
the \xcd"finally"
block is evaluated, and then control resumes at
the statement following the \xcd"try" statement, at the branch target, or at
the caller as appropriate.
If the \xcd"try" block completes
exceptionally, the \xcd"finally" block is evaluated after the
matching \xcd"catch" block, if any, and then the
exception is rethrown.

\section{Return statement}
\label{ReturnStatement}
\index{ReturnStatement}
\begin{grammar}
Statement \: ReturnStatement \\
ReturnStatement \: \xcd"return" Expression \xcd";" \\
             \| \xcd"return" \xcd";" \\
\end{grammar}

Methods and closures may return values using a return statement.
If the method's return type is expliclty declared \xcd"Void",
the method may return without a value; otherwise, it must return
a value of the appropriate type.
	

\chapter{Places}
\label{XtenPlaces}
\index{place}

An \Xten{} place is a repository for data and activities, corresponding
loosely to a process or a processor. Places induce a concept of ``local''. The
activities running in a place may access data items located at that place with
the efficiency of on-chip access. Accesses to remote places may take orders of
magnitude longer. X10's system of places is designed to make this obvious.
Programmers are aware of the places of their data, and know when they are
incurring communication costs, but the actual operation to do so is easy. It's
not hard to use non-local data; it's simply hard to to do so accidentally.

The set of places available to a computation is determined at the time that
the program is started, and remains fixed through the run of the program. See
the {\tt README} documentation on how to set command line and configuration
options to set the number of places.

Places are first-class values in X10, as instances 
\xcd"x10.lang.Place".   \xcd`Place` provides a number of useful ways to
query places, such as \xcd`Place.places`, which is a  \xcd`Sequence[Place]` of 
the places
available to the current run of the program.

Objects and structs (with one exception) are created in a single place -- the
place that the constructor call was running in. They cannot change places.
They can be {\em copied} to other places, and the special library struct
\Xcd{GlobalRef} allows values at one place to point to values at another.  

\section{The Structure of Places}
\index{place!MAX\_PLACES}
\index{place!FIRST\_PLACES}
\index{MAX\_PLACES}
\index{FIRST\_PLACE}

%~~exp~~`~~`~~ ~~ ^^^ Places10
Places are numbered 0 through \xcd`Place.MAX_PLACES-1`; the number is stored
in the field 
\xcd`pl.id`.  The \xcd`Sequence[Place]` \xcd`Place.places()` contains the places of the
program, in numeric order. 
The program starts by executing a \xcd`main` method at
%~~exp~~`~~`~~ ~~ ^^^ Places20
\xcd`Place.FIRST_PLACE`, which is 
%~~exp~~`~~`~~ ~~ ^^^ Placesoik
\xcd`Place.places()(0)`; see
\Sref{initial-computation}. 

Operations on places include \xcd`pl.next()`, which gives the next entry
(looping around) in \xcd`Place.places` and its opposite \xcd`pl.prev()`. 
In multi-place executions, 
\xcd`here.next()` is a convenient way to express ``a place other than \xcd`here`''.
There are also tests, like  
%~~exp~~`~~`~~pl:Place ~~ ^^^ Placesoid
\xcd`pl.isCUDA()`, which test for particular kinds of processors.


\section{{\tt here}}\index{here}\label{Here}

The variable \xcd"here" is always bound to the place at which the current
computation is running, in the same way that \xcd`this` is always bound to the
instance of the current class (for non-static code), or \xcd`self` is bound to
the instance of the type currently being constrained.  
\xcd`here` may denote different places in the same method body or even the
same expression, due to
place-shifting operations.


This is not unusual for automatic variables:  \Xcd{self} denotes 
two different values (one \xcd`List`, one \xcd`Long`) 
when one describes a non-null list of non-zero numbers as
\xcd`List[Long{self!=0}]{self!=null}`. In the following 
code, \xcd`here` has one value at 
\xcd`h0`, and a different one at \xcd`h1` (unless there is only one place).
%~~gen ^^^ Placesoijo
% package places.are.For.Graces;
% class Example {
% def example() {
%~~vis
\begin{xten}
val h0 = here;
at (here.next()) {
  val h1 = here; 
  assert (h0 != h1);
}
\end{xten}
%~~siv
%} } 
% 
%~~neg
\noindent
(Similar examples show that \xcd`self` and \xcd`this` have the same behavior:
\xcd`self` can be shadowed by constrained types appearing inside of type
constraints, and \xcd`this` by inner classes.)



The following example looks through a list of references to \Xcd{Thing}s.  
It finds those references to things that are \Xcd{here}, and deals with them.  
%~~gen ^^^ Places70
%package Places.Are.For.Graces.2;
%import x10.util.*;
%abstract class Thing {}
%class DoMine {
%  static def dealWith(Thing) {}	
%~~vis
\begin{xten}
  public static def deal(things: List[GlobalRef[Thing]]) {
     for(gr in things) {
        if (gr.home == here) {
           val grHere = 
               gr as GlobalRef[Thing]{gr.home == here};
           val thing <: Thing = grHere();
           dealWith(thing);
        }
     }
  }
\end{xten}
%~~siv
%}
% 
%~~neg

\section{ {\tt at}: Place Changing}\label{AtStatement}
\index{at}
\index{place!changing}

An activity may change place synchronously using the \xcd"at" statement or
\xcd"at" expression. Like any distributed operation, it is 
potentially expensive, as it requires, at a minimum, two messages
and the copying of all data used in the operation, and must be used with care
-- but it provides the basis for distributed programming in X10.

%##(AtStatement AtExp
\begin{bbgrammar}
%(FROM #(prod:AtStmt)#)
              AtStmt \: \xcd"at" \xcd"(" Exp \xcd")" Stmt & (\ref{prod:AtStmt}) \\
%(FROM #(prod:AtExp)#)
               AtExp \: \xcd"at" \xcd"(" Exp \xcd")" ClosureBody & (\ref{prod:AtExp}) \\
\end{bbgrammar}
%##)

The {\it PlaceExp} must be an expression of type \xcd`Place` or some
subtype. For programming convenience, if {\it PlaceExp} is of type
\xcd`GlobalRef[T]` then the \xcd'home' property of \xcd'GlobalRef' is
used as the value of {\it PlaceExp}.

An activity may also spawn an asynchronous remote child activity.  For
optimal performance, it is desirable for the spawning activity to
continue executing locally without waiting for the message creating
the remote child activity to arrive at the destination place. \Xten{}
supports this ``fire-and-forget'' style of remote activity creation by
special handling of the combination of \xcd'at (P) async S'.  In
particular, any exceptions raised during deserialization
(\Sref{sect:at-init-val}) at the remote place will be reported
asynchronously (as if they occured after the remote activity
\xcd`async S` was spawned).

%%AT-COPY%% The \xcd`at`-statment \xcd`at(p;F)S` first evaluates \xcd`p` to a place, then
%%AT-COPY%% copies information to that place as determined by \xcd`F`, and then executes
%%AT-COPY%% \xcd`S` using the resulting copies.  The \xcd`at`-{\em expression}
%%AT-COPY%% \xcd`at(p;F)E` is similar, but it copies the result of the expression \xcd`E`
%%AT-COPY%% and returns the copy as its result.
%%AT-COPY%% 
%%AT-COPY%% The clause \xcd`F` in \xcd`at(p;F)S` is a list of zero or more {\em copy
%%AT-COPY%% specifiers}, explaining what values are to be copied to the place \xcd`p`, and
%%AT-COPY%% how they are to be referred to at \xcd`p`.  
%%AT-COPY%% 

%%AT-COPY%% \begin{ex}
%%AT-COPY%% The following example creates a rail \xcd`a` located \xcd`here`, and copies
%%AT-COPY%% it to another place, giving the copy the name \xcd`a2` there.  The copy is
%%AT-COPY%% modified and examined.  After the \xcd`at` finishes, the original is also
%%AT-COPY%% examined, and (since only the copy, not the original, was modified) is observed
%%AT-COPY%% to be unchanged. 
%%AT-COPY%% %~x~gen ^^^ Places6e1o
%%AT-COPY%% % package Places6e1o;
%%AT-COPY%% % KNOWNFAIL-at
%%AT-COPY%% % class Example { static def example() { 
%%AT-COPY%% %~x~vis
%%AT-COPY%% \begin{xten}
%%AT-COPY%% val a = [1,2,3];
%%AT-COPY%% at(here.next(); a2 = a) {
%%AT-COPY%%   a2(1) = 4;
%%AT-COPY%%   assert a2(0)==1 && a2(1)==4 && a2(2)==3; 
%%AT-COPY%%   // 'a' is not accessible here
%%AT-COPY%% }
%%AT-COPY%% assert  a(0)==1 && a(1)==2 && a(2)==3; 
%%AT-COPY%% \end{xten}
%%AT-COPY%% %~x~siv
%%AT-COPY%% %} } 
%%AT-COPY%% % class Hook { def run() { Example.example(); return true; }}
%%AT-COPY%% %~x~neg
%%AT-COPY%% \end{ex}
%%AT-COPY%% 

\begin{ex}
The following example creates a rail \xcd`a` located \xcd`here`, and copies
it to another place.  \xcd`a` in the second place (\xcd`here.next()`) refers
to the copy.  The copy is
modified and examined.  After the \xcd`at` finishes, the original is also
examined, and (since only the copy, not the original, was modified) is observed
to be unchanged. 
%~~gen ^^^ Places6e1o
% package Places6e1o;
% KNOWNFAIL-at
% class Example { static def example() { 
%~~vis
\begin{xten}
val a = [1,2,3];
at(here.next()) {
  a(1) = 4;
  assert a(0)==1 && a(1)==4 && a(2)==3; 
}
assert  a(0)==1 && a(1)==2 && a(2)==3; 
\end{xten}
%~~siv
%} } 
% class Hook { def run() { Example.example(); return true; }}
%~~neg
\end{ex}

%%AT-COPY%% \subsection{Copy Specifiers}
%%AT-COPY%% \label{sect:copy-spec}
%%AT-COPY%% \index{copy specifier}
%%AT-COPY%% \index{at!copy specifier}
%%AT-COPY%% 
%%AT-COPY%% A single copy specifier can be one of the following forms.   
%%AT-COPY%% Each copy specifier determines an {\em original-expression}, saying what value
%%AT-COPY%% will be copied, and a {\em target variable}, saying what it will be called.
%%AT-COPY%% 
%%AT-COPY%% \begin{itemize}
%%AT-COPY%% 
%%AT-COPY%% \item \xcd`val x = E`, and its usual variants \xcd`val x:T = E`, 
%%AT-COPY%%       \xcd`x : T = E`, and 
%%AT-COPY%%       \xcd`val x <: T = E`, evaluate the expression \xcd`E` at the initial
%%AT-COPY%%       place, copy it to \xcd`p`, and bind \xcd`x` to the copy, as normal for a
%%AT-COPY%%       local \xcd`val` binding.  If a type is supplied, it is checked
%%AT-COPY%%       statically in the usual way.  
%%AT-COPY%%       The original-expression is \xcd`E`, and the target variable is \xcd`x`.
%%AT-COPY%% 
%%AT-COPY%% \begin{ex}
%%AT-COPY%% The following code copies a variable \xcd`a` located \xcd`here` to a variable
%%AT-COPY%% \xcd`d` located \xcd`there`.  
%%AT-COPY%% Note that, while the copy \xcd`d` is available \xcd`there` inside of the \xcd`at`-block,
%%AT-COPY%% the original \xcd`a` is not.  (\xcd`a` could not be available in the block in
%%AT-COPY%% any case; it is not located \xcd`there`.)
%%AT-COPY%% %~~gen ^^^ Places9v2e1
%%AT-COPY%% % package Places9v2e1;
%%AT-COPY%% % KNOWNFAIL-at
%%AT-COPY%% % class Example{ 
%%AT-COPY%% % static def use(Any) = 1;
%%AT-COPY%% % static def example() { 
%%AT-COPY%% %  val there = here.next();
%%AT-COPY%% %~~vis
%%AT-COPY%% \begin{xten}
%%AT-COPY%% var a : Long = 1;
%%AT-COPY%% at(there; val d = a) {
%%AT-COPY%%    assert d == 1;
%%AT-COPY%%    // ERROR: assert a == 1;
%%AT-COPY%% }
%%AT-COPY%% \end{xten}
%%AT-COPY%% %~~siv
%%AT-COPY%% % } } 
%%AT-COPY%% % class Hook{ def run() {Example.example(); return true;}}
%%AT-COPY%% %~~neg
%%AT-COPY%% \end{ex}
%%AT-COPY%% 
%%AT-COPY%% \item \xcd`var x : T = E` evaluates \xcd`E` at the initial place, copies it to
%%AT-COPY%%       \xcd`p`, and binds \xcd`x` to a new \xcd`var` whose initial value is the
%%AT-COPY%%       copy, as normal for a local \xcd`var` binding.
%%AT-COPY%%       If a type is supplied, it is checked
%%AT-COPY%%       statically in the usual way.
%%AT-COPY%%       The original-expression is \xcd`E`, and the target variable is \xcd`x`.
%%AT-COPY%%       Note that, like a \xcd`var` parameter to a method, \xcd`x` is a local
%%AT-COPY%%       variable.  Changes to \xcd`x` will not change anything else. In
%%AT-COPY%%       particular, even if \xcd`x` has the same name as a \xcd`var` variable
%%AT-COPY%%       outside, the two \xcd`var`s are unconnected.  
%%AT-COPY%%       See \Sref{sect:athome} for the way to modify a variable from the
%%AT-COPY%%       surrounding scope.
%%AT-COPY%% 
%%AT-COPY%% \begin{ex}
%%AT-COPY%% The following code copies \xcd`a` to a \xcd`var` named \xcd`e`.  Changing
%%AT-COPY%% \xcd`e` does not change \xcd`a`; the two \xcd`var`s have no ongoing relationship.
%%AT-COPY%% %~~gen ^^^ Places9v2e2
%%AT-COPY%% % package Places9v2e2;
%%AT-COPY%% % KNOWNFAIL-at
%%AT-COPY%% % class Example{ 
%%AT-COPY%% % static def use(Any) = 1;
%%AT-COPY%% % static def example() { 
%%AT-COPY%% %  val there = here.next();
%%AT-COPY%% %~~vis
%%AT-COPY%% \begin{xten}
%%AT-COPY%% var a : Long = 1;
%%AT-COPY%% assert a == 1;
%%AT-COPY%% at(there; var e = a) { 
%%AT-COPY%%    assert e == 1;
%%AT-COPY%%    e += 1;
%%AT-COPY%%    assert e == 2;
%%AT-COPY%% }
%%AT-COPY%% assert a == 1; 
%%AT-COPY%% \end{xten}
%%AT-COPY%% %~~siv
%%AT-COPY%% % 
%%AT-COPY%% % }  } 
%%AT-COPY%% % class Hook{ def run() {Example.example(); return true;}}
%%AT-COPY%% %~~neg
%%AT-COPY%% \end{ex}
%%AT-COPY%% 
%%AT-COPY%% \item \xcd`x = E`, as a copy specifier, is equivalent to \xcd`val x = E`.
%%AT-COPY%%       Note that this abbreviated form is not available as a local variable
%%AT-COPY%%       definition, (because it is used as an assignment statement), but in a
%%AT-COPY%%       copy specifier there are no assignment statements and so the
%%AT-COPY%%       abbreviation is allowed.
%%AT-COPY%%       The original-expression is \xcd`E`, and the target variable is \xcd`x`.
%%AT-COPY%% 
%%AT-COPY%% \begin{ex}
%%AT-COPY%% The following code evaluates an expression \xcd`a+b(0)`.  The result of this
%%AT-COPY%% expression is stored \xcd`there`, in the \xcd`val` variable \xcd`f`, but is
%%AT-COPY%% not stored \xcd`here`. 
%%AT-COPY%% %~~gen ^^^ Places9v2e3
%%AT-COPY%% % package Places9v2e3;
%%AT-COPY%% % KNOWNFAIL-at
%%AT-COPY%% % class Example{ 
%%AT-COPY%% % static def use(Any) = 1;
%%AT-COPY%% % static def example() { 
%%AT-COPY%% %  val there = here.next();
%%AT-COPY%% %~~vis
%%AT-COPY%% \begin{xten}
%%AT-COPY%% var a : Long = 1;
%%AT-COPY%% var b : Rail[Long] = [2,3,4];
%%AT-COPY%% at(there; f = a + b(0)) {
%%AT-COPY%%    assert f == 3;
%%AT-COPY%% }
%%AT-COPY%% \end{xten}
%%AT-COPY%% %~~siv
%%AT-COPY%% % }  } 
%%AT-COPY%% % class Hook{ def run() {Example.example(); return true;}}
%%AT-COPY%% % 
%%AT-COPY%% %~~neg
%%AT-COPY%% 
%%AT-COPY%% 
%%AT-COPY%% \end{ex}
%%AT-COPY%% 
%%AT-COPY%% \item \xcd`x` alone, as a copy specifier, is equivalent to \xcd`val x = x`.
%%AT-COPY%%       It says that the variable \xcd`x` will be copied, and the copy will also
%%AT-COPY%%       be named \xcd`x`.  
%%AT-COPY%%       The original-expression is \xcd`x`, and the target variable is \xcd`x`.
%%AT-COPY%% 
%%AT-COPY%% \begin{ex}
%%AT-COPY%% The following code copies \xcd`b` to \xcd`there`.  The copy is also called
%%AT-COPY%% \xcd`b`.  The two \xcd`b`'s are not connected; \eg, changing one does not
%%AT-COPY%% change the other.
%%AT-COPY%% %~~gen ^^^ Places9v2e4
%%AT-COPY%% % package Places9v2e4;
%%AT-COPY%% % KNOWNFAIL-at
%%AT-COPY%% % class Example{ 
%%AT-COPY%% % static def use(Any) = 1;
%%AT-COPY%% % static def example() { 
%%AT-COPY%% %  val there = here.next();
%%AT-COPY%% %~~vis
%%AT-COPY%% \begin{xten}
%%AT-COPY%% var b : Rail[Long] = [2,3,4];
%%AT-COPY%% assert b(0) == 2;
%%AT-COPY%% at(there; b) {
%%AT-COPY%%   b(0) = 200;  // Modify copy of b.
%%AT-COPY%%   assert b(0) == 200;
%%AT-COPY%% }
%%AT-COPY%% assert b(0) == 2; 
%%AT-COPY%% \end{xten}
%%AT-COPY%% %~~siv
%%AT-COPY%% % 
%%AT-COPY%% % }  } 
%%AT-COPY%% % class Hook{ def run() {Example.example(); return true;}}
%%AT-COPY%% %~~neg
%%AT-COPY%% \end{ex}
%%AT-COPY%% 
%%AT-COPY%% \item A field assignment statements \xcdmath"a.fld = $E_2$", evaluates 
%%AT-COPY%%       \xcd`a` and $E_2$ on the sending side to values $v_1$ and {$v_2$}.  
%%AT-COPY%%       {$v_1$} must be an object with a mutable field \xcd`fld`.  {$v_1$} and
%%AT-COPY%%       {$v_2$} are sent to place \xcd`p`, and the field assignment is performed
%%AT-COPY%%       there.  The modified version of {$v_1$} is available as a \xcd`val`
%%AT-COPY%%       variable \xcd`a`.   The compiler may optimize this, \eg, by neglecting to
%%AT-COPY%%       deserialize \xcdmath"$v_1$.fld", and deserializing {$v_2$} directly into
%%AT-COPY%%       that field rather than into a separate buffer.
%%AT-COPY%% 
%%AT-COPY%% \begin{ex}
%%AT-COPY%% %~~gen ^^^ Places9v2e5
%%AT-COPY%% % package Places9v2e5;
%%AT-COPY%% % KNOWNFAIL
%%AT-COPY%% % class Example {
%%AT-COPY%% % static def use(Any) = 1;
%%AT-COPY%% % static def example() { 
%%AT-COPY%% %  val there = here.next();
%%AT-COPY%% %~~vis
%%AT-COPY%% \begin{xten}
%%AT-COPY%% class Example{ 
%%AT-COPY%%    var f : Long = 1;
%%AT-COPY%%    var g : Long = 2;
%%AT-COPY%%    static def example() { 
%%AT-COPY%%       val there = here.next();
%%AT-COPY%%       val e : Example = new Example();
%%AT-COPY%%       assert e.f == 1 && e.g == 2;
%%AT-COPY%%       at(there; e.f = 3) {
%%AT-COPY%%           assert e.f == 3; && e.g == 2;
%%AT-COPY%%       }
%%AT-COPY%%       assert e.f == 1 && e.g == 2;
%%AT-COPY%%    }
%%AT-COPY%% }
%%AT-COPY%% \end{xten}
%%AT-COPY%% %~~siv
%%AT-COPY%% % class Hook{ def run() {Example.example(); return true;}}
%%AT-COPY%% %~~neg
%%AT-COPY%% %
%%AT-COPY%% \end{ex}
%%AT-COPY%% 
%%AT-COPY%% \item A rail-element assignment 
%%AT-COPY%%       \xcdmath"a($E_1$, $\ldots$, $E_n$) = $E_+$".
%%AT-COPY%%       This copies and transmits \xcd`a` as normal for a rail.  In addition,
%%AT-COPY%%       and 
%%AT-COPY%%       much like a field assignment, it also evaluates all the expressions $E_i$
%%AT-COPY%%       at the sending side to values $v_i$, and transmits them.  \xcd`a`'s value must
%%AT-COPY%%       admit a suitably-typed $n$-ary subscripting operation.  That operation
%%AT-COPY%%       is applied after the values are deserialized at \xcd`p`.  The compiler
%%AT-COPY%%       may optimize this, \eg, by neglecting to deserialize one element of the
%%AT-COPY%%       rail $v_0$, and deserializing $v_+$ directly into that location.  
%%AT-COPY%% 
%%AT-COPY%% 
%%AT-COPY%% \begin{ex}
%%AT-COPY%% The following code sends a modified \xcd`b` to \xcd`there`, while (as always)
%%AT-COPY%% keeping an unmodified version \xcd`here`.   X10 may perform optimizations to
%%AT-COPY%% avoid transmitting the original value of \xcd`b(1)`, since it will be
%%AT-COPY%% overwritten immediately in any case.
%%AT-COPY%% %~~gen ^^^ Places9v2e6
%%AT-COPY%% % package Places9v2e6;
%%AT-COPY%% % KNOWNFAIL
%%AT-COPY%% % class Example{ 
%%AT-COPY%% % static def use(Any) = 1;
%%AT-COPY%% % static def example() { 
%%AT-COPY%% %  val there = here.next();
%%AT-COPY%% %~~vis
%%AT-COPY%% \begin{xten}
%%AT-COPY%% var b = [2,3,4];
%%AT-COPY%% assert b(0) == 2 && b(1) == 3;
%%AT-COPY%% at(there; b(1) = 300) {
%%AT-COPY%%   assert b(0) == 2 && b(1) == 300;
%%AT-COPY%% }
%%AT-COPY%% assert b(0) == 2 && b(1) == 3;
%%AT-COPY%% \end{xten}
%%AT-COPY%% %~~siv
%%AT-COPY%% % 
%%AT-COPY%% %~~neg
%%AT-COPY%% % }  }
%%AT-COPY%% % class Hook{ def run() {Example.example(); return true;}}
%%AT-COPY%% \end{ex}
%%AT-COPY%% 
%%AT-COPY%% \item \xcd`*` may appear as the last copy specifier in the list, indicating
%%AT-COPY%%       that all \xcd`val` variables from outside \xcd`S` which are used in
%%AT-COPY%%       \xcd`S` should be copied. Specifically, let 
%%AT-COPY%%       \xcdmath"x$_1, \ldots, $x$_n$" be all the \xcd`val` variables defined
%%AT-COPY%%       outside of \xcd`S` 
%%AT-COPY%%       mentioned in \xcd`S`. The \xcd`*` copy specifier is equivalent to 
%%AT-COPY%%       the list of variables 
%%AT-COPY%%       \xcdmath"x$_1, \ldots, $x$_n$".
%%AT-COPY%% 
%%AT-COPY%% \begin{ex}
%%AT-COPY%% %~~gen ^^^ Places9v2e7
%%AT-COPY%% % package Places9v2e7;
%%AT-COPY%% % KNOWNFAIL-at
%%AT-COPY%% % class Example{ 
%%AT-COPY%% % static def use(Any) = 1;
%%AT-COPY%% % static def example() { 
%%AT-COPY%% %  val there = here.next();
%%AT-COPY%% %~~vis
%%AT-COPY%% \begin{xten}
%%AT-COPY%% var a : Long = 1;
%%AT-COPY%% val b = [2,3,4];
%%AT-COPY%% at(there; *) {
%%AT-COPY%%   assert a + b(0) == b(1);
%%AT-COPY%% }
%%AT-COPY%% \end{xten}
%%AT-COPY%% %~~siv
%%AT-COPY%% % }  }
%%AT-COPY%% % class Hook{ def run() {Example.example(); return true;}}
%%AT-COPY%% %~~neg
%%AT-COPY%% 
%%AT-COPY%% \end{ex}
%%AT-COPY%% 
%%AT-COPY%% \end{itemize}
%%AT-COPY%% 
%%AT-COPY%% As an important special case, \xcd`at(p;)S` copies {\em nothing} to \xcd`S`.
%%AT-COPY%% This must not be confused with \xcd`at(p)S`, which copies {\em everything}.
%%AT-COPY%% 
%%AT-COPY%% 
%%AT-COPY%% 
%%AT-COPY%% Note that \xcd`at(p;x,*)use(x,y);` is equivalent to \xcd`at(p;*)use(x,y);`.
%%AT-COPY%% In both statements, the \xcd`*` indicates that all variables used in the body
%%AT-COPY%% are to be copied in.  The former makes clear that \xcd`x` is one of the things
%%AT-COPY%% being copied, but, from the \xcd`*`, there may be others. 
%%AT-COPY%% 
%%AT-COPY%% However, other copy specifiers may be used to compute
%%AT-COPY%% values in \xcd`S` which are not available (and thus need not be stored)
%%AT-COPY%% outside of it.  
%%AT-COPY%% 
%%AT-COPY%% \begin{ex}The following code may end up with a large object \xcd`c` in
%%AT-COPY%% memory at \xcd`p` but not at the initial place: 
%%AT-COPY%% %~~gen ^^^ Places3q9u
%%AT-COPY%% % package Places3q9u;
%%AT-COPY%% % KNOWNFAIL-at
%%AT-COPY%% % class Example { 
%%AT-COPY%% % def use(Example, Example, Example) = 1;
%%AT-COPY%% % def Elephant(Example) = 1;
%%AT-COPY%% % static def example(a: Example, b:Example, p:Place) { 
%%AT-COPY%% %~~vis
%%AT-COPY%% \begin{xten}
%%AT-COPY%% at(p; c = a.Elephant(b), *) {
%%AT-COPY%%   use(a,b,c);
%%AT-COPY%% }
%%AT-COPY%% \end{xten}
%%AT-COPY%% %~~siv
%%AT-COPY%% %} } 
%%AT-COPY%% %~~neg
%%AT-COPY%% \end{ex}
%%AT-COPY%% 
%%AT-COPY%% The blanket \xcd`at`-statement \xcd`at(p)S` copies everything.  It is an
%%AT-COPY%% abbreviation for \xcd`at(p;*)S`.  
%%AT-COPY%% When this manual refers to a generic \xcd`at`-statement as \xcd`at(p;F)S`, it
%%AT-COPY%% should be understood as including the blanket \xcd`at` statement \xcd`at(p)S`
%%AT-COPY%% with this interpretation.
%%AT-COPY%% 

\subsection{Copying Values}
%%AT-COPY%% An activity executing statement \xcd"at (q;F) S" at a place \xcd`p`
%%AT-COPY%% evaluates \xcd`q` at \xcd`p` and then moves to \xcd`q` to execute
%%AT-COPY%% \xcd`S`.  
%%AT-COPY%% The original-expressions of \xcd`F` are evaluated at \xcd`p`.
%%AT-COPY%% Their values are copied (\Sref{sect:at-init-val}) to \xcd`q`, and bound to 
%%AT-COPY%% names there, as specified by \xcd`F`.  
%%AT-COPY%% \xcd`S` is evaluated in an environment containing the target variables of
%%AT-COPY%% \xcd`F`, and \xcd`here` and {\em no} other variables.  (In particular, if this
%%AT-COPY%% statement appears in an instance method body and \xcd`this` is not copied,
%%AT-COPY%% \xcd`this` is not accessible.  This fact is important: it allows the
%%AT-COPY%% programmer to control when \xcd`this` is copied, which may be expensive for
%%AT-COPY%% large containers.)

An activity executing \xcd`at(q)S` at a place \xcd`p` evaluates \xcd`q` at
place \xcd`p`, which should be a \xcd`Place`.  It then moves to place \xcd`q`
to execute \xcd`S`.  The values variables that \xcd`S` refers to are copied
(\Sref{sect:at-init-val}) to \xcd`q`, and bound to the variables of the same
name.   If the \xcd`at` is inside of an instance method and \xcd`S` uses
\xcd`this`, \xcd`this` is copied as well.  Note that a field reference
\xcd`this.fld` or a method call \xcd`this.meth()` will cause \xcd`this` to be
copied --- as will their abbreviated forms \xcd`fld` and \xcd`meth()`, despite
the lack of a visible \xcd`this`.  


Note that the value obtained by evaluating \xcd`q`
is not necessarily distinct from \xcd`p` (\eg, \xcd`q` may be
\xcd`here`). 
This does not alter the behavior of \xcd`at`.  
%%AT-COPY%%  \xcd`at(here;F)S` will copy all the values specified by \xcd`F`, 
%%AT-COPY%% even though there is no actual change of place, and even though the original
%%AT-COPY%% values already exist there.
\xcd`at(here)S` will copy all the values mentioned in \xcd`S`, even though
there is no actual change of place, and even though the original values
already exist there. 

On normal termination of \xcd`S` control returns to \xcd`p` and
execution is continued with the statement following 
%%AT-COPY%% \xcd`at (q;F) S`. 
\xcd`at (q) S`. 
If
\xcd`S` terminates abruptly with exception \xcd`E`, \xcd`E` is
serialized into a buffer, the buffer is communicated to \xcd`p` where
it is deserialized into an exception \xcd`E1` and \xcd`at (p) S`
throws \xcd`E1`.

Since 
%%AT-COPY%% \xcd`at(p;F) S` 
\xcd`at(p) S` 
is a synchronous construct, usual control-flow
constructs such as \xcd`break`, \xcd`continue`, \xcd`return` and 
\xcd`throw` are permitted in \xcd`S`.  All concurrency related
constructs -- \xcd`async`, \xcd`finish`, \xcd`atomic`, \xcd`when` are
also permitted.

The \xcd`at`-expression 
%%AT-COPY%% \xcd`at(p;F)E` 
\xcd`at(p)E` 
is similar, except that, in the case of
normal termination of \xcd`E`, the value that \xcd`E` produces is serialized
into a buffer, transported to the starting place, and deserialized, and the
value of the \xcd`at`-expression is the result of deserialization.

\limitation{
X10 does not currently allow {\tt break}, {\tt continue}, or {\tt return}
to exit from an {\tt at}.
}



\subsection{How {\tt at} Copies Values}
\label{sect:at-init-val}

%%AT-COPY%% The values of the original-expressions  specified by \xcd`F` in 
%%AT-COPY%% \xcd`at (p;F)S` are copied to \xcd`p`, as follows.

The values mentioned in \xcd`S` are copied to place \xcd`p` by \xcd`at(p)S` as follows.

First, the original-expressions are evaluated to give a vector of X10 values.
Consider the graph of all values reachable from these values (except for 
\xcd`transient` fields 
(\Sref{sect:transient}, \xcd`GlobalRef`s (\Sref{GlobalRef}); also custom
serialization (\Sref{sect:ser+deser} may alter this behavior)). 

Second this graph is {\em
serialized} into a buffer and transmitted to place \xcd`q`.  Third,
the vector of X10 values is 
re-created at \xcd`q` 
by deserializing the buffer at
\xcd`q`. Fourth, \xcd`S` is executed at \xcd`q`, in an environment in
which each variable \xcd`v` declared in \xcd`F` 
refers to the corresponding deserialized value.  

Note that since values accessed across an \xcd`at` boundary are
copied, the programmer may wish to adopt the discipline that either
variables accessed across an \xcd`at` boundary  contain only structs 
or stateless objects, or the methods invoked on them do not access any
mutable state on the objects. Otherwise the programmer has to ensure
that side effects are made to the correct copy of the object. For this
the struct \xcd`x10.lang.GlobalRef[T]` is often useful.


\subsubsection{Serialization and deserialization.}
\label{sect:ser+deser}
\index{transient}
\index{field!transient}
The X10 runtime provides a default mechanism for
serializing/deserializing an object graph with a given set of roots.
This mechanism may be overridden by the programmer on a per class or
struct basis as described in the API documentation for
\xcd`x10.io.CustomSerialization`.  
The default mechanism performs a
deep copy of the object graph (that is, it copies the object or struct
and, recursively, the values contained in its fields), but does not
traverse or copy \xcd`transient` fields. \xcd`transient` fields are omitted from the
serialized data.   On deserialization, \xcd`transient` fields are initialized
with their default values (\Sref{DefaultValues}).    The types of
\xcd`transient` fields must therefore have default values.

The default serialization/deserialization mechanism will not (modulo
error conditions like \xcd`OutOfMemoryError`) throw any exceptions. However,
user code running during serialization/deserialization via
\xcd`CustomSerialization` may raise exceptions.  These exceptions are
handled like any other exception raised during the execution of an X10
activity.  However, due to the special treatment of \xcd`at (p) async S` 
(\Sref{sect:AtStatement}) any exception raised during
deserialization will be handled as if it was raised by \xcd`async S`,
not by the \xcd`at` statement itself.


A struct \xcd`s` of type \xcd`x10.lang.GlobalRef[T]` \ref{GlobalRef}
is serialized as a unique global reference to its contained object
\xcd`o` (of type \xcd`T`).  Please see the documentation
of \xcd`x10.lang.GlobalRef[T]` for more details.



\subsection{{\tt at} and Activities}
%%AT-COPY%% \xcd`at(p;F)S` 
\xcd`at(p)S` 
does {\em not} start a new activity.  It should be thought of as
transporting the current activity to \xcd`p`, running \xcd`S` there, and then
transporting it back.  \xcd`async` is the only construct in the
language that starts a new activity. In different contexts, each one
of the following makes sense:
%%AT-COPY%% (1)~\xcd`async at(p;F) S` 
(1)~\xcd`async at(p) S` 
(spawn an activity locally to execute \xcd`S` at
\xcd`p`; here \xcd`p` is evaluated by the spawned activity) , 
%%AT-COPY%% (2)~\xcd`at(p;F) async S` 
(2)~\xcd`at(p) async S` 
(evaluate \xcd`p` and then at \xcd`p` spawn an
activity to execute \xcd`S`), and,
%%AT-COPY%% (3)~\xcd`async at(p;F) async S`. 
(3)~\xcd`async at(p) async S`. 
%%AT-COPY%% In most cases, \xcd`at(p;F) async S` is preferred to
%%\xcd`async at(p;F)`, since In most cases, \xcd`at(p) async S` is
preferred to \xcd`async at(p) S`, since the former form enables a more
efficient runtime implementation.  In the first case, the expression
\xcd`p` is evaluated synchronously by the current activity and then a
single remote async is spawned.  In the second case, \xcd`p` is
semantically required to be evaluated asynchronously with the parent
async as it is contained in the body of an async.  Therefore, if the
compiler cannot prove that "async at (p)" can be safely rewritten into
"at (p) async", a first local async is spawned to evaluate \xcd`p`
then a remote async is spawned to evaluate \xcd`S`.

Since 
%%AT-COPY%% \Xcd{at(p;F) S} 
\Xcd{at(p) S} 
does not start a new activity, 
\xcd`S` may contain constructs which only make sense
within a single activity.  
For example, 
\begin{xten}
    for(x in globalRefsToThings) 
      if (at(x.home) x().isNice()) 
        return x();
\end{xten}
returns the first nice thing in a collection.   If we had used 
\xcd`async at(x.home)`, this would not be allowed; 
you can't \xcd`return` from an
\xcd`async`. 

\limitation{
X10 does not currently allow {\tt break}, {\tt continue}, or {\tt return}
to exit from an {\tt at}.
}



\subsection{Copying from {\tt at} }
\index{at!copying}

%%AT-COPY%% \xcd`at(p;F)S` copies data as specified by \xcd`F`, and sends it
\xcd`at(p)S` copies data required in \xcd`S`, and sends it
to place \xcd`p`, before executing \xcd`S` there. The only things that are not
copied are values only reachable through \xcd`GlobalRef`s and \xcd`transient`
fields, and data omitted by custom serialization.    
%%AT-COPY%% Several choices of copy specifier use the same identifier for the original
%%AT-COPY%% variable outside of 
%%AT-COPY%% \xcd`at(p)S` 
%%AT-COPY%% and its copy inside of \xcd`S`.  
%%AT-COPY%% 

\begin{ex}
%%AT-COPY%% 
%%AT-COPY%% %~~gen ^^^ Places_implicit_copy_from_at_example_1
%%AT-COPY%% % package Places.implicitcopyfromat;
%%AT-COPY%% % class Example {
%%AT-COPY%% % static def example() {
%%AT-COPY%% % 
%%AT-COPY%% %~~vis
%%AT-COPY%% \begin{xten}
%%AT-COPY%% val c = new Cell[Long](9); // (1)
%%AT-COPY%% at (here;c) {             // (2)
%%AT-COPY%%    assert(c() == 9);      // (3)
%%AT-COPY%%    c.set(8);              // (4)
%%AT-COPY%%    assert(c() == 8);      // (5)
%%AT-COPY%% }
%%AT-COPY%% assert(c() == 9);         // (6)
%%AT-COPY%% \end{xten}
%%AT-COPY%% %~~siv
%%AT-COPY%% %}}
%%AT-COPY%% % class Hook{ def run() { Example.example(); return true; } }
%%AT-COPY%% %~~neg
%%AT-COPY%% 

%~~gen ^^^ Places_implicit_copy_from_at_example_1
% package Places.implicitcopyfromat;
% class Example {
% static def example() {
% 
%~~vis
\begin{xten}
val c = new Cell[Long](9); // (1)
at (here) {               // (2) 
   assert(c() == 9);      // (3)
   c.set(8);              // (4)
   assert(c() == 8);      // (5)
}
assert(c() == 9);         // (6)
\end{xten}
%~~siv
%}}
% class Hook{ def run() { Example.example(); return true; } }
%~~neg


The \xcd`at` statement copies the \xcd`Cell` and its contents.  
After \xcd`(1)`, \xcd`c` is a \xcd`Cell` containing 9; call that cell {$c_1$}
At \xcd`(2)`, that cell is copied, resulting in another cell {$c_2$} whose
contents are also 9, as tested at \xcd`(3)`.
(Note that the copying behavior of \xcd`at` happens {\em even when the
destination place is the same as the starting place}--- even with
\xcd`at(here)`.)
At \xcd`(4)`, the contents of {$c_2$} are changed to 8, as confirmed at \xcd`(5)`; the contents of
{$c_1$} are of course untouched.    Finally, at \xcd`(c)`, outside the scope
of the \xcd`at` started at line \xcd`(2)`, \xcd`c` refers to its original
value {$c_1$} rather than the copy {$c_2$}.  
\end{ex}

The \xcd`at` statement induces a {\em deep copy}.  Not only does it copy the
values of variables, it copies values that they refer to through zero or more
levels of reference.  Structures are preserved as well: if two fields
\xcd`x.f` and \xcd`x.g` refer to the same object {$o_1$} in the original, then
\xcd`x.f` and \xcd`x.g` will both refer to the same object {$o_2$} in the
copy.  

\begin{ex}
In the following variation of the preceding example,
\xcd`a`'s original value {$a_1$} is a rail with two references to the same
\xcd`Cell[Long]` {$c_1$}.  The fact that {$a_1(0)$} and {$a_1(1)$} are both
identical to {$c_1$} is demonstrated in \xcd`(A)`-\xcd`(C)`, as {$a_1(0)$} is modified
and {$a_1(1)$} is observed to change.  In \xcd`(D)`-\xcd`(F)`, the copy
{$a_2$} is tested in the same way, showing that {$a_2(0)$} and {$a_2(1)$} both
refer to the same \xcd`Cell[Long]` {$c_2$}.  However, the test at \xcd`(G)`
shows that {$c_2$} is a different cell from {$c_1$}, because changes to
{$c_2$} did not propagate to {$c_1$}.  

%%AT-COPY%% %~~gen ^^^ PlacesAtCopy
%%AT-COPY%% %package Places.AtCopy2;
%%AT-COPY%% %class example {
%%AT-COPY%% %static def Example() {
%%AT-COPY%% %
%%AT-COPY%% %~~vis
%%AT-COPY%% \begin{xten}
%%AT-COPY%% val c = new Cell[Long](5);
%%AT-COPY%% val a : Rail[Cell[Long]] = [c,c as Cell[Long]];
%%AT-COPY%% assert(a(0)() == 5 && a(1)() == 5);     // (A)
%%AT-COPY%% c.set(6);                               // (B)
%%AT-COPY%% assert(a(0)() == 6 && a(1)() == 6);     // (C)
%%AT-COPY%% at(here;a) {
%%AT-COPY%%   assert(a(0)() == 6 && a(1)() == 6);   // (D)
%%AT-COPY%%   c.set(7);                             // (E)
%%AT-COPY%%   assert(a(0)() == 7 && a(1)() == 7);   // (F)
%%AT-COPY%% }
%%AT-COPY%% assert(a(0)() == 6 && a(1)() == 6);     // (G)
%%AT-COPY%% \end{xten}
%%AT-COPY%% %~~siv
%%AT-COPY%% %}}
%%AT-COPY%% %class Hook{ def run() { example.Example(); return true; } }
%%AT-COPY%% %~~neg

%~~gen ^^^ PlacesAtCopy
%package Places.AtCopy2;
%class example {
%static def Example() {
%
%~~vis
\begin{xten}
val c = new Cell[Long](5);
val a : Rail[Cell[Long]] = [c,c as Cell[Long]];
assert(a(0)() == 5 && a(1)() == 5);     // (A)
c.set(6);                               // (B)
assert(a(0)() == 6 && a(1)() == 6);     // (C)
at(here) {
  assert(a(0)() == 6 && a(1)() == 6);   // (D)
  c.set(7);                             // (E)
  assert(a(0)() == 7 && a(1)() == 7);   // (F)
}
assert(a(0)() == 6 && a(1)() == 6);     // (G)
\end{xten}
%~~siv
%}}
%class Hook{ def run() { example.Example(); return true; } }
%~~neg


\end{ex}

\subsection{Copying and Transient Fields}
\label{sect:transient}
\index{at!transient fields and}
\index{transient}
\index{field!transient}

Recall that fields of classes and structs marked \xcd`transient` are not copied by
\xcd`at`.  Instead, they are set to the default values for their types. Types
that do not have default values cannot be used in \xcd`transient` fields.

\begin{ex}
Every \xcd`Trans` object has an \xcd`a`-field equal
to 1.  However, despite the initializer on the \xcd`b` field, it is not the
case that every \xcd`Trans` has \xcd`b==2`.  Since \xcd`b` is \xcd`transient`,
when the \xcd`Trans` value \xcd`this` is copied at \xcd`at(here){...}` in
\xcd`example()`, its \xcd`b` field is not copied, and the default value for an
\xcd`Long`, 0, is used instead.  
Note that we could not make a transient field \xcd`c : Long{c != 0}`, since the
type has no default value, and copying would in fact set it to zero.

%%AT-COPY%% %~~gen ^^^ Places40
%%AT-COPY%% %package Places_transient_a;
%%AT-COPY%% % 
%%AT-COPY%% %~~vis
%%AT-COPY%% \begin{xten}
%%AT-COPY%% class Trans {
%%AT-COPY%%    val a : Long = 1;
%%AT-COPY%%    transient val b : Long = 2;
%%AT-COPY%%    //ERROR transient val c : Long{c != 0} = 3;
%%AT-COPY%%    def example() {
%%AT-COPY%%      assert(a == 1 && b == 2);
%%AT-COPY%%      at(here;a) {
%%AT-COPY%%         assert(a == 1 && b == 0);
%%AT-COPY%%      }
%%AT-COPY%%    }
%%AT-COPY%% }
%%AT-COPY%% \end{xten}
%%AT-COPY%% %~~siv
%%AT-COPY%% %class Hook{ def run() { (new Trans()).example(); return true; } }
%%AT-COPY%% %~~neg

%~~gen ^^^ Places40
%package Places_transient_a;
% 
%~~vis
\begin{xten}
class Trans {
   val a : Long = 1;
   transient val b : Long = 2;
   //ERROR: transient val c : Long{c != 0} = 3;
   def example() {
     assert(a == 1 && b == 2);
     at(here) {
        assert(a == 1 && b == 0);
     }
   }
}
\end{xten}
%~~siv
%class Hook{ def run() { (new Trans()).example(); return true; } }
%~~neg



\end{ex}

\subsection{Copying and GlobalRef}
\label{GlobalRef}
\index{at!GlobalRef}
\index{at!blocking copying}

%%The other barrier to the potentially copious copying behavior of \xcd`at`
%%is the \xcd`GlobalRef` struct.  
A \xcd`GlobalRef[T]` (say \xcd`g`) contains a reference to
a value \xcd`v` of type \xcd`T`, in a form which can be transmitted, and a \xcd`Place`
\xcd`g.home` indicating where the value lives. When a 
\xcd`GlobalRef` is serialized an opaque, globally unique handle to
\xcd`v` is created.  

\begin{ex}The following example does not copy the value \xcd`huge`.  However, \xcd`huge`
would have been copied if it had been put into a \xcd`Cell`, or simply used
directly. 

%%AT-COPY%% %~~gen ^^^ Places50
%%AT-COPY%% %package Places.copyingblockingwithglobref;
%%AT-COPY%% % class GR {
%%AT-COPY%% %  static def use(Any){}
%%AT-COPY%% %  static def example() {
%%AT-COPY%% % 
%%AT-COPY%% %~~vis
%%AT-COPY%% \begin{xten}
%%AT-COPY%% val huge = "A potentially big thing";
%%AT-COPY%% val href = GlobalRef(huge);
%%AT-COPY%% at (here;href) {
%%AT-COPY%%    use(href);
%%AT-COPY%%   }
%%AT-COPY%% }
%%AT-COPY%% \end{xten}
%%AT-COPY%% %~~siv
%%AT-COPY%% %}
%%AT-COPY%% % class Hook{ def run() { GR.example(); return true; } }
%%AT-COPY%% %~~neg

%~~gen ^^^ Places50
%package Places.copyingblockingwithglobref;
% class GR {
%  static def use(Any){}
%  static def example() {
% 
%~~vis
\begin{xten}
val huge = "A potentially big thing";
val href = GlobalRef(huge);
at (here) {
   use(href);
  }
}
\end{xten}
%~~siv
%}
% class Hook{ def run() { GR.example(); return true; } }
%~~neg


\end{ex}

Values protected in \xcd`GlobalRef`s can be retrieved by the application
%~~exp~~`~~`~~ g:GlobalRef[Any]{here == g.home}~~ ^^^Places4e7q
operation \xcd`g()`.  \xcd`g()` is guarded; it can 
only be called when \xcd`g.home == here`.  If you  want to do anything other
than pass a global reference around or compare two of them for equality, you
need to placeshift back to the home place of the reference, often with
\xcd`at(g.home)`.   

\begin{ex}The following program, for reasons best known to the programmer,
modifies the 
command-line argument array.

%%AT-COPY%% 
%%AT-COPY%% %~~gen ^^^ Places60
%%AT-COPY%% % package Places.Atsome.Globref2;
%%AT-COPY%% % class GR2 {
%%AT-COPY%% % 
%%AT-COPY%% %~~vis
%%AT-COPY%% \begin{xten}
%%AT-COPY%%   public static def main(argv:Rail[String]) {
%%AT-COPY%%     val argref = GlobalRef[Rail[String]](argv);
%%AT-COPY%%     at(here.next(); argref) 
%%AT-COPY%%         use(argref);
%%AT-COPY%%   }
%%AT-COPY%%   static def use(argref : GlobalRef[Rail[String]]) {
%%AT-COPY%%     at(argref.home; argref) {
%%AT-COPY%%       val argv = argref();
%%AT-COPY%%       argv(0) = "Hi!";
%%AT-COPY%%     }
%%AT-COPY%%   }
%%AT-COPY%% \end{xten}
%%AT-COPY%% %~~siv
%%AT-COPY%% %} 
%%AT-COPY%% % class Hook{ def run() { GR2.main(["what, me weasel?" as String]); return true; }}
%%AT-COPY%% %~~neg
%%AT-COPY%% 

%~~gen ^^^ Places60
% package Places.Atsome.Globref2;
% class GR2 {
% 
%~~vis
\begin{xten}
  public static def main(argv:Rail[String]) {
    val argref = GlobalRef[Rail[String]](argv);
    at(here.next()) 
        use(argref);
  }
  static def use(argref : GlobalRef[Rail[String]]) {
    at(argref) {
      val argv = argref();
      argv(0) = "Hi!";
    }
  }
\end{xten}
%~~siv
%} 
% class Hook{ def run() { GR2.main(["what, me weasel?" as String]); return true; }}
%~~neg

\end{ex}

There is an implicit coercion from \xcd`GlobalRef[T]` to \xcd`Place`, so
\xcd`at(argref)S` goes to \xcd`argref.home`.  


\subsection{Warnings about \xcd`at`}
There are two dangers involved with \xcd`at`: 
\begin{itemize}
\item Careless use of \xcd`at` can result in copying and transmission
of very large data structures.  
%%AT-COPY%% This is particularly an issue with the blanket
%%AT-COPY%% \xcd`at` statement, \xcd`at(p)S`, where everything used in \xcd`S` is copied.  
In particular, it is very easy to capture
\xcd`this` -- a field reference will do it -- and accidentally copy everything
that \xcd`this` refers to, which can be very large.  A disciplined use of copy
specifiers to make explicit just what gets copied can ameliorate this issue.

\item As seen in the examples above, a local variable reference
  \xcd`x` may refer to different objects in different nested \xcd`at`
  scopes. The programmer must either ensure that a variable accessed
  across an \xcd`at` boundary has no mutable state or be prepared to
  reason about which copy gets modified.   A disciplined use of copy specifiers to give
  different names to variables can ameliorate this concern.
\end{itemize}


%%AT-COPY%% \section{{\tt athome}: Returning Values from {\tt at}-Blocks}
%%AT-COPY%% \label{sect:athome}
%%AT-COPY%% \index{athome}
%%AT-COPY%% 
%%AT-COPY%% The 
%%AT-COPY%% \xcd`at(p;F)S` 
%%AT-COPY%% construct renders external variables unavailable within
%%AT-COPY%% \xcd`S`.  However, it is often useful to transmit values back from \xcd`S`,
%%AT-COPY%% and store them in external variables. 
%%AT-COPY%% 
%%AT-COPY%% The \xcd`athome(V;F)S` construct provides
%%AT-COPY%% this ability.  \xcd`V` is a list of variables, which must all be defined at
%%AT-COPY%% the same place.  \xcd`athome(V;F)S` goes to the place where the variables are
%%AT-COPY%% defined, copying \xcd`F` as for \xcd`at(p;F)S`, and executes \xcd`S` ---
%%AT-COPY%% allowing reading, assignment and initialization of the listed variables in
%%AT-COPY%% \xcd`V`. 
%%AT-COPY%% 
%%AT-COPY%% \xcd`V`, the list of variables, may include one or more variables.  It is a
%%AT-COPY%% static error if X10 cannot determine that all the variables in the list are
%%AT-COPY%% defined at the same place.
%%AT-COPY%% 
%%AT-COPY%% 
%%AT-COPY%% 
%%AT-COPY%% 
%%AT-COPY%% \begin{ex}
%%AT-COPY%% \xcd`athome` allows returning multiple pieces of information from an
%%AT-COPY%% \xcd`at`-statement.  In the following example, we return two data: 
%%AT-COPY%% one as a \xcd`val` named \xcd`square`, and the other as an addition in to a
%%AT-COPY%% partially-computed polynomial named \xcd`poly`.  
%%AT-COPY%% %~~gen ^^^ Places5f9g
%%AT-COPY%% % package Places5f9g;
%%AT-COPY%% % % KNOWNFAIL-at
%%AT-COPY%% % class Example { 
%%AT-COPY%% %~~vis
%%AT-COPY%% \begin{xten}
%%AT-COPY%% static def example(a: Long, mathProc: Place) { 
%%AT-COPY%%   val square : Long;
%%AT-COPY%%   var poly : Long = 1 + a; // will be 1+a+a*a
%%AT-COPY%%   at(mathProc; a) {
%%AT-COPY%%     val sq = a*a; 
%%AT-COPY%%     athome(square, poly; sq) {
%%AT-COPY%%        square = sq;  // initialization
%%AT-COPY%%        poly += sq;   // read and update
%%AT-COPY%%     }
%%AT-COPY%%   return [square, poly];
%%AT-COPY%%   }
%%AT-COPY%% \end{xten}
%%AT-COPY%% %~~siv
%%AT-COPY%% %}}
%%AT-COPY%% % class Hook { def run() { 
%%AT-COPY%% %   val e = example(2, here);
%%AT-COPY%% %   assert e(0) == 4 && e(1) == 7;
%%AT-COPY%% %   return true;
%%AT-COPY%% % }} 
%%AT-COPY%% %~~neg
%%AT-COPY%% \end{ex}
%%AT-COPY%% 
%%AT-COPY%% The abbreviated forms 
%%AT-COPY%% \xcd`athome (*) S` and 
%%AT-COPY%% \xcd`athome S` 
%%AT-COPY%% allow a block of assignments without specifying the variables being assigned
%%AT-COPY%% to, which is convenient for a small set of assignments. 
%%AT-COPY%% They 
%%AT-COPY%% are both equivalent to \xcd`athome(V;F)S`,
%%AT-COPY%% where: 
%%AT-COPY%% \begin{itemize}
%%AT-COPY%% \item \xcd`V` is the list of all variables appearing on the left-hand side of
%%AT-COPY%%       an assignment or update statement in \xcd`S`, excluding those which
%%AT-COPY%%       appear inside the body of an \xcd`at` or \xcd`athome` statement in \xcd`S`;
%%AT-COPY%% \item \xcd`F` is the same as for \xcd`at(p)S` (\Sref{sect:copy-spec})
%%AT-COPY%% \end{itemize}
%%AT-COPY%% 
%%AT-COPY%% 
%%AT-COPY%% \begin{ex}
%%AT-COPY%% 
%%AT-COPY%% Much as the blanket \xcd`at` construct \xcd`at(p)S` is convenient for
%%AT-COPY%% executing a small code body at another place, the blanket \xcd`athome`
%%AT-COPY%% construct \xcd`athome(*) S` 
%%AT-COPY%% (which may be written as simply \xcd`athome S`)
%%AT-COPY%% is convenient for returning a result or two.   The
%%AT-COPY%% preceding example could have been written using blanket statements.
%%AT-COPY%% 
%%AT-COPY%% %~~gen ^^^ Places5f9gblanket
%%AT-COPY%% % package Places5f9gblanket;
%%AT-COPY%% % class Example { 
%%AT-COPY%% % KNOWNFAIL-at
%%AT-COPY%% %~~vis
%%AT-COPY%% \begin{xten}
%%AT-COPY%% static def example(a: Long, mathProc: Place) { 
%%AT-COPY%%   val square : Long;
%%AT-COPY%%   var poly : Long = 1 + a; // will be 1+a+a*a
%%AT-COPY%%   at(mathProc) {
%%AT-COPY%%     val sq = a*a; 
%%AT-COPY%%     athome {
%%AT-COPY%%        square = sq;  // initialization
%%AT-COPY%%        poly += sq;   // read and update
%%AT-COPY%%     }
%%AT-COPY%%   return [square, poly];
%%AT-COPY%%   }
%%AT-COPY%% \end{xten}
%%AT-COPY%% %~~siv
%%AT-COPY%% %}}
%%AT-COPY%% % class Hook { def run() { 
%%AT-COPY%% %   val e = example(2, here);
%%AT-COPY%% %   assert e(0) == 4 && e(1) == 7;
%%AT-COPY%% %   return true;
%%AT-COPY%% % }} 
%%AT-COPY%% %~~neg
%%AT-COPY%% \end{ex}
%%AT-COPY%% 
%%AT-COPY%% {\bf Design:} It is not fundamentally essential to distinguish \xcd`at` from
%%AT-COPY%% \xcd`athome`.  \xcd`at(p;F)S` could allow writing to variables whose homes are
%%AT-COPY%% known at compile-time to be equal to \xcd`p`.  Indeed, in earlier versions of
%%AT-COPY%% X10, it did so.    This required an idiom in which programmers had to manage
%%AT-COPY%% the home locations of variables directly, and keep track of which home
%%AT-COPY%% location corresponded to which variable.  The \xcd`athome` construct makes
%%AT-COPY%% this idiom more convenient. 
	
\chapter{Activities}\label{XtenActivities}

An \Xten{} computation may have many concurrent {\em activities} ``in
flight'' at any give time. We use the term activity to denote a
dynamic execution instance of a piece of code (with references to
data). An activity is intended to execute in parallel with other
activities. An activity may be thought of as a very light-weight
thread.  In \XtenCurrVer{}, an activity may not be interrupted,
suspended or resumed as the result of actions taken by any other
activity.

An activity is spawned in a given place and stays in that place for
its lifetime.  An activity may be {\em running}, {\em blocked} on some
condition or {\em terminated}. When the statement associated with an
activity terminates normally, the activity terminates normally; when
it terminates abruptly with some reason $R$, the activity terminates
with the same reason (\Sref{ExceptionModel}).

An activity may be long-running and may invoke recursive methods (thus
may have a stack associated with it). On the other hand, an activity
may be short-running, involving a fine-grained operation such as a
single read or write.

% An activity may have an {\em activitylocal} heap accessible only
%to the activity. 

An activity may asynchronously and in parallel launch activities at
other places.

\Xten{} distinguishes between {\em local} termination and {\em global}
termination of a statement. The execution of a statement by an
activity is said to terminate locally when the activity has finished
all its computation related to that statement. (For instance the
creation of an asynchronous activity terminates locally when the
activity has been created.)  It is said to terminate globally when it
has terminated locally and all activities that it may have spawned at
any place (if any) have, recursively, terminated globally.

An \Xten{} computation is initiated as a single activity from the
command line. This activity is the {\em root activity}\index{root
activity} for the entire computation. The entire computation
terminates when (and only when) this activity globally
terminates. Thus \Xten{} does not permit the creation of so called
``daemon threads''---threads that outlive the lifetime of the root
activity. We say that an \Xten{} computation is {\em rooted}
(\Sref{initial-computation}).

\futureext{ We may permit the initial activity to be a daemon activity
to permit reactive computations, such as webservers, that may not
terminate.}

\section{The \Xten{} rooted exception model}
\label{ExceptionModel}
\index{Exception!model}

The rooted nature of \Xten{} computations permits the definition of a
{\em rooted} exception model. In multi-threaded programming languages
there is a natural parent-child relationship between a thread and a
thread that it spawns. Typically the parent thread continues execution
in parallel with the child thread. Therefore the parent thread cannot
serve to catch any exceptions thrown by the child thread. 

The presence of a root activity permits \Xten{} to adopt a different
model.  In any state of the computation, say that an activity $A$ is
{\em a root of} an activity $B$ if $A$ is an ancestor of $B$ and $A$
is suspended at a statement (such as the \xcd"finish" statement
\Sref{finish}) awaiting the termination of $B$ (and possibly other
activities). For every \Xten{} computation, the
\emph{root-of} relation
is guaranteed to be a tree. The root of the tree is the root activity
of the entire computation. If $A$ is the nearest root of $B$, the path
from $A$ to $B$ is called the {\em activation path} for the
activity.\footnote{Note that depending on the state of the computation
the activation path may traverse activities that are running,
suspended or terminated.}

We may now state the exception model for \Xten.  An uncaught exception
propagates up the activation path to its nearest root activity, where
it may be handled locally or propagated up the \emph{root-of} tree when
the activity terminates (based on the semantics of the statement being
executed by the activity).\footnote{In \XtenCurrVer{} the \xcd"finish"
statement is the only statement that marks its activity as a root
activity. Future versions of the language may introduce more such
statements.}  Thus, unlike concurrent languages such as \java{}, no
exception is ``thrown on the floor''.

\section{Spawning an activity}\label{AsynchronousActivity}\label{AsyncActivity}

Asynchronous activities serve as a single abstraction for supporting a
wide range of concurrency constructs such as message passing, threads,
DMA, streaming, data prefetching. (In general, asynchronous operations
are better suited for supporting scalability than synchronous
operations.)

An activity is created by executing the statement:

\begin{grammar}
Statement \: AsyncStatement \\
AsyncStatement \: \xcd"async" PlaceExpressionSingleList\opt Statement \\
PlaceExpressionSingleList \: \xcd"(" PlaceExpression \xcd")" \\
PlaceExpression \: Expression 
\end{grammar} 

The place expression \xcd"e" is expected to be of type \xcd"Place",
e.g., \xcd"here" or \xcd"d(p)" for some
distribution \xcd"d" and point \xcd"p" (\Sref{XtenPlaces}).  
If not, the compiler replaces
\xcd"e" with \xcd"e.home" if
\xcd"e" is of type \xcd"x10.lang.Object". Otherwise the compiler reports a type error. 

Note specifically that the expression \xcd"a(i)" when used as a place
expression may evaluate to \xcd"a(i).home", which may not be
the same place as \xcd"a.dist(i)". The programmer must be 
careful to choose the right expression, appropriate for the statement.
Accesses to \xcd"a(i)" within \grammarrule{Statement} should typically be guarded 
by the place expression \xcd"a.dist(i)".

In many cases the compiler may infer the unique place at which the
statement is to be executed by an analysis of the types of the
variables occurring in the statement. (The place must be such that the
statement can be executed safely, without generating a
\xcd"BadPlaceException".) In such cases the programmer may omit the
place designator; the compiler will throw an error if it cannot
determine the unique designated place.\footnote{\XtenCurrVer{} does
not specify a particular algorithm; this will be fixed in future
versions.}

An activity $A$ executes the statement \xcd"async (P) S" by launching
a new activity $B$ at the designated place, to execute the specified
statement. The statement terminates locally as soon as $B$ is
launched.  The activation path for $B$ is that of $A$, augmented with
information about the line number at which $B$ was spawned.  $B$
terminates normally when $S$ terminates normally.  It terminates
abruptly if $S$ throws an (uncaught) exception. The exception is
propagated to $A$ if $A$ is a root activity (see \Sref{finish}),
otherwise through $A$ to $A$'s root activity. Note that while an
activity is running, exceptions thrown by activities it has already
generated may propagate through it up to its root activity.

Multiple activities launched by a single activity at another place are
not ordered in any way. They are added to the pool of activities at
the target place and will be executed in sequence or in parallel based
on the local scheduler's decisions. If the programmer wishes to
sequence their execution s/he must use \Xten{} constructs, such as
clocks and \xcd"finish" to obtain the desired effect.  Further, the
\Xten{} implementations are not required to have fair schedulers,
though every implementation should make a best faith effort to ensure
that every activity eventually gets a chance to make forward progress.

\begin{staticrule*}
The statement in the body of an \xcd"async" is subject to the
restriction that it must be acceptable as the body of a \xcd"void"
method for an anonymous inner class declared at that point in the code,
which throws no checked exceptions. As such, it may reference
variables in lexically enclosing scopes (including \xcd"clock"
variables, \Sref{XtenClocks}) provided that such variables are
(implicitly or explicitly) \xcd"val".
\end{staticrule*}

\section{Place changes}\label{AtStatement}

An activity may change place using the \xcd"at" statement or
\xcd"at" expression:

\begin{grammar}
Statement \: AtStatement \\
AtStatement \: \xcd"at" PlaceExpressionSingleList Statement \\
Expression \: AtExpression \\
AtExpression \: \xcd"at" PlaceExpressionSingleList ClosureBody 
\end{grammar}

The statement \xcd"at (p) S" executes the statement \xcd"S"
synchronously at place \xcd"p".
The expression \xcd"at (p) E" executes the statement \xcd"E"
synchronously at place \xcd"p", returning the result to the
originating place.

\section{Finish}\index{finish}\label{finish}
The statement \xcd"finish S" converts global termination to local
termination and introduces a root activity. 

\begin{grammar}
Statement \: FinishStatement \\
FinishStatement \: \xcd"finish" Statement 
\end{grammar}

An activity $A$ executes \xcd"finish S" by executing \xcd"S".  The
execution of \xcd"S" may spawn other asynchronous activities (here or
at other places).  Uncaught exceptions thrown or propagated by any
activity spawned by \xcd"S" are accumulated at \xcd"finish S".
\xcd"finish S" terminates locally when all activities spawned by
\xcd"S" terminate globally (either abruptly or normally). If \xcd"S"
terminates normally, then \xcd"finish S" terminates normally and $A$
continues execution with the next statement after \xcd"finish S".  If
\xcd"S" terminates abruptly, then \xcd"finish S" terminates abruptly
and throws a single exception, \Xcd{x10.lang.MultipleExceptions}
formed from the collection of exceptions accumulated at \xcd"finish S".

Thus a \xcd"finish S" statement serves as a collection point for
uncaught exceptions generated during the execution of \xcd"S".

Note that repeatedly \xcd"finish"ing a statement has no effect after
the first \xcd"finish": \xcd"finish finish S" is indistinguishable
from \xcd"finish S".

\paragraph{Interaction with clocks.}\label{sec:finish:clock-rule}
\xcd"finish S" interacts with clocks (\Sref{XtenClocks}). 

While executing \xcd"S", an activity must not spawn any \xcd"clocked"
asyncs. (Asyncs spawned during the execution of \xcd"S" may spawn
clocked asyncs.) A
\xcd"ClockUseException"\index{clock!ClockUseException} is thrown
if (and when) this condition is violated.

In \XtenCurrVer{} this condition is checked dynamically; future
versions of the language will introduce type qualifiers which permit
this condition to be checked statically.

\futureext{
The semantics of \xcd"finish S" is conjunctive; it terminates when all
the activities created during the execution of \xcd"S" (recursively)
terminate. In many situations (e.g., nondeterministic search) it is
natural to require a statement to terminate when any {\em one} of the
activities it has spawned succeeds. The other activities may then be
safely aborted. Future versions of the language may introduce a
\xcd"finishone S" construct to support such speculative or nondeterministic
computation.
}
%% Need an example here.

\section{Initial activity}\label{initial-computation}\index{initial activity}

An \Xten{} computation is initiated from the command line on the
presentation of a classname \xcd"C". The class must have a
\xcd"public static def main(a: array[String])" method, otherwise an
exception is thrown
and the computation terminates.  The single statement
\begin{xten}
finish async (Place.FIRST_PLACE) {
  C.main(s);
}
\end{xten} 
\noindent is executed where \xcd"s" is an array of strings created
from command line arguments. This single activity is the root activity
for the entire computation. (See \Sref{XtenPlaces} for a discussion of
places.)

%% Say something about configuration information? 

\section{Foreach statements}\index{\Xcd{foreach}}\label{foreach-section}


\begin{grammar}
Statement \: ForEachStatement \\
ForEachStatement \: 
      \xcd"foreach" \xcd"(" Formal \xcd"in" Expression \xcd")"
          Statement 
\end{grammar}


The \xcd"foreach" statement is a parallel version of the enhanced \xcd"for"
statement (\Sref{ForAllLoop}). \xcd`for(x in C)S` executes \xcd`S` {\em
  sequentially}, with everything happening \xcd`here`. \xcd`foreach(x in C)S`
executes \xcd`S` for each iteration of the loop {\em in parallel}, located at
\xcd`x.home`. It is thus equivalent to:
\begin{xten}
foreach (x in C)
  async at (x.home) S
\end{xten}

As a common and useful special case, \xcd`C` may be a \xcd`Dist` or an
\xcd`Array`.  For both of these, \xcd`foreach(x in C)S` is treated just like 
\xcd`foreach(x in C.region)S`.  \xcd`x` ranges over the \xcd`Point`s of the
region.  Each activity that \xcd`foreach` starts is located at \xcd`here` --
the same place that the \xcd`foreach` statement itself is executing.  (If you
want to start an activity at the place where the array element \xcd`C(p)` is
located, use \xcd`ateach` (\Sref{ateach-section}) instead of \xcd`foreach`.)

Exceptions thrown by \xcd`S`, like other exceptions in \xcd`async`s, are
propagated to the root activity of the \xcd`foreach`.  

%FOREACH%  An activity executes a \xcd"foreach" statement in a similar fashion
%FOREACH%  except that separate \xcd"async" activities are launched in parallel
%FOREACH%  in the local place of each object returned by the iteration.
%FOREACH%  The statement
%FOREACH%  terminates locally when all the activities have been spawned. It never
%FOREACH%  throws an exception, though exceptions thrown by the spawned
%FOREACH%  activities are propagated through to the root activity.
%FOREACH%  
%FOREACH%  In a common case, the
%FOREACH%  the collection is intended to be of type
%FOREACH%  \xcd"Region" and the formal parameter is of type \xcd"Point".  Expressions \xcd"e" of type \xcd"Dist" and
%FOREACH%  \xcd"Array" are also accepted, and treated as if they were \xcd"e.region".





\section{Ateach statements}\index{\Xcd{ateach}}\label{ateach-section}

\begin{grammar}
Statement \: AtEachStatement \\
AtEachStatement \:
      \xcd"ateach" \xcd"(" Formal \xcd"in" Expression \xcd")"
         Statement 
\end{grammar}

The \xcd"ateach" statement is similar to the \xcd"foreach"
statement, but it spawns activites at each place of a distribution. 
In \xcd`ateach(p in D) S`, 
\xcd`D` must be either of type \xcd"Dist" or of type
\xcd`Array[T]`, 
and \xcd`p` will be of type \xcd"Point".

This statement differs from \xcd"foreach" only in
that each activity is spawned at the place specified by the
distribution for the point. That is, if \xcd`D` is a \xcd`Dist`, 
\xcd"ateach(p in D) S" could be implemented as:
\begin{xten}
foreach (p in D.region) 
  async (D(p)) S
\end{xten}

However, the compiler may implement it more efficiently to avoid extraneous
communications.  In particular, it is recommended that \xcd`ateach(p in D)S`
be implemented as the following code, which coordinates with each place of
\xcd`D` just once, rather than once per element of \xcd`D` at that place: 
\begin{xten}
foreach (p in D.places()) at (p) {
    foreach (pt in D|here) {
        S
    }
}
\end{xten}

If \xcd`e` is an \xcd`Array[T]`, then \xcd`ateach (p in e)S` is identical to
\xcd`ateach(p in e.dist)S`; the iteration is over the array's underlying
distribution.   \xcd`ateach(p in A)dealWith(A(p));` is a common and generally
efficient idiom for working with the elements of an array.



\section{Futures}\label{XtenFutures}

\Xten{} provides syntactic support for {\em asynchronous expressions}, also
known as futures:

\begin{grammar}
Primary \: FutureExpression \\
FutureExpression \:
  \xcd"future" PlaceExpressionSingleList\opt ClosureBody
\end{grammar} 

Intuitively such an expression evaluates its body asynchronously at
the given place. The resulting value may be obtained from the future
returned by this expression, by using the \xcd"force" operation.

In more detail, in an expression \xcd"future (Q) e", the place
expression \xcd"Q" is treated as in an \xcd"async" statement. \xcd"e"
is an expression of some type \xcd"T". \xcd"e" may reference only
those variables in the enclosing lexical environment which are
declared to be \xcd"val".

If the type of \xcd"e" is \xcd"T" then the type of
\xcd"future (Q) e" is \xcd"Future[T]".  This 
type \xcd"Future[T]" is defined as if by:
\begin{xten}
package x10.lang;
public interface Future[T] implements () => T {
  global def forced(): Boolean;
  global def force(): T;
}
\end{xten}

Evaluation of \xcd"future (Q) e" terminates locally with the creation
of a value \xcd"f" of type \xcd"Future[T]".  This value may be
stored in objects, passed as arguments to methods, returned from
method invocation etc. 

At any point, the method \xcd"forced" may be invoked on \xcd"f". This
method returns without blocking, with the value \xcd"true" if the
asynchronous evaluation of \xcd"e" has terminated globally and with
the value \xcd"false" if it has not.

\xcd"Future[T]" is a subtype of the function type \xcd"() => T".
Invoking---\emph{forcing}---the future \xcd"f" blocks until the
asynchronous evaluation of \xcd"e" has terminated globally. If the
evaluation terminates successfully with value \xcd"v", then the method
invocation returns \xcd"v". If the evaluation terminates abruptly with
exception \xcd"z", then the method throws exception \xcd"z". Multiple
invocations of the function (by this or any other activity) do not
result in multiple evaluations of \xcd"e". The results of the first
evaluation are stored in the future \xcd"f" and used to respond to all
queries.


\begin{xten}
promise: Future[T] = future (a.dist(3)) a(3);
value: T = promise();
\end{xten}


\section{At expressions}

\begin{grammar}
Expression \: \xcd"at" \xcd"(" Expression \xcd")" Expression
\end{grammar}

An \Xcd{at} expression evaluates an expression synchronously at the
given place and returns its value. Note that expression evaluation may
spawn asynchronous activities. The \Xcd{at} expression will return
without waiting for those activities to terminate. That is, \Xcd{at}
does not have built-in \Xcd{finish} semantics.

\section{Shared variables}
\label{Shared}

{\bf Compiler Limitation: Shared variables are not currently implemented.}

A {\em shared local variable} is declared with the annotation \xcd"shared".
It can be accessed within any control construct in its scope, including
\Xcd{async}, \Xcd{at}, \Xcd{future} and closures.

Note that the lifetime of some of these constructs may outlast the
lifetime of the scope -- requiring the implementation to allocate them
outside the current stack frame.

\section{Atomic blocks}\label{AtomicBlocks}\index{atomic blocks}
Languages such as \java{} use low-level synchronization locks to allow
multiple interacting threads to coordinate the mutation of shared
data. \Xten{} eschews locks in favor of a very simple high-level
construct, the {\em atomic block}.

A programmer may use atomic blocks to guarantee that invariants of
shared data-structures are maintained even as they are being accessed
simultaneously by multiple activities running in the same place.

\subsection{Unconditional atomic blocks}
The simplest form of an atomic block is the {\em unconditional
atomic block}:

\begin{grammar}
Statement \: AtomicStatement \\
AtomicStatement \: \xcd"atomic"  Statement \\
MethodModifier \: \xcd"atomic" \\
\end{grammar}

For the sake of efficient implementation \XtenCurrVer{} requires
that the atomic block be {\em analyzable}, that is, the set of
locations that are read and written by the \grammarrule{BlockStatement} are
bounded and determined statically.\footnote{A static bound is a constant
that depends only on the program text, and is independent 
of any runtime parameters. }
The exact algorithm to be used by
the compiler to perform this analysis will be specified in future
versions of the language.
\tbd{}

Such a statement is executed by an activity as if in a single step
during which all other concurrent activities in the same place are
suspended. If execution of the statement may throw an exception, it is
the programmer's responsibility to wrap the atomic block within a
\xcd"try"/{\xcd"finally" clause and include undo code in the finally
clause. Thus the \xcd"atomic" statement only guarantees atomicity on
successful execution, not on a faulty execution.

%% A compiler is allowed to reorder two atomic blocks that have no
%%data-dependency between them, just as it may reorder any two
%%statements which have no data-dependencies between them. For the
%%purposes of data dependency analysis, an atomic block is deemed to
%%have read and written all data at a single program point, the
%%beginning of the atomic block.
%%%% I dont believe we need to say at some point in the atomic block.
%%
We allow methods of an object to be annotated with \xcd"atomic". Such
a method is taken to stand for a method whose body is wrapped within an
\xcd"atomic" statement.

Atomic blocks are closely related to non-blocking synchronization
constructs \cite{herlihy91waitfree}, and can be used to implement 
non-blocking concurrent algorithms.

\begin{staticrule*}
In \xcd"atomic S", \xcd"S" may include method calls,
conditionals, etc.

It may {\em not} include an \xcd"async" activity (such as creation
of a \Xcd{future}).

It may {\em not} include any statement that may potentially block at
runtime (e.g., \xcd"when", \xcd"force" operations, \xcd"next"
operations on clocks, \xcd"finish"). 

All locations accessed in an atomic block must statically satisfy the
{\em locality condition}: they must belong to the place of the current
activity.\index{locality condition}\label{LocalityCondition} 

\end{staticrule*}


The compiler checks for this condition by checking whether the statement
could be the body of a \xcd"void" method annotated with \xcd"safe" at
that point in the code (\Sref{SafeAnnotation}).

\paragraph{Consequences.}
Note an important property of an (unconditional) atomic block:

\begin{eqnarray}
 \mbox{\xcd"atomic \{s1; atomic s2\}"} &=& \mbox{\xcd"atomic \{s1; s2\}"}
\end{eqnarray}

Further, an atomic block will eventually terminate successfully or
thrown an exception; it may not introduce a deadlock.

\subsubsection{Example}

The following class method implements a (generic) compare and swap (CAS) operation:

\begin{xten}
// target defined in lexically enclosing environment.
public atomic def CAS(old: Object, new: Object): Boolean {
   if (target.equals(old)) {
     target = new;
     return true;
   }
   return false;
}
\end{xten}

\subsection{Conditional atomic blocks}

Conditional atomic blocks are of the form:

\begin{grammar}
Statement \:  WhenStatement \\
WhenStatement \:  \xcd"when" \xcd"(" Expression \xcd")" Statement \\
            \| WhenStatement \xcd"or" \xcd"(" Expression \xcd")" Statement 
\end{grammar}

In such a statement the one or more expressions are called {\em
guards} and must be \xcd"Boolean" expressions. The statements are the
corresponding {\em guarded statements}. The first pair of expression
and statement is called the {\em main clause} and the additional pairs
are called {\em auxiliary clauses}. A statement must have a main
clause and may have no auxiliary clauses.

An activity executing such a statement suspends until such time as any
one of the guards is true in the current state. In that state, the
statement corresponding to the first guard that is true is executed.
The checking of the guards and the execution of the corresponding
guarded statement is done atomically. 


\Xten{} does not guarantee that a conditional atomic block
will execute if its condition holds only intermmittently. For, based on
the vagaries of the scheduler, the precise instant at which a
condition holds may be missed. Therefore the programmer is advised to
ensure that conditions being tested by conditional atomic blocks are
eventually stable, i.e., they will continue to hold until the block
executes (the action in the body of the block may cause the condition
to not hold any more).

%%Fourth, \Xten{} does not guarantees only {\em weak fairness} when executing
%%conditional atomic blocks. Let $c$ be the guard of some conditional
%%atomic block $A$. $A$ is required to make forward progress only if
%%$c$ is {\em eventually stable}. That is, any execution $s_1, s_2,
%%\ldots$ of the program is considered illegal only if there is a $j$
%%such that $c$ holds in all states $s_k$ for $k > j$ and in which $A$
%%does not execute. Specifically, if the system executes in such a way
%%that $c$ holds only intermmitently (that is, for some state in which
%%$c$ holds there is always a later state in which $c$ does not hold),
%%$A$ is not required to be executed (though it may be executed).

\begin{rationale}
The guarantee provided by \xcd"wait"/\xcd"notify" in \java{} is no
stronger. Indeed conditional atomic blocks may be thought of as a
replacement for \java's wait/notify functionality.
\end{rationale} 

We note two common abbreviations. The statement \xcd"when (true) S" is
behaviorally identical to \xcd"atomic S": it never suspends. Second,
\xcd"when (c) {;}" may be abbreviated to \xcd"await(c);"---it
simply indicates that the thread must await the occurrence of a
certain condition before proceeding.  Finally note that a \xcd"when"
statement with multiple branches is behaviorally identical to a
\xcd"when" statement with a single branch that checks the disjunction of
the condition of each branch, and whose body contains an
\xcd"if"/\xcd"then"/\xcd"else" checking each of the branch conditions.

\begin{staticrule*}
For the sake of efficient implementation certain restrictions are
placed on the guards and statements in a conditional atomic
block. 
\end{staticrule*}

Guards are required not to have side-effects, not to spawn
asynchronous activities and to have a statically determinable upper
bound on their execution. These conditions are expected to be checked
statically by the compiler.

The body of a \xcd"when" statement must satisfy the conditions
for the body of an \xcd"atomic" block.
%Second, as for unconditional atomic blocks, the set of memory
%locations accessed by a guarded statements are required to be bounded
%and statically analyzable.

Note that this implies that guarded statements are required to be {\em
flat}, that is, they may not contain conditional atomic blocks. (The
implementation of nested conditional atomic blocks may require
sophisticated operational techniques such as rollbacks.)

\paragraph{Sample usage.} 
There are many ways to ensure that a guard is eventually
stable. Typically the set of activities are divided into those that
may enable a condition and those that are blocked on the
condition. Then it is sufficient to require that the threads that may
enable a condition do not disable it once it is enabled. Instead the
condition may be disabled in a guarded statement guarded by the
condition. This will ensure forward progress, given the weak-fairness
guarantee.

\begin{example}
The following class shows how to implement a bounded buffer of size
$1$ in \Xten{} for repeated communication between a sender and a
receiver.

\begin{xten}
class OneBuffer {
  datum: Object = null;
  filled: Boolean = false;
  public def send(v: Object) {
    when (!filled) {
      this.datum = v;
      this.filled = true;
    }
  }
  public def receive(): Object {
    when (filled) {
      v: Object = datum;
      datum = null;
      filled = false;
      return v;
    }
  }
}
\end{xten}
\end{example}

\eat{
\paragraph{Implementing a future with a latch.}\label{future-imp}
The following class shows how to implement a {\em latch}. A latch is
an object that is initially created in a state called the {\em
unlatched} state. During its lifetime it may transition once to a {\em
forced} state. Once forced, it stays forced for its lifetime. The
latch may be queried to determine if it is forced, and if so, an
associated value may be retrieved. Below, we will consider a latch set
when some activity invokes a \xcd"setValue" method on it. This method
provides two values, a normal value and an exceptional value. The
method \xcd"force" blocks until the latch is set. If an exceptional
value was specified when the latch was set, that value is thrown on
any attempt to read the latch. Otherwise the normal value is returned.

\begin{xten}
public interface Future[T] {
   def forced(): Boolean;
   def apply(): T;
}
public class Latch implements Future {
  protected var forced: Boolean = false;
  protected var result: Box[T] = null;
  protected var z: Box[Exception] = null;

  public atomic def setValue(val: T): Boolean {
    return setValue(val, null);
  }
  public atomic def setValue(z: Exception): Boolean {
    return setValue(null, z);
  }
  public atomic def setValue(val: T,
                             z: Exception): Boolean {
    if (forced) return false;
    // these assignment happens only once.
    this.result = val;
    this.z = z;
    this.forced = true;
    return true;
  }
  public atomic def forced(): Boolean {
    return forced;
  }
  public def apply(): T {
    when (forced) {
      if (z != null) throw z;
      return result to T;
    }
  }
}
\end{xten}

Latches, \xcd"aync" operations and \xcd"finish" operations may be used
to implement futures as follows. The expression \xcd"future(P) e"
can be translated to:
\begin{xten}
(() => {
    L: Latch = new Latch();
    async (P) {
      X: Object;
      try {
        finish X = e;
        async (L) {
          L.setValue(X); 
        }
      }
      catch (Z: Exception) {
        async (L) {
          L.setValue(Z);
        }
      }
    }
    return L;
  })()
\end{xten}

Here we assume that \xcd"RunnableLatch" is an interface defined by:
\begin{xten} 
public interface RunnableLatch {
  def run(): Latch;
}
\end{xten}

We use the standard \java{} idiom of wrapping the core translation
inside an inner class definition/method invocation pair (i.e.,
\xcd"new RunnableLatch() {....}.run()") so as to keep the resulting
expression completely self-contained, while executing statements
inside the evaluation of an expression.

Execution of a \xcd"future(P) e" causes a new latch to be created,
and an \xcd"async" activity spawned at \xcd"P". The activity attempts
to \xcd"finish" the assigned \xcd"x = e", where \xcd"x" is a local
variable.  This may cause new activities to be spawned, based on
\xcd"e". If the assignment terminates successfully, another activity is
spawned to invoke the \xcd"setValue" method on the latch.  Exceptions
thrown by these activities (if any) are accumulated at the \xcd"finish"
statement and thrown after global termination of all
activities spawned by \xcd"x=e". The exception will be caught by the 
\xcd"catch" clause and stored with the latch. 


\oldtodo{Conditional atomic blocks should be powerful enough to implement clocks as well.}

\paragraph{A future to execute a statement.}
Consider an expression \xcd"onFinish {S}". This should return
a \xcd"Boolean" latch which should be forced when \xcd"S" has terminated
globally. Unlike \xcd"finish S", the evaluation of \xcd"onFinish {S}"
should locally terminate immediately, returning a latch. The
latch may be passed around in method invocations and stored in
objects. An activity may perform \xcd"force"/\xcd"forced" method
invocations on the latch whenever it desires to determine whether \xcd"S"
has terminated.

Such an expression can be written as:
\begin{xten}
(=> {
    L: Latch = new Latch();
    async (here) {
      try {
        finish S;
        L.setValue(true);
      }
      catch (Z: Exception) {
        L.setValue(Z);
      }
    }
    return L;
  }
)()
\end{xten}
}
	
\chapter{Clocks}\label{XtenClocks}\index{clocks}

The standard library for \Xten{}, \xcd"x10.lang" defines a
final class", \xcd"Clock" intended for repeated quiescence detection
of arbitrary, data-dependent collection of activities. Clocks are a
generalization of {\em barriers}. They permit dynamically created
activities to register and deregister. An activity may be registered
with multiple clocks at the same time. In particular, nested clocks
are permitted: an activity may create a nested clock and within one
phase of the outer clock schedule activities to run to completion on
the nested clock.  Nevertheless the design of clocks ensures that
deadlock cannot be introduced by using clock operations, and that
clock operations do not introduce any races.

This chapter describes the syntax and semantics of clocks and
statements in the language that have parameters of type \xcd"Clock". 

The key invariants associated with clocks are as follows.  At any
stage of the computation, a clock has zero or more {\em registered}
activities. An activity may perform operations only on those clocks it
is registered with (these clocks constitute its {\em clock set}).  An
activity is registered with one or more clocks when it is created.
During its lifetime the only additional clocks it is registered with
are exactly those that it creates. In particular it is not possible
for an activity to register itself with a clock it discovers by
reading a data-structure.

An activity may perform the following operations on a clock \xcd"c".
It may {\em unregister} with \xcd"c" by executing \xcd"c.drop();".
After this, it may perform no further actions on \xcd"c"
for its lifetime. It may {\em check} to see if it is unregistered on a
clock. It may {\em register} a newly forked activity with \xcd"c".
%% It may {\em post} a statement \xcd"S" for completion in the current phase
%% of \xcd"c" by executing the statement \xcd"now(c) S". 
Once registered and "active" (see below), it may also perform the following operations.
It may {\em resume} the clock by executing \xcd"c.resume();". This
indicates to \xcd"c" that it has finished posting all statements it
wishes to perform in the current phase. Finally, it may {\em block}
(by executing \xcd"next;") on all the clocks that it is registered
with. (This operation implicitly \xcd"resume"'s all clocks for the
activity.) It will resume from this statement only when all these
clocks are ready to advance to the next phase.

A clock becomes ready to advance to the next phase when every activity
registered with the clock has executed at least one \xcd"resume"
operation on that clock and all statements posted for completion in
the current phase have been completed.

Though clocks introduce a blocking statement (\xcd"next") an important
property of \Xten{} is that clocks cannot introduce deadlocks. That
is, the system cannot reach a quiescent state (in which no activity is
progressing) from which it is unable to progress. For, before blocking
each activity resumes all clocks it is registered with. Thus if a
configuration were to be stuck (that is, no activity can progress) all
clocks will have been resumed. But this implies that all activities
blocked on \xcd"next" may continue and the configuration is not stuck.
The only other possibility is that an activity may be stuck on
\xcd"finish". But the interaction rule between \xcd"finish" and clocks
(\Sref{sec:finish:clock-rule}) guarantees that this cannot cause a cycle
in the wait-for graph. A more rigorous proof may be found in \cite{X10-concur05}.

\section{Clock operations}\label{sec:clock}
The special statements introduced for clock operations are listed below.
%%479 NowStatement ::= 
%%      now ( Clock ) Statement

\begin{grammar}
Statement \: ClockedStatement \\
ClockedStatement \: \xcd"clocked" \xcd"(" ClockList \xcd")" Statement \\
NextStatement \: \xcd"next" \xcd";" \\
\end{grammar}

Note that \xcd"x10.lang.Clock" provides several useful methods on
clocks (e.g. \xcd"drop").

\subsection{Creating new clocks}\index{clock!creation}\label{sec:clock:create}

Clocks are created using a factory method on \xcd"x10.lang.Clock":

\begin{xten}
timeSynchronizer: Clock = Clock.make();
\end{xten}

\eat{All clocked variables are implicitly final. The initializer for a
local variable declaration of type \xcd"Clock" must be a new clock
expression. Thus \Xten{} does not permit aliasing of clocks.
Clocks are created in the place global heap and hence outlive the
lifetime of the creating activity.  Clocks are structs, hence may be freely
copied from place to 
place. (Clock instances typically contain references to mutable state
that maintains the current state of the clock.)
}
The current activity is automatically registered with the newly
created clock.  It may deregister using the \xcd"drop" method on
clocks (see the documentation of \xcd"x10.lang.Clock"). All activities
are automatically deregistered from all clocks they are registered
with on termination (normal or abrupt).

\subsection{Registering new activities on clocks}
\index{clock!clocked statements}\label{sec:clock:register}

The programmer may specify which clocks a new activity is to be
registered with using the \xcd"clocked" clause.

An activity may transmit only those clocks that is registered with and
has not quiesced on (\Sref{resume}). 
A \xcd"ClockUseException"\index{clock!ClockUseException} is
thrown if (and when) this condition is violated.

An activity may check that it is registered on a clock \xcd"c" by
executing:
\begin{xten}
c.registered()
\end{xten}
\noindent This call returns the \xcd"Boolean" value \xcd"true" iff the
activity is registered on \xcd"c"; otherwise it returns \xcd"false".

\begin{note}
\Xten{} does not contain a ``register'' statement that would allow an
activity to discover a clock in a datastructure and register itself on
it. Therefore, while clocks may be stored in a datastructure by one
activity and read from that by another, the new activity cannot
``use'' the clock unless it is already registered with it.
\end{note}

\oldtodo{Add text on arrays of clocks.}

\subsection{Resuming clocks}\index{clock!resume}\label{resume}\label{sec:clock:resume}
\Xten{} permits {\em split phase} clocks. An activity may wish
to indicate that it has completed whatever work it wishes to perform
in the current phase of a  clock \xcd"c" it is registered with, without
suspending all activity. It may do so  by executing the method
invocation:
\begin{xten}
c.resume();
\end{xten}

An activity may invoke this method only on a clock it is registered
with, and has not yet dropped (\Sref{sec:clock:drop}). A \xcd"ClockUseException"\index{clock!ClockUseException} is thrown if (and
when) this condition is violated.  Nothing happens if the activity has
already invoked a \xcd"resume" on this clock in the current phase.
Otherwise execution of this statement indicates that the activity will
not transmit \xcd"c" to an \xcd"async" (through a \xcd"clocked"
clause),
% or invoke \xcd"now" 
until it terminates, drops \xcd"c" or executes a \xcd"next". 

\begin{staticrule*}
The compiler should issue an error if any activity has a potentially
live execution path from a \xcd"resume" statement on a clock \xcd"c"
to a
%\xcd"now" or
async spawn statement (which registers the new
activity on \xcd"c") unless the path goes through a \xcd"next"
statement. (A \xcd"c.drop()" following a \xcd"c.resume()" is legal,
as is \xcd"c.resume()" following a \xcd"c.resume()".)
\end{staticrule*}

\subsection{Advancing clocks}\index{clock!next}\label{sec:clock:next}
An activity may execute the statement
\begin{xten}
next;
\end{xten}

\noindent 
Execution of this statement blocks until all the clocks that the
activity is registered with (if any) have advanced. (The activity
implicitly issues a \xcd"resume" on all clocks it is registered
with before suspending.)

An \Xten{} computation is said to be {\em quiescent} on a clock
\xcd"c" if each activity registered with \xcd"c" has resumed \xcd"c".
Note that once a computation is quiescent on \xcd"c", it will remain
quiescent on \xcd"c" forever (unless the system takes some action),
since no other activity can become registered with \xcd"c".  That is,
quiescence on a clock is a {\em stable property}.

Once the implementation has detected quiescence on \xcd"c", the system
marks all activities registered with \xcd"c" as being able to progress
on \xcd"c". An activity blocked on \xcd"next" resumes execution once
it is marked for progress by all the clocks it is registered with.

\subsection{Dropping clocks}\index{clock!drop}\label{sec:clock:drop}
An activity may drop a clock by executing:
\begin{xten}
c.drop();
\end{xten}

\noindent{} The activity is no longer considered registered with this
clock.  A \xcd"ClockUseException" is thrown if the activity has
already dropped \xcd"c".


%\subsection{Posting statements on a clock}\index{clock!now}\label{sec:clock:now}
\Xten{} provides syntactic support for a common idiom. Often it may be
necessary for an activity $A$ to require that a certain set of
statements be executed to completion before a clock $c$ can move
forward, without $A$ actually waiting for the completion
of the statement. We introduce the syntax:
\begin{x10}
461 Statement ::= NowStatement
471 StatementNoShortIf ::= 
       NowStatementNoShortIf
479 NowStatement ::= 
       now ( Clock ) Statement
489 NowStatementNoShortIf ::= 
       now ( Clock ) StatementNoShortIf
\end{x10}
\noindent 

A statement {\tt now (c) s} may be considered as shorthand for
\begin{x10}
  async clocked(c) \{ 
     finish async s; 
  \}
\end{x10}

\paragraph{Note.} Because of the static semantics of {\tt finish}
it is not possible to nest {\cf now} statements. Instead if it proves
useful, we may introduce a multi-clocked {\tt now} statement,
which permits the statement to be posted on multiple clocks
simultaneously.
\begin{x10}
479' NowStatement ::= 
       now ( ClockList ) Statement
489' NowStatementNoShortIf ::= 
       now ( ClockList ) StatementNoShortIf  
\end{x10}


\section{Program equivalences}
From the discussion above it should be clear that the following
equivalences hold:

\begin{eqnarray}
 \mbox{\xcd"c.resume(); next;"}       &=& \mbox{\xcd"next;"}\\
 \mbox{\xcd"c.resume(); d.resume();"} &=& \mbox{\xcd"d.resume(); c.resume();"}\\
 \mbox{\xcd"c.resume(); c.resume();"} &=& \mbox{\xcd"c.resume();"}
\end{eqnarray}

Note that \xcd"next; next;" is not the same as \xcd"next;". The
first will wait for clocks to advance twice, and the second
once.  

%\notinfouro{\subsection{Implementation Notes}
Clocks may be implemented efficiently with message passing, e.g.{} by
using short-circuit ideas in \cite{SaraswatPODC88}.  Recall that every
activity is spawned with references to a fixed number of clocks. Each
reference should be thought of as a global pointer to a location in
some place representing the clock. (We shall discuss a further
optimization below.) Each clock keeps two counters: the total number
of outstanding references to the clock, and the number of activities
that are currently suspended on the clock.

When an activity $A$ spawns another activity $B$ that will reference a
clock $c$ referenced by $A$, $A$ {\em splits} the reference by sending
a message to the clock. Whenever an activity drops a reference to a
clock, or suspends on it, it sends a message to the clock. 

The optimization is that the clock can be represented in a distributed
fashion. Each place keeps a local counter for each clock that is
referenced by an activity in that place. The global location for the
clock simply keeps track of the places that have references and that
are quiescent. This can reduce the inter-place message traffic
significantly.
}
%\notinfouro{\section{Clocked types}\index{types!clocked}

We allow types to specify clocks, via a {\cf clocked(c)} modifier,
e.g.{}

\begin{x10}
  clocked(c) int r;
\end{x10}

This declaration asserts that {\cf r} is accessible
(readable/writable) only by those statements that are clocked on {\cf
c}. Thus propagation of the clock provides some control over the
``visibility'' of {\cf r}.

The declaration 

\begin{x10}
  clocked(c) final int l = r;
\end{x10}

\noindent asserts additionally that in each clock instant {\cf l} is final, 
i.e.{} the value of {\cf l} may be reset at the beginning of each phase
of {\tt c} but stays constant during the phase.

This statement terminates when the computation of {\tt r} has
terminated and the assignment has been performed.

\todo{Generalize the syntax so that aggregate variables can be clocked with an aggregate clock of the same shape.}

\subsection{Clocked assignment}\index{assignment!clocked}
We expect that most arrays containing application data will be
declared to be {\cf clocked final}. We support this very powerful type
declaration by providing a new statement:
{\footnotesize
\begin{verbatim}
  next(c) l = r; 
\end{verbatim}}


\noindent 
for a variable $l$ declared to be clocked on $c$. The statement
assigns $r$ to the {\em next} value of $l$. There may be multiple such
assignments before the clock advances. The last such assignment
specifies the value of the variable that will be visible after the
clock has advanced.  If the variable is {\cf clocked final} it is
guaranteed that {\em all} readers of the variable throughout this
phase will see the value $r$.

The expression {\tt r} is implicitly treated as {\tt now(c) r}. That
is, the clock {\tt c} will not advance until the computation of {\tt r} has
terminated.

}
%\notinfouro{%III. Applied constrained calculi. (3 pages)
%
%For each example below, formal static and dynamic semantics rules for
%new constructs extension over the core CFJ. Subject-reduction and
%type-soundness theorems. Proofs to be found in fuller version of
%paper.
%
%(a) arrays, region, distributions -- type safe implies no arrayoutofbounds
%exceptions, only ClassCastExceptions (when dynamic checks fail).
%
%Use Satish's conditional constraints example.
%-- emphasize what is new over DML. 
%
%(b) places, concurrency -- place types.
%
%(c) ownership types, alias control.
%
The following section presents example uses of constrained types
using several different
constraint systems.
%
\eat{
Many common object-oriented
idioms and
object-oriented type systems can be captured naturally using
constrained types: specifically we discuss types for places,
aliases,
ownership, arrays and clocks.  \ref{TODO}
}

\eat{
\ref{TODO}
Many of these constraint systems are more
expressive than the constraint system implemented in the current
\Xten{} compiler and have not (yet) been implemented.
}

\eat{
\ref{TODO}
In the following we will use the shorthand $\tt C(\bar{t}:c)$ for the
type $\tt C(:\bar{f}=\bar{t},c)$ where the declaration of the class
{\tt C} is $\tt \class\ C(\bar{\tt T}\ \bar{\tt f}:c)\ldots$  Also,
we abbreviate $\tt C(\bar{t}:\true)$ as $\tt C(\bar{t})$.
Finally, we use the shorthand $\tt T\;x=t;~c$ for the constraint
$\tt T\;x;~x=t;~c$.
}

\eat{
Finally, we
will also have need to use the shorthand
${\tt C}_1(\bar{t}_1:{\tt c}_1)\& \ldots {\tt C}_k(\bar{\tt t}_k:{\tt c}_k)$
for the type
${\tt C}_1(:\bar{\tt f}_1=\bar{\tt t}_1, \ldots,
            \bar{\tt f}_k=\bar{\tt t}_k,{\tt c}_1,\ldots,{\tt c}_k)$ 
provided that the ${\tt C}_i$ form a subtype chain
and the declared fields of ${\tt C}_i$ are ${\tt f}_i$.

Constraints naturally allow for partial specification
(e.g., inequalities) or incomplete specification (no constraint on a
variable) with the same simple syntax. In the example below,
the type of {\tt a} does not place any constraint on the second
dimension of {\tt a}, but this dimension can be used in other
types (e.g., the return type).
\begin{xten}
  class Matrix(int m, int n) {
    Matrix(m,a.n) mul(Matrix(:m==this.n) a) {...}
    ...
  }
\end{xten}

Constraints also naturally permit the expression of existential types:
\begin{xten}
  class List(int length) { 
    List(:length <= this.length) filter(Comparator k) {...} 
    ...
  }
\end{xten}
\noindent
Here, the length of the list returned by the \xcd{filter} method is 
unknown, but is bound by the length of the original list.
}

\if 0
\subsection{Presburger constraints: array bounds}

Xi and Pfenning proposed using dependent types for eliminating
array bounds checks~\cite{xi98array}.
\Xten{} does not (yet) support generic types, however XXX
%
In CFJ, an array of type \xcd{T[]} indexed by (signed) integers
can be modeled as a class with the following
signature:\footnote{For this example, we assume generics
are supported.}
\begin{xten}
interface Array<T>(int(:self >= 0) length) {
  T get(int(:0 <= self, self < this.length) i);
  void set(int(:0 <= self, self < this.length) i, T v);
}
\end{xten}

Bounds can be checked using a constraint system based on
Presburger arithmetic~\cite{omega}.  Constraint terms include
integer constraints, scalar multiplication, and addition;
constraints include inequalities:
\fi


\eat{
Some code that iterates over an array (sugaring {\tt get} and {\tt set}):
\begin{xten}
double dot(double[] x, double[] y
         : x.length == y.length) {
  double r = 0.; 
  for (int(:self >= 0, self < x.length)
       i = 0; i < x.length; i++) {
    r += x[i] * y[i];
  }
  return r;
}
\end{xten}
}

\eat{
Another one:
\begin{xten}
double[](:length = x.length) saxpy(double a, double[] x, double[] y : x.length = y.length) {
    double[](:length = x.length) result = new double[x.length];
    for (int(:self >= 0, self < x.length) i = 0; i < x.length; i++) {
        result[i] = a * x[i] + y[i];
    }
    return result;
}
\end{xten}
}

% \subsection{Presburger constraints: blocked LU factorization}

\subsection{Equality constraints}

The \Xten{} compiler includes a simple equality-based constraint
system, described in Section~\ref{sec:lang}.
Equalities constraints
are used throughout \Xten{} programs.  For example, to ensure
$n$-dimensional arrays are indexed only be $n$-dimensional
index points, the array access operation requires that the
array's \xcd{rank} property be equal to the index's \xcd{rank}.

Equality constraints specified in the X10 run-time library are used by the
compiler to generate efficient code.  For instance, an iteration over
the points in a region can be optimized to a set of nested loops
if the constraint on the region's type specifies that the region
is rectangular and of constant rank.


\eat{
\subsection{Equality constraints with disjunction: place types}

This example is due to Satish Chandra. We wish to specify a balanced
distributed tree with the property that its right child is always at
the same place as its parent, and once the left child is at the same
place then the entire subtree is at that place.  In
\Xten{}, every object has a field {\tt location} of type
{\tt place} that specifies the location at which the object is located.
%
The desired property may be specified thus:
\begin{xten}
class Tree(boolean localLeft) extends Object {
  Tree(:!this.localLeft || (location==here && self.localLeft)) left; 
  Tree(:location==here) right);
  ...
}
\end{xten}
The constraint on \xcd{left} states that if the \xcd{localLeft} property is
true for the current node, then the location of the \xcd{left} child must be
\xcd{here} and its \xcd{localLeft} property must be set.  This ensures,
recursively, that the entire left subtree will be located at the same place.
}

\subsection{Presburger constraints}

Presburger constraints are linear integer inequalities.
%A constraint solver plugin was implemented using a port to Java of the
%Omega library.~\cite{omega,scale}
%A separate implementation
%of a Presburger constraint solver was implemented using
%CVC3~\cite{cvc}. 
A Presburger constraint solver plugin was implemented using
CVC3~\cite{cvclite,cvc}.  The list example in
Figure~\ref{fig:list-example} type-checks using this constraint system.

Presburger constraints are particularly useful in a
high-performance computing setting where array operations are
pervasive.
Xi and Pfenning proposed using dependent types for eliminating
array bounds checks~\cite{xi98array}.  A Presburger constraint
system can be used to keep track of array dimensions and array
indices to ensure bounds violations do not occur.

\subsection{Set constraints: region-based arrays}

Rather than using Presburger constraints, 
\Xten{} takes another approach:
following ZPL~\cite{ZPL}, arrays in \Xten{}
are defined over
{\em regions},
sets of $n$-dimensional {\em index points}~\cite{gps06-arrays}.
For instance, the region \xcd{[0:200,}\xcd{1:100]} specifies a
collection of two-dimensional points \xcd{(i,j)} with \xcd{i}
ranging from \xcd{0} to \xcd{200} and \xcd{j} ranging from
\xcd{1} to \xcd{100}.

Regions and points were modeled in CVC3~\cite{cvc} to create a
constraint solver that ensures array bounds
violations do not occur:
an array access type-checks if the index point can be statically
determined to be in the region over which the array is defined.

Region constraints are subset constraints
written as calls to the \xcd{contains}
method of the \xcd{region} class.
The constraint solver does not actually evaluate the calls to
the \xcd{contains} method, rather it interprets these calls
symbolically
as subset constraints at compile time.

Constraints have the following syntax:

{
\small
\begin{tabular}{r@{\quad}rcl}
\\
  (Constraint)   &\xcd{c} &::=& \xcd{r.contains(r)} \bnf \dots \\
  (Region) &\xcd{r} &::=& \xcd{t} \bnf [${\tt b}_1$:${\tt d}_1$,\ldots,${\tt b}_k$:${\tt d}_k$]
           \\
           &        &  \bnf &
           \xcd{r | r} \bnf \xcd{r & r} \bnf \xcd{r - r}
           \\
           &        &  \bnf &
           \xcd{r + p} \bnf \xcd{r - p} \\
  (Point)  &\xcd{p} &::=& \xcd{t} \bnf $[{\tt b}_1,\ldots,{\tt b}_k]$ \\
(Integer)&\xcd{b},\xcd{d} &::=& \xcd{t} \bnf \xcd{n} \\
\\
\end{tabular}
}

\noindent
where \xcd{t} are constraint terms (properties and final variables)
and \xcd{n} are integer literals.

Regions used in constraints are either constraint terms \xcd{t},
region constants, unions (\xcd{|}), intersections (\xcd{&}),
or differences (\xcd{-}), or regions where each point is
offset by another point \xcd{p} using \xcd{+} or \xcd{-}.

% $\xcd{r}_1$\xcd{.contains(}$\xcd{r}_2$\xcd{)}.

\begin{figure}[t]
\footnotesize

\inputxten{sor.x10}

\caption{Successive over-relaxation with regions}
\label{fig:sor}
\end{figure}

For example, the code in Figure~\ref{fig:sor} performs a successive
over-relaxation~\cite{sor} of a matrix \tcd{G} with rank 2.
The function declares a region variable \tcd{outer} as an alias for
\tcd{G}'s region and a region variable \tcd{inner} to be 
the subset of \tcd{outer} that excludes the boundary points,
formed by intersecting the \tcd{outer} region with itself shifted up, down,
left, and right by one.
The function then declares two more regions \tcd{d0} and \tcd{d1},
where ${\tt d}_i$ is set of points ${\tt x}_i$ where
$({\tt x}_0, {\tt x}_1)$ is in \tcd{inner}.  The function
iterates multiple times over points \tcd{i} in \tcd{d0}.
The syntax \tcd{finish} \tcd{foreach} (line 22) tells the
compiler to execute each loop iteration in parallel and to wait
for all concurrent activities to terminate.
The inner loop (lines 24--28) iterates over a subregion of
\tcd{inner}.

The type checker establishes that the \tcd{region} property of
the point \tcd{ij} (line 24) is \tcd{inner}
\xcd{&}~\xcd{[i..i,d1min..d1max]}, and that this region is a
subset of \tcd{inner}, which is in turn a subset of \tcd{outer},
the region of the array \tcd{G}.
Thus, the accesses to the array in the loop body
do not violate the bounds of the array.

A key to making the program type-check is that the region
intersection that defines \tcd{inner} (lines 10--11)
is explicitly intersected with \tcd{outer} so that the 
constraint solver can determine that
the result is a subset of \tcd{outer}.


\eat{
\subsection{AVL trees and red--black trees}

AVL trees and red-black trees can be modeled so that the
data structure invariant is enforced statically.

\begin{xten}
class AVLTree(int(:self >= 0) height) {...}
class Leaf(Object key) extends AVLTree(0) {...}
class Node(Object key, AVLTree l, AVLTree r
         : int d=l.height-r.height; -1 <= d, d <= 1) 
    extends AVLTree(max(l.height,r.height)+1) {...}
\end{xten}

Red--black trees may be modeled similarly. Such trees have the
invariant that (a) all leaves are black, (b) each non-leaf node has
the same number of black nodes on every path to a leaf (the black
height), (c) the immediate children of every red node are black.
\begin{xten}
class RBTree(int blackHeight) {...}
class Leaf extends RBTree(0) { int value; ... }
class Node(boolean isBlack, 
           RBTree(:this.isBlack || isBlack) l, 
           RBTree(:this.isBlack || isBlack,
                   blackHeight=l.blackHeight) r)
    extends RBTree(l.blackHeight+1) { int value; ... }
\end{xten}
}

\eat{
\subsection{Self types and binary methods}

Self types~\cite{bsg95,bfp-ecoop97-match} can be implemented
using a {\tt klass} property on objects.  The {\tt klass}
property represents the run-time class of the object.
Self types can be used to solve the binary method problem \cite{bruce-binary}.

In the example below, the {\tt Set} interface has a {\tt union} method
whose argument must be of the same class as {\tt this}.
\noindent This enables the {\tt IntSet} class's {\tt union}
method to access the {\tt bits} field of its argument {\tt s}.
\begin{xten}
  interface Set(:Class klass) {
    Set(this.klass) union(Set(this.klass) s);
  }
  class IntSet(:Class klass) implements Set(klass) {
    long bits;

    IntSet(IntSet.class)() { property(IntSet.class); }

    IntSet(IntSet.class)(int(:0 <= self, self <= 63) i) {
      property(IntSet.class);
      bits = 1 << i; }

    Set(this.klass) union(Set(this.klass) s) {
      IntSet(this.klass) r = new IntSet(this.klass);
      r.bits = this.bits | s.bits;
      return r; }
  }
\end{xten}
\noindent
The key to ensuring that this code type-checks is the
\rn{T-constr}
rule.
With a constraint system ${\cal C}_{\mathsf{klass}}$ aware of
the {\tt klass} property, the rule 
\rn{T-var} is used to subsume an expression of type
${\tt Set(this.class)}$ to type ${\tt IntSet(this.class)}$
when {\tt this} is known to be an {\tt IntSet}:
{\footnotesize
\[
\from{\begin{array}{c}
{\tt IntSet}~{\tt this}, {\tt Set}({\tt this.klass})~{\tt s}
        \vdash {\tt Set}({\tt this.klass})~{\tt s} \\
{\tt IntSet}~{\tt this}, {\tt Set}({\tt this.klass})~{\tt s}
        \vdash_{{\cal C}_{\mathsf{klass}}} {\tt IntSet}({\tt this.klass})~{\tt s} \\
\end{array}}
\infer{
{\tt IntSet}~{\tt this}, {\tt Set}({\tt this.klass})~{\tt s}
        \vdash {\tt IntSet}({\tt this.klass})~{\tt s}}
\]}
}


\eat{
\subsection{Binary search}

An informal study by Jon Bentley~\cite{programming-pearls}
found that x\% of professional programmers attending in a course
could not correctly implement binary search.

Dependent types can help here by adding the invariants to the
index types.

\subsection{Quicksort}

\begin{xten}
int(:left <= self & self <= right)
partition(T[] array, int left, int right, int pivotIndex : left <= pivotIndex & pivotIndex <= right) {
     T pivotValue = array[pivotIndex];

     // Move pivot to end
     swap(array, pivotIndex, right);

     int(:left <= self & self <= right) storeIndex;
     storeIndex = left;
     for (int(:left <= self & self <= right-1) i = left; i < right; i++) {
         if (array[i] <= pivotValue) {
             swap(array, storeIndex, i);
             storeIndex++;
         }
     }

     // Move pivot to its final place
     swap(array, right, storeIndex)
     return storeIndex;
}

void swap(T[] array,
          int(:0 <= self & self < array.length i,
          int(:0 <= self & self < array.length j) {
    T tmp = array[i];
    array[i] = array[j];
    array[j] = tmp;
}

void quicksort(T[] array, int left, int right : left <= right) {
    if (left < right) {
         // select a pivot index
         int(:left <= self & self <= right) pivotIndex = (left + right) / 2;
         pivotNewIndex = partition(array, left, right, pivotIndex)
         quicksort(array, left, pivotNewIndex-1)
         quicksort(array, pivotNewIndex+1, right)
    }
}
\end{xten}
}


\newif\ifowner
\ownerfalse

\ifowner

\subsection{Ownership constraints}

\begin{figure}[t]
\inputxten{LO.x10}
\caption{Ownership types}
\label{fig:ownership}
\end{figure}

Using a simple extension of \Xten{}'s built-in equality
constraint system,
constrained types can also be used to encode a form of ownership
types~\cite{ownership-types,liskov-popl2003}.

Figure~\ref{fig:ownership} shows a fragment of a \xcd{List}
class with ownership types.
Each \xcd{Owned} object has an \xcd{owner} property.  Objects
also have properties that are used as owner parameters.
%
The \xcd{List} class has an \xcd{owner} property inherited from
\xcd{Owned} and also declares a \xcd{valOwner} property that is
instantiated with the owner of the values in the list, stored in
the \xcd{head} field of each element.  The \xcd{tail} of the
list is owned by the list object itself.

\Xten{}'s equality-based constraint system is sufficient for
tracking object ownership, however is does not capture all
properties of ownership type systems.
Ownership type systems enforce an ``owners as dominators''
property: the ownership relation forms a tree within the object
graph; a reference is not permitted to point directly to objects
with a different owner.
%
To encode this property, the owner of
the values \xcd{valOwner} must be contained within the owner
of the list itself; that is, \xcd{valOwner} must be \xcd{owner}
or \xcd{valOwner}'s owner must be contained in \xcd{owner}.
This is captured by the constraint \xcd{self.owns(owner)} on
\xcd{valOwner}.  Calls to the \xcd{owns} method in constraints
are interpreted by the ownership constraint solver as the
disjunction of conditions shown in the body of \xcd{owns}.
The object \xcd{world} is the root of the ownership tree; 
all objects are transitively owned by \xcd{world}.

For example, the type \xcd{List(:owner==world & valOwner == this)}
is invalid, because
its constraint is interpreted as
\xcd{owner == world & valOwner == this & this.owns(world)},
which is satisfiable only when \xcd{this == world} (which it is not).

An additional check is needed to ensure objects owned by
\xcd{this} are encapsulated.
The \xcd{tail()} method for instance, incorrectly leaks the
list's \xcd{tail} field.  To eliminate this case, the ownership
constraint system must additionally check that owner parameters
are bound only to 
\xcd{this}, \xcd{world}, or an owner property of \xcd{this}.
These conditions ensure that \xcd{tail()} can be called only on
\xcd{this} because its return type is otherwise not valid.
For instance, in the following code, the type of \xcd{ys} is
not valid because the \xcd{owner} property is bound to \xcd{xs}:
\begin{xten}
    final Owned o = ...;
    final List(:owner==o & valOwner==o) xs;
    List(:owner==xs & valOwner==o) ys = xs.tail();
\end{xten}

\fi

\if 0
\subsection{Disequalities: non-null types}

A constraint system that supports disequalities can be used to
enforce a non-null invariant on reference types.
A non-null type \xcd{C} can be written simply as \xcd{C(:self != null)}.
\fi

\eat{
\subsection{Clocked types}

Clocks are barriers that are adapted to a context where activities may be
dynamically created, and are designed so that all clock operations are
determinate.

For each arity $n$, we introduce a {\em Gentzen predicate}
${\tt clocked(\bar{t})}$. A $k$-ary Gentzen predicate $a$ satisfies the
property that $a(t_1,\ldots, t_k) \vdash a(s_1,\ldots,s_n)$ iff $k=n$
and $t_i=s_i$ for $i\leq k$.

Such a \xcd{clocked} atom is added to the context by an \xcd{clocked async}:
$$
\from{\Gamma, {\tt clocked(\bar{\tt v})} \vdash {\tt T}\ {\tt e}}
\infer{\Gamma \vdash {\tt T}\ {\tt async}\ {\tt clocked}(\bar{\tt v}) {\tt e}}
$$

A programmer can require that a method may be invoked only if the
invoking activity is registered on the clocks $\bar{\tt k}$ by adding
a \xcd{clocked} clause. The rule for method elaboration and method invocation then change:
$$
\begin{array}{l}
\from{ \bar{\tt T}\ \bar{\tt x}, {\tt C}\ \this, {\tt c},\clocked(\bar{\tt k}) \vdash {\tt S}\ {\tt e}, {\tt S} \subtype {\tt T} }   
\infer{\tt T\ m(\bar{\tt T}\,\bar{\tt x} : c) \clocked(\bar{\tt  k})\{\return\ e;\}\ \mbox{OK in}\ C} 
\\ \quad\\ 
\rname{T-Invk}%
\from{\begin{array}{l}
\Gamma \vdash {\tt T}_{0:n} \ {\tt e}_{0:n}  \\
\mtype({\tt  T}_0,{\tt  m},{\tt  z}_0)= \tt {\tt  Z}_{1:n}\ {\tt  z}_{1:n}:c,clocked(\bar{\tt  k}) \rightarrow {\tt  S} \\
\Gamma, {\tt  T}_{0:n}\ {\tt  z}_{0:n} \vdash {\tt  T}_{1:n} \subtype {\tt  Z}_{1:n}\\
\sigma(\Gamma, {\tt  T}_{0:n}\ {\tt  z}_{0:n}) \vdash_{\cal C} {\tt  c} \ \ \ 
\mbox {(${\tt  z}_{0:n}$ fresh)} \\
\Gamma \vdash \clocked(\bar{\tt  k})\\
\end{array}}
\infer{\Gamma \vdash ({\tt  T}_{0:n}\ {\tt  z}_{0:n}; S)\ {\tt  e}_0.{\tt  m({\tt  e}_{1:n})}}
\end{array}
$$
}

\eat{
\subsection{Capabilities}

Capabilities (from Radha and Vijay's paper on neighborhoods)
}

\eat{
\subsection{Activity-local objects}

Parallelism in \Xten{} is supported through lightweight asynchronous {\em
activities}, created by {\tt  async} statements.
It is often useful to restrict objects so that they are {\em local} to a
particular activity.
A local object may be accessed only by
the activity that created it or by an ancestor of that activity.
% it may be written only by the activity that created
% it or by a descendant of that activity.
Local objects are declared and created by qualifying their type
with {\tt  local}:
\begin{xten}
  local C o = new local C();
\end{xten}

To encode local objects in \Xten{}, we add an {\tt  activity}
property to objects:
\begin{xten}
  class Object(Activity activity) { ... }
\end{xten}
\noindent
where {\tt  Activity} has a possibly null {\tt  parent} property:
\begin{xten}
  class Activity(Activity parent) { ... }
\end{xten}
\noindent

To track the current activity ({\tt  z}), we augment typing judgments
as follows:
\[
  {\tt  z};~\Gamma \vdash {\tt  T}\ {\tt  e}
\]
\noindent where ${\tt  Activity}({{\tt  z}'})~{\tt  z} \in \Gamma$.
When the current activity is {\tt  z},
we encode the type {\tt  local C} as ${\tt  C}({\tt  z})$.

Spawning a new activity with an {\tt  async} statement
introduces a fresh activity ${\tt  z}'$:
\[
\from{
{\tt  z}';~\Gamma,~{\tt  Activity}({\tt  z})~{\tt  z'} \vdash {\tt  T}\ {\tt  e}\ \ \ 
\mbox{(${\tt  z}'$ fresh)}
}
\infer{
{\tt  z};~\Gamma \vdash {\tt  T}\ ({\tt  async}\ {\tt  e})
}
\]
The rule \rn{T-Field} is strengthened to require that reads 
only be performed on objects whose {\tt  activity} property is a
descendant of the current activity.
%\rname{T-Field-Local}%
\[
\from{
\begin{array}{ll}
{\tt  z};~\Gamma \vdash {\tt  T}_0\ {\tt  e} \\
\mathit{fields}({\tt  T}_0,{\tt  z}_0)= \bar{\tt  U}\ \bar{\tt  f}_i &
\mbox{(${\tt  z}_0$ fresh)} \\
{\tt  z};~\Gamma \vdash {\tt  T}_0 \subtype {\tt  C}(:{\tt  activity} = {\tt  z}') &
\Gamma \vdash {\tt  z}~\mathsf{spawns}~{\tt  z}'
\end{array}
}
\infer{{\tt  z};~\Gamma \vdash ({\tt  T}_0\ {\tt  z}_0; {\tt  z}_0.{\tt  f}_i=\self;{\tt  U}_i)\ {\tt  e.f}_i}
\]

%\Gamma \vdash {\tt  z}_0.{\tt  activity} = {\tt  z}' &

\noindent
where the $\mathsf{spawns}$ relation is defined as follows:
\[
\Gamma \vdash {\tt  z}~\mathsf{spawns}~{\tt  z}
\]
\[
\from{
\Gamma \vdash {\tt  z_1}~\mathsf{spawns}~{\tt  z_2} \ \ \ 
\Gamma \vdash {\tt  z_2}~\mathsf{spawns}~{\tt  z_3}}
\infer{\Gamma \vdash {\tt  z_1}~\mathsf{spawns}~{\tt  z_3}}
\]
\[
\from{\Gamma \vdash {\tt  z_2}.{\tt  parent} = {\tt  z_1}}
\infer{\Gamma \vdash {\tt  z_1}~\mathsf{spawns}~{\tt  z_2}}
\]

\eat{
local objects owned by activity that created it.

locals cannot be read by contained asyncs.

locals can be written by contained asyncs.

locals created by an activity are inherited by the parent when
the activity terminates.

\begin{xten}
C(:thread = current) x = ...;
finish foreach (...) {
  C(:thread = current) y = x; // no!
  x = y;
}
\end{xten}

// can read if thread prop is current, or an ancestor of current
// can write if thread prop is current or a child of current

e : C(:thread = x)
current owns x
fields(...) = Ti fi
-----------------------
e.fi : Ti

extensions:

1. add thread to context
2. strengthen T-field rule
}
}

\eat{
\subsection{Discussion}

\paragraph{Control-flow.}
Tricky to encode.  Need something like {\tt pc} label~\cite{jif}.

\paragraph{Type state.}
Type state depends on the mutable state of the 
objects.  Cannot do in this framework.

Dependent types are of use in annotations~\cite{ns07-x10anno}.
}
}

	
\chapter{Local and Distributed Arrays}\label{XtenArrays}\index{array}

\section{Overview}

Indexable memory is fundamental abstraction for a programming
language. X10 includes
\begin{itemize}
\item Rails -- intrinsic one dimensional arrays
\item Local multi-dimensional arrays; both simplearray and regionarray
\item Distributed multi-dimensional arrays; both simplearray and regionarray
\end{itemize}

\section{Rails}

WRITE ME

\section{Simple Arrays}

WRITE ME

\section{Region-based Arrays}

Classes in the \Xcd{x10.regionarray} package provide the most general and 
flexible array abstraction that support mapping arbitrary multi-dimensional
index spaces to data elements. Although they are significantly more
flexible than \Xcd{Rail}s or the classes of the \Xcd{x10.simplearray}
package, this flexibility does carry with it an expectation of lower
runtime performance. 

\Xcd{Array}s provide indexed access to data at a single \Xcd{Place}, {\em via}
\Xcd{Point}s---indices of any dimensionality. \Xcd{DistArray}s is similar, but
spreads the data across multiple \xcd`Place`s, {\em via} \Xcd{Dist}s.  

\subsection{Points}\label{point-syntax}
\index{point}
\index{point!syntax}

Both kinds of arrays are indexed by \xcd`Point`s, which are $n$-dimensional tuples of
integers.  The \xcd`rank`
property of a point gives its dimensionality.  Points can be constructed from
integers, or \xcd`Rail[Int] by the \xcd`Point.make` factory methods:
%~~gen ^^^ArraysPointsExample1
% package Arrays.Points.Example1;
% import x10.regionarray.*;
% class Example1 {
% def example1() {
%~~vis
\begin{xten}
val origin_1 : Point{rank==1} = Point.make(0);
val origin_2 : Point{rank==2} = Point.make(0,0);
val origin_5 : Point = Point.make(new Rail[Int](5));
\end{xten}
%~~siv
% } } 
%~~neg

There is an implicit conversion from \xcd`Rail[Int]` to 
\xcd`Point`, giving
a convenient syntax for constructing points: 

%~~gen ^^^ Arrays30
% package Arrays.Points.Example2;
% import x10.regionarray.*;
% class Example{
% def example() {
%~~vis
\begin{xten}
val p : Point = [1,2,3];
val q : Point{rank==5} = [1,2,3,4,5];
val r : Point(3) = [11,22,33];
\end{xten}
%~~siv
% } } 
%~~neg

The coordinates of a point are available by function application, or, if you
prefer, by subscripting; \xcd`p(i)` is the
\xcd`i`th coordinate of the point \xcd`p`. 
\xcdmath`Point($n$)` is a \Xcd{type}-defined shorthand  for 
\xcdmath`Point{rank==$n$}`.


\subsection{Regions}\label{XtenRegions}\index{region}
\index{region!syntax}

A {\em region} is a set of points of the same rank.  {}\Xten{}
provides a built-in class, \xcd`x10.regionarray.Region`, to allow the
creation of new regions and to perform operations on regions. 
Each region \xcd`R` has a property \xcd`R.rank`, giving the dimensionality of
all the points in it.

\begin{ex}
%~~gen ^^^ Arrays40
% package Arrays40;
% import x10.regionarray.*;
% class Example {
% static def example() {
%~~vis
\begin{xten}
val MAX_HEIGHT=20;
val Null = Region.makeUnit(); //Empty 0-dimensional region
val R1 = Region.make(1, 100); // Region 1..100
val R2 = Region.make(1..100);  // Region 1..100
val R3 = Region.make(0..99, -1..MAX_HEIGHT);
val R4 = Region.makeUpperTriangular(10);
val R5 = R4 && R3; // intersection of two regions
\end{xten}
%~~siv
% } } 
%~~neg

The \xcd`IntRange` value \xcd`1..100` can be used to construct
the one-dimensional \xcd`Region` consisting of the points
$\{$\xcdmath`[m]`, \dots, \xcdmath`[n]`$\}$
\xcd`Region` by using the \xcd`Region.make` factory method.  
\xcd`IntRange`s are useful in building up regions, especially rectangular regions.  
\end{ex}

By a special dispensation, the compiler knows that, if \xcd`r : Region(m)` and
\xcd`s : Region(n)`, then \xcd`r*s : Region(m+n)`.  (The X10 type system
ordinarily could not specify the sum; the best it could do 
would be \xcd`r*s : Region`, with the rank of the region unknown.)  This
feature allows more convenient use of arrays; in particular, one does not need
to keep track of ranks nearly so much.

Various built-in regions are provided through  factory
methods on \xcd`Region`.  
\begin{itemize}
%~~exp~~"~~"~~ n:Int ~~ import x10.regionarray.*; ^^^Arrays3s5h
\item \xcd"Region.makeEmpty(n)" returns an empty region of rank \xcd"n".
%~~exp~~"~~"~~ n:Int ~~ import x10.regionarray.*; ^^^Arrays3x4j
\item \xcd"Region.makeFull(n)" returns the region containing all points of
      rank \xcd"n".  
%~~exp~~"~~"~~ ~~ import x10.regionarray.*; ^^^Arrays7l3d
\item \xcd"Region.makeUnit()" returns the region of rank 0 containing the
      unique point of rank 0.  It is useful as the identity for Cartesian
      product of regions.
%~~exp~~"~~"~~ normal:Point, k:Int ~~ import x10.regionarray.*; ^^^Arrays3l7z
\item \xcd"Region.makeHalfspace(normal, k)",
      where \xcd`normal` is a \xcd`Point` and \xcd`k` an \xcd`Int`, 
      returns the unbounded
      half-space of rank \xcd"normal.rank", consisting of all points \xcd"p"
      satisfying the vector inequality \xcdmath`p$\cdot$normal $\le$ k`.
%~~exp~~"~~"~~ min:Rail[Long], max:Rail[Long] ~~ import x10.regionarray.*; ^^^Arrays3i3n
\item \xcd"Region.makeRectangular(min, max)", 
      where \xcd"min" and \xcd"max"
      are rank-1 length-\xcd`n` integer arrays, returns a
      \xcd"Region(n)" equal to: 
      \xcdmath`[min(0) .. max(0), $\ldots$, min(n-1)..max(n-1)]`.
%~~exp~~"~~"~~ size: int, a: int, b: int~~ import x10.regionarray.*; ^^^Arrays2f2y
\item \xcd"Region.makeBanded(size, a, b)" constructs the
      banded \xcd"Region(2)" of size \xcd"size", with \Xcd{a} bands above
      and \Xcd{b} bands below the diagonal.
%~~exp~~"~~"~~size:Int ~~ import x10.regionarray.*; ^^^Arrays5s3q
\item \xcd"Region.makeBanded(size)" constructs the banded \Xcd{Region(2)} with
      just the main diagonal.
%~~exp~~`~~`~~N:Int ~~ import x10.regionarray.*; ^^^Arrays5s3qtri
\item \xcd`Region.makeUpperTriangular(N)` returns a region corresponding
to the non-zero indices in an upper-triangular \xcd`N x N` matrix.
%~~exp~~`~~`~~N:Int ~~ import x10.regionarray.*; ^^^Arrays5s3qlowertri
\item \xcd`Region.makeLowerTriangular(N)` returns a region corresponding
to the non-zero indices in a lower-triangular \xcd`N x N` matrix.
\item 
  If \xcd`R` is a region, and \xcd`p` a Point of the same rank, then 
%~~exp~~`~~`~~R:Region, p:Point(R.rank) ~~ import x10.regionarray.*; ^^^ Arrays50
  \xcd`R+p` is \xcd`R` translated forwards by 
  \xcd`p` -- the region whose
%~~exp~~`~~`~~r:Point, p:Point(r.rank) ~~ import x10.regionarray.*; ^^^ Arrays60
  points are \xcd`r+p` 
  for each \xcd`r` in \xcd`R`.
\item 
  If \xcd`R` is a region, and \xcd`p` a Point of the same rank, then 
%~~exp~~`~~`~~R:Region, p:Point(R.rank) ~~ import x10.regionarray.*; ^^^ Arrays70
  \xcd`R-p` is \xcd`R` translated backwards by 
  \xcd`p` -- the region whose
%~~exp~~`~~`~~r:Point, p:Point(r.rank) ~~ import x10.regionarray.*; ^^^ Arrays80
  points are \xcd`r-p` 
  for each \xcd`r` in \xcd`R`.

\end{itemize}

All the points in a region are ordered canonically by the
lexicographic total order. Thus the points of the region \xcd`(1..2)*(1..2)`
are ordered as 
\begin{xten}
(1,1), (1,2), (2,1), (2,2)
\end{xten}
Sequential iteration statements such as \xcd`for` (\Sref{ForAllLoop})
iterate over the points in a region in the canonical order.

A region is said to be {\em rectangular}\index{region!convex} if it is of
the form \xcdmath`(T$_1$ * $\cdots$ * T$_k$)` for some set of intervals
\xcdmath`T$_i = $ l$_i$ .. h$_i$ `. 
In particular an \xcd`IntRange` turned into a \xcd`Region` is rectangular: 
%~~exp~~`~~`~~ ~~ import x10.regionarray.*; ^^^Arrays3x4z
\xcd`Region.make(1..10)`.
Such a
region satisfies the property that if two points $p_1$ and $p_3$ are
in the region, then so is every point $p_2$ between them (that is, it is {\em convex}). 
(Banded and triangular regions are not rectangular.)
The operation
%~~exp~~`~~`~~R:Region ~~ import x10.regionarray.*; ^^^ Arrays90
\xcd`R.boundingBox()` gives the smallest rectangular region containing
\xcd`R`.

\subsubsection{Operations on regions}
\index{region!operations}

Let \xcd`R` be a region. A {\em sub-region} is a subset of \xcd"R".
\index{region!sub-region}

Let \xcdmath`R1` and \xcdmath`R2` be two regions whose types establish that
they are of the same rank. Let \xcdmath`S` be another region; its rank is
irrelevant. 

\xcdmath`R1 && R2` is the intersection of \xcdmath`R1` and
\xcdmath`R2`, \viz, the region containing all points which are in both
\Xcd{R1} and \Xcd{R2}.  \index{region!intersection}
%~~exp~~`~~`~~ ~~ import x10.regionarray.*; ^^^ Arrays100
For example, \xcd`Region.make(1..10) && Region.make(2..20)` is \Xcd{2..10}.


\xcdmath`R1 * S` is the Cartesian product of \xcdmath`R1` and
\xcdmath`S`,  formed by pairing each point in \xcdmath`R1` with every  point in \xcdmath`S`.
\index{region!product}
%~~exp~~`~~`~~ ~~ import x10.regionarray.*; ^^^ Arrays110
Thus, \xcd`Region.make(1..2)*Region.make(3..4)*Region.make(5..6)`
is the region of rank \Xcd{3} containing the eight points with coordinates
\xcd`[1,3,5]`, \xcd`[1,3,6]`, \xcd`[1,4,5]`, \xcd`[1,4,6]`,
\xcd`[2,3,5]`, \xcd`[2,3,6]`, \xcd`[2,4,5]`, \xcd`[2,4,6]`.


For a region \xcdmath`R` and point \xcdmath`p` of the same rank,
%~~exp~~`~~`~~R:Region, p:Point{p.rank==R.rank} ~~ import x10.regionarray.*; ^^^ Arrays120
\xcd`R+p` 
and
%~~exp~~`~~`~~R:Region, p:Point{p.rank==R.rank} ~~ import x10.regionarray.*; ^^^ Arrays130
\xcd`R-p` 
represent the translation of the region
forward 
and backward 
by \xcdmath`p`. That is, \Xcd{R+p} is the set of points
\Xcd{p+q} for all \Xcd{q} in \Xcd{R}, and \Xcd{R-p} is the set of \Xcd{q-p}.

More \Xcd{Region} methods are described in the API documentation.

\subsection{Arrays}
\index{array}

Arrays are organized data, arranged so that it can be accessed by subscript.
An \xcd`Array[T]` \Xcd{A} has a \Xcd{Region} \Xcd{A.region}, telling which
\Xcd{Point}s are in \Xcd{A}.  For each point \Xcd{p} in \Xcd{A.region},
\Xcd{A(p)} is the datum of type \Xcd{T} associated with \Xcd{p}.  X10
implementations should 
attempt to store \xcd`Array`s efficiently, and to make array element accesses
quick---\eg, avoiding constructing \Xcd{Point}s when unnecessary.

This generalizes the concepts of arrays appearing in many other programming
languages.  A \Xcd{Point} may have any number of coordinates, so an
\xcd`Array` can have, in effect, any number of integer subscripts.  

\begin{ex}Indeed, it is possible to write code that works on \Xcd{Array}s regardless 
of dimension.  For example, to add one \Xcd{Array[Int]} \Xcd{src} into another
\Xcd{dest}, 
%~~gen ^^^ Arrays140
%package Arrays.Arrays.Arrays.Example;
%import x10.regionarray.*;
% class Example{
%~~vis
\begin{xten}
static def addInto(src: Array[Int], dest:Array[Int])
  {src.region == dest.region}
  = {
    for (p in src.region) 
       dest(p) += src(p);
  }
\end{xten}
%~~siv
%}
% class Hook{
%   def run() { 
%     val a = new Array[Int](3, [1,2,3]);
%     val b = new Array[Int](a.region, (p:Point(1)) => 10*a(p) );
%     Example.addInto(a, b);
%     return b(0) == 11 && b(1) == 22 && b(2) == 33;
% }}
%~~neg
\noindent
Since \Xcd{p} is a \Xcd{Point}, it can hold as many coordinates as are
necessary for the arrays \Xcd{src} and \Xcd{dest}.
\end{ex}

The basic operation on arrays is subscripting: if \Xcd{A} is an \Xcd{Array[T]}
and \Xcd{p} a point with the same rank as \xcd`A.region`, then
%~~exp~~`~~`~~A:Array[Int], p:Point{self.rank == A.region.rank} ~~ import x10.regionarray.*; ^^^ Arrays150
\xcd`A(p)`
is the value of type \Xcd{T} associated with point \Xcd{p}.
This is the same operation as function application
(\Sref{sect:FunctionApplication}); arrays implement function types, and can be
used as functions.

Array elements can be changed by assignment. If \Xcd{t:T}, 
%~~gen ^^^ Arrays160
%package Arrays.Arrays.Subscripting.Is.From.Mars;
%import x10.regionarray.*; 
%class Example{
%def example[T](A:Array[T], p: Point{rank == A.region.rank}, t:T){
%~~vis
\begin{xten}
A(p) = t;
\end{xten}
%~~siv
%} } 
%~~neg
modifies the value associated with \Xcd{p} to be \Xcd{t}, and leaves all other
values in \Xcd{A} unchanged.

An \Xcd{Array[T]} named \Xcd{a} has: 
\begin{itemize}
%~~exp~~`~~`~~a:Array[Int] ~~ import x10.regionarray.*; ^^^ Arrays170
\item \xcd`a.region`: the \Xcd{Region} upon which \Xcd{a} is defined.
%~~exp~~`~~`~~a:Array[Int] ~~ import x10.regionarray.*; ^^^ Arrays180
\item \xcd`a.size`: the number of elements in \Xcd{a}.
%~~exp~~`~~`~~a:Array[Int] ~~ import x10.regionarray.*; ^^^ Arrays190
\item \xcd`a.rank`, the rank of the points usable to subscript \Xcd{a}. 
      \xcd`a.rank` is a cached copy of 
      \Xcd{a.region.rank}.
\end{itemize}

\subsubsection{Array Constructors}
\index{array!constructor}

To construct an array whose elements all have the same value \Xcd{init}, call
\Xcd{new Array[T](R, init)}. 
For example, an array of a thousand \xcd`"oh!"`s can be made by:
%~~exp~~`~~`~~ ~~ import x10.regionarray.*; ^^^ Arrays200
\xcd`new Array[String](1000, "oh!")`.


To construct and initialize an array, call the two-argument constructor. 
\Xcd{new Array[T](R, f)} constructs an array of elements of type \Xcd{T} on
region \Xcd{R}, with \Xcd{a(p)} initialized to \Xcd{f(p)} for each point
\Xcd{p} in \Xcd{R}.  \Xcd{f} must be a function taking a point of rank
\Xcd{R.rank} to a value of type \Xcd{T}.  

\begin{ex}
One way to construct the array \xcd`[11, 22, 33]` is with an array constructor
%~~exp~~`~~`~~ ~~ import x10.regionarray.*; ^^^ Arrays210
\xcd`new Array[Int](3, (i:long)=>(11*i) as Int)`. 
To construct a multiplication table, call
%~~exp~~`~~`~~ ~~ import x10.regionarray.*; ^^^ Arrays220
\xcd`new Array[Long](Region.make(0..9, 0..9), (p:Point(2)) => p(0)*p(1))`.
\end{ex}

Other constructors are available; see the API documentation and
\Sref{sect:RailCtors}. 

\subsubsection{Array Operations}
\index{array!operations on}

The basic operation on \Xcd{Array}s is subscripting.  If \Xcd{a:Array[T]} and 
\xcd`p:Point{rank == a.rank}`, then \Xcd{a(p)} is the value of type \Xcd{T}
appearing at position \Xcd{p} in \Xcd{a}.    The syntax is identical to
function application, and, indeed, arrays may be used as functions.
\Xcd{a(p)} may be assigned to, as well, by the usual assignment syntax
%~~exp~~`~~`~~a:Array[Int], p:Point{rank == a.rank}, t:Int ~~ import x10.regionarray.*; ^^^ Arrays230
\xcd`a(p)=t`.
(This uses the application and setting syntactic sugar, as given in \Sref{set-and-apply}.)

Sometimes it is more convenient to subscript by integers.  Arrays of rank 1-4
can, in fact, be accessed by integers: 
%~~gen ^^^ Arrays240
%package Arrays240;
%import x10.regionarray.*;
%class Example{
%static def example(){
%~~vis
\begin{xten}
val A1 = new Array[Int](10, 0);
A1(4) = A1(4) + 1;
val A4 = new Array[Int](Region.make(1..2, 1..3, 1..4, 1..5), 0);
A4(2,3,4,5) = A4(1,1,1,1)+1;
\end{xten}
%~~siv
% assert A1(4) == 1 && A4(2,3,4,5) == 1;
%}}
% class Hook{ def run() {Example.example(); return true;}}
%~~neg



Iteration over an \Xcd{Array} is defined, and produces the \Xcd{Point}s of the
array's region.  If you want to use the values in the array, you have to
subscript it.  For example, you could take the logarithm of every element of an
\Xcd{Array[Double]} by: 
%~~gen ^^^ Arrays250
%package Arrays250;
%import x10.regionarray.*;
%class Example{
%static def example(a:Array[Double]) {
%~~vis
\begin{xten}
for (p in a) a(p) = Math.log(a(p));
\end{xten}
%~~siv
%}}
% class Hook{ def run() { val a = new Array[Double](2, [1.0,2.0]); Example.example(a); return a(0)==Math.log(1.0) && a(1)==Math.log(2.0); }}

%~~neg



\subsection{Distributions}\label{XtenDistributions}
\index{distribution}

Distributed arrays are spread across multiple \xcd`Place`s.  
A {\em distribution}, a mapping from a region to a set of places, 
describes where each element of a distributed array is kept.
Distributions are embodied by the class \Xcd{x10.regionarray.Dist} and its
subclasses. 
The {\em rank} of a distribution is the rank of the underlying region, and
thus the rank of every point that the distribution applies to.


\begin{ex}
%~~gen ^^^ Arrays260
%package Arrays.Dist_example_a;
%import x10.regionarray.*;
% class Example{
% def example() {
%~~vis
\begin{xten}
val R  <: Region = Region.make(1..100);
val D1 <: Dist = Dist.makeBlock(R);
val D2 <: Dist = Dist.makeConstant(R, here);
\end{xten}
%~~siv
% } } 
%~~neg

\xcd`D1` distributes the region \xcd`R` in blocks, with a set of consecutive
points at each place, as evenly as possible.  \xcd`D2` maps all the points in
\xcd`R` to \xcd`here`.  
\end{ex}

Let \xcd`D` be a distribution. 
%~~exp~~`~~`~~D:Dist ~~ import x10.regionarray.*; ^^^ Arrays270
\xcd`D.region` 
denotes the underlying
region. 
Given a point \xcd`p`, the expression
%~~exp~~`~~`~~ D:Dist, p:Point{p.rank == D.rank}~~ import x10.regionarray.*; ^^^ Arrays280
\xcd`D(p)` represents the application of \xcd`D` to \xcd`p`, that is,
the place that \xcd`p` is mapped to by \xcd`D`. The evaluation of the
expression \xcd`D(p)` throws an \xcd`ArrayIndexOutofBoundsException`
if \xcd`p` does not lie in the underlying region.


\subsubsection{{\tt PlaceGroup}s}

A \xcd`PlaceGroup` represents an ordered set of \xcd`Place`s.
\xcd`PlaceGroup`s exist for performance and scaleability: they are more
efficient, in certain critical places, than general collections of
\xcd`Place`. \xcd`PlaceGroup` implements \xcd`Sequence[Place]`, and thus
provides familiar operations -- \xcd`pg.size()` for the number of places,
\xcd`pg.iterator()` to iterate over them, etc.  

\xcd`PlaceGroup` is an abstract class.  The concrete class
\xcd`SparsePlaceGroup` is intended for a small group of places. 
%~~exp~~`~~`~~ somePlace:Place ~~ ^^^Arrays1j6q
\xcd`new SparsePlaceGroup(somePlace)` is a good \xcd`PlaceGroup` containing
one place.  
%~~exp~~`~~`~~ seqPlaces: Rail[Place] ~~ ^^^Arrays9g6f
\xcd`new SparsePlaceGroup(seqPlaces)`
constructs a sparse place group from a Rail of places.

\subsubsection{Operations returning distributions}
\index{distribution!operations}



Let \xcd`R` be a region, \xcd`Q` 
a \xcd`PlaceGroup`, and \xcd`P` a place.

\paragraph{Unique distribution} \index{distribution!unique}
%~~exp~~`~~`~~Q:PlaceGroup ~~ import x10.regionarray.*; ^^^ Arrays290
The distribution \xcd`Dist.makeUnique(Q)` is the unique distribution from the
region \xcd`Region.make(1..k)` to \xcd`Q` mapping each point \xcd`i` to
\xcd`pi`.


\paragraph{Constant distributions.} \index{distribution!constant}
%~~exp~~`~~`~~R:Region, P:Place ~~ import x10.regionarray.*; ^^^ Arrays300
The distribution \xcd`Dist.makeConstant(R,P)` maps every point in region
\xcd`R` to place \xcd`P`.  
%~~exp~~`~~`~~R:Region ~~ import x10.regionarray.*; ^^^Arrays9n5n
The special case \xcd`Dist.makeConstant(R)` maps every point in \xcd`R` to
\xcd`here`. 

\paragraph{Block distributions.}\index{distribution!block}
%~~exp~~`~~`~~R:Region ~~ import x10.regionarray.*; ^^^ Arrays320
The distribution \xcd`Dist.makeBlock(R)` distributes the elements of \xcd`R`,
in approximately-even blocks, over all the places available to the program. 
There are other \xcd`Dist.makeBlock` methods capable of controlling the
distribution and the set of places used; see the API documentation.


\paragraph{Domain Restriction.} \index{distribution!restriction!region}

If \xcd`D` is a distribution and \xcd`R` is a sub-region of {\cf
%~~exp~~`~~`~~D:Dist,R :Region{R.rank==D.rank} ~~ import x10.regionarray.*; ^^^ Arrays330
D.region}, then \xcd`D | R` represents the restriction of \xcd`D` to
\xcd`R`---that is, the distribution that takes each point \xcd`p` in \xcd`R`
to 
%~~exp~~`~~`~~D:Dist, p:Point{p.rank==D.rank} ~~ import x10.regionarray.*; ^^^ Arrays340
\xcd`D(p)`, 
but doesn't apply to any points but those in \xcd`R`.

\paragraph{Range Restriction.}\index{distribution!restriction!range}

If \xcd`D` is a distribution and \xcd`P` a place expression, the term
%~~exp~~`~~`~~ D:Dist, P:Place~~ import x10.regionarray.*; ^^^ Arrays350
\xcd`D | P` 
denotes the sub-distribution of \xcd`D` defined over all the
points in the region of \xcd`D` mapped to \xcd`P`.

Note that \xcd`D | here` does not necessarily contain adjacent points
in \xcd`D.region`. For instance, if \xcd`D` is a cyclic distribution,
\xcd`D | here` will typically contain points that differ by the number of
places. 
An implementation may find a
way to still represent them in contiguous memory, \eg, using an arithmetic
function to map from the region index to an index 
into the array.


\subsection{Distributed Arrays}
\index{array!distributed}
\index{distributed array}
\index{\Xcd{DistArray}}
\index{DistArray}

Distributed arrays, instances of \xcd`DistArray[T]`, are very much like
\xcd`Array`s, except that they distribute information among multiple
\xcd`Place`s according to a \xcd`Dist` value passed in as a constructor
argument.  

\begin{ex}The following code creates a distributed array holding
a thousand cells, each initialized to 0.0, distributed via a block
distribution over all places.
%~~gen ^^^ Arrays360
% package Arrays.Distarrays.basic.example;
% import x10.regionarray.*;
% class Example {
% def example() {
%~~vis
\begin{xten}
val R <: Region = Region.make(1..1000);
val D <: Dist = Dist.makeBlock(R);
val da <: DistArray[Float] 
       = DistArray.make[Float](D, (Point(1))=>0.0f);
\end{xten}
%~~siv
%}}
%~~neg
\end{ex}



\subsection{Distributed Array Construction}\label{ArrayInitializer}
\index{distributed array!creation}
\index{\Xcd{DistArray}!creation}
\index{DistArray!creation}

\xcd`DistArray`s are instantiated by invoking one of the \xcd`make` factory
methods of the \xcd`DistArray` class.
A \xcd`DistArray` creation 
must take either an \xcd`Int` as an argument or a \xcd`Dist`. In the first
case,  a distributed array is created over the distribution 
%~~exp~~`~~`~~N:Int ~~ import x10.regionarray.*; ^^^Arrays1s6g
\xcd`Dist.makeConstant(Region.make(0, N-1),here)`;
in the second over the given distribution. 

\begin{ex}A distributed array creation operation may also specify an initializer
function.
The function is applied in parallel
at all points in the domain of the distribution. The
construction operation terminates locally only when the \xcd`DistArray` has been
fully created and initialized (at all places in the range of the
distribution).

For instance:
%~~gen ^^^ Arrays370
% package Arrays.DistArray.Construction.FeralWolf;
% import x10.regionarray.*;
% class Example {
% def example() {
%~~vis
\begin{xten}
val ident = ([i]:Point(1)) => i;
val data : DistArray[Long]
    = DistArray.make[Long](Dist.makeConstant(Region.make(1, 9)), ident);
val blk = Dist.makeBlock(Region.make(1..9, 1..9));
val data2 : DistArray[Long]
    = DistArray.make[Long](blk, ([i,j]:Point(2)) => i*j);
\end{xten}
%~~siv
% }  }
%~~neg




{}\noindent 
The first declaration stores in \xcd`data` a reference to a mutable
distributed array with \xcd`9` elements each of which is located in the
same place as the array. The element at \Xcd{[i]} is initialized to its index
\xcd`i`. 

The second declaration stores in \xcd`data2` a reference to a mutable
two-dimensional distributed array, whose coordinates both range from 1 to
9, distributed in blocks over all \xcd`Place`s, 
initialized with \xcd`i*j`
at point \xcd`[i,j]`.
\end{ex}


\subsection{Operations on Arrays and Distributed Arrays}

Arrays and distributed arrays share many operations.
In the following, let \xcd`a` be an array with base type T, and \xcd`da` be an
array with distribution \xcd`D` and base type \xcd`T`.




\subsubsection{Element operations}\index{array!access}
The value of \xcd`a` at a point \xcd`p` in its region of definition is
%~~exp~~`~~`~~a:Array[Int](3), p:Point(3) ~~ import x10.regionarray.*; ^^^ Arrays380
obtained by using the indexing operation \xcd`a(p)`. 
The value of \xcd`da` at \xcd`p` is similarly
%~~exp~~`~~`~~da:DistArray[Int](3), p:Point(3) ~~ import x10.regionarray.*; ^^^ Arrays390
\xcd`da(p)`.
This operation
may be used on the left hand side of an assignment operation to update
the value: 
%~~stmt~~`~~`~~a:Array[Int](3), p:Point(3), t:Int ~~ import x10.regionarray.*; ^^^ Arrays400
\xcd`a(p)=t;`
and 
%~~stmt~~`~~`~~da:DistArray[Int](3), p:Point(3), t:Int ~~ import x10.regionarray.*; ^^^ Arrays410
\xcd`da(p)=t;`
The operator assignments, \xcd`a(i) += e` and so on,  are also
available. 

It is a runtime error to 
access arrays, with \xcd`da(p)` or \xcd`da(p)=v`, at a place
other than \xcd`da.dist(p)`, \viz{} at the place that the element exists. 


\subsubsection{Arrays of Single Values}\label{ConstantArray}
\index{array!constant promotion}

For a region \xcd`R` and a value \xcd`v` of type \xcd`T`, the expression 
%~~genexp~~`~~`~~T~~R:Region{self!=null}, v:T ~~ import x10.regionarray.*; ^^^ Arrays420
\xcd`new Array[T](R, v)` 
produces an array on region \xcd`R` initialized with value \xcd`v`.
Similarly, 
for a distribution \xcd`D` and a value \xcd`v` of
type \xcd`T` the expression 
\begin{xtenmath}
DistArray.make[T](D, (Point(D.rank))=>v)
\end{xtenmath}
constructs a distributed array with
distribution \xcd`D` and base type \xcd`T` initialized with \xcd`v`
at every point.

Note that \xcd`Array`s are constructed by constructor calls, but
\xcd`DistArrays` are constructed by calls to the factory methods
\xcd`DistArray.make`. This is because \xcd`Array`s are fairly simple objects,
but \xcd`DistArray`s may be implemented by different classes for different
distributions. The use of the factory method gives the library writer the
freedom to select appropriate implementations.


\subsubsection{Restriction of an array}\index{array!restriction}

Let \xcd`R` be a sub-region of \xcd`da.region`. Then 
%~~exp~~`~~`~~da:DistArray[Int](3), p:Point(3), R: Region(da.rank) ~~ import x10.regionarray.*; ^^^ Arrays440
\xcd`da | R`
represents the sub-\xcd`DistArray` of \xcd`da` on the region \xcd`R`.
That is, \xcd`da | R` has the same values as \xcd`da` when subscripted by a
%~~exp~~`~~`~~R:Region, da: DistArray[Int]{da.region.rank == R.rank} ~~ import x10.regionarray.*; ^^^ Arrays450
point in region \xcd`R && da.region`, and is undefined elsewhere.

Recall that a rich set of operators are available on distributions
(\Sref{XtenDistributions}) to obtain sub-distributions
(e.g. restricting to a sub-region, to a specific place etc).


\subsubsection{Operations on Whole Arrays}

\paragraph{Pointwise operations}\label{ArrayPointwise}\index{array!pointwise operations}
The unary \xcd`map` operation applies a function to each element of
a distributed or non-distributed array, returning a new distributed array with
the same distribution, or a non-distributed array with the same region.

The following produces an array of cubes: 
%~~gen ^^^ Arrays460
%package Arrays_pointwise_a;
%import x10.regionarray.*;
%class Example{
%static def example() {
%~~vis
\begin{xten}
val A = new Array[Int](11, (i:long)=>i as Int);
assert A(3) == 3 && A(4) == 4 && A(10) == 10; 
val cube = (i:Int) => i*i*i;
val B = A.map(cube);
assert B(3) == 27 && B(4) == 64 && B(10) == 1000; 
\end{xten}
%~~siv
%} } 
% class Hook{ def run() {Example.example(); return true;}}
%~~neg

A variant operation lets you specify the array \Xcd{B} into which the result
will be stored, 
%~~gen ^^^ Arrays470
%package Arrays.map_with_result;
%import x10.regionarray.*;
%class Example{
%static def example() {
%~~vis
\begin{xten}
val A = new Array[Int](11, (i:long)=>i as Int);
assert A(3) == 3 && A(4) == 4 && A(10) == 10; 
val cube = (i:Int) => i*i*i;
val B = new Array[Int](A.region); // B = 0,0,0,0,0,0,0,0,0,0,0
A.map(B, cube);
assert B(3) == 27 && B(4) == 64 && B(10) == 1000; 
\end{xten}
%~~siv
%} } 
% class Hook{ def run() {Example.example(); return true;}}
%~~neg
\noindent
This is convenient if you have an already-allocated array lying around unused.
In particular, it can be used if you don't need \Xcd{A} afterwards and want to
reuse its space:
%~~gen ^^^ Arrays480
%package Arrays.map_reusing_space;
%import x10.regionarray.*;
%class Example{
%static def example() {
%~~vis
\begin{xten}
val A = new Array[Int](11, (i:long)=>i as Int);
assert A(3) == 3 && A(4) == 4 && A(10) == 10; 
val cube = (i:Int) => i*i*i;
A.map(A, cube);
assert A(3) == 27 && A(4) == 64 && A(10) == 1000; 
\end{xten}
%~~siv
%} } 
% class Hook{ def run() {Example.example(); return true;}}
%~~neg


The binary \xcd`map` operation takes a binary function and
another
array over the same region or distributed array over the same  distribution,
and applies the function 
pointwise to corresponding elements of the two arrays, returning
a new array or distributed array of the same shape.
The following code adds two distributed arrays: 
%~~gen ^^^ Arrays490
% package Arrays.Pointwise.Pointless.Map2;
% import x10.regionarray.*;
% class Example{
%~~vis
\begin{xten}
static def add(da:DistArray[Int], db: DistArray[Int])
    {da.dist==db.dist}
    = da.map(db, (a:Int,b:Int)=>a+b);
\end{xten}
%~~siv
%}
%~~neg



\paragraph{Reductions}\label{ArrayReductions}\index{array!reductions}

Let \xcd`f` be a function of type \xcd`(T,T)=>T`.  Let
\xcd`a` be an array over base type \xcd`T`.
Let \xcd`unit` be a value of type \xcd`T`.
Then the
%~~genexp~~`~~`~~ T ~~ f:(T,T)=>T, a : Array[T], unit:T ~~ import x10.regionarray.*; ^^^ Arrays500
operation \xcd`a.reduce(f, unit)` returns a value of type \xcd`T` obtained
by combining all the elements of \xcd`a` by use of  \xcd`f` in some unspecified order
(perhaps in parallel).   
The following code gives one method which 
meets the definition of \Xcd{reduce},
having a running total \Xcd{r}, and accumulating each value \xcd`a(p)` into it
using \Xcd{f} in turn.  (This code is simply given as an example; \Xcd{Array}
has this operation defined already.)
%~~gen ^^^ Arrays510
%package Arrays.Reductions.And.Eliminations;
%import x10.regionarray.*;
% class Example {
%~~vis
\begin{xten}
def oneWayToReduce[T](a:Array[T], f:(T,T)=>T, unit:T):T {
  var r : T = unit;
  for(p in a.region) r = f(r, a(p));
  return r;
}
\end{xten}
%~~siv
%}
%~~neg


For example,  the following sums an array of integers.  \Xcd{f} is addition,
and \Xcd{unit} is zero.  
%~~gen ^^^ Arrays520
% package Arrays.Reductions.And.Emulsions;
%import x10.regionarray.*;
% class Example {
% static def example() {
%~~vis
\begin{xten}
val a = new Array[Int](4, (i:long)=>(i+1) as Int);
val sum = a.reduce((a:Int,b:Int)=>a+b, 0); 
assert(sum == 10); // 10 == 1+2+3+4
\end{xten}
%~~siv
%}}
% class Hook{ def run() {Example.example(); return true;}}
%~~neg

Other orders of evaluation, degrees of parallelism, and applications of
\Xcd{f(x,unit)} and \xcd`f(unit,x)`are also correct.
In order to guarantee that the result is precisely
determined, the  function \xcd`f` should be associative and
commutative, and the value \xcd`unit` should satisfy
\xcd`f(unit,x)` \xcd`==` \xcd`x` \xcd`==` \xcd`f(x,unit)`
for all \Xcd{x:T}.  




\xcd`DistArray`s have the same operation.
This operation involves communication between the places over which
the \xcd`DistArray` is distributed. The \Xten{} implementation guarantees that
only one value of type \xcd`T` is communicated from a place as part of
this reduction process.

\paragraph{Scans}\label{ArrayScans}\index{array!scans}


Let \xcd`f:(T,T)=>T`, \xcd`unit:T`, and \xcd`a` be an \xcd`Array[T]` or
\xcd`DistArray[T]`.  Then \xcd`a.scan(f,unit)` is the array or distributed
array of type \xcd`T` whose {$i$}th element in canonical order is the
reduction by \xcd`f` with unit \xcd`unit` of the first {$i$} elements of
\xcd`a`. 


This operation involves communication between the places over which the
distributed array is distributed. The \Xten{} implementation will endeavour to
minimize the communication between places to implement this operation.

Other operations on arrays, distributed arrays, and the related classes may be
found in the \xcd`x10.regionarray` package.
	
\chapter{Annotations}\label{XtenAnnotations}\index{annotations}


\Xten{} provides an 
an annotation system  system for to allow the
compiler to be extended with new static analyses and new
transformations.

Annotations are constraint-free interface types that decorate the abstract syntax tree
of an \Xten{} program.  The \Xten{} type-checker ensures that an annotation
is a legal interface type.
In \Xten{}, interfaces may declare
both methods and properties.  Therefore, like any interface type, an
annotation may instantiate
one or more of its interface's properties.
%%PLUGINNERY%%  Unlike with Java
%%PLUGINNERY%%  annotations,
%%PLUGINNERY%%  property initializers need not be
%%PLUGINNERY%%  compile-time constants;
%%PLUGINNERY%%  however, a given compiler plugin
%%PLUGINNERY%%  may do additional checks to constrain the allowable
%%PLUGINNERY%%  initializer expressions.
%%PLUGINNERY%%  The \Xten{} type-checker does not check that
%%PLUGINNERY%%  all properties of an annotation are initialized,
%%PLUGINNERY%%  although this could be enforced by
%%PLUGINNERY%%  a compiler plugin.

\section{Annotation syntax}

The annotation syntax consists of an ``\texttt{@}'' followed by an interface type.

%##(Annotations Annotation
\begin{bbgrammar}
%(FROM #(prod:Annotations)#)
         Annotations \: Annotation & (\ref{prod:Annotations}) \\
                    \| Annotations Annotation \\
%(FROM #(prod:Annotation)#)
          Annotation \: \xcd"@" NamedType & (\ref{prod:Annotation}) \\
\end{bbgrammar}
%##)

Annotations can be applied to most syntactic constructs in the language
including class declarations, constructors, methods, field declarations,
local variable declarations and formal parameters, statements,
expressions, and types.
Multiple occurrences of the same annotation (i.e., multiple
annotations with the same interface type) on the same entity are permitted.

%%OBSOLETE%% \begin{grammar}
%%OBSOLETE%% ClassModifier \: Annotation \\
%%OBSOLETE%% InterfaceModifier \: Annotation \\
%%OBSOLETE%% FieldModifier \: Annotation \\
%%OBSOLETE%% MethodModifier \: Annotation \\
%%OBSOLETE%% VariableModifier \: Annotation \\
%%OBSOLETE%% ConstructorModifier \: Annotation \\
%%OBSOLETE%% AbstractMethodModifier \: Annotation \\
%%OBSOLETE%% ConstantModifier \: Annotation \\
%%OBSOLETE%% Type \: AnnotatedType \\
%%OBSOLETE%% AnnotatedType \: Annotation\plus Type \\
%%OBSOLETE%% Statement \: AnnotatedStatement \\
%%OBSOLETE%% AnnotatedStatement \: Annotation\plus Statement \\
%%OBSOLETE%% Expression \: AnnotatedExpression \\
%%OBSOLETE%% AnnotatedExpression \: Annotation\plus Expression \\
%%OBSOLETE%% \end{grammar}

\noindent
Recall that interface types may have dependent parameters.

\noindent
The following examples illustrate the syntax:

\begin{itemize}
\item Declaration annotations:
\begin{xtennoindent}
  // class annotation
  @Value
  class Cons { ... }

  // method annotation
  @PreCondition(0 <= i && i < this.size)
  public def get(i: Int): Object { ... }

  // constructor annotation
  @Where(x != null)
  def this(x: T) { ... }

  // constructor return type annotation
  def this(x: T): C@Initialized { ... }

  // variable annotation
  @Unique x: A;
\end{xtennoindent}
\item Type annotations:
\begin{xtennoindent}
  List@Nonempty

  Int@Range(1,4)

  Array[Array[Double]]@Size(n * n)
\end{xtennoindent}
\item Expression annotations:
\begin{xtennoindent}
  m() : @RemoteCall
\end{xtennoindent}
\item Statement annotations:
\begin{xtennoindent}
  @Atomic { ... }

  @MinIterations(0)
  @MaxIterations(n)
  for (var i: Int = 0; i < n; i++) { ... }

  // An annotated empty statement ;
  @Assert(x < y);
\end{xtennoindent}
\end{itemize}

\section{Annotation declarations}

Annotations are declared as interfaces.  They must be
subtypes of the interface \texttt{x10.lang.annotation.Annotation}.
Annotations on particular static entities must extend the corresponding
\xcd`Annotation` subclasses, as follows: 
\begin{itemize}
\item Expressions---\xcd`ExpressionAnnotation`
\item Statements---\xcd`StatementAnnotation`
\item Classes---\xcd`ClassAnnotation`
\item Fields---\xcd`FieldAnnotation`
\item Methods---\xcd`MethodAnnotation`
\item Imports---\xcd`ImportAnnotation`
\item Packages---\xcd`PackageAnnotation`
\end{itemize}

\chapter{Interoperability with Other Languages}
\label{NativeCode}
\index{native code}
\label{Interoperability}
\index{interoperability}

The ability to interoperate with other programming languages is an
essential feature of the \Xten{} implementation.  Cross-language
interoperability enables both the incremental adoption of \Xten{} in
existing applications and the usage of existing libraries and
frameworks by newly developed \Xten{} programs. 

There are two primary interoperability scenarios that are supported by
\XtenCurrVer{}: inline substitution of fragments of foreign code for
\Xten program fragments (expressions, statements) and external linkage
to foreign code.

\section{Embedded Native Code Fragments}

The
\xcd`@Native(lang,code) Construct` annotation from \xcd`x10.compiler.Native` in
\Xten{} can be used to tell the \Xten{} compiler to substitute \xcd`code` for
whatever it would have generated when compiling \xcd`Construct`
with the \xcd`lang` back end.

The compiler cannot analyze native code the same way it analyzes \Xten{} code.  In
particular, \xcd`@Native` fields and methods must be explicitly typed; the
compiler will not infer types.

\subsection{Native {\tt static} Methods}

\xcd`static` methods can be given native implementations.  Note that these
implementations are syntactically {\em expressions}, not statements, in C++ or
Java.   Also, it is possible (and common) to provide native implementations
into both Java and C++ for the same method.
%~~gen ^^^ extern10
% package Extern.or_current_turn;
%~~vis
\begin{xten}
import x10.compiler.Native;
class Son {
  @Native("c++", "printf(\"Hi!\")")
  @Native("java", "System.out.println(\"Hi!\")")
  static def printNatively():void = {};
}
\end{xten}
%~~siv
%
%~~neg

If only some back-end languages are given, the \Xten{} code will be used for the
remaining back ends: 
%~~gen ^^^ extern20
% package Extern.or.burn;
%~~vis
\begin{xten}
import x10.compiler.Native;
class Land {
  @Native("c++", "printf(\"Hi from C++!\")")
  static def example():void = {
    x10.io.Console.OUT.println("Hi from X10!");
  };
}
\end{xten}
%~~siv
%
%~~neg

The \xcd`native` modifier on methods indicates that the method must not have
an \Xten{} code body, and \xcd`@Native` implementations must be given for all back
ends:
%~~gen ^^^ extern30
% package Extern.or_maybe_getting_back_together;
%~~vis
\begin{xten}
import x10.compiler.Native;
class Plants {
  @Native("c++", "printf(\"Hi!\")")
  @Native("java", "System.out.println(\"Hi!\")")
  static native def printNatively():void;
}
\end{xten}
%~~siv
%
%~~neg


Values may be returned from external code to \Xten{}.  Scalar types in Java and
C++ correspond directly to the analogous types in \Xten{}.  
%~~gen ^^^ extern40
% package Extern.hardy;
%~~vis
\begin{xten}
import x10.compiler.Native;
class Return {
  @Native("c++", "1")
  @Native("java", "1")
  static native def one():Int;
}
\end{xten}
%~~siv
%
%~~neg
Types are {\em not} inferred for methods marked as \xcd`@Native`.

Parameters may be passed to external code.  \xcd`(#1)`  is the first parameter,
\xcd`(#2)` the second, and so forth.  (\xcd`(#0)` is the name of the enclosing
class, or the \xcd`this` variable.)  Be aware that this is macro substitution
rather than normal parameter 
passing; \eg, if the first actual parameter is \xcd`i++`, and \xcd`(#1)`
appears twice in the external code, \xcd`i` will be incremented twice.
For example, a (ridiculous) way to print the sum of two numbers is: 
%~~gen ^^^ extern50
% package Extern.or_turnabout_is_fair_play;
%~~vis
\begin{xten}
import x10.compiler.Native;
class Species {
  @Native("c++","printf(\"Sum=%d\", ((#1)+(#2)) )")
  @Native("java","System.out.println(\"\" + ((#1)+(#2)))")
  static native def printNatively(x:Int, y:Int):void;
}
\end{xten}
%~~siv
%
%~~neg


Static variables in the class are available in the external code.  For Java,
the static variables are used with their \Xten{} names.  For C++, the names
must be mangled, by use of the \xcd`FMGL` macro.  

%~~gen ^^^ extern60
%package Extern.or.Die;
%~~vis
\begin{xten}
import x10.compiler.Native;
class Ability {
  static val A : Int = 1n;
  @Native("java", "A+2")
  @Native("c++", "Ability::FMGL(A)+2")
  static native def fromStatic():Int;
}
\end{xten}
%~~siv
%
%~~neg




\subsection{Native Blocks}

Any block may be annotated with \xcd`@Native(lang,stmt)`, indicating that, in
the given back end, it should be implemented as \xcd`stmt`. All 
variables  from the surrounding context are available inside \xcd`stmt`. For
example, the method call \xcd`born.example(10)`, if compiled to Java, changes
the field \xcd`y` of a \xcd`Born` object to 10. If compiled to C++ (for which
there is no \xcd`@Native`), it sets it to 3. 
%~~gen ^^^ extern70
%package Extern.me.plz; 
%~~vis
\begin{xten}
import x10.compiler.Native;
class Born {
  var y : Int = 1n; 
  public def example(x:Int):Int{
    @Native("java", "y=x;") 
    {y = 3n;}
    return y;
  }
}
\end{xten}
%~~siv
%
%~~neg

Note that the code being replaced is a statement -- the block \xcd`{y = 3;}`
in this case -- so the replacement should also be a statement. 


Other \Xten{} constructs may or may not be available in Java and/or C++ code.  For
example, type variables do not correspond exactly to type variables in either
language, and may not be available there.  The exact compilation scheme is
{\em not} fully specified.  You may inspect the generated Java or C++ code and
see how to do specific things, but there is no guarantee that fancy external
coding will continue to work in later versions of \Xten{}.



The full facilities of C++ or Java are available in native code blocks.
However, there is no guarantee that advanced features behave sensibly. You
must follow the exact conventions that the code generator does, or you will
get unpredictable results.  Furthermore, the code generator's conventions may
change without notice or documentation from version to version.  In most cases
the  code should either be a very simple expression, or a method
or function call to external code.


\section{Interoperability with External Java Code}

With Managed X10, we can seamlessly call existing Java code from \Xten{},
and call \Xten{} code from Java.  We call this the 
\emph{Java interoperability}~\cite{TakeuchiX1013} feature.

By combining Java interoperability with X10's distributed
execution features, we can even make existing Java applications, which
are originally designed to run on a single Java VM, scale-out with
minor modifications.

\subsection{How Java program is seen in X10}

Managed X10 does not pre-process the existing Java code to make it
accessible from X10.  X10 programs compiled with Managed X10 will call
existing Java code as is.

\paragraph{Types}

In X10, both at compile time and run time, there is no way to
distinguish Java types from X10 types.  Java types can be referred to
with regular \xcd{import} statement, or their qualified names.  The
package \xcd{java.lang} is not auto-imported into \Xten.  In Managed
x10, the resolver is enhanced to resolve types with X10 source files
in the source path first, then resolve them with Java class files in
the class path. Note that the resolver does not resolve types with
Java source files, therefore Java source files must be compiled in
advance.  To refer to Java types listed in
Tables~\ref{tab:specialtypes}, and \ref{tab:otherspecialtypes}, which
include all Java primitive types, use the corresponding X10 type
(e.g. use \xcd{x10.lang.Int} (or in short, \xcd{Int}) instead of
\xcd{int}).

\paragraph{Fields}

Fields of Java types are seen as fields of X10 types.

Managed X10 does not change the static initialization semantics of
Java types, which is per-class, at load time, and per-place (Java VM),
therefore, it is subtly different than the per-field lazy
initialization semantics of X10 static fields.

\paragraph{Methods}

Methods of Java types are seen as methods of X10 types.

\paragraph{Generic types}

Generic Java types are seen as their raw types 
(\S 4.8 in~\cite{java-lang-spec2005}).  Raw type is a mechanism to handle generic
Java types as non-generic types, where the type parameters are assumed
as \verb|java.lang.Object| or their upperbound if they have it.  Java
introduced generics and raw type at the same time to facilitate
generic Java code interfacing with non-generic legacy Java code.
Managed X10 uses this mechanism for a slightly different purpose.
Java erases type parameters at compile time, whereas X10 preserves
their values at run time.  To manifest this semantic gap in generics,
Managed X10 represents Java generic types as raw types and eliminates
type parameters at source code level.  For more detailed discussions,
please refer to~\cite{TakeuchiX1011,TakeuchiX1012}.

\paragraph{Arrays}

X10 rail and array types are generic types whose representation is different
from Java array types.

Managed X10 provides a special X10 type
\xcd{x10.interop.Java.array[T]} which represents Java array type
\xcd{T[]}.  Note that for X10 types in Table~\ref{tab:specialtypes},
this type means the Java array type of their primary type.  For
example, \xcd{array[Int]} and \xcd{array[String]} mean
\xcd{int[]} and \xcd{java.lang.String[]}, respectively.  Managed X10
also provides conversion methods between X10 \xcd`Rail`s and Java
arrays (\xcd{Java.convert[T](a:Rail[T]):array[T]} and
\xcd{Java.convert[T](a:array[T]):Rail[T]}),
and creation methods for Java arrays 
(\xcd{Java.newArray[T](d0:Int):array[T]}
etc.).

\paragraph{Exceptions}

The \Xten{} 2.3 exception hierarchy has been designed so that there is a
natural correspondence with the Java exception hierarchy. As shown in
Table~\ref{tab:otherspecialtypes}, many commonly used Java
exception types are directly mapped to X10 exception types. 
Exception types that are thus aliased may be caught (and thrown) using
either their Java or \Xten types.  In \Xten code, it is stylistically
preferable to use the \Xten type to refer to the exception as shown in 
Figure~\ref{fig:javaexceptions}.

%----------------
\begin{figure}
\begin{xten}
import x10.interop.Java;
public class XClass {   
  public static def main(args:Rail[String]):void {
    try {
      val a = Java.newArray[Int](2);
      a(0) = 0;
      a(1) = 1;
      a(2) = 2;
    } catch (e:x10.lang.ArrayIndexOutOfBoundsException) {
      Console.OUT.println(e);
    }
  }
}
\end{xten}
%\vspace{-2mm}%@@ADJUST
\begin{verbatim}
> x10c -d bin src/XClass.x10
> x10 -cp bin XClass
x10.lang.ArrayIndexOutOfBoundsException: Array index out of range: 2
\end{verbatim}
\caption{Java exceptions in X10}
%\vspace{-4mm}%@@ADJUST
\label{fig:javaexceptions}
\end{figure}
%----------------

\paragraph{Compiling and executing X10 programs}

We can compile and run X10 programs that call existing Java code with
the same \verb|x10c| and \verb|x10| command by specifying the location
of Java class files or jar files that your X10 programs refer to, with
\verb|-classpath| (or in short, \verb|-cp|) option.

\subsection{How X10 program is translated to Java}

Managed X10 translates X10 programs to Java class files. 

X10 does not provide a Java reflection-like mechanism to resolve X10
types, methods, and fields with their names at runtime, nor a code
generation tool, such as \verb|javah|, to generate stub code to access
them from other languages.  Java programmers, therefore, need to
access X10 types, methods, and fields in the generated Java code
directly as they access Java types, methods, and fields.  To make it
possible, Java programmers need to understand how X10 programs are
translated to Java.

Some aspects of the X10 to Java translation scheme may change in
future version of \Xten{}; therefore in this document only a stable
subset of translation scheme will be explained.  Although it is a
subset, it has been extensively used by X10 core team and proved to be
useful to develop Java Hadoop interop layer for a Main-memory Map
Reduce (M3R) engine~\cite{Shinnar12M3R} in X10.

In the following discussions, we deliberately ignore generic X10
types because the translation of generics is an area of active
development and will undergo some changes in future versions of \Xten{}.
For those who are interested in the implementation of generics
in Managed X10, please consult~\cite{TakeuchiX1012}.  We also don't
cover function types, function values, and all non-static methods.
Although slightly outdated, another paper~\cite{TakeuchiX1011}, which
describes translation scheme in X10 2.1.2, is still useful to
understand the overview of Java code generation in Managed X10.


\paragraph{Types}

X10 classes and structs are translated to Java classes with the same
names.  X10 interfaces are translated to Java interfaces with the same
names.

Table~\ref{tab:specialtypes} shows the list of special types that are
mapped to Java primitives.  Primitives are their primary
representations that are useful for good performance.  Wrapper classes
are used when the reference types are needed.  Each wrapper class has
two static methods \verb|$box()| and \verb|$unbox()| to convert its
value from primary representation to wrapper class, and vice versa,
and Java backend inserts their calls as needed.  An you notice, every
unsigned type uses the same Java primitive as its corresponding signed
type for its representation.

Table~\ref{tab:otherspecialtypes} shows a non-exhaustive list of
another kind of special types that are mapped (not translated) to Java
types.  As you notice, since an interface \verb|Any| is mapped to a
class |java.lang.Object| and \verb|Object| is hidden from the
language, there is no direct way to create an instance of
\verb|Object|. As a workaround, \verb|Java.newObject()| is provided.

As you also notice, \verb|x10.lang.Comparable[T]| is mapped to \verb|java.lang.Comparable|.
This is needed to map \verb|x10.lang.String|, which implements \verb|x10.lang.Compatable[String]|, to \verb|java.lang.String| for performance, but as a trade off, this mapping results in the lost of runtime type information for \verb|Comparable[T]| in Managed X10.
The runtime of Managed X10 has built-in knowledge for \verb|String|, but for other Java classes that implement \verb|java.lang.Comparable|, \verb|instanceof Comparable[Int]| etc. may return incorrect results.
In principle, it is impossible to map X10 generic type to the existing Java generic type without losing runtime type information.

%----------------
\begin{table}
%\scriptsize
\small
\centering
\mbox{
%\hspace{-4mm}%@@ADJUST
\begin{tabular}{|lr|lr|l|}												   \hline
\multicolumn{2}{|c|}{\textbf{X10}}	& \multicolumn{2}{|c|}{\textbf{Java (primary)}}	& \textbf{Java (wrapper class)}	\\ \hline
															   \hline
{\tt x10.lang.Byte}	& {\tt 1y}	& {\tt byte}		& {\tt (byte)1}		& {\tt x10.core.Byte}		\\ \hline
{\tt x10.lang.UByte}	& {\tt 1uy}	& {\tt byte}		& {\tt (byte)1}		& {\tt x10.core.UByte}		\\ \hline
{\tt x10.lang.Short}	& {\tt 1s}	& {\tt short}		& {\tt (short)1}	& {\tt x10.core.Short}		\\ \hline
{\tt x10.lang.UShort}	& {\tt 1us}	& {\tt short}		& {\tt (short)1}	& {\tt x10.core.UShort} 	\\ \hline
{\tt x10.lang.Int}	& {\tt 1}	& {\tt int}		& {\tt 1}		& {\tt x10.core.Int}		\\ \hline
{\tt x10.lang.UInt}	& {\tt 1u}	& {\tt int}		& {\tt 1}		& {\tt x10.core.UInt}		\\ \hline
{\tt x10.lang.Long}	& {\tt 1l}	& {\tt long}		& {\tt 1l}		& {\tt x10.core.Long}	 	\\ \hline
{\tt x10.lang.ULong}	& {\tt 1ul}	& {\tt long}		& {\tt 1l}		& {\tt x10.core.ULong}	 	\\ \hline
{\tt x10.lang.Float}	& {\tt 1.0f}	& {\tt float}		& {\tt 1.0f}		& {\tt x10.core.Float}	 	\\ \hline
{\tt x10.lang.Double}	& {\tt 1.0}	& {\tt double}		& {\tt 1.0}		& {\tt x10.core.Double} 	\\ \hline
{\tt x10.lang.Char}	& {\tt 'c'}	& {\tt char}		& {\tt 'c'}		& {\tt x10.core.Char}		\\ \hline
{\tt x10.lang.Boolean}	& {\tt true}	& {\tt boolean}		& {\tt true}		& {\tt x10.core.Boolean}	\\ \hline
%{\tt x10.lang.String} 	& {\tt "abc"}	& {\tt java.lang.String}& {\tt "abc"}		& {\tt x10.core.String}		\\ \hline
\end{tabular}
}
\caption{X10 types that are mapped to Java primitives}
%\vspace{-4mm}%@@ADJUST
\label{tab:specialtypes}
\end{table}
%----------------


%----------------
\begin{table}
%\scriptsize
\small
\centering
\mbox{
%\hspace{-4mm}%@@ADJUST
\begin{tabular}{|l|l|}										   \hline
\multicolumn{1}{|c|}{\textbf{X10}}		& \multicolumn{1}{|c|}{\textbf{Java}}		\\ \hline
												   \hline
{\tt x10.lang.Any} 				& {\tt java.lang.Object}			\\ \hline
{\tt x10.lang.Comparable[T]} 			& {\tt java.lang.Comparable}			\\ \hline
{\tt x10.lang.String}		 		& {\tt java.lang.String}			\\ \hline
{\tt x10.lang.CheckedThrowable}		 	& {\tt java.lang.Throwable}			\\ \hline
{\tt x10.lang.CheckedException}		 	& {\tt java.lang.Exception}			\\ \hline
{\tt x10.lang.Exception} 			& {\tt java.lang.RuntimeException}		\\ \hline
{\tt x10.lang.ArithmeticException} 		& {\tt java.lang.ArithmeticException}		\\ \hline
{\tt x10.lang.ClassCastException} 		& {\tt java.lang.ClassCastException}		\\ \hline
{\tt x10.lang.IllegalArgumentException} 	& {\tt java.lang.IllegalArgumentException}	\\ \hline
{\tt x10.util.NoSuchElementException}	 	& {\tt java.util.NoSuchElementException}	\\ \hline
{\tt x10.lang.NullPointerException} 		& {\tt java.lang.NullPointerException}		\\ \hline
{\tt x10.lang.NumberFormatException} 		& {\tt java.lang.NumberFormatException}		\\ \hline
{\tt x10.lang.UnsupportedOperationException} 	& {\tt java.lang.UnsupportedOperationException}	\\ \hline
{\tt x10.lang.IndexOutOfBoundsException} 	& {\tt java.lang.IndexOutOfBoundsException}	\\ \hline
{\tt x10.lang.ArrayIndexOutOfBoundsException} 	& {\tt java.lang.ArrayIndexOutOfBoundsException}\\ \hline
{\tt x10.lang.StringIndexOutOfBoundsException} 	& {\tt java.lang.StringIndexOutOfBoundsException}\\ \hline
{\tt x10.lang.Error} 				& {\tt java.lang.Error}				\\ \hline
{\tt x10.lang.AssertionError} 			& {\tt java.lang.AssertionError}		\\ \hline
{\tt x10.lang.OutOfMemoryError} 		& {\tt java.lang.OutOfMemoryError}		\\ \hline
{\tt x10.lang.StackOverflowError} 		& {\tt java.lang.StackOverflowError}		\\ \hline
{\tt void} 					& {\tt void}					\\ \hline
\end{tabular}
}
\caption{X10 types that are mapped (not translated) to Java types}
%\vspace{-4mm}%@@ADJUST
\label{tab:otherspecialtypes}
\end{table}
%----------------


\paragraph{Fields}

As shown in Figure~\ref{fig:fields}, instance fields of X10 classes and structs are translated to the instance fields of the same names of the generated Java classes.
Static fields of X10 classes and structs are translated to the static methods of the generated Java classes, whose name has \verb|get$| prefix.
Static fields of X10 interfaces are translated to the static methods of the special nested class named \verb|$Shadow| of the generated Java interfaces.

%----------------
\begin{figure}
\begin{xten}
class C {
  static val a:Int = ...;
  var b:Int;
}
interface I {
  val x:Int = ...;
}
\end{xten}
%\vspace{-4mm}%@@ADJUST
\begin{xten}
class C {
  static int get$a() { return ...; }
  int b;
}
interface I {
  abstract static class $Shadow {
    static int get$x() { return ...; }
  }
}
\end{xten}
%\vspace{-2mm}%@@ADJUST
\caption{X10 fields in Java}
%\vspace{-4mm}%@@ADJUST
\label{fig:fields}
\end{figure}
%----------------


\paragraph{Methods}

As shown in Figure~\ref{fig:methods}, methods of X10 classes or structs are translated to the methods of the same names of the generated Java classes.
Methods of X10 interfaces are translated to the methods of the same names of the generated Java interfaces.

Every method whose return type has two representations, such as the types in Table~\ref{tab:specialtypes}, will have \verb|$O| suffix with its name.
For example, \verb|def f():Int| in X10 will be compiled as \verb|int f$O()| in Java.
For those who are interested in the reason, please look at our paper~\cite{TakeuchiX1012}.

%----------------
\begin{figure}
\begin{xten}
interface I {
  def f():Int;
  def g():Any;
}
class C implements I {
  static def s():Int = 0;
  static def t():Any = null;
  public def f():Int = 1;
  public def g():Any = null;
}
\end{xten}
%\vspace{-4mm}%@@ADJUST
\begin{xten}
interface I {
  int f$O();
  java.lang.Object g();
}
class C implements I {
  static int s$O() { return 0; }
  static java.lang.Object t() { return null; }
  public int f$O() { return 1; }
  public java.lang.Object g() { return null; }
}
\end{xten}
%\vspace{-2mm}%@@ADJUST
\caption{X10 methods in Java}
%\vspace{-4mm}%@@ADJUST
\label{fig:methods}
\end{figure}
%----------------


\paragraph{Compiling Java programs}

To compile Java program that calls X10 code, you should use
\verb|x10cj| command instead of javac (or whatever your Java
compiler). It invokes the post Java-compiler of \verb|x10c| with the
appropriate options. You need to specify the location of X10-generated
class files or jar files that your Java program refers to.

\verb|x10cj -cp MyX10Lib.jar MyJavaProg.java|


\paragraph{Executing Java programs}

Before executing any X10-generated Java code, the runtime of Managed
X10 needs to be set up properly at each place.  To set up the runtime,
a special launcher named \verb|runjava| is used to run Java programs.
All Java programs that call X10 code should be launched with it, and
no other mechanisms, including direct execution with java command, are
supported.

\begin{verbatim}
Usage: runjava <Java-main-class> [arg0 arg1 ...]
\end{verbatim}


\section{Interoperability with External C and C++ Code}

C and C++ code can be linked to X10 code, either by writing auxiliary C++ files and
adding them with suitable annotations, or by linking libraries.

\subsection{Auxiliary C++ Files}

Auxiliary C++ code can be written in \xcd`.h` and \xcd`.cc` files, which
should be put in the same directory as the the X10 file using them.
Connecting with the library uses the \xcd`@NativeCPPInclude(dot_h_file_name)`
annotation to include the header file, and the 
\xcd`@NativeCPPCompilationUnit(dot_cc_file_name)` annotation to include the
C++ code proper.  For example: 

{\bf MyCppCode.h}: 
\begin{xten}
void foo();
\end{xten}


{\bf MyCppCode.cc}:
\begin{xten}
#include <cstdlib>
#include <cstdio>
void foo() {
    printf("Hello World!\n");
}
\end{xten}

{\bf Test.x10}:
\begin{xten}
import x10.compiler.Native;
import x10.compiler.NativeCPPInclude;
import x10.compiler.NativeCPPCompilationUnit;

@NativeCPPInclude("MyCPPCode.h")
@NativeCPPCompilationUnit("MyCPPCode.cc")
public class Test {
    public static def main (args:Rail[String]) {
        { @Native("c++","foo();") {} }
    }
}
\end{xten}

\subsection{C++ System Libraries}

If we want to additionally link to more libraries in \xcd`/usr/lib` for
example, it is necessary to adjust the post-compilation directly.  The
post-compilation is the compilation of the C++ which the X10-to-C++ compiler
\xcd`x10c++` produces.  

The primary mechanism used for this is the \xcd`-cxx-prearg` and
\xcd`-cxx-postarg` command line arguments to
\xcd`x10c++`. The values of any \xcd`-cxx-prearg` commands are placed
in the post compiler command before the list of .cc files to compile.
The values of any \xcd`-cxx-postarg` commands are placed in the post
compiler command after the list of .cc files to compile. Typically
pre-args are arguments intended for the C++ compiler itself, while
post-args are arguments intended for the linker. 

The following example shows how to compile \xcd`blas` into the
executable via these commands. The command must be issued on one line.

\begin{xten}
x10c++ Test.x10 -cxx-prearg -I/usr/local/blas 
  -cxx-postarg -L/usr/local/blas -cxx-postarg -lblas'
\end{xten}


\chapter{Definite Assignment}
\label{sect:DefiniteAssignment}
\index{definite assignment}
\index{assignment!definite}
\index{definitely assigned}
\index{definitely not assigned}

X10 requires that every variable be set before it is read.
Sometimes this is easy, as when a variable is declared and assigned together: 
%~~gen ^^^ DefiniteAssignment4x1u
% package DefiniteAssignment4x1u;
% class Example {
% def example() {
%~~vis
\begin{xten}
  var x : Long = 0;
  assert x == 0;
\end{xten}
%~~siv
%}}
%~~neg
However, it is convenient to allow programs to make decisions before
initializing variables.
%~~gen ^^^ DefiniteAssignment4u7z
% package DefiniteAssignment4u7z;
% class Example {
%~~vis
\begin{xten}
def example(a:Long, b:Long) {
  val max:Long;
  //ERROR: assert max==max; // can't read 'max'
  if (a > b) max = a;
  else max = b;
  assert max >= a && max >= b;
}
\end{xten}
%~~siv
%}
%~~neg
This is particularly useful for \xcd`val` variables.  \xcd`var`s could be
initialized to a default value and then reassigned with the right value.
\xcd`val`s must be initialized once and cannot be changed, so they must be
initialized with the correct value. 

However, one must be careful -- and the X10 compiler enforces this care.
Without the \xcd`else` clause, the preceding code might not give \xcd`max` a
value by the time \xcd`assert` is invoked.  

This leads to the concept of {\em definite assignment} \cite{jls2}.
A variable is {\em definitely assigned} at a point in code if, no matter how that
point in code is reached, the variable has been assigned to.  In X10,
variables must be definitely assigned before they can be read.


As X10 requires that \xcd`val` variables {\em not} be initialized
twice,  we need the dual concept as well.  A variable is {\em definitely
unassigned} at a point in code if it cannot have been assigned no
matter how that point in code is reached.  For example, immediately
after \xcd`val x:Long`, \xcd`x` is definitely unassigned.  

Finally, we need the concept of {\em singly} and {\em multiply assigned}.
A variable is singly assigned in a block if it is assigned precisely
once; it is multiply assigned if it could possibly be assigned more than once.  
\xcd`var`s can  multiply assigned as desired. \xcd`val`s must be singly
assigned.  For example, the code \xcd`x = 1; x = 2;` is perfectly fine if
\xcd`x` is a \xcd`var`, but incorrect (even in a constructor) if \xcd`x` is a
\xcd`val`.  

At some points in code, a variable might be neither definitely assigned nor
definitely unassigned.    Such states are not always useful.  
%~~gen ^^^ DefiniteAssignment4f5z
% package DefiniteAssignment4f5z;
% class Example {
% 
%~~vis
\begin{xten}
def example(flag : Boolean) {
  var x : Long;
  if (flag) x = 1;
  // x is neither def. assigned nor unassigned.
  x = 2; 
  // x is def. assigned.
\end{xten}
%~~siv
% } } 
%~~neg
This shows that we cannot simply define ``definitely unassigned'' as ``not
definitely assigned''.   If \xcd`x` had been a \xcd`val` rather than a
\xcd`var`, the previous example would not be allowed.    

Unfortunately, a completely accurate definition of ``definitely assigned''
or ``definitely unassigned'' is undecidable -- impossible for the compiler to
determine.  So, X10 takes a {\em conservative approximation} of these
concepts.  If X10's definition says that \xcd`x` is definitely assigned (or
definitely unassigned), then it will be assigned (or not assigned) in every
execution of the program.  

However, there are programs which X10's algorithm says are incorrect, but
which actually would behave properly if they were executed.   In the following
example, \xcd`flag` is either \xcd`true` or \xcd`false`, and in either case
\xcd`x` will be initialized.  However, X10's analysis does not understand this
--- thought it {\em would} understand if the example were coded with an
\xcd`if-else` rather than a pair of \xcd`if`s.  So, after the two \xcd`if`
statements, \xcd`x` is not definitely assigned, and thus the \xcd`assert`
statement, which reads it, is forbidden.  
%~~gen ^^^ DefiniteAssignment3x6i
% package DefiniteAssignment3x6i;
% class Example{ 
%~~vis
\begin{xten}
def example(flag:Boolean) {
  var x : Long;
  if (flag) x = 1;
  if (!flag) x = 2;
  // ERROR: assert x < 3;
}
\end{xten}
%~~siv
%}
%~~neg

\section{Asynchronous Definite Assignment}


Local variables and instance fields allow {\em asynchronous assignment}. A local
variable can be assigned in an \xcd`async` statement, and, when the
\xcd`async` is \xcd`finish`ed, the variable is definitely assigned.  

\begin{ex}
%~~gen ^^^ DefiniteAssignment4a6n
% package DefiniteAssignment4a6n;
% class Example {
% def example() {
%~~vis
\begin{xten}
val a : Long;
finish {
  async {
    a = 1;
  } 
  // a is not definitely assigned here
}
// a is definitely assigned after 'finish'
assert a==1; 
\end{xten}
%~~siv
%} } 
%~~neg
\end{ex}

This concept supports a core X10 programming idiom.  A \xcd`val` variable may
be initialized asynchronously, thereby providing a means for returning a value
from an \xcd`async` to be used after the enclosing \xcd`finish`.  

\section{Characteristics of Definite Assignment}

The properties ``definitely assigned'', ``singly assigned'', and
``definitely unassigned'' are computed by a conservative approximation of
X10's evaluation rules.

The precise details are up to the implementation. 
Many basic cases must be handled accurately; \eg, \xcd`x=1;` definitely and
singly assigns \xcd`x`.  

However, in more complicated cases, a conforming X10 may mark as invalid 
some code which, when executed, would actually be correct.  
For example, the following
program fragment will always result in \xcd`x` being definitely and singly
assigned:  
\begin{xten}
val x : Long;
var b : Boolean = mysterious();
if (b) x = cryptic();
if (!b) x = unknown();
\end{xten}
However, most conservative approximations of program execution won't mark
\xcd`x` as properly initialized, though it is.   For \xcd`x` to be properly
initialized, precisely one of the 
two assignments to \xcd`x` must be executed. If \xcd`b` were true initially,
it would still be true after the call to \xcd`cryptic()` --- since methods
cannot modify their caller's local variables -- and so the first but not the
second assignment would happen. If \xcd`b` were false initially, it would
still be false when \xcd`!b` is tested, and so the second but not the first
assignment would happen.  Either way, \xcd`x` is definitely and singly assigned.

However, for a slightly different program, this analysis would be wrong. \Eg,
if  \xcd`b` were a field of \xcd`this` rather than a local variable,
\xcd`cryptic()` could change \xcd`b`; if \xcd`b` were true initially, both
assignments might happen, which is incorrect for a \xcd`val`.  

This sort of reasoning is beyond  most conservative approximation algorithms.
(Indeed, many do not bother checking that \xcd`!b` late in the program is the
opposite of \xcd`b` earlier.)
Algorithms that pay attention to such details and subtleties tend to be
fairly expensive, which would lead to very slow compilation for X10 -- for the
sake of obscure cases.

X10's analysis provides at least the following guarantees. We describe them in
terms of a statement \xcd`S` performing some collection of possible numbers of
assignments to variables --- on a scale of ``0'', ``1'', and ``many''. For
example, \xcd`if (b) x=1; else {x=1;x=2;y=2;}` might assign to \xcd`x` one or
many times, and might assign to \xcd`y` zero or one time. Hence, after it,
\xcd`x` is definitely assigned and may be multiply assigned, and \xcd`y` is
neither definitely assigned nor definitely unassigned.  

These descriptions are combined in natural ways.  For example, if \xcd`R` says
that \xcd`x` will be assigned 0 or 1 times, and \xcd`S` says it will be
assigned precisely once, then \xcd`R;S` will assign it one or many times.  If
only one or \xcd`R` or \xcd`S` will occur, as from \xcd`if (b) R; else S;`, 
then \xcd`x` may be assigned 0 or 1 times. 

This information is sufficient for the tests X10 makes.  If \xcd`x` can is
assigned one or many times in \xcd`S`, it is definitely assigned.  It is an
error if 
\xcd`x` is ever read at a point where it have been assigned zero times.  It is
an error if a \xcd`val` may be assigned many times.

We do not guarantee that any particular X10 compiler uses this algorithm;
indeed, as of the time of writing, the X10 compiler uses a somewhat more
precise one.  However, any conformant X10 compiler must provide results which
are at least as accurate as this analysis.

\subsubsection{Assignment: {\tt x = e}}   

\xcd`x = e` assigns to \xcd`x`, in addition to whatever assignments
\xcd`e` makes.   For example, if \xcd`this.setX(y)` sets a field \xcd`x` to
\xcd`y` and returns \xcd`y`, then \xcd`x = this.setX(y)` definitely and
multiply assigns \xcd`x`.  

\subsubsection{{\tt async} and {\tt finish}}

By itself, \xcd`async S` provides few guarantees.  After an activity
executes \xcd`async{x=1;}` we know that there is a separate activity
which (on being scheduled) will set \xcd`x` to \xcd`1`.  We do not
know that this has happened yet.

However, if there is a \xcd`finish` around the \xcd`async`, the situation is
clearer.  After \xcd`finish async x=1;`, \xcd`x` has definitely been
assigned.  

In general, if an \xcd`async S` appears in the body of a \xcd`finish` in a way
that guarantees that it will be executed, then, after the \xcd`finish`, the
assignments made by \xcd`S` will have occurred.  For example, if \xcd`S`
definitely assigns to \xcd`x`, and the body of the \xcd`finish` guarantees
that \xcd`async S` will be executed, then \xcd`finish{...async S...}`
definitely assigns \xcd`x`.

\subsubsection{{\tt if} and {\tt switch}}

When \xcd`if(E) S else T` finishes, it will have performed the assignments of
\xcd`E`, together with those of either \xcd`S` or \xcd`T` but not both.  For
example, \xcd`if (b) x=1; else x=2;` definitely assigns \xcd`x`,
but \xcd`if (b) x=1;` does not.

{\tt switch} is more complex, but follows the same principles as \xcd`if`.
For example, \xcd`switch(E){case 1: A; break; case 2: B; default: C;}`  
performs the assignments of \xcd`E`, and those of precisely one of \xcd`A`, or
\xcd`B;C`, or \xcd`C`.  Note that case \xcd`2` falls through to the default
case, so it performs the same assignments as \xcd`B;C`.

\subsubsection{Sequencing}

When \xcd`R;S` finishes, it will have performed the assignments of \xcd`R` and
those of \xcd`S`, if \xcd`R` and \xcd`S` terminate normally. If
\xcd`R` terminates abruptly, then only the assignments of \xcd`R`
executed till the point of termination will have been executed. if
\xcd`R` terminates normally, but \xcd`S` terminates abruptly then the
assignments of \xcd`R` will have been executed and those of \xcd`S`
executed till the point of termination. 

For example, \xcd`x=1;y=2;` definitely assigns \xcd`x` and 
\xcd`y`, and \xcd`x=1;x=2;` multiply assigns \xcd`x`. 


\subsubsection{Loops}

\xcd`while(E)S` performs the assignments of \xcd`E` one or more times, and
those of \xcd`S` zero or more times.  For example, if \xcd`while(b()){x=1;}`
might assign to \xcd`x` zero, one, or many times.  
\xcd`do S while(E)` performs the assignments of \xcd`E` one or more times, and
those of \xcd`S` one or more times. 

\xcd`for(A;B;C)D` performs the assignments of \xcd`A` once, those of \xcd`B`
one or more times, and those of \xcd`C` and \xcd`D` one or more times.
\xcd`for(x in E)S` performs the assignments of \xcd`E` once and those of
\xcd`S` zero or more times.  

Loops are of very little value for providing definite assignments, since X10
does not in general know how many times they will be executed. 

\xcd`continue` and \xcd`break` inside of a loop are hard to describe in simple
terms.  They may be conservatively assumed to cause the loop to give no
information about the variables assigned inside of it.
For example, the analysis may conservatively conclude that 
\xcd`do{ x = 1; if (true) break; } while(true)` may 
assign to \xcd`x` zero, one, or many times, overlooking the more precise fact
that it is assigned once.  

\subsubsection{Method Calls}

A method call \xcd`E.m(A,B)` performs the assignments of \xcd`E`, \xcd`A`, and
\xcd`B` once each, and also those of \xcd`m`.  This implies that X10 must be
aware of the possible assignments performed by each method.

If X10 has complete information about \xcd`m` (as when \xcd`m` is a
\xcd`private` or \xcd`final` method), this is straightforward.  When such
information is fundamentally impossible to acquire, as when \xcd`m` is a
non-final method invocation, X10 has no choice but to assume that \xcd`m`
might do anything that a method can do.    (For this reason, the only methods
that can be called from within a constructor on a raw --
incompletely-constructed -- object) are the \xcd`private` and
\xcd`final` ones.)  
\begin{itemize}
\item \xcd`m` cannot assign to local variables of the caller; methods have no
      such power.
\item Let \xcd`m` be an instance method. \xcd`m` can assign to \xcd`var` fields of \xcd`this` freely,
\item Let \xcd`m` be an instance method. \xcd`m` cannot initialize \xcd`val` fields of \xcd`this`.  (But see
      \Sref{sect:call-another-ctor}; when one constructor calls another as the
      first statement of its body, the other constructor can initialize
      v\xcd`val` fields. This is a constructor call, not a method call.) 
\end{itemize}

Recall that every container must be fully initialized upon exit
from its constructor.  
X10 places certain restrictions on which methods can be called from a
constructor; see \Sref{sect:nonescaping}.  One of these restrictions is that
methods called before object initialization is complete must be \xcd`final` or
\xcd`private` --- and hence, available for static analysis.  So, when checking
field initialization, X10 will ensure: 
\begin{enumerate}
\item Each \xcd`val` field is initialized before it is read.   
      A method that does not read a \xcd`val` field \xcd`f` {\em may} be
      called before \xcd`f` is initialized; a method that reads \xcd`f` must
      not be called until \xcd`f` is initialized.        
      For example, 
      a constructor may have the form:
%~~gen ^^^ DefiniteAssignment4x6k
% package DefiniteAssignment4x6k;
%~~vis
\begin{xten}
class C {
  val f : Long;
  val g : String;
  def this() {
     f = fless();
     g = useF();
  }
  private def fless() = "f not used here".length();
  private def useF() = "f=" + this.f;
}
\end{xten}
%~~siv
%
%~~neg

\item \xcd`var` fields require a deeper analysis.  Consider a \xcd`var`
      field \xcd`var x:T`  without initializer.  If \xcd`T` has a default
      value, \xcd`x` may be read inside of a constructor before it is
      otherwise written, and it will 
      have its default value.   

      If \xcd`T` has no default value, an analysis
      like that used for \xcd`val`s must be performed to determine that
      \xcd`x` is initialized before it is used.  The situation is 
      more complex than for \xcd`val`s, however, because a method can assign to
      \xcd`x` as well read from it.  The X10 compiler computes a conservative
      approximation of which methods
      read and write which \xcd`var` fields. (Doing this carefully 
      requires finding a solution of a set of equations over sets of
      variables, with each callable method having equations describing what it
      reads and writes.)    

\end{enumerate}


\subsubsection{{\tt at} 
%and \xcd`athome`
}

%%AT-COPY%% \xcd`at(E)S` performs the assignments of \xcd`E`. Within \xcd`S`, only those
%%AT-COPY%% assignments to variables \xcd`x` from the surrounding environment which take
%%AT-COPY%% place within a suitable \xcd`athome(x)R` are counted. 
%%AT-COPY%% 
%%AT-COPY%% \begin{ex}
%%AT-COPY%% In the following code, the outer variable named \xcd`a` is definitely assigned
%%AT-COPY%% once, by the assignment \xcd`a = 3;`.  The inner variable (also named \xcd`a`)
%%AT-COPY%% is definitely multiply assigned 
%%AT-COPY%% by the two assignments \xcd`a = 1;` and \xcd`a = 2;` 
%%AT-COPY%% between the \xcd`at` and the \xcd`athome`.  
%%AT-COPY%% 
%%AT-COPY%% %~~gen ^^^ DefiniteAssignment3n5q
%%AT-COPY%% % package DefiniteAssignment3n5q;
%%AT-COPY%% % KNOWNFAIL-at
%%AT-COPY%% % class DefAss { def defass() { 
%%AT-COPY%% %~~vis
%%AT-COPY%% \begin{xten}
%%AT-COPY%% var a : Long;
%%AT-COPY%% at(here.next(); var a : Long = a) {
%%AT-COPY%%   a = 1;
%%AT-COPY%%   a = 2; 
%%AT-COPY%%   athome(a) a = 3;
%%AT-COPY%% }
%%AT-COPY%% \end{xten}
%%AT-COPY%% %~~siv
%%AT-COPY%% % } } 
%%AT-COPY%% %~~neg
%%AT-COPY%% 
%%AT-COPY%% 
%%AT-COPY%% \end{ex}
%%AT-COPY%% 

% vj Wed Sep 18 04:19:05 EDT 2013
% Hmm. This used to be incorrect.
\xcd`at(p)S` performs precisely the assignments of \xcd`p` and those
of \xcd`S`. Note that \xcd`S` is executed at the place named by
\xcd`p` in an environment in which all variables used in \xcd`S` but
defined outside \xcd`S` are bound to copies (made at \xcd`p`) of the
values they had at the \xcd`at(p)S` statement (\Sref{AtStatement}).

% vj Wed Sep 18 04:19:05 EDT 2013
% Hmm. Commented this out. This is not true :-(
%\xcd`this` cannot be read or written by an \xcd`at`-statement.

\subsubsection{{\tt atomic}}

\xcd`atomic S` performs the assignments of \xcd`S`, 
and \xcd`when(E)S` performs those of \xcd`E` and \xcd`S`.  Note that
\xcd`E` or \xcd`S` may terminate abruptly.

\subsubsection{{\tt try}}

\xcd`try S catch(x:T1) E1 ... catch(x:Tn) En finally F` 
performs some or all of the assignments of \xcd`S`, plus all the assignments
of zero or one of the \xcd`E`'s, plus those of \xcd`F`.  
For example,
\begin{xten}
try {
  x = boomy();
  x = 0;
}
catch(e:Boom) { y = 1; }
finally { z = 1; }
\end{xten}
\noindent 
assigns \xcd`x` zero, one, or many times\footnote{A more precise
analysis could discover that \xcd`x` cannot be initialized only once.}, 
assigns \xcd`y` zero or one time, and assigns \xcd`z` exactly once.

\subsubsection{Expression Statements}

Expression statements \xcd`E;`, and other statements that execute an
expression and do something innocuous with it (local variable declaration and
\xcd`assert`) have the same effects as \xcd`E`. 

\subsubsection{{\tt return}, {\tt throw}}

Statements that do not finish normally, such as \xcd`return` and \xcd`throw`,
do not initialize anything (though the computation of the return or thrown
value may).    They also terminate a line of computation.  For example, 
\xcd`if(b) {x=1; return;}  x=2;` definitely and singly assigns \xcd`x`.  

%% vj Thu Sep 19 06:00:59 EDT 2013
%% No changes made for v2.4. 

\chapter{Grammar}\label{Grammar}


In this grammar, $X^?$ denotes an optional $X$ element.


\begin{bbgrammarappendix}{3.9in}

(\arabic{equation}) & AdditiveExp \refstepcounter{equation}\label{prod:AdditiveExp}  \: MultiplicativeExp  \\

 &    \| AdditiveExp \xcd"+" MultiplicativeExp \\
 &    \| AdditiveExp \xcd"-" MultiplicativeExp \\

\end{bbgrammarappendix}

\begin{bbgrammarappendix}{4.4in}

(\arabic{equation}) & AndExp \refstepcounter{equation}\label{prod:AndExp}  \: EqualityExp  \\

 &    \| AndExp \xcd"&" EqualityExp \\

\end{bbgrammarappendix}

\begin{bbgrammarappendix}{3.7in}

(\arabic{equation}) & AnnotatedType \refstepcounter{equation}\label{prod:AnnotatedType}  \: Type Annotations  \\


\end{bbgrammarappendix}

\begin{bbgrammarappendix}{4.0in}

(\arabic{equation}) & Annotation \refstepcounter{equation}\label{prod:Annotation}  \: \xcd"@" NamedTypeNoConstraints  \\


\end{bbgrammarappendix}

\begin{bbgrammarappendix}{3.6in}

(\arabic{equation}) & AnnotationStmt \refstepcounter{equation}\label{prod:AnnotationStmt}  \: Annotations\opt NonExpStmt  \\


\end{bbgrammarappendix}

\begin{bbgrammarappendix}{3.9in}

(\arabic{equation}) & Annotations \refstepcounter{equation}\label{prod:Annotations}  \: Annotation  \\

 &    \| Annotations Annotation \\

\end{bbgrammarappendix}

\begin{bbgrammarappendix}{3.8in}

(\arabic{equation}) & ApplyOpDecln \refstepcounter{equation}\label{prod:ApplyOpDecln}  \: MethMods \xcd"operator" \xcd"this" TypeParams\opt Formals Guard\opt HasResultType\opt MethodBody  \\


\end{bbgrammarappendix}

\begin{bbgrammarappendix}{3.8in}

(\arabic{equation}) & ArgumentList \refstepcounter{equation}\label{prod:ArgumentList}  \: Exp  \\

 &    \| ArgumentList \xcd"," Exp \\

\end{bbgrammarappendix}

\begin{bbgrammarappendix}{4.1in}

(\arabic{equation}) & Arguments \refstepcounter{equation}\label{prod:Arguments}  \: \xcd"(" ArgumentList \xcd")"  \\


\end{bbgrammarappendix}

\begin{bbgrammarappendix}{4.0in}

(\arabic{equation}) & AssertStmt \refstepcounter{equation}\label{prod:AssertStmt}  \: \xcd"assert" Exp \xcd";"  \\

 &    \| \xcd"assert" Exp  \xcd":" Exp  \xcd";" \\

\end{bbgrammarappendix}

\begin{bbgrammarappendix}{3.6in}

(\arabic{equation}) & AssignPropCall \refstepcounter{equation}\label{prod:AssignPropCall}  \: \xcd"property" TypeArgs\opt \xcd"(" ArgumentList\opt \xcd")" \xcd";"  \\


\end{bbgrammarappendix}

\begin{bbgrammarappendix}{4.0in}

(\arabic{equation}) & Assignment \refstepcounter{equation}\label{prod:Assignment}  \: LeftHandSide AsstOp AsstExp  \\

 &    \| ExpName  \xcd"(" ArgumentList\opt \xcd")" AsstOp AsstExp \\
 &    \| Primary  \xcd"(" ArgumentList\opt \xcd")" AsstOp AsstExp \\

\end{bbgrammarappendix}

\begin{bbgrammarappendix}{4.3in}

(\arabic{equation}) & AsstExp \refstepcounter{equation}\label{prod:AsstExp}  \: Assignment  \\

 &    \| ConditionalExp \\

\end{bbgrammarappendix}

\begin{bbgrammarappendix}{4.4in}

(\arabic{equation}) & AsstOp \refstepcounter{equation}\label{prod:AsstOp}  \: \xcd"="  \\

 &    \| \xcd"*=" \\
 &    \| \xcd"/=" \\
 &    \| \xcd"%=" \\
 &    \| \xcd"+=" \\
 &    \| \xcd"-=" \\
 &    \| \xcd"<<=" \\
 &    \| \xcd">>=" \\
 &    \| \xcd">>>=" \\
 &    \| \xcd"&=" \\
 &    \| \xcd"^=" \\
 &    \| \xcd"|=" \\

\end{bbgrammarappendix}

\begin{bbgrammarappendix}{4.1in}

(\arabic{equation}) & AsyncStmt \refstepcounter{equation}\label{prod:AsyncStmt}  \: \xcd"async" ClockedClause\opt Stmt  \\

 &    \| \xcd"clocked" \xcd"async" Stmt \\

\end{bbgrammarappendix}

\begin{bbgrammarappendix}{3.6in}

(\arabic{equation}) & AtCaptureDeclr \refstepcounter{equation}\label{prod:AtCaptureDeclr}  \: Mods\opt VarKeyword\opt VariableDeclr  \\

 &    \| Id \\
 &    \| \xcd"this" \\

\end{bbgrammarappendix}

\begin{bbgrammarappendix}{3.5in}

(\arabic{equation}) & AtCaptureDeclrs \refstepcounter{equation}\label{prod:AtCaptureDeclrs}  \: AtCaptureDeclr  \\

 &    \| AtCaptureDeclrs \xcd"," AtCaptureDeclr \\

\end{bbgrammarappendix}

\begin{bbgrammarappendix}{4.0in}

(\arabic{equation}) & AtEachStmt \refstepcounter{equation}\label{prod:AtEachStmt}  \: \xcd"ateach" \xcd"(" LoopIndex \xcd"in" Exp \xcd")" ClockedClause\opt Stmt  \\

 &    \| \xcd"ateach" \xcd"(" Exp \xcd")" Stmt \\

\end{bbgrammarappendix}

\begin{bbgrammarappendix}{4.5in}

(\arabic{equation}) & AtExp \refstepcounter{equation}\label{prod:AtExp}  \: \xcd"at" \xcd"(" Exp \xcd")" ClosureBody  \\


\end{bbgrammarappendix}

\begin{bbgrammarappendix}{4.4in}

(\arabic{equation}) & AtStmt \refstepcounter{equation}\label{prod:AtStmt}  \: \xcd"at" \xcd"(" Exp \xcd")" Stmt  \\


\end{bbgrammarappendix}

\begin{bbgrammarappendix}{4.0in}

(\arabic{equation}) & AtomicStmt \refstepcounter{equation}\label{prod:AtomicStmt}  \: \xcd"atomic" Stmt  \\


\end{bbgrammarappendix}

\begin{bbgrammarappendix}{3.8in}

(\arabic{equation}) & BasicForStmt \refstepcounter{equation}\label{prod:BasicForStmt}  \: \xcd"for" \xcd"(" ForInit\opt \xcd";" Exp\opt \xcd";" ForUpdate\opt \xcd")" Stmt  \\


\end{bbgrammarappendix}

\begin{bbgrammarappendix}{4.5in}

(\arabic{equation}) & BinOp \refstepcounter{equation}\label{prod:BinOp}  \: \xcd"+"  \\

 &    \| \xcd"-" \\
 &    \| \xcd"*" \\
 &    \| \xcd"/" \\
 &    \| \xcd"%" \\
 &    \| \xcd"&" \\
 &    \| \xcd"|" \\
 &    \| \xcd"^" \\
 &    \| \xcd"&&" \\
 &    \| \xcd"||" \\
 &    \| \xcd"<<" \\
 &    \| \xcd">>" \\
 &    \| \xcd">>>" \\
 &    \| \xcd">=" \\
 &    \| \xcd"<=" \\
 &    \| \xcd">" \\
 &    \| \xcd"<" \\
 &    \| \xcd"==" \\
 &    \| \xcd"!=" \\
 &    \| \xcd".." \\
 &    \| \xcd"->" \\
 &    \| \xcd"<-" \\
 &    \| \xcd"-<" \\
 &    \| \xcd">-" \\
 &    \| \xcd"**" \\
 &    \| \xcd"~" \\
 &    \| \xcd"!~" \\
 &    \| \xcd"!" \\

\end{bbgrammarappendix}

\begin{bbgrammarappendix}{4.0in}

(\arabic{equation}) & BinOpDecln \refstepcounter{equation}\label{prod:BinOpDecln}  \: MethMods \xcd"operator" TypeParams\opt \xcd"(" Formal  \xcd")" BinOp \xcd"(" Formal  \xcd")" Guard\opt HasResultType\opt MethodBody  \\

 &    \| MethMods \xcd"operator" TypeParams\opt \xcd"this" BinOp \xcd"(" Formal  \xcd")" Guard\opt HasResultType\opt MethodBody \\
 &    \| MethMods \xcd"operator" TypeParams\opt \xcd"(" Formal  \xcd")" BinOp \xcd"this" Guard\opt HasResultType\opt MethodBody \\

\end{bbgrammarappendix}

\begin{bbgrammarappendix}{4.5in}

(\arabic{equation}) & Block \refstepcounter{equation}\label{prod:Block}  \: \xcd"{" BlockStmts\opt \xcd"}"  \\


\end{bbgrammarappendix}

\begin{bbgrammarappendix}{3.3in}

(\arabic{equation}) & BlockInteriorStmt \refstepcounter{equation}\label{prod:BlockInteriorStmt}  \: LocVarDeclnStmt  \\

 &    \| ClassDecln \\
 &    \| StructDecln \\
 &    \| TypeDefDecln \\
 &    \| Stmt \\

\end{bbgrammarappendix}

\begin{bbgrammarappendix}{4.0in}

(\arabic{equation}) & BlockStmts \refstepcounter{equation}\label{prod:BlockStmts}  \: BlockInteriorStmt  \\

 &    \| BlockStmts BlockInteriorStmt \\

\end{bbgrammarappendix}

\begin{bbgrammarappendix}{3.6in}

(\arabic{equation}) & BooleanLiteral \refstepcounter{equation}\label{prod:BooleanLiteral}  \: \xcd"true"   \\

 &    \| \xcd"false"  \\

\end{bbgrammarappendix}

\begin{bbgrammarappendix}{4.1in}

(\arabic{equation}) & BreakStmt \refstepcounter{equation}\label{prod:BreakStmt}  \: \xcd"break" Id\opt \xcd";"  \\


\end{bbgrammarappendix}

\begin{bbgrammarappendix}{4.3in}

(\arabic{equation}) & CastExp \refstepcounter{equation}\label{prod:CastExp}  \: Primary  \\

 &    \| ExpName \\
 &    \| CastExp \xcd"as" Type \\

\end{bbgrammarappendix}

\begin{bbgrammarappendix}{3.9in}

(\arabic{equation}) & CatchClause \refstepcounter{equation}\label{prod:CatchClause}  \: \xcd"catch" \xcd"(" Formal \xcd")" Block  \\


\end{bbgrammarappendix}

\begin{bbgrammarappendix}{4.3in}

(\arabic{equation}) & Catches \refstepcounter{equation}\label{prod:Catches}  \: CatchClause  \\

 &    \| Catches CatchClause \\

\end{bbgrammarappendix}

\begin{bbgrammarappendix}{4.1in}

(\arabic{equation}) & ClassBody \refstepcounter{equation}\label{prod:ClassBody}  \: \xcd"{" ClassMemberDeclns\opt \xcd"}"  \\


\end{bbgrammarappendix}

\begin{bbgrammarappendix}{4.0in}

(\arabic{equation}) & ClassDecln \refstepcounter{equation}\label{prod:ClassDecln}  \: Mods\opt \xcd"class" Id TypeParamsI\opt Properties\opt Guard\opt Super\opt Interfaces\opt ClassBody  \\


\end{bbgrammarappendix}

\begin{bbgrammarappendix}{3.4in}

(\arabic{equation}) & ClassMemberDecln \refstepcounter{equation}\label{prod:ClassMemberDecln}  \: InterfaceMemberDecln  \\

 &    \| CtorDecln \\

\end{bbgrammarappendix}

\begin{bbgrammarappendix}{3.3in}

(\arabic{equation}) & ClassMemberDeclns \refstepcounter{equation}\label{prod:ClassMemberDeclns}  \: ClassMemberDecln  \\

 &    \| ClassMemberDeclns ClassMemberDecln \\

\end{bbgrammarappendix}

\begin{bbgrammarappendix}{4.1in}

(\arabic{equation}) & ClassName \refstepcounter{equation}\label{prod:ClassName}  \: TypeName  \\


\end{bbgrammarappendix}

\begin{bbgrammarappendix}{4.1in}

(\arabic{equation}) & ClassType \refstepcounter{equation}\label{prod:ClassType}  \: NamedType  \\


\end{bbgrammarappendix}

\begin{bbgrammarappendix}{3.7in}

(\arabic{equation}) & ClockedClause \refstepcounter{equation}\label{prod:ClockedClause}  \: \xcd"clocked" Arguments  \\


\end{bbgrammarappendix}

\begin{bbgrammarappendix}{3.9in}

(\arabic{equation}) & ClosureBody \refstepcounter{equation}\label{prod:ClosureBody}  \: Exp  \\

 &    \| Annotations\opt \xcd"{" BlockStmts\opt LastExp \xcd"}" \\
 &    \| Annotations\opt Block \\

\end{bbgrammarappendix}

\begin{bbgrammarappendix}{4.0in}

(\arabic{equation}) & ClosureExp \refstepcounter{equation}\label{prod:ClosureExp}  \: Formals Guard\opt HasResultType\opt \xcd"=>" ClosureBody  \\


\end{bbgrammarappendix}

\begin{bbgrammarappendix}{3.5in}

(\arabic{equation}) & CompilationUnit \refstepcounter{equation}\label{prod:CompilationUnit}  \: PackageDecln\opt TypeDeclns\opt  \\

 &    \| PackageDecln\opt ImportDeclns TypeDeclns\opt \\
 &    \| ImportDeclns PackageDecln  ImportDeclns\opt  TypeDeclns\opt \\
 &    \| PackageDecln ImportDeclns PackageDecln  ImportDeclns\opt  TypeDeclns\opt \\

\end{bbgrammarappendix}

\begin{bbgrammarappendix}{3.3in}

(\arabic{equation}) & ConditionalAndExp \refstepcounter{equation}\label{prod:ConditionalAndExp}  \: InclusiveOrExp  \\

 &    \| ConditionalAndExp \xcd"&&" InclusiveOrExp \\

\end{bbgrammarappendix}

\begin{bbgrammarappendix}{3.6in}

(\arabic{equation}) & ConditionalExp \refstepcounter{equation}\label{prod:ConditionalExp}  \: ConditionalOrExp  \\

 &    \| ClosureExp \\
 &    \| AtExp \\
 &    \| ConditionalOrExp \xcd"?" Exp \xcd":" ConditionalExp \\

\end{bbgrammarappendix}

\begin{bbgrammarappendix}{3.4in}

(\arabic{equation}) & ConditionalOrExp \refstepcounter{equation}\label{prod:ConditionalOrExp}  \: ConditionalAndExp  \\

 &    \| ConditionalOrExp \xcd"||" ConditionalAndExp \\

\end{bbgrammarappendix}

\begin{bbgrammarappendix}{3.9in}

(\arabic{equation}) & ConstantExp \refstepcounter{equation}\label{prod:ConstantExp}  \: Exp  \\


\end{bbgrammarappendix}

\begin{bbgrammarappendix}{3.5in}

(\arabic{equation}) & ConstrainedType \refstepcounter{equation}\label{prod:ConstrainedType}  \: NamedType  \\

 &    \| AnnotatedType \\

\end{bbgrammarappendix}

\begin{bbgrammarappendix}{2.9in}

(\arabic{equation}) & ConstraintConjunction \refstepcounter{equation}\label{prod:ConstraintConjunction}  \: Exp  \\

 &    \| ConstraintConjunction \xcd"," Exp \\

\end{bbgrammarappendix}

\begin{bbgrammarappendix}{3.8in}

(\arabic{equation}) & ContinueStmt \refstepcounter{equation}\label{prod:ContinueStmt}  \: \xcd"continue" Id\opt \xcd";"  \\


\end{bbgrammarappendix}

\begin{bbgrammarappendix}{3.3in}

(\arabic{equation}) & ConversionOpDecln \refstepcounter{equation}\label{prod:ConversionOpDecln}  \: ExplConvOpDecln  \\

 &    \| ImplConvOpDecln \\

\end{bbgrammarappendix}

\begin{bbgrammarappendix}{4.1in}

(\arabic{equation}) & CtorBlock \refstepcounter{equation}\label{prod:CtorBlock}  \: \xcd"{" ExplicitCtorInvo\opt BlockStmts\opt \xcd"}"  \\


\end{bbgrammarappendix}

\begin{bbgrammarappendix}{4.2in}

(\arabic{equation}) & CtorBody \refstepcounter{equation}\label{prod:CtorBody}  \: \xcd"=" CtorBlock  \\

 &    \| CtorBlock \\
 &    \| \xcd"=" ExplicitCtorInvo \\
 &    \| \xcd"=" AssignPropCall \\
 &    \| \xcd";" \\

\end{bbgrammarappendix}

\begin{bbgrammarappendix}{4.1in}

(\arabic{equation}) & CtorDecln \refstepcounter{equation}\label{prod:CtorDecln}  \: Mods\opt \xcd"def" \xcd"this" TypeParams\opt Formals Guard\opt HasResultType\opt CtorBody  \\


\end{bbgrammarappendix}

\begin{bbgrammarappendix}{3.8in}

(\arabic{equation}) & DepNamedType \refstepcounter{equation}\label{prod:DepNamedType}  \: SimpleNamedType DepParams  \\

 &    \| ParamizedNamedType DepParams \\

\end{bbgrammarappendix}

\begin{bbgrammarappendix}{4.4in}

(\arabic{equation}) & DoStmt \refstepcounter{equation}\label{prod:DoStmt}  \: \xcd"do" Stmt \xcd"while" \xcd"(" Exp \xcd")" \xcd";"  \\


\end{bbgrammarappendix}

\begin{bbgrammarappendix}{4.1in}

(\arabic{equation}) & EmptyStmt \refstepcounter{equation}\label{prod:EmptyStmt}  \: \xcd";"  \\


\end{bbgrammarappendix}

\begin{bbgrammarappendix}{3.5in}

(\arabic{equation}) & EnhancedForStmt \refstepcounter{equation}\label{prod:EnhancedForStmt}  \: \xcd"for" \xcd"(" LoopIndex \xcd"in" Exp \xcd")" Stmt  \\

 &    \| \xcd"for" \xcd"(" Exp \xcd")" Stmt \\

\end{bbgrammarappendix}

\begin{bbgrammarappendix}{3.9in}

(\arabic{equation}) & EqualityExp \refstepcounter{equation}\label{prod:EqualityExp}  \: RelationalExp  \\

 &    \| EqualityExp \xcd"==" RelationalExp \\
 &    \| EqualityExp \xcd"!=" RelationalExp \\
 &    \| Type  \xcd"==" Type  \\
 &    \| EqualityExp \xcd"~" RelationalExp \\
 &    \| EqualityExp \xcd"!~" RelationalExp \\

\end{bbgrammarappendix}

\begin{bbgrammarappendix}{3.6in}

(\arabic{equation}) & ExclusiveOrExp \refstepcounter{equation}\label{prod:ExclusiveOrExp}  \: AndExp  \\

 &    \| ExclusiveOrExp \xcd"^" AndExp \\

\end{bbgrammarappendix}

\begin{bbgrammarappendix}{4.7in}

(\arabic{equation}) & Exp \refstepcounter{equation}\label{prod:Exp}  \: AsstExp  \\


\end{bbgrammarappendix}

\begin{bbgrammarappendix}{4.3in}

(\arabic{equation}) & ExpName \refstepcounter{equation}\label{prod:ExpName}  \: Id  \\

 &    \| FullyQualifiedName \xcd"." Id \\

\end{bbgrammarappendix}

\begin{bbgrammarappendix}{4.3in}

(\arabic{equation}) & ExpStmt \refstepcounter{equation}\label{prod:ExpStmt}  \: StmtExp \xcd";"  \\


\end{bbgrammarappendix}

\begin{bbgrammarappendix}{3.5in}

(\arabic{equation}) & ExplConvOpDecln \refstepcounter{equation}\label{prod:ExplConvOpDecln}  \: MethMods \xcd"operator" TypeParams\opt \xcd"(" Formal  \xcd")" \xcd"as" Type Guard\opt MethodBody  \\

 &    \| MethMods \xcd"operator" TypeParams\opt \xcd"(" Formal  \xcd")" \xcd"as" \xcd"?" Guard\opt HasResultType\opt MethodBody \\

\end{bbgrammarappendix}

\begin{bbgrammarappendix}{3.4in}

(\arabic{equation}) & ExplicitCtorInvo \refstepcounter{equation}\label{prod:ExplicitCtorInvo}  \: \xcd"this" TypeArgs\opt \xcd"(" ArgumentList\opt \xcd")" \xcd";"  \\

 &    \| \xcd"super" TypeArgs\opt \xcd"(" ArgumentList\opt \xcd")" \xcd";" \\
 &    \| Primary \xcd"." \xcd"this" TypeArgs\opt \xcd"(" ArgumentList\opt \xcd")" \xcd";" \\
 &    \| Primary \xcd"." \xcd"super" TypeArgs\opt \xcd"(" ArgumentList\opt \xcd")" \xcd";" \\

\end{bbgrammarappendix}

\begin{bbgrammarappendix}{3.3in}

(\arabic{equation}) & ExtendsInterfaces \refstepcounter{equation}\label{prod:ExtendsInterfaces}  \: \xcd"extends" Type  \\

 &    \| ExtendsInterfaces \xcd"," Type \\

\end{bbgrammarappendix}

\begin{bbgrammarappendix}{3.9in}

(\arabic{equation}) & FieldAccess \refstepcounter{equation}\label{prod:FieldAccess}  \: Primary \xcd"." Id  \\

 &    \| \xcd"super" \xcd"." Id \\
 &    \| ClassName \xcd"." \xcd"super"  \xcd"." Id \\

\end{bbgrammarappendix}

\begin{bbgrammarappendix}{4.0in}

(\arabic{equation}) & FieldDecln \refstepcounter{equation}\label{prod:FieldDecln}  \: Mods\opt VarKeyword FieldDeclrs \xcd";"  \\

 &    \| Mods\opt FieldDeclrs \xcd";" \\

\end{bbgrammarappendix}

\begin{bbgrammarappendix}{4.0in}

(\arabic{equation}) & FieldDeclr \refstepcounter{equation}\label{prod:FieldDeclr}  \: Id HasResultType  \\

 &    \| Id HasResultType\opt \xcd"=" VariableInitializer \\

\end{bbgrammarappendix}

\begin{bbgrammarappendix}{3.9in}

(\arabic{equation}) & FieldDeclrs \refstepcounter{equation}\label{prod:FieldDeclrs}  \: FieldDeclr  \\

 &    \| FieldDeclrs \xcd"," FieldDeclr \\

\end{bbgrammarappendix}

\begin{bbgrammarappendix}{4.3in}

(\arabic{equation}) & Finally \refstepcounter{equation}\label{prod:Finally}  \: \xcd"finally" Block  \\


\end{bbgrammarappendix}

\begin{bbgrammarappendix}{4.0in}

(\arabic{equation}) & FinishStmt \refstepcounter{equation}\label{prod:FinishStmt}  \: \xcd"finish" Stmt  \\

 &    \| \xcd"clocked" \xcd"finish" Stmt \\

\end{bbgrammarappendix}

\begin{bbgrammarappendix}{4.3in}

(\arabic{equation}) & ForInit \refstepcounter{equation}\label{prod:ForInit}  \: StmtExpList  \\

 &    \| LocVarDecln \\

\end{bbgrammarappendix}

\begin{bbgrammarappendix}{4.3in}

(\arabic{equation}) & ForStmt \refstepcounter{equation}\label{prod:ForStmt}  \: BasicForStmt  \\

 &    \| EnhancedForStmt \\

\end{bbgrammarappendix}

\begin{bbgrammarappendix}{4.1in}

(\arabic{equation}) & ForUpdate \refstepcounter{equation}\label{prod:ForUpdate}  \: StmtExpList  \\


\end{bbgrammarappendix}

\begin{bbgrammarappendix}{4.4in}

(\arabic{equation}) & Formal \refstepcounter{equation}\label{prod:Formal}  \: Mods\opt FormalDeclr  \\

 &    \| Mods\opt VarKeyword FormalDeclr \\
 &    \| Type \\

\end{bbgrammarappendix}

\begin{bbgrammarappendix}{3.9in}

(\arabic{equation}) & FormalDeclr \refstepcounter{equation}\label{prod:FormalDeclr}  \: Id ResultType  \\

 &    \| \xcd"[" IdList \xcd"]" ResultType \\
 &    \| Id \xcd"[" IdList \xcd"]" ResultType \\

\end{bbgrammarappendix}

\begin{bbgrammarappendix}{3.8in}

(\arabic{equation}) & FormalDeclrs \refstepcounter{equation}\label{prod:FormalDeclrs}  \: FormalDeclr  \\

 &    \| FormalDeclrs \xcd"," FormalDeclr \\

\end{bbgrammarappendix}

\begin{bbgrammarappendix}{4.0in}

(\arabic{equation}) & FormalList \refstepcounter{equation}\label{prod:FormalList}  \: Formal  \\

 &    \| FormalList \xcd"," Formal \\

\end{bbgrammarappendix}

\begin{bbgrammarappendix}{4.3in}

(\arabic{equation}) & Formals \refstepcounter{equation}\label{prod:Formals}  \: \xcd"(" FormalList\opt \xcd")"  \\


\end{bbgrammarappendix}

\begin{bbgrammarappendix}{3.2in}

(\arabic{equation}) & FullyQualifiedName \refstepcounter{equation}\label{prod:FullyQualifiedName}  \: Id  \\

 &    \| FullyQualifiedName \xcd"." Id \\

\end{bbgrammarappendix}

\begin{bbgrammarappendix}{3.8in}

(\arabic{equation}) & FunctionType \refstepcounter{equation}\label{prod:FunctionType}  \: TypeParams\opt \xcd"(" FormalList\opt \xcd")" Guard\opt \xcd"=>" Type  \\


\end{bbgrammarappendix}

\begin{bbgrammarappendix}{4.5in}

(\arabic{equation}) & Guard \refstepcounter{equation}\label{prod:Guard}  \: DepParams  \\


\end{bbgrammarappendix}


\begin{bbgrammarappendix}{4.5in}

(\arabic{equation}) & Throws \refstepcounter{equation}\label{prod:Throws}  \: \xcd"throws" ThrowsList  \\

\end{bbgrammarappendix}

\begin{bbgrammarappendix}{3.3in}

(\arabic{equation}) & ThrowsList \refstepcounter{equation}\label{prod:ThrowsList}  \: Type  \\

 &    \| ThrowsList \xcd"," Type \\

\end{bbgrammarappendix}

\begin{bbgrammarappendix}{3.7in}

(\arabic{equation}) & HasResultType \refstepcounter{equation}\label{prod:HasResultType}  \: ResultType  \\

 &    \| \xcd"<:" Type \\

\end{bbgrammarappendix}

\begin{bbgrammarappendix}{3.3in}

(\arabic{equation}) & HasZeroConstraint \refstepcounter{equation}\label{prod:HasZeroConstraint}  \: Type  \xcd"haszero"  \\


\end{bbgrammarappendix}

\begin{bbgrammarappendix}{3.8in}

(\arabic{equation}) & HomeVariable \refstepcounter{equation}\label{prod:HomeVariable}  \: Id  \\

 &    \| \xcd"this" \\

\end{bbgrammarappendix}

\begin{bbgrammarappendix}{3.4in}

(\arabic{equation}) & HomeVariableList \refstepcounter{equation}\label{prod:HomeVariableList}  \: HomeVariable  \\

 &    \| HomeVariableList \xcd"," HomeVariable \\

\end{bbgrammarappendix}

\begin{bbgrammarappendix}{4.8in}

(\arabic{equation}) & Id \refstepcounter{equation}\label{prod:Id}  \: \xcd"IDENTIFIER"   \\


\end{bbgrammarappendix}

\begin{bbgrammarappendix}{4.4in}

(\arabic{equation}) & IdList \refstepcounter{equation}\label{prod:IdList}  \: Id  \\

 &    \| IdList \xcd"," Id \\

\end{bbgrammarappendix}

\begin{bbgrammarappendix}{3.6in}

(\arabic{equation}) & IfThenElseStmt \refstepcounter{equation}\label{prod:IfThenElseStmt}  \: \xcd"if" \xcd"(" Exp \xcd")" Stmt  \xcd"else" Stmt   \\


\end{bbgrammarappendix}

\begin{bbgrammarappendix}{4.0in}

(\arabic{equation}) & IfThenStmt \refstepcounter{equation}\label{prod:IfThenStmt}  \: \xcd"if" \xcd"(" Exp \xcd")" Stmt  \\


\end{bbgrammarappendix}

\begin{bbgrammarappendix}{3.5in}

(\arabic{equation}) & ImplConvOpDecln \refstepcounter{equation}\label{prod:ImplConvOpDecln}  \: MethMods \xcd"operator" TypeParams\opt \xcd"(" Formal  \xcd")" Guard\opt HasResultType\opt MethodBody  \\


\end{bbgrammarappendix}

\begin{bbgrammarappendix}{3.9in}

(\arabic{equation}) & ImportDecln \refstepcounter{equation}\label{prod:ImportDecln}  \: SingleTypeImportDecln  \\

 &    \| TypeImportOnDemandDecln \\

\end{bbgrammarappendix}

\begin{bbgrammarappendix}{3.8in}

(\arabic{equation}) & ImportDeclns \refstepcounter{equation}\label{prod:ImportDeclns}  \: ImportDecln  \\

 &    \| ImportDeclns ImportDecln \\

\end{bbgrammarappendix}

\begin{bbgrammarappendix}{3.6in}

(\arabic{equation}) & InclusiveOrExp \refstepcounter{equation}\label{prod:InclusiveOrExp}  \: ExclusiveOrExp  \\

 &    \| InclusiveOrExp \xcd"|" ExclusiveOrExp \\

\end{bbgrammarappendix}

\begin{bbgrammarappendix}{3.7in}

(\arabic{equation}) & InterfaceBody \refstepcounter{equation}\label{prod:InterfaceBody}  \: \xcd"{" InterfaceMemberDeclns\opt \xcd"}"  \\


\end{bbgrammarappendix}

\begin{bbgrammarappendix}{3.6in}

(\arabic{equation}) & InterfaceDecln \refstepcounter{equation}\label{prod:InterfaceDecln}  \: Mods\opt \xcd"interface" Id TypeParamsI\opt Properties\opt Guard\opt ExtendsInterfaces\opt InterfaceBody  \\


\end{bbgrammarappendix}

\begin{bbgrammarappendix}{3.0in}

(\arabic{equation}) & InterfaceMemberDecln \refstepcounter{equation}\label{prod:InterfaceMemberDecln}  \: MethodDecln  \\

 &    \| PropMethodDecln \\
 &    \| FieldDecln \\
 &    \| TypeDecln \\

\end{bbgrammarappendix}

\begin{bbgrammarappendix}{2.9in}

(\arabic{equation}) & InterfaceMemberDeclns \refstepcounter{equation}\label{prod:InterfaceMemberDeclns}  \: InterfaceMemberDecln  \\

 &    \| InterfaceMemberDeclns InterfaceMemberDecln \\

\end{bbgrammarappendix}

\begin{bbgrammarappendix}{3.3in}

(\arabic{equation}) & InterfaceTypeList \refstepcounter{equation}\label{prod:InterfaceTypeList}  \: Type  \\

 &    \| InterfaceTypeList \xcd"," Type \\

\end{bbgrammarappendix}

\begin{bbgrammarappendix}{4.0in}

(\arabic{equation}) & Interfaces \refstepcounter{equation}\label{prod:Interfaces}  \: \xcd"implements" InterfaceTypeList  \\


\end{bbgrammarappendix}

\begin{bbgrammarappendix}{3.9in}

(\arabic{equation}) & LabeledStmt \refstepcounter{equation}\label{prod:LabeledStmt}  \: Id \xcd":" LoopStmt  \\


\end{bbgrammarappendix}

\begin{bbgrammarappendix}{4.3in}

(\arabic{equation}) & LastExp \refstepcounter{equation}\label{prod:LastExp}  \: Exp  \\


\end{bbgrammarappendix}

\begin{bbgrammarappendix}{3.8in}

(\arabic{equation}) & LeftHandSide \refstepcounter{equation}\label{prod:LeftHandSide}  \: ExpName  \\

 &    \| FieldAccess \\

\end{bbgrammarappendix}

\begin{bbgrammarappendix}{4.3in}

(\arabic{equation}) & Literal \refstepcounter{equation}\label{prod:Literal}  \: \xcd"IntegerLiteral"   \\

 &    \| \xcd"LongLiteral"  \\
 &    \| \xcd"ByteLiteral" \\
 &    \| \xcd"UnsignedByteLiteral" \\
 &    \| \xcd"ShortLiteral" \\
 &    \| \xcd"UnsignedShortLiteral" \\
 &    \| \xcd"UnsignedIntegerLiteral"  \\
 &    \| \xcd"UnsignedLongLiteral"  \\
 &    \| \xcd"FloatingPointLiteral"  \\
 &    \| \xcd"DoubleLiteral"  \\
 &    \| BooleanLiteral \\
 &    \| \xcd"CharacterLiteral"  \\
 &    \| \xcd"StringLiteral"  \\
 &    \| \xcd"null" \\

\end{bbgrammarappendix}

\begin{bbgrammarappendix}{3.9in}

(\arabic{equation}) & LocVarDecln \refstepcounter{equation}\label{prod:LocVarDecln}  \: Mods\opt VarKeyword VariableDeclrs  \\

 &    \| Mods\opt VarDeclsWType \\
 &    \| Mods\opt VarKeyword FormalDeclrs \\

\end{bbgrammarappendix}

\begin{bbgrammarappendix}{3.5in}

(\arabic{equation}) & LocVarDeclnStmt \refstepcounter{equation}\label{prod:LocVarDeclnStmt}  \: LocVarDecln \xcd";"  \\


\end{bbgrammarappendix}

\begin{bbgrammarappendix}{4.1in}

(\arabic{equation}) & LoopIndex \refstepcounter{equation}\label{prod:LoopIndex}  \: Mods\opt LoopIndexDeclr  \\

 &    \| Mods\opt VarKeyword LoopIndexDeclr \\

\end{bbgrammarappendix}

\begin{bbgrammarappendix}{3.6in}

(\arabic{equation}) & LoopIndexDeclr \refstepcounter{equation}\label{prod:LoopIndexDeclr}  \: Id HasResultType\opt  \\

 &    \| \xcd"[" IdList \xcd"]" HasResultType\opt \\
 &    \| Id \xcd"[" IdList \xcd"]" HasResultType\opt \\

\end{bbgrammarappendix}

\begin{bbgrammarappendix}{4.2in}

(\arabic{equation}) & LoopStmt \refstepcounter{equation}\label{prod:LoopStmt}  \: ForStmt  \\

 &    \| WhileStmt \\
 &    \| DoStmt \\
 &    \| AtEachStmt \\

\end{bbgrammarappendix}

\begin{bbgrammarappendix}{4.2in}

(\arabic{equation}) & MethMods \refstepcounter{equation}\label{prod:MethMods}  \: Mods\opt  \\

 &    \| MethMods \xcd"property"  \\
 &    \| MethMods Mod \\

\end{bbgrammarappendix}

\begin{bbgrammarappendix}{4.0in}

(\arabic{equation}) & MethodBody \refstepcounter{equation}\label{prod:MethodBody}  \: \xcd"=" LastExp \xcd";"  \\

 &    \| \xcd"=" Annotations\opt \xcd"{" BlockStmts\opt LastExp \xcd"}" \\
 &    \| \xcd"=" Annotations\opt Block \\
 &    \| Annotations\opt Block \\
 &    \| \xcd";" \\

\end{bbgrammarappendix}

\begin{bbgrammarappendix}{3.9in}

(\arabic{equation}) & MethodDecln
  \refstepcounter{equation}\label{prod:MethodDecln}  \: MethMods \xcd"def" Id TypeParams\opt Formals Guard\opt Throws\opt HasResultType\opt MethodBody  \\

 &    \| BinOpDecln \\
 &    \| PrefixOpDecln \\
 &    \| ApplyOpDecln \\
 &    \| SetOpDecln \\
 &    \| ConversionOpDecln \\

\end{bbgrammarappendix}

\begin{bbgrammarappendix}{4.0in}

(\arabic{equation}) & MethodInvo \refstepcounter{equation}\label{prod:MethodInvo}  \: MethodName TypeArgs\opt \xcd"(" ArgumentList\opt \xcd")"  \\

 &    \| Primary \xcd"." Id TypeArgs\opt \xcd"(" ArgumentList\opt \xcd")" \\
 &    \| \xcd"super" \xcd"." Id TypeArgs\opt \xcd"(" ArgumentList\opt \xcd")" \\
 &    \| ClassName \xcd"." \xcd"super"  \xcd"." Id TypeArgs\opt \xcd"(" ArgumentList\opt \xcd")" \\
 &    \| Primary TypeArgs\opt \xcd"(" ArgumentList\opt \xcd")" \\

\end{bbgrammarappendix}

\begin{bbgrammarappendix}{4.0in}

(\arabic{equation}) & MethodName \refstepcounter{equation}\label{prod:MethodName}  \: Id  \\

 &    \| FullyQualifiedName \xcd"." Id \\

\end{bbgrammarappendix}

\begin{bbgrammarappendix}{4.7in}

(\arabic{equation}) & Mod \refstepcounter{equation}\label{prod:Mod}  \: \xcd"abstract"  \\

 &    \| Annotation \\
 &    \| \xcd"atomic" \\
 &    \| \xcd"final" \\
 &    \| \xcd"native" \\
 &    \| \xcd"private" \\
 &    \| \xcd"protected" \\
 &    \| \xcd"public" \\
 &    \| \xcd"static" \\
 &    \| \xcd"transient" \\
 &    \| \xcd"clocked" \\

\end{bbgrammarappendix}

\begin{bbgrammarappendix}{3.3in}

(\arabic{equation}) & MultiplicativeExp \refstepcounter{equation}\label{prod:MultiplicativeExp}  \: RangeExp  \\

 &    \| MultiplicativeExp \xcd"*" RangeExp \\
 &    \| MultiplicativeExp \xcd"/" RangeExp \\
 &    \| MultiplicativeExp \xcd"%" RangeExp \\
 &    \| MultiplicativeExp \xcd"**" RangeExp \\

\end{bbgrammarappendix}

\begin{bbgrammarappendix}{4.1in}

(\arabic{equation}) & NamedType \refstepcounter{equation}\label{prod:NamedType}  \: NamedTypeNoConstraints  \\

 &    \| DepNamedType \\

\end{bbgrammarappendix}

\begin{bbgrammarappendix}{2.8in}

(\arabic{equation}) & NamedTypeNoConstraints \refstepcounter{equation}\label{prod:NamedTypeNoConstraints}  \: SimpleNamedType  \\

 &    \| ParamizedNamedType \\

\end{bbgrammarappendix}

\begin{bbgrammarappendix}{4.0in}

(\arabic{equation}) & NonExpStmt \refstepcounter{equation}\label{prod:NonExpStmt}  \: Block  \\

 &    \| EmptyStmt \\
 &    \| AssertStmt \\
 &    \| SwitchStmt \\
 &    \| DoStmt \\
 &    \| BreakStmt \\
 &    \| ContinueStmt \\
 &    \| ReturnStmt \\
 &    \| ThrowStmt \\
 &    \| TryStmt \\
 &    \| LabeledStmt \\
 &    \| IfThenStmt \\
 &    \| IfThenElseStmt \\
 &    \| WhileStmt \\
 &    \| ForStmt \\
 &    \| AsyncStmt \\
 &    \| AtStmt \\
 &    \| AtomicStmt \\
 &    \| WhenStmt \\
 &    \| AtEachStmt \\
 &    \| FinishStmt \\
 &    \| AssignPropCall \\

\end{bbgrammarappendix}

\begin{bbgrammarappendix}{3.7in}

(\arabic{equation}) & ObCreationExp \refstepcounter{equation}\label{prod:ObCreationExp}  \: \xcd"new" TypeName TypeArgs\opt \xcd"(" ArgumentList\opt \xcd")" ClassBody\opt  \\

 &    \| Primary \xcd"." \xcd"new" Id TypeArgs\opt \xcd"(" ArgumentList\opt \xcd")" ClassBody\opt \\
 &    \| FullyQualifiedName \xcd"." \xcd"new" Id TypeArgs\opt \xcd"(" ArgumentList\opt \xcd")" ClassBody\opt \\

\end{bbgrammarappendix}

\begin{bbgrammarappendix}{3.8in}

(\arabic{equation}) & PackageDecln \refstepcounter{equation}\label{prod:PackageDecln}  \: Annotations\opt \xcd"package" PackageName \xcd";"  \\


\end{bbgrammarappendix}

\begin{bbgrammarappendix}{3.9in}

(\arabic{equation}) & PackageName \refstepcounter{equation}\label{prod:PackageName}  \: Id  \\

 &    \| PackageName \xcd"." Id \\

\end{bbgrammarappendix}

\begin{bbgrammarappendix}{3.3in}

(\arabic{equation}) & PackageOrTypeName \refstepcounter{equation}\label{prod:PackageOrTypeName}  \: Id  \\

 &    \| PackageOrTypeName \xcd"." Id \\

\end{bbgrammarappendix}

\begin{bbgrammarappendix}{3.2in}

(\arabic{equation}) & ParamizedNamedType \refstepcounter{equation}\label{prod:ParamizedNamedType}  \: SimpleNamedType Arguments  \\

 &    \| SimpleNamedType TypeArgs \\
 &    \| SimpleNamedType TypeArgs Arguments \\

\end{bbgrammarappendix}

\begin{bbgrammarappendix}{3.4in}

(\arabic{equation}) & PostDecrementExp \refstepcounter{equation}\label{prod:PostDecrementExp}  \: PostfixExp \xcd"--"  \\


\end{bbgrammarappendix}

\begin{bbgrammarappendix}{3.4in}

(\arabic{equation}) & PostIncrementExp \refstepcounter{equation}\label{prod:PostIncrementExp}  \: PostfixExp \xcd"++"  \\


\end{bbgrammarappendix}

\begin{bbgrammarappendix}{4.0in}

(\arabic{equation}) & PostfixExp \refstepcounter{equation}\label{prod:PostfixExp}  \: CastExp  \\

 &    \| PostIncrementExp \\
 &    \| PostDecrementExp \\

\end{bbgrammarappendix}

\begin{bbgrammarappendix}{3.5in}

(\arabic{equation}) & PreDecrementExp \refstepcounter{equation}\label{prod:PreDecrementExp}  \: \xcd"--" UnaryExpNotPlusMinus  \\


\end{bbgrammarappendix}

\begin{bbgrammarappendix}{3.5in}

(\arabic{equation}) & PreIncrementExp \refstepcounter{equation}\label{prod:PreIncrementExp}  \: \xcd"++" UnaryExpNotPlusMinus  \\


\end{bbgrammarappendix}

\begin{bbgrammarappendix}{4.2in}

(\arabic{equation}) & PrefixOp \refstepcounter{equation}\label{prod:PrefixOp}  \: \xcd"+"  \\

 &    \| \xcd"-" \\
 &    \| \xcd"!" \\
 &    \| \xcd"~" \\
 &    \| \xcd"^" \\
 &    \| \xcd"|" \\
 &    \| \xcd"&" \\
 &    \| \xcd"*" \\
 &    \| \xcd"/" \\
 &    \| \xcd"%" \\

\end{bbgrammarappendix}

\begin{bbgrammarappendix}{3.7in}

(\arabic{equation}) & PrefixOpDecln \refstepcounter{equation}\label{prod:PrefixOpDecln}  \: MethMods \xcd"operator" TypeParams\opt PrefixOp \xcd"(" Formal  \xcd")" Guard\opt HasResultType\opt MethodBody  \\

 &    \| MethMods \xcd"operator" TypeParams\opt PrefixOp \xcd"this" Guard\opt HasResultType\opt MethodBody \\

\end{bbgrammarappendix}

\begin{bbgrammarappendix}{4.3in}

(\arabic{equation}) & Primary \refstepcounter{equation}\label{prod:Primary}  \: \xcd"here"  \\

 &    \| \xcd"[" ArgumentList\opt \xcd"]" \\
 &    \| Literal \\
 &    \| \xcd"self" \\
 &    \| \xcd"this" \\
 &    \| ClassName \xcd"." \xcd"this" \\
 &    \| \xcd"(" Exp \xcd")" \\
 &    \| ObCreationExp \\
 &    \| FieldAccess \\
 &    \| MethodInvo \\

\end{bbgrammarappendix}

\begin{bbgrammarappendix}{4.6in}

(\arabic{equation}) & Prop \refstepcounter{equation}\label{prod:Prop}  \: Annotations\opt Id ResultType  \\


\end{bbgrammarappendix}

\begin{bbgrammarappendix}{4.2in}

(\arabic{equation}) & PropList \refstepcounter{equation}\label{prod:PropList}  \: Prop  \\

 &    \| PropList \xcd"," Prop \\

\end{bbgrammarappendix}

\begin{bbgrammarappendix}{3.5in}

(\arabic{equation}) & PropMethodDecln \refstepcounter{equation}\label{prod:PropMethodDecln}  \: MethMods Id TypeParams\opt Formals Guard\opt HasResultType\opt MethodBody  \\

 &    \| MethMods Id Guard\opt HasResultType\opt MethodBody \\

\end{bbgrammarappendix}

\begin{bbgrammarappendix}{4.0in}

(\arabic{equation}) & Properties \refstepcounter{equation}\label{prod:Properties}  \: \xcd"(" PropList \xcd")"  \\


\end{bbgrammarappendix}

\begin{bbgrammarappendix}{4.2in}

(\arabic{equation}) & RangeExp \refstepcounter{equation}\label{prod:RangeExp}  \: UnaryExp  \\

 &    \| RangeExp  \xcd".." UnaryExp  \\

\end{bbgrammarappendix}

\begin{bbgrammarappendix}{3.7in}

(\arabic{equation}) & RelationalExp \refstepcounter{equation}\label{prod:RelationalExp}  \: ShiftExp  \\

 &    \| HasZeroConstraint \\
 &    \| SubtypeConstraint \\
 &    \| RelationalExp \xcd"<" ShiftExp \\
 &    \| RelationalExp \xcd">" ShiftExp \\
 &    \| RelationalExp \xcd"<=" ShiftExp \\
 &    \| RelationalExp \xcd">=" ShiftExp \\
 &    \| RelationalExp \xcd"instanceof" Type \\

\end{bbgrammarappendix}

\begin{bbgrammarappendix}{4.0in}

(\arabic{equation}) & ResultType \refstepcounter{equation}\label{prod:ResultType}  \: \xcd":" Type  \\


\end{bbgrammarappendix}

\begin{bbgrammarappendix}{4.0in}

(\arabic{equation}) & ReturnStmt \refstepcounter{equation}\label{prod:ReturnStmt}  \: \xcd"return" Exp\opt \xcd";"  \\


\end{bbgrammarappendix}

\begin{bbgrammarappendix}{4.0in}

(\arabic{equation}) & SetOpDecln \refstepcounter{equation}\label{prod:SetOpDecln}  \: MethMods \xcd"operator" \xcd"this" TypeParams\opt Formals \xcd"=" \xcd"(" Formal  \xcd")" Guard\opt HasResultType\opt MethodBody  \\


\end{bbgrammarappendix}

\begin{bbgrammarappendix}{4.2in}

(\arabic{equation}) & ShiftExp \refstepcounter{equation}\label{prod:ShiftExp}  \: AdditiveExp  \\

 &    \| ShiftExp \xcd"<<" AdditiveExp \\
 &    \| ShiftExp \xcd">>" AdditiveExp \\
 &    \| ShiftExp \xcd">>>" AdditiveExp \\
 &    \| ShiftExp  \xcd"->" AdditiveExp  \\
 &    \| ShiftExp  \xcd"<-" AdditiveExp  \\
 &    \| ShiftExp  \xcd"-<" AdditiveExp  \\
 &    \| ShiftExp  \xcd">-" AdditiveExp  \\
 &    \| ShiftExp  \xcd"!" AdditiveExp  \\

\end{bbgrammarappendix}

\begin{bbgrammarappendix}{3.5in}

(\arabic{equation}) & SimpleNamedType \refstepcounter{equation}\label{prod:SimpleNamedType}  \: TypeName  \\

 &    \| Primary \xcd"." Id \\
 &    \| ParamizedNamedType \xcd"." Id \\
 &    \| DepNamedType \xcd"." Id \\

\end{bbgrammarappendix}

\begin{bbgrammarappendix}{2.9in}

(\arabic{equation}) & SingleTypeImportDecln \refstepcounter{equation}\label{prod:SingleTypeImportDecln}  \: \xcd"import" TypeName \xcd";"  \\


\end{bbgrammarappendix}

\begin{bbgrammarappendix}{4.6in}

(\arabic{equation}) & Stmt \refstepcounter{equation}\label{prod:Stmt}  \: AnnotationStmt  \\

 &    \| ExpStmt \\

\end{bbgrammarappendix}

\begin{bbgrammarappendix}{4.3in}

(\arabic{equation}) & StmtExp \refstepcounter{equation}\label{prod:StmtExp}  \: Assignment  \\

 &    \| PreIncrementExp \\
 &    \| PreDecrementExp \\
 &    \| PostIncrementExp \\
 &    \| PostDecrementExp \\
 &    \| MethodInvo \\
 &    \| ObCreationExp \\

\end{bbgrammarappendix}

\begin{bbgrammarappendix}{3.9in}

(\arabic{equation}) & StmtExpList \refstepcounter{equation}\label{prod:StmtExpList}  \: StmtExp  \\

 &    \| StmtExpList \xcd"," StmtExp \\

\end{bbgrammarappendix}

\begin{bbgrammarappendix}{3.9in}

(\arabic{equation}) & StructDecln \refstepcounter{equation}\label{prod:StructDecln}  \: Mods\opt \xcd"struct" Id TypeParamsI\opt Properties\opt Guard\opt Interfaces\opt ClassBody  \\


\end{bbgrammarappendix}

\begin{bbgrammarappendix}{3.3in}

(\arabic{equation}) & SubtypeConstraint \refstepcounter{equation}\label{prod:SubtypeConstraint}  \: Type  \xcd"<:" Type   \\

 &    \| Type  \xcd":>" Type  \\

\end{bbgrammarappendix}

\begin{bbgrammarappendix}{4.5in}

(\arabic{equation}) & Super \refstepcounter{equation}\label{prod:Super}  \: \xcd"extends" ClassType  \\


\end{bbgrammarappendix}

\begin{bbgrammarappendix}{3.9in}

(\arabic{equation}) & SwitchBlock \refstepcounter{equation}\label{prod:SwitchBlock}  \: \xcd"{" SwitchBlockGroups\opt SwitchLabels\opt \xcd"}"  \\


\end{bbgrammarappendix}

\begin{bbgrammarappendix}{3.4in}

(\arabic{equation}) & SwitchBlockGroup \refstepcounter{equation}\label{prod:SwitchBlockGroup}  \: SwitchLabels BlockStmts  \\


\end{bbgrammarappendix}

\begin{bbgrammarappendix}{3.3in}

(\arabic{equation}) & SwitchBlockGroups \refstepcounter{equation}\label{prod:SwitchBlockGroups}  \: SwitchBlockGroup  \\

 &    \| SwitchBlockGroups SwitchBlockGroup \\

\end{bbgrammarappendix}

\begin{bbgrammarappendix}{3.9in}

(\arabic{equation}) & SwitchLabel \refstepcounter{equation}\label{prod:SwitchLabel}  \: \xcd"case" ConstantExp \xcd":"  \\

 &    \| \xcd"default" \xcd":" \\

\end{bbgrammarappendix}

\begin{bbgrammarappendix}{3.8in}

(\arabic{equation}) & SwitchLabels \refstepcounter{equation}\label{prod:SwitchLabels}  \: SwitchLabel  \\

 &    \| SwitchLabels SwitchLabel \\

\end{bbgrammarappendix}

\begin{bbgrammarappendix}{4.0in}

(\arabic{equation}) & SwitchStmt \refstepcounter{equation}\label{prod:SwitchStmt}  \: \xcd"switch" \xcd"(" Exp \xcd")" SwitchBlock  \\


\end{bbgrammarappendix}

\begin{bbgrammarappendix}{4.1in}

(\arabic{equation}) & ThrowStmt \refstepcounter{equation}\label{prod:ThrowStmt}  \: \xcd"throw" Exp \xcd";"  \\


\end{bbgrammarappendix}

\begin{bbgrammarappendix}{4.3in}

(\arabic{equation}) & TryStmt \refstepcounter{equation}\label{prod:TryStmt}  \: \xcd"try" Block Catches  \\

 &    \| \xcd"try" Block Catches\opt Finally \\

\end{bbgrammarappendix}

\begin{bbgrammarappendix}{4.6in}

(\arabic{equation}) & Type \refstepcounter{equation}\label{prod:Type}  \: FunctionType  \\

 &    \| ConstrainedType \\
 &    \| Void \\

\end{bbgrammarappendix}

\begin{bbgrammarappendix}{4.2in}

(\arabic{equation}) & TypeArgs \refstepcounter{equation}\label{prod:TypeArgs}  \: \xcd"[" TypeArgumentList \xcd"]"  \\


\end{bbgrammarappendix}

\begin{bbgrammarappendix}{3.4in}

(\arabic{equation}) & TypeArgumentList \refstepcounter{equation}\label{prod:TypeArgumentList}  \: Type  \\

 &    \| TypeArgumentList \xcd"," Type \\

\end{bbgrammarappendix}

\begin{bbgrammarappendix}{4.1in}

(\arabic{equation}) & TypeDecln \refstepcounter{equation}\label{prod:TypeDecln}  \: ClassDecln  \\

 &    \| StructDecln \\
 &    \| InterfaceDecln \\
 &    \| TypeDefDecln \\
 &    \| \xcd";" \\

\end{bbgrammarappendix}

\begin{bbgrammarappendix}{4.0in}

(\arabic{equation}) & TypeDeclns \refstepcounter{equation}\label{prod:TypeDeclns}  \: TypeDecln  \\

 &    \| TypeDeclns TypeDecln \\

\end{bbgrammarappendix}

\begin{bbgrammarappendix}{3.8in}

(\arabic{equation}) & TypeDefDecln \refstepcounter{equation}\label{prod:TypeDefDecln}  \: Mods\opt \xcd"type" Id TypeParams\opt Guard\opt \xcd"=" Type \xcd";"  \\

 &    \| Mods\opt \xcd"type" Id TypeParams\opt \xcd"(" FormalList \xcd")" Guard\opt \xcd"=" Type \xcd";" \\

\end{bbgrammarappendix}

\begin{bbgrammarappendix}{2.7in}

(\arabic{equation}) & TypeImportOnDemandDecln \refstepcounter{equation}\label{prod:TypeImportOnDemandDecln}  \: \xcd"import" PackageOrTypeName \xcd"." \xcd"*" \xcd";"  \\


\end{bbgrammarappendix}

\begin{bbgrammarappendix}{4.2in}

(\arabic{equation}) & TypeName \refstepcounter{equation}\label{prod:TypeName}  \: Id  \\

 &    \| TypeName \xcd"." Id \\

\end{bbgrammarappendix}

\begin{bbgrammarappendix}{4.1in}

(\arabic{equation}) & TypeParam \refstepcounter{equation}\label{prod:TypeParam}  \: Id  \\


\end{bbgrammarappendix}

\begin{bbgrammarappendix}{3.6in}

(\arabic{equation}) & TypeParamIList \refstepcounter{equation}\label{prod:TypeParamIList}  \: TypeParam  \\

 &    \| TypeParamIList \xcd"," TypeParam \\
 &    \| TypeParamIList \xcd"," \\

\end{bbgrammarappendix}

\begin{bbgrammarappendix}{3.7in}

(\arabic{equation}) & TypeParamList \refstepcounter{equation}\label{prod:TypeParamList}  \: TypeParam  \\

 &    \| TypeParamList \xcd"," TypeParam \\

\end{bbgrammarappendix}

\begin{bbgrammarappendix}{4.0in}

(\arabic{equation}) & TypeParams \refstepcounter{equation}\label{prod:TypeParams}  \: \xcd"[" TypeParamList \xcd"]"  \\


\end{bbgrammarappendix}

\begin{bbgrammarappendix}{3.9in}

(\arabic{equation}) & TypeParamsI \refstepcounter{equation}\label{prod:TypeParamsI}  \: \xcd"[" TypeParamIList \xcd"]"  \\


\end{bbgrammarappendix}

\begin{bbgrammarappendix}{3.1in}

(\arabic{equation}) & UnannotatedUnaryExp \refstepcounter{equation}\label{prod:UnannotatedUnaryExp}  \: PreIncrementExp  \\

 &    \| PreDecrementExp \\
 &    \| \xcd"+" UnaryExpNotPlusMinus \\
 &    \| \xcd"-" UnaryExpNotPlusMinus \\
 &    \| UnaryExpNotPlusMinus \\

\end{bbgrammarappendix}

\begin{bbgrammarappendix}{4.2in}

(\arabic{equation}) & UnaryExp \refstepcounter{equation}\label{prod:UnaryExp}  \: UnannotatedUnaryExp  \\

 &    \| Annotations UnannotatedUnaryExp \\

\end{bbgrammarappendix}

\begin{bbgrammarappendix}{3.0in}

(\arabic{equation}) & UnaryExpNotPlusMinus \refstepcounter{equation}\label{prod:UnaryExpNotPlusMinus}  \: PostfixExp  \\

 &    \| \xcd"~" UnaryExp \\
 &    \| \xcd"!" UnaryExp \\
 &    \| \xcd"^" UnaryExp \\
 &    \| \xcd"|" UnaryExp \\
 &    \| \xcd"&" UnaryExp \\
 &    \| \xcd"*" UnaryExp \\
 &    \| \xcd"/" UnaryExp \\
 &    \| \xcd"%" UnaryExp \\

\end{bbgrammarappendix}

\begin{bbgrammarappendix}{3.8in}

(\arabic{equation}) & VarDeclWType \refstepcounter{equation}\label{prod:VarDeclWType}  \: Id HasResultType \xcd"=" VariableInitializer  \\

 &    \| \xcd"[" IdList \xcd"]" HasResultType \xcd"=" VariableInitializer \\
 &    \| Id \xcd"[" IdList \xcd"]" HasResultType \xcd"=" VariableInitializer \\

\end{bbgrammarappendix}

\begin{bbgrammarappendix}{3.7in}

(\arabic{equation}) & VarDeclsWType \refstepcounter{equation}\label{prod:VarDeclsWType}  \: VarDeclWType  \\

 &    \| VarDeclsWType \xcd"," VarDeclWType \\

\end{bbgrammarappendix}

\begin{bbgrammarappendix}{4.0in}

(\arabic{equation}) & VarKeyword \refstepcounter{equation}\label{prod:VarKeyword}  \: \xcd"val"  \\

 &    \| \xcd"var" \\

\end{bbgrammarappendix}

\begin{bbgrammarappendix}{3.7in}

(\arabic{equation}) & VariableDeclr \refstepcounter{equation}\label{prod:VariableDeclr}  \: Id HasResultType\opt \xcd"=" VariableInitializer  \\

 &    \| \xcd"[" IdList \xcd"]" HasResultType\opt \xcd"=" VariableInitializer \\
 &    \| Id \xcd"[" IdList \xcd"]" HasResultType\opt \xcd"=" VariableInitializer \\

\end{bbgrammarappendix}

\begin{bbgrammarappendix}{3.6in}

(\arabic{equation}) & VariableDeclrs \refstepcounter{equation}\label{prod:VariableDeclrs}  \: VariableDeclr  \\

 &    \| VariableDeclrs \xcd"," VariableDeclr \\

\end{bbgrammarappendix}

\begin{bbgrammarappendix}{3.1in}

(\arabic{equation}) & VariableInitializer \refstepcounter{equation}\label{prod:VariableInitializer}  \: Exp  \\


\end{bbgrammarappendix}

\begin{bbgrammarappendix}{4.6in}

(\arabic{equation}) & Void \refstepcounter{equation}\label{prod:Void}  \: \xcd"void"  \\


\end{bbgrammarappendix}

\begin{bbgrammarappendix}{4.2in}

(\arabic{equation}) & WhenStmt \refstepcounter{equation}\label{prod:WhenStmt}  \: \xcd"when" \xcd"(" Exp \xcd")" Stmt  \\


\end{bbgrammarappendix}

\begin{bbgrammarappendix}{4.1in}

(\arabic{equation}) & WhileStmt \refstepcounter{equation}\label{prod:WhileStmt}  \: \xcd"while" \xcd"(" Exp \xcd")" Stmt  \\


\end{bbgrammarappendix}



\clearpage
\addcontentsline{toc}{chapter}{References}
\renewcommand{\bibname}{References}
\bibliographystyle{plain}
\bibliography{master}

%\documentclass[10pt,twoside,twocolumn,notitlepage]{report}
\documentclass[12pt,twoside,notitlepage]{report}
\usepackage{tex/x10}
\usepackage{tex/tenv}
\def\Hat{{\tt \char`\^}}
\usepackage{url}
\usepackage{times}
\usepackage{tex/txtt}
\usepackage{ifpdf}
\usepackage{tocloft}
\usepackage{tex/bcprules}
\usepackage{xspace}

\newif\ifdraft
%\drafttrue
\draftfalse

\pagestyle{headings}
\showboxdepth=0
\makeindex

\usepackage{tex/commands}

\usepackage[
pdfauthor={Vijay Saraswat, Bard Bloom, Igor Peshansky, Olivier Tardieu, and David Grove},
pdftitle={X10 Language Specification},
pdfcreator={pdftex},
pdfkeywords={X10},
linkcolor=blue,
citecolor=blue,
urlcolor=blue
]{hyperref}

\ifpdf
          \pdfinfo {
              /Author   (Vijay Saraswat, Bard Bloom, Igor Peshansky, Olivier Tardieu, and David Grove)
              /Title    (X10 Language Specification)
              /Keywords (X10)
              /Subject  ()
              /Creator  (TeX)
              /Producer (PDFLaTeX)
          }
\fi

\def\headertitle{The \XtenCurrVer{} Report (Draft) }
\def\integerversion{2.2}

% Sizes and dimensions

%\topmargin -.375in       %    Nominal distance from top of page to top of
                         %    box containing running head.
%\headsep 15pt            %    Space between running head and text.

%\textheight 9.0in        % Height of text (including footnotes and figures, 
                         % excluding running head and foot).

%\textwidth 5.5in         % Width of text line.
\columnsep 15pt          % Space between columns 
\columnseprule 0pt       % Width of rule between columns.

\parskip 5pt plus 2pt minus 2pt % Extra vertical space between paragraphs.
\parindent 0pt                  % Width of paragraph indentation.
%\topsep 0pt plus 2pt            % Extra vertical space, in addition to 
                                % \parskip, added above and below list and
                                % paragraphing environments.


\newif\iftwocolumn

\makeatletter
\twocolumnfalse
\if@twocolumn
\twocolumntrue
\fi
\makeatother

\iftwocolumn

\oddsidemargin  0in    % Left margin on odd-numbered pages.
\evensidemargin 0in    % Left margin on even-numbered pages.

\else

\oddsidemargin  .5in    % Left margin on odd-numbered pages.
\evensidemargin .5in    % Left margin on even-numbered pages.

\fi


\newtenv{example}{Example}[section]
\newtenv{planned}{Planned}[section]

\begin{document}

% \section{Work In Progress}
% \begin{itemize}
%     \item Rewrite first chapter
%     \item Describe library classes, including such fundamentals as Object and String
%     \item Examples for covariant/contravariant generics are wrong -- use Nate's examples
%     \item Describe local classes.
%     \item Reduce the use of \xcd`self` in constraints.
%     \item Copy sections of grammar to relevant sections of text.
%     \item Do something about 4.12.3
% \end{itemize}
% 
% {\bf Feedback:} 
% To help us the most, we would appreciate comments in one of these formats: 
% \begin{itemize}
% \item An annotated copy of the PDF document, if it's convenient.  Acrobat
%       Writer can produce helpful highlighting and sticky notes.  If you don't
%       use Acrobat Writer, don't fuss.
% \item Text comments.  Since the document is still being edited, page numbers
%       are going to be useless as pointers to the text.  If possible, we'd like
%       pointers to sections by number and title: {\em In 12.1, ``Empty
%       Statement'', please discuss side effects and performance implications
%       for this construct''}  If it's a long section, giving us a couple words
%       we can grep for would help too.
% \end{itemize}
% 
% Thank you very much!




% \parindent 0pt %!! 15pt                    % Width of paragraph indentation.

%\hfil {\bf 7 Feb 2005}
%\hfil \today{}

% First page

\thispagestyle{empty}

% \todo{"another" report?}

\topnewpage[{
\begin{center}   
{\huge\bf Report on the Experimental Language \Xten{}}
\vskip 1ex
$$
\begin{tabular}{l@{\extracolsep{.5in}}lll}
\multicolumn{4}{c}{\sc  Version 1.1}\\
\multicolumn{4}{c}{\sc Please send comments to 
V\authorsc{IJAY} S\authorsc{ARASWAT} at 
{\tt vsaraswa@us.ibm.com}}\\
%\multicolumn{4}{c}{({\sc IBM Confidential})}

%\ldots
\end{tabular}
$$
\vskip 2ex
% {\it Dedicated to the Memory of APL} % vj
{\bf Jun 30, 2007}
\vskip 2.6ex
\end{center}


}]


\chapter*{Summary}
This draft report provides an initial description of the programming
language \Xten. \Xten{} is a single-inheritance class-based object-oriented
(OO) programming language designed for high-performance, high-productivity
computing on high-end computers supporting $\approx 10^5$ hardware threads
and $\approx 10^{15}$ operations per second. 

{}\Xten{} is based on state-of-the-art object-oriented programming
languages and deviates from them only as necessary to support its
design goals. The language is intended to have a simple and clear
semantics and be readily accessible to mainstream OO programmers. It
is intended to support a wide variety of concurrent programming
idioms.
%, incuding data parallelism, task parallelism, pipelining.
%producer/consumer and divide and conquer.

%We expect to revise this document in the light of experience gained in implementing
%and using this language.

The \Xten{} design team consists of
D\authorsc{AVID} B\authorsc{ACON}, 
R\authorsc{AJ} B\authorsc{ARIK}, 
G\authorsc{ANESH} B\authorsc{IKSHANDI}, 
B\authorsc{OB} B\authorsc{LAINEY}, 
P\authorsc{HILIPPE} C\authorsc{HARLES}, 
P\authorsc{ERRY} C\authorsc{HENG}, 
C\authorsc{HRISTOPHER} D\authorsc{ONAWA}, 
J\authorsc{ULIAN} D\authorsc{OLBY}, 
K\authorsc{EMAL} E\authorsc{BCIO\u{G}LU},
R\authorsc{OBERT} F\authorsc{UHRER},
P\authorsc{ATRICK} G\authorsc{ALLOP}, 
C\authorsc{HRISTIAN} G\authorsc{ROTHOFF}, 
A\authorsc{LLAN} K\authorsc{IELSTRA}, 
S\authorsc{REEDHAR} K\authorsc{ODALI}, 
S\authorsc{RIRAM} K\authorsc{RISHNAMOORTHY}, 
N\authorsc{ATHANIEL} N\authorsc{YSTROM}, 
I\authorsc{IGOR} P\authorsc{ESHANSKY}, 
V\authorsc{IJAY} S\authorsc{ARASWAT} (contact author), 
V\authorsc{IVEK} S\authorsc{ARKAR},
A\authorsc{RMANDO} S\authorsc{OLAR-LEZAMA},  
C\authorsc{HRISTOPH von} P\authorsc{RAUN},
P\authorsc{RADEEP} V\authorsc{ARMA},
K\authorsc{RISHNA} V\authorsc{ENKATA},
J\authorsc{AN} V\authorsc{ITEK}, and
T\authorsc{ONG} W\authorsc{EN}.

For extended discussions and support we would like to thank: 
Robert Callahan, Calin
Cascaval, Norman Cohen, Elmootaz Elnozahy, John Field, Bob Fuhrer,
Orren Krieger, Doug Lea, John McCalpin, Paul McKenney, Ram Rajamony,
R.K.~Shyamasundar, Filip Pizlo, V.T.~Rajan, Frank Tip, and Mandana Vaziri.

We thank Jonathan Rhees and William Clinger with help in obtaining the
\LaTeX{} style file and macros used in producing the Scheme report,
after which this document is based. We acknowledge the influence of
the $\mbox{\Java}^{\mbox{TM}}$ Language Specification \cite{jls2}
document, as evidenced by the numerous citations in the text.

This document revises Version {\cf 1.01} of the Report, released in
December 2006. It documents the language corresponding to the first
revision of the first version of the implementation.  This
revision was done by
R\authorsc{AJ} B\authorsc{ARIK}, 
P\authorsc{HILIPPE} C\authorsc{HARLES}, 
C\authorsc{HRISTOPHER} D\authorsc{ONAWA}, 
R\authorsc{OBERT} F\authorsc{UHRER},
N\authorsc{ATHANIEL} N\authorsc{YSTROM},  
V\authorsc{IJAY} S\authorsc{ARASWAT},
V\authorsc{IVEK} S\authorsc{ARKAR},
P\authorsc{RADEEP} V\authorsc{ARMA} and
K\authorsc{RISHNA} V\authorsc{ENKATA}.
(Earlier implementations benefited from significant contributions by
C\authorsc{HRISTIAN} G\authorsc{ROTHOFF} and 
C\authorsc{HRISTOPH von} P\authorsc{RAUN}.)
T\authorsc{ONG} W\authorsc{EN} has written many application programs
in \Xten{}. G\authorsc{UOJING} C\authorsc{ONG} has helped in the
development of many applications.


%\vfill
%\begin{center}
%{\large \bf
%*** DRAFT*** \\
%%August 31, 1989
%\today
%}\end{center}

\vfill
\eject


\chapter*{Contents}
\addvspace{3.5pt}                  % don't shrink this gap
\renewcommand{\tocshrink}{-3.5pt}  % value determined experimentally
{\footnotesize
\tableofcontents
}

\vfill
\eject


 

\clearpage

{\parskip 0pt
\addtolength{\cftsecnumwidth}{0.5em}
\addtolength{\cftsubsecnumwidth}{0.5em}
%\addtolength{\cftsecindent}{0.5em}
\addtolength{\cftsubsecindent}{0.5em}
\tableofcontents
}


\chapter{Introduction}

\subsection*{Background}
Larger computational problems require more powerful computers capable of
performing a larger number of operations per second. The era of
increasing performance by simply increasing clocking frequency now
seems to be behind us. It is becoming increasingly difficult
to mange chip power and heat.  Instead, computer
designers are starting to look at {\em scale out} systems in which the
system's computational capacity is increased by adding additional
nodes of comparable power to existing nodes, and connecting nodes with
a high-speed communication network.

A central problem with scale out systems is a definition of the {\em
memory model}, that is, a model of the interaction between shared
memory and  simultaneous (read, write) operations on that
memory by multiple processors. The traditional ``one operation at a
time, to completion'' model that underlies Lamport's notion of {\em
sequential consistency} (SC) proves too expensive to implement in
hardware, at scale. Various models of {\em relaxed consistency} have
proven too difficult for programmers to work with.  

One response to this problem has been to move to a {\em fragmented
memory model}. Multiple processors are made to interact via a
relatively language-neutral message-passing format such as MPI
\cite{mpi}. This model has enjoyed some success: several
high-performance applications have been written in this
style. Unfortunately, this model leads to a {\em loss of programmer
productivity}: the message-passing format is integrated into the host
language by means of an application-programming interface (API), the
programmer must explicitly represent and manage the interaction
between multiple processes and choreograph their data exchange; large
data-structures (such as distributed arrays, graphs, hash-tables) that
are conceptually unitary must be thought of as fragmented across
different nodes; all processors must generally execute the same code
(in an SPMD fashion) etc.

One response to this problem has been the advent of the {\em
partitioned global address space} (PGAS) model underlying languages
such as UPC, Titanium and Co-Array Fortran \cite{pgas,titanium}. These
languages permit the programmer to think of a single computation
running across multiple processors, sharing a common address
space. All data resides at some processors, which is said to have {\em
affinity} to the data.  Each processor may operate directly on the
data it contains but must use some indirect mechanism to access or
update data at other processors. Some kind of global {\em barriers}
are used to ensure that processors remain roughly in lock-step.

\Xten{} is a modern object-oriented programming language
in the PGAS family. The fundamental goal of \Xten{} is to enable
scalable, high-performance, high-productivity transformational
programming for high-end computers---for traditional numerical
computation workloads (such as weather simulation, molecular dynamics,
particle transport problems etc) as well as commercial server
workloads.

\Xten{} is based on state-of-the-art object-oriented
programming ideas primarily to take advantage of their proven
flexibility and ease-of-use for a wide spectrum of programming
problems. \Xten{} takes advantage of several years of research (e.g.,
in the context of the Java Grande forum,
\cite{moreira00java,kava}) on how to adapt such languages to the context of
high-performance numerical computing. Thus \Xten{} provides support
for user-defined {\em struct types} (such as \xcd"Int", \xcd"Float",
\xcd"Complex" etc), supports a very
flexible form of multi-dimensional arrays (based on ideas in ZPL
\cite{zpl}) and supports IEEE-standard floating point arithmetic.
Some capabilities for supporting operator overloading are also provided.

{}\Xten{} introduces a flexible treatment of concurrency, distribution
and locality, within an integrated type system. \Xten{} extends the
PGAS model with {\em asynchrony} (yielding the {\em APGAS} programming
model). {}\Xten{} introduces {\em places} as an abstraction for a
computational context with a locally synchronous view of shared
memory. An \Xten{} computation runs over a large collection of places.
Each place hosts some data and runs one or more {\em
activities}. Activities are extremely lightweight threads of
execution. An activity may synchronously (and {\em atomically}) use
one or more memory locations in the place in which it resides,
leveraging current symmetric multiprocessor (SMP) technology.  
To access or update memory at other places, it must 
spawn activities asynchronously (either explicitly or implicitly).
\Xten{} provides weaker ordering guarantees for
inter-place data access, enabling applications to scale.  {\em
Immutable} data needs no consistency management and may be freely
copied by the implementation between places.  One or more {\em clocks}
may be used to order activities running in multiple
places.  Arrays may be distributed across multiple
places. Arrays support parallel collective operations. A novel
exception flow model ensures that exceptions thrown by asynchronous
activities can be caught at a suitable parent activity.  The type
system tracks which memory accesses are local. The programmer may
introduce place casts which verify the access is local at run time.
Linking with native code is supported.

\XtenCurrVer builds on v1.7 to support the following features: {\em
  structs} (i.e., ``header-less'', inlinable objects), type rules for
preventing escape of \xcd{this} from a constructor, 
the introduction of a global object model, permitting user-specified
(immutable) fields to be replicated with the object reference.
\xcd{value} classes are no longer supported; their functionality is
accomplished by using structs or global fields and methods.


Several representative idioms for concurrency and communication have
already found pleasant expression in \Xten. We intend to develop
several full-scale applications to get better experience with the
language, and revisit the design in the light of this experience.


\chapter{Overview of \Xten}

\Xten{} is a statically typed object-oriented language, extending a sequential
core language with \emph{places}, \emph{activities}, \emph{clocks},
(distributed, multi-dimensional) \emph{arrays} and \emph{struct} types. All
these changes are motivated by the desire to use the new language for
high-end, high-performance, high-productivity computing.

\section{Object-oriented features}

The sequential core of \Xten{} is a {\em container-based} object-oriented language
similar to \java{} and C++, and more recent languages such as Scala.  
Programmers write \Xten{} code by defining containers for data and behavior
called 
\emph{classes}
(\Sref{XtenClasses}) and
\emph{structs}
(\Sref{XtenStructs}), 
often abstracted as 
\emph{interfaces}
(\Sref{XtenInterfaces}).
X10 provides inheritance and subtyping in fairly traditional ways. 

\begin{ex}

\xcd`Normed` describes entities with a \xcd`norm()` method. \xcd`Normed` is
intended to be used for entities with a position in some coordinate system,
and \xcd`norm()` gives the distance between the entity and the origin. A
\xcd`Slider` is an object which can be moved around on a line; a
\xcd`PlanePoint` is a fixed position in a plane. Both \xcd`Slider`s and
\xcd`PlanePoint`s have a sensible \xcd`norm()` method, and implement
\xcd`Normed`.

%~~gen ^^^ Overview10
% package Overview;
%~~vis
\begin{xten}
interface Normed {
  def norm():Double;
}
class Slider implements Normed {
  var x : Double = 0;
  public def norm() = Math.abs(x);
  public def move(dx:Double) { x += dx; }
}
struct PlanePoint implements Normed {
  val x : Double; val y:Double;
  public def this(x:Double, y:Double) {
    this.x = x; this.y = y;
  }
  public def norm() = Math.sqrt(x*x+y*y);
}
\end{xten}
%~~siv
%
%~~neg
\end{ex}

\paragraph{Interfaces}

An \Xten{} interface specifies a collection of abstract methods; \xcd`Normed`
specifies just \xcd`norm()`. Classes and
structs can be specified to {\em implement} interfaces, as \xcd`Slider` and
\xcd`PlanePoint` implement \xcd`Normed`, and, when they do so, must provide
all the methods that the interface demands.

Interfaces are
purely abstract. Every value of type \xcd`Normed` must be an instance of some
class like \xcd`Slider` or some struct like \xcd`PlanePoint` which implements
\xcd`Normed`; no value can be \xcd`Normed` and nothing else. 


\paragraph{Classes and Structs}



There are two kinds of containers: \emph{classes}
(\Sref{ReferenceClasses}) and \emph{structs} (\Sref{Structs}). Containers hold
data in {\em fields}, and give concrete implementations of 
methods, as \xcd`Slider` and \xcd`PlainPoint` above.

Classes are organized in a single-inheritance tree: a class may have only a
single parent class, though it may implement many interfaces and have many
subclasses. Classes may have mutable fields, as \xcd`Slider` does.

In contrast, structs are headerless values, lacking the internal organs
which give objects their intricate behavior.  This makes them less powerful
than objects (\eg, structs cannot inherit methods, though objects can), but also
cheaper (\eg, they can be inlined, and they require less space than objects).  
Structs are immutable, though their fields may be immutably set to objects
which are themselves mutable.  They behave like objects in all ways consistent
with these limitations; \eg, while they cannot {\em inherit} methods, they can
have them -- as \xcd`PlanePoint` does.

\Xten{} has no primitive classes per se. However, the standard library
\xcd"x10.lang" supplies structs and objects \xcd"Boolean", \xcd"Byte",
\xcd"Short", \xcd"Char", \xcd"Int", \xcd"Long", \xcd"Float", \xcd"Double",
\xcd"Complex" and \xcd"String". The user may defined additional arithmetic
structs using the facilities of the language.



\paragraph{Functions.}

X10 provides functions (\Sref{Closures}) to allow code to be used
as values.  Functions are first-class data: they can be stored in lists,
passed between activities, and so on.  \xcd`square`, below, is a function
which squares an \xcd`Long`.  \xcd`of4` takes an \xcd`Long`-to-\xcd`Long`
function and applies it to the number \xcd`4`.  So, \xcd`fourSquared` computes
\xcd`of4(square)`, which is \xcd`square(4)`, which is 16, in a fairly
complicated way.
%~~gen ^^^ Overview20
% package Overview.of.Functions.one;
% class Whatever{
% def chkplz() {
%~~vis
\begin{xten}
  val square = (i:Long) => i*i;
  val of4 = (f: (Long)=>Long) => f(4);
  val fourSquared = of4(square);
\end{xten}
%~~siv
%}}
%~~neg



Functions are used extensively in X10
programs.  For example, a common way to construct and initialize an \xcd`Rail[Long]` --
that is, a fixed-length one-dimensional array of numbers, like an \xcd`long[]` in \java{} -- is to
pass two arguments to a factory method: the first argument being the length of
the rail, and the second being a function which computes the initial value of
the \xcd`i`{$^{th}$} element.  The following code constructs a 1-dimensional
rail 
initialized to the squares of 0,1,...,9: \xcd`r(0) == 0`, \xcd`r(5)==25`, etc. 
%~~gen ^^^ Overview30
% package Overview.of.Functions.two;
% class Whatevermore {
%  def plzchk(){
%    val square = (i:Long) => (i*i);
%~~vis
\begin{xten}
  val r : Rail[Long] = new Rail[Long](10, square);
\end{xten}
%~~siv
%}}
%~~neg








\paragraph{Constrained Types}

X10 containers may declare {\em properties}, which are fields bound immutably
at the creation of the container.  The static analysis system understands
properties, and can work with them logically.   


For example, an implementation of matrices \xcd`Mat` might have the numbers of
rows and columns as properties.  A little bit of care in definitions allows
the definition of a \xcd`+` operation that works on matrices of the same
shape, and \xcd`*` that works on matrices with appropriately matching shapes.
%~~gen ^^^ Overview40
%package Overview.Mat2;
%~~vis
\begin{xten}
abstract class Mat(rows:Long, cols:Long) {
 static type Mat(r:Long, c:Long) = Mat{rows==r&&cols==c};
 abstract operator this + (y:Mat(this.rows,this.cols))
                 :Mat(this.rows, this.cols);
 abstract operator this * (y:Mat) {this.cols == y.rows} 
                 :Mat(this.rows, y.cols);
\end{xten}
%~~siv
%  static def makeMat(r:Long,c:Long) : Mat(r,c) = null;
%  static def example(a:Long, b:Long, c:Long) {
%    val axb1 : Mat(a,b) = makeMat(a,b);
%    val axb2 : Mat(a,b) = makeMat(a,b);
%    val bxc  : Mat(b,c) = makeMat(b,c);
%    val axc  : Mat(a,c) = (axb1 +axb2) * bxc;
%  }
%}
%~~neg



The following code typechecks (assuming that \xcd`makeMat(m,n)` is a function
which creates an \xcdmath"m$\times$n" matrix).
However, an attempt to compute \xcd`axb1 + bxc` or
\xcd`bxc * axb1` would result in a compile-time type error:
%~~gen ^^^ Overview50
%package Overview.Mat1;
%//OPTIONS: -STATIC_CHECKS
%abstract class Mat(rows:Long, cols:Long) {
%  static type Mat(r:Long, c:Long) = Mat{rows==r&&cols==c};
%  public def this(r:Long, c:Long) : Mat(r,c) = {property(r,c);}
%  static def makeMat(r:Long,c:Long) : Mat(r,c) = null;
%  abstract  operator this + (y:Mat(this.rows,this.cols)):Mat(this.rows, this.cols);
%  abstract  operator this * (y:Mat) {this.cols == y.rows} : Mat(this.rows, y.cols);
%~~vis
\begin{xten}
  static def example(a:Long, b:Long, c:Long) {
    val axb1 : Mat(a,b) = makeMat(a,b);
    val axb2 : Mat(a,b) = makeMat(a,b);
    val bxc  : Mat(b,c) = makeMat(b,c);
    val axc  : Mat(a,c) = (axb1 +axb2) * bxc;
    //ERROR: val wrong1 = axb1 + bxc;
    //ERROR: val wrong2 = bxc * axb1;
  }

\end{xten}
%~~siv
%}
%~~neg

The ``little bit of care'' shows off many of the features of constrained
types.    
The \xcd`(rows:Long, cols:Long)` in the class definition declares two
properties, \xcd`rows` and \xcd`cols`.\footnote{The class is officially declared
abstract to allow for multiple implementations, like sparse and band matrices,
but in fact is abstract to avoid having to write the actual definitions of
\xcd`+` and \xcd`*`.}  

A constrained type looks like \xcd`Mat{rows==r && cols==c}`: a type
name, followed by a Boolean expression in braces.  
The \xcd`type` declaration on the second line makes
\xcd`Mat(r,c)` be a synonym for \xcd`Mat{rows==r && cols==c}`,
allowing for compact types in many places.

Functions can return constrained types.  
The \xcd`makeMat(r,c)` method returns a \xcd`Mat(r,c)` -- a matrix whose shape
is given by the arguments to the method.    In
particular, constructors can have constrained return types to provide specific
information about the constructed values.

The arguments of methods can have type constraints as well.  The 
\xcd`operator this +` line lets \xcd`A+B` add two matrices.  The type of the
second argument \xcd`y` is constrained to have the same number of rows and
columns as the first argument \xcd`this`. Attempts to add mismatched matrices
will be flagged as type errors at compilation.

At times it is more convenient to put the constraint on the method as a whole,
as seen in the \xcd`operator this *` line. Unlike for \xcd`+`, there is no
need to constrain both dimensions; we simply need to check that the columns of
the left factor match the rows of the right. This constraint is written in
\xcd`{...}` after the argument list.  The shape of the result is computed from
the shapes of the arguments.

And that is all that is necessary for a user-defined class of matrices to have
shape-checking for matrix addition and multiplication.  The \xcd`example`
method compiles under those definitions.








\paragraph{Generic types}

Containers may have type parameters, permitting the definition of
{\em generic types}.  Type parameters may be instantiated by any X10 type.  It
is thus possible to make a list of integers \xcd`List[Long]`, a list of
non-zero integers \xcd`List[Long{self != 0}]`, or a list of people
\xcd`List[Person]`.  In the definition of \xcd`List`, \xcd`T` is a type
parameter; it can be instantiated with any type.
%~~gen ^^^ Overview60
%~~vis
\begin{xten}
class List[T] {
    var head: T;
    var tail: List[T];
    def this(h: T, t: List[T]) { head = h; tail = t; }
    def add(x: T) {
        if (this.tail == null)
            this.tail = new List[T](x, null);
        else
            this.tail.add(x);
    }
}
\end{xten}
%~~siv
%~~neg
The constructor (\xcd"def this") initializes the fields of the new object.
The \xcd"add" method appends an element to the list.
\xcd"List" is a generic type.  When  instances of \xcd"List" are
allocated, the type \param{} \xcd"T" must be bound to a concrete
type.  \xcd"List[Long]" is the type of lists of element type
\xcd"Long", \xcd"List[List[String]]" is the type of lists whose elements are
themselves lists of string, and so on.

%%BARD-HERE

\section{The sequential core of X10}

The sequential aspects of X10 are mostly familiar from C and its progeny.
\Xten{} enjoys the familiar control flow constructs: \xcd"if" statements,
\xcd"while" loops, \xcd"for" loops, \xcd"switch" statements, \xcd`throw` to
raise exceptions and \xcd`try...catch` to handle them, and so on.

X10 has both implicit coercions and explicit conversions, and both can be
defined on user-defined types.  Explicit conversions are written with the
\xcd`as` operation: \xcd`n as Long`.  The types can be constrained: 
%~~exp~~`~~`~~n:Long~~ ^^^ Overview70
\xcd`n as Long{self != 0}` converts \xcd`n` to a non-zero integer, and throws a
runtime exception if its value as an integer is zero.

\section{Places and activities}

The full power of X10 starts to emerge with concurrency.
An \Xten{} program is intended to run on a wide range of computers,
from uniprocessors to large clusters of parallel processors supporting
millions of concurrent operations. To support this scale, \Xten{}
introduces the central concept of \emph{place} (\Sref{XtenPlaces}).
A place can be thought of as a virtual shared-memory multi-processor:
a computational unit with a finite (though perhaps changing) number of
hardware threads and a bounded amount of shared memory, uniformly
accessible by all threads.



An \Xten{} computation acts on \emph{values}(\Sref{XtenObjects}) through
the execution of lightweight threads called
\emph{activities}(\Sref{XtenActivities}). 
An {\em object}
 has a small, statically fixed set of fields, each of
which has a distinct name. A scalar object is located at a single place and
stays at that place throughout its lifetime. An \emph{aggregate} object has
many fields (the number may be known only when the object is created),
uniformly accessed through an index (\eg, an integer) and may be distributed
across many places. The distribution of an aggregate object remains unchanged
throughout the computation, thought different aggregates may be distributed
differently. Objects are garbage-collected when no longer useable; there are
no operations in the language to allow a programmer to explicitly release
memory.

{}\Xten{} has a \emph{unified} or \emph{global address space}. This means that
an activity can reference objects at other places. However, an activity may
synchronously access data items only in the current place, the place in which
it is running. It may atomically update one or more data items, but only in
the current place.   If it becomes necessary to read or modify an object at
some other place \xcd`q`, the {\em place-shifting} operation \xcd`at(q;F)` can
be used, to move part of the activity to \xcd`q`.  \xcd`F` is a specification
of what information will be sent to \xcd`q` for use by that part of the
computation. 
It is easy to compute
across multiple places, but the expensive operations (\eg, those which require
communication) are readily visible in the code. 

\paragraph{Atomic blocks.}

X10 has a control construct \xcd"atomic S" where \xcd"S" is a statement with
certain restrictions. \xcd`S` will be executed atomically, without
interruption by other activities. This is a common primitive used in
concurrent algorithms, though rarely provided in this degree of generality by
concurrent programming languages.

More powerfully -- and more expensively -- X10 allows conditional atomic
blocks, \xcd`when(B)S`, which are executed atomically at some point when
\xcd`B` is true.  Conditional atomic blocks are one of the strongest
primitives used in concurrent algorithms, and one of the least-often
available. 

\paragraph{Asynchronous activities.}

An asynchronous activity is created by a statement \xcd"async S", which starts
up a new activity running \xcd`S`.  It does not wait for the new activity to
finish; there is a separate statement (\xcd`finish`) to do that.


\section{Distributed heap management}

\Xten{} is the language for parallel and distributed computing, which is based on the APGAS (Asynchronous Partitioned Global Address Space) programming model. In (A)PGAS, the address space is partitioned into multiple semi-spaces. The semi-space is called \emph{place} in \Xten{}. In Managed \Xten{} (\Xten{} on \java{} VMs), a place is represented as a single \java{} VM and the semi-space is mapped to the heap of the \java{} VM.

\Xten{} supports garbage collection. Objects in a local heap (local objects) are collected with (local) garbage collection and there is no way to explicitly free them. The reference to local objects is called \emph{local reference}.

In addition, \Xten{} has another type of reference called \emph{remote reference}. Unlike local reference, remote reference can reference objects at both local and remote places.

With remote reference, an activity (something like thread, it runs on a place at a time but it can move itself to different places) can access objects at a remote place (remote objects) when the activity has moved to the remote place. The place where an object is created is the home place of the object and it does not change for the lifetime.

To guarantee an activity can access remote objects at their home place, the objects with remote reference are protected from (local) garbage collection at their home place even if they have no local reference. Objects can be garbage collected only when they have neither local nor remote reference. The garbage collection that takes care of remote reference is called distributed garbage collection and it is supported in Managed \Xten{}.

Distributed garbage collection in Managed \Xten{} \cite{KawachiyaX1012} tracks the lifetime of remote reference with reference counting. When the local garbage collection at a remote place detects the remote reference is no longer needed at the place, the count is decremented. When the count becomes zero, the local garbage collection at the home place is ready to collect the referenced object in the ordinary way.

This mechanism works in most cases, but when there is unbalance in heap allocation rate between places, there is a risk of out of memory error at a frequently allocating place. This is because remote reference from infrequently allocating (i.e. infrequently garbage collected) places could retain remotely referenced objects longer than needed.

To avoid the out of memory error even with unbalanced heap allocation rate, there is a way to explicitly release remote reference.

A single call of \xcd`PlaceLocalHandle.destroy()` (\xcd`PlaceLocalHandle` is an \Xten{} type that bundles multiple remote references to the objects at different places) releases all remote references immediately, thus the local garbage collection at each place becomes ready to collect the referenced object in the ordinary way. It can be called at the point where the all objects referenced by the handle are no longer needed to be accessible with the handle. Local reference to the object at each place won't be affected.



\section{Clocks}
The MPI style of coordinating the activity of multiple processes with
a single barrier is not suitable for the dynamic network of heterogeneous
activities in an \Xten{} computation.  
X10 allows multiple barriers in a form that supports determinate,
deadlock-free parallel computation, via the \xcd`Clock` type.

A single \xcd`Clock` represents a computation that occurs in phases.
At any given time, an activity is {\em registered} with zero or more clocks.
The X10 statement \xcd`next` tells all of an activity's registered clocks that
the activity has finished the current phase, and causes it to wait for the
next phase.  Other operations allow waiting on a single clock, starting
new clocks or new activities registered on an extant clock, and so on. 

%%INTRO-CLOCK%  Activities may use clocks to repeatedly detect quiescence of arbitrary
%%INTRO-CLOCK%  programmer-specified, data-dependent set of activities. Each activity
%%INTRO-CLOCK%  is spawned with a known set of clocks and may dynamically create new
%%INTRO-CLOCK%  clocks. At any given time an activity is \emph{registered} with zero or
%%INTRO-CLOCK%  more clocks. It may register newly created activities with a clock,
%%INTRO-CLOCK%  un-register itself with a clock, suspend on a clock or require that a
%%INTRO-CLOCK%  statement (possibly involving execution of new async activities) be
%%INTRO-CLOCK%  executed to completion before the clock can advance.  At any given
%%INTRO-CLOCK%  step of the execution a clock is in a given phase. It advances to the
%%INTRO-CLOCK%  next phase only when all its registered activities have \emph{quiesced}
%%INTRO-CLOCK%  (by executing a \xcd"next" operation on the clock).
%%INTRO-CLOCK%  When a clock advances, all its activities may now resume execution.
%%INTRO-CLOCK%  

Clocks act as {barriers} for a dynamically varying collection of activities.
They generalize the barriers found in MPI style program in that an activity
may use multiple clocks simultaneously. Yet programs using clocks properly are
guaranteed not to suffer from deadlock.

%%HERE

\section{Arrays, regions and distributions}

X10 provides \xcd`DistArray`s, {\em distributed arrays}, which spread data
across many places. An underlying \xcd`Dist` object provides the {\em
distribution}, telling which elements of the \xcd`DistArray` go in which
place. \xcd`Dist` uses subsidiary \xcd`Region` objects to abstract over the
shape and even the dimensionality of arrays.
Specialized X10 control statements such as \xcd`ateach` provide efficient
parallel iteration over distributed arrays.


\section{Annotations}

\Xten{} supports annotations on classes and interfaces, methods
and constructors,
variables, types, expressions and statements.
These annotations may be processed by compiler plugins.

\section{Translating MPI programs to \Xten{}}

While \Xten{} permits considerably greater flexibility in writing
distributed programs and data structures than MPI, it is instructive
to examine how to translate MPI programs to \Xten.

Each separate MPI process can be translated into an \Xten{}
place. Async activities may be used to read and write variables
located at different processes. A single clock may be used for barrier
synchronization between multiple MPI processes. \Xten{} collective
operations may be used to implement MPI collective operations.
\Xten{} is more general than MPI in (a)~not requiring synchronization
between two processes in order to enable one to read and write the
other's values, (b)~permitting the use of high-level atomic blocks
within a process to obtain mutual exclusion between multiple
activities running in the same node (c)~permitting the use of multiple
clocks to combine the expression of different physics (e.g.,
computations modeling blood coagulation together with computations
involving the flow of blood), (d)~not requiring an SPMD style of
computation.


%\note{Relaxed exception model}
\section{Summary and future work}
\subsection{Design for scalability}
\Xten{} is designed for scalability, by encouraging working with local data,
and limiting the ability of events at one place to delay those at another. For
example, an activity may atomically access only multiple locations in the
current place. Unconditional atomic blocks are dynamically guaranteed to be
non-blocking, and may be implemented using non-blocking techniques that avoid
mutual exclusion bottlenecks. 
%TODO: yoav says: ``no idea what [the following] means''
Data-flow synchronization permits point-to-point
coordination between reader/writer activities, obviating the need for
barrier-based or lock-based synchronization in many cases.

\subsection{Design for productivity}
\Xten{} is designed for productivity.

\paragraph{Safety and correctness.}



Programs written in \Xten{} are guaranteed to be statically
\emph{type safe}, \emph{memory safe} and \emph{pointer safe},
with certain exceptions given in \Sref{sect:LimitationOfStrictTyping}.

Static type safety guarantees that every location contains only values whose
dynamic type agrees with the location's static type. The compiler allows a
choice of how to handle method calls. In strict mode, method calls are
statically checked to be permitted by the static types of operands. In lax
mode, dynamic checks are inserted when calls may or may not be correct,
providing weaker static correctness guarantees but more programming
convenience. 

Memory safety guarantees that an object may only access memory within its
representation, and other objects it has a reference to. \Xten{} does not
permit 
pointer arithmetic, and bound-checks array accesses dynamically if necessary.
\Xten{} uses garbage collection to collect objects no longer referenced by any
activity. \Xten{} guarantees that no object can retain a reference to an
object whose memory has been reclaimed. Further, \Xten{} guarantees that every
location is initialized at run time before it is read, and every value read
from a word of memory has previously been written into that word.

%XXX
%Pointer safety guarantees that a null pointer exception cannot be
%thrown by an operation on a value of a non-nullable type.

%Because places are reflected in the type system, static type safety
%also implies \emph{place safety}. All operations that need to be performed
%locally are, in fact, performed locally.  All data which is declared to be
%stored locally are, in fact, stored locally.

\Xten{} programs that use only the common, specified clock idioms and unconditional atomic
blocks are guaranteed not to deadlock. Unconditional atomic blocks
are non-blocking, hence cannot introduce deadlocks.
Many concurrent programs can be shown to be determinate (hence
race-free) statically.

\paragraph{Integration.}
A key issue for any new programming language is how well it can be
integrated with existing (external) languages, system environments,
libraries and tools.

%TODO: Yoav asks ``you mean interop''?
We believe that \Xten{}, like \java{}, will be able to support a large
number of libraries and tools. An area where we expect future versions
of \Xten{} to improve on \java{} like languages is \emph{native
integration} (\Sref{NativeCode}). Specifically, \Xten{} will 
permit multi-dimensional local arrays to be operated on natively by
native code.

\subsection{Conclusion}
{}\Xten{} is considerably higher-level than thread-based languages in
that it supports dynamically spawning lightweight activities, the
use of atomic operations for mutual exclusion, and the use of clocks
for repeated quiescence detection.

Yet it is much more concrete than languages like HPF in that it forces
the programmer to explicitly deal with distribution of data
objects. In this the language reflects the designers' belief that
issues of locality and distribution cannot be hidden from the
programmer of high-performance code in high-end computing.  A
performance model that distinguishes between computation and
communication must be made explicit and transparent.\footnote{In this
\Xten{} is similar to more modern languages such as ZPL \cite{zpl}.}
At the same time we believe that the place-based type system and
support for generic programming will allow the \Xten{} programmer to
be highly productive; many of the tedious details of
distribution-specific code can be handled in a generic fashion.

\chapter{Lexical structure}

In general, \Xten{} follows \java{} rules \cite[Chapter 3]{jls2} for
lexical structure.

Lexically a program consists of a stream of white space, comments,
identifiers, keywords, literals, separators and operators.

\paragraph{Whitespace}
% Whitespace \index{whitespace} follows \java{} rules \cite[Chapter 3.6]{jls2}.
ASCII space, horizontal tab (HT), form feed (FF) and line
terminators constitute white space.

\paragraph{Comments}
% Comments \index{comments} follows \java{} rules
% \cite[Chapter 3.7]{jls2}. 
All text included within the ASCII characters ``\xcd"/*"'' and
``\xcd"*/"'' is
considered a comment and ignored; nested comments are not
allowed.  All text from the ASCII characters
``\xcd"//"'' to the end of line is considered a comment and is ignored.

\paragraph{Identifiers}

Identifiers consist of a single letter followed by zero or more
letters or digits.
Letters are defined as the characters for which the \java{}
method \xcd"Character.isJavaIdentifierStart" returns true.
Digits are defined as the ASCII characters \xcd"0" through \xcd"9".

\paragraph{Keywords}
\Xten{} reserves the following keywords:
\begin{xten}
abstract       do             in             public         
as             else           instanceof     return         
assert         extends        interface      self           
async          false          native         static         
ateach         final          new            struct         
break          finally        null           super          
case           finish         offers         switch         
catch          for            operator       this           
class          goto           package        throw          
continue       if             private        transient      
def            implements     property       true           
default        import         protected      try            
\end{xten}
Note that the primitive types are not considered keywords.

\paragraph{Literals}\label{Literals}\index{literals}

Briefly, \XtenCurrVer{} uses fairly standard syntax for its literals:
integers, unsigned integers, floating point numbers, booleans, 
characters, strings, and \xcd"null".  The most exotic points are (1) unsigned
numbers are marked by a \xcd`u` and cannot have a sign; (2) \xcd`true` and
\xcd`false` are the literals for the booleans; and (3) floating point numbers
are \xcd`Double` unless marked with an \xcd`f` for \xcd`Float`. 

Less briefly, we use the following abbreviations: 
\begin{displaymath}
\begin{array}{rcll}
d &=& \mbox{one or more decimal digits}\\
d_8 &=& \mbox{one or more octal digits}\\
d_{16} &=& \mbox{one or more hexadecimal digits, using \xcd`a`-\xcd`f`
for 10-15}\\
i &=& d 
        \mathbin{|} {\tt 0} d_8 
        \mathbin{|} {\tt 0x} d_{16}
        \mathbin{|} {\tt 0X} d_{16}
\\
s &=& \mbox{optional \xcd`+` or \xcd`-`}\\
b &=& d 
          \mathbin{|} d {\tt .}
          \mathbin{|} d {\tt .} d
          \mathbin{|}  {\tt .} d \\
x &=& ({\tt e } \mathbin{|} {\tt E})
         s
         d \\
f &=& b x
\end{array}
\end{displaymath}

\begin{itemize}

\item \xcd`true` and \xcd`false` are the \xcd`Boolean` literals.

\item \xcd`null` is a literal for the null value.  It has type
      \xcd`Any{self==null}`. 

\item \xcd`Int` literals have the form {$si$}; \eg, \xcd`123`,
      \xcd`-321` are decimal \xcd`Int`s, \xcd`0123` and \xcd`-0321` are octal
      \xcd`Int`s, and \xcd`0x123`, \xcd`-0X321`,  \xcd`0xBED`, and \xcd`0XEBEC` are
      hexadecimal \xcd`Int`s.  

\item \xcd`Long` literals have the form {$si{\tt l}$} or
      {$si{\tt L}$}. \Eg, \xcd`1234567890L`  and \xcd`0xBAGEL` are \xcd`Long` literals. 

\item \xcd`UInt` literals have the form {$i{\tt u}$} or {$i {\tt U}$}.
      \Eg, \xcd`123u`, \xcd`0123u`, and \xcd`0xBEAU` are \xcd`UInt` literals.

\item \xcd`ULong` literals have the form {$i {\tt ul}$} or {$i {\tt
      lu}$}, or capital versions of those.  For example, 
      \xcd`123ul`, \xcd`0124567012ul`,  \xcd`0xFLU`, \xcd`OXba1eful`, and \xcd`0xDecafC0ffeefUL` are \xcd`ULong`
      literals. 

\item \xcd`Float` literals have the form {$s f {\tt f}$} or  {$s
      f {\tt F}$}.  Note that the floating-point marker letter \xcd`f` is
      required: unmarked floating-point-looking literals are \xcd`Double`. 
      \Eg, \xcd`1f`, \xcd`6.023E+32f`, \xcd`6.626068E-34F` are \xcd`Float`
      literals. 

\item \xcd`Double` literals have the form {$s f$}\footnote{Except that
      literals like \xcd`1` 
      which match both {$i$} and {$f$} are counted as
      integers, not \xcd`Double`; \xcd`Double`s require a decimal
      point, an exponent, or the \xcd`d` marker.
      }, {$s f {\tt
      D}$}, and {$s f {\tt d}$}.  
      \Eg, \xcd`0.0`, \xcd`0e100`, \xcd`229792458d`, and \xcd`314159265e-8`
      are \xcd`Double` literals.

\item \xcd`Char` literals have one of the following forms: 
      \begin{itemize}
      \item \xcd`'`{\it c}\xcd`'` where {\em c} is any printing ASCII
            character other than 
            \xcd`\` or \xcd`'`, representing the character {\em c} itself; 
            \eg, \xcd`'!'`;
      \item \xcd`'\b'`, representing backspace;
      \item \xcd`'\t'`, representing tab;
      \item \xcd`'\n'`, representing newline;
      \item \xcd`'\f'`, representing form feed;
      \item \xcd`'\r'`, representing return;
      \item \xcd`'\''`, representing single-quote;
      \item \xcd`'\"'`, representing double-quote;
      \item \xcd`'\\'`, representing backslash;
      \item \xcd`'\`{\em dd}\xcd`'`, where {\em dd} is one or more octal
            digits, representing the one-byte character numbered {\em dd}; it
            is an error if {\em dd}{$>255$}.      
      \end{itemize}

\item \xcd`String` literals consist of a double-quote \xcd`"`, followed by
      zero or more of the contents of a \xcd`Char` literal, followed by
      another double quote.  \Eg, \xcd`"hi!"`, \xcd`""`.

\item There are no literals of type \xcd`Byte`, \xcd`UByte`, \xcd`Short`, or
      \xcd`UShort`.  

\end{itemize}



\paragraph{Separators}
\Xten{} has the following separators and delimiters:
\begin{xten}
( )  { }  [ ]  ;  ,  .
\end{xten}

\paragraph{Operators}
\Xten{} has the following operators:
\begin{xten}
==  !=  <   >   <=  >=
&&  ||  &   |   ^
<<  >>  >>>
+   -   *   /   %
++  --  !   ~
&=  |=  ^=
<<= >>= >>>=
+=  -=  *=  /=  %=
=   ?   :   =>  ->
<:  :>  @   ..
\end{xten}





\chapter{Types}
\label{XtenTypes}\index{types}

{}\Xten{} is a {\em strongly typed} object-oriented language: every
variable and expression has a type that is known at compile-time.
Types limit the values that variables can hold and specify the places
at which these values can lie.


{}\Xten{} supports three kinds of runtime entities, {\em objects},
{\em structs}, and {\em functions}. Objects are instances of {\em
  classes} (\Sref{ReferenceClasses}). They may contain mutable fields
and stay resident in the place in which they were
created. 
Objects are said to be {\em boxed} in that variables of a
class type are implemented through a single memory location that
contains a reference to the memory containing the declared state of
the object (and other meta-information such as the list of methods of
the object). Thus objects are represented through an extra level of
indirection. A consequence of this flexibility is that every class
type contains the value \Xcd{null} corresponding to the invalid
reference. \Xcd{null} is often useful as a default value. Further, two
objects may be compared for identity (\Xcd{==}) in constant time by
simply containing references to the memory used to represent the
objects.

Structs are instances of {\em struct types} (\Sref{StructClasses}). They are a
restricted variant of classes, lacking meta-information; this makes them less
flexible, but in many cases more efficient. When it is semantically
meaningful, converting a class into a struct or vice-versa is quite easy.
Structs are immutable and may be freely copied from place to place. Further,
they may be allocated inline, using only as much memory as necessary to hold
and align the fields of the struct.

Functions, called closures or lambda-expressions in other languages, are
instances of {\em function types|} {\Sref{Functions}). Functions can refer to
%~~exp~~`~~`~~y:Int ~~
variables from the surrounding environment; \eg, \xcd`(x:Int)=>x*y` is a unary
integer function which multiplies its argument by the variable \xcd`y` from
the surrounding block.  
Functions may be freely copied from place to place and may be repeatedly
applied to a set of arguments.

These runtime entities are classified by {\em types}. Types are used in
variable declarations, explicit coercions and conversions, object creation,
array creation, class literals, static state and method accessors, and
\xcd"instanceof" expressions.

The basic relationship between values and types is {\em instantiation}. For
example, \xcd`1` is an instance of type of integers, \xcd`Int`. It is also an
instance of type of all entities \xcd`Any`, and of type of nonzero integers
\xcd`Int{self != 0}`, and many others.

The basic relationship between types is {\em subtyping}: \xcd`T <: U` holds if
every instance of \xcd`T` is also an instance of \xcd`U`. Two important kinds
of subtyping are {\em subclassing} and {\em strengthening}.  Subclassing is a
familiar notion from object-oriented programming.  In a class
hierarchy with classes \xcd`Animal` and \xcd`Cat` arranged in the usual way,
every \xcd`Cat` is an \xcd`Animal`, so \xcd`Cat <: Animal` by subclassing.  
Strengthening is an equally familiar notion from logic.   The instances of
\xcd`Int{self != 0}` are all elements of \xcd`Int{true}` as well, because
\xcd`self != 0` logically implies \xcd`true`; so 
\xcd`Int{self != 0} <: Int{true} == Int` by strengthening.  X10 uses both
notions of subtyping.




\subsection*{The Grammar of Types}

Types are described by the following grammar: 
\bard{Is this still correct?}
\begin{grammar}
Type \: FunctionType \\
    \| ConstrainedType  \\

FunctionType \: TypeParameters\opt \xcd"(" Formals\opt \xcd")"
Constraint\opt Throws\opt \xcd"=>" Type \\
TypeParameters \: \xcd"[" TypeParameter ( \xcd"," TypeParameter )\star \xcd"]" \\
TypeParameter \: Identifier \\
Throws \: \xcd"throws" TypeName ( \xcd"," TypeName )\star \\

ConstrainedType \: Annotation\star BaseType Constraint\opt
PlaceConstraint\opt \\

BaseType \: ClassBaseType \\
     \| InterfaceBaseType \\
     \| PathType \\
     \| \xcd"(" Type \xcd")" \\

ClassType \: Annotation\star ClassBaseType Constraint\opt
PlaceConstraint\opt \\
InterfaceType \: Annotation\star InterfaceBaseType Constraint\opt
PlaceConstraint\opt \\

PathType \: Expression \xcd"." Identifier \\

Annotation \: \xcd"@" InterfaceBaseType Constraint\opt \\

ClassOrInterfaceType \: ClassType \\ \| InterfaceType \\
ClassBaseType \: TypeName \\
InterfaceBaseType \: TypeName \\
\end{grammar}

% \section{Type definitions and type constructors}
% 
% Types in \Xten{} are specified through declarations and through
% type constructors:

% \paragraph{Class types.}

\section{\xcd`Any`}

It is quite convenient to have a type which all values are instances of; that
is, a supertype of all types.\footnote{Java, for one, suffers a number of
  inconveniences because some built-in types like \xcd`int` and \xcd`char`
  aren't subtypes of anything else.}  X10's universal supertype is the
  interface \xcd`Any`. 


\begin{xten}
package x10.lang;
public interface Any {
  property def home():Place;
  property def at(p:Object):Boolean;
  property def at(p:Place):Boolean;
  global safe def toString():String;
  global safe def typeName():String;
  global safe def equals(Any):Boolean;
  global safe def hashCode():Int;
}
\end{xten}

\xcd`Any` provides a handful of essential methods that make sense and are
useful for everything.\footnote{The behavioral annotation \xcd`property` is
  explained in \Sref{properties}; \xcd`safe` in \Sref{SafeAnnotation}, and
  \xcd`global` in \Sref{GlobalAnnotation}.} \xcd`a.toString()` produces a
string representation of \xcd`a`, and \xcd`a.typeName()` the string
representation of its type; both are useful for debugging.  \xcd`aequals(b)`
is the programmer-overridable equality test, and \xcd`a.hashCode()` an integer
useful for hashing.  \xcd`at()` and \xcd`home()` are used in multi-place
computing. 



\section{Classes and interfaces}
\label{ReferenceTypes}

\subsection{Class types}

\index{types!class types}
\index{class}
\index{class declaration}
\index{declaration!class declaration}
\index{declaration!reference class declaration}

A {\em class declaration} (\Sref{XtenClasses}) introduces a {\em class type}
containing all instances of the class.  The \xcd`Position` class below
could describe the position of a slider control, for example.

%~~gen
% package Types.By.Cripes.Classes;
%~~vis
\begin{xten}
class Position {
  private var x : Int = 0;
  public def move(dx:Int) { x += dx; }
  public def pos() : Int = x;
}
\end{xten}
%~~siv
%
%~~neg

Class instances, also called objects, are created via constructor calls. Class
instances have fields and methods, type members, and value properties bound at
construction time. In addition, classes have static members: constant fields,
type definitions, and member classes and member interfaces.

A class with type parameters is {\em generic}. A class type is instantiatable
only if all of its parameters are instantiated on concrete types.  The
\xcd`Cell[T]` class provides a container capable of holding a value of type
\xcd`T`, or being empty.

%~~gen
% package Types.For.Gripes.Of.Wesley.Snipes;
%~~vis
\begin{xten}
class Cell[T] {
  var empty : Boolean = true;
  var contents : T;
  public def putIn(t:T) { 
    contents = t; empty = false; 
  }
  public def emptyOut() { empty = true; }
  public def isEmpty() = empty;
  public def getOut():T throws Exception {
     if (empty) throw new Exception("Empty!");
     return contents ;
  }
}
\end{xten}
%~~siv
%
%~~neg


\Xten{} does not permit mutable static state. A fundamental principle of the
X10 model of computation is that all mutable state be local to some place
(\Sref{XtenPlaces}), and, as static variables are globally available, they
cannot be mutable. When mutable global state is necessary, programmers should
use singleton classes, putting the state in an object and using place-shifting
commands (\Sref{AtStatement}) and atomicity (\Sref{AtomicBlocks}) as necessary
to mutate it safely.

\index{\Xcd{Object}}
\index{\Xcd{x10.lang.Object}}

Classes are structured in a single-inheritance hierarchy. All classes extend
the class \xcd"x10.lang.Object", directly or indirectly. Each class other than
\xcd`Object` extends a single parent class.  \xcd`Object` provides no behavior
of its own, beyond that required by \xcd`Any`.

\index{class!reference class}
\index{reference class type}
\index{\Xcd{Object}}
\index{\Xcd{x10.lang.Object}}


\index{null}
\bard{We've got to say this better.}
Variables of class type may contain the value \xcd"null". 

\subsection{Interface types}
\label{InterfaceTypes}

\index{types!interface types}
\index{interface}
\index{interface declaration}
\index{declaration!interface declaration}

An {\em interface declaration} (\Sref{XtenInterfaces}) defines an {\em
interface type}, specifying a set of methods, type members, and
properties which must be provided by any class declared to implement the
interface. 


Interfaces can also have static members: constant fields, type definitions,
and member classes and interfaces.  However, interfaces cannot specify that
implementing classes must provide static members.

An interface may extend multiple interfaces.  
%~~gen
%package Types.For.Snipes.Interfaces;
%~~vis
\begin{xten}
interface Named {
  def name():String;
}
interface Mobile {
  def move(howFar:Int):Void;
}
interface Person extends Named, Mobile {}
interface NamedPoint extends Named, Mobile{} 
\end{xten}
%~~siv
%
%~~neg


Classes may be declared to implement multiple interfaces.
Semantically, the interface type is the set of all objects that are
instances of classes that implement the interface. A class implements
an interface if it is declared to and if it implements all the methods
and properties defined in the interface.  For example, \xcd`KimThePoint`
implements \xcd`Person`, and hence \xcd`Named` and \xcd`Mobile`.  It would be
a static error if \xcd`KimThePoint` had no \xcd`name` method.

%~~gen
%interface Named {
%   def name():String;
% }
% interface Mobile {
%   def move(howFar:Int):Void;
% }
% interface Person extends Named, Mobile {}
% interface NamedPoint extends Named, Mobile{} 
%~~vis
\begin{xten}
class KimThePoint implements Person {
   var pos : Int = 0;
   public def name() = "Kim (" + pos + ")";
   public def move(dPos:Int) { pos += dPos; }
}
\end{xten}
%~~siv
%
%~~neg


\subsection{Properties}
\index{properties}
\label{properties}

Classes, interfaces, and structs may have {\em properties}, public \xcd`val` instance
fields bound on object creation. For example, the following code declares a
class named \xcd"Coords" with properties \xcd"x" and \xcd"y" and a \xcd"move"
method. The properties are bound using the \xcd"property" statement in the
constructor.

%~~gen
%package not.x10.lang;
%~~vis
\begin{xten}
class Coords(x: Int, y: Int) {
  def this(x: Int, y: Int) : Int{this.x==x, this.y==y} 
    = { property(x, y); }
  def move(dx: Int, dy: Int) = new Coords(x+dx, y+dy);
}
\end{xten}
%~~siv
%~~neg

Properties, unlike other public \xcd`val` fields, can be used  
at compile time in {\em constraints}. This allows us
to specify subtypes based on properties, by appending a boolean expression to
the type. For example, the type \xcd"Coords{x==0}" is the set of all points
whose \xcd"x" property is \xcd"0".  Details of this substantial topic are
found in \Sref{ConstrainedTypes}.



\section{Type parameters and Generic Types}
\label{TypeParameters}

\index{types!type parameters}
\index{methods!parametrized methods}
\index{constructors!parametrized constructors}
\index{closures!parametrized closures}
\label{Generics}
\index{types!generic types}

A class, interface, method, closure, or type definition  may have type
parameters.  Type parameters can be used as types, and will be bound to types
on instantiation.  For example, a generic stack class may be defined as 
\xcd`Stack[T]{...}`.  Stacks can hold values of any type; \eg, 
%~~type~~`~~`~~ ~~class Stack[T]{}
\xcd`Stack[Int]` is a stack of integers, and 
%~~type~~`~~`~~ ~~class Stack[T]{}
\xcd`Stack[Point{self!=null}]`is a stack of non-null \xcd`Point`s.
Generics {\em must} be instantiated when they are used: \xcd`Stack`, by
itself, is not a valid type.
Type parameters may be constrained by a guard on the declaration
(\Sref{ClassGuard}, \Sref{TypeDefGuard},
\Sref{MethodGuard},\Sref{ClosureGuard}).

\index{types!concrete types}
\index{concrete type}
A {\em generic type} is a class, struct,  interface, or type declared with one or
more type parameters.  When instantiated with concrete (\viz, non-generic)
types for its parameters, a generic type becomes a concrete type and can be
used like any other type. For example,
\xcd`Stack` is a generic type, 
%~~type~~`~~`~~ ~~class Stack[T]{}
\xcd`Stack[Int]` is a concrete type, and can be used as one: 
%~~stmt~~`~~`~~ ~~class Stack[T]{}
\xcd`var stack : Stack[Int];`


A \xcd`Cell[T]` is a generic object, capable of holding a value of type
\xcd`T`.  For example, a \xcd`Cell[Int]` can hold an \xcd`Int`, and a
\xcd`Cell[Cell[Int]{self!=0}]` can hold a \xcd`Cell[Int]` which in turn can
only hold non-zero numbers.  \xcd`Cell`s are actually useful in situations
where values must be bound immutably for one reason, but need to be mutable.
%~~gen
% package ch4;
%~~vis
\begin{xten}
class Cell[T] {
    var x: T;
    def this(x: T) { this.x = x; }
    def get(): T = x;
    def set(x: T) = { this.x = x; }
}
\end{xten}
%~~siv
%~~neg


\xcd"Cell[Int]" is the type of \xcd`Int`-holding cells.  
The \xcd"get" method on a \xcd`Cell[Int]` returns an \xcd"Int"; the
\xcd"set" method takes an \xcd"Int" as argument.  Note that
\xcd"Cell" alone is not a legal type because the parameter is
not bound.

\subsection{Variance of Type Parameters}
\index{covariant}
\index{contravariant}
\index{invariant}
\index{type parameter!covariant}
\index{type parameter!contravariant}
\index{type parameter!invariant}

Consider classes \xcd`Person :> Child`.  Every child is a person, but there
are people who are not children.  What is the relationship between
\xcd`Cell[Person]` and \xcd`Cell[Child]`?  

\subsubsection{Why Variance Is Necessary}

In this case, \xcd`Cell[Person]` and \xcd`Cell[Child]` should be unrelated.  
If we had \xcd`Cell[Person] :> Cell[Child]`, the following code would let us
assign a \xcd`old` (a \xcd`Person` but not a \xcd`Child`) to a
variable \xcd`young` of type \xcd`Child`, thereby breaking the type system: 
\begin{xten}
// INCORRECTLY assuming Cell[Person] :> Cell[Child]
val cc : Cell[Child] = new Cell[Child]();
val cp : Cell[Person] = cc; // legal upcast
cp.set(old);       // legal since old : Person
val young : Child = cc.get(); 
\end{xten}

Similarly, if \xcd`Cell[Person] <: Cell[Child]`: 
\begin{xten}
// INCORRECTLY assuming Cell[Person] <: Cell[Child]
val cp : Cell[Person] = new Cell[Person];
val cc : Cell[Child] = cp; // legal upcast
val cp.set(old); 
val young : Child = cc.get();
\end{xten}

So, there cannot be a subtyping relationship in either direction between the
two. And indeed, neither of these programs passes the X10 typechecker.


\subsubsection{Legitimate Variance}

The \xcd`Cell[Person]`-vs-\xcd`Cell[Child]` problems occur because it is
possible to both store and retrieve values from the same object. However,
entities with only one of the two capabilities {\em can} sensibly have some
subtyping relations. Furthermore, both sorts of entity are useful. An entity
which can store values but not retrieve them can nonetheless summarize them.
An object which can retrieve values but not store values can be constructed
with an initial value, providing a read-only cell.

So, X10 provides {\em variance} to support these options.  Type parameters
may be defined in one of three forms.  
\begin{enumerate}
\item {\em invariant}: Given a definition \xcd`class C[T]{...}`, \xcd`C[Person]` and
      \xcd`C[Child]` are unrelated classes; neither is a subclass of the
      other.
\item {\em covariant}: Given a definition \xcd`class C[+T]{...}` (the \xcd`+` indicates
      covariance), \xcd`C[Person] :> C[Child]`.  This is appropriate when
      \xcd`C` allows retrieving values but not setting them.
\item {\em contravariant}: Given a definition \xcd`class C[-T]{...}` (the \xcd`-` indicates
      contravariance), \xcd`C[Person] <: C[Child]`.  This is appropriate when
      \xcd`C` allows storing values but not retrieving them.
\end{enumerate}


The \xcd"T" parameter of \xcd"Cell" above is
invariant.  

A typical example of covariance is \xcd`Get`.  As the \xcd`example()` method
shows, a \xcd`Get[T]` must be constructed with its value, and will return that
value whenever desired.
%~~gen
% package ch4;
%~~vis
\begin{xten}
class Get[+T] {
  var x: T;
  def this(x: T) { this.x = x; }
  def get(): T = x;
  static def example() {
     val g : Get[Int]! = new Get[Int](31);
     val n : Int = g.get();
     x10.io.Console.OUT.print("It's " + n);
     x10.io.Console.OUT.print("It's still " + g.get());
  }
}
\end{xten}
%~~siv
%~~neg


A typical example of contravariance is \xcd`Set`.  As the \xcd`example()`
method shows,  a variety of objects\footnote{Objects but no structs.  If we
had wanted structs too, we could have used a \xcd`Cell[Any]`.}  can be put into a
\xcd`Set[Object]`.  While the object itself cannot be retrieved, some summary
information about it -- in this case, its \xcd`typeName` -- can be.  
%~~gen
% package ch4;
%~~vis
\begin{xten}
class Set[-T] {
  var x: T;
  def this(x: T) { this.x = x; }
  def set(x: T) = { this.x = x; } 
  def summary(): String = this.x.typeName();
  static def example() {
    val s : Set[Object]! = new Set[Object](new Throwable());
    s.summary(); // == "x10.lang.Throwable"
    s.set("A String");
    s.summary(); // == "x10.lang.String";
  }    
}
\end{xten}
%~~siv
%
%~~neg


Given types \xcd"S" and \xcd"T": 
\begin{itemize}
\item
If the parameter of \xcd"Get" is covariant, then
\xcd"Get[S]" is a subtype of \xcd"Get[T]" if
\xcd"S" is a {\em subtype} of \xcd"T".

\item
If the parameter of \xcd"Set" is contravariant, then
\xcd"Set[S]" is a subtype of \xcd"Set[T]" if
\xcd"S" is a {\em supertype} of \xcd"T".

\item
If the parameter of \xcd"Cell" is invariant, then
\xcd"Cell[S]" is a subtype of \xcd"Cell[T]" if
\xcd"S" is a {\em equal} to \xcd"T".
\end{itemize}


In order to make types marked as covariant and contravariant semantically
sound, X10 performs extra checks.  
A covariant type parameter is permitted to appear only in covariant type positions,
and a contravariant type parameter in contravariant positions. 
\begin{itemize}
\item The return type of a method is a covariant position.
\item The argument types of a method are contravariant positions.
\item Whether a type argument position of a generic class, interface or struct type \Xcd{C}
is covariant or contravariant is determined by the \Xcd{+} or \Xcd{-} annotation
at that position in the declaration of \Xcd{C}.
\end{itemize}

There are similar restrictions on use of covariant and contravariant values. 
\bard{Get them!  What are they?}


\section{Type definitions}
\label{TypeDefs}

\index{types!type definitions}
\index{declarations!type definitions}

\section{Type definitions}

With value arguments, type arguments, and constraints, the
syntax for \Xten{} types can often be verbose;
\Xten{} therefore provides {\em type definitions}
to allow aliases to be defined for types.
Type definitions have the following syntax:

\begin{grammar}
TypeDefinition \: 
                \xcd"type"~Identifier
                           ( \xcd"[" TypeParameters \xcd"]" )\opt \\
                        && ( \xcd"(" Formals \xcd")" )\opt
                            Constraint\opt \xcd"=" Type \\
\end{grammar}

\noindent
A type definition can be thought of as a type-valued function,
mapping type parameters and value parameters to a concrete type.
%
The following examples are legal type definitions:
\begin{xten}
type StringSet = Set[String];
type MapToList[K,V] = Map[K,List[V]];
type Nat = Int{self>=0};
type Int(x: Int) = Int{self==x};
type Int(lo: Int, hi: Int) = Int{lo <= self, self <= hi};
\end{xten}

As the two definitions of \xcd"Int" demonstrate, type definitions may 
be overloaded: a type definition with a different number of type
parameters or with different types of value
parameters---according to the method overloading rules
(\Sref{MethodOverload}) define distinct types.

Type definitions may appear as class members or in the body of a
method, constructor, or initializer.  Type definitions that are
members of a class are \xcd"static"; type properties can be used
for non-static type definitions.

Type definitions are applicative, not generative; that is, they
are define aliases for types and do not introduce new types.
Thus, the following code is legal:
\begin{xten}
type A = Int;
type B = String;
type C = String;
a: A = 3;
b: B = new C("Hi");
c: C = b + ", Mom!";
\end{xten}
A type defined by a type definition
has the same constructors as its defining type; however, a
constructor may not be invoked using a given type definition
name if the constructor return type is not a subtype of the
defined type.

All type definitions are members of their enclosing package or
class.  A compilation unit may have one or more type definitions
or class or interface declarations with the same name, as long
as the types are unique by overloading.




\section{Constrained types}
\label{ConstrainedTypes}
\label{DepType:DepType}
\label{DepTypes}

\index{dependent types}
\index{constrained types}
\index{generic types}
\index{types!constrained types}
\index{types!dependent types}
\index{types!generic types}


Basic types, like \xcd`Int` and \xcd`List[String]`, provide useful
descriptions of data.  Indeed, most typed programming languages get by with no
more specific descriptions.

However, there are a lot of things that one frequently wants to say about
data.  One might want to know that a \xcd`String` variable is not \xcd`null`,
or that a matrix is square, or that one matrix has the same number of columns
that another has rows (so they can be multiplied).  In the multicore setting,
one might wish to know that two values are located at the same processor.

In most languages, there is simply no way to say these things statically.
Programmers must made do with comments, \xcd`assert` statements, and dynamic
tests.  X10 can do better, with {\em constraints} on types (and methods and
other things).

A constraint is a boolean expression \xcd`e` attached to a basic type \xcd`T`,
written \xcd`T{e}`.  (Only a limited selection of boolean expressions is
available.)  The values of type \xcd`T{e}` are the values of \xcd`T` for which
\xcd`e` is true.  For example: 

\begin{itemize}
%~~type~~`~~`~~ ~~
\item \xcd`String{self != null}` is the type of non-null strings.  \xcd`self`
      is a special variable available only in constraints; it refers to the
      datum being constrained.   
\item If \xcd`Matrix` has properties \xcd`rows` and \xcd`cols`, 
%~~type~~`~~`~~ ~~class Matrix(rows:Int,cols:Int){}
      \xcd`Matrix{rows == cols}` is the type of square matrices.
\item One way to say that \xcd`a` has the same number of columns that \xcd`b`
      has rows (so that \xcd`a*b` is a valid matrix product), one could say: 
%~~gen
% package Types.cripes.whered.you.get.those.gripes;
% class Matrix(rows:Int, cols:Int){
% public static def someMatrix(): Matrix = null;
% public static def example(){
%~~vis
\begin{xten}
  val a : Matrix = someMatrix() ;
  var b : Matrix{b.rows == a.cols} ;
\end{xten}
%~~siv
%}}
%~~neg

\item One way to say that objects \xcd`c` and \xcd`d` are located at the same
      place is: 
%~~gen
% package Types.flipes.knipes.shipes.wipes;
% class Exampler {
% static def someObject(): Object = null;
% static def example() {
%~~vis
\begin{xten}
  val a : Object = someObject();
  var b : Object{a.home == b.home};
\end{xten}
%~~siv
%}}
%~~neg
\end{itemize}



%%BARD-HERE


Given a type \xcd"T", a {\em constrained type} \xcd"T{e}" may be constructed
by constraining its properties with a boolean expression of a limited sort
\xcd"e". The values of \xcd`T{e}` are those values of type \xcd`T` for which
\xcd`e` evaluates to \xcd`true`.  \Eg, \xcd`Point` has a property
\xcd`rank:Int`.  If \xcd`p : Point`, \xcd`p` may have any \xcd`rank`.
\xcd`Point{rank == 3}` is the point type constrained to only those values
whose \xcd`rank` property is 3.   

A common use of constrained types is to explain where objects are located.
Every object has a \xcd`home` property. If \xcd`Person` is a type of people,
then \xcd`Person{home==here}` is the type of people whose data is stored at
the current location.  As explained in \Sref{XtenPlaces}, certain operations
can only be performed at an object's home, so having this expressible as a
type is crucial.


\xcd"T{e}" is a {\em dependent type}, that is, a type dependent on values. The
type \xcd"T" is called the {\em base type} and \xcd"e" is called the {\em
constraint}. 

For brevity, the constraint may be omitted and
interpreted as \xcd"true".

Constraints may refer to values in the local environment: 
%~~gen
% class ConstraintsMayReferToValues {
% def thoseValues() {
%~~vis
\begin{xten}
     val n = 1;
     var p : Point{rank == n};
\end{xten}
%~~siv
%}}
%~~neg
Indeed, there is technically no need for a constraint to refer to the
properties of its type; it can refer entirely to the environment, thus: 
%~~gen
% class ConstraintsMayReferToValuesTwo {
% def thoseValues() {
%~~vis
\begin{xten}
     val m = 1;
     val n = 2;
     var p : Point{m != n};
\end{xten}
%~~siv
%}}
%~~neg

Constraints on properties induce a natural subtyping relationship:
\xcd"C{c}" is a subtype of
\xcd"D{d}" if \xcd"C" is a subclass of \xcd"D" and
\xcd"c" entails \xcd"d".

Type parameters cannot be constrained.

\subsection{Constraints}

\def\withmath#1{\relax\ifmmode#1\else{$#1$}\fi}
\def\LL#1{\withmath{\lbrack\!\lbrack #1\rbrack\!\rbrack}}

Expressions used as constraints are restricted by the constraint
system in use to ensure that the constraints can be solved at compile
time.  The \Xten{} compiler allows compiler plugins to be installed to
extend the constraint language and the constraint system.  Constraints
must be of type \xcd"Boolean".  The compiler supports the following
constraint syntax.


\begin{grammar}
Constraint \: ValueArguments     Guard\opt \\
           \| ValueArguments\opt Guard     \\
           \\
ValueArguments   \:  \xcd"(" ArgumentList\opt \xcd")" \\
ArgumentList     \:  Expression ( \xcd"," Expression )\star \\
Guard            \: \xcd"{" DepExpression \xcd"}" \\
DepExpression    \: ( Formal \xcd";" )\star ArgumentList \\
\end{grammar}

In \XtenCurrVer{} value constraints may be equalities (\xcd"=="),
disequalities (\xcd"!=") and conjunctions thereof.  The terms over
which these constraints are specified include literals and
(accessible, immutable) variables and fields, property methods, and the special
constants {\tt here}, {\tt self}, and {\tt this}. Additionally, place
types are permitted (\Sref{PlaceTypes}).

\index{self}
When constraining a value of type \xcd`T`, \xcd`self` refers to the object of
type \xcd`T` which is being constrained.  For example, \xcd`Int{self == 4}` is
the type of \xcd`Int`s which are equal to 4 -- the best possible description
of \xcd`4`, and a very difficult type to express without using \xcd`self`.  


Type constraints may be subtyping and supertyping (\xcd"<:" and
\xcd":>") expressions over types.

The static constraint checker approximates computational reality in some
cases.  For example, it assumes that built-in types are infinite. This is a
good approximation for \xcd`Int`.  It is a poor approximation for \xcd`Boolean`,
as the checker believes that \xcd`a != b && a != c && b != c` is satisfiable
over \xcd`Boolean`, which it is not.  However, the checker is always correct
when computing the truth or falsehood of a constraint.


% //, and existential quantification over typed variables.

\emph{
Subsequent implementations are intended to support boolean algebra,
arithmetic, relational algebra, etc., to permit types over regions and
distributions. We envision this as a major step towards removing most,
if not all, dynamic array bounds and place checks from \Xten{}.
}


\subsubsection{Acyclicity restriction}

To ensure that type-checking is decidable, we
require that property graphs be acyclic.
That is, it should not be the case at runtime that
a set of objects can be created such that the
graph formed by taking objects as nodes and adding an edge from $m$ to
$n$ if $m$ has a property whose value is $n$ has a cycle in it.

Currently this restriction is not checked by the compiler. Future
versions of the compiler will check this restriction by introducing
rules on escaping of \Xcd{this} (\Sref{protorules}) before the invocation of
\Xcd{property} calls.


\subsection{Place constraints}
\label{PlaceTypes}
\label{PlaceType}
\index{place types}
\label{DepType:PlaceType}\index{placetype}

An \Xten{} computation spans multiple places
(\Sref{XtenPlaces}). Each place constains data and activities that
operate on that data.  \XtenCurrVer{} does not permit the dynamic
creation of a place. Each \Xten{} computation is initiated with a
fixed number of places, as determined by a configuration parameter.
In this section we discuss how the programmer may supply place type
information, thereby allowing the compiler to check data locality,
i.e., that data items being accessed in an atomic section are local.

\begin{grammar}
PlaceConstraint     \: \xcd"!" Place\opt \\
Place              \:   Expression \\
\end{grammar}

Because of the importance of places in the \Xten{} design, special
syntactic support is provided for constrained types involving places.

All \Xten{} classes extend the class
\xcd"x10.lang.Object", which defines a property
\xcd"home" of type
\xcd"Place".

If a constrained reference type \xcd"T" has an \xcd"!p" suffix,
the constraint for \xcd"T" is implicitly assumed to contain the clause
\xcd"self.home==p"; that is,
\xcd"C{c}!p" is equivalent to \xcd"C{self.home==p && c}".

The place \xcd"p" may be ommitted. It defaults to \xcd"this" 
for types in field declarations, and to \xcd"here" elsewhere.


% The place specifier \xcd"any" specifies that the object can be
% located anywhere.  Thus, the location is unconstrained; that is,
% \xcd"C{c}!any" is equivalent to \xcd"C{c}".

% XXX ARRAY
%The place specifier \xcd"current" on an array base type
%specifies that an object with that type at point \xcd"p"
%in the array 
%is located at \xcd"dist(p)".  The \xcd"current" specifier can be
%used only with array types.




\subsection{Constraint semantics}

\begin{staticrule}{Variable occurrence}
In a dependent type \xcd"T" = \xcd"C{c}", the only variables that may
occur in \xcd"c" are (a)
\xcd"self", (b) properties visible at \xcd"T", (c)  local \xcd`val`s, \xcd`val`
method parameters or \xcd`val` constructor parameters visible at \xcd"T", (d)
\xcd`val` fields visible at \xcd"T"'s lexical place in the source program.  
\end{staticrule}

\begin{staticrule}{Restrictions on \xcd"this"}
  The special variable \xcd"this" may be used in a dependent clause for a type \xcd"T"
  only if \xcd`this` may be used in an expression at that point.  \Viz, if 
  \xcd"this" occurs in (a) a property declaration for a
  class, (b) an instance method, (c) an
  instance field, or (d) an instance initializer.

  In particular, \xcd"this" may not be used in types that occur in a static
  context, or in the arguments, body or return type of a constructor or
  in the extends or implements clauses of class and interface
  definitions.  In these contexts, the object that \xcd"this" would
  correspond to is not defined.
\end{staticrule}

\begin{staticrule}{Variable visibility}
  If a type \xcd"T" occurs in a field, method or constructor
  declaration, then all variables used in \xcd"T" must have at least the
  same visibility as the declaration.  The relation ``at least the same
  visibility as'' is given by the transitive closure of:

\begin{xten}
public > protected > package > private
\end{xten}

All inherited properties of a type \xcd"T" are visible in the property
list of \xcd"T", and the body of \xcd"T".

\end{staticrule}

In general, variables (i.e., local variables, parameters,
properties, fields) are visible at
\xcd"T" if they are defined before \xcd"T" in the program. This rule applies to
types in property lists as well as parameter lists (for methods and
constructors).
A formal parameter is visible in the types of all other formal
parameters of the same method, constructor, or type definition,
as well as in the method or constructor body itself.
Properties are accessible via their containing object--\xcd"this"
within the body of their class declaration.  The special
variable \xcd"this" is in scope at each property
declaration, constructor signatures and bodies, instance method signatures
and bodies,
and instance field signatures and initializers, but not in scope
at \xcd"static" method or field declarations or \xcd"static"
initializers.  

We permit variable declarations \xcd"v: T" where \xcd"T" is obtained
from a dependent type \xcd"C{c}" by replacing one or more occurrences
of \xcd"self" in \xcd"c" by \xcd"v". (If such a declaration \xcd"v: T"
is type-correct, it must be the case that the variable \xcd"v" is not
visible at the type \xcd"T". Hence we can always recover the
underlying dependent type \xcd"C{c}" by replacing all occurrences of \xcd"v"
in the constraint of \xcd"T" by \xcd"self".)

For instance, \xcd"v: Int{v == 0}" is shorthand for \xcd"v: Int{self == 0}".

\begin{staticrule}{Constraint type}
The type of a constraint \xcd"c" must be \xcd"Boolean".  
\end{staticrule}

A variable occurring in the constraint \xcd"c" of a dependent type, other than
\xcd"self" or a property of \xcd"self", is said to be a {\em
parameter} of \xcd"c".\label{DepType:Parameter} \index{parameter}

An instance \xcd"o" of \xcd"C" is said to be of type \xcd"C{c}"
(or: {\em belong to}
\xcd"C{c}") if the predicate \xcd"c" evaluates to \xcd"true" in the current lexical
environment, augmented with the binding \xcd"self" $\mapsto$ \xcd"o". We shall
use the function \LL{\mbox{\Xcd{C\{c\}}}} to denote the set of
objects that belong to \xcd"C{c}". 


\subsection{Consistency of dependent types}\label{DepType:Consistency}\index{dependent type,consistency}

A dependent type \xcd"C{c}" may contain zero or more parameters. We require
that a type never be empty---so that it is possible for a variable of
the type to contain a value. This is accomplished by requiring that
the constraint \xcd"c" must be satisfiable {\em regardless} of the value assumed
by parameters to the constraint (if any). Formally, consider a type
\xcd"T" = \xcd"C{c}", with the variables
\xcdmath"f$_1$: F$_1$, $\dots$, f$_k$: F$_k$"
free in \xcd"c".  Let 
\xcdmath"$S$ = {f$_1$: F$_1$, $\dots$, f$_k$: F$_k$, f$_{k+1}$: F$_{k+1}$, $\dots$, f$_n$: F$_n$}"
be the smallest set of
declarations containing
\xcdmath"f$_1$: F$_1$, $\dots$, f$_k$: F$_k$"
and closed under the rule:
\xcd"f: F" in $S$ if a reference to variable \xcd"f" (which
is declared as \xcd"f: F") occurs in a type in $S$.

(NOTE: The syntax rules for the language ensure that $S$ is always
finite. The type for a variable \xcd"v" cannot reference a variable whose
type depends on \xcd"v".)

We say that \xcd"T" = \xcd"C{c}" is {\em parametrically consistent} (in brief:
{\em consistent}) if:

\begin{itemize}
\item Each type \xcdmath"F$_1$, $\dots$, F$_n$" is (recursively) parametrically consistent, and
\item It can be established that
\xcdmath"$\forall$f$_1$: F$_1$, $\dots$, f$_n$: F$_n$. $\exists$self: C. c && $\mathit{inv}$(C)".
\end{itemize}

\noindent
where \xcdmath"$\mathit{inv}$(C)" is the invariant associated
with the type \xcd"C" (\Sref{DepType:TypeInvariant}).  Note by
definition of $S$ the formula above has no free variables.

\begin{staticrule*}
For a declaration \xcd"v: T" to be type-correct, \xcd"T" must be parametrically
consistent. The compiler issues an error if it cannot determine
the type is parametrically consistent.
\end{staticrule*}

\begin{example}

A class that represents a line has two distinct points:\footnote{We call them
\xcd`Position` to avoid confusion with the built-in class \xcd`Point`}

%~~gen
% 
%~~vis
\begin{xten}
class Position(x: Int, y: Int) {
   def this(x:Int,y:Int){property(x,y);}
   }
class Line(start: Position, 
           end: Position{self != start}) {}
\end{xten}
\end{example}
%~~siv
%~~neg


\begin{example}
One can use dependent type to define other closed geometric figures as well.

To see that the declaration \xcd"end: Position{self != start}" is
parametrically consistent, note that the following formula is valid:
\begin{xtenmath}
$\forall$this: Line. $\exists$self: Position. self != this.start  
\end{xtenmath}
\noindent since the set of all \xcd"Position"s has more than one element.
\end{example}

\begin{example}
A triangle has three lines sharing three vertices.

%~~gen
%package triangleExample;
% class Position(x: Int, y: Int) {
%    def this(x:Int,y:Int){property(x,y);}
%    }
% class Line(start: Position, 
%            end: Position{self != start}) {}
% 
%~~vis
\begin{xten}
class Triangle 
 (a: Line, 
  b: Line{a.end == b.start}, 
  c: Line{b.end == c.start && c.end == a.start}) 
 {
   def this(a:Line,
            b: Line{a.end == b.start}, 
            c: Line{b.end == c.start && c.end == a.start}) 
   {property(a,b,c);}
 }
\end{xten}
%~~siv
%
%~~neg

Given \xcd"a: Line", the type \xcd"b: Line{a.end == b.start}" is consistent,
and
given the two, the type \xcd"c: Line{b.end == c.start, c.end == a.start}"
is consistent.

%%Similarly:
%%
%%   // A class with properties a, b,c,x satisfying the 
%%   // given constraints.
%%   class SolvableQuad(a: Int, b: Int, 
%%                      c: Int{b*b - 4*a*c >= 0},
%%                      x: Int{a*x*x + b*x + c==0}) { 
%%     ...
%%   }
%%
%%  // A class with properties r, x, and y satisfying
%%  // the conditions for (x,y) to lie on a circle with center (0,0)
%%  // and radius r.
%%   class Circle (r: Int{r > 0},
%%                 x: Int{r*r - x*x >= 0},
%%                 y: Int{y*y == r*r -x*x}) { 
%%   ...
%%   }
\end{example}

\section{Function types}
\label{FunctionTypes}
\label{FunctionType}
\index{function!types}
\index{types!function types}

        Function types are defined via the \xcd"=>" type
        constructor.  Closures (\Sref{Closures}) and method
        selectors (\Sref{MethodSelectors}) are of function type.
        The general form of a function type is:
\begin{xtenmath}
(x$_1$: T$_1$, $\dots$, x$_n$: T$_n$){c} => T
        throws S$_1$, $\dots$, S$_k$
\end{xtenmath}
        This
        is the type of functions that take 
        value parameters
        \xcdmath"x$_i$"
        of types
        \xcdmath"T$_i$"
        such that the guard \xcd"c" holds
        and returns a value of type \xcd"T" or throws
        exceptions of 
        types S$_i$.

The value parameters are in scope throughout the function
signature---they may be used in the types of other formal parameters
and in the return type.  Value parameters names  may be
omitted if they are not used.  The guard specifies a condition that 
must hold for an application to be well-typed.

\begin{grammar}
FunctionType \: TypeParameters\opt \xcd"(" Formals\opt \xcd")" Constraint\opt
\xcd"=>" Type Throws\opt \\
TypeParameters \: \xcd"[" TypeParameter ( \xcd"," TypeParameter
)\star \xcd"]" \\
TypeParameter \: Identifier \\
Formals \: Formal ( \xcd"," Formal )\star \\
\end{grammar}


For every sequence of types \xcd"T1,..., Tn,T", and \xcd"n" distinct variables
\xcd"x1,...,xn" and constraint \xcd"c", the expression
\xcd"(x1:T1,...,xn:Tn){c}=>T" is a \emph{function type}. It stands for
 the set of all functions \xcd"f" which can be applied in a place \xcd"p" to a
 list of values \xcd"(v1,...,vn)" provided that the constraint
 \xcd"c[v1,...,vn,p/x1,...,xn,here]" is true, and which returns a value of
 type \xcd"T[v1,...vn/x1,...,xn]". When \xcd"c" is true, the clause \xcd"{c}" can be
 omitted. When \xcd"x1,...,xn" do not occur in \xcd"c" or \xcd"T", they can be
 omitted. Thus the type \xcd"(T1,...,Tn)=>T" is actually shorthand for
 \xcd"(x1:T1,...,xn:Tn){true}=>T", for some variables \xcd"x1,...,xn".


Juxtaposition is used to express function application: the expression
\xcd"f(a1,..,an)" expresses the application of a function \xcd"f" to the argument
list \xcd"a1,...,an".

\index{Exception!unchecked}
Note that function invocation may throw unchecked exceptions. 

A function type is covariant in its result type and contravariant in
each of its argument types. That is, let 
\xcd"S1,...,Sn,S,T1,...Tn,T" be any
types satisfying \xcd"Si <: Ti" and \xcd"S <: T". Then
\xcd"(x1:T1,...,xn:Tn){c}=>S" is a subtype of
\xcd"(x1:S1,...,xn:Sn){c}=>T".


A value \xcd"f" of a function type \xcd"(x1:T1,...,xn:Tn){c}=>T" also
has all the methods of \Xcd{Any} associated with it (see \Sref{FunctionAnyMethods}).


A function type \xcd"F=(x1:T1,...,xn:Tn){c}=>T" can be used as the declared type of local variables, parameters, loop variables, return types of methods and in \xcd"_� instanceof F" and \xcd"_ as F" expressions. 


A class or struct definition may use a function type \xcd"F" in its
implements clause; this declares an abstract method 
\xcd"def apply(x1:T1,...,xn:Tn){c}:T" on that class. Similarly, an interface
definition may specify a function type "F" in its "extends" clause. A
class or struct implementing such an interface implicitly defines an
abstract method \xcd"def apply(x1:T1,..,xn:Tn){c}:T". Expressions of such
a struct, class or interface type can be assigned to variables of type
\xcd"F" and can be applied via juxtaposition to an argument list of the
right type.


Thus, objects and structs in \Xten{} may behave like functions. 

A function type \xcd"F" is not a class type in that it does not extend any
type or implement any interfaces, or support equality tests. \xcd"F" cannot be extended by any type. It
is not an interface type in that it is not a subtype of
\xcd"x10.lang.Object". (Values of type \xcd"F" cannot be assigned to variables of
type \xcd"x10.lang.Object".) It is not a struct type in that it has no
defined fields and hence no notion of structural equality.

\xcd"null" is a legal value for a function type. 


\section{Annotated types}
\label{AnnotatedTypes}

\index{types!annotated types}
\index{annotations!type annotations}

        Any \Xten{} type may be annotated with zero or more
        user-defined \emph{type annotations}
        (\Sref{XtenAnnotations}).  

        Annotations are defined as (constrained) interface types and are
        processed by compiler plugins, which may interpret the
        annotation symbolically.

        A type \xcd"T" is annotated by interface types
        \xcdmath"A$_1$", \dots,
        \xcdmath"A$_n$"
        using the syntax
        \xcdmath"@A$_1$ $\dots$ @A$_n$ T".

\section{Subtyping and type equivalence}\label{DepType:Equivalence}
\index{type equivalence}
\index{subtyping}

Subtyping is relation between types.  It is the
reflexive, transitive 
closure of the {\em direct subtyping} relation, defined as
follows.

\paragraph{Class types.}  A class type is a direct subtype of
any
class it is declared to extend.  A class type is direct subtype
of any interfaces it is declared to implement.

\paragraph{Interface types.}  An interface type is a direct
subtype of any interfaces it is declared to extend.

\paragraph{Function types.}

Function types are covariant on their return type and
contravariant on their argument types.
For instance,
a function type
\xcd"(S1) => T1" 
is a subtype of another function type
\xcd"(S2) => T2" 
if \xcd"S2" is a subtype of \xcd"S1"
and \xcd"T1" is a subtype of \xcd"T2".

\paragraph{Constrained types.}

Two dependent types \xcd"C{c}" and \xcd"C{d}" are said to be {\em equivalent} if 
\xcd"c" is true whenever \xcd"d" is, and vice versa. Thus, 
$\LL{\mbox{\Xcd{C\{c\}}}} = \LL{\mbox{\Xcd{C\{d\}}}}$.

Note that two dependent type that are syntactically different may be
equivalent. For instance, \xcd"Int{self >= 0}" and
\xcd"Int{self == 0 || self > 0}" are equivalent though they are syntactically
distinct. The \Java{} type system is essentially a nominal system---two
types are the same if and only if they have the same name. The \Xten{}
type system extends the nominal type system of \Java{} to permit
constraint-based equivalence.

A dependent type \xcd"C{c}" is a subtype of a type \xcd"C{d}" if
\xcd"c" implies \xcd"d".  When this subtyping relationship holds, 
$\LL{\mbox{\Xcd{C\{c\}}}}$ is a
subset of $\LL{\mbox{\Xcd{C\{d\}}}}$. All dependent types
defined on a class \xcd"C"
refine the unconstrained class type \xcd"C"; \xcd"C" is
equivalent to \xcd"C{true}".

\paragraph{Type parameters.}

A type parameter \xcd"X" of a class or interface \xcd"C"
is a subtype of a type \xcd"T" if
the 
class invariant of \xcd"C" implies that \xcd"X" is a subtype of \xcd"T".
Similarly, \xcd"T" is a subtype of parameter \xcd"X" if the
class invariant implies the relationship.

A type parameter \xcd"X" of a method
\xcd"m"
is a subtype of a type \xcd"T" if
the 
guard of \xcd"m" implies that \xcd"X" is a subtype of \xcd"T".
Similarly, \xcd"T" is a subtype of parameter \xcd"X" if the
guard implies the relationship.


\section{Least common ancestor of types}
\label{LCA}

To compute the type of conditional expressions
(\Sref{Conditional}),
and of rail constructors
(\Sref{RailConstructors}), the least common ancestor of types
must be computed.

The least common ancestor of two  types
\xcdmath"T$_1$" and \xcdmath"T$_2$"
is the
unique most-specific type
that is a supertype of both
\xcdmath"T$_1$" and \xcdmath"T$_2$".

If the most-specific type is not unique (which can happen when
\xcdmath"T$_1$" and \xcdmath"T$_2$" both implement two
or more incomparable interfaces), then
least common ancestor type is \xcd"x10.lang.Any".

\section{Coercions and conversions}
\label{XtenConversions}
\label{User-definedCoercions}
\index{conversions}\index{coercions}

\XtenCurrVer{} supports the following coercions and conversions

\subsection{Coercions}

A {\em coercion} does not change object identity; a coerced object may
be explicitly coerced back to its original type through a cast. A {\em
  conversion} may change object identity if the type being converted
to is not the same as the type converted from. \Xten{} permits
user-defined conversions (\Sref{sec:user-defined-conversions}).

\paragraph{Subsumption coercion.}
A subtype may be implicitly coerced to any supertype.
\index{coercions!subsumption coercion}

\paragraph{Explicit coercion (casting with \xcd"as")}
A reference type may be explicitly coerced to any other
reference type using the \xcd"as" operation.
If the value coerced is not an instance of the target type,
a \xcd"ClassCastException" is thrown.  Casting to a constrained
type may require a run-time check that the constraint is
satisfied.
\index{coercions!explicit coercion}
\index{casting}
\index{\Xcd{as}}

\subsection{Conversions}

\paragraph{Narrowing conversion.}
A value class may be explicitly converted to any superclass
using the \xcd"as" operation.

\index{conversions!narrowing conversions}

\paragraph{Widening numeric conversion.}
A numeric type may be implicitly converted to a wider numeric type. In
particular, an implicit conversion may be performed between a numeric
type and a type to its right, below:

\begin{xten}
Byte < Short < Int < Long < Float < Double
\end{xten}

\index{conversions!widening conversions}
\index{conversions!numeric conversions}

\paragraph{String conversion.}
Any object that is an operand of the binary
\xcd"+" operator may
be converted to \xcd"String" if the other operand is a \xcd"String".
A conversion to \xcd"String" is performed by invoking the \xcd"toString()"
method of the object.

\index{conversions!string conversion}

\paragraph{User defined conversions.}\label{sec:user-defined-conversions}
\index{conversions!user defined}

The user may define conversion operators from type \Xcd{A} {\em to} a
container type \Xcd{B} by specifying a method on \Xcd{B} as follows:

\begin{xten}
  public static operator (r: A): T = ... 
\end{xten}

The return type \Xcd{T} should be a subtype of \Xcd{B}. The return
type need not be specified explicitly; it will be computed in the
usual fashion if it is not. However, it is good practice for the
programmer to specify the return type for such operators explicitly.

For instance, the code for \Xcd{x10.lang.Point} contains:

\begin{xten}
  public static global safe operator (r: Rail[int])
     : Point(r.length) = make(r);
\end{xten}

The compiler looks for such operators on the container type \Xcd{B}
when it encounters an expression of the form \Xcd{r as B} (where
\Xcd{r} is of type \Xcd{A}). If it finds such a method, it sets the
type of the expression \Xcd{r as B} to be the return type of the
method. Thus the type of \Xcd{r as B} is guaranteed to be some subtype
of \Xcd{B}.

\begin{example}
Consider the following code:  
\begin{xten}
val p  = [2, 2, 2, 2, 2] as Point;
val q = [1, 1, 1, 1, 1] as Point;
val a = p - q;    
\end{xten}
This code fragment compiles successfully, given the above operator definition. 
The type of \Xcd{p} is inferred to be \Xcd{Point(5)} (i.e.{} the type 
\xcd"Point{self.rank==5}".
Similarly for \Xcd{q}. Hence the application of the operator ``\Xcd{-}'' is legal (it requires both arguments to have the same rank). The type of \Xcd{a} is computed as \Xcd{Point(5)}.
\end{example}


%\subsection{Syntactic abbreviations}\label{DepType:SyntaxAbbrev}

\section{Built-in types}

The package \xcd"x10.lang" provides a number of built-in class and
interface declarations that can be used to construct types.

\subsection{The class \Xcd{Object}}
\label{Object}
\index{\Xcd{Object}}
\index{\Xcd{x10.lang.Object}}

The class \xcd"x10.lang.Object" is the supertype of all classes.
A variable of this type can hold a reference to any object.
The code for this class (with annotations removed) is:
\begin{xten}
public class Object (home: Place) 
     implements Any
{
    public native def this();
    public property def home() = home;
    public property def at(p:Place) = home==p;
    public property def at(r:Object) = home==r.home;
    public global safe native def toString() : String;
    public global safe native def typeName() : String;
    public global safe def equals(x:Any) = this == x;
    public global safe native def hashCode():Int;
}
\end{xten}

\subsection{The class \Xcd{String}}
\label{String}\index{\Xcd{String}}\index{\Xcd{x10.lang.String}}

Strings in \Xten{} are instances of the class \xcd"x10.lang.String", and are
all immutable.
Strings are one of the few types with literals, rather than simply
      constructors.  String literals are the familiar \xcd`"`-delimited
      strings, like \xcd`"this"` and \xcd`"that"`.

Every X10 value has a \xcd`String` print representation, given by
      \xcd`whatever.toString()`.   
All values can be implicitly converted to strings by the concatenation
      operation \xcd`+`, which calls their \xcd`toString()` methods if they
      are not strings already.  For example, 
%~~exp~~`~~`~~ ~~
      \xcd`"one " + 2 + here` 
      evaluates to something like \xcd`one 2(Place 0)`.  



\section{Array Type Constructors}
\label{ArrayypeConstructors}\index{array types}

{}\XtenCurrVer{} does not have array class declarations
(\S~\ref{XtenArrays}). This means that user cannot define new array
class types. Instead arrays are created as instances of array types
constructed through the application of {\em array type constructors}
(\S~\ref{XtenArrays}).

The array type constructor takes as argument a type (the {\em base
type}), an optional distribution (\S~\ref{XtenDistributions}), and
optionally either the keyword {\cf reference} or {\cf value} (the
default is reference):
\begin{x10}
18    ArrayType ::= Type [ ]
438   ArrayType ::= X10ArrayType
439   X10ArrayType ::= Type [ . ]
440     | Type reference [ . ]
441     | Type value [ . ]
442     | Type [ DepParameterExpr ]
443     | Type reference [ DepParameterExpr ]
444     | Type value [ DepParameterExpr ]
\end{x10}

The array type {\cf Type[ ] } is the type of all arrays of base
type {\tt Type} defined over the distribution {\tt 0:N -> here}
for some positive integer {\tt N}.

The qualifier {\tt value} ({\tt reference}) specifies that the array
is a {\tt value}({\tt reference}) array. The array elements of a {\tt
value} array are all {\tt final}.\footnote{Note that the base type of a {\tt value} array can be a value class or a reference class, just as the 
type of a {\tt final} variable can be a value class or a reference class.
}If the qualifier is not specified,
the array is a {\tt reference} array.

The array type {\cf Type reference [.]} is the type of all (reference)
arrays of basetype {\tt Type}. Such an array can take on any
distribution, over any region. Similarly, {\cf Type value [.]} is the
type of all value arrays of basetype {\tt Type}.

\XtenCurrVer{} also allows a distribution to be specified between {\tt
[} and {\tt ]}. The distribution must be an expression of type
{\tt distribution} (e.g.{} a {\tt final} variable) whose
value does not depend on the value of any mutable variable.

Future extensions to \Xten{} will support a more general syntax for
arrays which allows for the specification of dependent types, 
e.g.{} {\tt double[:rank 3]}, the type of all arrays of 
{\tt double} of rank {\tt 3}.



\subsection{Future types}

The class \xcd"x10.lang.Future[T]"
is the type of all \xcd"future" expressions.
The type represents a value which when forced will return a value of type
\xcd"T". The class makes available the following methods:

\begin{xten}
package x10.lang;
public class Future[T] implements () => T {
  public def apply(): T = force();
  public def force(): T = ...;
  public def forced(): Boolean = ...;
}
\end{xten}
  


\section{Type inference}
\label{TypeInference}
\index{types!inference}
\index{type inference}

\XtenCurrVer{} supports limited local type inference, permitting
variable types and return types to be elided.
It is a static error if an omitted type cannot be inferred or
uniquely determined.

\subsection{Variable declarations}

The type of a variable declaration can be omitted if the
declaration has an initializer.  The inferred type of the
variable is the computed type of the initializer.

\subsection{Return types}

The return type of a method can be omitted if the method has a
body (i.e., is not \xcd"abstract" or \xcd"extern").  The
inferred return type is the computed type of the body.

The return type of a closure can be omitted.
The inferred return type is the computed type of the body.

The return type of a constructor can be omitted if the
constructor has a body (i.e., is not \xcd"extern").
The inferred return type is the enclosing class type with
properties bound to the arguments in the constructor's \xcd"property"
statement, if any, or to the unconstrained class type.

\index{Void}
The inferred type of a method or closure body is the least common ancestor
of the types of the expressions in \xcd"return" statements
in the body.  If the method does not return a value, the
inferred type is \xcd"Void".

\subsection{Type arguments}

A call to a polymorphic method %, closure, or constructor 
may omit the
explicit type arguments.  If the method has a type parameter
\xcd"T", the type argument corresponding to \xcd"T" is inferred
to be the least common ancestor of the types of any formal
parameters of type \xcd"T".

%TODO--check this!
Consider the following method:
\begin{xten}
def choose[T](a: T, b: T): T { ... }
\end{xten}
%
Given \xcd"Set[T] <: Collection[T]", 
\xcd"List[T] <: Collection[T]",
and \xcd"SubClass <: SuperClass",
in the following snippet, the algorithm will infer the type
\xcd"Collection[Any]" for \xcd"x".
\begin{xten}
def m(intSet: Set[Int], stringList: List[String]) {
  val x = choose(intSet, stringList);
...
}
\end{xten}
%
And in this snippet, the algorithm should infer the type
\xcd"Collection[Int]" for \xcd"y".
\begin{xten}
def m(intSet: Set[Int], intList: List[Int]) {
  val y = choose(intSet, intList);
  ...
}
\end{xten}
%
Finally, in this snippet, the algorithm should infer the type
\xcd"Collection{T <: SuperClass}" for \xcd"z".
\begin{xten}
def m(intSet: Set[SubClass], numList: List{T <: SuperClass}) {
  val z = choose(intSet, numList);
  ...
}
\end{xten}

	

\chapter{Variables}\label{XtenVariables}\index{variable}

%%OLDA variable is a storage location.  \Xten{} supports seven kinds of
%%OLDvariables: constant {\em class variables} (static variables), {\em
%%OLD  instance variables} (the instance fields of a class), {\em array
%%OLD  components}, {\em method parameters}, {\em constructor parameters},
%%OLD{\em exception-handler parameters} and {\em local variables}.

A {\em variable} is an X10 identifier associated with a value within some
context. Variable bindings have these essential properties:
\begin{itemize}
\item {\bf Type:} What sorts of values can be bound to the identifier;
\item {\bf Scope:} The region of code in which the identifier is associated
      with the entity;
\item {\bf Lifetime:} The interval of time in which the identifier is
      associated with the entity.
\item {\bf Visibility:} Which parts of the program can read or manipulate the
      value through the variable.
\end{itemize}



X10 has many varieties of variables, used for a number of purposes. They will
be described in more detail in this chapter.  
\begin{itemize}
\item Class variables, also known as the static fields of a class, which hold
      their values for the lifetime of the class.  
\item Instance variables, which hold their values for the lifetime of an
      object;
\item Array elements, which are not individually named and hold their values
      for the lifetime of an array;
\item Formal parameters to methods, functions, and constructors, which hold
      their values for the duration of method (etc.) invocation;
\item Local variables, which hold their values for the duration of execution
      of a block.
\item Exception-handler parameters, which hold their values for the execution
      of the exception being handled. 
\end{itemize}
A few other kinds of things are called variables for historical reasons; \eg,
type parameters are often called type variables, despite not being variables
in this sense because they do not refer to X10 values.  Other named entities,
such as classes and methods, are not called variables.  However, all
name-to-whatever bindings enjoy similar concepts of scope and visibility.  

In the following example, \xcd`n` is an instance variable, and \xcd`nxt` is a
local variable defined within the method \xcd`bump`.\footnote{This code is
unnecessarily turgid for the sake of the example.  One would generally write
\xcd`public def bump() = ++n;`.   }
%~~gen
% package Vars.For.Squares;
%~~vis
\begin{xten}
class Counter {
  private var n : Int = 0;
  public def bump() : Int {
    val nxt = n+1;
    n = nxt;
    return nxt;
    }
}
\end{xten}
%~~siv
%
%~~neg
Both variables have type \xcd`Int` (or
perhaps something more specific).    The scope of \xcd`n` is the body of
\xcd`Counter`; the scope of \xcd`nxt` is the body of \xcd`bump`.  The
lifetime of \xcd`n` is the lifetime of the \xcd`Counter` object holding it;
the lifetime of \xcd`nxt` is the duration of the call to \xcd`bump`. Neither
variable can be seen from outside of its scope.

\label{exploded-syntax}
\label{VariableDeclarations}
\index{variable declaration}


Variables whose value may not be changed after initialization are said to be
{\em immutable}, or {\em constants} (\Sref{FinalVariables}), or simply
\xcd`val` variables. Variables whose value may change are {\em mutable} or
simply \xcd`var` variables. \xcd`var` variables are declared by the \xcd`var`
keyword. \xcd`val` variables may be declared by the \xcd`val` keyword; when a
variable declaration does not include either \xcd`var` or \xcd`val`, it is
considered \xcd`val`. 


%~~gen
%package Vars.For.Bears.In.Chairs;
%class VarExample{
%static def example() {
%~~vis
\begin{xten}
val a : Int = 0;               // Full 'val' syntax
b : Int = 0;                   // 'val' implied
val c = 0;                     // Type inferred
var d : Int = 0;               // Full 'var' syntax
var e : Int;                   // Not initialized
var f : Int{self != 100} = 0;  // Constrained type
\end{xten}
%~~siv
%}}
%~~neg







\section{Immutable variables}
\label{FinalVariables}
\index{variable!immutable}
\index{immutable variable}
\index{variable!val}
\index{val}

Immutable (\xcd`val`) variables can be given values (by initialization or assignment) at
most once, and must be given values before they are used.  Usually this is
achieved by declaring and initializing the variable in a single statement.
%~~gen
% package Vars.In.Snares;
% class ABitTedious{
% def example() {
%~~vis
\begin{xten}
val a : Int = 10;
val b = (a+1)*(a-1);
\end{xten}
%~~siv
%}}
%~~neg
\xcd`a` and \xcd`b` cannot be assigned to further.

In other cases, the declaration and assignment are separate.  One such
case is how constructors give values to \xcd`val` fields of objects.  The
\xcd`Example` class has an immutable field \xcd`n`, which is given different
values depending on which constructor was called. \xcd`n` can't be given its
value by initialization when it is declared, since it is not knowable which
constructor is called at that point.  
%~~gen
% package Vars.For.Cares;
%~~vis
\begin{xten}
class Example {
  val n : Int; // not initialized here
  def this() { n = 1; }
  def this(dummy:Boolean) { n = 2;}
}
\end{xten}
%~~siv
%
%~~neg

Another common case of separating declaration and assignment is in function
and method call.  The formal parameters are bound to the corresponding actual
parameters, but the binding does not happen until the function is called.  In
the code below, \xcd`x` is initialized to \xcd`3` in the first call and
\xcd`4` in the second.
%~~gen
%package Vars.For.Swears;
%class Examplement {
%static def whatever() {
%~~vis
\begin{xten}
val sq = (x:Int) => x*x;
x10.io.Console.OUT.println("3 squared = " + sq(3));
x10.io.Console.OUT.println("4 squared = " + sq(4));
\end{xten}
%~~siv
%}}
%~~neg





%%IMMUTABLE%% An immutable variable satisfies two conditions: 
%%IMMUTABLE%% \begin{itemize}
%%IMMUTABLE%% \item it can be assigned to at most once, 
%%IMMUTABLE%% \item it must be assigned to before use. 
%%IMMUTABLE%% \end{itemize}
%%IMMUTABLE%% 
%%IMMUTABLE%% \Xten{} follows \java{} language rules in this respect \cite[\S
%%IMMUTABLE%% 4.5.4,8.3.1.2,16]{jls2}. Briefly, the compiler must undertake a
%%IMMUTABLE%% specific analysis to statically guarantee the two properties above.
%%IMMUTABLE%% 
%%IMMUTABLE%% Immutable local variables and fields are defined by the \xcd"val"
%%IMMUTABLE%% keyword.  Elements of value arrays are also immutable.
%%IMMUTABLE%% 
%%IMMUTABLE%% \oldtodo{Check if this analysis needs to be revisited.}

\section{Initial values of variables}
\label{NullaryConstructor}\index{nullary constructor}
\index{initial value}
\index{initialization}


Every assignment, binding, or initialization to a variable of type \xcd`T{c}`
must be an instance of type \xcd`T` satisfying the constraint \xcd`{c}`.
Variables must be given a value before they are used. This may be done by
initialization, which is the only way for immutable (\xcd`val`) variables and
one option for mutable (\xcd`var`) ones: 

%~~gen
%package Vars.For.Bears;
%class VarsForBears{
%def check() {
%~~vis
\begin{xten}
  val immut : Int = 3;
  var mutab : Int = immut;
  val use = immut + mutab;
\end{xten}
%~~siv
%}}
%~~neg
Or, for mutable variables, it may be done by a later assignment.  

%~~gen
%package Vars.For.Stars;
%class VarsForStars{
%def check() {
%~~vis
\begin{xten}
  var muta2 : Int;
  muta2 = 4;
  val use = muta2 * 10;
\end{xten}
%~~siv
%}}
%~~neg


Every class variable must be initialized before it is read, through
the execution of an explicit initializer. Every
instance variable must be initialized before it is read, through the
execution of an explicit or implicit initializer or a constructor.
Implicit initializers initialize \xcd`var`s to the default values of their
types (\Sref{DefaultValues}). Variables of types which do not have default
values are not implicitly initialized.



Each method and constructor parameter is initialized to the
corresponding argument value provided by the invoker of the method. An
exception-handling parameter is initialized to the object thrown by
the exception. A local variable must be explicitly given a value by
initialization or assignment, in a way that the compiler can verify
using the rules for definite assignment \cite[\S~16]{jls2}.


\section{Destructuring syntax}
\index{variable declarator!destructuring}
\index{destructuring}
\Xten{} permits a \emph{destructuring} syntax for local variable
declarations with explicit initializers,  and for formal parameters, of type \xcd`Point`, \Sref{point-syntax}.
(Future versions of X10 may allow destructuring of other types as well.) 
A point is a sequence of {$r \ge 0$} \xcd`Int`-valued coordinates.  
It is often useful to get at the coordinates directly, in variables. 

The following code makes an anonymous point with one coordinate \xcd`11`, and
binds \xcd`i` to \xcd`11`.  Then it makes a point with coordinates \xcd`22`
and \xcd`33`, binds \xcd`p` to that point, and \xcd`j` and \xcd`k` to \xcd`22`
and \xcd`33` respectively.
%~~gen
% package Vars.For.Glares;
% class DestructuringEx1 {
% def whyJustForLocals() {
%~~vis
\begin{xten}
val [i] : Point = Point.make(11);
val p[j,k] = Point.make(22,33);
val q[l,m] = [44,55]; // coerces an array to a point.
\end{xten}
%~~siv
%}}
%~~neg

A useful idiom for iterating over a range of numbers is: 
%~~gen
%package Vars.For.Bears;
% class ForBear {
% def forbear() {
%~~vis
\begin{xten}
var sum : Int = 0;
for ([i] in 1..100) sum += i;
\end{xten}
%~~siv
% ; } } 
%~~neg
\noindent
The brackets in \xcd`[i]` introduce destructuring, making X10 treat \xcd`i`
as an \xcd`Int`; without them, it would be a \xcd`Point`.  

In general, a pattern of the form \xcdmath"[i$_1$,$\ldots$,i$_n$]" matches a
point with {$n$} coordinates, binding \xcdmath"i$_j$" to coordinate {$j$}.  
A pattern of the form \xcdmath"p[i$_1$,$\ldots$,i$_n$]" does the same,  but
also binds \xcd`p` to the point.

\section{Formal parameters}
\index{formal parameter}
\index{parameter}

\begin{bbgrammar}
 FormalParams    \: \xcd"(" FormalParamList\opt \xcd")" & (\ref{prod:FormalParams})\\%<FROM #(prod:FormalParams)#
 FormalParamList    \: FormalParam & (\ref{prod:FormalParamList})\\%<FROM #(prod:FormalParamList)#
    \| FormalParamList \xcd"," FormalParam\\
 FormalParam    \: Mods\opt FormalDeclarator & (\ref{prod:FormalParam})\\%<FROM #(prod:FormalParam)#
    \| Mods\opt VarKeyword FormalDeclarator\\
    \| Type\\
 FormalDeclarators    \: FormalDeclarator & (\ref{prod:FormalDeclarators})\\%<FROM #(prod:FormalDeclarators)#
    \| FormalDeclarators \xcd"," FormalDeclarator\\
 FormalDeclarator    \: Id ResultType & (\ref{prod:FormalDeclarator})\\%<FROM #(prod:FormalDeclarator)#
    \| \xcd"[" IdList \xcd"]" ResultType\\
    \| Id \xcd"[" IdList \xcd"]" ResultType\\
 ResultType    \: \xcd":" Type & (\ref{prod:ResultType})\\%<FROM #(prod:ResultType)#
\end{bbgrammar}

Formal parameters are the variables which hold values transmitted into a
method or function.  
They are always declared with a type.  (Type inference is not
available, because there is no single expression to deduce a type from.)
The variable name can be omitted if it is not to be used in the
scope of the declaration, as in the type of the method 
\xcd`static def main(Array[String]):void` executed at the start of a program that
does not use its command-line arguments.

\xcd`var` and \xcd`val` behave just as they do for local
variables, \Sref{local-variables}.  In particular, the following \xcd`inc`
method is allowed, but, unlike some languages, does {\em not} increment its
actual parameter.  \xcd`inc(j)` creates a new local 
variable \xcd`i` for the method call, initializes \xcd`i` with the value of
\xcd`j`, increments \xcd`i`, and then returns.  \xcd`j` is never changed.
%~~gen
% package Vars.For.Squares.Of.Mares;
% class Ink {
%~~vis
\begin{xten}
static def inc(var i:Int) { i += 1; }
\end{xten}
%~~siv
%}
%~~neg


\section{Local variables}\label{local-variables}
\index{variable!local}
\index{local variable}
Local variables are declared in a limited scope, and, dynamically, keep their
values only for so long as the scope is being executed.  They may be \xcd`var`
or \xcd`val`.  
They may have 
initializer expressions: \xcd`var i:Int = 1;` introduces 
a variable \xcd`i` and initializes it to 1.
If the variable is immutable
(\xcd"val")
the type may be omitted and
inferred from the initializer type (\Sref{TypeInference}).

The variable declaration \xcd`val x:T=e;` confirms that \xcd`e`'s value is of
type \xcd`T`, and then introduces the variable \xcd`x` with type \xcd`T`.  For
example, consider a class Tub with a property \xcd`p`.
%~~gen
% package Vars.Local;
%~~vis
\begin{xten}
class Tub(p:Int){
  def this(pp:Int):Tub{self.p==pp} {property(pp);}
  def example() {
    val t : Tub = new Tub(3);
  }
}
\end{xten}
%~~siv
%
%~~neg
\noindent
produces a variable \xcd`t` of type \xcd`Tub`, even though the expression
\xcd`new Tub(3)` produces a value of type \xcd`Tub{self.p==3}` -- that is, a
\xcd`Tub`  whose \xcd`p` field is 3.  This can be inconvenient when the
constraint information is required.

\index{\Xcd{<:}}
Including type information in variable declarations is generally good
programming practice: it explains to both the compiler and human readers
something of the intent of the variable.  However, including types in 
\xcd`val t:T=e` can obliterate helpful information.  So, X10 allows a {\em
documentation type declaration}, written \xcd`val t <: T = e`.  This 
has the same effect as \xcd`val t = e`, giving \xcd`t` the full type inferred
from \xcd`e`; but it also confirms statically that that type is at least
\xcd`T`.  For example, the following gives \xcd`t` the type \xcd`Tub{self.p==3}` as
desired: 
%~~gen
% package Vars.Local;
% class TubBounded{
% def example() {
%~~vis
\begin{xten}
   val t <: Tub = new Tub(3);
\end{xten}
%~~siv
%}}
%~~neg
\noindent
However, replacing \xcd`Tub` by \xcd`Int` would result in a compilation error. 

Variables do not need to be initialized at the time of definition -- not even
\xcd`val`s. They must be initialized by the time of use, and \xcd`val`s may
only be assigned to once. The X10 compiler performs static checks guaranteeing
this restriction. The following is correct, albeit obtuse: 
%~~gen
%package Vars.Local;
% class NotInitVal {
%~~vis
\begin{xten}
static def main(r: Array[String](1)):void {
  val a : Int;
  a = r.size;
  val b : String;
  if (a == 5) b = "five?"; else b = "" + a + " args"; 
  // ... 
\end{xten}
%~~siv
%} }
%~~neg



\section{Fields}
\index{field}
\index{object!field}
\index{struct!field}
\index{class!field}

\begin{bbgrammar}
 FieldDeclarators    \: FieldDeclarator & (\ref{prod:FieldDeclarators})\\%<FROM #(prod:FieldDeclarators)#
    \| FieldDeclarators \xcd"," FieldDeclarator\\
 FieldDecl    \: Mods\opt FieldKeyword FieldDeclarators \xcd";" & (\ref{prod:FieldDecl})\\%<FROM #(prod:FieldDecl)#
    \| Mods\opt FieldDeclarators \xcd";"\\
 FieldDeclarator    \: Id HasResultType & (\ref{prod:FieldDeclarator})\\%<FROM #(prod:FieldDeclarator)#
    \| Id HasResultType\opt \xcd"=" VariableInitializer\\
 HasResultType    \: \xcd":" Type & (\ref{prod:HasResultType})\\%<FROM #(prod:HasResultType)#
    \| \xcd"<:" Type\\
 FieldKeyword    \: \xcd"val" & (\ref{prod:FieldKeyword})\\%<FROM #(prod:FieldKeyword)#
    \| \xcd"var"\\
 Mod    \: \xcd"abstract" & (\ref{prod:Mod})\\%<FROM #(prod:Mod)#
    \| Annotation\\
    \| \xcd"atomic"\\
    \| \xcd"final"\\
    \| \xcd"native"\\
    \| \xcd"private"\\
    \| \xcd"protected"\\
    \| \xcd"public"\\
    \| \xcd"static"\\
    \| \xcd"transient"\\
    \| \xcd"clocked"\\

\end{bbgrammar}

Like most other kinds of variables in X10, 
the fields of an object can be either \xcd`val` or \xcd`var`. 
Fields can be \xcd`static`,\xcd`global`, or \xcd`property`; see
\Sref{FieldDefinitions} and \Sref{PropertiesInClasses}.
Field declarations may have optional
initializer expressions, as for local variables, \Sref{local-variables}.
\xcd`var` fields without an initializer are initialized with the default value
of their type. \xcd`val` fields without an initializer must be initialized by
each constructor.


For \xcd`val` fields, as for \xcd`val` local variables, the type may be
omitted and inferred from the initializer type (\Sref{TypeInference}).
\xcd`var` files, like \xcd`var` local variables, must be declared with a type.



%%GRAM%% \begin{grammar}
%%GRAM%% FieldDeclaration
%%GRAM%%         \: FieldModifier\star \xcd"var" FieldDeclaratorsWithType \\&& ( \xcd"," FieldDeclaratorsWithType )\star \\
%%GRAM%%         \| FieldModifier\star \xcd"val" FieldDeclarators \\&& ( \xcd"," FieldDeclarators )\star \\
%%GRAM%%         \| FieldModifier\star FieldDeclaratorsWithType \\&& ( \xcd"," FieldDeclaratorsWithType )\star \\
%%GRAM%% FieldDeclarators
%%GRAM%%         \: FieldDeclaratorsWithType \\
%%GRAM%%         \: FieldDeclaratorWithInit \\
%%GRAM%% FieldDeclaratorId
%%GRAM%%         \: Identifier  \\
%%GRAM%% FieldDeclaratorWithInit
%%GRAM%%         \: FieldDeclaratorId Init \\
%%GRAM%%         \| FieldDeclaratorId ResultType Init \\
%%GRAM%% FieldDeclaratorsWithType
%%GRAM%%         \: FieldDeclaratorId ( \xcd"," FieldDeclaratorId )\star ResultType \\
%%GRAM%% FieldModifier \: Annotation \\
%%GRAM%%                 \| \xcd"static" \\ \| \xcd`property` \\ \| \xcd`global` \\
%%GRAM%% \end{grammar}
%%GRAM%% 
%%GRAM%% 

%%ACC%%  \section{Accumulator Variables}
%%ACC%%  
%%ACC%%  Accumulator variables allow the accumulation of partial results to produce a
%%ACC%%  final result.  For example, an accumulator variable could compute a running
%%ACC%%  sum, product, maximum, or minimum of a collection of numbers.  In particular,
%%ACC%%  many concurrent activites can accumulate safely into the {\em same} local
%%ACC%%  variable, without need for \Xcd{atomic} blocks or other explicit coordination.  
%%ACC%%  
%%ACC%%  An accumulator variable is associated with a {\em reducer}, which explains how
%%ACC%%  new partial values are accumulated.
%%ACC%%  
%%ACC%%  \subsection{Reducers}
%%ACC%%  
%%ACC%%  A notion of accumulation has two aspects: 
%%ACC%%  \begin{enumerate}
%%ACC%%  \item A {\bf zero} value, which is the initial value of the accumulator,
%%ACC%%        before any partial results have been included.  When accumulating a sum,
%%ACC%%        the zero value is \Xcd{0}; when accumulating a product, it is \Xcd{1}.
%%ACC%%  \item A {\bf combining function}, explaining how to combine two partial
%%ACC%%        accumulations into a whole one.  When accumulating a sum, partial sums
%%ACC%%        should be added together; for a product, they should be multiplied.  
%%ACC%%  \end{enumerate}
%%ACC%%  
%%ACC%%  In X10, this is represented as a value of type
%%ACC%%  \Xcd{x10.lang.Reducer[T]}: 
%%ACC%%  %~acc~gen
%%ACC%%  %package Vars.Notx10lang.Reducerererer;
%%ACC%%  %~acc~vis
%%ACC%%  \begin{xten}
%%ACC%%  struct Reducer[T](zero:T, apply: (T,T)=>T){}
%%ACC%%  \end{xten}
%%ACC%%  %~acc~siv
%%ACC%%  %
%%ACC%%  %~acc~neg
%%ACC%%  \noindent 
%%ACC%%  If \Xcd{r:Reducer[T]}, then \Xcd{r.zero} is the zero element, and
%%ACC%%  \Xcd{r(a,b)} --- which can also be written \Xcd{r.apply(a,b)} --- is the
%%ACC%%  combination of \Xcd{a} and \Xcd{b}.
%%ACC%%  
%%ACC%%  For example, the reducers for adding and multiplying integers are: 
%%ACC%%  %~acc~gen
%%ACC%%  %package Vars.Notx10lang.Reducererererererer;
%%ACC%%  %struct Reducer[T](zero:T, apply: (T,T)=>T){}
%%ACC%%  %class Example{
%%ACC%%  %~acc~vis
%%ACC%%  \begin{xten}
%%ACC%%  val summer = Reducer[Int](0, Int.+);
%%ACC%%  val producter = Reducer[Int](1, Int.*);
%%ACC%%  \end{xten}
%%ACC%%  %~acc~siv
%%ACC%%  %}
%%ACC%%  %~acc~neg
%%ACC%%  
%%ACC%%  
%%ACC%%  Reduction is guaranteed to be deterministic if the reducer is {\em
%%ACC%%  Abelian},\footnote{This term is borrowed from abstract algebra, where such a
%%ACC%%  reducer, together with its type, forms an Abelian monoid.}
%%ACC%%  that is, 
%%ACC%%  \begin{enumerate}
%%ACC%%  \item \Xcd{r.apply} is pure; that is, has no side effects;
%%ACC%%  \item \Xcd{r.apply} is commutative; that is, \Xcd{r(a,b) == r(b,a)} for all
%%ACC%%        inputs \Xcd{a} and \Xcd{b};
%%ACC%%  \item \Xcd{r.apply} is associative; that is, 
%%ACC%%        \Xcd{r(a,r(b,c)) == r(r(a,b),c)} for all \Xcd{a}, \Xcd{b}, and \Xcd{c}.
%%ACC%%  \item \Xcd{r.zero} is the identity element for \Xcd{r.apply}; that is, 
%%ACC%%        \Xcd{r(a, r.zero) == a}
%%ACC%%        for all \Xcd{a}.
%%ACC%%  \end{enumerate}
%%ACC%%  
%%ACC%%  
%%ACC%%  
%%ACC%%  
%%ACC%%  \Xcd{summer} and \Xcd{producter} satisfy all these conditions, and give
%%ACC%%  determinate reductions. The compiler does not require or check these, though.
%%ACC%%  
%%ACC%%  
%%ACC%%  \subsection{Accumulators}
%%ACC%%  
%%ACC%%  If \Xcd{r} is a  value of type \Xcd{Reducer[T]}, then an accumulator of type
%%ACC%%  \Xcd{T} using \Xcd{r} is declared as:
%%ACC%%  %~accTODO~gen
%%ACC%%  % package Vars.Accumulators.Basic.Little.Idea;
%%ACC%%  % class C[T]{
%%ACC%%  % static def example (r:Reducer[T]) {
%%ACC%%  %~accTODO~vis
%%ACC%%  \begin{xten}
%%ACC%%  acc(r) x : T;
%%ACC%%  acc(r) y; 
%%ACC%%  \end{xten}
%%ACC%%  %~accTODO~siv
%%ACC%%  %
%%ACC%%  %~accTODO~neg
%%ACC%%  The type declaration \Xcd{T} is optional; if specified, it must be the same
%%ACC%%  type that the reducer \Xcd{r} uses.
%%ACC%%  
%%ACC%%  \subsection{Sequential Use of Accumulators}
%%ACC%%  
%%ACC%%  The sequential use of accumulator variables is straightforward, and could be
%%ACC%%  done as easily without accumulators.  (The power of accumulators is in their
%%ACC%%  concurrent use, \Sref{ConcurrentUseOfAccumulators}.)
%%ACC%%  
%%ACC%%  A variable declared as \Xcd{acc(r) x:T;} is initialized to \Xcd{r.zero}.  
%%ACC%%  
%%ACC%%  Assignment of values of \Xcd{acc} variables has nonstandard semantics.
%%ACC%%  \Xcd{x = v;} causes the value \Xcd{r(v,x)} to be stored in \Xcd{x} --- in
%%ACC%%  particular, {\em not} the value of \Xcd{v}.
%%ACC%%  
%%ACC%%  Reading a value from an accumulator retrieves the current accumulation.
%%ACC%%  
%%ACC%%  For example, the sum and product of a list \Xcd{L} of integers can be computed
%%ACC%%  by: 
%%ACC%%  %~accTODO~gen
%%ACC%%  %package Vars.Accumulators.Are.For.Bisimulators;
%%ACC%%  % import java.util.*;
%%ACC%%  % class Example{
%%ACC%%  % static def example(L: List[Int]) {
%%ACC%%  %~accTODO~vis
%%ACC%%  \begin{xten}
%%ACC%%  val summer = Reducer[Int](0, Int.+);
%%ACC%%  val producter = Reducer[Int](1, Int.*);
%%ACC%%  acc(summer) sum;
%%ACC%%  acc(producter) prod;
%%ACC%%  for (x in L) {
%%ACC%%    sum = x;
%%ACC%%    prod = x;
%%ACC%%  }
%%ACC%%  x10.io.Console.OUT.println("Sum = " + sum + "; Product = " + prod);
%%ACC%%  \end{xten}
%%ACC%%  %~accTODO~siv
%%ACC%%  %
%%ACC%%  %~accTODO~neg
%%ACC%%  
%%ACC%%  
%%ACC%%  
%%ACC%%  \subsection{Concurrent Use of Accumulators}
%%ACC%%  \label{ConcurrentUseOfAccumulators}
%%ACC%%  \index{accumulator!and activities}
%%ACC%%  
%%ACC%%  Accumulator variables are restricted and synchronized in ways that make them
%%ACC%%  ideally suited for concurrent accumulation of data.   The {\em governing
%%ACC%%  activity} of an accumulator is the activity in which the \Xcd{acc} variable is
%%ACC%%  declared.  
%%ACC%%  
%%ACC%%  \begin{enumerate}
%%ACC%%  \item The governing activity can read the accumulator at any point that it has
%%ACC%%        no running sub-activities.  
%%ACC%%  \item Any activity that has lexical access to the accumulator can write to it.  
%%ACC%%        All
%%ACC%%        writes are performed atomically, without need for \Xcd{atomic} or other
%%ACC%%        concurrency control.
%%ACC%%  \end{enumerate}
%%ACC%%  
%%ACC%%  If the reducer is Abelian, this guarantees that \Xcd{acc} variables cannot
%%ACC%%  cause race conditions; the result of such a computation is determinate,
%%ACC%%  independent of the scheduling of activities. Read-read conflicts are
%%ACC%%  impossible, as only a single activity, the governing activity, can read the
%%ACC%%  \Xcd{acc} variable. Read-write conflicts are impossible, as reads are only
%%ACC%%  allowed at points where the only activity which can refer to the \Xcd{acc}
%%ACC%%  variable is the governing activity. Two activities may try to write the
%%ACC%%  \Xcd{acc} variable at the same time. The writes are performed atomically, so
%%ACC%%  they behave as if they happened in some (arbitrary) order---and, because the
%%ACC%%  reducer is Abelian, the order of writes doesn't matter.
%%ACC%%  
%%ACC%%  If the reducer is not Abelian---\eg, it is accumulating a string result by
%%ACC%%  concatenating a lot of partial strings together---the result is indeterminate.
%%ACC%%  However, because the accumulator operations are atomic, it will be the result
%%ACC%%  of {\em some} combination of the individual elements by the reduction
%%ACC%%  operation, \eg, the concatenation of the partial strings in {\em some} order.  
%%ACC%%  
%%ACC%%  
%%ACC%%  
%%ACC%%  For example, the following code computes triangle numbers {$\sum_{i=1}^{n}i$}
%%ACC%%  concurrently.\footnote{This program is highly inefficient. Even ignoring the
%%ACC%%    constant-time formula {$\sum_{i=1}^{n}i = \frac{n(n+1)}{2}$}, this program
%%ACC%%    incurs the cost of starting {$n$} activities and coordinating {$n$} accesses
%%ACC%%    to the accumulator. Accumulator variables are of most value in multi-place,
%%ACC%%    multi-core computations.}
%%ACC%%  
%%ACC%%  
%%ACC%%  %~accTODO~gen
%%ACC%%  %package Vars.Accumulator.Concurrency.Example;
%%ACC%%  %class Example{
%%ACC%%  %
%%ACC%%  %~accTODO~vis
%%ACC%%  \begin{xten}
%%ACC%%  def triangle(n:Int) {
%%ACC%%    val summer = Reducer[Int](0, Int.+);
%%ACC%%    acc(summer) sum; 
%%ACC%%    finish {
%%ACC%%      for([i] in 1..n) async {
%%ACC%%        sum = i;  // (A)
%%ACC%%      }
%%ACC%%      // (C)
%%ACC%%    }
%%ACC%%    return sum; // (B)
%%ACC%%  }
%%ACC%%  \end{xten}
%%ACC%%  %~accTODO~siv
%%ACC%%  %}
%%ACC%%  %~accTODO~neg
%%ACC%%  
%%ACC%%  The governing activity of the \Xcd{acc} variable \Xcd{sum} is the activity
%%ACC%%  including the body of \Xcd{triangle}.  It starts up \Xcd{n} sub-activities,
%%ACC%%  each of which adds one value to \Xcd{sum} at point \Xcd{(A)}.  Note that these
%%ACC%%  activities cannot {\em read} the value of \Xcd{sum}---only the governing
%%ACC%%  activity can do that---but they can update it.  
%%ACC%%  
%%ACC%%  At point \Xcd{(B)}, \Xcd{triangle} returns the value in \Xcd{sum}. It is
%%ACC%%  clear, from the \Xcd{finish} statement, that all sub-activities started by the
%%ACC%%  governing process have finished at this point. X10 forbids reading of
%%ACC%%  \Xcd{sum}, even by the governing process, at point \Xcd{(C)}, since
%%ACC%%  sub-activities writing into it could still be active when the governing
%%ACC%%  activity reaches this point.  The \Xcd{return sum;} statement could not be
%%ACC%%  moved to \Xcd{(C)}, which is good, because the program would be wrong if it
%%ACC%%  were there.
%%ACC%%  
%%ACC%%  
%%ACC%%  
%%ACC%%  
%%ACC%%  \subsubsection{Accumulators and Places}
%%ACC%%  \index{accumulator!and places} Activity variables can be read and written from
%%ACC%%  any place, without need for \Xcd{GlobalRef}s. We may spread the previous
%%ACC%%  computation out among all the available processors by simply sticking in an
%%ACC%%  \Xcd{at(...)} statement at point \Xcd{(D)}, without need for other
%%ACC%%  restructuring of the program.
%%ACC%%  
%%ACC%%  %~accTODO~gen
%%ACC%%  %package Vars.Accumulator.Concurrency.Example.Multiplacey;
%%ACC%%  %class Example{
%%ACC%%  %~accTODO~vis
%%ACC%%  \begin{xten}
%%ACC%%  def triangle(n:Int) {
%%ACC%%    val summer = Reducer[Int](0, Int.+);
%%ACC%%    acc(summer) sum; 
%%ACC%%    finish {
%%ACC%%      for([i] in 1..n) async 
%%ACC%%        at(Places.place(i % Places.MAX_PLACES) { //(D)
%%ACC%%          sum = i;  // (A)
%%ACC%%      }
%%ACC%%    }
%%ACC%%    return sum; // (B)
%%ACC%%  }
%%ACC%%  \end{xten}
%%ACC%%  %~accTODO~siv
%%ACC%%  %}
%%ACC%%  %~accTODO~neg
%%ACC%%  
%%ACC%%  \subsubsection{Accumulator Parameters}
%%ACC%%  \index{accumulator variables!as parameters}
%%ACC%%  \index{parameters!accumulator}
%%ACC%%  
%%ACC%%  Accumulators can be passed to methods and closures, by giving the keyword 
%%ACC%%  \Xcd{acc} instead of \Xcd{var} or \Xcd{val}.  Reducers are not specified; each
%%ACC%%  accumulator comes with its own reducer.  However, the type \Xcd{T} of the
%%ACC%%  accumulator {\em is} required.
%%ACC%%  
%%ACC%%  For example, the following method takes a list of numbers, and accumulates
%%ACC%%  those that are divisible by 2 in \Xcd{evens}, and those that are divisible by
%%ACC%%  3 in \Xcd{triples}: 
%%ACC%%  %~accTODO~gen
%%ACC%%  %package Vars.accumulators.parameters.oscillators.convulsitors.proximators;
%%ACC%%  %import x10.util.*;
%%ACC%%  %class Whatever {
%%ACC%%  %~accTODO~vis
%%ACC%%  \begin{xten}
%%ACC%%  static def split23(L:List[Int], acc evens:Int, acc triples:Int) {
%%ACC%%    for(n in L) {
%%ACC%%       if (n % 2 == 0) evens = n;
%%ACC%%       if (n % 3 == 0) triples = n;
%%ACC%%    }
%%ACC%%  }
%%ACC%%  static val summer = Reducer[Int](0, Int.+);
%%ACC%%  static val producter = Reducer[Int](1, Int.*);
%%ACC%%  static def sumEvenPlusProdTriple(L:List[Int]) {
%%ACC%%    acc(summer) sumEven;
%%ACC%%    acc(producter) prodTriple;
%%ACC%%    split23(L, sumEven, prodTriple);
%%ACC%%    return sumEven + prodTriple;
%%ACC%%  }
%%ACC%%  \end{xten}
%%ACC%%  %~accTODO~siv
%%ACC%%  %}
%%ACC%%  %~accTODO~neg
%%ACC%%  
%%ACC%%  \subsection{Indexed Accumulators}
%%ACC%%  \index{accumulator!indexed}
%%ACC%%  \index{accumulator!array}
%%ACC%%  
%%ACC%%  
%%ACC%%  \noo{Define this!}
%%ACC%%  
%%ACC%%  %~accTODO~gen
%%ACC%%  % package Vars.Indexed.Accumulators;
%%ACC%%  %~accTODO~vis
%%ACC%%  \begin{xten}
%%ACC%%  class BoolAccum implements SelfAccumulator[Boolean, Int] {
%%ACC%%    var sumTrue = 0, sumFalse = 0;
%%ACC%%    def update(k:Boolean, v:Int) { 
%%ACC%%       if (k) sumTrue += k; else sumFalse += k;
%%ACC%%    }
%%ACC%%    def update(ks:Array[Boolean]{rail}, vs:Array[Int]{ks.size == vs.size}) {
%%ACC%%       for([i] in ks.region) update(ks(i), vs(i));  }
%%ACC%%    
%%ACC%%  }
%%ACC%%  \end{xten}
%%ACC%%  %~accTODO~siv
%%ACC%%  %
%%ACC%%  %~accTODO~neg

\chapter{Names and packages}
\label{packages} \index{name}\index{package}

\section{Names}

An X10 program consists largely of giving names to entities, and then
manipulating the entities by their names. The entities involved may be
compile-time constructs, like packages, types and classes, or run-time
constructs, like numbers and strings and objects.  

X10 names can be {\em simple names}, which look like identifiers: \xcd`vj`,
\xcd`x10`, \xcd`AndSoOn`. Or, they can be {\em qualified names}, which are
sequences of two or more identifiers separated by dots: \xcd`x10.lang.String`, 
\xcd`somePack.someType`, \xcd`a.b.c.d.e.f`.   Some entities have only simple
names; some have both simple and qualified names.

Every declaration that introduces a name has a {\em scope}: the region of the
program in which the named entity can be referred to by a simple name.  
In some cases, entities may be referred to by qualified names outside of their
scope.  \Eg, a \xcd`public` class \xcd`C` defined in package \xcd`p` can be
referred to by the simple name \xcd`C` inside of \xcd`p`, or by the qualified
name \xcd`p.C` from anywhere.  

Many sorts of entities have {\em members}.  Packages have classes, structs,
and interfaces as members.  Those, in turn, have fields, methods, types, and
so forth as members.  The member \xcd`x` of an entity named \xcd`E` (as a
simple or qualified name) has the name \xcd`E.x`; it may also have other
names.  

\subsection{Shadowing}
\index{shadowing}
\index{namespace}

One declaration $d$ may {\em shadow} another declaration $d'$ in part of the
scope of $d'$, if $d$ and $d'$ declare variables with the same simple name $n$.
When $d$ shadows $d'$, a use of $n$ might refer to $d$'s $n$ (unless some
$d''$ in turn shadows $d$), but will never refer to $d'$'s $n$.

X10 has four namespaces:
\begin{itemize}
\item {\bf Types:} for classes, interfaces, structs, and defined types.
\item {\bf Values:} for \xcd`val`- and \xcd`var`-bound variables; fields;
      and formal parameters of all sorts.
\item {\bf Methods:} for methods of classes, interfaces, and structs.
\item {\bf Packages:} for packages.
\end{itemize}

A declaration $d$ in one namespace, binding a name $n$ to an entity $e$,
shadows all other declarations of that name $n$ in scope at the point where
$d$ is declared. This shadowing is in effect for the entire scope of $d$.  
Declarations in different namespaces do not shadow each other.
Thus, a local variable declaration may shadow a field declaration, but not a
class declaration.

Declarations which only introduce qualified names --- in X10, this is only
package declarations --- cannot shadow anything.

The rules for shadowing of imported names are given in \Sref{sect:ImportDecl}.

\subsection{Hiding}
\index{hiding}
\label{sect:Hiding}

Shadowing is ubiquituous in X10. Another, and considerably rarer, way that one
definition of a given simpl ename can render another definition of the same
name unavailable is {\em hiding}. If a class \xcd`Super` defines a field named
\xcd`x`, and a subclass \xcd`Sub` of \xcd`Super` also defines a field named
\xcd`x`, then, for \xcd`Sub`s, references to the \xcd`x` field get \xcd`Sub`'s
\xcd`x` rather than \xcd`Super`'s. In this case, \xcd`Super`'s \xcd`x` is said
to be {\em hidden}.

Hiding is technically different from shadowing, because hiding applies in more
circumstances: a use of class \xcd`Sub`, such as \xcd`sub.x`, may involve
hiding of name \xcd`x`, though it could not involve shadowing of \xcd`x`
because \xcd`x` is need not be declared as a name at that point.

\subsection{Obscuring}
\index{obscuring}
\label{sect:Obscuring}

The third way in which a definition of a simple name may become unavailable is
{\em obscuring}. This well-named concept says that, if \xcd`n` can be
interpreted as two or more of: a variable, a type, and a package, then it will
be interpreted as a variable if that is possible, or a type if it cannot be
interpreted as a variable. In this case, the unavailable interpretations are
{\em obscured}. 

\begin{ex}
In the \xcd`example` method of the following code, both a struct and a local
variable are named \xcd`eg`.  Following the obscuring rules, The call
\xcd`eg.ow()` in the first \xcd`assert` uses the variable rather than the struct.  
As the second \xcd`assert` demonstrates, the struct can be accessed through
its fully-qualified name.   Note that none of this would have happened if the
coder had followed the convention that structs have capitalized names,
\xcd`Eg`, and variables have lower-case ones, \xcd`eg`. 

%~~gen ^^^ Packages5t5g
% NOTEST
%~~vis
\begin{xten}
package obscuring;
struct eg {
   static def ow()= 1;
   static struct Bite {
      def ow() = 2;
   }
   def example() {
       val eg = Bite();
       assert eg.ow() == 2;
       assert obscuring.eg.ow() == 1;
     }
}

\end{xten}
%~~siv
% class Hook{ def run() { (eg()).example(); return true; } }
%~~neg

\end{ex}

Due to obscuring, it may be impossible to refer to a type or a package via a
simple name in some circumstances.  Obscuring does not block qualified names.



\subsection{Ambiguity and Disambiguation}

Neither simple nor qualified names are necessarily unique.  There can be, in
general, many entities that have the same name.  This is perfectly ordinary,
and, when done well, considered good programming practice.   Various forms of
{\em disambiguation} are used to tell which entity is meant by a given name;
\eg, methods with the same name may be disambiguated by the types of their
arguments (\Sref{sect:MethodResolution}).

\begin{ex}
In the following example, there are three static methods with 
qualified name \xcd`DisambEx.Example.m`; they can be disambiguated by their
different arguments.   Inside the body of the third, the simple name \xcd`i`
refers to both the \xcd`Int` formal of \xcd`m`, and to the static method 
\xcd`DisambEx.Example.i`.  
%~~gen ^^^ Packages9e6r
%~~vis
\begin{xten}
package DisambEx; 
class Example {
  static def m() = 1;
  static def m(Boolean) = 2;
  static def i() = 3;
  static def m(i:Int) {
    if (i > 10) {
      return i() + 1;
    }
    return i;
  }
  static def example() {
    assert m() == 1;
    assert m(true) == 2;
    assert m(20) == 4;
  }
}
\end{xten}
%~~siv
% class Hook{ def run() { Example.example(); return true; } }
%~~neg
\end{ex}



\section{Access Control}
\index{public}\index{protected}\index{private}

X10 allows programmers {\em access control}, that is, the ability to determine
statically where identifiers of most sorts are visible.  In particular, X10
allows {\em information hiding}, wherein certain data can be accessed from
only limited parts of the program. 

There are four access control modes: 
\xcd"public" , \xcd"protected", \xcd"private"
and uninflected package-specific scopes, much like those of Java. 
Most things can be public or private; a few things (\eg, class members) can
also be protected or package-scoped.  

Accessibility of one X10 entity (package, container, member, etc.) from within
a package or container is defined as follows: 
\begin{itemize}
\item Packages are always accessible.
\item If a container \xcd`C` is public, and, if it is inside of another
      container \xcd`D`,
      container \xcd`D` is accessible, then \xcd`C` is accessible.  
\item A member \xcd`m` of a container \xcd`C` is accessible from within
      another  \xcd`E`
      if \xcd`C` is
      accessible, and: 
      \begin{itemize}
      \item \xcd`m` is declared \xcd`public`; or
      \item \xcd`C` is an interface; or
      \item \xcd`m` is declared \xcd`protected`, and either the access is from
            within the same package that \xcd`C` is defined in, or from within
            the body of a subclass of \xcd`C` (but see
            \Sref{sect:protected-details} for some fine points); or
      \item \xcd`m` is declared \xcd`private`, and the access is from within
            the top-level class which contains the definition of \xcd`C` ---
            which may be \xcd`C` itself, or, if \xcd`C` is a nested container, an
            outer class around \xcd`C`; or
      \item \xcd`m` has no explicit class declaration (hence using the
            implicit ``package''-level access control), and the access occurs
            from the same package that \xcd`C` is declared in.
      \end{itemize}
\end{itemize}

\subsection{Details of \xcd`protected`}
\label{sect:protected-details}

\xcd`protected` access has a few fine points. 
Within the body of a subclass \xcd`D` of the class \xcd`C` containing
the definition of a protected member \xcd`m`, 

\begin{itemize}

\item An access \xcd`e.fld` to a field, or \xcd`e.m(...)` to a method, is
      permitted precisely when the type of \xcd`e` is either \xcd`D` or a
      subtype of \xcd`D`.  
For example, the access to \xcd`that.f` in the following code is acceptable, but
the access to \xcd`xhax.f` is not.  
%~~gen ^^^ Packages9q4y
% package Packages9q4y;
%~~vis
\begin{xten}
class C {
  protected var f : Int = 0;
}
class X extends C {}
class D extends C {
  def usef(that:D, xhax:X) {
     this.f += that.f; 
     // ERROR: this.f += xhax.f;
}
\end{xten}
%~~siv
%
%~~neg

\limitation{Some X10 compilers improperly allow access to {\tt xhax} -- as,
indeed, some Java compilers do, despite Java having the analogous rule.
However, X10 allows all permitted accesses, so the workaround is trivial.}

\item An access through a qualified name \xcd`Q.N` is permitted precisely when
      the type of \xcd`Q` is \xcd`D` or a subtype of \xcd`D`. 

\end{itemize}

Qualified access to a protected constructor is subtle.  Let \xcd`C` be a class
with a \xcd`protected` constructor $c$, and let \xcd`S` be the innermost
class containing a use $u$ of $c$.  There are three cases: 

\begin{itemize}
\item Super superclass construction invocations, \xcd`super(...)` or
      \xcd`E.super(...)`, are permitted.
\item Anonymous class instance creations, \xcd`new C(...){...}`
      and \xcd`E.new C(...){...}`, are
      permitted.
\item No other accesses are permitted. 
\end{itemize}

\section{Packages}

A package is a named collection of top-level type declarations, \viz, class,
interface, and struct declarations. Package names are sequences of
identifiers, like \xcd`x10.lang` and \xcd`com.ibm.museum`. The multiple names
are simply a convenience, though there is a tenuous relationship between
packages \xcd`a`, \xcd`a.b`, and \xcd`a.c`.   Packages can be accessed by
name from anywhere: a package may contain private elements, but may not itself
be private. 

Packages and protection modifiers determine which top-level names can be used
where. Only the \xcd`public` members of package \xcd`pack.age` can be accessed
outside of \xcd`pack.age` itself.  
%~~gen~~Stimulus ^^^ Stimulus
% NOTEST 
%~~vis
\begin{xten}
package pack.age;
class Deal {
  public def make() {}
}
public class Stimulus {
  private def taxCut() = true;
  protected def benefits() = true;
  public def jobCreation() = true;
  /*package*/ def jumpstart() = true;
}
\end{xten}
%~~siv
% 
%
%~~neg

The class \xcd`Stimulus` can be referred to from anywhere outside of
\xcd`pack.age` by its full name of \xcd`pack.age.Stimulus`, or can be imported
and referred to simply as \xcd`Stimulus`.  The public \xcd`jobCreation()`
method of a \xcd`Stimulus` can be referred to from anywhere as well; the other
methods have smaller visibility.  The non-\xcd`public` class \xcd`Deal` cannot
be used from outside of \xcd`pack.age`.  



\subsection{Name Collisions}

It is a static error for a package to have two members with the same name. For
example, package \xcd`pack.age` cannot define two classes both named
\xcd`Crash`, nor a class and an interface with that name.

Furthermore, \xcd`pack.age` cannot define a member \xcd`Crash` if there is
another package named \xcd`pack.age.Crash`, nor vice-versa. (This prohibition
is the only actual relationship between the two packages.)  This prevents the
ambiguity of whether \xcd`pack.age.Crash` refers to the class or the package.  
Note that the naming convention that package names are lower-case and package
members are capitalized prevents such collisions.


\section{{\tt import} Declarations}
\label{sect:ImportDecl}
\index{import}

Any public member of a package can be referred to from anywhere through a
fully-qualified name: \xcd`pack.age.Stimulus`.    

Often, this is too awkward.  X10 has two ways to allow code outside of a class
to refer to the class by its short name (\xcd`Stimulus`): single-type imports
and on-demand imports.   

Imports of either kind appear at the start of the file, immediately after the
\xcd`package` directive if there is one; their scope is the whole file.

\subsection{Single-Type Import}

The declaration \xcd`import ` {\em TypeName} \xcd`;` imports a single type
into the current namespace.  The type it imports must be a fully-qualified
name of an extant type, and it must either be in the same package (in which
case the \xcd`import` is redundant) or be declared \xcd`public`.  

Furthermore, when importing \xcd`pack.age.T`, there must not be another type
named \xcd`T` at that point: neither a  \xcd`T` declared in \xcd`pack.age`,
nor a \xcd`inst.ant.T` imported from some other package.

The declaration \xcd`import E.n;`, appearing in file $f$ of a package named
\xcd`P`, shadows the following types named \xcd`n` when they appear in $f$: 
\begin{itemize}
\item Top-level types named \xcd`n` appearing in other files of \xcd`P`, and 
\item Types named \xcd`n` imported by automatic imports
      (\Sref{sect:AutomaticImport}) in $f$.
\end{itemize}
\noindent


\subsection{Automatic Import}
\label{sect:AutomaticImport}

The automatic import \xcd`import pack.age.*;`, loosely, imports all the public
members of \xcd`pack.age`.  In fact, it does so somewhat carefully, avoiding
certain errors that could occur if it were done naively.  Types defined in the
current package, and those imported by single-type imports, shadow those
imported by automatic imports.   Names automatically imported never shadow any
other names.



\subsection{Implicit Imports}

The packages \xcd`x10.lang` and \xcd`x10.array` are automatically imported in all files
without need for further specification.

%%BARD-HERE



\section{Conventions on Type Names}

%##(TypeName PackageName
\begin{bbgrammar}
%(FROM #(prod:TypeName)#)
            TypeName \: Id & (\ref{prod:TypeName}) \\
                    \| TypeName \xcd"." Id \\
%(FROM #(prod:PackageName)#)
         PackageName \: Id & (\ref{prod:PackageName}) \\
                    \| PackageName \xcd"." Id \\
\end{bbgrammar}
%##)


While not enforced by the compiler, classes and interfaces
in the \Xten{} library follow the following naming conventions.
Names of types---including classes,
type parameters, and types specified by type definitions---are in
CamelCase and begin with an uppercase letter.  (Type variables are often
single capital letters, such as \xcd`T`.)
For backward
compatibility with languages such as C and \java{}, type
definitions are provided to allow primitive types
such as \xcd"int" and \xcd"boolean" to be written in lowercase.
Names of methods, fields, value properties, and packages are in camelCase and
begin with a lowercase letter. 
Names of \xcd"static val" fields are in all uppercase with words
separated by \xcd"_"'s.

\chapter{Interfaces}
\label{XtenInterfaces}\index{interface}

An interface specifies signatures for zero or more public methods, property
methods,
\xcd`static val`s, 
classes, structs, interfaces, types
and an invariant. 

The following puny example illustrates all these features: 
% TODO Well, it would if there weren't a compiler bug in the way.
%~~gen ^^^Interfaces_static_val
% package Interfaces_static_val;
% 
%~~vis
\begin{xten}
interface Pushable{prio() != 0} {
  def push(): void;
  static val MAX_PRIO = 100;
  abstract class Pushedness{}
  struct Pushy{}
  interface Pushing{}
  static type Shove = Int;
  property text():String;
  property prio():Int;
}
class MessageButton(text:String)
  implements Pushable{self.prio()==Pushable.MAX_PRIO} {
  public def push() { 
    x10.io.Console.OUT.println(text + " pushed");
  }
  public property text() = text;
  public property prio() = Pushable.MAX_PRIO;
}
\end{xten}
%~~siv
%
%~~neg
\noindent
\xcd`Pushable` defines two property methods, one normal method, and a static
value.  It also 
establishes an invariant, that \xcd`prio() != 0`. 
\xcd`MessageButton` implements a constrained version of \xcd`Pushable`,
\viz\ one with maximum priority.  It
defines the \xcd`push()` method given in the interface, as a \xcd`public`
method---interface methods are implicitly \xcd`public`.

\limitation{X10 may not always detect that type invariants of interfaces are
satisfied, even when they obviously are.}

A container---a class or struct---can {\em implement} an interface,
typically by having all the methods and property methods that the interface
requires, and by providing a suitable \xcd`implements` clause in its definition.

A variable may be declared to be of interface type.  Such a variable has all
the property and normal methods declared (directly or indirectly) by the
interface; 
nothing else is statically available.  Values of any concrete type which
implement the interface may be stored in the variable.  

\begin{ex}
The following code puts two quite different objects into the variable
\xcd`star`, both of which satisfy the interface \xcd`Star`.
%~~gen ^^^ Interfaces6l3f
% package Interfaces6l3f;
%~~vis
\begin{xten}
interface Star { def rise():void; }
class AlphaCentauri implements Star {
   public def rise() {}
}
class ElvisPresley implements Star {
   public def rise() {}
}
class Example {
   static def example() {
      var star : Star;
      star = new AlphaCentauri();
      star.rise();
      star = new ElvisPresley();
      star.rise();
   }
}
\end{xten}
%~~siv
%
%~~neg
\end{ex}
An interface may extend several interfaces, giving
X10 a large fraction of the power of multiple inheritance at a tiny fraction
of the cost.

\begin{ex}
%~~gen ^^^ Interfaces6g4u
% package Interfaces6g4u;
%~~vis
\begin{xten}
interface Star{}
interface Dog{}
class Sirius implements Dog, Star{}
class Lassie implements Dog, Star{}
\end{xten}
%~~siv
%
%~~neg
\end{ex}


\section{Interface Syntax}

\label{DepType:Interface}

%##(NormalInterfaceDecl TypeParamsI TypeParamI Guard ExtendsInterfaces InterfaceBody InterfaceMemberDecl
\begin{bbgrammar}
%(FROM #(prod:NormalInterfaceDecl)#)
 NormalInterfaceDecl \: Mods\opt \xcd"interface" Id TypeParamsI\opt Properties\opt Guard\opt ExtendsInterfaces\opt InterfaceBody & (\ref{prod:NormalInterfaceDecl}) \\
%(FROM #(prod:TypeParamsI)#)
         TypeParamsI \: \xcd"[" TypeParamIList \xcd"]" & (\ref{prod:TypeParamsI}) \\
%(FROM #(prod:TypeParamI)#)
          TypeParamI \: Id & (\ref{prod:TypeParamI}) \\
                     \| \xcd"+" Id \\
                     \| \xcd"-" Id \\
%(FROM #(prod:Guard)#)
               Guard \: DepParams & (\ref{prod:Guard}) \\
%(FROM #(prod:ExtendsInterfaces)#)
   ExtendsInterfaces \: \xcd"extends" Type & (\ref{prod:ExtendsInterfaces}) \\
                     \| ExtendsInterfaces \xcd"," Type \\
%(FROM #(prod:InterfaceBody)#)
       InterfaceBody \: \xcd"{" InterfaceMemberDecls\opt \xcd"}" & (\ref{prod:InterfaceBody}) \\
%(FROM #(prod:InterfaceMemberDecl)#)
 InterfaceMemberDecl \: MethodDecl & (\ref{prod:InterfaceMemberDecl}) \\
                     \| PropertyMethodDecl \\
                     \| FieldDecl \\
                     \| ClassDecl \\
                     \| InterfaceDecl \\
                     \| TypeDefDecl \\
                     \| \xcd";" \\
\end{bbgrammar}
%##)


\noindent
The invariant associated with an interface is the conjunction of the
invariants associated with its superinterfaces and the invariant
defined at the interface. 



A class \xcd"C"  implements an interface \xcd"I" if
\begin{itemize}
\item \xcd`I`, or a subtype of \xcd`I`, appears in the \xcd`implements` list
      of \xcd`C`, 
\item \xcd`C`'s property methods include all the property methods  of \xcd"I",
\item Each method \xcd`m` defined by \xcd`I` is also a method of \xcd`C` --
      with the {\em  \xcd`public`} modifier added.   These methods may be
      \xcd`abstract` if \xcd`C` is \xcd`abstract`.
\end{itemize}


If \xcd`C` implements \xcd`I`, then the class invariant
(\Sref{DepType:ClassGuardDef}) for \xcd`C`,   $\mathit{inv}($\xcd"C"$)$, implies
the class invariant for \xcd`I`, $\mathit{inv}($\xcd"I"$)$.  That is, if the
interface \xcd`I` specifies some requirement, then every class \xcd`C` that
implements it satisfies that requirement.

\section{Access to Members}

All interface members are \xcd`public`, whether or not they are declared
public.  There is little purpose to non-public methods of an interface; they
would specify that implementing classes and structs have methods that cannot
be seen.

\section{Property Methods}

An interface may declare \xcd`property` methods.  All non-\xcd`abstract`
containers implementing such an interface must provide all the \xcd`property`
methods specified.  

\section{Field Definitions}
\index{interface!field definition in}

An interface may declare a \xcd`val` field, with a value.  This field is implicitly
\xcd`public static val`.  In particular, it is {\em not} an instance field. 
%~~gen ^^^ Interfaces10
% package Interface.Field;
%~~vis
\begin{xten}
interface KnowsPi {
  PI = 3.14159265358;
}
\end{xten}
%~~siv
%
%~~neg

Classes and structs implementing such an interface get the interface's fields as
\xcd`public static` fields.  Unlike  methods, there is no need
for the implementing class to declare them. 
%~~gen ^^^ Interfaces20
% package Interface.Field.Two;
% interface KnowsPi {PI = 3.14159265358;}
%~~vis
\begin{xten}
class Circle implements KnowsPi {
  static def area(r:Double) = PI * r * r;
}
class UsesPi {
  def circumf(r:Double) = 2 * r * KnowsPi.PI;
}
\end{xten}
%~~siv
%
%~~neg

\subsection{Fine Points of Fields}

If two parent interfaces give different static fields of the same name, 
those fields must be referred to by qualified names.
%~~gen ^^^ Interface_field_name_collision
% 
%~~vis
\begin{xten}
interface E1 {static val a = 1;}
interface E2 {static val a = 2;}
interface E3 extends E1, E2{}
class Example implements E3 {
  def example() = E1.a + E2.a;
}
\end{xten}
%~~siv
%
%~~neg

If the {\em same} field \xcd`a` is inherited through many paths, there is no need to
disambiguate it:
%~~gen ^^^ Interfaces_multi
% package Interfaces.Mult.Inher.Field;
%~~vis
\begin{xten}
interface I1 { static val a = 1;} 
interface I2 extends I1 {}
interface I3 extends I1 {}
interface I4 extends I2,I3 {}
class Example implements I4 {
  def example() = a;
}
\end{xten}
%~~siv
%
%~~neg

The initializer of a field in an interface may be any expression.  It is
evaluated under the same rules as a \xcd`static` field of a class. 

\begin{ex}
In this example, a local class (\Sref{sect:LocalClasses}) \xcd`B` is defined,
with an inner interface \xcd`I`.  The field \xcd`V` of \xcd`I` uses a variable
\xcd`n` which is global to \xcd`B`.   In this case it is a truly baroque way
to bind a \xcd`val`, but other uses are nontrivial.

%~~gen ^^^ Interfaces3l4a
% package Interfaces3l4a;
%~~vis
\begin{xten}
class TheOne {
  static val ONE = 1;
  interface WelshOrFrench {
    val UN = 1;
  }
  static class Onesome implements WelshOrFrench {
    static def example() {
      assert UN == ONE;
    }
  }
}
\end{xten}
%~~siv
% class Hook{ def run() {TheOne.Onesome.example(); return true;}}
%~~neg
\end{ex}

\section{Generic Interfaces}

Interfaces, like classes and structs, can have type parameters.  
The discussion of generics in \Sref{TypeParameters} applies to interfaces,
without modification.

\begin{ex}
%~~gen ^^^ Interfaces7n1z
% package Interfaces7n1z;
%~~vis
\begin{xten}
interface ListOfFuns[T,U] extends x10.util.List[(T)=>U] {}
\end{xten}
%~~siv
%
%~~neg

\end{ex}

\section{Interface Inheritance}

The {\em direct superinterfaces} of a non-generic interface \xcd`I` are the interfaces
(if any) mentioned in the \xcd`extends` clause of \xcd`I`'s definition.
If \xcd`I`  is generic, the direct superinterfaces are of an instantiation of
\xcd`I` are the corresponding instantiations of those interfaces.
A {\em superinterface} of \xcd`I` is either \xcd`I` itself, or a direct
superinterface of a superinterface of \xcd`I`, and similarly for generic
interfaces.    

\xcd`I` inherits the members of all of its superinterfaces. Any class or
struct that has \xcd`I` in its \xcd`implements` clause also implements all of
\xcd`I`'s superinterfaces. 






\section{Members of an Interface}

The members of an interface \xcd`I` are the union of the following sets: 
\begin{enumerate}
\item All of the members appearing in \xcd`I`'s declaration;
\item All the members of its direct super-interfaces, except those which are
      hidden (\Sref{sect:Hiding}) by \xcd`I`
\item The members of \xcd`Any`.
\end{enumerate}

Overriding for instances is defined as for classes, \Sref{MethodOverload}

\chapter{Classes}
\label{XtenClasses}\index{class}
\label{ReferenceClasses}

\section{Principles of X10 Objects}\label{XtenObjects}\index{object}
\index{class}

\subsection{Basic Design}

Objects are instances of classes: the most common and most powerful sort of
value in X10.  The other kinds of values, structs and functions, are more
specialized, better in some circumstances but not in all.
\xcd"x10.lang.Object" is the most general class; all other classes inherit
from it, directly or indirectly. 


Classes are structured in a single-inheritance code
hierarchy.   They may have any or all of these features: 
\begin{itemize}
\item Implementing any number of interfaces;
\item Static and instance \xcd`val` fields; 
\item Instance \xcd`var` fields; 
\item Static and instance methods;
\item Constructors;
\item Properties;
\item Static and instance nested containers.
\end{itemize}


\Xten{} objects do not have locks associated with them.
Programmers should use atomic blocks (\Sref{AtomicBlocks}) for mutual
exclusion and clocks (\Sref{XtenClocks}) for sequencing multiple parallel
operations.

An object exists in a single location: the place that it was created.  One
place cannot use or even directly refer to an object in a different place.   A
special type, \Xcd{GlobalRef[T]}, allows explicit cross-place references. 

The basic operations on objects are:
\begin{itemize}

{}\item Field access (\Sref{FieldAccess}). 
The static and instance fields of an object can be retrieved; \xcd`var` fields
can be set.  

{}\item Method invocation (\Sref{MethodInvocation}).  
Static and instance methods of an object can be invoked.

{}\item Casting (\Sref{ClassCast}) and instance testing with \xcd`instanceof`
(\Sref{instanceOf}) Objects can be cast or type-tested.  

\item The equality operators \xcd"==" and \xcd"!="
Objects can be compared for equality with the \Xcd{==} operation.  This checks
object {\em identity}: two objects are \Xcd{==} iff they are the same object.

\end{itemize}

  
 
\subsection{Class Declaration Syntax}

The {\em class declaration} has a list of type parameters, a list of
properties, a constraint (the {\em class invariant}), a single superclass,
zero or more interfaces that it implements, and a class body containing the
the definition of fields, properties, methods, and member types. Each such
declaration introduces a class type (\Sref{ReferenceTypes}).

%##(NormalClassDecl TypeParamsWithVariance TypeParamWithVarianceList Properties PropertyList Property WhereClause Super Interfaces InterfaceTypeList ClassBody ClassBodyDecls ClassMemberDecl
\begin{bbgrammar}
%(FROM #(prod:NormalClassDecl)#)
     NormalClassDecl \: Mods\opt \xcd"class" Id TypeParamsWithVariance\opt Properties\opt WhereClause\opt Super\opt Interfaces\opt ClassBody & (\ref{prod:NormalClassDecl}) \\
%(FROM #(prod:TypeParamsWithVariance)#)
TypeParamsWithVariance \: \xcd"[" TypeParamWithVarianceList \xcd"]" & (\ref{prod:TypeParamsWithVariance}) \\
%(FROM #(prod:TypeParamWithVarianceList)#)
TypeParamWithVarianceList \: TypeParamWithVariance & (\ref{prod:TypeParamWithVarianceList}) \\
                    \| TypeParamWithVarianceList \xcd"," TypeParamWithVariance \\
%(FROM #(prod:Properties)#)
          Properties \: \xcd"(" PropertyList \xcd")" & (\ref{prod:Properties}) \\
%(FROM #(prod:PropertyList)#)
        PropertyList \: Property & (\ref{prod:PropertyList}) \\
                    \| PropertyList \xcd"," Property \\
%(FROM #(prod:Property)#)
            Property \: Annotations\opt Id ResultType & (\ref{prod:Property}) \\
%(FROM #(prod:WhereClause)#)
         WhereClause \: DepParams & (\ref{prod:WhereClause}) \\
%(FROM #(prod:Super)#)
               Super \: \xcd"extends" ClassType & (\ref{prod:Super}) \\
%(FROM #(prod:Interfaces)#)
          Interfaces \: \xcd"implements" InterfaceTypeList & (\ref{prod:Interfaces}) \\
%(FROM #(prod:InterfaceTypeList)#)
   InterfaceTypeList \: Type & (\ref{prod:InterfaceTypeList}) \\
                    \| InterfaceTypeList \xcd"," Type \\
%(FROM #(prod:ClassBody)#)
           ClassBody \: \xcd"{" ClassBodyDecls\opt \xcd"}" & (\ref{prod:ClassBody}) \\
%(FROM #(prod:ClassBodyDecls)#)
      ClassBodyDecls \: ClassBodyDecl & (\ref{prod:ClassBodyDecls}) \\
                    \| ClassBodyDecls ClassBodyDecl \\
%(FROM #(prod:ClassMemberDecl)#)
     ClassMemberDecl \: FieldDecl & (\ref{prod:ClassMemberDecl}) \\
                    \| MethodDecl \\
                    \| PropertyMethodDecl \\
                    \| TypeDefDecl \\
                    \| ClassDecl \\
                    \| InterfaceDecl \\
                    \| \xcd";" \\
\end{bbgrammar}
%##)




\section{Fields}
\label{FieldDefinitions}
\index{object!field}
\index{field}

Objects may have {\em instance fields}, or simply {\em fields} (called
``instance variables'' in C++ and Smalltalk, and ``slots'' in CLOS): places to
store data that is pertinent to the object.  Fields, like variables, may be
mutable (\xcd`var`) or immutable (\xcd`val`).  

Class may have {\em static fields}, which store data pertinent to the
entire class of objects.  
See \Sref{StaticInitialization} for more information.

No two fields of the same class may have the same name.  A field may have the
same name as a method, although for fields of functional type there is a
subtlety (\Sref{sect:disambiguations}).  

\subsection{Field Initialization}
\index{field!initialization}
\index{initialization!of field}

Fields may be given values via {\em field initialization expressions}:
\xcd`val f1 = E;` and \xcd`var f2 : Int = F;`. Other fields of \xcd`this` may
be referenced, but only those that {\em precede} the field being initialized.
For example, the following is correct, but would not be if the fields were
reversed:

%~~gen ^^^ Classes10
%package Classes_field_init_expr_a;
%~~vis
\begin{xten}
class Fld{
  val a = 1;
  val b = 2+a;
}
\end{xten}
%~~siv
%
%~~neg


\subsection{Field hiding}
\index{hiding}
\index{field|hiding}


A subclass that defines a field \xcd"f" hides any field \xcd"f"
declared in a superclass, regardless of their types.  The
superclass field \xcd"f" may be accessed within the body of
the subclass via the reference \xcd"super.f".

With inner classes, it is occasionally necessary to 
write \xcd`Cls.super.f` to get at a hidden field \xcd`f` of an outer class
\xcd`Cls`. 

\begin{ex}
The \xcd`f` field in \xcd`Sub` hides the \xcd`f` field in \xcd`Super`
The \xcd`superf`` method provides access to the \xcd`f` field in \xcd`Super`.
%~~gen ^^^ Classes20
% package classes.fields.primus;
%~~vis
\begin{xten}
class Super{ 
  public val f = 1; 
}
class Sub extends Super {
  val f = true;
  def superf() : Int = super.f; // 1
}
\end{xten}
%~~siv
% class Hook { def run() { return (new Sub()).superf() == 1; }} 
%~~neg
\end{ex}

%~~gen ^^^ Classes30
% package classes.fields.secundus; 
% NOTEST
%~~vis
\begin{xten}
class A {
   val f = 3;
}
class B extends A {
   val f = 4;
   class C extends B {
      // C is both a subclass and inner class of B
      val f = 5;
       def example() {
         assert f         == 5 : "field of C";
         assert super.f   == 4 : "field of superclass";
         assert B.this.f  == 4 : "field of outer instance";
         assert B.super.f == 3 : "super.f of outer instance";
       }
    }
}
\end{xten}
%~~siv
% class Hook { def run() { ((new B()).new C()).example(); return true; } }
%~~neg


\subsection{Field qualifiers}
\label{FieldQualifier}
\index{qualifier!field}
\index{field!qualifier}

The behavior of a field may be changed by a field qualifier, such as
\xcd`static` or \xcd`transient`.  


\subsubsection{\Xcd{static} qualifier}
\index{field!static}

A \xcd`val` field may be declared to be {\em static}, as described in
\Sref{FieldDefinitions}. 

\subsubsection{\Xcd{transient} Qualifier}
\label{TransientFields}
\index{transient}
\index{field!transient}

A field may be declared to be {\em transient}.  Transient fields are excluded
from the deep copying that happens when information is sent from place to
place in an \Xcd{at} statement.    The value of a transient field of a copied
object is the default value of its type, regardless of the value of the field
in the original.  If the type of a field has no
default value, it cannot be marked \Xcd{transient}.
%~~gen ^^^ Classes40
% package Classes.Transient.Example;
% KNOWNFAIL-at
%~~vis
\begin{xten}
class Trans { 
   val copied = "copied";
   transient var transy : String = "a very long string";
   def example() {
      at (here; this) { // causes copying of 'this'
         assert(this.copied.equals("copied"));
         assert(this.transy == null);
      }
   }
}
\end{xten}
%~~siv
%
%~~neg


\section{Properties}
\label{PropertiesInClasses}
\index{property}

The properties of an object (or struct) are  public \xcd`val` fields
usable at compile time in constraints.\footnote{In many cases, a 
\xcd`val` field can be upgraded to a \xcd`property`, which 
entails no compile-time or runtime cost.  Some cannot be, \eg, in cases where
cyclic structures of \xcd`val` fields are required.} 
For example,  every array has a \xcd`rank` telling
how many subscripts it takes.  User-defined classes can have whatever
properties are desired. 

Properties are defined in parentheses, after the name of the class.  They are
given values by the \xcd`property` command in constructors.

%~~gen ^^^ Classes50
% package Classes.Toss.Freedom.Disk2;
%~~vis
\begin{xten}
class Proper(t:Int) {
  def this(t:Int) {property(t);}
}
\end{xten}
%~~siv
%
%~~neg




It is a static error for a class
defining a property \xcd"x: T" to have a subclass class that defines
a property or a field with the name \xcd"x".


A property \xcd`x:T` induces a field with the same name and type, 
as if defined with: 
%~~gen ^^^ Classes60
% package Classes.For.Masses.Of.NevermindTheRest;
% class Exampll[T] {
%~~vis
\begin{xten}
public val x : T;
\end{xten} 
%~~siv
% def this(y:T) { x=y; }
% }
%~~neg
\noindent It also defines a nullary getter method, 
%~~gen ^^^ Classes70
% package Classes_nullary_getter_a;
% class Exampllll[T] {
% public val x : T;
% def this(y:T) { x=y; }
%~~vis
\begin{xten}
public final def x()=x;
\end{xten}
%~~siv
%}
%~~neg





\index{property!initialization}
Properties are initialized by the invocation of a special \Xcd{property}
statement: 
\begin{xten}
property(e1,..., en);
\end{xten}
The number and types of arguments to the \Xcd{property} statement must match
the number and types of the properties in the class declaration.  
Every constructor of a class with properties must invoke \xcd`property(...)`
precisely once; it is a static error if X10 cannot prove that this holds.
The requirement to use the \xcd`property` statement means that all properties
must be given values at the same time.  

By construction, the graph whose nodes are values and whose edges are
properties is acyclic.  \Eg, there cannot be values \xcd`a` and \xcd`b` with
properties \xcd`c` and \xcd`d` such that \xcd`a.c == b` and \xcd`b.d == a`.


\subsection{Properties and Fields}

A container with a property named \xcd`p`, or a nullary property method named
\xcd`p()`, cannot have a field named \xcd`p` --- either defined in that
container, or inherited from a superclass.

\subsection{Acyclicity of Properties}
\index{properties!acyclic}

X10 has certain restrictions that, ultimately, require that properties are
simpler than their containers.  For example, \xcd`class A(a:A){}` is not
allowed.  
Formally, this requirement is that there is  a total order $\preceq$ 
on all classes and
structs such that, if $A$ extends $B$, then $A \prec B$, and
if $A$ has a property of type $B$, then $A \prec B$, where $A \prec B$ means
$A \preceq B$ and $A \ne B$.   
For example, the preceding class \xcd`A` is ruled out because we would need
\xcd`A`$\prec$\xcd`A`, which violates the definition of $\prec$.
The programmer need not (and cannot) specify
$\preceq$, and rarely need worry about its existence.  

Similarly, 
the type of a property may not simply be a type parameter.  
For example, \xcd`class A[X](x:X){}` is illegal.





\section{Methods}
\index{method}
\index{signature}
\index{method!signature}
\index{method!instance}
\index{method!static}

As is common in object-oriented languages, objects can have {\em methods}, of
two sorts.  {\em Static methods} are functions, conceptually associated with a
class and defined in its namespace.  {\em Instance methods} are parameterized
code bodies associated with an instance of the class, which execute with
convenient access to that instance's fields. 

Each method has a {\em signature}, telling what arguments it accepts, what
type it returns, and what precondition it requires. Method definitions may be
overridden by subclasses; the overriding definition may have a declared return
type that is a subtype of the return type of the definition being overridden.
Multiple methods with the same name but different signatures may be provided
\index{overloading}
\index{polymorphism}
on a class (called ``overloading'' or ``ad hoc polymorphism''). Methods may be
declared \Xcd{public}, \Xcd{private}, \Xcd{protected}, or given default package-level access
rights.

%##(MethMods MethodDecl TypeParams FormalParams FormalParamList HasResultType MethodBody
\begin{bbgrammar}
%(FROM #(prod:MethMods)#)
            MethMods \: Mods\opt & (\ref{prod:MethMods}) \\
                    \| MethMods \xcd"property"  \\
                    \| MethMods Mod \\
%(FROM #(prod:MethodDecl)#)
          MethodDecl \: MethMods \xcd"def" Id TypeParams\opt FormalParams WhereClause\opt HasResultType\opt Offers\opt MethodBody & (\ref{prod:MethodDecl}) \\
                    \| MethMods \xcd"operator" TypeParams\opt \xcd"(" FormalParam  \xcd")" BinOp \xcd"(" FormalParam  \xcd")" WhereClause\opt HasResultType\opt Offers\opt MethodBody \\
                    \| MethMods \xcd"operator" TypeParams\opt PrefixOp \xcd"(" FormalParam  \xcd")" WhereClause\opt HasResultType\opt Offers\opt MethodBody \\
                    \| MethMods \xcd"operator" TypeParams\opt \xcd"this" BinOp \xcd"(" FormalParam  \xcd")" WhereClause\opt HasResultType\opt Offers\opt MethodBody \\
                    \| MethMods \xcd"operator" TypeParams\opt \xcd"(" FormalParam  \xcd")" BinOp \xcd"this" WhereClause\opt HasResultType\opt Offers\opt MethodBody \\
                    \| MethMods \xcd"operator" TypeParams\opt PrefixOp \xcd"this" WhereClause\opt HasResultType\opt Offers\opt MethodBody \\
                    \| MethMods \xcd"operator" \xcd"this" TypeParams\opt FormalParams WhereClause\opt HasResultType\opt Offers\opt MethodBody \\
                    \| MethMods \xcd"operator" \xcd"this" TypeParams\opt FormalParams \xcd"=" \xcd"(" FormalParam  \xcd")" WhereClause\opt HasResultType\opt Offers\opt MethodBody \\
                    \| MethMods \xcd"operator" TypeParams\opt \xcd"(" FormalParam  \xcd")" \xcd"as" Type WhereClause\opt Offers\opt MethodBody \\
                    \| MethMods \xcd"operator" TypeParams\opt \xcd"(" FormalParam  \xcd")" \xcd"as" \xcd"?" WhereClause\opt HasResultType\opt Offers\opt MethodBody \\
                    \| MethMods \xcd"operator" TypeParams\opt \xcd"(" FormalParam  \xcd")" WhereClause\opt HasResultType\opt Offers\opt MethodBody \\
%(FROM #(prod:TypeParams)#)
          TypeParams \: \xcd"[" TypeParamList \xcd"]" & (\ref{prod:TypeParams}) \\
%(FROM #(prod:FormalParams)#)
        FormalParams \: \xcd"(" FormalParamList\opt \xcd")" & (\ref{prod:FormalParams}) \\
%(FROM #(prod:FormalParamList)#)
     FormalParamList \: FormalParam & (\ref{prod:FormalParamList}) \\
                    \| FormalParamList \xcd"," FormalParam \\
%(FROM #(prod:HasResultType)#)
       HasResultType \: \xcd":" Type & (\ref{prod:HasResultType}) \\
                    \| \xcd"<:" Type \\
%(FROM #(prod:MethodBody)#)
          MethodBody \: \xcd"=" LastExp \xcd";" & (\ref{prod:MethodBody}) \\
                    \| \xcd"=" Annotations\opt \xcd"{" BlockStatements\opt LastExp \xcd"}" \\
                    \| \xcd"=" Annotations\opt Block \\
                    \| Annotations\opt Block \\
                    \| \xcd";" \\
\end{bbgrammar}
%##)


\index{parameter!var}
\index{parameter!val}
A formal parameter may have a \xcd"val" or \xcd"var"
% , or \Xcd{ref}
modifier; \xcd`val` is the default.
The body of the method is executed in an environment in which 
each formal parameter corresponds to a local variable (\xcd`var` iff the
formal parameter is \xcd`var`)
and is initialized with the value of the actual parameter.

\subsection{Generic Instance Methods}
\index{method!generic instance}

\limitationx{}
In X10, an instance method may be generic: 
%~~gen ^^^ Classes1b7z
% package Classes1b7z;
% NOTEST
%~~vis
\begin{xten}
class Example {
  def example[T](t:T) = "I like " + t;
}
\end{xten}
%~~siv
%
%~~neg

However, the C++ back end does not currently support generic virtual instance
methods like \xcd`example`.  It does allow generic instance methods which are
\xcd`final` or \xcd`private`, and it does allow generic static methods.  


\subsection{Method Guards}
\label{MethodGuard}
\index{method!guard}
\index{guard!on method}

Often, a method will only make sense to invoke under certain
statically-determinable conditions.  For example, \xcd`example(x)` is only
well-defined when \xcd`x != null`, as \xcd`null.toString()` throws a null
pointer exception: 
%~~gen ^^^ Classes80
% package Classes.methodwithconstraintthingie;
% 
%~~vis
\begin{xten}
class Example {
   var f : String = "";
   def example(x:Object){x != null} = {
      this.f = x.toString();
   }
}
\end{xten}
%~~siv
%
%~~neg
\noindent
(We could have used a constrained type \xcd`Object{self!=null}` for \xcd`x`
instead; in
most cases it is a matter of personal preference or convenience of expression
which one to use.) 

The requirement of having a method guard is that callers must demonstrate to
the X10
compiler that the guard is satisfied.  (As usual with static constraint
checking, there is no runtime cost.  Indeed, this code can be more efficient
than usual, as it is statically provable that \xcd`x != null`.)
This may require a cast: 
%~~gen ^^^ Classes90
% package Classes.methodguardnadacastthingie;
% 
% class Example {var f : String = ""; def example(x:Object){x != null} = {this.f = x.toString();}}
% class Eyample {
%~~vis
\begin{xten}
  def exam(e:Example, x:Object) {
    if (x != null) 
       e.example(x as Object{x != null});
    // WRONG: if (x != null) e.example(x);
  }
\end{xten}
%~~siv
%}
%~~neg



The guard \xcd`{c}` 
in a guarded method 
\xcd`def m(){c} = E;`
specifies a constraint \xcd"c" on the
properties of the class \xcd"C" on which the method is being defined. The
method, in effect, only exists  for those instances of \xcd"C" which satisfy
\xcd"c".  It is 
illegal for code to invoke the method on objects whose static type is
not a subtype of \xcd"C{c}".

Specifically: 
    the compiler checks that every method invocation
    \xcdmath"o.m(e$_1$, $\dots$, e$_n$)"
    is type correct. Each argument
    \xcdmath"e$_i$" must have a
    static type \xcdmath"S$_i$" that is a subtype of the declared type
    \xcdmath"T$_i$" for the $i$th
    argument of the method, and the conjunction of the constraints on the
    static types 
    of the arguments must entail the guard in the parameter list
    of the method.

    The compiler checks that in every method invocation
    \xcdmath"o.m(e$_1$, $\dots$, e$_n$)"
    the static type of \xcd"o", \xcd"S", is a subtype of \xcd"C{c}", where the method
    is defined in class \xcd"C" and the guard for \xcd"m" is equivalent to
    \xcd"c".

    Finally, if the declared return type of the method is
    \xcd"D{d}", the
    return type computed for the call is
    \xcdmath"D{a: S; x$_1$: S$_1$; $\dots$; x$_n$: S$_n$; d[a/this]}",
    where \xcd"a" is a new
    variable that does not occur in
    \xcdmath"d, S, S$_1$, $\dots$, S$_n$", and
    \xcdmath"x$_1$, $\dots$, x$_n$" are the formal
    parameters of the method.


\limitation{
Using a reference to an outer class, \xcd`Outer.this`, in a constraint, is not supported.
}


\subsection{Property methods}
\index{method!property}
\index{property method}

Property methods are methods that can be evaluated in constraints.  
For example, the \xcd`eq()` method below tells if the \xcd`x` and \xcd`y`
properties are equal; the \xcd`is(z)` method tells if they are both equal to
\xcd`z`.  These can be used in constraints, as illustrated in the
\xcd`example()` method.
%~~gen ^^^ Classes100
%package Classes.PropertyMethods;
%~~vis
\begin{xten}
class Example(x:Int, y:Int) {
   def this(x:Int, y:Int) { property(x,y); }
   property eq() = (x==y);
   property is(z:Int) = x==z && y==z;
   def example( a : Example{eq()}, b : Example{is(3)} ) {}
}
\end{xten}
%~~siv
%
%~~neg


A method declared with the modifier \xcd"property" may be used
in constraints.  A property method declared in a class must have
a body and must not be \xcd"void".  The body of the method must
consist of only a single \xcd"return" statement or a single
expression.  It is a static error if the expression cannot be
represented in the constraint system.   Property methods may be \xcd`abstract`
in \xcd`abstract` classes, but are implicitly \xcd`final` in
non-\xcd`abstract` classes. 

The expression may contain invocations of other property methods. It is the
responsibility of the programmer to ensure that the evaluation of a property
terminates at compile-time, otherwise the type-checker will not terminate and
the program will fail to compile in a potentially most unfortunate way.

Property methods in classes are implicitly \xcd"final"; they cannot be
overridden.  It is a static error if a superclass has a property method with a
given signature, and a subclass has a method or property method with the same
signature.   It is a static error if a superclass has a property with some
name \xcd`p`, and a subclass has a nullary method of any kind (instance,
static, or property) also named \xcd`p`. 

% We should fix it so that a property p (or a property method p())  in the super class  cannot be shadowed by a field p in a subclass, or overridden by a method p() in the subclass.


A nullary property method definition may omit the formal parameters and
the \xcd"def" keyword.  That is, the following are equivalent:



%~~gen ^^^ Classes110
% package classes.waifsome1;
% class Waif(rect:Boolean, onePlace:Place, zeroBased:Boolean) {
%~~vis
\begin{xten}
property def rail(): Boolean = rect && onePlace == here && zeroBased;
\end{xten}
%~~siv
%}
%~~neg
and
%~~gen ^^^ Classes120
% package classes.waifsome2;
% class Waif(rect:Boolean, onePlace:Place, zeroBased:Boolean) {
%~~vis
\begin{xten}
property rail(): Boolean = rect && onePlace == here && zeroBased;
\end{xten}
%~~siv
%}
%~~neg

Similarly, nullary property methods can be inspected in constraints without
\xcd`()`. If \xcd`ob`'s type has a property \xcd`p`, then \xcd`ob.p` is that
property. Otherwise, if it has a nullary property method \xcd`p()`, \xcd`ob.p`
is equivalent to \xcd`ob.p()`. As a consequence, if the type provides both a
property \xcd`p` and a nullary method \xcd`p()`, then the property can be
accessed as \xcd`ob.p` and the method as \xcd`ob.p()`.\footnote{This only
applies to nullary property methods, not nullary instance methods.  Nullary
property methods perform limited computations, have no side effects, and
always return the same value, since
they have to be expressed in the constraint sublanguage.  In this sense, a
nullary property method does not behave hugely different from a property.
indeed, a compilation scheme which cached the value of the property method
would all but erase the distinction.  Other methods may
have more behavior, \eg, side effects, so we keep the \xcd`()` to make it
clear that a method call is potentially large.
}

%~~longexp~~`~~` ^^^ Classes130
% package classes.not.weasels;
% class Waif(rect:Boolean, onePlace:Place, zeroBased:Boolean) {
%   def this(rect:Boolean, onePlace:Place, zeroBased:Boolean) 
%          :Waif{self.rect==rect, self.onePlace==onePlace, self.zeroBased==zeroBased}
%          = {property(rect, onePlace, zeroBased);}
%   property rail(): Boolean = rect && onePlace == here && zeroBased;
%   static def zoink() {
%      val w : Waif{
%~~vis
\xcd`w.rail`, with either definition above, 
% }= new Waif(true, here, true);
% }}
%~~pxegnol
is equivalent to 
%~~longexp~~`~~` ^^^ Classes140
% package classes.not.ferrets;
% class Waif(rect:Boolean, onePlace:Place, zeroBased:Boolean) {
%   def this(rect:Boolean, onePlace:Place, zeroBased:Boolean) 
%          :Waif{self.rect==rect, self.onePlace==onePlace, self.zeroBased==zeroBased}
%          = {property(rect, onePlace, zeroBased);}
%   property rail(): Boolean = rect && onePlace == here && zeroBased;
%   static def zoink() {
%      val w : Waif{
%~~vis
\xcd`w.rail()`
% }= new Waif(true, here, true);
% }}
%~~pxegnol




\subsection{Method overloading, overriding, hiding, shadowing and obscuring}
\label{MethodOverload}
\index{method!overloading}



The definitions of method overloading, overriding, hiding, shadowing and
obscuring in \Xten{} are familiar from languages such as Java, modulo the
following considerations motivated by type parameters and dependent types.



Two or more methods of a class or interface may have the same
name if they have a different number of type parameters, or
they have formal parameters of different types.  \Eg, the following is legal: 

%~~gen ^^^ Classes150
% package Classes.Mful;
%~~vis
\begin{xten}
class Mful{
   def m() = 1;
   def m[T]() = 2;
   def m(x:Int) = 3;
   def m[T](x:Int) = 4;
}
\end{xten}
%~~siv
%
%~~neg

\XtenCurrVer{} does not permit overloading based on constraints. That is, the
following is {\em not} legal, although either method definition individually
is legal:
\begin{xten}
   def n(x:Int){x==1} = "one";
   def n(x:Int){x!=1} = "not";
\end{xten}


The definition of a method declaration \xcdmath"m$_1$" ``having the same signature
as'' a method declaration \xcdmath"m$_2$" involves identity of types. 

\noo{The following few paragraphs need to be stared at carefully.  I think
they are contradictory and/or wrong.}

The {\em constraint erasure} of a type \xcdmath"T" is defined as follows.
The constraint erasure of  (a)~a class, interface or struct type \xcdmath"T" is 
\xcdmath"T"; (b)~a type \xcdmath"T{c}" is the constraint erasure of 
\xcdmath"T"; (b)~a type \xcdmath"T[S$_1$,$\ldots$,S$_n$]" 
is \xcdmath"T'[S$_1$',$\ldots$,S$_n$']" where each primed type is the erasure of 
the corresponding unprimed type.
 Two methods are said to have {\em equivalent signatures} if (a) they have the
 same number of type parameters, 
(b) they have the same number of formal (value) parameters, and (c)
for each formal parameter the constraint erasure of its types are equivalent. It is a
compile-time error for there to be two methods with the same name and
equivalent signatures in a class (either defined in that class or in a
superclass), unless the signatures are identical and one of the methods is
defined in a superclass (in which case the superclass's method is overridden
by the subclass's).

 

A class \xcd"C" may not have two non-identical but equivalent 
declarations for
a method named 
\xcd"m"---either 
defined at \xcd"C" or inherited:
\begin{xtenmath}
def m[X$_1$, $\dots$, X$_m$](v$_1$: T$_1$, $\dots$, v$_n$: T$_n$){tc}: T {...}
def m[X$_1$, $\dots$, X$_m$](v$_1$: S$_1$, $\dots$, v$_n$: S$_n$){sc}: S {...}
\end{xtenmath}
\noindent
if it is the case that the constraint erasures of the types \xcdmath"T$_1$",
\dots, \xcdmath"T$_n$" are
equivalent to the constraint erasures of the types \xcdmath"S$_1$, $\dots$, T$_n$"
respectively.



In addition, the guard of a overriding method must be 
no stronger than the guard of the overridden method.   This
ensures that any virtual call to the method
satisfies the guard of the callee.


  If a class \xcd"C" overrides a method of a class or interface
  \xcd"B", the guard of the method in \xcd"B" must entail
  the guard of the method in \xcd"C".


A class \xcd"C" inherits from its direct superclass and superinterfaces all
their methods visible according to the access modifiers
of the superclass/superinterfaces that are not hidden or overridden. A method \xcdmath"M$_1$" in a class
\xcd"C" overrides
a method \xcdmath"M$_2$" in a superclass \xcd"D" if
\xcdmath"M$_1$" and \xcdmath"M$_2$" have the same signature with constraints erased.
Methods are overriden on a signature-by-signature basis.

\section{Constructors}
\index{constructor}

Instances of classes are created by the \xcd`new` expression: \\
%##(ClassInstCreationExp
\begin{bbgrammar}
%(FROM #(prod:ClassInstCreationExp)#)
ClassInstCreationExp \: \xcd"new" TypeName TypeArguments\opt \xcd"(" ArgumentList\opt \xcd")" ClassBody\opt & (\ref{prod:ClassInstCreationExp}) \\
                    \| \xcd"new" TypeName \xcd"[" Type \xcd"]" \xcd"[" ArgumentList\opt \xcd"]" \\
                    \| Primary \xcd"." \xcd"new" Id TypeArguments\opt \xcd"(" ArgumentList\opt \xcd")" ClassBody\opt \\
                    \| AmbiguousName \xcd"." \xcd"new" Id TypeArguments\opt \xcd"(" ArgumentList\opt \xcd")" ClassBody\opt \\
\end{bbgrammar}
%##)

This constructs a new object, and calls some code, called a {\em constructor},
to initialize the newly-created object properly.

Constructors are defined like methods, except that they are named \xcd`this`
and ordinary methods may not be.    The content of a constructor body has
certain capabilities (\eg, \xcd`val` fields of the object may be initialized)
and certain restrictions (\eg, most methods cannot be called); see
\Sref{ObjectInitialization} for the details.

\begin{ex}

The following class provides two constructors.  The unary constructor 
\xcd`def this(b : Int)` allows initialization of the \xcd`a` field to an 
arbitrary value.  The nullary constructor \xcd`def this()` gives it a default
value of 10.  The \xcd`example` method illustrates both of these calls.


%~~gen ^^^ ClassesCtor10
% package ClassesCtor10;
%~~vis
\begin{xten}
class C {
  public val a : Int;
  def this(b : Int) { a = b; } 
  def this()        { a = 10; }
  static def example() {
     val two = new C(2);
     assert two.a == 2;
     val ten = new C(); 
     assert ten.a == 10;
  }
}
\end{xten}
%~~siv
%
%~~neg
\end{ex}

\subsection{Automatic Generation of Constructors}
\index{constructor!generated}

Classes that have no constructors written in the class declaration are
automatically given a constructor which sets the class properties and does
nothing else. If this automatically-generated constructor is not valid (\eg,
if the class has \xcd`val` fields that need to be initialized in a
constructor), the class has no constructor, which is a static error.

\begin{ex}
The following class has no explicit constructor.
Its implicit constructor is 
\xcd`def this(x:Int){property(x);}`
This implicit constructor is valid, and so is the class. 
%~~gen ^^^ ClassesCtor20
% package ClassesCtor20;
%~~vis
\begin{xten}
class C(x:Int) {
  static def example() {
    val c : C = new C(4);
    assert c.x == 4;
  }
}
\end{xten}
%~~siv
%
%~~neg
\noindent 


The following class has the same default constructor.  However, that
constructor does not initialize \xcd`d`, and thus is invalid.  This 
class does not compile; it needs an explicit constructor.
%~~gen ^^^ ClassCtor30_MustFailCompile
% NOCOMPILE
%~~vis
\begin{xten}
// THIS CODE DOES NOT COMPILE
class Cfail(x:Int) {
  val d: Int;
  static def example() {
    val wrong = new Cfail(40);
  }
}
\end{xten}
%~~siv
%
%~~neg


\end{ex}

\subsection{Calling Other Constructors}

The {\em first} statement of a constructor body may be a call of the form 
\xcd`this(a,b,c)` or \xcd`super(a,b,c)`.  The former will execute the body of
the matching constructor of the current class; the latter, of the superclass. 
This allows a measure of abstraction in constructor definitions; one may be
defined in terms of another.

\begin{ex}
The following class has two constructors.  \xcd`new Ctors(123)` constructs a
new \xcd`Ctors` object with parameter 123.  \xcd`new Ctors()` constructs one
whose parameter has a default value of 100: 
%~~gen ^^^ Classes5q6q
% package Classes5q6q;
%~~vis
\begin{xten}
class Ctors {
  val a : Int;
  def this(a:Int) { this.a = a; }
  def this() {
    this(100);
  }
}
\end{xten}
%~~siv
%
%~~neg
\end{ex}

In the case of a class which implements \xcd`operator ()` 
--- or any other constructor and application with the same signature --- 
this can be ambiguous.  If \xcd`this()` appears as the first statement of a
constructor body, it could, in principle, mean either a constructor call or an
operator evaluation.   This ambiguity is resolved so that \xcd`this()` always
means the constructor invocation.  If, for some reason, it is necessary to
invoke an application operator early in a constructor, precede it with a dummy
statement, such as \xcd`if(false);`  

\section{Static initialization}
\label{StaticInitialization}
\index{initialization!static}
The \Xten{} runtime implements the following procedure to ensure
reliable initialization of the static state of classes.


Execution (of an entire X10 program) commences with a single thread executing
the 
\emph{initialization} phase of an \Xten{} computation at place \Xcd{0}. This
phase must complete successfully before the body of the \Xcd{main} method is
executed.

The initialization phase should be thought of as if it is implemented in
the following fashion. (The implementation may do something more
efficient as long as it is faithful to this semantics.)

\begin{xten}
Within the scope of a new finish
for every static field f of every class C 
   (with type T and initializer e):
async {
  val l = e; 
  ateach (Dist.makeUnique()) {
     assign l to the static f field of 
         the local C class object;
     mark the f field of the local C 
         class object as initialized;
  }
}
\end{xten}

During this phase, any read of a static field \Xcd{C.f} (where \Xcd{f} is of type \Xcd{T})
is replaced by a call to the method \Xcd{C.read\_f():T} defined on class \Xcd{C}
as follows

\begin{xten}
def read_f():T {
   when (initialized(C.f)){};
   return C.f;
}
\end{xten}
 

If all these activities terminate normally, all static fields have values of
their declared types, 
and the \Xcd{finish} terminates normally. If
any activity throws an exception, the \Xcd{finish} throws an
exception. Since no user code is executing which can catch exceptions
thrown by the finish, such exceptions are printed on the console, and
computation aborts.

If the activities deadlock, the implementation deadlocks.

In all cases, the main method is executed only once all static fields
have been initialized correctly.

Since static state is immutable and is replicated to all places via 
the initialization phase as described above, it can be accessed from
any place.



\section{User-Defined Operators}
\label{sect:operators}
\index{operator}
\index{operator!user-defined}

It is often convenient to have methods named by symbols rather than words.
For example, suppose that we wish to define a \xcd`Poly` class of
polynomials -- for the sake of illustration, single-variable polynomials with
\xcd`Int` coefficients.  It would be very nice to be able to manipulate these
polynomials by the usual operations: \xcd`+` to add, \xcd`*` to multiply,
\xcd`-` to subtract, and \xcd`p(x)` to compute the value of the polynomial at
argument \xcd`x`.  We would like to write code thus: 
%~~gen ^^^ Classes160
% package Classes.In.Poly101;
% // Integer-coefficient polynomials of one variable.
% class Poly {
%   public val coeff : Array[Int](1);
%   public def this(coeff: Array[Int](1)) { this.coeff = coeff;}
%   public def degree() = coeff.size-1;
%   public def a(i:Int) = (i<0 || i>this.degree()) ? 0 : coeff(i);
%
%   public static operator (c : Int) as Poly = new Poly([c]);
%
%   public operator this(x:Int) {
%     val d = this.degree();
%     var s : Int = this.a(d);
%     for( i in 1 .. this.degree() ) {
%        s = x * s + a(d-i);
%     }
%     return s;
%   }
%
%   public operator this + (p:Poly) =  new Poly(
%      new Array[Int](
%         Math.max(this.coeff.size, p.coeff.size),
%         (i:Int) => this.a(i) + p.a(i)
%      ));
%   public operator this - (p:Poly) = this + (-1)*p;
%
%   public operator this * (p:Poly) = new Poly(
%      new Array[Int](
%        this.degree() + p.degree() + 1,
%        (k:Int) => sumDeg(k, this, p)
%        )
%      );
%
%
%   public operator (n : Int) + this = (n as Poly) + this;
%   public operator this + (n : Int) = (n as Poly) + this;
%
%   public operator (n : Int) - this = (n as Poly) + (-1) * this;
%   public operator this - (n : Int) = ((-n) as Poly) + this;
%
%   public operator (n : Int) * this = new Poly(
%      new Array[Int](
%        this.degree()+1,
%        (k:Int) => n * this.a(k)
%      ));
%   private static def sumDeg(k:Int, a:Poly, b:Poly) {
%      var s : Int = 0;
%      for( i in 0 .. k ) s += a.a(i) * b.a(k-i);
%        // x10.io.Console.OUT.println("sumdeg(" + k + "," + a + "," + b + ")=" + s);
%      return s;
%      };
%   public final def toString() = {
%      var allZeroSoFar : Boolean = true;
%      var s : String ="";
%      for( i in 0..this.degree() ) {
%        val ai = this.a(i);
%        if (ai == 0) continue;
%        if (allZeroSoFar) {
%           allZeroSoFar = false;
%           s = term(ai, i);
%        }
%        else
%           s +=
%              (ai > 0 ? " + " : " - ")
%             +term(ai, i);
%      }
%      if (allZeroSoFar) s = "0";
%      return s;
%   }
%   private final def term(ai: Int, n:Int) = {
%      val xpow = (n==0) ? "" : (n==1) ? "x" : "x^" + n ;
%      return (ai == 1) ? xpow : "" + Math.abs(ai) + xpow;
%   }
%
%   public static def Main(ss:Array[String](1)) = main(ss);
%


%~~vis
\begin{xten}
  public static def main(Array[String](1)):void {
     val X = new Poly([0,1]);
     val t <: Poly = 7 * X + 6 * X * X * X; 
     val u <: Poly = 3 + 5*X - 7*X*X;
     val v <: Poly = t * u - 1;
     for( i in -3 .. 3) {
       x10.io.Console.OUT.println(
         "" + i + "	X:" + X(i) + "	t:" + t(i) 
         + "	u:" + u(i) + "	v:" + v(i)
         );
     }
  }

\end{xten}
%~~siv
%}
%~~neg

Writing the same code with method calls, while possible, is far less elegant: 
%~~gen ^^^ Classes170

%package Classes.In.Remedial.Poly101;
% // Integer-coefficient polynomials of one variable.
% class UglyPoly {
%   public val coeff : Array[Int](1);
%   public def this(coeff: Array[Int](1)) { this.coeff = coeff;}
%   public def degree() = coeff.size-1;
%   public  def  a(i:Int) = (i<0 || i>this.degree()) ? 0 : coeff(i);
%
%   public static operator (c : Int) as UglyPoly = new UglyPoly([c]);
%
%   public def apply(x:Int) {
%     val d = this.degree();
%     var s : Int = this.a(d);
%     for( i in 1 .. this.degree() ) {
%        s = x * s + a(d-i);
%     }
%     return s;
%   }
%
%   public operator this + (p:UglyPoly) =  new UglyPoly(
%      new Array[Int](
%         Math.max(this.coeff.size, p.coeff.size),
%         (i:Int) => this.a(i) + p.a(i)
%      ));
%   public operator this - (p:UglyPoly) = this + (-1)*p;
%
%   public operator this * (p:UglyPoly) = new UglyPoly(
%      new Array[Int](
%        this.degree() + p.degree() + 1,
%        (k:Int) => sumDeg(k, this, p)
%        )
%      );
%
%
%   public operator (n : Int) + this = (n as UglyPoly) + this;
%   public operator this + (n : Int) = (n as UglyPoly) + this;
%
%   public operator (n : Int) - this = (n as UglyPoly) + (-1) * this;
%   public operator this - (n : Int) = ((-n) as UglyPoly) + this;
%
%   public operator (n : Int) * this = new UglyPoly(
%      new Array[Int](
%        this.degree()+1,
%        (k:Int) => n * this.a(k)
%      ));
%   private static def sumDeg(k:Int, a:UglyPoly, b:UglyPoly) {
%      var s : Int = 0;
%      for( i in 0 .. k ) s += a.a(i) * b.a(k-i);
%        // x10.io.Console.OUT.println("sumdeg(" + k + "," + a + "," + b + ")=" + s);
%      return s;
%      };
%   public final def toString() = {
%      var allZeroSoFar : Boolean = true;
%      var s : String ="";
%      for( i in 0..this.degree() ) {
%        val ai = this.a(i);
%        if (ai == 0) continue;
%        if (allZeroSoFar) {
%           allZeroSoFar = false;
%           s = term(ai, i);
%        }
%        else
%           s +=
%              (ai > 0 ? " + " : " - ")
%             +term(ai, i);
%      }
%      if (allZeroSoFar) s = "0";
%      return s;
%   }
%   private final def term(ai: Int, n:Int) = {
%      val xpow = (n==0) ? "" : (n==1) ? "x" : "x^" + n ;
%      return (ai == 1) ? xpow : "" + Math.abs(ai) + xpow;
%   }
%
%   def mult(p:UglyPoly) = this * p;
%   def mult(n:Int) = n * this;
%   def plus(p:UglyPoly) = this + p;
%   def plus(n:Int) = n + this;
%   def minus(p:UglyPoly) = this - p;
%   def minus(n:Int) = this - n;
%   static def const(n:Int) = n as UglyPoly;
%
%
%~~vis
\begin{xten}
  public static def uglymain() {
     val X = new UglyPoly([0,1]);
     val t <: UglyPoly = X.mult(7).plus(
                          X.mult(X).mult(X).mult(6));  
     val u <: UglyPoly = const(3).plus(
                           X.mult(5)).minus(X.mult(X).mult(7));
     val v <: UglyPoly = t.mult(u).minus(1);
     for( i in -3 .. 3) {
       x10.io.Console.OUT.println(
         "" + i + "	X:" + X.apply(i) + "	t:" + t.apply(i) 
          + "	u:" + u.apply(i) + "	v:" + v.apply(i)
         );
     }
  }
\end{xten}
%~~siv
%}
%~~neg

The operator-using code can be written in X10, though a few variations are
necessary to handle such exotic cases as \xcd`1+X`.

%% HERE!!

Most X10 operators can be given definitions.\footnote{Indeed, even for the
standard types, these operators are defined in the library.  Not even as basic
an operation as integer addition is built into the language.  Conversely, if
you define a full-featured numeric type, it will have most of the privileges that
the standard ones enjoy.  The missing priveleges are (1) literals; (2) 
the \xcd`..` operator won't compute the \xcd`zeroBased` and \xcd`rail`
properties as it does for \xcd`Int` ranges; (3) \xcd`*` won't track ranks, as
it does for \xcd`Region`s; 
(4) \xcd`&&` and \xcd`||` won't short-circuit, as they do for \xcd`Boolean`s, 
 and (5) \xcd`a==b` will only coincide with
\xcd`a.equals(b)` if coded that way.  For example, a \xcd`Polar` type might
have many representations for the origin, as radius 0 and any angle; these
will be \xcd`equals()`, but will not be \xcd`==`.}  (However, \xcd`&&` and
\xcd`||` 
are only short-circuiting for \xcd`Boolean` expressions; user-defined versions
of these operators have no special execution behavior.)

The user-definable operations are (in order of precedence): \\
\begin{tabular}{l}
implicit type coercions\\
postfix \xcd`()`\\
\xcd`as T`\\
unary \xcd`-`, unary \xcd`+`, \xcd`!`, \xcd`~`\\
\xcd`..`\\
\xcd`*     `  \xcd`/     `  \xcd`%` \\
\xcd`+` \xcd`     -` \\
\xcd`<<    ` \xcd`>>    ` \xcd`>>>   ` \xcd`->` \\
\xcd`>     ` \xcd`>=    ` \xcd`<     ` \xcd`<=     ` 
\xcd`in     ` 
\xcd`&` \\
\xcd`^` \\
\xcd`|` \\
\xcd`&&` \\
\xcd`||` \\
\end{tabular}



\subsection{Binary Operators}

Defining the sum \xcd`P+Q` of two polynomials looks much like a method
definition.  It uses the \xcd`operator` keyword instead of \xcd`def`, and
\xcd`this` appears in the definition in the place that a \xcd`Poly` would
appear in a use of the operator.  So, 
\xcd`operator this + (p:Poly)` explains how to add \xcd`this` to a
\xcd`Poly` value.
%~~gen ^^^ Classes180
% package Classes.In.Poly102;
%~~vis
\begin{xten}
class Poly {
  public val coeff : Array[Int](1);
  public def this(coeff: Array[Int](1)) { this.coeff = coeff;}
  public def degree() = coeff.size-1;
  public def  a(i:Int) = (i<0 || i>this.degree()) ? 0 : coeff(i);

  public operator this + (p:Poly) =  new Poly(
     new Array[Int](
        Math.max(this.coeff.size, p.coeff.size),
        (i:Int) => this.a(i) + p.a(i)
     )); 
  // ... 
\end{xten}
%~~siv
%   public operator (n : Int) + this = new Poly([n]) + this;
%   public operator this + (n : Int) = new Poly([n]) + this;
% 
%   def makeSureItWorks() {
%      val x = new Poly([0,1]);
%      val p <: Poly = x+x+x;
%      val q <: Poly = 1+x;
%      val r <: Poly = x+1;
%   }
%     
% }
%~~neg


The sum of a polynomial and an integer, \xcd`P+3`, looks like
an overloaded method definition.  
%~~gen ^^^ Classes190
% package Classes.In.Poly103;
% class Poly {
%   public val coeff : Array[Int](1);
%   public def this(coeff: Array[Int](1)) { this.coeff = coeff;}
%   public def degree() = coeff.size-1;
%   public def  a(i:Int) = (i<0 || i>this.degree()) ? 0 : coeff(i);
% 
%   public operator this + (p:Poly) =  new Poly(
%      new Array[Int](
%         Math.max(this.coeff.size, p.coeff.size),
%         (i:Int) => this.a(i) + p.a(i)
%      ));
%    public operator (n : Int) + this = new Poly([n]) + this;
%~~vis
\begin{xten}
   public operator this + (n : Int) = new Poly([n]) + this;
\end{xten}
%~~siv
% 
%   def makeSureItWorks() {
%      val x = new Poly([0,1]);
%      val p <: Poly = x+x+x;
%      val q <: Poly = 1+x;
%      val r <: Poly = x+1;
%   }
%     
% }
%~~neg


However, we want to allow the sum of an integer and a polynomial as well:
\xcd`3+P`.  It would be quite inconvenient to have to define this as a method
on \xcd`Int`; changing \xcd`Int` is far outside of normal coding.  So, we
allow it as a method on \xcd`Poly` as well.


%~~gen ^^^ Classes200
% package Classes.In.Poly104o;
% class Poly {
%   public val coeff : Array[Int](1);
%   public def this(coeff: Array[Int](1)) { this.coeff = coeff;}
%   public def degree() = coeff.size-1;
%   public def  a(i:Int) = (i<0 || i>this.degree()) ? 0 : coeff(i);
% 
%   public operator this + (p:Poly) =  new Poly(
%      new Array[Int](
%         Math.max(this.coeff.size, p.coeff.size),
%         (i:Int) => this.a(i) + p.a(i)
%      ));
%~~vis
\begin{xten}
   public operator (n : Int) + this = new Poly([n]) + this;
\end{xten}
%~~siv
% 
%   public operator this + (n : Int) = new Poly([n]) + this;
%   def makeSureItWorks() {
%      val x = new Poly([0,1]);
%      val p <: Poly = x+x+x;
%      val q <: Poly = 1+x;
%      val r <: Poly = x+1;
%   }
%     
% }
%~~neg

Furthermore, it is sometimes convenient to express a binary operation as a
static method on a class. 
The definition for the sum of two
\xcd`Poly`s could have been written:
%~~gen ^^^ Classes210
% package Classes.In.Poly105;
% class Poly {
%   public val coeff : Array[Int](1);
%   public def this(coeff: Array[Int](1)) { this.coeff = coeff;}
%   public def degree() = coeff.size-1;
%   public def  a(i:Int) = (i<0 || i>this.degree()) ? 0 : coeff(i);
%~~vis
\begin{xten}
  public static operator (p:Poly) + (q:Poly) =  new Poly(
     new Array[Int](
        Math.max(q.coeff.size, p.coeff.size),
        (i:Int) => q.a(i) + p.a(i)
     ));
\end{xten}
%~~siv
%
%   public operator (n : Int) + this = new Poly([n]) + this;
%   public operator this + (n : Int) = new Poly([n]) + this;
% 
%   def makeSureItWorks() {
%      val x = new Poly([0,1]);
%      val p <: Poly = x+x+x;
%      val q <: Poly = 1+x;
%      val r <: Poly = x+1;
%   }
%     
% }
%~~neg

This requires the following grammar: \\
%##(MethodDecl
\begin{bbgrammar}
%(FROM #(prod:MethodDecl)#)
          MethodDecl \: MethMods \xcd"def" Id TypeParams\opt FormalParams WhereClause\opt HasResultType\opt Offers\opt MethodBody & (\ref{prod:MethodDecl}) \\
                    \| MethMods \xcd"operator" TypeParams\opt \xcd"(" FormalParam  \xcd")" BinOp \xcd"(" FormalParam  \xcd")" WhereClause\opt HasResultType\opt Offers\opt MethodBody \\
                    \| MethMods \xcd"operator" TypeParams\opt PrefixOp \xcd"(" FormalParam  \xcd")" WhereClause\opt HasResultType\opt Offers\opt MethodBody \\
                    \| MethMods \xcd"operator" TypeParams\opt \xcd"this" BinOp \xcd"(" FormalParam  \xcd")" WhereClause\opt HasResultType\opt Offers\opt MethodBody \\
                    \| MethMods \xcd"operator" TypeParams\opt \xcd"(" FormalParam  \xcd")" BinOp \xcd"this" WhereClause\opt HasResultType\opt Offers\opt MethodBody \\
                    \| MethMods \xcd"operator" TypeParams\opt PrefixOp \xcd"this" WhereClause\opt HasResultType\opt Offers\opt MethodBody \\
                    \| MethMods \xcd"operator" \xcd"this" TypeParams\opt FormalParams WhereClause\opt HasResultType\opt Offers\opt MethodBody \\
                    \| MethMods \xcd"operator" \xcd"this" TypeParams\opt FormalParams \xcd"=" \xcd"(" FormalParam  \xcd")" WhereClause\opt HasResultType\opt Offers\opt MethodBody \\
                    \| MethMods \xcd"operator" TypeParams\opt \xcd"(" FormalParam  \xcd")" \xcd"as" Type WhereClause\opt Offers\opt MethodBody \\
                    \| MethMods \xcd"operator" TypeParams\opt \xcd"(" FormalParam  \xcd")" \xcd"as" \xcd"?" WhereClause\opt HasResultType\opt Offers\opt MethodBody \\
                    \| MethMods \xcd"operator" TypeParams\opt \xcd"(" FormalParam  \xcd")" WhereClause\opt HasResultType\opt Offers\opt MethodBody \\
\end{bbgrammar}
%##)
When X10 attempts to typecheck a binary operator expression like \xcd`P+Q`, it
first typechecks \xcd`P` and \xcd`Q`. Then, it looks for operator declarations
for \xcd`+` in the types of \xcd`P` and \xcd`Q`. If there are none, it is a
static error. If there is precisely one, that one will be used. If there are
several, X10 looks for a {\em best-matching} operation, \viz{} one which does
not require the operands to be converted to another type. For example,
\xcd`operator this + (n:Long)` and \xcd`operator this + (n:Int)` both apply to
\xcd`p+1`, because \xcd`1` can be converted from an \xcd`Int` to a \xcd`Long`.
However, the \xcd`Int` version will be chosen because it does not require a
conversion. If even the best-matching operation is not uniquely determined,
the compiler will report a static error.

The main difference between expressing a binary operation as an instance
method (with a \xcd`this` in the definition) and a static one (no \xcd`this`)
is that instance methods don't apply any conversions, while static methods
attempt to convert both arguments. 




\subsection{Unary Operators}

Unary operators are defined in a similar way, with \xcd`this` appearing in the
\xcd`operator` definition where an actual value would occur in a unary
expression.  The operator to negate a polynomial is: 

%~~gen ^^^ Classes220
% package Classes.In.Poly106;
% class Poly {
%   public val coeff : Array[Int](1);
%   public def this(coeff: Array[Int](1)) { this.coeff = coeff;}
%   public def degree() = coeff.size-1;
%   public def  a(i:Int) = (i<0 || i>this.degree()) ? 0 : coeff(i);
%~~vis
\begin{xten}
  public operator - this = new Poly(
    new Array[Int](coeff.size, (i:Int) => -coeff(i))
    );
\end{xten}
%~~siv
%   def makeSureItWorks() {
%      val x = new Poly([0,1]);
%      val p <: Poly = -x;
%   }
% }
%~~neg

The rules for typechecking a unary operation are the same as for methods; the
complexities of binary operations are not needed.

\bard{List the operators which this works for, in precedence order}


\subsection{Type Conversions}
\index{type conversion!user-defined}

Explicit type conversions, \xcd`e as T{c}`, can be defined as operators on
class \xcd`T`.

%~~gen ^^^ Classes230
% package Classes_explicit_type_conversions_a;
%~~vis
\begin{xten}
class Poly {
  public val coeff : Array[Int](1);
  public def this(coeff: Array[Int](1)) { this.coeff = coeff;}
  public static operator (a:Int) as Poly = new Poly([a]);
  public static def main(Array[String](1)):void {
     val three : Poly = 3 as Poly;
  }
}
\end{xten}
%~~siv
%
%~~neg


Furthermore, \xcd`T` may be written as \xcd`?` in the definition of a type
conversion operator (and only there) to have it inferred from context: 

%~~gen ^^^ Classes9x1k
% package Classes9x1k;
%~~vis
\begin{xten}
class Caster(n:Int) {
  public static operator (a:Int) as ? = new Caster(a); 
  public static def example() {
    val c : Caster{n==3} = 3 as Caster{n==3};
  }
}
\end{xten}
%~~siv
%
%~~neg


The \xcd`?` may be given a bound, such as \xcd`as ? <: Caster`, if desired.

% TODO

%%TODO%%  You may define a type conversion to a constrained type, like \xcd`Poly` in
%%TODO%%  the previous example.   If you convert to a more specific constraint, X10 will use
%%TODO%%  the conversion, but insert a dynamic check to make sure that you have
%%TODO%%  satisfied the more specific constraint.  
%%TODO%%  For example: 
%%TODO%%  %~x~gen
%%TODO%%  %package Classes.And.Type.Conversions;
%%TODO%%  %~x~vis
%%TODO%%  \begin{xten}
%%TODO%%  class Uni(n:Int) {
%%TODO%%  
%%TODO%%    public def this(n:Int) : Uni{self.n==n} = {property(n);}
%%TODO%%    static operator (String) as Uni{self.n != 9} = new Uni(3);
%%TODO%%    public static def main(Array[String](1)):void {
%%TODO%%      val u = "" as Uni{self.n != 9 && self.n != 3};
%%TODO%%    }
%%TODO%%  }
%%TODO%%  \end{xten}
%%TODO%%  %~x~siv
%%TODO%%  %
%%TODO%%  %~x~neg
%%TODO%%  The string \xcd`""` is converted to \xcd`Uni{self.n != 9}` via the defined
%%TODO%%  conversion operator, and that value is checked against the remaining
%%TODO%%  constraints \xcd`{self.n != 3}` at runtime.  (In this case it will fail.)
%%TODO%%  
%%TODO%%  There may be many conversions from different types to \xcd`T`, but there may
%%TODO%%  be at most one conversion from any given type to \xcd`T`. 
%%TODO%%  
\bard{Syntax}

\subsection{Implicit Type Coercions}
\label{sect:ImplicitCoercion}
\index{type conversion!implicit}

You may also define {\em implicit} type coercions to \xcd`T{c}` as static
operators in class \xcd`T`.  The syntax for this is
\xcd`static operator (x:U) : T{c} = e`.
Implicit coercions are used automatically by the compiler on method calls 
(\Sref{sect:MethodResolution}) and assignments (\Sref{domedomedome}).



For example, we can define an implicit coercion from \xcd`Int` to \xcd`Poly`,
and avoid having to define the sum of an integer and a polynomial
as many special cases.  In the following example, we only define \xcd`+` on
two polynomials (using a \xcd`static` operator, so that implicit coercions
will be used -- they would not be for an instance method operator).  The
calculation \xcd`1+x` coerces \xcd`1` to a polynomial and uses polynomial
addition to add it to \xcd`x`.

%~~gen ^^^ Classes240
% package Classes.And.Implicit.Coercions;
% class Poly {
%   public val coeff : Array[Int](1);
%   public def this(coeff: Array[Int](1)) { this.coeff = coeff;}
%   public def degree() = coeff.size-1;
%   public def  a(i:Int) = (i<0 || i>this.degree()) ? 0 : coeff(i);
%   public final def toString() = {
%      var allZeroSoFar : Boolean = true;
%      var s : String ="";
%      for( i in 0..this.degree() ) {
%        val ai = this.a(i);
%        if (ai == 0) continue;
%        if (allZeroSoFar) {
%           allZeroSoFar = false;
%           s = term(ai, i);
%        }
%        else 
%           s += 
%              (ai > 0 ? " + " : " - ")
%             +term(ai, i);
%      }
%      if (allZeroSoFar) s = "0";
%      return s;
%   }
%   private final def term(ai: Int, n:Int) = {
%      val xpow = (n==0) ? "" : (n==1) ? "x" : "x^" + n ;
%      return (ai == 1) ? xpow : "" + Math.abs(ai) + xpow;
%   }

%~~vis
\begin{xten}
  public static operator (c : Int) : Poly = new Poly([c]);

  public static operator (p:Poly) + (q:Poly) = new Poly(
      new Array[Int](
        Math.max(p.coeff.size, q.coeff.size),
        (i:Int) => p.a(i) + q.a(i)
     ));

  public static def main(Array[String](1)):void {
     val x = new Poly([0,1]);
     x10.io.Console.OUT.println("1+x=" + (1+x));
  }
\end{xten}
%~~siv
%}
%~~neg

\bard{Syntax}

\subsection{Assignment and Application Operators}
\index{assignment operator}
\index{application operator}
\index{()}
\index{()=}
\label{set-and-apply}
X10 allows types to implement the subscripting / function application
operator, and indexed assignment.  The \xcd`Array`-like classes take advantage
of both of these in \xcd`a(i) = a(i) + 1`.  

\xcd`a(b,c,d)`
is an operator call, to an operator defined with 
\xcd`public operator this(b:B, c:C, d:D)`.  It may be overloaded.
For
example, an ordered dictionary structure could allow subscripting by numbers
with \xcd`public operator this(i:Int)`, and by string-valued keys with 
\xcd`public operator this(s:String)`.  


\xcd`a(i,j)=b` is an \xcd`operator` as well, with zero or more indices
\xcd`i,j`.  It may also be overloaded. 

The update operations \xcd`a(i) += b` are defined to be the same as the
corresponding \xcd`a(i) = a(i) + b`. This applies for all arities of
arguments, and all types, and all binary operations. Of course to use this,
both the application and assignment \xcd`operator`s must be defined.


\begin{ex}

The \xcd`Oddvec` class of somewhat peculiar vectors illustrates this.
\xcd`a()` returns a string representation of the oddvec, which probably should
be done by \xcd`toString()` instead.  
\xcd`a(i)` sensibly picks out one of the three
coordinates of \xcd`a`.
\xcd`a()=b` sets all the coordinates of \xcd`a` to \xcd`b`.
\xcd`a(i)=b` assigns to one of the
coordinates.  \xcd`a(i,j)=b` assigns different values to \xcd`a(i)` and
\xcd`a(j)`, purely for the sake of the example.

%~~gen ^^^ Classes250
% package Classes.Assignments1_oddvec;
%~~vis
\begin{xten}
class Oddvec {
  var v : Array[Int](1) = new Array[Int](3, (Int)=>0);
  public operator this () = 
      "(" + v(0) + "," + v(1) + "," + v(2) + ")";
  public operator this () = (newval: Int) { 
    for(p in v) v(p) = newval;
  }
  public operator this(i:Int) = v(i);
  public operator this(i:Int, j:Int) = [v(i),v(j)];
  public operator this(i:Int) = (newval:Int) = {v(i) = newval;}
  public operator this(i:Int, j:Int) = (newval:Int) = {
       v(i) = newval; v(j) = newval+1;} 
  public def example() {
    this(1) = 6;   assert this(1) == 6;
    this(1) += 7;  assert this(1) == 13;
  }
\end{xten}
%~~siv
% }
%  class Hook { def run() {
%     val a = new Oddvec();
%     assert a().equals("(0,0,0)");
%     a() = 1;
%     assert a().equals("(1,1,1)");
%     a(1) = 4;
%     assert a().equals("(1,4,1)");
%     a(0,2) = 5;
%     assert a().equals("(5,4,6)");
%     return true;
%   }
% }
%~~neg

\end{ex}

\section{Class Guards and Invariants}\label{DepType:ClassGuard}
\index{type invariants}
\index{class invariants}
\index{invariant!type}
\index{invariant!class}
\index{guard}


Classes (and structs and interfaces) may specify a {\em class guard}, a
constraint which must hold on all values of the class.    In the following
example, a \xcd`Line` is defined by two distinct \xcd`Pt`s\footnote{We use \xcd`Pt`
to avoid any possible confusion with the built-in class \xcd`Point`.}
%~~gen ^^^ Classes260
% package classes.guards.invariants.glurp;
%~~vis
\begin{xten}
class Pt(x:Int, y:Int){}
class Line(a:Pt, b:Pt){a != b} {}
\end{xten}
%~~siv
%
%~~neg

In most cases the class guard could be phrased as a type constraint on a property of
the class instead, if preferred.  Arguably, a symmetric constraint like two
points being different is better expressed as a class guard, rather than
asymmetrically as a constraint on one type: 
%~~gen ^^^ Classes270
% package classes.guards.invariants.glurp2;
% class Pt(x:Int, y:Int){}
%~~vis
\begin{xten}
class Line(a:Pt, b:Pt{a != b}) {}
\end{xten}
%~~siv
%
%~~neg



\label{DepType:TypeInvariant}
\index{class invariant}
\index{invariant!class}
\index{class!invariant}
\label{DepType:ClassGuardDef}



With every defined class, struct,  or interface \xcd"T" we associate a {\em type
invariant} $\mathit{inv}($\xcd"T"$)$, which describes the guarantees on the
properties of values of type \xcd`T`.  

Every value of \xcd`T` satisfies $\mathit{inv}($\xcd"T"$)$ at all times.  This
is somewhat stronger than the concept of type invariant in most languages
(which only requires that the invariant holds when no method calls are
active).  X10 invariants only concern properties, which are immutable; thus,
once established, they cannot be falsified.

The type
invariant associated with \xcd"x10.lang.Any"
is 
\xcd"true".

The type invariant associated with any interface or struct \xcd"I" that extends
interfaces \xcdmath"I$_1$, $\dots$, I$_k$" and defines properties
\xcdmath"x$_1$: P$_1$, $\dots$, x$_n$: P$_n$" and
specifies a guard \xcd"c" is given by:

\begin{xtenmath}
$\mathit{inv}$(I$_1$) && $\dots$ && $\mathit{inv}$(I$_k$) 
    && self.x$_1$ instanceof P$_1$ &&  $\dots$ &&  self.x$_n$ instanceof P$_n$ 
    && c  
\end{xtenmath}

Similarly the type invariant associated with any class \xcd"C" that
implements interfaces \xcdmath"I$_1$, $\dots$, I$_k$",
extends class \xcd"D" and defines properties
\xcdmath"x$_1$: P$_1$, $\dots$, x$_n$: P$_n$" and
specifies a guard \xcd"c" is
given by the same thing with the invariant of the superclass \xcd`D` conjoined:
\begin{xtenmath}
$\mathit{inv}$(I$_1$) && $\dots$ && $\mathit{inv}$(I$_k$) 
    && self.x$_1$ instanceof P$_1$ &&  $\dots$ &&  self.x$_n$ instanceof P$_n$ 
    && c  
    && $\mathit{inv}$(D)
\end{xtenmath}


Note that the type invariant associated with a class entails the type
invariants of each interface that it implements (directly or indirectly), and
the type invariant of each ancestor class.
It is guaranteed that for any variable \xcd"v" of
type \xcd"T{c}" (where \xcd"T" is an interface name or a class name) the only
objects \xcd"o" that may be stored in \xcd"v" are such that \xcd"o" satisfies
$\mathit{inv}(\mbox{\xcd"T"}[\mbox{\xcd"o"}/\mbox{\xcd"this"}])
\wedge \mbox{\xcd"c"}[\mbox{\xcd"o"}/\mbox{\xcd"self"}]$.



\subsection{Invariants for {\tt implements} and {\tt extends} clauses}\label{DepType:Implements}
\label{DepType:Extends}
\index{type-checking!implements clause}
\index{type-checking!extends clause}
\index{implements}
\index{extends}
Consider a class definition
\begin{xtenmath}
$\mbox{\emph{ClassModifiers}}^{\mbox{?}}$
class C(x$_1$: P$_1$, $\dots$, x$_n$: P$_n$) extends D{d}
   implements I$_1${c$_1$}, $\dots$, I$_k${c$_k$}
$\mbox{\emph{ClassBody}}$
\end{xtenmath}

Each of the following static semantics rules must be satisfied:

\noo{reformat this}

The type invariant \xcdmath"$\mathit{inv}$(C)" of \xcd"C" must entail
\xcdmath"c$_i$[this/self]" for each $i$ in $\{1, \dots, k\}$



The return type \xcd"c" of each constructor in a class \xcd`C`
must entail the invariant \xcdmath"$\mathit{inv}$(C)".


\subsection{Invariants and constructor definitions}
\index{invariant!and constructor}
\index{constructor!and invariant}

A constructor for a class \xcd"C" is guaranteed to return an object of the
class on successful termination. This object must satisfy  \xcdmath"$\mathit{inv}$(C)", the
class invariant associated with \xcd"C" (\Sref{DepType:TypeInvariant}).
However,
often the objects returned by a constructor may satisfy {\em stronger}
properties than the class invariant. \Xten{}'s dependent type system
permits these extra properties to be asserted with the constructor in
the form of a constrained type (the ``return type'' of the constructor):

%##(CtorDecl
\begin{bbgrammar}
%(FROM #(prod:CtorDecl)#)
            CtorDecl \: Mods\opt \xcd"def" \xcd"this" TypeParams\opt FormalParams WhereClause\opt HasResultType\opt  CtorBody & (\ref{prod:CtorDecl}) \\
\end{bbgrammar}
%##)

\label{ConstructorGuard}

The parameter list for the constructor
may specify a \emph{guard} that is to be satisfied by the parameters
to the list.

\begin{ex}
%%TODO--rewrite this
Here is another example, constructed as a simplified 
version of \Xcd{x10.array.Region}.  The \xcd`mockUnion` method 
has the type, though not the value, that a true \xcd`union` method would have.

%~~gen ^^^ Classes280
%package Classes.SimplifiedRegion;
%~~vis
\begin{xten}
class MyRegion(rank:Int) {
  static type MyRegion(n:Int)=MyRegion{rank==n};
  def this(r:Int):MyRegion(r) {
    property(r);
  }
  def this(diag:Array[Int](1)):MyRegion(diag.size){ 
    property(diag.size);
  }
  def mockUnion(r:MyRegion(rank)):MyRegion(rank) = this;
  def example() {
    val R1 : MyRegion(3) = new MyRegion([4,4,4]); 
    val R2 : MyRegion(3) = new MyRegion([5,4,1]); 
    val R3 = R1.mockUnion(R2); // inferred type MyRegion(3)
  }
}
\end{xten}
%~~siv
%
%~~neg
The first constructor returns the empty region of rank \Xcd{r}.  The
second constructor takes a \Xcd{Array[Int](1)} of arbitrary length
\Xcd{n} and returns a \Xcd{MyRegion(n)} (intended to represent the set
of points in the rectangular parallelopiped between the origin and the
\Xcd{diag}.)

The code in \xcd`example` typechecks, and \xcd`R3`'s type is inferred as
\xcd`MyRegion(3)`.  


\end{ex}

   Let \xcd"C" be a class with properties
   \xcdmath"p$_1$: P$_1$, $\dots$, p$_n$: P$_n$", and invariant \xcd"c"
   extending the constrained type \xcd"D{d}" (where \xcd"D" is the name of a
   class).



   For every constructor in \xcd"C" the compiler checks that the call to
   super invokes a constructor for \xcd"D" whose return type is strong enough
   to entail \xcd"d". Specifically, if the call to super is of the form 
     \xcdmath"super(e$_1$, $\dots$, e$_k$)"
   and the static type of each expression \xcdmath"e$_i$" is
   \xcdmath"S$_i$", and the invocation
   is statically resolved to a constructor
\xcdmath"def this(x$_1$: T$_1$, $\dots$, x$_k$: T$_k$){c}: D{d$_1$}"
   then it must be the case that 
\begin{xtenmath}
x$_1$: S$_1$, $\dots$, x$_i$: S$_i$ entails x$_i$: T$_i$  (for $i \in \{1, \dots, k\}$)
x$_1$: S$_1$, $\dots$, x$_k$: S$_k$ entails c  
d$_1$[a/self], x$_1$: S$_1$, ..., x$_k$: S$_k$ entails d[a/self]      
\end{xtenmath}
\noindent where \xcd"a" is a constant that does not appear in 
\xcdmath"x$_1$: S$_1$ $\wedge$ ... $\wedge$ x$_k$: S$_k$".

   The compiler checks that every constructor for \xcd"C" ensures that
   the properties \xcdmath"p$_1$,..., p$_n$" are initialized with values which satisfy
   $\mathit{inv}($\xcd"T"$)$, and its own return type \xcd"c'" as follows.  In each constructor, the
   compiler checks that the static types \xcdmath"T$_i$" of the expressions \xcdmath"e$_i$"
   assigned to \xcdmath"p$_i$" are such that the following is
   true:
\begin{xtenmath}
p$_1$: T$_1$, $\dots$, p$_n$: T$_n$ entails $\mathit{inv}($T$)$ $\wedge$ c'     
\end{xtenmath}

(Note that for the assignment of \xcdmath"e$_i$" to \xcdmath"p$_i$"
to be type-correct it must be the
    case that \xcdmath"p$_i$: T$_i$ $\wedge$ p$_i$: P$_i$".) 



The compiler must check that every invocation \xcdmath"C(e$_1$, $\dots$, e$_n$)" to a
constructor is type correct: each argument \xcdmath"e$_i$" must have a static type
that is a subtype of the declared type \xcdmath"T$_i$" for the $i$th
argument of the
constructor, and the conjunction of static types of the argument must
entail the constraint in the parameter list of the constructor.



\subsection{Object Initialization}
\label{ObjectInitialization}
\index{initialization}
\index{constructor}
\index{object!constructor}
\index{struct!constructor}


X10 does object initialization safely.  It avoids certain bad things which
trouble some other languages:
\begin{enumerate}
\item Use of a field before the field has been initialized.
\item A program reading two different values from a \xcd`val` field of a
      container. 
\item \Xcd{this} escaping from a constructor, which can cause problems as
      noted below. 

\end{enumerate}

It should be unsurprising that fields must not be used before they are
initialized. At best, it is uncertain what value will be in them, as in
\Xcd{x} below. Worse, the value might not even be an allowable value; \Xcd{y},
declared to be nonzero in the following example, might be zero before it is
initialized.
\begin{xten}
// Not correct X10
class ThisIsWrong {
  val x : Int;
  val y : Int{y != 0};
  def this() {
    x10.io.Console.OUT.println("x=" + x + "; y=" + y);
    x = 1; y = 2;
  }
}
\end{xten}

One particularly insidious way to read uninitialized fields is to allow
\Xcd{this} to escape from a constructor. For example, the constructor could
put \Xcd{this} into a data structure before initializing it, and another
activity could read it from the data structure and look at its fields:
\begin{xten}
class Wrong {
  val shouldBe8 : Int;
  static Cell[Wrong] wrongCell = new Cell[Wrong]();
  static def doItWrong() {
     finish {
       async { new Wrong(); } // (A)
       assert( wrongCell().shouldBe8 == 8); // (B)
     }
  }
  def this() {
     wrongCell.set(this); // (C) - ILLEGAL
     this.shouldBe8 = 8; // (D)
  }
}
\end{xten}
\noindent
In this example, the underconstructed \Xcd{Wrong} object is leaked into a
storage cell at line \Xcd{(C)}, and then initialized.  The \Xcd{doItWrong}
method constructs a new \Xcd{Wrong} object, and looks at the \Xcd{Wrong}
object in the storage cell to check on its \Xcd{shouldBe8} field.  One
possible order of events is the following:
\begin{enumerate}
\item \Xcd{doItWrong()} is called.
\item \Xcd{(A)} is started.  Space for a new \Xcd{Wrong} object is allocated.
      Its \Xcd{shouldBe8} field, not yet initialized, contains some garbage
      value.
\item \Xcd{(C)} is executed, as part of the process of constructing a new
      \Xcd{Wrong} object.  The new, uninitialized object is stored in
      \Xcd{wrongCell}.
\item Now, the initialization activity is paused, and execution of the main activity
      proceeds from \Xcd{(B)}.
\item The value in \Xcd{wrongCell} is retrieved, and is \Xcd{shouldBe8} field
      is read.  This field contains garbage, and the assertion fails.
\item Now let the initialization activity proceed with \Xcd{(D)},
      initializing \Xcd{shouldBe8} --- too late.
\end{enumerate}

The \xcd`at` statement (\Sref{AtStatement}) introduces the potential for
escape as well. The following class prints an uninitialized value: 
%~~gen ^^^ ThisEscapingViaAt_MustFailCompile
% package ObjInit_at;
% NOCOMPILE
%~~vis
\begin{xten}
class Example {
  val a: Int;
  def this() { 
    at(here.next()) {
      // Recall that 'this' is a copy of 'this' outside 'at'.
      Console.OUT.println("this.a = " + this.a);
    }
    this.a = 1;
  }
}
\end{xten}
%~~siv
%
%~~neg


X10 must protect against such possibilities.  The rules explaining how
constructors can be written are somewhat intricate; they are designed to allow
as much programming as possible without leading to potential problems.
Ultimately, they simply are elaborations of the fundamental principles that
uninitialized fields must never be read, and \Xcd{this} must never be leaked.

\subsection{Raw and Cooked Objects}
\index{raw}
\index{cooked}

An object is {\em raw} before its constructor ends, and {\em cooked} after its
constructor ends. Note that, when an object is cooked, all its subobjects are
cooked.  




\subsection{Constructors and NonEscaping Methods}
\index{non-escaping}
\label{sect:nonescaping}

In general, constructors must not be allowed to call methods with \Xcd{this} as
an argument or receiver. Such calls could leak references to \Xcd{this},
either directly from a call to \Xcd{cell.set(this)}, or indirectly because
\Xcd{toString} leaks \Xcd{this}, and the concatenation
\Xcd`"Escaper = "+this` calls \Xcd{toString}.\footnote{This is abominable behavior for
\Xcd{toString}, but it cannot be prevented -- save by a scheme such as we
present in this section.}
%~WRONG~gen
%package ObjectInit.CtorAndNonEscaping.One;
%~WRONG~vis
\begin{xten}
class Escaper {
  static val Cell[Escaper] cell = new Cell[Escaper]();
  def toString() {
    cell.set(this);
    return "Evil!";
  }
  def this() {
    cell.set(this);
    x10.io.Console.OUT.println("Escaper = " + this);
  }
}
\end{xten}
%~WRONG~siv
%
%~WRONG~neg

However, it is convenient to be able to call methods from constructors; {\em
e.g.}, a class might have eleven constructors whose common behavior is best
described by three methods.
Under certain stringent conditions, it {\em is}
safe to call a method: the method called must not leak references to
\Xcd{this}, and must not read \Xcd{val}s or \Xcd{var}s which might not have
been assigned.

So, X10 performs a static dataflow analysis, sufficient to guarantee that
method calls in constructors are safe.  This analysis requires having access
to or guarantees about all the code that could possibly be called.  This can
be accomplished in two ways:
\begin{enumerate}
\item Ensuring that only code from the class itself can be called, by
      forbidding overriding of
      methods called from the constructor: they can be marked \Xcd{final} or
      \Xcd{private}, or the whole class can be \Xcd{final}.
\item Marking the methods called from the constructor by
      \xcd`@NonEscaping`.
\end{enumerate}

\subsubsection{Non-Escaping Methods}
\index{method!non-escaping}
\index{method!implicitly non-escaping}
\index{method!NonEscaping}
\index{implicitly non-escaping}
\index{non-escaping}
\index{non-escaping!implicitly}
\index{NonEscaping}


A method may be annotated with \xcd`@NonEscaping`.  This
imposes several restrictions on the method body, and on all methods overriding
it.  However, it is the only way that a method can be called from
constructors.  The
\Xcd{@NonEscaping} annotation makes explicit all the X10 compiler's needs for
constructor-safety.

A method can, however, be safe to call from constructors without being marked
\Xcd{@NonEscaping}. We call such methods {\em implicitly non-escaping}.
Implicitly non-escaping methods need to obey the same constraints on
\Xcd{this}, \Xcd{super}, and variable usage as \Xcd{@NonEscaping} methods. An
implicitly non-escaping method {\em could} be marked as
\xcd`@NonEscaping` for some list of variables; the compiler, in
effect, infers the annotation. In addition, implicitly non-escaping methods
must be \Xcd{private} or \Xcd{final} or members of a \Xcd{final} class; this
corresponds to the hereditary nature of \Xcd{@NonEscaping} (by forbidding
inheritance of implicitly non-escaping methods).

We say that a method is {\em non-escaping} if it is either implicitly
non-escaping, or annotated \Xcd{@NonEscaping}.

The first requirement on non-escaping methods is that they do not allow
\Xcd{this} to escape. Inside of their bodies, \Xcd{this} and \Xcd{super} may
only be used for field access and assignment, and as the receiver of
non-escaping methods.

Finally, if a method \Xcd{m} in class \Xcd{C} is marked
\xcd`@NonEscaping`, then every method which overrides \Xcd{m} in any
subclass of \Xcd{C} must be annotated with precisely the same annotation,
\xcd`@NonEscaping`, as well.

The following example uses most of the possible variations (leaving out
\Xcd{final} class).  \Xcd{aplomb()} explicitly forbids reading any field but
\Xcd{a}. \Xcd{boric()} is called after \Xcd{a} and \Xcd{b} are set, but
\Xcd{c} is not.
The \xcd`@NonEscaping` annotation on \xcd`boric()` is optional, but the
compiler will print a warning if it is left out.
\Xcd{cajoled()} is only called after all fields are set, so it
can read anything; its annotation, too, is not required.   \Xcd{SeeAlso} is able to override \Xcd{aplomb()}, because
\Xcd{aplomb()} is \xcd`@NonEscaping("a")`; it cannot override the final method
\Xcd{boric()} or the private one \Xcd{cajoled()}.  Even for overriding
\Xcd{aplomb()}, it is crucial that \Xcd{SeeAlso.aplomb()} be
declared \xcd`@NonEscaping("a")`, just like \Xcd{C2.aplomb()}.
%~~gen ^^^ ObjectInitialization10
%package ObjInit.C2;
%~~vis
\begin{xten}
import x10.compiler.*;

final class C2 {
  protected val a:Int, b:Int, c:Int;
  protected var x:Int, y:Int, z:Int;
  def this() {
    a = 1;
    this.aplomb();
    b = 2;
    this.boric();
    c = 3;
    this.cajoled();
  }
  @NonEscaping def aplomb() {
    x = a;
    // this.boric(); // not allowed; boric reads b.
    // z = b; // not allowed -- only 'a' can be read here
  }
  @NonEscaping final def boric() {
    y = b;
    this.aplomb(); // allowed; a is definitely set before boric is called
    // z = c; // not allowed; c is not definitely written
  }
  @NonEscaping private def cajoled() {
    z = c;
  }
}

\end{xten}
%~~siv
%
%~~neg



\subsection{Fine Structure of Constructors}
\label{SFineStructCtors}

The code of a constructor consists of four segments, three of them optional
and one of them implicit.
\begin{enumerate}
\item The first segment is an optional call to \Xcd{this(...)} or
      \Xcd{super(...)}.  If this is supplied, it must be the first statement
      of the constructor.  If it is not supplied, the compiler treats it as a
      nullary super-call \Xcd{super()};
\item If the class or struct has properties, there must be a single
      \Xcd{property(...)} command in the constructor.  Every execution path
      through the constructor must go through this \Xcd{property(...)} command
      precisely once.   The second segment of the constructor is the code
      following the first segment, up to and including the \Xcd{property()}
      statement.

      If the class or struct has no properties, the \Xcd{property()} call must
      be omitted. If it is present, the second segment is defined as before.  If
      it is absent, the second segment is empty.
\item The third segment is automatically generated.  Fields with initializers
      are initialized immediately after the \Xcd{property} statement.
      In the following example, \Xcd{b} is initialized to \Xcd{y*9000} in
      segment three.  The initialization makes sense and does the right
      thing; \Xcd{b} will be \Xcd{y*9000} for every \Xcd{Overdone} object.
      (This would not be possible if field initializers were processed
      earlier, before properties were set.)
\item The fourth segment is the remainder of the constructor body.
\end{enumerate}

The segments in the following code are shown in the comments.
%~~gen ^^^ ObjectInitialization20
% package ObjectInitialization.ShowingSegments;
%~~vis
\begin{xten}
class Overlord(x:Int) {
  def this(x:Int) { property(x); }
}//Overlord
class Overdone(y:Int) extends Overlord  {
  val a : Int;
  val b =  y * 9000;
  def this(r:Int) {
    super(r);                      // (1)
    x10.io.Console.OUT.println(r); // (2)
    val rp1 = r+1;
    property(rp1);                 // (2)
    // field initializations here  // (3)
    a = r + 2;                     // (4)
  }
}//Overdone
\end{xten}
%~~siv
%
%~~neg

The rules of what is allowed in the three segments are different, though
unsurprising.  For example, properties of the current class can only be read
in segment 3 or 4---naturally, because they are set at the end of segment 2.

\subsubsection{Initialization and Inner Classses}
\index{constructor!inner classes in}

Constructors of inner classes are tantamount to method calls on \Xcd{this}.
For example, the constructor for Inner {\bf is} acceptable.  It does not leak
\Xcd{this}.  It leaks \Xcd{Outer.this}, which is an utterly different object.
So, the call to \Xcd{this.new Inner()} in the \Xcd{Outer} constructor {\em
is} illegal.  It would leak \Xcd{this}.  There is no special rule in effect
preventing this; a constructor call of an inner class is no
different from a method as far as leaking is concerned.
%~~gen ^^^ ObjectInitialization30
% package ObjInit.InnerClass; 
% NOTEST
%~~vis
\begin{xten}
class Outer {
  static val leak : Cell[Outer] = new Cell[Outer](null);
  class Inner {
     def this() {Outer.leak.set(Outer.this);}
  }
  def /*Outer*/this() {
     //ERROR: val inner = this.new Inner();
  }
}
\end{xten}
%~~siv
%
%~~neg



\subsubsection{Initialization and Closures}
\index{constructor!closure in}

Closures in constructors may not refer to \xcd`this`.  They may not even refer
to fields of \xcd`this` that have been initialized.   For example, the
closure \xcd`bad1` is not allowed because it refers to \xcd`this`; \xcd`bad2`
is not allowed because it mentions \xcd`a` --- which is, of course, identical
to \xcd`this.a`. 

%%-deleted-%% valid if they were invoked (or inlined) at the
%%-deleted-%%place of creation. For example, \Xcd{closure} below is acceptable because it
%%-deleted-%%only refers to fields defined at the point it was written.  \Xcd{badClosure}
%%-deleted-%%would not be acceptable, because it refers to fields that were not defined at
%%-deleted-%%that point, although they were defined later.
%~~gen ^^^ ObjectInitialization40
% package ObjectInitialization.Closures; 
%~~vis
\begin{xten}
class C {
  val a:Int;
  def this() {
    this.a = 1;
    //ERROR: val bad1 = () => this; 
    //ERROR: val bad2 = () => a*10;
  }
}
\end{xten}
%~~siv
%
%~~neg


\subsection{Definite Initialization in Constructors}


An instance field \Xcd{var x:T}, when \Xcd{T} has a default value, need not be
explicitly initialized.  In this case, \Xcd{x} will be initialized to the
default value of type \Xcd{T}.  For example, a \Xcd{Score} object will have
its \Xcd{currently} field initialized to zero, below:
%~~gen ^^^ ObjectInitialization50
% package ObjectInit.DefaultInit;
%~~vis
\begin{xten}
class Score {
  public var currently : Int;
}
\end{xten}
%~~siv
%
%~~neg

All other sorts of instance fields do need to be initialized before they can
be used.  \Xcd{val} fields must be initialized, even if their type has a
default value.  It would be silly to have a field \Xcd{val z : Int} that was
always given default value of \Xcd{0} and, since it is \Xcd{val}, can never be
changed.  \Xcd{var} fields whose type has no default value must be initialized
as well, such as \xcd`var y : Int{y != 0}`, since it cannot be assigned a
sensible initial value.

The fundamental principles are:
\begin{enumerate}
\item \Xcd{val} fields must be assigned precisely once in each constructor on every
possible execution path.
\item \Xcd{var} fields of defaultless type must be
assigned at least once on every possible execution path, but may be assigned
more than once.
\item No variable may be read before it is guaranteed to have been
assigned.
\item Initialization may be by field initialization expressions (\Xcd{val x :
      Int = y+z}), or by uninitialized fields \Xcd{val x : Int;} plus
an initializing assignment \Xcd{x = y+z}.  Recall that field initialization
expressions are performed after the \Xcd{property} statement, in segment 3 in
the terminology of \Sref{SFineStructCtors}.
\end{enumerate}



\subsection{Summary of Restrictions on Classes and Constructors}

The following table tells whether a given feature is (yes), is not (no) or is
with some conditions (note) allowed in a given context.   For example, a
property method is allowed with the type of another property, as long as it
only mentions the preceding properties. The first column of the table gives
examples, by line of the following code body.

\begin{tabular}{||l|l|c|c|c|c|c|c||}
\hline
~
  & {\bf Example}
  & {\bf Prop.}
  & {\bf {\tt \small self==this}(1)}
  & {\bf Prop.Meth.}
  & {\bf {\tt this}}
  & {\bf {fields}}
\\\hline
Type of property
  & (A)
  & %?properties
    yes (2)
  & no %? self==this
  & no %? property methods
  & no %? this may be used
  & no %? fields may be used
\\\hline
Class Invariant
  & (B)
  & yes %?properties
  & yes %? self==this
  & yes %? property methods
  & yes %? this may be used
  & no %? fields may be used
\\\hline
Supertype (3)
  & (C), (D)
  & yes%?properties
  & yes%? self==this
  & yes%? property methods
  & no%? this may be used
  & no%? fields may be used
\\\hline
Property Method Body
  & (E)
  & yes %?properties
  & yes %? self==this
  & yes %? property methods
  & yes %? this may be used
  & no %? fields may be used
\\\hline

Static field (4)
  & (F) (G)
  & no %?properties
  & no %? self==this
  & no %? property methods
  & no %? this may be used
  & no %? fields may be used
\\\hline

Instance field (5)
  & (H), (I)
  & yes %?properties
  & yes %? self==this
  & yes %? property methods
  & yes %? this may be used
  & yes %? fields may be used
\\\hline

Constructor arg. type
  & (J)
  & no %?properties
  & no %? self==this
  & no  %? property methods
  & no %? this may be used
  & no %? fields may be used
\\\hline

Constructor guard
  & (K)
  & no %?properties
  & no %? self==this
  & no %? property methods
  & no %? this may be used
  & no %? fields may be used
\\\hline

Constructor ret. type
  & (L)
  & yes %?properties
  & yes %? self==this
  & yes %? property methods
  & yes %? this may be used
  & yes %? fields may be used
\\\hline

Constructor segment 1
  & (M)
  & no%?properties
  & yes%? self==this
  & no%? property methods
  & no%? this may be used
  & no%? fields may be used
\\\hline


Constructor segment 2
  & (N)
  & no%?properties
  & yes%? self==this
  & no%? property methods
  & no%? this may be used
  & no%? fields may be used
\\\hline

Constructor segment 4
  & (O)
  & yes%?properties
  & yes%? self==this
  & yes%? property methods
  & yes%? this may be used
  & yes%? fields may be used
\\\hline

Methods
  & (P)
  & yes %?properties
  & yes %? self==this
  & yes %? property methods
  & yes %? this may be used
  & yes %? fields may be used
\\\hline



\iffalse
place
  & (pos)
  & %?properties
  & %? self==this
  & %? property methods
  & %? this may be used
  & %? fields may be used
\\\hline
\fi
\end{tabular}

Details:

\begin{itemize}
\item (1) {Top-level {\tt self} only.}
\item (2) {The type of the {$i^{th}$} property may only mention
                 properties {$1$} through {$i$}.}
\item (3) Super-interfaces follow the same rules as supertypes.
\item (4) The same rules apply to types and initializers.
\end{itemize}



The example indices refer to the following code:
%~~gen ^^^ ObjectInitialization60
% package ObjectInit.GrandExample;
% class Supertype[T]{}
% interface SuperInterface[T]{}
%~~vis
\begin{xten}
class Example (
   prop : Int,
   proq : Int{prop != proq},                    // (A)
   pror : Int
   )
   {prop != 0}                                  // (B)
   extends Supertype[Int{self != prop}]         // (C)
   implements SuperInterface[Int{self != prop}] // (D)
{
   property def propmeth() = (prop == pror);    // (E)
   static staticField
      : Cell[Int{self != 0}]                    // (F)
      = new Cell[Int{self != 0}](1);            // (G)
   var instanceField
      : Int {self != prop}                      // (H)
      = (prop + 1) as Int{self != prop};        // (I)
   def this(
      a : Int{a != 0},
      b : Int{b != a}                           // (J)
      )
      {a != b}                                  // (K)
      : Example{self.prop == a && self.proq==b} // (L)
   {
      super();                                  // (M)
      property(a,b,a);                          // (N)
      // fields initialized here
      instanceField = b as Int{self != prop};   // (O)
   }

   def someMethod() =
        prop + staticField + instanceField;     // (P)
}
\end{xten}
%~~siv
%
%~~neg


\section{Method Resolution}
\index{method!resolution}
\index{method!which one will get called}
\label{sect:MethodResolution}

Method resolution is the problem of determining, statically, which method (or
constructor or operator)
should be invoked, when there are several choices that could be invoked.  For
example, the following class has two overloaded \xcd`zap` methods, one taking
an \Xcd{Object}, and the other a \Xcd{Resolve}.  Method resolution will figure
out that the call \Xcd{zap(1..4)} should call \xcd`zap(Object)`, and
\Xcd{zap(new Resolve())} should call \xcd`zap(Resolve)`.  

%~~gen ^^^ MethodResolution10
%package MethodResolution.yousayyouwantaresolution;
%~~vis
\begin{xten}
class Resolve {
  static def zap(Object) = "Object";
  static def zap(Resolve) = "Resolve";
  public static def main(argv:Array[String](1)) {
    Console.OUT.println(zap(1..4));
    Console.OUT.println(zap(new Resolve()));
  }
}
\end{xten}
%~~siv
%
%~~neg

The basic concept of method resolution is quite straightforward: 
\begin{enumerate}
\item List all the methods that could possibly be used (counting implicit
      coercions); 
\item Pick the most specific one; 
\item Fail if there is not a unique most specific one.
\end{enumerate}
\noindent
In the presence of X10's highly-detailed type system, some subtleties arise. 
One point, at least, is {\em not} subtle. The same procedure is used, {\em
mutatis mutandis} for method, constructor, and operator resolution.  

Generics introduce several subtleties, especially with the inference of
generic types. 


For the purposes of method resolution, all that matters about a method,
constructor, or operator \xcd`M` --- we use the word ``method'' to include all
three choices for this section --- is its signature, plus which method it is.
So, a typical \xcd`M` might look like 
\xcdmath"def m[G$_1$,$\ldots$, G$_g$](x$_1$:T$_1$,$\ldots$, x$_f$:T$_f$){c} =...".  The code body \xcd`...` is irrelevant for the purpose of whether a
given method call means \xcd`M` or not, so we ignore it for this section.

All that matters about a method definition, for the purposes of method
resolution, is: 
\begin{enumerate}
\item The method name \xcd`m`;
\item The generic type parameters of the method \xcd`M`,  \xcdmath"G$_1$,$\ldots$, G$_g$".  If there
      are no generic type parameters, {$g=0$}.  
\item The types \xcdmath"x$_1$:T$_1$,$\ldots$, x$_f$:T$_f$" of the formal parameters.  If
      there are no formal parameters, {$f=0$}. In the case of an instance
      method, the receiver will be the first formal parameter.\footnote{The
      variable names are relevant because one formal can be mentioned in a
      later type, or even a constraint: {\tt def f(a:Int, b:Point\{rank==a\})=...}.}
\item The constraint \xcd`c` of the method \xcd`M`. If no constraint is specified, \xcd`c` is
      \xcd`true`. 
\item A {\em unique identifier} \xcd`id`, sufficient to tell the compiler
      which method body is intended.  A file name and position in that file
      would suffice.  The details of the identifier are not relevant.
\end{enumerate}

For the purposes of understanding method resolution, we assume that all the
actual parameters of an invocation are simply variables: \xcd`x1.meth(x2,x3)`.
This is done routinely by the compiler in any case; the code 
\xcd`tbl(i).meth(true, a+1)` would be treated roughly as 
\begin{xten}
val x1 = tbl(i);
val x2 = true;
val x3 = a+1;
x1.meth(x2,x3);
\end{xten}

All that matters about an invocation \xcd`I` is: 
\begin{enumerate}
\item The method name \xcdmath"m$'$";
\item The generic type parameters \xcdmath"G$'_1$,$\ldots$, G$'_g$".  If there
      are no generic type parameters, {$g=0$}.  
\item The names and types \xcdmath"x$_1$:T$'_1$,$\ldots$, x$_f$:T$'_f$" of the
      actual parameters.
      If
      there are no actual parameters, {$f=0$}. In the case of an instance
      method, the receiver is the first actual parameter.
\end{enumerate}

The signature of the method resolution procedure is: 
\xcd`resolve(invo : Invocation, context: Set[Method]) : MethodID`.  
Given a particular invocation and the set \xcd`context` of all methods
which could be called at that point of code, method resolution either returns
the unique identifier of the method that should be called, or (conceptually)
throws an exception if the call cannot be resolved.

The procedure for computing \xcd`resolve(invo, context)` is: 
\begin{enumerate}
\item Eliminate from \xcd`context` those methods which are not {\em
      acceptable}; \viz, those whose name, type parameters, formal parameters,
      and constraint do not suitably match \xcd`invo`.  In more detail:
      \begin{itemize}
      \item The method name \xcd`m` must simply equal the invocation name \xcdmath"m$'$";
      \item X10 infers type parameters, by an algorithm given in \Sref{TypeParamInfer}.
      \item The method's type parameters are bound to the invocation's for the
            remainder of the acceptability test.
      \item The actual parameter types must be subtypes of the formal
            parameter types, or be coercible to such subtypes.  Parameter $i$
            is a subtype if \xcdmath"T$'_i$ <: T$_i$".  It is implicitly
            coercible to a subtype if there is an implicit coercion operator
            defined from \xcdmath"T$'_i$" to some type \xcd`U`, and 
            \xcdmath"U <: T$_i$". \index{method resolution!implicit coercions
            and} \index{implicit coercion}\index{coercion}.  If coercions are
            used to resolve the method, they will be called on the arguments
            before the method is invoked.
            
      \item The formal constraint \xcd`c` must be satisfied in the invoking
            context. 
      \end{itemize}
\item Eliminate from \xcd`context` those methods which are not {\em
      available}; \viz, those which cannot be called due to visibility
      constraints, such as methods from other classes marked \xcd`private`.
      The remaining methods are both acceptable and available; they might be
      the one that is intended.
\item From the remaining methods, find the unique \xcd`ms` which is more specific than all the
      others, \viz, for which \xcd`specific(ms,mo) = true` for all other
      methods \xcd`mo`.
      The specificity test \xcd`specific` is given next.
      \begin{itemize}
      \item If there is a unique such \xcd`ms`, then
            \xcd`resolve(invo,context)` returns the \xcd`id` of \xcd`ms`.  
      \item If there is not a unique such \xcd`ms`, then \xcd`resolve` reports
            an error.
      \end{itemize}

\end{enumerate}

The subsidiary procedure \xcd`specific(m1, m2)` determines whether method
\xcd`m1` is equally or more specific than \xcd`m2`.  \xcd`specific` is not a
total order: is is possible for each one to be considered more specific than
the other, or either to be more specific.  \xcd`specific` is computed as: 
\begin{enumerate}
\item Construct an invocation \xcd`invo1` based on \xcd`m1`: 
      \begin{itemize}
      \item \xcd`invo1`'s method name is \xcd`m1`'s method name;
      \item \xcd`invo1`'s generic parameters are those of \xcd`m1`--- simply
            some type variables.
      \item \xcd`invo1`'s parameters are those of \xcd`m1`.
      \end{itemize}
\item If \xcd`m2` is acceptable for the invocation \xcd`invo1`,
      \xcd`specific(m1,m2)` returns true; 
\item Construct an invocation \xcd`invo2p`, which is \xcd`invo1` with the
      generic parameters erased.  Let \xcd`invo2` be \xcd`invo2p` with generic
      parameters as inferred by X10's type inference algorithm.  If type
      inference fails, \xcd`specific(m1,m2)` returns false.
\item If \xcd`m2` is acceptable for the invocation \xcd`invo2`,
      \xcd`specific(m1,m2)` returns true; 
\item Otherwise, \xcd`specific(m1,m2)` returns false.
\end{enumerate}

\subsection{Other Disambiguations}

It is possible to have a field of the same name as a method.
Indeed, it is a common pattern to have private field and a public
method of the same name to access it:
\begin{ex}
%~~gen ^^^ MethodResolution_disamb_a
%package MethodResolution_disamb_a;
%~~vis
\begin{xten}
class Xhaver {
  private var x: Int = 0;
  public def x() = x;
  public def bumpX() { x ++; }
}
\end{xten}
%~~siv
%
%~~neg
\end{ex}

\begin{ex}
However, this can lead to syntactic ambiguity in the case where the field
\Xcd{f} of object \xcd`a` is a
function, array, list, or the like, and where \xcd`a` has a method also named
\xcd`f`.  The term \Xcd{a.f(b)} could either mean ``call method \xcd`f` of \xcd`a` upon
\xcd`b`'', or ``apply the function \xcd`a.f` to argument \xcd`b`''.  

%~~gen  ^^^ MethodResolution_disamb_b
%package MethodResolution_disamb_b;
%NOCOMPILE
%~~vis
\begin{xten}
class Ambig {
  public val f : (Int)=>Int =  (x:Int) => x*x;
  public def f(y:int) = y+1;
  public def example() {
      val v = this.f(10);
      // is v 100, or 11?
  }
}
\end{xten}
%~~siv
%
%~~neg
\end{ex}

In the case where a syntactic form \xcdmath"E.m(F$_1$, $\ldots$, F$_n$)" could
be resolved as either a method call, or the application of a field \xcd`E.m`
to some arguments, it will be treated as a method call.  
The application of \xcd`E.m` to some arguments can be specified by adding
parentheses:  \xcdmath"(E.m)(F$_1$, $\ldots$, F$_n$)".

\begin{ex}

%~~gen ^^^ MethodResolution_disamb_c
%package MethodResolution_disamb_c;
%NOCOMPILE
%~~vis
\begin{xten}
class Disambig {
  public val f : (Int)=>Int =  (x:Int) => x*x;
  public def f(y:int) = y+1;
  public def example() {
      assert(  this.f(10)  == 11  );
      assert( (this.f)(10) == 100 );
  }
}
\end{xten}
%~~siv
%
%~~neg

\end{ex}


\subsection{Static Nested Classes}
\index{class!static nested}
\index{class!nested}
\index{static nested class}

One class (or struct or interface) may be nested within another.  The simplest
way to do this is as a \xcd`static` nested class. 
For most purposes, a static nested class behaves like a top-level class.
However, a static inner class has access to private static
fields and methods of its containing class.  

Nested interfaces and static structs are permitted as well.

%~~gen
% package Classes.StaticNested;
%~~vis
\begin{xten}
class Outer {
  private static val priv = 1;
  private static def special(n:Int) = n*n;
  public static class StaticNested {
     static def reveal(n:Int) = special(n) + priv;
  }
}
\end{xten}
%~~siv
%
%~~neg

\subsection{Inner Classes}
\index{class!inner}
\index{inner class}


Non-static nested classes are called {\em inner classes}. An inner class
instance can be thought of as a very elaborate member of an object --- one
with a full class structure of its own.   The crucial characteristic of an
inner class instance is that it has an implicit reference to an instance of
its containing class.  


This feature is particularly useful when an instance of the inner class makes
no sense without reference to an instance of the outer, and is closely tied to
it.  For example, consider a range class, describing a span of integers {$m$}
to {$n$}, and an iterator over the range.  The iterator might as well have
access to the range object, and there is little point to discussing
iterators-over-ranges without discussing ranges as well.
In the following example, the inner class \xcd`RangeIter` iterates over the
enclosing \xcd`Range`.  

It has its own private cursor field \xcd`n`, telling
where it is in the iteration; different iterations over the same \xcd`Range`
can exist, and will each have their own cursor.
It is perhaps unwise to use the name \xcd`n` for a field of the inner class,
since it is also a field of the outer class, but it is legal.  (It can happen
by accident as well -- \eg, if a programmer were to add a field \xcd`n` to a
superclass of the  outer class, the inner class would still work.)
It does not even
interfere with the inner class's ability to refer to the outer class's \xcd`n`
field: the cursor initialization 
refers to the \xcd`Range`'s lower bound through a fully qualified name
\xcd`Range.this.n`.
Its \xcd`hasNext()` method refers to the outer class's \xcd`m` field, which is
not shadowed.


%~~gen
% package Classes.InnerClasses.Range.Against.The.Machine;
%~~vis
\begin{xten}
class Range(m:Int, n:Int) implements Iterable[Int]{
  public def iterator ()  = new RangeIter();
  private class RangeIter implements Iterator[Int] {
     private var n : Int = m;
     public def hasNext() = n <= Range.this.n;
     public def next() = n++;
  }
  public static def main(argv:Array[String](1)) {
    val r = new Range(3,5);
    for(i in r) Console.OUT.println("i=" + i);
  }
}
\end{xten}
%~~siv
%
%~~neg

An inner class has full access to the members of its enclosing class, both
static and instance.  In particular, it can access \xcd`private` information,
just as methods of the enclosing class can.  

An inner class can have its own members.  
Inside instance methods of an inner class, \xcd`this` refers to the instance
of the {\em inner} class.  The instance of the outer class can be accessed as
{\em Outer}\xcd`.this` (where {\em Outer} is the name of the outer class).
If, for some dire reason, it is necessary to have an inner class within an inner
class, the innermost class can refer to the \xcd`this` of either outer class
by using its name.

An inner class can inherit from any class in scope,
with no special restrictions. \xcd`super` inside an inner class refers to the
inner class's superclass. If it is necessary to refer to the outer classes's
superclass, use a qualified name of the form {\em Outer}\xcd`.super`.

The only restriction placed on the members of inner classes is that static
fields of an inner class must be compile-time constant expressions. 

\index{inner class!extending}
An inner class \xcd`IC1` of some outer class \xcd`OC1` can be extended by
another class \xcd`IC2`. However, since an \xcd`IC1` only exists as a
dependent of an \xcd`OC1`, each \xcd`IC2` must be associated with an \xcd`OC1`
--- or a subtype thereof --- as well.   For example, one often extends an
inner class when one extends its outer class: 
%~~gen
% package Classes.Innerclasses.Are.For.Innermasses;
%~~vis
\begin{xten}
class OC1 {
   class IC1 {}
}
class OC2 extends OC1 {
   class IC2 extends IC1 {} 
}
\end{xten}
%~~siv
%
%~~neg

The hiding of method names has one fine point.  If an inner class defines a
method named \xcd`doit`, then {\em all} methods named \xcd`doit` from the
outer class are hidden --- even if they have different argument types than the
one defined in the inner class.
They are still accessible via
\xcd`Outer.this.doit()`, but not simply via \xcd`doit()`.  The following code
is correct, but would not be correct if the ERROR line were uncommented.

%~~gen
% package Classes.Innerclasses.StupidOverloading;
%~~vis
\begin{xten}
class Outer {
  def doit() {}
  def doit(String) {}
  class Inner { 
     def doit(Boolean, Outer) {}
     def example() {
        doit(true, Outer.this);
        Outer.this.doit();
        //ERROR: doit("fails");
     }
  }
}
\end{xten}
%~~siv
%
%~~neg


\subsubsection{Constructors and Inner Classes}
\index{inner class!constructor}

If \xcd`IC` is an inner class of \xcd`OC`, then instance code in the body of
\xcd`OC` can create instances of \xcd`IC` simply by calling a constructor
\xcd`new IC(...)`: 
%~~gen
% package Classes.Innerclasses.Constructors.Easy;
%~~vis
\begin{xten}
class OC {
  class IC {}
  def method(){
    val ic = new IC();
  }
}
\end{xten}
%~~siv
%
%~~neg

Instances of \xcd`IC` can be constructed from elsewhere as well.  Since every
instance of \xcd`IC` is associated with an instance of \xcd`OC`, an \xcd`OC`
must be supplied to the \xcd`IC` constructor.  The syntax for doing so is: 
\xcd`oc.new IC()`.  For example: 
%~~gen
% package Classes.Innerclasses.Constructors.Whythesnorkisthissocomplicated;
%~~vis
\begin{xten}
class OC {
  class IC {}
  static val oc1 = new OC();
  static val oc2 = new OC();
  static val ic1 = oc1.new IC();
  static val ic2 = oc2.new IC();
}
class Elsewhere{
  def method(oc : OC) {
    val ic = oc.new IC();
  }
}
\end{xten}
%~~siv
%
%~~neg




\noo{Local Classes}


%% vj Thu Sep 19 21:34:13 EDT 2013
% updated for v2.4 -- no change.
\chapter{Structs}
\label{XtenStructs}
\label{StructClasses}
\label{Structs}
\index{struct}


X10 objects are a powerful general-purpose programming tool. However, the
power must be paid for in space and time. In space, a typical object
implementation requires some extra memory for run-time class information, as
well as a pointer for each reference to the object. In time, a typical object
requires an extra indirection to read or write data, and some run-time
computation to figure out which method body to call.

For high-performance computing, this overhead may not be acceptable for all
objects. X10 provides structs, which are stripped-down objects. They are less
powerful than objects; in particular they lack inheritance and mutable fields.
Without inheritance, method calls do not need to do any lookup; they can be
implemented directly. Accordingly, structs can be implemented and used more
cheaply than objects, potentially avoiding the space and time overhead.
(Currently, the C++ back end avoids the overhead, but the Java back end
implements structs as Java objects and does not avoid it.)



Structs and classes are interoperable. Both can implement interfaces; in
particular, like all X10 values they implement \xcd`Any`.  Subroutines 
whose arguments are defined by interfaces can take both structs and classes.
(Some caution is necessary here: referring to a struct through an interface
requires overhead similar to that required for an object.)



In many cases structs can be converted to classes or classes to structs,
within the constraints of structs. If you start off defining a struct and
decide you need a class instead, the code change required is simply changing
the keyword \xcd`struct` to \xcd`class`. If you have a class that does not use
inheritance or mutable fields, it can be converted to a struct by changing its
keyword. Client code using the struct that was a class will need certain
changes: \eg, the \xcd`new` keyword must be added in constructor calls, and
structs (unlike classes) cannot be \xcd`null`.    





\section{Struct declaration}
\index{struct!declaration}

%##(StructDecln TypeParamsI Properties Guard Interfaces ClassBody
\begin{bbgrammar}
%(FROM #(prod:StructDecln)#)
         StructDecln \: Mods\opt \xcd"struct" Id TypeParamsI\opt Properties\opt Guard\opt Interfaces\opt ClassBody & (\ref{prod:StructDecln}) \\
%(FROM #(prod:TypeParamsI)#)
         TypeParamsI \: \xcd"[" TypeParamIList \xcd"]" & (\ref{prod:TypeParamsI}) \\
%(FROM #(prod:Properties)#)
          Properties \: \xcd"(" PropList \xcd")" & (\ref{prod:Properties}) \\
%(FROM #(prod:Guard)#)
               Guard \: DepParams & (\ref{prod:Guard}) \\
%(FROM #(prod:Interfaces)#)
          Interfaces \: \xcd"implements" InterfaceTypeList & (\ref{prod:Interfaces}) \\
%(FROM #(prod:ClassBody)#)
           ClassBody \: \xcd"{" ClassMemberDeclns\opt \xcd"}" & (\ref{prod:ClassBody}) \\
\end{bbgrammar}
%##)



All fields of a struct must be \xcd`val`.

A struct \Xcd{S} cannot contain a field of type \Xcd{S}, or a field of struct
type \Xcd{T} which, recursively, contains a field of type \Xcd{S}.  This
restriction is necessary to permit \xcd`S` to be implemented as a contiguous
block of memory of size equal to the sum of the sizes of its fields.  


Values of a struct \Xcd{C} type can be created by invoking a constructor
defined in \Xcd{C}.  Unlike for classes, the  \Xcd{new} keyword is optional
for struct constructors.  

\begin{ex}
Leaving out \xcd`new` can improve readability in some cases: 
%~~gen ^^^ Structs10
% package Structs.For.Muckts;
%~~vis
\begin{xten}
struct Polar(r:Double, theta:Double){
  def this(r:Double, theta:Double) {property(r,theta);}
  static val Origin = Polar(0,0);
  static val x0y1   = Polar(1, 3.14159/2);
  static val x1y0   = new Polar(1, 0); 
}
\end{xten}
%~~siv
%
%~~neg


When a struct and a method have the same name (often in violation of the X10
capitalization convention), 
\xcd`new` may be used to resolve to the struct's constructor.  
%~~gen ^^^ Structs2w3o
% package Structs2w3o;
%~~vis
\begin{xten}
struct Ambig(x:Long) {
  static def Ambig(x:Long) = "ambiguity please";
  static def example() { 
    val useMethod      = Ambig(1);
    val useConstructor = new Ambig(2);
  }
}
\end{xten}
%~~siv
%
%~~neg

\end{ex}

Structs support the same notions of generics, properties, and constrained
types that classes do.  

\begin{ex}

%~~gen ^^^ Structs6i5t
% package Structs6i5t;
%~~vis
\begin{xten}
struct Exam[T](nQuestions:Long){T <: Question} {
  public static interface Question {}
  // ... 
}
\end{xten}
%~~siv
%
%~~neg


\end{ex}

%%NOW_GONE%% \begin{ex}The \xcd`Pair` type below provides pairs
%%NOW_GONE%% of values; the \xcd`diag()` method applies only when the two elements of the
%%NOW_GONE%% pair are equal, and returns that common value: 
%%NOW_GONE%% %~x~gen ^^^ Structs20
%%NOW_GONE%% % package Structs20;
%%NOW_GONE%% %~x~vis
%%NOW_GONE%% \begin{xten}
%%NOW_GONE%% struct Pair[T,U](t:T, u:U) {
%%NOW_GONE%%   def this(t:T, u:U) { property(t,u); }
%%NOW_GONE%%   def diag(){T==U && t==u} = t;
%%NOW_GONE%% }
%%NOW_GONE%% \end{xten}
%%NOW_GONE%% %~x~siv
%%NOW_GONE%% % class Hook{ def run() {
%%NOW_GONE%% %   val p = Pair(3,3); 
%%NOW_GONE%% %   return p.diag() == 3;
%%NOW_GONE%% % }}
%%NOW_GONE%% %~x~neg
%%NOW_GONE%% \end{ex}

\section{Boxing of structs}
\index{auto-boxing!struct to interface}
\index{struct!auto-boxing}
\index{struct!casting to interface}
\label{auto-boxing} 
If a struct \Xcd{S} implements an interface \Xcd{I} (\eg, \Xcd{Any}),
a value \Xcd{v} of type \Xcd{S} can be assigned to a variable of type
\Xcd{I}. The implementation creates an object \Xcd{o} that is an
instance of an anonymous class implementing \Xcd{I} and containing
\Xcd{v}.  The result of invoking a method of \Xcd{I} on \Xcd{o} is the
same as invoking it on \Xcd{v}. This operation is termed {\em auto-boxing}.
It allows full interoperability of structs and objects---at the cost of losing
the extra efficiency of the structs when they are boxed.

In a generic class or struct obtained by instantiating a type parameter
\Xcd{T} with a struct \Xcd{S}, variables declared at type \Xcd{T} in the body
of the class are not boxed. They are implemented as if they were declared at
type \Xcd{S}.

\begin{ex}
The rail \xcd`aa` in the following example is a \xcd`Rail[Any]`.  It
initially holds two objects.  Then, its elements are replaced by two structs,
both of which are auto-boxed.  Note that no fussing is required to put an
integer into a \xcd`Rail[Any]`.  
However, a rail of structs, such as \xcd`ah`, holds {\em unboxed} structs
and does not incur boxing overhead.
%~~gen ^^^ Structs3q6l
% package Structs3q6l;
%~~vis
\begin{xten}
struct Horse(x:Long){
  static def example(){
    val aa : Rail[Any] = ["a String" as Any, "another one"];
    aa(0) = Horse(8);
    aa(1) = 13;
    val ah : Rail[Horse] = [Horse(7), Horse(13)];
  }
}
\end{xten}
%~~siv
%
%~~neg


\end{ex}

\section{Optional Implementation of {\tt Any} methods}
\label{StructAnyMethods}
\index{Any!structs}

Two
structs are equal (\Xcd{==}) if and only if their corresponding fields
are equal (\Xcd{==}). 

All structs implement \Xcd{x10.lang.Any}. 
Structs are required to implement the following methods from \xcd`Any`.  
Programmers need not provide them; X10 will produce them automatically if 
the program does not include them. 
\begin{xten}
  public def equals(Any):Boolean;
  public def hashCode():Int;
  public def typeName():String;
  public def toString():String;  
\end{xten}


A programmer who provides an explicit implementation
of \Xcd{equals(Any)} for a struct \Xcd{S} should also consider
supplying a definition for \Xcd{equals(S):Boolean}. This will often
yield better performance since the cost of an upcast to \Xcd{Any} and
then a downcast to \Xcd{S} can be avoided.

\section{Primitive Types}
\index{types!primitive}
\index{primitive types}
\index{Int}
\index{UInt}
\index{Long}
\index{ULong}
\index{Char}
\index{Byte}
\index{UByte}
\index{Boolean}
\index{Short}
\index{UShort}
\index{Float}
\index{Double}

Certain types that might be built in to other languages are in fact
implemented as structs in package \xcd`x10.lang` in X10. Their methods and
operations are often provided with \xcd`@Native` (\Sref{NativeCode}) rather
than X10 code, however. These types are:
\begin{xten}
Boolean, Char, Byte, Short, Int, Long
Float, Double, UByte, UShort, UInt, ULong
\end{xten}

\subsection{Signed and Unsigned Integers}
\index{types!unsigned}
\index{integers!unsigned}
\index{unsigned}

X10 has an unsigned integer type corresponding to each integer type:
\xcd`UInt` is an unsigned \xcd`Int`, and so on. These types can be used for
binary programming, or when an extra bit of precision for counters or other
non-negative numbers is needed in integer arithmetic. However, X10 does not
otherwise encourage the use of unsigned arithmetic.




 
%%WRONG%% \section{Generic programming with structs}
%%WRONG%% \index{struct!generic}
%%WRONG%% \index{generic!struct}
%%WRONG%% 
%%WRONG%% 
%%WRONG%% 
%%WRONG%% The programmer must be aware of the different interpretations of
%%WRONG%% equality (\Sref{StableEquality}) for structs and classes and ensure that the
%%WRONG%% code is correctly written for both cases. 
%%WRONG%% 
%%WRONG%% \index{isObject}
%%WRONG%% \index{isStruct}
%%WRONG%% \index{isFunction}
%%WRONG%% Three static methods on \xcd`x10.lang.System` 
%%WRONG%% allow you to tell what kind of value \xcd`x` is: object,
%%WRONG%% struct, or function.  
%%WRONG%% \xcd`System.isObject(x)` returns true if \xcd`x` is a value of \xcd`Object`
%%WRONG%% type, including \xcd`null`; \xcd`System.isStruct(x)` returns true if \xcd`x`
%%WRONG%% is a \xcd`struct`; \xcd`System.isFunction(x)` returns true if \xcd`x` is a
%%WRONG%% closure value.  Precisely one of these three functions returns true for any
%%WRONG%% X10 value \xcd`x`.  
%%WRONG%% 
%%WRONG%% \begin{xten}
%%WRONG%% val x:X = ...;
%%WRONG%% if (System.isObject(x)) { // x is a real object
%%WRONG%%    val x2 = x as Object; // this cast will always succeed.
%%WRONG%%    ...
%%WRONG%% } else if (System.isStruct(x)) { // x is a struct
%%WRONG%%    ...
%%WRONG%% } else {  
%%WRONG%%   assert System.isFunction(x);
%%WRONG%% }
%%WRONG%% \end{xten}
%%WRONG%%  
  
\section{Example structs}

\xcd`x10.lang.Complex` provides a detailed example of a practical struct,
suitable for use in a library.  For a shorter example, we define the
\xcd`Pair` struct.   A \xcd`Pair` packages
two values of possibly unrelated type together in a single value, \eg, to
return two values from a function.  

\xcd`divmod` computes the quotient and remainder of \xcdmath"a $\div$ b" (naively).
It returns both, packaged as a \xcd`Pair[UInt, UInt]`.  Note that the
constructor uses type inference, and that the quotient and remainder are
accessed through the \xcd`first` and \xcd`second` fields.
%~~gen ^^^ Structs30
% package Structs30Pair;
%~~vis
\begin{xten}
struct Pair[T,U] {
    public val first:T;
    public val second:U;
    public def this(first:T, second:U):Pair[T,U] {
        this.first = first;
        this.second = second;
    }
    public def toString() 
        = "(" + first + ", " + second + ")";
}
class Example {
  static def divmod(var a:UInt, b:UInt): Pair[UInt, UInt] {
     assert b > 0u;
     var q : UInt = 0un;
     while (a > b) {q += 1un; a -= b;}
     return Pair(q, a); 
  }
  static def example() {
     val qr = divmod(22un, 7un);
     assert qr.first == 3un && qr.second == 1un;
  }
}
\end{xten}
%~~siv
%class Hook{ def run() { Example.example(); return true; } } 
%~~neg

\section{Nested Structs}
\index{struct!static nested}
\index{static nested struct}

Static nested structs may be defined, essentially as static nested classes
except for making them structs
(\Sref{StaticNestedClasses}).  Inner structs may be defined, essentially as
inner classes except making them structs (\Sref{InnerClasses}).
\limitationx{} Nested structs must be currently be declared static.

\section{Default Values of Structs}
\label{sect:DefaultValuesOfStructs}


If all fields of a struct have default values, then the struct has a
default value, \viz, the struct whose fields are all set to their
default values.  If some field does not have a default value, neither
does the struct.

\begin{ex}

In the following code, the \xcd`Example` struct has a default value whose
\xcd`i` field is \xcd`0`.  If an \xcd`Example` is ever constructed by the
constructor, its \xcd`i` field will be \xcd`1`.  This program does a slightly
subtle dance to get ahold of a default \xcd`Example`, by having an instance
\xcd`var` (which, unlike most kinds of variables, does not need to get
initialized before use (though that exemption only applies if its type has a
default value)).   As the \xcd`assert` confirms, the default \xcd`Example`
does indeed have an \xcd`i` field of \xcd`0`.

%~~gen ^^^ Structs6r3w
% package Structs6r3w;
% 
%~~vis
\begin{xten}
class StructDefault {
  static struct Example {
    val i : Long;
    def this() { i = 1; }
  }
  var ex : Example; 
  static def example() {
     val ex = (new StructDefault()).ex;
     assert ex.i == 0;
  }
\end{xten}
%~~siv
% }
%  class Hook { def run() { StructDefault.example(); return true; } } 
%~~neg


\end{ex}


\section{Converting Between Classes And Structs}

Code written using structs can be modified to use classes, or vice versa.
Caution must be used in certain places. 

Class and struct {\em definitions} are syntactically nearly identical:
change the \xcd`class` keyword to \xcd`struct` or vice versa.  Of course,
certain important class features can't be used with structs, such as
inheritance and \xcd`var` fields. 

Converting code that {\em uses} the class or struct requires a certain amount
of caution.
Suppose, in particular, that we want to convert the class \xcd`Class2Struct`
to a struct, and \xcd`Struct2Class` to a class.
%~~gen ^^^ Structs40
%package Structs.Converting;
%~~vis
\begin{xten}
class Class2Struct {
  val a : Long;
  def this(a:Long) { this.a = a; }
  def m() = a;
}
struct Struct2Class { 
  val a : Long;
  def this(a:Long) { this.a = a; }
  def m() = a;
}
\end{xten}
%~~siv
%
%~~neg

\begin{enumerate}

\item Class constructors require the \xcd`new` keyword; struct constructors
      allow  it but do not require it.  
      \xcd`Struct2Class(3)` to will need to be converted to 
      \xcd`new Struct2Class(3)`.

\item Objects and structs have different notions of \xcd`==`.  
      For objects, \xcd`==` means ``same object''; for structs, it means
      ``same contents''. Before conversion, both \xcd`assert`s in the
      following program succeed.  After converting and fixing constructors,
      both of them fail.
%~~gen ^^^ Structs50
%package Structs.Converting.And.Conniving;
% class Class2Struct {
%   val a : Long;
%   def this(a:Long) { this.a = a; }
%   def m() = a;
% }
% struct Struct2Class { 
%   val a : Long;
%   def this(a:Long) { this.a = a; }
%   def m() = a;
% }
%class Example {
% static def Examplle() {
%~~vis
\begin{xten}
val a = new Class2Struct(2);
val b = new Class2Struct(2);
assert a != b;
val c = Struct2Class(3);
val d = Struct2Class(3);
assert c==d;
\end{xten}
%~~siv
%}}
%~~neg

\item Objects can be set to \xcd`null`.  Structs cannot.  

\item The rules for default values are quite different.  
The default value of an object type (if it exists) is \xcd`null`, which behaves quite
differently from an ordinary object of that type; \eg, you cannot call methods
on \xcd`null`, whereas you can on an ordinary object. The default value for
a struct type (if it exists) is a struct like any other of its type, and you
can call methods on it as for any other.


\end{enumerate}



\chapter{Functions}
\label{Functions}
\label{functions}
\index{functions}
\label{Closures}

\section{Overview}
Functions, the last of the three kinds of values in X10, encapsulate pieces of
code which can be applied to a vector of arguments to produce a value.
Functions, when applied, can do nearly anything that any other code could do:
fail to terminate, throw an exception, modify variables, spawn activities,
execute in several places, and so on. X10 functions are not mathematical
functions: the \xcd`f(1)` may return \xcd`true` on one call and \xcd`false` on
an immediately following call.

It is a limitation of \XtenCurrVer{} that functions do not support
type arguments. This limitation may be removed in future versions of
the language.

A \emph{function literal} \xcd"(x1:T1,..,xn:Tn){c}:T=>e" creates a function of
type\\ \xcd"(x1:T1,...,xn:Tn){c}=>T" (\Sref{FunctionType}).  For example, 
\xcd`(x:Int) => x*x` is a function literal describing the squaring function on
integers.   
\xcd`null` is also a function value.

\limitationx{} Function literals do not currently support guards. 

Function application is written \xcd`f(a,b,c)`, following common mathematical
usage. 
\index{Exception!unchecked}


The function body may be a block.  To compute integer squares by repeated
addition (inefficiently), one may write: 
%~~gen
% package Functions.Are.For.Spunctions;
% class Examplllll {
% static 
%~~vis
\begin{xten}
val sq: (Int) => Int 
      = (n:Int) => {
           var s : Int = 0;
           val abs_n = n < 0 ? -n : n;
           for ([i] in 1..abs_n) s += abs_n;
           s
        };
\end{xten}
%~~siv
%}
%~~neg




A function literal evaluates to a function entity {$\phi$}. When {$\phi$} is
applied to a suitable list of actual parameters \xcd`a1`-\xcd`an`, it
evaluates \xcd`e` with the formal parameters bound to the actual parameters.
So, the following are equivalent, where \xcd`e` is an expression involving
\xcd`x1` and \xcd`x2`\footnote{Strictly, there are a few other requirements;
  \eg, \xcd`result` must be a \xcd`var` of type \xcd`T` defined outside the
  outer block, the variables \xcd`a1` and \xcd`a2` had better not appear in
  \xcd`e`, and everything in sight had better typecheck properly.}

%~~gen
% package functions2.why.is.there.a.two;
% abstract class FunctionsTooManyFlippingFunctions[T, T1, T2]{
% abstract def arg1():T1;
% abstract def arg2():T2;
% def thing1(e:T) {var result:T;
%~~vis
\begin{xten}
{
  val f = (x1:T1,x2:T2){true}:T => e;
  val a1 : T1 = arg1();
  val a2 : T2 = arg2();
  result = f(a1,a2);
}
\end{xten}
%~~siv
%}}
%~~neg
and 
%~~gen
% package functions2.why.is.there.a.two.but.here.is.the.other.one;
% abstract class FunctionsTooManyFlippingFunctions[T, T1, T2]{
% abstract def arg1():T1;
% abstract def arg2():T2;
% def thing1(e:T) {var result:T;
%~~vis
\begin{xten}
{
  val a1 : T1 = arg1();
  val a2 : T2 = arg2();
  {
     val x1 : T1 = a1;
     val x2 : T2 = a2;
     result = e;
  }  
}
\end{xten}
%~~siv
%}}
%~~neg
\noindent
This doesn't quite work if the body is a statement rather than an expression.
A few language features are forbidden (\xcd`break` or \xcd`continue` of a loop
that surrounds the function literal) or mean something different (\xcd`return`
inside a function returns from the function). 





The \emph{method selector expression} \Xcd{e.m.(x1:T1,...,xn:Tn)} (\Sref{MethodSelectors})
permits the specification of the function underlying
the method \Xcd{m}, which takes arguments of type \Xcd{(x1:T1,..., xn:Tn)}.
Within this function, \Xcd{this} is bound to the result of evaluating \Xcd{e}.

Function types may be used in \Xcd{implements} clauses of class
definitions. Instances of such classes may be used as functions of the
given type.  Indeed, an object may behave like any (fixed) number of
functions, since the class it is an instance of may implement any
(fixed) number of function types.

%\section{Implementation Notes}
%\begin{itemize}
%
%\item Note that e.m.(T1,...,Tn) will evaluate e to create a
%  function. This function will be applied later to given
%  arguments. Thus this syntax can be used to evaluate the receiver of
%  a method call ahead of the actual invocation. The resulting function
%  can be used multiple times, of course.
%\end{itemize}


\section{Function Literals}
\index{literal!function}
\label{FunctionLiteral}

\Xten{} provides first-class, typed functions, including
\emph{closures}, \emph{operator functions}, and \emph{method
  selectors}.

\begin{grammar}
ClosureExpression \:
        \xcd"("
        Formals\opt
        \xcd")"
\\ &&
        Guard\opt
        ReturnType\opt
        \xcd"=>" ClosureBody \\
ClosureBody \:
        Expression \\
        \| \xcd"{" Statement\star \xcd"}" \\
        \| \xcd"{" Statement\star Expression \xcd"}" \\
\end{grammar}

Functions have zero or more formal parameters and an optional return type.
The body has the 
same syntax as a method body; it may be either an expression, a block
of statements, or a block terminated by an expression to return. In
particular, a value may be returned from the body of the function
using a return statement (\Sref{ReturnStatement}). 

The type of a
function is a function type (\Sref{FunctionType}).  In some cases the
return type \Xcd{T} is also optional and defaults to the type of the
body. If a formal \Xcd{xi} does not occur in any
\Xcd{Tj}, \Xcd{c}, \Xcd{T} or \Xcd{e}, the declaration \Xcd{xi:Ti} may
be replaced by just \Xcd{Ti}: \xcd`(Int)=>7` is the integer function returning
7 for all inputs.

\label{ClosureGuard}

As with methods, a function may declare a guard to
constrain the actual parameters with which it may be invoked.
The guard may refer to the type parameters, formal parameters,
and any \xcd`val`s in scope at the function expression.

The body of the function is evaluated when the function is
invoked by a call expression (\Sref{Call}), not at the function's
place in the program text.

As with methods, a function with return type \xcd"void" cannot
have a terminating expression. 
If the return type is omitted, it is inferred, as described in
\Sref{TypeInference}.
It is a static error if the return type cannot be inferred.  \Eg,
\xcd`(Int)=>null` is not well-defined; X10 does not know which type of
\xcd`null` is intended.  
%~~exp~~`~~`~~ ~~
But \xcd`(Int):Array[Double](1) => null` is legal.


\begin{example}
The following method takes a function parameter and uses it to
test each element of the list, returning the first matching
element.  It returns \xcd`absent` if no element matches.

%~~gen
% package functions2.oh.no;
% import x10.util.*;
% class Finder {
% static 
%~~vis
\begin{xten}

def find[T](f: (T) => Boolean, xs: List[T], absent:T): T = {
  for (x: T in xs)
    if (f(x)) return x;
  absent
  }
\end{xten}
%~~siv
% }
%~~neg

The method may be invoked thus:
%~~gen
% package functions2.oh.no.my.ears;
% import x10.util.*;
% class Finderator {
% static def find[T](f: (T) => Boolean, xs: x10.util.List[T], absent:T): T = {
%  for (x: T in xs)
%    if (f(x)) return x;
%  absent
%}
% static def checkery() {
%~~vis
\begin{xten}
xs: List[Int] = new ArrayList[Int]();
x: Int = find((x: Int) => x>0, xs, 0);
\end{xten}
%~~siv
%}}
%~~neg

\end{example}



\subsection{Outer variable access}

In a function
\xcdmath"(x$_1$: T$_1$, $\dots$, x$_n$: T$_n$){c} => { s }"
the types \xcdmath"T$_i$", the guard \xcd"c" and the body \xcd"s"
may access many, though not all, sorts of variables from outer scopes.  
Specifically, they can access: 
\begin{itemize}
\item All fields of the enclosing object and class;
\item All type parameters;
\item All \xcd`val` variables;
\end{itemize}
\noindent
\xcd`var` variables cannot be accessed.


The function body may refer to instances of enclosing classes using
the syntax \xcd"C.this", where \xcd"C" is the name of the
enclosing class.  \xcd`this` refers to the instance of the immediately
enclosing class, as usual.

For example, the following is legal.  
However, the commented-out line would not be legal.
Note that \xcd`a` is not a local \xcd`var` variable. It is a field of
\xcd`this`. A reference to \xcd`a` is simply short for \xcd`this.a`, which is
a use of a \xcd`val` variable (\xcd`this`).  
%~~gen
% package Functions.areLikeGrunctions.fromConjunctionJunctions;
%~~vis
\begin{xten}
class Lambda {
   var a : Int = 0;
   val b = 0;
   def m(var c : Int, val d : Int) {
      var e : Int = 0;
      val f : Int = 0;
      val closure = (var i: Int, val j: Int) => {
    	  return a + b + d + f + j + this.a + Lambda.this.a;
          // ILLEGAL: return c + e + i;
      };
      return closure;
   }
}
\end{xten}
%~~siv
%
%~~neg

%%SHARED%% 
%%SHARED%% 
%%SHARED%% Access to variables is not automatically atomic.  As
%%SHARED%% with any code that might mutate shared data concurrently, be sure to protect
%%SHARED%% references to mutable shared state with \xcd`atomic`. For example, the
%%SHARED%% following code returns a pair of closures which operate on the same shared
%%SHARED%% variable \xcd`a`, which are concurrency-safe---even if invoked many times
%%SHARED%% simultaneously. Without \xcd`atomic`, it would no longer be concurrency-safe.
%%SHARED%% 
%%SHARED%% 
%%SHARED%% %~s~gen
%%SHARED%% % package Functions2.Are.All.Too.Much;
%%SHARED%% % class Fun2Frivols {
%%SHARED%% %~s~vis
%%SHARED%% \begin{xten}
%%SHARED%%   def counters() {
%%SHARED%%       var a : Int = 0;
%%SHARED%%        return [
%%SHARED%%           () => {atomic a ++;},
%%SHARED%%           () => {atomic return a;}
%%SHARED%%           ];
%%SHARED%%    }
%%SHARED%% \end{xten}
%%SHARED%% %~s~siv
%%SHARED%% %}
%%SHARED%% %
%%SHARED%% %~s~neg


%SHARED% \begin{note}
%SHARED% The main activity may run in parallel with any
%SHARED% functions it creates. Hence even the read of an outer variable by the
%SHARED% body of a function may result in a race condition. Since functions are
%SHARED% first-class, the analysis of whether a function may execute in parallel
%SHARED% with the activity that created it may be difficult.
%SHARED% \end{note}

%% vj: This should be verified.
%\begin{note}
%The rule for accessing outer variables from function bodies
%should be the same as the rule for accessing outer variables from local
%or anonymous classes.
%\end{note}

\section{Method selectors}
\label{MethodSelectors}
\index{function!method selector}
\index{method!underlying function}

A method selector expression allows a method to be used as a
first-class function, without writing a function expression for it.
For example, consider a class \xcd`Span` defining ranges of integers.  

%~~gen
% package Functions2.Span;
%~~vis
\begin{xten}
class Span(low:Int, high:Int) {
  def this(low:Int, high:Int) {property(low,high);}
  def between(n:Int) = low <= n && n <= high;
  def example() {
    val digit = new Span(0,9);
    val isDigit : (Int) => Boolean = digit.between.(Int);
    if (isDigit(8)) Console.OUT.println("8 is!");
  }
}
\end{xten}
%~~siv
%
%~~neg
\noindent


In \xcd`example()`, 
%~~exp~~`~~`~~ digit:Span~~class Span(low:Int, high:Int) {def this(low:Int, high:Int) {property(low,high);} def between(n:Int) = low <= n && n <= high;}
\xcd`digit.between.(Int)` 
is a unary function testing whether its argument is between zero
and nine.  It could also be written 
%~~exp~~`~~`~~ digit:Span~~class Span(low:Int, high:Int) {def this(low:Int, high:Int) {property(low,high);} def between(n:Int) = low <= n && n <= high;}
\xcd`(n:Int) => digit.between(n)`.

%%GRAMMAR%% This is formalized thus:
%%GRAMMAR%% 
%%GRAMMAR%% \begin{grammar}
%%GRAMMAR%% MethodSelector \:
%%GRAMMAR%%         Primary \xcd"."
%%GRAMMAR%%         MethodName \xcd"."
%%GRAMMAR%%                 TypeParameters\opt \xcd"(" Formals\opt \xcd")" \\
%%GRAMMAR%%       \|
%%GRAMMAR%%         TypeName \xcd"."
%%GRAMMAR%%         MethodName \xcd"."
%%GRAMMAR%%                 TypeParameters\opt \xcd"(" Formals\opt \xcd")" \\
%%GRAMMAR%% \end{grammar}

The \emph{method selector expression} \Xcd{e.m.(T1,...,Tn)} is type
correct only if  the static type of \Xcd{e} is a
class or struct or interface \xcd`V` with a method
\Xcd{m(x1:T1,...xn:Tn)\{c\}:T} defined on it (for some
\Xcd{x1,...,xn,c,T)}. At runtime the evaluation of this expression
evaluates \Xcd{e} to a value \Xcd{v} and creates a function \Xcd{f}
which, when applied to an argument list \Xcd{(a1,...,an)} (of the right
type) yields the value obtained by evaluating \Xcd{v.m(a1,...,an)}.

Thus, the method selector

\begin{xtenmath}
e.m.(T$_1$, $\dots$, T$_n$)
\end{xtenmath}
\noindent behaves as if it were the function
\begin{xtenmath}
((v:V)=>
  (x$_1$: T$_1$, $\dots$, x$_n$: T$_n$){c} 
  => v.m(x$_1$, $\dots$, x$_n$))
(e)
\end{xtenmath}



Because of overloading, a method name is not sufficient to
uniquely identify a function for a given class.
One needs the argument type information as well.
The selector syntax (dot) is used to distinguish \xcd"e.m()" (a
method invocation on \xcd"e" of method named \xcd"m" with no arguments)
from \xcd"e.m.()"
(the function bound to the method). 

A static method provides a binding from a name to a function that is
independent of any instance of a class; rather it is associated with the
class itself. The static function selector
\xcdmath"T.m.(T$_1$, $\dots$, T$_n$)" denotes the
function bound to the static method named \xcd"m", with argument types
\xcdmath"(T$_1$, $\dots$, T$_n$)" for the type \xcd"T". The return type
of the function is specified by the declaration of \xcd"T.m".

There is no difference between using a function defined directly 
directly using the function syntax, or obtained via static or
instance function selectors.


\section{Operator functions}
\label{OperatorFunction}
\index{function!operator}
Every binary operator (e.g.,
\xcd"+",
\xcd"-",
\xcd"*",
\xcd"/",
\dots) has a family of functions, one for
each type on which the operator is defined. The function can be
selected using the ``\xcd`.`'' syntax:


\begin{xtenmath}
String.+             $\equiv$ (x: String, y: String): String => x + y
Long.-               $\equiv$ (x: Long, y: Long): Long => x - y
Float.-              $\equiv$ (x: Float, y: Float): Float => x - y
Boolean.&            $\equiv$ (x: Boolean, y: Boolean): Boolean => x & y
Int.<                $\equiv$ (x: Int, y: Int): Boolean => x < y
\end{xtenmath}

%~~gen
% package Functions.Operatorfunctionsgracklegrackle;
% class JustATest {
% val dummy = [String.+,
%  Long.-,
%  Float.-,
%  Boolean.&,
%  Int.<
%  ];
% }
%~~vis
\begin{xten}
\end{xten}
%~~siv
%
%~~neg


%%TODO -- fix commented-out lines!

%~~gen
% package Functions2.For.The.Lose;
% class TypecheckThatSillyExample {
%   def checker() {
%    val l1 : (String, String) => String = String.+;
%    val r1 : (String, String) => String = (x: String, y: String): String => x + y;
%    val l2 : (Long,Long) => Long = Long.-;
%    val r2 : (Long,Long) => Long = (x: Long, y: Long): Long => x - y;
%//var v1 : (Float,Float) => Float = Float.-(Float,Float) ;
%var v2 : (Float,Float) => Float = (x: Float, y: Float): Float => x - y;
%//var v3 : (Int) => Int =  Int.-(Int)     ;      ;
%var v4  : (Int) => Int  =  (x: Int): Int => -x;
%var v5 : (Boolean,Boolean) => Boolean = Boolean.&            ;
%var v6 : (Boolean,Boolean) => Boolean =  (x: Boolean, y: Boolean): Boolean => x & y;
%//var v7 : (Boolean) => Boolean = Boolean.!            ;
%var v8 : (Boolean) => Boolean =  (x: Boolean): Boolean => !x;
%//var v9 : (Int,Int) => Boolean = Int.<(Int,Int)       ;
%var v10: (Int,Int) => Boolean =  (x: Int, y: Int): Boolean => x < y;
%//var v11 : (Dist,Place)=>Dist = Dist.|(Place)        ;
%var v12 : (Dist,Place)=>Dist=  (d: Dist, p: Place): Dist => d | p;
%}
% }
%~~vis
%~~siv
%
%~~neg

Unary and binary promotion (\Sref{XtenPromotions}) is not performed
when invoking these
operations; instead, the operands are coerced individually via implicit
coercions (\Sref{XtenConversions}), as appropriate.


%%WE-NEVER-GOT-TO-IT%%  \begin{planned}
%%WE-NEVER-GOT-TO-IT%%  
%%WE-NEVER-GOT-TO-IT%%  {\bf The following is not implemented in version 2.0.3:}
%%WE-NEVER-GOT-TO-IT%%  
%%WE-NEVER-GOT-TO-IT%%  Additionally, for every expression \xcd"e" of a type \xcd"T" at which a binary
%%WE-NEVER-GOT-TO-IT%%  operator \xcd"OP" is defined, the expression \xcd"e.OP" or
%%WE-NEVER-GOT-TO-IT%%  \xcd"e.OP(T)" represents the function
%%WE-NEVER-GOT-TO-IT%%  defined by:
%%WE-NEVER-GOT-TO-IT%%  
%%WE-NEVER-GOT-TO-IT%%  \begin{xten}
%%WE-NEVER-GOT-TO-IT%%  (x: T): T => { e OP x }
%%WE-NEVER-GOT-TO-IT%%  \end{xten}
%%WE-NEVER-GOT-TO-IT%%  
%%WE-NEVER-GOT-TO-IT%%  \begin{grammar}
%%WE-NEVER-GOT-TO-IT%%  Primary \: Expr \xcd"." Operator \xcd"(" Formals\opt \xcd")" \\
%%WE-NEVER-GOT-TO-IT%%          \| Expr \xcd"." Operator \\
%%WE-NEVER-GOT-TO-IT%%  \end{grammar}
%%WE-NEVER-GOT-TO-IT%%  
%%WE-NEVER-GOT-TO-IT%%  %% For every expression \xcd"e" of a type \xcd"T" at which a unary
%%WE-NEVER-GOT-TO-IT%%  %%operator \xcd"OP" is defined, the expression \xcd"e.OP()"
%%WE-NEVER-GOT-TO-IT%%  %% represents the function defined by:
%%WE-NEVER-GOT-TO-IT%%  
%%WE-NEVER-GOT-TO-IT%%  %% \begin{xten}
%%WE-NEVER-GOT-TO-IT%%  %% (): T => { OP e }
%%WE-NEVER-GOT-TO-IT%%  %% \end{xten}
%%WE-NEVER-GOT-TO-IT%%  
%%WE-NEVER-GOT-TO-IT%%  For example,
%%WE-NEVER-GOT-TO-IT%%  one may write an expression that adds one to each member of a
%%WE-NEVER-GOT-TO-IT%%  list \xcd"xs" by:
%%WE-NEVER-GOT-TO-IT%%  
%%WE-NEVER-GOT-TO-IT%%  %%TODO -- when this topic works, make the example wwork too.
%%WE-NEVER-GOT-TO-IT%%  %~x~gen
%%WE-NEVER-GOT-TO-IT%%  % package Functions2.Wants.A.Dinner.Reservation;
%%WE-NEVER-GOT-TO-IT%%  % import x10.util.*;
%%WE-NEVER-GOT-TO-IT%%  % class Reservation {
%%WE-NEVER-GOT-TO-IT%%  % def smerp() {
%%WE-NEVER-GOT-TO-IT%%  %   val xs = new ArrayList[Int]();
%%WE-NEVER-GOT-TO-IT%%  %~x~vis
%%WE-NEVER-GOT-TO-IT%%  \begin{xten}
%%WE-NEVER-GOT-TO-IT%%  xs.map(1.+);
%%WE-NEVER-GOT-TO-IT%%  \end{xten}
%%WE-NEVER-GOT-TO-IT%%  %~x~siv
%%WE-NEVER-GOT-TO-IT%%  % }
%%WE-NEVER-GOT-TO-IT%%  % }
%%WE-NEVER-GOT-TO-IT%%  %
%%WE-NEVER-GOT-TO-IT%%  %~x~neg
%%WE-NEVER-GOT-TO-IT%%  \end{planned}
%%WE-NEVER-GOT-TO-IT%%  
%%WE-NEVER-GOT-TO-IT%%  
\section{Functions as objects of type \Xcd{Any}}
\label{FunctionAnyMethods}

\label{FunctionEquality}
\index{function!equality} \index{equality!function} Two functions \Xcd{f} and
\Xcd{g} are equal if both were obtained by the same evaluation of a function
literal.\footnote{A literal may occur in program text within a loop, and hence
  may be evaluated multiple times.} Further, it is guaranteed that if two
functions are equal then they refer to the same locations in the environment
and represent the same code, so their executions in an identical situation are
indistinguishable. (Specifically, if \xcd`f == g`, then \xcd`f(1)` can be
substituted for \xcd`g(1)` and the result will be identical. However, there is
no guarantee that \xcd`f(1)==g(1)` will evaluate to true. Indeed, there is no
guarantee that \xcd`f(1)==f(1)` will evaluate to true either, as \xcd`f` might
be a function which returns {$n$} on its {$n^{th}$} invocation. However,
\xcd`f(1)==f(1)` and \xcd`f(1)==g(1)` are interchangeable.)
\index{function!==}


Every function type implements all the methods of \Xcd{Any}.
\xcd`f.equals(g)` is equivalent to \xcd`f==g`.  \xcd`f.hashCode()`, 
\xcd`f.toString()`, and \xcd`f.typeName()` are implementation-dependent, but
respect \xcd`equals` and the basic contracts of \xcd`Any`. 

\index{function!equals}
\index{function!hashCode}
\index{function!toString}
\index{function!typeName}
\index{function!home}
\index{function!at(Place)}
\index{function!at(Object)}



\chapter{Expressions}\label{XtenExpressions}\index{expression}

\Xten{} has a rich expression language.
Evaluating an expression produces a value, or, in a few cases, no value. 
Expression evaluation may have side effects, such as change of the value of a 
\xcd`var` variable or a data structure, allocation of new values, or throwing
an exception. 



\section{Literals}
\index{literal}

Literals denote fixed values of built-in types. 
The syntax for literals is given in \Sref{Literals}. 

The type that \Xten{} gives a literal often includes its value. \Eg, \xcd`1`
is of type \xcd`Int{self==1}`, and \xcd`true` is of type
\xcd`Boolean{self==true}`.

\section{{\tt this}}
\index{this}
\index{\Xcd{this}}

%##(Primary
\begin{bbgrammar}
%(FROM #(prod:Primary)#)
             Primary \: \xcd"here" & (\ref{prod:Primary}) \\
                    \| \xcd"[" ArgumentList\opt \xcd"]" \\
                    \| Literal \\
                    \| \xcd"self" \\
                    \| \xcd"this" \\
                    \| ClassName \xcd"." \xcd"this" \\
                    \| \xcd"(" Exp \xcd")" \\
                    \| ClassInstCreationExp \\
                    \| FieldAccess \\
                    \| MethodInvocation \\
                    \| MethodSelection \\
                    \| OperatorFunction \\
\end{bbgrammar}
%##)

The expression \xcd"this" is a  local \xcd`val` containing a reference
to an instance of the lexically enclosing class.
It may be used only within the body of an instance method, a
constructor, or in the initializer of a instance field -- that is, the places
where there is an instance of the class under consideration.

Within an inner class, \xcd"this" may be qualified with the
name of a lexically enclosing class.  In this case, it
represents an instance of that enclosing class.  
For example, \xcd`Outer` is a class containing \xcd`Inner`.  Each instance of
\xcd`Inner` has a reference \xcd`Outer.this` to the \xcd`Outer` involved in its
creation.  \xcd`Inner` has access to the fields of \xcd`Outer.this`, as seen
in the \xcd`outerThree` and \xcd`alwaysTrue` methods.  Note that \xcd`Inner`
has its own \xcd`three` field, which is different from and not even the same
type as \xcd`Outer.this.three`. 
%~~gen ^^^ Expressions10
% package exp.vexp.pexp.lexp.shexp; 
% NOTEST
%~~vis 
\begin{xten}
class Outer {
  val three = 3;
  class Inner {
     val three = "THREE";
     def outerThree() = Outer.this.three;
     def alwaysTrue() = outerThree() == 3;
  }
}
\end{xten}
%~~siv
%
%~~neg

The type of a \xcd"this" expression is the
innermost enclosing class, or the qualifying class,
constrained by the class invariant and the
method guard, if any.

The \xcd"this" expression may also be used within constraints in
a class or interface header (the class invariant and
\xcd"extends" and \xcd"implements" clauses).  Here, the type of
\xcd"this" is restricted so that only properties declared in the
class header itself, and specifically not any members declared in the class
body or in supertypes, are accessible through \xcd"this".

\section{Local variables}

%##(Id
\begin{bbgrammar}
%(FROM #(prod:Id)#)
                  Id \: identifier & (\ref{prod:Id}) \\
\end{bbgrammar}
%##)

A local variable expression consists simply of the name of the local variable,
field of the current object, formal parameter in scope, etc. It evaluates to
the value of the local variable. \xcd`n` in the second line below is a local
variable expression. 
%~~gen  ^^^ Expressions20
% package exp.loc.al.varia.ble; 
% class Example {
% def example() { 
%~~vis
\begin{xten}
val n = 22;
val m = n + 56;
\end{xten}
%~~siv
%} }
%~~neg



\section{Field access}
\label{FieldAccess}
\index{field!access to}

%##(FieldAccess
\begin{bbgrammar}
%(FROM #(prod:FieldAccess)#)
         FieldAccess \: Primary \xcd"." Id & (\ref{prod:FieldAccess}) \\
                    \| \xcd"super" \xcd"." Id \\
                    \| ClassName \xcd"." \xcd"super"  \xcd"." Id \\
                    \| Primary \xcd"." \xcd"class"  \\
                    \| \xcd"super" \xcd"." \xcd"class"  \\
                    \| ClassName \xcd"." \xcd"super"  \xcd"." \xcd"class"  \\
\end{bbgrammar}
%##)

A field of an object instance may be  accessed
with a field access expression.

The type of the access is the declared type of the field with the
actual target substituted for \xcd"this" in the type. 
% If the actual
%target is not a final access path (\Sref{FinalAccessPath}),
%an anonymous path is substituted for \xcd"this".

The field accessed is selected from the fields and value properties
of the static type of the target and its superclasses.

If the field target is given by the keyword \xcd"super", the target's type is
the superclass of the enclosing class.  This form is used to access fields of
the parent class shadowed by same-named fields of the current class.

If the field target is \xcd`Cls.super`, then the target's type is \xcd`Cls`,
which must be an  enclosing class.  This (admittedly
obscure) form is used to access fields of an ancestor class which are shadowed
by same-named fields of some more recent ancestor.  The following example
illustrates all four cases:

%~~gen ^^^ Expressions30
% package exp.re.ssio.ns.fiel.dacc.ess;
% NOTEST
%~~vis
\begin{xten}
class Uncle {
  public static val f = 1;
}
class Parent {
  public val f = 2;
}
class Ego extends Parent {
  public val f = 3;
  class Child extends Ego {
     public val f = 4;
     def expDotId() = this.f; // 4
     def superDotId() = super.f; // 3
     def classNameDotId() = Uncle.f; // 1;
     def cnDotSuperDotId() = Ego.super.f; // 2
  }
}
\end{xten}
%~~siv
%
%~~neg


If the field target is \xcd"null", a \xcd"NullPointerException"
is thrown.

If the field target is a class name, a static field is selected.

It is illegal to access  a field that is not visible from
the current context.
It is illegal to access a non-static field
through a static field access expression.

\section{Function Literals}
Function literals are described in \Sref{Functions}.

\section{Calls}
\label{Call}
\label{MethodInvocation}
\label{MethodInvocationSubstitution}
\index{invocation}
\index{call}
\index{invocation!method}
\index{call!method}
\index{invocation!function}
\index{call!function}
\index{method!calling}
\index{method!invoking}


%##(MethodInvocation ArgumentList
\begin{bbgrammar}
%(FROM #(prod:MethodInvocation)#)
    MethodInvocation \: MethodPrimaryPrefix \xcd"(" ArgumentList\opt \xcd")" & (\ref{prod:MethodInvocation}) \\
                    \| MethodSuperPrefix \xcd"(" ArgumentList\opt \xcd")" \\
                    \| MethodClassNameSuperPrefix \xcd"(" ArgumentList\opt \xcd")" \\
                    \| MethodName TypeArguments\opt \xcd"(" ArgumentList\opt \xcd")" \\
                    \| Primary \xcd"." Id TypeArguments\opt \xcd"(" ArgumentList\opt \xcd")" \\
                    \| \xcd"super" \xcd"." Id TypeArguments\opt \xcd"(" ArgumentList\opt \xcd")" \\
                    \| ClassName \xcd"." \xcd"super"  \xcd"." Id TypeArguments\opt \xcd"(" ArgumentList\opt \xcd")" \\
                    \| Primary TypeArguments\opt \xcd"(" ArgumentList\opt \xcd")" \\
%(FROM #(prod:ArgumentList)#)
        ArgumentList \: Exp & (\ref{prod:ArgumentList}) \\
                    \| ArgumentList \xcd"," Exp \\
\end{bbgrammar}
%##)


A \grammarrule{MethodInvocation} may be to either a \xcd"static" method, an
instance method, or a closure.

The syntax is ambiguous; the target must be type-checked to determine if it is
the name of a method or if it refers to a field containing a closure. It is a
static error if a call may resolve to both a closure call or to a method call.
%~~gen ^^^ Expressions40
% package expres.sio.nsca.lls;
%~~vis
\begin{xten}
class Callsome {
  static val closure = () => 1;
  static def method () = 2;
  static val closureEvaluated = Callsome.closure();
  static val methodEvaluated = Callsome.method();
}
\end{xten}
%~~siv
%
%~~neg
However, adding a static method called \xcd`closure` makes \xcd`Callsome.closure()`
ambiguous: it could be a call to the closure, or to the static method: 

%~~gen ^^^ Expressions50
% package expres.sio.nsca.lls.twoooo;
% class Callsome {static val closure = () => 1; static def method () = 2; static val methodEvaluated = Callsome.method();
%~~vis
\begin{xten}
  static def closure () = 3;
  // ERROR: static errory = Callsome.closure();
\end{xten}
%~~siv
% }
%~~neg

A closure call \xcdmath"e($\dots$)" is shorthand for a method call
\xcdmath"e.apply($\dots$)". 

Method selection rules are given in \Sref{sect:MethodResolution}.

It is a static error if a method's \grammarrule{Guard} is not satisfied by the
caller.  For example: 
%~~gen ^^^ Expressions60
%package Expressions.Calls.Guarded.By.Walls;
%~~vis
\begin{xten}
class DivideBy(denom:Int) {
  def doIt(numer:Int){denom != 0} = numer / denom;
  def example() {
     //ERROR: denom might be zero: this.doIt(100); 
     (this as DivideBy{self.denom != 0}).doIt(100);
  }
}
\end{xten}
%~~siv
%
%~~neg


\section{Assignment}\index{assignment}\label{AssignmentStatement}

%##(Assignment LeftHandSide AssignmentOperator
\begin{bbgrammar}
%(FROM #(prod:Assignment)#)
          Assignment \: LeftHandSide AssignmentOperator AssignmentExp & (\ref{prod:Assignment}) \\
                    \| ExpName  \xcd"(" ArgumentList\opt \xcd")" AssignmentOperator AssignmentExp \\
                    \| Primary  \xcd"(" ArgumentList\opt \xcd")" AssignmentOperator AssignmentExp \\
%(FROM #(prod:LeftHandSide)#)
        LeftHandSide \: ExpName & (\ref{prod:LeftHandSide}) \\
                    \| FieldAccess \\
%(FROM #(prod:AssignmentOperator)#)
  AssignmentOperator \: \xcd"=" & (\ref{prod:AssignmentOperator}) \\
                    \| \xcd"*=" \\
                    \| \xcd"/=" \\
                    \| \xcd"%=" \\
                    \| \xcd"+=" \\
                    \| \xcd"-=" \\
                    \| \xcd"<<=" \\
                    \| \xcd">>=" \\
                    \| \xcd">>>=" \\
                    \| \xcd"&=" \\
                    \| \xcd"^=" \\
                    \| \xcd"|=" \\
\end{bbgrammar}
%##)



The assignment expression \xcd"x = e" assigns a value given by
expression \xcd"e"
to a variable \xcd"x".  
Most often, \xcd`x` is a mutable (\xcd`var` variable).  The same syntax is
used for delayed initialization of a \xcd`val`, but \xcd`val`s can only be
initialized once.
%~~gen ^^^ Expressions70
% package express.ions.ass.ignment;
% class Example {
% static def exasmple() {
%~~vis
\begin{xten}
  var x : Int;
  val y : Int;
  x = 1;
  y = 2; // Correct; initializes y
  x = 3; 
  // Incorrect: y = 4;
\end{xten}
%~~siv
% } } 
%~~neg


There are three syntactic forms of
assignment: 
\begin{enumerate}
\item \xcd`x = e;`, assigning to a local variable, formal parameter, field of
      \xcd`this`, etc. 
\item \xcd`x.f = e;`, assigning to a field of an object.
\item \xcdmath`a(i$_1$,$\ldots$,i$_n$) = v;`, where {$n \ge 0$}, assigning to
      an element of an array or some other such structure. This is syntactic
      sugar for a method call: \xcdmath`a.set(v,i$_1$,$\ldots$,i$_n$)`.
      Naturally, it is a static error if no suitable \xcd`set` method exists
      for \xcd`a`.
\end{enumerate}

For a binary operator $\diamond$, the $\diamond$-assignment expression
\xcdmath"x $\diamond$= e" combines the current value of \xcd`x` with the value
of \xcd`e` by {$\diamond$}, and stores the result back into \xcd`x`.  
\xcd`i += 2`, for example, adds 2 to \xcd`i`. For variables and fields, 
\xcdmath"x $\diamond$= e" behaves just like \xcdmath"x = x $\diamond$ e". 

The subscripting forms of \xcdmath"a(i) $\diamond$= b" are slightly subtle.
Subexpressions of \xcd`a` and \xcd`i` are only evaluated once.  However,
\xcd`a(i)` and \xcd`a(i)=c` are each executed once---in particular, there is
one call to \xcd`a.apply(i)` and one to \xcd`a.set(i,c)`, the desugared forms
of \xcd`a(i)` and \xcd`a(i)=c`.  If subscripting is implemented strangely for
the class of \xcd`a`, the behavior is {\em not} necessarily updating a single
storage location. Specifically, \xcd`A()(I()) += B()` is tantamount to: 
%~~gen ^^^ Expressions80
% package expressions.stupid.addab;
% class Example {
% def example(A:()=>Rail[Int], I: () => Int, B: () => Int ) {
%~~vis
\begin{xten}
{
  val aa = A();  // Evaluate A() once
  val ii = I();  // Evaluate I() once
  val bb = B();  // Evaluate B() once
  val tmp = aa(ii) + bb; // read aa(ii)
  aa(ii) = tmp;  // write sum back to aa(ii)
}
\end{xten}
%~~siv
%}}
%~~neg

\limitation{+= does not currently meet this specification.}




\section{Increment and decrement}
\index{increment}
\index{decrement}
\index{\Xcd{++}}
\index{\Xcd{--}}


The operators \xcd"++" and \xcd"--" increment and decrement
a variable, respectively.  
\xcd`x++` and \xcd`++x` both increment \xcd`x`, just as the statement 
\xcd`x += 1` would, and similarly for \xcd`--`.  

The difference between the two is the return value.  
\xcd`++x` returns the {\em new} value of \xcd`x`, after incrementing.
\xcd`x++` returns the {\em old} value of \xcd`, before incrementing.`

\limitation{This currently only works for numeric types.}

\section{Numeric Operations}
\label{XtenPromotions}
\index{promotion}
\index{numeric promotion}
\index{numeric operations}
\index{operation!numeric}

Numeric types (\xcd`Byte`, \xcd`Short`, \xcd`Int`, \xcd`Long`, \xcd`Float`,
\xcd`Double`, and unsigned variants of fixed-point types) are normal X10
structs, though most of their methods are implemented via native code. They
obey the same general rules as other X10 structs. For example, numeric
operations are defined by \xcd`operator` definitions, the same way you could
for any struct.

Promoting a numeric value to a longer numeric type preserves the sign of the
value.  For example, \xcd`(255 as UByte) as UInt` is 255.  

\subsection{Conversions and coercions}

Specifically, each numeric type can be converted or coerced into each other
numeric type, perhaps with loss of accuracy.
%~~gen ^^^ Expressions90
% package exp.ress.io.ns.numeric.conversions;
% class ExampleOfConversionAndStuff {
% def example() {
%~~vis
\begin{xten}
val n : Byte = 123 as Byte; // explicit 
val f : (Int)=>Boolean = (Int) => true; 
val ok = f(n); // implicit
\end{xten}
%~~siv
% } }
%~~neg



\subsection{Unary plus and unary minus}

The unary \xcd`+` operation on numbers is an identity function.
The unary \xcd`-` operation on numbers is a negation function.
On unsigned numbers, these are two's-complement.  For example, 
\xcd`-(0x0F as UByte)` is 
\xcd`(0xF1 as UByte)`.
\bard{UInts and such are closed under negation -- the negative of a UInt is
done binarily.  }



\section{Bitwise complement}

The unary \xcd"~" operator, only defined on integral types, complements each
bit in its operand.  

\section{Binary arithmetic operations} 

The binary arithmetic operators perform the familiar binary arithmetic
operations: \xcd`+` adds, \xcd`-` subtracts, \xcd`*` multiplies, 
\xcd`/` divides, and \xcd`%`
computes remainder.

On integers, the operands are coerced to the longer of their two types, and
then operated upon.  
Floating point operations are determined by the IEEE 754
standard. 
The integer \xcd"/" and \xcd"%" throw an exception 
if the right operand is zero.



\section{Binary shift operations}

The operands of the binary shift operations must be of integral type.
The type of the result is the type of the left operand.

If the promoted type of the left operand is \xcd"Int",
the right operand is masked with \xcd"0x1f" using the bitwise
AND (\xcd"&") operator, giving a number at most the number of bits in an
\xcd`Int`. 
If the promoted type of the left operand is \xcd"Long",
the right operand is masked with \xcd"0x3f" using the bitwise
AND (\xcd"&") operator, giving a number at most the number of bits in a
\xcd`Long`. 

The \xcd"<<" operator left-shifts the left operand by the number of
bits given by the right operand.
The \xcd">>" operator right-shifts the left operand by the number of
bits given by the right operand.  The result is sign extended;
that is, if the right operand is $k$,
the most significant $k$ bits of the result are set to the most
significant bit of the operand.

The \xcd">>>" operator right-shifts the left operand by the number of
bits given by the right operand.  The result is not sign extended;
that is, if the right operand is $k$,
the most significant $k$ bits of the result are set to \xcd"0".
This operation is deprecated, and may be removed in a later version of the
language. 

\section{Binary bitwise operations}

The binary bitwise operations operate on integral types, which are promoted to
the longer of the two types.
The \xcd"&" operator  performs the bitwise AND of the promoted operands.
The \xcd"|" operator  performs the bitwise inclusive OR of the promoted operands.
The \xcd"^" operator  performs the bitwise exclusive OR of the promoted operands.

\section{String concatenation}
\index{string!concatenation}

The \xcd"+"  operator is used for string concatenation 
 as well as addition.
If either operand is of static type \xcd"x10.lang.String",
 the other operand is converted to a \xcd"String" , if needed,
  and  the two strings  are concatenated.
 String conversion of a non-\xcd"null" value is  performed by invoking the
 \xcd"toString()" method of the value.
  If the value is \xcd"null", the value is converted to 
  \xcd'"null"'.

The type of the result is \xcd"String".

 For example, 
%~~exp~~`~~`~~ ~~ ^^^ Expressions100
      \xcd`"one " + 2 + here` 
      evaluates to  \xcd`one 2(Place 0)`.  

\section{Logical negation}

The operand of the  unary \xcd"!" operator 
must be of type \xcd"x10.lang.Boolean".
The type of the result is \xcd"Boolean".
If the value of the operand is \xcd"true", the result is \xcd"false"; if
if the value of the operand  is \xcd"false", the result is \xcd"true".

\section{Boolean logical operations}

Operands of the binary boolean logical operators must be of type \xcd"Boolean".
The type of the result is \xcd"Boolean"

The \xcd"&" operator  evaluates to \xcd"true" if both of its
operands evaluate to \xcd"true"; otherwise, the operator
evaluates to \xcd"false".

The \xcd"|" operator  evaluates to \xcd"false" if both of its
operands evaluate to \xcd"false"; otherwise, the operator
evaluates to \xcd"true".

\section{Boolean conditional operations}

Operands of the binary boolean conditional operators must be of type
\xcd"Boolean". 
The type of the result is \xcd"Boolean"

The \xcd"&&" operator  evaluates to \xcd"true" if both of its
operands evaluate to \xcd"true"; otherwise, the operator
evaluates to \xcd"false".
Unlike the logical operator \xcd"&",
if the first operand is \xcd"false",
the second operand is not evaluated.

The \xcd"||" operator  evaluates to \xcd"false" if both of its
operands evaluate to \xcd"false"; otherwise, the operator
evaluates to \xcd"true".
Unlike the logical operator \xcd"||",
if the first operand is \xcd"true",
the second operand is not evaluated.

\section{Relational operations} 

The relational operations compare numbers, producing \xcd`Boolean` results.  

The \xcd"<" operator evaluates to \xcd"true" if the left operand is
less than the right.
The \xcd"<=" operator evaluates to \xcd"true" if the left operand is
less than or equal to the right.
The \xcd">" operator evaluates to \xcd"true" if the left operand is
greater than the right.
The \xcd">=" operator evaluates to \xcd"true" if the left operand is
greater than or equal to the right.

Floating point comparison is determined by the IEEE 754
standard.  Thus,
if either operand is NaN, the result is \xcd"false".
Negative zero and positive zero are considered to be equal.
All finite values are less than positive infinity and greater
than negative infinity.

\section{Conditional expressions}
\index{\Xcd{? :}}
\index{conditional expression}
\index{expression!conditional}
\label{Conditional}

%##(ConditionalExp
\begin{bbgrammar}
%(FROM #(prod:ConditionalExp)#)
      ConditionalExp \: ConditionalOrExp & (\ref{prod:ConditionalExp}) \\
                    \| ClosureExp \\
                    \| AtExp \\
                    \| FinishExp \\
                    \| ConditionalOrExp \xcd"?" Exp \xcd":" ConditionalExp \\
\end{bbgrammar}
%##)

A conditional expression evaluates its first subexpression (the
condition); if \xcd"true"
the second subexpression (the consequent) is evaluated; otherwise,
the third subexpression (the alternative) is evaluated.

The type of the condition must be \xcd"Boolean".
The type of the conditional expression is some common 
ancestor (as constrained by \Sref{LCA}) of the types of the consequent and the
alternative. 

For example, 
%~~exp~~`~~`~~a:Int,b:Int ~~ ^^^ Expressions110
\xcd`a == b ? 1 : 2`
evaluates to \xcd`1` if \xcd`a` and \xcd`b` are the same, and \xcd`2` if they
are different.   As the type of \xcd`1` is \xcd`Int{self==1}` and of \xcd`2`
is \xcd`Int{self==2}`, the type of the conditional expression has the form
\xcd`Int{c}`, where \xcd`self==1` and \xcd`self==2` both imply \xcd`c`.  For
example, it might be \xcd`Int{true}` -- or perhaps it might be 
\xcd`Int{self != 8}`. Note that this term has no most accurate type in the X10
type system.

\section{Stable equality}
\label{StableEquality}
\index{\Xcd{==}}
\index{equality}

\begin{bbgrammar}
 EqualityExp    \: RelationalExp & (\ref{prod:EqualityExp})\\%<FROM #(prod:EqualityExp)#
    \| EqualityExp \xcd"==" RelationalExp\\
    \| EqualityExp \xcd"!=" RelationalExp\\
    \| Type  \xcd"==" Type \\
\end{bbgrammar}


The \xcd"==" and \xcd"!=" operators provide a fundamental, though
non-abstract, notion of equality.  \xcd`a==b` is true if the values of \xcd`a`
and \xcd`b` are extremely identical.

\begin{itemize}
\item If \xcd`a` and \xcd`b` are values of object type, then \xcd`a==b` holds
      if \xcd`a` and \xcd`b` are the same object.
\item If one operand is \xcd`null`, then \xcd`a==b` holds iff the other is
      also \xcd`null`.
\item If the operands both have struct type, then they must be structurally equal;
that is, they must be instances of the same struct
and all their fields or components must be \xcd"==". 
\item The definition of equality for function types is specified in
      \Sref{FunctionEquality}.
\item If the operands have numeric types, they are coerced into the larger of
      the two types (see \Sref{WideningConversions}) and then compared for numeric equality.
\end{itemize}

\xcd`a != b`
is true iff \xcd`a==b` is false.

The predicates \xcd"==" and \xcd"!=" may not be overridden by the programmer.
Note that \xcd`a==b` is a form of \emph{stable equality}; that is, the result of
the equality operation is not affected by the mutable state of the program,
after evaluation of \xcd`a` and \xcd`b`. 


\section{Allocation}
\label{ClassCreation}
\index{new}
\index{allocation}
\index{class!instantation}
\index{class!construction}
\index{struct!instantation}
\index{struct!construction}
\index{instantation}

%##(ClassInstCreationExp
\begin{bbgrammar}
%(FROM #(prod:ClassInstCreationExp)#)
ClassInstCreationExp \: \xcd"new" TypeName TypeArguments\opt \xcd"(" ArgumentList\opt \xcd")" ClassBody\opt & (\ref{prod:ClassInstCreationExp}) \\
                    \| \xcd"new" TypeName \xcd"[" Type \xcd"]" \xcd"[" ArgumentList\opt \xcd"]" \\
                    \| Primary \xcd"." \xcd"new" Id TypeArguments\opt \xcd"(" ArgumentList\opt \xcd")" ClassBody\opt \\
                    \| AmbiguousName \xcd"." \xcd"new" Id TypeArguments\opt \xcd"(" ArgumentList\opt \xcd")" ClassBody\opt \\
\end{bbgrammar}
%##)

An allocation expression creates a new instance of a class and
invokes a constructor of the class.
The expression designates the class name and passes
type and value arguments to the constructor.

The allocation expression may have an optional class body.
In this case, an anonymous subclass of the given class is
allocated.   An anonymous class allocation may also specify a
single super-interface rather than a superclass; the superclass
of the anonymous class is \xcd"x10.lang.Object".

If the class is anonymous---that is, if a class body is
provided---then the constructor is selected from the superclass.
The constructor to invoke is selected using the same rules as
for method invocation (\Sref{MethodInvocation}).

The type of an allocation expression
is the return type of the constructor invoked, with appropriate
substitutions  of actual arguments for formal parameters, as
specified in \Sref{MethodInvocationSubstitution}.

It is illegal to allocate an instance of an \xcd"abstract" class.
It is illegal to allocate an instance of a class or to invoke a
constructor that is not visible at
the allocation expression.

Note that instantiating a struct type uses function application syntax, not
\xcd`new`.  As structs do not have subclassing, there is no need or
possibility of a {\em ClassBody}.


\section{Casts}\label{ClassCast}\index{cast}
\index{type conversion}

The cast operation may be used to cast an expression to a given type:

%##(CastExp
\begin{bbgrammar}
%(FROM #(prod:CastExp)#)
             CastExp \: Primary & (\ref{prod:CastExp}) \\
                    \| ExpName \\
                    \| CastExp \xcd"as" Type \\
\end{bbgrammar}
%##)

The result of this operation is a value of the given type if the cast
is permissible at run time, and either a compile-time error or a runtime
exception 
(\xcd`x10.lang.TypeCastException`) if it is not.  

When evaluating \xcd`E as T{c}`, first the value of \xcd`E` is converted to
type \xcd`T` (which may fail), and then the constraint \xcd`{c}` is checked. 



\begin{itemize}
\item If \xcd`T` is a primitive type, then \xcd`E`'s value is converted to type
      \xcd`T` according to the rules of
      \Sref{sec:effects-of-explicit-numeric-coercions}. 
      
\item If \xcd`T` is a class, then the first half of the cast succeeds if the
      run-time value of \xcd`E` is an instance of class \xcd`T`, or of a
      subclass. 

\item If \xcd`T` is an interface, then the first half of the cast succeeds if
      the run-time value of \xcd`E` is an instance of a class implementing
      \xcd`T`. 

\item If \xcd`T` is a struct type, then the first half of the cast succeeds if
      the run-time value of \xcd`E` is an instance of \xcd`T`.  

\item If \xcd`T` is a function type, then the first half of the cast succeeds
      if the run-time value of \xcd`X` is a function of that type, or a
      subtype of it.
\end{itemize}

If the first half of the cast succeeds, the second half -- the constraint
\xcd`{c}` -- must be checked.  In general this will be done at runtime, though
in special cases it can be checked at compile time.   For example, 
\xcd`n as Int{self != w}` succeeds if \xcd`n != w` --- even if \xcd`w` is a value
read from input, and thus not determined at compile time.

The compiler may forbid casts that it knows cannot possibly work. If there is
no way for the value of \xcd`E` to be of type \xcd`T{c}`, then 
\xcd`E as T{c}` can result in a static error, rather than a runtime error.  
For example, \xcd`1 as Int{self==2}` may fail to compile, because the compiler
knows that \xcd`1`, which has type \xcd`Int{self==1}`, cannot possibly be of
type \xcd`Int{self==2}`. 


%BB% \bard{This section need serious whomping.  The Java mention needs to go.  The
%BB% rules for coercions are given in \Sref{sec:effects-of-explicit-numeric-coercions}.
%BB% If the \xcd`Type` has a constraint, the constraint will be checked at runtime. 
%BB% We need to give examples. 
%BB% }
%BB% 
%BB% Type conversion is checked according to the
%BB% rules of the \java{} language (e.g., \cite[\S 5.5]{jls2}).
%BB% For constrained types, both the base
%BB% type and the constraint are checked.
%BB% If the
%BB% value cannot be cast to the appropriate type, a
%BB% \xcd"ClassCastException"
%BB% is thrown. 



% {\bf Conversions of numeric values}
% {\bf Can't do (a as T) if a can't be a T.}


%If the value cannot be cast to the
%appropriate place type a \xcd"BadPlaceException" is thrown. 

% Any attempt to cast an expression of a reference type to a value type
% (or vice versa) results in a compile-time error. Some casts---such as
% those that seek to cast a value of a subtype to a supertype---are
% known to succeed at compile-time. Such casts should not cause extra
% computational overhead at run time.

\section{\Xcd{instanceof}}
\label{instanceOf}
\index{\Xcd{instanceof}}
\index{instanceof}

\Xten{} permits types to be used in an in instanceof expression
to determine whether an object is an instance of the given type:

%##(RelationalExp
\begin{bbgrammar}
%(FROM #(prod:RelationalExp)#)
       RelationalExp \: RangeExp & (\ref{prod:RelationalExp}) \\
                    \| SubtypeConstraint \\
                    \| RelationalExp \xcd"<" RangeExp \\
                    \| RelationalExp \xcd">" RangeExp \\
                    \| RelationalExp \xcd"<=" RangeExp \\
                    \| RelationalExp \xcd">=" RangeExp \\
                    \| RelationalExp \xcd"instanceof" Type \\
                    \| RelationalExp \xcd"in" ShiftExp \\
\end{bbgrammar}
%##)

In the above expression, \grammarrule{Type} is any type. At run time, the
result of this operator is \xcd"true" if the
\grammarrule{RelationalExpression} can be coerced to \grammarrule{Type}
without a \xcd"TypeCastException" being thrown or static error occurring.
Otherwise the result is \xcd"false". This determination may involve checking
that the constraint, if any, associated with the type is true for the given
expression.

%~~exp~~`~~`~~x:Int~~ ^^^ Expressions120
For example, \xcd`3 instanceof Int{self==x}` is an overly-complicated way of
saying \xcd`3==x`.


However, it is a static error if \xcd`e` cannot possibly be an instance of
\xcd`C{c}`; the compiler will reject \xcd`1 instanceof Int{self == 2}` because
\xcd`1` can never satisfy \xcd`Int{self == 2}`. Similarly, \Xcd{1 instanceof
String} is a static error, rather than an expression always returning false. 

\limitationx
X10 does not currently handle \xcd`instanceof` of generics in the way you
%~NO~exp~~`~~`~~r:Array[Int](1) ~~
might expect.  For example, \xcd`r instanceof Array[Int{self != 0}]` does
not test that every element of \xcd`r` is non-zero; instead, the compiler
rejects it.


\section{Subtyping expressions}
\index{\Xcd{<:}}
\index{\Xcd{:>}}
\index{subtype!test}


%##(SubtypeConstraint
\begin{bbgrammar}
%(FROM #(prod:SubtypeConstraint)#)
   SubtypeConstraint \: Type  \xcd"<:" Type  & (\ref{prod:SubtypeConstraint}) \\
                    \| Type  \xcd":>" Type  \\
\end{bbgrammar}
%##)

The subtyping expression \xcdmath"T$_1$ <: T$_2$" evaluates to \xcd"true"
\xcdmath"T$_1$" is a subtype of \xcdmath"T$_2$".

The expression \xcdmath"T$_1$ :> T$_2$" evaluates to \xcd"true"
\xcdmath"T$_2$" is a subtype of \xcdmath"T$_1$".

The expression \xcdmath"T$_1$ == T$_2$"
evaluates to  \xcd"true" \xcdmath"T$_1$" is a subtype of \xcdmath"T$_2$" and
if \xcdmath"T$_2$" is a subtype of \xcdmath"T$_1$".

Subtyping expressions are particularly useful in giving constraints on generic
types.  \xcd`x10.util.Ordered[T]` is an interface whose values can be compared
with values of type \xcd`T`. 
In particular, \xcd`T <: x10.util.Ordered[T]` is
true if values of type \xcd`T` can be compared to other values of type
\xcd`T`.  So, if we wish to define a generic class \xcd`OrderedList[T]`, of
lists whose elements are kept in the right order, we need the elements to be
ordered.  This is phrased as a constraint on \xcd`T`: 
%~~gen ^^^ Expressions130
% package expre.ssi.onsfgua.rde.dq.uantification;
%~~vis
\begin{xten}
class OrderedList[T]{T <: x10.util.Ordered[T]} {
  // ...
}
\end{xten}
%~~siv
%
%~~neg




\section{Contains expressions}
\index{in}

%##(RelationalExp
\begin{bbgrammar}
%(FROM #(prod:RelationalExp)#)
       RelationalExp \: RangeExp & (\ref{prod:RelationalExp}) \\
                    \| SubtypeConstraint \\
                    \| RelationalExp \xcd"<" RangeExp \\
                    \| RelationalExp \xcd">" RangeExp \\
                    \| RelationalExp \xcd"<=" RangeExp \\
                    \| RelationalExp \xcd">=" RangeExp \\
                    \| RelationalExp \xcd"instanceof" Type \\
                    \| RelationalExp \xcd"in" ShiftExp \\
\end{bbgrammar}
%##)

The expression \xcd"p in r" tests if a value \xcd"p" is in a collection
\xcd"r"; it evaluates to \xcd"r.contains(p)".
The collection \xcd"r"
must be of type \xcd"Collection[T]" and the value \xcd"p" must
be of type \xcd"T".

\section{Array Constructors}
\label{sect:ArrayCtors}
\index{array!construction}
\index{array!literal}

%##(Primary ClassInstCreationExp
\begin{bbgrammar}
%(FROM #(prod:Primary)#)
             Primary \: \xcd"here" & (\ref{prod:Primary}) \\
                    \| \xcd"[" ArgumentList\opt \xcd"]" \\
                    \| Literal \\
                    \| \xcd"self" \\
                    \| \xcd"this" \\
                    \| ClassName \xcd"." \xcd"this" \\
                    \| \xcd"(" Exp \xcd")" \\
                    \| ClassInstCreationExp \\
                    \| FieldAccess \\
                    \| MethodInvocation \\
                    \| MethodSelection \\
                    \| OperatorFunction \\
%(FROM #(prod:ClassInstCreationExp)#)
ClassInstCreationExp \: \xcd"new" TypeName TypeArguments\opt \xcd"(" ArgumentList\opt \xcd")" ClassBody\opt & (\ref{prod:ClassInstCreationExp}) \\
                    \| \xcd"new" TypeName \xcd"[" Type \xcd"]" \xcd"[" ArgumentList\opt \xcd"]" \\
                    \| Primary \xcd"." \xcd"new" Id TypeArguments\opt \xcd"(" ArgumentList\opt \xcd")" ClassBody\opt \\
                    \| AmbiguousName \xcd"." \xcd"new" Id TypeArguments\opt \xcd"(" ArgumentList\opt \xcd")" ClassBody\opt \\
\end{bbgrammar}
%##)

X10 includes short syntactic forms for constructing one-dimensional arrays.
The shortest form is to enclose some expressions in brackets: 
%~~gen ^^^ Expressions140
% package Expressions.ArrayCtor.Primo;
% class Example {
% def example() {
%~~vis
\begin{xten}
val ints <: Array[Int](1) = [1,3,7,21];
\end{xten}
%~~siv
%}}
%~~neg

The expression \Xcd{[e1,e2,e3, ..., en]} produces an \Xcd{n}-element
\xcd`Array[T](1)`, where \xcd`T` is the least common supertype of the {\bf
  base types} of the expressions \xcd`ei`. For example, the type of
\xcd`[0,1,2]` is \Xcd{Array[Int](1)}.    

More importantly, the type of 
\xcd`[0]` is also \xcd`Array[Int](1)`.  It is {\em not} 
\xcd`Array[Int{self==0}](1)`, even though all the elements are all 
of type \xcd`Int{self==0}`.  This is subtle but important. There are many
functions that take \xcd`Array[Int](1)`s, such as conversions to \xcd`Point`.
These functions do {\em not} take
\xcd`Array[Int{self==0}]`'s.

(Suppose that the function took \xcd`a:Array[Int](1)` and did 
the operation \xcd`a(i)=100`.   This operation is perfectly fine for
an \xcd`Array[Int](1)`, which is all the compiler knows about \xcd`a`.  
However, it is invalid for an \xcd`Array[Int{self==0}](1)`, because it assigns
a non-zero value to an element of the array, violating the type constraint
which says that all the elements are zero.  So, \xcd`Array[Int{self==0}](1)`
is not a subtype of \xcd`Array[Int](1)` --- the two types are simply unrelated.)
%~~type~~`~~`~~ ~~ ^^^ Expressions150
Since there are far more uses for \xcd`Array[Int](1)` than
%~~type~~`~~`~~ ~~ ^^^ Expressions160
\xcd`Array[Int{self==0}](1)`, the compiler produces the former.

Still, occasionally one does actually need \xcd`Array[Int{self==0}](1)`, 
or, say, \xcd`Array[Eel{self != null}](1)`, an array of non-null \xcd`Eel`s.  
For these cases, X10 provides an array constructor which does allow
specification of the element type: \xcd`new Array[T][e1...en]`.  Each
element \xcd`ei` must be of type \xcd`T`.  The resulting array is of type
\xcd`Array[T](1)`.  
%~~gen ^^^ Expressions170
%package Expressions.ArrayCtor.Details;
%class Eel{}
%class Example{
%def example(){
%~~vis
\begin{xten}
val zero <: Array[Int{self == 0}](1) = new Array[Int{self == 0}][0];
val non1 <: Array[Int{self != 1}](1) = new Array[Int{self != 1}][0];
val eels <: Array[Eel{self != null}](1) = 
    new Array[Eel{self != null}][ new Eel() ];
\end{xten}
%~~siv
%}}
%~~neg



%%OLD-RAIL%% 
%%OLD-RAIL%% 
%%OLD-RAIL%% \noo{This is now an Array ctor and the text needs revision}
%%OLD-RAIL%% \label{RailConstructors}
%%OLD-RAIL%% 
%%OLD-RAIL%% \begin{grammar}
%%OLD-RAIL%% RailConstructor \: \xcd"[" Expressions \xcd"]" \\
%%OLD-RAIL%% Expressions \: Expression ( \xcd"," Expression )\star \\
%%OLD-RAIL%% \end{grammar}
%%OLD-RAIL%% 
%%OLD-RAIL%% The rail constructor \xcdmath"[a$_0$, $\dots$, a$_{k-1}$]"
%%OLD-RAIL%% creates an instance of \xcd"ValRail" with length $k$, 
%%OLD-RAIL%% whose $i$th element is
%%OLD-RAIL%% \xcdmath"a$_i$".  The element type of the rail is a common ancestor of the
%%OLD-RAIL%% types of the \xcdmath"a$_i$"'s, as per \Sref{LCA}.
%%OLD-RAIL%% %~s~gen
%%OLD-RAIL%% % package ex.pre.ssio.nsandrailconstructors;
%%OLD-RAIL%% % class Example {
%%OLD-RAIL%% % def example() {
%%OLD-RAIL%% %~s~vis
%%OLD-RAIL%% \begin{xten}
%%OLD-RAIL%% val a <: Array[Int] = [1,3,5];
%%OLD-RAIL%% val b <: Array[Any] = [1, a, "please"];
%%OLD-RAIL%% \end{xten}
%%OLD-RAIL%% %~s~siv
%%OLD-RAIL%% % } } 
%%OLD-RAIL%% %~s~neg
%%OLD-RAIL%% 
%%OLD-RAIL%% Since arrays are subtypes of \xcd"(Point) => T",
%%OLD-RAIL%% rail constructors can be passed into the \xcd"Array" and
%%OLD-RAIL%% \xcd"ValArray" constructors as initializer functions.
%%OLD-RAIL%% 
%%OLD-RAIL%% Rail constructors of type \xcd"ValRail[Int]" and length \xcd"n" 
%%OLD-RAIL%% may be implicitly converted to type \xcd"Point{rank==n}".
%%OLD-RAIL%% Rail constructors of type \xcd"ValRail[Region]" and length \xcd"n" 
%%OLD-RAIL%% may be implicitly converted to type \xcd"Region{rank==n}".
%%OLD-RAIL%% 
%%OLD-RAIL%% %~s~gen
%%OLD-RAIL%% % package ex.pre.ssio.nsandrailconstructors;
%%OLD-RAIL%% % class Exympyl {
%%OLD-RAIL%% % def example() {
%%OLD-RAIL%% %~s~vis
%%OLD-RAIL%% \begin{xten}
%%OLD-RAIL%% val a : Point{rank==4} = [1,2,3,4];
%%OLD-RAIL%% val b : Region{rank==2} = (-1 .. 1) * (-2 .. 2);
%%OLD-RAIL%% \end{xten}
%%OLD-RAIL%% %~s~siv
%%OLD-RAIL%% % } } 
%%OLD-RAIL%% %~s~neg
%%OLD-RAIL%% 

\section{Coercions and conversions}
\label{XtenConversions}
\label{User-definedCoercions}
\index{conversion}\index{coercion}
\index{type!conversion}\index{type!coercion}

\XtenCurrVer{} supports the following coercions and conversions.

\subsection{Coercions}

%##(CastExp
\begin{bbgrammar}
%(FROM #(prod:CastExp)#)
             CastExp \: Primary & (\ref{prod:CastExp}) \\
                    \| ExpName \\
                    \| CastExp \xcd"as" Type \\
\end{bbgrammar}
%##)


A {\em coercion} does not change object identity; a coerced object may
be explicitly coerced back to its original type through a cast. A {\em
  conversion} may change object identity if the type being converted
to is not the same as the type converted from. \Xten{} permits
user-defined conversions (\Sref{sec:user-defined-conversions}).

\paragraph{Subsumption coercion.}
A subtype may be implicitly coerced to any supertype.
\index{coercion!subsumption}

\paragraph{Explicit coercion (casting with \xcd"as")}



An object of any class may be explicitly coerced to any other
class type using the \xcd"as" operation.  If \xcd`Child <: Person` and
\xcd`rhys:Child`, then 
%~~gen ^^^ Expressions180
% package Types.Coercions;
%  class Person {}
%  class Child extends Person{} 
%  class Exampllllle { 
%    def example(rhys:Child) =
%~~vis
\begin{xten}
  rhys as Person
\end{xten}
%~~siv
%;}
%~~neg
is an expression of type \xcd`Person`.  

If the value coerced is not an instance of the target type,
a \xcd"ClassCastException" is thrown.  Casting to a constrained
type may require a run-time check that the constraint is
satisfied.
\index{coercion!explicit}
\index{cast}
\index{\Xcd{as}}

\limitation{It is currently a static error, rather than the specified
\xcd`ClassCastException`, when the cast is statically determinable to be
impossible.}

\paragraph{Effects of explicit numeric coercion}
\label{sec:effects-of-explicit-numeric-coercions}

Coercing a number of one type to another type gives the best approximation of
the number in the result type, or a suitable disaster value if no
approximation is good enough.  

\begin{itemize}
\item Casting a number to a {\em wider} numeric type is safe and effective,
      and can be done by an implicit conversion as well as an explicit
%~~exp~~`~~`~~ ~~ ^^^ Expressions190
      coercion.  For example, \xcd`4 as Long` produces the \xcd`Long` value of
      4. 
\item Casting a floating-point value to an integer value truncates the digits
      after the decimal point, thereby rounding the number towards zero.  
%~~exp~~`~~`~~ ^^^ Expressions200
      \xcd`54.321 as Int` is \xcd`54`, and 
%~~exp~~`~~`~~ ~~ ^^^ Expressions210
      \xcd`-54.321 as Int` is \xcd`-54`.
      If the floating-point value is too large to represent as that kind of
      integer, the coercion returns the largest or smallest value of that type
      instead: \xcd`1e110 as Int` is 
      \xcd`Int.MAX_VALUE`, \xcd`2147483647`. 

\item Casting a \xcd`Double` to a \xcd`Float` normally truncates digits: 
%~~exp~~`~~`~~ ~~ ^^^ Expressions220
      \xcd`0.12345678901234567890 as Float` is \xcd`0.12345679f`.  This can
      turn a nonzero \xcd`Double` into \xcd`0.0f`, the zero of type
      \xcd`Float`: 
%~~exp~~`~~`~~ ~~ ^^^ Expressions230
      \xcd`1e-100 as Float` is \xcd`0.0f`.  Since 
      \xcd`Double`s can be as large as about \xcd`1.79E308` and \xcd`Float`s
      can only be as large as about \xcd`3.4E38f`, a large \xcd`Double` will
      be converted to the special \xcd`Float` value of \xcd`Infinity`: 
%~~exp~~`~~`~~ ~~ ^^^ Expressions240
      \xcd`1e100 as Float` is \xcd`Infinity`.
\item Integers are coerced to smaller integer types by truncating the
      high-order bits. If the value of the large integer fits into the smaller
      integer's range, this gives the same number in the smaller type: 
%~~exp~~`~~`~~ ~~ ^^^ Expressions250
      \xcd`12 as Byte` is the \xcd`Byte`-sized 12, 
%~~exp~~`~~`~~ ~~ ^^^ Expressions260
      \xcd`-12 as Byte` is -12. 
      However, if the larger integer {\em doesn't} fit in the smaller type,
%~~exp~~`~~`~~ ~~ ^^^ Expressions270
      the numeric value and even the sign can change: \xcd`254 as Byte` is
      \xcd`Byte`-sized \xcd`-2`.  


\end{itemize}

\subsection{Conversions}
\index{conversion}
\index{type!conversion}

\paragraph{Widening numeric conversion.}
\label{WideningConversions}
A numeric type may be implicitly converted to a wider numeric type. In
particular, an implicit conversion may be performed between a numeric
type and a type to its right, below:

\begin{xten}
Byte < Short < Int < Long < Float < Double
\end{xten}

\index{conversion!widening}
\index{conversion!numeric}

\paragraph{String conversion.}
Any value that is an operand of the binary
\xcd"+" operator may
be converted to \xcd"String" if the other operand is a \xcd"String".
A conversion to \xcd"String" is performed by invoking the \xcd"toString()"
method.

\index{conversion!string}

\paragraph{User defined conversions.}\label{sec:user-defined-conversions}
\index{conversion!user-defined}

The user may define conversion operators from type \Xcd{A} {\em to} a
container type \Xcd{B} by specifying a method on \Xcd{B} as follows:

\begin{xten}
  public static operator (r: A): T = ... 
\end{xten}

The return type \Xcd{T} should be a subtype of \Xcd{B}. The return
type need not be specified explicitly; it will be computed in the
usual fashion if it is not. However, it is good practice for the
programmer to specify the return type for such operators explicitly.

For instance, the code for \Xcd{x10.lang.Point} contains:

\begin{xten}
  public operator (r: Array[Int](1)): Point(r.length) = make(r);
\end{xten}

The compiler looks for such operators on the container type \Xcd{B}
when it encounters an expression of the form \Xcd{r as B} (where
\Xcd{r} is of type \Xcd{A}). If it finds such a method, it sets the
type of the expression \Xcd{r as B} to be the return type of the
method. Thus the type of \Xcd{r as B} is guaranteed to be some subtype
of \Xcd{B}.

\begin{example}
Consider the following code:  



%~~stmt~~\begin{xten}~~\end{xten}~~ ~~ ^^^ Expressions280
\begin{xten}
val p  = [2, 2, 2, 2, 2] as Point;
val q = [1, 1, 1, 1, 1] as Point;
val a = p - q;    
\end{xten}
This code fragment compiles successfully, given the above operator definition. 
The type of \Xcd{p} is inferred to be \Xcd{Point(5)} (i.e.{} the type 
%~~type~~`~~`~~ ~~ ^^^ Expressions290
\xcd`Point{self.rank==5}`.
Similarly for \Xcd{q}. Hence the application of the operator ``\Xcd{-}'' is legal (it requires both arguments to have the same rank). The type of \Xcd{a} is computed as \Xcd{Point(5)}.
\end{example}
	
\chapter{Statements}\label{XtenStatements}\index{statements}

This chapter describes the statements in the sequential core of
\Xten{}.  Statements involving concurrency and distribution
are described in \Sref{XtenActivities}.

\section{Empty statement}

\begin{grammar}
Statement \: \xcd";" \\
\end{grammar}

The empty statement \xcd";" does nothing.  It is useful when a
loop header is evaluated for its side effects.  For example,
the following code sums the elements of an array.
\begin{xten}
var sum: Int = 0;
for (i: Int = 0; i < a.length; i++, sum += a[i])
    ;
\end{xten}

\section{Local variable declaration}

\begin{grammar}
Statement \: LocalVariableDeclarationStatement \\
             LocalVariableDeclarationStatement \:
             LocalVariableDeclaration \xcd";" \\
\end{grammar}

The syntax of local variables declarations is described in
\Sref{VariableDeclarations}.

Local variables may be declared only within a block statement
(\Sref{Blocks}).
The scope of a local variable declaration is the 
statement itself and the subsequent statements in the block.

\section{Block statement}
\label{Blocks}

\begin{grammar}
Statement \: BlockStatement \\
BlockStatement \: \xcd"{" Statement\star \xcd"}" \\
\end{grammar}

A block statement consists of a sequence of statements delimited
by ``\xcd"{"'' and ``\xcd"}"''.  Statements are evaluated in
order.  The scope of local variables introduced within the block  
is the remainder of the block following the variable declaration.

\section{Expression statement}

\begin{grammar}
Statement \: ExpressionStatement \\
ExpressionStatement \: StatementExpression \xcd";" \\
StatementExpression \: Assignment \\
          \| Allocation \\
          \| Call \\
\end{grammar}

The expression statement evaluates an expression, ignoring the
result.  The expression must be either an assignment, an
allocation, or a call.

\section{Labeled statement}

\begin{grammar}
Statement \: LabeledStatement \\
LabeledStatement \: Identifier \xcd":" Statement \\
\end{grammar}

Statements may be labeled.  The label may be used as the target
of a break or continue statement.  The scope of a label is the
statement labeled.

\section{Break statement}

\begin{grammar}
Statement \: BreakStatement \\
BreakStatement \: \xcd"break" Identifier\opt \\
\end{grammar}

An unlabeled break statement exits the currently enclosing loop
or switch statement.

An labeled break statement exits the enclosing loop
or switch statement with the given label.

It is illegal to break out of a loop not defined in the current
method, constructor, initializer, or closure.

The following code searches for an element of a two-dimensional
array and breaks out of the loop when found:

\begin{xten}
var found: Boolean = false;
for (i: Int = 0; i < a.length; i++)
    for (j: Int = 0; j < a(i).length; j++)
        if (a(i)(j) == v) {
            found = true;
            break;
        }
\end{xten}

\section{Continue statement}

\begin{grammar}
Statement \: ContinueStatement \\
ContinueStatement \: \xcd"continue" Identifier\opt \\
\end{grammar}

An unlabeled continue statement branches to the top of the
currently enclosing loop.

An labeled break statement branches to the top of the enclosing loop
with the given label.

It is illegal to continue a loop not defined in the current
method, constructor, initializer, or closure.

\section{If statement}

\begin{grammar}
Statement \: IfThenStatement \\
          \| IfThenElseStatement \\
IfThenStatement \: \xcd"if" \xcd"(" Expression \xcd")" Statement \\
IfThenElseStatement \: \xcd"if" \xcd"(" Expression \xcd")" Statement \xcd"else" Statement \\
\end{grammar}

An if statement comes in two forms: with and without an else
clause.

The if-then statement evaluates a condition expression and 
evaluates the consequent expression if the condition is
\xcd"true".  If the 
condition is \xcd"false",
the if-then statement completes normally.

The if-then-else statement evaluates a condition expression and 
evaluates the consequent expression if the condition is
\xcd"true"; otherwise, the alternative statement is evaluated.

The condition must be of type \xcd"Boolean".

\section{Switch statement}

\begin{grammar}
Statement \: SwitchStatement \\
SwitchStatement \: \xcd"switch" \xcd"(" Expression \xcd")" \xcd"{" Case\plus \xcd"}" \\
Case \: \xcd"case" Expression \xcd":" Statement\star \\
     \| \xcd"default" \xcd":" Statement\star \\
\end{grammar}

A switch statement evaluates an index expression and then branches to
a case whose value equal to the value of the index expression.
If no such case exists, the switch branches to the 
\xcd"default" case, if any.

Statements in each case branch evaluated in sequence.  At the
end of the branch, normal control-flow falls through to the next case, if
any.  To prevent fall-through, a case branch may be exited using
a \xcd"break" statement.

The index expression must be of type \xcd"Int".

Case labels must be of type \xcd"Int" and must be compile-time
constants.  Case labels cannot be duplicated within the
\xcd"switch" statement.

\section{While statement}

\index{while@\xcd"while"}

\begin{grammar}
Statement \: WhileStatement \\
WhileStatement \: \xcd"while" \xcd"(" Expression \xcd")" Statement \\
\end{grammar}

A while statement evaluates a condition and executes a loop body
if \xcd"true".  If the loop body completes normally (either by reaching
the end or via a \xcd"continue" statement with the loop header
as target), the condition is reevaluated and the loop repeats if
\xcd"true".  If the condition is \xcd"false", the loop
exits.

The condition must be of type \xcd"Boolean".

\section{Do--while statement}

\index{do@\xcd"do"}

\begin{grammar}
Statement \: DoWhileStatement \\
DoWhileStatement \: \xcd"do" Statement \xcd"while" \xcd"(" Expression \xcd")" \xcd";" \\
\end{grammar}


A do-while statement executes a loop body, and then evaluates a
condition expression.  If \xcd"true", the loop repeats.
Otherwise, the loop exits.

The condition must be of type \xcd"Boolean".

\section{For statement}

\index{for@\xcd"for"}

\begin{grammar}
Statement \: ForStatement \\
          \| EnhancedForStatement \\
ForStatement \: \xcd"for" \xcd"("
        ForInit\opt \xcd";" Expression\opt \xcd";" ForUpdate\opt
        \xcd")" Statement \\
ForInit \:
        StatementExpression ( \xcd"," StatementExpression )\star
        \\
      \| LocalVariableDeclaration \\
EnhancedForStatement \: \xcd"for" \xcd"("
        Formal \xcd"in" Expression 
        \xcd")" Statement \\
\end{grammar}

\Xten{} provides two forms of for statement: a basic for
statement and an enhanced for statement.

A basic for statement consists of an initializer, a condition, an
iterator, and a body.  First, the initializer is evaluated.
The initializer may introduce local variables that are in scope
throughout the for statement.  An empty initializer is
permitted.
Next, the condition is evaluated.  If \xcd"true", the loop body
is executed; otherwise, the loop exits.
The condition may be omitted, in which case the condition is
considered \xcd"true".
If the loop completes normally (either by reaching the end
or via a \xcd"continue" statement with the loop header as
target),
the iterator is evaluated and then the condition is reevaluated
and the loop repeats if
\xcd"true".  If the condition is \xcd"false", the loop
exits.

The condition must be of type \xcd"Boolean".
The initializer and iterator are statements, not expressions
and so do not have types.

\label{ForAllLoop}

% XXX REGION

An enhanced for statement is used to iterate over a collection.
If the formal parameter is of type \xcd"T",
the collection expression must be of type \xcd"Iterable[T]".
Exploded
syntax may
be used for the formal parameter (\Sref{exploded-syntax}).
Each iteration of the loop
binds the parameter to another element of the collection.
If the parameter is final, it may not be assigned within the
loop body.

In a common case, the
the collection is intended to be of type
\xcd"Region" and the formal parameter is of type \xcd"Point".
Expressions \xcd"e" of type \xcd"Dist" and
\xcd"Array" are also accepted, and treated as if they were
\xcd"e.region".
If the collection is a region, the \xcd"for" statement
enumerates the points in the region in canonical order.



\section{Throw statement}
\index{throw}

\begin{grammar}
Statement \: ThrowStatement \\
ThrowStatement \: \xcd"throw" Expression \xcd";"
\end{grammar}

The \xcd"throw" statement throws an exception.  The exception
must be a subclass of the value class \xcd"x10.lang.Throwable". 
% null not allowed since a value class;
% If the exception is
% \xcd"null", a \xcd"NullPointerException" is thrown.

\begin{example}
The following statement checks if an index is in range and
throws an exception if not.
\begin{xten}
if (i < 0 || i > x.length)
    throw new IndexOutOfBoundsException();
\end{xten}
\end{example}

\section{Try--catch statement}

\begin{grammar}
Statement \: TryStatement \\
TryStatement \: \xcd"try" BlockStatement Catch\plus Finally\opt \\
             \| \xcd"try" BlockStatement Catch\star Finally \\
Catch \: \xcd"catch" \xcd"(" Formal \xcd")" BlockStatement \\
Finally \: \xcd"finally" BlockStatement \\
\end{grammar}

Exceptions are handled with a \xcd"try" statement.
A \xcd"try" statement consists of a \xcd"try" block, zero or more
\xcd"catch" blocks, and an optional \xcd"finally" block.

First, the \xcd"try" block is evaluated.  If the block throws an
exception, control transfers to the first matching \xcd"catch"
block, if any.  A \xcd"catch" matches if the value of the
exception thrown is a subclass of the \xcd"catch" block's formal
parameter type.

The \xcd"finally" block, if present, is evaluated on all normal
and exceptional control-flow paths from the \xcd"try" block.
If the \xcd"try" block completes normally
or via a \xcd"return", a \xcd"break", or a
\xcd"continue" statement, 
the \xcd"finally"
block is evaluated, and then control resumes at
the statement following the \xcd"try" statement, at the branch target, or at
the caller as appropriate.
If the \xcd"try" block completes
exceptionally, the \xcd"finally" block is evaluated after the
matching \xcd"catch" block, if any, and then the
exception is rethrown.

\section{Return statement}
\label{ReturnStatement}
\index{ReturnStatement}
\begin{grammar}
Statement \: ReturnStatement \\
ReturnStatement \: \xcd"return" Expression \xcd";" \\
             \| \xcd"return" \xcd";" \\
\end{grammar}

Methods and closures may return values using a return statement.
If the method's return type is expliclty declared \xcd"Void",
the method may return without a value; otherwise, it must return
a value of the appropriate type.
	

\chapter{Places}
\label{XtenPlaces}
\index{place}

An \Xten{} place is a repository for data and activities, corresponding
loosely to a process or a processor. Places induce a concept of ``local''. The
activities running in a place may access data items located at that place with
the efficiency of on-chip access. Accesses to remote places may take orders of
magnitude longer. X10's system of places is designed to make this obvious.
Programmers are aware of the places of their data, and know when they are
incurring communication costs, but the actual operation to do so is easy. It's
not hard to use non-local data; it's simply hard to to do so accidentally.

The set of places available to a computation is determined at the time that
the program is started, and remains fixed through the run of the program. See
the {\tt README} documentation on how to set command line and configuration
options to set the number of places.

Places are first-class values in X10, as instances 
\xcd"x10.lang.Place".   \xcd`Place` provides a number of useful ways to
query places, such as \xcd`Place.places`, which is a  \xcd`Sequence[Place]` of 
the places
available to the current run of the program.

Objects and structs (with one exception) are created in a single place -- the
place that the constructor call was running in. They cannot change places.
They can be {\em copied} to other places, and the special library struct
\Xcd{GlobalRef} allows values at one place to point to values at another.  

\section{The Structure of Places}
\index{place!MAX\_PLACES}
\index{place!FIRST\_PLACES}
\index{MAX\_PLACES}
\index{FIRST\_PLACE}

%~~exp~~`~~`~~ ~~ ^^^ Places10
Places are numbered 0 through \xcd`Place.MAX_PLACES-1`; the number is stored
in the field 
\xcd`pl.id`.  The \xcd`Sequence[Place]` \xcd`Place.places()` contains the places of the
program, in numeric order. 
The program starts by executing a \xcd`main` method at
%~~exp~~`~~`~~ ~~ ^^^ Places20
\xcd`Place.FIRST_PLACE`, which is 
%~~exp~~`~~`~~ ~~ ^^^ Placesoik
\xcd`Place.places()(0)`; see
\Sref{initial-computation}. 

Operations on places include \xcd`pl.next()`, which gives the next entry
(looping around) in \xcd`Place.places` and its opposite \xcd`pl.prev()`. 
In multi-place executions, 
\xcd`here.next()` is a convenient way to express ``a place other than \xcd`here`''.
There are also tests, like  
%~~exp~~`~~`~~pl:Place ~~ ^^^ Placesoid
\xcd`pl.isCUDA()`, which test for particular kinds of processors.


\section{{\tt here}}\index{here}\label{Here}

The variable \xcd"here" is always bound to the place at which the current
computation is running, in the same way that \xcd`this` is always bound to the
instance of the current class (for non-static code), or \xcd`self` is bound to
the instance of the type currently being constrained.  
\xcd`here` may denote different places in the same method body or even the
same expression, due to
place-shifting operations.


This is not unusual for automatic variables:  \Xcd{self} denotes 
two different values (one \xcd`List`, one \xcd`Long`) 
when one describes a non-null list of non-zero numbers as
\xcd`List[Long{self!=0}]{self!=null}`. In the following 
code, \xcd`here` has one value at 
\xcd`h0`, and a different one at \xcd`h1` (unless there is only one place).
%~~gen ^^^ Placesoijo
% package places.are.For.Graces;
% class Example {
% def example() {
%~~vis
\begin{xten}
val h0 = here;
at (here.next()) {
  val h1 = here; 
  assert (h0 != h1);
}
\end{xten}
%~~siv
%} } 
% 
%~~neg
\noindent
(Similar examples show that \xcd`self` and \xcd`this` have the same behavior:
\xcd`self` can be shadowed by constrained types appearing inside of type
constraints, and \xcd`this` by inner classes.)



The following example looks through a list of references to \Xcd{Thing}s.  
It finds those references to things that are \Xcd{here}, and deals with them.  
%~~gen ^^^ Places70
%package Places.Are.For.Graces.2;
%import x10.util.*;
%abstract class Thing {}
%class DoMine {
%  static def dealWith(Thing) {}	
%~~vis
\begin{xten}
  public static def deal(things: List[GlobalRef[Thing]]) {
     for(gr in things) {
        if (gr.home == here) {
           val grHere = 
               gr as GlobalRef[Thing]{gr.home == here};
           val thing <: Thing = grHere();
           dealWith(thing);
        }
     }
  }
\end{xten}
%~~siv
%}
% 
%~~neg

\section{ {\tt at}: Place Changing}\label{AtStatement}
\index{at}
\index{place!changing}

An activity may change place synchronously using the \xcd"at" statement or
\xcd"at" expression. Like any distributed operation, it is 
potentially expensive, as it requires, at a minimum, two messages
and the copying of all data used in the operation, and must be used with care
-- but it provides the basis for distributed programming in X10.

%##(AtStatement AtExp
\begin{bbgrammar}
%(FROM #(prod:AtStmt)#)
              AtStmt \: \xcd"at" \xcd"(" Exp \xcd")" Stmt & (\ref{prod:AtStmt}) \\
%(FROM #(prod:AtExp)#)
               AtExp \: \xcd"at" \xcd"(" Exp \xcd")" ClosureBody & (\ref{prod:AtExp}) \\
\end{bbgrammar}
%##)

The {\it PlaceExp} must be an expression of type \xcd`Place` or some
subtype. For programming convenience, if {\it PlaceExp} is of type
\xcd`GlobalRef[T]` then the \xcd'home' property of \xcd'GlobalRef' is
used as the value of {\it PlaceExp}.

An activity may also spawn an asynchronous remote child activity.  For
optimal performance, it is desirable for the spawning activity to
continue executing locally without waiting for the message creating
the remote child activity to arrive at the destination place. \Xten{}
supports this ``fire-and-forget'' style of remote activity creation by
special handling of the combination of \xcd'at (P) async S'.  In
particular, any exceptions raised during deserialization
(\Sref{sect:at-init-val}) at the remote place will be reported
asynchronously (as if they occured after the remote activity
\xcd`async S` was spawned).

%%AT-COPY%% The \xcd`at`-statment \xcd`at(p;F)S` first evaluates \xcd`p` to a place, then
%%AT-COPY%% copies information to that place as determined by \xcd`F`, and then executes
%%AT-COPY%% \xcd`S` using the resulting copies.  The \xcd`at`-{\em expression}
%%AT-COPY%% \xcd`at(p;F)E` is similar, but it copies the result of the expression \xcd`E`
%%AT-COPY%% and returns the copy as its result.
%%AT-COPY%% 
%%AT-COPY%% The clause \xcd`F` in \xcd`at(p;F)S` is a list of zero or more {\em copy
%%AT-COPY%% specifiers}, explaining what values are to be copied to the place \xcd`p`, and
%%AT-COPY%% how they are to be referred to at \xcd`p`.  
%%AT-COPY%% 

%%AT-COPY%% \begin{ex}
%%AT-COPY%% The following example creates a rail \xcd`a` located \xcd`here`, and copies
%%AT-COPY%% it to another place, giving the copy the name \xcd`a2` there.  The copy is
%%AT-COPY%% modified and examined.  After the \xcd`at` finishes, the original is also
%%AT-COPY%% examined, and (since only the copy, not the original, was modified) is observed
%%AT-COPY%% to be unchanged. 
%%AT-COPY%% %~x~gen ^^^ Places6e1o
%%AT-COPY%% % package Places6e1o;
%%AT-COPY%% % KNOWNFAIL-at
%%AT-COPY%% % class Example { static def example() { 
%%AT-COPY%% %~x~vis
%%AT-COPY%% \begin{xten}
%%AT-COPY%% val a = [1,2,3];
%%AT-COPY%% at(here.next(); a2 = a) {
%%AT-COPY%%   a2(1) = 4;
%%AT-COPY%%   assert a2(0)==1 && a2(1)==4 && a2(2)==3; 
%%AT-COPY%%   // 'a' is not accessible here
%%AT-COPY%% }
%%AT-COPY%% assert  a(0)==1 && a(1)==2 && a(2)==3; 
%%AT-COPY%% \end{xten}
%%AT-COPY%% %~x~siv
%%AT-COPY%% %} } 
%%AT-COPY%% % class Hook { def run() { Example.example(); return true; }}
%%AT-COPY%% %~x~neg
%%AT-COPY%% \end{ex}
%%AT-COPY%% 

\begin{ex}
The following example creates a rail \xcd`a` located \xcd`here`, and copies
it to another place.  \xcd`a` in the second place (\xcd`here.next()`) refers
to the copy.  The copy is
modified and examined.  After the \xcd`at` finishes, the original is also
examined, and (since only the copy, not the original, was modified) is observed
to be unchanged. 
%~~gen ^^^ Places6e1o
% package Places6e1o;
% KNOWNFAIL-at
% class Example { static def example() { 
%~~vis
\begin{xten}
val a = [1,2,3];
at(here.next()) {
  a(1) = 4;
  assert a(0)==1 && a(1)==4 && a(2)==3; 
}
assert  a(0)==1 && a(1)==2 && a(2)==3; 
\end{xten}
%~~siv
%} } 
% class Hook { def run() { Example.example(); return true; }}
%~~neg
\end{ex}

%%AT-COPY%% \subsection{Copy Specifiers}
%%AT-COPY%% \label{sect:copy-spec}
%%AT-COPY%% \index{copy specifier}
%%AT-COPY%% \index{at!copy specifier}
%%AT-COPY%% 
%%AT-COPY%% A single copy specifier can be one of the following forms.   
%%AT-COPY%% Each copy specifier determines an {\em original-expression}, saying what value
%%AT-COPY%% will be copied, and a {\em target variable}, saying what it will be called.
%%AT-COPY%% 
%%AT-COPY%% \begin{itemize}
%%AT-COPY%% 
%%AT-COPY%% \item \xcd`val x = E`, and its usual variants \xcd`val x:T = E`, 
%%AT-COPY%%       \xcd`x : T = E`, and 
%%AT-COPY%%       \xcd`val x <: T = E`, evaluate the expression \xcd`E` at the initial
%%AT-COPY%%       place, copy it to \xcd`p`, and bind \xcd`x` to the copy, as normal for a
%%AT-COPY%%       local \xcd`val` binding.  If a type is supplied, it is checked
%%AT-COPY%%       statically in the usual way.  
%%AT-COPY%%       The original-expression is \xcd`E`, and the target variable is \xcd`x`.
%%AT-COPY%% 
%%AT-COPY%% \begin{ex}
%%AT-COPY%% The following code copies a variable \xcd`a` located \xcd`here` to a variable
%%AT-COPY%% \xcd`d` located \xcd`there`.  
%%AT-COPY%% Note that, while the copy \xcd`d` is available \xcd`there` inside of the \xcd`at`-block,
%%AT-COPY%% the original \xcd`a` is not.  (\xcd`a` could not be available in the block in
%%AT-COPY%% any case; it is not located \xcd`there`.)
%%AT-COPY%% %~~gen ^^^ Places9v2e1
%%AT-COPY%% % package Places9v2e1;
%%AT-COPY%% % KNOWNFAIL-at
%%AT-COPY%% % class Example{ 
%%AT-COPY%% % static def use(Any) = 1;
%%AT-COPY%% % static def example() { 
%%AT-COPY%% %  val there = here.next();
%%AT-COPY%% %~~vis
%%AT-COPY%% \begin{xten}
%%AT-COPY%% var a : Long = 1;
%%AT-COPY%% at(there; val d = a) {
%%AT-COPY%%    assert d == 1;
%%AT-COPY%%    // ERROR: assert a == 1;
%%AT-COPY%% }
%%AT-COPY%% \end{xten}
%%AT-COPY%% %~~siv
%%AT-COPY%% % } } 
%%AT-COPY%% % class Hook{ def run() {Example.example(); return true;}}
%%AT-COPY%% %~~neg
%%AT-COPY%% \end{ex}
%%AT-COPY%% 
%%AT-COPY%% \item \xcd`var x : T = E` evaluates \xcd`E` at the initial place, copies it to
%%AT-COPY%%       \xcd`p`, and binds \xcd`x` to a new \xcd`var` whose initial value is the
%%AT-COPY%%       copy, as normal for a local \xcd`var` binding.
%%AT-COPY%%       If a type is supplied, it is checked
%%AT-COPY%%       statically in the usual way.
%%AT-COPY%%       The original-expression is \xcd`E`, and the target variable is \xcd`x`.
%%AT-COPY%%       Note that, like a \xcd`var` parameter to a method, \xcd`x` is a local
%%AT-COPY%%       variable.  Changes to \xcd`x` will not change anything else. In
%%AT-COPY%%       particular, even if \xcd`x` has the same name as a \xcd`var` variable
%%AT-COPY%%       outside, the two \xcd`var`s are unconnected.  
%%AT-COPY%%       See \Sref{sect:athome} for the way to modify a variable from the
%%AT-COPY%%       surrounding scope.
%%AT-COPY%% 
%%AT-COPY%% \begin{ex}
%%AT-COPY%% The following code copies \xcd`a` to a \xcd`var` named \xcd`e`.  Changing
%%AT-COPY%% \xcd`e` does not change \xcd`a`; the two \xcd`var`s have no ongoing relationship.
%%AT-COPY%% %~~gen ^^^ Places9v2e2
%%AT-COPY%% % package Places9v2e2;
%%AT-COPY%% % KNOWNFAIL-at
%%AT-COPY%% % class Example{ 
%%AT-COPY%% % static def use(Any) = 1;
%%AT-COPY%% % static def example() { 
%%AT-COPY%% %  val there = here.next();
%%AT-COPY%% %~~vis
%%AT-COPY%% \begin{xten}
%%AT-COPY%% var a : Long = 1;
%%AT-COPY%% assert a == 1;
%%AT-COPY%% at(there; var e = a) { 
%%AT-COPY%%    assert e == 1;
%%AT-COPY%%    e += 1;
%%AT-COPY%%    assert e == 2;
%%AT-COPY%% }
%%AT-COPY%% assert a == 1; 
%%AT-COPY%% \end{xten}
%%AT-COPY%% %~~siv
%%AT-COPY%% % 
%%AT-COPY%% % }  } 
%%AT-COPY%% % class Hook{ def run() {Example.example(); return true;}}
%%AT-COPY%% %~~neg
%%AT-COPY%% \end{ex}
%%AT-COPY%% 
%%AT-COPY%% \item \xcd`x = E`, as a copy specifier, is equivalent to \xcd`val x = E`.
%%AT-COPY%%       Note that this abbreviated form is not available as a local variable
%%AT-COPY%%       definition, (because it is used as an assignment statement), but in a
%%AT-COPY%%       copy specifier there are no assignment statements and so the
%%AT-COPY%%       abbreviation is allowed.
%%AT-COPY%%       The original-expression is \xcd`E`, and the target variable is \xcd`x`.
%%AT-COPY%% 
%%AT-COPY%% \begin{ex}
%%AT-COPY%% The following code evaluates an expression \xcd`a+b(0)`.  The result of this
%%AT-COPY%% expression is stored \xcd`there`, in the \xcd`val` variable \xcd`f`, but is
%%AT-COPY%% not stored \xcd`here`. 
%%AT-COPY%% %~~gen ^^^ Places9v2e3
%%AT-COPY%% % package Places9v2e3;
%%AT-COPY%% % KNOWNFAIL-at
%%AT-COPY%% % class Example{ 
%%AT-COPY%% % static def use(Any) = 1;
%%AT-COPY%% % static def example() { 
%%AT-COPY%% %  val there = here.next();
%%AT-COPY%% %~~vis
%%AT-COPY%% \begin{xten}
%%AT-COPY%% var a : Long = 1;
%%AT-COPY%% var b : Rail[Long] = [2,3,4];
%%AT-COPY%% at(there; f = a + b(0)) {
%%AT-COPY%%    assert f == 3;
%%AT-COPY%% }
%%AT-COPY%% \end{xten}
%%AT-COPY%% %~~siv
%%AT-COPY%% % }  } 
%%AT-COPY%% % class Hook{ def run() {Example.example(); return true;}}
%%AT-COPY%% % 
%%AT-COPY%% %~~neg
%%AT-COPY%% 
%%AT-COPY%% 
%%AT-COPY%% \end{ex}
%%AT-COPY%% 
%%AT-COPY%% \item \xcd`x` alone, as a copy specifier, is equivalent to \xcd`val x = x`.
%%AT-COPY%%       It says that the variable \xcd`x` will be copied, and the copy will also
%%AT-COPY%%       be named \xcd`x`.  
%%AT-COPY%%       The original-expression is \xcd`x`, and the target variable is \xcd`x`.
%%AT-COPY%% 
%%AT-COPY%% \begin{ex}
%%AT-COPY%% The following code copies \xcd`b` to \xcd`there`.  The copy is also called
%%AT-COPY%% \xcd`b`.  The two \xcd`b`'s are not connected; \eg, changing one does not
%%AT-COPY%% change the other.
%%AT-COPY%% %~~gen ^^^ Places9v2e4
%%AT-COPY%% % package Places9v2e4;
%%AT-COPY%% % KNOWNFAIL-at
%%AT-COPY%% % class Example{ 
%%AT-COPY%% % static def use(Any) = 1;
%%AT-COPY%% % static def example() { 
%%AT-COPY%% %  val there = here.next();
%%AT-COPY%% %~~vis
%%AT-COPY%% \begin{xten}
%%AT-COPY%% var b : Rail[Long] = [2,3,4];
%%AT-COPY%% assert b(0) == 2;
%%AT-COPY%% at(there; b) {
%%AT-COPY%%   b(0) = 200;  // Modify copy of b.
%%AT-COPY%%   assert b(0) == 200;
%%AT-COPY%% }
%%AT-COPY%% assert b(0) == 2; 
%%AT-COPY%% \end{xten}
%%AT-COPY%% %~~siv
%%AT-COPY%% % 
%%AT-COPY%% % }  } 
%%AT-COPY%% % class Hook{ def run() {Example.example(); return true;}}
%%AT-COPY%% %~~neg
%%AT-COPY%% \end{ex}
%%AT-COPY%% 
%%AT-COPY%% \item A field assignment statements \xcdmath"a.fld = $E_2$", evaluates 
%%AT-COPY%%       \xcd`a` and $E_2$ on the sending side to values $v_1$ and {$v_2$}.  
%%AT-COPY%%       {$v_1$} must be an object with a mutable field \xcd`fld`.  {$v_1$} and
%%AT-COPY%%       {$v_2$} are sent to place \xcd`p`, and the field assignment is performed
%%AT-COPY%%       there.  The modified version of {$v_1$} is available as a \xcd`val`
%%AT-COPY%%       variable \xcd`a`.   The compiler may optimize this, \eg, by neglecting to
%%AT-COPY%%       deserialize \xcdmath"$v_1$.fld", and deserializing {$v_2$} directly into
%%AT-COPY%%       that field rather than into a separate buffer.
%%AT-COPY%% 
%%AT-COPY%% \begin{ex}
%%AT-COPY%% %~~gen ^^^ Places9v2e5
%%AT-COPY%% % package Places9v2e5;
%%AT-COPY%% % KNOWNFAIL
%%AT-COPY%% % class Example {
%%AT-COPY%% % static def use(Any) = 1;
%%AT-COPY%% % static def example() { 
%%AT-COPY%% %  val there = here.next();
%%AT-COPY%% %~~vis
%%AT-COPY%% \begin{xten}
%%AT-COPY%% class Example{ 
%%AT-COPY%%    var f : Long = 1;
%%AT-COPY%%    var g : Long = 2;
%%AT-COPY%%    static def example() { 
%%AT-COPY%%       val there = here.next();
%%AT-COPY%%       val e : Example = new Example();
%%AT-COPY%%       assert e.f == 1 && e.g == 2;
%%AT-COPY%%       at(there; e.f = 3) {
%%AT-COPY%%           assert e.f == 3; && e.g == 2;
%%AT-COPY%%       }
%%AT-COPY%%       assert e.f == 1 && e.g == 2;
%%AT-COPY%%    }
%%AT-COPY%% }
%%AT-COPY%% \end{xten}
%%AT-COPY%% %~~siv
%%AT-COPY%% % class Hook{ def run() {Example.example(); return true;}}
%%AT-COPY%% %~~neg
%%AT-COPY%% %
%%AT-COPY%% \end{ex}
%%AT-COPY%% 
%%AT-COPY%% \item A rail-element assignment 
%%AT-COPY%%       \xcdmath"a($E_1$, $\ldots$, $E_n$) = $E_+$".
%%AT-COPY%%       This copies and transmits \xcd`a` as normal for a rail.  In addition,
%%AT-COPY%%       and 
%%AT-COPY%%       much like a field assignment, it also evaluates all the expressions $E_i$
%%AT-COPY%%       at the sending side to values $v_i$, and transmits them.  \xcd`a`'s value must
%%AT-COPY%%       admit a suitably-typed $n$-ary subscripting operation.  That operation
%%AT-COPY%%       is applied after the values are deserialized at \xcd`p`.  The compiler
%%AT-COPY%%       may optimize this, \eg, by neglecting to deserialize one element of the
%%AT-COPY%%       rail $v_0$, and deserializing $v_+$ directly into that location.  
%%AT-COPY%% 
%%AT-COPY%% 
%%AT-COPY%% \begin{ex}
%%AT-COPY%% The following code sends a modified \xcd`b` to \xcd`there`, while (as always)
%%AT-COPY%% keeping an unmodified version \xcd`here`.   X10 may perform optimizations to
%%AT-COPY%% avoid transmitting the original value of \xcd`b(1)`, since it will be
%%AT-COPY%% overwritten immediately in any case.
%%AT-COPY%% %~~gen ^^^ Places9v2e6
%%AT-COPY%% % package Places9v2e6;
%%AT-COPY%% % KNOWNFAIL
%%AT-COPY%% % class Example{ 
%%AT-COPY%% % static def use(Any) = 1;
%%AT-COPY%% % static def example() { 
%%AT-COPY%% %  val there = here.next();
%%AT-COPY%% %~~vis
%%AT-COPY%% \begin{xten}
%%AT-COPY%% var b = [2,3,4];
%%AT-COPY%% assert b(0) == 2 && b(1) == 3;
%%AT-COPY%% at(there; b(1) = 300) {
%%AT-COPY%%   assert b(0) == 2 && b(1) == 300;
%%AT-COPY%% }
%%AT-COPY%% assert b(0) == 2 && b(1) == 3;
%%AT-COPY%% \end{xten}
%%AT-COPY%% %~~siv
%%AT-COPY%% % 
%%AT-COPY%% %~~neg
%%AT-COPY%% % }  }
%%AT-COPY%% % class Hook{ def run() {Example.example(); return true;}}
%%AT-COPY%% \end{ex}
%%AT-COPY%% 
%%AT-COPY%% \item \xcd`*` may appear as the last copy specifier in the list, indicating
%%AT-COPY%%       that all \xcd`val` variables from outside \xcd`S` which are used in
%%AT-COPY%%       \xcd`S` should be copied. Specifically, let 
%%AT-COPY%%       \xcdmath"x$_1, \ldots, $x$_n$" be all the \xcd`val` variables defined
%%AT-COPY%%       outside of \xcd`S` 
%%AT-COPY%%       mentioned in \xcd`S`. The \xcd`*` copy specifier is equivalent to 
%%AT-COPY%%       the list of variables 
%%AT-COPY%%       \xcdmath"x$_1, \ldots, $x$_n$".
%%AT-COPY%% 
%%AT-COPY%% \begin{ex}
%%AT-COPY%% %~~gen ^^^ Places9v2e7
%%AT-COPY%% % package Places9v2e7;
%%AT-COPY%% % KNOWNFAIL-at
%%AT-COPY%% % class Example{ 
%%AT-COPY%% % static def use(Any) = 1;
%%AT-COPY%% % static def example() { 
%%AT-COPY%% %  val there = here.next();
%%AT-COPY%% %~~vis
%%AT-COPY%% \begin{xten}
%%AT-COPY%% var a : Long = 1;
%%AT-COPY%% val b = [2,3,4];
%%AT-COPY%% at(there; *) {
%%AT-COPY%%   assert a + b(0) == b(1);
%%AT-COPY%% }
%%AT-COPY%% \end{xten}
%%AT-COPY%% %~~siv
%%AT-COPY%% % }  }
%%AT-COPY%% % class Hook{ def run() {Example.example(); return true;}}
%%AT-COPY%% %~~neg
%%AT-COPY%% 
%%AT-COPY%% \end{ex}
%%AT-COPY%% 
%%AT-COPY%% \end{itemize}
%%AT-COPY%% 
%%AT-COPY%% As an important special case, \xcd`at(p;)S` copies {\em nothing} to \xcd`S`.
%%AT-COPY%% This must not be confused with \xcd`at(p)S`, which copies {\em everything}.
%%AT-COPY%% 
%%AT-COPY%% 
%%AT-COPY%% 
%%AT-COPY%% Note that \xcd`at(p;x,*)use(x,y);` is equivalent to \xcd`at(p;*)use(x,y);`.
%%AT-COPY%% In both statements, the \xcd`*` indicates that all variables used in the body
%%AT-COPY%% are to be copied in.  The former makes clear that \xcd`x` is one of the things
%%AT-COPY%% being copied, but, from the \xcd`*`, there may be others. 
%%AT-COPY%% 
%%AT-COPY%% However, other copy specifiers may be used to compute
%%AT-COPY%% values in \xcd`S` which are not available (and thus need not be stored)
%%AT-COPY%% outside of it.  
%%AT-COPY%% 
%%AT-COPY%% \begin{ex}The following code may end up with a large object \xcd`c` in
%%AT-COPY%% memory at \xcd`p` but not at the initial place: 
%%AT-COPY%% %~~gen ^^^ Places3q9u
%%AT-COPY%% % package Places3q9u;
%%AT-COPY%% % KNOWNFAIL-at
%%AT-COPY%% % class Example { 
%%AT-COPY%% % def use(Example, Example, Example) = 1;
%%AT-COPY%% % def Elephant(Example) = 1;
%%AT-COPY%% % static def example(a: Example, b:Example, p:Place) { 
%%AT-COPY%% %~~vis
%%AT-COPY%% \begin{xten}
%%AT-COPY%% at(p; c = a.Elephant(b), *) {
%%AT-COPY%%   use(a,b,c);
%%AT-COPY%% }
%%AT-COPY%% \end{xten}
%%AT-COPY%% %~~siv
%%AT-COPY%% %} } 
%%AT-COPY%% %~~neg
%%AT-COPY%% \end{ex}
%%AT-COPY%% 
%%AT-COPY%% The blanket \xcd`at`-statement \xcd`at(p)S` copies everything.  It is an
%%AT-COPY%% abbreviation for \xcd`at(p;*)S`.  
%%AT-COPY%% When this manual refers to a generic \xcd`at`-statement as \xcd`at(p;F)S`, it
%%AT-COPY%% should be understood as including the blanket \xcd`at` statement \xcd`at(p)S`
%%AT-COPY%% with this interpretation.
%%AT-COPY%% 

\subsection{Copying Values}
%%AT-COPY%% An activity executing statement \xcd"at (q;F) S" at a place \xcd`p`
%%AT-COPY%% evaluates \xcd`q` at \xcd`p` and then moves to \xcd`q` to execute
%%AT-COPY%% \xcd`S`.  
%%AT-COPY%% The original-expressions of \xcd`F` are evaluated at \xcd`p`.
%%AT-COPY%% Their values are copied (\Sref{sect:at-init-val}) to \xcd`q`, and bound to 
%%AT-COPY%% names there, as specified by \xcd`F`.  
%%AT-COPY%% \xcd`S` is evaluated in an environment containing the target variables of
%%AT-COPY%% \xcd`F`, and \xcd`here` and {\em no} other variables.  (In particular, if this
%%AT-COPY%% statement appears in an instance method body and \xcd`this` is not copied,
%%AT-COPY%% \xcd`this` is not accessible.  This fact is important: it allows the
%%AT-COPY%% programmer to control when \xcd`this` is copied, which may be expensive for
%%AT-COPY%% large containers.)

An activity executing \xcd`at(q)S` at a place \xcd`p` evaluates \xcd`q` at
place \xcd`p`, which should be a \xcd`Place`.  It then moves to place \xcd`q`
to execute \xcd`S`.  The values variables that \xcd`S` refers to are copied
(\Sref{sect:at-init-val}) to \xcd`q`, and bound to the variables of the same
name.   If the \xcd`at` is inside of an instance method and \xcd`S` uses
\xcd`this`, \xcd`this` is copied as well.  Note that a field reference
\xcd`this.fld` or a method call \xcd`this.meth()` will cause \xcd`this` to be
copied --- as will their abbreviated forms \xcd`fld` and \xcd`meth()`, despite
the lack of a visible \xcd`this`.  


Note that the value obtained by evaluating \xcd`q`
is not necessarily distinct from \xcd`p` (\eg, \xcd`q` may be
\xcd`here`). 
This does not alter the behavior of \xcd`at`.  
%%AT-COPY%%  \xcd`at(here;F)S` will copy all the values specified by \xcd`F`, 
%%AT-COPY%% even though there is no actual change of place, and even though the original
%%AT-COPY%% values already exist there.
\xcd`at(here)S` will copy all the values mentioned in \xcd`S`, even though
there is no actual change of place, and even though the original values
already exist there. 

On normal termination of \xcd`S` control returns to \xcd`p` and
execution is continued with the statement following 
%%AT-COPY%% \xcd`at (q;F) S`. 
\xcd`at (q) S`. 
If
\xcd`S` terminates abruptly with exception \xcd`E`, \xcd`E` is
serialized into a buffer, the buffer is communicated to \xcd`p` where
it is deserialized into an exception \xcd`E1` and \xcd`at (p) S`
throws \xcd`E1`.

Since 
%%AT-COPY%% \xcd`at(p;F) S` 
\xcd`at(p) S` 
is a synchronous construct, usual control-flow
constructs such as \xcd`break`, \xcd`continue`, \xcd`return` and 
\xcd`throw` are permitted in \xcd`S`.  All concurrency related
constructs -- \xcd`async`, \xcd`finish`, \xcd`atomic`, \xcd`when` are
also permitted.

The \xcd`at`-expression 
%%AT-COPY%% \xcd`at(p;F)E` 
\xcd`at(p)E` 
is similar, except that, in the case of
normal termination of \xcd`E`, the value that \xcd`E` produces is serialized
into a buffer, transported to the starting place, and deserialized, and the
value of the \xcd`at`-expression is the result of deserialization.

\limitation{
X10 does not currently allow {\tt break}, {\tt continue}, or {\tt return}
to exit from an {\tt at}.
}



\subsection{How {\tt at} Copies Values}
\label{sect:at-init-val}

%%AT-COPY%% The values of the original-expressions  specified by \xcd`F` in 
%%AT-COPY%% \xcd`at (p;F)S` are copied to \xcd`p`, as follows.

The values mentioned in \xcd`S` are copied to place \xcd`p` by \xcd`at(p)S` as follows.

First, the original-expressions are evaluated to give a vector of X10 values.
Consider the graph of all values reachable from these values (except for 
\xcd`transient` fields 
(\Sref{sect:transient}, \xcd`GlobalRef`s (\Sref{GlobalRef}); also custom
serialization (\Sref{sect:ser+deser} may alter this behavior)). 

Second this graph is {\em
serialized} into a buffer and transmitted to place \xcd`q`.  Third,
the vector of X10 values is 
re-created at \xcd`q` 
by deserializing the buffer at
\xcd`q`. Fourth, \xcd`S` is executed at \xcd`q`, in an environment in
which each variable \xcd`v` declared in \xcd`F` 
refers to the corresponding deserialized value.  

Note that since values accessed across an \xcd`at` boundary are
copied, the programmer may wish to adopt the discipline that either
variables accessed across an \xcd`at` boundary  contain only structs 
or stateless objects, or the methods invoked on them do not access any
mutable state on the objects. Otherwise the programmer has to ensure
that side effects are made to the correct copy of the object. For this
the struct \xcd`x10.lang.GlobalRef[T]` is often useful.


\subsubsection{Serialization and deserialization.}
\label{sect:ser+deser}
\index{transient}
\index{field!transient}
The X10 runtime provides a default mechanism for
serializing/deserializing an object graph with a given set of roots.
This mechanism may be overridden by the programmer on a per class or
struct basis as described in the API documentation for
\xcd`x10.io.CustomSerialization`.  
The default mechanism performs a
deep copy of the object graph (that is, it copies the object or struct
and, recursively, the values contained in its fields), but does not
traverse or copy \xcd`transient` fields. \xcd`transient` fields are omitted from the
serialized data.   On deserialization, \xcd`transient` fields are initialized
with their default values (\Sref{DefaultValues}).    The types of
\xcd`transient` fields must therefore have default values.

The default serialization/deserialization mechanism will not (modulo
error conditions like \xcd`OutOfMemoryError`) throw any exceptions. However,
user code running during serialization/deserialization via
\xcd`CustomSerialization` may raise exceptions.  These exceptions are
handled like any other exception raised during the execution of an X10
activity.  However, due to the special treatment of \xcd`at (p) async S` 
(\Sref{sect:AtStatement}) any exception raised during
deserialization will be handled as if it was raised by \xcd`async S`,
not by the \xcd`at` statement itself.


A struct \xcd`s` of type \xcd`x10.lang.GlobalRef[T]` \ref{GlobalRef}
is serialized as a unique global reference to its contained object
\xcd`o` (of type \xcd`T`).  Please see the documentation
of \xcd`x10.lang.GlobalRef[T]` for more details.



\subsection{{\tt at} and Activities}
%%AT-COPY%% \xcd`at(p;F)S` 
\xcd`at(p)S` 
does {\em not} start a new activity.  It should be thought of as
transporting the current activity to \xcd`p`, running \xcd`S` there, and then
transporting it back.  \xcd`async` is the only construct in the
language that starts a new activity. In different contexts, each one
of the following makes sense:
%%AT-COPY%% (1)~\xcd`async at(p;F) S` 
(1)~\xcd`async at(p) S` 
(spawn an activity locally to execute \xcd`S` at
\xcd`p`; here \xcd`p` is evaluated by the spawned activity) , 
%%AT-COPY%% (2)~\xcd`at(p;F) async S` 
(2)~\xcd`at(p) async S` 
(evaluate \xcd`p` and then at \xcd`p` spawn an
activity to execute \xcd`S`), and,
%%AT-COPY%% (3)~\xcd`async at(p;F) async S`. 
(3)~\xcd`async at(p) async S`. 
%%AT-COPY%% In most cases, \xcd`at(p;F) async S` is preferred to
%%\xcd`async at(p;F)`, since In most cases, \xcd`at(p) async S` is
preferred to \xcd`async at(p) S`, since the former form enables a more
efficient runtime implementation.  In the first case, the expression
\xcd`p` is evaluated synchronously by the current activity and then a
single remote async is spawned.  In the second case, \xcd`p` is
semantically required to be evaluated asynchronously with the parent
async as it is contained in the body of an async.  Therefore, if the
compiler cannot prove that "async at (p)" can be safely rewritten into
"at (p) async", a first local async is spawned to evaluate \xcd`p`
then a remote async is spawned to evaluate \xcd`S`.

Since 
%%AT-COPY%% \Xcd{at(p;F) S} 
\Xcd{at(p) S} 
does not start a new activity, 
\xcd`S` may contain constructs which only make sense
within a single activity.  
For example, 
\begin{xten}
    for(x in globalRefsToThings) 
      if (at(x.home) x().isNice()) 
        return x();
\end{xten}
returns the first nice thing in a collection.   If we had used 
\xcd`async at(x.home)`, this would not be allowed; 
you can't \xcd`return` from an
\xcd`async`. 

\limitation{
X10 does not currently allow {\tt break}, {\tt continue}, or {\tt return}
to exit from an {\tt at}.
}



\subsection{Copying from {\tt at} }
\index{at!copying}

%%AT-COPY%% \xcd`at(p;F)S` copies data as specified by \xcd`F`, and sends it
\xcd`at(p)S` copies data required in \xcd`S`, and sends it
to place \xcd`p`, before executing \xcd`S` there. The only things that are not
copied are values only reachable through \xcd`GlobalRef`s and \xcd`transient`
fields, and data omitted by custom serialization.    
%%AT-COPY%% Several choices of copy specifier use the same identifier for the original
%%AT-COPY%% variable outside of 
%%AT-COPY%% \xcd`at(p)S` 
%%AT-COPY%% and its copy inside of \xcd`S`.  
%%AT-COPY%% 

\begin{ex}
%%AT-COPY%% 
%%AT-COPY%% %~~gen ^^^ Places_implicit_copy_from_at_example_1
%%AT-COPY%% % package Places.implicitcopyfromat;
%%AT-COPY%% % class Example {
%%AT-COPY%% % static def example() {
%%AT-COPY%% % 
%%AT-COPY%% %~~vis
%%AT-COPY%% \begin{xten}
%%AT-COPY%% val c = new Cell[Long](9); // (1)
%%AT-COPY%% at (here;c) {             // (2)
%%AT-COPY%%    assert(c() == 9);      // (3)
%%AT-COPY%%    c.set(8);              // (4)
%%AT-COPY%%    assert(c() == 8);      // (5)
%%AT-COPY%% }
%%AT-COPY%% assert(c() == 9);         // (6)
%%AT-COPY%% \end{xten}
%%AT-COPY%% %~~siv
%%AT-COPY%% %}}
%%AT-COPY%% % class Hook{ def run() { Example.example(); return true; } }
%%AT-COPY%% %~~neg
%%AT-COPY%% 

%~~gen ^^^ Places_implicit_copy_from_at_example_1
% package Places.implicitcopyfromat;
% class Example {
% static def example() {
% 
%~~vis
\begin{xten}
val c = new Cell[Long](9); // (1)
at (here) {               // (2) 
   assert(c() == 9);      // (3)
   c.set(8);              // (4)
   assert(c() == 8);      // (5)
}
assert(c() == 9);         // (6)
\end{xten}
%~~siv
%}}
% class Hook{ def run() { Example.example(); return true; } }
%~~neg


The \xcd`at` statement copies the \xcd`Cell` and its contents.  
After \xcd`(1)`, \xcd`c` is a \xcd`Cell` containing 9; call that cell {$c_1$}
At \xcd`(2)`, that cell is copied, resulting in another cell {$c_2$} whose
contents are also 9, as tested at \xcd`(3)`.
(Note that the copying behavior of \xcd`at` happens {\em even when the
destination place is the same as the starting place}--- even with
\xcd`at(here)`.)
At \xcd`(4)`, the contents of {$c_2$} are changed to 8, as confirmed at \xcd`(5)`; the contents of
{$c_1$} are of course untouched.    Finally, at \xcd`(c)`, outside the scope
of the \xcd`at` started at line \xcd`(2)`, \xcd`c` refers to its original
value {$c_1$} rather than the copy {$c_2$}.  
\end{ex}

The \xcd`at` statement induces a {\em deep copy}.  Not only does it copy the
values of variables, it copies values that they refer to through zero or more
levels of reference.  Structures are preserved as well: if two fields
\xcd`x.f` and \xcd`x.g` refer to the same object {$o_1$} in the original, then
\xcd`x.f` and \xcd`x.g` will both refer to the same object {$o_2$} in the
copy.  

\begin{ex}
In the following variation of the preceding example,
\xcd`a`'s original value {$a_1$} is a rail with two references to the same
\xcd`Cell[Long]` {$c_1$}.  The fact that {$a_1(0)$} and {$a_1(1)$} are both
identical to {$c_1$} is demonstrated in \xcd`(A)`-\xcd`(C)`, as {$a_1(0)$} is modified
and {$a_1(1)$} is observed to change.  In \xcd`(D)`-\xcd`(F)`, the copy
{$a_2$} is tested in the same way, showing that {$a_2(0)$} and {$a_2(1)$} both
refer to the same \xcd`Cell[Long]` {$c_2$}.  However, the test at \xcd`(G)`
shows that {$c_2$} is a different cell from {$c_1$}, because changes to
{$c_2$} did not propagate to {$c_1$}.  

%%AT-COPY%% %~~gen ^^^ PlacesAtCopy
%%AT-COPY%% %package Places.AtCopy2;
%%AT-COPY%% %class example {
%%AT-COPY%% %static def Example() {
%%AT-COPY%% %
%%AT-COPY%% %~~vis
%%AT-COPY%% \begin{xten}
%%AT-COPY%% val c = new Cell[Long](5);
%%AT-COPY%% val a : Rail[Cell[Long]] = [c,c as Cell[Long]];
%%AT-COPY%% assert(a(0)() == 5 && a(1)() == 5);     // (A)
%%AT-COPY%% c.set(6);                               // (B)
%%AT-COPY%% assert(a(0)() == 6 && a(1)() == 6);     // (C)
%%AT-COPY%% at(here;a) {
%%AT-COPY%%   assert(a(0)() == 6 && a(1)() == 6);   // (D)
%%AT-COPY%%   c.set(7);                             // (E)
%%AT-COPY%%   assert(a(0)() == 7 && a(1)() == 7);   // (F)
%%AT-COPY%% }
%%AT-COPY%% assert(a(0)() == 6 && a(1)() == 6);     // (G)
%%AT-COPY%% \end{xten}
%%AT-COPY%% %~~siv
%%AT-COPY%% %}}
%%AT-COPY%% %class Hook{ def run() { example.Example(); return true; } }
%%AT-COPY%% %~~neg

%~~gen ^^^ PlacesAtCopy
%package Places.AtCopy2;
%class example {
%static def Example() {
%
%~~vis
\begin{xten}
val c = new Cell[Long](5);
val a : Rail[Cell[Long]] = [c,c as Cell[Long]];
assert(a(0)() == 5 && a(1)() == 5);     // (A)
c.set(6);                               // (B)
assert(a(0)() == 6 && a(1)() == 6);     // (C)
at(here) {
  assert(a(0)() == 6 && a(1)() == 6);   // (D)
  c.set(7);                             // (E)
  assert(a(0)() == 7 && a(1)() == 7);   // (F)
}
assert(a(0)() == 6 && a(1)() == 6);     // (G)
\end{xten}
%~~siv
%}}
%class Hook{ def run() { example.Example(); return true; } }
%~~neg


\end{ex}

\subsection{Copying and Transient Fields}
\label{sect:transient}
\index{at!transient fields and}
\index{transient}
\index{field!transient}

Recall that fields of classes and structs marked \xcd`transient` are not copied by
\xcd`at`.  Instead, they are set to the default values for their types. Types
that do not have default values cannot be used in \xcd`transient` fields.

\begin{ex}
Every \xcd`Trans` object has an \xcd`a`-field equal
to 1.  However, despite the initializer on the \xcd`b` field, it is not the
case that every \xcd`Trans` has \xcd`b==2`.  Since \xcd`b` is \xcd`transient`,
when the \xcd`Trans` value \xcd`this` is copied at \xcd`at(here){...}` in
\xcd`example()`, its \xcd`b` field is not copied, and the default value for an
\xcd`Long`, 0, is used instead.  
Note that we could not make a transient field \xcd`c : Long{c != 0}`, since the
type has no default value, and copying would in fact set it to zero.

%%AT-COPY%% %~~gen ^^^ Places40
%%AT-COPY%% %package Places_transient_a;
%%AT-COPY%% % 
%%AT-COPY%% %~~vis
%%AT-COPY%% \begin{xten}
%%AT-COPY%% class Trans {
%%AT-COPY%%    val a : Long = 1;
%%AT-COPY%%    transient val b : Long = 2;
%%AT-COPY%%    //ERROR transient val c : Long{c != 0} = 3;
%%AT-COPY%%    def example() {
%%AT-COPY%%      assert(a == 1 && b == 2);
%%AT-COPY%%      at(here;a) {
%%AT-COPY%%         assert(a == 1 && b == 0);
%%AT-COPY%%      }
%%AT-COPY%%    }
%%AT-COPY%% }
%%AT-COPY%% \end{xten}
%%AT-COPY%% %~~siv
%%AT-COPY%% %class Hook{ def run() { (new Trans()).example(); return true; } }
%%AT-COPY%% %~~neg

%~~gen ^^^ Places40
%package Places_transient_a;
% 
%~~vis
\begin{xten}
class Trans {
   val a : Long = 1;
   transient val b : Long = 2;
   //ERROR: transient val c : Long{c != 0} = 3;
   def example() {
     assert(a == 1 && b == 2);
     at(here) {
        assert(a == 1 && b == 0);
     }
   }
}
\end{xten}
%~~siv
%class Hook{ def run() { (new Trans()).example(); return true; } }
%~~neg



\end{ex}

\subsection{Copying and GlobalRef}
\label{GlobalRef}
\index{at!GlobalRef}
\index{at!blocking copying}

%%The other barrier to the potentially copious copying behavior of \xcd`at`
%%is the \xcd`GlobalRef` struct.  
A \xcd`GlobalRef[T]` (say \xcd`g`) contains a reference to
a value \xcd`v` of type \xcd`T`, in a form which can be transmitted, and a \xcd`Place`
\xcd`g.home` indicating where the value lives. When a 
\xcd`GlobalRef` is serialized an opaque, globally unique handle to
\xcd`v` is created.  

\begin{ex}The following example does not copy the value \xcd`huge`.  However, \xcd`huge`
would have been copied if it had been put into a \xcd`Cell`, or simply used
directly. 

%%AT-COPY%% %~~gen ^^^ Places50
%%AT-COPY%% %package Places.copyingblockingwithglobref;
%%AT-COPY%% % class GR {
%%AT-COPY%% %  static def use(Any){}
%%AT-COPY%% %  static def example() {
%%AT-COPY%% % 
%%AT-COPY%% %~~vis
%%AT-COPY%% \begin{xten}
%%AT-COPY%% val huge = "A potentially big thing";
%%AT-COPY%% val href = GlobalRef(huge);
%%AT-COPY%% at (here;href) {
%%AT-COPY%%    use(href);
%%AT-COPY%%   }
%%AT-COPY%% }
%%AT-COPY%% \end{xten}
%%AT-COPY%% %~~siv
%%AT-COPY%% %}
%%AT-COPY%% % class Hook{ def run() { GR.example(); return true; } }
%%AT-COPY%% %~~neg

%~~gen ^^^ Places50
%package Places.copyingblockingwithglobref;
% class GR {
%  static def use(Any){}
%  static def example() {
% 
%~~vis
\begin{xten}
val huge = "A potentially big thing";
val href = GlobalRef(huge);
at (here) {
   use(href);
  }
}
\end{xten}
%~~siv
%}
% class Hook{ def run() { GR.example(); return true; } }
%~~neg


\end{ex}

Values protected in \xcd`GlobalRef`s can be retrieved by the application
%~~exp~~`~~`~~ g:GlobalRef[Any]{here == g.home}~~ ^^^Places4e7q
operation \xcd`g()`.  \xcd`g()` is guarded; it can 
only be called when \xcd`g.home == here`.  If you  want to do anything other
than pass a global reference around or compare two of them for equality, you
need to placeshift back to the home place of the reference, often with
\xcd`at(g.home)`.   

\begin{ex}The following program, for reasons best known to the programmer,
modifies the 
command-line argument array.

%%AT-COPY%% 
%%AT-COPY%% %~~gen ^^^ Places60
%%AT-COPY%% % package Places.Atsome.Globref2;
%%AT-COPY%% % class GR2 {
%%AT-COPY%% % 
%%AT-COPY%% %~~vis
%%AT-COPY%% \begin{xten}
%%AT-COPY%%   public static def main(argv:Rail[String]) {
%%AT-COPY%%     val argref = GlobalRef[Rail[String]](argv);
%%AT-COPY%%     at(here.next(); argref) 
%%AT-COPY%%         use(argref);
%%AT-COPY%%   }
%%AT-COPY%%   static def use(argref : GlobalRef[Rail[String]]) {
%%AT-COPY%%     at(argref.home; argref) {
%%AT-COPY%%       val argv = argref();
%%AT-COPY%%       argv(0) = "Hi!";
%%AT-COPY%%     }
%%AT-COPY%%   }
%%AT-COPY%% \end{xten}
%%AT-COPY%% %~~siv
%%AT-COPY%% %} 
%%AT-COPY%% % class Hook{ def run() { GR2.main(["what, me weasel?" as String]); return true; }}
%%AT-COPY%% %~~neg
%%AT-COPY%% 

%~~gen ^^^ Places60
% package Places.Atsome.Globref2;
% class GR2 {
% 
%~~vis
\begin{xten}
  public static def main(argv:Rail[String]) {
    val argref = GlobalRef[Rail[String]](argv);
    at(here.next()) 
        use(argref);
  }
  static def use(argref : GlobalRef[Rail[String]]) {
    at(argref) {
      val argv = argref();
      argv(0) = "Hi!";
    }
  }
\end{xten}
%~~siv
%} 
% class Hook{ def run() { GR2.main(["what, me weasel?" as String]); return true; }}
%~~neg

\end{ex}

There is an implicit coercion from \xcd`GlobalRef[T]` to \xcd`Place`, so
\xcd`at(argref)S` goes to \xcd`argref.home`.  


\subsection{Warnings about \xcd`at`}
There are two dangers involved with \xcd`at`: 
\begin{itemize}
\item Careless use of \xcd`at` can result in copying and transmission
of very large data structures.  
%%AT-COPY%% This is particularly an issue with the blanket
%%AT-COPY%% \xcd`at` statement, \xcd`at(p)S`, where everything used in \xcd`S` is copied.  
In particular, it is very easy to capture
\xcd`this` -- a field reference will do it -- and accidentally copy everything
that \xcd`this` refers to, which can be very large.  A disciplined use of copy
specifiers to make explicit just what gets copied can ameliorate this issue.

\item As seen in the examples above, a local variable reference
  \xcd`x` may refer to different objects in different nested \xcd`at`
  scopes. The programmer must either ensure that a variable accessed
  across an \xcd`at` boundary has no mutable state or be prepared to
  reason about which copy gets modified.   A disciplined use of copy specifiers to give
  different names to variables can ameliorate this concern.
\end{itemize}


%%AT-COPY%% \section{{\tt athome}: Returning Values from {\tt at}-Blocks}
%%AT-COPY%% \label{sect:athome}
%%AT-COPY%% \index{athome}
%%AT-COPY%% 
%%AT-COPY%% The 
%%AT-COPY%% \xcd`at(p;F)S` 
%%AT-COPY%% construct renders external variables unavailable within
%%AT-COPY%% \xcd`S`.  However, it is often useful to transmit values back from \xcd`S`,
%%AT-COPY%% and store them in external variables. 
%%AT-COPY%% 
%%AT-COPY%% The \xcd`athome(V;F)S` construct provides
%%AT-COPY%% this ability.  \xcd`V` is a list of variables, which must all be defined at
%%AT-COPY%% the same place.  \xcd`athome(V;F)S` goes to the place where the variables are
%%AT-COPY%% defined, copying \xcd`F` as for \xcd`at(p;F)S`, and executes \xcd`S` ---
%%AT-COPY%% allowing reading, assignment and initialization of the listed variables in
%%AT-COPY%% \xcd`V`. 
%%AT-COPY%% 
%%AT-COPY%% \xcd`V`, the list of variables, may include one or more variables.  It is a
%%AT-COPY%% static error if X10 cannot determine that all the variables in the list are
%%AT-COPY%% defined at the same place.
%%AT-COPY%% 
%%AT-COPY%% 
%%AT-COPY%% 
%%AT-COPY%% 
%%AT-COPY%% \begin{ex}
%%AT-COPY%% \xcd`athome` allows returning multiple pieces of information from an
%%AT-COPY%% \xcd`at`-statement.  In the following example, we return two data: 
%%AT-COPY%% one as a \xcd`val` named \xcd`square`, and the other as an addition in to a
%%AT-COPY%% partially-computed polynomial named \xcd`poly`.  
%%AT-COPY%% %~~gen ^^^ Places5f9g
%%AT-COPY%% % package Places5f9g;
%%AT-COPY%% % % KNOWNFAIL-at
%%AT-COPY%% % class Example { 
%%AT-COPY%% %~~vis
%%AT-COPY%% \begin{xten}
%%AT-COPY%% static def example(a: Long, mathProc: Place) { 
%%AT-COPY%%   val square : Long;
%%AT-COPY%%   var poly : Long = 1 + a; // will be 1+a+a*a
%%AT-COPY%%   at(mathProc; a) {
%%AT-COPY%%     val sq = a*a; 
%%AT-COPY%%     athome(square, poly; sq) {
%%AT-COPY%%        square = sq;  // initialization
%%AT-COPY%%        poly += sq;   // read and update
%%AT-COPY%%     }
%%AT-COPY%%   return [square, poly];
%%AT-COPY%%   }
%%AT-COPY%% \end{xten}
%%AT-COPY%% %~~siv
%%AT-COPY%% %}}
%%AT-COPY%% % class Hook { def run() { 
%%AT-COPY%% %   val e = example(2, here);
%%AT-COPY%% %   assert e(0) == 4 && e(1) == 7;
%%AT-COPY%% %   return true;
%%AT-COPY%% % }} 
%%AT-COPY%% %~~neg
%%AT-COPY%% \end{ex}
%%AT-COPY%% 
%%AT-COPY%% The abbreviated forms 
%%AT-COPY%% \xcd`athome (*) S` and 
%%AT-COPY%% \xcd`athome S` 
%%AT-COPY%% allow a block of assignments without specifying the variables being assigned
%%AT-COPY%% to, which is convenient for a small set of assignments. 
%%AT-COPY%% They 
%%AT-COPY%% are both equivalent to \xcd`athome(V;F)S`,
%%AT-COPY%% where: 
%%AT-COPY%% \begin{itemize}
%%AT-COPY%% \item \xcd`V` is the list of all variables appearing on the left-hand side of
%%AT-COPY%%       an assignment or update statement in \xcd`S`, excluding those which
%%AT-COPY%%       appear inside the body of an \xcd`at` or \xcd`athome` statement in \xcd`S`;
%%AT-COPY%% \item \xcd`F` is the same as for \xcd`at(p)S` (\Sref{sect:copy-spec})
%%AT-COPY%% \end{itemize}
%%AT-COPY%% 
%%AT-COPY%% 
%%AT-COPY%% \begin{ex}
%%AT-COPY%% 
%%AT-COPY%% Much as the blanket \xcd`at` construct \xcd`at(p)S` is convenient for
%%AT-COPY%% executing a small code body at another place, the blanket \xcd`athome`
%%AT-COPY%% construct \xcd`athome(*) S` 
%%AT-COPY%% (which may be written as simply \xcd`athome S`)
%%AT-COPY%% is convenient for returning a result or two.   The
%%AT-COPY%% preceding example could have been written using blanket statements.
%%AT-COPY%% 
%%AT-COPY%% %~~gen ^^^ Places5f9gblanket
%%AT-COPY%% % package Places5f9gblanket;
%%AT-COPY%% % class Example { 
%%AT-COPY%% % KNOWNFAIL-at
%%AT-COPY%% %~~vis
%%AT-COPY%% \begin{xten}
%%AT-COPY%% static def example(a: Long, mathProc: Place) { 
%%AT-COPY%%   val square : Long;
%%AT-COPY%%   var poly : Long = 1 + a; // will be 1+a+a*a
%%AT-COPY%%   at(mathProc) {
%%AT-COPY%%     val sq = a*a; 
%%AT-COPY%%     athome {
%%AT-COPY%%        square = sq;  // initialization
%%AT-COPY%%        poly += sq;   // read and update
%%AT-COPY%%     }
%%AT-COPY%%   return [square, poly];
%%AT-COPY%%   }
%%AT-COPY%% \end{xten}
%%AT-COPY%% %~~siv
%%AT-COPY%% %}}
%%AT-COPY%% % class Hook { def run() { 
%%AT-COPY%% %   val e = example(2, here);
%%AT-COPY%% %   assert e(0) == 4 && e(1) == 7;
%%AT-COPY%% %   return true;
%%AT-COPY%% % }} 
%%AT-COPY%% %~~neg
%%AT-COPY%% \end{ex}
%%AT-COPY%% 
%%AT-COPY%% {\bf Design:} It is not fundamentally essential to distinguish \xcd`at` from
%%AT-COPY%% \xcd`athome`.  \xcd`at(p;F)S` could allow writing to variables whose homes are
%%AT-COPY%% known at compile-time to be equal to \xcd`p`.  Indeed, in earlier versions of
%%AT-COPY%% X10, it did so.    This required an idiom in which programmers had to manage
%%AT-COPY%% the home locations of variables directly, and keep track of which home
%%AT-COPY%% location corresponded to which variable.  The \xcd`athome` construct makes
%%AT-COPY%% this idiom more convenient. 
	
\chapter{Activities}\label{XtenActivities}

An \Xten{} computation may have many concurrent {\em activities} ``in
flight'' at any give time. We use the term activity to denote a
dynamic execution instance of a piece of code (with references to
data). An activity is intended to execute in parallel with other
activities. An activity may be thought of as a very light-weight
thread.  In \XtenCurrVer{}, an activity may not be interrupted,
suspended or resumed as the result of actions taken by any other
activity.

An activity is spawned in a given place and stays in that place for
its lifetime.  An activity may be {\em running}, {\em blocked} on some
condition or {\em terminated}. When the statement associated with an
activity terminates normally, the activity terminates normally; when
it terminates abruptly with some reason $R$, the activity terminates
with the same reason (\Sref{ExceptionModel}).

An activity may be long-running and may invoke recursive methods (thus
may have a stack associated with it). On the other hand, an activity
may be short-running, involving a fine-grained operation such as a
single read or write.

% An activity may have an {\em activitylocal} heap accessible only
%to the activity. 

An activity may asynchronously and in parallel launch activities at
other places.

\Xten{} distinguishes between {\em local} termination and {\em global}
termination of a statement. The execution of a statement by an
activity is said to terminate locally when the activity has finished
all its computation related to that statement. (For instance the
creation of an asynchronous activity terminates locally when the
activity has been created.)  It is said to terminate globally when it
has terminated locally and all activities that it may have spawned at
any place (if any) have, recursively, terminated globally.

An \Xten{} computation is initiated as a single activity from the
command line. This activity is the {\em root activity}\index{root
activity} for the entire computation. The entire computation
terminates when (and only when) this activity globally
terminates. Thus \Xten{} does not permit the creation of so called
``daemon threads''---threads that outlive the lifetime of the root
activity. We say that an \Xten{} computation is {\em rooted}
(\Sref{initial-computation}).

\futureext{ We may permit the initial activity to be a daemon activity
to permit reactive computations, such as webservers, that may not
terminate.}

\section{The \Xten{} rooted exception model}
\label{ExceptionModel}
\index{Exception!model}

The rooted nature of \Xten{} computations permits the definition of a
{\em rooted} exception model. In multi-threaded programming languages
there is a natural parent-child relationship between a thread and a
thread that it spawns. Typically the parent thread continues execution
in parallel with the child thread. Therefore the parent thread cannot
serve to catch any exceptions thrown by the child thread. 

The presence of a root activity permits \Xten{} to adopt a different
model.  In any state of the computation, say that an activity $A$ is
{\em a root of} an activity $B$ if $A$ is an ancestor of $B$ and $A$
is suspended at a statement (such as the \xcd"finish" statement
\Sref{finish}) awaiting the termination of $B$ (and possibly other
activities). For every \Xten{} computation, the
\emph{root-of} relation
is guaranteed to be a tree. The root of the tree is the root activity
of the entire computation. If $A$ is the nearest root of $B$, the path
from $A$ to $B$ is called the {\em activation path} for the
activity.\footnote{Note that depending on the state of the computation
the activation path may traverse activities that are running,
suspended or terminated.}

We may now state the exception model for \Xten.  An uncaught exception
propagates up the activation path to its nearest root activity, where
it may be handled locally or propagated up the \emph{root-of} tree when
the activity terminates (based on the semantics of the statement being
executed by the activity).\footnote{In \XtenCurrVer{} the \xcd"finish"
statement is the only statement that marks its activity as a root
activity. Future versions of the language may introduce more such
statements.}  Thus, unlike concurrent languages such as \java{}, no
exception is ``thrown on the floor''.

\section{Spawning an activity}\label{AsynchronousActivity}\label{AsyncActivity}

Asynchronous activities serve as a single abstraction for supporting a
wide range of concurrency constructs such as message passing, threads,
DMA, streaming, data prefetching. (In general, asynchronous operations
are better suited for supporting scalability than synchronous
operations.)

An activity is created by executing the statement:

\begin{grammar}
Statement \: AsyncStatement \\
AsyncStatement \: \xcd"async" PlaceExpressionSingleList\opt Statement \\
PlaceExpressionSingleList \: \xcd"(" PlaceExpression \xcd")" \\
PlaceExpression \: Expression 
\end{grammar} 

The place expression \xcd"e" is expected to be of type \xcd"Place",
e.g., \xcd"here" or \xcd"d(p)" for some
distribution \xcd"d" and point \xcd"p" (\Sref{XtenPlaces}).  
If not, the compiler replaces
\xcd"e" with \xcd"e.home" if
\xcd"e" is of type \xcd"x10.lang.Object". Otherwise the compiler reports a type error. 

Note specifically that the expression \xcd"a(i)" when used as a place
expression may evaluate to \xcd"a(i).home", which may not be
the same place as \xcd"a.dist(i)". The programmer must be 
careful to choose the right expression, appropriate for the statement.
Accesses to \xcd"a(i)" within \grammarrule{Statement} should typically be guarded 
by the place expression \xcd"a.dist(i)".

In many cases the compiler may infer the unique place at which the
statement is to be executed by an analysis of the types of the
variables occurring in the statement. (The place must be such that the
statement can be executed safely, without generating a
\xcd"BadPlaceException".) In such cases the programmer may omit the
place designator; the compiler will throw an error if it cannot
determine the unique designated place.\footnote{\XtenCurrVer{} does
not specify a particular algorithm; this will be fixed in future
versions.}

An activity $A$ executes the statement \xcd"async (P) S" by launching
a new activity $B$ at the designated place, to execute the specified
statement. The statement terminates locally as soon as $B$ is
launched.  The activation path for $B$ is that of $A$, augmented with
information about the line number at which $B$ was spawned.  $B$
terminates normally when $S$ terminates normally.  It terminates
abruptly if $S$ throws an (uncaught) exception. The exception is
propagated to $A$ if $A$ is a root activity (see \Sref{finish}),
otherwise through $A$ to $A$'s root activity. Note that while an
activity is running, exceptions thrown by activities it has already
generated may propagate through it up to its root activity.

Multiple activities launched by a single activity at another place are
not ordered in any way. They are added to the pool of activities at
the target place and will be executed in sequence or in parallel based
on the local scheduler's decisions. If the programmer wishes to
sequence their execution s/he must use \Xten{} constructs, such as
clocks and \xcd"finish" to obtain the desired effect.  Further, the
\Xten{} implementations are not required to have fair schedulers,
though every implementation should make a best faith effort to ensure
that every activity eventually gets a chance to make forward progress.

\begin{staticrule*}
The statement in the body of an \xcd"async" is subject to the
restriction that it must be acceptable as the body of a \xcd"void"
method for an anonymous inner class declared at that point in the code,
which throws no checked exceptions. As such, it may reference
variables in lexically enclosing scopes (including \xcd"clock"
variables, \Sref{XtenClocks}) provided that such variables are
(implicitly or explicitly) \xcd"val".
\end{staticrule*}

\section{Place changes}\label{AtStatement}

An activity may change place using the \xcd"at" statement or
\xcd"at" expression:

\begin{grammar}
Statement \: AtStatement \\
AtStatement \: \xcd"at" PlaceExpressionSingleList Statement \\
Expression \: AtExpression \\
AtExpression \: \xcd"at" PlaceExpressionSingleList ClosureBody 
\end{grammar}

The statement \xcd"at (p) S" executes the statement \xcd"S"
synchronously at place \xcd"p".
The expression \xcd"at (p) E" executes the statement \xcd"E"
synchronously at place \xcd"p", returning the result to the
originating place.

\section{Finish}\index{finish}\label{finish}
The statement \xcd"finish S" converts global termination to local
termination and introduces a root activity. 

\begin{grammar}
Statement \: FinishStatement \\
FinishStatement \: \xcd"finish" Statement 
\end{grammar}

An activity $A$ executes \xcd"finish S" by executing \xcd"S".  The
execution of \xcd"S" may spawn other asynchronous activities (here or
at other places).  Uncaught exceptions thrown or propagated by any
activity spawned by \xcd"S" are accumulated at \xcd"finish S".
\xcd"finish S" terminates locally when all activities spawned by
\xcd"S" terminate globally (either abruptly or normally). If \xcd"S"
terminates normally, then \xcd"finish S" terminates normally and $A$
continues execution with the next statement after \xcd"finish S".  If
\xcd"S" terminates abruptly, then \xcd"finish S" terminates abruptly
and throws a single exception, \Xcd{x10.lang.MultipleExceptions}
formed from the collection of exceptions accumulated at \xcd"finish S".

Thus a \xcd"finish S" statement serves as a collection point for
uncaught exceptions generated during the execution of \xcd"S".

Note that repeatedly \xcd"finish"ing a statement has no effect after
the first \xcd"finish": \xcd"finish finish S" is indistinguishable
from \xcd"finish S".

\paragraph{Interaction with clocks.}\label{sec:finish:clock-rule}
\xcd"finish S" interacts with clocks (\Sref{XtenClocks}). 

While executing \xcd"S", an activity must not spawn any \xcd"clocked"
asyncs. (Asyncs spawned during the execution of \xcd"S" may spawn
clocked asyncs.) A
\xcd"ClockUseException"\index{clock!ClockUseException} is thrown
if (and when) this condition is violated.

In \XtenCurrVer{} this condition is checked dynamically; future
versions of the language will introduce type qualifiers which permit
this condition to be checked statically.

\futureext{
The semantics of \xcd"finish S" is conjunctive; it terminates when all
the activities created during the execution of \xcd"S" (recursively)
terminate. In many situations (e.g., nondeterministic search) it is
natural to require a statement to terminate when any {\em one} of the
activities it has spawned succeeds. The other activities may then be
safely aborted. Future versions of the language may introduce a
\xcd"finishone S" construct to support such speculative or nondeterministic
computation.
}
%% Need an example here.

\section{Initial activity}\label{initial-computation}\index{initial activity}

An \Xten{} computation is initiated from the command line on the
presentation of a classname \xcd"C". The class must have a
\xcd"public static def main(a: array[String])" method, otherwise an
exception is thrown
and the computation terminates.  The single statement
\begin{xten}
finish async (Place.FIRST_PLACE) {
  C.main(s);
}
\end{xten} 
\noindent is executed where \xcd"s" is an array of strings created
from command line arguments. This single activity is the root activity
for the entire computation. (See \Sref{XtenPlaces} for a discussion of
places.)

%% Say something about configuration information? 

\section{Foreach statements}\index{\Xcd{foreach}}\label{foreach-section}


\begin{grammar}
Statement \: ForEachStatement \\
ForEachStatement \: 
      \xcd"foreach" \xcd"(" Formal \xcd"in" Expression \xcd")"
          Statement 
\end{grammar}


The \xcd"foreach" statement is a parallel version of the enhanced \xcd"for"
statement (\Sref{ForAllLoop}). \xcd`for(x in C)S` executes \xcd`S` {\em
  sequentially}, with everything happening \xcd`here`. \xcd`foreach(x in C)S`
executes \xcd`S` for each iteration of the loop {\em in parallel}, located at
\xcd`x.home`. It is thus equivalent to:
\begin{xten}
foreach (x in C)
  async at (x.home) S
\end{xten}

As a common and useful special case, \xcd`C` may be a \xcd`Dist` or an
\xcd`Array`.  For both of these, \xcd`foreach(x in C)S` is treated just like 
\xcd`foreach(x in C.region)S`.  \xcd`x` ranges over the \xcd`Point`s of the
region.  Each activity that \xcd`foreach` starts is located at \xcd`here` --
the same place that the \xcd`foreach` statement itself is executing.  (If you
want to start an activity at the place where the array element \xcd`C(p)` is
located, use \xcd`ateach` (\Sref{ateach-section}) instead of \xcd`foreach`.)

Exceptions thrown by \xcd`S`, like other exceptions in \xcd`async`s, are
propagated to the root activity of the \xcd`foreach`.  

%FOREACH%  An activity executes a \xcd"foreach" statement in a similar fashion
%FOREACH%  except that separate \xcd"async" activities are launched in parallel
%FOREACH%  in the local place of each object returned by the iteration.
%FOREACH%  The statement
%FOREACH%  terminates locally when all the activities have been spawned. It never
%FOREACH%  throws an exception, though exceptions thrown by the spawned
%FOREACH%  activities are propagated through to the root activity.
%FOREACH%  
%FOREACH%  In a common case, the
%FOREACH%  the collection is intended to be of type
%FOREACH%  \xcd"Region" and the formal parameter is of type \xcd"Point".  Expressions \xcd"e" of type \xcd"Dist" and
%FOREACH%  \xcd"Array" are also accepted, and treated as if they were \xcd"e.region".





\section{Ateach statements}\index{\Xcd{ateach}}\label{ateach-section}

\begin{grammar}
Statement \: AtEachStatement \\
AtEachStatement \:
      \xcd"ateach" \xcd"(" Formal \xcd"in" Expression \xcd")"
         Statement 
\end{grammar}

The \xcd"ateach" statement is similar to the \xcd"foreach"
statement, but it spawns activites at each place of a distribution. 
In \xcd`ateach(p in D) S`, 
\xcd`D` must be either of type \xcd"Dist" or of type
\xcd`Array[T]`, 
and \xcd`p` will be of type \xcd"Point".

This statement differs from \xcd"foreach" only in
that each activity is spawned at the place specified by the
distribution for the point. That is, if \xcd`D` is a \xcd`Dist`, 
\xcd"ateach(p in D) S" could be implemented as:
\begin{xten}
foreach (p in D.region) 
  async (D(p)) S
\end{xten}

However, the compiler may implement it more efficiently to avoid extraneous
communications.  In particular, it is recommended that \xcd`ateach(p in D)S`
be implemented as the following code, which coordinates with each place of
\xcd`D` just once, rather than once per element of \xcd`D` at that place: 
\begin{xten}
foreach (p in D.places()) at (p) {
    foreach (pt in D|here) {
        S
    }
}
\end{xten}

If \xcd`e` is an \xcd`Array[T]`, then \xcd`ateach (p in e)S` is identical to
\xcd`ateach(p in e.dist)S`; the iteration is over the array's underlying
distribution.   \xcd`ateach(p in A)dealWith(A(p));` is a common and generally
efficient idiom for working with the elements of an array.



\section{Futures}\label{XtenFutures}

\Xten{} provides syntactic support for {\em asynchronous expressions}, also
known as futures:

\begin{grammar}
Primary \: FutureExpression \\
FutureExpression \:
  \xcd"future" PlaceExpressionSingleList\opt ClosureBody
\end{grammar} 

Intuitively such an expression evaluates its body asynchronously at
the given place. The resulting value may be obtained from the future
returned by this expression, by using the \xcd"force" operation.

In more detail, in an expression \xcd"future (Q) e", the place
expression \xcd"Q" is treated as in an \xcd"async" statement. \xcd"e"
is an expression of some type \xcd"T". \xcd"e" may reference only
those variables in the enclosing lexical environment which are
declared to be \xcd"val".

If the type of \xcd"e" is \xcd"T" then the type of
\xcd"future (Q) e" is \xcd"Future[T]".  This 
type \xcd"Future[T]" is defined as if by:
\begin{xten}
package x10.lang;
public interface Future[T] implements () => T {
  global def forced(): Boolean;
  global def force(): T;
}
\end{xten}

Evaluation of \xcd"future (Q) e" terminates locally with the creation
of a value \xcd"f" of type \xcd"Future[T]".  This value may be
stored in objects, passed as arguments to methods, returned from
method invocation etc. 

At any point, the method \xcd"forced" may be invoked on \xcd"f". This
method returns without blocking, with the value \xcd"true" if the
asynchronous evaluation of \xcd"e" has terminated globally and with
the value \xcd"false" if it has not.

\xcd"Future[T]" is a subtype of the function type \xcd"() => T".
Invoking---\emph{forcing}---the future \xcd"f" blocks until the
asynchronous evaluation of \xcd"e" has terminated globally. If the
evaluation terminates successfully with value \xcd"v", then the method
invocation returns \xcd"v". If the evaluation terminates abruptly with
exception \xcd"z", then the method throws exception \xcd"z". Multiple
invocations of the function (by this or any other activity) do not
result in multiple evaluations of \xcd"e". The results of the first
evaluation are stored in the future \xcd"f" and used to respond to all
queries.


\begin{xten}
promise: Future[T] = future (a.dist(3)) a(3);
value: T = promise();
\end{xten}


\section{At expressions}

\begin{grammar}
Expression \: \xcd"at" \xcd"(" Expression \xcd")" Expression
\end{grammar}

An \Xcd{at} expression evaluates an expression synchronously at the
given place and returns its value. Note that expression evaluation may
spawn asynchronous activities. The \Xcd{at} expression will return
without waiting for those activities to terminate. That is, \Xcd{at}
does not have built-in \Xcd{finish} semantics.

\section{Shared variables}
\label{Shared}

{\bf Compiler Limitation: Shared variables are not currently implemented.}

A {\em shared local variable} is declared with the annotation \xcd"shared".
It can be accessed within any control construct in its scope, including
\Xcd{async}, \Xcd{at}, \Xcd{future} and closures.

Note that the lifetime of some of these constructs may outlast the
lifetime of the scope -- requiring the implementation to allocate them
outside the current stack frame.

\section{Atomic blocks}\label{AtomicBlocks}\index{atomic blocks}
Languages such as \java{} use low-level synchronization locks to allow
multiple interacting threads to coordinate the mutation of shared
data. \Xten{} eschews locks in favor of a very simple high-level
construct, the {\em atomic block}.

A programmer may use atomic blocks to guarantee that invariants of
shared data-structures are maintained even as they are being accessed
simultaneously by multiple activities running in the same place.

\subsection{Unconditional atomic blocks}
The simplest form of an atomic block is the {\em unconditional
atomic block}:

\begin{grammar}
Statement \: AtomicStatement \\
AtomicStatement \: \xcd"atomic"  Statement \\
MethodModifier \: \xcd"atomic" \\
\end{grammar}

For the sake of efficient implementation \XtenCurrVer{} requires
that the atomic block be {\em analyzable}, that is, the set of
locations that are read and written by the \grammarrule{BlockStatement} are
bounded and determined statically.\footnote{A static bound is a constant
that depends only on the program text, and is independent 
of any runtime parameters. }
The exact algorithm to be used by
the compiler to perform this analysis will be specified in future
versions of the language.
\tbd{}

Such a statement is executed by an activity as if in a single step
during which all other concurrent activities in the same place are
suspended. If execution of the statement may throw an exception, it is
the programmer's responsibility to wrap the atomic block within a
\xcd"try"/{\xcd"finally" clause and include undo code in the finally
clause. Thus the \xcd"atomic" statement only guarantees atomicity on
successful execution, not on a faulty execution.

%% A compiler is allowed to reorder two atomic blocks that have no
%%data-dependency between them, just as it may reorder any two
%%statements which have no data-dependencies between them. For the
%%purposes of data dependency analysis, an atomic block is deemed to
%%have read and written all data at a single program point, the
%%beginning of the atomic block.
%%%% I dont believe we need to say at some point in the atomic block.
%%
We allow methods of an object to be annotated with \xcd"atomic". Such
a method is taken to stand for a method whose body is wrapped within an
\xcd"atomic" statement.

Atomic blocks are closely related to non-blocking synchronization
constructs \cite{herlihy91waitfree}, and can be used to implement 
non-blocking concurrent algorithms.

\begin{staticrule*}
In \xcd"atomic S", \xcd"S" may include method calls,
conditionals, etc.

It may {\em not} include an \xcd"async" activity (such as creation
of a \Xcd{future}).

It may {\em not} include any statement that may potentially block at
runtime (e.g., \xcd"when", \xcd"force" operations, \xcd"next"
operations on clocks, \xcd"finish"). 

All locations accessed in an atomic block must statically satisfy the
{\em locality condition}: they must belong to the place of the current
activity.\index{locality condition}\label{LocalityCondition} 

\end{staticrule*}


The compiler checks for this condition by checking whether the statement
could be the body of a \xcd"void" method annotated with \xcd"safe" at
that point in the code (\Sref{SafeAnnotation}).

\paragraph{Consequences.}
Note an important property of an (unconditional) atomic block:

\begin{eqnarray}
 \mbox{\xcd"atomic \{s1; atomic s2\}"} &=& \mbox{\xcd"atomic \{s1; s2\}"}
\end{eqnarray}

Further, an atomic block will eventually terminate successfully or
thrown an exception; it may not introduce a deadlock.

\subsubsection{Example}

The following class method implements a (generic) compare and swap (CAS) operation:

\begin{xten}
// target defined in lexically enclosing environment.
public atomic def CAS(old: Object, new: Object): Boolean {
   if (target.equals(old)) {
     target = new;
     return true;
   }
   return false;
}
\end{xten}

\subsection{Conditional atomic blocks}

Conditional atomic blocks are of the form:

\begin{grammar}
Statement \:  WhenStatement \\
WhenStatement \:  \xcd"when" \xcd"(" Expression \xcd")" Statement \\
            \| WhenStatement \xcd"or" \xcd"(" Expression \xcd")" Statement 
\end{grammar}

In such a statement the one or more expressions are called {\em
guards} and must be \xcd"Boolean" expressions. The statements are the
corresponding {\em guarded statements}. The first pair of expression
and statement is called the {\em main clause} and the additional pairs
are called {\em auxiliary clauses}. A statement must have a main
clause and may have no auxiliary clauses.

An activity executing such a statement suspends until such time as any
one of the guards is true in the current state. In that state, the
statement corresponding to the first guard that is true is executed.
The checking of the guards and the execution of the corresponding
guarded statement is done atomically. 


\Xten{} does not guarantee that a conditional atomic block
will execute if its condition holds only intermmittently. For, based on
the vagaries of the scheduler, the precise instant at which a
condition holds may be missed. Therefore the programmer is advised to
ensure that conditions being tested by conditional atomic blocks are
eventually stable, i.e., they will continue to hold until the block
executes (the action in the body of the block may cause the condition
to not hold any more).

%%Fourth, \Xten{} does not guarantees only {\em weak fairness} when executing
%%conditional atomic blocks. Let $c$ be the guard of some conditional
%%atomic block $A$. $A$ is required to make forward progress only if
%%$c$ is {\em eventually stable}. That is, any execution $s_1, s_2,
%%\ldots$ of the program is considered illegal only if there is a $j$
%%such that $c$ holds in all states $s_k$ for $k > j$ and in which $A$
%%does not execute. Specifically, if the system executes in such a way
%%that $c$ holds only intermmitently (that is, for some state in which
%%$c$ holds there is always a later state in which $c$ does not hold),
%%$A$ is not required to be executed (though it may be executed).

\begin{rationale}
The guarantee provided by \xcd"wait"/\xcd"notify" in \java{} is no
stronger. Indeed conditional atomic blocks may be thought of as a
replacement for \java's wait/notify functionality.
\end{rationale} 

We note two common abbreviations. The statement \xcd"when (true) S" is
behaviorally identical to \xcd"atomic S": it never suspends. Second,
\xcd"when (c) {;}" may be abbreviated to \xcd"await(c);"---it
simply indicates that the thread must await the occurrence of a
certain condition before proceeding.  Finally note that a \xcd"when"
statement with multiple branches is behaviorally identical to a
\xcd"when" statement with a single branch that checks the disjunction of
the condition of each branch, and whose body contains an
\xcd"if"/\xcd"then"/\xcd"else" checking each of the branch conditions.

\begin{staticrule*}
For the sake of efficient implementation certain restrictions are
placed on the guards and statements in a conditional atomic
block. 
\end{staticrule*}

Guards are required not to have side-effects, not to spawn
asynchronous activities and to have a statically determinable upper
bound on their execution. These conditions are expected to be checked
statically by the compiler.

The body of a \xcd"when" statement must satisfy the conditions
for the body of an \xcd"atomic" block.
%Second, as for unconditional atomic blocks, the set of memory
%locations accessed by a guarded statements are required to be bounded
%and statically analyzable.

Note that this implies that guarded statements are required to be {\em
flat}, that is, they may not contain conditional atomic blocks. (The
implementation of nested conditional atomic blocks may require
sophisticated operational techniques such as rollbacks.)

\paragraph{Sample usage.} 
There are many ways to ensure that a guard is eventually
stable. Typically the set of activities are divided into those that
may enable a condition and those that are blocked on the
condition. Then it is sufficient to require that the threads that may
enable a condition do not disable it once it is enabled. Instead the
condition may be disabled in a guarded statement guarded by the
condition. This will ensure forward progress, given the weak-fairness
guarantee.

\begin{example}
The following class shows how to implement a bounded buffer of size
$1$ in \Xten{} for repeated communication between a sender and a
receiver.

\begin{xten}
class OneBuffer {
  datum: Object = null;
  filled: Boolean = false;
  public def send(v: Object) {
    when (!filled) {
      this.datum = v;
      this.filled = true;
    }
  }
  public def receive(): Object {
    when (filled) {
      v: Object = datum;
      datum = null;
      filled = false;
      return v;
    }
  }
}
\end{xten}
\end{example}

\eat{
\paragraph{Implementing a future with a latch.}\label{future-imp}
The following class shows how to implement a {\em latch}. A latch is
an object that is initially created in a state called the {\em
unlatched} state. During its lifetime it may transition once to a {\em
forced} state. Once forced, it stays forced for its lifetime. The
latch may be queried to determine if it is forced, and if so, an
associated value may be retrieved. Below, we will consider a latch set
when some activity invokes a \xcd"setValue" method on it. This method
provides two values, a normal value and an exceptional value. The
method \xcd"force" blocks until the latch is set. If an exceptional
value was specified when the latch was set, that value is thrown on
any attempt to read the latch. Otherwise the normal value is returned.

\begin{xten}
public interface Future[T] {
   def forced(): Boolean;
   def apply(): T;
}
public class Latch implements Future {
  protected var forced: Boolean = false;
  protected var result: Box[T] = null;
  protected var z: Box[Exception] = null;

  public atomic def setValue(val: T): Boolean {
    return setValue(val, null);
  }
  public atomic def setValue(z: Exception): Boolean {
    return setValue(null, z);
  }
  public atomic def setValue(val: T,
                             z: Exception): Boolean {
    if (forced) return false;
    // these assignment happens only once.
    this.result = val;
    this.z = z;
    this.forced = true;
    return true;
  }
  public atomic def forced(): Boolean {
    return forced;
  }
  public def apply(): T {
    when (forced) {
      if (z != null) throw z;
      return result to T;
    }
  }
}
\end{xten}

Latches, \xcd"aync" operations and \xcd"finish" operations may be used
to implement futures as follows. The expression \xcd"future(P) e"
can be translated to:
\begin{xten}
(() => {
    L: Latch = new Latch();
    async (P) {
      X: Object;
      try {
        finish X = e;
        async (L) {
          L.setValue(X); 
        }
      }
      catch (Z: Exception) {
        async (L) {
          L.setValue(Z);
        }
      }
    }
    return L;
  })()
\end{xten}

Here we assume that \xcd"RunnableLatch" is an interface defined by:
\begin{xten} 
public interface RunnableLatch {
  def run(): Latch;
}
\end{xten}

We use the standard \java{} idiom of wrapping the core translation
inside an inner class definition/method invocation pair (i.e.,
\xcd"new RunnableLatch() {....}.run()") so as to keep the resulting
expression completely self-contained, while executing statements
inside the evaluation of an expression.

Execution of a \xcd"future(P) e" causes a new latch to be created,
and an \xcd"async" activity spawned at \xcd"P". The activity attempts
to \xcd"finish" the assigned \xcd"x = e", where \xcd"x" is a local
variable.  This may cause new activities to be spawned, based on
\xcd"e". If the assignment terminates successfully, another activity is
spawned to invoke the \xcd"setValue" method on the latch.  Exceptions
thrown by these activities (if any) are accumulated at the \xcd"finish"
statement and thrown after global termination of all
activities spawned by \xcd"x=e". The exception will be caught by the 
\xcd"catch" clause and stored with the latch. 


\oldtodo{Conditional atomic blocks should be powerful enough to implement clocks as well.}

\paragraph{A future to execute a statement.}
Consider an expression \xcd"onFinish {S}". This should return
a \xcd"Boolean" latch which should be forced when \xcd"S" has terminated
globally. Unlike \xcd"finish S", the evaluation of \xcd"onFinish {S}"
should locally terminate immediately, returning a latch. The
latch may be passed around in method invocations and stored in
objects. An activity may perform \xcd"force"/\xcd"forced" method
invocations on the latch whenever it desires to determine whether \xcd"S"
has terminated.

Such an expression can be written as:
\begin{xten}
(=> {
    L: Latch = new Latch();
    async (here) {
      try {
        finish S;
        L.setValue(true);
      }
      catch (Z: Exception) {
        L.setValue(Z);
      }
    }
    return L;
  }
)()
\end{xten}
}
	
\chapter{Clocks}\label{XtenClocks}\index{clocks}

The standard library for \Xten{}, \xcd"x10.lang" defines a
final class", \xcd"Clock" intended for repeated quiescence detection
of arbitrary, data-dependent collection of activities. Clocks are a
generalization of {\em barriers}. They permit dynamically created
activities to register and deregister. An activity may be registered
with multiple clocks at the same time. In particular, nested clocks
are permitted: an activity may create a nested clock and within one
phase of the outer clock schedule activities to run to completion on
the nested clock.  Nevertheless the design of clocks ensures that
deadlock cannot be introduced by using clock operations, and that
clock operations do not introduce any races.

This chapter describes the syntax and semantics of clocks and
statements in the language that have parameters of type \xcd"Clock". 

The key invariants associated with clocks are as follows.  At any
stage of the computation, a clock has zero or more {\em registered}
activities. An activity may perform operations only on those clocks it
is registered with (these clocks constitute its {\em clock set}).  An
activity is registered with one or more clocks when it is created.
During its lifetime the only additional clocks it is registered with
are exactly those that it creates. In particular it is not possible
for an activity to register itself with a clock it discovers by
reading a data-structure.

An activity may perform the following operations on a clock \xcd"c".
It may {\em unregister} with \xcd"c" by executing \xcd"c.drop();".
After this, it may perform no further actions on \xcd"c"
for its lifetime. It may {\em check} to see if it is unregistered on a
clock. It may {\em register} a newly forked activity with \xcd"c".
%% It may {\em post} a statement \xcd"S" for completion in the current phase
%% of \xcd"c" by executing the statement \xcd"now(c) S". 
Once registered and "active" (see below), it may also perform the following operations.
It may {\em resume} the clock by executing \xcd"c.resume();". This
indicates to \xcd"c" that it has finished posting all statements it
wishes to perform in the current phase. Finally, it may {\em block}
(by executing \xcd"next;") on all the clocks that it is registered
with. (This operation implicitly \xcd"resume"'s all clocks for the
activity.) It will resume from this statement only when all these
clocks are ready to advance to the next phase.

A clock becomes ready to advance to the next phase when every activity
registered with the clock has executed at least one \xcd"resume"
operation on that clock and all statements posted for completion in
the current phase have been completed.

Though clocks introduce a blocking statement (\xcd"next") an important
property of \Xten{} is that clocks cannot introduce deadlocks. That
is, the system cannot reach a quiescent state (in which no activity is
progressing) from which it is unable to progress. For, before blocking
each activity resumes all clocks it is registered with. Thus if a
configuration were to be stuck (that is, no activity can progress) all
clocks will have been resumed. But this implies that all activities
blocked on \xcd"next" may continue and the configuration is not stuck.
The only other possibility is that an activity may be stuck on
\xcd"finish". But the interaction rule between \xcd"finish" and clocks
(\Sref{sec:finish:clock-rule}) guarantees that this cannot cause a cycle
in the wait-for graph. A more rigorous proof may be found in \cite{X10-concur05}.

\section{Clock operations}\label{sec:clock}
The special statements introduced for clock operations are listed below.
%%479 NowStatement ::= 
%%      now ( Clock ) Statement

\begin{grammar}
Statement \: ClockedStatement \\
ClockedStatement \: \xcd"clocked" \xcd"(" ClockList \xcd")" Statement \\
NextStatement \: \xcd"next" \xcd";" \\
\end{grammar}

Note that \xcd"x10.lang.Clock" provides several useful methods on
clocks (e.g. \xcd"drop").

\subsection{Creating new clocks}\index{clock!creation}\label{sec:clock:create}

Clocks are created using a factory method on \xcd"x10.lang.Clock":

\begin{xten}
timeSynchronizer: Clock = Clock.make();
\end{xten}

\eat{All clocked variables are implicitly final. The initializer for a
local variable declaration of type \xcd"Clock" must be a new clock
expression. Thus \Xten{} does not permit aliasing of clocks.
Clocks are created in the place global heap and hence outlive the
lifetime of the creating activity.  Clocks are structs, hence may be freely
copied from place to 
place. (Clock instances typically contain references to mutable state
that maintains the current state of the clock.)
}
The current activity is automatically registered with the newly
created clock.  It may deregister using the \xcd"drop" method on
clocks (see the documentation of \xcd"x10.lang.Clock"). All activities
are automatically deregistered from all clocks they are registered
with on termination (normal or abrupt).

\subsection{Registering new activities on clocks}
\index{clock!clocked statements}\label{sec:clock:register}

The programmer may specify which clocks a new activity is to be
registered with using the \xcd"clocked" clause.

An activity may transmit only those clocks that is registered with and
has not quiesced on (\Sref{resume}). 
A \xcd"ClockUseException"\index{clock!ClockUseException} is
thrown if (and when) this condition is violated.

An activity may check that it is registered on a clock \xcd"c" by
executing:
\begin{xten}
c.registered()
\end{xten}
\noindent This call returns the \xcd"Boolean" value \xcd"true" iff the
activity is registered on \xcd"c"; otherwise it returns \xcd"false".

\begin{note}
\Xten{} does not contain a ``register'' statement that would allow an
activity to discover a clock in a datastructure and register itself on
it. Therefore, while clocks may be stored in a datastructure by one
activity and read from that by another, the new activity cannot
``use'' the clock unless it is already registered with it.
\end{note}

\oldtodo{Add text on arrays of clocks.}

\subsection{Resuming clocks}\index{clock!resume}\label{resume}\label{sec:clock:resume}
\Xten{} permits {\em split phase} clocks. An activity may wish
to indicate that it has completed whatever work it wishes to perform
in the current phase of a  clock \xcd"c" it is registered with, without
suspending all activity. It may do so  by executing the method
invocation:
\begin{xten}
c.resume();
\end{xten}

An activity may invoke this method only on a clock it is registered
with, and has not yet dropped (\Sref{sec:clock:drop}). A \xcd"ClockUseException"\index{clock!ClockUseException} is thrown if (and
when) this condition is violated.  Nothing happens if the activity has
already invoked a \xcd"resume" on this clock in the current phase.
Otherwise execution of this statement indicates that the activity will
not transmit \xcd"c" to an \xcd"async" (through a \xcd"clocked"
clause),
% or invoke \xcd"now" 
until it terminates, drops \xcd"c" or executes a \xcd"next". 

\begin{staticrule*}
The compiler should issue an error if any activity has a potentially
live execution path from a \xcd"resume" statement on a clock \xcd"c"
to a
%\xcd"now" or
async spawn statement (which registers the new
activity on \xcd"c") unless the path goes through a \xcd"next"
statement. (A \xcd"c.drop()" following a \xcd"c.resume()" is legal,
as is \xcd"c.resume()" following a \xcd"c.resume()".)
\end{staticrule*}

\subsection{Advancing clocks}\index{clock!next}\label{sec:clock:next}
An activity may execute the statement
\begin{xten}
next;
\end{xten}

\noindent 
Execution of this statement blocks until all the clocks that the
activity is registered with (if any) have advanced. (The activity
implicitly issues a \xcd"resume" on all clocks it is registered
with before suspending.)

An \Xten{} computation is said to be {\em quiescent} on a clock
\xcd"c" if each activity registered with \xcd"c" has resumed \xcd"c".
Note that once a computation is quiescent on \xcd"c", it will remain
quiescent on \xcd"c" forever (unless the system takes some action),
since no other activity can become registered with \xcd"c".  That is,
quiescence on a clock is a {\em stable property}.

Once the implementation has detected quiescence on \xcd"c", the system
marks all activities registered with \xcd"c" as being able to progress
on \xcd"c". An activity blocked on \xcd"next" resumes execution once
it is marked for progress by all the clocks it is registered with.

\subsection{Dropping clocks}\index{clock!drop}\label{sec:clock:drop}
An activity may drop a clock by executing:
\begin{xten}
c.drop();
\end{xten}

\noindent{} The activity is no longer considered registered with this
clock.  A \xcd"ClockUseException" is thrown if the activity has
already dropped \xcd"c".


%\subsection{Posting statements on a clock}\index{clock!now}\label{sec:clock:now}
\Xten{} provides syntactic support for a common idiom. Often it may be
necessary for an activity $A$ to require that a certain set of
statements be executed to completion before a clock $c$ can move
forward, without $A$ actually waiting for the completion
of the statement. We introduce the syntax:
\begin{x10}
461 Statement ::= NowStatement
471 StatementNoShortIf ::= 
       NowStatementNoShortIf
479 NowStatement ::= 
       now ( Clock ) Statement
489 NowStatementNoShortIf ::= 
       now ( Clock ) StatementNoShortIf
\end{x10}
\noindent 

A statement {\tt now (c) s} may be considered as shorthand for
\begin{x10}
  async clocked(c) \{ 
     finish async s; 
  \}
\end{x10}

\paragraph{Note.} Because of the static semantics of {\tt finish}
it is not possible to nest {\cf now} statements. Instead if it proves
useful, we may introduce a multi-clocked {\tt now} statement,
which permits the statement to be posted on multiple clocks
simultaneously.
\begin{x10}
479' NowStatement ::= 
       now ( ClockList ) Statement
489' NowStatementNoShortIf ::= 
       now ( ClockList ) StatementNoShortIf  
\end{x10}


\section{Program equivalences}
From the discussion above it should be clear that the following
equivalences hold:

\begin{eqnarray}
 \mbox{\xcd"c.resume(); next;"}       &=& \mbox{\xcd"next;"}\\
 \mbox{\xcd"c.resume(); d.resume();"} &=& \mbox{\xcd"d.resume(); c.resume();"}\\
 \mbox{\xcd"c.resume(); c.resume();"} &=& \mbox{\xcd"c.resume();"}
\end{eqnarray}

Note that \xcd"next; next;" is not the same as \xcd"next;". The
first will wait for clocks to advance twice, and the second
once.  

%\notinfouro{\subsection{Implementation Notes}
Clocks may be implemented efficiently with message passing, e.g.{} by
using short-circuit ideas in \cite{SaraswatPODC88}.  Recall that every
activity is spawned with references to a fixed number of clocks. Each
reference should be thought of as a global pointer to a location in
some place representing the clock. (We shall discuss a further
optimization below.) Each clock keeps two counters: the total number
of outstanding references to the clock, and the number of activities
that are currently suspended on the clock.

When an activity $A$ spawns another activity $B$ that will reference a
clock $c$ referenced by $A$, $A$ {\em splits} the reference by sending
a message to the clock. Whenever an activity drops a reference to a
clock, or suspends on it, it sends a message to the clock. 

The optimization is that the clock can be represented in a distributed
fashion. Each place keeps a local counter for each clock that is
referenced by an activity in that place. The global location for the
clock simply keeps track of the places that have references and that
are quiescent. This can reduce the inter-place message traffic
significantly.
}
%\notinfouro{\section{Clocked types}\index{types!clocked}

We allow types to specify clocks, via a {\cf clocked(c)} modifier,
e.g.{}

\begin{x10}
  clocked(c) int r;
\end{x10}

This declaration asserts that {\cf r} is accessible
(readable/writable) only by those statements that are clocked on {\cf
c}. Thus propagation of the clock provides some control over the
``visibility'' of {\cf r}.

The declaration 

\begin{x10}
  clocked(c) final int l = r;
\end{x10}

\noindent asserts additionally that in each clock instant {\cf l} is final, 
i.e.{} the value of {\cf l} may be reset at the beginning of each phase
of {\tt c} but stays constant during the phase.

This statement terminates when the computation of {\tt r} has
terminated and the assignment has been performed.

\todo{Generalize the syntax so that aggregate variables can be clocked with an aggregate clock of the same shape.}

\subsection{Clocked assignment}\index{assignment!clocked}
We expect that most arrays containing application data will be
declared to be {\cf clocked final}. We support this very powerful type
declaration by providing a new statement:
{\footnotesize
\begin{verbatim}
  next(c) l = r; 
\end{verbatim}}


\noindent 
for a variable $l$ declared to be clocked on $c$. The statement
assigns $r$ to the {\em next} value of $l$. There may be multiple such
assignments before the clock advances. The last such assignment
specifies the value of the variable that will be visible after the
clock has advanced.  If the variable is {\cf clocked final} it is
guaranteed that {\em all} readers of the variable throughout this
phase will see the value $r$.

The expression {\tt r} is implicitly treated as {\tt now(c) r}. That
is, the clock {\tt c} will not advance until the computation of {\tt r} has
terminated.

}
%\notinfouro{%III. Applied constrained calculi. (3 pages)
%
%For each example below, formal static and dynamic semantics rules for
%new constructs extension over the core CFJ. Subject-reduction and
%type-soundness theorems. Proofs to be found in fuller version of
%paper.
%
%(a) arrays, region, distributions -- type safe implies no arrayoutofbounds
%exceptions, only ClassCastExceptions (when dynamic checks fail).
%
%Use Satish's conditional constraints example.
%-- emphasize what is new over DML. 
%
%(b) places, concurrency -- place types.
%
%(c) ownership types, alias control.
%
The following section presents example uses of constrained types
using several different
constraint systems.
%
\eat{
Many common object-oriented
idioms and
object-oriented type systems can be captured naturally using
constrained types: specifically we discuss types for places,
aliases,
ownership, arrays and clocks.  \ref{TODO}
}

\eat{
\ref{TODO}
Many of these constraint systems are more
expressive than the constraint system implemented in the current
\Xten{} compiler and have not (yet) been implemented.
}

\eat{
\ref{TODO}
In the following we will use the shorthand $\tt C(\bar{t}:c)$ for the
type $\tt C(:\bar{f}=\bar{t},c)$ where the declaration of the class
{\tt C} is $\tt \class\ C(\bar{\tt T}\ \bar{\tt f}:c)\ldots$  Also,
we abbreviate $\tt C(\bar{t}:\true)$ as $\tt C(\bar{t})$.
Finally, we use the shorthand $\tt T\;x=t;~c$ for the constraint
$\tt T\;x;~x=t;~c$.
}

\eat{
Finally, we
will also have need to use the shorthand
${\tt C}_1(\bar{t}_1:{\tt c}_1)\& \ldots {\tt C}_k(\bar{\tt t}_k:{\tt c}_k)$
for the type
${\tt C}_1(:\bar{\tt f}_1=\bar{\tt t}_1, \ldots,
            \bar{\tt f}_k=\bar{\tt t}_k,{\tt c}_1,\ldots,{\tt c}_k)$ 
provided that the ${\tt C}_i$ form a subtype chain
and the declared fields of ${\tt C}_i$ are ${\tt f}_i$.

Constraints naturally allow for partial specification
(e.g., inequalities) or incomplete specification (no constraint on a
variable) with the same simple syntax. In the example below,
the type of {\tt a} does not place any constraint on the second
dimension of {\tt a}, but this dimension can be used in other
types (e.g., the return type).
\begin{xten}
  class Matrix(int m, int n) {
    Matrix(m,a.n) mul(Matrix(:m==this.n) a) {...}
    ...
  }
\end{xten}

Constraints also naturally permit the expression of existential types:
\begin{xten}
  class List(int length) { 
    List(:length <= this.length) filter(Comparator k) {...} 
    ...
  }
\end{xten}
\noindent
Here, the length of the list returned by the \xcd{filter} method is 
unknown, but is bound by the length of the original list.
}

\if 0
\subsection{Presburger constraints: array bounds}

Xi and Pfenning proposed using dependent types for eliminating
array bounds checks~\cite{xi98array}.
\Xten{} does not (yet) support generic types, however XXX
%
In CFJ, an array of type \xcd{T[]} indexed by (signed) integers
can be modeled as a class with the following
signature:\footnote{For this example, we assume generics
are supported.}
\begin{xten}
interface Array<T>(int(:self >= 0) length) {
  T get(int(:0 <= self, self < this.length) i);
  void set(int(:0 <= self, self < this.length) i, T v);
}
\end{xten}

Bounds can be checked using a constraint system based on
Presburger arithmetic~\cite{omega}.  Constraint terms include
integer constraints, scalar multiplication, and addition;
constraints include inequalities:
\fi


\eat{
Some code that iterates over an array (sugaring {\tt get} and {\tt set}):
\begin{xten}
double dot(double[] x, double[] y
         : x.length == y.length) {
  double r = 0.; 
  for (int(:self >= 0, self < x.length)
       i = 0; i < x.length; i++) {
    r += x[i] * y[i];
  }
  return r;
}
\end{xten}
}

\eat{
Another one:
\begin{xten}
double[](:length = x.length) saxpy(double a, double[] x, double[] y : x.length = y.length) {
    double[](:length = x.length) result = new double[x.length];
    for (int(:self >= 0, self < x.length) i = 0; i < x.length; i++) {
        result[i] = a * x[i] + y[i];
    }
    return result;
}
\end{xten}
}

% \subsection{Presburger constraints: blocked LU factorization}

\subsection{Equality constraints}

The \Xten{} compiler includes a simple equality-based constraint
system, described in Section~\ref{sec:lang}.
Equalities constraints
are used throughout \Xten{} programs.  For example, to ensure
$n$-dimensional arrays are indexed only be $n$-dimensional
index points, the array access operation requires that the
array's \xcd{rank} property be equal to the index's \xcd{rank}.

Equality constraints specified in the X10 run-time library are used by the
compiler to generate efficient code.  For instance, an iteration over
the points in a region can be optimized to a set of nested loops
if the constraint on the region's type specifies that the region
is rectangular and of constant rank.


\eat{
\subsection{Equality constraints with disjunction: place types}

This example is due to Satish Chandra. We wish to specify a balanced
distributed tree with the property that its right child is always at
the same place as its parent, and once the left child is at the same
place then the entire subtree is at that place.  In
\Xten{}, every object has a field {\tt location} of type
{\tt place} that specifies the location at which the object is located.
%
The desired property may be specified thus:
\begin{xten}
class Tree(boolean localLeft) extends Object {
  Tree(:!this.localLeft || (location==here && self.localLeft)) left; 
  Tree(:location==here) right);
  ...
}
\end{xten}
The constraint on \xcd{left} states that if the \xcd{localLeft} property is
true for the current node, then the location of the \xcd{left} child must be
\xcd{here} and its \xcd{localLeft} property must be set.  This ensures,
recursively, that the entire left subtree will be located at the same place.
}

\subsection{Presburger constraints}

Presburger constraints are linear integer inequalities.
%A constraint solver plugin was implemented using a port to Java of the
%Omega library.~\cite{omega,scale}
%A separate implementation
%of a Presburger constraint solver was implemented using
%CVC3~\cite{cvc}. 
A Presburger constraint solver plugin was implemented using
CVC3~\cite{cvclite,cvc}.  The list example in
Figure~\ref{fig:list-example} type-checks using this constraint system.

Presburger constraints are particularly useful in a
high-performance computing setting where array operations are
pervasive.
Xi and Pfenning proposed using dependent types for eliminating
array bounds checks~\cite{xi98array}.  A Presburger constraint
system can be used to keep track of array dimensions and array
indices to ensure bounds violations do not occur.

\subsection{Set constraints: region-based arrays}

Rather than using Presburger constraints, 
\Xten{} takes another approach:
following ZPL~\cite{ZPL}, arrays in \Xten{}
are defined over
{\em regions},
sets of $n$-dimensional {\em index points}~\cite{gps06-arrays}.
For instance, the region \xcd{[0:200,}\xcd{1:100]} specifies a
collection of two-dimensional points \xcd{(i,j)} with \xcd{i}
ranging from \xcd{0} to \xcd{200} and \xcd{j} ranging from
\xcd{1} to \xcd{100}.

Regions and points were modeled in CVC3~\cite{cvc} to create a
constraint solver that ensures array bounds
violations do not occur:
an array access type-checks if the index point can be statically
determined to be in the region over which the array is defined.

Region constraints are subset constraints
written as calls to the \xcd{contains}
method of the \xcd{region} class.
The constraint solver does not actually evaluate the calls to
the \xcd{contains} method, rather it interprets these calls
symbolically
as subset constraints at compile time.

Constraints have the following syntax:

{
\small
\begin{tabular}{r@{\quad}rcl}
\\
  (Constraint)   &\xcd{c} &::=& \xcd{r.contains(r)} \bnf \dots \\
  (Region) &\xcd{r} &::=& \xcd{t} \bnf [${\tt b}_1$:${\tt d}_1$,\ldots,${\tt b}_k$:${\tt d}_k$]
           \\
           &        &  \bnf &
           \xcd{r | r} \bnf \xcd{r & r} \bnf \xcd{r - r}
           \\
           &        &  \bnf &
           \xcd{r + p} \bnf \xcd{r - p} \\
  (Point)  &\xcd{p} &::=& \xcd{t} \bnf $[{\tt b}_1,\ldots,{\tt b}_k]$ \\
(Integer)&\xcd{b},\xcd{d} &::=& \xcd{t} \bnf \xcd{n} \\
\\
\end{tabular}
}

\noindent
where \xcd{t} are constraint terms (properties and final variables)
and \xcd{n} are integer literals.

Regions used in constraints are either constraint terms \xcd{t},
region constants, unions (\xcd{|}), intersections (\xcd{&}),
or differences (\xcd{-}), or regions where each point is
offset by another point \xcd{p} using \xcd{+} or \xcd{-}.

% $\xcd{r}_1$\xcd{.contains(}$\xcd{r}_2$\xcd{)}.

\begin{figure}[t]
\footnotesize

\inputxten{sor.x10}

\caption{Successive over-relaxation with regions}
\label{fig:sor}
\end{figure}

For example, the code in Figure~\ref{fig:sor} performs a successive
over-relaxation~\cite{sor} of a matrix \tcd{G} with rank 2.
The function declares a region variable \tcd{outer} as an alias for
\tcd{G}'s region and a region variable \tcd{inner} to be 
the subset of \tcd{outer} that excludes the boundary points,
formed by intersecting the \tcd{outer} region with itself shifted up, down,
left, and right by one.
The function then declares two more regions \tcd{d0} and \tcd{d1},
where ${\tt d}_i$ is set of points ${\tt x}_i$ where
$({\tt x}_0, {\tt x}_1)$ is in \tcd{inner}.  The function
iterates multiple times over points \tcd{i} in \tcd{d0}.
The syntax \tcd{finish} \tcd{foreach} (line 22) tells the
compiler to execute each loop iteration in parallel and to wait
for all concurrent activities to terminate.
The inner loop (lines 24--28) iterates over a subregion of
\tcd{inner}.

The type checker establishes that the \tcd{region} property of
the point \tcd{ij} (line 24) is \tcd{inner}
\xcd{&}~\xcd{[i..i,d1min..d1max]}, and that this region is a
subset of \tcd{inner}, which is in turn a subset of \tcd{outer},
the region of the array \tcd{G}.
Thus, the accesses to the array in the loop body
do not violate the bounds of the array.

A key to making the program type-check is that the region
intersection that defines \tcd{inner} (lines 10--11)
is explicitly intersected with \tcd{outer} so that the 
constraint solver can determine that
the result is a subset of \tcd{outer}.


\eat{
\subsection{AVL trees and red--black trees}

AVL trees and red-black trees can be modeled so that the
data structure invariant is enforced statically.

\begin{xten}
class AVLTree(int(:self >= 0) height) {...}
class Leaf(Object key) extends AVLTree(0) {...}
class Node(Object key, AVLTree l, AVLTree r
         : int d=l.height-r.height; -1 <= d, d <= 1) 
    extends AVLTree(max(l.height,r.height)+1) {...}
\end{xten}

Red--black trees may be modeled similarly. Such trees have the
invariant that (a) all leaves are black, (b) each non-leaf node has
the same number of black nodes on every path to a leaf (the black
height), (c) the immediate children of every red node are black.
\begin{xten}
class RBTree(int blackHeight) {...}
class Leaf extends RBTree(0) { int value; ... }
class Node(boolean isBlack, 
           RBTree(:this.isBlack || isBlack) l, 
           RBTree(:this.isBlack || isBlack,
                   blackHeight=l.blackHeight) r)
    extends RBTree(l.blackHeight+1) { int value; ... }
\end{xten}
}

\eat{
\subsection{Self types and binary methods}

Self types~\cite{bsg95,bfp-ecoop97-match} can be implemented
using a {\tt klass} property on objects.  The {\tt klass}
property represents the run-time class of the object.
Self types can be used to solve the binary method problem \cite{bruce-binary}.

In the example below, the {\tt Set} interface has a {\tt union} method
whose argument must be of the same class as {\tt this}.
\noindent This enables the {\tt IntSet} class's {\tt union}
method to access the {\tt bits} field of its argument {\tt s}.
\begin{xten}
  interface Set(:Class klass) {
    Set(this.klass) union(Set(this.klass) s);
  }
  class IntSet(:Class klass) implements Set(klass) {
    long bits;

    IntSet(IntSet.class)() { property(IntSet.class); }

    IntSet(IntSet.class)(int(:0 <= self, self <= 63) i) {
      property(IntSet.class);
      bits = 1 << i; }

    Set(this.klass) union(Set(this.klass) s) {
      IntSet(this.klass) r = new IntSet(this.klass);
      r.bits = this.bits | s.bits;
      return r; }
  }
\end{xten}
\noindent
The key to ensuring that this code type-checks is the
\rn{T-constr}
rule.
With a constraint system ${\cal C}_{\mathsf{klass}}$ aware of
the {\tt klass} property, the rule 
\rn{T-var} is used to subsume an expression of type
${\tt Set(this.class)}$ to type ${\tt IntSet(this.class)}$
when {\tt this} is known to be an {\tt IntSet}:
{\footnotesize
\[
\from{\begin{array}{c}
{\tt IntSet}~{\tt this}, {\tt Set}({\tt this.klass})~{\tt s}
        \vdash {\tt Set}({\tt this.klass})~{\tt s} \\
{\tt IntSet}~{\tt this}, {\tt Set}({\tt this.klass})~{\tt s}
        \vdash_{{\cal C}_{\mathsf{klass}}} {\tt IntSet}({\tt this.klass})~{\tt s} \\
\end{array}}
\infer{
{\tt IntSet}~{\tt this}, {\tt Set}({\tt this.klass})~{\tt s}
        \vdash {\tt IntSet}({\tt this.klass})~{\tt s}}
\]}
}


\eat{
\subsection{Binary search}

An informal study by Jon Bentley~\cite{programming-pearls}
found that x\% of professional programmers attending in a course
could not correctly implement binary search.

Dependent types can help here by adding the invariants to the
index types.

\subsection{Quicksort}

\begin{xten}
int(:left <= self & self <= right)
partition(T[] array, int left, int right, int pivotIndex : left <= pivotIndex & pivotIndex <= right) {
     T pivotValue = array[pivotIndex];

     // Move pivot to end
     swap(array, pivotIndex, right);

     int(:left <= self & self <= right) storeIndex;
     storeIndex = left;
     for (int(:left <= self & self <= right-1) i = left; i < right; i++) {
         if (array[i] <= pivotValue) {
             swap(array, storeIndex, i);
             storeIndex++;
         }
     }

     // Move pivot to its final place
     swap(array, right, storeIndex)
     return storeIndex;
}

void swap(T[] array,
          int(:0 <= self & self < array.length i,
          int(:0 <= self & self < array.length j) {
    T tmp = array[i];
    array[i] = array[j];
    array[j] = tmp;
}

void quicksort(T[] array, int left, int right : left <= right) {
    if (left < right) {
         // select a pivot index
         int(:left <= self & self <= right) pivotIndex = (left + right) / 2;
         pivotNewIndex = partition(array, left, right, pivotIndex)
         quicksort(array, left, pivotNewIndex-1)
         quicksort(array, pivotNewIndex+1, right)
    }
}
\end{xten}
}


\newif\ifowner
\ownerfalse

\ifowner

\subsection{Ownership constraints}

\begin{figure}[t]
\inputxten{LO.x10}
\caption{Ownership types}
\label{fig:ownership}
\end{figure}

Using a simple extension of \Xten{}'s built-in equality
constraint system,
constrained types can also be used to encode a form of ownership
types~\cite{ownership-types,liskov-popl2003}.

Figure~\ref{fig:ownership} shows a fragment of a \xcd{List}
class with ownership types.
Each \xcd{Owned} object has an \xcd{owner} property.  Objects
also have properties that are used as owner parameters.
%
The \xcd{List} class has an \xcd{owner} property inherited from
\xcd{Owned} and also declares a \xcd{valOwner} property that is
instantiated with the owner of the values in the list, stored in
the \xcd{head} field of each element.  The \xcd{tail} of the
list is owned by the list object itself.

\Xten{}'s equality-based constraint system is sufficient for
tracking object ownership, however is does not capture all
properties of ownership type systems.
Ownership type systems enforce an ``owners as dominators''
property: the ownership relation forms a tree within the object
graph; a reference is not permitted to point directly to objects
with a different owner.
%
To encode this property, the owner of
the values \xcd{valOwner} must be contained within the owner
of the list itself; that is, \xcd{valOwner} must be \xcd{owner}
or \xcd{valOwner}'s owner must be contained in \xcd{owner}.
This is captured by the constraint \xcd{self.owns(owner)} on
\xcd{valOwner}.  Calls to the \xcd{owns} method in constraints
are interpreted by the ownership constraint solver as the
disjunction of conditions shown in the body of \xcd{owns}.
The object \xcd{world} is the root of the ownership tree; 
all objects are transitively owned by \xcd{world}.

For example, the type \xcd{List(:owner==world & valOwner == this)}
is invalid, because
its constraint is interpreted as
\xcd{owner == world & valOwner == this & this.owns(world)},
which is satisfiable only when \xcd{this == world} (which it is not).

An additional check is needed to ensure objects owned by
\xcd{this} are encapsulated.
The \xcd{tail()} method for instance, incorrectly leaks the
list's \xcd{tail} field.  To eliminate this case, the ownership
constraint system must additionally check that owner parameters
are bound only to 
\xcd{this}, \xcd{world}, or an owner property of \xcd{this}.
These conditions ensure that \xcd{tail()} can be called only on
\xcd{this} because its return type is otherwise not valid.
For instance, in the following code, the type of \xcd{ys} is
not valid because the \xcd{owner} property is bound to \xcd{xs}:
\begin{xten}
    final Owned o = ...;
    final List(:owner==o & valOwner==o) xs;
    List(:owner==xs & valOwner==o) ys = xs.tail();
\end{xten}

\fi

\if 0
\subsection{Disequalities: non-null types}

A constraint system that supports disequalities can be used to
enforce a non-null invariant on reference types.
A non-null type \xcd{C} can be written simply as \xcd{C(:self != null)}.
\fi

\eat{
\subsection{Clocked types}

Clocks are barriers that are adapted to a context where activities may be
dynamically created, and are designed so that all clock operations are
determinate.

For each arity $n$, we introduce a {\em Gentzen predicate}
${\tt clocked(\bar{t})}$. A $k$-ary Gentzen predicate $a$ satisfies the
property that $a(t_1,\ldots, t_k) \vdash a(s_1,\ldots,s_n)$ iff $k=n$
and $t_i=s_i$ for $i\leq k$.

Such a \xcd{clocked} atom is added to the context by an \xcd{clocked async}:
$$
\from{\Gamma, {\tt clocked(\bar{\tt v})} \vdash {\tt T}\ {\tt e}}
\infer{\Gamma \vdash {\tt T}\ {\tt async}\ {\tt clocked}(\bar{\tt v}) {\tt e}}
$$

A programmer can require that a method may be invoked only if the
invoking activity is registered on the clocks $\bar{\tt k}$ by adding
a \xcd{clocked} clause. The rule for method elaboration and method invocation then change:
$$
\begin{array}{l}
\from{ \bar{\tt T}\ \bar{\tt x}, {\tt C}\ \this, {\tt c},\clocked(\bar{\tt k}) \vdash {\tt S}\ {\tt e}, {\tt S} \subtype {\tt T} }   
\infer{\tt T\ m(\bar{\tt T}\,\bar{\tt x} : c) \clocked(\bar{\tt  k})\{\return\ e;\}\ \mbox{OK in}\ C} 
\\ \quad\\ 
\rname{T-Invk}%
\from{\begin{array}{l}
\Gamma \vdash {\tt T}_{0:n} \ {\tt e}_{0:n}  \\
\mtype({\tt  T}_0,{\tt  m},{\tt  z}_0)= \tt {\tt  Z}_{1:n}\ {\tt  z}_{1:n}:c,clocked(\bar{\tt  k}) \rightarrow {\tt  S} \\
\Gamma, {\tt  T}_{0:n}\ {\tt  z}_{0:n} \vdash {\tt  T}_{1:n} \subtype {\tt  Z}_{1:n}\\
\sigma(\Gamma, {\tt  T}_{0:n}\ {\tt  z}_{0:n}) \vdash_{\cal C} {\tt  c} \ \ \ 
\mbox {(${\tt  z}_{0:n}$ fresh)} \\
\Gamma \vdash \clocked(\bar{\tt  k})\\
\end{array}}
\infer{\Gamma \vdash ({\tt  T}_{0:n}\ {\tt  z}_{0:n}; S)\ {\tt  e}_0.{\tt  m({\tt  e}_{1:n})}}
\end{array}
$$
}

\eat{
\subsection{Capabilities}

Capabilities (from Radha and Vijay's paper on neighborhoods)
}

\eat{
\subsection{Activity-local objects}

Parallelism in \Xten{} is supported through lightweight asynchronous {\em
activities}, created by {\tt  async} statements.
It is often useful to restrict objects so that they are {\em local} to a
particular activity.
A local object may be accessed only by
the activity that created it or by an ancestor of that activity.
% it may be written only by the activity that created
% it or by a descendant of that activity.
Local objects are declared and created by qualifying their type
with {\tt  local}:
\begin{xten}
  local C o = new local C();
\end{xten}

To encode local objects in \Xten{}, we add an {\tt  activity}
property to objects:
\begin{xten}
  class Object(Activity activity) { ... }
\end{xten}
\noindent
where {\tt  Activity} has a possibly null {\tt  parent} property:
\begin{xten}
  class Activity(Activity parent) { ... }
\end{xten}
\noindent

To track the current activity ({\tt  z}), we augment typing judgments
as follows:
\[
  {\tt  z};~\Gamma \vdash {\tt  T}\ {\tt  e}
\]
\noindent where ${\tt  Activity}({{\tt  z}'})~{\tt  z} \in \Gamma$.
When the current activity is {\tt  z},
we encode the type {\tt  local C} as ${\tt  C}({\tt  z})$.

Spawning a new activity with an {\tt  async} statement
introduces a fresh activity ${\tt  z}'$:
\[
\from{
{\tt  z}';~\Gamma,~{\tt  Activity}({\tt  z})~{\tt  z'} \vdash {\tt  T}\ {\tt  e}\ \ \ 
\mbox{(${\tt  z}'$ fresh)}
}
\infer{
{\tt  z};~\Gamma \vdash {\tt  T}\ ({\tt  async}\ {\tt  e})
}
\]
The rule \rn{T-Field} is strengthened to require that reads 
only be performed on objects whose {\tt  activity} property is a
descendant of the current activity.
%\rname{T-Field-Local}%
\[
\from{
\begin{array}{ll}
{\tt  z};~\Gamma \vdash {\tt  T}_0\ {\tt  e} \\
\mathit{fields}({\tt  T}_0,{\tt  z}_0)= \bar{\tt  U}\ \bar{\tt  f}_i &
\mbox{(${\tt  z}_0$ fresh)} \\
{\tt  z};~\Gamma \vdash {\tt  T}_0 \subtype {\tt  C}(:{\tt  activity} = {\tt  z}') &
\Gamma \vdash {\tt  z}~\mathsf{spawns}~{\tt  z}'
\end{array}
}
\infer{{\tt  z};~\Gamma \vdash ({\tt  T}_0\ {\tt  z}_0; {\tt  z}_0.{\tt  f}_i=\self;{\tt  U}_i)\ {\tt  e.f}_i}
\]

%\Gamma \vdash {\tt  z}_0.{\tt  activity} = {\tt  z}' &

\noindent
where the $\mathsf{spawns}$ relation is defined as follows:
\[
\Gamma \vdash {\tt  z}~\mathsf{spawns}~{\tt  z}
\]
\[
\from{
\Gamma \vdash {\tt  z_1}~\mathsf{spawns}~{\tt  z_2} \ \ \ 
\Gamma \vdash {\tt  z_2}~\mathsf{spawns}~{\tt  z_3}}
\infer{\Gamma \vdash {\tt  z_1}~\mathsf{spawns}~{\tt  z_3}}
\]
\[
\from{\Gamma \vdash {\tt  z_2}.{\tt  parent} = {\tt  z_1}}
\infer{\Gamma \vdash {\tt  z_1}~\mathsf{spawns}~{\tt  z_2}}
\]

\eat{
local objects owned by activity that created it.

locals cannot be read by contained asyncs.

locals can be written by contained asyncs.

locals created by an activity are inherited by the parent when
the activity terminates.

\begin{xten}
C(:thread = current) x = ...;
finish foreach (...) {
  C(:thread = current) y = x; // no!
  x = y;
}
\end{xten}

// can read if thread prop is current, or an ancestor of current
// can write if thread prop is current or a child of current

e : C(:thread = x)
current owns x
fields(...) = Ti fi
-----------------------
e.fi : Ti

extensions:

1. add thread to context
2. strengthen T-field rule
}
}

\eat{
\subsection{Discussion}

\paragraph{Control-flow.}
Tricky to encode.  Need something like {\tt pc} label~\cite{jif}.

\paragraph{Type state.}
Type state depends on the mutable state of the 
objects.  Cannot do in this framework.

Dependent types are of use in annotations~\cite{ns07-x10anno}.
}
}

	
\chapter{Local and Distributed Arrays}\label{XtenArrays}\index{array}

\section{Overview}

Indexable memory is fundamental abstraction for a programming
language. X10 includes
\begin{itemize}
\item Rails -- intrinsic one dimensional arrays
\item Local multi-dimensional arrays; both simplearray and regionarray
\item Distributed multi-dimensional arrays; both simplearray and regionarray
\end{itemize}

\section{Rails}

WRITE ME

\section{Simple Arrays}

WRITE ME

\section{Region-based Arrays}

Classes in the \Xcd{x10.regionarray} package provide the most general and 
flexible array abstraction that support mapping arbitrary multi-dimensional
index spaces to data elements. Although they are significantly more
flexible than \Xcd{Rail}s or the classes of the \Xcd{x10.simplearray}
package, this flexibility does carry with it an expectation of lower
runtime performance. 

\Xcd{Array}s provide indexed access to data at a single \Xcd{Place}, {\em via}
\Xcd{Point}s---indices of any dimensionality. \Xcd{DistArray}s is similar, but
spreads the data across multiple \xcd`Place`s, {\em via} \Xcd{Dist}s.  

\subsection{Points}\label{point-syntax}
\index{point}
\index{point!syntax}

Both kinds of arrays are indexed by \xcd`Point`s, which are $n$-dimensional tuples of
integers.  The \xcd`rank`
property of a point gives its dimensionality.  Points can be constructed from
integers, or \xcd`Rail[Int] by the \xcd`Point.make` factory methods:
%~~gen ^^^ArraysPointsExample1
% package Arrays.Points.Example1;
% import x10.regionarray.*;
% class Example1 {
% def example1() {
%~~vis
\begin{xten}
val origin_1 : Point{rank==1} = Point.make(0);
val origin_2 : Point{rank==2} = Point.make(0,0);
val origin_5 : Point = Point.make(new Rail[Int](5));
\end{xten}
%~~siv
% } } 
%~~neg

There is an implicit conversion from \xcd`Rail[Int]` to 
\xcd`Point`, giving
a convenient syntax for constructing points: 

%~~gen ^^^ Arrays30
% package Arrays.Points.Example2;
% import x10.regionarray.*;
% class Example{
% def example() {
%~~vis
\begin{xten}
val p : Point = [1,2,3];
val q : Point{rank==5} = [1,2,3,4,5];
val r : Point(3) = [11,22,33];
\end{xten}
%~~siv
% } } 
%~~neg

The coordinates of a point are available by function application, or, if you
prefer, by subscripting; \xcd`p(i)` is the
\xcd`i`th coordinate of the point \xcd`p`. 
\xcdmath`Point($n$)` is a \Xcd{type}-defined shorthand  for 
\xcdmath`Point{rank==$n$}`.


\subsection{Regions}\label{XtenRegions}\index{region}
\index{region!syntax}

A {\em region} is a set of points of the same rank.  {}\Xten{}
provides a built-in class, \xcd`x10.regionarray.Region`, to allow the
creation of new regions and to perform operations on regions. 
Each region \xcd`R` has a property \xcd`R.rank`, giving the dimensionality of
all the points in it.

\begin{ex}
%~~gen ^^^ Arrays40
% package Arrays40;
% import x10.regionarray.*;
% class Example {
% static def example() {
%~~vis
\begin{xten}
val MAX_HEIGHT=20;
val Null = Region.makeUnit(); //Empty 0-dimensional region
val R1 = Region.make(1, 100); // Region 1..100
val R2 = Region.make(1..100);  // Region 1..100
val R3 = Region.make(0..99, -1..MAX_HEIGHT);
val R4 = Region.makeUpperTriangular(10);
val R5 = R4 && R3; // intersection of two regions
\end{xten}
%~~siv
% } } 
%~~neg

The \xcd`IntRange` value \xcd`1..100` can be used to construct
the one-dimensional \xcd`Region` consisting of the points
$\{$\xcdmath`[m]`, \dots, \xcdmath`[n]`$\}$
\xcd`Region` by using the \xcd`Region.make` factory method.  
\xcd`IntRange`s are useful in building up regions, especially rectangular regions.  
\end{ex}

By a special dispensation, the compiler knows that, if \xcd`r : Region(m)` and
\xcd`s : Region(n)`, then \xcd`r*s : Region(m+n)`.  (The X10 type system
ordinarily could not specify the sum; the best it could do 
would be \xcd`r*s : Region`, with the rank of the region unknown.)  This
feature allows more convenient use of arrays; in particular, one does not need
to keep track of ranks nearly so much.

Various built-in regions are provided through  factory
methods on \xcd`Region`.  
\begin{itemize}
%~~exp~~"~~"~~ n:Int ~~ import x10.regionarray.*; ^^^Arrays3s5h
\item \xcd"Region.makeEmpty(n)" returns an empty region of rank \xcd"n".
%~~exp~~"~~"~~ n:Int ~~ import x10.regionarray.*; ^^^Arrays3x4j
\item \xcd"Region.makeFull(n)" returns the region containing all points of
      rank \xcd"n".  
%~~exp~~"~~"~~ ~~ import x10.regionarray.*; ^^^Arrays7l3d
\item \xcd"Region.makeUnit()" returns the region of rank 0 containing the
      unique point of rank 0.  It is useful as the identity for Cartesian
      product of regions.
%~~exp~~"~~"~~ normal:Point, k:Int ~~ import x10.regionarray.*; ^^^Arrays3l7z
\item \xcd"Region.makeHalfspace(normal, k)",
      where \xcd`normal` is a \xcd`Point` and \xcd`k` an \xcd`Int`, 
      returns the unbounded
      half-space of rank \xcd"normal.rank", consisting of all points \xcd"p"
      satisfying the vector inequality \xcdmath`p$\cdot$normal $\le$ k`.
%~~exp~~"~~"~~ min:Rail[Long], max:Rail[Long] ~~ import x10.regionarray.*; ^^^Arrays3i3n
\item \xcd"Region.makeRectangular(min, max)", 
      where \xcd"min" and \xcd"max"
      are rank-1 length-\xcd`n` integer arrays, returns a
      \xcd"Region(n)" equal to: 
      \xcdmath`[min(0) .. max(0), $\ldots$, min(n-1)..max(n-1)]`.
%~~exp~~"~~"~~ size: int, a: int, b: int~~ import x10.regionarray.*; ^^^Arrays2f2y
\item \xcd"Region.makeBanded(size, a, b)" constructs the
      banded \xcd"Region(2)" of size \xcd"size", with \Xcd{a} bands above
      and \Xcd{b} bands below the diagonal.
%~~exp~~"~~"~~size:Int ~~ import x10.regionarray.*; ^^^Arrays5s3q
\item \xcd"Region.makeBanded(size)" constructs the banded \Xcd{Region(2)} with
      just the main diagonal.
%~~exp~~`~~`~~N:Int ~~ import x10.regionarray.*; ^^^Arrays5s3qtri
\item \xcd`Region.makeUpperTriangular(N)` returns a region corresponding
to the non-zero indices in an upper-triangular \xcd`N x N` matrix.
%~~exp~~`~~`~~N:Int ~~ import x10.regionarray.*; ^^^Arrays5s3qlowertri
\item \xcd`Region.makeLowerTriangular(N)` returns a region corresponding
to the non-zero indices in a lower-triangular \xcd`N x N` matrix.
\item 
  If \xcd`R` is a region, and \xcd`p` a Point of the same rank, then 
%~~exp~~`~~`~~R:Region, p:Point(R.rank) ~~ import x10.regionarray.*; ^^^ Arrays50
  \xcd`R+p` is \xcd`R` translated forwards by 
  \xcd`p` -- the region whose
%~~exp~~`~~`~~r:Point, p:Point(r.rank) ~~ import x10.regionarray.*; ^^^ Arrays60
  points are \xcd`r+p` 
  for each \xcd`r` in \xcd`R`.
\item 
  If \xcd`R` is a region, and \xcd`p` a Point of the same rank, then 
%~~exp~~`~~`~~R:Region, p:Point(R.rank) ~~ import x10.regionarray.*; ^^^ Arrays70
  \xcd`R-p` is \xcd`R` translated backwards by 
  \xcd`p` -- the region whose
%~~exp~~`~~`~~r:Point, p:Point(r.rank) ~~ import x10.regionarray.*; ^^^ Arrays80
  points are \xcd`r-p` 
  for each \xcd`r` in \xcd`R`.

\end{itemize}

All the points in a region are ordered canonically by the
lexicographic total order. Thus the points of the region \xcd`(1..2)*(1..2)`
are ordered as 
\begin{xten}
(1,1), (1,2), (2,1), (2,2)
\end{xten}
Sequential iteration statements such as \xcd`for` (\Sref{ForAllLoop})
iterate over the points in a region in the canonical order.

A region is said to be {\em rectangular}\index{region!convex} if it is of
the form \xcdmath`(T$_1$ * $\cdots$ * T$_k$)` for some set of intervals
\xcdmath`T$_i = $ l$_i$ .. h$_i$ `. 
In particular an \xcd`IntRange` turned into a \xcd`Region` is rectangular: 
%~~exp~~`~~`~~ ~~ import x10.regionarray.*; ^^^Arrays3x4z
\xcd`Region.make(1..10)`.
Such a
region satisfies the property that if two points $p_1$ and $p_3$ are
in the region, then so is every point $p_2$ between them (that is, it is {\em convex}). 
(Banded and triangular regions are not rectangular.)
The operation
%~~exp~~`~~`~~R:Region ~~ import x10.regionarray.*; ^^^ Arrays90
\xcd`R.boundingBox()` gives the smallest rectangular region containing
\xcd`R`.

\subsubsection{Operations on regions}
\index{region!operations}

Let \xcd`R` be a region. A {\em sub-region} is a subset of \xcd"R".
\index{region!sub-region}

Let \xcdmath`R1` and \xcdmath`R2` be two regions whose types establish that
they are of the same rank. Let \xcdmath`S` be another region; its rank is
irrelevant. 

\xcdmath`R1 && R2` is the intersection of \xcdmath`R1` and
\xcdmath`R2`, \viz, the region containing all points which are in both
\Xcd{R1} and \Xcd{R2}.  \index{region!intersection}
%~~exp~~`~~`~~ ~~ import x10.regionarray.*; ^^^ Arrays100
For example, \xcd`Region.make(1..10) && Region.make(2..20)` is \Xcd{2..10}.


\xcdmath`R1 * S` is the Cartesian product of \xcdmath`R1` and
\xcdmath`S`,  formed by pairing each point in \xcdmath`R1` with every  point in \xcdmath`S`.
\index{region!product}
%~~exp~~`~~`~~ ~~ import x10.regionarray.*; ^^^ Arrays110
Thus, \xcd`Region.make(1..2)*Region.make(3..4)*Region.make(5..6)`
is the region of rank \Xcd{3} containing the eight points with coordinates
\xcd`[1,3,5]`, \xcd`[1,3,6]`, \xcd`[1,4,5]`, \xcd`[1,4,6]`,
\xcd`[2,3,5]`, \xcd`[2,3,6]`, \xcd`[2,4,5]`, \xcd`[2,4,6]`.


For a region \xcdmath`R` and point \xcdmath`p` of the same rank,
%~~exp~~`~~`~~R:Region, p:Point{p.rank==R.rank} ~~ import x10.regionarray.*; ^^^ Arrays120
\xcd`R+p` 
and
%~~exp~~`~~`~~R:Region, p:Point{p.rank==R.rank} ~~ import x10.regionarray.*; ^^^ Arrays130
\xcd`R-p` 
represent the translation of the region
forward 
and backward 
by \xcdmath`p`. That is, \Xcd{R+p} is the set of points
\Xcd{p+q} for all \Xcd{q} in \Xcd{R}, and \Xcd{R-p} is the set of \Xcd{q-p}.

More \Xcd{Region} methods are described in the API documentation.

\subsection{Arrays}
\index{array}

Arrays are organized data, arranged so that it can be accessed by subscript.
An \xcd`Array[T]` \Xcd{A} has a \Xcd{Region} \Xcd{A.region}, telling which
\Xcd{Point}s are in \Xcd{A}.  For each point \Xcd{p} in \Xcd{A.region},
\Xcd{A(p)} is the datum of type \Xcd{T} associated with \Xcd{p}.  X10
implementations should 
attempt to store \xcd`Array`s efficiently, and to make array element accesses
quick---\eg, avoiding constructing \Xcd{Point}s when unnecessary.

This generalizes the concepts of arrays appearing in many other programming
languages.  A \Xcd{Point} may have any number of coordinates, so an
\xcd`Array` can have, in effect, any number of integer subscripts.  

\begin{ex}Indeed, it is possible to write code that works on \Xcd{Array}s regardless 
of dimension.  For example, to add one \Xcd{Array[Int]} \Xcd{src} into another
\Xcd{dest}, 
%~~gen ^^^ Arrays140
%package Arrays.Arrays.Arrays.Example;
%import x10.regionarray.*;
% class Example{
%~~vis
\begin{xten}
static def addInto(src: Array[Int], dest:Array[Int])
  {src.region == dest.region}
  = {
    for (p in src.region) 
       dest(p) += src(p);
  }
\end{xten}
%~~siv
%}
% class Hook{
%   def run() { 
%     val a = new Array[Int](3, [1,2,3]);
%     val b = new Array[Int](a.region, (p:Point(1)) => 10*a(p) );
%     Example.addInto(a, b);
%     return b(0) == 11 && b(1) == 22 && b(2) == 33;
% }}
%~~neg
\noindent
Since \Xcd{p} is a \Xcd{Point}, it can hold as many coordinates as are
necessary for the arrays \Xcd{src} and \Xcd{dest}.
\end{ex}

The basic operation on arrays is subscripting: if \Xcd{A} is an \Xcd{Array[T]}
and \Xcd{p} a point with the same rank as \xcd`A.region`, then
%~~exp~~`~~`~~A:Array[Int], p:Point{self.rank == A.region.rank} ~~ import x10.regionarray.*; ^^^ Arrays150
\xcd`A(p)`
is the value of type \Xcd{T} associated with point \Xcd{p}.
This is the same operation as function application
(\Sref{sect:FunctionApplication}); arrays implement function types, and can be
used as functions.

Array elements can be changed by assignment. If \Xcd{t:T}, 
%~~gen ^^^ Arrays160
%package Arrays.Arrays.Subscripting.Is.From.Mars;
%import x10.regionarray.*; 
%class Example{
%def example[T](A:Array[T], p: Point{rank == A.region.rank}, t:T){
%~~vis
\begin{xten}
A(p) = t;
\end{xten}
%~~siv
%} } 
%~~neg
modifies the value associated with \Xcd{p} to be \Xcd{t}, and leaves all other
values in \Xcd{A} unchanged.

An \Xcd{Array[T]} named \Xcd{a} has: 
\begin{itemize}
%~~exp~~`~~`~~a:Array[Int] ~~ import x10.regionarray.*; ^^^ Arrays170
\item \xcd`a.region`: the \Xcd{Region} upon which \Xcd{a} is defined.
%~~exp~~`~~`~~a:Array[Int] ~~ import x10.regionarray.*; ^^^ Arrays180
\item \xcd`a.size`: the number of elements in \Xcd{a}.
%~~exp~~`~~`~~a:Array[Int] ~~ import x10.regionarray.*; ^^^ Arrays190
\item \xcd`a.rank`, the rank of the points usable to subscript \Xcd{a}. 
      \xcd`a.rank` is a cached copy of 
      \Xcd{a.region.rank}.
\end{itemize}

\subsubsection{Array Constructors}
\index{array!constructor}

To construct an array whose elements all have the same value \Xcd{init}, call
\Xcd{new Array[T](R, init)}. 
For example, an array of a thousand \xcd`"oh!"`s can be made by:
%~~exp~~`~~`~~ ~~ import x10.regionarray.*; ^^^ Arrays200
\xcd`new Array[String](1000, "oh!")`.


To construct and initialize an array, call the two-argument constructor. 
\Xcd{new Array[T](R, f)} constructs an array of elements of type \Xcd{T} on
region \Xcd{R}, with \Xcd{a(p)} initialized to \Xcd{f(p)} for each point
\Xcd{p} in \Xcd{R}.  \Xcd{f} must be a function taking a point of rank
\Xcd{R.rank} to a value of type \Xcd{T}.  

\begin{ex}
One way to construct the array \xcd`[11, 22, 33]` is with an array constructor
%~~exp~~`~~`~~ ~~ import x10.regionarray.*; ^^^ Arrays210
\xcd`new Array[Int](3, (i:long)=>(11*i) as Int)`. 
To construct a multiplication table, call
%~~exp~~`~~`~~ ~~ import x10.regionarray.*; ^^^ Arrays220
\xcd`new Array[Long](Region.make(0..9, 0..9), (p:Point(2)) => p(0)*p(1))`.
\end{ex}

Other constructors are available; see the API documentation and
\Sref{sect:RailCtors}. 

\subsubsection{Array Operations}
\index{array!operations on}

The basic operation on \Xcd{Array}s is subscripting.  If \Xcd{a:Array[T]} and 
\xcd`p:Point{rank == a.rank}`, then \Xcd{a(p)} is the value of type \Xcd{T}
appearing at position \Xcd{p} in \Xcd{a}.    The syntax is identical to
function application, and, indeed, arrays may be used as functions.
\Xcd{a(p)} may be assigned to, as well, by the usual assignment syntax
%~~exp~~`~~`~~a:Array[Int], p:Point{rank == a.rank}, t:Int ~~ import x10.regionarray.*; ^^^ Arrays230
\xcd`a(p)=t`.
(This uses the application and setting syntactic sugar, as given in \Sref{set-and-apply}.)

Sometimes it is more convenient to subscript by integers.  Arrays of rank 1-4
can, in fact, be accessed by integers: 
%~~gen ^^^ Arrays240
%package Arrays240;
%import x10.regionarray.*;
%class Example{
%static def example(){
%~~vis
\begin{xten}
val A1 = new Array[Int](10, 0);
A1(4) = A1(4) + 1;
val A4 = new Array[Int](Region.make(1..2, 1..3, 1..4, 1..5), 0);
A4(2,3,4,5) = A4(1,1,1,1)+1;
\end{xten}
%~~siv
% assert A1(4) == 1 && A4(2,3,4,5) == 1;
%}}
% class Hook{ def run() {Example.example(); return true;}}
%~~neg



Iteration over an \Xcd{Array} is defined, and produces the \Xcd{Point}s of the
array's region.  If you want to use the values in the array, you have to
subscript it.  For example, you could take the logarithm of every element of an
\Xcd{Array[Double]} by: 
%~~gen ^^^ Arrays250
%package Arrays250;
%import x10.regionarray.*;
%class Example{
%static def example(a:Array[Double]) {
%~~vis
\begin{xten}
for (p in a) a(p) = Math.log(a(p));
\end{xten}
%~~siv
%}}
% class Hook{ def run() { val a = new Array[Double](2, [1.0,2.0]); Example.example(a); return a(0)==Math.log(1.0) && a(1)==Math.log(2.0); }}

%~~neg



\subsection{Distributions}\label{XtenDistributions}
\index{distribution}

Distributed arrays are spread across multiple \xcd`Place`s.  
A {\em distribution}, a mapping from a region to a set of places, 
describes where each element of a distributed array is kept.
Distributions are embodied by the class \Xcd{x10.regionarray.Dist} and its
subclasses. 
The {\em rank} of a distribution is the rank of the underlying region, and
thus the rank of every point that the distribution applies to.


\begin{ex}
%~~gen ^^^ Arrays260
%package Arrays.Dist_example_a;
%import x10.regionarray.*;
% class Example{
% def example() {
%~~vis
\begin{xten}
val R  <: Region = Region.make(1..100);
val D1 <: Dist = Dist.makeBlock(R);
val D2 <: Dist = Dist.makeConstant(R, here);
\end{xten}
%~~siv
% } } 
%~~neg

\xcd`D1` distributes the region \xcd`R` in blocks, with a set of consecutive
points at each place, as evenly as possible.  \xcd`D2` maps all the points in
\xcd`R` to \xcd`here`.  
\end{ex}

Let \xcd`D` be a distribution. 
%~~exp~~`~~`~~D:Dist ~~ import x10.regionarray.*; ^^^ Arrays270
\xcd`D.region` 
denotes the underlying
region. 
Given a point \xcd`p`, the expression
%~~exp~~`~~`~~ D:Dist, p:Point{p.rank == D.rank}~~ import x10.regionarray.*; ^^^ Arrays280
\xcd`D(p)` represents the application of \xcd`D` to \xcd`p`, that is,
the place that \xcd`p` is mapped to by \xcd`D`. The evaluation of the
expression \xcd`D(p)` throws an \xcd`ArrayIndexOutofBoundsException`
if \xcd`p` does not lie in the underlying region.


\subsubsection{{\tt PlaceGroup}s}

A \xcd`PlaceGroup` represents an ordered set of \xcd`Place`s.
\xcd`PlaceGroup`s exist for performance and scaleability: they are more
efficient, in certain critical places, than general collections of
\xcd`Place`. \xcd`PlaceGroup` implements \xcd`Sequence[Place]`, and thus
provides familiar operations -- \xcd`pg.size()` for the number of places,
\xcd`pg.iterator()` to iterate over them, etc.  

\xcd`PlaceGroup` is an abstract class.  The concrete class
\xcd`SparsePlaceGroup` is intended for a small group of places. 
%~~exp~~`~~`~~ somePlace:Place ~~ ^^^Arrays1j6q
\xcd`new SparsePlaceGroup(somePlace)` is a good \xcd`PlaceGroup` containing
one place.  
%~~exp~~`~~`~~ seqPlaces: Rail[Place] ~~ ^^^Arrays9g6f
\xcd`new SparsePlaceGroup(seqPlaces)`
constructs a sparse place group from a Rail of places.

\subsubsection{Operations returning distributions}
\index{distribution!operations}



Let \xcd`R` be a region, \xcd`Q` 
a \xcd`PlaceGroup`, and \xcd`P` a place.

\paragraph{Unique distribution} \index{distribution!unique}
%~~exp~~`~~`~~Q:PlaceGroup ~~ import x10.regionarray.*; ^^^ Arrays290
The distribution \xcd`Dist.makeUnique(Q)` is the unique distribution from the
region \xcd`Region.make(1..k)` to \xcd`Q` mapping each point \xcd`i` to
\xcd`pi`.


\paragraph{Constant distributions.} \index{distribution!constant}
%~~exp~~`~~`~~R:Region, P:Place ~~ import x10.regionarray.*; ^^^ Arrays300
The distribution \xcd`Dist.makeConstant(R,P)` maps every point in region
\xcd`R` to place \xcd`P`.  
%~~exp~~`~~`~~R:Region ~~ import x10.regionarray.*; ^^^Arrays9n5n
The special case \xcd`Dist.makeConstant(R)` maps every point in \xcd`R` to
\xcd`here`. 

\paragraph{Block distributions.}\index{distribution!block}
%~~exp~~`~~`~~R:Region ~~ import x10.regionarray.*; ^^^ Arrays320
The distribution \xcd`Dist.makeBlock(R)` distributes the elements of \xcd`R`,
in approximately-even blocks, over all the places available to the program. 
There are other \xcd`Dist.makeBlock` methods capable of controlling the
distribution and the set of places used; see the API documentation.


\paragraph{Domain Restriction.} \index{distribution!restriction!region}

If \xcd`D` is a distribution and \xcd`R` is a sub-region of {\cf
%~~exp~~`~~`~~D:Dist,R :Region{R.rank==D.rank} ~~ import x10.regionarray.*; ^^^ Arrays330
D.region}, then \xcd`D | R` represents the restriction of \xcd`D` to
\xcd`R`---that is, the distribution that takes each point \xcd`p` in \xcd`R`
to 
%~~exp~~`~~`~~D:Dist, p:Point{p.rank==D.rank} ~~ import x10.regionarray.*; ^^^ Arrays340
\xcd`D(p)`, 
but doesn't apply to any points but those in \xcd`R`.

\paragraph{Range Restriction.}\index{distribution!restriction!range}

If \xcd`D` is a distribution and \xcd`P` a place expression, the term
%~~exp~~`~~`~~ D:Dist, P:Place~~ import x10.regionarray.*; ^^^ Arrays350
\xcd`D | P` 
denotes the sub-distribution of \xcd`D` defined over all the
points in the region of \xcd`D` mapped to \xcd`P`.

Note that \xcd`D | here` does not necessarily contain adjacent points
in \xcd`D.region`. For instance, if \xcd`D` is a cyclic distribution,
\xcd`D | here` will typically contain points that differ by the number of
places. 
An implementation may find a
way to still represent them in contiguous memory, \eg, using an arithmetic
function to map from the region index to an index 
into the array.


\subsection{Distributed Arrays}
\index{array!distributed}
\index{distributed array}
\index{\Xcd{DistArray}}
\index{DistArray}

Distributed arrays, instances of \xcd`DistArray[T]`, are very much like
\xcd`Array`s, except that they distribute information among multiple
\xcd`Place`s according to a \xcd`Dist` value passed in as a constructor
argument.  

\begin{ex}The following code creates a distributed array holding
a thousand cells, each initialized to 0.0, distributed via a block
distribution over all places.
%~~gen ^^^ Arrays360
% package Arrays.Distarrays.basic.example;
% import x10.regionarray.*;
% class Example {
% def example() {
%~~vis
\begin{xten}
val R <: Region = Region.make(1..1000);
val D <: Dist = Dist.makeBlock(R);
val da <: DistArray[Float] 
       = DistArray.make[Float](D, (Point(1))=>0.0f);
\end{xten}
%~~siv
%}}
%~~neg
\end{ex}



\subsection{Distributed Array Construction}\label{ArrayInitializer}
\index{distributed array!creation}
\index{\Xcd{DistArray}!creation}
\index{DistArray!creation}

\xcd`DistArray`s are instantiated by invoking one of the \xcd`make` factory
methods of the \xcd`DistArray` class.
A \xcd`DistArray` creation 
must take either an \xcd`Int` as an argument or a \xcd`Dist`. In the first
case,  a distributed array is created over the distribution 
%~~exp~~`~~`~~N:Int ~~ import x10.regionarray.*; ^^^Arrays1s6g
\xcd`Dist.makeConstant(Region.make(0, N-1),here)`;
in the second over the given distribution. 

\begin{ex}A distributed array creation operation may also specify an initializer
function.
The function is applied in parallel
at all points in the domain of the distribution. The
construction operation terminates locally only when the \xcd`DistArray` has been
fully created and initialized (at all places in the range of the
distribution).

For instance:
%~~gen ^^^ Arrays370
% package Arrays.DistArray.Construction.FeralWolf;
% import x10.regionarray.*;
% class Example {
% def example() {
%~~vis
\begin{xten}
val ident = ([i]:Point(1)) => i;
val data : DistArray[Long]
    = DistArray.make[Long](Dist.makeConstant(Region.make(1, 9)), ident);
val blk = Dist.makeBlock(Region.make(1..9, 1..9));
val data2 : DistArray[Long]
    = DistArray.make[Long](blk, ([i,j]:Point(2)) => i*j);
\end{xten}
%~~siv
% }  }
%~~neg




{}\noindent 
The first declaration stores in \xcd`data` a reference to a mutable
distributed array with \xcd`9` elements each of which is located in the
same place as the array. The element at \Xcd{[i]} is initialized to its index
\xcd`i`. 

The second declaration stores in \xcd`data2` a reference to a mutable
two-dimensional distributed array, whose coordinates both range from 1 to
9, distributed in blocks over all \xcd`Place`s, 
initialized with \xcd`i*j`
at point \xcd`[i,j]`.
\end{ex}


\subsection{Operations on Arrays and Distributed Arrays}

Arrays and distributed arrays share many operations.
In the following, let \xcd`a` be an array with base type T, and \xcd`da` be an
array with distribution \xcd`D` and base type \xcd`T`.




\subsubsection{Element operations}\index{array!access}
The value of \xcd`a` at a point \xcd`p` in its region of definition is
%~~exp~~`~~`~~a:Array[Int](3), p:Point(3) ~~ import x10.regionarray.*; ^^^ Arrays380
obtained by using the indexing operation \xcd`a(p)`. 
The value of \xcd`da` at \xcd`p` is similarly
%~~exp~~`~~`~~da:DistArray[Int](3), p:Point(3) ~~ import x10.regionarray.*; ^^^ Arrays390
\xcd`da(p)`.
This operation
may be used on the left hand side of an assignment operation to update
the value: 
%~~stmt~~`~~`~~a:Array[Int](3), p:Point(3), t:Int ~~ import x10.regionarray.*; ^^^ Arrays400
\xcd`a(p)=t;`
and 
%~~stmt~~`~~`~~da:DistArray[Int](3), p:Point(3), t:Int ~~ import x10.regionarray.*; ^^^ Arrays410
\xcd`da(p)=t;`
The operator assignments, \xcd`a(i) += e` and so on,  are also
available. 

It is a runtime error to 
access arrays, with \xcd`da(p)` or \xcd`da(p)=v`, at a place
other than \xcd`da.dist(p)`, \viz{} at the place that the element exists. 


\subsubsection{Arrays of Single Values}\label{ConstantArray}
\index{array!constant promotion}

For a region \xcd`R` and a value \xcd`v` of type \xcd`T`, the expression 
%~~genexp~~`~~`~~T~~R:Region{self!=null}, v:T ~~ import x10.regionarray.*; ^^^ Arrays420
\xcd`new Array[T](R, v)` 
produces an array on region \xcd`R` initialized with value \xcd`v`.
Similarly, 
for a distribution \xcd`D` and a value \xcd`v` of
type \xcd`T` the expression 
\begin{xtenmath}
DistArray.make[T](D, (Point(D.rank))=>v)
\end{xtenmath}
constructs a distributed array with
distribution \xcd`D` and base type \xcd`T` initialized with \xcd`v`
at every point.

Note that \xcd`Array`s are constructed by constructor calls, but
\xcd`DistArrays` are constructed by calls to the factory methods
\xcd`DistArray.make`. This is because \xcd`Array`s are fairly simple objects,
but \xcd`DistArray`s may be implemented by different classes for different
distributions. The use of the factory method gives the library writer the
freedom to select appropriate implementations.


\subsubsection{Restriction of an array}\index{array!restriction}

Let \xcd`R` be a sub-region of \xcd`da.region`. Then 
%~~exp~~`~~`~~da:DistArray[Int](3), p:Point(3), R: Region(da.rank) ~~ import x10.regionarray.*; ^^^ Arrays440
\xcd`da | R`
represents the sub-\xcd`DistArray` of \xcd`da` on the region \xcd`R`.
That is, \xcd`da | R` has the same values as \xcd`da` when subscripted by a
%~~exp~~`~~`~~R:Region, da: DistArray[Int]{da.region.rank == R.rank} ~~ import x10.regionarray.*; ^^^ Arrays450
point in region \xcd`R && da.region`, and is undefined elsewhere.

Recall that a rich set of operators are available on distributions
(\Sref{XtenDistributions}) to obtain sub-distributions
(e.g. restricting to a sub-region, to a specific place etc).


\subsubsection{Operations on Whole Arrays}

\paragraph{Pointwise operations}\label{ArrayPointwise}\index{array!pointwise operations}
The unary \xcd`map` operation applies a function to each element of
a distributed or non-distributed array, returning a new distributed array with
the same distribution, or a non-distributed array with the same region.

The following produces an array of cubes: 
%~~gen ^^^ Arrays460
%package Arrays_pointwise_a;
%import x10.regionarray.*;
%class Example{
%static def example() {
%~~vis
\begin{xten}
val A = new Array[Int](11, (i:long)=>i as Int);
assert A(3) == 3 && A(4) == 4 && A(10) == 10; 
val cube = (i:Int) => i*i*i;
val B = A.map(cube);
assert B(3) == 27 && B(4) == 64 && B(10) == 1000; 
\end{xten}
%~~siv
%} } 
% class Hook{ def run() {Example.example(); return true;}}
%~~neg

A variant operation lets you specify the array \Xcd{B} into which the result
will be stored, 
%~~gen ^^^ Arrays470
%package Arrays.map_with_result;
%import x10.regionarray.*;
%class Example{
%static def example() {
%~~vis
\begin{xten}
val A = new Array[Int](11, (i:long)=>i as Int);
assert A(3) == 3 && A(4) == 4 && A(10) == 10; 
val cube = (i:Int) => i*i*i;
val B = new Array[Int](A.region); // B = 0,0,0,0,0,0,0,0,0,0,0
A.map(B, cube);
assert B(3) == 27 && B(4) == 64 && B(10) == 1000; 
\end{xten}
%~~siv
%} } 
% class Hook{ def run() {Example.example(); return true;}}
%~~neg
\noindent
This is convenient if you have an already-allocated array lying around unused.
In particular, it can be used if you don't need \Xcd{A} afterwards and want to
reuse its space:
%~~gen ^^^ Arrays480
%package Arrays.map_reusing_space;
%import x10.regionarray.*;
%class Example{
%static def example() {
%~~vis
\begin{xten}
val A = new Array[Int](11, (i:long)=>i as Int);
assert A(3) == 3 && A(4) == 4 && A(10) == 10; 
val cube = (i:Int) => i*i*i;
A.map(A, cube);
assert A(3) == 27 && A(4) == 64 && A(10) == 1000; 
\end{xten}
%~~siv
%} } 
% class Hook{ def run() {Example.example(); return true;}}
%~~neg


The binary \xcd`map` operation takes a binary function and
another
array over the same region or distributed array over the same  distribution,
and applies the function 
pointwise to corresponding elements of the two arrays, returning
a new array or distributed array of the same shape.
The following code adds two distributed arrays: 
%~~gen ^^^ Arrays490
% package Arrays.Pointwise.Pointless.Map2;
% import x10.regionarray.*;
% class Example{
%~~vis
\begin{xten}
static def add(da:DistArray[Int], db: DistArray[Int])
    {da.dist==db.dist}
    = da.map(db, (a:Int,b:Int)=>a+b);
\end{xten}
%~~siv
%}
%~~neg



\paragraph{Reductions}\label{ArrayReductions}\index{array!reductions}

Let \xcd`f` be a function of type \xcd`(T,T)=>T`.  Let
\xcd`a` be an array over base type \xcd`T`.
Let \xcd`unit` be a value of type \xcd`T`.
Then the
%~~genexp~~`~~`~~ T ~~ f:(T,T)=>T, a : Array[T], unit:T ~~ import x10.regionarray.*; ^^^ Arrays500
operation \xcd`a.reduce(f, unit)` returns a value of type \xcd`T` obtained
by combining all the elements of \xcd`a` by use of  \xcd`f` in some unspecified order
(perhaps in parallel).   
The following code gives one method which 
meets the definition of \Xcd{reduce},
having a running total \Xcd{r}, and accumulating each value \xcd`a(p)` into it
using \Xcd{f} in turn.  (This code is simply given as an example; \Xcd{Array}
has this operation defined already.)
%~~gen ^^^ Arrays510
%package Arrays.Reductions.And.Eliminations;
%import x10.regionarray.*;
% class Example {
%~~vis
\begin{xten}
def oneWayToReduce[T](a:Array[T], f:(T,T)=>T, unit:T):T {
  var r : T = unit;
  for(p in a.region) r = f(r, a(p));
  return r;
}
\end{xten}
%~~siv
%}
%~~neg


For example,  the following sums an array of integers.  \Xcd{f} is addition,
and \Xcd{unit} is zero.  
%~~gen ^^^ Arrays520
% package Arrays.Reductions.And.Emulsions;
%import x10.regionarray.*;
% class Example {
% static def example() {
%~~vis
\begin{xten}
val a = new Array[Int](4, (i:long)=>(i+1) as Int);
val sum = a.reduce((a:Int,b:Int)=>a+b, 0); 
assert(sum == 10); // 10 == 1+2+3+4
\end{xten}
%~~siv
%}}
% class Hook{ def run() {Example.example(); return true;}}
%~~neg

Other orders of evaluation, degrees of parallelism, and applications of
\Xcd{f(x,unit)} and \xcd`f(unit,x)`are also correct.
In order to guarantee that the result is precisely
determined, the  function \xcd`f` should be associative and
commutative, and the value \xcd`unit` should satisfy
\xcd`f(unit,x)` \xcd`==` \xcd`x` \xcd`==` \xcd`f(x,unit)`
for all \Xcd{x:T}.  




\xcd`DistArray`s have the same operation.
This operation involves communication between the places over which
the \xcd`DistArray` is distributed. The \Xten{} implementation guarantees that
only one value of type \xcd`T` is communicated from a place as part of
this reduction process.

\paragraph{Scans}\label{ArrayScans}\index{array!scans}


Let \xcd`f:(T,T)=>T`, \xcd`unit:T`, and \xcd`a` be an \xcd`Array[T]` or
\xcd`DistArray[T]`.  Then \xcd`a.scan(f,unit)` is the array or distributed
array of type \xcd`T` whose {$i$}th element in canonical order is the
reduction by \xcd`f` with unit \xcd`unit` of the first {$i$} elements of
\xcd`a`. 


This operation involves communication between the places over which the
distributed array is distributed. The \Xten{} implementation will endeavour to
minimize the communication between places to implement this operation.

Other operations on arrays, distributed arrays, and the related classes may be
found in the \xcd`x10.regionarray` package.
	
\chapter{Annotations}\label{XtenAnnotations}\index{annotations}


\Xten{} provides an 
an annotation system  system for to allow the
compiler to be extended with new static analyses and new
transformations.

Annotations are constraint-free interface types that decorate the abstract syntax tree
of an \Xten{} program.  The \Xten{} type-checker ensures that an annotation
is a legal interface type.
In \Xten{}, interfaces may declare
both methods and properties.  Therefore, like any interface type, an
annotation may instantiate
one or more of its interface's properties.
%%PLUGINNERY%%  Unlike with Java
%%PLUGINNERY%%  annotations,
%%PLUGINNERY%%  property initializers need not be
%%PLUGINNERY%%  compile-time constants;
%%PLUGINNERY%%  however, a given compiler plugin
%%PLUGINNERY%%  may do additional checks to constrain the allowable
%%PLUGINNERY%%  initializer expressions.
%%PLUGINNERY%%  The \Xten{} type-checker does not check that
%%PLUGINNERY%%  all properties of an annotation are initialized,
%%PLUGINNERY%%  although this could be enforced by
%%PLUGINNERY%%  a compiler plugin.

\section{Annotation syntax}

The annotation syntax consists of an ``\texttt{@}'' followed by an interface type.

%##(Annotations Annotation
\begin{bbgrammar}
%(FROM #(prod:Annotations)#)
         Annotations \: Annotation & (\ref{prod:Annotations}) \\
                    \| Annotations Annotation \\
%(FROM #(prod:Annotation)#)
          Annotation \: \xcd"@" NamedType & (\ref{prod:Annotation}) \\
\end{bbgrammar}
%##)

Annotations can be applied to most syntactic constructs in the language
including class declarations, constructors, methods, field declarations,
local variable declarations and formal parameters, statements,
expressions, and types.
Multiple occurrences of the same annotation (i.e., multiple
annotations with the same interface type) on the same entity are permitted.

%%OBSOLETE%% \begin{grammar}
%%OBSOLETE%% ClassModifier \: Annotation \\
%%OBSOLETE%% InterfaceModifier \: Annotation \\
%%OBSOLETE%% FieldModifier \: Annotation \\
%%OBSOLETE%% MethodModifier \: Annotation \\
%%OBSOLETE%% VariableModifier \: Annotation \\
%%OBSOLETE%% ConstructorModifier \: Annotation \\
%%OBSOLETE%% AbstractMethodModifier \: Annotation \\
%%OBSOLETE%% ConstantModifier \: Annotation \\
%%OBSOLETE%% Type \: AnnotatedType \\
%%OBSOLETE%% AnnotatedType \: Annotation\plus Type \\
%%OBSOLETE%% Statement \: AnnotatedStatement \\
%%OBSOLETE%% AnnotatedStatement \: Annotation\plus Statement \\
%%OBSOLETE%% Expression \: AnnotatedExpression \\
%%OBSOLETE%% AnnotatedExpression \: Annotation\plus Expression \\
%%OBSOLETE%% \end{grammar}

\noindent
Recall that interface types may have dependent parameters.

\noindent
The following examples illustrate the syntax:

\begin{itemize}
\item Declaration annotations:
\begin{xtennoindent}
  // class annotation
  @Value
  class Cons { ... }

  // method annotation
  @PreCondition(0 <= i && i < this.size)
  public def get(i: Int): Object { ... }

  // constructor annotation
  @Where(x != null)
  def this(x: T) { ... }

  // constructor return type annotation
  def this(x: T): C@Initialized { ... }

  // variable annotation
  @Unique x: A;
\end{xtennoindent}
\item Type annotations:
\begin{xtennoindent}
  List@Nonempty

  Int@Range(1,4)

  Array[Array[Double]]@Size(n * n)
\end{xtennoindent}
\item Expression annotations:
\begin{xtennoindent}
  m() : @RemoteCall
\end{xtennoindent}
\item Statement annotations:
\begin{xtennoindent}
  @Atomic { ... }

  @MinIterations(0)
  @MaxIterations(n)
  for (var i: Int = 0; i < n; i++) { ... }

  // An annotated empty statement ;
  @Assert(x < y);
\end{xtennoindent}
\end{itemize}

\section{Annotation declarations}

Annotations are declared as interfaces.  They must be
subtypes of the interface \texttt{x10.lang.annotation.Annotation}.
Annotations on particular static entities must extend the corresponding
\xcd`Annotation` subclasses, as follows: 
\begin{itemize}
\item Expressions---\xcd`ExpressionAnnotation`
\item Statements---\xcd`StatementAnnotation`
\item Classes---\xcd`ClassAnnotation`
\item Fields---\xcd`FieldAnnotation`
\item Methods---\xcd`MethodAnnotation`
\item Imports---\xcd`ImportAnnotation`
\item Packages---\xcd`PackageAnnotation`
\end{itemize}

\chapter{Interoperability with Other Languages}
\label{NativeCode}
\index{native code}
\label{Interoperability}
\index{interoperability}

The ability to interoperate with other programming languages is an
essential feature of the \Xten{} implementation.  Cross-language
interoperability enables both the incremental adoption of \Xten{} in
existing applications and the usage of existing libraries and
frameworks by newly developed \Xten{} programs. 

There are two primary interoperability scenarios that are supported by
\XtenCurrVer{}: inline substitution of fragments of foreign code for
\Xten program fragments (expressions, statements) and external linkage
to foreign code.

\section{Embedded Native Code Fragments}

The
\xcd`@Native(lang,code) Construct` annotation from \xcd`x10.compiler.Native` in
\Xten{} can be used to tell the \Xten{} compiler to substitute \xcd`code` for
whatever it would have generated when compiling \xcd`Construct`
with the \xcd`lang` back end.

The compiler cannot analyze native code the same way it analyzes \Xten{} code.  In
particular, \xcd`@Native` fields and methods must be explicitly typed; the
compiler will not infer types.

\subsection{Native {\tt static} Methods}

\xcd`static` methods can be given native implementations.  Note that these
implementations are syntactically {\em expressions}, not statements, in C++ or
Java.   Also, it is possible (and common) to provide native implementations
into both Java and C++ for the same method.
%~~gen ^^^ extern10
% package Extern.or_current_turn;
%~~vis
\begin{xten}
import x10.compiler.Native;
class Son {
  @Native("c++", "printf(\"Hi!\")")
  @Native("java", "System.out.println(\"Hi!\")")
  static def printNatively():void = {};
}
\end{xten}
%~~siv
%
%~~neg

If only some back-end languages are given, the \Xten{} code will be used for the
remaining back ends: 
%~~gen ^^^ extern20
% package Extern.or.burn;
%~~vis
\begin{xten}
import x10.compiler.Native;
class Land {
  @Native("c++", "printf(\"Hi from C++!\")")
  static def example():void = {
    x10.io.Console.OUT.println("Hi from X10!");
  };
}
\end{xten}
%~~siv
%
%~~neg

The \xcd`native` modifier on methods indicates that the method must not have
an \Xten{} code body, and \xcd`@Native` implementations must be given for all back
ends:
%~~gen ^^^ extern30
% package Extern.or_maybe_getting_back_together;
%~~vis
\begin{xten}
import x10.compiler.Native;
class Plants {
  @Native("c++", "printf(\"Hi!\")")
  @Native("java", "System.out.println(\"Hi!\")")
  static native def printNatively():void;
}
\end{xten}
%~~siv
%
%~~neg


Values may be returned from external code to \Xten{}.  Scalar types in Java and
C++ correspond directly to the analogous types in \Xten{}.  
%~~gen ^^^ extern40
% package Extern.hardy;
%~~vis
\begin{xten}
import x10.compiler.Native;
class Return {
  @Native("c++", "1")
  @Native("java", "1")
  static native def one():Int;
}
\end{xten}
%~~siv
%
%~~neg
Types are {\em not} inferred for methods marked as \xcd`@Native`.

Parameters may be passed to external code.  \xcd`(#1)`  is the first parameter,
\xcd`(#2)` the second, and so forth.  (\xcd`(#0)` is the name of the enclosing
class, or the \xcd`this` variable.)  Be aware that this is macro substitution
rather than normal parameter 
passing; \eg, if the first actual parameter is \xcd`i++`, and \xcd`(#1)`
appears twice in the external code, \xcd`i` will be incremented twice.
For example, a (ridiculous) way to print the sum of two numbers is: 
%~~gen ^^^ extern50
% package Extern.or_turnabout_is_fair_play;
%~~vis
\begin{xten}
import x10.compiler.Native;
class Species {
  @Native("c++","printf(\"Sum=%d\", ((#1)+(#2)) )")
  @Native("java","System.out.println(\"\" + ((#1)+(#2)))")
  static native def printNatively(x:Int, y:Int):void;
}
\end{xten}
%~~siv
%
%~~neg


Static variables in the class are available in the external code.  For Java,
the static variables are used with their \Xten{} names.  For C++, the names
must be mangled, by use of the \xcd`FMGL` macro.  

%~~gen ^^^ extern60
%package Extern.or.Die;
%~~vis
\begin{xten}
import x10.compiler.Native;
class Ability {
  static val A : Int = 1n;
  @Native("java", "A+2")
  @Native("c++", "Ability::FMGL(A)+2")
  static native def fromStatic():Int;
}
\end{xten}
%~~siv
%
%~~neg




\subsection{Native Blocks}

Any block may be annotated with \xcd`@Native(lang,stmt)`, indicating that, in
the given back end, it should be implemented as \xcd`stmt`. All 
variables  from the surrounding context are available inside \xcd`stmt`. For
example, the method call \xcd`born.example(10)`, if compiled to Java, changes
the field \xcd`y` of a \xcd`Born` object to 10. If compiled to C++ (for which
there is no \xcd`@Native`), it sets it to 3. 
%~~gen ^^^ extern70
%package Extern.me.plz; 
%~~vis
\begin{xten}
import x10.compiler.Native;
class Born {
  var y : Int = 1n; 
  public def example(x:Int):Int{
    @Native("java", "y=x;") 
    {y = 3n;}
    return y;
  }
}
\end{xten}
%~~siv
%
%~~neg

Note that the code being replaced is a statement -- the block \xcd`{y = 3;}`
in this case -- so the replacement should also be a statement. 


Other \Xten{} constructs may or may not be available in Java and/or C++ code.  For
example, type variables do not correspond exactly to type variables in either
language, and may not be available there.  The exact compilation scheme is
{\em not} fully specified.  You may inspect the generated Java or C++ code and
see how to do specific things, but there is no guarantee that fancy external
coding will continue to work in later versions of \Xten{}.



The full facilities of C++ or Java are available in native code blocks.
However, there is no guarantee that advanced features behave sensibly. You
must follow the exact conventions that the code generator does, or you will
get unpredictable results.  Furthermore, the code generator's conventions may
change without notice or documentation from version to version.  In most cases
the  code should either be a very simple expression, or a method
or function call to external code.


\section{Interoperability with External Java Code}

With Managed X10, we can seamlessly call existing Java code from \Xten{},
and call \Xten{} code from Java.  We call this the 
\emph{Java interoperability}~\cite{TakeuchiX1013} feature.

By combining Java interoperability with X10's distributed
execution features, we can even make existing Java applications, which
are originally designed to run on a single Java VM, scale-out with
minor modifications.

\subsection{How Java program is seen in X10}

Managed X10 does not pre-process the existing Java code to make it
accessible from X10.  X10 programs compiled with Managed X10 will call
existing Java code as is.

\paragraph{Types}

In X10, both at compile time and run time, there is no way to
distinguish Java types from X10 types.  Java types can be referred to
with regular \xcd{import} statement, or their qualified names.  The
package \xcd{java.lang} is not auto-imported into \Xten.  In Managed
x10, the resolver is enhanced to resolve types with X10 source files
in the source path first, then resolve them with Java class files in
the class path. Note that the resolver does not resolve types with
Java source files, therefore Java source files must be compiled in
advance.  To refer to Java types listed in
Tables~\ref{tab:specialtypes}, and \ref{tab:otherspecialtypes}, which
include all Java primitive types, use the corresponding X10 type
(e.g. use \xcd{x10.lang.Int} (or in short, \xcd{Int}) instead of
\xcd{int}).

\paragraph{Fields}

Fields of Java types are seen as fields of X10 types.

Managed X10 does not change the static initialization semantics of
Java types, which is per-class, at load time, and per-place (Java VM),
therefore, it is subtly different than the per-field lazy
initialization semantics of X10 static fields.

\paragraph{Methods}

Methods of Java types are seen as methods of X10 types.

\paragraph{Generic types}

Generic Java types are seen as their raw types 
(\S 4.8 in~\cite{java-lang-spec2005}).  Raw type is a mechanism to handle generic
Java types as non-generic types, where the type parameters are assumed
as \verb|java.lang.Object| or their upperbound if they have it.  Java
introduced generics and raw type at the same time to facilitate
generic Java code interfacing with non-generic legacy Java code.
Managed X10 uses this mechanism for a slightly different purpose.
Java erases type parameters at compile time, whereas X10 preserves
their values at run time.  To manifest this semantic gap in generics,
Managed X10 represents Java generic types as raw types and eliminates
type parameters at source code level.  For more detailed discussions,
please refer to~\cite{TakeuchiX1011,TakeuchiX1012}.

\paragraph{Arrays}

X10 rail and array types are generic types whose representation is different
from Java array types.

Managed X10 provides a special X10 type
\xcd{x10.interop.Java.array[T]} which represents Java array type
\xcd{T[]}.  Note that for X10 types in Table~\ref{tab:specialtypes},
this type means the Java array type of their primary type.  For
example, \xcd{array[Int]} and \xcd{array[String]} mean
\xcd{int[]} and \xcd{java.lang.String[]}, respectively.  Managed X10
also provides conversion methods between X10 \xcd`Rail`s and Java
arrays (\xcd{Java.convert[T](a:Rail[T]):array[T]} and
\xcd{Java.convert[T](a:array[T]):Rail[T]}),
and creation methods for Java arrays 
(\xcd{Java.newArray[T](d0:Int):array[T]}
etc.).

\paragraph{Exceptions}

The \Xten{} 2.3 exception hierarchy has been designed so that there is a
natural correspondence with the Java exception hierarchy. As shown in
Table~\ref{tab:otherspecialtypes}, many commonly used Java
exception types are directly mapped to X10 exception types. 
Exception types that are thus aliased may be caught (and thrown) using
either their Java or \Xten types.  In \Xten code, it is stylistically
preferable to use the \Xten type to refer to the exception as shown in 
Figure~\ref{fig:javaexceptions}.

%----------------
\begin{figure}
\begin{xten}
import x10.interop.Java;
public class XClass {   
  public static def main(args:Rail[String]):void {
    try {
      val a = Java.newArray[Int](2);
      a(0) = 0;
      a(1) = 1;
      a(2) = 2;
    } catch (e:x10.lang.ArrayIndexOutOfBoundsException) {
      Console.OUT.println(e);
    }
  }
}
\end{xten}
%\vspace{-2mm}%@@ADJUST
\begin{verbatim}
> x10c -d bin src/XClass.x10
> x10 -cp bin XClass
x10.lang.ArrayIndexOutOfBoundsException: Array index out of range: 2
\end{verbatim}
\caption{Java exceptions in X10}
%\vspace{-4mm}%@@ADJUST
\label{fig:javaexceptions}
\end{figure}
%----------------

\paragraph{Compiling and executing X10 programs}

We can compile and run X10 programs that call existing Java code with
the same \verb|x10c| and \verb|x10| command by specifying the location
of Java class files or jar files that your X10 programs refer to, with
\verb|-classpath| (or in short, \verb|-cp|) option.

\subsection{How X10 program is translated to Java}

Managed X10 translates X10 programs to Java class files. 

X10 does not provide a Java reflection-like mechanism to resolve X10
types, methods, and fields with their names at runtime, nor a code
generation tool, such as \verb|javah|, to generate stub code to access
them from other languages.  Java programmers, therefore, need to
access X10 types, methods, and fields in the generated Java code
directly as they access Java types, methods, and fields.  To make it
possible, Java programmers need to understand how X10 programs are
translated to Java.

Some aspects of the X10 to Java translation scheme may change in
future version of \Xten{}; therefore in this document only a stable
subset of translation scheme will be explained.  Although it is a
subset, it has been extensively used by X10 core team and proved to be
useful to develop Java Hadoop interop layer for a Main-memory Map
Reduce (M3R) engine~\cite{Shinnar12M3R} in X10.

In the following discussions, we deliberately ignore generic X10
types because the translation of generics is an area of active
development and will undergo some changes in future versions of \Xten{}.
For those who are interested in the implementation of generics
in Managed X10, please consult~\cite{TakeuchiX1012}.  We also don't
cover function types, function values, and all non-static methods.
Although slightly outdated, another paper~\cite{TakeuchiX1011}, which
describes translation scheme in X10 2.1.2, is still useful to
understand the overview of Java code generation in Managed X10.


\paragraph{Types}

X10 classes and structs are translated to Java classes with the same
names.  X10 interfaces are translated to Java interfaces with the same
names.

Table~\ref{tab:specialtypes} shows the list of special types that are
mapped to Java primitives.  Primitives are their primary
representations that are useful for good performance.  Wrapper classes
are used when the reference types are needed.  Each wrapper class has
two static methods \verb|$box()| and \verb|$unbox()| to convert its
value from primary representation to wrapper class, and vice versa,
and Java backend inserts their calls as needed.  An you notice, every
unsigned type uses the same Java primitive as its corresponding signed
type for its representation.

Table~\ref{tab:otherspecialtypes} shows a non-exhaustive list of
another kind of special types that are mapped (not translated) to Java
types.  As you notice, since an interface \verb|Any| is mapped to a
class |java.lang.Object| and \verb|Object| is hidden from the
language, there is no direct way to create an instance of
\verb|Object|. As a workaround, \verb|Java.newObject()| is provided.

As you also notice, \verb|x10.lang.Comparable[T]| is mapped to \verb|java.lang.Comparable|.
This is needed to map \verb|x10.lang.String|, which implements \verb|x10.lang.Compatable[String]|, to \verb|java.lang.String| for performance, but as a trade off, this mapping results in the lost of runtime type information for \verb|Comparable[T]| in Managed X10.
The runtime of Managed X10 has built-in knowledge for \verb|String|, but for other Java classes that implement \verb|java.lang.Comparable|, \verb|instanceof Comparable[Int]| etc. may return incorrect results.
In principle, it is impossible to map X10 generic type to the existing Java generic type without losing runtime type information.

%----------------
\begin{table}
%\scriptsize
\small
\centering
\mbox{
%\hspace{-4mm}%@@ADJUST
\begin{tabular}{|lr|lr|l|}												   \hline
\multicolumn{2}{|c|}{\textbf{X10}}	& \multicolumn{2}{|c|}{\textbf{Java (primary)}}	& \textbf{Java (wrapper class)}	\\ \hline
															   \hline
{\tt x10.lang.Byte}	& {\tt 1y}	& {\tt byte}		& {\tt (byte)1}		& {\tt x10.core.Byte}		\\ \hline
{\tt x10.lang.UByte}	& {\tt 1uy}	& {\tt byte}		& {\tt (byte)1}		& {\tt x10.core.UByte}		\\ \hline
{\tt x10.lang.Short}	& {\tt 1s}	& {\tt short}		& {\tt (short)1}	& {\tt x10.core.Short}		\\ \hline
{\tt x10.lang.UShort}	& {\tt 1us}	& {\tt short}		& {\tt (short)1}	& {\tt x10.core.UShort} 	\\ \hline
{\tt x10.lang.Int}	& {\tt 1}	& {\tt int}		& {\tt 1}		& {\tt x10.core.Int}		\\ \hline
{\tt x10.lang.UInt}	& {\tt 1u}	& {\tt int}		& {\tt 1}		& {\tt x10.core.UInt}		\\ \hline
{\tt x10.lang.Long}	& {\tt 1l}	& {\tt long}		& {\tt 1l}		& {\tt x10.core.Long}	 	\\ \hline
{\tt x10.lang.ULong}	& {\tt 1ul}	& {\tt long}		& {\tt 1l}		& {\tt x10.core.ULong}	 	\\ \hline
{\tt x10.lang.Float}	& {\tt 1.0f}	& {\tt float}		& {\tt 1.0f}		& {\tt x10.core.Float}	 	\\ \hline
{\tt x10.lang.Double}	& {\tt 1.0}	& {\tt double}		& {\tt 1.0}		& {\tt x10.core.Double} 	\\ \hline
{\tt x10.lang.Char}	& {\tt 'c'}	& {\tt char}		& {\tt 'c'}		& {\tt x10.core.Char}		\\ \hline
{\tt x10.lang.Boolean}	& {\tt true}	& {\tt boolean}		& {\tt true}		& {\tt x10.core.Boolean}	\\ \hline
%{\tt x10.lang.String} 	& {\tt "abc"}	& {\tt java.lang.String}& {\tt "abc"}		& {\tt x10.core.String}		\\ \hline
\end{tabular}
}
\caption{X10 types that are mapped to Java primitives}
%\vspace{-4mm}%@@ADJUST
\label{tab:specialtypes}
\end{table}
%----------------


%----------------
\begin{table}
%\scriptsize
\small
\centering
\mbox{
%\hspace{-4mm}%@@ADJUST
\begin{tabular}{|l|l|}										   \hline
\multicolumn{1}{|c|}{\textbf{X10}}		& \multicolumn{1}{|c|}{\textbf{Java}}		\\ \hline
												   \hline
{\tt x10.lang.Any} 				& {\tt java.lang.Object}			\\ \hline
{\tt x10.lang.Comparable[T]} 			& {\tt java.lang.Comparable}			\\ \hline
{\tt x10.lang.String}		 		& {\tt java.lang.String}			\\ \hline
{\tt x10.lang.CheckedThrowable}		 	& {\tt java.lang.Throwable}			\\ \hline
{\tt x10.lang.CheckedException}		 	& {\tt java.lang.Exception}			\\ \hline
{\tt x10.lang.Exception} 			& {\tt java.lang.RuntimeException}		\\ \hline
{\tt x10.lang.ArithmeticException} 		& {\tt java.lang.ArithmeticException}		\\ \hline
{\tt x10.lang.ClassCastException} 		& {\tt java.lang.ClassCastException}		\\ \hline
{\tt x10.lang.IllegalArgumentException} 	& {\tt java.lang.IllegalArgumentException}	\\ \hline
{\tt x10.util.NoSuchElementException}	 	& {\tt java.util.NoSuchElementException}	\\ \hline
{\tt x10.lang.NullPointerException} 		& {\tt java.lang.NullPointerException}		\\ \hline
{\tt x10.lang.NumberFormatException} 		& {\tt java.lang.NumberFormatException}		\\ \hline
{\tt x10.lang.UnsupportedOperationException} 	& {\tt java.lang.UnsupportedOperationException}	\\ \hline
{\tt x10.lang.IndexOutOfBoundsException} 	& {\tt java.lang.IndexOutOfBoundsException}	\\ \hline
{\tt x10.lang.ArrayIndexOutOfBoundsException} 	& {\tt java.lang.ArrayIndexOutOfBoundsException}\\ \hline
{\tt x10.lang.StringIndexOutOfBoundsException} 	& {\tt java.lang.StringIndexOutOfBoundsException}\\ \hline
{\tt x10.lang.Error} 				& {\tt java.lang.Error}				\\ \hline
{\tt x10.lang.AssertionError} 			& {\tt java.lang.AssertionError}		\\ \hline
{\tt x10.lang.OutOfMemoryError} 		& {\tt java.lang.OutOfMemoryError}		\\ \hline
{\tt x10.lang.StackOverflowError} 		& {\tt java.lang.StackOverflowError}		\\ \hline
{\tt void} 					& {\tt void}					\\ \hline
\end{tabular}
}
\caption{X10 types that are mapped (not translated) to Java types}
%\vspace{-4mm}%@@ADJUST
\label{tab:otherspecialtypes}
\end{table}
%----------------


\paragraph{Fields}

As shown in Figure~\ref{fig:fields}, instance fields of X10 classes and structs are translated to the instance fields of the same names of the generated Java classes.
Static fields of X10 classes and structs are translated to the static methods of the generated Java classes, whose name has \verb|get$| prefix.
Static fields of X10 interfaces are translated to the static methods of the special nested class named \verb|$Shadow| of the generated Java interfaces.

%----------------
\begin{figure}
\begin{xten}
class C {
  static val a:Int = ...;
  var b:Int;
}
interface I {
  val x:Int = ...;
}
\end{xten}
%\vspace{-4mm}%@@ADJUST
\begin{xten}
class C {
  static int get$a() { return ...; }
  int b;
}
interface I {
  abstract static class $Shadow {
    static int get$x() { return ...; }
  }
}
\end{xten}
%\vspace{-2mm}%@@ADJUST
\caption{X10 fields in Java}
%\vspace{-4mm}%@@ADJUST
\label{fig:fields}
\end{figure}
%----------------


\paragraph{Methods}

As shown in Figure~\ref{fig:methods}, methods of X10 classes or structs are translated to the methods of the same names of the generated Java classes.
Methods of X10 interfaces are translated to the methods of the same names of the generated Java interfaces.

Every method whose return type has two representations, such as the types in Table~\ref{tab:specialtypes}, will have \verb|$O| suffix with its name.
For example, \verb|def f():Int| in X10 will be compiled as \verb|int f$O()| in Java.
For those who are interested in the reason, please look at our paper~\cite{TakeuchiX1012}.

%----------------
\begin{figure}
\begin{xten}
interface I {
  def f():Int;
  def g():Any;
}
class C implements I {
  static def s():Int = 0;
  static def t():Any = null;
  public def f():Int = 1;
  public def g():Any = null;
}
\end{xten}
%\vspace{-4mm}%@@ADJUST
\begin{xten}
interface I {
  int f$O();
  java.lang.Object g();
}
class C implements I {
  static int s$O() { return 0; }
  static java.lang.Object t() { return null; }
  public int f$O() { return 1; }
  public java.lang.Object g() { return null; }
}
\end{xten}
%\vspace{-2mm}%@@ADJUST
\caption{X10 methods in Java}
%\vspace{-4mm}%@@ADJUST
\label{fig:methods}
\end{figure}
%----------------


\paragraph{Compiling Java programs}

To compile Java program that calls X10 code, you should use
\verb|x10cj| command instead of javac (or whatever your Java
compiler). It invokes the post Java-compiler of \verb|x10c| with the
appropriate options. You need to specify the location of X10-generated
class files or jar files that your Java program refers to.

\verb|x10cj -cp MyX10Lib.jar MyJavaProg.java|


\paragraph{Executing Java programs}

Before executing any X10-generated Java code, the runtime of Managed
X10 needs to be set up properly at each place.  To set up the runtime,
a special launcher named \verb|runjava| is used to run Java programs.
All Java programs that call X10 code should be launched with it, and
no other mechanisms, including direct execution with java command, are
supported.

\begin{verbatim}
Usage: runjava <Java-main-class> [arg0 arg1 ...]
\end{verbatim}


\section{Interoperability with External C and C++ Code}

C and C++ code can be linked to X10 code, either by writing auxiliary C++ files and
adding them with suitable annotations, or by linking libraries.

\subsection{Auxiliary C++ Files}

Auxiliary C++ code can be written in \xcd`.h` and \xcd`.cc` files, which
should be put in the same directory as the the X10 file using them.
Connecting with the library uses the \xcd`@NativeCPPInclude(dot_h_file_name)`
annotation to include the header file, and the 
\xcd`@NativeCPPCompilationUnit(dot_cc_file_name)` annotation to include the
C++ code proper.  For example: 

{\bf MyCppCode.h}: 
\begin{xten}
void foo();
\end{xten}


{\bf MyCppCode.cc}:
\begin{xten}
#include <cstdlib>
#include <cstdio>
void foo() {
    printf("Hello World!\n");
}
\end{xten}

{\bf Test.x10}:
\begin{xten}
import x10.compiler.Native;
import x10.compiler.NativeCPPInclude;
import x10.compiler.NativeCPPCompilationUnit;

@NativeCPPInclude("MyCPPCode.h")
@NativeCPPCompilationUnit("MyCPPCode.cc")
public class Test {
    public static def main (args:Rail[String]) {
        { @Native("c++","foo();") {} }
    }
}
\end{xten}

\subsection{C++ System Libraries}

If we want to additionally link to more libraries in \xcd`/usr/lib` for
example, it is necessary to adjust the post-compilation directly.  The
post-compilation is the compilation of the C++ which the X10-to-C++ compiler
\xcd`x10c++` produces.  

The primary mechanism used for this is the \xcd`-cxx-prearg` and
\xcd`-cxx-postarg` command line arguments to
\xcd`x10c++`. The values of any \xcd`-cxx-prearg` commands are placed
in the post compiler command before the list of .cc files to compile.
The values of any \xcd`-cxx-postarg` commands are placed in the post
compiler command after the list of .cc files to compile. Typically
pre-args are arguments intended for the C++ compiler itself, while
post-args are arguments intended for the linker. 

The following example shows how to compile \xcd`blas` into the
executable via these commands. The command must be issued on one line.

\begin{xten}
x10c++ Test.x10 -cxx-prearg -I/usr/local/blas 
  -cxx-postarg -L/usr/local/blas -cxx-postarg -lblas'
\end{xten}


\chapter{Definite Assignment}
\label{sect:DefiniteAssignment}
\index{definite assignment}
\index{assignment!definite}
\index{definitely assigned}
\index{definitely not assigned}

X10 requires that every variable be set before it is read.
Sometimes this is easy, as when a variable is declared and assigned together: 
%~~gen ^^^ DefiniteAssignment4x1u
% package DefiniteAssignment4x1u;
% class Example {
% def example() {
%~~vis
\begin{xten}
  var x : Long = 0;
  assert x == 0;
\end{xten}
%~~siv
%}}
%~~neg
However, it is convenient to allow programs to make decisions before
initializing variables.
%~~gen ^^^ DefiniteAssignment4u7z
% package DefiniteAssignment4u7z;
% class Example {
%~~vis
\begin{xten}
def example(a:Long, b:Long) {
  val max:Long;
  //ERROR: assert max==max; // can't read 'max'
  if (a > b) max = a;
  else max = b;
  assert max >= a && max >= b;
}
\end{xten}
%~~siv
%}
%~~neg
This is particularly useful for \xcd`val` variables.  \xcd`var`s could be
initialized to a default value and then reassigned with the right value.
\xcd`val`s must be initialized once and cannot be changed, so they must be
initialized with the correct value. 

However, one must be careful -- and the X10 compiler enforces this care.
Without the \xcd`else` clause, the preceding code might not give \xcd`max` a
value by the time \xcd`assert` is invoked.  

This leads to the concept of {\em definite assignment} \cite{jls2}.
A variable is {\em definitely assigned} at a point in code if, no matter how that
point in code is reached, the variable has been assigned to.  In X10,
variables must be definitely assigned before they can be read.


As X10 requires that \xcd`val` variables {\em not} be initialized
twice,  we need the dual concept as well.  A variable is {\em definitely
unassigned} at a point in code if it cannot have been assigned no
matter how that point in code is reached.  For example, immediately
after \xcd`val x:Long`, \xcd`x` is definitely unassigned.  

Finally, we need the concept of {\em singly} and {\em multiply assigned}.
A variable is singly assigned in a block if it is assigned precisely
once; it is multiply assigned if it could possibly be assigned more than once.  
\xcd`var`s can  multiply assigned as desired. \xcd`val`s must be singly
assigned.  For example, the code \xcd`x = 1; x = 2;` is perfectly fine if
\xcd`x` is a \xcd`var`, but incorrect (even in a constructor) if \xcd`x` is a
\xcd`val`.  

At some points in code, a variable might be neither definitely assigned nor
definitely unassigned.    Such states are not always useful.  
%~~gen ^^^ DefiniteAssignment4f5z
% package DefiniteAssignment4f5z;
% class Example {
% 
%~~vis
\begin{xten}
def example(flag : Boolean) {
  var x : Long;
  if (flag) x = 1;
  // x is neither def. assigned nor unassigned.
  x = 2; 
  // x is def. assigned.
\end{xten}
%~~siv
% } } 
%~~neg
This shows that we cannot simply define ``definitely unassigned'' as ``not
definitely assigned''.   If \xcd`x` had been a \xcd`val` rather than a
\xcd`var`, the previous example would not be allowed.    

Unfortunately, a completely accurate definition of ``definitely assigned''
or ``definitely unassigned'' is undecidable -- impossible for the compiler to
determine.  So, X10 takes a {\em conservative approximation} of these
concepts.  If X10's definition says that \xcd`x` is definitely assigned (or
definitely unassigned), then it will be assigned (or not assigned) in every
execution of the program.  

However, there are programs which X10's algorithm says are incorrect, but
which actually would behave properly if they were executed.   In the following
example, \xcd`flag` is either \xcd`true` or \xcd`false`, and in either case
\xcd`x` will be initialized.  However, X10's analysis does not understand this
--- thought it {\em would} understand if the example were coded with an
\xcd`if-else` rather than a pair of \xcd`if`s.  So, after the two \xcd`if`
statements, \xcd`x` is not definitely assigned, and thus the \xcd`assert`
statement, which reads it, is forbidden.  
%~~gen ^^^ DefiniteAssignment3x6i
% package DefiniteAssignment3x6i;
% class Example{ 
%~~vis
\begin{xten}
def example(flag:Boolean) {
  var x : Long;
  if (flag) x = 1;
  if (!flag) x = 2;
  // ERROR: assert x < 3;
}
\end{xten}
%~~siv
%}
%~~neg

\section{Asynchronous Definite Assignment}


Local variables and instance fields allow {\em asynchronous assignment}. A local
variable can be assigned in an \xcd`async` statement, and, when the
\xcd`async` is \xcd`finish`ed, the variable is definitely assigned.  

\begin{ex}
%~~gen ^^^ DefiniteAssignment4a6n
% package DefiniteAssignment4a6n;
% class Example {
% def example() {
%~~vis
\begin{xten}
val a : Long;
finish {
  async {
    a = 1;
  } 
  // a is not definitely assigned here
}
// a is definitely assigned after 'finish'
assert a==1; 
\end{xten}
%~~siv
%} } 
%~~neg
\end{ex}

This concept supports a core X10 programming idiom.  A \xcd`val` variable may
be initialized asynchronously, thereby providing a means for returning a value
from an \xcd`async` to be used after the enclosing \xcd`finish`.  

\section{Characteristics of Definite Assignment}

The properties ``definitely assigned'', ``singly assigned'', and
``definitely unassigned'' are computed by a conservative approximation of
X10's evaluation rules.

The precise details are up to the implementation. 
Many basic cases must be handled accurately; \eg, \xcd`x=1;` definitely and
singly assigns \xcd`x`.  

However, in more complicated cases, a conforming X10 may mark as invalid 
some code which, when executed, would actually be correct.  
For example, the following
program fragment will always result in \xcd`x` being definitely and singly
assigned:  
\begin{xten}
val x : Long;
var b : Boolean = mysterious();
if (b) x = cryptic();
if (!b) x = unknown();
\end{xten}
However, most conservative approximations of program execution won't mark
\xcd`x` as properly initialized, though it is.   For \xcd`x` to be properly
initialized, precisely one of the 
two assignments to \xcd`x` must be executed. If \xcd`b` were true initially,
it would still be true after the call to \xcd`cryptic()` --- since methods
cannot modify their caller's local variables -- and so the first but not the
second assignment would happen. If \xcd`b` were false initially, it would
still be false when \xcd`!b` is tested, and so the second but not the first
assignment would happen.  Either way, \xcd`x` is definitely and singly assigned.

However, for a slightly different program, this analysis would be wrong. \Eg,
if  \xcd`b` were a field of \xcd`this` rather than a local variable,
\xcd`cryptic()` could change \xcd`b`; if \xcd`b` were true initially, both
assignments might happen, which is incorrect for a \xcd`val`.  

This sort of reasoning is beyond  most conservative approximation algorithms.
(Indeed, many do not bother checking that \xcd`!b` late in the program is the
opposite of \xcd`b` earlier.)
Algorithms that pay attention to such details and subtleties tend to be
fairly expensive, which would lead to very slow compilation for X10 -- for the
sake of obscure cases.

X10's analysis provides at least the following guarantees. We describe them in
terms of a statement \xcd`S` performing some collection of possible numbers of
assignments to variables --- on a scale of ``0'', ``1'', and ``many''. For
example, \xcd`if (b) x=1; else {x=1;x=2;y=2;}` might assign to \xcd`x` one or
many times, and might assign to \xcd`y` zero or one time. Hence, after it,
\xcd`x` is definitely assigned and may be multiply assigned, and \xcd`y` is
neither definitely assigned nor definitely unassigned.  

These descriptions are combined in natural ways.  For example, if \xcd`R` says
that \xcd`x` will be assigned 0 or 1 times, and \xcd`S` says it will be
assigned precisely once, then \xcd`R;S` will assign it one or many times.  If
only one or \xcd`R` or \xcd`S` will occur, as from \xcd`if (b) R; else S;`, 
then \xcd`x` may be assigned 0 or 1 times. 

This information is sufficient for the tests X10 makes.  If \xcd`x` can is
assigned one or many times in \xcd`S`, it is definitely assigned.  It is an
error if 
\xcd`x` is ever read at a point where it have been assigned zero times.  It is
an error if a \xcd`val` may be assigned many times.

We do not guarantee that any particular X10 compiler uses this algorithm;
indeed, as of the time of writing, the X10 compiler uses a somewhat more
precise one.  However, any conformant X10 compiler must provide results which
are at least as accurate as this analysis.

\subsubsection{Assignment: {\tt x = e}}   

\xcd`x = e` assigns to \xcd`x`, in addition to whatever assignments
\xcd`e` makes.   For example, if \xcd`this.setX(y)` sets a field \xcd`x` to
\xcd`y` and returns \xcd`y`, then \xcd`x = this.setX(y)` definitely and
multiply assigns \xcd`x`.  

\subsubsection{{\tt async} and {\tt finish}}

By itself, \xcd`async S` provides few guarantees.  After an activity
executes \xcd`async{x=1;}` we know that there is a separate activity
which (on being scheduled) will set \xcd`x` to \xcd`1`.  We do not
know that this has happened yet.

However, if there is a \xcd`finish` around the \xcd`async`, the situation is
clearer.  After \xcd`finish async x=1;`, \xcd`x` has definitely been
assigned.  

In general, if an \xcd`async S` appears in the body of a \xcd`finish` in a way
that guarantees that it will be executed, then, after the \xcd`finish`, the
assignments made by \xcd`S` will have occurred.  For example, if \xcd`S`
definitely assigns to \xcd`x`, and the body of the \xcd`finish` guarantees
that \xcd`async S` will be executed, then \xcd`finish{...async S...}`
definitely assigns \xcd`x`.

\subsubsection{{\tt if} and {\tt switch}}

When \xcd`if(E) S else T` finishes, it will have performed the assignments of
\xcd`E`, together with those of either \xcd`S` or \xcd`T` but not both.  For
example, \xcd`if (b) x=1; else x=2;` definitely assigns \xcd`x`,
but \xcd`if (b) x=1;` does not.

{\tt switch} is more complex, but follows the same principles as \xcd`if`.
For example, \xcd`switch(E){case 1: A; break; case 2: B; default: C;}`  
performs the assignments of \xcd`E`, and those of precisely one of \xcd`A`, or
\xcd`B;C`, or \xcd`C`.  Note that case \xcd`2` falls through to the default
case, so it performs the same assignments as \xcd`B;C`.

\subsubsection{Sequencing}

When \xcd`R;S` finishes, it will have performed the assignments of \xcd`R` and
those of \xcd`S`, if \xcd`R` and \xcd`S` terminate normally. If
\xcd`R` terminates abruptly, then only the assignments of \xcd`R`
executed till the point of termination will have been executed. if
\xcd`R` terminates normally, but \xcd`S` terminates abruptly then the
assignments of \xcd`R` will have been executed and those of \xcd`S`
executed till the point of termination. 

For example, \xcd`x=1;y=2;` definitely assigns \xcd`x` and 
\xcd`y`, and \xcd`x=1;x=2;` multiply assigns \xcd`x`. 


\subsubsection{Loops}

\xcd`while(E)S` performs the assignments of \xcd`E` one or more times, and
those of \xcd`S` zero or more times.  For example, if \xcd`while(b()){x=1;}`
might assign to \xcd`x` zero, one, or many times.  
\xcd`do S while(E)` performs the assignments of \xcd`E` one or more times, and
those of \xcd`S` one or more times. 

\xcd`for(A;B;C)D` performs the assignments of \xcd`A` once, those of \xcd`B`
one or more times, and those of \xcd`C` and \xcd`D` one or more times.
\xcd`for(x in E)S` performs the assignments of \xcd`E` once and those of
\xcd`S` zero or more times.  

Loops are of very little value for providing definite assignments, since X10
does not in general know how many times they will be executed. 

\xcd`continue` and \xcd`break` inside of a loop are hard to describe in simple
terms.  They may be conservatively assumed to cause the loop to give no
information about the variables assigned inside of it.
For example, the analysis may conservatively conclude that 
\xcd`do{ x = 1; if (true) break; } while(true)` may 
assign to \xcd`x` zero, one, or many times, overlooking the more precise fact
that it is assigned once.  

\subsubsection{Method Calls}

A method call \xcd`E.m(A,B)` performs the assignments of \xcd`E`, \xcd`A`, and
\xcd`B` once each, and also those of \xcd`m`.  This implies that X10 must be
aware of the possible assignments performed by each method.

If X10 has complete information about \xcd`m` (as when \xcd`m` is a
\xcd`private` or \xcd`final` method), this is straightforward.  When such
information is fundamentally impossible to acquire, as when \xcd`m` is a
non-final method invocation, X10 has no choice but to assume that \xcd`m`
might do anything that a method can do.    (For this reason, the only methods
that can be called from within a constructor on a raw --
incompletely-constructed -- object) are the \xcd`private` and
\xcd`final` ones.)  
\begin{itemize}
\item \xcd`m` cannot assign to local variables of the caller; methods have no
      such power.
\item Let \xcd`m` be an instance method. \xcd`m` can assign to \xcd`var` fields of \xcd`this` freely,
\item Let \xcd`m` be an instance method. \xcd`m` cannot initialize \xcd`val` fields of \xcd`this`.  (But see
      \Sref{sect:call-another-ctor}; when one constructor calls another as the
      first statement of its body, the other constructor can initialize
      v\xcd`val` fields. This is a constructor call, not a method call.) 
\end{itemize}

Recall that every container must be fully initialized upon exit
from its constructor.  
X10 places certain restrictions on which methods can be called from a
constructor; see \Sref{sect:nonescaping}.  One of these restrictions is that
methods called before object initialization is complete must be \xcd`final` or
\xcd`private` --- and hence, available for static analysis.  So, when checking
field initialization, X10 will ensure: 
\begin{enumerate}
\item Each \xcd`val` field is initialized before it is read.   
      A method that does not read a \xcd`val` field \xcd`f` {\em may} be
      called before \xcd`f` is initialized; a method that reads \xcd`f` must
      not be called until \xcd`f` is initialized.        
      For example, 
      a constructor may have the form:
%~~gen ^^^ DefiniteAssignment4x6k
% package DefiniteAssignment4x6k;
%~~vis
\begin{xten}
class C {
  val f : Long;
  val g : String;
  def this() {
     f = fless();
     g = useF();
  }
  private def fless() = "f not used here".length();
  private def useF() = "f=" + this.f;
}
\end{xten}
%~~siv
%
%~~neg

\item \xcd`var` fields require a deeper analysis.  Consider a \xcd`var`
      field \xcd`var x:T`  without initializer.  If \xcd`T` has a default
      value, \xcd`x` may be read inside of a constructor before it is
      otherwise written, and it will 
      have its default value.   

      If \xcd`T` has no default value, an analysis
      like that used for \xcd`val`s must be performed to determine that
      \xcd`x` is initialized before it is used.  The situation is 
      more complex than for \xcd`val`s, however, because a method can assign to
      \xcd`x` as well read from it.  The X10 compiler computes a conservative
      approximation of which methods
      read and write which \xcd`var` fields. (Doing this carefully 
      requires finding a solution of a set of equations over sets of
      variables, with each callable method having equations describing what it
      reads and writes.)    

\end{enumerate}


\subsubsection{{\tt at} 
%and \xcd`athome`
}

%%AT-COPY%% \xcd`at(E)S` performs the assignments of \xcd`E`. Within \xcd`S`, only those
%%AT-COPY%% assignments to variables \xcd`x` from the surrounding environment which take
%%AT-COPY%% place within a suitable \xcd`athome(x)R` are counted. 
%%AT-COPY%% 
%%AT-COPY%% \begin{ex}
%%AT-COPY%% In the following code, the outer variable named \xcd`a` is definitely assigned
%%AT-COPY%% once, by the assignment \xcd`a = 3;`.  The inner variable (also named \xcd`a`)
%%AT-COPY%% is definitely multiply assigned 
%%AT-COPY%% by the two assignments \xcd`a = 1;` and \xcd`a = 2;` 
%%AT-COPY%% between the \xcd`at` and the \xcd`athome`.  
%%AT-COPY%% 
%%AT-COPY%% %~~gen ^^^ DefiniteAssignment3n5q
%%AT-COPY%% % package DefiniteAssignment3n5q;
%%AT-COPY%% % KNOWNFAIL-at
%%AT-COPY%% % class DefAss { def defass() { 
%%AT-COPY%% %~~vis
%%AT-COPY%% \begin{xten}
%%AT-COPY%% var a : Long;
%%AT-COPY%% at(here.next(); var a : Long = a) {
%%AT-COPY%%   a = 1;
%%AT-COPY%%   a = 2; 
%%AT-COPY%%   athome(a) a = 3;
%%AT-COPY%% }
%%AT-COPY%% \end{xten}
%%AT-COPY%% %~~siv
%%AT-COPY%% % } } 
%%AT-COPY%% %~~neg
%%AT-COPY%% 
%%AT-COPY%% 
%%AT-COPY%% \end{ex}
%%AT-COPY%% 

% vj Wed Sep 18 04:19:05 EDT 2013
% Hmm. This used to be incorrect.
\xcd`at(p)S` performs precisely the assignments of \xcd`p` and those
of \xcd`S`. Note that \xcd`S` is executed at the place named by
\xcd`p` in an environment in which all variables used in \xcd`S` but
defined outside \xcd`S` are bound to copies (made at \xcd`p`) of the
values they had at the \xcd`at(p)S` statement (\Sref{AtStatement}).

% vj Wed Sep 18 04:19:05 EDT 2013
% Hmm. Commented this out. This is not true :-(
%\xcd`this` cannot be read or written by an \xcd`at`-statement.

\subsubsection{{\tt atomic}}

\xcd`atomic S` performs the assignments of \xcd`S`, 
and \xcd`when(E)S` performs those of \xcd`E` and \xcd`S`.  Note that
\xcd`E` or \xcd`S` may terminate abruptly.

\subsubsection{{\tt try}}

\xcd`try S catch(x:T1) E1 ... catch(x:Tn) En finally F` 
performs some or all of the assignments of \xcd`S`, plus all the assignments
of zero or one of the \xcd`E`'s, plus those of \xcd`F`.  
For example,
\begin{xten}
try {
  x = boomy();
  x = 0;
}
catch(e:Boom) { y = 1; }
finally { z = 1; }
\end{xten}
\noindent 
assigns \xcd`x` zero, one, or many times\footnote{A more precise
analysis could discover that \xcd`x` cannot be initialized only once.}, 
assigns \xcd`y` zero or one time, and assigns \xcd`z` exactly once.

\subsubsection{Expression Statements}

Expression statements \xcd`E;`, and other statements that execute an
expression and do something innocuous with it (local variable declaration and
\xcd`assert`) have the same effects as \xcd`E`. 

\subsubsection{{\tt return}, {\tt throw}}

Statements that do not finish normally, such as \xcd`return` and \xcd`throw`,
do not initialize anything (though the computation of the return or thrown
value may).    They also terminate a line of computation.  For example, 
\xcd`if(b) {x=1; return;}  x=2;` definitely and singly assigns \xcd`x`.  

%% vj Thu Sep 19 06:00:59 EDT 2013
%% No changes made for v2.4. 

\chapter{Grammar}\label{Grammar}


In this grammar, $X^?$ denotes an optional $X$ element.


\begin{bbgrammarappendix}{3.9in}

(\arabic{equation}) & AdditiveExp \refstepcounter{equation}\label{prod:AdditiveExp}  \: MultiplicativeExp  \\

 &    \| AdditiveExp \xcd"+" MultiplicativeExp \\
 &    \| AdditiveExp \xcd"-" MultiplicativeExp \\

\end{bbgrammarappendix}

\begin{bbgrammarappendix}{4.4in}

(\arabic{equation}) & AndExp \refstepcounter{equation}\label{prod:AndExp}  \: EqualityExp  \\

 &    \| AndExp \xcd"&" EqualityExp \\

\end{bbgrammarappendix}

\begin{bbgrammarappendix}{3.7in}

(\arabic{equation}) & AnnotatedType \refstepcounter{equation}\label{prod:AnnotatedType}  \: Type Annotations  \\


\end{bbgrammarappendix}

\begin{bbgrammarappendix}{4.0in}

(\arabic{equation}) & Annotation \refstepcounter{equation}\label{prod:Annotation}  \: \xcd"@" NamedTypeNoConstraints  \\


\end{bbgrammarappendix}

\begin{bbgrammarappendix}{3.6in}

(\arabic{equation}) & AnnotationStmt \refstepcounter{equation}\label{prod:AnnotationStmt}  \: Annotations\opt NonExpStmt  \\


\end{bbgrammarappendix}

\begin{bbgrammarappendix}{3.9in}

(\arabic{equation}) & Annotations \refstepcounter{equation}\label{prod:Annotations}  \: Annotation  \\

 &    \| Annotations Annotation \\

\end{bbgrammarappendix}

\begin{bbgrammarappendix}{3.8in}

(\arabic{equation}) & ApplyOpDecln \refstepcounter{equation}\label{prod:ApplyOpDecln}  \: MethMods \xcd"operator" \xcd"this" TypeParams\opt Formals Guard\opt HasResultType\opt MethodBody  \\


\end{bbgrammarappendix}

\begin{bbgrammarappendix}{3.8in}

(\arabic{equation}) & ArgumentList \refstepcounter{equation}\label{prod:ArgumentList}  \: Exp  \\

 &    \| ArgumentList \xcd"," Exp \\

\end{bbgrammarappendix}

\begin{bbgrammarappendix}{4.1in}

(\arabic{equation}) & Arguments \refstepcounter{equation}\label{prod:Arguments}  \: \xcd"(" ArgumentList \xcd")"  \\


\end{bbgrammarappendix}

\begin{bbgrammarappendix}{4.0in}

(\arabic{equation}) & AssertStmt \refstepcounter{equation}\label{prod:AssertStmt}  \: \xcd"assert" Exp \xcd";"  \\

 &    \| \xcd"assert" Exp  \xcd":" Exp  \xcd";" \\

\end{bbgrammarappendix}

\begin{bbgrammarappendix}{3.6in}

(\arabic{equation}) & AssignPropCall \refstepcounter{equation}\label{prod:AssignPropCall}  \: \xcd"property" TypeArgs\opt \xcd"(" ArgumentList\opt \xcd")" \xcd";"  \\


\end{bbgrammarappendix}

\begin{bbgrammarappendix}{4.0in}

(\arabic{equation}) & Assignment \refstepcounter{equation}\label{prod:Assignment}  \: LeftHandSide AsstOp AsstExp  \\

 &    \| ExpName  \xcd"(" ArgumentList\opt \xcd")" AsstOp AsstExp \\
 &    \| Primary  \xcd"(" ArgumentList\opt \xcd")" AsstOp AsstExp \\

\end{bbgrammarappendix}

\begin{bbgrammarappendix}{4.3in}

(\arabic{equation}) & AsstExp \refstepcounter{equation}\label{prod:AsstExp}  \: Assignment  \\

 &    \| ConditionalExp \\

\end{bbgrammarappendix}

\begin{bbgrammarappendix}{4.4in}

(\arabic{equation}) & AsstOp \refstepcounter{equation}\label{prod:AsstOp}  \: \xcd"="  \\

 &    \| \xcd"*=" \\
 &    \| \xcd"/=" \\
 &    \| \xcd"%=" \\
 &    \| \xcd"+=" \\
 &    \| \xcd"-=" \\
 &    \| \xcd"<<=" \\
 &    \| \xcd">>=" \\
 &    \| \xcd">>>=" \\
 &    \| \xcd"&=" \\
 &    \| \xcd"^=" \\
 &    \| \xcd"|=" \\

\end{bbgrammarappendix}

\begin{bbgrammarappendix}{4.1in}

(\arabic{equation}) & AsyncStmt \refstepcounter{equation}\label{prod:AsyncStmt}  \: \xcd"async" ClockedClause\opt Stmt  \\

 &    \| \xcd"clocked" \xcd"async" Stmt \\

\end{bbgrammarappendix}

\begin{bbgrammarappendix}{3.6in}

(\arabic{equation}) & AtCaptureDeclr \refstepcounter{equation}\label{prod:AtCaptureDeclr}  \: Mods\opt VarKeyword\opt VariableDeclr  \\

 &    \| Id \\
 &    \| \xcd"this" \\

\end{bbgrammarappendix}

\begin{bbgrammarappendix}{3.5in}

(\arabic{equation}) & AtCaptureDeclrs \refstepcounter{equation}\label{prod:AtCaptureDeclrs}  \: AtCaptureDeclr  \\

 &    \| AtCaptureDeclrs \xcd"," AtCaptureDeclr \\

\end{bbgrammarappendix}

\begin{bbgrammarappendix}{4.0in}

(\arabic{equation}) & AtEachStmt \refstepcounter{equation}\label{prod:AtEachStmt}  \: \xcd"ateach" \xcd"(" LoopIndex \xcd"in" Exp \xcd")" ClockedClause\opt Stmt  \\

 &    \| \xcd"ateach" \xcd"(" Exp \xcd")" Stmt \\

\end{bbgrammarappendix}

\begin{bbgrammarappendix}{4.5in}

(\arabic{equation}) & AtExp \refstepcounter{equation}\label{prod:AtExp}  \: \xcd"at" \xcd"(" Exp \xcd")" ClosureBody  \\


\end{bbgrammarappendix}

\begin{bbgrammarappendix}{4.4in}

(\arabic{equation}) & AtStmt \refstepcounter{equation}\label{prod:AtStmt}  \: \xcd"at" \xcd"(" Exp \xcd")" Stmt  \\


\end{bbgrammarappendix}

\begin{bbgrammarappendix}{4.0in}

(\arabic{equation}) & AtomicStmt \refstepcounter{equation}\label{prod:AtomicStmt}  \: \xcd"atomic" Stmt  \\


\end{bbgrammarappendix}

\begin{bbgrammarappendix}{3.8in}

(\arabic{equation}) & BasicForStmt \refstepcounter{equation}\label{prod:BasicForStmt}  \: \xcd"for" \xcd"(" ForInit\opt \xcd";" Exp\opt \xcd";" ForUpdate\opt \xcd")" Stmt  \\


\end{bbgrammarappendix}

\begin{bbgrammarappendix}{4.5in}

(\arabic{equation}) & BinOp \refstepcounter{equation}\label{prod:BinOp}  \: \xcd"+"  \\

 &    \| \xcd"-" \\
 &    \| \xcd"*" \\
 &    \| \xcd"/" \\
 &    \| \xcd"%" \\
 &    \| \xcd"&" \\
 &    \| \xcd"|" \\
 &    \| \xcd"^" \\
 &    \| \xcd"&&" \\
 &    \| \xcd"||" \\
 &    \| \xcd"<<" \\
 &    \| \xcd">>" \\
 &    \| \xcd">>>" \\
 &    \| \xcd">=" \\
 &    \| \xcd"<=" \\
 &    \| \xcd">" \\
 &    \| \xcd"<" \\
 &    \| \xcd"==" \\
 &    \| \xcd"!=" \\
 &    \| \xcd".." \\
 &    \| \xcd"->" \\
 &    \| \xcd"<-" \\
 &    \| \xcd"-<" \\
 &    \| \xcd">-" \\
 &    \| \xcd"**" \\
 &    \| \xcd"~" \\
 &    \| \xcd"!~" \\
 &    \| \xcd"!" \\

\end{bbgrammarappendix}

\begin{bbgrammarappendix}{4.0in}

(\arabic{equation}) & BinOpDecln \refstepcounter{equation}\label{prod:BinOpDecln}  \: MethMods \xcd"operator" TypeParams\opt \xcd"(" Formal  \xcd")" BinOp \xcd"(" Formal  \xcd")" Guard\opt HasResultType\opt MethodBody  \\

 &    \| MethMods \xcd"operator" TypeParams\opt \xcd"this" BinOp \xcd"(" Formal  \xcd")" Guard\opt HasResultType\opt MethodBody \\
 &    \| MethMods \xcd"operator" TypeParams\opt \xcd"(" Formal  \xcd")" BinOp \xcd"this" Guard\opt HasResultType\opt MethodBody \\

\end{bbgrammarappendix}

\begin{bbgrammarappendix}{4.5in}

(\arabic{equation}) & Block \refstepcounter{equation}\label{prod:Block}  \: \xcd"{" BlockStmts\opt \xcd"}"  \\


\end{bbgrammarappendix}

\begin{bbgrammarappendix}{3.3in}

(\arabic{equation}) & BlockInteriorStmt \refstepcounter{equation}\label{prod:BlockInteriorStmt}  \: LocVarDeclnStmt  \\

 &    \| ClassDecln \\
 &    \| StructDecln \\
 &    \| TypeDefDecln \\
 &    \| Stmt \\

\end{bbgrammarappendix}

\begin{bbgrammarappendix}{4.0in}

(\arabic{equation}) & BlockStmts \refstepcounter{equation}\label{prod:BlockStmts}  \: BlockInteriorStmt  \\

 &    \| BlockStmts BlockInteriorStmt \\

\end{bbgrammarappendix}

\begin{bbgrammarappendix}{3.6in}

(\arabic{equation}) & BooleanLiteral \refstepcounter{equation}\label{prod:BooleanLiteral}  \: \xcd"true"   \\

 &    \| \xcd"false"  \\

\end{bbgrammarappendix}

\begin{bbgrammarappendix}{4.1in}

(\arabic{equation}) & BreakStmt \refstepcounter{equation}\label{prod:BreakStmt}  \: \xcd"break" Id\opt \xcd";"  \\


\end{bbgrammarappendix}

\begin{bbgrammarappendix}{4.3in}

(\arabic{equation}) & CastExp \refstepcounter{equation}\label{prod:CastExp}  \: Primary  \\

 &    \| ExpName \\
 &    \| CastExp \xcd"as" Type \\

\end{bbgrammarappendix}

\begin{bbgrammarappendix}{3.9in}

(\arabic{equation}) & CatchClause \refstepcounter{equation}\label{prod:CatchClause}  \: \xcd"catch" \xcd"(" Formal \xcd")" Block  \\


\end{bbgrammarappendix}

\begin{bbgrammarappendix}{4.3in}

(\arabic{equation}) & Catches \refstepcounter{equation}\label{prod:Catches}  \: CatchClause  \\

 &    \| Catches CatchClause \\

\end{bbgrammarappendix}

\begin{bbgrammarappendix}{4.1in}

(\arabic{equation}) & ClassBody \refstepcounter{equation}\label{prod:ClassBody}  \: \xcd"{" ClassMemberDeclns\opt \xcd"}"  \\


\end{bbgrammarappendix}

\begin{bbgrammarappendix}{4.0in}

(\arabic{equation}) & ClassDecln \refstepcounter{equation}\label{prod:ClassDecln}  \: Mods\opt \xcd"class" Id TypeParamsI\opt Properties\opt Guard\opt Super\opt Interfaces\opt ClassBody  \\


\end{bbgrammarappendix}

\begin{bbgrammarappendix}{3.4in}

(\arabic{equation}) & ClassMemberDecln \refstepcounter{equation}\label{prod:ClassMemberDecln}  \: InterfaceMemberDecln  \\

 &    \| CtorDecln \\

\end{bbgrammarappendix}

\begin{bbgrammarappendix}{3.3in}

(\arabic{equation}) & ClassMemberDeclns \refstepcounter{equation}\label{prod:ClassMemberDeclns}  \: ClassMemberDecln  \\

 &    \| ClassMemberDeclns ClassMemberDecln \\

\end{bbgrammarappendix}

\begin{bbgrammarappendix}{4.1in}

(\arabic{equation}) & ClassName \refstepcounter{equation}\label{prod:ClassName}  \: TypeName  \\


\end{bbgrammarappendix}

\begin{bbgrammarappendix}{4.1in}

(\arabic{equation}) & ClassType \refstepcounter{equation}\label{prod:ClassType}  \: NamedType  \\


\end{bbgrammarappendix}

\begin{bbgrammarappendix}{3.7in}

(\arabic{equation}) & ClockedClause \refstepcounter{equation}\label{prod:ClockedClause}  \: \xcd"clocked" Arguments  \\


\end{bbgrammarappendix}

\begin{bbgrammarappendix}{3.9in}

(\arabic{equation}) & ClosureBody \refstepcounter{equation}\label{prod:ClosureBody}  \: Exp  \\

 &    \| Annotations\opt \xcd"{" BlockStmts\opt LastExp \xcd"}" \\
 &    \| Annotations\opt Block \\

\end{bbgrammarappendix}

\begin{bbgrammarappendix}{4.0in}

(\arabic{equation}) & ClosureExp \refstepcounter{equation}\label{prod:ClosureExp}  \: Formals Guard\opt HasResultType\opt \xcd"=>" ClosureBody  \\


\end{bbgrammarappendix}

\begin{bbgrammarappendix}{3.5in}

(\arabic{equation}) & CompilationUnit \refstepcounter{equation}\label{prod:CompilationUnit}  \: PackageDecln\opt TypeDeclns\opt  \\

 &    \| PackageDecln\opt ImportDeclns TypeDeclns\opt \\
 &    \| ImportDeclns PackageDecln  ImportDeclns\opt  TypeDeclns\opt \\
 &    \| PackageDecln ImportDeclns PackageDecln  ImportDeclns\opt  TypeDeclns\opt \\

\end{bbgrammarappendix}

\begin{bbgrammarappendix}{3.3in}

(\arabic{equation}) & ConditionalAndExp \refstepcounter{equation}\label{prod:ConditionalAndExp}  \: InclusiveOrExp  \\

 &    \| ConditionalAndExp \xcd"&&" InclusiveOrExp \\

\end{bbgrammarappendix}

\begin{bbgrammarappendix}{3.6in}

(\arabic{equation}) & ConditionalExp \refstepcounter{equation}\label{prod:ConditionalExp}  \: ConditionalOrExp  \\

 &    \| ClosureExp \\
 &    \| AtExp \\
 &    \| ConditionalOrExp \xcd"?" Exp \xcd":" ConditionalExp \\

\end{bbgrammarappendix}

\begin{bbgrammarappendix}{3.4in}

(\arabic{equation}) & ConditionalOrExp \refstepcounter{equation}\label{prod:ConditionalOrExp}  \: ConditionalAndExp  \\

 &    \| ConditionalOrExp \xcd"||" ConditionalAndExp \\

\end{bbgrammarappendix}

\begin{bbgrammarappendix}{3.9in}

(\arabic{equation}) & ConstantExp \refstepcounter{equation}\label{prod:ConstantExp}  \: Exp  \\


\end{bbgrammarappendix}

\begin{bbgrammarappendix}{3.5in}

(\arabic{equation}) & ConstrainedType \refstepcounter{equation}\label{prod:ConstrainedType}  \: NamedType  \\

 &    \| AnnotatedType \\

\end{bbgrammarappendix}

\begin{bbgrammarappendix}{2.9in}

(\arabic{equation}) & ConstraintConjunction \refstepcounter{equation}\label{prod:ConstraintConjunction}  \: Exp  \\

 &    \| ConstraintConjunction \xcd"," Exp \\

\end{bbgrammarappendix}

\begin{bbgrammarappendix}{3.8in}

(\arabic{equation}) & ContinueStmt \refstepcounter{equation}\label{prod:ContinueStmt}  \: \xcd"continue" Id\opt \xcd";"  \\


\end{bbgrammarappendix}

\begin{bbgrammarappendix}{3.3in}

(\arabic{equation}) & ConversionOpDecln \refstepcounter{equation}\label{prod:ConversionOpDecln}  \: ExplConvOpDecln  \\

 &    \| ImplConvOpDecln \\

\end{bbgrammarappendix}

\begin{bbgrammarappendix}{4.1in}

(\arabic{equation}) & CtorBlock \refstepcounter{equation}\label{prod:CtorBlock}  \: \xcd"{" ExplicitCtorInvo\opt BlockStmts\opt \xcd"}"  \\


\end{bbgrammarappendix}

\begin{bbgrammarappendix}{4.2in}

(\arabic{equation}) & CtorBody \refstepcounter{equation}\label{prod:CtorBody}  \: \xcd"=" CtorBlock  \\

 &    \| CtorBlock \\
 &    \| \xcd"=" ExplicitCtorInvo \\
 &    \| \xcd"=" AssignPropCall \\
 &    \| \xcd";" \\

\end{bbgrammarappendix}

\begin{bbgrammarappendix}{4.1in}

(\arabic{equation}) & CtorDecln \refstepcounter{equation}\label{prod:CtorDecln}  \: Mods\opt \xcd"def" \xcd"this" TypeParams\opt Formals Guard\opt HasResultType\opt CtorBody  \\


\end{bbgrammarappendix}

\begin{bbgrammarappendix}{3.8in}

(\arabic{equation}) & DepNamedType \refstepcounter{equation}\label{prod:DepNamedType}  \: SimpleNamedType DepParams  \\

 &    \| ParamizedNamedType DepParams \\

\end{bbgrammarappendix}

\begin{bbgrammarappendix}{4.4in}

(\arabic{equation}) & DoStmt \refstepcounter{equation}\label{prod:DoStmt}  \: \xcd"do" Stmt \xcd"while" \xcd"(" Exp \xcd")" \xcd";"  \\


\end{bbgrammarappendix}

\begin{bbgrammarappendix}{4.1in}

(\arabic{equation}) & EmptyStmt \refstepcounter{equation}\label{prod:EmptyStmt}  \: \xcd";"  \\


\end{bbgrammarappendix}

\begin{bbgrammarappendix}{3.5in}

(\arabic{equation}) & EnhancedForStmt \refstepcounter{equation}\label{prod:EnhancedForStmt}  \: \xcd"for" \xcd"(" LoopIndex \xcd"in" Exp \xcd")" Stmt  \\

 &    \| \xcd"for" \xcd"(" Exp \xcd")" Stmt \\

\end{bbgrammarappendix}

\begin{bbgrammarappendix}{3.9in}

(\arabic{equation}) & EqualityExp \refstepcounter{equation}\label{prod:EqualityExp}  \: RelationalExp  \\

 &    \| EqualityExp \xcd"==" RelationalExp \\
 &    \| EqualityExp \xcd"!=" RelationalExp \\
 &    \| Type  \xcd"==" Type  \\
 &    \| EqualityExp \xcd"~" RelationalExp \\
 &    \| EqualityExp \xcd"!~" RelationalExp \\

\end{bbgrammarappendix}

\begin{bbgrammarappendix}{3.6in}

(\arabic{equation}) & ExclusiveOrExp \refstepcounter{equation}\label{prod:ExclusiveOrExp}  \: AndExp  \\

 &    \| ExclusiveOrExp \xcd"^" AndExp \\

\end{bbgrammarappendix}

\begin{bbgrammarappendix}{4.7in}

(\arabic{equation}) & Exp \refstepcounter{equation}\label{prod:Exp}  \: AsstExp  \\


\end{bbgrammarappendix}

\begin{bbgrammarappendix}{4.3in}

(\arabic{equation}) & ExpName \refstepcounter{equation}\label{prod:ExpName}  \: Id  \\

 &    \| FullyQualifiedName \xcd"." Id \\

\end{bbgrammarappendix}

\begin{bbgrammarappendix}{4.3in}

(\arabic{equation}) & ExpStmt \refstepcounter{equation}\label{prod:ExpStmt}  \: StmtExp \xcd";"  \\


\end{bbgrammarappendix}

\begin{bbgrammarappendix}{3.5in}

(\arabic{equation}) & ExplConvOpDecln \refstepcounter{equation}\label{prod:ExplConvOpDecln}  \: MethMods \xcd"operator" TypeParams\opt \xcd"(" Formal  \xcd")" \xcd"as" Type Guard\opt MethodBody  \\

 &    \| MethMods \xcd"operator" TypeParams\opt \xcd"(" Formal  \xcd")" \xcd"as" \xcd"?" Guard\opt HasResultType\opt MethodBody \\

\end{bbgrammarappendix}

\begin{bbgrammarappendix}{3.4in}

(\arabic{equation}) & ExplicitCtorInvo \refstepcounter{equation}\label{prod:ExplicitCtorInvo}  \: \xcd"this" TypeArgs\opt \xcd"(" ArgumentList\opt \xcd")" \xcd";"  \\

 &    \| \xcd"super" TypeArgs\opt \xcd"(" ArgumentList\opt \xcd")" \xcd";" \\
 &    \| Primary \xcd"." \xcd"this" TypeArgs\opt \xcd"(" ArgumentList\opt \xcd")" \xcd";" \\
 &    \| Primary \xcd"." \xcd"super" TypeArgs\opt \xcd"(" ArgumentList\opt \xcd")" \xcd";" \\

\end{bbgrammarappendix}

\begin{bbgrammarappendix}{3.3in}

(\arabic{equation}) & ExtendsInterfaces \refstepcounter{equation}\label{prod:ExtendsInterfaces}  \: \xcd"extends" Type  \\

 &    \| ExtendsInterfaces \xcd"," Type \\

\end{bbgrammarappendix}

\begin{bbgrammarappendix}{3.9in}

(\arabic{equation}) & FieldAccess \refstepcounter{equation}\label{prod:FieldAccess}  \: Primary \xcd"." Id  \\

 &    \| \xcd"super" \xcd"." Id \\
 &    \| ClassName \xcd"." \xcd"super"  \xcd"." Id \\

\end{bbgrammarappendix}

\begin{bbgrammarappendix}{4.0in}

(\arabic{equation}) & FieldDecln \refstepcounter{equation}\label{prod:FieldDecln}  \: Mods\opt VarKeyword FieldDeclrs \xcd";"  \\

 &    \| Mods\opt FieldDeclrs \xcd";" \\

\end{bbgrammarappendix}

\begin{bbgrammarappendix}{4.0in}

(\arabic{equation}) & FieldDeclr \refstepcounter{equation}\label{prod:FieldDeclr}  \: Id HasResultType  \\

 &    \| Id HasResultType\opt \xcd"=" VariableInitializer \\

\end{bbgrammarappendix}

\begin{bbgrammarappendix}{3.9in}

(\arabic{equation}) & FieldDeclrs \refstepcounter{equation}\label{prod:FieldDeclrs}  \: FieldDeclr  \\

 &    \| FieldDeclrs \xcd"," FieldDeclr \\

\end{bbgrammarappendix}

\begin{bbgrammarappendix}{4.3in}

(\arabic{equation}) & Finally \refstepcounter{equation}\label{prod:Finally}  \: \xcd"finally" Block  \\


\end{bbgrammarappendix}

\begin{bbgrammarappendix}{4.0in}

(\arabic{equation}) & FinishStmt \refstepcounter{equation}\label{prod:FinishStmt}  \: \xcd"finish" Stmt  \\

 &    \| \xcd"clocked" \xcd"finish" Stmt \\

\end{bbgrammarappendix}

\begin{bbgrammarappendix}{4.3in}

(\arabic{equation}) & ForInit \refstepcounter{equation}\label{prod:ForInit}  \: StmtExpList  \\

 &    \| LocVarDecln \\

\end{bbgrammarappendix}

\begin{bbgrammarappendix}{4.3in}

(\arabic{equation}) & ForStmt \refstepcounter{equation}\label{prod:ForStmt}  \: BasicForStmt  \\

 &    \| EnhancedForStmt \\

\end{bbgrammarappendix}

\begin{bbgrammarappendix}{4.1in}

(\arabic{equation}) & ForUpdate \refstepcounter{equation}\label{prod:ForUpdate}  \: StmtExpList  \\


\end{bbgrammarappendix}

\begin{bbgrammarappendix}{4.4in}

(\arabic{equation}) & Formal \refstepcounter{equation}\label{prod:Formal}  \: Mods\opt FormalDeclr  \\

 &    \| Mods\opt VarKeyword FormalDeclr \\
 &    \| Type \\

\end{bbgrammarappendix}

\begin{bbgrammarappendix}{3.9in}

(\arabic{equation}) & FormalDeclr \refstepcounter{equation}\label{prod:FormalDeclr}  \: Id ResultType  \\

 &    \| \xcd"[" IdList \xcd"]" ResultType \\
 &    \| Id \xcd"[" IdList \xcd"]" ResultType \\

\end{bbgrammarappendix}

\begin{bbgrammarappendix}{3.8in}

(\arabic{equation}) & FormalDeclrs \refstepcounter{equation}\label{prod:FormalDeclrs}  \: FormalDeclr  \\

 &    \| FormalDeclrs \xcd"," FormalDeclr \\

\end{bbgrammarappendix}

\begin{bbgrammarappendix}{4.0in}

(\arabic{equation}) & FormalList \refstepcounter{equation}\label{prod:FormalList}  \: Formal  \\

 &    \| FormalList \xcd"," Formal \\

\end{bbgrammarappendix}

\begin{bbgrammarappendix}{4.3in}

(\arabic{equation}) & Formals \refstepcounter{equation}\label{prod:Formals}  \: \xcd"(" FormalList\opt \xcd")"  \\


\end{bbgrammarappendix}

\begin{bbgrammarappendix}{3.2in}

(\arabic{equation}) & FullyQualifiedName \refstepcounter{equation}\label{prod:FullyQualifiedName}  \: Id  \\

 &    \| FullyQualifiedName \xcd"." Id \\

\end{bbgrammarappendix}

\begin{bbgrammarappendix}{3.8in}

(\arabic{equation}) & FunctionType \refstepcounter{equation}\label{prod:FunctionType}  \: TypeParams\opt \xcd"(" FormalList\opt \xcd")" Guard\opt \xcd"=>" Type  \\


\end{bbgrammarappendix}

\begin{bbgrammarappendix}{4.5in}

(\arabic{equation}) & Guard \refstepcounter{equation}\label{prod:Guard}  \: DepParams  \\


\end{bbgrammarappendix}


\begin{bbgrammarappendix}{4.5in}

(\arabic{equation}) & Throws \refstepcounter{equation}\label{prod:Throws}  \: \xcd"throws" ThrowsList  \\

\end{bbgrammarappendix}

\begin{bbgrammarappendix}{3.3in}

(\arabic{equation}) & ThrowsList \refstepcounter{equation}\label{prod:ThrowsList}  \: Type  \\

 &    \| ThrowsList \xcd"," Type \\

\end{bbgrammarappendix}

\begin{bbgrammarappendix}{3.7in}

(\arabic{equation}) & HasResultType \refstepcounter{equation}\label{prod:HasResultType}  \: ResultType  \\

 &    \| \xcd"<:" Type \\

\end{bbgrammarappendix}

\begin{bbgrammarappendix}{3.3in}

(\arabic{equation}) & HasZeroConstraint \refstepcounter{equation}\label{prod:HasZeroConstraint}  \: Type  \xcd"haszero"  \\


\end{bbgrammarappendix}

\begin{bbgrammarappendix}{3.8in}

(\arabic{equation}) & HomeVariable \refstepcounter{equation}\label{prod:HomeVariable}  \: Id  \\

 &    \| \xcd"this" \\

\end{bbgrammarappendix}

\begin{bbgrammarappendix}{3.4in}

(\arabic{equation}) & HomeVariableList \refstepcounter{equation}\label{prod:HomeVariableList}  \: HomeVariable  \\

 &    \| HomeVariableList \xcd"," HomeVariable \\

\end{bbgrammarappendix}

\begin{bbgrammarappendix}{4.8in}

(\arabic{equation}) & Id \refstepcounter{equation}\label{prod:Id}  \: \xcd"IDENTIFIER"   \\


\end{bbgrammarappendix}

\begin{bbgrammarappendix}{4.4in}

(\arabic{equation}) & IdList \refstepcounter{equation}\label{prod:IdList}  \: Id  \\

 &    \| IdList \xcd"," Id \\

\end{bbgrammarappendix}

\begin{bbgrammarappendix}{3.6in}

(\arabic{equation}) & IfThenElseStmt \refstepcounter{equation}\label{prod:IfThenElseStmt}  \: \xcd"if" \xcd"(" Exp \xcd")" Stmt  \xcd"else" Stmt   \\


\end{bbgrammarappendix}

\begin{bbgrammarappendix}{4.0in}

(\arabic{equation}) & IfThenStmt \refstepcounter{equation}\label{prod:IfThenStmt}  \: \xcd"if" \xcd"(" Exp \xcd")" Stmt  \\


\end{bbgrammarappendix}

\begin{bbgrammarappendix}{3.5in}

(\arabic{equation}) & ImplConvOpDecln \refstepcounter{equation}\label{prod:ImplConvOpDecln}  \: MethMods \xcd"operator" TypeParams\opt \xcd"(" Formal  \xcd")" Guard\opt HasResultType\opt MethodBody  \\


\end{bbgrammarappendix}

\begin{bbgrammarappendix}{3.9in}

(\arabic{equation}) & ImportDecln \refstepcounter{equation}\label{prod:ImportDecln}  \: SingleTypeImportDecln  \\

 &    \| TypeImportOnDemandDecln \\

\end{bbgrammarappendix}

\begin{bbgrammarappendix}{3.8in}

(\arabic{equation}) & ImportDeclns \refstepcounter{equation}\label{prod:ImportDeclns}  \: ImportDecln  \\

 &    \| ImportDeclns ImportDecln \\

\end{bbgrammarappendix}

\begin{bbgrammarappendix}{3.6in}

(\arabic{equation}) & InclusiveOrExp \refstepcounter{equation}\label{prod:InclusiveOrExp}  \: ExclusiveOrExp  \\

 &    \| InclusiveOrExp \xcd"|" ExclusiveOrExp \\

\end{bbgrammarappendix}

\begin{bbgrammarappendix}{3.7in}

(\arabic{equation}) & InterfaceBody \refstepcounter{equation}\label{prod:InterfaceBody}  \: \xcd"{" InterfaceMemberDeclns\opt \xcd"}"  \\


\end{bbgrammarappendix}

\begin{bbgrammarappendix}{3.6in}

(\arabic{equation}) & InterfaceDecln \refstepcounter{equation}\label{prod:InterfaceDecln}  \: Mods\opt \xcd"interface" Id TypeParamsI\opt Properties\opt Guard\opt ExtendsInterfaces\opt InterfaceBody  \\


\end{bbgrammarappendix}

\begin{bbgrammarappendix}{3.0in}

(\arabic{equation}) & InterfaceMemberDecln \refstepcounter{equation}\label{prod:InterfaceMemberDecln}  \: MethodDecln  \\

 &    \| PropMethodDecln \\
 &    \| FieldDecln \\
 &    \| TypeDecln \\

\end{bbgrammarappendix}

\begin{bbgrammarappendix}{2.9in}

(\arabic{equation}) & InterfaceMemberDeclns \refstepcounter{equation}\label{prod:InterfaceMemberDeclns}  \: InterfaceMemberDecln  \\

 &    \| InterfaceMemberDeclns InterfaceMemberDecln \\

\end{bbgrammarappendix}

\begin{bbgrammarappendix}{3.3in}

(\arabic{equation}) & InterfaceTypeList \refstepcounter{equation}\label{prod:InterfaceTypeList}  \: Type  \\

 &    \| InterfaceTypeList \xcd"," Type \\

\end{bbgrammarappendix}

\begin{bbgrammarappendix}{4.0in}

(\arabic{equation}) & Interfaces \refstepcounter{equation}\label{prod:Interfaces}  \: \xcd"implements" InterfaceTypeList  \\


\end{bbgrammarappendix}

\begin{bbgrammarappendix}{3.9in}

(\arabic{equation}) & LabeledStmt \refstepcounter{equation}\label{prod:LabeledStmt}  \: Id \xcd":" LoopStmt  \\


\end{bbgrammarappendix}

\begin{bbgrammarappendix}{4.3in}

(\arabic{equation}) & LastExp \refstepcounter{equation}\label{prod:LastExp}  \: Exp  \\


\end{bbgrammarappendix}

\begin{bbgrammarappendix}{3.8in}

(\arabic{equation}) & LeftHandSide \refstepcounter{equation}\label{prod:LeftHandSide}  \: ExpName  \\

 &    \| FieldAccess \\

\end{bbgrammarappendix}

\begin{bbgrammarappendix}{4.3in}

(\arabic{equation}) & Literal \refstepcounter{equation}\label{prod:Literal}  \: \xcd"IntegerLiteral"   \\

 &    \| \xcd"LongLiteral"  \\
 &    \| \xcd"ByteLiteral" \\
 &    \| \xcd"UnsignedByteLiteral" \\
 &    \| \xcd"ShortLiteral" \\
 &    \| \xcd"UnsignedShortLiteral" \\
 &    \| \xcd"UnsignedIntegerLiteral"  \\
 &    \| \xcd"UnsignedLongLiteral"  \\
 &    \| \xcd"FloatingPointLiteral"  \\
 &    \| \xcd"DoubleLiteral"  \\
 &    \| BooleanLiteral \\
 &    \| \xcd"CharacterLiteral"  \\
 &    \| \xcd"StringLiteral"  \\
 &    \| \xcd"null" \\

\end{bbgrammarappendix}

\begin{bbgrammarappendix}{3.9in}

(\arabic{equation}) & LocVarDecln \refstepcounter{equation}\label{prod:LocVarDecln}  \: Mods\opt VarKeyword VariableDeclrs  \\

 &    \| Mods\opt VarDeclsWType \\
 &    \| Mods\opt VarKeyword FormalDeclrs \\

\end{bbgrammarappendix}

\begin{bbgrammarappendix}{3.5in}

(\arabic{equation}) & LocVarDeclnStmt \refstepcounter{equation}\label{prod:LocVarDeclnStmt}  \: LocVarDecln \xcd";"  \\


\end{bbgrammarappendix}

\begin{bbgrammarappendix}{4.1in}

(\arabic{equation}) & LoopIndex \refstepcounter{equation}\label{prod:LoopIndex}  \: Mods\opt LoopIndexDeclr  \\

 &    \| Mods\opt VarKeyword LoopIndexDeclr \\

\end{bbgrammarappendix}

\begin{bbgrammarappendix}{3.6in}

(\arabic{equation}) & LoopIndexDeclr \refstepcounter{equation}\label{prod:LoopIndexDeclr}  \: Id HasResultType\opt  \\

 &    \| \xcd"[" IdList \xcd"]" HasResultType\opt \\
 &    \| Id \xcd"[" IdList \xcd"]" HasResultType\opt \\

\end{bbgrammarappendix}

\begin{bbgrammarappendix}{4.2in}

(\arabic{equation}) & LoopStmt \refstepcounter{equation}\label{prod:LoopStmt}  \: ForStmt  \\

 &    \| WhileStmt \\
 &    \| DoStmt \\
 &    \| AtEachStmt \\

\end{bbgrammarappendix}

\begin{bbgrammarappendix}{4.2in}

(\arabic{equation}) & MethMods \refstepcounter{equation}\label{prod:MethMods}  \: Mods\opt  \\

 &    \| MethMods \xcd"property"  \\
 &    \| MethMods Mod \\

\end{bbgrammarappendix}

\begin{bbgrammarappendix}{4.0in}

(\arabic{equation}) & MethodBody \refstepcounter{equation}\label{prod:MethodBody}  \: \xcd"=" LastExp \xcd";"  \\

 &    \| \xcd"=" Annotations\opt \xcd"{" BlockStmts\opt LastExp \xcd"}" \\
 &    \| \xcd"=" Annotations\opt Block \\
 &    \| Annotations\opt Block \\
 &    \| \xcd";" \\

\end{bbgrammarappendix}

\begin{bbgrammarappendix}{3.9in}

(\arabic{equation}) & MethodDecln
  \refstepcounter{equation}\label{prod:MethodDecln}  \: MethMods \xcd"def" Id TypeParams\opt Formals Guard\opt Throws\opt HasResultType\opt MethodBody  \\

 &    \| BinOpDecln \\
 &    \| PrefixOpDecln \\
 &    \| ApplyOpDecln \\
 &    \| SetOpDecln \\
 &    \| ConversionOpDecln \\

\end{bbgrammarappendix}

\begin{bbgrammarappendix}{4.0in}

(\arabic{equation}) & MethodInvo \refstepcounter{equation}\label{prod:MethodInvo}  \: MethodName TypeArgs\opt \xcd"(" ArgumentList\opt \xcd")"  \\

 &    \| Primary \xcd"." Id TypeArgs\opt \xcd"(" ArgumentList\opt \xcd")" \\
 &    \| \xcd"super" \xcd"." Id TypeArgs\opt \xcd"(" ArgumentList\opt \xcd")" \\
 &    \| ClassName \xcd"." \xcd"super"  \xcd"." Id TypeArgs\opt \xcd"(" ArgumentList\opt \xcd")" \\
 &    \| Primary TypeArgs\opt \xcd"(" ArgumentList\opt \xcd")" \\

\end{bbgrammarappendix}

\begin{bbgrammarappendix}{4.0in}

(\arabic{equation}) & MethodName \refstepcounter{equation}\label{prod:MethodName}  \: Id  \\

 &    \| FullyQualifiedName \xcd"." Id \\

\end{bbgrammarappendix}

\begin{bbgrammarappendix}{4.7in}

(\arabic{equation}) & Mod \refstepcounter{equation}\label{prod:Mod}  \: \xcd"abstract"  \\

 &    \| Annotation \\
 &    \| \xcd"atomic" \\
 &    \| \xcd"final" \\
 &    \| \xcd"native" \\
 &    \| \xcd"private" \\
 &    \| \xcd"protected" \\
 &    \| \xcd"public" \\
 &    \| \xcd"static" \\
 &    \| \xcd"transient" \\
 &    \| \xcd"clocked" \\

\end{bbgrammarappendix}

\begin{bbgrammarappendix}{3.3in}

(\arabic{equation}) & MultiplicativeExp \refstepcounter{equation}\label{prod:MultiplicativeExp}  \: RangeExp  \\

 &    \| MultiplicativeExp \xcd"*" RangeExp \\
 &    \| MultiplicativeExp \xcd"/" RangeExp \\
 &    \| MultiplicativeExp \xcd"%" RangeExp \\
 &    \| MultiplicativeExp \xcd"**" RangeExp \\

\end{bbgrammarappendix}

\begin{bbgrammarappendix}{4.1in}

(\arabic{equation}) & NamedType \refstepcounter{equation}\label{prod:NamedType}  \: NamedTypeNoConstraints  \\

 &    \| DepNamedType \\

\end{bbgrammarappendix}

\begin{bbgrammarappendix}{2.8in}

(\arabic{equation}) & NamedTypeNoConstraints \refstepcounter{equation}\label{prod:NamedTypeNoConstraints}  \: SimpleNamedType  \\

 &    \| ParamizedNamedType \\

\end{bbgrammarappendix}

\begin{bbgrammarappendix}{4.0in}

(\arabic{equation}) & NonExpStmt \refstepcounter{equation}\label{prod:NonExpStmt}  \: Block  \\

 &    \| EmptyStmt \\
 &    \| AssertStmt \\
 &    \| SwitchStmt \\
 &    \| DoStmt \\
 &    \| BreakStmt \\
 &    \| ContinueStmt \\
 &    \| ReturnStmt \\
 &    \| ThrowStmt \\
 &    \| TryStmt \\
 &    \| LabeledStmt \\
 &    \| IfThenStmt \\
 &    \| IfThenElseStmt \\
 &    \| WhileStmt \\
 &    \| ForStmt \\
 &    \| AsyncStmt \\
 &    \| AtStmt \\
 &    \| AtomicStmt \\
 &    \| WhenStmt \\
 &    \| AtEachStmt \\
 &    \| FinishStmt \\
 &    \| AssignPropCall \\

\end{bbgrammarappendix}

\begin{bbgrammarappendix}{3.7in}

(\arabic{equation}) & ObCreationExp \refstepcounter{equation}\label{prod:ObCreationExp}  \: \xcd"new" TypeName TypeArgs\opt \xcd"(" ArgumentList\opt \xcd")" ClassBody\opt  \\

 &    \| Primary \xcd"." \xcd"new" Id TypeArgs\opt \xcd"(" ArgumentList\opt \xcd")" ClassBody\opt \\
 &    \| FullyQualifiedName \xcd"." \xcd"new" Id TypeArgs\opt \xcd"(" ArgumentList\opt \xcd")" ClassBody\opt \\

\end{bbgrammarappendix}

\begin{bbgrammarappendix}{3.8in}

(\arabic{equation}) & PackageDecln \refstepcounter{equation}\label{prod:PackageDecln}  \: Annotations\opt \xcd"package" PackageName \xcd";"  \\


\end{bbgrammarappendix}

\begin{bbgrammarappendix}{3.9in}

(\arabic{equation}) & PackageName \refstepcounter{equation}\label{prod:PackageName}  \: Id  \\

 &    \| PackageName \xcd"." Id \\

\end{bbgrammarappendix}

\begin{bbgrammarappendix}{3.3in}

(\arabic{equation}) & PackageOrTypeName \refstepcounter{equation}\label{prod:PackageOrTypeName}  \: Id  \\

 &    \| PackageOrTypeName \xcd"." Id \\

\end{bbgrammarappendix}

\begin{bbgrammarappendix}{3.2in}

(\arabic{equation}) & ParamizedNamedType \refstepcounter{equation}\label{prod:ParamizedNamedType}  \: SimpleNamedType Arguments  \\

 &    \| SimpleNamedType TypeArgs \\
 &    \| SimpleNamedType TypeArgs Arguments \\

\end{bbgrammarappendix}

\begin{bbgrammarappendix}{3.4in}

(\arabic{equation}) & PostDecrementExp \refstepcounter{equation}\label{prod:PostDecrementExp}  \: PostfixExp \xcd"--"  \\


\end{bbgrammarappendix}

\begin{bbgrammarappendix}{3.4in}

(\arabic{equation}) & PostIncrementExp \refstepcounter{equation}\label{prod:PostIncrementExp}  \: PostfixExp \xcd"++"  \\


\end{bbgrammarappendix}

\begin{bbgrammarappendix}{4.0in}

(\arabic{equation}) & PostfixExp \refstepcounter{equation}\label{prod:PostfixExp}  \: CastExp  \\

 &    \| PostIncrementExp \\
 &    \| PostDecrementExp \\

\end{bbgrammarappendix}

\begin{bbgrammarappendix}{3.5in}

(\arabic{equation}) & PreDecrementExp \refstepcounter{equation}\label{prod:PreDecrementExp}  \: \xcd"--" UnaryExpNotPlusMinus  \\


\end{bbgrammarappendix}

\begin{bbgrammarappendix}{3.5in}

(\arabic{equation}) & PreIncrementExp \refstepcounter{equation}\label{prod:PreIncrementExp}  \: \xcd"++" UnaryExpNotPlusMinus  \\


\end{bbgrammarappendix}

\begin{bbgrammarappendix}{4.2in}

(\arabic{equation}) & PrefixOp \refstepcounter{equation}\label{prod:PrefixOp}  \: \xcd"+"  \\

 &    \| \xcd"-" \\
 &    \| \xcd"!" \\
 &    \| \xcd"~" \\
 &    \| \xcd"^" \\
 &    \| \xcd"|" \\
 &    \| \xcd"&" \\
 &    \| \xcd"*" \\
 &    \| \xcd"/" \\
 &    \| \xcd"%" \\

\end{bbgrammarappendix}

\begin{bbgrammarappendix}{3.7in}

(\arabic{equation}) & PrefixOpDecln \refstepcounter{equation}\label{prod:PrefixOpDecln}  \: MethMods \xcd"operator" TypeParams\opt PrefixOp \xcd"(" Formal  \xcd")" Guard\opt HasResultType\opt MethodBody  \\

 &    \| MethMods \xcd"operator" TypeParams\opt PrefixOp \xcd"this" Guard\opt HasResultType\opt MethodBody \\

\end{bbgrammarappendix}

\begin{bbgrammarappendix}{4.3in}

(\arabic{equation}) & Primary \refstepcounter{equation}\label{prod:Primary}  \: \xcd"here"  \\

 &    \| \xcd"[" ArgumentList\opt \xcd"]" \\
 &    \| Literal \\
 &    \| \xcd"self" \\
 &    \| \xcd"this" \\
 &    \| ClassName \xcd"." \xcd"this" \\
 &    \| \xcd"(" Exp \xcd")" \\
 &    \| ObCreationExp \\
 &    \| FieldAccess \\
 &    \| MethodInvo \\

\end{bbgrammarappendix}

\begin{bbgrammarappendix}{4.6in}

(\arabic{equation}) & Prop \refstepcounter{equation}\label{prod:Prop}  \: Annotations\opt Id ResultType  \\


\end{bbgrammarappendix}

\begin{bbgrammarappendix}{4.2in}

(\arabic{equation}) & PropList \refstepcounter{equation}\label{prod:PropList}  \: Prop  \\

 &    \| PropList \xcd"," Prop \\

\end{bbgrammarappendix}

\begin{bbgrammarappendix}{3.5in}

(\arabic{equation}) & PropMethodDecln \refstepcounter{equation}\label{prod:PropMethodDecln}  \: MethMods Id TypeParams\opt Formals Guard\opt HasResultType\opt MethodBody  \\

 &    \| MethMods Id Guard\opt HasResultType\opt MethodBody \\

\end{bbgrammarappendix}

\begin{bbgrammarappendix}{4.0in}

(\arabic{equation}) & Properties \refstepcounter{equation}\label{prod:Properties}  \: \xcd"(" PropList \xcd")"  \\


\end{bbgrammarappendix}

\begin{bbgrammarappendix}{4.2in}

(\arabic{equation}) & RangeExp \refstepcounter{equation}\label{prod:RangeExp}  \: UnaryExp  \\

 &    \| RangeExp  \xcd".." UnaryExp  \\

\end{bbgrammarappendix}

\begin{bbgrammarappendix}{3.7in}

(\arabic{equation}) & RelationalExp \refstepcounter{equation}\label{prod:RelationalExp}  \: ShiftExp  \\

 &    \| HasZeroConstraint \\
 &    \| SubtypeConstraint \\
 &    \| RelationalExp \xcd"<" ShiftExp \\
 &    \| RelationalExp \xcd">" ShiftExp \\
 &    \| RelationalExp \xcd"<=" ShiftExp \\
 &    \| RelationalExp \xcd">=" ShiftExp \\
 &    \| RelationalExp \xcd"instanceof" Type \\

\end{bbgrammarappendix}

\begin{bbgrammarappendix}{4.0in}

(\arabic{equation}) & ResultType \refstepcounter{equation}\label{prod:ResultType}  \: \xcd":" Type  \\


\end{bbgrammarappendix}

\begin{bbgrammarappendix}{4.0in}

(\arabic{equation}) & ReturnStmt \refstepcounter{equation}\label{prod:ReturnStmt}  \: \xcd"return" Exp\opt \xcd";"  \\


\end{bbgrammarappendix}

\begin{bbgrammarappendix}{4.0in}

(\arabic{equation}) & SetOpDecln \refstepcounter{equation}\label{prod:SetOpDecln}  \: MethMods \xcd"operator" \xcd"this" TypeParams\opt Formals \xcd"=" \xcd"(" Formal  \xcd")" Guard\opt HasResultType\opt MethodBody  \\


\end{bbgrammarappendix}

\begin{bbgrammarappendix}{4.2in}

(\arabic{equation}) & ShiftExp \refstepcounter{equation}\label{prod:ShiftExp}  \: AdditiveExp  \\

 &    \| ShiftExp \xcd"<<" AdditiveExp \\
 &    \| ShiftExp \xcd">>" AdditiveExp \\
 &    \| ShiftExp \xcd">>>" AdditiveExp \\
 &    \| ShiftExp  \xcd"->" AdditiveExp  \\
 &    \| ShiftExp  \xcd"<-" AdditiveExp  \\
 &    \| ShiftExp  \xcd"-<" AdditiveExp  \\
 &    \| ShiftExp  \xcd">-" AdditiveExp  \\
 &    \| ShiftExp  \xcd"!" AdditiveExp  \\

\end{bbgrammarappendix}

\begin{bbgrammarappendix}{3.5in}

(\arabic{equation}) & SimpleNamedType \refstepcounter{equation}\label{prod:SimpleNamedType}  \: TypeName  \\

 &    \| Primary \xcd"." Id \\
 &    \| ParamizedNamedType \xcd"." Id \\
 &    \| DepNamedType \xcd"." Id \\

\end{bbgrammarappendix}

\begin{bbgrammarappendix}{2.9in}

(\arabic{equation}) & SingleTypeImportDecln \refstepcounter{equation}\label{prod:SingleTypeImportDecln}  \: \xcd"import" TypeName \xcd";"  \\


\end{bbgrammarappendix}

\begin{bbgrammarappendix}{4.6in}

(\arabic{equation}) & Stmt \refstepcounter{equation}\label{prod:Stmt}  \: AnnotationStmt  \\

 &    \| ExpStmt \\

\end{bbgrammarappendix}

\begin{bbgrammarappendix}{4.3in}

(\arabic{equation}) & StmtExp \refstepcounter{equation}\label{prod:StmtExp}  \: Assignment  \\

 &    \| PreIncrementExp \\
 &    \| PreDecrementExp \\
 &    \| PostIncrementExp \\
 &    \| PostDecrementExp \\
 &    \| MethodInvo \\
 &    \| ObCreationExp \\

\end{bbgrammarappendix}

\begin{bbgrammarappendix}{3.9in}

(\arabic{equation}) & StmtExpList \refstepcounter{equation}\label{prod:StmtExpList}  \: StmtExp  \\

 &    \| StmtExpList \xcd"," StmtExp \\

\end{bbgrammarappendix}

\begin{bbgrammarappendix}{3.9in}

(\arabic{equation}) & StructDecln \refstepcounter{equation}\label{prod:StructDecln}  \: Mods\opt \xcd"struct" Id TypeParamsI\opt Properties\opt Guard\opt Interfaces\opt ClassBody  \\


\end{bbgrammarappendix}

\begin{bbgrammarappendix}{3.3in}

(\arabic{equation}) & SubtypeConstraint \refstepcounter{equation}\label{prod:SubtypeConstraint}  \: Type  \xcd"<:" Type   \\

 &    \| Type  \xcd":>" Type  \\

\end{bbgrammarappendix}

\begin{bbgrammarappendix}{4.5in}

(\arabic{equation}) & Super \refstepcounter{equation}\label{prod:Super}  \: \xcd"extends" ClassType  \\


\end{bbgrammarappendix}

\begin{bbgrammarappendix}{3.9in}

(\arabic{equation}) & SwitchBlock \refstepcounter{equation}\label{prod:SwitchBlock}  \: \xcd"{" SwitchBlockGroups\opt SwitchLabels\opt \xcd"}"  \\


\end{bbgrammarappendix}

\begin{bbgrammarappendix}{3.4in}

(\arabic{equation}) & SwitchBlockGroup \refstepcounter{equation}\label{prod:SwitchBlockGroup}  \: SwitchLabels BlockStmts  \\


\end{bbgrammarappendix}

\begin{bbgrammarappendix}{3.3in}

(\arabic{equation}) & SwitchBlockGroups \refstepcounter{equation}\label{prod:SwitchBlockGroups}  \: SwitchBlockGroup  \\

 &    \| SwitchBlockGroups SwitchBlockGroup \\

\end{bbgrammarappendix}

\begin{bbgrammarappendix}{3.9in}

(\arabic{equation}) & SwitchLabel \refstepcounter{equation}\label{prod:SwitchLabel}  \: \xcd"case" ConstantExp \xcd":"  \\

 &    \| \xcd"default" \xcd":" \\

\end{bbgrammarappendix}

\begin{bbgrammarappendix}{3.8in}

(\arabic{equation}) & SwitchLabels \refstepcounter{equation}\label{prod:SwitchLabels}  \: SwitchLabel  \\

 &    \| SwitchLabels SwitchLabel \\

\end{bbgrammarappendix}

\begin{bbgrammarappendix}{4.0in}

(\arabic{equation}) & SwitchStmt \refstepcounter{equation}\label{prod:SwitchStmt}  \: \xcd"switch" \xcd"(" Exp \xcd")" SwitchBlock  \\


\end{bbgrammarappendix}

\begin{bbgrammarappendix}{4.1in}

(\arabic{equation}) & ThrowStmt \refstepcounter{equation}\label{prod:ThrowStmt}  \: \xcd"throw" Exp \xcd";"  \\


\end{bbgrammarappendix}

\begin{bbgrammarappendix}{4.3in}

(\arabic{equation}) & TryStmt \refstepcounter{equation}\label{prod:TryStmt}  \: \xcd"try" Block Catches  \\

 &    \| \xcd"try" Block Catches\opt Finally \\

\end{bbgrammarappendix}

\begin{bbgrammarappendix}{4.6in}

(\arabic{equation}) & Type \refstepcounter{equation}\label{prod:Type}  \: FunctionType  \\

 &    \| ConstrainedType \\
 &    \| Void \\

\end{bbgrammarappendix}

\begin{bbgrammarappendix}{4.2in}

(\arabic{equation}) & TypeArgs \refstepcounter{equation}\label{prod:TypeArgs}  \: \xcd"[" TypeArgumentList \xcd"]"  \\


\end{bbgrammarappendix}

\begin{bbgrammarappendix}{3.4in}

(\arabic{equation}) & TypeArgumentList \refstepcounter{equation}\label{prod:TypeArgumentList}  \: Type  \\

 &    \| TypeArgumentList \xcd"," Type \\

\end{bbgrammarappendix}

\begin{bbgrammarappendix}{4.1in}

(\arabic{equation}) & TypeDecln \refstepcounter{equation}\label{prod:TypeDecln}  \: ClassDecln  \\

 &    \| StructDecln \\
 &    \| InterfaceDecln \\
 &    \| TypeDefDecln \\
 &    \| \xcd";" \\

\end{bbgrammarappendix}

\begin{bbgrammarappendix}{4.0in}

(\arabic{equation}) & TypeDeclns \refstepcounter{equation}\label{prod:TypeDeclns}  \: TypeDecln  \\

 &    \| TypeDeclns TypeDecln \\

\end{bbgrammarappendix}

\begin{bbgrammarappendix}{3.8in}

(\arabic{equation}) & TypeDefDecln \refstepcounter{equation}\label{prod:TypeDefDecln}  \: Mods\opt \xcd"type" Id TypeParams\opt Guard\opt \xcd"=" Type \xcd";"  \\

 &    \| Mods\opt \xcd"type" Id TypeParams\opt \xcd"(" FormalList \xcd")" Guard\opt \xcd"=" Type \xcd";" \\

\end{bbgrammarappendix}

\begin{bbgrammarappendix}{2.7in}

(\arabic{equation}) & TypeImportOnDemandDecln \refstepcounter{equation}\label{prod:TypeImportOnDemandDecln}  \: \xcd"import" PackageOrTypeName \xcd"." \xcd"*" \xcd";"  \\


\end{bbgrammarappendix}

\begin{bbgrammarappendix}{4.2in}

(\arabic{equation}) & TypeName \refstepcounter{equation}\label{prod:TypeName}  \: Id  \\

 &    \| TypeName \xcd"." Id \\

\end{bbgrammarappendix}

\begin{bbgrammarappendix}{4.1in}

(\arabic{equation}) & TypeParam \refstepcounter{equation}\label{prod:TypeParam}  \: Id  \\


\end{bbgrammarappendix}

\begin{bbgrammarappendix}{3.6in}

(\arabic{equation}) & TypeParamIList \refstepcounter{equation}\label{prod:TypeParamIList}  \: TypeParam  \\

 &    \| TypeParamIList \xcd"," TypeParam \\
 &    \| TypeParamIList \xcd"," \\

\end{bbgrammarappendix}

\begin{bbgrammarappendix}{3.7in}

(\arabic{equation}) & TypeParamList \refstepcounter{equation}\label{prod:TypeParamList}  \: TypeParam  \\

 &    \| TypeParamList \xcd"," TypeParam \\

\end{bbgrammarappendix}

\begin{bbgrammarappendix}{4.0in}

(\arabic{equation}) & TypeParams \refstepcounter{equation}\label{prod:TypeParams}  \: \xcd"[" TypeParamList \xcd"]"  \\


\end{bbgrammarappendix}

\begin{bbgrammarappendix}{3.9in}

(\arabic{equation}) & TypeParamsI \refstepcounter{equation}\label{prod:TypeParamsI}  \: \xcd"[" TypeParamIList \xcd"]"  \\


\end{bbgrammarappendix}

\begin{bbgrammarappendix}{3.1in}

(\arabic{equation}) & UnannotatedUnaryExp \refstepcounter{equation}\label{prod:UnannotatedUnaryExp}  \: PreIncrementExp  \\

 &    \| PreDecrementExp \\
 &    \| \xcd"+" UnaryExpNotPlusMinus \\
 &    \| \xcd"-" UnaryExpNotPlusMinus \\
 &    \| UnaryExpNotPlusMinus \\

\end{bbgrammarappendix}

\begin{bbgrammarappendix}{4.2in}

(\arabic{equation}) & UnaryExp \refstepcounter{equation}\label{prod:UnaryExp}  \: UnannotatedUnaryExp  \\

 &    \| Annotations UnannotatedUnaryExp \\

\end{bbgrammarappendix}

\begin{bbgrammarappendix}{3.0in}

(\arabic{equation}) & UnaryExpNotPlusMinus \refstepcounter{equation}\label{prod:UnaryExpNotPlusMinus}  \: PostfixExp  \\

 &    \| \xcd"~" UnaryExp \\
 &    \| \xcd"!" UnaryExp \\
 &    \| \xcd"^" UnaryExp \\
 &    \| \xcd"|" UnaryExp \\
 &    \| \xcd"&" UnaryExp \\
 &    \| \xcd"*" UnaryExp \\
 &    \| \xcd"/" UnaryExp \\
 &    \| \xcd"%" UnaryExp \\

\end{bbgrammarappendix}

\begin{bbgrammarappendix}{3.8in}

(\arabic{equation}) & VarDeclWType \refstepcounter{equation}\label{prod:VarDeclWType}  \: Id HasResultType \xcd"=" VariableInitializer  \\

 &    \| \xcd"[" IdList \xcd"]" HasResultType \xcd"=" VariableInitializer \\
 &    \| Id \xcd"[" IdList \xcd"]" HasResultType \xcd"=" VariableInitializer \\

\end{bbgrammarappendix}

\begin{bbgrammarappendix}{3.7in}

(\arabic{equation}) & VarDeclsWType \refstepcounter{equation}\label{prod:VarDeclsWType}  \: VarDeclWType  \\

 &    \| VarDeclsWType \xcd"," VarDeclWType \\

\end{bbgrammarappendix}

\begin{bbgrammarappendix}{4.0in}

(\arabic{equation}) & VarKeyword \refstepcounter{equation}\label{prod:VarKeyword}  \: \xcd"val"  \\

 &    \| \xcd"var" \\

\end{bbgrammarappendix}

\begin{bbgrammarappendix}{3.7in}

(\arabic{equation}) & VariableDeclr \refstepcounter{equation}\label{prod:VariableDeclr}  \: Id HasResultType\opt \xcd"=" VariableInitializer  \\

 &    \| \xcd"[" IdList \xcd"]" HasResultType\opt \xcd"=" VariableInitializer \\
 &    \| Id \xcd"[" IdList \xcd"]" HasResultType\opt \xcd"=" VariableInitializer \\

\end{bbgrammarappendix}

\begin{bbgrammarappendix}{3.6in}

(\arabic{equation}) & VariableDeclrs \refstepcounter{equation}\label{prod:VariableDeclrs}  \: VariableDeclr  \\

 &    \| VariableDeclrs \xcd"," VariableDeclr \\

\end{bbgrammarappendix}

\begin{bbgrammarappendix}{3.1in}

(\arabic{equation}) & VariableInitializer \refstepcounter{equation}\label{prod:VariableInitializer}  \: Exp  \\


\end{bbgrammarappendix}

\begin{bbgrammarappendix}{4.6in}

(\arabic{equation}) & Void \refstepcounter{equation}\label{prod:Void}  \: \xcd"void"  \\


\end{bbgrammarappendix}

\begin{bbgrammarappendix}{4.2in}

(\arabic{equation}) & WhenStmt \refstepcounter{equation}\label{prod:WhenStmt}  \: \xcd"when" \xcd"(" Exp \xcd")" Stmt  \\


\end{bbgrammarappendix}

\begin{bbgrammarappendix}{4.1in}

(\arabic{equation}) & WhileStmt \refstepcounter{equation}\label{prod:WhileStmt}  \: \xcd"while" \xcd"(" Exp \xcd")" Stmt  \\


\end{bbgrammarappendix}



\renewcommand{\bibname}{References}
\bibliographystyle{plain}
\bibliography{master}


\clearpage
%\documentclass[10pt,twoside,twocolumn,notitlepage]{report}
\documentclass[12pt,twoside,notitlepage]{report}
\usepackage{tex/x10}
\usepackage{tex/tenv}
\def\Hat{{\tt \char`\^}}
\usepackage{url}
\usepackage{times}
\usepackage{tex/txtt}
\usepackage{ifpdf}
\usepackage{tocloft}
\usepackage{tex/bcprules}
\usepackage{xspace}

\newif\ifdraft
%\drafttrue
\draftfalse

\pagestyle{headings}
\showboxdepth=0
\makeindex

\usepackage{tex/commands}

\usepackage[
pdfauthor={Vijay Saraswat, Bard Bloom, Igor Peshansky, Olivier Tardieu, and David Grove},
pdftitle={X10 Language Specification},
pdfcreator={pdftex},
pdfkeywords={X10},
linkcolor=blue,
citecolor=blue,
urlcolor=blue
]{hyperref}

\ifpdf
          \pdfinfo {
              /Author   (Vijay Saraswat, Bard Bloom, Igor Peshansky, Olivier Tardieu, and David Grove)
              /Title    (X10 Language Specification)
              /Keywords (X10)
              /Subject  ()
              /Creator  (TeX)
              /Producer (PDFLaTeX)
          }
\fi

\def\headertitle{The \XtenCurrVer{} Report (Draft) }
\def\integerversion{2.2}

% Sizes and dimensions

%\topmargin -.375in       %    Nominal distance from top of page to top of
                         %    box containing running head.
%\headsep 15pt            %    Space between running head and text.

%\textheight 9.0in        % Height of text (including footnotes and figures, 
                         % excluding running head and foot).

%\textwidth 5.5in         % Width of text line.
\columnsep 15pt          % Space between columns 
\columnseprule 0pt       % Width of rule between columns.

\parskip 5pt plus 2pt minus 2pt % Extra vertical space between paragraphs.
\parindent 0pt                  % Width of paragraph indentation.
%\topsep 0pt plus 2pt            % Extra vertical space, in addition to 
                                % \parskip, added above and below list and
                                % paragraphing environments.


\newif\iftwocolumn

\makeatletter
\twocolumnfalse
\if@twocolumn
\twocolumntrue
\fi
\makeatother

\iftwocolumn

\oddsidemargin  0in    % Left margin on odd-numbered pages.
\evensidemargin 0in    % Left margin on even-numbered pages.

\else

\oddsidemargin  .5in    % Left margin on odd-numbered pages.
\evensidemargin .5in    % Left margin on even-numbered pages.

\fi


\newtenv{example}{Example}[section]
\newtenv{planned}{Planned}[section]

\begin{document}

% \section{Work In Progress}
% \begin{itemize}
%     \item Rewrite first chapter
%     \item Describe library classes, including such fundamentals as Object and String
%     \item Examples for covariant/contravariant generics are wrong -- use Nate's examples
%     \item Describe local classes.
%     \item Reduce the use of \xcd`self` in constraints.
%     \item Copy sections of grammar to relevant sections of text.
%     \item Do something about 4.12.3
% \end{itemize}
% 
% {\bf Feedback:} 
% To help us the most, we would appreciate comments in one of these formats: 
% \begin{itemize}
% \item An annotated copy of the PDF document, if it's convenient.  Acrobat
%       Writer can produce helpful highlighting and sticky notes.  If you don't
%       use Acrobat Writer, don't fuss.
% \item Text comments.  Since the document is still being edited, page numbers
%       are going to be useless as pointers to the text.  If possible, we'd like
%       pointers to sections by number and title: {\em In 12.1, ``Empty
%       Statement'', please discuss side effects and performance implications
%       for this construct''}  If it's a long section, giving us a couple words
%       we can grep for would help too.
% \end{itemize}
% 
% Thank you very much!




% \parindent 0pt %!! 15pt                    % Width of paragraph indentation.

%\hfil {\bf 7 Feb 2005}
%\hfil \today{}

% First page

\thispagestyle{empty}

% \todo{"another" report?}

\topnewpage[{
\begin{center}   
{\huge\bf Report on the Experimental Language \Xten{}}
\vskip 1ex
$$
\begin{tabular}{l@{\extracolsep{.5in}}lll}
\multicolumn{4}{c}{\sc  Version 1.1}\\
\multicolumn{4}{c}{\sc Please send comments to 
V\authorsc{IJAY} S\authorsc{ARASWAT} at 
{\tt vsaraswa@us.ibm.com}}\\
%\multicolumn{4}{c}{({\sc IBM Confidential})}

%\ldots
\end{tabular}
$$
\vskip 2ex
% {\it Dedicated to the Memory of APL} % vj
{\bf Jun 30, 2007}
\vskip 2.6ex
\end{center}


}]


\chapter*{Summary}
This draft report provides an initial description of the programming
language \Xten. \Xten{} is a single-inheritance class-based object-oriented
(OO) programming language designed for high-performance, high-productivity
computing on high-end computers supporting $\approx 10^5$ hardware threads
and $\approx 10^{15}$ operations per second. 

{}\Xten{} is based on state-of-the-art object-oriented programming
languages and deviates from them only as necessary to support its
design goals. The language is intended to have a simple and clear
semantics and be readily accessible to mainstream OO programmers. It
is intended to support a wide variety of concurrent programming
idioms.
%, incuding data parallelism, task parallelism, pipelining.
%producer/consumer and divide and conquer.

%We expect to revise this document in the light of experience gained in implementing
%and using this language.

The \Xten{} design team consists of
D\authorsc{AVID} B\authorsc{ACON}, 
R\authorsc{AJ} B\authorsc{ARIK}, 
G\authorsc{ANESH} B\authorsc{IKSHANDI}, 
B\authorsc{OB} B\authorsc{LAINEY}, 
P\authorsc{HILIPPE} C\authorsc{HARLES}, 
P\authorsc{ERRY} C\authorsc{HENG}, 
C\authorsc{HRISTOPHER} D\authorsc{ONAWA}, 
J\authorsc{ULIAN} D\authorsc{OLBY}, 
K\authorsc{EMAL} E\authorsc{BCIO\u{G}LU},
R\authorsc{OBERT} F\authorsc{UHRER},
P\authorsc{ATRICK} G\authorsc{ALLOP}, 
C\authorsc{HRISTIAN} G\authorsc{ROTHOFF}, 
A\authorsc{LLAN} K\authorsc{IELSTRA}, 
S\authorsc{REEDHAR} K\authorsc{ODALI}, 
S\authorsc{RIRAM} K\authorsc{RISHNAMOORTHY}, 
N\authorsc{ATHANIEL} N\authorsc{YSTROM}, 
I\authorsc{IGOR} P\authorsc{ESHANSKY}, 
V\authorsc{IJAY} S\authorsc{ARASWAT} (contact author), 
V\authorsc{IVEK} S\authorsc{ARKAR},
A\authorsc{RMANDO} S\authorsc{OLAR-LEZAMA},  
C\authorsc{HRISTOPH von} P\authorsc{RAUN},
P\authorsc{RADEEP} V\authorsc{ARMA},
K\authorsc{RISHNA} V\authorsc{ENKATA},
J\authorsc{AN} V\authorsc{ITEK}, and
T\authorsc{ONG} W\authorsc{EN}.

For extended discussions and support we would like to thank: 
Robert Callahan, Calin
Cascaval, Norman Cohen, Elmootaz Elnozahy, John Field, Bob Fuhrer,
Orren Krieger, Doug Lea, John McCalpin, Paul McKenney, Ram Rajamony,
R.K.~Shyamasundar, Filip Pizlo, V.T.~Rajan, Frank Tip, and Mandana Vaziri.

We thank Jonathan Rhees and William Clinger with help in obtaining the
\LaTeX{} style file and macros used in producing the Scheme report,
after which this document is based. We acknowledge the influence of
the $\mbox{\Java}^{\mbox{TM}}$ Language Specification \cite{jls2}
document, as evidenced by the numerous citations in the text.

This document revises Version {\cf 1.01} of the Report, released in
December 2006. It documents the language corresponding to the first
revision of the first version of the implementation.  This
revision was done by
R\authorsc{AJ} B\authorsc{ARIK}, 
P\authorsc{HILIPPE} C\authorsc{HARLES}, 
C\authorsc{HRISTOPHER} D\authorsc{ONAWA}, 
R\authorsc{OBERT} F\authorsc{UHRER},
N\authorsc{ATHANIEL} N\authorsc{YSTROM},  
V\authorsc{IJAY} S\authorsc{ARASWAT},
V\authorsc{IVEK} S\authorsc{ARKAR},
P\authorsc{RADEEP} V\authorsc{ARMA} and
K\authorsc{RISHNA} V\authorsc{ENKATA}.
(Earlier implementations benefited from significant contributions by
C\authorsc{HRISTIAN} G\authorsc{ROTHOFF} and 
C\authorsc{HRISTOPH von} P\authorsc{RAUN}.)
T\authorsc{ONG} W\authorsc{EN} has written many application programs
in \Xten{}. G\authorsc{UOJING} C\authorsc{ONG} has helped in the
development of many applications.


%\vfill
%\begin{center}
%{\large \bf
%*** DRAFT*** \\
%%August 31, 1989
%\today
%}\end{center}

\vfill
\eject


\chapter*{Contents}
\addvspace{3.5pt}                  % don't shrink this gap
\renewcommand{\tocshrink}{-3.5pt}  % value determined experimentally
{\footnotesize
\tableofcontents
}

\vfill
\eject


 

\clearpage

{\parskip 0pt
\addtolength{\cftsecnumwidth}{0.5em}
\addtolength{\cftsubsecnumwidth}{0.5em}
%\addtolength{\cftsecindent}{0.5em}
\addtolength{\cftsubsecindent}{0.5em}
\tableofcontents
}


\chapter{Introduction}

\subsection*{Background}
Larger computational problems require more powerful computers capable of
performing a larger number of operations per second. The era of
increasing performance by simply increasing clocking frequency now
seems to be behind us. It is becoming increasingly difficult
to mange chip power and heat.  Instead, computer
designers are starting to look at {\em scale out} systems in which the
system's computational capacity is increased by adding additional
nodes of comparable power to existing nodes, and connecting nodes with
a high-speed communication network.

A central problem with scale out systems is a definition of the {\em
memory model}, that is, a model of the interaction between shared
memory and  simultaneous (read, write) operations on that
memory by multiple processors. The traditional ``one operation at a
time, to completion'' model that underlies Lamport's notion of {\em
sequential consistency} (SC) proves too expensive to implement in
hardware, at scale. Various models of {\em relaxed consistency} have
proven too difficult for programmers to work with.  

One response to this problem has been to move to a {\em fragmented
memory model}. Multiple processors are made to interact via a
relatively language-neutral message-passing format such as MPI
\cite{mpi}. This model has enjoyed some success: several
high-performance applications have been written in this
style. Unfortunately, this model leads to a {\em loss of programmer
productivity}: the message-passing format is integrated into the host
language by means of an application-programming interface (API), the
programmer must explicitly represent and manage the interaction
between multiple processes and choreograph their data exchange; large
data-structures (such as distributed arrays, graphs, hash-tables) that
are conceptually unitary must be thought of as fragmented across
different nodes; all processors must generally execute the same code
(in an SPMD fashion) etc.

One response to this problem has been the advent of the {\em
partitioned global address space} (PGAS) model underlying languages
such as UPC, Titanium and Co-Array Fortran \cite{pgas,titanium}. These
languages permit the programmer to think of a single computation
running across multiple processors, sharing a common address
space. All data resides at some processors, which is said to have {\em
affinity} to the data.  Each processor may operate directly on the
data it contains but must use some indirect mechanism to access or
update data at other processors. Some kind of global {\em barriers}
are used to ensure that processors remain roughly in lock-step.

\Xten{} is a modern object-oriented programming language
in the PGAS family. The fundamental goal of \Xten{} is to enable
scalable, high-performance, high-productivity transformational
programming for high-end computers---for traditional numerical
computation workloads (such as weather simulation, molecular dynamics,
particle transport problems etc) as well as commercial server
workloads.

\Xten{} is based on state-of-the-art object-oriented
programming ideas primarily to take advantage of their proven
flexibility and ease-of-use for a wide spectrum of programming
problems. \Xten{} takes advantage of several years of research (e.g.,
in the context of the Java Grande forum,
\cite{moreira00java,kava}) on how to adapt such languages to the context of
high-performance numerical computing. Thus \Xten{} provides support
for user-defined {\em struct types} (such as \xcd"Int", \xcd"Float",
\xcd"Complex" etc), supports a very
flexible form of multi-dimensional arrays (based on ideas in ZPL
\cite{zpl}) and supports IEEE-standard floating point arithmetic.
Some capabilities for supporting operator overloading are also provided.

{}\Xten{} introduces a flexible treatment of concurrency, distribution
and locality, within an integrated type system. \Xten{} extends the
PGAS model with {\em asynchrony} (yielding the {\em APGAS} programming
model). {}\Xten{} introduces {\em places} as an abstraction for a
computational context with a locally synchronous view of shared
memory. An \Xten{} computation runs over a large collection of places.
Each place hosts some data and runs one or more {\em
activities}. Activities are extremely lightweight threads of
execution. An activity may synchronously (and {\em atomically}) use
one or more memory locations in the place in which it resides,
leveraging current symmetric multiprocessor (SMP) technology.  
To access or update memory at other places, it must 
spawn activities asynchronously (either explicitly or implicitly).
\Xten{} provides weaker ordering guarantees for
inter-place data access, enabling applications to scale.  {\em
Immutable} data needs no consistency management and may be freely
copied by the implementation between places.  One or more {\em clocks}
may be used to order activities running in multiple
places.  Arrays may be distributed across multiple
places. Arrays support parallel collective operations. A novel
exception flow model ensures that exceptions thrown by asynchronous
activities can be caught at a suitable parent activity.  The type
system tracks which memory accesses are local. The programmer may
introduce place casts which verify the access is local at run time.
Linking with native code is supported.

\XtenCurrVer builds on v1.7 to support the following features: {\em
  structs} (i.e., ``header-less'', inlinable objects), type rules for
preventing escape of \xcd{this} from a constructor, 
the introduction of a global object model, permitting user-specified
(immutable) fields to be replicated with the object reference.
\xcd{value} classes are no longer supported; their functionality is
accomplished by using structs or global fields and methods.


Several representative idioms for concurrency and communication have
already found pleasant expression in \Xten. We intend to develop
several full-scale applications to get better experience with the
language, and revisit the design in the light of this experience.


\chapter{Overview of \Xten}

\Xten{} is a statically typed object-oriented language, extending a sequential
core language with \emph{places}, \emph{activities}, \emph{clocks},
(distributed, multi-dimensional) \emph{arrays} and \emph{struct} types. All
these changes are motivated by the desire to use the new language for
high-end, high-performance, high-productivity computing.

\section{Object-oriented features}

The sequential core of \Xten{} is a {\em container-based} object-oriented language
similar to \java{} and C++, and more recent languages such as Scala.  
Programmers write \Xten{} code by defining containers for data and behavior
called 
\emph{classes}
(\Sref{XtenClasses}) and
\emph{structs}
(\Sref{XtenStructs}), 
often abstracted as 
\emph{interfaces}
(\Sref{XtenInterfaces}).
X10 provides inheritance and subtyping in fairly traditional ways. 

\begin{ex}

\xcd`Normed` describes entities with a \xcd`norm()` method. \xcd`Normed` is
intended to be used for entities with a position in some coordinate system,
and \xcd`norm()` gives the distance between the entity and the origin. A
\xcd`Slider` is an object which can be moved around on a line; a
\xcd`PlanePoint` is a fixed position in a plane. Both \xcd`Slider`s and
\xcd`PlanePoint`s have a sensible \xcd`norm()` method, and implement
\xcd`Normed`.

%~~gen ^^^ Overview10
% package Overview;
%~~vis
\begin{xten}
interface Normed {
  def norm():Double;
}
class Slider implements Normed {
  var x : Double = 0;
  public def norm() = Math.abs(x);
  public def move(dx:Double) { x += dx; }
}
struct PlanePoint implements Normed {
  val x : Double; val y:Double;
  public def this(x:Double, y:Double) {
    this.x = x; this.y = y;
  }
  public def norm() = Math.sqrt(x*x+y*y);
}
\end{xten}
%~~siv
%
%~~neg
\end{ex}

\paragraph{Interfaces}

An \Xten{} interface specifies a collection of abstract methods; \xcd`Normed`
specifies just \xcd`norm()`. Classes and
structs can be specified to {\em implement} interfaces, as \xcd`Slider` and
\xcd`PlanePoint` implement \xcd`Normed`, and, when they do so, must provide
all the methods that the interface demands.

Interfaces are
purely abstract. Every value of type \xcd`Normed` must be an instance of some
class like \xcd`Slider` or some struct like \xcd`PlanePoint` which implements
\xcd`Normed`; no value can be \xcd`Normed` and nothing else. 


\paragraph{Classes and Structs}



There are two kinds of containers: \emph{classes}
(\Sref{ReferenceClasses}) and \emph{structs} (\Sref{Structs}). Containers hold
data in {\em fields}, and give concrete implementations of 
methods, as \xcd`Slider` and \xcd`PlainPoint` above.

Classes are organized in a single-inheritance tree: a class may have only a
single parent class, though it may implement many interfaces and have many
subclasses. Classes may have mutable fields, as \xcd`Slider` does.

In contrast, structs are headerless values, lacking the internal organs
which give objects their intricate behavior.  This makes them less powerful
than objects (\eg, structs cannot inherit methods, though objects can), but also
cheaper (\eg, they can be inlined, and they require less space than objects).  
Structs are immutable, though their fields may be immutably set to objects
which are themselves mutable.  They behave like objects in all ways consistent
with these limitations; \eg, while they cannot {\em inherit} methods, they can
have them -- as \xcd`PlanePoint` does.

\Xten{} has no primitive classes per se. However, the standard library
\xcd"x10.lang" supplies structs and objects \xcd"Boolean", \xcd"Byte",
\xcd"Short", \xcd"Char", \xcd"Int", \xcd"Long", \xcd"Float", \xcd"Double",
\xcd"Complex" and \xcd"String". The user may defined additional arithmetic
structs using the facilities of the language.



\paragraph{Functions.}

X10 provides functions (\Sref{Closures}) to allow code to be used
as values.  Functions are first-class data: they can be stored in lists,
passed between activities, and so on.  \xcd`square`, below, is a function
which squares an \xcd`Long`.  \xcd`of4` takes an \xcd`Long`-to-\xcd`Long`
function and applies it to the number \xcd`4`.  So, \xcd`fourSquared` computes
\xcd`of4(square)`, which is \xcd`square(4)`, which is 16, in a fairly
complicated way.
%~~gen ^^^ Overview20
% package Overview.of.Functions.one;
% class Whatever{
% def chkplz() {
%~~vis
\begin{xten}
  val square = (i:Long) => i*i;
  val of4 = (f: (Long)=>Long) => f(4);
  val fourSquared = of4(square);
\end{xten}
%~~siv
%}}
%~~neg



Functions are used extensively in X10
programs.  For example, a common way to construct and initialize an \xcd`Rail[Long]` --
that is, a fixed-length one-dimensional array of numbers, like an \xcd`long[]` in \java{} -- is to
pass two arguments to a factory method: the first argument being the length of
the rail, and the second being a function which computes the initial value of
the \xcd`i`{$^{th}$} element.  The following code constructs a 1-dimensional
rail 
initialized to the squares of 0,1,...,9: \xcd`r(0) == 0`, \xcd`r(5)==25`, etc. 
%~~gen ^^^ Overview30
% package Overview.of.Functions.two;
% class Whatevermore {
%  def plzchk(){
%    val square = (i:Long) => (i*i);
%~~vis
\begin{xten}
  val r : Rail[Long] = new Rail[Long](10, square);
\end{xten}
%~~siv
%}}
%~~neg








\paragraph{Constrained Types}

X10 containers may declare {\em properties}, which are fields bound immutably
at the creation of the container.  The static analysis system understands
properties, and can work with them logically.   


For example, an implementation of matrices \xcd`Mat` might have the numbers of
rows and columns as properties.  A little bit of care in definitions allows
the definition of a \xcd`+` operation that works on matrices of the same
shape, and \xcd`*` that works on matrices with appropriately matching shapes.
%~~gen ^^^ Overview40
%package Overview.Mat2;
%~~vis
\begin{xten}
abstract class Mat(rows:Long, cols:Long) {
 static type Mat(r:Long, c:Long) = Mat{rows==r&&cols==c};
 abstract operator this + (y:Mat(this.rows,this.cols))
                 :Mat(this.rows, this.cols);
 abstract operator this * (y:Mat) {this.cols == y.rows} 
                 :Mat(this.rows, y.cols);
\end{xten}
%~~siv
%  static def makeMat(r:Long,c:Long) : Mat(r,c) = null;
%  static def example(a:Long, b:Long, c:Long) {
%    val axb1 : Mat(a,b) = makeMat(a,b);
%    val axb2 : Mat(a,b) = makeMat(a,b);
%    val bxc  : Mat(b,c) = makeMat(b,c);
%    val axc  : Mat(a,c) = (axb1 +axb2) * bxc;
%  }
%}
%~~neg



The following code typechecks (assuming that \xcd`makeMat(m,n)` is a function
which creates an \xcdmath"m$\times$n" matrix).
However, an attempt to compute \xcd`axb1 + bxc` or
\xcd`bxc * axb1` would result in a compile-time type error:
%~~gen ^^^ Overview50
%package Overview.Mat1;
%//OPTIONS: -STATIC_CHECKS
%abstract class Mat(rows:Long, cols:Long) {
%  static type Mat(r:Long, c:Long) = Mat{rows==r&&cols==c};
%  public def this(r:Long, c:Long) : Mat(r,c) = {property(r,c);}
%  static def makeMat(r:Long,c:Long) : Mat(r,c) = null;
%  abstract  operator this + (y:Mat(this.rows,this.cols)):Mat(this.rows, this.cols);
%  abstract  operator this * (y:Mat) {this.cols == y.rows} : Mat(this.rows, y.cols);
%~~vis
\begin{xten}
  static def example(a:Long, b:Long, c:Long) {
    val axb1 : Mat(a,b) = makeMat(a,b);
    val axb2 : Mat(a,b) = makeMat(a,b);
    val bxc  : Mat(b,c) = makeMat(b,c);
    val axc  : Mat(a,c) = (axb1 +axb2) * bxc;
    //ERROR: val wrong1 = axb1 + bxc;
    //ERROR: val wrong2 = bxc * axb1;
  }

\end{xten}
%~~siv
%}
%~~neg

The ``little bit of care'' shows off many of the features of constrained
types.    
The \xcd`(rows:Long, cols:Long)` in the class definition declares two
properties, \xcd`rows` and \xcd`cols`.\footnote{The class is officially declared
abstract to allow for multiple implementations, like sparse and band matrices,
but in fact is abstract to avoid having to write the actual definitions of
\xcd`+` and \xcd`*`.}  

A constrained type looks like \xcd`Mat{rows==r && cols==c}`: a type
name, followed by a Boolean expression in braces.  
The \xcd`type` declaration on the second line makes
\xcd`Mat(r,c)` be a synonym for \xcd`Mat{rows==r && cols==c}`,
allowing for compact types in many places.

Functions can return constrained types.  
The \xcd`makeMat(r,c)` method returns a \xcd`Mat(r,c)` -- a matrix whose shape
is given by the arguments to the method.    In
particular, constructors can have constrained return types to provide specific
information about the constructed values.

The arguments of methods can have type constraints as well.  The 
\xcd`operator this +` line lets \xcd`A+B` add two matrices.  The type of the
second argument \xcd`y` is constrained to have the same number of rows and
columns as the first argument \xcd`this`. Attempts to add mismatched matrices
will be flagged as type errors at compilation.

At times it is more convenient to put the constraint on the method as a whole,
as seen in the \xcd`operator this *` line. Unlike for \xcd`+`, there is no
need to constrain both dimensions; we simply need to check that the columns of
the left factor match the rows of the right. This constraint is written in
\xcd`{...}` after the argument list.  The shape of the result is computed from
the shapes of the arguments.

And that is all that is necessary for a user-defined class of matrices to have
shape-checking for matrix addition and multiplication.  The \xcd`example`
method compiles under those definitions.








\paragraph{Generic types}

Containers may have type parameters, permitting the definition of
{\em generic types}.  Type parameters may be instantiated by any X10 type.  It
is thus possible to make a list of integers \xcd`List[Long]`, a list of
non-zero integers \xcd`List[Long{self != 0}]`, or a list of people
\xcd`List[Person]`.  In the definition of \xcd`List`, \xcd`T` is a type
parameter; it can be instantiated with any type.
%~~gen ^^^ Overview60
%~~vis
\begin{xten}
class List[T] {
    var head: T;
    var tail: List[T];
    def this(h: T, t: List[T]) { head = h; tail = t; }
    def add(x: T) {
        if (this.tail == null)
            this.tail = new List[T](x, null);
        else
            this.tail.add(x);
    }
}
\end{xten}
%~~siv
%~~neg
The constructor (\xcd"def this") initializes the fields of the new object.
The \xcd"add" method appends an element to the list.
\xcd"List" is a generic type.  When  instances of \xcd"List" are
allocated, the type \param{} \xcd"T" must be bound to a concrete
type.  \xcd"List[Long]" is the type of lists of element type
\xcd"Long", \xcd"List[List[String]]" is the type of lists whose elements are
themselves lists of string, and so on.

%%BARD-HERE

\section{The sequential core of X10}

The sequential aspects of X10 are mostly familiar from C and its progeny.
\Xten{} enjoys the familiar control flow constructs: \xcd"if" statements,
\xcd"while" loops, \xcd"for" loops, \xcd"switch" statements, \xcd`throw` to
raise exceptions and \xcd`try...catch` to handle them, and so on.

X10 has both implicit coercions and explicit conversions, and both can be
defined on user-defined types.  Explicit conversions are written with the
\xcd`as` operation: \xcd`n as Long`.  The types can be constrained: 
%~~exp~~`~~`~~n:Long~~ ^^^ Overview70
\xcd`n as Long{self != 0}` converts \xcd`n` to a non-zero integer, and throws a
runtime exception if its value as an integer is zero.

\section{Places and activities}

The full power of X10 starts to emerge with concurrency.
An \Xten{} program is intended to run on a wide range of computers,
from uniprocessors to large clusters of parallel processors supporting
millions of concurrent operations. To support this scale, \Xten{}
introduces the central concept of \emph{place} (\Sref{XtenPlaces}).
A place can be thought of as a virtual shared-memory multi-processor:
a computational unit with a finite (though perhaps changing) number of
hardware threads and a bounded amount of shared memory, uniformly
accessible by all threads.



An \Xten{} computation acts on \emph{values}(\Sref{XtenObjects}) through
the execution of lightweight threads called
\emph{activities}(\Sref{XtenActivities}). 
An {\em object}
 has a small, statically fixed set of fields, each of
which has a distinct name. A scalar object is located at a single place and
stays at that place throughout its lifetime. An \emph{aggregate} object has
many fields (the number may be known only when the object is created),
uniformly accessed through an index (\eg, an integer) and may be distributed
across many places. The distribution of an aggregate object remains unchanged
throughout the computation, thought different aggregates may be distributed
differently. Objects are garbage-collected when no longer useable; there are
no operations in the language to allow a programmer to explicitly release
memory.

{}\Xten{} has a \emph{unified} or \emph{global address space}. This means that
an activity can reference objects at other places. However, an activity may
synchronously access data items only in the current place, the place in which
it is running. It may atomically update one or more data items, but only in
the current place.   If it becomes necessary to read or modify an object at
some other place \xcd`q`, the {\em place-shifting} operation \xcd`at(q;F)` can
be used, to move part of the activity to \xcd`q`.  \xcd`F` is a specification
of what information will be sent to \xcd`q` for use by that part of the
computation. 
It is easy to compute
across multiple places, but the expensive operations (\eg, those which require
communication) are readily visible in the code. 

\paragraph{Atomic blocks.}

X10 has a control construct \xcd"atomic S" where \xcd"S" is a statement with
certain restrictions. \xcd`S` will be executed atomically, without
interruption by other activities. This is a common primitive used in
concurrent algorithms, though rarely provided in this degree of generality by
concurrent programming languages.

More powerfully -- and more expensively -- X10 allows conditional atomic
blocks, \xcd`when(B)S`, which are executed atomically at some point when
\xcd`B` is true.  Conditional atomic blocks are one of the strongest
primitives used in concurrent algorithms, and one of the least-often
available. 

\paragraph{Asynchronous activities.}

An asynchronous activity is created by a statement \xcd"async S", which starts
up a new activity running \xcd`S`.  It does not wait for the new activity to
finish; there is a separate statement (\xcd`finish`) to do that.


\section{Distributed heap management}

\Xten{} is the language for parallel and distributed computing, which is based on the APGAS (Asynchronous Partitioned Global Address Space) programming model. In (A)PGAS, the address space is partitioned into multiple semi-spaces. The semi-space is called \emph{place} in \Xten{}. In Managed \Xten{} (\Xten{} on \java{} VMs), a place is represented as a single \java{} VM and the semi-space is mapped to the heap of the \java{} VM.

\Xten{} supports garbage collection. Objects in a local heap (local objects) are collected with (local) garbage collection and there is no way to explicitly free them. The reference to local objects is called \emph{local reference}.

In addition, \Xten{} has another type of reference called \emph{remote reference}. Unlike local reference, remote reference can reference objects at both local and remote places.

With remote reference, an activity (something like thread, it runs on a place at a time but it can move itself to different places) can access objects at a remote place (remote objects) when the activity has moved to the remote place. The place where an object is created is the home place of the object and it does not change for the lifetime.

To guarantee an activity can access remote objects at their home place, the objects with remote reference are protected from (local) garbage collection at their home place even if they have no local reference. Objects can be garbage collected only when they have neither local nor remote reference. The garbage collection that takes care of remote reference is called distributed garbage collection and it is supported in Managed \Xten{}.

Distributed garbage collection in Managed \Xten{} \cite{KawachiyaX1012} tracks the lifetime of remote reference with reference counting. When the local garbage collection at a remote place detects the remote reference is no longer needed at the place, the count is decremented. When the count becomes zero, the local garbage collection at the home place is ready to collect the referenced object in the ordinary way.

This mechanism works in most cases, but when there is unbalance in heap allocation rate between places, there is a risk of out of memory error at a frequently allocating place. This is because remote reference from infrequently allocating (i.e. infrequently garbage collected) places could retain remotely referenced objects longer than needed.

To avoid the out of memory error even with unbalanced heap allocation rate, there is a way to explicitly release remote reference.

A single call of \xcd`PlaceLocalHandle.destroy()` (\xcd`PlaceLocalHandle` is an \Xten{} type that bundles multiple remote references to the objects at different places) releases all remote references immediately, thus the local garbage collection at each place becomes ready to collect the referenced object in the ordinary way. It can be called at the point where the all objects referenced by the handle are no longer needed to be accessible with the handle. Local reference to the object at each place won't be affected.



\section{Clocks}
The MPI style of coordinating the activity of multiple processes with
a single barrier is not suitable for the dynamic network of heterogeneous
activities in an \Xten{} computation.  
X10 allows multiple barriers in a form that supports determinate,
deadlock-free parallel computation, via the \xcd`Clock` type.

A single \xcd`Clock` represents a computation that occurs in phases.
At any given time, an activity is {\em registered} with zero or more clocks.
The X10 statement \xcd`next` tells all of an activity's registered clocks that
the activity has finished the current phase, and causes it to wait for the
next phase.  Other operations allow waiting on a single clock, starting
new clocks or new activities registered on an extant clock, and so on. 

%%INTRO-CLOCK%  Activities may use clocks to repeatedly detect quiescence of arbitrary
%%INTRO-CLOCK%  programmer-specified, data-dependent set of activities. Each activity
%%INTRO-CLOCK%  is spawned with a known set of clocks and may dynamically create new
%%INTRO-CLOCK%  clocks. At any given time an activity is \emph{registered} with zero or
%%INTRO-CLOCK%  more clocks. It may register newly created activities with a clock,
%%INTRO-CLOCK%  un-register itself with a clock, suspend on a clock or require that a
%%INTRO-CLOCK%  statement (possibly involving execution of new async activities) be
%%INTRO-CLOCK%  executed to completion before the clock can advance.  At any given
%%INTRO-CLOCK%  step of the execution a clock is in a given phase. It advances to the
%%INTRO-CLOCK%  next phase only when all its registered activities have \emph{quiesced}
%%INTRO-CLOCK%  (by executing a \xcd"next" operation on the clock).
%%INTRO-CLOCK%  When a clock advances, all its activities may now resume execution.
%%INTRO-CLOCK%  

Clocks act as {barriers} for a dynamically varying collection of activities.
They generalize the barriers found in MPI style program in that an activity
may use multiple clocks simultaneously. Yet programs using clocks properly are
guaranteed not to suffer from deadlock.

%%HERE

\section{Arrays, regions and distributions}

X10 provides \xcd`DistArray`s, {\em distributed arrays}, which spread data
across many places. An underlying \xcd`Dist` object provides the {\em
distribution}, telling which elements of the \xcd`DistArray` go in which
place. \xcd`Dist` uses subsidiary \xcd`Region` objects to abstract over the
shape and even the dimensionality of arrays.
Specialized X10 control statements such as \xcd`ateach` provide efficient
parallel iteration over distributed arrays.


\section{Annotations}

\Xten{} supports annotations on classes and interfaces, methods
and constructors,
variables, types, expressions and statements.
These annotations may be processed by compiler plugins.

\section{Translating MPI programs to \Xten{}}

While \Xten{} permits considerably greater flexibility in writing
distributed programs and data structures than MPI, it is instructive
to examine how to translate MPI programs to \Xten.

Each separate MPI process can be translated into an \Xten{}
place. Async activities may be used to read and write variables
located at different processes. A single clock may be used for barrier
synchronization between multiple MPI processes. \Xten{} collective
operations may be used to implement MPI collective operations.
\Xten{} is more general than MPI in (a)~not requiring synchronization
between two processes in order to enable one to read and write the
other's values, (b)~permitting the use of high-level atomic blocks
within a process to obtain mutual exclusion between multiple
activities running in the same node (c)~permitting the use of multiple
clocks to combine the expression of different physics (e.g.,
computations modeling blood coagulation together with computations
involving the flow of blood), (d)~not requiring an SPMD style of
computation.


%\note{Relaxed exception model}
\section{Summary and future work}
\subsection{Design for scalability}
\Xten{} is designed for scalability, by encouraging working with local data,
and limiting the ability of events at one place to delay those at another. For
example, an activity may atomically access only multiple locations in the
current place. Unconditional atomic blocks are dynamically guaranteed to be
non-blocking, and may be implemented using non-blocking techniques that avoid
mutual exclusion bottlenecks. 
%TODO: yoav says: ``no idea what [the following] means''
Data-flow synchronization permits point-to-point
coordination between reader/writer activities, obviating the need for
barrier-based or lock-based synchronization in many cases.

\subsection{Design for productivity}
\Xten{} is designed for productivity.

\paragraph{Safety and correctness.}



Programs written in \Xten{} are guaranteed to be statically
\emph{type safe}, \emph{memory safe} and \emph{pointer safe},
with certain exceptions given in \Sref{sect:LimitationOfStrictTyping}.

Static type safety guarantees that every location contains only values whose
dynamic type agrees with the location's static type. The compiler allows a
choice of how to handle method calls. In strict mode, method calls are
statically checked to be permitted by the static types of operands. In lax
mode, dynamic checks are inserted when calls may or may not be correct,
providing weaker static correctness guarantees but more programming
convenience. 

Memory safety guarantees that an object may only access memory within its
representation, and other objects it has a reference to. \Xten{} does not
permit 
pointer arithmetic, and bound-checks array accesses dynamically if necessary.
\Xten{} uses garbage collection to collect objects no longer referenced by any
activity. \Xten{} guarantees that no object can retain a reference to an
object whose memory has been reclaimed. Further, \Xten{} guarantees that every
location is initialized at run time before it is read, and every value read
from a word of memory has previously been written into that word.

%XXX
%Pointer safety guarantees that a null pointer exception cannot be
%thrown by an operation on a value of a non-nullable type.

%Because places are reflected in the type system, static type safety
%also implies \emph{place safety}. All operations that need to be performed
%locally are, in fact, performed locally.  All data which is declared to be
%stored locally are, in fact, stored locally.

\Xten{} programs that use only the common, specified clock idioms and unconditional atomic
blocks are guaranteed not to deadlock. Unconditional atomic blocks
are non-blocking, hence cannot introduce deadlocks.
Many concurrent programs can be shown to be determinate (hence
race-free) statically.

\paragraph{Integration.}
A key issue for any new programming language is how well it can be
integrated with existing (external) languages, system environments,
libraries and tools.

%TODO: Yoav asks ``you mean interop''?
We believe that \Xten{}, like \java{}, will be able to support a large
number of libraries and tools. An area where we expect future versions
of \Xten{} to improve on \java{} like languages is \emph{native
integration} (\Sref{NativeCode}). Specifically, \Xten{} will 
permit multi-dimensional local arrays to be operated on natively by
native code.

\subsection{Conclusion}
{}\Xten{} is considerably higher-level than thread-based languages in
that it supports dynamically spawning lightweight activities, the
use of atomic operations for mutual exclusion, and the use of clocks
for repeated quiescence detection.

Yet it is much more concrete than languages like HPF in that it forces
the programmer to explicitly deal with distribution of data
objects. In this the language reflects the designers' belief that
issues of locality and distribution cannot be hidden from the
programmer of high-performance code in high-end computing.  A
performance model that distinguishes between computation and
communication must be made explicit and transparent.\footnote{In this
\Xten{} is similar to more modern languages such as ZPL \cite{zpl}.}
At the same time we believe that the place-based type system and
support for generic programming will allow the \Xten{} programmer to
be highly productive; many of the tedious details of
distribution-specific code can be handled in a generic fashion.

\chapter{Lexical structure}

In general, \Xten{} follows \java{} rules \cite[Chapter 3]{jls2} for
lexical structure.

Lexically a program consists of a stream of white space, comments,
identifiers, keywords, literals, separators and operators.

\paragraph{Whitespace}
% Whitespace \index{whitespace} follows \java{} rules \cite[Chapter 3.6]{jls2}.
ASCII space, horizontal tab (HT), form feed (FF) and line
terminators constitute white space.

\paragraph{Comments}
% Comments \index{comments} follows \java{} rules
% \cite[Chapter 3.7]{jls2}. 
All text included within the ASCII characters ``\xcd"/*"'' and
``\xcd"*/"'' is
considered a comment and ignored; nested comments are not
allowed.  All text from the ASCII characters
``\xcd"//"'' to the end of line is considered a comment and is ignored.

\paragraph{Identifiers}

Identifiers consist of a single letter followed by zero or more
letters or digits.
Letters are defined as the characters for which the \java{}
method \xcd"Character.isJavaIdentifierStart" returns true.
Digits are defined as the ASCII characters \xcd"0" through \xcd"9".

\paragraph{Keywords}
\Xten{} reserves the following keywords:
\begin{xten}
abstract       do             in             public         
as             else           instanceof     return         
assert         extends        interface      self           
async          false          native         static         
ateach         final          new            struct         
break          finally        null           super          
case           finish         offers         switch         
catch          for            operator       this           
class          goto           package        throw          
continue       if             private        transient      
def            implements     property       true           
default        import         protected      try            
\end{xten}
Note that the primitive types are not considered keywords.

\paragraph{Literals}\label{Literals}\index{literals}

Briefly, \XtenCurrVer{} uses fairly standard syntax for its literals:
integers, unsigned integers, floating point numbers, booleans, 
characters, strings, and \xcd"null".  The most exotic points are (1) unsigned
numbers are marked by a \xcd`u` and cannot have a sign; (2) \xcd`true` and
\xcd`false` are the literals for the booleans; and (3) floating point numbers
are \xcd`Double` unless marked with an \xcd`f` for \xcd`Float`. 

Less briefly, we use the following abbreviations: 
\begin{displaymath}
\begin{array}{rcll}
d &=& \mbox{one or more decimal digits}\\
d_8 &=& \mbox{one or more octal digits}\\
d_{16} &=& \mbox{one or more hexadecimal digits, using \xcd`a`-\xcd`f`
for 10-15}\\
i &=& d 
        \mathbin{|} {\tt 0} d_8 
        \mathbin{|} {\tt 0x} d_{16}
        \mathbin{|} {\tt 0X} d_{16}
\\
s &=& \mbox{optional \xcd`+` or \xcd`-`}\\
b &=& d 
          \mathbin{|} d {\tt .}
          \mathbin{|} d {\tt .} d
          \mathbin{|}  {\tt .} d \\
x &=& ({\tt e } \mathbin{|} {\tt E})
         s
         d \\
f &=& b x
\end{array}
\end{displaymath}

\begin{itemize}

\item \xcd`true` and \xcd`false` are the \xcd`Boolean` literals.

\item \xcd`null` is a literal for the null value.  It has type
      \xcd`Any{self==null}`. 

\item \xcd`Int` literals have the form {$si$}; \eg, \xcd`123`,
      \xcd`-321` are decimal \xcd`Int`s, \xcd`0123` and \xcd`-0321` are octal
      \xcd`Int`s, and \xcd`0x123`, \xcd`-0X321`,  \xcd`0xBED`, and \xcd`0XEBEC` are
      hexadecimal \xcd`Int`s.  

\item \xcd`Long` literals have the form {$si{\tt l}$} or
      {$si{\tt L}$}. \Eg, \xcd`1234567890L`  and \xcd`0xBAGEL` are \xcd`Long` literals. 

\item \xcd`UInt` literals have the form {$i{\tt u}$} or {$i {\tt U}$}.
      \Eg, \xcd`123u`, \xcd`0123u`, and \xcd`0xBEAU` are \xcd`UInt` literals.

\item \xcd`ULong` literals have the form {$i {\tt ul}$} or {$i {\tt
      lu}$}, or capital versions of those.  For example, 
      \xcd`123ul`, \xcd`0124567012ul`,  \xcd`0xFLU`, \xcd`OXba1eful`, and \xcd`0xDecafC0ffeefUL` are \xcd`ULong`
      literals. 

\item \xcd`Float` literals have the form {$s f {\tt f}$} or  {$s
      f {\tt F}$}.  Note that the floating-point marker letter \xcd`f` is
      required: unmarked floating-point-looking literals are \xcd`Double`. 
      \Eg, \xcd`1f`, \xcd`6.023E+32f`, \xcd`6.626068E-34F` are \xcd`Float`
      literals. 

\item \xcd`Double` literals have the form {$s f$}\footnote{Except that
      literals like \xcd`1` 
      which match both {$i$} and {$f$} are counted as
      integers, not \xcd`Double`; \xcd`Double`s require a decimal
      point, an exponent, or the \xcd`d` marker.
      }, {$s f {\tt
      D}$}, and {$s f {\tt d}$}.  
      \Eg, \xcd`0.0`, \xcd`0e100`, \xcd`229792458d`, and \xcd`314159265e-8`
      are \xcd`Double` literals.

\item \xcd`Char` literals have one of the following forms: 
      \begin{itemize}
      \item \xcd`'`{\it c}\xcd`'` where {\em c} is any printing ASCII
            character other than 
            \xcd`\` or \xcd`'`, representing the character {\em c} itself; 
            \eg, \xcd`'!'`;
      \item \xcd`'\b'`, representing backspace;
      \item \xcd`'\t'`, representing tab;
      \item \xcd`'\n'`, representing newline;
      \item \xcd`'\f'`, representing form feed;
      \item \xcd`'\r'`, representing return;
      \item \xcd`'\''`, representing single-quote;
      \item \xcd`'\"'`, representing double-quote;
      \item \xcd`'\\'`, representing backslash;
      \item \xcd`'\`{\em dd}\xcd`'`, where {\em dd} is one or more octal
            digits, representing the one-byte character numbered {\em dd}; it
            is an error if {\em dd}{$>255$}.      
      \end{itemize}

\item \xcd`String` literals consist of a double-quote \xcd`"`, followed by
      zero or more of the contents of a \xcd`Char` literal, followed by
      another double quote.  \Eg, \xcd`"hi!"`, \xcd`""`.

\item There are no literals of type \xcd`Byte`, \xcd`UByte`, \xcd`Short`, or
      \xcd`UShort`.  

\end{itemize}



\paragraph{Separators}
\Xten{} has the following separators and delimiters:
\begin{xten}
( )  { }  [ ]  ;  ,  .
\end{xten}

\paragraph{Operators}
\Xten{} has the following operators:
\begin{xten}
==  !=  <   >   <=  >=
&&  ||  &   |   ^
<<  >>  >>>
+   -   *   /   %
++  --  !   ~
&=  |=  ^=
<<= >>= >>>=
+=  -=  *=  /=  %=
=   ?   :   =>  ->
<:  :>  @   ..
\end{xten}





\chapter{Types}
\label{XtenTypes}\index{types}

{}\Xten{} is a {\em strongly typed} object-oriented language: every
variable and expression has a type that is known at compile-time.
Types limit the values that variables can hold and specify the places
at which these values can lie.


{}\Xten{} supports three kinds of runtime entities, {\em objects},
{\em structs}, and {\em functions}. Objects are instances of {\em
  classes} (\Sref{ReferenceClasses}). They may contain mutable fields
and stay resident in the place in which they were
created. 
Objects are said to be {\em boxed} in that variables of a
class type are implemented through a single memory location that
contains a reference to the memory containing the declared state of
the object (and other meta-information such as the list of methods of
the object). Thus objects are represented through an extra level of
indirection. A consequence of this flexibility is that every class
type contains the value \Xcd{null} corresponding to the invalid
reference. \Xcd{null} is often useful as a default value. Further, two
objects may be compared for identity (\Xcd{==}) in constant time by
simply containing references to the memory used to represent the
objects.

Structs are instances of {\em struct types} (\Sref{StructClasses}). They are a
restricted variant of classes, lacking meta-information; this makes them less
flexible, but in many cases more efficient. When it is semantically
meaningful, converting a class into a struct or vice-versa is quite easy.
Structs are immutable and may be freely copied from place to place. Further,
they may be allocated inline, using only as much memory as necessary to hold
and align the fields of the struct.

Functions, called closures or lambda-expressions in other languages, are
instances of {\em function types|} {\Sref{Functions}). Functions can refer to
%~~exp~~`~~`~~y:Int ~~
variables from the surrounding environment; \eg, \xcd`(x:Int)=>x*y` is a unary
integer function which multiplies its argument by the variable \xcd`y` from
the surrounding block.  
Functions may be freely copied from place to place and may be repeatedly
applied to a set of arguments.

These runtime entities are classified by {\em types}. Types are used in
variable declarations, explicit coercions and conversions, object creation,
array creation, class literals, static state and method accessors, and
\xcd"instanceof" expressions.

The basic relationship between values and types is {\em instantiation}. For
example, \xcd`1` is an instance of type of integers, \xcd`Int`. It is also an
instance of type of all entities \xcd`Any`, and of type of nonzero integers
\xcd`Int{self != 0}`, and many others.

The basic relationship between types is {\em subtyping}: \xcd`T <: U` holds if
every instance of \xcd`T` is also an instance of \xcd`U`. Two important kinds
of subtyping are {\em subclassing} and {\em strengthening}.  Subclassing is a
familiar notion from object-oriented programming.  In a class
hierarchy with classes \xcd`Animal` and \xcd`Cat` arranged in the usual way,
every \xcd`Cat` is an \xcd`Animal`, so \xcd`Cat <: Animal` by subclassing.  
Strengthening is an equally familiar notion from logic.   The instances of
\xcd`Int{self != 0}` are all elements of \xcd`Int{true}` as well, because
\xcd`self != 0` logically implies \xcd`true`; so 
\xcd`Int{self != 0} <: Int{true} == Int` by strengthening.  X10 uses both
notions of subtyping.




\subsection*{The Grammar of Types}

Types are described by the following grammar: 
\bard{Is this still correct?}
\begin{grammar}
Type \: FunctionType \\
    \| ConstrainedType  \\

FunctionType \: TypeParameters\opt \xcd"(" Formals\opt \xcd")"
Constraint\opt Throws\opt \xcd"=>" Type \\
TypeParameters \: \xcd"[" TypeParameter ( \xcd"," TypeParameter )\star \xcd"]" \\
TypeParameter \: Identifier \\
Throws \: \xcd"throws" TypeName ( \xcd"," TypeName )\star \\

ConstrainedType \: Annotation\star BaseType Constraint\opt
PlaceConstraint\opt \\

BaseType \: ClassBaseType \\
     \| InterfaceBaseType \\
     \| PathType \\
     \| \xcd"(" Type \xcd")" \\

ClassType \: Annotation\star ClassBaseType Constraint\opt
PlaceConstraint\opt \\
InterfaceType \: Annotation\star InterfaceBaseType Constraint\opt
PlaceConstraint\opt \\

PathType \: Expression \xcd"." Identifier \\

Annotation \: \xcd"@" InterfaceBaseType Constraint\opt \\

ClassOrInterfaceType \: ClassType \\ \| InterfaceType \\
ClassBaseType \: TypeName \\
InterfaceBaseType \: TypeName \\
\end{grammar}

% \section{Type definitions and type constructors}
% 
% Types in \Xten{} are specified through declarations and through
% type constructors:

% \paragraph{Class types.}

\section{\xcd`Any`}

It is quite convenient to have a type which all values are instances of; that
is, a supertype of all types.\footnote{Java, for one, suffers a number of
  inconveniences because some built-in types like \xcd`int` and \xcd`char`
  aren't subtypes of anything else.}  X10's universal supertype is the
  interface \xcd`Any`. 


\begin{xten}
package x10.lang;
public interface Any {
  property def home():Place;
  property def at(p:Object):Boolean;
  property def at(p:Place):Boolean;
  global safe def toString():String;
  global safe def typeName():String;
  global safe def equals(Any):Boolean;
  global safe def hashCode():Int;
}
\end{xten}

\xcd`Any` provides a handful of essential methods that make sense and are
useful for everything.\footnote{The behavioral annotation \xcd`property` is
  explained in \Sref{properties}; \xcd`safe` in \Sref{SafeAnnotation}, and
  \xcd`global` in \Sref{GlobalAnnotation}.} \xcd`a.toString()` produces a
string representation of \xcd`a`, and \xcd`a.typeName()` the string
representation of its type; both are useful for debugging.  \xcd`aequals(b)`
is the programmer-overridable equality test, and \xcd`a.hashCode()` an integer
useful for hashing.  \xcd`at()` and \xcd`home()` are used in multi-place
computing. 



\section{Classes and interfaces}
\label{ReferenceTypes}

\subsection{Class types}

\index{types!class types}
\index{class}
\index{class declaration}
\index{declaration!class declaration}
\index{declaration!reference class declaration}

A {\em class declaration} (\Sref{XtenClasses}) introduces a {\em class type}
containing all instances of the class.  The \xcd`Position` class below
could describe the position of a slider control, for example.

%~~gen
% package Types.By.Cripes.Classes;
%~~vis
\begin{xten}
class Position {
  private var x : Int = 0;
  public def move(dx:Int) { x += dx; }
  public def pos() : Int = x;
}
\end{xten}
%~~siv
%
%~~neg

Class instances, also called objects, are created via constructor calls. Class
instances have fields and methods, type members, and value properties bound at
construction time. In addition, classes have static members: constant fields,
type definitions, and member classes and member interfaces.

A class with type parameters is {\em generic}. A class type is instantiatable
only if all of its parameters are instantiated on concrete types.  The
\xcd`Cell[T]` class provides a container capable of holding a value of type
\xcd`T`, or being empty.

%~~gen
% package Types.For.Gripes.Of.Wesley.Snipes;
%~~vis
\begin{xten}
class Cell[T] {
  var empty : Boolean = true;
  var contents : T;
  public def putIn(t:T) { 
    contents = t; empty = false; 
  }
  public def emptyOut() { empty = true; }
  public def isEmpty() = empty;
  public def getOut():T throws Exception {
     if (empty) throw new Exception("Empty!");
     return contents ;
  }
}
\end{xten}
%~~siv
%
%~~neg


\Xten{} does not permit mutable static state. A fundamental principle of the
X10 model of computation is that all mutable state be local to some place
(\Sref{XtenPlaces}), and, as static variables are globally available, they
cannot be mutable. When mutable global state is necessary, programmers should
use singleton classes, putting the state in an object and using place-shifting
commands (\Sref{AtStatement}) and atomicity (\Sref{AtomicBlocks}) as necessary
to mutate it safely.

\index{\Xcd{Object}}
\index{\Xcd{x10.lang.Object}}

Classes are structured in a single-inheritance hierarchy. All classes extend
the class \xcd"x10.lang.Object", directly or indirectly. Each class other than
\xcd`Object` extends a single parent class.  \xcd`Object` provides no behavior
of its own, beyond that required by \xcd`Any`.

\index{class!reference class}
\index{reference class type}
\index{\Xcd{Object}}
\index{\Xcd{x10.lang.Object}}


\index{null}
\bard{We've got to say this better.}
Variables of class type may contain the value \xcd"null". 

\subsection{Interface types}
\label{InterfaceTypes}

\index{types!interface types}
\index{interface}
\index{interface declaration}
\index{declaration!interface declaration}

An {\em interface declaration} (\Sref{XtenInterfaces}) defines an {\em
interface type}, specifying a set of methods, type members, and
properties which must be provided by any class declared to implement the
interface. 


Interfaces can also have static members: constant fields, type definitions,
and member classes and interfaces.  However, interfaces cannot specify that
implementing classes must provide static members.

An interface may extend multiple interfaces.  
%~~gen
%package Types.For.Snipes.Interfaces;
%~~vis
\begin{xten}
interface Named {
  def name():String;
}
interface Mobile {
  def move(howFar:Int):Void;
}
interface Person extends Named, Mobile {}
interface NamedPoint extends Named, Mobile{} 
\end{xten}
%~~siv
%
%~~neg


Classes may be declared to implement multiple interfaces.
Semantically, the interface type is the set of all objects that are
instances of classes that implement the interface. A class implements
an interface if it is declared to and if it implements all the methods
and properties defined in the interface.  For example, \xcd`KimThePoint`
implements \xcd`Person`, and hence \xcd`Named` and \xcd`Mobile`.  It would be
a static error if \xcd`KimThePoint` had no \xcd`name` method.

%~~gen
%interface Named {
%   def name():String;
% }
% interface Mobile {
%   def move(howFar:Int):Void;
% }
% interface Person extends Named, Mobile {}
% interface NamedPoint extends Named, Mobile{} 
%~~vis
\begin{xten}
class KimThePoint implements Person {
   var pos : Int = 0;
   public def name() = "Kim (" + pos + ")";
   public def move(dPos:Int) { pos += dPos; }
}
\end{xten}
%~~siv
%
%~~neg


\subsection{Properties}
\index{properties}
\label{properties}

Classes, interfaces, and structs may have {\em properties}, public \xcd`val` instance
fields bound on object creation. For example, the following code declares a
class named \xcd"Coords" with properties \xcd"x" and \xcd"y" and a \xcd"move"
method. The properties are bound using the \xcd"property" statement in the
constructor.

%~~gen
%package not.x10.lang;
%~~vis
\begin{xten}
class Coords(x: Int, y: Int) {
  def this(x: Int, y: Int) : Int{this.x==x, this.y==y} 
    = { property(x, y); }
  def move(dx: Int, dy: Int) = new Coords(x+dx, y+dy);
}
\end{xten}
%~~siv
%~~neg

Properties, unlike other public \xcd`val` fields, can be used  
at compile time in {\em constraints}. This allows us
to specify subtypes based on properties, by appending a boolean expression to
the type. For example, the type \xcd"Coords{x==0}" is the set of all points
whose \xcd"x" property is \xcd"0".  Details of this substantial topic are
found in \Sref{ConstrainedTypes}.



\section{Type parameters and Generic Types}
\label{TypeParameters}

\index{types!type parameters}
\index{methods!parametrized methods}
\index{constructors!parametrized constructors}
\index{closures!parametrized closures}
\label{Generics}
\index{types!generic types}

A class, interface, method, closure, or type definition  may have type
parameters.  Type parameters can be used as types, and will be bound to types
on instantiation.  For example, a generic stack class may be defined as 
\xcd`Stack[T]{...}`.  Stacks can hold values of any type; \eg, 
%~~type~~`~~`~~ ~~class Stack[T]{}
\xcd`Stack[Int]` is a stack of integers, and 
%~~type~~`~~`~~ ~~class Stack[T]{}
\xcd`Stack[Point{self!=null}]`is a stack of non-null \xcd`Point`s.
Generics {\em must} be instantiated when they are used: \xcd`Stack`, by
itself, is not a valid type.
Type parameters may be constrained by a guard on the declaration
(\Sref{ClassGuard}, \Sref{TypeDefGuard},
\Sref{MethodGuard},\Sref{ClosureGuard}).

\index{types!concrete types}
\index{concrete type}
A {\em generic type} is a class, struct,  interface, or type declared with one or
more type parameters.  When instantiated with concrete (\viz, non-generic)
types for its parameters, a generic type becomes a concrete type and can be
used like any other type. For example,
\xcd`Stack` is a generic type, 
%~~type~~`~~`~~ ~~class Stack[T]{}
\xcd`Stack[Int]` is a concrete type, and can be used as one: 
%~~stmt~~`~~`~~ ~~class Stack[T]{}
\xcd`var stack : Stack[Int];`


A \xcd`Cell[T]` is a generic object, capable of holding a value of type
\xcd`T`.  For example, a \xcd`Cell[Int]` can hold an \xcd`Int`, and a
\xcd`Cell[Cell[Int]{self!=0}]` can hold a \xcd`Cell[Int]` which in turn can
only hold non-zero numbers.  \xcd`Cell`s are actually useful in situations
where values must be bound immutably for one reason, but need to be mutable.
%~~gen
% package ch4;
%~~vis
\begin{xten}
class Cell[T] {
    var x: T;
    def this(x: T) { this.x = x; }
    def get(): T = x;
    def set(x: T) = { this.x = x; }
}
\end{xten}
%~~siv
%~~neg


\xcd"Cell[Int]" is the type of \xcd`Int`-holding cells.  
The \xcd"get" method on a \xcd`Cell[Int]` returns an \xcd"Int"; the
\xcd"set" method takes an \xcd"Int" as argument.  Note that
\xcd"Cell" alone is not a legal type because the parameter is
not bound.

\subsection{Variance of Type Parameters}
\index{covariant}
\index{contravariant}
\index{invariant}
\index{type parameter!covariant}
\index{type parameter!contravariant}
\index{type parameter!invariant}

Consider classes \xcd`Person :> Child`.  Every child is a person, but there
are people who are not children.  What is the relationship between
\xcd`Cell[Person]` and \xcd`Cell[Child]`?  

\subsubsection{Why Variance Is Necessary}

In this case, \xcd`Cell[Person]` and \xcd`Cell[Child]` should be unrelated.  
If we had \xcd`Cell[Person] :> Cell[Child]`, the following code would let us
assign a \xcd`old` (a \xcd`Person` but not a \xcd`Child`) to a
variable \xcd`young` of type \xcd`Child`, thereby breaking the type system: 
\begin{xten}
// INCORRECTLY assuming Cell[Person] :> Cell[Child]
val cc : Cell[Child] = new Cell[Child]();
val cp : Cell[Person] = cc; // legal upcast
cp.set(old);       // legal since old : Person
val young : Child = cc.get(); 
\end{xten}

Similarly, if \xcd`Cell[Person] <: Cell[Child]`: 
\begin{xten}
// INCORRECTLY assuming Cell[Person] <: Cell[Child]
val cp : Cell[Person] = new Cell[Person];
val cc : Cell[Child] = cp; // legal upcast
val cp.set(old); 
val young : Child = cc.get();
\end{xten}

So, there cannot be a subtyping relationship in either direction between the
two. And indeed, neither of these programs passes the X10 typechecker.


\subsubsection{Legitimate Variance}

The \xcd`Cell[Person]`-vs-\xcd`Cell[Child]` problems occur because it is
possible to both store and retrieve values from the same object. However,
entities with only one of the two capabilities {\em can} sensibly have some
subtyping relations. Furthermore, both sorts of entity are useful. An entity
which can store values but not retrieve them can nonetheless summarize them.
An object which can retrieve values but not store values can be constructed
with an initial value, providing a read-only cell.

So, X10 provides {\em variance} to support these options.  Type parameters
may be defined in one of three forms.  
\begin{enumerate}
\item {\em invariant}: Given a definition \xcd`class C[T]{...}`, \xcd`C[Person]` and
      \xcd`C[Child]` are unrelated classes; neither is a subclass of the
      other.
\item {\em covariant}: Given a definition \xcd`class C[+T]{...}` (the \xcd`+` indicates
      covariance), \xcd`C[Person] :> C[Child]`.  This is appropriate when
      \xcd`C` allows retrieving values but not setting them.
\item {\em contravariant}: Given a definition \xcd`class C[-T]{...}` (the \xcd`-` indicates
      contravariance), \xcd`C[Person] <: C[Child]`.  This is appropriate when
      \xcd`C` allows storing values but not retrieving them.
\end{enumerate}


The \xcd"T" parameter of \xcd"Cell" above is
invariant.  

A typical example of covariance is \xcd`Get`.  As the \xcd`example()` method
shows, a \xcd`Get[T]` must be constructed with its value, and will return that
value whenever desired.
%~~gen
% package ch4;
%~~vis
\begin{xten}
class Get[+T] {
  var x: T;
  def this(x: T) { this.x = x; }
  def get(): T = x;
  static def example() {
     val g : Get[Int]! = new Get[Int](31);
     val n : Int = g.get();
     x10.io.Console.OUT.print("It's " + n);
     x10.io.Console.OUT.print("It's still " + g.get());
  }
}
\end{xten}
%~~siv
%~~neg


A typical example of contravariance is \xcd`Set`.  As the \xcd`example()`
method shows,  a variety of objects\footnote{Objects but no structs.  If we
had wanted structs too, we could have used a \xcd`Cell[Any]`.}  can be put into a
\xcd`Set[Object]`.  While the object itself cannot be retrieved, some summary
information about it -- in this case, its \xcd`typeName` -- can be.  
%~~gen
% package ch4;
%~~vis
\begin{xten}
class Set[-T] {
  var x: T;
  def this(x: T) { this.x = x; }
  def set(x: T) = { this.x = x; } 
  def summary(): String = this.x.typeName();
  static def example() {
    val s : Set[Object]! = new Set[Object](new Throwable());
    s.summary(); // == "x10.lang.Throwable"
    s.set("A String");
    s.summary(); // == "x10.lang.String";
  }    
}
\end{xten}
%~~siv
%
%~~neg


Given types \xcd"S" and \xcd"T": 
\begin{itemize}
\item
If the parameter of \xcd"Get" is covariant, then
\xcd"Get[S]" is a subtype of \xcd"Get[T]" if
\xcd"S" is a {\em subtype} of \xcd"T".

\item
If the parameter of \xcd"Set" is contravariant, then
\xcd"Set[S]" is a subtype of \xcd"Set[T]" if
\xcd"S" is a {\em supertype} of \xcd"T".

\item
If the parameter of \xcd"Cell" is invariant, then
\xcd"Cell[S]" is a subtype of \xcd"Cell[T]" if
\xcd"S" is a {\em equal} to \xcd"T".
\end{itemize}


In order to make types marked as covariant and contravariant semantically
sound, X10 performs extra checks.  
A covariant type parameter is permitted to appear only in covariant type positions,
and a contravariant type parameter in contravariant positions. 
\begin{itemize}
\item The return type of a method is a covariant position.
\item The argument types of a method are contravariant positions.
\item Whether a type argument position of a generic class, interface or struct type \Xcd{C}
is covariant or contravariant is determined by the \Xcd{+} or \Xcd{-} annotation
at that position in the declaration of \Xcd{C}.
\end{itemize}

There are similar restrictions on use of covariant and contravariant values. 
\bard{Get them!  What are they?}


\section{Type definitions}
\label{TypeDefs}

\index{types!type definitions}
\index{declarations!type definitions}
\input{TypeDefs.tex}


\section{Constrained types}
\label{ConstrainedTypes}
\label{DepType:DepType}
\label{DepTypes}

\index{dependent types}
\index{constrained types}
\index{generic types}
\index{types!constrained types}
\index{types!dependent types}
\index{types!generic types}


Basic types, like \xcd`Int` and \xcd`List[String]`, provide useful
descriptions of data.  Indeed, most typed programming languages get by with no
more specific descriptions.

However, there are a lot of things that one frequently wants to say about
data.  One might want to know that a \xcd`String` variable is not \xcd`null`,
or that a matrix is square, or that one matrix has the same number of columns
that another has rows (so they can be multiplied).  In the multicore setting,
one might wish to know that two values are located at the same processor.

In most languages, there is simply no way to say these things statically.
Programmers must made do with comments, \xcd`assert` statements, and dynamic
tests.  X10 can do better, with {\em constraints} on types (and methods and
other things).

A constraint is a boolean expression \xcd`e` attached to a basic type \xcd`T`,
written \xcd`T{e}`.  (Only a limited selection of boolean expressions is
available.)  The values of type \xcd`T{e}` are the values of \xcd`T` for which
\xcd`e` is true.  For example: 

\begin{itemize}
%~~type~~`~~`~~ ~~
\item \xcd`String{self != null}` is the type of non-null strings.  \xcd`self`
      is a special variable available only in constraints; it refers to the
      datum being constrained.   
\item If \xcd`Matrix` has properties \xcd`rows` and \xcd`cols`, 
%~~type~~`~~`~~ ~~class Matrix(rows:Int,cols:Int){}
      \xcd`Matrix{rows == cols}` is the type of square matrices.
\item One way to say that \xcd`a` has the same number of columns that \xcd`b`
      has rows (so that \xcd`a*b` is a valid matrix product), one could say: 
%~~gen
% package Types.cripes.whered.you.get.those.gripes;
% class Matrix(rows:Int, cols:Int){
% public static def someMatrix(): Matrix = null;
% public static def example(){
%~~vis
\begin{xten}
  val a : Matrix = someMatrix() ;
  var b : Matrix{b.rows == a.cols} ;
\end{xten}
%~~siv
%}}
%~~neg

\item One way to say that objects \xcd`c` and \xcd`d` are located at the same
      place is: 
%~~gen
% package Types.flipes.knipes.shipes.wipes;
% class Exampler {
% static def someObject(): Object = null;
% static def example() {
%~~vis
\begin{xten}
  val a : Object = someObject();
  var b : Object{a.home == b.home};
\end{xten}
%~~siv
%}}
%~~neg
\end{itemize}



%%BARD-HERE


Given a type \xcd"T", a {\em constrained type} \xcd"T{e}" may be constructed
by constraining its properties with a boolean expression of a limited sort
\xcd"e". The values of \xcd`T{e}` are those values of type \xcd`T` for which
\xcd`e` evaluates to \xcd`true`.  \Eg, \xcd`Point` has a property
\xcd`rank:Int`.  If \xcd`p : Point`, \xcd`p` may have any \xcd`rank`.
\xcd`Point{rank == 3}` is the point type constrained to only those values
whose \xcd`rank` property is 3.   

A common use of constrained types is to explain where objects are located.
Every object has a \xcd`home` property. If \xcd`Person` is a type of people,
then \xcd`Person{home==here}` is the type of people whose data is stored at
the current location.  As explained in \Sref{XtenPlaces}, certain operations
can only be performed at an object's home, so having this expressible as a
type is crucial.


\xcd"T{e}" is a {\em dependent type}, that is, a type dependent on values. The
type \xcd"T" is called the {\em base type} and \xcd"e" is called the {\em
constraint}. 

For brevity, the constraint may be omitted and
interpreted as \xcd"true".

Constraints may refer to values in the local environment: 
%~~gen
% class ConstraintsMayReferToValues {
% def thoseValues() {
%~~vis
\begin{xten}
     val n = 1;
     var p : Point{rank == n};
\end{xten}
%~~siv
%}}
%~~neg
Indeed, there is technically no need for a constraint to refer to the
properties of its type; it can refer entirely to the environment, thus: 
%~~gen
% class ConstraintsMayReferToValuesTwo {
% def thoseValues() {
%~~vis
\begin{xten}
     val m = 1;
     val n = 2;
     var p : Point{m != n};
\end{xten}
%~~siv
%}}
%~~neg

Constraints on properties induce a natural subtyping relationship:
\xcd"C{c}" is a subtype of
\xcd"D{d}" if \xcd"C" is a subclass of \xcd"D" and
\xcd"c" entails \xcd"d".

Type parameters cannot be constrained.

\subsection{Constraints}

\def\withmath#1{\relax\ifmmode#1\else{$#1$}\fi}
\def\LL#1{\withmath{\lbrack\!\lbrack #1\rbrack\!\rbrack}}

Expressions used as constraints are restricted by the constraint
system in use to ensure that the constraints can be solved at compile
time.  The \Xten{} compiler allows compiler plugins to be installed to
extend the constraint language and the constraint system.  Constraints
must be of type \xcd"Boolean".  The compiler supports the following
constraint syntax.


\begin{grammar}
Constraint \: ValueArguments     Guard\opt \\
           \| ValueArguments\opt Guard     \\
           \\
ValueArguments   \:  \xcd"(" ArgumentList\opt \xcd")" \\
ArgumentList     \:  Expression ( \xcd"," Expression )\star \\
Guard            \: \xcd"{" DepExpression \xcd"}" \\
DepExpression    \: ( Formal \xcd";" )\star ArgumentList \\
\end{grammar}

In \XtenCurrVer{} value constraints may be equalities (\xcd"=="),
disequalities (\xcd"!=") and conjunctions thereof.  The terms over
which these constraints are specified include literals and
(accessible, immutable) variables and fields, property methods, and the special
constants {\tt here}, {\tt self}, and {\tt this}. Additionally, place
types are permitted (\Sref{PlaceTypes}).

\index{self}
When constraining a value of type \xcd`T`, \xcd`self` refers to the object of
type \xcd`T` which is being constrained.  For example, \xcd`Int{self == 4}` is
the type of \xcd`Int`s which are equal to 4 -- the best possible description
of \xcd`4`, and a very difficult type to express without using \xcd`self`.  


Type constraints may be subtyping and supertyping (\xcd"<:" and
\xcd":>") expressions over types.

The static constraint checker approximates computational reality in some
cases.  For example, it assumes that built-in types are infinite. This is a
good approximation for \xcd`Int`.  It is a poor approximation for \xcd`Boolean`,
as the checker believes that \xcd`a != b && a != c && b != c` is satisfiable
over \xcd`Boolean`, which it is not.  However, the checker is always correct
when computing the truth or falsehood of a constraint.


% //, and existential quantification over typed variables.

\emph{
Subsequent implementations are intended to support boolean algebra,
arithmetic, relational algebra, etc., to permit types over regions and
distributions. We envision this as a major step towards removing most,
if not all, dynamic array bounds and place checks from \Xten{}.
}


\subsubsection{Acyclicity restriction}

To ensure that type-checking is decidable, we
require that property graphs be acyclic.
That is, it should not be the case at runtime that
a set of objects can be created such that the
graph formed by taking objects as nodes and adding an edge from $m$ to
$n$ if $m$ has a property whose value is $n$ has a cycle in it.

Currently this restriction is not checked by the compiler. Future
versions of the compiler will check this restriction by introducing
rules on escaping of \Xcd{this} (\Sref{protorules}) before the invocation of
\Xcd{property} calls.


\input{PlaceTypes}

\subsection{Constraint semantics}

\begin{staticrule}{Variable occurrence}
In a dependent type \xcd"T" = \xcd"C{c}", the only variables that may
occur in \xcd"c" are (a)
\xcd"self", (b) properties visible at \xcd"T", (c)  local \xcd`val`s, \xcd`val`
method parameters or \xcd`val` constructor parameters visible at \xcd"T", (d)
\xcd`val` fields visible at \xcd"T"'s lexical place in the source program.  
\end{staticrule}

\begin{staticrule}{Restrictions on \xcd"this"}
  The special variable \xcd"this" may be used in a dependent clause for a type \xcd"T"
  only if \xcd`this` may be used in an expression at that point.  \Viz, if 
  \xcd"this" occurs in (a) a property declaration for a
  class, (b) an instance method, (c) an
  instance field, or (d) an instance initializer.

  In particular, \xcd"this" may not be used in types that occur in a static
  context, or in the arguments, body or return type of a constructor or
  in the extends or implements clauses of class and interface
  definitions.  In these contexts, the object that \xcd"this" would
  correspond to is not defined.
\end{staticrule}

\begin{staticrule}{Variable visibility}
  If a type \xcd"T" occurs in a field, method or constructor
  declaration, then all variables used in \xcd"T" must have at least the
  same visibility as the declaration.  The relation ``at least the same
  visibility as'' is given by the transitive closure of:

\begin{xten}
public > protected > package > private
\end{xten}

All inherited properties of a type \xcd"T" are visible in the property
list of \xcd"T", and the body of \xcd"T".

\end{staticrule}

In general, variables (i.e., local variables, parameters,
properties, fields) are visible at
\xcd"T" if they are defined before \xcd"T" in the program. This rule applies to
types in property lists as well as parameter lists (for methods and
constructors).
A formal parameter is visible in the types of all other formal
parameters of the same method, constructor, or type definition,
as well as in the method or constructor body itself.
Properties are accessible via their containing object--\xcd"this"
within the body of their class declaration.  The special
variable \xcd"this" is in scope at each property
declaration, constructor signatures and bodies, instance method signatures
and bodies,
and instance field signatures and initializers, but not in scope
at \xcd"static" method or field declarations or \xcd"static"
initializers.  

We permit variable declarations \xcd"v: T" where \xcd"T" is obtained
from a dependent type \xcd"C{c}" by replacing one or more occurrences
of \xcd"self" in \xcd"c" by \xcd"v". (If such a declaration \xcd"v: T"
is type-correct, it must be the case that the variable \xcd"v" is not
visible at the type \xcd"T". Hence we can always recover the
underlying dependent type \xcd"C{c}" by replacing all occurrences of \xcd"v"
in the constraint of \xcd"T" by \xcd"self".)

For instance, \xcd"v: Int{v == 0}" is shorthand for \xcd"v: Int{self == 0}".

\begin{staticrule}{Constraint type}
The type of a constraint \xcd"c" must be \xcd"Boolean".  
\end{staticrule}

A variable occurring in the constraint \xcd"c" of a dependent type, other than
\xcd"self" or a property of \xcd"self", is said to be a {\em
parameter} of \xcd"c".\label{DepType:Parameter} \index{parameter}

An instance \xcd"o" of \xcd"C" is said to be of type \xcd"C{c}"
(or: {\em belong to}
\xcd"C{c}") if the predicate \xcd"c" evaluates to \xcd"true" in the current lexical
environment, augmented with the binding \xcd"self" $\mapsto$ \xcd"o". We shall
use the function \LL{\mbox{\Xcd{C\{c\}}}} to denote the set of
objects that belong to \xcd"C{c}". 


\subsection{Consistency of dependent types}\label{DepType:Consistency}\index{dependent type,consistency}

A dependent type \xcd"C{c}" may contain zero or more parameters. We require
that a type never be empty---so that it is possible for a variable of
the type to contain a value. This is accomplished by requiring that
the constraint \xcd"c" must be satisfiable {\em regardless} of the value assumed
by parameters to the constraint (if any). Formally, consider a type
\xcd"T" = \xcd"C{c}", with the variables
\xcdmath"f$_1$: F$_1$, $\dots$, f$_k$: F$_k$"
free in \xcd"c".  Let 
\xcdmath"$S$ = {f$_1$: F$_1$, $\dots$, f$_k$: F$_k$, f$_{k+1}$: F$_{k+1}$, $\dots$, f$_n$: F$_n$}"
be the smallest set of
declarations containing
\xcdmath"f$_1$: F$_1$, $\dots$, f$_k$: F$_k$"
and closed under the rule:
\xcd"f: F" in $S$ if a reference to variable \xcd"f" (which
is declared as \xcd"f: F") occurs in a type in $S$.

(NOTE: The syntax rules for the language ensure that $S$ is always
finite. The type for a variable \xcd"v" cannot reference a variable whose
type depends on \xcd"v".)

We say that \xcd"T" = \xcd"C{c}" is {\em parametrically consistent} (in brief:
{\em consistent}) if:

\begin{itemize}
\item Each type \xcdmath"F$_1$, $\dots$, F$_n$" is (recursively) parametrically consistent, and
\item It can be established that
\xcdmath"$\forall$f$_1$: F$_1$, $\dots$, f$_n$: F$_n$. $\exists$self: C. c && $\mathit{inv}$(C)".
\end{itemize}

\noindent
where \xcdmath"$\mathit{inv}$(C)" is the invariant associated
with the type \xcd"C" (\Sref{DepType:TypeInvariant}).  Note by
definition of $S$ the formula above has no free variables.

\begin{staticrule*}
For a declaration \xcd"v: T" to be type-correct, \xcd"T" must be parametrically
consistent. The compiler issues an error if it cannot determine
the type is parametrically consistent.
\end{staticrule*}

\begin{example}

A class that represents a line has two distinct points:\footnote{We call them
\xcd`Position` to avoid confusion with the built-in class \xcd`Point`}

%~~gen
% 
%~~vis
\begin{xten}
class Position(x: Int, y: Int) {
   def this(x:Int,y:Int){property(x,y);}
   }
class Line(start: Position, 
           end: Position{self != start}) {}
\end{xten}
\end{example}
%~~siv
%~~neg


\begin{example}
One can use dependent type to define other closed geometric figures as well.

To see that the declaration \xcd"end: Position{self != start}" is
parametrically consistent, note that the following formula is valid:
\begin{xtenmath}
$\forall$this: Line. $\exists$self: Position. self != this.start  
\end{xtenmath}
\noindent since the set of all \xcd"Position"s has more than one element.
\end{example}

\begin{example}
A triangle has three lines sharing three vertices.

%~~gen
%package triangleExample;
% class Position(x: Int, y: Int) {
%    def this(x:Int,y:Int){property(x,y);}
%    }
% class Line(start: Position, 
%            end: Position{self != start}) {}
% 
%~~vis
\begin{xten}
class Triangle 
 (a: Line, 
  b: Line{a.end == b.start}, 
  c: Line{b.end == c.start && c.end == a.start}) 
 {
   def this(a:Line,
            b: Line{a.end == b.start}, 
            c: Line{b.end == c.start && c.end == a.start}) 
   {property(a,b,c);}
 }
\end{xten}
%~~siv
%
%~~neg

Given \xcd"a: Line", the type \xcd"b: Line{a.end == b.start}" is consistent,
and
given the two, the type \xcd"c: Line{b.end == c.start, c.end == a.start}"
is consistent.

%%Similarly:
%%
%%   // A class with properties a, b,c,x satisfying the 
%%   // given constraints.
%%   class SolvableQuad(a: Int, b: Int, 
%%                      c: Int{b*b - 4*a*c >= 0},
%%                      x: Int{a*x*x + b*x + c==0}) { 
%%     ...
%%   }
%%
%%  // A class with properties r, x, and y satisfying
%%  // the conditions for (x,y) to lie on a circle with center (0,0)
%%  // and radius r.
%%   class Circle (r: Int{r > 0},
%%                 x: Int{r*r - x*x >= 0},
%%                 y: Int{y*y == r*r -x*x}) { 
%%   ...
%%   }
\end{example}

\section{Function types}
\label{FunctionTypes}
\label{FunctionType}
\index{function!types}
\index{types!function types}

        Function types are defined via the \xcd"=>" type
        constructor.  Closures (\Sref{Closures}) and method
        selectors (\Sref{MethodSelectors}) are of function type.
        The general form of a function type is:
\begin{xtenmath}
(x$_1$: T$_1$, $\dots$, x$_n$: T$_n$){c} => T
        throws S$_1$, $\dots$, S$_k$
\end{xtenmath}
        This
        is the type of functions that take 
        value parameters
        \xcdmath"x$_i$"
        of types
        \xcdmath"T$_i$"
        such that the guard \xcd"c" holds
        and returns a value of type \xcd"T" or throws
        exceptions of 
        types S$_i$.

The value parameters are in scope throughout the function
signature---they may be used in the types of other formal parameters
and in the return type.  Value parameters names  may be
omitted if they are not used.  The guard specifies a condition that 
must hold for an application to be well-typed.

\begin{grammar}
FunctionType \: TypeParameters\opt \xcd"(" Formals\opt \xcd")" Constraint\opt
\xcd"=>" Type Throws\opt \\
TypeParameters \: \xcd"[" TypeParameter ( \xcd"," TypeParameter
)\star \xcd"]" \\
TypeParameter \: Identifier \\
Formals \: Formal ( \xcd"," Formal )\star \\
\end{grammar}


For every sequence of types \xcd"T1,..., Tn,T", and \xcd"n" distinct variables
\xcd"x1,...,xn" and constraint \xcd"c", the expression
\xcd"(x1:T1,...,xn:Tn){c}=>T" is a \emph{function type}. It stands for
 the set of all functions \xcd"f" which can be applied in a place \xcd"p" to a
 list of values \xcd"(v1,...,vn)" provided that the constraint
 \xcd"c[v1,...,vn,p/x1,...,xn,here]" is true, and which returns a value of
 type \xcd"T[v1,...vn/x1,...,xn]". When \xcd"c" is true, the clause \xcd"{c}" can be
 omitted. When \xcd"x1,...,xn" do not occur in \xcd"c" or \xcd"T", they can be
 omitted. Thus the type \xcd"(T1,...,Tn)=>T" is actually shorthand for
 \xcd"(x1:T1,...,xn:Tn){true}=>T", for some variables \xcd"x1,...,xn".


Juxtaposition is used to express function application: the expression
\xcd"f(a1,..,an)" expresses the application of a function \xcd"f" to the argument
list \xcd"a1,...,an".

\index{Exception!unchecked}
Note that function invocation may throw unchecked exceptions. 

A function type is covariant in its result type and contravariant in
each of its argument types. That is, let 
\xcd"S1,...,Sn,S,T1,...Tn,T" be any
types satisfying \xcd"Si <: Ti" and \xcd"S <: T". Then
\xcd"(x1:T1,...,xn:Tn){c}=>S" is a subtype of
\xcd"(x1:S1,...,xn:Sn){c}=>T".


A value \xcd"f" of a function type \xcd"(x1:T1,...,xn:Tn){c}=>T" also
has all the methods of \Xcd{Any} associated with it (see \Sref{FunctionAnyMethods}).


A function type \xcd"F=(x1:T1,...,xn:Tn){c}=>T" can be used as the declared type of local variables, parameters, loop variables, return types of methods and in \xcd"_� instanceof F" and \xcd"_ as F" expressions. 


A class or struct definition may use a function type \xcd"F" in its
implements clause; this declares an abstract method 
\xcd"def apply(x1:T1,...,xn:Tn){c}:T" on that class. Similarly, an interface
definition may specify a function type "F" in its "extends" clause. A
class or struct implementing such an interface implicitly defines an
abstract method \xcd"def apply(x1:T1,..,xn:Tn){c}:T". Expressions of such
a struct, class or interface type can be assigned to variables of type
\xcd"F" and can be applied via juxtaposition to an argument list of the
right type.


Thus, objects and structs in \Xten{} may behave like functions. 

A function type \xcd"F" is not a class type in that it does not extend any
type or implement any interfaces, or support equality tests. \xcd"F" cannot be extended by any type. It
is not an interface type in that it is not a subtype of
\xcd"x10.lang.Object". (Values of type \xcd"F" cannot be assigned to variables of
type \xcd"x10.lang.Object".) It is not a struct type in that it has no
defined fields and hence no notion of structural equality.

\xcd"null" is a legal value for a function type. 


\section{Annotated types}
\label{AnnotatedTypes}

\index{types!annotated types}
\index{annotations!type annotations}

        Any \Xten{} type may be annotated with zero or more
        user-defined \emph{type annotations}
        (\Sref{XtenAnnotations}).  

        Annotations are defined as (constrained) interface types and are
        processed by compiler plugins, which may interpret the
        annotation symbolically.

        A type \xcd"T" is annotated by interface types
        \xcdmath"A$_1$", \dots,
        \xcdmath"A$_n$"
        using the syntax
        \xcdmath"@A$_1$ $\dots$ @A$_n$ T".

\section{Subtyping and type equivalence}\label{DepType:Equivalence}
\index{type equivalence}
\index{subtyping}

Subtyping is relation between types.  It is the
reflexive, transitive 
closure of the {\em direct subtyping} relation, defined as
follows.

\paragraph{Class types.}  A class type is a direct subtype of
any
class it is declared to extend.  A class type is direct subtype
of any interfaces it is declared to implement.

\paragraph{Interface types.}  An interface type is a direct
subtype of any interfaces it is declared to extend.

\paragraph{Function types.}

Function types are covariant on their return type and
contravariant on their argument types.
For instance,
a function type
\xcd"(S1) => T1" 
is a subtype of another function type
\xcd"(S2) => T2" 
if \xcd"S2" is a subtype of \xcd"S1"
and \xcd"T1" is a subtype of \xcd"T2".

\paragraph{Constrained types.}

Two dependent types \xcd"C{c}" and \xcd"C{d}" are said to be {\em equivalent} if 
\xcd"c" is true whenever \xcd"d" is, and vice versa. Thus, 
$\LL{\mbox{\Xcd{C\{c\}}}} = \LL{\mbox{\Xcd{C\{d\}}}}$.

Note that two dependent type that are syntactically different may be
equivalent. For instance, \xcd"Int{self >= 0}" and
\xcd"Int{self == 0 || self > 0}" are equivalent though they are syntactically
distinct. The \Java{} type system is essentially a nominal system---two
types are the same if and only if they have the same name. The \Xten{}
type system extends the nominal type system of \Java{} to permit
constraint-based equivalence.

A dependent type \xcd"C{c}" is a subtype of a type \xcd"C{d}" if
\xcd"c" implies \xcd"d".  When this subtyping relationship holds, 
$\LL{\mbox{\Xcd{C\{c\}}}}$ is a
subset of $\LL{\mbox{\Xcd{C\{d\}}}}$. All dependent types
defined on a class \xcd"C"
refine the unconstrained class type \xcd"C"; \xcd"C" is
equivalent to \xcd"C{true}".

\paragraph{Type parameters.}

A type parameter \xcd"X" of a class or interface \xcd"C"
is a subtype of a type \xcd"T" if
the 
class invariant of \xcd"C" implies that \xcd"X" is a subtype of \xcd"T".
Similarly, \xcd"T" is a subtype of parameter \xcd"X" if the
class invariant implies the relationship.

A type parameter \xcd"X" of a method
\xcd"m"
is a subtype of a type \xcd"T" if
the 
guard of \xcd"m" implies that \xcd"X" is a subtype of \xcd"T".
Similarly, \xcd"T" is a subtype of parameter \xcd"X" if the
guard implies the relationship.


\section{Least common ancestor of types}
\label{LCA}

To compute the type of conditional expressions
(\Sref{Conditional}),
and of rail constructors
(\Sref{RailConstructors}), the least common ancestor of types
must be computed.

The least common ancestor of two  types
\xcdmath"T$_1$" and \xcdmath"T$_2$"
is the
unique most-specific type
that is a supertype of both
\xcdmath"T$_1$" and \xcdmath"T$_2$".

If the most-specific type is not unique (which can happen when
\xcdmath"T$_1$" and \xcdmath"T$_2$" both implement two
or more incomparable interfaces), then
least common ancestor type is \xcd"x10.lang.Any".

\input{Coercions}

%\subsection{Syntactic abbreviations}\label{DepType:SyntaxAbbrev}

\section{Built-in types}

The package \xcd"x10.lang" provides a number of built-in class and
interface declarations that can be used to construct types.

\subsection{The class \Xcd{Object}}
\label{Object}
\index{\Xcd{Object}}
\index{\Xcd{x10.lang.Object}}

The class \xcd"x10.lang.Object" is the supertype of all classes.
A variable of this type can hold a reference to any object.
The code for this class (with annotations removed) is:
\begin{xten}
public class Object (home: Place) 
     implements Any
{
    public native def this();
    public property def home() = home;
    public property def at(p:Place) = home==p;
    public property def at(r:Object) = home==r.home;
    public global safe native def toString() : String;
    public global safe native def typeName() : String;
    public global safe def equals(x:Any) = this == x;
    public global safe native def hashCode():Int;
}
\end{xten}

\subsection{The class \Xcd{String}}
\label{String}\index{\Xcd{String}}\index{\Xcd{x10.lang.String}}

Strings in \Xten{} are instances of the class \xcd"x10.lang.String", and are
all immutable.
Strings are one of the few types with literals, rather than simply
      constructors.  String literals are the familiar \xcd`"`-delimited
      strings, like \xcd`"this"` and \xcd`"that"`.

Every X10 value has a \xcd`String` print representation, given by
      \xcd`whatever.toString()`.   
All values can be implicitly converted to strings by the concatenation
      operation \xcd`+`, which calls their \xcd`toString()` methods if they
      are not strings already.  For example, 
%~~exp~~`~~`~~ ~~
      \xcd`"one " + 2 + here` 
      evaluates to something like \xcd`one 2(Place 0)`.  



\input{ArrayTypes}
\input{FutureTypes}

\section{Type inference}
\label{TypeInference}
\index{types!inference}
\index{type inference}

\XtenCurrVer{} supports limited local type inference, permitting
variable types and return types to be elided.
It is a static error if an omitted type cannot be inferred or
uniquely determined.

\subsection{Variable declarations}

The type of a variable declaration can be omitted if the
declaration has an initializer.  The inferred type of the
variable is the computed type of the initializer.

\subsection{Return types}

The return type of a method can be omitted if the method has a
body (i.e., is not \xcd"abstract" or \xcd"extern").  The
inferred return type is the computed type of the body.

The return type of a closure can be omitted.
The inferred return type is the computed type of the body.

The return type of a constructor can be omitted if the
constructor has a body (i.e., is not \xcd"extern").
The inferred return type is the enclosing class type with
properties bound to the arguments in the constructor's \xcd"property"
statement, if any, or to the unconstrained class type.

\index{Void}
The inferred type of a method or closure body is the least common ancestor
of the types of the expressions in \xcd"return" statements
in the body.  If the method does not return a value, the
inferred type is \xcd"Void".

\subsection{Type arguments}

A call to a polymorphic method %, closure, or constructor 
may omit the
explicit type arguments.  If the method has a type parameter
\xcd"T", the type argument corresponding to \xcd"T" is inferred
to be the least common ancestor of the types of any formal
parameters of type \xcd"T".

%TODO--check this!
Consider the following method:
\begin{xten}
def choose[T](a: T, b: T): T { ... }
\end{xten}
%
Given \xcd"Set[T] <: Collection[T]", 
\xcd"List[T] <: Collection[T]",
and \xcd"SubClass <: SuperClass",
in the following snippet, the algorithm will infer the type
\xcd"Collection[Any]" for \xcd"x".
\begin{xten}
def m(intSet: Set[Int], stringList: List[String]) {
  val x = choose(intSet, stringList);
...
}
\end{xten}
%
And in this snippet, the algorithm should infer the type
\xcd"Collection[Int]" for \xcd"y".
\begin{xten}
def m(intSet: Set[Int], intList: List[Int]) {
  val y = choose(intSet, intList);
  ...
}
\end{xten}
%
Finally, in this snippet, the algorithm should infer the type
\xcd"Collection{T <: SuperClass}" for \xcd"z".
\begin{xten}
def m(intSet: Set[SubClass], numList: List{T <: SuperClass}) {
  val z = choose(intSet, numList);
  ...
}
\end{xten}

	

\chapter{Variables}\label{XtenVariables}\index{variable}

%%OLDA variable is a storage location.  \Xten{} supports seven kinds of
%%OLDvariables: constant {\em class variables} (static variables), {\em
%%OLD  instance variables} (the instance fields of a class), {\em array
%%OLD  components}, {\em method parameters}, {\em constructor parameters},
%%OLD{\em exception-handler parameters} and {\em local variables}.

A {\em variable} is an X10 identifier associated with a value within some
context. Variable bindings have these essential properties:
\begin{itemize}
\item {\bf Type:} What sorts of values can be bound to the identifier;
\item {\bf Scope:} The region of code in which the identifier is associated
      with the entity;
\item {\bf Lifetime:} The interval of time in which the identifier is
      associated with the entity.
\item {\bf Visibility:} Which parts of the program can read or manipulate the
      value through the variable.
\end{itemize}



X10 has many varieties of variables, used for a number of purposes. They will
be described in more detail in this chapter.  
\begin{itemize}
\item Class variables, also known as the static fields of a class, which hold
      their values for the lifetime of the class.  
\item Instance variables, which hold their values for the lifetime of an
      object;
\item Array elements, which are not individually named and hold their values
      for the lifetime of an array;
\item Formal parameters to methods, functions, and constructors, which hold
      their values for the duration of method (etc.) invocation;
\item Local variables, which hold their values for the duration of execution
      of a block.
\item Exception-handler parameters, which hold their values for the execution
      of the exception being handled. 
\end{itemize}
A few other kinds of things are called variables for historical reasons; \eg,
type parameters are often called type variables, despite not being variables
in this sense because they do not refer to X10 values.  Other named entities,
such as classes and methods, are not called variables.  However, all
name-to-whatever bindings enjoy similar concepts of scope and visibility.  

In the following example, \xcd`n` is an instance variable, and \xcd`nxt` is a
local variable defined within the method \xcd`bump`.\footnote{This code is
unnecessarily turgid for the sake of the example.  One would generally write
\xcd`public def bump() = ++n;`.   }
%~~gen
% package Vars.For.Squares;
%~~vis
\begin{xten}
class Counter {
  private var n : Int = 0;
  public def bump() : Int {
    val nxt = n+1;
    n = nxt;
    return nxt;
    }
}
\end{xten}
%~~siv
%
%~~neg
Both variables have type \xcd`Int` (or
perhaps something more specific).    The scope of \xcd`n` is the body of
\xcd`Counter`; the scope of \xcd`nxt` is the body of \xcd`bump`.  The
lifetime of \xcd`n` is the lifetime of the \xcd`Counter` object holding it;
the lifetime of \xcd`nxt` is the duration of the call to \xcd`bump`. Neither
variable can be seen from outside of its scope.

\label{exploded-syntax}
\label{VariableDeclarations}
\index{variable declaration}


Variables whose value may not be changed after initialization are said to be
{\em immutable}, or {\em constants} (\Sref{FinalVariables}), or simply
\xcd`val` variables. Variables whose value may change are {\em mutable} or
simply \xcd`var` variables. \xcd`var` variables are declared by the \xcd`var`
keyword. \xcd`val` variables may be declared by the \xcd`val` keyword; when a
variable declaration does not include either \xcd`var` or \xcd`val`, it is
considered \xcd`val`. 


%~~gen
%package Vars.For.Bears.In.Chairs;
%class VarExample{
%static def example() {
%~~vis
\begin{xten}
val a : Int = 0;               // Full 'val' syntax
b : Int = 0;                   // 'val' implied
val c = 0;                     // Type inferred
var d : Int = 0;               // Full 'var' syntax
var e : Int;                   // Not initialized
var f : Int{self != 100} = 0;  // Constrained type
\end{xten}
%~~siv
%}}
%~~neg







\section{Immutable variables}
\label{FinalVariables}
\index{variable!immutable}
\index{immutable variable}
\index{variable!val}
\index{val}

Immutable (\xcd`val`) variables can be given values (by initialization or assignment) at
most once, and must be given values before they are used.  Usually this is
achieved by declaring and initializing the variable in a single statement.
%~~gen
% package Vars.In.Snares;
% class ABitTedious{
% def example() {
%~~vis
\begin{xten}
val a : Int = 10;
val b = (a+1)*(a-1);
\end{xten}
%~~siv
%}}
%~~neg
\xcd`a` and \xcd`b` cannot be assigned to further.

In other cases, the declaration and assignment are separate.  One such
case is how constructors give values to \xcd`val` fields of objects.  The
\xcd`Example` class has an immutable field \xcd`n`, which is given different
values depending on which constructor was called. \xcd`n` can't be given its
value by initialization when it is declared, since it is not knowable which
constructor is called at that point.  
%~~gen
% package Vars.For.Cares;
%~~vis
\begin{xten}
class Example {
  val n : Int; // not initialized here
  def this() { n = 1; }
  def this(dummy:Boolean) { n = 2;}
}
\end{xten}
%~~siv
%
%~~neg

Another common case of separating declaration and assignment is in function
and method call.  The formal parameters are bound to the corresponding actual
parameters, but the binding does not happen until the function is called.  In
the code below, \xcd`x` is initialized to \xcd`3` in the first call and
\xcd`4` in the second.
%~~gen
%package Vars.For.Swears;
%class Examplement {
%static def whatever() {
%~~vis
\begin{xten}
val sq = (x:Int) => x*x;
x10.io.Console.OUT.println("3 squared = " + sq(3));
x10.io.Console.OUT.println("4 squared = " + sq(4));
\end{xten}
%~~siv
%}}
%~~neg





%%IMMUTABLE%% An immutable variable satisfies two conditions: 
%%IMMUTABLE%% \begin{itemize}
%%IMMUTABLE%% \item it can be assigned to at most once, 
%%IMMUTABLE%% \item it must be assigned to before use. 
%%IMMUTABLE%% \end{itemize}
%%IMMUTABLE%% 
%%IMMUTABLE%% \Xten{} follows \java{} language rules in this respect \cite[\S
%%IMMUTABLE%% 4.5.4,8.3.1.2,16]{jls2}. Briefly, the compiler must undertake a
%%IMMUTABLE%% specific analysis to statically guarantee the two properties above.
%%IMMUTABLE%% 
%%IMMUTABLE%% Immutable local variables and fields are defined by the \xcd"val"
%%IMMUTABLE%% keyword.  Elements of value arrays are also immutable.
%%IMMUTABLE%% 
%%IMMUTABLE%% \oldtodo{Check if this analysis needs to be revisited.}

\section{Initial values of variables}
\label{NullaryConstructor}\index{nullary constructor}
\index{initial value}
\index{initialization}


Every assignment, binding, or initialization to a variable of type \xcd`T{c}`
must be an instance of type \xcd`T` satisfying the constraint \xcd`{c}`.
Variables must be given a value before they are used. This may be done by
initialization, which is the only way for immutable (\xcd`val`) variables and
one option for mutable (\xcd`var`) ones: 

%~~gen
%package Vars.For.Bears;
%class VarsForBears{
%def check() {
%~~vis
\begin{xten}
  val immut : Int = 3;
  var mutab : Int = immut;
  val use = immut + mutab;
\end{xten}
%~~siv
%}}
%~~neg
Or, for mutable variables, it may be done by a later assignment.  

%~~gen
%package Vars.For.Stars;
%class VarsForStars{
%def check() {
%~~vis
\begin{xten}
  var muta2 : Int;
  muta2 = 4;
  val use = muta2 * 10;
\end{xten}
%~~siv
%}}
%~~neg


Every class variable must be initialized before it is read, through
the execution of an explicit initializer. Every
instance variable must be initialized before it is read, through the
execution of an explicit or implicit initializer or a constructor.
Implicit initializers initialize \xcd`var`s to the default values of their
types (\Sref{DefaultValues}). Variables of types which do not have default
values are not implicitly initialized.



Each method and constructor parameter is initialized to the
corresponding argument value provided by the invoker of the method. An
exception-handling parameter is initialized to the object thrown by
the exception. A local variable must be explicitly given a value by
initialization or assignment, in a way that the compiler can verify
using the rules for definite assignment \cite[\S~16]{jls2}.


\section{Destructuring syntax}
\index{variable declarator!destructuring}
\index{destructuring}
\Xten{} permits a \emph{destructuring} syntax for local variable
declarations with explicit initializers,  and for formal parameters, of type \xcd`Point`, \Sref{point-syntax}.
(Future versions of X10 may allow destructuring of other types as well.) 
A point is a sequence of {$r \ge 0$} \xcd`Int`-valued coordinates.  
It is often useful to get at the coordinates directly, in variables. 

The following code makes an anonymous point with one coordinate \xcd`11`, and
binds \xcd`i` to \xcd`11`.  Then it makes a point with coordinates \xcd`22`
and \xcd`33`, binds \xcd`p` to that point, and \xcd`j` and \xcd`k` to \xcd`22`
and \xcd`33` respectively.
%~~gen
% package Vars.For.Glares;
% class DestructuringEx1 {
% def whyJustForLocals() {
%~~vis
\begin{xten}
val [i] : Point = Point.make(11);
val p[j,k] = Point.make(22,33);
val q[l,m] = [44,55]; // coerces an array to a point.
\end{xten}
%~~siv
%}}
%~~neg

A useful idiom for iterating over a range of numbers is: 
%~~gen
%package Vars.For.Bears;
% class ForBear {
% def forbear() {
%~~vis
\begin{xten}
var sum : Int = 0;
for ([i] in 1..100) sum += i;
\end{xten}
%~~siv
% ; } } 
%~~neg
\noindent
The brackets in \xcd`[i]` introduce destructuring, making X10 treat \xcd`i`
as an \xcd`Int`; without them, it would be a \xcd`Point`.  

In general, a pattern of the form \xcdmath"[i$_1$,$\ldots$,i$_n$]" matches a
point with {$n$} coordinates, binding \xcdmath"i$_j$" to coordinate {$j$}.  
A pattern of the form \xcdmath"p[i$_1$,$\ldots$,i$_n$]" does the same,  but
also binds \xcd`p` to the point.

\section{Formal parameters}
\index{formal parameter}
\index{parameter}

\begin{bbgrammar}
 FormalParams    \: \xcd"(" FormalParamList\opt \xcd")" & (\ref{prod:FormalParams})\\%<FROM #(prod:FormalParams)#
 FormalParamList    \: FormalParam & (\ref{prod:FormalParamList})\\%<FROM #(prod:FormalParamList)#
    \| FormalParamList \xcd"," FormalParam\\
 FormalParam    \: Mods\opt FormalDeclarator & (\ref{prod:FormalParam})\\%<FROM #(prod:FormalParam)#
    \| Mods\opt VarKeyword FormalDeclarator\\
    \| Type\\
 FormalDeclarators    \: FormalDeclarator & (\ref{prod:FormalDeclarators})\\%<FROM #(prod:FormalDeclarators)#
    \| FormalDeclarators \xcd"," FormalDeclarator\\
 FormalDeclarator    \: Id ResultType & (\ref{prod:FormalDeclarator})\\%<FROM #(prod:FormalDeclarator)#
    \| \xcd"[" IdList \xcd"]" ResultType\\
    \| Id \xcd"[" IdList \xcd"]" ResultType\\
 ResultType    \: \xcd":" Type & (\ref{prod:ResultType})\\%<FROM #(prod:ResultType)#
\end{bbgrammar}

Formal parameters are the variables which hold values transmitted into a
method or function.  
They are always declared with a type.  (Type inference is not
available, because there is no single expression to deduce a type from.)
The variable name can be omitted if it is not to be used in the
scope of the declaration, as in the type of the method 
\xcd`static def main(Array[String]):void` executed at the start of a program that
does not use its command-line arguments.

\xcd`var` and \xcd`val` behave just as they do for local
variables, \Sref{local-variables}.  In particular, the following \xcd`inc`
method is allowed, but, unlike some languages, does {\em not} increment its
actual parameter.  \xcd`inc(j)` creates a new local 
variable \xcd`i` for the method call, initializes \xcd`i` with the value of
\xcd`j`, increments \xcd`i`, and then returns.  \xcd`j` is never changed.
%~~gen
% package Vars.For.Squares.Of.Mares;
% class Ink {
%~~vis
\begin{xten}
static def inc(var i:Int) { i += 1; }
\end{xten}
%~~siv
%}
%~~neg


\section{Local variables}\label{local-variables}
\index{variable!local}
\index{local variable}
Local variables are declared in a limited scope, and, dynamically, keep their
values only for so long as the scope is being executed.  They may be \xcd`var`
or \xcd`val`.  
They may have 
initializer expressions: \xcd`var i:Int = 1;` introduces 
a variable \xcd`i` and initializes it to 1.
If the variable is immutable
(\xcd"val")
the type may be omitted and
inferred from the initializer type (\Sref{TypeInference}).

The variable declaration \xcd`val x:T=e;` confirms that \xcd`e`'s value is of
type \xcd`T`, and then introduces the variable \xcd`x` with type \xcd`T`.  For
example, consider a class Tub with a property \xcd`p`.
%~~gen
% package Vars.Local;
%~~vis
\begin{xten}
class Tub(p:Int){
  def this(pp:Int):Tub{self.p==pp} {property(pp);}
  def example() {
    val t : Tub = new Tub(3);
  }
}
\end{xten}
%~~siv
%
%~~neg
\noindent
produces a variable \xcd`t` of type \xcd`Tub`, even though the expression
\xcd`new Tub(3)` produces a value of type \xcd`Tub{self.p==3}` -- that is, a
\xcd`Tub`  whose \xcd`p` field is 3.  This can be inconvenient when the
constraint information is required.

\index{\Xcd{<:}}
Including type information in variable declarations is generally good
programming practice: it explains to both the compiler and human readers
something of the intent of the variable.  However, including types in 
\xcd`val t:T=e` can obliterate helpful information.  So, X10 allows a {\em
documentation type declaration}, written \xcd`val t <: T = e`.  This 
has the same effect as \xcd`val t = e`, giving \xcd`t` the full type inferred
from \xcd`e`; but it also confirms statically that that type is at least
\xcd`T`.  For example, the following gives \xcd`t` the type \xcd`Tub{self.p==3}` as
desired: 
%~~gen
% package Vars.Local;
% class TubBounded{
% def example() {
%~~vis
\begin{xten}
   val t <: Tub = new Tub(3);
\end{xten}
%~~siv
%}}
%~~neg
\noindent
However, replacing \xcd`Tub` by \xcd`Int` would result in a compilation error. 

Variables do not need to be initialized at the time of definition -- not even
\xcd`val`s. They must be initialized by the time of use, and \xcd`val`s may
only be assigned to once. The X10 compiler performs static checks guaranteeing
this restriction. The following is correct, albeit obtuse: 
%~~gen
%package Vars.Local;
% class NotInitVal {
%~~vis
\begin{xten}
static def main(r: Array[String](1)):void {
  val a : Int;
  a = r.size;
  val b : String;
  if (a == 5) b = "five?"; else b = "" + a + " args"; 
  // ... 
\end{xten}
%~~siv
%} }
%~~neg



\section{Fields}
\index{field}
\index{object!field}
\index{struct!field}
\index{class!field}

\begin{bbgrammar}
 FieldDeclarators    \: FieldDeclarator & (\ref{prod:FieldDeclarators})\\%<FROM #(prod:FieldDeclarators)#
    \| FieldDeclarators \xcd"," FieldDeclarator\\
 FieldDecl    \: Mods\opt FieldKeyword FieldDeclarators \xcd";" & (\ref{prod:FieldDecl})\\%<FROM #(prod:FieldDecl)#
    \| Mods\opt FieldDeclarators \xcd";"\\
 FieldDeclarator    \: Id HasResultType & (\ref{prod:FieldDeclarator})\\%<FROM #(prod:FieldDeclarator)#
    \| Id HasResultType\opt \xcd"=" VariableInitializer\\
 HasResultType    \: \xcd":" Type & (\ref{prod:HasResultType})\\%<FROM #(prod:HasResultType)#
    \| \xcd"<:" Type\\
 FieldKeyword    \: \xcd"val" & (\ref{prod:FieldKeyword})\\%<FROM #(prod:FieldKeyword)#
    \| \xcd"var"\\
 Mod    \: \xcd"abstract" & (\ref{prod:Mod})\\%<FROM #(prod:Mod)#
    \| Annotation\\
    \| \xcd"atomic"\\
    \| \xcd"final"\\
    \| \xcd"native"\\
    \| \xcd"private"\\
    \| \xcd"protected"\\
    \| \xcd"public"\\
    \| \xcd"static"\\
    \| \xcd"transient"\\
    \| \xcd"clocked"\\

\end{bbgrammar}

Like most other kinds of variables in X10, 
the fields of an object can be either \xcd`val` or \xcd`var`. 
Fields can be \xcd`static`,\xcd`global`, or \xcd`property`; see
\Sref{FieldDefinitions} and \Sref{PropertiesInClasses}.
Field declarations may have optional
initializer expressions, as for local variables, \Sref{local-variables}.
\xcd`var` fields without an initializer are initialized with the default value
of their type. \xcd`val` fields without an initializer must be initialized by
each constructor.


For \xcd`val` fields, as for \xcd`val` local variables, the type may be
omitted and inferred from the initializer type (\Sref{TypeInference}).
\xcd`var` files, like \xcd`var` local variables, must be declared with a type.



%%GRAM%% \begin{grammar}
%%GRAM%% FieldDeclaration
%%GRAM%%         \: FieldModifier\star \xcd"var" FieldDeclaratorsWithType \\&& ( \xcd"," FieldDeclaratorsWithType )\star \\
%%GRAM%%         \| FieldModifier\star \xcd"val" FieldDeclarators \\&& ( \xcd"," FieldDeclarators )\star \\
%%GRAM%%         \| FieldModifier\star FieldDeclaratorsWithType \\&& ( \xcd"," FieldDeclaratorsWithType )\star \\
%%GRAM%% FieldDeclarators
%%GRAM%%         \: FieldDeclaratorsWithType \\
%%GRAM%%         \: FieldDeclaratorWithInit \\
%%GRAM%% FieldDeclaratorId
%%GRAM%%         \: Identifier  \\
%%GRAM%% FieldDeclaratorWithInit
%%GRAM%%         \: FieldDeclaratorId Init \\
%%GRAM%%         \| FieldDeclaratorId ResultType Init \\
%%GRAM%% FieldDeclaratorsWithType
%%GRAM%%         \: FieldDeclaratorId ( \xcd"," FieldDeclaratorId )\star ResultType \\
%%GRAM%% FieldModifier \: Annotation \\
%%GRAM%%                 \| \xcd"static" \\ \| \xcd`property` \\ \| \xcd`global` \\
%%GRAM%% \end{grammar}
%%GRAM%% 
%%GRAM%% 

%%ACC%%  \section{Accumulator Variables}
%%ACC%%  
%%ACC%%  Accumulator variables allow the accumulation of partial results to produce a
%%ACC%%  final result.  For example, an accumulator variable could compute a running
%%ACC%%  sum, product, maximum, or minimum of a collection of numbers.  In particular,
%%ACC%%  many concurrent activites can accumulate safely into the {\em same} local
%%ACC%%  variable, without need for \Xcd{atomic} blocks or other explicit coordination.  
%%ACC%%  
%%ACC%%  An accumulator variable is associated with a {\em reducer}, which explains how
%%ACC%%  new partial values are accumulated.
%%ACC%%  
%%ACC%%  \subsection{Reducers}
%%ACC%%  
%%ACC%%  A notion of accumulation has two aspects: 
%%ACC%%  \begin{enumerate}
%%ACC%%  \item A {\bf zero} value, which is the initial value of the accumulator,
%%ACC%%        before any partial results have been included.  When accumulating a sum,
%%ACC%%        the zero value is \Xcd{0}; when accumulating a product, it is \Xcd{1}.
%%ACC%%  \item A {\bf combining function}, explaining how to combine two partial
%%ACC%%        accumulations into a whole one.  When accumulating a sum, partial sums
%%ACC%%        should be added together; for a product, they should be multiplied.  
%%ACC%%  \end{enumerate}
%%ACC%%  
%%ACC%%  In X10, this is represented as a value of type
%%ACC%%  \Xcd{x10.lang.Reducer[T]}: 
%%ACC%%  %~acc~gen
%%ACC%%  %package Vars.Notx10lang.Reducerererer;
%%ACC%%  %~acc~vis
%%ACC%%  \begin{xten}
%%ACC%%  struct Reducer[T](zero:T, apply: (T,T)=>T){}
%%ACC%%  \end{xten}
%%ACC%%  %~acc~siv
%%ACC%%  %
%%ACC%%  %~acc~neg
%%ACC%%  \noindent 
%%ACC%%  If \Xcd{r:Reducer[T]}, then \Xcd{r.zero} is the zero element, and
%%ACC%%  \Xcd{r(a,b)} --- which can also be written \Xcd{r.apply(a,b)} --- is the
%%ACC%%  combination of \Xcd{a} and \Xcd{b}.
%%ACC%%  
%%ACC%%  For example, the reducers for adding and multiplying integers are: 
%%ACC%%  %~acc~gen
%%ACC%%  %package Vars.Notx10lang.Reducererererererer;
%%ACC%%  %struct Reducer[T](zero:T, apply: (T,T)=>T){}
%%ACC%%  %class Example{
%%ACC%%  %~acc~vis
%%ACC%%  \begin{xten}
%%ACC%%  val summer = Reducer[Int](0, Int.+);
%%ACC%%  val producter = Reducer[Int](1, Int.*);
%%ACC%%  \end{xten}
%%ACC%%  %~acc~siv
%%ACC%%  %}
%%ACC%%  %~acc~neg
%%ACC%%  
%%ACC%%  
%%ACC%%  Reduction is guaranteed to be deterministic if the reducer is {\em
%%ACC%%  Abelian},\footnote{This term is borrowed from abstract algebra, where such a
%%ACC%%  reducer, together with its type, forms an Abelian monoid.}
%%ACC%%  that is, 
%%ACC%%  \begin{enumerate}
%%ACC%%  \item \Xcd{r.apply} is pure; that is, has no side effects;
%%ACC%%  \item \Xcd{r.apply} is commutative; that is, \Xcd{r(a,b) == r(b,a)} for all
%%ACC%%        inputs \Xcd{a} and \Xcd{b};
%%ACC%%  \item \Xcd{r.apply} is associative; that is, 
%%ACC%%        \Xcd{r(a,r(b,c)) == r(r(a,b),c)} for all \Xcd{a}, \Xcd{b}, and \Xcd{c}.
%%ACC%%  \item \Xcd{r.zero} is the identity element for \Xcd{r.apply}; that is, 
%%ACC%%        \Xcd{r(a, r.zero) == a}
%%ACC%%        for all \Xcd{a}.
%%ACC%%  \end{enumerate}
%%ACC%%  
%%ACC%%  
%%ACC%%  
%%ACC%%  
%%ACC%%  \Xcd{summer} and \Xcd{producter} satisfy all these conditions, and give
%%ACC%%  determinate reductions. The compiler does not require or check these, though.
%%ACC%%  
%%ACC%%  
%%ACC%%  \subsection{Accumulators}
%%ACC%%  
%%ACC%%  If \Xcd{r} is a  value of type \Xcd{Reducer[T]}, then an accumulator of type
%%ACC%%  \Xcd{T} using \Xcd{r} is declared as:
%%ACC%%  %~accTODO~gen
%%ACC%%  % package Vars.Accumulators.Basic.Little.Idea;
%%ACC%%  % class C[T]{
%%ACC%%  % static def example (r:Reducer[T]) {
%%ACC%%  %~accTODO~vis
%%ACC%%  \begin{xten}
%%ACC%%  acc(r) x : T;
%%ACC%%  acc(r) y; 
%%ACC%%  \end{xten}
%%ACC%%  %~accTODO~siv
%%ACC%%  %
%%ACC%%  %~accTODO~neg
%%ACC%%  The type declaration \Xcd{T} is optional; if specified, it must be the same
%%ACC%%  type that the reducer \Xcd{r} uses.
%%ACC%%  
%%ACC%%  \subsection{Sequential Use of Accumulators}
%%ACC%%  
%%ACC%%  The sequential use of accumulator variables is straightforward, and could be
%%ACC%%  done as easily without accumulators.  (The power of accumulators is in their
%%ACC%%  concurrent use, \Sref{ConcurrentUseOfAccumulators}.)
%%ACC%%  
%%ACC%%  A variable declared as \Xcd{acc(r) x:T;} is initialized to \Xcd{r.zero}.  
%%ACC%%  
%%ACC%%  Assignment of values of \Xcd{acc} variables has nonstandard semantics.
%%ACC%%  \Xcd{x = v;} causes the value \Xcd{r(v,x)} to be stored in \Xcd{x} --- in
%%ACC%%  particular, {\em not} the value of \Xcd{v}.
%%ACC%%  
%%ACC%%  Reading a value from an accumulator retrieves the current accumulation.
%%ACC%%  
%%ACC%%  For example, the sum and product of a list \Xcd{L} of integers can be computed
%%ACC%%  by: 
%%ACC%%  %~accTODO~gen
%%ACC%%  %package Vars.Accumulators.Are.For.Bisimulators;
%%ACC%%  % import java.util.*;
%%ACC%%  % class Example{
%%ACC%%  % static def example(L: List[Int]) {
%%ACC%%  %~accTODO~vis
%%ACC%%  \begin{xten}
%%ACC%%  val summer = Reducer[Int](0, Int.+);
%%ACC%%  val producter = Reducer[Int](1, Int.*);
%%ACC%%  acc(summer) sum;
%%ACC%%  acc(producter) prod;
%%ACC%%  for (x in L) {
%%ACC%%    sum = x;
%%ACC%%    prod = x;
%%ACC%%  }
%%ACC%%  x10.io.Console.OUT.println("Sum = " + sum + "; Product = " + prod);
%%ACC%%  \end{xten}
%%ACC%%  %~accTODO~siv
%%ACC%%  %
%%ACC%%  %~accTODO~neg
%%ACC%%  
%%ACC%%  
%%ACC%%  
%%ACC%%  \subsection{Concurrent Use of Accumulators}
%%ACC%%  \label{ConcurrentUseOfAccumulators}
%%ACC%%  \index{accumulator!and activities}
%%ACC%%  
%%ACC%%  Accumulator variables are restricted and synchronized in ways that make them
%%ACC%%  ideally suited for concurrent accumulation of data.   The {\em governing
%%ACC%%  activity} of an accumulator is the activity in which the \Xcd{acc} variable is
%%ACC%%  declared.  
%%ACC%%  
%%ACC%%  \begin{enumerate}
%%ACC%%  \item The governing activity can read the accumulator at any point that it has
%%ACC%%        no running sub-activities.  
%%ACC%%  \item Any activity that has lexical access to the accumulator can write to it.  
%%ACC%%        All
%%ACC%%        writes are performed atomically, without need for \Xcd{atomic} or other
%%ACC%%        concurrency control.
%%ACC%%  \end{enumerate}
%%ACC%%  
%%ACC%%  If the reducer is Abelian, this guarantees that \Xcd{acc} variables cannot
%%ACC%%  cause race conditions; the result of such a computation is determinate,
%%ACC%%  independent of the scheduling of activities. Read-read conflicts are
%%ACC%%  impossible, as only a single activity, the governing activity, can read the
%%ACC%%  \Xcd{acc} variable. Read-write conflicts are impossible, as reads are only
%%ACC%%  allowed at points where the only activity which can refer to the \Xcd{acc}
%%ACC%%  variable is the governing activity. Two activities may try to write the
%%ACC%%  \Xcd{acc} variable at the same time. The writes are performed atomically, so
%%ACC%%  they behave as if they happened in some (arbitrary) order---and, because the
%%ACC%%  reducer is Abelian, the order of writes doesn't matter.
%%ACC%%  
%%ACC%%  If the reducer is not Abelian---\eg, it is accumulating a string result by
%%ACC%%  concatenating a lot of partial strings together---the result is indeterminate.
%%ACC%%  However, because the accumulator operations are atomic, it will be the result
%%ACC%%  of {\em some} combination of the individual elements by the reduction
%%ACC%%  operation, \eg, the concatenation of the partial strings in {\em some} order.  
%%ACC%%  
%%ACC%%  
%%ACC%%  
%%ACC%%  For example, the following code computes triangle numbers {$\sum_{i=1}^{n}i$}
%%ACC%%  concurrently.\footnote{This program is highly inefficient. Even ignoring the
%%ACC%%    constant-time formula {$\sum_{i=1}^{n}i = \frac{n(n+1)}{2}$}, this program
%%ACC%%    incurs the cost of starting {$n$} activities and coordinating {$n$} accesses
%%ACC%%    to the accumulator. Accumulator variables are of most value in multi-place,
%%ACC%%    multi-core computations.}
%%ACC%%  
%%ACC%%  
%%ACC%%  %~accTODO~gen
%%ACC%%  %package Vars.Accumulator.Concurrency.Example;
%%ACC%%  %class Example{
%%ACC%%  %
%%ACC%%  %~accTODO~vis
%%ACC%%  \begin{xten}
%%ACC%%  def triangle(n:Int) {
%%ACC%%    val summer = Reducer[Int](0, Int.+);
%%ACC%%    acc(summer) sum; 
%%ACC%%    finish {
%%ACC%%      for([i] in 1..n) async {
%%ACC%%        sum = i;  // (A)
%%ACC%%      }
%%ACC%%      // (C)
%%ACC%%    }
%%ACC%%    return sum; // (B)
%%ACC%%  }
%%ACC%%  \end{xten}
%%ACC%%  %~accTODO~siv
%%ACC%%  %}
%%ACC%%  %~accTODO~neg
%%ACC%%  
%%ACC%%  The governing activity of the \Xcd{acc} variable \Xcd{sum} is the activity
%%ACC%%  including the body of \Xcd{triangle}.  It starts up \Xcd{n} sub-activities,
%%ACC%%  each of which adds one value to \Xcd{sum} at point \Xcd{(A)}.  Note that these
%%ACC%%  activities cannot {\em read} the value of \Xcd{sum}---only the governing
%%ACC%%  activity can do that---but they can update it.  
%%ACC%%  
%%ACC%%  At point \Xcd{(B)}, \Xcd{triangle} returns the value in \Xcd{sum}. It is
%%ACC%%  clear, from the \Xcd{finish} statement, that all sub-activities started by the
%%ACC%%  governing process have finished at this point. X10 forbids reading of
%%ACC%%  \Xcd{sum}, even by the governing process, at point \Xcd{(C)}, since
%%ACC%%  sub-activities writing into it could still be active when the governing
%%ACC%%  activity reaches this point.  The \Xcd{return sum;} statement could not be
%%ACC%%  moved to \Xcd{(C)}, which is good, because the program would be wrong if it
%%ACC%%  were there.
%%ACC%%  
%%ACC%%  
%%ACC%%  
%%ACC%%  
%%ACC%%  \subsubsection{Accumulators and Places}
%%ACC%%  \index{accumulator!and places} Activity variables can be read and written from
%%ACC%%  any place, without need for \Xcd{GlobalRef}s. We may spread the previous
%%ACC%%  computation out among all the available processors by simply sticking in an
%%ACC%%  \Xcd{at(...)} statement at point \Xcd{(D)}, without need for other
%%ACC%%  restructuring of the program.
%%ACC%%  
%%ACC%%  %~accTODO~gen
%%ACC%%  %package Vars.Accumulator.Concurrency.Example.Multiplacey;
%%ACC%%  %class Example{
%%ACC%%  %~accTODO~vis
%%ACC%%  \begin{xten}
%%ACC%%  def triangle(n:Int) {
%%ACC%%    val summer = Reducer[Int](0, Int.+);
%%ACC%%    acc(summer) sum; 
%%ACC%%    finish {
%%ACC%%      for([i] in 1..n) async 
%%ACC%%        at(Places.place(i % Places.MAX_PLACES) { //(D)
%%ACC%%          sum = i;  // (A)
%%ACC%%      }
%%ACC%%    }
%%ACC%%    return sum; // (B)
%%ACC%%  }
%%ACC%%  \end{xten}
%%ACC%%  %~accTODO~siv
%%ACC%%  %}
%%ACC%%  %~accTODO~neg
%%ACC%%  
%%ACC%%  \subsubsection{Accumulator Parameters}
%%ACC%%  \index{accumulator variables!as parameters}
%%ACC%%  \index{parameters!accumulator}
%%ACC%%  
%%ACC%%  Accumulators can be passed to methods and closures, by giving the keyword 
%%ACC%%  \Xcd{acc} instead of \Xcd{var} or \Xcd{val}.  Reducers are not specified; each
%%ACC%%  accumulator comes with its own reducer.  However, the type \Xcd{T} of the
%%ACC%%  accumulator {\em is} required.
%%ACC%%  
%%ACC%%  For example, the following method takes a list of numbers, and accumulates
%%ACC%%  those that are divisible by 2 in \Xcd{evens}, and those that are divisible by
%%ACC%%  3 in \Xcd{triples}: 
%%ACC%%  %~accTODO~gen
%%ACC%%  %package Vars.accumulators.parameters.oscillators.convulsitors.proximators;
%%ACC%%  %import x10.util.*;
%%ACC%%  %class Whatever {
%%ACC%%  %~accTODO~vis
%%ACC%%  \begin{xten}
%%ACC%%  static def split23(L:List[Int], acc evens:Int, acc triples:Int) {
%%ACC%%    for(n in L) {
%%ACC%%       if (n % 2 == 0) evens = n;
%%ACC%%       if (n % 3 == 0) triples = n;
%%ACC%%    }
%%ACC%%  }
%%ACC%%  static val summer = Reducer[Int](0, Int.+);
%%ACC%%  static val producter = Reducer[Int](1, Int.*);
%%ACC%%  static def sumEvenPlusProdTriple(L:List[Int]) {
%%ACC%%    acc(summer) sumEven;
%%ACC%%    acc(producter) prodTriple;
%%ACC%%    split23(L, sumEven, prodTriple);
%%ACC%%    return sumEven + prodTriple;
%%ACC%%  }
%%ACC%%  \end{xten}
%%ACC%%  %~accTODO~siv
%%ACC%%  %}
%%ACC%%  %~accTODO~neg
%%ACC%%  
%%ACC%%  \subsection{Indexed Accumulators}
%%ACC%%  \index{accumulator!indexed}
%%ACC%%  \index{accumulator!array}
%%ACC%%  
%%ACC%%  
%%ACC%%  \noo{Define this!}
%%ACC%%  
%%ACC%%  %~accTODO~gen
%%ACC%%  % package Vars.Indexed.Accumulators;
%%ACC%%  %~accTODO~vis
%%ACC%%  \begin{xten}
%%ACC%%  class BoolAccum implements SelfAccumulator[Boolean, Int] {
%%ACC%%    var sumTrue = 0, sumFalse = 0;
%%ACC%%    def update(k:Boolean, v:Int) { 
%%ACC%%       if (k) sumTrue += k; else sumFalse += k;
%%ACC%%    }
%%ACC%%    def update(ks:Array[Boolean]{rail}, vs:Array[Int]{ks.size == vs.size}) {
%%ACC%%       for([i] in ks.region) update(ks(i), vs(i));  }
%%ACC%%    
%%ACC%%  }
%%ACC%%  \end{xten}
%%ACC%%  %~accTODO~siv
%%ACC%%  %
%%ACC%%  %~accTODO~neg

\chapter{Names and packages}
\label{packages} \index{name}\index{package}

\section{Names}

An X10 program consists largely of giving names to entities, and then
manipulating the entities by their names. The entities involved may be
compile-time constructs, like packages, types and classes, or run-time
constructs, like numbers and strings and objects.  

X10 names can be {\em simple names}, which look like identifiers: \xcd`vj`,
\xcd`x10`, \xcd`AndSoOn`. Or, they can be {\em qualified names}, which are
sequences of two or more identifiers separated by dots: \xcd`x10.lang.String`, 
\xcd`somePack.someType`, \xcd`a.b.c.d.e.f`.   Some entities have only simple
names; some have both simple and qualified names.

Every declaration that introduces a name has a {\em scope}: the region of the
program in which the named entity can be referred to by a simple name.  
In some cases, entities may be referred to by qualified names outside of their
scope.  \Eg, a \xcd`public` class \xcd`C` defined in package \xcd`p` can be
referred to by the simple name \xcd`C` inside of \xcd`p`, or by the qualified
name \xcd`p.C` from anywhere.  

Many sorts of entities have {\em members}.  Packages have classes, structs,
and interfaces as members.  Those, in turn, have fields, methods, types, and
so forth as members.  The member \xcd`x` of an entity named \xcd`E` (as a
simple or qualified name) has the name \xcd`E.x`; it may also have other
names.  

\subsection{Shadowing}
\index{shadowing}
\index{namespace}

One declaration $d$ may {\em shadow} another declaration $d'$ in part of the
scope of $d'$, if $d$ and $d'$ declare variables with the same simple name $n$.
When $d$ shadows $d'$, a use of $n$ might refer to $d$'s $n$ (unless some
$d''$ in turn shadows $d$), but will never refer to $d'$'s $n$.

X10 has four namespaces:
\begin{itemize}
\item {\bf Types:} for classes, interfaces, structs, and defined types.
\item {\bf Values:} for \xcd`val`- and \xcd`var`-bound variables; fields;
      and formal parameters of all sorts.
\item {\bf Methods:} for methods of classes, interfaces, and structs.
\item {\bf Packages:} for packages.
\end{itemize}

A declaration $d$ in one namespace, binding a name $n$ to an entity $e$,
shadows all other declarations of that name $n$ in scope at the point where
$d$ is declared. This shadowing is in effect for the entire scope of $d$.  
Declarations in different namespaces do not shadow each other.
Thus, a local variable declaration may shadow a field declaration, but not a
class declaration.

Declarations which only introduce qualified names --- in X10, this is only
package declarations --- cannot shadow anything.

The rules for shadowing of imported names are given in \Sref{sect:ImportDecl}.

\subsection{Hiding}
\index{hiding}
\label{sect:Hiding}

Shadowing is ubiquituous in X10. Another, and considerably rarer, way that one
definition of a given simpl ename can render another definition of the same
name unavailable is {\em hiding}. If a class \xcd`Super` defines a field named
\xcd`x`, and a subclass \xcd`Sub` of \xcd`Super` also defines a field named
\xcd`x`, then, for \xcd`Sub`s, references to the \xcd`x` field get \xcd`Sub`'s
\xcd`x` rather than \xcd`Super`'s. In this case, \xcd`Super`'s \xcd`x` is said
to be {\em hidden}.

Hiding is technically different from shadowing, because hiding applies in more
circumstances: a use of class \xcd`Sub`, such as \xcd`sub.x`, may involve
hiding of name \xcd`x`, though it could not involve shadowing of \xcd`x`
because \xcd`x` is need not be declared as a name at that point.

\subsection{Obscuring}
\index{obscuring}
\label{sect:Obscuring}

The third way in which a definition of a simple name may become unavailable is
{\em obscuring}. This well-named concept says that, if \xcd`n` can be
interpreted as two or more of: a variable, a type, and a package, then it will
be interpreted as a variable if that is possible, or a type if it cannot be
interpreted as a variable. In this case, the unavailable interpretations are
{\em obscured}. 

\begin{ex}
In the \xcd`example` method of the following code, both a struct and a local
variable are named \xcd`eg`.  Following the obscuring rules, The call
\xcd`eg.ow()` in the first \xcd`assert` uses the variable rather than the struct.  
As the second \xcd`assert` demonstrates, the struct can be accessed through
its fully-qualified name.   Note that none of this would have happened if the
coder had followed the convention that structs have capitalized names,
\xcd`Eg`, and variables have lower-case ones, \xcd`eg`. 

%~~gen ^^^ Packages5t5g
% NOTEST
%~~vis
\begin{xten}
package obscuring;
struct eg {
   static def ow()= 1;
   static struct Bite {
      def ow() = 2;
   }
   def example() {
       val eg = Bite();
       assert eg.ow() == 2;
       assert obscuring.eg.ow() == 1;
     }
}

\end{xten}
%~~siv
% class Hook{ def run() { (eg()).example(); return true; } }
%~~neg

\end{ex}

Due to obscuring, it may be impossible to refer to a type or a package via a
simple name in some circumstances.  Obscuring does not block qualified names.



\subsection{Ambiguity and Disambiguation}

Neither simple nor qualified names are necessarily unique.  There can be, in
general, many entities that have the same name.  This is perfectly ordinary,
and, when done well, considered good programming practice.   Various forms of
{\em disambiguation} are used to tell which entity is meant by a given name;
\eg, methods with the same name may be disambiguated by the types of their
arguments (\Sref{sect:MethodResolution}).

\begin{ex}
In the following example, there are three static methods with 
qualified name \xcd`DisambEx.Example.m`; they can be disambiguated by their
different arguments.   Inside the body of the third, the simple name \xcd`i`
refers to both the \xcd`Int` formal of \xcd`m`, and to the static method 
\xcd`DisambEx.Example.i`.  
%~~gen ^^^ Packages9e6r
%~~vis
\begin{xten}
package DisambEx; 
class Example {
  static def m() = 1;
  static def m(Boolean) = 2;
  static def i() = 3;
  static def m(i:Int) {
    if (i > 10) {
      return i() + 1;
    }
    return i;
  }
  static def example() {
    assert m() == 1;
    assert m(true) == 2;
    assert m(20) == 4;
  }
}
\end{xten}
%~~siv
% class Hook{ def run() { Example.example(); return true; } }
%~~neg
\end{ex}



\section{Access Control}
\index{public}\index{protected}\index{private}

X10 allows programmers {\em access control}, that is, the ability to determine
statically where identifiers of most sorts are visible.  In particular, X10
allows {\em information hiding}, wherein certain data can be accessed from
only limited parts of the program. 

There are four access control modes: 
\xcd"public" , \xcd"protected", \xcd"private"
and uninflected package-specific scopes, much like those of Java. 
Most things can be public or private; a few things (\eg, class members) can
also be protected or package-scoped.  

Accessibility of one X10 entity (package, container, member, etc.) from within
a package or container is defined as follows: 
\begin{itemize}
\item Packages are always accessible.
\item If a container \xcd`C` is public, and, if it is inside of another
      container \xcd`D`,
      container \xcd`D` is accessible, then \xcd`C` is accessible.  
\item A member \xcd`m` of a container \xcd`C` is accessible from within
      another  \xcd`E`
      if \xcd`C` is
      accessible, and: 
      \begin{itemize}
      \item \xcd`m` is declared \xcd`public`; or
      \item \xcd`C` is an interface; or
      \item \xcd`m` is declared \xcd`protected`, and either the access is from
            within the same package that \xcd`C` is defined in, or from within
            the body of a subclass of \xcd`C` (but see
            \Sref{sect:protected-details} for some fine points); or
      \item \xcd`m` is declared \xcd`private`, and the access is from within
            the top-level class which contains the definition of \xcd`C` ---
            which may be \xcd`C` itself, or, if \xcd`C` is a nested container, an
            outer class around \xcd`C`; or
      \item \xcd`m` has no explicit class declaration (hence using the
            implicit ``package''-level access control), and the access occurs
            from the same package that \xcd`C` is declared in.
      \end{itemize}
\end{itemize}

\subsection{Details of \xcd`protected`}
\label{sect:protected-details}

\xcd`protected` access has a few fine points. 
Within the body of a subclass \xcd`D` of the class \xcd`C` containing
the definition of a protected member \xcd`m`, 

\begin{itemize}

\item An access \xcd`e.fld` to a field, or \xcd`e.m(...)` to a method, is
      permitted precisely when the type of \xcd`e` is either \xcd`D` or a
      subtype of \xcd`D`.  
For example, the access to \xcd`that.f` in the following code is acceptable, but
the access to \xcd`xhax.f` is not.  
%~~gen ^^^ Packages9q4y
% package Packages9q4y;
%~~vis
\begin{xten}
class C {
  protected var f : Int = 0;
}
class X extends C {}
class D extends C {
  def usef(that:D, xhax:X) {
     this.f += that.f; 
     // ERROR: this.f += xhax.f;
}
\end{xten}
%~~siv
%
%~~neg

\limitation{Some X10 compilers improperly allow access to {\tt xhax} -- as,
indeed, some Java compilers do, despite Java having the analogous rule.
However, X10 allows all permitted accesses, so the workaround is trivial.}

\item An access through a qualified name \xcd`Q.N` is permitted precisely when
      the type of \xcd`Q` is \xcd`D` or a subtype of \xcd`D`. 

\end{itemize}

Qualified access to a protected constructor is subtle.  Let \xcd`C` be a class
with a \xcd`protected` constructor $c$, and let \xcd`S` be the innermost
class containing a use $u$ of $c$.  There are three cases: 

\begin{itemize}
\item Super superclass construction invocations, \xcd`super(...)` or
      \xcd`E.super(...)`, are permitted.
\item Anonymous class instance creations, \xcd`new C(...){...}`
      and \xcd`E.new C(...){...}`, are
      permitted.
\item No other accesses are permitted. 
\end{itemize}

\section{Packages}

A package is a named collection of top-level type declarations, \viz, class,
interface, and struct declarations. Package names are sequences of
identifiers, like \xcd`x10.lang` and \xcd`com.ibm.museum`. The multiple names
are simply a convenience, though there is a tenuous relationship between
packages \xcd`a`, \xcd`a.b`, and \xcd`a.c`.   Packages can be accessed by
name from anywhere: a package may contain private elements, but may not itself
be private. 

Packages and protection modifiers determine which top-level names can be used
where. Only the \xcd`public` members of package \xcd`pack.age` can be accessed
outside of \xcd`pack.age` itself.  
%~~gen~~Stimulus ^^^ Stimulus
% NOTEST 
%~~vis
\begin{xten}
package pack.age;
class Deal {
  public def make() {}
}
public class Stimulus {
  private def taxCut() = true;
  protected def benefits() = true;
  public def jobCreation() = true;
  /*package*/ def jumpstart() = true;
}
\end{xten}
%~~siv
% 
%
%~~neg

The class \xcd`Stimulus` can be referred to from anywhere outside of
\xcd`pack.age` by its full name of \xcd`pack.age.Stimulus`, or can be imported
and referred to simply as \xcd`Stimulus`.  The public \xcd`jobCreation()`
method of a \xcd`Stimulus` can be referred to from anywhere as well; the other
methods have smaller visibility.  The non-\xcd`public` class \xcd`Deal` cannot
be used from outside of \xcd`pack.age`.  



\subsection{Name Collisions}

It is a static error for a package to have two members with the same name. For
example, package \xcd`pack.age` cannot define two classes both named
\xcd`Crash`, nor a class and an interface with that name.

Furthermore, \xcd`pack.age` cannot define a member \xcd`Crash` if there is
another package named \xcd`pack.age.Crash`, nor vice-versa. (This prohibition
is the only actual relationship between the two packages.)  This prevents the
ambiguity of whether \xcd`pack.age.Crash` refers to the class or the package.  
Note that the naming convention that package names are lower-case and package
members are capitalized prevents such collisions.


\section{{\tt import} Declarations}
\label{sect:ImportDecl}
\index{import}

Any public member of a package can be referred to from anywhere through a
fully-qualified name: \xcd`pack.age.Stimulus`.    

Often, this is too awkward.  X10 has two ways to allow code outside of a class
to refer to the class by its short name (\xcd`Stimulus`): single-type imports
and on-demand imports.   

Imports of either kind appear at the start of the file, immediately after the
\xcd`package` directive if there is one; their scope is the whole file.

\subsection{Single-Type Import}

The declaration \xcd`import ` {\em TypeName} \xcd`;` imports a single type
into the current namespace.  The type it imports must be a fully-qualified
name of an extant type, and it must either be in the same package (in which
case the \xcd`import` is redundant) or be declared \xcd`public`.  

Furthermore, when importing \xcd`pack.age.T`, there must not be another type
named \xcd`T` at that point: neither a  \xcd`T` declared in \xcd`pack.age`,
nor a \xcd`inst.ant.T` imported from some other package.

The declaration \xcd`import E.n;`, appearing in file $f$ of a package named
\xcd`P`, shadows the following types named \xcd`n` when they appear in $f$: 
\begin{itemize}
\item Top-level types named \xcd`n` appearing in other files of \xcd`P`, and 
\item Types named \xcd`n` imported by automatic imports
      (\Sref{sect:AutomaticImport}) in $f$.
\end{itemize}
\noindent


\subsection{Automatic Import}
\label{sect:AutomaticImport}

The automatic import \xcd`import pack.age.*;`, loosely, imports all the public
members of \xcd`pack.age`.  In fact, it does so somewhat carefully, avoiding
certain errors that could occur if it were done naively.  Types defined in the
current package, and those imported by single-type imports, shadow those
imported by automatic imports.   Names automatically imported never shadow any
other names.



\subsection{Implicit Imports}

The packages \xcd`x10.lang` and \xcd`x10.array` are automatically imported in all files
without need for further specification.

%%BARD-HERE



\section{Conventions on Type Names}

%##(TypeName PackageName
\begin{bbgrammar}
%(FROM #(prod:TypeName)#)
            TypeName \: Id & (\ref{prod:TypeName}) \\
                    \| TypeName \xcd"." Id \\
%(FROM #(prod:PackageName)#)
         PackageName \: Id & (\ref{prod:PackageName}) \\
                    \| PackageName \xcd"." Id \\
\end{bbgrammar}
%##)


While not enforced by the compiler, classes and interfaces
in the \Xten{} library follow the following naming conventions.
Names of types---including classes,
type parameters, and types specified by type definitions---are in
CamelCase and begin with an uppercase letter.  (Type variables are often
single capital letters, such as \xcd`T`.)
For backward
compatibility with languages such as C and \java{}, type
definitions are provided to allow primitive types
such as \xcd"int" and \xcd"boolean" to be written in lowercase.
Names of methods, fields, value properties, and packages are in camelCase and
begin with a lowercase letter. 
Names of \xcd"static val" fields are in all uppercase with words
separated by \xcd"_"'s.

\chapter{Interfaces}
\label{XtenInterfaces}\index{interface}

An interface specifies signatures for zero or more public methods, property
methods,
\xcd`static val`s, 
classes, structs, interfaces, types
and an invariant. 

The following puny example illustrates all these features: 
% TODO Well, it would if there weren't a compiler bug in the way.
%~~gen ^^^Interfaces_static_val
% package Interfaces_static_val;
% 
%~~vis
\begin{xten}
interface Pushable{prio() != 0} {
  def push(): void;
  static val MAX_PRIO = 100;
  abstract class Pushedness{}
  struct Pushy{}
  interface Pushing{}
  static type Shove = Int;
  property text():String;
  property prio():Int;
}
class MessageButton(text:String)
  implements Pushable{self.prio()==Pushable.MAX_PRIO} {
  public def push() { 
    x10.io.Console.OUT.println(text + " pushed");
  }
  public property text() = text;
  public property prio() = Pushable.MAX_PRIO;
}
\end{xten}
%~~siv
%
%~~neg
\noindent
\xcd`Pushable` defines two property methods, one normal method, and a static
value.  It also 
establishes an invariant, that \xcd`prio() != 0`. 
\xcd`MessageButton` implements a constrained version of \xcd`Pushable`,
\viz\ one with maximum priority.  It
defines the \xcd`push()` method given in the interface, as a \xcd`public`
method---interface methods are implicitly \xcd`public`.

\limitation{X10 may not always detect that type invariants of interfaces are
satisfied, even when they obviously are.}

A container---a class or struct---can {\em implement} an interface,
typically by having all the methods and property methods that the interface
requires, and by providing a suitable \xcd`implements` clause in its definition.

A variable may be declared to be of interface type.  Such a variable has all
the property and normal methods declared (directly or indirectly) by the
interface; 
nothing else is statically available.  Values of any concrete type which
implement the interface may be stored in the variable.  

\begin{ex}
The following code puts two quite different objects into the variable
\xcd`star`, both of which satisfy the interface \xcd`Star`.
%~~gen ^^^ Interfaces6l3f
% package Interfaces6l3f;
%~~vis
\begin{xten}
interface Star { def rise():void; }
class AlphaCentauri implements Star {
   public def rise() {}
}
class ElvisPresley implements Star {
   public def rise() {}
}
class Example {
   static def example() {
      var star : Star;
      star = new AlphaCentauri();
      star.rise();
      star = new ElvisPresley();
      star.rise();
   }
}
\end{xten}
%~~siv
%
%~~neg
\end{ex}
An interface may extend several interfaces, giving
X10 a large fraction of the power of multiple inheritance at a tiny fraction
of the cost.

\begin{ex}
%~~gen ^^^ Interfaces6g4u
% package Interfaces6g4u;
%~~vis
\begin{xten}
interface Star{}
interface Dog{}
class Sirius implements Dog, Star{}
class Lassie implements Dog, Star{}
\end{xten}
%~~siv
%
%~~neg
\end{ex}


\section{Interface Syntax}

\label{DepType:Interface}

%##(NormalInterfaceDecl TypeParamsI TypeParamI Guard ExtendsInterfaces InterfaceBody InterfaceMemberDecl
\begin{bbgrammar}
%(FROM #(prod:NormalInterfaceDecl)#)
 NormalInterfaceDecl \: Mods\opt \xcd"interface" Id TypeParamsI\opt Properties\opt Guard\opt ExtendsInterfaces\opt InterfaceBody & (\ref{prod:NormalInterfaceDecl}) \\
%(FROM #(prod:TypeParamsI)#)
         TypeParamsI \: \xcd"[" TypeParamIList \xcd"]" & (\ref{prod:TypeParamsI}) \\
%(FROM #(prod:TypeParamI)#)
          TypeParamI \: Id & (\ref{prod:TypeParamI}) \\
                     \| \xcd"+" Id \\
                     \| \xcd"-" Id \\
%(FROM #(prod:Guard)#)
               Guard \: DepParams & (\ref{prod:Guard}) \\
%(FROM #(prod:ExtendsInterfaces)#)
   ExtendsInterfaces \: \xcd"extends" Type & (\ref{prod:ExtendsInterfaces}) \\
                     \| ExtendsInterfaces \xcd"," Type \\
%(FROM #(prod:InterfaceBody)#)
       InterfaceBody \: \xcd"{" InterfaceMemberDecls\opt \xcd"}" & (\ref{prod:InterfaceBody}) \\
%(FROM #(prod:InterfaceMemberDecl)#)
 InterfaceMemberDecl \: MethodDecl & (\ref{prod:InterfaceMemberDecl}) \\
                     \| PropertyMethodDecl \\
                     \| FieldDecl \\
                     \| ClassDecl \\
                     \| InterfaceDecl \\
                     \| TypeDefDecl \\
                     \| \xcd";" \\
\end{bbgrammar}
%##)


\noindent
The invariant associated with an interface is the conjunction of the
invariants associated with its superinterfaces and the invariant
defined at the interface. 



A class \xcd"C"  implements an interface \xcd"I" if
\begin{itemize}
\item \xcd`I`, or a subtype of \xcd`I`, appears in the \xcd`implements` list
      of \xcd`C`, 
\item \xcd`C`'s property methods include all the property methods  of \xcd"I",
\item Each method \xcd`m` defined by \xcd`I` is also a method of \xcd`C` --
      with the {\em  \xcd`public`} modifier added.   These methods may be
      \xcd`abstract` if \xcd`C` is \xcd`abstract`.
\end{itemize}


If \xcd`C` implements \xcd`I`, then the class invariant
(\Sref{DepType:ClassGuardDef}) for \xcd`C`,   $\mathit{inv}($\xcd"C"$)$, implies
the class invariant for \xcd`I`, $\mathit{inv}($\xcd"I"$)$.  That is, if the
interface \xcd`I` specifies some requirement, then every class \xcd`C` that
implements it satisfies that requirement.

\section{Access to Members}

All interface members are \xcd`public`, whether or not they are declared
public.  There is little purpose to non-public methods of an interface; they
would specify that implementing classes and structs have methods that cannot
be seen.

\section{Property Methods}

An interface may declare \xcd`property` methods.  All non-\xcd`abstract`
containers implementing such an interface must provide all the \xcd`property`
methods specified.  

\section{Field Definitions}
\index{interface!field definition in}

An interface may declare a \xcd`val` field, with a value.  This field is implicitly
\xcd`public static val`.  In particular, it is {\em not} an instance field. 
%~~gen ^^^ Interfaces10
% package Interface.Field;
%~~vis
\begin{xten}
interface KnowsPi {
  PI = 3.14159265358;
}
\end{xten}
%~~siv
%
%~~neg

Classes and structs implementing such an interface get the interface's fields as
\xcd`public static` fields.  Unlike  methods, there is no need
for the implementing class to declare them. 
%~~gen ^^^ Interfaces20
% package Interface.Field.Two;
% interface KnowsPi {PI = 3.14159265358;}
%~~vis
\begin{xten}
class Circle implements KnowsPi {
  static def area(r:Double) = PI * r * r;
}
class UsesPi {
  def circumf(r:Double) = 2 * r * KnowsPi.PI;
}
\end{xten}
%~~siv
%
%~~neg

\subsection{Fine Points of Fields}

If two parent interfaces give different static fields of the same name, 
those fields must be referred to by qualified names.
%~~gen ^^^ Interface_field_name_collision
% 
%~~vis
\begin{xten}
interface E1 {static val a = 1;}
interface E2 {static val a = 2;}
interface E3 extends E1, E2{}
class Example implements E3 {
  def example() = E1.a + E2.a;
}
\end{xten}
%~~siv
%
%~~neg

If the {\em same} field \xcd`a` is inherited through many paths, there is no need to
disambiguate it:
%~~gen ^^^ Interfaces_multi
% package Interfaces.Mult.Inher.Field;
%~~vis
\begin{xten}
interface I1 { static val a = 1;} 
interface I2 extends I1 {}
interface I3 extends I1 {}
interface I4 extends I2,I3 {}
class Example implements I4 {
  def example() = a;
}
\end{xten}
%~~siv
%
%~~neg

The initializer of a field in an interface may be any expression.  It is
evaluated under the same rules as a \xcd`static` field of a class. 

\begin{ex}
In this example, a local class (\Sref{sect:LocalClasses}) \xcd`B` is defined,
with an inner interface \xcd`I`.  The field \xcd`V` of \xcd`I` uses a variable
\xcd`n` which is global to \xcd`B`.   In this case it is a truly baroque way
to bind a \xcd`val`, but other uses are nontrivial.

%~~gen ^^^ Interfaces3l4a
% package Interfaces3l4a;
%~~vis
\begin{xten}
class TheOne {
  static val ONE = 1;
  interface WelshOrFrench {
    val UN = 1;
  }
  static class Onesome implements WelshOrFrench {
    static def example() {
      assert UN == ONE;
    }
  }
}
\end{xten}
%~~siv
% class Hook{ def run() {TheOne.Onesome.example(); return true;}}
%~~neg
\end{ex}

\section{Generic Interfaces}

Interfaces, like classes and structs, can have type parameters.  
The discussion of generics in \Sref{TypeParameters} applies to interfaces,
without modification.

\begin{ex}
%~~gen ^^^ Interfaces7n1z
% package Interfaces7n1z;
%~~vis
\begin{xten}
interface ListOfFuns[T,U] extends x10.util.List[(T)=>U] {}
\end{xten}
%~~siv
%
%~~neg

\end{ex}

\section{Interface Inheritance}

The {\em direct superinterfaces} of a non-generic interface \xcd`I` are the interfaces
(if any) mentioned in the \xcd`extends` clause of \xcd`I`'s definition.
If \xcd`I`  is generic, the direct superinterfaces are of an instantiation of
\xcd`I` are the corresponding instantiations of those interfaces.
A {\em superinterface} of \xcd`I` is either \xcd`I` itself, or a direct
superinterface of a superinterface of \xcd`I`, and similarly for generic
interfaces.    

\xcd`I` inherits the members of all of its superinterfaces. Any class or
struct that has \xcd`I` in its \xcd`implements` clause also implements all of
\xcd`I`'s superinterfaces. 






\section{Members of an Interface}

The members of an interface \xcd`I` are the union of the following sets: 
\begin{enumerate}
\item All of the members appearing in \xcd`I`'s declaration;
\item All the members of its direct super-interfaces, except those which are
      hidden (\Sref{sect:Hiding}) by \xcd`I`
\item The members of \xcd`Any`.
\end{enumerate}

Overriding for instances is defined as for classes, \Sref{MethodOverload}

\chapter{Classes}
\label{XtenClasses}\index{class}
\label{ReferenceClasses}

\section{Principles of X10 Objects}\label{XtenObjects}\index{object}
\index{class}

\subsection{Basic Design}

Objects are instances of classes: the most common and most powerful sort of
value in X10.  The other kinds of values, structs and functions, are more
specialized, better in some circumstances but not in all.
\xcd"x10.lang.Object" is the most general class; all other classes inherit
from it, directly or indirectly. 


Classes are structured in a single-inheritance code
hierarchy.   They may have any or all of these features: 
\begin{itemize}
\item Implementing any number of interfaces;
\item Static and instance \xcd`val` fields; 
\item Instance \xcd`var` fields; 
\item Static and instance methods;
\item Constructors;
\item Properties;
\item Static and instance nested containers.
\end{itemize}


\Xten{} objects do not have locks associated with them.
Programmers should use atomic blocks (\Sref{AtomicBlocks}) for mutual
exclusion and clocks (\Sref{XtenClocks}) for sequencing multiple parallel
operations.

An object exists in a single location: the place that it was created.  One
place cannot use or even directly refer to an object in a different place.   A
special type, \Xcd{GlobalRef[T]}, allows explicit cross-place references. 

The basic operations on objects are:
\begin{itemize}

{}\item Field access (\Sref{FieldAccess}). 
The static and instance fields of an object can be retrieved; \xcd`var` fields
can be set.  

{}\item Method invocation (\Sref{MethodInvocation}).  
Static and instance methods of an object can be invoked.

{}\item Casting (\Sref{ClassCast}) and instance testing with \xcd`instanceof`
(\Sref{instanceOf}) Objects can be cast or type-tested.  

\item The equality operators \xcd"==" and \xcd"!="
Objects can be compared for equality with the \Xcd{==} operation.  This checks
object {\em identity}: two objects are \Xcd{==} iff they are the same object.

\end{itemize}

  
 
\subsection{Class Declaration Syntax}

The {\em class declaration} has a list of type parameters, a list of
properties, a constraint (the {\em class invariant}), a single superclass,
zero or more interfaces that it implements, and a class body containing the
the definition of fields, properties, methods, and member types. Each such
declaration introduces a class type (\Sref{ReferenceTypes}).

%##(NormalClassDecl TypeParamsWithVariance TypeParamWithVarianceList Properties PropertyList Property WhereClause Super Interfaces InterfaceTypeList ClassBody ClassBodyDecls ClassMemberDecl
\begin{bbgrammar}
%(FROM #(prod:NormalClassDecl)#)
     NormalClassDecl \: Mods\opt \xcd"class" Id TypeParamsWithVariance\opt Properties\opt WhereClause\opt Super\opt Interfaces\opt ClassBody & (\ref{prod:NormalClassDecl}) \\
%(FROM #(prod:TypeParamsWithVariance)#)
TypeParamsWithVariance \: \xcd"[" TypeParamWithVarianceList \xcd"]" & (\ref{prod:TypeParamsWithVariance}) \\
%(FROM #(prod:TypeParamWithVarianceList)#)
TypeParamWithVarianceList \: TypeParamWithVariance & (\ref{prod:TypeParamWithVarianceList}) \\
                    \| TypeParamWithVarianceList \xcd"," TypeParamWithVariance \\
%(FROM #(prod:Properties)#)
          Properties \: \xcd"(" PropertyList \xcd")" & (\ref{prod:Properties}) \\
%(FROM #(prod:PropertyList)#)
        PropertyList \: Property & (\ref{prod:PropertyList}) \\
                    \| PropertyList \xcd"," Property \\
%(FROM #(prod:Property)#)
            Property \: Annotations\opt Id ResultType & (\ref{prod:Property}) \\
%(FROM #(prod:WhereClause)#)
         WhereClause \: DepParams & (\ref{prod:WhereClause}) \\
%(FROM #(prod:Super)#)
               Super \: \xcd"extends" ClassType & (\ref{prod:Super}) \\
%(FROM #(prod:Interfaces)#)
          Interfaces \: \xcd"implements" InterfaceTypeList & (\ref{prod:Interfaces}) \\
%(FROM #(prod:InterfaceTypeList)#)
   InterfaceTypeList \: Type & (\ref{prod:InterfaceTypeList}) \\
                    \| InterfaceTypeList \xcd"," Type \\
%(FROM #(prod:ClassBody)#)
           ClassBody \: \xcd"{" ClassBodyDecls\opt \xcd"}" & (\ref{prod:ClassBody}) \\
%(FROM #(prod:ClassBodyDecls)#)
      ClassBodyDecls \: ClassBodyDecl & (\ref{prod:ClassBodyDecls}) \\
                    \| ClassBodyDecls ClassBodyDecl \\
%(FROM #(prod:ClassMemberDecl)#)
     ClassMemberDecl \: FieldDecl & (\ref{prod:ClassMemberDecl}) \\
                    \| MethodDecl \\
                    \| PropertyMethodDecl \\
                    \| TypeDefDecl \\
                    \| ClassDecl \\
                    \| InterfaceDecl \\
                    \| \xcd";" \\
\end{bbgrammar}
%##)




\section{Fields}
\label{FieldDefinitions}
\index{object!field}
\index{field}

Objects may have {\em instance fields}, or simply {\em fields} (called
``instance variables'' in C++ and Smalltalk, and ``slots'' in CLOS): places to
store data that is pertinent to the object.  Fields, like variables, may be
mutable (\xcd`var`) or immutable (\xcd`val`).  

Class may have {\em static fields}, which store data pertinent to the
entire class of objects.  
See \Sref{StaticInitialization} for more information.

No two fields of the same class may have the same name.  A field may have the
same name as a method, although for fields of functional type there is a
subtlety (\Sref{sect:disambiguations}).  

\subsection{Field Initialization}
\index{field!initialization}
\index{initialization!of field}

Fields may be given values via {\em field initialization expressions}:
\xcd`val f1 = E;` and \xcd`var f2 : Int = F;`. Other fields of \xcd`this` may
be referenced, but only those that {\em precede} the field being initialized.
For example, the following is correct, but would not be if the fields were
reversed:

%~~gen ^^^ Classes10
%package Classes_field_init_expr_a;
%~~vis
\begin{xten}
class Fld{
  val a = 1;
  val b = 2+a;
}
\end{xten}
%~~siv
%
%~~neg


\subsection{Field hiding}
\index{hiding}
\index{field|hiding}


A subclass that defines a field \xcd"f" hides any field \xcd"f"
declared in a superclass, regardless of their types.  The
superclass field \xcd"f" may be accessed within the body of
the subclass via the reference \xcd"super.f".

With inner classes, it is occasionally necessary to 
write \xcd`Cls.super.f` to get at a hidden field \xcd`f` of an outer class
\xcd`Cls`. 

\begin{ex}
The \xcd`f` field in \xcd`Sub` hides the \xcd`f` field in \xcd`Super`
The \xcd`superf`` method provides access to the \xcd`f` field in \xcd`Super`.
%~~gen ^^^ Classes20
% package classes.fields.primus;
%~~vis
\begin{xten}
class Super{ 
  public val f = 1; 
}
class Sub extends Super {
  val f = true;
  def superf() : Int = super.f; // 1
}
\end{xten}
%~~siv
% class Hook { def run() { return (new Sub()).superf() == 1; }} 
%~~neg
\end{ex}

%~~gen ^^^ Classes30
% package classes.fields.secundus; 
% NOTEST
%~~vis
\begin{xten}
class A {
   val f = 3;
}
class B extends A {
   val f = 4;
   class C extends B {
      // C is both a subclass and inner class of B
      val f = 5;
       def example() {
         assert f         == 5 : "field of C";
         assert super.f   == 4 : "field of superclass";
         assert B.this.f  == 4 : "field of outer instance";
         assert B.super.f == 3 : "super.f of outer instance";
       }
    }
}
\end{xten}
%~~siv
% class Hook { def run() { ((new B()).new C()).example(); return true; } }
%~~neg


\subsection{Field qualifiers}
\label{FieldQualifier}
\index{qualifier!field}
\index{field!qualifier}

The behavior of a field may be changed by a field qualifier, such as
\xcd`static` or \xcd`transient`.  


\subsubsection{\Xcd{static} qualifier}
\index{field!static}

A \xcd`val` field may be declared to be {\em static}, as described in
\Sref{FieldDefinitions}. 

\subsubsection{\Xcd{transient} Qualifier}
\label{TransientFields}
\index{transient}
\index{field!transient}

A field may be declared to be {\em transient}.  Transient fields are excluded
from the deep copying that happens when information is sent from place to
place in an \Xcd{at} statement.    The value of a transient field of a copied
object is the default value of its type, regardless of the value of the field
in the original.  If the type of a field has no
default value, it cannot be marked \Xcd{transient}.
%~~gen ^^^ Classes40
% package Classes.Transient.Example;
% KNOWNFAIL-at
%~~vis
\begin{xten}
class Trans { 
   val copied = "copied";
   transient var transy : String = "a very long string";
   def example() {
      at (here; this) { // causes copying of 'this'
         assert(this.copied.equals("copied"));
         assert(this.transy == null);
      }
   }
}
\end{xten}
%~~siv
%
%~~neg


\section{Properties}
\label{PropertiesInClasses}
\index{property}

The properties of an object (or struct) are  public \xcd`val` fields
usable at compile time in constraints.\footnote{In many cases, a 
\xcd`val` field can be upgraded to a \xcd`property`, which 
entails no compile-time or runtime cost.  Some cannot be, \eg, in cases where
cyclic structures of \xcd`val` fields are required.} 
For example,  every array has a \xcd`rank` telling
how many subscripts it takes.  User-defined classes can have whatever
properties are desired. 

Properties are defined in parentheses, after the name of the class.  They are
given values by the \xcd`property` command in constructors.

%~~gen ^^^ Classes50
% package Classes.Toss.Freedom.Disk2;
%~~vis
\begin{xten}
class Proper(t:Int) {
  def this(t:Int) {property(t);}
}
\end{xten}
%~~siv
%
%~~neg




It is a static error for a class
defining a property \xcd"x: T" to have a subclass class that defines
a property or a field with the name \xcd"x".


A property \xcd`x:T` induces a field with the same name and type, 
as if defined with: 
%~~gen ^^^ Classes60
% package Classes.For.Masses.Of.NevermindTheRest;
% class Exampll[T] {
%~~vis
\begin{xten}
public val x : T;
\end{xten} 
%~~siv
% def this(y:T) { x=y; }
% }
%~~neg
\noindent It also defines a nullary getter method, 
%~~gen ^^^ Classes70
% package Classes_nullary_getter_a;
% class Exampllll[T] {
% public val x : T;
% def this(y:T) { x=y; }
%~~vis
\begin{xten}
public final def x()=x;
\end{xten}
%~~siv
%}
%~~neg





\index{property!initialization}
Properties are initialized by the invocation of a special \Xcd{property}
statement: 
\begin{xten}
property(e1,..., en);
\end{xten}
The number and types of arguments to the \Xcd{property} statement must match
the number and types of the properties in the class declaration.  
Every constructor of a class with properties must invoke \xcd`property(...)`
precisely once; it is a static error if X10 cannot prove that this holds.
The requirement to use the \xcd`property` statement means that all properties
must be given values at the same time.  

By construction, the graph whose nodes are values and whose edges are
properties is acyclic.  \Eg, there cannot be values \xcd`a` and \xcd`b` with
properties \xcd`c` and \xcd`d` such that \xcd`a.c == b` and \xcd`b.d == a`.


\subsection{Properties and Fields}

A container with a property named \xcd`p`, or a nullary property method named
\xcd`p()`, cannot have a field named \xcd`p` --- either defined in that
container, or inherited from a superclass.

\subsection{Acyclicity of Properties}
\index{properties!acyclic}

X10 has certain restrictions that, ultimately, require that properties are
simpler than their containers.  For example, \xcd`class A(a:A){}` is not
allowed.  
Formally, this requirement is that there is  a total order $\preceq$ 
on all classes and
structs such that, if $A$ extends $B$, then $A \prec B$, and
if $A$ has a property of type $B$, then $A \prec B$, where $A \prec B$ means
$A \preceq B$ and $A \ne B$.   
For example, the preceding class \xcd`A` is ruled out because we would need
\xcd`A`$\prec$\xcd`A`, which violates the definition of $\prec$.
The programmer need not (and cannot) specify
$\preceq$, and rarely need worry about its existence.  

Similarly, 
the type of a property may not simply be a type parameter.  
For example, \xcd`class A[X](x:X){}` is illegal.





\section{Methods}
\index{method}
\index{signature}
\index{method!signature}
\index{method!instance}
\index{method!static}

As is common in object-oriented languages, objects can have {\em methods}, of
two sorts.  {\em Static methods} are functions, conceptually associated with a
class and defined in its namespace.  {\em Instance methods} are parameterized
code bodies associated with an instance of the class, which execute with
convenient access to that instance's fields. 

Each method has a {\em signature}, telling what arguments it accepts, what
type it returns, and what precondition it requires. Method definitions may be
overridden by subclasses; the overriding definition may have a declared return
type that is a subtype of the return type of the definition being overridden.
Multiple methods with the same name but different signatures may be provided
\index{overloading}
\index{polymorphism}
on a class (called ``overloading'' or ``ad hoc polymorphism''). Methods may be
declared \Xcd{public}, \Xcd{private}, \Xcd{protected}, or given default package-level access
rights.

%##(MethMods MethodDecl TypeParams FormalParams FormalParamList HasResultType MethodBody
\begin{bbgrammar}
%(FROM #(prod:MethMods)#)
            MethMods \: Mods\opt & (\ref{prod:MethMods}) \\
                    \| MethMods \xcd"property"  \\
                    \| MethMods Mod \\
%(FROM #(prod:MethodDecl)#)
          MethodDecl \: MethMods \xcd"def" Id TypeParams\opt FormalParams WhereClause\opt HasResultType\opt Offers\opt MethodBody & (\ref{prod:MethodDecl}) \\
                    \| MethMods \xcd"operator" TypeParams\opt \xcd"(" FormalParam  \xcd")" BinOp \xcd"(" FormalParam  \xcd")" WhereClause\opt HasResultType\opt Offers\opt MethodBody \\
                    \| MethMods \xcd"operator" TypeParams\opt PrefixOp \xcd"(" FormalParam  \xcd")" WhereClause\opt HasResultType\opt Offers\opt MethodBody \\
                    \| MethMods \xcd"operator" TypeParams\opt \xcd"this" BinOp \xcd"(" FormalParam  \xcd")" WhereClause\opt HasResultType\opt Offers\opt MethodBody \\
                    \| MethMods \xcd"operator" TypeParams\opt \xcd"(" FormalParam  \xcd")" BinOp \xcd"this" WhereClause\opt HasResultType\opt Offers\opt MethodBody \\
                    \| MethMods \xcd"operator" TypeParams\opt PrefixOp \xcd"this" WhereClause\opt HasResultType\opt Offers\opt MethodBody \\
                    \| MethMods \xcd"operator" \xcd"this" TypeParams\opt FormalParams WhereClause\opt HasResultType\opt Offers\opt MethodBody \\
                    \| MethMods \xcd"operator" \xcd"this" TypeParams\opt FormalParams \xcd"=" \xcd"(" FormalParam  \xcd")" WhereClause\opt HasResultType\opt Offers\opt MethodBody \\
                    \| MethMods \xcd"operator" TypeParams\opt \xcd"(" FormalParam  \xcd")" \xcd"as" Type WhereClause\opt Offers\opt MethodBody \\
                    \| MethMods \xcd"operator" TypeParams\opt \xcd"(" FormalParam  \xcd")" \xcd"as" \xcd"?" WhereClause\opt HasResultType\opt Offers\opt MethodBody \\
                    \| MethMods \xcd"operator" TypeParams\opt \xcd"(" FormalParam  \xcd")" WhereClause\opt HasResultType\opt Offers\opt MethodBody \\
%(FROM #(prod:TypeParams)#)
          TypeParams \: \xcd"[" TypeParamList \xcd"]" & (\ref{prod:TypeParams}) \\
%(FROM #(prod:FormalParams)#)
        FormalParams \: \xcd"(" FormalParamList\opt \xcd")" & (\ref{prod:FormalParams}) \\
%(FROM #(prod:FormalParamList)#)
     FormalParamList \: FormalParam & (\ref{prod:FormalParamList}) \\
                    \| FormalParamList \xcd"," FormalParam \\
%(FROM #(prod:HasResultType)#)
       HasResultType \: \xcd":" Type & (\ref{prod:HasResultType}) \\
                    \| \xcd"<:" Type \\
%(FROM #(prod:MethodBody)#)
          MethodBody \: \xcd"=" LastExp \xcd";" & (\ref{prod:MethodBody}) \\
                    \| \xcd"=" Annotations\opt \xcd"{" BlockStatements\opt LastExp \xcd"}" \\
                    \| \xcd"=" Annotations\opt Block \\
                    \| Annotations\opt Block \\
                    \| \xcd";" \\
\end{bbgrammar}
%##)


\index{parameter!var}
\index{parameter!val}
A formal parameter may have a \xcd"val" or \xcd"var"
% , or \Xcd{ref}
modifier; \xcd`val` is the default.
The body of the method is executed in an environment in which 
each formal parameter corresponds to a local variable (\xcd`var` iff the
formal parameter is \xcd`var`)
and is initialized with the value of the actual parameter.

\subsection{Generic Instance Methods}
\index{method!generic instance}

\limitationx{}
In X10, an instance method may be generic: 
%~~gen ^^^ Classes1b7z
% package Classes1b7z;
% NOTEST
%~~vis
\begin{xten}
class Example {
  def example[T](t:T) = "I like " + t;
}
\end{xten}
%~~siv
%
%~~neg

However, the C++ back end does not currently support generic virtual instance
methods like \xcd`example`.  It does allow generic instance methods which are
\xcd`final` or \xcd`private`, and it does allow generic static methods.  


\subsection{Method Guards}
\label{MethodGuard}
\index{method!guard}
\index{guard!on method}

Often, a method will only make sense to invoke under certain
statically-determinable conditions.  For example, \xcd`example(x)` is only
well-defined when \xcd`x != null`, as \xcd`null.toString()` throws a null
pointer exception: 
%~~gen ^^^ Classes80
% package Classes.methodwithconstraintthingie;
% 
%~~vis
\begin{xten}
class Example {
   var f : String = "";
   def example(x:Object){x != null} = {
      this.f = x.toString();
   }
}
\end{xten}
%~~siv
%
%~~neg
\noindent
(We could have used a constrained type \xcd`Object{self!=null}` for \xcd`x`
instead; in
most cases it is a matter of personal preference or convenience of expression
which one to use.) 

The requirement of having a method guard is that callers must demonstrate to
the X10
compiler that the guard is satisfied.  (As usual with static constraint
checking, there is no runtime cost.  Indeed, this code can be more efficient
than usual, as it is statically provable that \xcd`x != null`.)
This may require a cast: 
%~~gen ^^^ Classes90
% package Classes.methodguardnadacastthingie;
% 
% class Example {var f : String = ""; def example(x:Object){x != null} = {this.f = x.toString();}}
% class Eyample {
%~~vis
\begin{xten}
  def exam(e:Example, x:Object) {
    if (x != null) 
       e.example(x as Object{x != null});
    // WRONG: if (x != null) e.example(x);
  }
\end{xten}
%~~siv
%}
%~~neg



The guard \xcd`{c}` 
in a guarded method 
\xcd`def m(){c} = E;`
specifies a constraint \xcd"c" on the
properties of the class \xcd"C" on which the method is being defined. The
method, in effect, only exists  for those instances of \xcd"C" which satisfy
\xcd"c".  It is 
illegal for code to invoke the method on objects whose static type is
not a subtype of \xcd"C{c}".

Specifically: 
    the compiler checks that every method invocation
    \xcdmath"o.m(e$_1$, $\dots$, e$_n$)"
    is type correct. Each argument
    \xcdmath"e$_i$" must have a
    static type \xcdmath"S$_i$" that is a subtype of the declared type
    \xcdmath"T$_i$" for the $i$th
    argument of the method, and the conjunction of the constraints on the
    static types 
    of the arguments must entail the guard in the parameter list
    of the method.

    The compiler checks that in every method invocation
    \xcdmath"o.m(e$_1$, $\dots$, e$_n$)"
    the static type of \xcd"o", \xcd"S", is a subtype of \xcd"C{c}", where the method
    is defined in class \xcd"C" and the guard for \xcd"m" is equivalent to
    \xcd"c".

    Finally, if the declared return type of the method is
    \xcd"D{d}", the
    return type computed for the call is
    \xcdmath"D{a: S; x$_1$: S$_1$; $\dots$; x$_n$: S$_n$; d[a/this]}",
    where \xcd"a" is a new
    variable that does not occur in
    \xcdmath"d, S, S$_1$, $\dots$, S$_n$", and
    \xcdmath"x$_1$, $\dots$, x$_n$" are the formal
    parameters of the method.


\limitation{
Using a reference to an outer class, \xcd`Outer.this`, in a constraint, is not supported.
}


\subsection{Property methods}
\index{method!property}
\index{property method}

Property methods are methods that can be evaluated in constraints.  
For example, the \xcd`eq()` method below tells if the \xcd`x` and \xcd`y`
properties are equal; the \xcd`is(z)` method tells if they are both equal to
\xcd`z`.  These can be used in constraints, as illustrated in the
\xcd`example()` method.
%~~gen ^^^ Classes100
%package Classes.PropertyMethods;
%~~vis
\begin{xten}
class Example(x:Int, y:Int) {
   def this(x:Int, y:Int) { property(x,y); }
   property eq() = (x==y);
   property is(z:Int) = x==z && y==z;
   def example( a : Example{eq()}, b : Example{is(3)} ) {}
}
\end{xten}
%~~siv
%
%~~neg


A method declared with the modifier \xcd"property" may be used
in constraints.  A property method declared in a class must have
a body and must not be \xcd"void".  The body of the method must
consist of only a single \xcd"return" statement or a single
expression.  It is a static error if the expression cannot be
represented in the constraint system.   Property methods may be \xcd`abstract`
in \xcd`abstract` classes, but are implicitly \xcd`final` in
non-\xcd`abstract` classes. 

The expression may contain invocations of other property methods. It is the
responsibility of the programmer to ensure that the evaluation of a property
terminates at compile-time, otherwise the type-checker will not terminate and
the program will fail to compile in a potentially most unfortunate way.

Property methods in classes are implicitly \xcd"final"; they cannot be
overridden.  It is a static error if a superclass has a property method with a
given signature, and a subclass has a method or property method with the same
signature.   It is a static error if a superclass has a property with some
name \xcd`p`, and a subclass has a nullary method of any kind (instance,
static, or property) also named \xcd`p`. 

% We should fix it so that a property p (or a property method p())  in the super class  cannot be shadowed by a field p in a subclass, or overridden by a method p() in the subclass.


A nullary property method definition may omit the formal parameters and
the \xcd"def" keyword.  That is, the following are equivalent:



%~~gen ^^^ Classes110
% package classes.waifsome1;
% class Waif(rect:Boolean, onePlace:Place, zeroBased:Boolean) {
%~~vis
\begin{xten}
property def rail(): Boolean = rect && onePlace == here && zeroBased;
\end{xten}
%~~siv
%}
%~~neg
and
%~~gen ^^^ Classes120
% package classes.waifsome2;
% class Waif(rect:Boolean, onePlace:Place, zeroBased:Boolean) {
%~~vis
\begin{xten}
property rail(): Boolean = rect && onePlace == here && zeroBased;
\end{xten}
%~~siv
%}
%~~neg

Similarly, nullary property methods can be inspected in constraints without
\xcd`()`. If \xcd`ob`'s type has a property \xcd`p`, then \xcd`ob.p` is that
property. Otherwise, if it has a nullary property method \xcd`p()`, \xcd`ob.p`
is equivalent to \xcd`ob.p()`. As a consequence, if the type provides both a
property \xcd`p` and a nullary method \xcd`p()`, then the property can be
accessed as \xcd`ob.p` and the method as \xcd`ob.p()`.\footnote{This only
applies to nullary property methods, not nullary instance methods.  Nullary
property methods perform limited computations, have no side effects, and
always return the same value, since
they have to be expressed in the constraint sublanguage.  In this sense, a
nullary property method does not behave hugely different from a property.
indeed, a compilation scheme which cached the value of the property method
would all but erase the distinction.  Other methods may
have more behavior, \eg, side effects, so we keep the \xcd`()` to make it
clear that a method call is potentially large.
}

%~~longexp~~`~~` ^^^ Classes130
% package classes.not.weasels;
% class Waif(rect:Boolean, onePlace:Place, zeroBased:Boolean) {
%   def this(rect:Boolean, onePlace:Place, zeroBased:Boolean) 
%          :Waif{self.rect==rect, self.onePlace==onePlace, self.zeroBased==zeroBased}
%          = {property(rect, onePlace, zeroBased);}
%   property rail(): Boolean = rect && onePlace == here && zeroBased;
%   static def zoink() {
%      val w : Waif{
%~~vis
\xcd`w.rail`, with either definition above, 
% }= new Waif(true, here, true);
% }}
%~~pxegnol
is equivalent to 
%~~longexp~~`~~` ^^^ Classes140
% package classes.not.ferrets;
% class Waif(rect:Boolean, onePlace:Place, zeroBased:Boolean) {
%   def this(rect:Boolean, onePlace:Place, zeroBased:Boolean) 
%          :Waif{self.rect==rect, self.onePlace==onePlace, self.zeroBased==zeroBased}
%          = {property(rect, onePlace, zeroBased);}
%   property rail(): Boolean = rect && onePlace == here && zeroBased;
%   static def zoink() {
%      val w : Waif{
%~~vis
\xcd`w.rail()`
% }= new Waif(true, here, true);
% }}
%~~pxegnol




\subsection{Method overloading, overriding, hiding, shadowing and obscuring}
\label{MethodOverload}
\index{method!overloading}



The definitions of method overloading, overriding, hiding, shadowing and
obscuring in \Xten{} are familiar from languages such as Java, modulo the
following considerations motivated by type parameters and dependent types.



Two or more methods of a class or interface may have the same
name if they have a different number of type parameters, or
they have formal parameters of different types.  \Eg, the following is legal: 

%~~gen ^^^ Classes150
% package Classes.Mful;
%~~vis
\begin{xten}
class Mful{
   def m() = 1;
   def m[T]() = 2;
   def m(x:Int) = 3;
   def m[T](x:Int) = 4;
}
\end{xten}
%~~siv
%
%~~neg

\XtenCurrVer{} does not permit overloading based on constraints. That is, the
following is {\em not} legal, although either method definition individually
is legal:
\begin{xten}
   def n(x:Int){x==1} = "one";
   def n(x:Int){x!=1} = "not";
\end{xten}


The definition of a method declaration \xcdmath"m$_1$" ``having the same signature
as'' a method declaration \xcdmath"m$_2$" involves identity of types. 

\noo{The following few paragraphs need to be stared at carefully.  I think
they are contradictory and/or wrong.}

The {\em constraint erasure} of a type \xcdmath"T" is defined as follows.
The constraint erasure of  (a)~a class, interface or struct type \xcdmath"T" is 
\xcdmath"T"; (b)~a type \xcdmath"T{c}" is the constraint erasure of 
\xcdmath"T"; (b)~a type \xcdmath"T[S$_1$,$\ldots$,S$_n$]" 
is \xcdmath"T'[S$_1$',$\ldots$,S$_n$']" where each primed type is the erasure of 
the corresponding unprimed type.
 Two methods are said to have {\em equivalent signatures} if (a) they have the
 same number of type parameters, 
(b) they have the same number of formal (value) parameters, and (c)
for each formal parameter the constraint erasure of its types are equivalent. It is a
compile-time error for there to be two methods with the same name and
equivalent signatures in a class (either defined in that class or in a
superclass), unless the signatures are identical and one of the methods is
defined in a superclass (in which case the superclass's method is overridden
by the subclass's).

 

A class \xcd"C" may not have two non-identical but equivalent 
declarations for
a method named 
\xcd"m"---either 
defined at \xcd"C" or inherited:
\begin{xtenmath}
def m[X$_1$, $\dots$, X$_m$](v$_1$: T$_1$, $\dots$, v$_n$: T$_n$){tc}: T {...}
def m[X$_1$, $\dots$, X$_m$](v$_1$: S$_1$, $\dots$, v$_n$: S$_n$){sc}: S {...}
\end{xtenmath}
\noindent
if it is the case that the constraint erasures of the types \xcdmath"T$_1$",
\dots, \xcdmath"T$_n$" are
equivalent to the constraint erasures of the types \xcdmath"S$_1$, $\dots$, T$_n$"
respectively.



In addition, the guard of a overriding method must be 
no stronger than the guard of the overridden method.   This
ensures that any virtual call to the method
satisfies the guard of the callee.


  If a class \xcd"C" overrides a method of a class or interface
  \xcd"B", the guard of the method in \xcd"B" must entail
  the guard of the method in \xcd"C".


A class \xcd"C" inherits from its direct superclass and superinterfaces all
their methods visible according to the access modifiers
of the superclass/superinterfaces that are not hidden or overridden. A method \xcdmath"M$_1$" in a class
\xcd"C" overrides
a method \xcdmath"M$_2$" in a superclass \xcd"D" if
\xcdmath"M$_1$" and \xcdmath"M$_2$" have the same signature with constraints erased.
Methods are overriden on a signature-by-signature basis.

\section{Constructors}
\index{constructor}

Instances of classes are created by the \xcd`new` expression: \\
%##(ClassInstCreationExp
\begin{bbgrammar}
%(FROM #(prod:ClassInstCreationExp)#)
ClassInstCreationExp \: \xcd"new" TypeName TypeArguments\opt \xcd"(" ArgumentList\opt \xcd")" ClassBody\opt & (\ref{prod:ClassInstCreationExp}) \\
                    \| \xcd"new" TypeName \xcd"[" Type \xcd"]" \xcd"[" ArgumentList\opt \xcd"]" \\
                    \| Primary \xcd"." \xcd"new" Id TypeArguments\opt \xcd"(" ArgumentList\opt \xcd")" ClassBody\opt \\
                    \| AmbiguousName \xcd"." \xcd"new" Id TypeArguments\opt \xcd"(" ArgumentList\opt \xcd")" ClassBody\opt \\
\end{bbgrammar}
%##)

This constructs a new object, and calls some code, called a {\em constructor},
to initialize the newly-created object properly.

Constructors are defined like methods, except that they are named \xcd`this`
and ordinary methods may not be.    The content of a constructor body has
certain capabilities (\eg, \xcd`val` fields of the object may be initialized)
and certain restrictions (\eg, most methods cannot be called); see
\Sref{ObjectInitialization} for the details.

\begin{ex}

The following class provides two constructors.  The unary constructor 
\xcd`def this(b : Int)` allows initialization of the \xcd`a` field to an 
arbitrary value.  The nullary constructor \xcd`def this()` gives it a default
value of 10.  The \xcd`example` method illustrates both of these calls.


%~~gen ^^^ ClassesCtor10
% package ClassesCtor10;
%~~vis
\begin{xten}
class C {
  public val a : Int;
  def this(b : Int) { a = b; } 
  def this()        { a = 10; }
  static def example() {
     val two = new C(2);
     assert two.a == 2;
     val ten = new C(); 
     assert ten.a == 10;
  }
}
\end{xten}
%~~siv
%
%~~neg
\end{ex}

\subsection{Automatic Generation of Constructors}
\index{constructor!generated}

Classes that have no constructors written in the class declaration are
automatically given a constructor which sets the class properties and does
nothing else. If this automatically-generated constructor is not valid (\eg,
if the class has \xcd`val` fields that need to be initialized in a
constructor), the class has no constructor, which is a static error.

\begin{ex}
The following class has no explicit constructor.
Its implicit constructor is 
\xcd`def this(x:Int){property(x);}`
This implicit constructor is valid, and so is the class. 
%~~gen ^^^ ClassesCtor20
% package ClassesCtor20;
%~~vis
\begin{xten}
class C(x:Int) {
  static def example() {
    val c : C = new C(4);
    assert c.x == 4;
  }
}
\end{xten}
%~~siv
%
%~~neg
\noindent 


The following class has the same default constructor.  However, that
constructor does not initialize \xcd`d`, and thus is invalid.  This 
class does not compile; it needs an explicit constructor.
%~~gen ^^^ ClassCtor30_MustFailCompile
% NOCOMPILE
%~~vis
\begin{xten}
// THIS CODE DOES NOT COMPILE
class Cfail(x:Int) {
  val d: Int;
  static def example() {
    val wrong = new Cfail(40);
  }
}
\end{xten}
%~~siv
%
%~~neg


\end{ex}

\subsection{Calling Other Constructors}

The {\em first} statement of a constructor body may be a call of the form 
\xcd`this(a,b,c)` or \xcd`super(a,b,c)`.  The former will execute the body of
the matching constructor of the current class; the latter, of the superclass. 
This allows a measure of abstraction in constructor definitions; one may be
defined in terms of another.

\begin{ex}
The following class has two constructors.  \xcd`new Ctors(123)` constructs a
new \xcd`Ctors` object with parameter 123.  \xcd`new Ctors()` constructs one
whose parameter has a default value of 100: 
%~~gen ^^^ Classes5q6q
% package Classes5q6q;
%~~vis
\begin{xten}
class Ctors {
  val a : Int;
  def this(a:Int) { this.a = a; }
  def this() {
    this(100);
  }
}
\end{xten}
%~~siv
%
%~~neg
\end{ex}

In the case of a class which implements \xcd`operator ()` 
--- or any other constructor and application with the same signature --- 
this can be ambiguous.  If \xcd`this()` appears as the first statement of a
constructor body, it could, in principle, mean either a constructor call or an
operator evaluation.   This ambiguity is resolved so that \xcd`this()` always
means the constructor invocation.  If, for some reason, it is necessary to
invoke an application operator early in a constructor, precede it with a dummy
statement, such as \xcd`if(false);`  

\section{Static initialization}
\label{StaticInitialization}
\index{initialization!static}
The \Xten{} runtime implements the following procedure to ensure
reliable initialization of the static state of classes.


Execution (of an entire X10 program) commences with a single thread executing
the 
\emph{initialization} phase of an \Xten{} computation at place \Xcd{0}. This
phase must complete successfully before the body of the \Xcd{main} method is
executed.

The initialization phase should be thought of as if it is implemented in
the following fashion. (The implementation may do something more
efficient as long as it is faithful to this semantics.)

\begin{xten}
Within the scope of a new finish
for every static field f of every class C 
   (with type T and initializer e):
async {
  val l = e; 
  ateach (Dist.makeUnique()) {
     assign l to the static f field of 
         the local C class object;
     mark the f field of the local C 
         class object as initialized;
  }
}
\end{xten}

During this phase, any read of a static field \Xcd{C.f} (where \Xcd{f} is of type \Xcd{T})
is replaced by a call to the method \Xcd{C.read\_f():T} defined on class \Xcd{C}
as follows

\begin{xten}
def read_f():T {
   when (initialized(C.f)){};
   return C.f;
}
\end{xten}
 

If all these activities terminate normally, all static fields have values of
their declared types, 
and the \Xcd{finish} terminates normally. If
any activity throws an exception, the \Xcd{finish} throws an
exception. Since no user code is executing which can catch exceptions
thrown by the finish, such exceptions are printed on the console, and
computation aborts.

If the activities deadlock, the implementation deadlocks.

In all cases, the main method is executed only once all static fields
have been initialized correctly.

Since static state is immutable and is replicated to all places via 
the initialization phase as described above, it can be accessed from
any place.



\section{User-Defined Operators}
\label{sect:operators}
\index{operator}
\index{operator!user-defined}

It is often convenient to have methods named by symbols rather than words.
For example, suppose that we wish to define a \xcd`Poly` class of
polynomials -- for the sake of illustration, single-variable polynomials with
\xcd`Int` coefficients.  It would be very nice to be able to manipulate these
polynomials by the usual operations: \xcd`+` to add, \xcd`*` to multiply,
\xcd`-` to subtract, and \xcd`p(x)` to compute the value of the polynomial at
argument \xcd`x`.  We would like to write code thus: 
%~~gen ^^^ Classes160
% package Classes.In.Poly101;
% // Integer-coefficient polynomials of one variable.
% class Poly {
%   public val coeff : Array[Int](1);
%   public def this(coeff: Array[Int](1)) { this.coeff = coeff;}
%   public def degree() = coeff.size-1;
%   public def a(i:Int) = (i<0 || i>this.degree()) ? 0 : coeff(i);
%
%   public static operator (c : Int) as Poly = new Poly([c]);
%
%   public operator this(x:Int) {
%     val d = this.degree();
%     var s : Int = this.a(d);
%     for( i in 1 .. this.degree() ) {
%        s = x * s + a(d-i);
%     }
%     return s;
%   }
%
%   public operator this + (p:Poly) =  new Poly(
%      new Array[Int](
%         Math.max(this.coeff.size, p.coeff.size),
%         (i:Int) => this.a(i) + p.a(i)
%      ));
%   public operator this - (p:Poly) = this + (-1)*p;
%
%   public operator this * (p:Poly) = new Poly(
%      new Array[Int](
%        this.degree() + p.degree() + 1,
%        (k:Int) => sumDeg(k, this, p)
%        )
%      );
%
%
%   public operator (n : Int) + this = (n as Poly) + this;
%   public operator this + (n : Int) = (n as Poly) + this;
%
%   public operator (n : Int) - this = (n as Poly) + (-1) * this;
%   public operator this - (n : Int) = ((-n) as Poly) + this;
%
%   public operator (n : Int) * this = new Poly(
%      new Array[Int](
%        this.degree()+1,
%        (k:Int) => n * this.a(k)
%      ));
%   private static def sumDeg(k:Int, a:Poly, b:Poly) {
%      var s : Int = 0;
%      for( i in 0 .. k ) s += a.a(i) * b.a(k-i);
%        // x10.io.Console.OUT.println("sumdeg(" + k + "," + a + "," + b + ")=" + s);
%      return s;
%      };
%   public final def toString() = {
%      var allZeroSoFar : Boolean = true;
%      var s : String ="";
%      for( i in 0..this.degree() ) {
%        val ai = this.a(i);
%        if (ai == 0) continue;
%        if (allZeroSoFar) {
%           allZeroSoFar = false;
%           s = term(ai, i);
%        }
%        else
%           s +=
%              (ai > 0 ? " + " : " - ")
%             +term(ai, i);
%      }
%      if (allZeroSoFar) s = "0";
%      return s;
%   }
%   private final def term(ai: Int, n:Int) = {
%      val xpow = (n==0) ? "" : (n==1) ? "x" : "x^" + n ;
%      return (ai == 1) ? xpow : "" + Math.abs(ai) + xpow;
%   }
%
%   public static def Main(ss:Array[String](1)) = main(ss);
%


%~~vis
\begin{xten}
  public static def main(Array[String](1)):void {
     val X = new Poly([0,1]);
     val t <: Poly = 7 * X + 6 * X * X * X; 
     val u <: Poly = 3 + 5*X - 7*X*X;
     val v <: Poly = t * u - 1;
     for( i in -3 .. 3) {
       x10.io.Console.OUT.println(
         "" + i + "	X:" + X(i) + "	t:" + t(i) 
         + "	u:" + u(i) + "	v:" + v(i)
         );
     }
  }

\end{xten}
%~~siv
%}
%~~neg

Writing the same code with method calls, while possible, is far less elegant: 
%~~gen ^^^ Classes170

%package Classes.In.Remedial.Poly101;
% // Integer-coefficient polynomials of one variable.
% class UglyPoly {
%   public val coeff : Array[Int](1);
%   public def this(coeff: Array[Int](1)) { this.coeff = coeff;}
%   public def degree() = coeff.size-1;
%   public  def  a(i:Int) = (i<0 || i>this.degree()) ? 0 : coeff(i);
%
%   public static operator (c : Int) as UglyPoly = new UglyPoly([c]);
%
%   public def apply(x:Int) {
%     val d = this.degree();
%     var s : Int = this.a(d);
%     for( i in 1 .. this.degree() ) {
%        s = x * s + a(d-i);
%     }
%     return s;
%   }
%
%   public operator this + (p:UglyPoly) =  new UglyPoly(
%      new Array[Int](
%         Math.max(this.coeff.size, p.coeff.size),
%         (i:Int) => this.a(i) + p.a(i)
%      ));
%   public operator this - (p:UglyPoly) = this + (-1)*p;
%
%   public operator this * (p:UglyPoly) = new UglyPoly(
%      new Array[Int](
%        this.degree() + p.degree() + 1,
%        (k:Int) => sumDeg(k, this, p)
%        )
%      );
%
%
%   public operator (n : Int) + this = (n as UglyPoly) + this;
%   public operator this + (n : Int) = (n as UglyPoly) + this;
%
%   public operator (n : Int) - this = (n as UglyPoly) + (-1) * this;
%   public operator this - (n : Int) = ((-n) as UglyPoly) + this;
%
%   public operator (n : Int) * this = new UglyPoly(
%      new Array[Int](
%        this.degree()+1,
%        (k:Int) => n * this.a(k)
%      ));
%   private static def sumDeg(k:Int, a:UglyPoly, b:UglyPoly) {
%      var s : Int = 0;
%      for( i in 0 .. k ) s += a.a(i) * b.a(k-i);
%        // x10.io.Console.OUT.println("sumdeg(" + k + "," + a + "," + b + ")=" + s);
%      return s;
%      };
%   public final def toString() = {
%      var allZeroSoFar : Boolean = true;
%      var s : String ="";
%      for( i in 0..this.degree() ) {
%        val ai = this.a(i);
%        if (ai == 0) continue;
%        if (allZeroSoFar) {
%           allZeroSoFar = false;
%           s = term(ai, i);
%        }
%        else
%           s +=
%              (ai > 0 ? " + " : " - ")
%             +term(ai, i);
%      }
%      if (allZeroSoFar) s = "0";
%      return s;
%   }
%   private final def term(ai: Int, n:Int) = {
%      val xpow = (n==0) ? "" : (n==1) ? "x" : "x^" + n ;
%      return (ai == 1) ? xpow : "" + Math.abs(ai) + xpow;
%   }
%
%   def mult(p:UglyPoly) = this * p;
%   def mult(n:Int) = n * this;
%   def plus(p:UglyPoly) = this + p;
%   def plus(n:Int) = n + this;
%   def minus(p:UglyPoly) = this - p;
%   def minus(n:Int) = this - n;
%   static def const(n:Int) = n as UglyPoly;
%
%
%~~vis
\begin{xten}
  public static def uglymain() {
     val X = new UglyPoly([0,1]);
     val t <: UglyPoly = X.mult(7).plus(
                          X.mult(X).mult(X).mult(6));  
     val u <: UglyPoly = const(3).plus(
                           X.mult(5)).minus(X.mult(X).mult(7));
     val v <: UglyPoly = t.mult(u).minus(1);
     for( i in -3 .. 3) {
       x10.io.Console.OUT.println(
         "" + i + "	X:" + X.apply(i) + "	t:" + t.apply(i) 
          + "	u:" + u.apply(i) + "	v:" + v.apply(i)
         );
     }
  }
\end{xten}
%~~siv
%}
%~~neg

The operator-using code can be written in X10, though a few variations are
necessary to handle such exotic cases as \xcd`1+X`.

%% HERE!!

Most X10 operators can be given definitions.\footnote{Indeed, even for the
standard types, these operators are defined in the library.  Not even as basic
an operation as integer addition is built into the language.  Conversely, if
you define a full-featured numeric type, it will have most of the privileges that
the standard ones enjoy.  The missing priveleges are (1) literals; (2) 
the \xcd`..` operator won't compute the \xcd`zeroBased` and \xcd`rail`
properties as it does for \xcd`Int` ranges; (3) \xcd`*` won't track ranks, as
it does for \xcd`Region`s; 
(4) \xcd`&&` and \xcd`||` won't short-circuit, as they do for \xcd`Boolean`s, 
 and (5) \xcd`a==b` will only coincide with
\xcd`a.equals(b)` if coded that way.  For example, a \xcd`Polar` type might
have many representations for the origin, as radius 0 and any angle; these
will be \xcd`equals()`, but will not be \xcd`==`.}  (However, \xcd`&&` and
\xcd`||` 
are only short-circuiting for \xcd`Boolean` expressions; user-defined versions
of these operators have no special execution behavior.)

The user-definable operations are (in order of precedence): \\
\begin{tabular}{l}
implicit type coercions\\
postfix \xcd`()`\\
\xcd`as T`\\
unary \xcd`-`, unary \xcd`+`, \xcd`!`, \xcd`~`\\
\xcd`..`\\
\xcd`*     `  \xcd`/     `  \xcd`%` \\
\xcd`+` \xcd`     -` \\
\xcd`<<    ` \xcd`>>    ` \xcd`>>>   ` \xcd`->` \\
\xcd`>     ` \xcd`>=    ` \xcd`<     ` \xcd`<=     ` 
\xcd`in     ` 
\xcd`&` \\
\xcd`^` \\
\xcd`|` \\
\xcd`&&` \\
\xcd`||` \\
\end{tabular}



\subsection{Binary Operators}

Defining the sum \xcd`P+Q` of two polynomials looks much like a method
definition.  It uses the \xcd`operator` keyword instead of \xcd`def`, and
\xcd`this` appears in the definition in the place that a \xcd`Poly` would
appear in a use of the operator.  So, 
\xcd`operator this + (p:Poly)` explains how to add \xcd`this` to a
\xcd`Poly` value.
%~~gen ^^^ Classes180
% package Classes.In.Poly102;
%~~vis
\begin{xten}
class Poly {
  public val coeff : Array[Int](1);
  public def this(coeff: Array[Int](1)) { this.coeff = coeff;}
  public def degree() = coeff.size-1;
  public def  a(i:Int) = (i<0 || i>this.degree()) ? 0 : coeff(i);

  public operator this + (p:Poly) =  new Poly(
     new Array[Int](
        Math.max(this.coeff.size, p.coeff.size),
        (i:Int) => this.a(i) + p.a(i)
     )); 
  // ... 
\end{xten}
%~~siv
%   public operator (n : Int) + this = new Poly([n]) + this;
%   public operator this + (n : Int) = new Poly([n]) + this;
% 
%   def makeSureItWorks() {
%      val x = new Poly([0,1]);
%      val p <: Poly = x+x+x;
%      val q <: Poly = 1+x;
%      val r <: Poly = x+1;
%   }
%     
% }
%~~neg


The sum of a polynomial and an integer, \xcd`P+3`, looks like
an overloaded method definition.  
%~~gen ^^^ Classes190
% package Classes.In.Poly103;
% class Poly {
%   public val coeff : Array[Int](1);
%   public def this(coeff: Array[Int](1)) { this.coeff = coeff;}
%   public def degree() = coeff.size-1;
%   public def  a(i:Int) = (i<0 || i>this.degree()) ? 0 : coeff(i);
% 
%   public operator this + (p:Poly) =  new Poly(
%      new Array[Int](
%         Math.max(this.coeff.size, p.coeff.size),
%         (i:Int) => this.a(i) + p.a(i)
%      ));
%    public operator (n : Int) + this = new Poly([n]) + this;
%~~vis
\begin{xten}
   public operator this + (n : Int) = new Poly([n]) + this;
\end{xten}
%~~siv
% 
%   def makeSureItWorks() {
%      val x = new Poly([0,1]);
%      val p <: Poly = x+x+x;
%      val q <: Poly = 1+x;
%      val r <: Poly = x+1;
%   }
%     
% }
%~~neg


However, we want to allow the sum of an integer and a polynomial as well:
\xcd`3+P`.  It would be quite inconvenient to have to define this as a method
on \xcd`Int`; changing \xcd`Int` is far outside of normal coding.  So, we
allow it as a method on \xcd`Poly` as well.


%~~gen ^^^ Classes200
% package Classes.In.Poly104o;
% class Poly {
%   public val coeff : Array[Int](1);
%   public def this(coeff: Array[Int](1)) { this.coeff = coeff;}
%   public def degree() = coeff.size-1;
%   public def  a(i:Int) = (i<0 || i>this.degree()) ? 0 : coeff(i);
% 
%   public operator this + (p:Poly) =  new Poly(
%      new Array[Int](
%         Math.max(this.coeff.size, p.coeff.size),
%         (i:Int) => this.a(i) + p.a(i)
%      ));
%~~vis
\begin{xten}
   public operator (n : Int) + this = new Poly([n]) + this;
\end{xten}
%~~siv
% 
%   public operator this + (n : Int) = new Poly([n]) + this;
%   def makeSureItWorks() {
%      val x = new Poly([0,1]);
%      val p <: Poly = x+x+x;
%      val q <: Poly = 1+x;
%      val r <: Poly = x+1;
%   }
%     
% }
%~~neg

Furthermore, it is sometimes convenient to express a binary operation as a
static method on a class. 
The definition for the sum of two
\xcd`Poly`s could have been written:
%~~gen ^^^ Classes210
% package Classes.In.Poly105;
% class Poly {
%   public val coeff : Array[Int](1);
%   public def this(coeff: Array[Int](1)) { this.coeff = coeff;}
%   public def degree() = coeff.size-1;
%   public def  a(i:Int) = (i<0 || i>this.degree()) ? 0 : coeff(i);
%~~vis
\begin{xten}
  public static operator (p:Poly) + (q:Poly) =  new Poly(
     new Array[Int](
        Math.max(q.coeff.size, p.coeff.size),
        (i:Int) => q.a(i) + p.a(i)
     ));
\end{xten}
%~~siv
%
%   public operator (n : Int) + this = new Poly([n]) + this;
%   public operator this + (n : Int) = new Poly([n]) + this;
% 
%   def makeSureItWorks() {
%      val x = new Poly([0,1]);
%      val p <: Poly = x+x+x;
%      val q <: Poly = 1+x;
%      val r <: Poly = x+1;
%   }
%     
% }
%~~neg

This requires the following grammar: \\
%##(MethodDecl
\begin{bbgrammar}
%(FROM #(prod:MethodDecl)#)
          MethodDecl \: MethMods \xcd"def" Id TypeParams\opt FormalParams WhereClause\opt HasResultType\opt Offers\opt MethodBody & (\ref{prod:MethodDecl}) \\
                    \| MethMods \xcd"operator" TypeParams\opt \xcd"(" FormalParam  \xcd")" BinOp \xcd"(" FormalParam  \xcd")" WhereClause\opt HasResultType\opt Offers\opt MethodBody \\
                    \| MethMods \xcd"operator" TypeParams\opt PrefixOp \xcd"(" FormalParam  \xcd")" WhereClause\opt HasResultType\opt Offers\opt MethodBody \\
                    \| MethMods \xcd"operator" TypeParams\opt \xcd"this" BinOp \xcd"(" FormalParam  \xcd")" WhereClause\opt HasResultType\opt Offers\opt MethodBody \\
                    \| MethMods \xcd"operator" TypeParams\opt \xcd"(" FormalParam  \xcd")" BinOp \xcd"this" WhereClause\opt HasResultType\opt Offers\opt MethodBody \\
                    \| MethMods \xcd"operator" TypeParams\opt PrefixOp \xcd"this" WhereClause\opt HasResultType\opt Offers\opt MethodBody \\
                    \| MethMods \xcd"operator" \xcd"this" TypeParams\opt FormalParams WhereClause\opt HasResultType\opt Offers\opt MethodBody \\
                    \| MethMods \xcd"operator" \xcd"this" TypeParams\opt FormalParams \xcd"=" \xcd"(" FormalParam  \xcd")" WhereClause\opt HasResultType\opt Offers\opt MethodBody \\
                    \| MethMods \xcd"operator" TypeParams\opt \xcd"(" FormalParam  \xcd")" \xcd"as" Type WhereClause\opt Offers\opt MethodBody \\
                    \| MethMods \xcd"operator" TypeParams\opt \xcd"(" FormalParam  \xcd")" \xcd"as" \xcd"?" WhereClause\opt HasResultType\opt Offers\opt MethodBody \\
                    \| MethMods \xcd"operator" TypeParams\opt \xcd"(" FormalParam  \xcd")" WhereClause\opt HasResultType\opt Offers\opt MethodBody \\
\end{bbgrammar}
%##)
When X10 attempts to typecheck a binary operator expression like \xcd`P+Q`, it
first typechecks \xcd`P` and \xcd`Q`. Then, it looks for operator declarations
for \xcd`+` in the types of \xcd`P` and \xcd`Q`. If there are none, it is a
static error. If there is precisely one, that one will be used. If there are
several, X10 looks for a {\em best-matching} operation, \viz{} one which does
not require the operands to be converted to another type. For example,
\xcd`operator this + (n:Long)` and \xcd`operator this + (n:Int)` both apply to
\xcd`p+1`, because \xcd`1` can be converted from an \xcd`Int` to a \xcd`Long`.
However, the \xcd`Int` version will be chosen because it does not require a
conversion. If even the best-matching operation is not uniquely determined,
the compiler will report a static error.

The main difference between expressing a binary operation as an instance
method (with a \xcd`this` in the definition) and a static one (no \xcd`this`)
is that instance methods don't apply any conversions, while static methods
attempt to convert both arguments. 




\subsection{Unary Operators}

Unary operators are defined in a similar way, with \xcd`this` appearing in the
\xcd`operator` definition where an actual value would occur in a unary
expression.  The operator to negate a polynomial is: 

%~~gen ^^^ Classes220
% package Classes.In.Poly106;
% class Poly {
%   public val coeff : Array[Int](1);
%   public def this(coeff: Array[Int](1)) { this.coeff = coeff;}
%   public def degree() = coeff.size-1;
%   public def  a(i:Int) = (i<0 || i>this.degree()) ? 0 : coeff(i);
%~~vis
\begin{xten}
  public operator - this = new Poly(
    new Array[Int](coeff.size, (i:Int) => -coeff(i))
    );
\end{xten}
%~~siv
%   def makeSureItWorks() {
%      val x = new Poly([0,1]);
%      val p <: Poly = -x;
%   }
% }
%~~neg

The rules for typechecking a unary operation are the same as for methods; the
complexities of binary operations are not needed.

\bard{List the operators which this works for, in precedence order}


\subsection{Type Conversions}
\index{type conversion!user-defined}

Explicit type conversions, \xcd`e as T{c}`, can be defined as operators on
class \xcd`T`.

%~~gen ^^^ Classes230
% package Classes_explicit_type_conversions_a;
%~~vis
\begin{xten}
class Poly {
  public val coeff : Array[Int](1);
  public def this(coeff: Array[Int](1)) { this.coeff = coeff;}
  public static operator (a:Int) as Poly = new Poly([a]);
  public static def main(Array[String](1)):void {
     val three : Poly = 3 as Poly;
  }
}
\end{xten}
%~~siv
%
%~~neg


Furthermore, \xcd`T` may be written as \xcd`?` in the definition of a type
conversion operator (and only there) to have it inferred from context: 

%~~gen ^^^ Classes9x1k
% package Classes9x1k;
%~~vis
\begin{xten}
class Caster(n:Int) {
  public static operator (a:Int) as ? = new Caster(a); 
  public static def example() {
    val c : Caster{n==3} = 3 as Caster{n==3};
  }
}
\end{xten}
%~~siv
%
%~~neg


The \xcd`?` may be given a bound, such as \xcd`as ? <: Caster`, if desired.

% TODO

%%TODO%%  You may define a type conversion to a constrained type, like \xcd`Poly` in
%%TODO%%  the previous example.   If you convert to a more specific constraint, X10 will use
%%TODO%%  the conversion, but insert a dynamic check to make sure that you have
%%TODO%%  satisfied the more specific constraint.  
%%TODO%%  For example: 
%%TODO%%  %~x~gen
%%TODO%%  %package Classes.And.Type.Conversions;
%%TODO%%  %~x~vis
%%TODO%%  \begin{xten}
%%TODO%%  class Uni(n:Int) {
%%TODO%%  
%%TODO%%    public def this(n:Int) : Uni{self.n==n} = {property(n);}
%%TODO%%    static operator (String) as Uni{self.n != 9} = new Uni(3);
%%TODO%%    public static def main(Array[String](1)):void {
%%TODO%%      val u = "" as Uni{self.n != 9 && self.n != 3};
%%TODO%%    }
%%TODO%%  }
%%TODO%%  \end{xten}
%%TODO%%  %~x~siv
%%TODO%%  %
%%TODO%%  %~x~neg
%%TODO%%  The string \xcd`""` is converted to \xcd`Uni{self.n != 9}` via the defined
%%TODO%%  conversion operator, and that value is checked against the remaining
%%TODO%%  constraints \xcd`{self.n != 3}` at runtime.  (In this case it will fail.)
%%TODO%%  
%%TODO%%  There may be many conversions from different types to \xcd`T`, but there may
%%TODO%%  be at most one conversion from any given type to \xcd`T`. 
%%TODO%%  
\bard{Syntax}

\subsection{Implicit Type Coercions}
\label{sect:ImplicitCoercion}
\index{type conversion!implicit}

You may also define {\em implicit} type coercions to \xcd`T{c}` as static
operators in class \xcd`T`.  The syntax for this is
\xcd`static operator (x:U) : T{c} = e`.
Implicit coercions are used automatically by the compiler on method calls 
(\Sref{sect:MethodResolution}) and assignments (\Sref{domedomedome}).



For example, we can define an implicit coercion from \xcd`Int` to \xcd`Poly`,
and avoid having to define the sum of an integer and a polynomial
as many special cases.  In the following example, we only define \xcd`+` on
two polynomials (using a \xcd`static` operator, so that implicit coercions
will be used -- they would not be for an instance method operator).  The
calculation \xcd`1+x` coerces \xcd`1` to a polynomial and uses polynomial
addition to add it to \xcd`x`.

%~~gen ^^^ Classes240
% package Classes.And.Implicit.Coercions;
% class Poly {
%   public val coeff : Array[Int](1);
%   public def this(coeff: Array[Int](1)) { this.coeff = coeff;}
%   public def degree() = coeff.size-1;
%   public def  a(i:Int) = (i<0 || i>this.degree()) ? 0 : coeff(i);
%   public final def toString() = {
%      var allZeroSoFar : Boolean = true;
%      var s : String ="";
%      for( i in 0..this.degree() ) {
%        val ai = this.a(i);
%        if (ai == 0) continue;
%        if (allZeroSoFar) {
%           allZeroSoFar = false;
%           s = term(ai, i);
%        }
%        else 
%           s += 
%              (ai > 0 ? " + " : " - ")
%             +term(ai, i);
%      }
%      if (allZeroSoFar) s = "0";
%      return s;
%   }
%   private final def term(ai: Int, n:Int) = {
%      val xpow = (n==0) ? "" : (n==1) ? "x" : "x^" + n ;
%      return (ai == 1) ? xpow : "" + Math.abs(ai) + xpow;
%   }

%~~vis
\begin{xten}
  public static operator (c : Int) : Poly = new Poly([c]);

  public static operator (p:Poly) + (q:Poly) = new Poly(
      new Array[Int](
        Math.max(p.coeff.size, q.coeff.size),
        (i:Int) => p.a(i) + q.a(i)
     ));

  public static def main(Array[String](1)):void {
     val x = new Poly([0,1]);
     x10.io.Console.OUT.println("1+x=" + (1+x));
  }
\end{xten}
%~~siv
%}
%~~neg

\bard{Syntax}

\subsection{Assignment and Application Operators}
\index{assignment operator}
\index{application operator}
\index{()}
\index{()=}
\label{set-and-apply}
X10 allows types to implement the subscripting / function application
operator, and indexed assignment.  The \xcd`Array`-like classes take advantage
of both of these in \xcd`a(i) = a(i) + 1`.  

\xcd`a(b,c,d)`
is an operator call, to an operator defined with 
\xcd`public operator this(b:B, c:C, d:D)`.  It may be overloaded.
For
example, an ordered dictionary structure could allow subscripting by numbers
with \xcd`public operator this(i:Int)`, and by string-valued keys with 
\xcd`public operator this(s:String)`.  


\xcd`a(i,j)=b` is an \xcd`operator` as well, with zero or more indices
\xcd`i,j`.  It may also be overloaded. 

The update operations \xcd`a(i) += b` are defined to be the same as the
corresponding \xcd`a(i) = a(i) + b`. This applies for all arities of
arguments, and all types, and all binary operations. Of course to use this,
both the application and assignment \xcd`operator`s must be defined.


\begin{ex}

The \xcd`Oddvec` class of somewhat peculiar vectors illustrates this.
\xcd`a()` returns a string representation of the oddvec, which probably should
be done by \xcd`toString()` instead.  
\xcd`a(i)` sensibly picks out one of the three
coordinates of \xcd`a`.
\xcd`a()=b` sets all the coordinates of \xcd`a` to \xcd`b`.
\xcd`a(i)=b` assigns to one of the
coordinates.  \xcd`a(i,j)=b` assigns different values to \xcd`a(i)` and
\xcd`a(j)`, purely for the sake of the example.

%~~gen ^^^ Classes250
% package Classes.Assignments1_oddvec;
%~~vis
\begin{xten}
class Oddvec {
  var v : Array[Int](1) = new Array[Int](3, (Int)=>0);
  public operator this () = 
      "(" + v(0) + "," + v(1) + "," + v(2) + ")";
  public operator this () = (newval: Int) { 
    for(p in v) v(p) = newval;
  }
  public operator this(i:Int) = v(i);
  public operator this(i:Int, j:Int) = [v(i),v(j)];
  public operator this(i:Int) = (newval:Int) = {v(i) = newval;}
  public operator this(i:Int, j:Int) = (newval:Int) = {
       v(i) = newval; v(j) = newval+1;} 
  public def example() {
    this(1) = 6;   assert this(1) == 6;
    this(1) += 7;  assert this(1) == 13;
  }
\end{xten}
%~~siv
% }
%  class Hook { def run() {
%     val a = new Oddvec();
%     assert a().equals("(0,0,0)");
%     a() = 1;
%     assert a().equals("(1,1,1)");
%     a(1) = 4;
%     assert a().equals("(1,4,1)");
%     a(0,2) = 5;
%     assert a().equals("(5,4,6)");
%     return true;
%   }
% }
%~~neg

\end{ex}

\section{Class Guards and Invariants}\label{DepType:ClassGuard}
\index{type invariants}
\index{class invariants}
\index{invariant!type}
\index{invariant!class}
\index{guard}


Classes (and structs and interfaces) may specify a {\em class guard}, a
constraint which must hold on all values of the class.    In the following
example, a \xcd`Line` is defined by two distinct \xcd`Pt`s\footnote{We use \xcd`Pt`
to avoid any possible confusion with the built-in class \xcd`Point`.}
%~~gen ^^^ Classes260
% package classes.guards.invariants.glurp;
%~~vis
\begin{xten}
class Pt(x:Int, y:Int){}
class Line(a:Pt, b:Pt){a != b} {}
\end{xten}
%~~siv
%
%~~neg

In most cases the class guard could be phrased as a type constraint on a property of
the class instead, if preferred.  Arguably, a symmetric constraint like two
points being different is better expressed as a class guard, rather than
asymmetrically as a constraint on one type: 
%~~gen ^^^ Classes270
% package classes.guards.invariants.glurp2;
% class Pt(x:Int, y:Int){}
%~~vis
\begin{xten}
class Line(a:Pt, b:Pt{a != b}) {}
\end{xten}
%~~siv
%
%~~neg



\label{DepType:TypeInvariant}
\index{class invariant}
\index{invariant!class}
\index{class!invariant}
\label{DepType:ClassGuardDef}



With every defined class, struct,  or interface \xcd"T" we associate a {\em type
invariant} $\mathit{inv}($\xcd"T"$)$, which describes the guarantees on the
properties of values of type \xcd`T`.  

Every value of \xcd`T` satisfies $\mathit{inv}($\xcd"T"$)$ at all times.  This
is somewhat stronger than the concept of type invariant in most languages
(which only requires that the invariant holds when no method calls are
active).  X10 invariants only concern properties, which are immutable; thus,
once established, they cannot be falsified.

The type
invariant associated with \xcd"x10.lang.Any"
is 
\xcd"true".

The type invariant associated with any interface or struct \xcd"I" that extends
interfaces \xcdmath"I$_1$, $\dots$, I$_k$" and defines properties
\xcdmath"x$_1$: P$_1$, $\dots$, x$_n$: P$_n$" and
specifies a guard \xcd"c" is given by:

\begin{xtenmath}
$\mathit{inv}$(I$_1$) && $\dots$ && $\mathit{inv}$(I$_k$) 
    && self.x$_1$ instanceof P$_1$ &&  $\dots$ &&  self.x$_n$ instanceof P$_n$ 
    && c  
\end{xtenmath}

Similarly the type invariant associated with any class \xcd"C" that
implements interfaces \xcdmath"I$_1$, $\dots$, I$_k$",
extends class \xcd"D" and defines properties
\xcdmath"x$_1$: P$_1$, $\dots$, x$_n$: P$_n$" and
specifies a guard \xcd"c" is
given by the same thing with the invariant of the superclass \xcd`D` conjoined:
\begin{xtenmath}
$\mathit{inv}$(I$_1$) && $\dots$ && $\mathit{inv}$(I$_k$) 
    && self.x$_1$ instanceof P$_1$ &&  $\dots$ &&  self.x$_n$ instanceof P$_n$ 
    && c  
    && $\mathit{inv}$(D)
\end{xtenmath}


Note that the type invariant associated with a class entails the type
invariants of each interface that it implements (directly or indirectly), and
the type invariant of each ancestor class.
It is guaranteed that for any variable \xcd"v" of
type \xcd"T{c}" (where \xcd"T" is an interface name or a class name) the only
objects \xcd"o" that may be stored in \xcd"v" are such that \xcd"o" satisfies
$\mathit{inv}(\mbox{\xcd"T"}[\mbox{\xcd"o"}/\mbox{\xcd"this"}])
\wedge \mbox{\xcd"c"}[\mbox{\xcd"o"}/\mbox{\xcd"self"}]$.



\subsection{Invariants for {\tt implements} and {\tt extends} clauses}\label{DepType:Implements}
\label{DepType:Extends}
\index{type-checking!implements clause}
\index{type-checking!extends clause}
\index{implements}
\index{extends}
Consider a class definition
\begin{xtenmath}
$\mbox{\emph{ClassModifiers}}^{\mbox{?}}$
class C(x$_1$: P$_1$, $\dots$, x$_n$: P$_n$) extends D{d}
   implements I$_1${c$_1$}, $\dots$, I$_k${c$_k$}
$\mbox{\emph{ClassBody}}$
\end{xtenmath}

Each of the following static semantics rules must be satisfied:

\noo{reformat this}

The type invariant \xcdmath"$\mathit{inv}$(C)" of \xcd"C" must entail
\xcdmath"c$_i$[this/self]" for each $i$ in $\{1, \dots, k\}$



The return type \xcd"c" of each constructor in a class \xcd`C`
must entail the invariant \xcdmath"$\mathit{inv}$(C)".


\subsection{Invariants and constructor definitions}
\index{invariant!and constructor}
\index{constructor!and invariant}

A constructor for a class \xcd"C" is guaranteed to return an object of the
class on successful termination. This object must satisfy  \xcdmath"$\mathit{inv}$(C)", the
class invariant associated with \xcd"C" (\Sref{DepType:TypeInvariant}).
However,
often the objects returned by a constructor may satisfy {\em stronger}
properties than the class invariant. \Xten{}'s dependent type system
permits these extra properties to be asserted with the constructor in
the form of a constrained type (the ``return type'' of the constructor):

%##(CtorDecl
\begin{bbgrammar}
%(FROM #(prod:CtorDecl)#)
            CtorDecl \: Mods\opt \xcd"def" \xcd"this" TypeParams\opt FormalParams WhereClause\opt HasResultType\opt  CtorBody & (\ref{prod:CtorDecl}) \\
\end{bbgrammar}
%##)

\label{ConstructorGuard}

The parameter list for the constructor
may specify a \emph{guard} that is to be satisfied by the parameters
to the list.

\begin{ex}
%%TODO--rewrite this
Here is another example, constructed as a simplified 
version of \Xcd{x10.array.Region}.  The \xcd`mockUnion` method 
has the type, though not the value, that a true \xcd`union` method would have.

%~~gen ^^^ Classes280
%package Classes.SimplifiedRegion;
%~~vis
\begin{xten}
class MyRegion(rank:Int) {
  static type MyRegion(n:Int)=MyRegion{rank==n};
  def this(r:Int):MyRegion(r) {
    property(r);
  }
  def this(diag:Array[Int](1)):MyRegion(diag.size){ 
    property(diag.size);
  }
  def mockUnion(r:MyRegion(rank)):MyRegion(rank) = this;
  def example() {
    val R1 : MyRegion(3) = new MyRegion([4,4,4]); 
    val R2 : MyRegion(3) = new MyRegion([5,4,1]); 
    val R3 = R1.mockUnion(R2); // inferred type MyRegion(3)
  }
}
\end{xten}
%~~siv
%
%~~neg
The first constructor returns the empty region of rank \Xcd{r}.  The
second constructor takes a \Xcd{Array[Int](1)} of arbitrary length
\Xcd{n} and returns a \Xcd{MyRegion(n)} (intended to represent the set
of points in the rectangular parallelopiped between the origin and the
\Xcd{diag}.)

The code in \xcd`example` typechecks, and \xcd`R3`'s type is inferred as
\xcd`MyRegion(3)`.  


\end{ex}

   Let \xcd"C" be a class with properties
   \xcdmath"p$_1$: P$_1$, $\dots$, p$_n$: P$_n$", and invariant \xcd"c"
   extending the constrained type \xcd"D{d}" (where \xcd"D" is the name of a
   class).



   For every constructor in \xcd"C" the compiler checks that the call to
   super invokes a constructor for \xcd"D" whose return type is strong enough
   to entail \xcd"d". Specifically, if the call to super is of the form 
     \xcdmath"super(e$_1$, $\dots$, e$_k$)"
   and the static type of each expression \xcdmath"e$_i$" is
   \xcdmath"S$_i$", and the invocation
   is statically resolved to a constructor
\xcdmath"def this(x$_1$: T$_1$, $\dots$, x$_k$: T$_k$){c}: D{d$_1$}"
   then it must be the case that 
\begin{xtenmath}
x$_1$: S$_1$, $\dots$, x$_i$: S$_i$ entails x$_i$: T$_i$  (for $i \in \{1, \dots, k\}$)
x$_1$: S$_1$, $\dots$, x$_k$: S$_k$ entails c  
d$_1$[a/self], x$_1$: S$_1$, ..., x$_k$: S$_k$ entails d[a/self]      
\end{xtenmath}
\noindent where \xcd"a" is a constant that does not appear in 
\xcdmath"x$_1$: S$_1$ $\wedge$ ... $\wedge$ x$_k$: S$_k$".

   The compiler checks that every constructor for \xcd"C" ensures that
   the properties \xcdmath"p$_1$,..., p$_n$" are initialized with values which satisfy
   $\mathit{inv}($\xcd"T"$)$, and its own return type \xcd"c'" as follows.  In each constructor, the
   compiler checks that the static types \xcdmath"T$_i$" of the expressions \xcdmath"e$_i$"
   assigned to \xcdmath"p$_i$" are such that the following is
   true:
\begin{xtenmath}
p$_1$: T$_1$, $\dots$, p$_n$: T$_n$ entails $\mathit{inv}($T$)$ $\wedge$ c'     
\end{xtenmath}

(Note that for the assignment of \xcdmath"e$_i$" to \xcdmath"p$_i$"
to be type-correct it must be the
    case that \xcdmath"p$_i$: T$_i$ $\wedge$ p$_i$: P$_i$".) 



The compiler must check that every invocation \xcdmath"C(e$_1$, $\dots$, e$_n$)" to a
constructor is type correct: each argument \xcdmath"e$_i$" must have a static type
that is a subtype of the declared type \xcdmath"T$_i$" for the $i$th
argument of the
constructor, and the conjunction of static types of the argument must
entail the constraint in the parameter list of the constructor.



\input{ObjectInitialization.tex}

\input{MethodResolution.tex}

\input{InnerClasses.tex}

%% vj Thu Sep 19 21:34:13 EDT 2013
% updated for v2.4 -- no change.
\chapter{Structs}
\label{XtenStructs}
\label{StructClasses}
\label{Structs}
\index{struct}


X10 objects are a powerful general-purpose programming tool. However, the
power must be paid for in space and time. In space, a typical object
implementation requires some extra memory for run-time class information, as
well as a pointer for each reference to the object. In time, a typical object
requires an extra indirection to read or write data, and some run-time
computation to figure out which method body to call.

For high-performance computing, this overhead may not be acceptable for all
objects. X10 provides structs, which are stripped-down objects. They are less
powerful than objects; in particular they lack inheritance and mutable fields.
Without inheritance, method calls do not need to do any lookup; they can be
implemented directly. Accordingly, structs can be implemented and used more
cheaply than objects, potentially avoiding the space and time overhead.
(Currently, the C++ back end avoids the overhead, but the Java back end
implements structs as Java objects and does not avoid it.)



Structs and classes are interoperable. Both can implement interfaces; in
particular, like all X10 values they implement \xcd`Any`.  Subroutines 
whose arguments are defined by interfaces can take both structs and classes.
(Some caution is necessary here: referring to a struct through an interface
requires overhead similar to that required for an object.)



In many cases structs can be converted to classes or classes to structs,
within the constraints of structs. If you start off defining a struct and
decide you need a class instead, the code change required is simply changing
the keyword \xcd`struct` to \xcd`class`. If you have a class that does not use
inheritance or mutable fields, it can be converted to a struct by changing its
keyword. Client code using the struct that was a class will need certain
changes: \eg, the \xcd`new` keyword must be added in constructor calls, and
structs (unlike classes) cannot be \xcd`null`.    





\section{Struct declaration}
\index{struct!declaration}

%##(StructDecln TypeParamsI Properties Guard Interfaces ClassBody
\begin{bbgrammar}
%(FROM #(prod:StructDecln)#)
         StructDecln \: Mods\opt \xcd"struct" Id TypeParamsI\opt Properties\opt Guard\opt Interfaces\opt ClassBody & (\ref{prod:StructDecln}) \\
%(FROM #(prod:TypeParamsI)#)
         TypeParamsI \: \xcd"[" TypeParamIList \xcd"]" & (\ref{prod:TypeParamsI}) \\
%(FROM #(prod:Properties)#)
          Properties \: \xcd"(" PropList \xcd")" & (\ref{prod:Properties}) \\
%(FROM #(prod:Guard)#)
               Guard \: DepParams & (\ref{prod:Guard}) \\
%(FROM #(prod:Interfaces)#)
          Interfaces \: \xcd"implements" InterfaceTypeList & (\ref{prod:Interfaces}) \\
%(FROM #(prod:ClassBody)#)
           ClassBody \: \xcd"{" ClassMemberDeclns\opt \xcd"}" & (\ref{prod:ClassBody}) \\
\end{bbgrammar}
%##)



All fields of a struct must be \xcd`val`.

A struct \Xcd{S} cannot contain a field of type \Xcd{S}, or a field of struct
type \Xcd{T} which, recursively, contains a field of type \Xcd{S}.  This
restriction is necessary to permit \xcd`S` to be implemented as a contiguous
block of memory of size equal to the sum of the sizes of its fields.  


Values of a struct \Xcd{C} type can be created by invoking a constructor
defined in \Xcd{C}.  Unlike for classes, the  \Xcd{new} keyword is optional
for struct constructors.  

\begin{ex}
Leaving out \xcd`new` can improve readability in some cases: 
%~~gen ^^^ Structs10
% package Structs.For.Muckts;
%~~vis
\begin{xten}
struct Polar(r:Double, theta:Double){
  def this(r:Double, theta:Double) {property(r,theta);}
  static val Origin = Polar(0,0);
  static val x0y1   = Polar(1, 3.14159/2);
  static val x1y0   = new Polar(1, 0); 
}
\end{xten}
%~~siv
%
%~~neg


When a struct and a method have the same name (often in violation of the X10
capitalization convention), 
\xcd`new` may be used to resolve to the struct's constructor.  
%~~gen ^^^ Structs2w3o
% package Structs2w3o;
%~~vis
\begin{xten}
struct Ambig(x:Long) {
  static def Ambig(x:Long) = "ambiguity please";
  static def example() { 
    val useMethod      = Ambig(1);
    val useConstructor = new Ambig(2);
  }
}
\end{xten}
%~~siv
%
%~~neg

\end{ex}

Structs support the same notions of generics, properties, and constrained
types that classes do.  

\begin{ex}

%~~gen ^^^ Structs6i5t
% package Structs6i5t;
%~~vis
\begin{xten}
struct Exam[T](nQuestions:Long){T <: Question} {
  public static interface Question {}
  // ... 
}
\end{xten}
%~~siv
%
%~~neg


\end{ex}

%%NOW_GONE%% \begin{ex}The \xcd`Pair` type below provides pairs
%%NOW_GONE%% of values; the \xcd`diag()` method applies only when the two elements of the
%%NOW_GONE%% pair are equal, and returns that common value: 
%%NOW_GONE%% %~x~gen ^^^ Structs20
%%NOW_GONE%% % package Structs20;
%%NOW_GONE%% %~x~vis
%%NOW_GONE%% \begin{xten}
%%NOW_GONE%% struct Pair[T,U](t:T, u:U) {
%%NOW_GONE%%   def this(t:T, u:U) { property(t,u); }
%%NOW_GONE%%   def diag(){T==U && t==u} = t;
%%NOW_GONE%% }
%%NOW_GONE%% \end{xten}
%%NOW_GONE%% %~x~siv
%%NOW_GONE%% % class Hook{ def run() {
%%NOW_GONE%% %   val p = Pair(3,3); 
%%NOW_GONE%% %   return p.diag() == 3;
%%NOW_GONE%% % }}
%%NOW_GONE%% %~x~neg
%%NOW_GONE%% \end{ex}

\section{Boxing of structs}
\index{auto-boxing!struct to interface}
\index{struct!auto-boxing}
\index{struct!casting to interface}
\label{auto-boxing} 
If a struct \Xcd{S} implements an interface \Xcd{I} (\eg, \Xcd{Any}),
a value \Xcd{v} of type \Xcd{S} can be assigned to a variable of type
\Xcd{I}. The implementation creates an object \Xcd{o} that is an
instance of an anonymous class implementing \Xcd{I} and containing
\Xcd{v}.  The result of invoking a method of \Xcd{I} on \Xcd{o} is the
same as invoking it on \Xcd{v}. This operation is termed {\em auto-boxing}.
It allows full interoperability of structs and objects---at the cost of losing
the extra efficiency of the structs when they are boxed.

In a generic class or struct obtained by instantiating a type parameter
\Xcd{T} with a struct \Xcd{S}, variables declared at type \Xcd{T} in the body
of the class are not boxed. They are implemented as if they were declared at
type \Xcd{S}.

\begin{ex}
The rail \xcd`aa` in the following example is a \xcd`Rail[Any]`.  It
initially holds two objects.  Then, its elements are replaced by two structs,
both of which are auto-boxed.  Note that no fussing is required to put an
integer into a \xcd`Rail[Any]`.  
However, a rail of structs, such as \xcd`ah`, holds {\em unboxed} structs
and does not incur boxing overhead.
%~~gen ^^^ Structs3q6l
% package Structs3q6l;
%~~vis
\begin{xten}
struct Horse(x:Long){
  static def example(){
    val aa : Rail[Any] = ["a String" as Any, "another one"];
    aa(0) = Horse(8);
    aa(1) = 13;
    val ah : Rail[Horse] = [Horse(7), Horse(13)];
  }
}
\end{xten}
%~~siv
%
%~~neg


\end{ex}

\section{Optional Implementation of {\tt Any} methods}
\label{StructAnyMethods}
\index{Any!structs}

Two
structs are equal (\Xcd{==}) if and only if their corresponding fields
are equal (\Xcd{==}). 

All structs implement \Xcd{x10.lang.Any}. 
Structs are required to implement the following methods from \xcd`Any`.  
Programmers need not provide them; X10 will produce them automatically if 
the program does not include them. 
\begin{xten}
  public def equals(Any):Boolean;
  public def hashCode():Int;
  public def typeName():String;
  public def toString():String;  
\end{xten}


A programmer who provides an explicit implementation
of \Xcd{equals(Any)} for a struct \Xcd{S} should also consider
supplying a definition for \Xcd{equals(S):Boolean}. This will often
yield better performance since the cost of an upcast to \Xcd{Any} and
then a downcast to \Xcd{S} can be avoided.

\section{Primitive Types}
\index{types!primitive}
\index{primitive types}
\index{Int}
\index{UInt}
\index{Long}
\index{ULong}
\index{Char}
\index{Byte}
\index{UByte}
\index{Boolean}
\index{Short}
\index{UShort}
\index{Float}
\index{Double}

Certain types that might be built in to other languages are in fact
implemented as structs in package \xcd`x10.lang` in X10. Their methods and
operations are often provided with \xcd`@Native` (\Sref{NativeCode}) rather
than X10 code, however. These types are:
\begin{xten}
Boolean, Char, Byte, Short, Int, Long
Float, Double, UByte, UShort, UInt, ULong
\end{xten}

\subsection{Signed and Unsigned Integers}
\index{types!unsigned}
\index{integers!unsigned}
\index{unsigned}

X10 has an unsigned integer type corresponding to each integer type:
\xcd`UInt` is an unsigned \xcd`Int`, and so on. These types can be used for
binary programming, or when an extra bit of precision for counters or other
non-negative numbers is needed in integer arithmetic. However, X10 does not
otherwise encourage the use of unsigned arithmetic.




 
%%WRONG%% \section{Generic programming with structs}
%%WRONG%% \index{struct!generic}
%%WRONG%% \index{generic!struct}
%%WRONG%% 
%%WRONG%% 
%%WRONG%% 
%%WRONG%% The programmer must be aware of the different interpretations of
%%WRONG%% equality (\Sref{StableEquality}) for structs and classes and ensure that the
%%WRONG%% code is correctly written for both cases. 
%%WRONG%% 
%%WRONG%% \index{isObject}
%%WRONG%% \index{isStruct}
%%WRONG%% \index{isFunction}
%%WRONG%% Three static methods on \xcd`x10.lang.System` 
%%WRONG%% allow you to tell what kind of value \xcd`x` is: object,
%%WRONG%% struct, or function.  
%%WRONG%% \xcd`System.isObject(x)` returns true if \xcd`x` is a value of \xcd`Object`
%%WRONG%% type, including \xcd`null`; \xcd`System.isStruct(x)` returns true if \xcd`x`
%%WRONG%% is a \xcd`struct`; \xcd`System.isFunction(x)` returns true if \xcd`x` is a
%%WRONG%% closure value.  Precisely one of these three functions returns true for any
%%WRONG%% X10 value \xcd`x`.  
%%WRONG%% 
%%WRONG%% \begin{xten}
%%WRONG%% val x:X = ...;
%%WRONG%% if (System.isObject(x)) { // x is a real object
%%WRONG%%    val x2 = x as Object; // this cast will always succeed.
%%WRONG%%    ...
%%WRONG%% } else if (System.isStruct(x)) { // x is a struct
%%WRONG%%    ...
%%WRONG%% } else {  
%%WRONG%%   assert System.isFunction(x);
%%WRONG%% }
%%WRONG%% \end{xten}
%%WRONG%%  
  
\section{Example structs}

\xcd`x10.lang.Complex` provides a detailed example of a practical struct,
suitable for use in a library.  For a shorter example, we define the
\xcd`Pair` struct.   A \xcd`Pair` packages
two values of possibly unrelated type together in a single value, \eg, to
return two values from a function.  

\xcd`divmod` computes the quotient and remainder of \xcdmath"a $\div$ b" (naively).
It returns both, packaged as a \xcd`Pair[UInt, UInt]`.  Note that the
constructor uses type inference, and that the quotient and remainder are
accessed through the \xcd`first` and \xcd`second` fields.
%~~gen ^^^ Structs30
% package Structs30Pair;
%~~vis
\begin{xten}
struct Pair[T,U] {
    public val first:T;
    public val second:U;
    public def this(first:T, second:U):Pair[T,U] {
        this.first = first;
        this.second = second;
    }
    public def toString() 
        = "(" + first + ", " + second + ")";
}
class Example {
  static def divmod(var a:UInt, b:UInt): Pair[UInt, UInt] {
     assert b > 0u;
     var q : UInt = 0un;
     while (a > b) {q += 1un; a -= b;}
     return Pair(q, a); 
  }
  static def example() {
     val qr = divmod(22un, 7un);
     assert qr.first == 3un && qr.second == 1un;
  }
}
\end{xten}
%~~siv
%class Hook{ def run() { Example.example(); return true; } } 
%~~neg

\section{Nested Structs}
\index{struct!static nested}
\index{static nested struct}

Static nested structs may be defined, essentially as static nested classes
except for making them structs
(\Sref{StaticNestedClasses}).  Inner structs may be defined, essentially as
inner classes except making them structs (\Sref{InnerClasses}).
\limitationx{} Nested structs must be currently be declared static.

\section{Default Values of Structs}
\label{sect:DefaultValuesOfStructs}


If all fields of a struct have default values, then the struct has a
default value, \viz, the struct whose fields are all set to their
default values.  If some field does not have a default value, neither
does the struct.

\begin{ex}

In the following code, the \xcd`Example` struct has a default value whose
\xcd`i` field is \xcd`0`.  If an \xcd`Example` is ever constructed by the
constructor, its \xcd`i` field will be \xcd`1`.  This program does a slightly
subtle dance to get ahold of a default \xcd`Example`, by having an instance
\xcd`var` (which, unlike most kinds of variables, does not need to get
initialized before use (though that exemption only applies if its type has a
default value)).   As the \xcd`assert` confirms, the default \xcd`Example`
does indeed have an \xcd`i` field of \xcd`0`.

%~~gen ^^^ Structs6r3w
% package Structs6r3w;
% 
%~~vis
\begin{xten}
class StructDefault {
  static struct Example {
    val i : Long;
    def this() { i = 1; }
  }
  var ex : Example; 
  static def example() {
     val ex = (new StructDefault()).ex;
     assert ex.i == 0;
  }
\end{xten}
%~~siv
% }
%  class Hook { def run() { StructDefault.example(); return true; } } 
%~~neg


\end{ex}


\section{Converting Between Classes And Structs}

Code written using structs can be modified to use classes, or vice versa.
Caution must be used in certain places. 

Class and struct {\em definitions} are syntactically nearly identical:
change the \xcd`class` keyword to \xcd`struct` or vice versa.  Of course,
certain important class features can't be used with structs, such as
inheritance and \xcd`var` fields. 

Converting code that {\em uses} the class or struct requires a certain amount
of caution.
Suppose, in particular, that we want to convert the class \xcd`Class2Struct`
to a struct, and \xcd`Struct2Class` to a class.
%~~gen ^^^ Structs40
%package Structs.Converting;
%~~vis
\begin{xten}
class Class2Struct {
  val a : Long;
  def this(a:Long) { this.a = a; }
  def m() = a;
}
struct Struct2Class { 
  val a : Long;
  def this(a:Long) { this.a = a; }
  def m() = a;
}
\end{xten}
%~~siv
%
%~~neg

\begin{enumerate}

\item Class constructors require the \xcd`new` keyword; struct constructors
      allow  it but do not require it.  
      \xcd`Struct2Class(3)` to will need to be converted to 
      \xcd`new Struct2Class(3)`.

\item Objects and structs have different notions of \xcd`==`.  
      For objects, \xcd`==` means ``same object''; for structs, it means
      ``same contents''. Before conversion, both \xcd`assert`s in the
      following program succeed.  After converting and fixing constructors,
      both of them fail.
%~~gen ^^^ Structs50
%package Structs.Converting.And.Conniving;
% class Class2Struct {
%   val a : Long;
%   def this(a:Long) { this.a = a; }
%   def m() = a;
% }
% struct Struct2Class { 
%   val a : Long;
%   def this(a:Long) { this.a = a; }
%   def m() = a;
% }
%class Example {
% static def Examplle() {
%~~vis
\begin{xten}
val a = new Class2Struct(2);
val b = new Class2Struct(2);
assert a != b;
val c = Struct2Class(3);
val d = Struct2Class(3);
assert c==d;
\end{xten}
%~~siv
%}}
%~~neg

\item Objects can be set to \xcd`null`.  Structs cannot.  

\item The rules for default values are quite different.  
The default value of an object type (if it exists) is \xcd`null`, which behaves quite
differently from an ordinary object of that type; \eg, you cannot call methods
on \xcd`null`, whereas you can on an ordinary object. The default value for
a struct type (if it exists) is a struct like any other of its type, and you
can call methods on it as for any other.


\end{enumerate}



\chapter{Functions}
\label{Functions}
\label{functions}
\index{functions}
\label{Closures}

\section{Overview}
Functions, the last of the three kinds of values in X10, encapsulate pieces of
code which can be applied to a vector of arguments to produce a value.
Functions, when applied, can do nearly anything that any other code could do:
fail to terminate, throw an exception, modify variables, spawn activities,
execute in several places, and so on. X10 functions are not mathematical
functions: the \xcd`f(1)` may return \xcd`true` on one call and \xcd`false` on
an immediately following call.

It is a limitation of \XtenCurrVer{} that functions do not support
type arguments. This limitation may be removed in future versions of
the language.

A \emph{function literal} \xcd"(x1:T1,..,xn:Tn){c}:T=>e" creates a function of
type\\ \xcd"(x1:T1,...,xn:Tn){c}=>T" (\Sref{FunctionType}).  For example, 
\xcd`(x:Int) => x*x` is a function literal describing the squaring function on
integers.   
\xcd`null` is also a function value.

\limitationx{} Function literals do not currently support guards. 

Function application is written \xcd`f(a,b,c)`, following common mathematical
usage. 
\index{Exception!unchecked}


The function body may be a block.  To compute integer squares by repeated
addition (inefficiently), one may write: 
%~~gen
% package Functions.Are.For.Spunctions;
% class Examplllll {
% static 
%~~vis
\begin{xten}
val sq: (Int) => Int 
      = (n:Int) => {
           var s : Int = 0;
           val abs_n = n < 0 ? -n : n;
           for ([i] in 1..abs_n) s += abs_n;
           s
        };
\end{xten}
%~~siv
%}
%~~neg




A function literal evaluates to a function entity {$\phi$}. When {$\phi$} is
applied to a suitable list of actual parameters \xcd`a1`-\xcd`an`, it
evaluates \xcd`e` with the formal parameters bound to the actual parameters.
So, the following are equivalent, where \xcd`e` is an expression involving
\xcd`x1` and \xcd`x2`\footnote{Strictly, there are a few other requirements;
  \eg, \xcd`result` must be a \xcd`var` of type \xcd`T` defined outside the
  outer block, the variables \xcd`a1` and \xcd`a2` had better not appear in
  \xcd`e`, and everything in sight had better typecheck properly.}

%~~gen
% package functions2.why.is.there.a.two;
% abstract class FunctionsTooManyFlippingFunctions[T, T1, T2]{
% abstract def arg1():T1;
% abstract def arg2():T2;
% def thing1(e:T) {var result:T;
%~~vis
\begin{xten}
{
  val f = (x1:T1,x2:T2){true}:T => e;
  val a1 : T1 = arg1();
  val a2 : T2 = arg2();
  result = f(a1,a2);
}
\end{xten}
%~~siv
%}}
%~~neg
and 
%~~gen
% package functions2.why.is.there.a.two.but.here.is.the.other.one;
% abstract class FunctionsTooManyFlippingFunctions[T, T1, T2]{
% abstract def arg1():T1;
% abstract def arg2():T2;
% def thing1(e:T) {var result:T;
%~~vis
\begin{xten}
{
  val a1 : T1 = arg1();
  val a2 : T2 = arg2();
  {
     val x1 : T1 = a1;
     val x2 : T2 = a2;
     result = e;
  }  
}
\end{xten}
%~~siv
%}}
%~~neg
\noindent
This doesn't quite work if the body is a statement rather than an expression.
A few language features are forbidden (\xcd`break` or \xcd`continue` of a loop
that surrounds the function literal) or mean something different (\xcd`return`
inside a function returns from the function). 





The \emph{method selector expression} \Xcd{e.m.(x1:T1,...,xn:Tn)} (\Sref{MethodSelectors})
permits the specification of the function underlying
the method \Xcd{m}, which takes arguments of type \Xcd{(x1:T1,..., xn:Tn)}.
Within this function, \Xcd{this} is bound to the result of evaluating \Xcd{e}.

Function types may be used in \Xcd{implements} clauses of class
definitions. Instances of such classes may be used as functions of the
given type.  Indeed, an object may behave like any (fixed) number of
functions, since the class it is an instance of may implement any
(fixed) number of function types.

%\section{Implementation Notes}
%\begin{itemize}
%
%\item Note that e.m.(T1,...,Tn) will evaluate e to create a
%  function. This function will be applied later to given
%  arguments. Thus this syntax can be used to evaluate the receiver of
%  a method call ahead of the actual invocation. The resulting function
%  can be used multiple times, of course.
%\end{itemize}


\section{Function Literals}
\index{literal!function}
\label{FunctionLiteral}

\Xten{} provides first-class, typed functions, including
\emph{closures}, \emph{operator functions}, and \emph{method
  selectors}.

\begin{grammar}
ClosureExpression \:
        \xcd"("
        Formals\opt
        \xcd")"
\\ &&
        Guard\opt
        ReturnType\opt
        \xcd"=>" ClosureBody \\
ClosureBody \:
        Expression \\
        \| \xcd"{" Statement\star \xcd"}" \\
        \| \xcd"{" Statement\star Expression \xcd"}" \\
\end{grammar}

Functions have zero or more formal parameters and an optional return type.
The body has the 
same syntax as a method body; it may be either an expression, a block
of statements, or a block terminated by an expression to return. In
particular, a value may be returned from the body of the function
using a return statement (\Sref{ReturnStatement}). 

The type of a
function is a function type (\Sref{FunctionType}).  In some cases the
return type \Xcd{T} is also optional and defaults to the type of the
body. If a formal \Xcd{xi} does not occur in any
\Xcd{Tj}, \Xcd{c}, \Xcd{T} or \Xcd{e}, the declaration \Xcd{xi:Ti} may
be replaced by just \Xcd{Ti}: \xcd`(Int)=>7` is the integer function returning
7 for all inputs.

\label{ClosureGuard}

As with methods, a function may declare a guard to
constrain the actual parameters with which it may be invoked.
The guard may refer to the type parameters, formal parameters,
and any \xcd`val`s in scope at the function expression.

The body of the function is evaluated when the function is
invoked by a call expression (\Sref{Call}), not at the function's
place in the program text.

As with methods, a function with return type \xcd"void" cannot
have a terminating expression. 
If the return type is omitted, it is inferred, as described in
\Sref{TypeInference}.
It is a static error if the return type cannot be inferred.  \Eg,
\xcd`(Int)=>null` is not well-defined; X10 does not know which type of
\xcd`null` is intended.  
%~~exp~~`~~`~~ ~~
But \xcd`(Int):Array[Double](1) => null` is legal.


\begin{example}
The following method takes a function parameter and uses it to
test each element of the list, returning the first matching
element.  It returns \xcd`absent` if no element matches.

%~~gen
% package functions2.oh.no;
% import x10.util.*;
% class Finder {
% static 
%~~vis
\begin{xten}

def find[T](f: (T) => Boolean, xs: List[T], absent:T): T = {
  for (x: T in xs)
    if (f(x)) return x;
  absent
  }
\end{xten}
%~~siv
% }
%~~neg

The method may be invoked thus:
%~~gen
% package functions2.oh.no.my.ears;
% import x10.util.*;
% class Finderator {
% static def find[T](f: (T) => Boolean, xs: x10.util.List[T], absent:T): T = {
%  for (x: T in xs)
%    if (f(x)) return x;
%  absent
%}
% static def checkery() {
%~~vis
\begin{xten}
xs: List[Int] = new ArrayList[Int]();
x: Int = find((x: Int) => x>0, xs, 0);
\end{xten}
%~~siv
%}}
%~~neg

\end{example}



\subsection{Outer variable access}

In a function
\xcdmath"(x$_1$: T$_1$, $\dots$, x$_n$: T$_n$){c} => { s }"
the types \xcdmath"T$_i$", the guard \xcd"c" and the body \xcd"s"
may access many, though not all, sorts of variables from outer scopes.  
Specifically, they can access: 
\begin{itemize}
\item All fields of the enclosing object and class;
\item All type parameters;
\item All \xcd`val` variables;
\end{itemize}
\noindent
\xcd`var` variables cannot be accessed.


The function body may refer to instances of enclosing classes using
the syntax \xcd"C.this", where \xcd"C" is the name of the
enclosing class.  \xcd`this` refers to the instance of the immediately
enclosing class, as usual.

For example, the following is legal.  
However, the commented-out line would not be legal.
Note that \xcd`a` is not a local \xcd`var` variable. It is a field of
\xcd`this`. A reference to \xcd`a` is simply short for \xcd`this.a`, which is
a use of a \xcd`val` variable (\xcd`this`).  
%~~gen
% package Functions.areLikeGrunctions.fromConjunctionJunctions;
%~~vis
\begin{xten}
class Lambda {
   var a : Int = 0;
   val b = 0;
   def m(var c : Int, val d : Int) {
      var e : Int = 0;
      val f : Int = 0;
      val closure = (var i: Int, val j: Int) => {
    	  return a + b + d + f + j + this.a + Lambda.this.a;
          // ILLEGAL: return c + e + i;
      };
      return closure;
   }
}
\end{xten}
%~~siv
%
%~~neg

%%SHARED%% 
%%SHARED%% 
%%SHARED%% Access to variables is not automatically atomic.  As
%%SHARED%% with any code that might mutate shared data concurrently, be sure to protect
%%SHARED%% references to mutable shared state with \xcd`atomic`. For example, the
%%SHARED%% following code returns a pair of closures which operate on the same shared
%%SHARED%% variable \xcd`a`, which are concurrency-safe---even if invoked many times
%%SHARED%% simultaneously. Without \xcd`atomic`, it would no longer be concurrency-safe.
%%SHARED%% 
%%SHARED%% 
%%SHARED%% %~s~gen
%%SHARED%% % package Functions2.Are.All.Too.Much;
%%SHARED%% % class Fun2Frivols {
%%SHARED%% %~s~vis
%%SHARED%% \begin{xten}
%%SHARED%%   def counters() {
%%SHARED%%       var a : Int = 0;
%%SHARED%%        return [
%%SHARED%%           () => {atomic a ++;},
%%SHARED%%           () => {atomic return a;}
%%SHARED%%           ];
%%SHARED%%    }
%%SHARED%% \end{xten}
%%SHARED%% %~s~siv
%%SHARED%% %}
%%SHARED%% %
%%SHARED%% %~s~neg


%SHARED% \begin{note}
%SHARED% The main activity may run in parallel with any
%SHARED% functions it creates. Hence even the read of an outer variable by the
%SHARED% body of a function may result in a race condition. Since functions are
%SHARED% first-class, the analysis of whether a function may execute in parallel
%SHARED% with the activity that created it may be difficult.
%SHARED% \end{note}

%% vj: This should be verified.
%\begin{note}
%The rule for accessing outer variables from function bodies
%should be the same as the rule for accessing outer variables from local
%or anonymous classes.
%\end{note}

\section{Method selectors}
\label{MethodSelectors}
\index{function!method selector}
\index{method!underlying function}

A method selector expression allows a method to be used as a
first-class function, without writing a function expression for it.
For example, consider a class \xcd`Span` defining ranges of integers.  

%~~gen
% package Functions2.Span;
%~~vis
\begin{xten}
class Span(low:Int, high:Int) {
  def this(low:Int, high:Int) {property(low,high);}
  def between(n:Int) = low <= n && n <= high;
  def example() {
    val digit = new Span(0,9);
    val isDigit : (Int) => Boolean = digit.between.(Int);
    if (isDigit(8)) Console.OUT.println("8 is!");
  }
}
\end{xten}
%~~siv
%
%~~neg
\noindent


In \xcd`example()`, 
%~~exp~~`~~`~~ digit:Span~~class Span(low:Int, high:Int) {def this(low:Int, high:Int) {property(low,high);} def between(n:Int) = low <= n && n <= high;}
\xcd`digit.between.(Int)` 
is a unary function testing whether its argument is between zero
and nine.  It could also be written 
%~~exp~~`~~`~~ digit:Span~~class Span(low:Int, high:Int) {def this(low:Int, high:Int) {property(low,high);} def between(n:Int) = low <= n && n <= high;}
\xcd`(n:Int) => digit.between(n)`.

%%GRAMMAR%% This is formalized thus:
%%GRAMMAR%% 
%%GRAMMAR%% \begin{grammar}
%%GRAMMAR%% MethodSelector \:
%%GRAMMAR%%         Primary \xcd"."
%%GRAMMAR%%         MethodName \xcd"."
%%GRAMMAR%%                 TypeParameters\opt \xcd"(" Formals\opt \xcd")" \\
%%GRAMMAR%%       \|
%%GRAMMAR%%         TypeName \xcd"."
%%GRAMMAR%%         MethodName \xcd"."
%%GRAMMAR%%                 TypeParameters\opt \xcd"(" Formals\opt \xcd")" \\
%%GRAMMAR%% \end{grammar}

The \emph{method selector expression} \Xcd{e.m.(T1,...,Tn)} is type
correct only if  the static type of \Xcd{e} is a
class or struct or interface \xcd`V` with a method
\Xcd{m(x1:T1,...xn:Tn)\{c\}:T} defined on it (for some
\Xcd{x1,...,xn,c,T)}. At runtime the evaluation of this expression
evaluates \Xcd{e} to a value \Xcd{v} and creates a function \Xcd{f}
which, when applied to an argument list \Xcd{(a1,...,an)} (of the right
type) yields the value obtained by evaluating \Xcd{v.m(a1,...,an)}.

Thus, the method selector

\begin{xtenmath}
e.m.(T$_1$, $\dots$, T$_n$)
\end{xtenmath}
\noindent behaves as if it were the function
\begin{xtenmath}
((v:V)=>
  (x$_1$: T$_1$, $\dots$, x$_n$: T$_n$){c} 
  => v.m(x$_1$, $\dots$, x$_n$))
(e)
\end{xtenmath}



Because of overloading, a method name is not sufficient to
uniquely identify a function for a given class.
One needs the argument type information as well.
The selector syntax (dot) is used to distinguish \xcd"e.m()" (a
method invocation on \xcd"e" of method named \xcd"m" with no arguments)
from \xcd"e.m.()"
(the function bound to the method). 

A static method provides a binding from a name to a function that is
independent of any instance of a class; rather it is associated with the
class itself. The static function selector
\xcdmath"T.m.(T$_1$, $\dots$, T$_n$)" denotes the
function bound to the static method named \xcd"m", with argument types
\xcdmath"(T$_1$, $\dots$, T$_n$)" for the type \xcd"T". The return type
of the function is specified by the declaration of \xcd"T.m".

There is no difference between using a function defined directly 
directly using the function syntax, or obtained via static or
instance function selectors.


\section{Operator functions}
\label{OperatorFunction}
\index{function!operator}
Every binary operator (e.g.,
\xcd"+",
\xcd"-",
\xcd"*",
\xcd"/",
\dots) has a family of functions, one for
each type on which the operator is defined. The function can be
selected using the ``\xcd`.`'' syntax:


\begin{xtenmath}
String.+             $\equiv$ (x: String, y: String): String => x + y
Long.-               $\equiv$ (x: Long, y: Long): Long => x - y
Float.-              $\equiv$ (x: Float, y: Float): Float => x - y
Boolean.&            $\equiv$ (x: Boolean, y: Boolean): Boolean => x & y
Int.<                $\equiv$ (x: Int, y: Int): Boolean => x < y
\end{xtenmath}

%~~gen
% package Functions.Operatorfunctionsgracklegrackle;
% class JustATest {
% val dummy = [String.+,
%  Long.-,
%  Float.-,
%  Boolean.&,
%  Int.<
%  ];
% }
%~~vis
\begin{xten}
\end{xten}
%~~siv
%
%~~neg


%%TODO -- fix commented-out lines!

%~~gen
% package Functions2.For.The.Lose;
% class TypecheckThatSillyExample {
%   def checker() {
%    val l1 : (String, String) => String = String.+;
%    val r1 : (String, String) => String = (x: String, y: String): String => x + y;
%    val l2 : (Long,Long) => Long = Long.-;
%    val r2 : (Long,Long) => Long = (x: Long, y: Long): Long => x - y;
%//var v1 : (Float,Float) => Float = Float.-(Float,Float) ;
%var v2 : (Float,Float) => Float = (x: Float, y: Float): Float => x - y;
%//var v3 : (Int) => Int =  Int.-(Int)     ;      ;
%var v4  : (Int) => Int  =  (x: Int): Int => -x;
%var v5 : (Boolean,Boolean) => Boolean = Boolean.&            ;
%var v6 : (Boolean,Boolean) => Boolean =  (x: Boolean, y: Boolean): Boolean => x & y;
%//var v7 : (Boolean) => Boolean = Boolean.!            ;
%var v8 : (Boolean) => Boolean =  (x: Boolean): Boolean => !x;
%//var v9 : (Int,Int) => Boolean = Int.<(Int,Int)       ;
%var v10: (Int,Int) => Boolean =  (x: Int, y: Int): Boolean => x < y;
%//var v11 : (Dist,Place)=>Dist = Dist.|(Place)        ;
%var v12 : (Dist,Place)=>Dist=  (d: Dist, p: Place): Dist => d | p;
%}
% }
%~~vis
%~~siv
%
%~~neg

Unary and binary promotion (\Sref{XtenPromotions}) is not performed
when invoking these
operations; instead, the operands are coerced individually via implicit
coercions (\Sref{XtenConversions}), as appropriate.


%%WE-NEVER-GOT-TO-IT%%  \begin{planned}
%%WE-NEVER-GOT-TO-IT%%  
%%WE-NEVER-GOT-TO-IT%%  {\bf The following is not implemented in version 2.0.3:}
%%WE-NEVER-GOT-TO-IT%%  
%%WE-NEVER-GOT-TO-IT%%  Additionally, for every expression \xcd"e" of a type \xcd"T" at which a binary
%%WE-NEVER-GOT-TO-IT%%  operator \xcd"OP" is defined, the expression \xcd"e.OP" or
%%WE-NEVER-GOT-TO-IT%%  \xcd"e.OP(T)" represents the function
%%WE-NEVER-GOT-TO-IT%%  defined by:
%%WE-NEVER-GOT-TO-IT%%  
%%WE-NEVER-GOT-TO-IT%%  \begin{xten}
%%WE-NEVER-GOT-TO-IT%%  (x: T): T => { e OP x }
%%WE-NEVER-GOT-TO-IT%%  \end{xten}
%%WE-NEVER-GOT-TO-IT%%  
%%WE-NEVER-GOT-TO-IT%%  \begin{grammar}
%%WE-NEVER-GOT-TO-IT%%  Primary \: Expr \xcd"." Operator \xcd"(" Formals\opt \xcd")" \\
%%WE-NEVER-GOT-TO-IT%%          \| Expr \xcd"." Operator \\
%%WE-NEVER-GOT-TO-IT%%  \end{grammar}
%%WE-NEVER-GOT-TO-IT%%  
%%WE-NEVER-GOT-TO-IT%%  %% For every expression \xcd"e" of a type \xcd"T" at which a unary
%%WE-NEVER-GOT-TO-IT%%  %%operator \xcd"OP" is defined, the expression \xcd"e.OP()"
%%WE-NEVER-GOT-TO-IT%%  %% represents the function defined by:
%%WE-NEVER-GOT-TO-IT%%  
%%WE-NEVER-GOT-TO-IT%%  %% \begin{xten}
%%WE-NEVER-GOT-TO-IT%%  %% (): T => { OP e }
%%WE-NEVER-GOT-TO-IT%%  %% \end{xten}
%%WE-NEVER-GOT-TO-IT%%  
%%WE-NEVER-GOT-TO-IT%%  For example,
%%WE-NEVER-GOT-TO-IT%%  one may write an expression that adds one to each member of a
%%WE-NEVER-GOT-TO-IT%%  list \xcd"xs" by:
%%WE-NEVER-GOT-TO-IT%%  
%%WE-NEVER-GOT-TO-IT%%  %%TODO -- when this topic works, make the example wwork too.
%%WE-NEVER-GOT-TO-IT%%  %~x~gen
%%WE-NEVER-GOT-TO-IT%%  % package Functions2.Wants.A.Dinner.Reservation;
%%WE-NEVER-GOT-TO-IT%%  % import x10.util.*;
%%WE-NEVER-GOT-TO-IT%%  % class Reservation {
%%WE-NEVER-GOT-TO-IT%%  % def smerp() {
%%WE-NEVER-GOT-TO-IT%%  %   val xs = new ArrayList[Int]();
%%WE-NEVER-GOT-TO-IT%%  %~x~vis
%%WE-NEVER-GOT-TO-IT%%  \begin{xten}
%%WE-NEVER-GOT-TO-IT%%  xs.map(1.+);
%%WE-NEVER-GOT-TO-IT%%  \end{xten}
%%WE-NEVER-GOT-TO-IT%%  %~x~siv
%%WE-NEVER-GOT-TO-IT%%  % }
%%WE-NEVER-GOT-TO-IT%%  % }
%%WE-NEVER-GOT-TO-IT%%  %
%%WE-NEVER-GOT-TO-IT%%  %~x~neg
%%WE-NEVER-GOT-TO-IT%%  \end{planned}
%%WE-NEVER-GOT-TO-IT%%  
%%WE-NEVER-GOT-TO-IT%%  
\section{Functions as objects of type \Xcd{Any}}
\label{FunctionAnyMethods}

\label{FunctionEquality}
\index{function!equality} \index{equality!function} Two functions \Xcd{f} and
\Xcd{g} are equal if both were obtained by the same evaluation of a function
literal.\footnote{A literal may occur in program text within a loop, and hence
  may be evaluated multiple times.} Further, it is guaranteed that if two
functions are equal then they refer to the same locations in the environment
and represent the same code, so their executions in an identical situation are
indistinguishable. (Specifically, if \xcd`f == g`, then \xcd`f(1)` can be
substituted for \xcd`g(1)` and the result will be identical. However, there is
no guarantee that \xcd`f(1)==g(1)` will evaluate to true. Indeed, there is no
guarantee that \xcd`f(1)==f(1)` will evaluate to true either, as \xcd`f` might
be a function which returns {$n$} on its {$n^{th}$} invocation. However,
\xcd`f(1)==f(1)` and \xcd`f(1)==g(1)` are interchangeable.)
\index{function!==}


Every function type implements all the methods of \Xcd{Any}.
\xcd`f.equals(g)` is equivalent to \xcd`f==g`.  \xcd`f.hashCode()`, 
\xcd`f.toString()`, and \xcd`f.typeName()` are implementation-dependent, but
respect \xcd`equals` and the basic contracts of \xcd`Any`. 

\index{function!equals}
\index{function!hashCode}
\index{function!toString}
\index{function!typeName}
\index{function!home}
\index{function!at(Place)}
\index{function!at(Object)}



\chapter{Expressions}\label{XtenExpressions}\index{expression}

\Xten{} has a rich expression language.
Evaluating an expression produces a value, or, in a few cases, no value. 
Expression evaluation may have side effects, such as change of the value of a 
\xcd`var` variable or a data structure, allocation of new values, or throwing
an exception. 



\section{Literals}
\index{literal}

Literals denote fixed values of built-in types. 
The syntax for literals is given in \Sref{Literals}. 

The type that \Xten{} gives a literal often includes its value. \Eg, \xcd`1`
is of type \xcd`Int{self==1}`, and \xcd`true` is of type
\xcd`Boolean{self==true}`.

\section{{\tt this}}
\index{this}
\index{\Xcd{this}}

%##(Primary
\begin{bbgrammar}
%(FROM #(prod:Primary)#)
             Primary \: \xcd"here" & (\ref{prod:Primary}) \\
                    \| \xcd"[" ArgumentList\opt \xcd"]" \\
                    \| Literal \\
                    \| \xcd"self" \\
                    \| \xcd"this" \\
                    \| ClassName \xcd"." \xcd"this" \\
                    \| \xcd"(" Exp \xcd")" \\
                    \| ClassInstCreationExp \\
                    \| FieldAccess \\
                    \| MethodInvocation \\
                    \| MethodSelection \\
                    \| OperatorFunction \\
\end{bbgrammar}
%##)

The expression \xcd"this" is a  local \xcd`val` containing a reference
to an instance of the lexically enclosing class.
It may be used only within the body of an instance method, a
constructor, or in the initializer of a instance field -- that is, the places
where there is an instance of the class under consideration.

Within an inner class, \xcd"this" may be qualified with the
name of a lexically enclosing class.  In this case, it
represents an instance of that enclosing class.  
For example, \xcd`Outer` is a class containing \xcd`Inner`.  Each instance of
\xcd`Inner` has a reference \xcd`Outer.this` to the \xcd`Outer` involved in its
creation.  \xcd`Inner` has access to the fields of \xcd`Outer.this`, as seen
in the \xcd`outerThree` and \xcd`alwaysTrue` methods.  Note that \xcd`Inner`
has its own \xcd`three` field, which is different from and not even the same
type as \xcd`Outer.this.three`. 
%~~gen ^^^ Expressions10
% package exp.vexp.pexp.lexp.shexp; 
% NOTEST
%~~vis 
\begin{xten}
class Outer {
  val three = 3;
  class Inner {
     val three = "THREE";
     def outerThree() = Outer.this.three;
     def alwaysTrue() = outerThree() == 3;
  }
}
\end{xten}
%~~siv
%
%~~neg

The type of a \xcd"this" expression is the
innermost enclosing class, or the qualifying class,
constrained by the class invariant and the
method guard, if any.

The \xcd"this" expression may also be used within constraints in
a class or interface header (the class invariant and
\xcd"extends" and \xcd"implements" clauses).  Here, the type of
\xcd"this" is restricted so that only properties declared in the
class header itself, and specifically not any members declared in the class
body or in supertypes, are accessible through \xcd"this".

\section{Local variables}

%##(Id
\begin{bbgrammar}
%(FROM #(prod:Id)#)
                  Id \: identifier & (\ref{prod:Id}) \\
\end{bbgrammar}
%##)

A local variable expression consists simply of the name of the local variable,
field of the current object, formal parameter in scope, etc. It evaluates to
the value of the local variable. \xcd`n` in the second line below is a local
variable expression. 
%~~gen  ^^^ Expressions20
% package exp.loc.al.varia.ble; 
% class Example {
% def example() { 
%~~vis
\begin{xten}
val n = 22;
val m = n + 56;
\end{xten}
%~~siv
%} }
%~~neg



\section{Field access}
\label{FieldAccess}
\index{field!access to}

%##(FieldAccess
\begin{bbgrammar}
%(FROM #(prod:FieldAccess)#)
         FieldAccess \: Primary \xcd"." Id & (\ref{prod:FieldAccess}) \\
                    \| \xcd"super" \xcd"." Id \\
                    \| ClassName \xcd"." \xcd"super"  \xcd"." Id \\
                    \| Primary \xcd"." \xcd"class"  \\
                    \| \xcd"super" \xcd"." \xcd"class"  \\
                    \| ClassName \xcd"." \xcd"super"  \xcd"." \xcd"class"  \\
\end{bbgrammar}
%##)

A field of an object instance may be  accessed
with a field access expression.

The type of the access is the declared type of the field with the
actual target substituted for \xcd"this" in the type. 
% If the actual
%target is not a final access path (\Sref{FinalAccessPath}),
%an anonymous path is substituted for \xcd"this".

The field accessed is selected from the fields and value properties
of the static type of the target and its superclasses.

If the field target is given by the keyword \xcd"super", the target's type is
the superclass of the enclosing class.  This form is used to access fields of
the parent class shadowed by same-named fields of the current class.

If the field target is \xcd`Cls.super`, then the target's type is \xcd`Cls`,
which must be an  enclosing class.  This (admittedly
obscure) form is used to access fields of an ancestor class which are shadowed
by same-named fields of some more recent ancestor.  The following example
illustrates all four cases:

%~~gen ^^^ Expressions30
% package exp.re.ssio.ns.fiel.dacc.ess;
% NOTEST
%~~vis
\begin{xten}
class Uncle {
  public static val f = 1;
}
class Parent {
  public val f = 2;
}
class Ego extends Parent {
  public val f = 3;
  class Child extends Ego {
     public val f = 4;
     def expDotId() = this.f; // 4
     def superDotId() = super.f; // 3
     def classNameDotId() = Uncle.f; // 1;
     def cnDotSuperDotId() = Ego.super.f; // 2
  }
}
\end{xten}
%~~siv
%
%~~neg


If the field target is \xcd"null", a \xcd"NullPointerException"
is thrown.

If the field target is a class name, a static field is selected.

It is illegal to access  a field that is not visible from
the current context.
It is illegal to access a non-static field
through a static field access expression.

\section{Function Literals}
Function literals are described in \Sref{Functions}.

\section{Calls}
\label{Call}
\label{MethodInvocation}
\label{MethodInvocationSubstitution}
\index{invocation}
\index{call}
\index{invocation!method}
\index{call!method}
\index{invocation!function}
\index{call!function}
\index{method!calling}
\index{method!invoking}


%##(MethodInvocation ArgumentList
\begin{bbgrammar}
%(FROM #(prod:MethodInvocation)#)
    MethodInvocation \: MethodPrimaryPrefix \xcd"(" ArgumentList\opt \xcd")" & (\ref{prod:MethodInvocation}) \\
                    \| MethodSuperPrefix \xcd"(" ArgumentList\opt \xcd")" \\
                    \| MethodClassNameSuperPrefix \xcd"(" ArgumentList\opt \xcd")" \\
                    \| MethodName TypeArguments\opt \xcd"(" ArgumentList\opt \xcd")" \\
                    \| Primary \xcd"." Id TypeArguments\opt \xcd"(" ArgumentList\opt \xcd")" \\
                    \| \xcd"super" \xcd"." Id TypeArguments\opt \xcd"(" ArgumentList\opt \xcd")" \\
                    \| ClassName \xcd"." \xcd"super"  \xcd"." Id TypeArguments\opt \xcd"(" ArgumentList\opt \xcd")" \\
                    \| Primary TypeArguments\opt \xcd"(" ArgumentList\opt \xcd")" \\
%(FROM #(prod:ArgumentList)#)
        ArgumentList \: Exp & (\ref{prod:ArgumentList}) \\
                    \| ArgumentList \xcd"," Exp \\
\end{bbgrammar}
%##)


A \grammarrule{MethodInvocation} may be to either a \xcd"static" method, an
instance method, or a closure.

The syntax is ambiguous; the target must be type-checked to determine if it is
the name of a method or if it refers to a field containing a closure. It is a
static error if a call may resolve to both a closure call or to a method call.
%~~gen ^^^ Expressions40
% package expres.sio.nsca.lls;
%~~vis
\begin{xten}
class Callsome {
  static val closure = () => 1;
  static def method () = 2;
  static val closureEvaluated = Callsome.closure();
  static val methodEvaluated = Callsome.method();
}
\end{xten}
%~~siv
%
%~~neg
However, adding a static method called \xcd`closure` makes \xcd`Callsome.closure()`
ambiguous: it could be a call to the closure, or to the static method: 

%~~gen ^^^ Expressions50
% package expres.sio.nsca.lls.twoooo;
% class Callsome {static val closure = () => 1; static def method () = 2; static val methodEvaluated = Callsome.method();
%~~vis
\begin{xten}
  static def closure () = 3;
  // ERROR: static errory = Callsome.closure();
\end{xten}
%~~siv
% }
%~~neg

A closure call \xcdmath"e($\dots$)" is shorthand for a method call
\xcdmath"e.apply($\dots$)". 

Method selection rules are given in \Sref{sect:MethodResolution}.

It is a static error if a method's \grammarrule{Guard} is not satisfied by the
caller.  For example: 
%~~gen ^^^ Expressions60
%package Expressions.Calls.Guarded.By.Walls;
%~~vis
\begin{xten}
class DivideBy(denom:Int) {
  def doIt(numer:Int){denom != 0} = numer / denom;
  def example() {
     //ERROR: denom might be zero: this.doIt(100); 
     (this as DivideBy{self.denom != 0}).doIt(100);
  }
}
\end{xten}
%~~siv
%
%~~neg


\section{Assignment}\index{assignment}\label{AssignmentStatement}

%##(Assignment LeftHandSide AssignmentOperator
\begin{bbgrammar}
%(FROM #(prod:Assignment)#)
          Assignment \: LeftHandSide AssignmentOperator AssignmentExp & (\ref{prod:Assignment}) \\
                    \| ExpName  \xcd"(" ArgumentList\opt \xcd")" AssignmentOperator AssignmentExp \\
                    \| Primary  \xcd"(" ArgumentList\opt \xcd")" AssignmentOperator AssignmentExp \\
%(FROM #(prod:LeftHandSide)#)
        LeftHandSide \: ExpName & (\ref{prod:LeftHandSide}) \\
                    \| FieldAccess \\
%(FROM #(prod:AssignmentOperator)#)
  AssignmentOperator \: \xcd"=" & (\ref{prod:AssignmentOperator}) \\
                    \| \xcd"*=" \\
                    \| \xcd"/=" \\
                    \| \xcd"%=" \\
                    \| \xcd"+=" \\
                    \| \xcd"-=" \\
                    \| \xcd"<<=" \\
                    \| \xcd">>=" \\
                    \| \xcd">>>=" \\
                    \| \xcd"&=" \\
                    \| \xcd"^=" \\
                    \| \xcd"|=" \\
\end{bbgrammar}
%##)



The assignment expression \xcd"x = e" assigns a value given by
expression \xcd"e"
to a variable \xcd"x".  
Most often, \xcd`x` is a mutable (\xcd`var` variable).  The same syntax is
used for delayed initialization of a \xcd`val`, but \xcd`val`s can only be
initialized once.
%~~gen ^^^ Expressions70
% package express.ions.ass.ignment;
% class Example {
% static def exasmple() {
%~~vis
\begin{xten}
  var x : Int;
  val y : Int;
  x = 1;
  y = 2; // Correct; initializes y
  x = 3; 
  // Incorrect: y = 4;
\end{xten}
%~~siv
% } } 
%~~neg


There are three syntactic forms of
assignment: 
\begin{enumerate}
\item \xcd`x = e;`, assigning to a local variable, formal parameter, field of
      \xcd`this`, etc. 
\item \xcd`x.f = e;`, assigning to a field of an object.
\item \xcdmath`a(i$_1$,$\ldots$,i$_n$) = v;`, where {$n \ge 0$}, assigning to
      an element of an array or some other such structure. This is syntactic
      sugar for a method call: \xcdmath`a.set(v,i$_1$,$\ldots$,i$_n$)`.
      Naturally, it is a static error if no suitable \xcd`set` method exists
      for \xcd`a`.
\end{enumerate}

For a binary operator $\diamond$, the $\diamond$-assignment expression
\xcdmath"x $\diamond$= e" combines the current value of \xcd`x` with the value
of \xcd`e` by {$\diamond$}, and stores the result back into \xcd`x`.  
\xcd`i += 2`, for example, adds 2 to \xcd`i`. For variables and fields, 
\xcdmath"x $\diamond$= e" behaves just like \xcdmath"x = x $\diamond$ e". 

The subscripting forms of \xcdmath"a(i) $\diamond$= b" are slightly subtle.
Subexpressions of \xcd`a` and \xcd`i` are only evaluated once.  However,
\xcd`a(i)` and \xcd`a(i)=c` are each executed once---in particular, there is
one call to \xcd`a.apply(i)` and one to \xcd`a.set(i,c)`, the desugared forms
of \xcd`a(i)` and \xcd`a(i)=c`.  If subscripting is implemented strangely for
the class of \xcd`a`, the behavior is {\em not} necessarily updating a single
storage location. Specifically, \xcd`A()(I()) += B()` is tantamount to: 
%~~gen ^^^ Expressions80
% package expressions.stupid.addab;
% class Example {
% def example(A:()=>Rail[Int], I: () => Int, B: () => Int ) {
%~~vis
\begin{xten}
{
  val aa = A();  // Evaluate A() once
  val ii = I();  // Evaluate I() once
  val bb = B();  // Evaluate B() once
  val tmp = aa(ii) + bb; // read aa(ii)
  aa(ii) = tmp;  // write sum back to aa(ii)
}
\end{xten}
%~~siv
%}}
%~~neg

\limitation{+= does not currently meet this specification.}




\section{Increment and decrement}
\index{increment}
\index{decrement}
\index{\Xcd{++}}
\index{\Xcd{--}}


The operators \xcd"++" and \xcd"--" increment and decrement
a variable, respectively.  
\xcd`x++` and \xcd`++x` both increment \xcd`x`, just as the statement 
\xcd`x += 1` would, and similarly for \xcd`--`.  

The difference between the two is the return value.  
\xcd`++x` returns the {\em new} value of \xcd`x`, after incrementing.
\xcd`x++` returns the {\em old} value of \xcd`, before incrementing.`

\limitation{This currently only works for numeric types.}

\section{Numeric Operations}
\label{XtenPromotions}
\index{promotion}
\index{numeric promotion}
\index{numeric operations}
\index{operation!numeric}

Numeric types (\xcd`Byte`, \xcd`Short`, \xcd`Int`, \xcd`Long`, \xcd`Float`,
\xcd`Double`, and unsigned variants of fixed-point types) are normal X10
structs, though most of their methods are implemented via native code. They
obey the same general rules as other X10 structs. For example, numeric
operations are defined by \xcd`operator` definitions, the same way you could
for any struct.

Promoting a numeric value to a longer numeric type preserves the sign of the
value.  For example, \xcd`(255 as UByte) as UInt` is 255.  

\subsection{Conversions and coercions}

Specifically, each numeric type can be converted or coerced into each other
numeric type, perhaps with loss of accuracy.
%~~gen ^^^ Expressions90
% package exp.ress.io.ns.numeric.conversions;
% class ExampleOfConversionAndStuff {
% def example() {
%~~vis
\begin{xten}
val n : Byte = 123 as Byte; // explicit 
val f : (Int)=>Boolean = (Int) => true; 
val ok = f(n); // implicit
\end{xten}
%~~siv
% } }
%~~neg



\subsection{Unary plus and unary minus}

The unary \xcd`+` operation on numbers is an identity function.
The unary \xcd`-` operation on numbers is a negation function.
On unsigned numbers, these are two's-complement.  For example, 
\xcd`-(0x0F as UByte)` is 
\xcd`(0xF1 as UByte)`.
\bard{UInts and such are closed under negation -- the negative of a UInt is
done binarily.  }



\section{Bitwise complement}

The unary \xcd"~" operator, only defined on integral types, complements each
bit in its operand.  

\section{Binary arithmetic operations} 

The binary arithmetic operators perform the familiar binary arithmetic
operations: \xcd`+` adds, \xcd`-` subtracts, \xcd`*` multiplies, 
\xcd`/` divides, and \xcd`%`
computes remainder.

On integers, the operands are coerced to the longer of their two types, and
then operated upon.  
Floating point operations are determined by the IEEE 754
standard. 
The integer \xcd"/" and \xcd"%" throw an exception 
if the right operand is zero.



\section{Binary shift operations}

The operands of the binary shift operations must be of integral type.
The type of the result is the type of the left operand.

If the promoted type of the left operand is \xcd"Int",
the right operand is masked with \xcd"0x1f" using the bitwise
AND (\xcd"&") operator, giving a number at most the number of bits in an
\xcd`Int`. 
If the promoted type of the left operand is \xcd"Long",
the right operand is masked with \xcd"0x3f" using the bitwise
AND (\xcd"&") operator, giving a number at most the number of bits in a
\xcd`Long`. 

The \xcd"<<" operator left-shifts the left operand by the number of
bits given by the right operand.
The \xcd">>" operator right-shifts the left operand by the number of
bits given by the right operand.  The result is sign extended;
that is, if the right operand is $k$,
the most significant $k$ bits of the result are set to the most
significant bit of the operand.

The \xcd">>>" operator right-shifts the left operand by the number of
bits given by the right operand.  The result is not sign extended;
that is, if the right operand is $k$,
the most significant $k$ bits of the result are set to \xcd"0".
This operation is deprecated, and may be removed in a later version of the
language. 

\section{Binary bitwise operations}

The binary bitwise operations operate on integral types, which are promoted to
the longer of the two types.
The \xcd"&" operator  performs the bitwise AND of the promoted operands.
The \xcd"|" operator  performs the bitwise inclusive OR of the promoted operands.
The \xcd"^" operator  performs the bitwise exclusive OR of the promoted operands.

\section{String concatenation}
\index{string!concatenation}

The \xcd"+"  operator is used for string concatenation 
 as well as addition.
If either operand is of static type \xcd"x10.lang.String",
 the other operand is converted to a \xcd"String" , if needed,
  and  the two strings  are concatenated.
 String conversion of a non-\xcd"null" value is  performed by invoking the
 \xcd"toString()" method of the value.
  If the value is \xcd"null", the value is converted to 
  \xcd'"null"'.

The type of the result is \xcd"String".

 For example, 
%~~exp~~`~~`~~ ~~ ^^^ Expressions100
      \xcd`"one " + 2 + here` 
      evaluates to  \xcd`one 2(Place 0)`.  

\section{Logical negation}

The operand of the  unary \xcd"!" operator 
must be of type \xcd"x10.lang.Boolean".
The type of the result is \xcd"Boolean".
If the value of the operand is \xcd"true", the result is \xcd"false"; if
if the value of the operand  is \xcd"false", the result is \xcd"true".

\section{Boolean logical operations}

Operands of the binary boolean logical operators must be of type \xcd"Boolean".
The type of the result is \xcd"Boolean"

The \xcd"&" operator  evaluates to \xcd"true" if both of its
operands evaluate to \xcd"true"; otherwise, the operator
evaluates to \xcd"false".

The \xcd"|" operator  evaluates to \xcd"false" if both of its
operands evaluate to \xcd"false"; otherwise, the operator
evaluates to \xcd"true".

\section{Boolean conditional operations}

Operands of the binary boolean conditional operators must be of type
\xcd"Boolean". 
The type of the result is \xcd"Boolean"

The \xcd"&&" operator  evaluates to \xcd"true" if both of its
operands evaluate to \xcd"true"; otherwise, the operator
evaluates to \xcd"false".
Unlike the logical operator \xcd"&",
if the first operand is \xcd"false",
the second operand is not evaluated.

The \xcd"||" operator  evaluates to \xcd"false" if both of its
operands evaluate to \xcd"false"; otherwise, the operator
evaluates to \xcd"true".
Unlike the logical operator \xcd"||",
if the first operand is \xcd"true",
the second operand is not evaluated.

\section{Relational operations} 

The relational operations compare numbers, producing \xcd`Boolean` results.  

The \xcd"<" operator evaluates to \xcd"true" if the left operand is
less than the right.
The \xcd"<=" operator evaluates to \xcd"true" if the left operand is
less than or equal to the right.
The \xcd">" operator evaluates to \xcd"true" if the left operand is
greater than the right.
The \xcd">=" operator evaluates to \xcd"true" if the left operand is
greater than or equal to the right.

Floating point comparison is determined by the IEEE 754
standard.  Thus,
if either operand is NaN, the result is \xcd"false".
Negative zero and positive zero are considered to be equal.
All finite values are less than positive infinity and greater
than negative infinity.

\section{Conditional expressions}
\index{\Xcd{? :}}
\index{conditional expression}
\index{expression!conditional}
\label{Conditional}

%##(ConditionalExp
\begin{bbgrammar}
%(FROM #(prod:ConditionalExp)#)
      ConditionalExp \: ConditionalOrExp & (\ref{prod:ConditionalExp}) \\
                    \| ClosureExp \\
                    \| AtExp \\
                    \| FinishExp \\
                    \| ConditionalOrExp \xcd"?" Exp \xcd":" ConditionalExp \\
\end{bbgrammar}
%##)

A conditional expression evaluates its first subexpression (the
condition); if \xcd"true"
the second subexpression (the consequent) is evaluated; otherwise,
the third subexpression (the alternative) is evaluated.

The type of the condition must be \xcd"Boolean".
The type of the conditional expression is some common 
ancestor (as constrained by \Sref{LCA}) of the types of the consequent and the
alternative. 

For example, 
%~~exp~~`~~`~~a:Int,b:Int ~~ ^^^ Expressions110
\xcd`a == b ? 1 : 2`
evaluates to \xcd`1` if \xcd`a` and \xcd`b` are the same, and \xcd`2` if they
are different.   As the type of \xcd`1` is \xcd`Int{self==1}` and of \xcd`2`
is \xcd`Int{self==2}`, the type of the conditional expression has the form
\xcd`Int{c}`, where \xcd`self==1` and \xcd`self==2` both imply \xcd`c`.  For
example, it might be \xcd`Int{true}` -- or perhaps it might be 
\xcd`Int{self != 8}`. Note that this term has no most accurate type in the X10
type system.

\section{Stable equality}
\label{StableEquality}
\index{\Xcd{==}}
\index{equality}

\begin{bbgrammar}
 EqualityExp    \: RelationalExp & (\ref{prod:EqualityExp})\\%<FROM #(prod:EqualityExp)#
    \| EqualityExp \xcd"==" RelationalExp\\
    \| EqualityExp \xcd"!=" RelationalExp\\
    \| Type  \xcd"==" Type \\
\end{bbgrammar}


The \xcd"==" and \xcd"!=" operators provide a fundamental, though
non-abstract, notion of equality.  \xcd`a==b` is true if the values of \xcd`a`
and \xcd`b` are extremely identical.

\begin{itemize}
\item If \xcd`a` and \xcd`b` are values of object type, then \xcd`a==b` holds
      if \xcd`a` and \xcd`b` are the same object.
\item If one operand is \xcd`null`, then \xcd`a==b` holds iff the other is
      also \xcd`null`.
\item If the operands both have struct type, then they must be structurally equal;
that is, they must be instances of the same struct
and all their fields or components must be \xcd"==". 
\item The definition of equality for function types is specified in
      \Sref{FunctionEquality}.
\item If the operands have numeric types, they are coerced into the larger of
      the two types (see \Sref{WideningConversions}) and then compared for numeric equality.
\end{itemize}

\xcd`a != b`
is true iff \xcd`a==b` is false.

The predicates \xcd"==" and \xcd"!=" may not be overridden by the programmer.
Note that \xcd`a==b` is a form of \emph{stable equality}; that is, the result of
the equality operation is not affected by the mutable state of the program,
after evaluation of \xcd`a` and \xcd`b`. 


\section{Allocation}
\label{ClassCreation}
\index{new}
\index{allocation}
\index{class!instantation}
\index{class!construction}
\index{struct!instantation}
\index{struct!construction}
\index{instantation}

%##(ClassInstCreationExp
\begin{bbgrammar}
%(FROM #(prod:ClassInstCreationExp)#)
ClassInstCreationExp \: \xcd"new" TypeName TypeArguments\opt \xcd"(" ArgumentList\opt \xcd")" ClassBody\opt & (\ref{prod:ClassInstCreationExp}) \\
                    \| \xcd"new" TypeName \xcd"[" Type \xcd"]" \xcd"[" ArgumentList\opt \xcd"]" \\
                    \| Primary \xcd"." \xcd"new" Id TypeArguments\opt \xcd"(" ArgumentList\opt \xcd")" ClassBody\opt \\
                    \| AmbiguousName \xcd"." \xcd"new" Id TypeArguments\opt \xcd"(" ArgumentList\opt \xcd")" ClassBody\opt \\
\end{bbgrammar}
%##)

An allocation expression creates a new instance of a class and
invokes a constructor of the class.
The expression designates the class name and passes
type and value arguments to the constructor.

The allocation expression may have an optional class body.
In this case, an anonymous subclass of the given class is
allocated.   An anonymous class allocation may also specify a
single super-interface rather than a superclass; the superclass
of the anonymous class is \xcd"x10.lang.Object".

If the class is anonymous---that is, if a class body is
provided---then the constructor is selected from the superclass.
The constructor to invoke is selected using the same rules as
for method invocation (\Sref{MethodInvocation}).

The type of an allocation expression
is the return type of the constructor invoked, with appropriate
substitutions  of actual arguments for formal parameters, as
specified in \Sref{MethodInvocationSubstitution}.

It is illegal to allocate an instance of an \xcd"abstract" class.
It is illegal to allocate an instance of a class or to invoke a
constructor that is not visible at
the allocation expression.

Note that instantiating a struct type uses function application syntax, not
\xcd`new`.  As structs do not have subclassing, there is no need or
possibility of a {\em ClassBody}.


\section{Casts}\label{ClassCast}\index{cast}
\index{type conversion}

The cast operation may be used to cast an expression to a given type:

%##(CastExp
\begin{bbgrammar}
%(FROM #(prod:CastExp)#)
             CastExp \: Primary & (\ref{prod:CastExp}) \\
                    \| ExpName \\
                    \| CastExp \xcd"as" Type \\
\end{bbgrammar}
%##)

The result of this operation is a value of the given type if the cast
is permissible at run time, and either a compile-time error or a runtime
exception 
(\xcd`x10.lang.TypeCastException`) if it is not.  

When evaluating \xcd`E as T{c}`, first the value of \xcd`E` is converted to
type \xcd`T` (which may fail), and then the constraint \xcd`{c}` is checked. 



\begin{itemize}
\item If \xcd`T` is a primitive type, then \xcd`E`'s value is converted to type
      \xcd`T` according to the rules of
      \Sref{sec:effects-of-explicit-numeric-coercions}. 
      
\item If \xcd`T` is a class, then the first half of the cast succeeds if the
      run-time value of \xcd`E` is an instance of class \xcd`T`, or of a
      subclass. 

\item If \xcd`T` is an interface, then the first half of the cast succeeds if
      the run-time value of \xcd`E` is an instance of a class implementing
      \xcd`T`. 

\item If \xcd`T` is a struct type, then the first half of the cast succeeds if
      the run-time value of \xcd`E` is an instance of \xcd`T`.  

\item If \xcd`T` is a function type, then the first half of the cast succeeds
      if the run-time value of \xcd`X` is a function of that type, or a
      subtype of it.
\end{itemize}

If the first half of the cast succeeds, the second half -- the constraint
\xcd`{c}` -- must be checked.  In general this will be done at runtime, though
in special cases it can be checked at compile time.   For example, 
\xcd`n as Int{self != w}` succeeds if \xcd`n != w` --- even if \xcd`w` is a value
read from input, and thus not determined at compile time.

The compiler may forbid casts that it knows cannot possibly work. If there is
no way for the value of \xcd`E` to be of type \xcd`T{c}`, then 
\xcd`E as T{c}` can result in a static error, rather than a runtime error.  
For example, \xcd`1 as Int{self==2}` may fail to compile, because the compiler
knows that \xcd`1`, which has type \xcd`Int{self==1}`, cannot possibly be of
type \xcd`Int{self==2}`. 


%BB% \bard{This section need serious whomping.  The Java mention needs to go.  The
%BB% rules for coercions are given in \Sref{sec:effects-of-explicit-numeric-coercions}.
%BB% If the \xcd`Type` has a constraint, the constraint will be checked at runtime. 
%BB% We need to give examples. 
%BB% }
%BB% 
%BB% Type conversion is checked according to the
%BB% rules of the \java{} language (e.g., \cite[\S 5.5]{jls2}).
%BB% For constrained types, both the base
%BB% type and the constraint are checked.
%BB% If the
%BB% value cannot be cast to the appropriate type, a
%BB% \xcd"ClassCastException"
%BB% is thrown. 



% {\bf Conversions of numeric values}
% {\bf Can't do (a as T) if a can't be a T.}


%If the value cannot be cast to the
%appropriate place type a \xcd"BadPlaceException" is thrown. 

% Any attempt to cast an expression of a reference type to a value type
% (or vice versa) results in a compile-time error. Some casts---such as
% those that seek to cast a value of a subtype to a supertype---are
% known to succeed at compile-time. Such casts should not cause extra
% computational overhead at run time.

\section{\Xcd{instanceof}}
\label{instanceOf}
\index{\Xcd{instanceof}}
\index{instanceof}

\Xten{} permits types to be used in an in instanceof expression
to determine whether an object is an instance of the given type:

%##(RelationalExp
\begin{bbgrammar}
%(FROM #(prod:RelationalExp)#)
       RelationalExp \: RangeExp & (\ref{prod:RelationalExp}) \\
                    \| SubtypeConstraint \\
                    \| RelationalExp \xcd"<" RangeExp \\
                    \| RelationalExp \xcd">" RangeExp \\
                    \| RelationalExp \xcd"<=" RangeExp \\
                    \| RelationalExp \xcd">=" RangeExp \\
                    \| RelationalExp \xcd"instanceof" Type \\
                    \| RelationalExp \xcd"in" ShiftExp \\
\end{bbgrammar}
%##)

In the above expression, \grammarrule{Type} is any type. At run time, the
result of this operator is \xcd"true" if the
\grammarrule{RelationalExpression} can be coerced to \grammarrule{Type}
without a \xcd"TypeCastException" being thrown or static error occurring.
Otherwise the result is \xcd"false". This determination may involve checking
that the constraint, if any, associated with the type is true for the given
expression.

%~~exp~~`~~`~~x:Int~~ ^^^ Expressions120
For example, \xcd`3 instanceof Int{self==x}` is an overly-complicated way of
saying \xcd`3==x`.


However, it is a static error if \xcd`e` cannot possibly be an instance of
\xcd`C{c}`; the compiler will reject \xcd`1 instanceof Int{self == 2}` because
\xcd`1` can never satisfy \xcd`Int{self == 2}`. Similarly, \Xcd{1 instanceof
String} is a static error, rather than an expression always returning false. 

\limitationx
X10 does not currently handle \xcd`instanceof` of generics in the way you
%~NO~exp~~`~~`~~r:Array[Int](1) ~~
might expect.  For example, \xcd`r instanceof Array[Int{self != 0}]` does
not test that every element of \xcd`r` is non-zero; instead, the compiler
rejects it.


\section{Subtyping expressions}
\index{\Xcd{<:}}
\index{\Xcd{:>}}
\index{subtype!test}


%##(SubtypeConstraint
\begin{bbgrammar}
%(FROM #(prod:SubtypeConstraint)#)
   SubtypeConstraint \: Type  \xcd"<:" Type  & (\ref{prod:SubtypeConstraint}) \\
                    \| Type  \xcd":>" Type  \\
\end{bbgrammar}
%##)

The subtyping expression \xcdmath"T$_1$ <: T$_2$" evaluates to \xcd"true"
\xcdmath"T$_1$" is a subtype of \xcdmath"T$_2$".

The expression \xcdmath"T$_1$ :> T$_2$" evaluates to \xcd"true"
\xcdmath"T$_2$" is a subtype of \xcdmath"T$_1$".

The expression \xcdmath"T$_1$ == T$_2$"
evaluates to  \xcd"true" \xcdmath"T$_1$" is a subtype of \xcdmath"T$_2$" and
if \xcdmath"T$_2$" is a subtype of \xcdmath"T$_1$".

Subtyping expressions are particularly useful in giving constraints on generic
types.  \xcd`x10.util.Ordered[T]` is an interface whose values can be compared
with values of type \xcd`T`. 
In particular, \xcd`T <: x10.util.Ordered[T]` is
true if values of type \xcd`T` can be compared to other values of type
\xcd`T`.  So, if we wish to define a generic class \xcd`OrderedList[T]`, of
lists whose elements are kept in the right order, we need the elements to be
ordered.  This is phrased as a constraint on \xcd`T`: 
%~~gen ^^^ Expressions130
% package expre.ssi.onsfgua.rde.dq.uantification;
%~~vis
\begin{xten}
class OrderedList[T]{T <: x10.util.Ordered[T]} {
  // ...
}
\end{xten}
%~~siv
%
%~~neg




\section{Contains expressions}
\index{in}

%##(RelationalExp
\begin{bbgrammar}
%(FROM #(prod:RelationalExp)#)
       RelationalExp \: RangeExp & (\ref{prod:RelationalExp}) \\
                    \| SubtypeConstraint \\
                    \| RelationalExp \xcd"<" RangeExp \\
                    \| RelationalExp \xcd">" RangeExp \\
                    \| RelationalExp \xcd"<=" RangeExp \\
                    \| RelationalExp \xcd">=" RangeExp \\
                    \| RelationalExp \xcd"instanceof" Type \\
                    \| RelationalExp \xcd"in" ShiftExp \\
\end{bbgrammar}
%##)

The expression \xcd"p in r" tests if a value \xcd"p" is in a collection
\xcd"r"; it evaluates to \xcd"r.contains(p)".
The collection \xcd"r"
must be of type \xcd"Collection[T]" and the value \xcd"p" must
be of type \xcd"T".

\section{Array Constructors}
\label{sect:ArrayCtors}
\index{array!construction}
\index{array!literal}

%##(Primary ClassInstCreationExp
\begin{bbgrammar}
%(FROM #(prod:Primary)#)
             Primary \: \xcd"here" & (\ref{prod:Primary}) \\
                    \| \xcd"[" ArgumentList\opt \xcd"]" \\
                    \| Literal \\
                    \| \xcd"self" \\
                    \| \xcd"this" \\
                    \| ClassName \xcd"." \xcd"this" \\
                    \| \xcd"(" Exp \xcd")" \\
                    \| ClassInstCreationExp \\
                    \| FieldAccess \\
                    \| MethodInvocation \\
                    \| MethodSelection \\
                    \| OperatorFunction \\
%(FROM #(prod:ClassInstCreationExp)#)
ClassInstCreationExp \: \xcd"new" TypeName TypeArguments\opt \xcd"(" ArgumentList\opt \xcd")" ClassBody\opt & (\ref{prod:ClassInstCreationExp}) \\
                    \| \xcd"new" TypeName \xcd"[" Type \xcd"]" \xcd"[" ArgumentList\opt \xcd"]" \\
                    \| Primary \xcd"." \xcd"new" Id TypeArguments\opt \xcd"(" ArgumentList\opt \xcd")" ClassBody\opt \\
                    \| AmbiguousName \xcd"." \xcd"new" Id TypeArguments\opt \xcd"(" ArgumentList\opt \xcd")" ClassBody\opt \\
\end{bbgrammar}
%##)

X10 includes short syntactic forms for constructing one-dimensional arrays.
The shortest form is to enclose some expressions in brackets: 
%~~gen ^^^ Expressions140
% package Expressions.ArrayCtor.Primo;
% class Example {
% def example() {
%~~vis
\begin{xten}
val ints <: Array[Int](1) = [1,3,7,21];
\end{xten}
%~~siv
%}}
%~~neg

The expression \Xcd{[e1,e2,e3, ..., en]} produces an \Xcd{n}-element
\xcd`Array[T](1)`, where \xcd`T` is the least common supertype of the {\bf
  base types} of the expressions \xcd`ei`. For example, the type of
\xcd`[0,1,2]` is \Xcd{Array[Int](1)}.    

More importantly, the type of 
\xcd`[0]` is also \xcd`Array[Int](1)`.  It is {\em not} 
\xcd`Array[Int{self==0}](1)`, even though all the elements are all 
of type \xcd`Int{self==0}`.  This is subtle but important. There are many
functions that take \xcd`Array[Int](1)`s, such as conversions to \xcd`Point`.
These functions do {\em not} take
\xcd`Array[Int{self==0}]`'s.

(Suppose that the function took \xcd`a:Array[Int](1)` and did 
the operation \xcd`a(i)=100`.   This operation is perfectly fine for
an \xcd`Array[Int](1)`, which is all the compiler knows about \xcd`a`.  
However, it is invalid for an \xcd`Array[Int{self==0}](1)`, because it assigns
a non-zero value to an element of the array, violating the type constraint
which says that all the elements are zero.  So, \xcd`Array[Int{self==0}](1)`
is not a subtype of \xcd`Array[Int](1)` --- the two types are simply unrelated.)
%~~type~~`~~`~~ ~~ ^^^ Expressions150
Since there are far more uses for \xcd`Array[Int](1)` than
%~~type~~`~~`~~ ~~ ^^^ Expressions160
\xcd`Array[Int{self==0}](1)`, the compiler produces the former.

Still, occasionally one does actually need \xcd`Array[Int{self==0}](1)`, 
or, say, \xcd`Array[Eel{self != null}](1)`, an array of non-null \xcd`Eel`s.  
For these cases, X10 provides an array constructor which does allow
specification of the element type: \xcd`new Array[T][e1...en]`.  Each
element \xcd`ei` must be of type \xcd`T`.  The resulting array is of type
\xcd`Array[T](1)`.  
%~~gen ^^^ Expressions170
%package Expressions.ArrayCtor.Details;
%class Eel{}
%class Example{
%def example(){
%~~vis
\begin{xten}
val zero <: Array[Int{self == 0}](1) = new Array[Int{self == 0}][0];
val non1 <: Array[Int{self != 1}](1) = new Array[Int{self != 1}][0];
val eels <: Array[Eel{self != null}](1) = 
    new Array[Eel{self != null}][ new Eel() ];
\end{xten}
%~~siv
%}}
%~~neg



%%OLD-RAIL%% 
%%OLD-RAIL%% 
%%OLD-RAIL%% \noo{This is now an Array ctor and the text needs revision}
%%OLD-RAIL%% \label{RailConstructors}
%%OLD-RAIL%% 
%%OLD-RAIL%% \begin{grammar}
%%OLD-RAIL%% RailConstructor \: \xcd"[" Expressions \xcd"]" \\
%%OLD-RAIL%% Expressions \: Expression ( \xcd"," Expression )\star \\
%%OLD-RAIL%% \end{grammar}
%%OLD-RAIL%% 
%%OLD-RAIL%% The rail constructor \xcdmath"[a$_0$, $\dots$, a$_{k-1}$]"
%%OLD-RAIL%% creates an instance of \xcd"ValRail" with length $k$, 
%%OLD-RAIL%% whose $i$th element is
%%OLD-RAIL%% \xcdmath"a$_i$".  The element type of the rail is a common ancestor of the
%%OLD-RAIL%% types of the \xcdmath"a$_i$"'s, as per \Sref{LCA}.
%%OLD-RAIL%% %~s~gen
%%OLD-RAIL%% % package ex.pre.ssio.nsandrailconstructors;
%%OLD-RAIL%% % class Example {
%%OLD-RAIL%% % def example() {
%%OLD-RAIL%% %~s~vis
%%OLD-RAIL%% \begin{xten}
%%OLD-RAIL%% val a <: Array[Int] = [1,3,5];
%%OLD-RAIL%% val b <: Array[Any] = [1, a, "please"];
%%OLD-RAIL%% \end{xten}
%%OLD-RAIL%% %~s~siv
%%OLD-RAIL%% % } } 
%%OLD-RAIL%% %~s~neg
%%OLD-RAIL%% 
%%OLD-RAIL%% Since arrays are subtypes of \xcd"(Point) => T",
%%OLD-RAIL%% rail constructors can be passed into the \xcd"Array" and
%%OLD-RAIL%% \xcd"ValArray" constructors as initializer functions.
%%OLD-RAIL%% 
%%OLD-RAIL%% Rail constructors of type \xcd"ValRail[Int]" and length \xcd"n" 
%%OLD-RAIL%% may be implicitly converted to type \xcd"Point{rank==n}".
%%OLD-RAIL%% Rail constructors of type \xcd"ValRail[Region]" and length \xcd"n" 
%%OLD-RAIL%% may be implicitly converted to type \xcd"Region{rank==n}".
%%OLD-RAIL%% 
%%OLD-RAIL%% %~s~gen
%%OLD-RAIL%% % package ex.pre.ssio.nsandrailconstructors;
%%OLD-RAIL%% % class Exympyl {
%%OLD-RAIL%% % def example() {
%%OLD-RAIL%% %~s~vis
%%OLD-RAIL%% \begin{xten}
%%OLD-RAIL%% val a : Point{rank==4} = [1,2,3,4];
%%OLD-RAIL%% val b : Region{rank==2} = (-1 .. 1) * (-2 .. 2);
%%OLD-RAIL%% \end{xten}
%%OLD-RAIL%% %~s~siv
%%OLD-RAIL%% % } } 
%%OLD-RAIL%% %~s~neg
%%OLD-RAIL%% 

\section{Coercions and conversions}
\label{XtenConversions}
\label{User-definedCoercions}
\index{conversion}\index{coercion}
\index{type!conversion}\index{type!coercion}

\XtenCurrVer{} supports the following coercions and conversions.

\subsection{Coercions}

%##(CastExp
\begin{bbgrammar}
%(FROM #(prod:CastExp)#)
             CastExp \: Primary & (\ref{prod:CastExp}) \\
                    \| ExpName \\
                    \| CastExp \xcd"as" Type \\
\end{bbgrammar}
%##)


A {\em coercion} does not change object identity; a coerced object may
be explicitly coerced back to its original type through a cast. A {\em
  conversion} may change object identity if the type being converted
to is not the same as the type converted from. \Xten{} permits
user-defined conversions (\Sref{sec:user-defined-conversions}).

\paragraph{Subsumption coercion.}
A subtype may be implicitly coerced to any supertype.
\index{coercion!subsumption}

\paragraph{Explicit coercion (casting with \xcd"as")}



An object of any class may be explicitly coerced to any other
class type using the \xcd"as" operation.  If \xcd`Child <: Person` and
\xcd`rhys:Child`, then 
%~~gen ^^^ Expressions180
% package Types.Coercions;
%  class Person {}
%  class Child extends Person{} 
%  class Exampllllle { 
%    def example(rhys:Child) =
%~~vis
\begin{xten}
  rhys as Person
\end{xten}
%~~siv
%;}
%~~neg
is an expression of type \xcd`Person`.  

If the value coerced is not an instance of the target type,
a \xcd"ClassCastException" is thrown.  Casting to a constrained
type may require a run-time check that the constraint is
satisfied.
\index{coercion!explicit}
\index{cast}
\index{\Xcd{as}}

\limitation{It is currently a static error, rather than the specified
\xcd`ClassCastException`, when the cast is statically determinable to be
impossible.}

\paragraph{Effects of explicit numeric coercion}
\label{sec:effects-of-explicit-numeric-coercions}

Coercing a number of one type to another type gives the best approximation of
the number in the result type, or a suitable disaster value if no
approximation is good enough.  

\begin{itemize}
\item Casting a number to a {\em wider} numeric type is safe and effective,
      and can be done by an implicit conversion as well as an explicit
%~~exp~~`~~`~~ ~~ ^^^ Expressions190
      coercion.  For example, \xcd`4 as Long` produces the \xcd`Long` value of
      4. 
\item Casting a floating-point value to an integer value truncates the digits
      after the decimal point, thereby rounding the number towards zero.  
%~~exp~~`~~`~~ ^^^ Expressions200
      \xcd`54.321 as Int` is \xcd`54`, and 
%~~exp~~`~~`~~ ~~ ^^^ Expressions210
      \xcd`-54.321 as Int` is \xcd`-54`.
      If the floating-point value is too large to represent as that kind of
      integer, the coercion returns the largest or smallest value of that type
      instead: \xcd`1e110 as Int` is 
      \xcd`Int.MAX_VALUE`, \xcd`2147483647`. 

\item Casting a \xcd`Double` to a \xcd`Float` normally truncates digits: 
%~~exp~~`~~`~~ ~~ ^^^ Expressions220
      \xcd`0.12345678901234567890 as Float` is \xcd`0.12345679f`.  This can
      turn a nonzero \xcd`Double` into \xcd`0.0f`, the zero of type
      \xcd`Float`: 
%~~exp~~`~~`~~ ~~ ^^^ Expressions230
      \xcd`1e-100 as Float` is \xcd`0.0f`.  Since 
      \xcd`Double`s can be as large as about \xcd`1.79E308` and \xcd`Float`s
      can only be as large as about \xcd`3.4E38f`, a large \xcd`Double` will
      be converted to the special \xcd`Float` value of \xcd`Infinity`: 
%~~exp~~`~~`~~ ~~ ^^^ Expressions240
      \xcd`1e100 as Float` is \xcd`Infinity`.
\item Integers are coerced to smaller integer types by truncating the
      high-order bits. If the value of the large integer fits into the smaller
      integer's range, this gives the same number in the smaller type: 
%~~exp~~`~~`~~ ~~ ^^^ Expressions250
      \xcd`12 as Byte` is the \xcd`Byte`-sized 12, 
%~~exp~~`~~`~~ ~~ ^^^ Expressions260
      \xcd`-12 as Byte` is -12. 
      However, if the larger integer {\em doesn't} fit in the smaller type,
%~~exp~~`~~`~~ ~~ ^^^ Expressions270
      the numeric value and even the sign can change: \xcd`254 as Byte` is
      \xcd`Byte`-sized \xcd`-2`.  


\end{itemize}

\subsection{Conversions}
\index{conversion}
\index{type!conversion}

\paragraph{Widening numeric conversion.}
\label{WideningConversions}
A numeric type may be implicitly converted to a wider numeric type. In
particular, an implicit conversion may be performed between a numeric
type and a type to its right, below:

\begin{xten}
Byte < Short < Int < Long < Float < Double
\end{xten}

\index{conversion!widening}
\index{conversion!numeric}

\paragraph{String conversion.}
Any value that is an operand of the binary
\xcd"+" operator may
be converted to \xcd"String" if the other operand is a \xcd"String".
A conversion to \xcd"String" is performed by invoking the \xcd"toString()"
method.

\index{conversion!string}

\paragraph{User defined conversions.}\label{sec:user-defined-conversions}
\index{conversion!user-defined}

The user may define conversion operators from type \Xcd{A} {\em to} a
container type \Xcd{B} by specifying a method on \Xcd{B} as follows:

\begin{xten}
  public static operator (r: A): T = ... 
\end{xten}

The return type \Xcd{T} should be a subtype of \Xcd{B}. The return
type need not be specified explicitly; it will be computed in the
usual fashion if it is not. However, it is good practice for the
programmer to specify the return type for such operators explicitly.

For instance, the code for \Xcd{x10.lang.Point} contains:

\begin{xten}
  public operator (r: Array[Int](1)): Point(r.length) = make(r);
\end{xten}

The compiler looks for such operators on the container type \Xcd{B}
when it encounters an expression of the form \Xcd{r as B} (where
\Xcd{r} is of type \Xcd{A}). If it finds such a method, it sets the
type of the expression \Xcd{r as B} to be the return type of the
method. Thus the type of \Xcd{r as B} is guaranteed to be some subtype
of \Xcd{B}.

\begin{example}
Consider the following code:  



%~~stmt~~\begin{xten}~~\end{xten}~~ ~~ ^^^ Expressions280
\begin{xten}
val p  = [2, 2, 2, 2, 2] as Point;
val q = [1, 1, 1, 1, 1] as Point;
val a = p - q;    
\end{xten}
This code fragment compiles successfully, given the above operator definition. 
The type of \Xcd{p} is inferred to be \Xcd{Point(5)} (i.e.{} the type 
%~~type~~`~~`~~ ~~ ^^^ Expressions290
\xcd`Point{self.rank==5}`.
Similarly for \Xcd{q}. Hence the application of the operator ``\Xcd{-}'' is legal (it requires both arguments to have the same rank). The type of \Xcd{a} is computed as \Xcd{Point(5)}.
\end{example}
	
\chapter{Statements}\label{XtenStatements}\index{statements}

This chapter describes the statements in the sequential core of
\Xten{}.  Statements involving concurrency and distribution
are described in \Sref{XtenActivities}.

\section{Empty statement}

\begin{grammar}
Statement \: \xcd";" \\
\end{grammar}

The empty statement \xcd";" does nothing.  It is useful when a
loop header is evaluated for its side effects.  For example,
the following code sums the elements of an array.
\begin{xten}
var sum: Int = 0;
for (i: Int = 0; i < a.length; i++, sum += a[i])
    ;
\end{xten}

\section{Local variable declaration}

\begin{grammar}
Statement \: LocalVariableDeclarationStatement \\
             LocalVariableDeclarationStatement \:
             LocalVariableDeclaration \xcd";" \\
\end{grammar}

The syntax of local variables declarations is described in
\Sref{VariableDeclarations}.

Local variables may be declared only within a block statement
(\Sref{Blocks}).
The scope of a local variable declaration is the 
statement itself and the subsequent statements in the block.

\section{Block statement}
\label{Blocks}

\begin{grammar}
Statement \: BlockStatement \\
BlockStatement \: \xcd"{" Statement\star \xcd"}" \\
\end{grammar}

A block statement consists of a sequence of statements delimited
by ``\xcd"{"'' and ``\xcd"}"''.  Statements are evaluated in
order.  The scope of local variables introduced within the block  
is the remainder of the block following the variable declaration.

\section{Expression statement}

\begin{grammar}
Statement \: ExpressionStatement \\
ExpressionStatement \: StatementExpression \xcd";" \\
StatementExpression \: Assignment \\
          \| Allocation \\
          \| Call \\
\end{grammar}

The expression statement evaluates an expression, ignoring the
result.  The expression must be either an assignment, an
allocation, or a call.

\section{Labeled statement}

\begin{grammar}
Statement \: LabeledStatement \\
LabeledStatement \: Identifier \xcd":" Statement \\
\end{grammar}

Statements may be labeled.  The label may be used as the target
of a break or continue statement.  The scope of a label is the
statement labeled.

\section{Break statement}

\begin{grammar}
Statement \: BreakStatement \\
BreakStatement \: \xcd"break" Identifier\opt \\
\end{grammar}

An unlabeled break statement exits the currently enclosing loop
or switch statement.

An labeled break statement exits the enclosing loop
or switch statement with the given label.

It is illegal to break out of a loop not defined in the current
method, constructor, initializer, or closure.

The following code searches for an element of a two-dimensional
array and breaks out of the loop when found:

\begin{xten}
var found: Boolean = false;
for (i: Int = 0; i < a.length; i++)
    for (j: Int = 0; j < a(i).length; j++)
        if (a(i)(j) == v) {
            found = true;
            break;
        }
\end{xten}

\section{Continue statement}

\begin{grammar}
Statement \: ContinueStatement \\
ContinueStatement \: \xcd"continue" Identifier\opt \\
\end{grammar}

An unlabeled continue statement branches to the top of the
currently enclosing loop.

An labeled break statement branches to the top of the enclosing loop
with the given label.

It is illegal to continue a loop not defined in the current
method, constructor, initializer, or closure.

\section{If statement}

\begin{grammar}
Statement \: IfThenStatement \\
          \| IfThenElseStatement \\
IfThenStatement \: \xcd"if" \xcd"(" Expression \xcd")" Statement \\
IfThenElseStatement \: \xcd"if" \xcd"(" Expression \xcd")" Statement \xcd"else" Statement \\
\end{grammar}

An if statement comes in two forms: with and without an else
clause.

The if-then statement evaluates a condition expression and 
evaluates the consequent expression if the condition is
\xcd"true".  If the 
condition is \xcd"false",
the if-then statement completes normally.

The if-then-else statement evaluates a condition expression and 
evaluates the consequent expression if the condition is
\xcd"true"; otherwise, the alternative statement is evaluated.

The condition must be of type \xcd"Boolean".

\section{Switch statement}

\begin{grammar}
Statement \: SwitchStatement \\
SwitchStatement \: \xcd"switch" \xcd"(" Expression \xcd")" \xcd"{" Case\plus \xcd"}" \\
Case \: \xcd"case" Expression \xcd":" Statement\star \\
     \| \xcd"default" \xcd":" Statement\star \\
\end{grammar}

A switch statement evaluates an index expression and then branches to
a case whose value equal to the value of the index expression.
If no such case exists, the switch branches to the 
\xcd"default" case, if any.

Statements in each case branch evaluated in sequence.  At the
end of the branch, normal control-flow falls through to the next case, if
any.  To prevent fall-through, a case branch may be exited using
a \xcd"break" statement.

The index expression must be of type \xcd"Int".

Case labels must be of type \xcd"Int" and must be compile-time
constants.  Case labels cannot be duplicated within the
\xcd"switch" statement.

\section{While statement}

\index{while@\xcd"while"}

\begin{grammar}
Statement \: WhileStatement \\
WhileStatement \: \xcd"while" \xcd"(" Expression \xcd")" Statement \\
\end{grammar}

A while statement evaluates a condition and executes a loop body
if \xcd"true".  If the loop body completes normally (either by reaching
the end or via a \xcd"continue" statement with the loop header
as target), the condition is reevaluated and the loop repeats if
\xcd"true".  If the condition is \xcd"false", the loop
exits.

The condition must be of type \xcd"Boolean".

\section{Do--while statement}

\index{do@\xcd"do"}

\begin{grammar}
Statement \: DoWhileStatement \\
DoWhileStatement \: \xcd"do" Statement \xcd"while" \xcd"(" Expression \xcd")" \xcd";" \\
\end{grammar}


A do-while statement executes a loop body, and then evaluates a
condition expression.  If \xcd"true", the loop repeats.
Otherwise, the loop exits.

The condition must be of type \xcd"Boolean".

\section{For statement}

\index{for@\xcd"for"}

\begin{grammar}
Statement \: ForStatement \\
          \| EnhancedForStatement \\
ForStatement \: \xcd"for" \xcd"("
        ForInit\opt \xcd";" Expression\opt \xcd";" ForUpdate\opt
        \xcd")" Statement \\
ForInit \:
        StatementExpression ( \xcd"," StatementExpression )\star
        \\
      \| LocalVariableDeclaration \\
EnhancedForStatement \: \xcd"for" \xcd"("
        Formal \xcd"in" Expression 
        \xcd")" Statement \\
\end{grammar}

\Xten{} provides two forms of for statement: a basic for
statement and an enhanced for statement.

A basic for statement consists of an initializer, a condition, an
iterator, and a body.  First, the initializer is evaluated.
The initializer may introduce local variables that are in scope
throughout the for statement.  An empty initializer is
permitted.
Next, the condition is evaluated.  If \xcd"true", the loop body
is executed; otherwise, the loop exits.
The condition may be omitted, in which case the condition is
considered \xcd"true".
If the loop completes normally (either by reaching the end
or via a \xcd"continue" statement with the loop header as
target),
the iterator is evaluated and then the condition is reevaluated
and the loop repeats if
\xcd"true".  If the condition is \xcd"false", the loop
exits.

The condition must be of type \xcd"Boolean".
The initializer and iterator are statements, not expressions
and so do not have types.

\label{ForAllLoop}

% XXX REGION

An enhanced for statement is used to iterate over a collection.
If the formal parameter is of type \xcd"T",
the collection expression must be of type \xcd"Iterable[T]".
Exploded
syntax may
be used for the formal parameter (\Sref{exploded-syntax}).
Each iteration of the loop
binds the parameter to another element of the collection.
If the parameter is final, it may not be assigned within the
loop body.

In a common case, the
the collection is intended to be of type
\xcd"Region" and the formal parameter is of type \xcd"Point".
Expressions \xcd"e" of type \xcd"Dist" and
\xcd"Array" are also accepted, and treated as if they were
\xcd"e.region".
If the collection is a region, the \xcd"for" statement
enumerates the points in the region in canonical order.



\section{Throw statement}
\index{throw}

\begin{grammar}
Statement \: ThrowStatement \\
ThrowStatement \: \xcd"throw" Expression \xcd";"
\end{grammar}

The \xcd"throw" statement throws an exception.  The exception
must be a subclass of the value class \xcd"x10.lang.Throwable". 
% null not allowed since a value class;
% If the exception is
% \xcd"null", a \xcd"NullPointerException" is thrown.

\begin{example}
The following statement checks if an index is in range and
throws an exception if not.
\begin{xten}
if (i < 0 || i > x.length)
    throw new IndexOutOfBoundsException();
\end{xten}
\end{example}

\section{Try--catch statement}

\begin{grammar}
Statement \: TryStatement \\
TryStatement \: \xcd"try" BlockStatement Catch\plus Finally\opt \\
             \| \xcd"try" BlockStatement Catch\star Finally \\
Catch \: \xcd"catch" \xcd"(" Formal \xcd")" BlockStatement \\
Finally \: \xcd"finally" BlockStatement \\
\end{grammar}

Exceptions are handled with a \xcd"try" statement.
A \xcd"try" statement consists of a \xcd"try" block, zero or more
\xcd"catch" blocks, and an optional \xcd"finally" block.

First, the \xcd"try" block is evaluated.  If the block throws an
exception, control transfers to the first matching \xcd"catch"
block, if any.  A \xcd"catch" matches if the value of the
exception thrown is a subclass of the \xcd"catch" block's formal
parameter type.

The \xcd"finally" block, if present, is evaluated on all normal
and exceptional control-flow paths from the \xcd"try" block.
If the \xcd"try" block completes normally
or via a \xcd"return", a \xcd"break", or a
\xcd"continue" statement, 
the \xcd"finally"
block is evaluated, and then control resumes at
the statement following the \xcd"try" statement, at the branch target, or at
the caller as appropriate.
If the \xcd"try" block completes
exceptionally, the \xcd"finally" block is evaluated after the
matching \xcd"catch" block, if any, and then the
exception is rethrown.

\section{Return statement}
\label{ReturnStatement}
\index{ReturnStatement}
\begin{grammar}
Statement \: ReturnStatement \\
ReturnStatement \: \xcd"return" Expression \xcd";" \\
             \| \xcd"return" \xcd";" \\
\end{grammar}

Methods and closures may return values using a return statement.
If the method's return type is expliclty declared \xcd"Void",
the method may return without a value; otherwise, it must return
a value of the appropriate type.
	

\chapter{Places}
\label{XtenPlaces}
\index{place}

An \Xten{} place is a repository for data and activities, corresponding
loosely to a process or a processor. Places induce a concept of ``local''. The
activities running in a place may access data items located at that place with
the efficiency of on-chip access. Accesses to remote places may take orders of
magnitude longer. X10's system of places is designed to make this obvious.
Programmers are aware of the places of their data, and know when they are
incurring communication costs, but the actual operation to do so is easy. It's
not hard to use non-local data; it's simply hard to to do so accidentally.

The set of places available to a computation is determined at the time that
the program is started, and remains fixed through the run of the program. See
the {\tt README} documentation on how to set command line and configuration
options to set the number of places.

Places are first-class values in X10, as instances 
\xcd"x10.lang.Place".   \xcd`Place` provides a number of useful ways to
query places, such as \xcd`Place.places`, which is a  \xcd`Sequence[Place]` of 
the places
available to the current run of the program.

Objects and structs (with one exception) are created in a single place -- the
place that the constructor call was running in. They cannot change places.
They can be {\em copied} to other places, and the special library struct
\Xcd{GlobalRef} allows values at one place to point to values at another.  

\section{The Structure of Places}
\index{place!MAX\_PLACES}
\index{place!FIRST\_PLACES}
\index{MAX\_PLACES}
\index{FIRST\_PLACE}

%~~exp~~`~~`~~ ~~ ^^^ Places10
Places are numbered 0 through \xcd`Place.MAX_PLACES-1`; the number is stored
in the field 
\xcd`pl.id`.  The \xcd`Sequence[Place]` \xcd`Place.places()` contains the places of the
program, in numeric order. 
The program starts by executing a \xcd`main` method at
%~~exp~~`~~`~~ ~~ ^^^ Places20
\xcd`Place.FIRST_PLACE`, which is 
%~~exp~~`~~`~~ ~~ ^^^ Placesoik
\xcd`Place.places()(0)`; see
\Sref{initial-computation}. 

Operations on places include \xcd`pl.next()`, which gives the next entry
(looping around) in \xcd`Place.places` and its opposite \xcd`pl.prev()`. 
In multi-place executions, 
\xcd`here.next()` is a convenient way to express ``a place other than \xcd`here`''.
There are also tests, like  
%~~exp~~`~~`~~pl:Place ~~ ^^^ Placesoid
\xcd`pl.isCUDA()`, which test for particular kinds of processors.


\section{{\tt here}}\index{here}\label{Here}

The variable \xcd"here" is always bound to the place at which the current
computation is running, in the same way that \xcd`this` is always bound to the
instance of the current class (for non-static code), or \xcd`self` is bound to
the instance of the type currently being constrained.  
\xcd`here` may denote different places in the same method body or even the
same expression, due to
place-shifting operations.


This is not unusual for automatic variables:  \Xcd{self} denotes 
two different values (one \xcd`List`, one \xcd`Long`) 
when one describes a non-null list of non-zero numbers as
\xcd`List[Long{self!=0}]{self!=null}`. In the following 
code, \xcd`here` has one value at 
\xcd`h0`, and a different one at \xcd`h1` (unless there is only one place).
%~~gen ^^^ Placesoijo
% package places.are.For.Graces;
% class Example {
% def example() {
%~~vis
\begin{xten}
val h0 = here;
at (here.next()) {
  val h1 = here; 
  assert (h0 != h1);
}
\end{xten}
%~~siv
%} } 
% 
%~~neg
\noindent
(Similar examples show that \xcd`self` and \xcd`this` have the same behavior:
\xcd`self` can be shadowed by constrained types appearing inside of type
constraints, and \xcd`this` by inner classes.)



The following example looks through a list of references to \Xcd{Thing}s.  
It finds those references to things that are \Xcd{here}, and deals with them.  
%~~gen ^^^ Places70
%package Places.Are.For.Graces.2;
%import x10.util.*;
%abstract class Thing {}
%class DoMine {
%  static def dealWith(Thing) {}	
%~~vis
\begin{xten}
  public static def deal(things: List[GlobalRef[Thing]]) {
     for(gr in things) {
        if (gr.home == here) {
           val grHere = 
               gr as GlobalRef[Thing]{gr.home == here};
           val thing <: Thing = grHere();
           dealWith(thing);
        }
     }
  }
\end{xten}
%~~siv
%}
% 
%~~neg

\section{ {\tt at}: Place Changing}\label{AtStatement}
\index{at}
\index{place!changing}

An activity may change place synchronously using the \xcd"at" statement or
\xcd"at" expression. Like any distributed operation, it is 
potentially expensive, as it requires, at a minimum, two messages
and the copying of all data used in the operation, and must be used with care
-- but it provides the basis for distributed programming in X10.

%##(AtStatement AtExp
\begin{bbgrammar}
%(FROM #(prod:AtStmt)#)
              AtStmt \: \xcd"at" \xcd"(" Exp \xcd")" Stmt & (\ref{prod:AtStmt}) \\
%(FROM #(prod:AtExp)#)
               AtExp \: \xcd"at" \xcd"(" Exp \xcd")" ClosureBody & (\ref{prod:AtExp}) \\
\end{bbgrammar}
%##)

The {\it PlaceExp} must be an expression of type \xcd`Place` or some
subtype. For programming convenience, if {\it PlaceExp} is of type
\xcd`GlobalRef[T]` then the \xcd'home' property of \xcd'GlobalRef' is
used as the value of {\it PlaceExp}.

An activity may also spawn an asynchronous remote child activity.  For
optimal performance, it is desirable for the spawning activity to
continue executing locally without waiting for the message creating
the remote child activity to arrive at the destination place. \Xten{}
supports this ``fire-and-forget'' style of remote activity creation by
special handling of the combination of \xcd'at (P) async S'.  In
particular, any exceptions raised during deserialization
(\Sref{sect:at-init-val}) at the remote place will be reported
asynchronously (as if they occured after the remote activity
\xcd`async S` was spawned).

%%AT-COPY%% The \xcd`at`-statment \xcd`at(p;F)S` first evaluates \xcd`p` to a place, then
%%AT-COPY%% copies information to that place as determined by \xcd`F`, and then executes
%%AT-COPY%% \xcd`S` using the resulting copies.  The \xcd`at`-{\em expression}
%%AT-COPY%% \xcd`at(p;F)E` is similar, but it copies the result of the expression \xcd`E`
%%AT-COPY%% and returns the copy as its result.
%%AT-COPY%% 
%%AT-COPY%% The clause \xcd`F` in \xcd`at(p;F)S` is a list of zero or more {\em copy
%%AT-COPY%% specifiers}, explaining what values are to be copied to the place \xcd`p`, and
%%AT-COPY%% how they are to be referred to at \xcd`p`.  
%%AT-COPY%% 

%%AT-COPY%% \begin{ex}
%%AT-COPY%% The following example creates a rail \xcd`a` located \xcd`here`, and copies
%%AT-COPY%% it to another place, giving the copy the name \xcd`a2` there.  The copy is
%%AT-COPY%% modified and examined.  After the \xcd`at` finishes, the original is also
%%AT-COPY%% examined, and (since only the copy, not the original, was modified) is observed
%%AT-COPY%% to be unchanged. 
%%AT-COPY%% %~x~gen ^^^ Places6e1o
%%AT-COPY%% % package Places6e1o;
%%AT-COPY%% % KNOWNFAIL-at
%%AT-COPY%% % class Example { static def example() { 
%%AT-COPY%% %~x~vis
%%AT-COPY%% \begin{xten}
%%AT-COPY%% val a = [1,2,3];
%%AT-COPY%% at(here.next(); a2 = a) {
%%AT-COPY%%   a2(1) = 4;
%%AT-COPY%%   assert a2(0)==1 && a2(1)==4 && a2(2)==3; 
%%AT-COPY%%   // 'a' is not accessible here
%%AT-COPY%% }
%%AT-COPY%% assert  a(0)==1 && a(1)==2 && a(2)==3; 
%%AT-COPY%% \end{xten}
%%AT-COPY%% %~x~siv
%%AT-COPY%% %} } 
%%AT-COPY%% % class Hook { def run() { Example.example(); return true; }}
%%AT-COPY%% %~x~neg
%%AT-COPY%% \end{ex}
%%AT-COPY%% 

\begin{ex}
The following example creates a rail \xcd`a` located \xcd`here`, and copies
it to another place.  \xcd`a` in the second place (\xcd`here.next()`) refers
to the copy.  The copy is
modified and examined.  After the \xcd`at` finishes, the original is also
examined, and (since only the copy, not the original, was modified) is observed
to be unchanged. 
%~~gen ^^^ Places6e1o
% package Places6e1o;
% KNOWNFAIL-at
% class Example { static def example() { 
%~~vis
\begin{xten}
val a = [1,2,3];
at(here.next()) {
  a(1) = 4;
  assert a(0)==1 && a(1)==4 && a(2)==3; 
}
assert  a(0)==1 && a(1)==2 && a(2)==3; 
\end{xten}
%~~siv
%} } 
% class Hook { def run() { Example.example(); return true; }}
%~~neg
\end{ex}

%%AT-COPY%% \subsection{Copy Specifiers}
%%AT-COPY%% \label{sect:copy-spec}
%%AT-COPY%% \index{copy specifier}
%%AT-COPY%% \index{at!copy specifier}
%%AT-COPY%% 
%%AT-COPY%% A single copy specifier can be one of the following forms.   
%%AT-COPY%% Each copy specifier determines an {\em original-expression}, saying what value
%%AT-COPY%% will be copied, and a {\em target variable}, saying what it will be called.
%%AT-COPY%% 
%%AT-COPY%% \begin{itemize}
%%AT-COPY%% 
%%AT-COPY%% \item \xcd`val x = E`, and its usual variants \xcd`val x:T = E`, 
%%AT-COPY%%       \xcd`x : T = E`, and 
%%AT-COPY%%       \xcd`val x <: T = E`, evaluate the expression \xcd`E` at the initial
%%AT-COPY%%       place, copy it to \xcd`p`, and bind \xcd`x` to the copy, as normal for a
%%AT-COPY%%       local \xcd`val` binding.  If a type is supplied, it is checked
%%AT-COPY%%       statically in the usual way.  
%%AT-COPY%%       The original-expression is \xcd`E`, and the target variable is \xcd`x`.
%%AT-COPY%% 
%%AT-COPY%% \begin{ex}
%%AT-COPY%% The following code copies a variable \xcd`a` located \xcd`here` to a variable
%%AT-COPY%% \xcd`d` located \xcd`there`.  
%%AT-COPY%% Note that, while the copy \xcd`d` is available \xcd`there` inside of the \xcd`at`-block,
%%AT-COPY%% the original \xcd`a` is not.  (\xcd`a` could not be available in the block in
%%AT-COPY%% any case; it is not located \xcd`there`.)
%%AT-COPY%% %~~gen ^^^ Places9v2e1
%%AT-COPY%% % package Places9v2e1;
%%AT-COPY%% % KNOWNFAIL-at
%%AT-COPY%% % class Example{ 
%%AT-COPY%% % static def use(Any) = 1;
%%AT-COPY%% % static def example() { 
%%AT-COPY%% %  val there = here.next();
%%AT-COPY%% %~~vis
%%AT-COPY%% \begin{xten}
%%AT-COPY%% var a : Long = 1;
%%AT-COPY%% at(there; val d = a) {
%%AT-COPY%%    assert d == 1;
%%AT-COPY%%    // ERROR: assert a == 1;
%%AT-COPY%% }
%%AT-COPY%% \end{xten}
%%AT-COPY%% %~~siv
%%AT-COPY%% % } } 
%%AT-COPY%% % class Hook{ def run() {Example.example(); return true;}}
%%AT-COPY%% %~~neg
%%AT-COPY%% \end{ex}
%%AT-COPY%% 
%%AT-COPY%% \item \xcd`var x : T = E` evaluates \xcd`E` at the initial place, copies it to
%%AT-COPY%%       \xcd`p`, and binds \xcd`x` to a new \xcd`var` whose initial value is the
%%AT-COPY%%       copy, as normal for a local \xcd`var` binding.
%%AT-COPY%%       If a type is supplied, it is checked
%%AT-COPY%%       statically in the usual way.
%%AT-COPY%%       The original-expression is \xcd`E`, and the target variable is \xcd`x`.
%%AT-COPY%%       Note that, like a \xcd`var` parameter to a method, \xcd`x` is a local
%%AT-COPY%%       variable.  Changes to \xcd`x` will not change anything else. In
%%AT-COPY%%       particular, even if \xcd`x` has the same name as a \xcd`var` variable
%%AT-COPY%%       outside, the two \xcd`var`s are unconnected.  
%%AT-COPY%%       See \Sref{sect:athome} for the way to modify a variable from the
%%AT-COPY%%       surrounding scope.
%%AT-COPY%% 
%%AT-COPY%% \begin{ex}
%%AT-COPY%% The following code copies \xcd`a` to a \xcd`var` named \xcd`e`.  Changing
%%AT-COPY%% \xcd`e` does not change \xcd`a`; the two \xcd`var`s have no ongoing relationship.
%%AT-COPY%% %~~gen ^^^ Places9v2e2
%%AT-COPY%% % package Places9v2e2;
%%AT-COPY%% % KNOWNFAIL-at
%%AT-COPY%% % class Example{ 
%%AT-COPY%% % static def use(Any) = 1;
%%AT-COPY%% % static def example() { 
%%AT-COPY%% %  val there = here.next();
%%AT-COPY%% %~~vis
%%AT-COPY%% \begin{xten}
%%AT-COPY%% var a : Long = 1;
%%AT-COPY%% assert a == 1;
%%AT-COPY%% at(there; var e = a) { 
%%AT-COPY%%    assert e == 1;
%%AT-COPY%%    e += 1;
%%AT-COPY%%    assert e == 2;
%%AT-COPY%% }
%%AT-COPY%% assert a == 1; 
%%AT-COPY%% \end{xten}
%%AT-COPY%% %~~siv
%%AT-COPY%% % 
%%AT-COPY%% % }  } 
%%AT-COPY%% % class Hook{ def run() {Example.example(); return true;}}
%%AT-COPY%% %~~neg
%%AT-COPY%% \end{ex}
%%AT-COPY%% 
%%AT-COPY%% \item \xcd`x = E`, as a copy specifier, is equivalent to \xcd`val x = E`.
%%AT-COPY%%       Note that this abbreviated form is not available as a local variable
%%AT-COPY%%       definition, (because it is used as an assignment statement), but in a
%%AT-COPY%%       copy specifier there are no assignment statements and so the
%%AT-COPY%%       abbreviation is allowed.
%%AT-COPY%%       The original-expression is \xcd`E`, and the target variable is \xcd`x`.
%%AT-COPY%% 
%%AT-COPY%% \begin{ex}
%%AT-COPY%% The following code evaluates an expression \xcd`a+b(0)`.  The result of this
%%AT-COPY%% expression is stored \xcd`there`, in the \xcd`val` variable \xcd`f`, but is
%%AT-COPY%% not stored \xcd`here`. 
%%AT-COPY%% %~~gen ^^^ Places9v2e3
%%AT-COPY%% % package Places9v2e3;
%%AT-COPY%% % KNOWNFAIL-at
%%AT-COPY%% % class Example{ 
%%AT-COPY%% % static def use(Any) = 1;
%%AT-COPY%% % static def example() { 
%%AT-COPY%% %  val there = here.next();
%%AT-COPY%% %~~vis
%%AT-COPY%% \begin{xten}
%%AT-COPY%% var a : Long = 1;
%%AT-COPY%% var b : Rail[Long] = [2,3,4];
%%AT-COPY%% at(there; f = a + b(0)) {
%%AT-COPY%%    assert f == 3;
%%AT-COPY%% }
%%AT-COPY%% \end{xten}
%%AT-COPY%% %~~siv
%%AT-COPY%% % }  } 
%%AT-COPY%% % class Hook{ def run() {Example.example(); return true;}}
%%AT-COPY%% % 
%%AT-COPY%% %~~neg
%%AT-COPY%% 
%%AT-COPY%% 
%%AT-COPY%% \end{ex}
%%AT-COPY%% 
%%AT-COPY%% \item \xcd`x` alone, as a copy specifier, is equivalent to \xcd`val x = x`.
%%AT-COPY%%       It says that the variable \xcd`x` will be copied, and the copy will also
%%AT-COPY%%       be named \xcd`x`.  
%%AT-COPY%%       The original-expression is \xcd`x`, and the target variable is \xcd`x`.
%%AT-COPY%% 
%%AT-COPY%% \begin{ex}
%%AT-COPY%% The following code copies \xcd`b` to \xcd`there`.  The copy is also called
%%AT-COPY%% \xcd`b`.  The two \xcd`b`'s are not connected; \eg, changing one does not
%%AT-COPY%% change the other.
%%AT-COPY%% %~~gen ^^^ Places9v2e4
%%AT-COPY%% % package Places9v2e4;
%%AT-COPY%% % KNOWNFAIL-at
%%AT-COPY%% % class Example{ 
%%AT-COPY%% % static def use(Any) = 1;
%%AT-COPY%% % static def example() { 
%%AT-COPY%% %  val there = here.next();
%%AT-COPY%% %~~vis
%%AT-COPY%% \begin{xten}
%%AT-COPY%% var b : Rail[Long] = [2,3,4];
%%AT-COPY%% assert b(0) == 2;
%%AT-COPY%% at(there; b) {
%%AT-COPY%%   b(0) = 200;  // Modify copy of b.
%%AT-COPY%%   assert b(0) == 200;
%%AT-COPY%% }
%%AT-COPY%% assert b(0) == 2; 
%%AT-COPY%% \end{xten}
%%AT-COPY%% %~~siv
%%AT-COPY%% % 
%%AT-COPY%% % }  } 
%%AT-COPY%% % class Hook{ def run() {Example.example(); return true;}}
%%AT-COPY%% %~~neg
%%AT-COPY%% \end{ex}
%%AT-COPY%% 
%%AT-COPY%% \item A field assignment statements \xcdmath"a.fld = $E_2$", evaluates 
%%AT-COPY%%       \xcd`a` and $E_2$ on the sending side to values $v_1$ and {$v_2$}.  
%%AT-COPY%%       {$v_1$} must be an object with a mutable field \xcd`fld`.  {$v_1$} and
%%AT-COPY%%       {$v_2$} are sent to place \xcd`p`, and the field assignment is performed
%%AT-COPY%%       there.  The modified version of {$v_1$} is available as a \xcd`val`
%%AT-COPY%%       variable \xcd`a`.   The compiler may optimize this, \eg, by neglecting to
%%AT-COPY%%       deserialize \xcdmath"$v_1$.fld", and deserializing {$v_2$} directly into
%%AT-COPY%%       that field rather than into a separate buffer.
%%AT-COPY%% 
%%AT-COPY%% \begin{ex}
%%AT-COPY%% %~~gen ^^^ Places9v2e5
%%AT-COPY%% % package Places9v2e5;
%%AT-COPY%% % KNOWNFAIL
%%AT-COPY%% % class Example {
%%AT-COPY%% % static def use(Any) = 1;
%%AT-COPY%% % static def example() { 
%%AT-COPY%% %  val there = here.next();
%%AT-COPY%% %~~vis
%%AT-COPY%% \begin{xten}
%%AT-COPY%% class Example{ 
%%AT-COPY%%    var f : Long = 1;
%%AT-COPY%%    var g : Long = 2;
%%AT-COPY%%    static def example() { 
%%AT-COPY%%       val there = here.next();
%%AT-COPY%%       val e : Example = new Example();
%%AT-COPY%%       assert e.f == 1 && e.g == 2;
%%AT-COPY%%       at(there; e.f = 3) {
%%AT-COPY%%           assert e.f == 3; && e.g == 2;
%%AT-COPY%%       }
%%AT-COPY%%       assert e.f == 1 && e.g == 2;
%%AT-COPY%%    }
%%AT-COPY%% }
%%AT-COPY%% \end{xten}
%%AT-COPY%% %~~siv
%%AT-COPY%% % class Hook{ def run() {Example.example(); return true;}}
%%AT-COPY%% %~~neg
%%AT-COPY%% %
%%AT-COPY%% \end{ex}
%%AT-COPY%% 
%%AT-COPY%% \item A rail-element assignment 
%%AT-COPY%%       \xcdmath"a($E_1$, $\ldots$, $E_n$) = $E_+$".
%%AT-COPY%%       This copies and transmits \xcd`a` as normal for a rail.  In addition,
%%AT-COPY%%       and 
%%AT-COPY%%       much like a field assignment, it also evaluates all the expressions $E_i$
%%AT-COPY%%       at the sending side to values $v_i$, and transmits them.  \xcd`a`'s value must
%%AT-COPY%%       admit a suitably-typed $n$-ary subscripting operation.  That operation
%%AT-COPY%%       is applied after the values are deserialized at \xcd`p`.  The compiler
%%AT-COPY%%       may optimize this, \eg, by neglecting to deserialize one element of the
%%AT-COPY%%       rail $v_0$, and deserializing $v_+$ directly into that location.  
%%AT-COPY%% 
%%AT-COPY%% 
%%AT-COPY%% \begin{ex}
%%AT-COPY%% The following code sends a modified \xcd`b` to \xcd`there`, while (as always)
%%AT-COPY%% keeping an unmodified version \xcd`here`.   X10 may perform optimizations to
%%AT-COPY%% avoid transmitting the original value of \xcd`b(1)`, since it will be
%%AT-COPY%% overwritten immediately in any case.
%%AT-COPY%% %~~gen ^^^ Places9v2e6
%%AT-COPY%% % package Places9v2e6;
%%AT-COPY%% % KNOWNFAIL
%%AT-COPY%% % class Example{ 
%%AT-COPY%% % static def use(Any) = 1;
%%AT-COPY%% % static def example() { 
%%AT-COPY%% %  val there = here.next();
%%AT-COPY%% %~~vis
%%AT-COPY%% \begin{xten}
%%AT-COPY%% var b = [2,3,4];
%%AT-COPY%% assert b(0) == 2 && b(1) == 3;
%%AT-COPY%% at(there; b(1) = 300) {
%%AT-COPY%%   assert b(0) == 2 && b(1) == 300;
%%AT-COPY%% }
%%AT-COPY%% assert b(0) == 2 && b(1) == 3;
%%AT-COPY%% \end{xten}
%%AT-COPY%% %~~siv
%%AT-COPY%% % 
%%AT-COPY%% %~~neg
%%AT-COPY%% % }  }
%%AT-COPY%% % class Hook{ def run() {Example.example(); return true;}}
%%AT-COPY%% \end{ex}
%%AT-COPY%% 
%%AT-COPY%% \item \xcd`*` may appear as the last copy specifier in the list, indicating
%%AT-COPY%%       that all \xcd`val` variables from outside \xcd`S` which are used in
%%AT-COPY%%       \xcd`S` should be copied. Specifically, let 
%%AT-COPY%%       \xcdmath"x$_1, \ldots, $x$_n$" be all the \xcd`val` variables defined
%%AT-COPY%%       outside of \xcd`S` 
%%AT-COPY%%       mentioned in \xcd`S`. The \xcd`*` copy specifier is equivalent to 
%%AT-COPY%%       the list of variables 
%%AT-COPY%%       \xcdmath"x$_1, \ldots, $x$_n$".
%%AT-COPY%% 
%%AT-COPY%% \begin{ex}
%%AT-COPY%% %~~gen ^^^ Places9v2e7
%%AT-COPY%% % package Places9v2e7;
%%AT-COPY%% % KNOWNFAIL-at
%%AT-COPY%% % class Example{ 
%%AT-COPY%% % static def use(Any) = 1;
%%AT-COPY%% % static def example() { 
%%AT-COPY%% %  val there = here.next();
%%AT-COPY%% %~~vis
%%AT-COPY%% \begin{xten}
%%AT-COPY%% var a : Long = 1;
%%AT-COPY%% val b = [2,3,4];
%%AT-COPY%% at(there; *) {
%%AT-COPY%%   assert a + b(0) == b(1);
%%AT-COPY%% }
%%AT-COPY%% \end{xten}
%%AT-COPY%% %~~siv
%%AT-COPY%% % }  }
%%AT-COPY%% % class Hook{ def run() {Example.example(); return true;}}
%%AT-COPY%% %~~neg
%%AT-COPY%% 
%%AT-COPY%% \end{ex}
%%AT-COPY%% 
%%AT-COPY%% \end{itemize}
%%AT-COPY%% 
%%AT-COPY%% As an important special case, \xcd`at(p;)S` copies {\em nothing} to \xcd`S`.
%%AT-COPY%% This must not be confused with \xcd`at(p)S`, which copies {\em everything}.
%%AT-COPY%% 
%%AT-COPY%% 
%%AT-COPY%% 
%%AT-COPY%% Note that \xcd`at(p;x,*)use(x,y);` is equivalent to \xcd`at(p;*)use(x,y);`.
%%AT-COPY%% In both statements, the \xcd`*` indicates that all variables used in the body
%%AT-COPY%% are to be copied in.  The former makes clear that \xcd`x` is one of the things
%%AT-COPY%% being copied, but, from the \xcd`*`, there may be others. 
%%AT-COPY%% 
%%AT-COPY%% However, other copy specifiers may be used to compute
%%AT-COPY%% values in \xcd`S` which are not available (and thus need not be stored)
%%AT-COPY%% outside of it.  
%%AT-COPY%% 
%%AT-COPY%% \begin{ex}The following code may end up with a large object \xcd`c` in
%%AT-COPY%% memory at \xcd`p` but not at the initial place: 
%%AT-COPY%% %~~gen ^^^ Places3q9u
%%AT-COPY%% % package Places3q9u;
%%AT-COPY%% % KNOWNFAIL-at
%%AT-COPY%% % class Example { 
%%AT-COPY%% % def use(Example, Example, Example) = 1;
%%AT-COPY%% % def Elephant(Example) = 1;
%%AT-COPY%% % static def example(a: Example, b:Example, p:Place) { 
%%AT-COPY%% %~~vis
%%AT-COPY%% \begin{xten}
%%AT-COPY%% at(p; c = a.Elephant(b), *) {
%%AT-COPY%%   use(a,b,c);
%%AT-COPY%% }
%%AT-COPY%% \end{xten}
%%AT-COPY%% %~~siv
%%AT-COPY%% %} } 
%%AT-COPY%% %~~neg
%%AT-COPY%% \end{ex}
%%AT-COPY%% 
%%AT-COPY%% The blanket \xcd`at`-statement \xcd`at(p)S` copies everything.  It is an
%%AT-COPY%% abbreviation for \xcd`at(p;*)S`.  
%%AT-COPY%% When this manual refers to a generic \xcd`at`-statement as \xcd`at(p;F)S`, it
%%AT-COPY%% should be understood as including the blanket \xcd`at` statement \xcd`at(p)S`
%%AT-COPY%% with this interpretation.
%%AT-COPY%% 

\subsection{Copying Values}
%%AT-COPY%% An activity executing statement \xcd"at (q;F) S" at a place \xcd`p`
%%AT-COPY%% evaluates \xcd`q` at \xcd`p` and then moves to \xcd`q` to execute
%%AT-COPY%% \xcd`S`.  
%%AT-COPY%% The original-expressions of \xcd`F` are evaluated at \xcd`p`.
%%AT-COPY%% Their values are copied (\Sref{sect:at-init-val}) to \xcd`q`, and bound to 
%%AT-COPY%% names there, as specified by \xcd`F`.  
%%AT-COPY%% \xcd`S` is evaluated in an environment containing the target variables of
%%AT-COPY%% \xcd`F`, and \xcd`here` and {\em no} other variables.  (In particular, if this
%%AT-COPY%% statement appears in an instance method body and \xcd`this` is not copied,
%%AT-COPY%% \xcd`this` is not accessible.  This fact is important: it allows the
%%AT-COPY%% programmer to control when \xcd`this` is copied, which may be expensive for
%%AT-COPY%% large containers.)

An activity executing \xcd`at(q)S` at a place \xcd`p` evaluates \xcd`q` at
place \xcd`p`, which should be a \xcd`Place`.  It then moves to place \xcd`q`
to execute \xcd`S`.  The values variables that \xcd`S` refers to are copied
(\Sref{sect:at-init-val}) to \xcd`q`, and bound to the variables of the same
name.   If the \xcd`at` is inside of an instance method and \xcd`S` uses
\xcd`this`, \xcd`this` is copied as well.  Note that a field reference
\xcd`this.fld` or a method call \xcd`this.meth()` will cause \xcd`this` to be
copied --- as will their abbreviated forms \xcd`fld` and \xcd`meth()`, despite
the lack of a visible \xcd`this`.  


Note that the value obtained by evaluating \xcd`q`
is not necessarily distinct from \xcd`p` (\eg, \xcd`q` may be
\xcd`here`). 
This does not alter the behavior of \xcd`at`.  
%%AT-COPY%%  \xcd`at(here;F)S` will copy all the values specified by \xcd`F`, 
%%AT-COPY%% even though there is no actual change of place, and even though the original
%%AT-COPY%% values already exist there.
\xcd`at(here)S` will copy all the values mentioned in \xcd`S`, even though
there is no actual change of place, and even though the original values
already exist there. 

On normal termination of \xcd`S` control returns to \xcd`p` and
execution is continued with the statement following 
%%AT-COPY%% \xcd`at (q;F) S`. 
\xcd`at (q) S`. 
If
\xcd`S` terminates abruptly with exception \xcd`E`, \xcd`E` is
serialized into a buffer, the buffer is communicated to \xcd`p` where
it is deserialized into an exception \xcd`E1` and \xcd`at (p) S`
throws \xcd`E1`.

Since 
%%AT-COPY%% \xcd`at(p;F) S` 
\xcd`at(p) S` 
is a synchronous construct, usual control-flow
constructs such as \xcd`break`, \xcd`continue`, \xcd`return` and 
\xcd`throw` are permitted in \xcd`S`.  All concurrency related
constructs -- \xcd`async`, \xcd`finish`, \xcd`atomic`, \xcd`when` are
also permitted.

The \xcd`at`-expression 
%%AT-COPY%% \xcd`at(p;F)E` 
\xcd`at(p)E` 
is similar, except that, in the case of
normal termination of \xcd`E`, the value that \xcd`E` produces is serialized
into a buffer, transported to the starting place, and deserialized, and the
value of the \xcd`at`-expression is the result of deserialization.

\limitation{
X10 does not currently allow {\tt break}, {\tt continue}, or {\tt return}
to exit from an {\tt at}.
}



\subsection{How {\tt at} Copies Values}
\label{sect:at-init-val}

%%AT-COPY%% The values of the original-expressions  specified by \xcd`F` in 
%%AT-COPY%% \xcd`at (p;F)S` are copied to \xcd`p`, as follows.

The values mentioned in \xcd`S` are copied to place \xcd`p` by \xcd`at(p)S` as follows.

First, the original-expressions are evaluated to give a vector of X10 values.
Consider the graph of all values reachable from these values (except for 
\xcd`transient` fields 
(\Sref{sect:transient}, \xcd`GlobalRef`s (\Sref{GlobalRef}); also custom
serialization (\Sref{sect:ser+deser} may alter this behavior)). 

Second this graph is {\em
serialized} into a buffer and transmitted to place \xcd`q`.  Third,
the vector of X10 values is 
re-created at \xcd`q` 
by deserializing the buffer at
\xcd`q`. Fourth, \xcd`S` is executed at \xcd`q`, in an environment in
which each variable \xcd`v` declared in \xcd`F` 
refers to the corresponding deserialized value.  

Note that since values accessed across an \xcd`at` boundary are
copied, the programmer may wish to adopt the discipline that either
variables accessed across an \xcd`at` boundary  contain only structs 
or stateless objects, or the methods invoked on them do not access any
mutable state on the objects. Otherwise the programmer has to ensure
that side effects are made to the correct copy of the object. For this
the struct \xcd`x10.lang.GlobalRef[T]` is often useful.


\subsubsection{Serialization and deserialization.}
\label{sect:ser+deser}
\index{transient}
\index{field!transient}
The X10 runtime provides a default mechanism for
serializing/deserializing an object graph with a given set of roots.
This mechanism may be overridden by the programmer on a per class or
struct basis as described in the API documentation for
\xcd`x10.io.CustomSerialization`.  
The default mechanism performs a
deep copy of the object graph (that is, it copies the object or struct
and, recursively, the values contained in its fields), but does not
traverse or copy \xcd`transient` fields. \xcd`transient` fields are omitted from the
serialized data.   On deserialization, \xcd`transient` fields are initialized
with their default values (\Sref{DefaultValues}).    The types of
\xcd`transient` fields must therefore have default values.

The default serialization/deserialization mechanism will not (modulo
error conditions like \xcd`OutOfMemoryError`) throw any exceptions. However,
user code running during serialization/deserialization via
\xcd`CustomSerialization` may raise exceptions.  These exceptions are
handled like any other exception raised during the execution of an X10
activity.  However, due to the special treatment of \xcd`at (p) async S` 
(\Sref{sect:AtStatement}) any exception raised during
deserialization will be handled as if it was raised by \xcd`async S`,
not by the \xcd`at` statement itself.


A struct \xcd`s` of type \xcd`x10.lang.GlobalRef[T]` \ref{GlobalRef}
is serialized as a unique global reference to its contained object
\xcd`o` (of type \xcd`T`).  Please see the documentation
of \xcd`x10.lang.GlobalRef[T]` for more details.



\subsection{{\tt at} and Activities}
%%AT-COPY%% \xcd`at(p;F)S` 
\xcd`at(p)S` 
does {\em not} start a new activity.  It should be thought of as
transporting the current activity to \xcd`p`, running \xcd`S` there, and then
transporting it back.  \xcd`async` is the only construct in the
language that starts a new activity. In different contexts, each one
of the following makes sense:
%%AT-COPY%% (1)~\xcd`async at(p;F) S` 
(1)~\xcd`async at(p) S` 
(spawn an activity locally to execute \xcd`S` at
\xcd`p`; here \xcd`p` is evaluated by the spawned activity) , 
%%AT-COPY%% (2)~\xcd`at(p;F) async S` 
(2)~\xcd`at(p) async S` 
(evaluate \xcd`p` and then at \xcd`p` spawn an
activity to execute \xcd`S`), and,
%%AT-COPY%% (3)~\xcd`async at(p;F) async S`. 
(3)~\xcd`async at(p) async S`. 
%%AT-COPY%% In most cases, \xcd`at(p;F) async S` is preferred to
%%\xcd`async at(p;F)`, since In most cases, \xcd`at(p) async S` is
preferred to \xcd`async at(p) S`, since the former form enables a more
efficient runtime implementation.  In the first case, the expression
\xcd`p` is evaluated synchronously by the current activity and then a
single remote async is spawned.  In the second case, \xcd`p` is
semantically required to be evaluated asynchronously with the parent
async as it is contained in the body of an async.  Therefore, if the
compiler cannot prove that "async at (p)" can be safely rewritten into
"at (p) async", a first local async is spawned to evaluate \xcd`p`
then a remote async is spawned to evaluate \xcd`S`.

Since 
%%AT-COPY%% \Xcd{at(p;F) S} 
\Xcd{at(p) S} 
does not start a new activity, 
\xcd`S` may contain constructs which only make sense
within a single activity.  
For example, 
\begin{xten}
    for(x in globalRefsToThings) 
      if (at(x.home) x().isNice()) 
        return x();
\end{xten}
returns the first nice thing in a collection.   If we had used 
\xcd`async at(x.home)`, this would not be allowed; 
you can't \xcd`return` from an
\xcd`async`. 

\limitation{
X10 does not currently allow {\tt break}, {\tt continue}, or {\tt return}
to exit from an {\tt at}.
}



\subsection{Copying from {\tt at} }
\index{at!copying}

%%AT-COPY%% \xcd`at(p;F)S` copies data as specified by \xcd`F`, and sends it
\xcd`at(p)S` copies data required in \xcd`S`, and sends it
to place \xcd`p`, before executing \xcd`S` there. The only things that are not
copied are values only reachable through \xcd`GlobalRef`s and \xcd`transient`
fields, and data omitted by custom serialization.    
%%AT-COPY%% Several choices of copy specifier use the same identifier for the original
%%AT-COPY%% variable outside of 
%%AT-COPY%% \xcd`at(p)S` 
%%AT-COPY%% and its copy inside of \xcd`S`.  
%%AT-COPY%% 

\begin{ex}
%%AT-COPY%% 
%%AT-COPY%% %~~gen ^^^ Places_implicit_copy_from_at_example_1
%%AT-COPY%% % package Places.implicitcopyfromat;
%%AT-COPY%% % class Example {
%%AT-COPY%% % static def example() {
%%AT-COPY%% % 
%%AT-COPY%% %~~vis
%%AT-COPY%% \begin{xten}
%%AT-COPY%% val c = new Cell[Long](9); // (1)
%%AT-COPY%% at (here;c) {             // (2)
%%AT-COPY%%    assert(c() == 9);      // (3)
%%AT-COPY%%    c.set(8);              // (4)
%%AT-COPY%%    assert(c() == 8);      // (5)
%%AT-COPY%% }
%%AT-COPY%% assert(c() == 9);         // (6)
%%AT-COPY%% \end{xten}
%%AT-COPY%% %~~siv
%%AT-COPY%% %}}
%%AT-COPY%% % class Hook{ def run() { Example.example(); return true; } }
%%AT-COPY%% %~~neg
%%AT-COPY%% 

%~~gen ^^^ Places_implicit_copy_from_at_example_1
% package Places.implicitcopyfromat;
% class Example {
% static def example() {
% 
%~~vis
\begin{xten}
val c = new Cell[Long](9); // (1)
at (here) {               // (2) 
   assert(c() == 9);      // (3)
   c.set(8);              // (4)
   assert(c() == 8);      // (5)
}
assert(c() == 9);         // (6)
\end{xten}
%~~siv
%}}
% class Hook{ def run() { Example.example(); return true; } }
%~~neg


The \xcd`at` statement copies the \xcd`Cell` and its contents.  
After \xcd`(1)`, \xcd`c` is a \xcd`Cell` containing 9; call that cell {$c_1$}
At \xcd`(2)`, that cell is copied, resulting in another cell {$c_2$} whose
contents are also 9, as tested at \xcd`(3)`.
(Note that the copying behavior of \xcd`at` happens {\em even when the
destination place is the same as the starting place}--- even with
\xcd`at(here)`.)
At \xcd`(4)`, the contents of {$c_2$} are changed to 8, as confirmed at \xcd`(5)`; the contents of
{$c_1$} are of course untouched.    Finally, at \xcd`(c)`, outside the scope
of the \xcd`at` started at line \xcd`(2)`, \xcd`c` refers to its original
value {$c_1$} rather than the copy {$c_2$}.  
\end{ex}

The \xcd`at` statement induces a {\em deep copy}.  Not only does it copy the
values of variables, it copies values that they refer to through zero or more
levels of reference.  Structures are preserved as well: if two fields
\xcd`x.f` and \xcd`x.g` refer to the same object {$o_1$} in the original, then
\xcd`x.f` and \xcd`x.g` will both refer to the same object {$o_2$} in the
copy.  

\begin{ex}
In the following variation of the preceding example,
\xcd`a`'s original value {$a_1$} is a rail with two references to the same
\xcd`Cell[Long]` {$c_1$}.  The fact that {$a_1(0)$} and {$a_1(1)$} are both
identical to {$c_1$} is demonstrated in \xcd`(A)`-\xcd`(C)`, as {$a_1(0)$} is modified
and {$a_1(1)$} is observed to change.  In \xcd`(D)`-\xcd`(F)`, the copy
{$a_2$} is tested in the same way, showing that {$a_2(0)$} and {$a_2(1)$} both
refer to the same \xcd`Cell[Long]` {$c_2$}.  However, the test at \xcd`(G)`
shows that {$c_2$} is a different cell from {$c_1$}, because changes to
{$c_2$} did not propagate to {$c_1$}.  

%%AT-COPY%% %~~gen ^^^ PlacesAtCopy
%%AT-COPY%% %package Places.AtCopy2;
%%AT-COPY%% %class example {
%%AT-COPY%% %static def Example() {
%%AT-COPY%% %
%%AT-COPY%% %~~vis
%%AT-COPY%% \begin{xten}
%%AT-COPY%% val c = new Cell[Long](5);
%%AT-COPY%% val a : Rail[Cell[Long]] = [c,c as Cell[Long]];
%%AT-COPY%% assert(a(0)() == 5 && a(1)() == 5);     // (A)
%%AT-COPY%% c.set(6);                               // (B)
%%AT-COPY%% assert(a(0)() == 6 && a(1)() == 6);     // (C)
%%AT-COPY%% at(here;a) {
%%AT-COPY%%   assert(a(0)() == 6 && a(1)() == 6);   // (D)
%%AT-COPY%%   c.set(7);                             // (E)
%%AT-COPY%%   assert(a(0)() == 7 && a(1)() == 7);   // (F)
%%AT-COPY%% }
%%AT-COPY%% assert(a(0)() == 6 && a(1)() == 6);     // (G)
%%AT-COPY%% \end{xten}
%%AT-COPY%% %~~siv
%%AT-COPY%% %}}
%%AT-COPY%% %class Hook{ def run() { example.Example(); return true; } }
%%AT-COPY%% %~~neg

%~~gen ^^^ PlacesAtCopy
%package Places.AtCopy2;
%class example {
%static def Example() {
%
%~~vis
\begin{xten}
val c = new Cell[Long](5);
val a : Rail[Cell[Long]] = [c,c as Cell[Long]];
assert(a(0)() == 5 && a(1)() == 5);     // (A)
c.set(6);                               // (B)
assert(a(0)() == 6 && a(1)() == 6);     // (C)
at(here) {
  assert(a(0)() == 6 && a(1)() == 6);   // (D)
  c.set(7);                             // (E)
  assert(a(0)() == 7 && a(1)() == 7);   // (F)
}
assert(a(0)() == 6 && a(1)() == 6);     // (G)
\end{xten}
%~~siv
%}}
%class Hook{ def run() { example.Example(); return true; } }
%~~neg


\end{ex}

\subsection{Copying and Transient Fields}
\label{sect:transient}
\index{at!transient fields and}
\index{transient}
\index{field!transient}

Recall that fields of classes and structs marked \xcd`transient` are not copied by
\xcd`at`.  Instead, they are set to the default values for their types. Types
that do not have default values cannot be used in \xcd`transient` fields.

\begin{ex}
Every \xcd`Trans` object has an \xcd`a`-field equal
to 1.  However, despite the initializer on the \xcd`b` field, it is not the
case that every \xcd`Trans` has \xcd`b==2`.  Since \xcd`b` is \xcd`transient`,
when the \xcd`Trans` value \xcd`this` is copied at \xcd`at(here){...}` in
\xcd`example()`, its \xcd`b` field is not copied, and the default value for an
\xcd`Long`, 0, is used instead.  
Note that we could not make a transient field \xcd`c : Long{c != 0}`, since the
type has no default value, and copying would in fact set it to zero.

%%AT-COPY%% %~~gen ^^^ Places40
%%AT-COPY%% %package Places_transient_a;
%%AT-COPY%% % 
%%AT-COPY%% %~~vis
%%AT-COPY%% \begin{xten}
%%AT-COPY%% class Trans {
%%AT-COPY%%    val a : Long = 1;
%%AT-COPY%%    transient val b : Long = 2;
%%AT-COPY%%    //ERROR transient val c : Long{c != 0} = 3;
%%AT-COPY%%    def example() {
%%AT-COPY%%      assert(a == 1 && b == 2);
%%AT-COPY%%      at(here;a) {
%%AT-COPY%%         assert(a == 1 && b == 0);
%%AT-COPY%%      }
%%AT-COPY%%    }
%%AT-COPY%% }
%%AT-COPY%% \end{xten}
%%AT-COPY%% %~~siv
%%AT-COPY%% %class Hook{ def run() { (new Trans()).example(); return true; } }
%%AT-COPY%% %~~neg

%~~gen ^^^ Places40
%package Places_transient_a;
% 
%~~vis
\begin{xten}
class Trans {
   val a : Long = 1;
   transient val b : Long = 2;
   //ERROR: transient val c : Long{c != 0} = 3;
   def example() {
     assert(a == 1 && b == 2);
     at(here) {
        assert(a == 1 && b == 0);
     }
   }
}
\end{xten}
%~~siv
%class Hook{ def run() { (new Trans()).example(); return true; } }
%~~neg



\end{ex}

\subsection{Copying and GlobalRef}
\label{GlobalRef}
\index{at!GlobalRef}
\index{at!blocking copying}

%%The other barrier to the potentially copious copying behavior of \xcd`at`
%%is the \xcd`GlobalRef` struct.  
A \xcd`GlobalRef[T]` (say \xcd`g`) contains a reference to
a value \xcd`v` of type \xcd`T`, in a form which can be transmitted, and a \xcd`Place`
\xcd`g.home` indicating where the value lives. When a 
\xcd`GlobalRef` is serialized an opaque, globally unique handle to
\xcd`v` is created.  

\begin{ex}The following example does not copy the value \xcd`huge`.  However, \xcd`huge`
would have been copied if it had been put into a \xcd`Cell`, or simply used
directly. 

%%AT-COPY%% %~~gen ^^^ Places50
%%AT-COPY%% %package Places.copyingblockingwithglobref;
%%AT-COPY%% % class GR {
%%AT-COPY%% %  static def use(Any){}
%%AT-COPY%% %  static def example() {
%%AT-COPY%% % 
%%AT-COPY%% %~~vis
%%AT-COPY%% \begin{xten}
%%AT-COPY%% val huge = "A potentially big thing";
%%AT-COPY%% val href = GlobalRef(huge);
%%AT-COPY%% at (here;href) {
%%AT-COPY%%    use(href);
%%AT-COPY%%   }
%%AT-COPY%% }
%%AT-COPY%% \end{xten}
%%AT-COPY%% %~~siv
%%AT-COPY%% %}
%%AT-COPY%% % class Hook{ def run() { GR.example(); return true; } }
%%AT-COPY%% %~~neg

%~~gen ^^^ Places50
%package Places.copyingblockingwithglobref;
% class GR {
%  static def use(Any){}
%  static def example() {
% 
%~~vis
\begin{xten}
val huge = "A potentially big thing";
val href = GlobalRef(huge);
at (here) {
   use(href);
  }
}
\end{xten}
%~~siv
%}
% class Hook{ def run() { GR.example(); return true; } }
%~~neg


\end{ex}

Values protected in \xcd`GlobalRef`s can be retrieved by the application
%~~exp~~`~~`~~ g:GlobalRef[Any]{here == g.home}~~ ^^^Places4e7q
operation \xcd`g()`.  \xcd`g()` is guarded; it can 
only be called when \xcd`g.home == here`.  If you  want to do anything other
than pass a global reference around or compare two of them for equality, you
need to placeshift back to the home place of the reference, often with
\xcd`at(g.home)`.   

\begin{ex}The following program, for reasons best known to the programmer,
modifies the 
command-line argument array.

%%AT-COPY%% 
%%AT-COPY%% %~~gen ^^^ Places60
%%AT-COPY%% % package Places.Atsome.Globref2;
%%AT-COPY%% % class GR2 {
%%AT-COPY%% % 
%%AT-COPY%% %~~vis
%%AT-COPY%% \begin{xten}
%%AT-COPY%%   public static def main(argv:Rail[String]) {
%%AT-COPY%%     val argref = GlobalRef[Rail[String]](argv);
%%AT-COPY%%     at(here.next(); argref) 
%%AT-COPY%%         use(argref);
%%AT-COPY%%   }
%%AT-COPY%%   static def use(argref : GlobalRef[Rail[String]]) {
%%AT-COPY%%     at(argref.home; argref) {
%%AT-COPY%%       val argv = argref();
%%AT-COPY%%       argv(0) = "Hi!";
%%AT-COPY%%     }
%%AT-COPY%%   }
%%AT-COPY%% \end{xten}
%%AT-COPY%% %~~siv
%%AT-COPY%% %} 
%%AT-COPY%% % class Hook{ def run() { GR2.main(["what, me weasel?" as String]); return true; }}
%%AT-COPY%% %~~neg
%%AT-COPY%% 

%~~gen ^^^ Places60
% package Places.Atsome.Globref2;
% class GR2 {
% 
%~~vis
\begin{xten}
  public static def main(argv:Rail[String]) {
    val argref = GlobalRef[Rail[String]](argv);
    at(here.next()) 
        use(argref);
  }
  static def use(argref : GlobalRef[Rail[String]]) {
    at(argref) {
      val argv = argref();
      argv(0) = "Hi!";
    }
  }
\end{xten}
%~~siv
%} 
% class Hook{ def run() { GR2.main(["what, me weasel?" as String]); return true; }}
%~~neg

\end{ex}

There is an implicit coercion from \xcd`GlobalRef[T]` to \xcd`Place`, so
\xcd`at(argref)S` goes to \xcd`argref.home`.  


\subsection{Warnings about \xcd`at`}
There are two dangers involved with \xcd`at`: 
\begin{itemize}
\item Careless use of \xcd`at` can result in copying and transmission
of very large data structures.  
%%AT-COPY%% This is particularly an issue with the blanket
%%AT-COPY%% \xcd`at` statement, \xcd`at(p)S`, where everything used in \xcd`S` is copied.  
In particular, it is very easy to capture
\xcd`this` -- a field reference will do it -- and accidentally copy everything
that \xcd`this` refers to, which can be very large.  A disciplined use of copy
specifiers to make explicit just what gets copied can ameliorate this issue.

\item As seen in the examples above, a local variable reference
  \xcd`x` may refer to different objects in different nested \xcd`at`
  scopes. The programmer must either ensure that a variable accessed
  across an \xcd`at` boundary has no mutable state or be prepared to
  reason about which copy gets modified.   A disciplined use of copy specifiers to give
  different names to variables can ameliorate this concern.
\end{itemize}


%%AT-COPY%% \section{{\tt athome}: Returning Values from {\tt at}-Blocks}
%%AT-COPY%% \label{sect:athome}
%%AT-COPY%% \index{athome}
%%AT-COPY%% 
%%AT-COPY%% The 
%%AT-COPY%% \xcd`at(p;F)S` 
%%AT-COPY%% construct renders external variables unavailable within
%%AT-COPY%% \xcd`S`.  However, it is often useful to transmit values back from \xcd`S`,
%%AT-COPY%% and store them in external variables. 
%%AT-COPY%% 
%%AT-COPY%% The \xcd`athome(V;F)S` construct provides
%%AT-COPY%% this ability.  \xcd`V` is a list of variables, which must all be defined at
%%AT-COPY%% the same place.  \xcd`athome(V;F)S` goes to the place where the variables are
%%AT-COPY%% defined, copying \xcd`F` as for \xcd`at(p;F)S`, and executes \xcd`S` ---
%%AT-COPY%% allowing reading, assignment and initialization of the listed variables in
%%AT-COPY%% \xcd`V`. 
%%AT-COPY%% 
%%AT-COPY%% \xcd`V`, the list of variables, may include one or more variables.  It is a
%%AT-COPY%% static error if X10 cannot determine that all the variables in the list are
%%AT-COPY%% defined at the same place.
%%AT-COPY%% 
%%AT-COPY%% 
%%AT-COPY%% 
%%AT-COPY%% 
%%AT-COPY%% \begin{ex}
%%AT-COPY%% \xcd`athome` allows returning multiple pieces of information from an
%%AT-COPY%% \xcd`at`-statement.  In the following example, we return two data: 
%%AT-COPY%% one as a \xcd`val` named \xcd`square`, and the other as an addition in to a
%%AT-COPY%% partially-computed polynomial named \xcd`poly`.  
%%AT-COPY%% %~~gen ^^^ Places5f9g
%%AT-COPY%% % package Places5f9g;
%%AT-COPY%% % % KNOWNFAIL-at
%%AT-COPY%% % class Example { 
%%AT-COPY%% %~~vis
%%AT-COPY%% \begin{xten}
%%AT-COPY%% static def example(a: Long, mathProc: Place) { 
%%AT-COPY%%   val square : Long;
%%AT-COPY%%   var poly : Long = 1 + a; // will be 1+a+a*a
%%AT-COPY%%   at(mathProc; a) {
%%AT-COPY%%     val sq = a*a; 
%%AT-COPY%%     athome(square, poly; sq) {
%%AT-COPY%%        square = sq;  // initialization
%%AT-COPY%%        poly += sq;   // read and update
%%AT-COPY%%     }
%%AT-COPY%%   return [square, poly];
%%AT-COPY%%   }
%%AT-COPY%% \end{xten}
%%AT-COPY%% %~~siv
%%AT-COPY%% %}}
%%AT-COPY%% % class Hook { def run() { 
%%AT-COPY%% %   val e = example(2, here);
%%AT-COPY%% %   assert e(0) == 4 && e(1) == 7;
%%AT-COPY%% %   return true;
%%AT-COPY%% % }} 
%%AT-COPY%% %~~neg
%%AT-COPY%% \end{ex}
%%AT-COPY%% 
%%AT-COPY%% The abbreviated forms 
%%AT-COPY%% \xcd`athome (*) S` and 
%%AT-COPY%% \xcd`athome S` 
%%AT-COPY%% allow a block of assignments without specifying the variables being assigned
%%AT-COPY%% to, which is convenient for a small set of assignments. 
%%AT-COPY%% They 
%%AT-COPY%% are both equivalent to \xcd`athome(V;F)S`,
%%AT-COPY%% where: 
%%AT-COPY%% \begin{itemize}
%%AT-COPY%% \item \xcd`V` is the list of all variables appearing on the left-hand side of
%%AT-COPY%%       an assignment or update statement in \xcd`S`, excluding those which
%%AT-COPY%%       appear inside the body of an \xcd`at` or \xcd`athome` statement in \xcd`S`;
%%AT-COPY%% \item \xcd`F` is the same as for \xcd`at(p)S` (\Sref{sect:copy-spec})
%%AT-COPY%% \end{itemize}
%%AT-COPY%% 
%%AT-COPY%% 
%%AT-COPY%% \begin{ex}
%%AT-COPY%% 
%%AT-COPY%% Much as the blanket \xcd`at` construct \xcd`at(p)S` is convenient for
%%AT-COPY%% executing a small code body at another place, the blanket \xcd`athome`
%%AT-COPY%% construct \xcd`athome(*) S` 
%%AT-COPY%% (which may be written as simply \xcd`athome S`)
%%AT-COPY%% is convenient for returning a result or two.   The
%%AT-COPY%% preceding example could have been written using blanket statements.
%%AT-COPY%% 
%%AT-COPY%% %~~gen ^^^ Places5f9gblanket
%%AT-COPY%% % package Places5f9gblanket;
%%AT-COPY%% % class Example { 
%%AT-COPY%% % KNOWNFAIL-at
%%AT-COPY%% %~~vis
%%AT-COPY%% \begin{xten}
%%AT-COPY%% static def example(a: Long, mathProc: Place) { 
%%AT-COPY%%   val square : Long;
%%AT-COPY%%   var poly : Long = 1 + a; // will be 1+a+a*a
%%AT-COPY%%   at(mathProc) {
%%AT-COPY%%     val sq = a*a; 
%%AT-COPY%%     athome {
%%AT-COPY%%        square = sq;  // initialization
%%AT-COPY%%        poly += sq;   // read and update
%%AT-COPY%%     }
%%AT-COPY%%   return [square, poly];
%%AT-COPY%%   }
%%AT-COPY%% \end{xten}
%%AT-COPY%% %~~siv
%%AT-COPY%% %}}
%%AT-COPY%% % class Hook { def run() { 
%%AT-COPY%% %   val e = example(2, here);
%%AT-COPY%% %   assert e(0) == 4 && e(1) == 7;
%%AT-COPY%% %   return true;
%%AT-COPY%% % }} 
%%AT-COPY%% %~~neg
%%AT-COPY%% \end{ex}
%%AT-COPY%% 
%%AT-COPY%% {\bf Design:} It is not fundamentally essential to distinguish \xcd`at` from
%%AT-COPY%% \xcd`athome`.  \xcd`at(p;F)S` could allow writing to variables whose homes are
%%AT-COPY%% known at compile-time to be equal to \xcd`p`.  Indeed, in earlier versions of
%%AT-COPY%% X10, it did so.    This required an idiom in which programmers had to manage
%%AT-COPY%% the home locations of variables directly, and keep track of which home
%%AT-COPY%% location corresponded to which variable.  The \xcd`athome` construct makes
%%AT-COPY%% this idiom more convenient. 
	
\chapter{Activities}\label{XtenActivities}

An \Xten{} computation may have many concurrent {\em activities} ``in
flight'' at any give time. We use the term activity to denote a
dynamic execution instance of a piece of code (with references to
data). An activity is intended to execute in parallel with other
activities. An activity may be thought of as a very light-weight
thread.  In \XtenCurrVer{}, an activity may not be interrupted,
suspended or resumed as the result of actions taken by any other
activity.

An activity is spawned in a given place and stays in that place for
its lifetime.  An activity may be {\em running}, {\em blocked} on some
condition or {\em terminated}. When the statement associated with an
activity terminates normally, the activity terminates normally; when
it terminates abruptly with some reason $R$, the activity terminates
with the same reason (\Sref{ExceptionModel}).

An activity may be long-running and may invoke recursive methods (thus
may have a stack associated with it). On the other hand, an activity
may be short-running, involving a fine-grained operation such as a
single read or write.

% An activity may have an {\em activitylocal} heap accessible only
%to the activity. 

An activity may asynchronously and in parallel launch activities at
other places.

\Xten{} distinguishes between {\em local} termination and {\em global}
termination of a statement. The execution of a statement by an
activity is said to terminate locally when the activity has finished
all its computation related to that statement. (For instance the
creation of an asynchronous activity terminates locally when the
activity has been created.)  It is said to terminate globally when it
has terminated locally and all activities that it may have spawned at
any place (if any) have, recursively, terminated globally.

An \Xten{} computation is initiated as a single activity from the
command line. This activity is the {\em root activity}\index{root
activity} for the entire computation. The entire computation
terminates when (and only when) this activity globally
terminates. Thus \Xten{} does not permit the creation of so called
``daemon threads''---threads that outlive the lifetime of the root
activity. We say that an \Xten{} computation is {\em rooted}
(\Sref{initial-computation}).

\futureext{ We may permit the initial activity to be a daemon activity
to permit reactive computations, such as webservers, that may not
terminate.}

\section{The \Xten{} rooted exception model}
\label{ExceptionModel}
\index{Exception!model}

The rooted nature of \Xten{} computations permits the definition of a
{\em rooted} exception model. In multi-threaded programming languages
there is a natural parent-child relationship between a thread and a
thread that it spawns. Typically the parent thread continues execution
in parallel with the child thread. Therefore the parent thread cannot
serve to catch any exceptions thrown by the child thread. 

The presence of a root activity permits \Xten{} to adopt a different
model.  In any state of the computation, say that an activity $A$ is
{\em a root of} an activity $B$ if $A$ is an ancestor of $B$ and $A$
is suspended at a statement (such as the \xcd"finish" statement
\Sref{finish}) awaiting the termination of $B$ (and possibly other
activities). For every \Xten{} computation, the
\emph{root-of} relation
is guaranteed to be a tree. The root of the tree is the root activity
of the entire computation. If $A$ is the nearest root of $B$, the path
from $A$ to $B$ is called the {\em activation path} for the
activity.\footnote{Note that depending on the state of the computation
the activation path may traverse activities that are running,
suspended or terminated.}

We may now state the exception model for \Xten.  An uncaught exception
propagates up the activation path to its nearest root activity, where
it may be handled locally or propagated up the \emph{root-of} tree when
the activity terminates (based on the semantics of the statement being
executed by the activity).\footnote{In \XtenCurrVer{} the \xcd"finish"
statement is the only statement that marks its activity as a root
activity. Future versions of the language may introduce more such
statements.}  Thus, unlike concurrent languages such as \java{}, no
exception is ``thrown on the floor''.

\section{Spawning an activity}\label{AsynchronousActivity}\label{AsyncActivity}

Asynchronous activities serve as a single abstraction for supporting a
wide range of concurrency constructs such as message passing, threads,
DMA, streaming, data prefetching. (In general, asynchronous operations
are better suited for supporting scalability than synchronous
operations.)

An activity is created by executing the statement:

\begin{grammar}
Statement \: AsyncStatement \\
AsyncStatement \: \xcd"async" PlaceExpressionSingleList\opt Statement \\
PlaceExpressionSingleList \: \xcd"(" PlaceExpression \xcd")" \\
PlaceExpression \: Expression 
\end{grammar} 

The place expression \xcd"e" is expected to be of type \xcd"Place",
e.g., \xcd"here" or \xcd"d(p)" for some
distribution \xcd"d" and point \xcd"p" (\Sref{XtenPlaces}).  
If not, the compiler replaces
\xcd"e" with \xcd"e.home" if
\xcd"e" is of type \xcd"x10.lang.Object". Otherwise the compiler reports a type error. 

Note specifically that the expression \xcd"a(i)" when used as a place
expression may evaluate to \xcd"a(i).home", which may not be
the same place as \xcd"a.dist(i)". The programmer must be 
careful to choose the right expression, appropriate for the statement.
Accesses to \xcd"a(i)" within \grammarrule{Statement} should typically be guarded 
by the place expression \xcd"a.dist(i)".

In many cases the compiler may infer the unique place at which the
statement is to be executed by an analysis of the types of the
variables occurring in the statement. (The place must be such that the
statement can be executed safely, without generating a
\xcd"BadPlaceException".) In such cases the programmer may omit the
place designator; the compiler will throw an error if it cannot
determine the unique designated place.\footnote{\XtenCurrVer{} does
not specify a particular algorithm; this will be fixed in future
versions.}

An activity $A$ executes the statement \xcd"async (P) S" by launching
a new activity $B$ at the designated place, to execute the specified
statement. The statement terminates locally as soon as $B$ is
launched.  The activation path for $B$ is that of $A$, augmented with
information about the line number at which $B$ was spawned.  $B$
terminates normally when $S$ terminates normally.  It terminates
abruptly if $S$ throws an (uncaught) exception. The exception is
propagated to $A$ if $A$ is a root activity (see \Sref{finish}),
otherwise through $A$ to $A$'s root activity. Note that while an
activity is running, exceptions thrown by activities it has already
generated may propagate through it up to its root activity.

Multiple activities launched by a single activity at another place are
not ordered in any way. They are added to the pool of activities at
the target place and will be executed in sequence or in parallel based
on the local scheduler's decisions. If the programmer wishes to
sequence their execution s/he must use \Xten{} constructs, such as
clocks and \xcd"finish" to obtain the desired effect.  Further, the
\Xten{} implementations are not required to have fair schedulers,
though every implementation should make a best faith effort to ensure
that every activity eventually gets a chance to make forward progress.

\begin{staticrule*}
The statement in the body of an \xcd"async" is subject to the
restriction that it must be acceptable as the body of a \xcd"void"
method for an anonymous inner class declared at that point in the code,
which throws no checked exceptions. As such, it may reference
variables in lexically enclosing scopes (including \xcd"clock"
variables, \Sref{XtenClocks}) provided that such variables are
(implicitly or explicitly) \xcd"val".
\end{staticrule*}

\section{Place changes}\label{AtStatement}

An activity may change place using the \xcd"at" statement or
\xcd"at" expression:

\begin{grammar}
Statement \: AtStatement \\
AtStatement \: \xcd"at" PlaceExpressionSingleList Statement \\
Expression \: AtExpression \\
AtExpression \: \xcd"at" PlaceExpressionSingleList ClosureBody 
\end{grammar}

The statement \xcd"at (p) S" executes the statement \xcd"S"
synchronously at place \xcd"p".
The expression \xcd"at (p) E" executes the statement \xcd"E"
synchronously at place \xcd"p", returning the result to the
originating place.

\section{Finish}\index{finish}\label{finish}
The statement \xcd"finish S" converts global termination to local
termination and introduces a root activity. 

\begin{grammar}
Statement \: FinishStatement \\
FinishStatement \: \xcd"finish" Statement 
\end{grammar}

An activity $A$ executes \xcd"finish S" by executing \xcd"S".  The
execution of \xcd"S" may spawn other asynchronous activities (here or
at other places).  Uncaught exceptions thrown or propagated by any
activity spawned by \xcd"S" are accumulated at \xcd"finish S".
\xcd"finish S" terminates locally when all activities spawned by
\xcd"S" terminate globally (either abruptly or normally). If \xcd"S"
terminates normally, then \xcd"finish S" terminates normally and $A$
continues execution with the next statement after \xcd"finish S".  If
\xcd"S" terminates abruptly, then \xcd"finish S" terminates abruptly
and throws a single exception, \Xcd{x10.lang.MultipleExceptions}
formed from the collection of exceptions accumulated at \xcd"finish S".

Thus a \xcd"finish S" statement serves as a collection point for
uncaught exceptions generated during the execution of \xcd"S".

Note that repeatedly \xcd"finish"ing a statement has no effect after
the first \xcd"finish": \xcd"finish finish S" is indistinguishable
from \xcd"finish S".

\paragraph{Interaction with clocks.}\label{sec:finish:clock-rule}
\xcd"finish S" interacts with clocks (\Sref{XtenClocks}). 

While executing \xcd"S", an activity must not spawn any \xcd"clocked"
asyncs. (Asyncs spawned during the execution of \xcd"S" may spawn
clocked asyncs.) A
\xcd"ClockUseException"\index{clock!ClockUseException} is thrown
if (and when) this condition is violated.

In \XtenCurrVer{} this condition is checked dynamically; future
versions of the language will introduce type qualifiers which permit
this condition to be checked statically.

\futureext{
The semantics of \xcd"finish S" is conjunctive; it terminates when all
the activities created during the execution of \xcd"S" (recursively)
terminate. In many situations (e.g., nondeterministic search) it is
natural to require a statement to terminate when any {\em one} of the
activities it has spawned succeeds. The other activities may then be
safely aborted. Future versions of the language may introduce a
\xcd"finishone S" construct to support such speculative or nondeterministic
computation.
}
%% Need an example here.

\section{Initial activity}\label{initial-computation}\index{initial activity}

An \Xten{} computation is initiated from the command line on the
presentation of a classname \xcd"C". The class must have a
\xcd"public static def main(a: array[String])" method, otherwise an
exception is thrown
and the computation terminates.  The single statement
\begin{xten}
finish async (Place.FIRST_PLACE) {
  C.main(s);
}
\end{xten} 
\noindent is executed where \xcd"s" is an array of strings created
from command line arguments. This single activity is the root activity
for the entire computation. (See \Sref{XtenPlaces} for a discussion of
places.)

%% Say something about configuration information? 

\section{Foreach statements}\index{\Xcd{foreach}}\label{foreach-section}


\begin{grammar}
Statement \: ForEachStatement \\
ForEachStatement \: 
      \xcd"foreach" \xcd"(" Formal \xcd"in" Expression \xcd")"
          Statement 
\end{grammar}


The \xcd"foreach" statement is a parallel version of the enhanced \xcd"for"
statement (\Sref{ForAllLoop}). \xcd`for(x in C)S` executes \xcd`S` {\em
  sequentially}, with everything happening \xcd`here`. \xcd`foreach(x in C)S`
executes \xcd`S` for each iteration of the loop {\em in parallel}, located at
\xcd`x.home`. It is thus equivalent to:
\begin{xten}
foreach (x in C)
  async at (x.home) S
\end{xten}

As a common and useful special case, \xcd`C` may be a \xcd`Dist` or an
\xcd`Array`.  For both of these, \xcd`foreach(x in C)S` is treated just like 
\xcd`foreach(x in C.region)S`.  \xcd`x` ranges over the \xcd`Point`s of the
region.  Each activity that \xcd`foreach` starts is located at \xcd`here` --
the same place that the \xcd`foreach` statement itself is executing.  (If you
want to start an activity at the place where the array element \xcd`C(p)` is
located, use \xcd`ateach` (\Sref{ateach-section}) instead of \xcd`foreach`.)

Exceptions thrown by \xcd`S`, like other exceptions in \xcd`async`s, are
propagated to the root activity of the \xcd`foreach`.  

%FOREACH%  An activity executes a \xcd"foreach" statement in a similar fashion
%FOREACH%  except that separate \xcd"async" activities are launched in parallel
%FOREACH%  in the local place of each object returned by the iteration.
%FOREACH%  The statement
%FOREACH%  terminates locally when all the activities have been spawned. It never
%FOREACH%  throws an exception, though exceptions thrown by the spawned
%FOREACH%  activities are propagated through to the root activity.
%FOREACH%  
%FOREACH%  In a common case, the
%FOREACH%  the collection is intended to be of type
%FOREACH%  \xcd"Region" and the formal parameter is of type \xcd"Point".  Expressions \xcd"e" of type \xcd"Dist" and
%FOREACH%  \xcd"Array" are also accepted, and treated as if they were \xcd"e.region".





\section{Ateach statements}\index{\Xcd{ateach}}\label{ateach-section}

\begin{grammar}
Statement \: AtEachStatement \\
AtEachStatement \:
      \xcd"ateach" \xcd"(" Formal \xcd"in" Expression \xcd")"
         Statement 
\end{grammar}

The \xcd"ateach" statement is similar to the \xcd"foreach"
statement, but it spawns activites at each place of a distribution. 
In \xcd`ateach(p in D) S`, 
\xcd`D` must be either of type \xcd"Dist" or of type
\xcd`Array[T]`, 
and \xcd`p` will be of type \xcd"Point".

This statement differs from \xcd"foreach" only in
that each activity is spawned at the place specified by the
distribution for the point. That is, if \xcd`D` is a \xcd`Dist`, 
\xcd"ateach(p in D) S" could be implemented as:
\begin{xten}
foreach (p in D.region) 
  async (D(p)) S
\end{xten}

However, the compiler may implement it more efficiently to avoid extraneous
communications.  In particular, it is recommended that \xcd`ateach(p in D)S`
be implemented as the following code, which coordinates with each place of
\xcd`D` just once, rather than once per element of \xcd`D` at that place: 
\begin{xten}
foreach (p in D.places()) at (p) {
    foreach (pt in D|here) {
        S
    }
}
\end{xten}

If \xcd`e` is an \xcd`Array[T]`, then \xcd`ateach (p in e)S` is identical to
\xcd`ateach(p in e.dist)S`; the iteration is over the array's underlying
distribution.   \xcd`ateach(p in A)dealWith(A(p));` is a common and generally
efficient idiom for working with the elements of an array.



\section{Futures}\label{XtenFutures}

\Xten{} provides syntactic support for {\em asynchronous expressions}, also
known as futures:

\begin{grammar}
Primary \: FutureExpression \\
FutureExpression \:
  \xcd"future" PlaceExpressionSingleList\opt ClosureBody
\end{grammar} 

Intuitively such an expression evaluates its body asynchronously at
the given place. The resulting value may be obtained from the future
returned by this expression, by using the \xcd"force" operation.

In more detail, in an expression \xcd"future (Q) e", the place
expression \xcd"Q" is treated as in an \xcd"async" statement. \xcd"e"
is an expression of some type \xcd"T". \xcd"e" may reference only
those variables in the enclosing lexical environment which are
declared to be \xcd"val".

If the type of \xcd"e" is \xcd"T" then the type of
\xcd"future (Q) e" is \xcd"Future[T]".  This 
type \xcd"Future[T]" is defined as if by:
\begin{xten}
package x10.lang;
public interface Future[T] implements () => T {
  global def forced(): Boolean;
  global def force(): T;
}
\end{xten}

Evaluation of \xcd"future (Q) e" terminates locally with the creation
of a value \xcd"f" of type \xcd"Future[T]".  This value may be
stored in objects, passed as arguments to methods, returned from
method invocation etc. 

At any point, the method \xcd"forced" may be invoked on \xcd"f". This
method returns without blocking, with the value \xcd"true" if the
asynchronous evaluation of \xcd"e" has terminated globally and with
the value \xcd"false" if it has not.

\xcd"Future[T]" is a subtype of the function type \xcd"() => T".
Invoking---\emph{forcing}---the future \xcd"f" blocks until the
asynchronous evaluation of \xcd"e" has terminated globally. If the
evaluation terminates successfully with value \xcd"v", then the method
invocation returns \xcd"v". If the evaluation terminates abruptly with
exception \xcd"z", then the method throws exception \xcd"z". Multiple
invocations of the function (by this or any other activity) do not
result in multiple evaluations of \xcd"e". The results of the first
evaluation are stored in the future \xcd"f" and used to respond to all
queries.


\begin{xten}
promise: Future[T] = future (a.dist(3)) a(3);
value: T = promise();
\end{xten}


\section{At expressions}

\begin{grammar}
Expression \: \xcd"at" \xcd"(" Expression \xcd")" Expression
\end{grammar}

An \Xcd{at} expression evaluates an expression synchronously at the
given place and returns its value. Note that expression evaluation may
spawn asynchronous activities. The \Xcd{at} expression will return
without waiting for those activities to terminate. That is, \Xcd{at}
does not have built-in \Xcd{finish} semantics.

\section{Shared variables}
\label{Shared}

{\bf Compiler Limitation: Shared variables are not currently implemented.}

A {\em shared local variable} is declared with the annotation \xcd"shared".
It can be accessed within any control construct in its scope, including
\Xcd{async}, \Xcd{at}, \Xcd{future} and closures.

Note that the lifetime of some of these constructs may outlast the
lifetime of the scope -- requiring the implementation to allocate them
outside the current stack frame.

\section{Atomic blocks}\label{AtomicBlocks}\index{atomic blocks}
Languages such as \java{} use low-level synchronization locks to allow
multiple interacting threads to coordinate the mutation of shared
data. \Xten{} eschews locks in favor of a very simple high-level
construct, the {\em atomic block}.

A programmer may use atomic blocks to guarantee that invariants of
shared data-structures are maintained even as they are being accessed
simultaneously by multiple activities running in the same place.

\subsection{Unconditional atomic blocks}
The simplest form of an atomic block is the {\em unconditional
atomic block}:

\begin{grammar}
Statement \: AtomicStatement \\
AtomicStatement \: \xcd"atomic"  Statement \\
MethodModifier \: \xcd"atomic" \\
\end{grammar}

For the sake of efficient implementation \XtenCurrVer{} requires
that the atomic block be {\em analyzable}, that is, the set of
locations that are read and written by the \grammarrule{BlockStatement} are
bounded and determined statically.\footnote{A static bound is a constant
that depends only on the program text, and is independent 
of any runtime parameters. }
The exact algorithm to be used by
the compiler to perform this analysis will be specified in future
versions of the language.
\tbd{}

Such a statement is executed by an activity as if in a single step
during which all other concurrent activities in the same place are
suspended. If execution of the statement may throw an exception, it is
the programmer's responsibility to wrap the atomic block within a
\xcd"try"/{\xcd"finally" clause and include undo code in the finally
clause. Thus the \xcd"atomic" statement only guarantees atomicity on
successful execution, not on a faulty execution.

%% A compiler is allowed to reorder two atomic blocks that have no
%%data-dependency between them, just as it may reorder any two
%%statements which have no data-dependencies between them. For the
%%purposes of data dependency analysis, an atomic block is deemed to
%%have read and written all data at a single program point, the
%%beginning of the atomic block.
%%%% I dont believe we need to say at some point in the atomic block.
%%
We allow methods of an object to be annotated with \xcd"atomic". Such
a method is taken to stand for a method whose body is wrapped within an
\xcd"atomic" statement.

Atomic blocks are closely related to non-blocking synchronization
constructs \cite{herlihy91waitfree}, and can be used to implement 
non-blocking concurrent algorithms.

\begin{staticrule*}
In \xcd"atomic S", \xcd"S" may include method calls,
conditionals, etc.

It may {\em not} include an \xcd"async" activity (such as creation
of a \Xcd{future}).

It may {\em not} include any statement that may potentially block at
runtime (e.g., \xcd"when", \xcd"force" operations, \xcd"next"
operations on clocks, \xcd"finish"). 

All locations accessed in an atomic block must statically satisfy the
{\em locality condition}: they must belong to the place of the current
activity.\index{locality condition}\label{LocalityCondition} 

\end{staticrule*}


The compiler checks for this condition by checking whether the statement
could be the body of a \xcd"void" method annotated with \xcd"safe" at
that point in the code (\Sref{SafeAnnotation}).

\paragraph{Consequences.}
Note an important property of an (unconditional) atomic block:

\begin{eqnarray}
 \mbox{\xcd"atomic \{s1; atomic s2\}"} &=& \mbox{\xcd"atomic \{s1; s2\}"}
\end{eqnarray}

Further, an atomic block will eventually terminate successfully or
thrown an exception; it may not introduce a deadlock.

\subsubsection{Example}

The following class method implements a (generic) compare and swap (CAS) operation:

\begin{xten}
// target defined in lexically enclosing environment.
public atomic def CAS(old: Object, new: Object): Boolean {
   if (target.equals(old)) {
     target = new;
     return true;
   }
   return false;
}
\end{xten}

\subsection{Conditional atomic blocks}

Conditional atomic blocks are of the form:

\begin{grammar}
Statement \:  WhenStatement \\
WhenStatement \:  \xcd"when" \xcd"(" Expression \xcd")" Statement \\
            \| WhenStatement \xcd"or" \xcd"(" Expression \xcd")" Statement 
\end{grammar}

In such a statement the one or more expressions are called {\em
guards} and must be \xcd"Boolean" expressions. The statements are the
corresponding {\em guarded statements}. The first pair of expression
and statement is called the {\em main clause} and the additional pairs
are called {\em auxiliary clauses}. A statement must have a main
clause and may have no auxiliary clauses.

An activity executing such a statement suspends until such time as any
one of the guards is true in the current state. In that state, the
statement corresponding to the first guard that is true is executed.
The checking of the guards and the execution of the corresponding
guarded statement is done atomically. 


\Xten{} does not guarantee that a conditional atomic block
will execute if its condition holds only intermmittently. For, based on
the vagaries of the scheduler, the precise instant at which a
condition holds may be missed. Therefore the programmer is advised to
ensure that conditions being tested by conditional atomic blocks are
eventually stable, i.e., they will continue to hold until the block
executes (the action in the body of the block may cause the condition
to not hold any more).

%%Fourth, \Xten{} does not guarantees only {\em weak fairness} when executing
%%conditional atomic blocks. Let $c$ be the guard of some conditional
%%atomic block $A$. $A$ is required to make forward progress only if
%%$c$ is {\em eventually stable}. That is, any execution $s_1, s_2,
%%\ldots$ of the program is considered illegal only if there is a $j$
%%such that $c$ holds in all states $s_k$ for $k > j$ and in which $A$
%%does not execute. Specifically, if the system executes in such a way
%%that $c$ holds only intermmitently (that is, for some state in which
%%$c$ holds there is always a later state in which $c$ does not hold),
%%$A$ is not required to be executed (though it may be executed).

\begin{rationale}
The guarantee provided by \xcd"wait"/\xcd"notify" in \java{} is no
stronger. Indeed conditional atomic blocks may be thought of as a
replacement for \java's wait/notify functionality.
\end{rationale} 

We note two common abbreviations. The statement \xcd"when (true) S" is
behaviorally identical to \xcd"atomic S": it never suspends. Second,
\xcd"when (c) {;}" may be abbreviated to \xcd"await(c);"---it
simply indicates that the thread must await the occurrence of a
certain condition before proceeding.  Finally note that a \xcd"when"
statement with multiple branches is behaviorally identical to a
\xcd"when" statement with a single branch that checks the disjunction of
the condition of each branch, and whose body contains an
\xcd"if"/\xcd"then"/\xcd"else" checking each of the branch conditions.

\begin{staticrule*}
For the sake of efficient implementation certain restrictions are
placed on the guards and statements in a conditional atomic
block. 
\end{staticrule*}

Guards are required not to have side-effects, not to spawn
asynchronous activities and to have a statically determinable upper
bound on their execution. These conditions are expected to be checked
statically by the compiler.

The body of a \xcd"when" statement must satisfy the conditions
for the body of an \xcd"atomic" block.
%Second, as for unconditional atomic blocks, the set of memory
%locations accessed by a guarded statements are required to be bounded
%and statically analyzable.

Note that this implies that guarded statements are required to be {\em
flat}, that is, they may not contain conditional atomic blocks. (The
implementation of nested conditional atomic blocks may require
sophisticated operational techniques such as rollbacks.)

\paragraph{Sample usage.} 
There are many ways to ensure that a guard is eventually
stable. Typically the set of activities are divided into those that
may enable a condition and those that are blocked on the
condition. Then it is sufficient to require that the threads that may
enable a condition do not disable it once it is enabled. Instead the
condition may be disabled in a guarded statement guarded by the
condition. This will ensure forward progress, given the weak-fairness
guarantee.

\begin{example}
The following class shows how to implement a bounded buffer of size
$1$ in \Xten{} for repeated communication between a sender and a
receiver.

\begin{xten}
class OneBuffer {
  datum: Object = null;
  filled: Boolean = false;
  public def send(v: Object) {
    when (!filled) {
      this.datum = v;
      this.filled = true;
    }
  }
  public def receive(): Object {
    when (filled) {
      v: Object = datum;
      datum = null;
      filled = false;
      return v;
    }
  }
}
\end{xten}
\end{example}

\eat{
\paragraph{Implementing a future with a latch.}\label{future-imp}
The following class shows how to implement a {\em latch}. A latch is
an object that is initially created in a state called the {\em
unlatched} state. During its lifetime it may transition once to a {\em
forced} state. Once forced, it stays forced for its lifetime. The
latch may be queried to determine if it is forced, and if so, an
associated value may be retrieved. Below, we will consider a latch set
when some activity invokes a \xcd"setValue" method on it. This method
provides two values, a normal value and an exceptional value. The
method \xcd"force" blocks until the latch is set. If an exceptional
value was specified when the latch was set, that value is thrown on
any attempt to read the latch. Otherwise the normal value is returned.

\begin{xten}
public interface Future[T] {
   def forced(): Boolean;
   def apply(): T;
}
public class Latch implements Future {
  protected var forced: Boolean = false;
  protected var result: Box[T] = null;
  protected var z: Box[Exception] = null;

  public atomic def setValue(val: T): Boolean {
    return setValue(val, null);
  }
  public atomic def setValue(z: Exception): Boolean {
    return setValue(null, z);
  }
  public atomic def setValue(val: T,
                             z: Exception): Boolean {
    if (forced) return false;
    // these assignment happens only once.
    this.result = val;
    this.z = z;
    this.forced = true;
    return true;
  }
  public atomic def forced(): Boolean {
    return forced;
  }
  public def apply(): T {
    when (forced) {
      if (z != null) throw z;
      return result to T;
    }
  }
}
\end{xten}

Latches, \xcd"aync" operations and \xcd"finish" operations may be used
to implement futures as follows. The expression \xcd"future(P) e"
can be translated to:
\begin{xten}
(() => {
    L: Latch = new Latch();
    async (P) {
      X: Object;
      try {
        finish X = e;
        async (L) {
          L.setValue(X); 
        }
      }
      catch (Z: Exception) {
        async (L) {
          L.setValue(Z);
        }
      }
    }
    return L;
  })()
\end{xten}

Here we assume that \xcd"RunnableLatch" is an interface defined by:
\begin{xten} 
public interface RunnableLatch {
  def run(): Latch;
}
\end{xten}

We use the standard \java{} idiom of wrapping the core translation
inside an inner class definition/method invocation pair (i.e.,
\xcd"new RunnableLatch() {....}.run()") so as to keep the resulting
expression completely self-contained, while executing statements
inside the evaluation of an expression.

Execution of a \xcd"future(P) e" causes a new latch to be created,
and an \xcd"async" activity spawned at \xcd"P". The activity attempts
to \xcd"finish" the assigned \xcd"x = e", where \xcd"x" is a local
variable.  This may cause new activities to be spawned, based on
\xcd"e". If the assignment terminates successfully, another activity is
spawned to invoke the \xcd"setValue" method on the latch.  Exceptions
thrown by these activities (if any) are accumulated at the \xcd"finish"
statement and thrown after global termination of all
activities spawned by \xcd"x=e". The exception will be caught by the 
\xcd"catch" clause and stored with the latch. 


\oldtodo{Conditional atomic blocks should be powerful enough to implement clocks as well.}

\paragraph{A future to execute a statement.}
Consider an expression \xcd"onFinish {S}". This should return
a \xcd"Boolean" latch which should be forced when \xcd"S" has terminated
globally. Unlike \xcd"finish S", the evaluation of \xcd"onFinish {S}"
should locally terminate immediately, returning a latch. The
latch may be passed around in method invocations and stored in
objects. An activity may perform \xcd"force"/\xcd"forced" method
invocations on the latch whenever it desires to determine whether \xcd"S"
has terminated.

Such an expression can be written as:
\begin{xten}
(=> {
    L: Latch = new Latch();
    async (here) {
      try {
        finish S;
        L.setValue(true);
      }
      catch (Z: Exception) {
        L.setValue(Z);
      }
    }
    return L;
  }
)()
\end{xten}
}
	
\chapter{Clocks}\label{XtenClocks}\index{clocks}

The standard library for \Xten{}, \xcd"x10.lang" defines a
final class", \xcd"Clock" intended for repeated quiescence detection
of arbitrary, data-dependent collection of activities. Clocks are a
generalization of {\em barriers}. They permit dynamically created
activities to register and deregister. An activity may be registered
with multiple clocks at the same time. In particular, nested clocks
are permitted: an activity may create a nested clock and within one
phase of the outer clock schedule activities to run to completion on
the nested clock.  Nevertheless the design of clocks ensures that
deadlock cannot be introduced by using clock operations, and that
clock operations do not introduce any races.

This chapter describes the syntax and semantics of clocks and
statements in the language that have parameters of type \xcd"Clock". 

The key invariants associated with clocks are as follows.  At any
stage of the computation, a clock has zero or more {\em registered}
activities. An activity may perform operations only on those clocks it
is registered with (these clocks constitute its {\em clock set}).  An
activity is registered with one or more clocks when it is created.
During its lifetime the only additional clocks it is registered with
are exactly those that it creates. In particular it is not possible
for an activity to register itself with a clock it discovers by
reading a data-structure.

An activity may perform the following operations on a clock \xcd"c".
It may {\em unregister} with \xcd"c" by executing \xcd"c.drop();".
After this, it may perform no further actions on \xcd"c"
for its lifetime. It may {\em check} to see if it is unregistered on a
clock. It may {\em register} a newly forked activity with \xcd"c".
%% It may {\em post} a statement \xcd"S" for completion in the current phase
%% of \xcd"c" by executing the statement \xcd"now(c) S". 
Once registered and "active" (see below), it may also perform the following operations.
It may {\em resume} the clock by executing \xcd"c.resume();". This
indicates to \xcd"c" that it has finished posting all statements it
wishes to perform in the current phase. Finally, it may {\em block}
(by executing \xcd"next;") on all the clocks that it is registered
with. (This operation implicitly \xcd"resume"'s all clocks for the
activity.) It will resume from this statement only when all these
clocks are ready to advance to the next phase.

A clock becomes ready to advance to the next phase when every activity
registered with the clock has executed at least one \xcd"resume"
operation on that clock and all statements posted for completion in
the current phase have been completed.

Though clocks introduce a blocking statement (\xcd"next") an important
property of \Xten{} is that clocks cannot introduce deadlocks. That
is, the system cannot reach a quiescent state (in which no activity is
progressing) from which it is unable to progress. For, before blocking
each activity resumes all clocks it is registered with. Thus if a
configuration were to be stuck (that is, no activity can progress) all
clocks will have been resumed. But this implies that all activities
blocked on \xcd"next" may continue and the configuration is not stuck.
The only other possibility is that an activity may be stuck on
\xcd"finish". But the interaction rule between \xcd"finish" and clocks
(\Sref{sec:finish:clock-rule}) guarantees that this cannot cause a cycle
in the wait-for graph. A more rigorous proof may be found in \cite{X10-concur05}.

\section{Clock operations}\label{sec:clock}
The special statements introduced for clock operations are listed below.
%%479 NowStatement ::= 
%%      now ( Clock ) Statement

\begin{grammar}
Statement \: ClockedStatement \\
ClockedStatement \: \xcd"clocked" \xcd"(" ClockList \xcd")" Statement \\
NextStatement \: \xcd"next" \xcd";" \\
\end{grammar}

Note that \xcd"x10.lang.Clock" provides several useful methods on
clocks (e.g. \xcd"drop").

\subsection{Creating new clocks}\index{clock!creation}\label{sec:clock:create}

Clocks are created using a factory method on \xcd"x10.lang.Clock":

\begin{xten}
timeSynchronizer: Clock = Clock.make();
\end{xten}

\eat{All clocked variables are implicitly final. The initializer for a
local variable declaration of type \xcd"Clock" must be a new clock
expression. Thus \Xten{} does not permit aliasing of clocks.
Clocks are created in the place global heap and hence outlive the
lifetime of the creating activity.  Clocks are structs, hence may be freely
copied from place to 
place. (Clock instances typically contain references to mutable state
that maintains the current state of the clock.)
}
The current activity is automatically registered with the newly
created clock.  It may deregister using the \xcd"drop" method on
clocks (see the documentation of \xcd"x10.lang.Clock"). All activities
are automatically deregistered from all clocks they are registered
with on termination (normal or abrupt).

\subsection{Registering new activities on clocks}
\index{clock!clocked statements}\label{sec:clock:register}

The programmer may specify which clocks a new activity is to be
registered with using the \xcd"clocked" clause.

An activity may transmit only those clocks that is registered with and
has not quiesced on (\Sref{resume}). 
A \xcd"ClockUseException"\index{clock!ClockUseException} is
thrown if (and when) this condition is violated.

An activity may check that it is registered on a clock \xcd"c" by
executing:
\begin{xten}
c.registered()
\end{xten}
\noindent This call returns the \xcd"Boolean" value \xcd"true" iff the
activity is registered on \xcd"c"; otherwise it returns \xcd"false".

\begin{note}
\Xten{} does not contain a ``register'' statement that would allow an
activity to discover a clock in a datastructure and register itself on
it. Therefore, while clocks may be stored in a datastructure by one
activity and read from that by another, the new activity cannot
``use'' the clock unless it is already registered with it.
\end{note}

\oldtodo{Add text on arrays of clocks.}

\subsection{Resuming clocks}\index{clock!resume}\label{resume}\label{sec:clock:resume}
\Xten{} permits {\em split phase} clocks. An activity may wish
to indicate that it has completed whatever work it wishes to perform
in the current phase of a  clock \xcd"c" it is registered with, without
suspending all activity. It may do so  by executing the method
invocation:
\begin{xten}
c.resume();
\end{xten}

An activity may invoke this method only on a clock it is registered
with, and has not yet dropped (\Sref{sec:clock:drop}). A \xcd"ClockUseException"\index{clock!ClockUseException} is thrown if (and
when) this condition is violated.  Nothing happens if the activity has
already invoked a \xcd"resume" on this clock in the current phase.
Otherwise execution of this statement indicates that the activity will
not transmit \xcd"c" to an \xcd"async" (through a \xcd"clocked"
clause),
% or invoke \xcd"now" 
until it terminates, drops \xcd"c" or executes a \xcd"next". 

\begin{staticrule*}
The compiler should issue an error if any activity has a potentially
live execution path from a \xcd"resume" statement on a clock \xcd"c"
to a
%\xcd"now" or
async spawn statement (which registers the new
activity on \xcd"c") unless the path goes through a \xcd"next"
statement. (A \xcd"c.drop()" following a \xcd"c.resume()" is legal,
as is \xcd"c.resume()" following a \xcd"c.resume()".)
\end{staticrule*}

\subsection{Advancing clocks}\index{clock!next}\label{sec:clock:next}
An activity may execute the statement
\begin{xten}
next;
\end{xten}

\noindent 
Execution of this statement blocks until all the clocks that the
activity is registered with (if any) have advanced. (The activity
implicitly issues a \xcd"resume" on all clocks it is registered
with before suspending.)

An \Xten{} computation is said to be {\em quiescent} on a clock
\xcd"c" if each activity registered with \xcd"c" has resumed \xcd"c".
Note that once a computation is quiescent on \xcd"c", it will remain
quiescent on \xcd"c" forever (unless the system takes some action),
since no other activity can become registered with \xcd"c".  That is,
quiescence on a clock is a {\em stable property}.

Once the implementation has detected quiescence on \xcd"c", the system
marks all activities registered with \xcd"c" as being able to progress
on \xcd"c". An activity blocked on \xcd"next" resumes execution once
it is marked for progress by all the clocks it is registered with.

\subsection{Dropping clocks}\index{clock!drop}\label{sec:clock:drop}
An activity may drop a clock by executing:
\begin{xten}
c.drop();
\end{xten}

\noindent{} The activity is no longer considered registered with this
clock.  A \xcd"ClockUseException" is thrown if the activity has
already dropped \xcd"c".


%\input{clock/now.tex}

\section{Program equivalences}
From the discussion above it should be clear that the following
equivalences hold:

\begin{eqnarray}
 \mbox{\xcd"c.resume(); next;"}       &=& \mbox{\xcd"next;"}\\
 \mbox{\xcd"c.resume(); d.resume();"} &=& \mbox{\xcd"d.resume(); c.resume();"}\\
 \mbox{\xcd"c.resume(); c.resume();"} &=& \mbox{\xcd"c.resume();"}
\end{eqnarray}

Note that \xcd"next; next;" is not the same as \xcd"next;". The
first will wait for clocks to advance twice, and the second
once.  

%\notinfouro{\input{clock/imp-notes.tex}}
%\notinfouro{\input{clock/clocked-types.tex}}
%\notinfouro{\input{clock/examples.tex}}

	
\chapter{Local and Distributed Arrays}\label{XtenArrays}\index{array}

\section{Overview}

Indexable memory is fundamental abstraction for a programming
language. X10 includes
\begin{itemize}
\item Rails -- intrinsic one dimensional arrays
\item Local multi-dimensional arrays; both simplearray and regionarray
\item Distributed multi-dimensional arrays; both simplearray and regionarray
\end{itemize}

\section{Rails}

WRITE ME

\section{Simple Arrays}

WRITE ME

\section{Region-based Arrays}

Classes in the \Xcd{x10.regionarray} package provide the most general and 
flexible array abstraction that support mapping arbitrary multi-dimensional
index spaces to data elements. Although they are significantly more
flexible than \Xcd{Rail}s or the classes of the \Xcd{x10.simplearray}
package, this flexibility does carry with it an expectation of lower
runtime performance. 

\Xcd{Array}s provide indexed access to data at a single \Xcd{Place}, {\em via}
\Xcd{Point}s---indices of any dimensionality. \Xcd{DistArray}s is similar, but
spreads the data across multiple \xcd`Place`s, {\em via} \Xcd{Dist}s.  

\subsection{Points}\label{point-syntax}
\index{point}
\index{point!syntax}

Both kinds of arrays are indexed by \xcd`Point`s, which are $n$-dimensional tuples of
integers.  The \xcd`rank`
property of a point gives its dimensionality.  Points can be constructed from
integers, or \xcd`Rail[Int] by the \xcd`Point.make` factory methods:
%~~gen ^^^ArraysPointsExample1
% package Arrays.Points.Example1;
% import x10.regionarray.*;
% class Example1 {
% def example1() {
%~~vis
\begin{xten}
val origin_1 : Point{rank==1} = Point.make(0);
val origin_2 : Point{rank==2} = Point.make(0,0);
val origin_5 : Point = Point.make(new Rail[Int](5));
\end{xten}
%~~siv
% } } 
%~~neg

There is an implicit conversion from \xcd`Rail[Int]` to 
\xcd`Point`, giving
a convenient syntax for constructing points: 

%~~gen ^^^ Arrays30
% package Arrays.Points.Example2;
% import x10.regionarray.*;
% class Example{
% def example() {
%~~vis
\begin{xten}
val p : Point = [1,2,3];
val q : Point{rank==5} = [1,2,3,4,5];
val r : Point(3) = [11,22,33];
\end{xten}
%~~siv
% } } 
%~~neg

The coordinates of a point are available by function application, or, if you
prefer, by subscripting; \xcd`p(i)` is the
\xcd`i`th coordinate of the point \xcd`p`. 
\xcdmath`Point($n$)` is a \Xcd{type}-defined shorthand  for 
\xcdmath`Point{rank==$n$}`.


\subsection{Regions}\label{XtenRegions}\index{region}
\index{region!syntax}

A {\em region} is a set of points of the same rank.  {}\Xten{}
provides a built-in class, \xcd`x10.regionarray.Region`, to allow the
creation of new regions and to perform operations on regions. 
Each region \xcd`R` has a property \xcd`R.rank`, giving the dimensionality of
all the points in it.

\begin{ex}
%~~gen ^^^ Arrays40
% package Arrays40;
% import x10.regionarray.*;
% class Example {
% static def example() {
%~~vis
\begin{xten}
val MAX_HEIGHT=20;
val Null = Region.makeUnit(); //Empty 0-dimensional region
val R1 = Region.make(1, 100); // Region 1..100
val R2 = Region.make(1..100);  // Region 1..100
val R3 = Region.make(0..99, -1..MAX_HEIGHT);
val R4 = Region.makeUpperTriangular(10);
val R5 = R4 && R3; // intersection of two regions
\end{xten}
%~~siv
% } } 
%~~neg

The \xcd`IntRange` value \xcd`1..100` can be used to construct
the one-dimensional \xcd`Region` consisting of the points
$\{$\xcdmath`[m]`, \dots, \xcdmath`[n]`$\}$
\xcd`Region` by using the \xcd`Region.make` factory method.  
\xcd`IntRange`s are useful in building up regions, especially rectangular regions.  
\end{ex}

By a special dispensation, the compiler knows that, if \xcd`r : Region(m)` and
\xcd`s : Region(n)`, then \xcd`r*s : Region(m+n)`.  (The X10 type system
ordinarily could not specify the sum; the best it could do 
would be \xcd`r*s : Region`, with the rank of the region unknown.)  This
feature allows more convenient use of arrays; in particular, one does not need
to keep track of ranks nearly so much.

Various built-in regions are provided through  factory
methods on \xcd`Region`.  
\begin{itemize}
%~~exp~~"~~"~~ n:Int ~~ import x10.regionarray.*; ^^^Arrays3s5h
\item \xcd"Region.makeEmpty(n)" returns an empty region of rank \xcd"n".
%~~exp~~"~~"~~ n:Int ~~ import x10.regionarray.*; ^^^Arrays3x4j
\item \xcd"Region.makeFull(n)" returns the region containing all points of
      rank \xcd"n".  
%~~exp~~"~~"~~ ~~ import x10.regionarray.*; ^^^Arrays7l3d
\item \xcd"Region.makeUnit()" returns the region of rank 0 containing the
      unique point of rank 0.  It is useful as the identity for Cartesian
      product of regions.
%~~exp~~"~~"~~ normal:Point, k:Int ~~ import x10.regionarray.*; ^^^Arrays3l7z
\item \xcd"Region.makeHalfspace(normal, k)",
      where \xcd`normal` is a \xcd`Point` and \xcd`k` an \xcd`Int`, 
      returns the unbounded
      half-space of rank \xcd"normal.rank", consisting of all points \xcd"p"
      satisfying the vector inequality \xcdmath`p$\cdot$normal $\le$ k`.
%~~exp~~"~~"~~ min:Rail[Long], max:Rail[Long] ~~ import x10.regionarray.*; ^^^Arrays3i3n
\item \xcd"Region.makeRectangular(min, max)", 
      where \xcd"min" and \xcd"max"
      are rank-1 length-\xcd`n` integer arrays, returns a
      \xcd"Region(n)" equal to: 
      \xcdmath`[min(0) .. max(0), $\ldots$, min(n-1)..max(n-1)]`.
%~~exp~~"~~"~~ size: int, a: int, b: int~~ import x10.regionarray.*; ^^^Arrays2f2y
\item \xcd"Region.makeBanded(size, a, b)" constructs the
      banded \xcd"Region(2)" of size \xcd"size", with \Xcd{a} bands above
      and \Xcd{b} bands below the diagonal.
%~~exp~~"~~"~~size:Int ~~ import x10.regionarray.*; ^^^Arrays5s3q
\item \xcd"Region.makeBanded(size)" constructs the banded \Xcd{Region(2)} with
      just the main diagonal.
%~~exp~~`~~`~~N:Int ~~ import x10.regionarray.*; ^^^Arrays5s3qtri
\item \xcd`Region.makeUpperTriangular(N)` returns a region corresponding
to the non-zero indices in an upper-triangular \xcd`N x N` matrix.
%~~exp~~`~~`~~N:Int ~~ import x10.regionarray.*; ^^^Arrays5s3qlowertri
\item \xcd`Region.makeLowerTriangular(N)` returns a region corresponding
to the non-zero indices in a lower-triangular \xcd`N x N` matrix.
\item 
  If \xcd`R` is a region, and \xcd`p` a Point of the same rank, then 
%~~exp~~`~~`~~R:Region, p:Point(R.rank) ~~ import x10.regionarray.*; ^^^ Arrays50
  \xcd`R+p` is \xcd`R` translated forwards by 
  \xcd`p` -- the region whose
%~~exp~~`~~`~~r:Point, p:Point(r.rank) ~~ import x10.regionarray.*; ^^^ Arrays60
  points are \xcd`r+p` 
  for each \xcd`r` in \xcd`R`.
\item 
  If \xcd`R` is a region, and \xcd`p` a Point of the same rank, then 
%~~exp~~`~~`~~R:Region, p:Point(R.rank) ~~ import x10.regionarray.*; ^^^ Arrays70
  \xcd`R-p` is \xcd`R` translated backwards by 
  \xcd`p` -- the region whose
%~~exp~~`~~`~~r:Point, p:Point(r.rank) ~~ import x10.regionarray.*; ^^^ Arrays80
  points are \xcd`r-p` 
  for each \xcd`r` in \xcd`R`.

\end{itemize}

All the points in a region are ordered canonically by the
lexicographic total order. Thus the points of the region \xcd`(1..2)*(1..2)`
are ordered as 
\begin{xten}
(1,1), (1,2), (2,1), (2,2)
\end{xten}
Sequential iteration statements such as \xcd`for` (\Sref{ForAllLoop})
iterate over the points in a region in the canonical order.

A region is said to be {\em rectangular}\index{region!convex} if it is of
the form \xcdmath`(T$_1$ * $\cdots$ * T$_k$)` for some set of intervals
\xcdmath`T$_i = $ l$_i$ .. h$_i$ `. 
In particular an \xcd`IntRange` turned into a \xcd`Region` is rectangular: 
%~~exp~~`~~`~~ ~~ import x10.regionarray.*; ^^^Arrays3x4z
\xcd`Region.make(1..10)`.
Such a
region satisfies the property that if two points $p_1$ and $p_3$ are
in the region, then so is every point $p_2$ between them (that is, it is {\em convex}). 
(Banded and triangular regions are not rectangular.)
The operation
%~~exp~~`~~`~~R:Region ~~ import x10.regionarray.*; ^^^ Arrays90
\xcd`R.boundingBox()` gives the smallest rectangular region containing
\xcd`R`.

\subsubsection{Operations on regions}
\index{region!operations}

Let \xcd`R` be a region. A {\em sub-region} is a subset of \xcd"R".
\index{region!sub-region}

Let \xcdmath`R1` and \xcdmath`R2` be two regions whose types establish that
they are of the same rank. Let \xcdmath`S` be another region; its rank is
irrelevant. 

\xcdmath`R1 && R2` is the intersection of \xcdmath`R1` and
\xcdmath`R2`, \viz, the region containing all points which are in both
\Xcd{R1} and \Xcd{R2}.  \index{region!intersection}
%~~exp~~`~~`~~ ~~ import x10.regionarray.*; ^^^ Arrays100
For example, \xcd`Region.make(1..10) && Region.make(2..20)` is \Xcd{2..10}.


\xcdmath`R1 * S` is the Cartesian product of \xcdmath`R1` and
\xcdmath`S`,  formed by pairing each point in \xcdmath`R1` with every  point in \xcdmath`S`.
\index{region!product}
%~~exp~~`~~`~~ ~~ import x10.regionarray.*; ^^^ Arrays110
Thus, \xcd`Region.make(1..2)*Region.make(3..4)*Region.make(5..6)`
is the region of rank \Xcd{3} containing the eight points with coordinates
\xcd`[1,3,5]`, \xcd`[1,3,6]`, \xcd`[1,4,5]`, \xcd`[1,4,6]`,
\xcd`[2,3,5]`, \xcd`[2,3,6]`, \xcd`[2,4,5]`, \xcd`[2,4,6]`.


For a region \xcdmath`R` and point \xcdmath`p` of the same rank,
%~~exp~~`~~`~~R:Region, p:Point{p.rank==R.rank} ~~ import x10.regionarray.*; ^^^ Arrays120
\xcd`R+p` 
and
%~~exp~~`~~`~~R:Region, p:Point{p.rank==R.rank} ~~ import x10.regionarray.*; ^^^ Arrays130
\xcd`R-p` 
represent the translation of the region
forward 
and backward 
by \xcdmath`p`. That is, \Xcd{R+p} is the set of points
\Xcd{p+q} for all \Xcd{q} in \Xcd{R}, and \Xcd{R-p} is the set of \Xcd{q-p}.

More \Xcd{Region} methods are described in the API documentation.

\subsection{Arrays}
\index{array}

Arrays are organized data, arranged so that it can be accessed by subscript.
An \xcd`Array[T]` \Xcd{A} has a \Xcd{Region} \Xcd{A.region}, telling which
\Xcd{Point}s are in \Xcd{A}.  For each point \Xcd{p} in \Xcd{A.region},
\Xcd{A(p)} is the datum of type \Xcd{T} associated with \Xcd{p}.  X10
implementations should 
attempt to store \xcd`Array`s efficiently, and to make array element accesses
quick---\eg, avoiding constructing \Xcd{Point}s when unnecessary.

This generalizes the concepts of arrays appearing in many other programming
languages.  A \Xcd{Point} may have any number of coordinates, so an
\xcd`Array` can have, in effect, any number of integer subscripts.  

\begin{ex}Indeed, it is possible to write code that works on \Xcd{Array}s regardless 
of dimension.  For example, to add one \Xcd{Array[Int]} \Xcd{src} into another
\Xcd{dest}, 
%~~gen ^^^ Arrays140
%package Arrays.Arrays.Arrays.Example;
%import x10.regionarray.*;
% class Example{
%~~vis
\begin{xten}
static def addInto(src: Array[Int], dest:Array[Int])
  {src.region == dest.region}
  = {
    for (p in src.region) 
       dest(p) += src(p);
  }
\end{xten}
%~~siv
%}
% class Hook{
%   def run() { 
%     val a = new Array[Int](3, [1,2,3]);
%     val b = new Array[Int](a.region, (p:Point(1)) => 10*a(p) );
%     Example.addInto(a, b);
%     return b(0) == 11 && b(1) == 22 && b(2) == 33;
% }}
%~~neg
\noindent
Since \Xcd{p} is a \Xcd{Point}, it can hold as many coordinates as are
necessary for the arrays \Xcd{src} and \Xcd{dest}.
\end{ex}

The basic operation on arrays is subscripting: if \Xcd{A} is an \Xcd{Array[T]}
and \Xcd{p} a point with the same rank as \xcd`A.region`, then
%~~exp~~`~~`~~A:Array[Int], p:Point{self.rank == A.region.rank} ~~ import x10.regionarray.*; ^^^ Arrays150
\xcd`A(p)`
is the value of type \Xcd{T} associated with point \Xcd{p}.
This is the same operation as function application
(\Sref{sect:FunctionApplication}); arrays implement function types, and can be
used as functions.

Array elements can be changed by assignment. If \Xcd{t:T}, 
%~~gen ^^^ Arrays160
%package Arrays.Arrays.Subscripting.Is.From.Mars;
%import x10.regionarray.*; 
%class Example{
%def example[T](A:Array[T], p: Point{rank == A.region.rank}, t:T){
%~~vis
\begin{xten}
A(p) = t;
\end{xten}
%~~siv
%} } 
%~~neg
modifies the value associated with \Xcd{p} to be \Xcd{t}, and leaves all other
values in \Xcd{A} unchanged.

An \Xcd{Array[T]} named \Xcd{a} has: 
\begin{itemize}
%~~exp~~`~~`~~a:Array[Int] ~~ import x10.regionarray.*; ^^^ Arrays170
\item \xcd`a.region`: the \Xcd{Region} upon which \Xcd{a} is defined.
%~~exp~~`~~`~~a:Array[Int] ~~ import x10.regionarray.*; ^^^ Arrays180
\item \xcd`a.size`: the number of elements in \Xcd{a}.
%~~exp~~`~~`~~a:Array[Int] ~~ import x10.regionarray.*; ^^^ Arrays190
\item \xcd`a.rank`, the rank of the points usable to subscript \Xcd{a}. 
      \xcd`a.rank` is a cached copy of 
      \Xcd{a.region.rank}.
\end{itemize}

\subsubsection{Array Constructors}
\index{array!constructor}

To construct an array whose elements all have the same value \Xcd{init}, call
\Xcd{new Array[T](R, init)}. 
For example, an array of a thousand \xcd`"oh!"`s can be made by:
%~~exp~~`~~`~~ ~~ import x10.regionarray.*; ^^^ Arrays200
\xcd`new Array[String](1000, "oh!")`.


To construct and initialize an array, call the two-argument constructor. 
\Xcd{new Array[T](R, f)} constructs an array of elements of type \Xcd{T} on
region \Xcd{R}, with \Xcd{a(p)} initialized to \Xcd{f(p)} for each point
\Xcd{p} in \Xcd{R}.  \Xcd{f} must be a function taking a point of rank
\Xcd{R.rank} to a value of type \Xcd{T}.  

\begin{ex}
One way to construct the array \xcd`[11, 22, 33]` is with an array constructor
%~~exp~~`~~`~~ ~~ import x10.regionarray.*; ^^^ Arrays210
\xcd`new Array[Int](3, (i:long)=>(11*i) as Int)`. 
To construct a multiplication table, call
%~~exp~~`~~`~~ ~~ import x10.regionarray.*; ^^^ Arrays220
\xcd`new Array[Long](Region.make(0..9, 0..9), (p:Point(2)) => p(0)*p(1))`.
\end{ex}

Other constructors are available; see the API documentation and
\Sref{sect:RailCtors}. 

\subsubsection{Array Operations}
\index{array!operations on}

The basic operation on \Xcd{Array}s is subscripting.  If \Xcd{a:Array[T]} and 
\xcd`p:Point{rank == a.rank}`, then \Xcd{a(p)} is the value of type \Xcd{T}
appearing at position \Xcd{p} in \Xcd{a}.    The syntax is identical to
function application, and, indeed, arrays may be used as functions.
\Xcd{a(p)} may be assigned to, as well, by the usual assignment syntax
%~~exp~~`~~`~~a:Array[Int], p:Point{rank == a.rank}, t:Int ~~ import x10.regionarray.*; ^^^ Arrays230
\xcd`a(p)=t`.
(This uses the application and setting syntactic sugar, as given in \Sref{set-and-apply}.)

Sometimes it is more convenient to subscript by integers.  Arrays of rank 1-4
can, in fact, be accessed by integers: 
%~~gen ^^^ Arrays240
%package Arrays240;
%import x10.regionarray.*;
%class Example{
%static def example(){
%~~vis
\begin{xten}
val A1 = new Array[Int](10, 0);
A1(4) = A1(4) + 1;
val A4 = new Array[Int](Region.make(1..2, 1..3, 1..4, 1..5), 0);
A4(2,3,4,5) = A4(1,1,1,1)+1;
\end{xten}
%~~siv
% assert A1(4) == 1 && A4(2,3,4,5) == 1;
%}}
% class Hook{ def run() {Example.example(); return true;}}
%~~neg



Iteration over an \Xcd{Array} is defined, and produces the \Xcd{Point}s of the
array's region.  If you want to use the values in the array, you have to
subscript it.  For example, you could take the logarithm of every element of an
\Xcd{Array[Double]} by: 
%~~gen ^^^ Arrays250
%package Arrays250;
%import x10.regionarray.*;
%class Example{
%static def example(a:Array[Double]) {
%~~vis
\begin{xten}
for (p in a) a(p) = Math.log(a(p));
\end{xten}
%~~siv
%}}
% class Hook{ def run() { val a = new Array[Double](2, [1.0,2.0]); Example.example(a); return a(0)==Math.log(1.0) && a(1)==Math.log(2.0); }}

%~~neg



\subsection{Distributions}\label{XtenDistributions}
\index{distribution}

Distributed arrays are spread across multiple \xcd`Place`s.  
A {\em distribution}, a mapping from a region to a set of places, 
describes where each element of a distributed array is kept.
Distributions are embodied by the class \Xcd{x10.regionarray.Dist} and its
subclasses. 
The {\em rank} of a distribution is the rank of the underlying region, and
thus the rank of every point that the distribution applies to.


\begin{ex}
%~~gen ^^^ Arrays260
%package Arrays.Dist_example_a;
%import x10.regionarray.*;
% class Example{
% def example() {
%~~vis
\begin{xten}
val R  <: Region = Region.make(1..100);
val D1 <: Dist = Dist.makeBlock(R);
val D2 <: Dist = Dist.makeConstant(R, here);
\end{xten}
%~~siv
% } } 
%~~neg

\xcd`D1` distributes the region \xcd`R` in blocks, with a set of consecutive
points at each place, as evenly as possible.  \xcd`D2` maps all the points in
\xcd`R` to \xcd`here`.  
\end{ex}

Let \xcd`D` be a distribution. 
%~~exp~~`~~`~~D:Dist ~~ import x10.regionarray.*; ^^^ Arrays270
\xcd`D.region` 
denotes the underlying
region. 
Given a point \xcd`p`, the expression
%~~exp~~`~~`~~ D:Dist, p:Point{p.rank == D.rank}~~ import x10.regionarray.*; ^^^ Arrays280
\xcd`D(p)` represents the application of \xcd`D` to \xcd`p`, that is,
the place that \xcd`p` is mapped to by \xcd`D`. The evaluation of the
expression \xcd`D(p)` throws an \xcd`ArrayIndexOutofBoundsException`
if \xcd`p` does not lie in the underlying region.


\subsubsection{{\tt PlaceGroup}s}

A \xcd`PlaceGroup` represents an ordered set of \xcd`Place`s.
\xcd`PlaceGroup`s exist for performance and scaleability: they are more
efficient, in certain critical places, than general collections of
\xcd`Place`. \xcd`PlaceGroup` implements \xcd`Sequence[Place]`, and thus
provides familiar operations -- \xcd`pg.size()` for the number of places,
\xcd`pg.iterator()` to iterate over them, etc.  

\xcd`PlaceGroup` is an abstract class.  The concrete class
\xcd`SparsePlaceGroup` is intended for a small group of places. 
%~~exp~~`~~`~~ somePlace:Place ~~ ^^^Arrays1j6q
\xcd`new SparsePlaceGroup(somePlace)` is a good \xcd`PlaceGroup` containing
one place.  
%~~exp~~`~~`~~ seqPlaces: Rail[Place] ~~ ^^^Arrays9g6f
\xcd`new SparsePlaceGroup(seqPlaces)`
constructs a sparse place group from a Rail of places.

\subsubsection{Operations returning distributions}
\index{distribution!operations}



Let \xcd`R` be a region, \xcd`Q` 
a \xcd`PlaceGroup`, and \xcd`P` a place.

\paragraph{Unique distribution} \index{distribution!unique}
%~~exp~~`~~`~~Q:PlaceGroup ~~ import x10.regionarray.*; ^^^ Arrays290
The distribution \xcd`Dist.makeUnique(Q)` is the unique distribution from the
region \xcd`Region.make(1..k)` to \xcd`Q` mapping each point \xcd`i` to
\xcd`pi`.


\paragraph{Constant distributions.} \index{distribution!constant}
%~~exp~~`~~`~~R:Region, P:Place ~~ import x10.regionarray.*; ^^^ Arrays300
The distribution \xcd`Dist.makeConstant(R,P)` maps every point in region
\xcd`R` to place \xcd`P`.  
%~~exp~~`~~`~~R:Region ~~ import x10.regionarray.*; ^^^Arrays9n5n
The special case \xcd`Dist.makeConstant(R)` maps every point in \xcd`R` to
\xcd`here`. 

\paragraph{Block distributions.}\index{distribution!block}
%~~exp~~`~~`~~R:Region ~~ import x10.regionarray.*; ^^^ Arrays320
The distribution \xcd`Dist.makeBlock(R)` distributes the elements of \xcd`R`,
in approximately-even blocks, over all the places available to the program. 
There are other \xcd`Dist.makeBlock` methods capable of controlling the
distribution and the set of places used; see the API documentation.


\paragraph{Domain Restriction.} \index{distribution!restriction!region}

If \xcd`D` is a distribution and \xcd`R` is a sub-region of {\cf
%~~exp~~`~~`~~D:Dist,R :Region{R.rank==D.rank} ~~ import x10.regionarray.*; ^^^ Arrays330
D.region}, then \xcd`D | R` represents the restriction of \xcd`D` to
\xcd`R`---that is, the distribution that takes each point \xcd`p` in \xcd`R`
to 
%~~exp~~`~~`~~D:Dist, p:Point{p.rank==D.rank} ~~ import x10.regionarray.*; ^^^ Arrays340
\xcd`D(p)`, 
but doesn't apply to any points but those in \xcd`R`.

\paragraph{Range Restriction.}\index{distribution!restriction!range}

If \xcd`D` is a distribution and \xcd`P` a place expression, the term
%~~exp~~`~~`~~ D:Dist, P:Place~~ import x10.regionarray.*; ^^^ Arrays350
\xcd`D | P` 
denotes the sub-distribution of \xcd`D` defined over all the
points in the region of \xcd`D` mapped to \xcd`P`.

Note that \xcd`D | here` does not necessarily contain adjacent points
in \xcd`D.region`. For instance, if \xcd`D` is a cyclic distribution,
\xcd`D | here` will typically contain points that differ by the number of
places. 
An implementation may find a
way to still represent them in contiguous memory, \eg, using an arithmetic
function to map from the region index to an index 
into the array.


\subsection{Distributed Arrays}
\index{array!distributed}
\index{distributed array}
\index{\Xcd{DistArray}}
\index{DistArray}

Distributed arrays, instances of \xcd`DistArray[T]`, are very much like
\xcd`Array`s, except that they distribute information among multiple
\xcd`Place`s according to a \xcd`Dist` value passed in as a constructor
argument.  

\begin{ex}The following code creates a distributed array holding
a thousand cells, each initialized to 0.0, distributed via a block
distribution over all places.
%~~gen ^^^ Arrays360
% package Arrays.Distarrays.basic.example;
% import x10.regionarray.*;
% class Example {
% def example() {
%~~vis
\begin{xten}
val R <: Region = Region.make(1..1000);
val D <: Dist = Dist.makeBlock(R);
val da <: DistArray[Float] 
       = DistArray.make[Float](D, (Point(1))=>0.0f);
\end{xten}
%~~siv
%}}
%~~neg
\end{ex}



\subsection{Distributed Array Construction}\label{ArrayInitializer}
\index{distributed array!creation}
\index{\Xcd{DistArray}!creation}
\index{DistArray!creation}

\xcd`DistArray`s are instantiated by invoking one of the \xcd`make` factory
methods of the \xcd`DistArray` class.
A \xcd`DistArray` creation 
must take either an \xcd`Int` as an argument or a \xcd`Dist`. In the first
case,  a distributed array is created over the distribution 
%~~exp~~`~~`~~N:Int ~~ import x10.regionarray.*; ^^^Arrays1s6g
\xcd`Dist.makeConstant(Region.make(0, N-1),here)`;
in the second over the given distribution. 

\begin{ex}A distributed array creation operation may also specify an initializer
function.
The function is applied in parallel
at all points in the domain of the distribution. The
construction operation terminates locally only when the \xcd`DistArray` has been
fully created and initialized (at all places in the range of the
distribution).

For instance:
%~~gen ^^^ Arrays370
% package Arrays.DistArray.Construction.FeralWolf;
% import x10.regionarray.*;
% class Example {
% def example() {
%~~vis
\begin{xten}
val ident = ([i]:Point(1)) => i;
val data : DistArray[Long]
    = DistArray.make[Long](Dist.makeConstant(Region.make(1, 9)), ident);
val blk = Dist.makeBlock(Region.make(1..9, 1..9));
val data2 : DistArray[Long]
    = DistArray.make[Long](blk, ([i,j]:Point(2)) => i*j);
\end{xten}
%~~siv
% }  }
%~~neg




{}\noindent 
The first declaration stores in \xcd`data` a reference to a mutable
distributed array with \xcd`9` elements each of which is located in the
same place as the array. The element at \Xcd{[i]} is initialized to its index
\xcd`i`. 

The second declaration stores in \xcd`data2` a reference to a mutable
two-dimensional distributed array, whose coordinates both range from 1 to
9, distributed in blocks over all \xcd`Place`s, 
initialized with \xcd`i*j`
at point \xcd`[i,j]`.
\end{ex}


\subsection{Operations on Arrays and Distributed Arrays}

Arrays and distributed arrays share many operations.
In the following, let \xcd`a` be an array with base type T, and \xcd`da` be an
array with distribution \xcd`D` and base type \xcd`T`.




\subsubsection{Element operations}\index{array!access}
The value of \xcd`a` at a point \xcd`p` in its region of definition is
%~~exp~~`~~`~~a:Array[Int](3), p:Point(3) ~~ import x10.regionarray.*; ^^^ Arrays380
obtained by using the indexing operation \xcd`a(p)`. 
The value of \xcd`da` at \xcd`p` is similarly
%~~exp~~`~~`~~da:DistArray[Int](3), p:Point(3) ~~ import x10.regionarray.*; ^^^ Arrays390
\xcd`da(p)`.
This operation
may be used on the left hand side of an assignment operation to update
the value: 
%~~stmt~~`~~`~~a:Array[Int](3), p:Point(3), t:Int ~~ import x10.regionarray.*; ^^^ Arrays400
\xcd`a(p)=t;`
and 
%~~stmt~~`~~`~~da:DistArray[Int](3), p:Point(3), t:Int ~~ import x10.regionarray.*; ^^^ Arrays410
\xcd`da(p)=t;`
The operator assignments, \xcd`a(i) += e` and so on,  are also
available. 

It is a runtime error to 
access arrays, with \xcd`da(p)` or \xcd`da(p)=v`, at a place
other than \xcd`da.dist(p)`, \viz{} at the place that the element exists. 


\subsubsection{Arrays of Single Values}\label{ConstantArray}
\index{array!constant promotion}

For a region \xcd`R` and a value \xcd`v` of type \xcd`T`, the expression 
%~~genexp~~`~~`~~T~~R:Region{self!=null}, v:T ~~ import x10.regionarray.*; ^^^ Arrays420
\xcd`new Array[T](R, v)` 
produces an array on region \xcd`R` initialized with value \xcd`v`.
Similarly, 
for a distribution \xcd`D` and a value \xcd`v` of
type \xcd`T` the expression 
\begin{xtenmath}
DistArray.make[T](D, (Point(D.rank))=>v)
\end{xtenmath}
constructs a distributed array with
distribution \xcd`D` and base type \xcd`T` initialized with \xcd`v`
at every point.

Note that \xcd`Array`s are constructed by constructor calls, but
\xcd`DistArrays` are constructed by calls to the factory methods
\xcd`DistArray.make`. This is because \xcd`Array`s are fairly simple objects,
but \xcd`DistArray`s may be implemented by different classes for different
distributions. The use of the factory method gives the library writer the
freedom to select appropriate implementations.


\subsubsection{Restriction of an array}\index{array!restriction}

Let \xcd`R` be a sub-region of \xcd`da.region`. Then 
%~~exp~~`~~`~~da:DistArray[Int](3), p:Point(3), R: Region(da.rank) ~~ import x10.regionarray.*; ^^^ Arrays440
\xcd`da | R`
represents the sub-\xcd`DistArray` of \xcd`da` on the region \xcd`R`.
That is, \xcd`da | R` has the same values as \xcd`da` when subscripted by a
%~~exp~~`~~`~~R:Region, da: DistArray[Int]{da.region.rank == R.rank} ~~ import x10.regionarray.*; ^^^ Arrays450
point in region \xcd`R && da.region`, and is undefined elsewhere.

Recall that a rich set of operators are available on distributions
(\Sref{XtenDistributions}) to obtain sub-distributions
(e.g. restricting to a sub-region, to a specific place etc).


\subsubsection{Operations on Whole Arrays}

\paragraph{Pointwise operations}\label{ArrayPointwise}\index{array!pointwise operations}
The unary \xcd`map` operation applies a function to each element of
a distributed or non-distributed array, returning a new distributed array with
the same distribution, or a non-distributed array with the same region.

The following produces an array of cubes: 
%~~gen ^^^ Arrays460
%package Arrays_pointwise_a;
%import x10.regionarray.*;
%class Example{
%static def example() {
%~~vis
\begin{xten}
val A = new Array[Int](11, (i:long)=>i as Int);
assert A(3) == 3 && A(4) == 4 && A(10) == 10; 
val cube = (i:Int) => i*i*i;
val B = A.map(cube);
assert B(3) == 27 && B(4) == 64 && B(10) == 1000; 
\end{xten}
%~~siv
%} } 
% class Hook{ def run() {Example.example(); return true;}}
%~~neg

A variant operation lets you specify the array \Xcd{B} into which the result
will be stored, 
%~~gen ^^^ Arrays470
%package Arrays.map_with_result;
%import x10.regionarray.*;
%class Example{
%static def example() {
%~~vis
\begin{xten}
val A = new Array[Int](11, (i:long)=>i as Int);
assert A(3) == 3 && A(4) == 4 && A(10) == 10; 
val cube = (i:Int) => i*i*i;
val B = new Array[Int](A.region); // B = 0,0,0,0,0,0,0,0,0,0,0
A.map(B, cube);
assert B(3) == 27 && B(4) == 64 && B(10) == 1000; 
\end{xten}
%~~siv
%} } 
% class Hook{ def run() {Example.example(); return true;}}
%~~neg
\noindent
This is convenient if you have an already-allocated array lying around unused.
In particular, it can be used if you don't need \Xcd{A} afterwards and want to
reuse its space:
%~~gen ^^^ Arrays480
%package Arrays.map_reusing_space;
%import x10.regionarray.*;
%class Example{
%static def example() {
%~~vis
\begin{xten}
val A = new Array[Int](11, (i:long)=>i as Int);
assert A(3) == 3 && A(4) == 4 && A(10) == 10; 
val cube = (i:Int) => i*i*i;
A.map(A, cube);
assert A(3) == 27 && A(4) == 64 && A(10) == 1000; 
\end{xten}
%~~siv
%} } 
% class Hook{ def run() {Example.example(); return true;}}
%~~neg


The binary \xcd`map` operation takes a binary function and
another
array over the same region or distributed array over the same  distribution,
and applies the function 
pointwise to corresponding elements of the two arrays, returning
a new array or distributed array of the same shape.
The following code adds two distributed arrays: 
%~~gen ^^^ Arrays490
% package Arrays.Pointwise.Pointless.Map2;
% import x10.regionarray.*;
% class Example{
%~~vis
\begin{xten}
static def add(da:DistArray[Int], db: DistArray[Int])
    {da.dist==db.dist}
    = da.map(db, (a:Int,b:Int)=>a+b);
\end{xten}
%~~siv
%}
%~~neg



\paragraph{Reductions}\label{ArrayReductions}\index{array!reductions}

Let \xcd`f` be a function of type \xcd`(T,T)=>T`.  Let
\xcd`a` be an array over base type \xcd`T`.
Let \xcd`unit` be a value of type \xcd`T`.
Then the
%~~genexp~~`~~`~~ T ~~ f:(T,T)=>T, a : Array[T], unit:T ~~ import x10.regionarray.*; ^^^ Arrays500
operation \xcd`a.reduce(f, unit)` returns a value of type \xcd`T` obtained
by combining all the elements of \xcd`a` by use of  \xcd`f` in some unspecified order
(perhaps in parallel).   
The following code gives one method which 
meets the definition of \Xcd{reduce},
having a running total \Xcd{r}, and accumulating each value \xcd`a(p)` into it
using \Xcd{f} in turn.  (This code is simply given as an example; \Xcd{Array}
has this operation defined already.)
%~~gen ^^^ Arrays510
%package Arrays.Reductions.And.Eliminations;
%import x10.regionarray.*;
% class Example {
%~~vis
\begin{xten}
def oneWayToReduce[T](a:Array[T], f:(T,T)=>T, unit:T):T {
  var r : T = unit;
  for(p in a.region) r = f(r, a(p));
  return r;
}
\end{xten}
%~~siv
%}
%~~neg


For example,  the following sums an array of integers.  \Xcd{f} is addition,
and \Xcd{unit} is zero.  
%~~gen ^^^ Arrays520
% package Arrays.Reductions.And.Emulsions;
%import x10.regionarray.*;
% class Example {
% static def example() {
%~~vis
\begin{xten}
val a = new Array[Int](4, (i:long)=>(i+1) as Int);
val sum = a.reduce((a:Int,b:Int)=>a+b, 0); 
assert(sum == 10); // 10 == 1+2+3+4
\end{xten}
%~~siv
%}}
% class Hook{ def run() {Example.example(); return true;}}
%~~neg

Other orders of evaluation, degrees of parallelism, and applications of
\Xcd{f(x,unit)} and \xcd`f(unit,x)`are also correct.
In order to guarantee that the result is precisely
determined, the  function \xcd`f` should be associative and
commutative, and the value \xcd`unit` should satisfy
\xcd`f(unit,x)` \xcd`==` \xcd`x` \xcd`==` \xcd`f(x,unit)`
for all \Xcd{x:T}.  




\xcd`DistArray`s have the same operation.
This operation involves communication between the places over which
the \xcd`DistArray` is distributed. The \Xten{} implementation guarantees that
only one value of type \xcd`T` is communicated from a place as part of
this reduction process.

\paragraph{Scans}\label{ArrayScans}\index{array!scans}


Let \xcd`f:(T,T)=>T`, \xcd`unit:T`, and \xcd`a` be an \xcd`Array[T]` or
\xcd`DistArray[T]`.  Then \xcd`a.scan(f,unit)` is the array or distributed
array of type \xcd`T` whose {$i$}th element in canonical order is the
reduction by \xcd`f` with unit \xcd`unit` of the first {$i$} elements of
\xcd`a`. 


This operation involves communication between the places over which the
distributed array is distributed. The \Xten{} implementation will endeavour to
minimize the communication between places to implement this operation.

Other operations on arrays, distributed arrays, and the related classes may be
found in the \xcd`x10.regionarray` package.
	
\chapter{Annotations}\label{XtenAnnotations}\index{annotations}


\Xten{} provides an 
an annotation system  system for to allow the
compiler to be extended with new static analyses and new
transformations.

Annotations are constraint-free interface types that decorate the abstract syntax tree
of an \Xten{} program.  The \Xten{} type-checker ensures that an annotation
is a legal interface type.
In \Xten{}, interfaces may declare
both methods and properties.  Therefore, like any interface type, an
annotation may instantiate
one or more of its interface's properties.
%%PLUGINNERY%%  Unlike with Java
%%PLUGINNERY%%  annotations,
%%PLUGINNERY%%  property initializers need not be
%%PLUGINNERY%%  compile-time constants;
%%PLUGINNERY%%  however, a given compiler plugin
%%PLUGINNERY%%  may do additional checks to constrain the allowable
%%PLUGINNERY%%  initializer expressions.
%%PLUGINNERY%%  The \Xten{} type-checker does not check that
%%PLUGINNERY%%  all properties of an annotation are initialized,
%%PLUGINNERY%%  although this could be enforced by
%%PLUGINNERY%%  a compiler plugin.

\section{Annotation syntax}

The annotation syntax consists of an ``\texttt{@}'' followed by an interface type.

%##(Annotations Annotation
\begin{bbgrammar}
%(FROM #(prod:Annotations)#)
         Annotations \: Annotation & (\ref{prod:Annotations}) \\
                    \| Annotations Annotation \\
%(FROM #(prod:Annotation)#)
          Annotation \: \xcd"@" NamedType & (\ref{prod:Annotation}) \\
\end{bbgrammar}
%##)

Annotations can be applied to most syntactic constructs in the language
including class declarations, constructors, methods, field declarations,
local variable declarations and formal parameters, statements,
expressions, and types.
Multiple occurrences of the same annotation (i.e., multiple
annotations with the same interface type) on the same entity are permitted.

%%OBSOLETE%% \begin{grammar}
%%OBSOLETE%% ClassModifier \: Annotation \\
%%OBSOLETE%% InterfaceModifier \: Annotation \\
%%OBSOLETE%% FieldModifier \: Annotation \\
%%OBSOLETE%% MethodModifier \: Annotation \\
%%OBSOLETE%% VariableModifier \: Annotation \\
%%OBSOLETE%% ConstructorModifier \: Annotation \\
%%OBSOLETE%% AbstractMethodModifier \: Annotation \\
%%OBSOLETE%% ConstantModifier \: Annotation \\
%%OBSOLETE%% Type \: AnnotatedType \\
%%OBSOLETE%% AnnotatedType \: Annotation\plus Type \\
%%OBSOLETE%% Statement \: AnnotatedStatement \\
%%OBSOLETE%% AnnotatedStatement \: Annotation\plus Statement \\
%%OBSOLETE%% Expression \: AnnotatedExpression \\
%%OBSOLETE%% AnnotatedExpression \: Annotation\plus Expression \\
%%OBSOLETE%% \end{grammar}

\noindent
Recall that interface types may have dependent parameters.

\noindent
The following examples illustrate the syntax:

\begin{itemize}
\item Declaration annotations:
\begin{xtennoindent}
  // class annotation
  @Value
  class Cons { ... }

  // method annotation
  @PreCondition(0 <= i && i < this.size)
  public def get(i: Int): Object { ... }

  // constructor annotation
  @Where(x != null)
  def this(x: T) { ... }

  // constructor return type annotation
  def this(x: T): C@Initialized { ... }

  // variable annotation
  @Unique x: A;
\end{xtennoindent}
\item Type annotations:
\begin{xtennoindent}
  List@Nonempty

  Int@Range(1,4)

  Array[Array[Double]]@Size(n * n)
\end{xtennoindent}
\item Expression annotations:
\begin{xtennoindent}
  m() : @RemoteCall
\end{xtennoindent}
\item Statement annotations:
\begin{xtennoindent}
  @Atomic { ... }

  @MinIterations(0)
  @MaxIterations(n)
  for (var i: Int = 0; i < n; i++) { ... }

  // An annotated empty statement ;
  @Assert(x < y);
\end{xtennoindent}
\end{itemize}

\section{Annotation declarations}

Annotations are declared as interfaces.  They must be
subtypes of the interface \texttt{x10.lang.annotation.Annotation}.
Annotations on particular static entities must extend the corresponding
\xcd`Annotation` subclasses, as follows: 
\begin{itemize}
\item Expressions---\xcd`ExpressionAnnotation`
\item Statements---\xcd`StatementAnnotation`
\item Classes---\xcd`ClassAnnotation`
\item Fields---\xcd`FieldAnnotation`
\item Methods---\xcd`MethodAnnotation`
\item Imports---\xcd`ImportAnnotation`
\item Packages---\xcd`PackageAnnotation`
\end{itemize}

\chapter{Interoperability with Other Languages}
\label{NativeCode}
\index{native code}
\label{Interoperability}
\index{interoperability}

The ability to interoperate with other programming languages is an
essential feature of the \Xten{} implementation.  Cross-language
interoperability enables both the incremental adoption of \Xten{} in
existing applications and the usage of existing libraries and
frameworks by newly developed \Xten{} programs. 

There are two primary interoperability scenarios that are supported by
\XtenCurrVer{}: inline substitution of fragments of foreign code for
\Xten program fragments (expressions, statements) and external linkage
to foreign code.

\section{Embedded Native Code Fragments}

The
\xcd`@Native(lang,code) Construct` annotation from \xcd`x10.compiler.Native` in
\Xten{} can be used to tell the \Xten{} compiler to substitute \xcd`code` for
whatever it would have generated when compiling \xcd`Construct`
with the \xcd`lang` back end.

The compiler cannot analyze native code the same way it analyzes \Xten{} code.  In
particular, \xcd`@Native` fields and methods must be explicitly typed; the
compiler will not infer types.

\subsection{Native {\tt static} Methods}

\xcd`static` methods can be given native implementations.  Note that these
implementations are syntactically {\em expressions}, not statements, in C++ or
Java.   Also, it is possible (and common) to provide native implementations
into both Java and C++ for the same method.
%~~gen ^^^ extern10
% package Extern.or_current_turn;
%~~vis
\begin{xten}
import x10.compiler.Native;
class Son {
  @Native("c++", "printf(\"Hi!\")")
  @Native("java", "System.out.println(\"Hi!\")")
  static def printNatively():void = {};
}
\end{xten}
%~~siv
%
%~~neg

If only some back-end languages are given, the \Xten{} code will be used for the
remaining back ends: 
%~~gen ^^^ extern20
% package Extern.or.burn;
%~~vis
\begin{xten}
import x10.compiler.Native;
class Land {
  @Native("c++", "printf(\"Hi from C++!\")")
  static def example():void = {
    x10.io.Console.OUT.println("Hi from X10!");
  };
}
\end{xten}
%~~siv
%
%~~neg

The \xcd`native` modifier on methods indicates that the method must not have
an \Xten{} code body, and \xcd`@Native` implementations must be given for all back
ends:
%~~gen ^^^ extern30
% package Extern.or_maybe_getting_back_together;
%~~vis
\begin{xten}
import x10.compiler.Native;
class Plants {
  @Native("c++", "printf(\"Hi!\")")
  @Native("java", "System.out.println(\"Hi!\")")
  static native def printNatively():void;
}
\end{xten}
%~~siv
%
%~~neg


Values may be returned from external code to \Xten{}.  Scalar types in Java and
C++ correspond directly to the analogous types in \Xten{}.  
%~~gen ^^^ extern40
% package Extern.hardy;
%~~vis
\begin{xten}
import x10.compiler.Native;
class Return {
  @Native("c++", "1")
  @Native("java", "1")
  static native def one():Int;
}
\end{xten}
%~~siv
%
%~~neg
Types are {\em not} inferred for methods marked as \xcd`@Native`.

Parameters may be passed to external code.  \xcd`(#1)`  is the first parameter,
\xcd`(#2)` the second, and so forth.  (\xcd`(#0)` is the name of the enclosing
class, or the \xcd`this` variable.)  Be aware that this is macro substitution
rather than normal parameter 
passing; \eg, if the first actual parameter is \xcd`i++`, and \xcd`(#1)`
appears twice in the external code, \xcd`i` will be incremented twice.
For example, a (ridiculous) way to print the sum of two numbers is: 
%~~gen ^^^ extern50
% package Extern.or_turnabout_is_fair_play;
%~~vis
\begin{xten}
import x10.compiler.Native;
class Species {
  @Native("c++","printf(\"Sum=%d\", ((#1)+(#2)) )")
  @Native("java","System.out.println(\"\" + ((#1)+(#2)))")
  static native def printNatively(x:Int, y:Int):void;
}
\end{xten}
%~~siv
%
%~~neg


Static variables in the class are available in the external code.  For Java,
the static variables are used with their \Xten{} names.  For C++, the names
must be mangled, by use of the \xcd`FMGL` macro.  

%~~gen ^^^ extern60
%package Extern.or.Die;
%~~vis
\begin{xten}
import x10.compiler.Native;
class Ability {
  static val A : Int = 1n;
  @Native("java", "A+2")
  @Native("c++", "Ability::FMGL(A)+2")
  static native def fromStatic():Int;
}
\end{xten}
%~~siv
%
%~~neg




\subsection{Native Blocks}

Any block may be annotated with \xcd`@Native(lang,stmt)`, indicating that, in
the given back end, it should be implemented as \xcd`stmt`. All 
variables  from the surrounding context are available inside \xcd`stmt`. For
example, the method call \xcd`born.example(10)`, if compiled to Java, changes
the field \xcd`y` of a \xcd`Born` object to 10. If compiled to C++ (for which
there is no \xcd`@Native`), it sets it to 3. 
%~~gen ^^^ extern70
%package Extern.me.plz; 
%~~vis
\begin{xten}
import x10.compiler.Native;
class Born {
  var y : Int = 1n; 
  public def example(x:Int):Int{
    @Native("java", "y=x;") 
    {y = 3n;}
    return y;
  }
}
\end{xten}
%~~siv
%
%~~neg

Note that the code being replaced is a statement -- the block \xcd`{y = 3;}`
in this case -- so the replacement should also be a statement. 


Other \Xten{} constructs may or may not be available in Java and/or C++ code.  For
example, type variables do not correspond exactly to type variables in either
language, and may not be available there.  The exact compilation scheme is
{\em not} fully specified.  You may inspect the generated Java or C++ code and
see how to do specific things, but there is no guarantee that fancy external
coding will continue to work in later versions of \Xten{}.



The full facilities of C++ or Java are available in native code blocks.
However, there is no guarantee that advanced features behave sensibly. You
must follow the exact conventions that the code generator does, or you will
get unpredictable results.  Furthermore, the code generator's conventions may
change without notice or documentation from version to version.  In most cases
the  code should either be a very simple expression, or a method
or function call to external code.


\section{Interoperability with External Java Code}

With Managed X10, we can seamlessly call existing Java code from \Xten{},
and call \Xten{} code from Java.  We call this the 
\emph{Java interoperability}~\cite{TakeuchiX1013} feature.

By combining Java interoperability with X10's distributed
execution features, we can even make existing Java applications, which
are originally designed to run on a single Java VM, scale-out with
minor modifications.

\subsection{How Java program is seen in X10}

Managed X10 does not pre-process the existing Java code to make it
accessible from X10.  X10 programs compiled with Managed X10 will call
existing Java code as is.

\paragraph{Types}

In X10, both at compile time and run time, there is no way to
distinguish Java types from X10 types.  Java types can be referred to
with regular \xcd{import} statement, or their qualified names.  The
package \xcd{java.lang} is not auto-imported into \Xten.  In Managed
x10, the resolver is enhanced to resolve types with X10 source files
in the source path first, then resolve them with Java class files in
the class path. Note that the resolver does not resolve types with
Java source files, therefore Java source files must be compiled in
advance.  To refer to Java types listed in
Tables~\ref{tab:specialtypes}, and \ref{tab:otherspecialtypes}, which
include all Java primitive types, use the corresponding X10 type
(e.g. use \xcd{x10.lang.Int} (or in short, \xcd{Int}) instead of
\xcd{int}).

\paragraph{Fields}

Fields of Java types are seen as fields of X10 types.

Managed X10 does not change the static initialization semantics of
Java types, which is per-class, at load time, and per-place (Java VM),
therefore, it is subtly different than the per-field lazy
initialization semantics of X10 static fields.

\paragraph{Methods}

Methods of Java types are seen as methods of X10 types.

\paragraph{Generic types}

Generic Java types are seen as their raw types 
(\S 4.8 in~\cite{java-lang-spec2005}).  Raw type is a mechanism to handle generic
Java types as non-generic types, where the type parameters are assumed
as \verb|java.lang.Object| or their upperbound if they have it.  Java
introduced generics and raw type at the same time to facilitate
generic Java code interfacing with non-generic legacy Java code.
Managed X10 uses this mechanism for a slightly different purpose.
Java erases type parameters at compile time, whereas X10 preserves
their values at run time.  To manifest this semantic gap in generics,
Managed X10 represents Java generic types as raw types and eliminates
type parameters at source code level.  For more detailed discussions,
please refer to~\cite{TakeuchiX1011,TakeuchiX1012}.

\paragraph{Arrays}

X10 rail and array types are generic types whose representation is different
from Java array types.

Managed X10 provides a special X10 type
\xcd{x10.interop.Java.array[T]} which represents Java array type
\xcd{T[]}.  Note that for X10 types in Table~\ref{tab:specialtypes},
this type means the Java array type of their primary type.  For
example, \xcd{array[Int]} and \xcd{array[String]} mean
\xcd{int[]} and \xcd{java.lang.String[]}, respectively.  Managed X10
also provides conversion methods between X10 \xcd`Rail`s and Java
arrays (\xcd{Java.convert[T](a:Rail[T]):array[T]} and
\xcd{Java.convert[T](a:array[T]):Rail[T]}),
and creation methods for Java arrays 
(\xcd{Java.newArray[T](d0:Int):array[T]}
etc.).

\paragraph{Exceptions}

The \Xten{} 2.3 exception hierarchy has been designed so that there is a
natural correspondence with the Java exception hierarchy. As shown in
Table~\ref{tab:otherspecialtypes}, many commonly used Java
exception types are directly mapped to X10 exception types. 
Exception types that are thus aliased may be caught (and thrown) using
either their Java or \Xten types.  In \Xten code, it is stylistically
preferable to use the \Xten type to refer to the exception as shown in 
Figure~\ref{fig:javaexceptions}.

%----------------
\begin{figure}
\begin{xten}
import x10.interop.Java;
public class XClass {   
  public static def main(args:Rail[String]):void {
    try {
      val a = Java.newArray[Int](2);
      a(0) = 0;
      a(1) = 1;
      a(2) = 2;
    } catch (e:x10.lang.ArrayIndexOutOfBoundsException) {
      Console.OUT.println(e);
    }
  }
}
\end{xten}
%\vspace{-2mm}%@@ADJUST
\begin{verbatim}
> x10c -d bin src/XClass.x10
> x10 -cp bin XClass
x10.lang.ArrayIndexOutOfBoundsException: Array index out of range: 2
\end{verbatim}
\caption{Java exceptions in X10}
%\vspace{-4mm}%@@ADJUST
\label{fig:javaexceptions}
\end{figure}
%----------------

\paragraph{Compiling and executing X10 programs}

We can compile and run X10 programs that call existing Java code with
the same \verb|x10c| and \verb|x10| command by specifying the location
of Java class files or jar files that your X10 programs refer to, with
\verb|-classpath| (or in short, \verb|-cp|) option.

\subsection{How X10 program is translated to Java}

Managed X10 translates X10 programs to Java class files. 

X10 does not provide a Java reflection-like mechanism to resolve X10
types, methods, and fields with their names at runtime, nor a code
generation tool, such as \verb|javah|, to generate stub code to access
them from other languages.  Java programmers, therefore, need to
access X10 types, methods, and fields in the generated Java code
directly as they access Java types, methods, and fields.  To make it
possible, Java programmers need to understand how X10 programs are
translated to Java.

Some aspects of the X10 to Java translation scheme may change in
future version of \Xten{}; therefore in this document only a stable
subset of translation scheme will be explained.  Although it is a
subset, it has been extensively used by X10 core team and proved to be
useful to develop Java Hadoop interop layer for a Main-memory Map
Reduce (M3R) engine~\cite{Shinnar12M3R} in X10.

In the following discussions, we deliberately ignore generic X10
types because the translation of generics is an area of active
development and will undergo some changes in future versions of \Xten{}.
For those who are interested in the implementation of generics
in Managed X10, please consult~\cite{TakeuchiX1012}.  We also don't
cover function types, function values, and all non-static methods.
Although slightly outdated, another paper~\cite{TakeuchiX1011}, which
describes translation scheme in X10 2.1.2, is still useful to
understand the overview of Java code generation in Managed X10.


\paragraph{Types}

X10 classes and structs are translated to Java classes with the same
names.  X10 interfaces are translated to Java interfaces with the same
names.

Table~\ref{tab:specialtypes} shows the list of special types that are
mapped to Java primitives.  Primitives are their primary
representations that are useful for good performance.  Wrapper classes
are used when the reference types are needed.  Each wrapper class has
two static methods \verb|$box()| and \verb|$unbox()| to convert its
value from primary representation to wrapper class, and vice versa,
and Java backend inserts their calls as needed.  An you notice, every
unsigned type uses the same Java primitive as its corresponding signed
type for its representation.

Table~\ref{tab:otherspecialtypes} shows a non-exhaustive list of
another kind of special types that are mapped (not translated) to Java
types.  As you notice, since an interface \verb|Any| is mapped to a
class |java.lang.Object| and \verb|Object| is hidden from the
language, there is no direct way to create an instance of
\verb|Object|. As a workaround, \verb|Java.newObject()| is provided.

As you also notice, \verb|x10.lang.Comparable[T]| is mapped to \verb|java.lang.Comparable|.
This is needed to map \verb|x10.lang.String|, which implements \verb|x10.lang.Compatable[String]|, to \verb|java.lang.String| for performance, but as a trade off, this mapping results in the lost of runtime type information for \verb|Comparable[T]| in Managed X10.
The runtime of Managed X10 has built-in knowledge for \verb|String|, but for other Java classes that implement \verb|java.lang.Comparable|, \verb|instanceof Comparable[Int]| etc. may return incorrect results.
In principle, it is impossible to map X10 generic type to the existing Java generic type without losing runtime type information.

%----------------
\begin{table}
%\scriptsize
\small
\centering
\mbox{
%\hspace{-4mm}%@@ADJUST
\begin{tabular}{|lr|lr|l|}												   \hline
\multicolumn{2}{|c|}{\textbf{X10}}	& \multicolumn{2}{|c|}{\textbf{Java (primary)}}	& \textbf{Java (wrapper class)}	\\ \hline
															   \hline
{\tt x10.lang.Byte}	& {\tt 1y}	& {\tt byte}		& {\tt (byte)1}		& {\tt x10.core.Byte}		\\ \hline
{\tt x10.lang.UByte}	& {\tt 1uy}	& {\tt byte}		& {\tt (byte)1}		& {\tt x10.core.UByte}		\\ \hline
{\tt x10.lang.Short}	& {\tt 1s}	& {\tt short}		& {\tt (short)1}	& {\tt x10.core.Short}		\\ \hline
{\tt x10.lang.UShort}	& {\tt 1us}	& {\tt short}		& {\tt (short)1}	& {\tt x10.core.UShort} 	\\ \hline
{\tt x10.lang.Int}	& {\tt 1}	& {\tt int}		& {\tt 1}		& {\tt x10.core.Int}		\\ \hline
{\tt x10.lang.UInt}	& {\tt 1u}	& {\tt int}		& {\tt 1}		& {\tt x10.core.UInt}		\\ \hline
{\tt x10.lang.Long}	& {\tt 1l}	& {\tt long}		& {\tt 1l}		& {\tt x10.core.Long}	 	\\ \hline
{\tt x10.lang.ULong}	& {\tt 1ul}	& {\tt long}		& {\tt 1l}		& {\tt x10.core.ULong}	 	\\ \hline
{\tt x10.lang.Float}	& {\tt 1.0f}	& {\tt float}		& {\tt 1.0f}		& {\tt x10.core.Float}	 	\\ \hline
{\tt x10.lang.Double}	& {\tt 1.0}	& {\tt double}		& {\tt 1.0}		& {\tt x10.core.Double} 	\\ \hline
{\tt x10.lang.Char}	& {\tt 'c'}	& {\tt char}		& {\tt 'c'}		& {\tt x10.core.Char}		\\ \hline
{\tt x10.lang.Boolean}	& {\tt true}	& {\tt boolean}		& {\tt true}		& {\tt x10.core.Boolean}	\\ \hline
%{\tt x10.lang.String} 	& {\tt "abc"}	& {\tt java.lang.String}& {\tt "abc"}		& {\tt x10.core.String}		\\ \hline
\end{tabular}
}
\caption{X10 types that are mapped to Java primitives}
%\vspace{-4mm}%@@ADJUST
\label{tab:specialtypes}
\end{table}
%----------------


%----------------
\begin{table}
%\scriptsize
\small
\centering
\mbox{
%\hspace{-4mm}%@@ADJUST
\begin{tabular}{|l|l|}										   \hline
\multicolumn{1}{|c|}{\textbf{X10}}		& \multicolumn{1}{|c|}{\textbf{Java}}		\\ \hline
												   \hline
{\tt x10.lang.Any} 				& {\tt java.lang.Object}			\\ \hline
{\tt x10.lang.Comparable[T]} 			& {\tt java.lang.Comparable}			\\ \hline
{\tt x10.lang.String}		 		& {\tt java.lang.String}			\\ \hline
{\tt x10.lang.CheckedThrowable}		 	& {\tt java.lang.Throwable}			\\ \hline
{\tt x10.lang.CheckedException}		 	& {\tt java.lang.Exception}			\\ \hline
{\tt x10.lang.Exception} 			& {\tt java.lang.RuntimeException}		\\ \hline
{\tt x10.lang.ArithmeticException} 		& {\tt java.lang.ArithmeticException}		\\ \hline
{\tt x10.lang.ClassCastException} 		& {\tt java.lang.ClassCastException}		\\ \hline
{\tt x10.lang.IllegalArgumentException} 	& {\tt java.lang.IllegalArgumentException}	\\ \hline
{\tt x10.util.NoSuchElementException}	 	& {\tt java.util.NoSuchElementException}	\\ \hline
{\tt x10.lang.NullPointerException} 		& {\tt java.lang.NullPointerException}		\\ \hline
{\tt x10.lang.NumberFormatException} 		& {\tt java.lang.NumberFormatException}		\\ \hline
{\tt x10.lang.UnsupportedOperationException} 	& {\tt java.lang.UnsupportedOperationException}	\\ \hline
{\tt x10.lang.IndexOutOfBoundsException} 	& {\tt java.lang.IndexOutOfBoundsException}	\\ \hline
{\tt x10.lang.ArrayIndexOutOfBoundsException} 	& {\tt java.lang.ArrayIndexOutOfBoundsException}\\ \hline
{\tt x10.lang.StringIndexOutOfBoundsException} 	& {\tt java.lang.StringIndexOutOfBoundsException}\\ \hline
{\tt x10.lang.Error} 				& {\tt java.lang.Error}				\\ \hline
{\tt x10.lang.AssertionError} 			& {\tt java.lang.AssertionError}		\\ \hline
{\tt x10.lang.OutOfMemoryError} 		& {\tt java.lang.OutOfMemoryError}		\\ \hline
{\tt x10.lang.StackOverflowError} 		& {\tt java.lang.StackOverflowError}		\\ \hline
{\tt void} 					& {\tt void}					\\ \hline
\end{tabular}
}
\caption{X10 types that are mapped (not translated) to Java types}
%\vspace{-4mm}%@@ADJUST
\label{tab:otherspecialtypes}
\end{table}
%----------------


\paragraph{Fields}

As shown in Figure~\ref{fig:fields}, instance fields of X10 classes and structs are translated to the instance fields of the same names of the generated Java classes.
Static fields of X10 classes and structs are translated to the static methods of the generated Java classes, whose name has \verb|get$| prefix.
Static fields of X10 interfaces are translated to the static methods of the special nested class named \verb|$Shadow| of the generated Java interfaces.

%----------------
\begin{figure}
\begin{xten}
class C {
  static val a:Int = ...;
  var b:Int;
}
interface I {
  val x:Int = ...;
}
\end{xten}
%\vspace{-4mm}%@@ADJUST
\begin{xten}
class C {
  static int get$a() { return ...; }
  int b;
}
interface I {
  abstract static class $Shadow {
    static int get$x() { return ...; }
  }
}
\end{xten}
%\vspace{-2mm}%@@ADJUST
\caption{X10 fields in Java}
%\vspace{-4mm}%@@ADJUST
\label{fig:fields}
\end{figure}
%----------------


\paragraph{Methods}

As shown in Figure~\ref{fig:methods}, methods of X10 classes or structs are translated to the methods of the same names of the generated Java classes.
Methods of X10 interfaces are translated to the methods of the same names of the generated Java interfaces.

Every method whose return type has two representations, such as the types in Table~\ref{tab:specialtypes}, will have \verb|$O| suffix with its name.
For example, \verb|def f():Int| in X10 will be compiled as \verb|int f$O()| in Java.
For those who are interested in the reason, please look at our paper~\cite{TakeuchiX1012}.

%----------------
\begin{figure}
\begin{xten}
interface I {
  def f():Int;
  def g():Any;
}
class C implements I {
  static def s():Int = 0;
  static def t():Any = null;
  public def f():Int = 1;
  public def g():Any = null;
}
\end{xten}
%\vspace{-4mm}%@@ADJUST
\begin{xten}
interface I {
  int f$O();
  java.lang.Object g();
}
class C implements I {
  static int s$O() { return 0; }
  static java.lang.Object t() { return null; }
  public int f$O() { return 1; }
  public java.lang.Object g() { return null; }
}
\end{xten}
%\vspace{-2mm}%@@ADJUST
\caption{X10 methods in Java}
%\vspace{-4mm}%@@ADJUST
\label{fig:methods}
\end{figure}
%----------------


\paragraph{Compiling Java programs}

To compile Java program that calls X10 code, you should use
\verb|x10cj| command instead of javac (or whatever your Java
compiler). It invokes the post Java-compiler of \verb|x10c| with the
appropriate options. You need to specify the location of X10-generated
class files or jar files that your Java program refers to.

\verb|x10cj -cp MyX10Lib.jar MyJavaProg.java|


\paragraph{Executing Java programs}

Before executing any X10-generated Java code, the runtime of Managed
X10 needs to be set up properly at each place.  To set up the runtime,
a special launcher named \verb|runjava| is used to run Java programs.
All Java programs that call X10 code should be launched with it, and
no other mechanisms, including direct execution with java command, are
supported.

\begin{verbatim}
Usage: runjava <Java-main-class> [arg0 arg1 ...]
\end{verbatim}


\section{Interoperability with External C and C++ Code}

C and C++ code can be linked to X10 code, either by writing auxiliary C++ files and
adding them with suitable annotations, or by linking libraries.

\subsection{Auxiliary C++ Files}

Auxiliary C++ code can be written in \xcd`.h` and \xcd`.cc` files, which
should be put in the same directory as the the X10 file using them.
Connecting with the library uses the \xcd`@NativeCPPInclude(dot_h_file_name)`
annotation to include the header file, and the 
\xcd`@NativeCPPCompilationUnit(dot_cc_file_name)` annotation to include the
C++ code proper.  For example: 

{\bf MyCppCode.h}: 
\begin{xten}
void foo();
\end{xten}


{\bf MyCppCode.cc}:
\begin{xten}
#include <cstdlib>
#include <cstdio>
void foo() {
    printf("Hello World!\n");
}
\end{xten}

{\bf Test.x10}:
\begin{xten}
import x10.compiler.Native;
import x10.compiler.NativeCPPInclude;
import x10.compiler.NativeCPPCompilationUnit;

@NativeCPPInclude("MyCPPCode.h")
@NativeCPPCompilationUnit("MyCPPCode.cc")
public class Test {
    public static def main (args:Rail[String]) {
        { @Native("c++","foo();") {} }
    }
}
\end{xten}

\subsection{C++ System Libraries}

If we want to additionally link to more libraries in \xcd`/usr/lib` for
example, it is necessary to adjust the post-compilation directly.  The
post-compilation is the compilation of the C++ which the X10-to-C++ compiler
\xcd`x10c++` produces.  

The primary mechanism used for this is the \xcd`-cxx-prearg` and
\xcd`-cxx-postarg` command line arguments to
\xcd`x10c++`. The values of any \xcd`-cxx-prearg` commands are placed
in the post compiler command before the list of .cc files to compile.
The values of any \xcd`-cxx-postarg` commands are placed in the post
compiler command after the list of .cc files to compile. Typically
pre-args are arguments intended for the C++ compiler itself, while
post-args are arguments intended for the linker. 

The following example shows how to compile \xcd`blas` into the
executable via these commands. The command must be issued on one line.

\begin{xten}
x10c++ Test.x10 -cxx-prearg -I/usr/local/blas 
  -cxx-postarg -L/usr/local/blas -cxx-postarg -lblas'
\end{xten}


\chapter{Definite Assignment}
\label{sect:DefiniteAssignment}
\index{definite assignment}
\index{assignment!definite}
\index{definitely assigned}
\index{definitely not assigned}

X10 requires that every variable be set before it is read.
Sometimes this is easy, as when a variable is declared and assigned together: 
%~~gen ^^^ DefiniteAssignment4x1u
% package DefiniteAssignment4x1u;
% class Example {
% def example() {
%~~vis
\begin{xten}
  var x : Long = 0;
  assert x == 0;
\end{xten}
%~~siv
%}}
%~~neg
However, it is convenient to allow programs to make decisions before
initializing variables.
%~~gen ^^^ DefiniteAssignment4u7z
% package DefiniteAssignment4u7z;
% class Example {
%~~vis
\begin{xten}
def example(a:Long, b:Long) {
  val max:Long;
  //ERROR: assert max==max; // can't read 'max'
  if (a > b) max = a;
  else max = b;
  assert max >= a && max >= b;
}
\end{xten}
%~~siv
%}
%~~neg
This is particularly useful for \xcd`val` variables.  \xcd`var`s could be
initialized to a default value and then reassigned with the right value.
\xcd`val`s must be initialized once and cannot be changed, so they must be
initialized with the correct value. 

However, one must be careful -- and the X10 compiler enforces this care.
Without the \xcd`else` clause, the preceding code might not give \xcd`max` a
value by the time \xcd`assert` is invoked.  

This leads to the concept of {\em definite assignment} \cite{jls2}.
A variable is {\em definitely assigned} at a point in code if, no matter how that
point in code is reached, the variable has been assigned to.  In X10,
variables must be definitely assigned before they can be read.


As X10 requires that \xcd`val` variables {\em not} be initialized
twice,  we need the dual concept as well.  A variable is {\em definitely
unassigned} at a point in code if it cannot have been assigned no
matter how that point in code is reached.  For example, immediately
after \xcd`val x:Long`, \xcd`x` is definitely unassigned.  

Finally, we need the concept of {\em singly} and {\em multiply assigned}.
A variable is singly assigned in a block if it is assigned precisely
once; it is multiply assigned if it could possibly be assigned more than once.  
\xcd`var`s can  multiply assigned as desired. \xcd`val`s must be singly
assigned.  For example, the code \xcd`x = 1; x = 2;` is perfectly fine if
\xcd`x` is a \xcd`var`, but incorrect (even in a constructor) if \xcd`x` is a
\xcd`val`.  

At some points in code, a variable might be neither definitely assigned nor
definitely unassigned.    Such states are not always useful.  
%~~gen ^^^ DefiniteAssignment4f5z
% package DefiniteAssignment4f5z;
% class Example {
% 
%~~vis
\begin{xten}
def example(flag : Boolean) {
  var x : Long;
  if (flag) x = 1;
  // x is neither def. assigned nor unassigned.
  x = 2; 
  // x is def. assigned.
\end{xten}
%~~siv
% } } 
%~~neg
This shows that we cannot simply define ``definitely unassigned'' as ``not
definitely assigned''.   If \xcd`x` had been a \xcd`val` rather than a
\xcd`var`, the previous example would not be allowed.    

Unfortunately, a completely accurate definition of ``definitely assigned''
or ``definitely unassigned'' is undecidable -- impossible for the compiler to
determine.  So, X10 takes a {\em conservative approximation} of these
concepts.  If X10's definition says that \xcd`x` is definitely assigned (or
definitely unassigned), then it will be assigned (or not assigned) in every
execution of the program.  

However, there are programs which X10's algorithm says are incorrect, but
which actually would behave properly if they were executed.   In the following
example, \xcd`flag` is either \xcd`true` or \xcd`false`, and in either case
\xcd`x` will be initialized.  However, X10's analysis does not understand this
--- thought it {\em would} understand if the example were coded with an
\xcd`if-else` rather than a pair of \xcd`if`s.  So, after the two \xcd`if`
statements, \xcd`x` is not definitely assigned, and thus the \xcd`assert`
statement, which reads it, is forbidden.  
%~~gen ^^^ DefiniteAssignment3x6i
% package DefiniteAssignment3x6i;
% class Example{ 
%~~vis
\begin{xten}
def example(flag:Boolean) {
  var x : Long;
  if (flag) x = 1;
  if (!flag) x = 2;
  // ERROR: assert x < 3;
}
\end{xten}
%~~siv
%}
%~~neg

\section{Asynchronous Definite Assignment}


Local variables and instance fields allow {\em asynchronous assignment}. A local
variable can be assigned in an \xcd`async` statement, and, when the
\xcd`async` is \xcd`finish`ed, the variable is definitely assigned.  

\begin{ex}
%~~gen ^^^ DefiniteAssignment4a6n
% package DefiniteAssignment4a6n;
% class Example {
% def example() {
%~~vis
\begin{xten}
val a : Long;
finish {
  async {
    a = 1;
  } 
  // a is not definitely assigned here
}
// a is definitely assigned after 'finish'
assert a==1; 
\end{xten}
%~~siv
%} } 
%~~neg
\end{ex}

This concept supports a core X10 programming idiom.  A \xcd`val` variable may
be initialized asynchronously, thereby providing a means for returning a value
from an \xcd`async` to be used after the enclosing \xcd`finish`.  

\section{Characteristics of Definite Assignment}

The properties ``definitely assigned'', ``singly assigned'', and
``definitely unassigned'' are computed by a conservative approximation of
X10's evaluation rules.

The precise details are up to the implementation. 
Many basic cases must be handled accurately; \eg, \xcd`x=1;` definitely and
singly assigns \xcd`x`.  

However, in more complicated cases, a conforming X10 may mark as invalid 
some code which, when executed, would actually be correct.  
For example, the following
program fragment will always result in \xcd`x` being definitely and singly
assigned:  
\begin{xten}
val x : Long;
var b : Boolean = mysterious();
if (b) x = cryptic();
if (!b) x = unknown();
\end{xten}
However, most conservative approximations of program execution won't mark
\xcd`x` as properly initialized, though it is.   For \xcd`x` to be properly
initialized, precisely one of the 
two assignments to \xcd`x` must be executed. If \xcd`b` were true initially,
it would still be true after the call to \xcd`cryptic()` --- since methods
cannot modify their caller's local variables -- and so the first but not the
second assignment would happen. If \xcd`b` were false initially, it would
still be false when \xcd`!b` is tested, and so the second but not the first
assignment would happen.  Either way, \xcd`x` is definitely and singly assigned.

However, for a slightly different program, this analysis would be wrong. \Eg,
if  \xcd`b` were a field of \xcd`this` rather than a local variable,
\xcd`cryptic()` could change \xcd`b`; if \xcd`b` were true initially, both
assignments might happen, which is incorrect for a \xcd`val`.  

This sort of reasoning is beyond  most conservative approximation algorithms.
(Indeed, many do not bother checking that \xcd`!b` late in the program is the
opposite of \xcd`b` earlier.)
Algorithms that pay attention to such details and subtleties tend to be
fairly expensive, which would lead to very slow compilation for X10 -- for the
sake of obscure cases.

X10's analysis provides at least the following guarantees. We describe them in
terms of a statement \xcd`S` performing some collection of possible numbers of
assignments to variables --- on a scale of ``0'', ``1'', and ``many''. For
example, \xcd`if (b) x=1; else {x=1;x=2;y=2;}` might assign to \xcd`x` one or
many times, and might assign to \xcd`y` zero or one time. Hence, after it,
\xcd`x` is definitely assigned and may be multiply assigned, and \xcd`y` is
neither definitely assigned nor definitely unassigned.  

These descriptions are combined in natural ways.  For example, if \xcd`R` says
that \xcd`x` will be assigned 0 or 1 times, and \xcd`S` says it will be
assigned precisely once, then \xcd`R;S` will assign it one or many times.  If
only one or \xcd`R` or \xcd`S` will occur, as from \xcd`if (b) R; else S;`, 
then \xcd`x` may be assigned 0 or 1 times. 

This information is sufficient for the tests X10 makes.  If \xcd`x` can is
assigned one or many times in \xcd`S`, it is definitely assigned.  It is an
error if 
\xcd`x` is ever read at a point where it have been assigned zero times.  It is
an error if a \xcd`val` may be assigned many times.

We do not guarantee that any particular X10 compiler uses this algorithm;
indeed, as of the time of writing, the X10 compiler uses a somewhat more
precise one.  However, any conformant X10 compiler must provide results which
are at least as accurate as this analysis.

\subsubsection{Assignment: {\tt x = e}}   

\xcd`x = e` assigns to \xcd`x`, in addition to whatever assignments
\xcd`e` makes.   For example, if \xcd`this.setX(y)` sets a field \xcd`x` to
\xcd`y` and returns \xcd`y`, then \xcd`x = this.setX(y)` definitely and
multiply assigns \xcd`x`.  

\subsubsection{{\tt async} and {\tt finish}}

By itself, \xcd`async S` provides few guarantees.  After an activity
executes \xcd`async{x=1;}` we know that there is a separate activity
which (on being scheduled) will set \xcd`x` to \xcd`1`.  We do not
know that this has happened yet.

However, if there is a \xcd`finish` around the \xcd`async`, the situation is
clearer.  After \xcd`finish async x=1;`, \xcd`x` has definitely been
assigned.  

In general, if an \xcd`async S` appears in the body of a \xcd`finish` in a way
that guarantees that it will be executed, then, after the \xcd`finish`, the
assignments made by \xcd`S` will have occurred.  For example, if \xcd`S`
definitely assigns to \xcd`x`, and the body of the \xcd`finish` guarantees
that \xcd`async S` will be executed, then \xcd`finish{...async S...}`
definitely assigns \xcd`x`.

\subsubsection{{\tt if} and {\tt switch}}

When \xcd`if(E) S else T` finishes, it will have performed the assignments of
\xcd`E`, together with those of either \xcd`S` or \xcd`T` but not both.  For
example, \xcd`if (b) x=1; else x=2;` definitely assigns \xcd`x`,
but \xcd`if (b) x=1;` does not.

{\tt switch} is more complex, but follows the same principles as \xcd`if`.
For example, \xcd`switch(E){case 1: A; break; case 2: B; default: C;}`  
performs the assignments of \xcd`E`, and those of precisely one of \xcd`A`, or
\xcd`B;C`, or \xcd`C`.  Note that case \xcd`2` falls through to the default
case, so it performs the same assignments as \xcd`B;C`.

\subsubsection{Sequencing}

When \xcd`R;S` finishes, it will have performed the assignments of \xcd`R` and
those of \xcd`S`, if \xcd`R` and \xcd`S` terminate normally. If
\xcd`R` terminates abruptly, then only the assignments of \xcd`R`
executed till the point of termination will have been executed. if
\xcd`R` terminates normally, but \xcd`S` terminates abruptly then the
assignments of \xcd`R` will have been executed and those of \xcd`S`
executed till the point of termination. 

For example, \xcd`x=1;y=2;` definitely assigns \xcd`x` and 
\xcd`y`, and \xcd`x=1;x=2;` multiply assigns \xcd`x`. 


\subsubsection{Loops}

\xcd`while(E)S` performs the assignments of \xcd`E` one or more times, and
those of \xcd`S` zero or more times.  For example, if \xcd`while(b()){x=1;}`
might assign to \xcd`x` zero, one, or many times.  
\xcd`do S while(E)` performs the assignments of \xcd`E` one or more times, and
those of \xcd`S` one or more times. 

\xcd`for(A;B;C)D` performs the assignments of \xcd`A` once, those of \xcd`B`
one or more times, and those of \xcd`C` and \xcd`D` one or more times.
\xcd`for(x in E)S` performs the assignments of \xcd`E` once and those of
\xcd`S` zero or more times.  

Loops are of very little value for providing definite assignments, since X10
does not in general know how many times they will be executed. 

\xcd`continue` and \xcd`break` inside of a loop are hard to describe in simple
terms.  They may be conservatively assumed to cause the loop to give no
information about the variables assigned inside of it.
For example, the analysis may conservatively conclude that 
\xcd`do{ x = 1; if (true) break; } while(true)` may 
assign to \xcd`x` zero, one, or many times, overlooking the more precise fact
that it is assigned once.  

\subsubsection{Method Calls}

A method call \xcd`E.m(A,B)` performs the assignments of \xcd`E`, \xcd`A`, and
\xcd`B` once each, and also those of \xcd`m`.  This implies that X10 must be
aware of the possible assignments performed by each method.

If X10 has complete information about \xcd`m` (as when \xcd`m` is a
\xcd`private` or \xcd`final` method), this is straightforward.  When such
information is fundamentally impossible to acquire, as when \xcd`m` is a
non-final method invocation, X10 has no choice but to assume that \xcd`m`
might do anything that a method can do.    (For this reason, the only methods
that can be called from within a constructor on a raw --
incompletely-constructed -- object) are the \xcd`private` and
\xcd`final` ones.)  
\begin{itemize}
\item \xcd`m` cannot assign to local variables of the caller; methods have no
      such power.
\item Let \xcd`m` be an instance method. \xcd`m` can assign to \xcd`var` fields of \xcd`this` freely,
\item Let \xcd`m` be an instance method. \xcd`m` cannot initialize \xcd`val` fields of \xcd`this`.  (But see
      \Sref{sect:call-another-ctor}; when one constructor calls another as the
      first statement of its body, the other constructor can initialize
      v\xcd`val` fields. This is a constructor call, not a method call.) 
\end{itemize}

Recall that every container must be fully initialized upon exit
from its constructor.  
X10 places certain restrictions on which methods can be called from a
constructor; see \Sref{sect:nonescaping}.  One of these restrictions is that
methods called before object initialization is complete must be \xcd`final` or
\xcd`private` --- and hence, available for static analysis.  So, when checking
field initialization, X10 will ensure: 
\begin{enumerate}
\item Each \xcd`val` field is initialized before it is read.   
      A method that does not read a \xcd`val` field \xcd`f` {\em may} be
      called before \xcd`f` is initialized; a method that reads \xcd`f` must
      not be called until \xcd`f` is initialized.        
      For example, 
      a constructor may have the form:
%~~gen ^^^ DefiniteAssignment4x6k
% package DefiniteAssignment4x6k;
%~~vis
\begin{xten}
class C {
  val f : Long;
  val g : String;
  def this() {
     f = fless();
     g = useF();
  }
  private def fless() = "f not used here".length();
  private def useF() = "f=" + this.f;
}
\end{xten}
%~~siv
%
%~~neg

\item \xcd`var` fields require a deeper analysis.  Consider a \xcd`var`
      field \xcd`var x:T`  without initializer.  If \xcd`T` has a default
      value, \xcd`x` may be read inside of a constructor before it is
      otherwise written, and it will 
      have its default value.   

      If \xcd`T` has no default value, an analysis
      like that used for \xcd`val`s must be performed to determine that
      \xcd`x` is initialized before it is used.  The situation is 
      more complex than for \xcd`val`s, however, because a method can assign to
      \xcd`x` as well read from it.  The X10 compiler computes a conservative
      approximation of which methods
      read and write which \xcd`var` fields. (Doing this carefully 
      requires finding a solution of a set of equations over sets of
      variables, with each callable method having equations describing what it
      reads and writes.)    

\end{enumerate}


\subsubsection{{\tt at} 
%and \xcd`athome`
}

%%AT-COPY%% \xcd`at(E)S` performs the assignments of \xcd`E`. Within \xcd`S`, only those
%%AT-COPY%% assignments to variables \xcd`x` from the surrounding environment which take
%%AT-COPY%% place within a suitable \xcd`athome(x)R` are counted. 
%%AT-COPY%% 
%%AT-COPY%% \begin{ex}
%%AT-COPY%% In the following code, the outer variable named \xcd`a` is definitely assigned
%%AT-COPY%% once, by the assignment \xcd`a = 3;`.  The inner variable (also named \xcd`a`)
%%AT-COPY%% is definitely multiply assigned 
%%AT-COPY%% by the two assignments \xcd`a = 1;` and \xcd`a = 2;` 
%%AT-COPY%% between the \xcd`at` and the \xcd`athome`.  
%%AT-COPY%% 
%%AT-COPY%% %~~gen ^^^ DefiniteAssignment3n5q
%%AT-COPY%% % package DefiniteAssignment3n5q;
%%AT-COPY%% % KNOWNFAIL-at
%%AT-COPY%% % class DefAss { def defass() { 
%%AT-COPY%% %~~vis
%%AT-COPY%% \begin{xten}
%%AT-COPY%% var a : Long;
%%AT-COPY%% at(here.next(); var a : Long = a) {
%%AT-COPY%%   a = 1;
%%AT-COPY%%   a = 2; 
%%AT-COPY%%   athome(a) a = 3;
%%AT-COPY%% }
%%AT-COPY%% \end{xten}
%%AT-COPY%% %~~siv
%%AT-COPY%% % } } 
%%AT-COPY%% %~~neg
%%AT-COPY%% 
%%AT-COPY%% 
%%AT-COPY%% \end{ex}
%%AT-COPY%% 

% vj Wed Sep 18 04:19:05 EDT 2013
% Hmm. This used to be incorrect.
\xcd`at(p)S` performs precisely the assignments of \xcd`p` and those
of \xcd`S`. Note that \xcd`S` is executed at the place named by
\xcd`p` in an environment in which all variables used in \xcd`S` but
defined outside \xcd`S` are bound to copies (made at \xcd`p`) of the
values they had at the \xcd`at(p)S` statement (\Sref{AtStatement}).

% vj Wed Sep 18 04:19:05 EDT 2013
% Hmm. Commented this out. This is not true :-(
%\xcd`this` cannot be read or written by an \xcd`at`-statement.

\subsubsection{{\tt atomic}}

\xcd`atomic S` performs the assignments of \xcd`S`, 
and \xcd`when(E)S` performs those of \xcd`E` and \xcd`S`.  Note that
\xcd`E` or \xcd`S` may terminate abruptly.

\subsubsection{{\tt try}}

\xcd`try S catch(x:T1) E1 ... catch(x:Tn) En finally F` 
performs some or all of the assignments of \xcd`S`, plus all the assignments
of zero or one of the \xcd`E`'s, plus those of \xcd`F`.  
For example,
\begin{xten}
try {
  x = boomy();
  x = 0;
}
catch(e:Boom) { y = 1; }
finally { z = 1; }
\end{xten}
\noindent 
assigns \xcd`x` zero, one, or many times\footnote{A more precise
analysis could discover that \xcd`x` cannot be initialized only once.}, 
assigns \xcd`y` zero or one time, and assigns \xcd`z` exactly once.

\subsubsection{Expression Statements}

Expression statements \xcd`E;`, and other statements that execute an
expression and do something innocuous with it (local variable declaration and
\xcd`assert`) have the same effects as \xcd`E`. 

\subsubsection{{\tt return}, {\tt throw}}

Statements that do not finish normally, such as \xcd`return` and \xcd`throw`,
do not initialize anything (though the computation of the return or thrown
value may).    They also terminate a line of computation.  For example, 
\xcd`if(b) {x=1; return;}  x=2;` definitely and singly assigns \xcd`x`.  

%% vj Thu Sep 19 06:00:59 EDT 2013
%% No changes made for v2.4. 

\chapter{Grammar}\label{Grammar}


In this grammar, $X^?$ denotes an optional $X$ element.


\begin{bbgrammarappendix}{3.9in}

(\arabic{equation}) & AdditiveExp \refstepcounter{equation}\label{prod:AdditiveExp}  \: MultiplicativeExp  \\

 &    \| AdditiveExp \xcd"+" MultiplicativeExp \\
 &    \| AdditiveExp \xcd"-" MultiplicativeExp \\

\end{bbgrammarappendix}

\begin{bbgrammarappendix}{4.4in}

(\arabic{equation}) & AndExp \refstepcounter{equation}\label{prod:AndExp}  \: EqualityExp  \\

 &    \| AndExp \xcd"&" EqualityExp \\

\end{bbgrammarappendix}

\begin{bbgrammarappendix}{3.7in}

(\arabic{equation}) & AnnotatedType \refstepcounter{equation}\label{prod:AnnotatedType}  \: Type Annotations  \\


\end{bbgrammarappendix}

\begin{bbgrammarappendix}{4.0in}

(\arabic{equation}) & Annotation \refstepcounter{equation}\label{prod:Annotation}  \: \xcd"@" NamedTypeNoConstraints  \\


\end{bbgrammarappendix}

\begin{bbgrammarappendix}{3.6in}

(\arabic{equation}) & AnnotationStmt \refstepcounter{equation}\label{prod:AnnotationStmt}  \: Annotations\opt NonExpStmt  \\


\end{bbgrammarappendix}

\begin{bbgrammarappendix}{3.9in}

(\arabic{equation}) & Annotations \refstepcounter{equation}\label{prod:Annotations}  \: Annotation  \\

 &    \| Annotations Annotation \\

\end{bbgrammarappendix}

\begin{bbgrammarappendix}{3.8in}

(\arabic{equation}) & ApplyOpDecln \refstepcounter{equation}\label{prod:ApplyOpDecln}  \: MethMods \xcd"operator" \xcd"this" TypeParams\opt Formals Guard\opt HasResultType\opt MethodBody  \\


\end{bbgrammarappendix}

\begin{bbgrammarappendix}{3.8in}

(\arabic{equation}) & ArgumentList \refstepcounter{equation}\label{prod:ArgumentList}  \: Exp  \\

 &    \| ArgumentList \xcd"," Exp \\

\end{bbgrammarappendix}

\begin{bbgrammarappendix}{4.1in}

(\arabic{equation}) & Arguments \refstepcounter{equation}\label{prod:Arguments}  \: \xcd"(" ArgumentList \xcd")"  \\


\end{bbgrammarappendix}

\begin{bbgrammarappendix}{4.0in}

(\arabic{equation}) & AssertStmt \refstepcounter{equation}\label{prod:AssertStmt}  \: \xcd"assert" Exp \xcd";"  \\

 &    \| \xcd"assert" Exp  \xcd":" Exp  \xcd";" \\

\end{bbgrammarappendix}

\begin{bbgrammarappendix}{3.6in}

(\arabic{equation}) & AssignPropCall \refstepcounter{equation}\label{prod:AssignPropCall}  \: \xcd"property" TypeArgs\opt \xcd"(" ArgumentList\opt \xcd")" \xcd";"  \\


\end{bbgrammarappendix}

\begin{bbgrammarappendix}{4.0in}

(\arabic{equation}) & Assignment \refstepcounter{equation}\label{prod:Assignment}  \: LeftHandSide AsstOp AsstExp  \\

 &    \| ExpName  \xcd"(" ArgumentList\opt \xcd")" AsstOp AsstExp \\
 &    \| Primary  \xcd"(" ArgumentList\opt \xcd")" AsstOp AsstExp \\

\end{bbgrammarappendix}

\begin{bbgrammarappendix}{4.3in}

(\arabic{equation}) & AsstExp \refstepcounter{equation}\label{prod:AsstExp}  \: Assignment  \\

 &    \| ConditionalExp \\

\end{bbgrammarappendix}

\begin{bbgrammarappendix}{4.4in}

(\arabic{equation}) & AsstOp \refstepcounter{equation}\label{prod:AsstOp}  \: \xcd"="  \\

 &    \| \xcd"*=" \\
 &    \| \xcd"/=" \\
 &    \| \xcd"%=" \\
 &    \| \xcd"+=" \\
 &    \| \xcd"-=" \\
 &    \| \xcd"<<=" \\
 &    \| \xcd">>=" \\
 &    \| \xcd">>>=" \\
 &    \| \xcd"&=" \\
 &    \| \xcd"^=" \\
 &    \| \xcd"|=" \\

\end{bbgrammarappendix}

\begin{bbgrammarappendix}{4.1in}

(\arabic{equation}) & AsyncStmt \refstepcounter{equation}\label{prod:AsyncStmt}  \: \xcd"async" ClockedClause\opt Stmt  \\

 &    \| \xcd"clocked" \xcd"async" Stmt \\

\end{bbgrammarappendix}

\begin{bbgrammarappendix}{3.6in}

(\arabic{equation}) & AtCaptureDeclr \refstepcounter{equation}\label{prod:AtCaptureDeclr}  \: Mods\opt VarKeyword\opt VariableDeclr  \\

 &    \| Id \\
 &    \| \xcd"this" \\

\end{bbgrammarappendix}

\begin{bbgrammarappendix}{3.5in}

(\arabic{equation}) & AtCaptureDeclrs \refstepcounter{equation}\label{prod:AtCaptureDeclrs}  \: AtCaptureDeclr  \\

 &    \| AtCaptureDeclrs \xcd"," AtCaptureDeclr \\

\end{bbgrammarappendix}

\begin{bbgrammarappendix}{4.0in}

(\arabic{equation}) & AtEachStmt \refstepcounter{equation}\label{prod:AtEachStmt}  \: \xcd"ateach" \xcd"(" LoopIndex \xcd"in" Exp \xcd")" ClockedClause\opt Stmt  \\

 &    \| \xcd"ateach" \xcd"(" Exp \xcd")" Stmt \\

\end{bbgrammarappendix}

\begin{bbgrammarappendix}{4.5in}

(\arabic{equation}) & AtExp \refstepcounter{equation}\label{prod:AtExp}  \: \xcd"at" \xcd"(" Exp \xcd")" ClosureBody  \\


\end{bbgrammarappendix}

\begin{bbgrammarappendix}{4.4in}

(\arabic{equation}) & AtStmt \refstepcounter{equation}\label{prod:AtStmt}  \: \xcd"at" \xcd"(" Exp \xcd")" Stmt  \\


\end{bbgrammarappendix}

\begin{bbgrammarappendix}{4.0in}

(\arabic{equation}) & AtomicStmt \refstepcounter{equation}\label{prod:AtomicStmt}  \: \xcd"atomic" Stmt  \\


\end{bbgrammarappendix}

\begin{bbgrammarappendix}{3.8in}

(\arabic{equation}) & BasicForStmt \refstepcounter{equation}\label{prod:BasicForStmt}  \: \xcd"for" \xcd"(" ForInit\opt \xcd";" Exp\opt \xcd";" ForUpdate\opt \xcd")" Stmt  \\


\end{bbgrammarappendix}

\begin{bbgrammarappendix}{4.5in}

(\arabic{equation}) & BinOp \refstepcounter{equation}\label{prod:BinOp}  \: \xcd"+"  \\

 &    \| \xcd"-" \\
 &    \| \xcd"*" \\
 &    \| \xcd"/" \\
 &    \| \xcd"%" \\
 &    \| \xcd"&" \\
 &    \| \xcd"|" \\
 &    \| \xcd"^" \\
 &    \| \xcd"&&" \\
 &    \| \xcd"||" \\
 &    \| \xcd"<<" \\
 &    \| \xcd">>" \\
 &    \| \xcd">>>" \\
 &    \| \xcd">=" \\
 &    \| \xcd"<=" \\
 &    \| \xcd">" \\
 &    \| \xcd"<" \\
 &    \| \xcd"==" \\
 &    \| \xcd"!=" \\
 &    \| \xcd".." \\
 &    \| \xcd"->" \\
 &    \| \xcd"<-" \\
 &    \| \xcd"-<" \\
 &    \| \xcd">-" \\
 &    \| \xcd"**" \\
 &    \| \xcd"~" \\
 &    \| \xcd"!~" \\
 &    \| \xcd"!" \\

\end{bbgrammarappendix}

\begin{bbgrammarappendix}{4.0in}

(\arabic{equation}) & BinOpDecln \refstepcounter{equation}\label{prod:BinOpDecln}  \: MethMods \xcd"operator" TypeParams\opt \xcd"(" Formal  \xcd")" BinOp \xcd"(" Formal  \xcd")" Guard\opt HasResultType\opt MethodBody  \\

 &    \| MethMods \xcd"operator" TypeParams\opt \xcd"this" BinOp \xcd"(" Formal  \xcd")" Guard\opt HasResultType\opt MethodBody \\
 &    \| MethMods \xcd"operator" TypeParams\opt \xcd"(" Formal  \xcd")" BinOp \xcd"this" Guard\opt HasResultType\opt MethodBody \\

\end{bbgrammarappendix}

\begin{bbgrammarappendix}{4.5in}

(\arabic{equation}) & Block \refstepcounter{equation}\label{prod:Block}  \: \xcd"{" BlockStmts\opt \xcd"}"  \\


\end{bbgrammarappendix}

\begin{bbgrammarappendix}{3.3in}

(\arabic{equation}) & BlockInteriorStmt \refstepcounter{equation}\label{prod:BlockInteriorStmt}  \: LocVarDeclnStmt  \\

 &    \| ClassDecln \\
 &    \| StructDecln \\
 &    \| TypeDefDecln \\
 &    \| Stmt \\

\end{bbgrammarappendix}

\begin{bbgrammarappendix}{4.0in}

(\arabic{equation}) & BlockStmts \refstepcounter{equation}\label{prod:BlockStmts}  \: BlockInteriorStmt  \\

 &    \| BlockStmts BlockInteriorStmt \\

\end{bbgrammarappendix}

\begin{bbgrammarappendix}{3.6in}

(\arabic{equation}) & BooleanLiteral \refstepcounter{equation}\label{prod:BooleanLiteral}  \: \xcd"true"   \\

 &    \| \xcd"false"  \\

\end{bbgrammarappendix}

\begin{bbgrammarappendix}{4.1in}

(\arabic{equation}) & BreakStmt \refstepcounter{equation}\label{prod:BreakStmt}  \: \xcd"break" Id\opt \xcd";"  \\


\end{bbgrammarappendix}

\begin{bbgrammarappendix}{4.3in}

(\arabic{equation}) & CastExp \refstepcounter{equation}\label{prod:CastExp}  \: Primary  \\

 &    \| ExpName \\
 &    \| CastExp \xcd"as" Type \\

\end{bbgrammarappendix}

\begin{bbgrammarappendix}{3.9in}

(\arabic{equation}) & CatchClause \refstepcounter{equation}\label{prod:CatchClause}  \: \xcd"catch" \xcd"(" Formal \xcd")" Block  \\


\end{bbgrammarappendix}

\begin{bbgrammarappendix}{4.3in}

(\arabic{equation}) & Catches \refstepcounter{equation}\label{prod:Catches}  \: CatchClause  \\

 &    \| Catches CatchClause \\

\end{bbgrammarappendix}

\begin{bbgrammarappendix}{4.1in}

(\arabic{equation}) & ClassBody \refstepcounter{equation}\label{prod:ClassBody}  \: \xcd"{" ClassMemberDeclns\opt \xcd"}"  \\


\end{bbgrammarappendix}

\begin{bbgrammarappendix}{4.0in}

(\arabic{equation}) & ClassDecln \refstepcounter{equation}\label{prod:ClassDecln}  \: Mods\opt \xcd"class" Id TypeParamsI\opt Properties\opt Guard\opt Super\opt Interfaces\opt ClassBody  \\


\end{bbgrammarappendix}

\begin{bbgrammarappendix}{3.4in}

(\arabic{equation}) & ClassMemberDecln \refstepcounter{equation}\label{prod:ClassMemberDecln}  \: InterfaceMemberDecln  \\

 &    \| CtorDecln \\

\end{bbgrammarappendix}

\begin{bbgrammarappendix}{3.3in}

(\arabic{equation}) & ClassMemberDeclns \refstepcounter{equation}\label{prod:ClassMemberDeclns}  \: ClassMemberDecln  \\

 &    \| ClassMemberDeclns ClassMemberDecln \\

\end{bbgrammarappendix}

\begin{bbgrammarappendix}{4.1in}

(\arabic{equation}) & ClassName \refstepcounter{equation}\label{prod:ClassName}  \: TypeName  \\


\end{bbgrammarappendix}

\begin{bbgrammarappendix}{4.1in}

(\arabic{equation}) & ClassType \refstepcounter{equation}\label{prod:ClassType}  \: NamedType  \\


\end{bbgrammarappendix}

\begin{bbgrammarappendix}{3.7in}

(\arabic{equation}) & ClockedClause \refstepcounter{equation}\label{prod:ClockedClause}  \: \xcd"clocked" Arguments  \\


\end{bbgrammarappendix}

\begin{bbgrammarappendix}{3.9in}

(\arabic{equation}) & ClosureBody \refstepcounter{equation}\label{prod:ClosureBody}  \: Exp  \\

 &    \| Annotations\opt \xcd"{" BlockStmts\opt LastExp \xcd"}" \\
 &    \| Annotations\opt Block \\

\end{bbgrammarappendix}

\begin{bbgrammarappendix}{4.0in}

(\arabic{equation}) & ClosureExp \refstepcounter{equation}\label{prod:ClosureExp}  \: Formals Guard\opt HasResultType\opt \xcd"=>" ClosureBody  \\


\end{bbgrammarappendix}

\begin{bbgrammarappendix}{3.5in}

(\arabic{equation}) & CompilationUnit \refstepcounter{equation}\label{prod:CompilationUnit}  \: PackageDecln\opt TypeDeclns\opt  \\

 &    \| PackageDecln\opt ImportDeclns TypeDeclns\opt \\
 &    \| ImportDeclns PackageDecln  ImportDeclns\opt  TypeDeclns\opt \\
 &    \| PackageDecln ImportDeclns PackageDecln  ImportDeclns\opt  TypeDeclns\opt \\

\end{bbgrammarappendix}

\begin{bbgrammarappendix}{3.3in}

(\arabic{equation}) & ConditionalAndExp \refstepcounter{equation}\label{prod:ConditionalAndExp}  \: InclusiveOrExp  \\

 &    \| ConditionalAndExp \xcd"&&" InclusiveOrExp \\

\end{bbgrammarappendix}

\begin{bbgrammarappendix}{3.6in}

(\arabic{equation}) & ConditionalExp \refstepcounter{equation}\label{prod:ConditionalExp}  \: ConditionalOrExp  \\

 &    \| ClosureExp \\
 &    \| AtExp \\
 &    \| ConditionalOrExp \xcd"?" Exp \xcd":" ConditionalExp \\

\end{bbgrammarappendix}

\begin{bbgrammarappendix}{3.4in}

(\arabic{equation}) & ConditionalOrExp \refstepcounter{equation}\label{prod:ConditionalOrExp}  \: ConditionalAndExp  \\

 &    \| ConditionalOrExp \xcd"||" ConditionalAndExp \\

\end{bbgrammarappendix}

\begin{bbgrammarappendix}{3.9in}

(\arabic{equation}) & ConstantExp \refstepcounter{equation}\label{prod:ConstantExp}  \: Exp  \\


\end{bbgrammarappendix}

\begin{bbgrammarappendix}{3.5in}

(\arabic{equation}) & ConstrainedType \refstepcounter{equation}\label{prod:ConstrainedType}  \: NamedType  \\

 &    \| AnnotatedType \\

\end{bbgrammarappendix}

\begin{bbgrammarappendix}{2.9in}

(\arabic{equation}) & ConstraintConjunction \refstepcounter{equation}\label{prod:ConstraintConjunction}  \: Exp  \\

 &    \| ConstraintConjunction \xcd"," Exp \\

\end{bbgrammarappendix}

\begin{bbgrammarappendix}{3.8in}

(\arabic{equation}) & ContinueStmt \refstepcounter{equation}\label{prod:ContinueStmt}  \: \xcd"continue" Id\opt \xcd";"  \\


\end{bbgrammarappendix}

\begin{bbgrammarappendix}{3.3in}

(\arabic{equation}) & ConversionOpDecln \refstepcounter{equation}\label{prod:ConversionOpDecln}  \: ExplConvOpDecln  \\

 &    \| ImplConvOpDecln \\

\end{bbgrammarappendix}

\begin{bbgrammarappendix}{4.1in}

(\arabic{equation}) & CtorBlock \refstepcounter{equation}\label{prod:CtorBlock}  \: \xcd"{" ExplicitCtorInvo\opt BlockStmts\opt \xcd"}"  \\


\end{bbgrammarappendix}

\begin{bbgrammarappendix}{4.2in}

(\arabic{equation}) & CtorBody \refstepcounter{equation}\label{prod:CtorBody}  \: \xcd"=" CtorBlock  \\

 &    \| CtorBlock \\
 &    \| \xcd"=" ExplicitCtorInvo \\
 &    \| \xcd"=" AssignPropCall \\
 &    \| \xcd";" \\

\end{bbgrammarappendix}

\begin{bbgrammarappendix}{4.1in}

(\arabic{equation}) & CtorDecln \refstepcounter{equation}\label{prod:CtorDecln}  \: Mods\opt \xcd"def" \xcd"this" TypeParams\opt Formals Guard\opt HasResultType\opt CtorBody  \\


\end{bbgrammarappendix}

\begin{bbgrammarappendix}{3.8in}

(\arabic{equation}) & DepNamedType \refstepcounter{equation}\label{prod:DepNamedType}  \: SimpleNamedType DepParams  \\

 &    \| ParamizedNamedType DepParams \\

\end{bbgrammarappendix}

\begin{bbgrammarappendix}{4.4in}

(\arabic{equation}) & DoStmt \refstepcounter{equation}\label{prod:DoStmt}  \: \xcd"do" Stmt \xcd"while" \xcd"(" Exp \xcd")" \xcd";"  \\


\end{bbgrammarappendix}

\begin{bbgrammarappendix}{4.1in}

(\arabic{equation}) & EmptyStmt \refstepcounter{equation}\label{prod:EmptyStmt}  \: \xcd";"  \\


\end{bbgrammarappendix}

\begin{bbgrammarappendix}{3.5in}

(\arabic{equation}) & EnhancedForStmt \refstepcounter{equation}\label{prod:EnhancedForStmt}  \: \xcd"for" \xcd"(" LoopIndex \xcd"in" Exp \xcd")" Stmt  \\

 &    \| \xcd"for" \xcd"(" Exp \xcd")" Stmt \\

\end{bbgrammarappendix}

\begin{bbgrammarappendix}{3.9in}

(\arabic{equation}) & EqualityExp \refstepcounter{equation}\label{prod:EqualityExp}  \: RelationalExp  \\

 &    \| EqualityExp \xcd"==" RelationalExp \\
 &    \| EqualityExp \xcd"!=" RelationalExp \\
 &    \| Type  \xcd"==" Type  \\
 &    \| EqualityExp \xcd"~" RelationalExp \\
 &    \| EqualityExp \xcd"!~" RelationalExp \\

\end{bbgrammarappendix}

\begin{bbgrammarappendix}{3.6in}

(\arabic{equation}) & ExclusiveOrExp \refstepcounter{equation}\label{prod:ExclusiveOrExp}  \: AndExp  \\

 &    \| ExclusiveOrExp \xcd"^" AndExp \\

\end{bbgrammarappendix}

\begin{bbgrammarappendix}{4.7in}

(\arabic{equation}) & Exp \refstepcounter{equation}\label{prod:Exp}  \: AsstExp  \\


\end{bbgrammarappendix}

\begin{bbgrammarappendix}{4.3in}

(\arabic{equation}) & ExpName \refstepcounter{equation}\label{prod:ExpName}  \: Id  \\

 &    \| FullyQualifiedName \xcd"." Id \\

\end{bbgrammarappendix}

\begin{bbgrammarappendix}{4.3in}

(\arabic{equation}) & ExpStmt \refstepcounter{equation}\label{prod:ExpStmt}  \: StmtExp \xcd";"  \\


\end{bbgrammarappendix}

\begin{bbgrammarappendix}{3.5in}

(\arabic{equation}) & ExplConvOpDecln \refstepcounter{equation}\label{prod:ExplConvOpDecln}  \: MethMods \xcd"operator" TypeParams\opt \xcd"(" Formal  \xcd")" \xcd"as" Type Guard\opt MethodBody  \\

 &    \| MethMods \xcd"operator" TypeParams\opt \xcd"(" Formal  \xcd")" \xcd"as" \xcd"?" Guard\opt HasResultType\opt MethodBody \\

\end{bbgrammarappendix}

\begin{bbgrammarappendix}{3.4in}

(\arabic{equation}) & ExplicitCtorInvo \refstepcounter{equation}\label{prod:ExplicitCtorInvo}  \: \xcd"this" TypeArgs\opt \xcd"(" ArgumentList\opt \xcd")" \xcd";"  \\

 &    \| \xcd"super" TypeArgs\opt \xcd"(" ArgumentList\opt \xcd")" \xcd";" \\
 &    \| Primary \xcd"." \xcd"this" TypeArgs\opt \xcd"(" ArgumentList\opt \xcd")" \xcd";" \\
 &    \| Primary \xcd"." \xcd"super" TypeArgs\opt \xcd"(" ArgumentList\opt \xcd")" \xcd";" \\

\end{bbgrammarappendix}

\begin{bbgrammarappendix}{3.3in}

(\arabic{equation}) & ExtendsInterfaces \refstepcounter{equation}\label{prod:ExtendsInterfaces}  \: \xcd"extends" Type  \\

 &    \| ExtendsInterfaces \xcd"," Type \\

\end{bbgrammarappendix}

\begin{bbgrammarappendix}{3.9in}

(\arabic{equation}) & FieldAccess \refstepcounter{equation}\label{prod:FieldAccess}  \: Primary \xcd"." Id  \\

 &    \| \xcd"super" \xcd"." Id \\
 &    \| ClassName \xcd"." \xcd"super"  \xcd"." Id \\

\end{bbgrammarappendix}

\begin{bbgrammarappendix}{4.0in}

(\arabic{equation}) & FieldDecln \refstepcounter{equation}\label{prod:FieldDecln}  \: Mods\opt VarKeyword FieldDeclrs \xcd";"  \\

 &    \| Mods\opt FieldDeclrs \xcd";" \\

\end{bbgrammarappendix}

\begin{bbgrammarappendix}{4.0in}

(\arabic{equation}) & FieldDeclr \refstepcounter{equation}\label{prod:FieldDeclr}  \: Id HasResultType  \\

 &    \| Id HasResultType\opt \xcd"=" VariableInitializer \\

\end{bbgrammarappendix}

\begin{bbgrammarappendix}{3.9in}

(\arabic{equation}) & FieldDeclrs \refstepcounter{equation}\label{prod:FieldDeclrs}  \: FieldDeclr  \\

 &    \| FieldDeclrs \xcd"," FieldDeclr \\

\end{bbgrammarappendix}

\begin{bbgrammarappendix}{4.3in}

(\arabic{equation}) & Finally \refstepcounter{equation}\label{prod:Finally}  \: \xcd"finally" Block  \\


\end{bbgrammarappendix}

\begin{bbgrammarappendix}{4.0in}

(\arabic{equation}) & FinishStmt \refstepcounter{equation}\label{prod:FinishStmt}  \: \xcd"finish" Stmt  \\

 &    \| \xcd"clocked" \xcd"finish" Stmt \\

\end{bbgrammarappendix}

\begin{bbgrammarappendix}{4.3in}

(\arabic{equation}) & ForInit \refstepcounter{equation}\label{prod:ForInit}  \: StmtExpList  \\

 &    \| LocVarDecln \\

\end{bbgrammarappendix}

\begin{bbgrammarappendix}{4.3in}

(\arabic{equation}) & ForStmt \refstepcounter{equation}\label{prod:ForStmt}  \: BasicForStmt  \\

 &    \| EnhancedForStmt \\

\end{bbgrammarappendix}

\begin{bbgrammarappendix}{4.1in}

(\arabic{equation}) & ForUpdate \refstepcounter{equation}\label{prod:ForUpdate}  \: StmtExpList  \\


\end{bbgrammarappendix}

\begin{bbgrammarappendix}{4.4in}

(\arabic{equation}) & Formal \refstepcounter{equation}\label{prod:Formal}  \: Mods\opt FormalDeclr  \\

 &    \| Mods\opt VarKeyword FormalDeclr \\
 &    \| Type \\

\end{bbgrammarappendix}

\begin{bbgrammarappendix}{3.9in}

(\arabic{equation}) & FormalDeclr \refstepcounter{equation}\label{prod:FormalDeclr}  \: Id ResultType  \\

 &    \| \xcd"[" IdList \xcd"]" ResultType \\
 &    \| Id \xcd"[" IdList \xcd"]" ResultType \\

\end{bbgrammarappendix}

\begin{bbgrammarappendix}{3.8in}

(\arabic{equation}) & FormalDeclrs \refstepcounter{equation}\label{prod:FormalDeclrs}  \: FormalDeclr  \\

 &    \| FormalDeclrs \xcd"," FormalDeclr \\

\end{bbgrammarappendix}

\begin{bbgrammarappendix}{4.0in}

(\arabic{equation}) & FormalList \refstepcounter{equation}\label{prod:FormalList}  \: Formal  \\

 &    \| FormalList \xcd"," Formal \\

\end{bbgrammarappendix}

\begin{bbgrammarappendix}{4.3in}

(\arabic{equation}) & Formals \refstepcounter{equation}\label{prod:Formals}  \: \xcd"(" FormalList\opt \xcd")"  \\


\end{bbgrammarappendix}

\begin{bbgrammarappendix}{3.2in}

(\arabic{equation}) & FullyQualifiedName \refstepcounter{equation}\label{prod:FullyQualifiedName}  \: Id  \\

 &    \| FullyQualifiedName \xcd"." Id \\

\end{bbgrammarappendix}

\begin{bbgrammarappendix}{3.8in}

(\arabic{equation}) & FunctionType \refstepcounter{equation}\label{prod:FunctionType}  \: TypeParams\opt \xcd"(" FormalList\opt \xcd")" Guard\opt \xcd"=>" Type  \\


\end{bbgrammarappendix}

\begin{bbgrammarappendix}{4.5in}

(\arabic{equation}) & Guard \refstepcounter{equation}\label{prod:Guard}  \: DepParams  \\


\end{bbgrammarappendix}


\begin{bbgrammarappendix}{4.5in}

(\arabic{equation}) & Throws \refstepcounter{equation}\label{prod:Throws}  \: \xcd"throws" ThrowsList  \\

\end{bbgrammarappendix}

\begin{bbgrammarappendix}{3.3in}

(\arabic{equation}) & ThrowsList \refstepcounter{equation}\label{prod:ThrowsList}  \: Type  \\

 &    \| ThrowsList \xcd"," Type \\

\end{bbgrammarappendix}

\begin{bbgrammarappendix}{3.7in}

(\arabic{equation}) & HasResultType \refstepcounter{equation}\label{prod:HasResultType}  \: ResultType  \\

 &    \| \xcd"<:" Type \\

\end{bbgrammarappendix}

\begin{bbgrammarappendix}{3.3in}

(\arabic{equation}) & HasZeroConstraint \refstepcounter{equation}\label{prod:HasZeroConstraint}  \: Type  \xcd"haszero"  \\


\end{bbgrammarappendix}

\begin{bbgrammarappendix}{3.8in}

(\arabic{equation}) & HomeVariable \refstepcounter{equation}\label{prod:HomeVariable}  \: Id  \\

 &    \| \xcd"this" \\

\end{bbgrammarappendix}

\begin{bbgrammarappendix}{3.4in}

(\arabic{equation}) & HomeVariableList \refstepcounter{equation}\label{prod:HomeVariableList}  \: HomeVariable  \\

 &    \| HomeVariableList \xcd"," HomeVariable \\

\end{bbgrammarappendix}

\begin{bbgrammarappendix}{4.8in}

(\arabic{equation}) & Id \refstepcounter{equation}\label{prod:Id}  \: \xcd"IDENTIFIER"   \\


\end{bbgrammarappendix}

\begin{bbgrammarappendix}{4.4in}

(\arabic{equation}) & IdList \refstepcounter{equation}\label{prod:IdList}  \: Id  \\

 &    \| IdList \xcd"," Id \\

\end{bbgrammarappendix}

\begin{bbgrammarappendix}{3.6in}

(\arabic{equation}) & IfThenElseStmt \refstepcounter{equation}\label{prod:IfThenElseStmt}  \: \xcd"if" \xcd"(" Exp \xcd")" Stmt  \xcd"else" Stmt   \\


\end{bbgrammarappendix}

\begin{bbgrammarappendix}{4.0in}

(\arabic{equation}) & IfThenStmt \refstepcounter{equation}\label{prod:IfThenStmt}  \: \xcd"if" \xcd"(" Exp \xcd")" Stmt  \\


\end{bbgrammarappendix}

\begin{bbgrammarappendix}{3.5in}

(\arabic{equation}) & ImplConvOpDecln \refstepcounter{equation}\label{prod:ImplConvOpDecln}  \: MethMods \xcd"operator" TypeParams\opt \xcd"(" Formal  \xcd")" Guard\opt HasResultType\opt MethodBody  \\


\end{bbgrammarappendix}

\begin{bbgrammarappendix}{3.9in}

(\arabic{equation}) & ImportDecln \refstepcounter{equation}\label{prod:ImportDecln}  \: SingleTypeImportDecln  \\

 &    \| TypeImportOnDemandDecln \\

\end{bbgrammarappendix}

\begin{bbgrammarappendix}{3.8in}

(\arabic{equation}) & ImportDeclns \refstepcounter{equation}\label{prod:ImportDeclns}  \: ImportDecln  \\

 &    \| ImportDeclns ImportDecln \\

\end{bbgrammarappendix}

\begin{bbgrammarappendix}{3.6in}

(\arabic{equation}) & InclusiveOrExp \refstepcounter{equation}\label{prod:InclusiveOrExp}  \: ExclusiveOrExp  \\

 &    \| InclusiveOrExp \xcd"|" ExclusiveOrExp \\

\end{bbgrammarappendix}

\begin{bbgrammarappendix}{3.7in}

(\arabic{equation}) & InterfaceBody \refstepcounter{equation}\label{prod:InterfaceBody}  \: \xcd"{" InterfaceMemberDeclns\opt \xcd"}"  \\


\end{bbgrammarappendix}

\begin{bbgrammarappendix}{3.6in}

(\arabic{equation}) & InterfaceDecln \refstepcounter{equation}\label{prod:InterfaceDecln}  \: Mods\opt \xcd"interface" Id TypeParamsI\opt Properties\opt Guard\opt ExtendsInterfaces\opt InterfaceBody  \\


\end{bbgrammarappendix}

\begin{bbgrammarappendix}{3.0in}

(\arabic{equation}) & InterfaceMemberDecln \refstepcounter{equation}\label{prod:InterfaceMemberDecln}  \: MethodDecln  \\

 &    \| PropMethodDecln \\
 &    \| FieldDecln \\
 &    \| TypeDecln \\

\end{bbgrammarappendix}

\begin{bbgrammarappendix}{2.9in}

(\arabic{equation}) & InterfaceMemberDeclns \refstepcounter{equation}\label{prod:InterfaceMemberDeclns}  \: InterfaceMemberDecln  \\

 &    \| InterfaceMemberDeclns InterfaceMemberDecln \\

\end{bbgrammarappendix}

\begin{bbgrammarappendix}{3.3in}

(\arabic{equation}) & InterfaceTypeList \refstepcounter{equation}\label{prod:InterfaceTypeList}  \: Type  \\

 &    \| InterfaceTypeList \xcd"," Type \\

\end{bbgrammarappendix}

\begin{bbgrammarappendix}{4.0in}

(\arabic{equation}) & Interfaces \refstepcounter{equation}\label{prod:Interfaces}  \: \xcd"implements" InterfaceTypeList  \\


\end{bbgrammarappendix}

\begin{bbgrammarappendix}{3.9in}

(\arabic{equation}) & LabeledStmt \refstepcounter{equation}\label{prod:LabeledStmt}  \: Id \xcd":" LoopStmt  \\


\end{bbgrammarappendix}

\begin{bbgrammarappendix}{4.3in}

(\arabic{equation}) & LastExp \refstepcounter{equation}\label{prod:LastExp}  \: Exp  \\


\end{bbgrammarappendix}

\begin{bbgrammarappendix}{3.8in}

(\arabic{equation}) & LeftHandSide \refstepcounter{equation}\label{prod:LeftHandSide}  \: ExpName  \\

 &    \| FieldAccess \\

\end{bbgrammarappendix}

\begin{bbgrammarappendix}{4.3in}

(\arabic{equation}) & Literal \refstepcounter{equation}\label{prod:Literal}  \: \xcd"IntegerLiteral"   \\

 &    \| \xcd"LongLiteral"  \\
 &    \| \xcd"ByteLiteral" \\
 &    \| \xcd"UnsignedByteLiteral" \\
 &    \| \xcd"ShortLiteral" \\
 &    \| \xcd"UnsignedShortLiteral" \\
 &    \| \xcd"UnsignedIntegerLiteral"  \\
 &    \| \xcd"UnsignedLongLiteral"  \\
 &    \| \xcd"FloatingPointLiteral"  \\
 &    \| \xcd"DoubleLiteral"  \\
 &    \| BooleanLiteral \\
 &    \| \xcd"CharacterLiteral"  \\
 &    \| \xcd"StringLiteral"  \\
 &    \| \xcd"null" \\

\end{bbgrammarappendix}

\begin{bbgrammarappendix}{3.9in}

(\arabic{equation}) & LocVarDecln \refstepcounter{equation}\label{prod:LocVarDecln}  \: Mods\opt VarKeyword VariableDeclrs  \\

 &    \| Mods\opt VarDeclsWType \\
 &    \| Mods\opt VarKeyword FormalDeclrs \\

\end{bbgrammarappendix}

\begin{bbgrammarappendix}{3.5in}

(\arabic{equation}) & LocVarDeclnStmt \refstepcounter{equation}\label{prod:LocVarDeclnStmt}  \: LocVarDecln \xcd";"  \\


\end{bbgrammarappendix}

\begin{bbgrammarappendix}{4.1in}

(\arabic{equation}) & LoopIndex \refstepcounter{equation}\label{prod:LoopIndex}  \: Mods\opt LoopIndexDeclr  \\

 &    \| Mods\opt VarKeyword LoopIndexDeclr \\

\end{bbgrammarappendix}

\begin{bbgrammarappendix}{3.6in}

(\arabic{equation}) & LoopIndexDeclr \refstepcounter{equation}\label{prod:LoopIndexDeclr}  \: Id HasResultType\opt  \\

 &    \| \xcd"[" IdList \xcd"]" HasResultType\opt \\
 &    \| Id \xcd"[" IdList \xcd"]" HasResultType\opt \\

\end{bbgrammarappendix}

\begin{bbgrammarappendix}{4.2in}

(\arabic{equation}) & LoopStmt \refstepcounter{equation}\label{prod:LoopStmt}  \: ForStmt  \\

 &    \| WhileStmt \\
 &    \| DoStmt \\
 &    \| AtEachStmt \\

\end{bbgrammarappendix}

\begin{bbgrammarappendix}{4.2in}

(\arabic{equation}) & MethMods \refstepcounter{equation}\label{prod:MethMods}  \: Mods\opt  \\

 &    \| MethMods \xcd"property"  \\
 &    \| MethMods Mod \\

\end{bbgrammarappendix}

\begin{bbgrammarappendix}{4.0in}

(\arabic{equation}) & MethodBody \refstepcounter{equation}\label{prod:MethodBody}  \: \xcd"=" LastExp \xcd";"  \\

 &    \| \xcd"=" Annotations\opt \xcd"{" BlockStmts\opt LastExp \xcd"}" \\
 &    \| \xcd"=" Annotations\opt Block \\
 &    \| Annotations\opt Block \\
 &    \| \xcd";" \\

\end{bbgrammarappendix}

\begin{bbgrammarappendix}{3.9in}

(\arabic{equation}) & MethodDecln
  \refstepcounter{equation}\label{prod:MethodDecln}  \: MethMods \xcd"def" Id TypeParams\opt Formals Guard\opt Throws\opt HasResultType\opt MethodBody  \\

 &    \| BinOpDecln \\
 &    \| PrefixOpDecln \\
 &    \| ApplyOpDecln \\
 &    \| SetOpDecln \\
 &    \| ConversionOpDecln \\

\end{bbgrammarappendix}

\begin{bbgrammarappendix}{4.0in}

(\arabic{equation}) & MethodInvo \refstepcounter{equation}\label{prod:MethodInvo}  \: MethodName TypeArgs\opt \xcd"(" ArgumentList\opt \xcd")"  \\

 &    \| Primary \xcd"." Id TypeArgs\opt \xcd"(" ArgumentList\opt \xcd")" \\
 &    \| \xcd"super" \xcd"." Id TypeArgs\opt \xcd"(" ArgumentList\opt \xcd")" \\
 &    \| ClassName \xcd"." \xcd"super"  \xcd"." Id TypeArgs\opt \xcd"(" ArgumentList\opt \xcd")" \\
 &    \| Primary TypeArgs\opt \xcd"(" ArgumentList\opt \xcd")" \\

\end{bbgrammarappendix}

\begin{bbgrammarappendix}{4.0in}

(\arabic{equation}) & MethodName \refstepcounter{equation}\label{prod:MethodName}  \: Id  \\

 &    \| FullyQualifiedName \xcd"." Id \\

\end{bbgrammarappendix}

\begin{bbgrammarappendix}{4.7in}

(\arabic{equation}) & Mod \refstepcounter{equation}\label{prod:Mod}  \: \xcd"abstract"  \\

 &    \| Annotation \\
 &    \| \xcd"atomic" \\
 &    \| \xcd"final" \\
 &    \| \xcd"native" \\
 &    \| \xcd"private" \\
 &    \| \xcd"protected" \\
 &    \| \xcd"public" \\
 &    \| \xcd"static" \\
 &    \| \xcd"transient" \\
 &    \| \xcd"clocked" \\

\end{bbgrammarappendix}

\begin{bbgrammarappendix}{3.3in}

(\arabic{equation}) & MultiplicativeExp \refstepcounter{equation}\label{prod:MultiplicativeExp}  \: RangeExp  \\

 &    \| MultiplicativeExp \xcd"*" RangeExp \\
 &    \| MultiplicativeExp \xcd"/" RangeExp \\
 &    \| MultiplicativeExp \xcd"%" RangeExp \\
 &    \| MultiplicativeExp \xcd"**" RangeExp \\

\end{bbgrammarappendix}

\begin{bbgrammarappendix}{4.1in}

(\arabic{equation}) & NamedType \refstepcounter{equation}\label{prod:NamedType}  \: NamedTypeNoConstraints  \\

 &    \| DepNamedType \\

\end{bbgrammarappendix}

\begin{bbgrammarappendix}{2.8in}

(\arabic{equation}) & NamedTypeNoConstraints \refstepcounter{equation}\label{prod:NamedTypeNoConstraints}  \: SimpleNamedType  \\

 &    \| ParamizedNamedType \\

\end{bbgrammarappendix}

\begin{bbgrammarappendix}{4.0in}

(\arabic{equation}) & NonExpStmt \refstepcounter{equation}\label{prod:NonExpStmt}  \: Block  \\

 &    \| EmptyStmt \\
 &    \| AssertStmt \\
 &    \| SwitchStmt \\
 &    \| DoStmt \\
 &    \| BreakStmt \\
 &    \| ContinueStmt \\
 &    \| ReturnStmt \\
 &    \| ThrowStmt \\
 &    \| TryStmt \\
 &    \| LabeledStmt \\
 &    \| IfThenStmt \\
 &    \| IfThenElseStmt \\
 &    \| WhileStmt \\
 &    \| ForStmt \\
 &    \| AsyncStmt \\
 &    \| AtStmt \\
 &    \| AtomicStmt \\
 &    \| WhenStmt \\
 &    \| AtEachStmt \\
 &    \| FinishStmt \\
 &    \| AssignPropCall \\

\end{bbgrammarappendix}

\begin{bbgrammarappendix}{3.7in}

(\arabic{equation}) & ObCreationExp \refstepcounter{equation}\label{prod:ObCreationExp}  \: \xcd"new" TypeName TypeArgs\opt \xcd"(" ArgumentList\opt \xcd")" ClassBody\opt  \\

 &    \| Primary \xcd"." \xcd"new" Id TypeArgs\opt \xcd"(" ArgumentList\opt \xcd")" ClassBody\opt \\
 &    \| FullyQualifiedName \xcd"." \xcd"new" Id TypeArgs\opt \xcd"(" ArgumentList\opt \xcd")" ClassBody\opt \\

\end{bbgrammarappendix}

\begin{bbgrammarappendix}{3.8in}

(\arabic{equation}) & PackageDecln \refstepcounter{equation}\label{prod:PackageDecln}  \: Annotations\opt \xcd"package" PackageName \xcd";"  \\


\end{bbgrammarappendix}

\begin{bbgrammarappendix}{3.9in}

(\arabic{equation}) & PackageName \refstepcounter{equation}\label{prod:PackageName}  \: Id  \\

 &    \| PackageName \xcd"." Id \\

\end{bbgrammarappendix}

\begin{bbgrammarappendix}{3.3in}

(\arabic{equation}) & PackageOrTypeName \refstepcounter{equation}\label{prod:PackageOrTypeName}  \: Id  \\

 &    \| PackageOrTypeName \xcd"." Id \\

\end{bbgrammarappendix}

\begin{bbgrammarappendix}{3.2in}

(\arabic{equation}) & ParamizedNamedType \refstepcounter{equation}\label{prod:ParamizedNamedType}  \: SimpleNamedType Arguments  \\

 &    \| SimpleNamedType TypeArgs \\
 &    \| SimpleNamedType TypeArgs Arguments \\

\end{bbgrammarappendix}

\begin{bbgrammarappendix}{3.4in}

(\arabic{equation}) & PostDecrementExp \refstepcounter{equation}\label{prod:PostDecrementExp}  \: PostfixExp \xcd"--"  \\


\end{bbgrammarappendix}

\begin{bbgrammarappendix}{3.4in}

(\arabic{equation}) & PostIncrementExp \refstepcounter{equation}\label{prod:PostIncrementExp}  \: PostfixExp \xcd"++"  \\


\end{bbgrammarappendix}

\begin{bbgrammarappendix}{4.0in}

(\arabic{equation}) & PostfixExp \refstepcounter{equation}\label{prod:PostfixExp}  \: CastExp  \\

 &    \| PostIncrementExp \\
 &    \| PostDecrementExp \\

\end{bbgrammarappendix}

\begin{bbgrammarappendix}{3.5in}

(\arabic{equation}) & PreDecrementExp \refstepcounter{equation}\label{prod:PreDecrementExp}  \: \xcd"--" UnaryExpNotPlusMinus  \\


\end{bbgrammarappendix}

\begin{bbgrammarappendix}{3.5in}

(\arabic{equation}) & PreIncrementExp \refstepcounter{equation}\label{prod:PreIncrementExp}  \: \xcd"++" UnaryExpNotPlusMinus  \\


\end{bbgrammarappendix}

\begin{bbgrammarappendix}{4.2in}

(\arabic{equation}) & PrefixOp \refstepcounter{equation}\label{prod:PrefixOp}  \: \xcd"+"  \\

 &    \| \xcd"-" \\
 &    \| \xcd"!" \\
 &    \| \xcd"~" \\
 &    \| \xcd"^" \\
 &    \| \xcd"|" \\
 &    \| \xcd"&" \\
 &    \| \xcd"*" \\
 &    \| \xcd"/" \\
 &    \| \xcd"%" \\

\end{bbgrammarappendix}

\begin{bbgrammarappendix}{3.7in}

(\arabic{equation}) & PrefixOpDecln \refstepcounter{equation}\label{prod:PrefixOpDecln}  \: MethMods \xcd"operator" TypeParams\opt PrefixOp \xcd"(" Formal  \xcd")" Guard\opt HasResultType\opt MethodBody  \\

 &    \| MethMods \xcd"operator" TypeParams\opt PrefixOp \xcd"this" Guard\opt HasResultType\opt MethodBody \\

\end{bbgrammarappendix}

\begin{bbgrammarappendix}{4.3in}

(\arabic{equation}) & Primary \refstepcounter{equation}\label{prod:Primary}  \: \xcd"here"  \\

 &    \| \xcd"[" ArgumentList\opt \xcd"]" \\
 &    \| Literal \\
 &    \| \xcd"self" \\
 &    \| \xcd"this" \\
 &    \| ClassName \xcd"." \xcd"this" \\
 &    \| \xcd"(" Exp \xcd")" \\
 &    \| ObCreationExp \\
 &    \| FieldAccess \\
 &    \| MethodInvo \\

\end{bbgrammarappendix}

\begin{bbgrammarappendix}{4.6in}

(\arabic{equation}) & Prop \refstepcounter{equation}\label{prod:Prop}  \: Annotations\opt Id ResultType  \\


\end{bbgrammarappendix}

\begin{bbgrammarappendix}{4.2in}

(\arabic{equation}) & PropList \refstepcounter{equation}\label{prod:PropList}  \: Prop  \\

 &    \| PropList \xcd"," Prop \\

\end{bbgrammarappendix}

\begin{bbgrammarappendix}{3.5in}

(\arabic{equation}) & PropMethodDecln \refstepcounter{equation}\label{prod:PropMethodDecln}  \: MethMods Id TypeParams\opt Formals Guard\opt HasResultType\opt MethodBody  \\

 &    \| MethMods Id Guard\opt HasResultType\opt MethodBody \\

\end{bbgrammarappendix}

\begin{bbgrammarappendix}{4.0in}

(\arabic{equation}) & Properties \refstepcounter{equation}\label{prod:Properties}  \: \xcd"(" PropList \xcd")"  \\


\end{bbgrammarappendix}

\begin{bbgrammarappendix}{4.2in}

(\arabic{equation}) & RangeExp \refstepcounter{equation}\label{prod:RangeExp}  \: UnaryExp  \\

 &    \| RangeExp  \xcd".." UnaryExp  \\

\end{bbgrammarappendix}

\begin{bbgrammarappendix}{3.7in}

(\arabic{equation}) & RelationalExp \refstepcounter{equation}\label{prod:RelationalExp}  \: ShiftExp  \\

 &    \| HasZeroConstraint \\
 &    \| SubtypeConstraint \\
 &    \| RelationalExp \xcd"<" ShiftExp \\
 &    \| RelationalExp \xcd">" ShiftExp \\
 &    \| RelationalExp \xcd"<=" ShiftExp \\
 &    \| RelationalExp \xcd">=" ShiftExp \\
 &    \| RelationalExp \xcd"instanceof" Type \\

\end{bbgrammarappendix}

\begin{bbgrammarappendix}{4.0in}

(\arabic{equation}) & ResultType \refstepcounter{equation}\label{prod:ResultType}  \: \xcd":" Type  \\


\end{bbgrammarappendix}

\begin{bbgrammarappendix}{4.0in}

(\arabic{equation}) & ReturnStmt \refstepcounter{equation}\label{prod:ReturnStmt}  \: \xcd"return" Exp\opt \xcd";"  \\


\end{bbgrammarappendix}

\begin{bbgrammarappendix}{4.0in}

(\arabic{equation}) & SetOpDecln \refstepcounter{equation}\label{prod:SetOpDecln}  \: MethMods \xcd"operator" \xcd"this" TypeParams\opt Formals \xcd"=" \xcd"(" Formal  \xcd")" Guard\opt HasResultType\opt MethodBody  \\


\end{bbgrammarappendix}

\begin{bbgrammarappendix}{4.2in}

(\arabic{equation}) & ShiftExp \refstepcounter{equation}\label{prod:ShiftExp}  \: AdditiveExp  \\

 &    \| ShiftExp \xcd"<<" AdditiveExp \\
 &    \| ShiftExp \xcd">>" AdditiveExp \\
 &    \| ShiftExp \xcd">>>" AdditiveExp \\
 &    \| ShiftExp  \xcd"->" AdditiveExp  \\
 &    \| ShiftExp  \xcd"<-" AdditiveExp  \\
 &    \| ShiftExp  \xcd"-<" AdditiveExp  \\
 &    \| ShiftExp  \xcd">-" AdditiveExp  \\
 &    \| ShiftExp  \xcd"!" AdditiveExp  \\

\end{bbgrammarappendix}

\begin{bbgrammarappendix}{3.5in}

(\arabic{equation}) & SimpleNamedType \refstepcounter{equation}\label{prod:SimpleNamedType}  \: TypeName  \\

 &    \| Primary \xcd"." Id \\
 &    \| ParamizedNamedType \xcd"." Id \\
 &    \| DepNamedType \xcd"." Id \\

\end{bbgrammarappendix}

\begin{bbgrammarappendix}{2.9in}

(\arabic{equation}) & SingleTypeImportDecln \refstepcounter{equation}\label{prod:SingleTypeImportDecln}  \: \xcd"import" TypeName \xcd";"  \\


\end{bbgrammarappendix}

\begin{bbgrammarappendix}{4.6in}

(\arabic{equation}) & Stmt \refstepcounter{equation}\label{prod:Stmt}  \: AnnotationStmt  \\

 &    \| ExpStmt \\

\end{bbgrammarappendix}

\begin{bbgrammarappendix}{4.3in}

(\arabic{equation}) & StmtExp \refstepcounter{equation}\label{prod:StmtExp}  \: Assignment  \\

 &    \| PreIncrementExp \\
 &    \| PreDecrementExp \\
 &    \| PostIncrementExp \\
 &    \| PostDecrementExp \\
 &    \| MethodInvo \\
 &    \| ObCreationExp \\

\end{bbgrammarappendix}

\begin{bbgrammarappendix}{3.9in}

(\arabic{equation}) & StmtExpList \refstepcounter{equation}\label{prod:StmtExpList}  \: StmtExp  \\

 &    \| StmtExpList \xcd"," StmtExp \\

\end{bbgrammarappendix}

\begin{bbgrammarappendix}{3.9in}

(\arabic{equation}) & StructDecln \refstepcounter{equation}\label{prod:StructDecln}  \: Mods\opt \xcd"struct" Id TypeParamsI\opt Properties\opt Guard\opt Interfaces\opt ClassBody  \\


\end{bbgrammarappendix}

\begin{bbgrammarappendix}{3.3in}

(\arabic{equation}) & SubtypeConstraint \refstepcounter{equation}\label{prod:SubtypeConstraint}  \: Type  \xcd"<:" Type   \\

 &    \| Type  \xcd":>" Type  \\

\end{bbgrammarappendix}

\begin{bbgrammarappendix}{4.5in}

(\arabic{equation}) & Super \refstepcounter{equation}\label{prod:Super}  \: \xcd"extends" ClassType  \\


\end{bbgrammarappendix}

\begin{bbgrammarappendix}{3.9in}

(\arabic{equation}) & SwitchBlock \refstepcounter{equation}\label{prod:SwitchBlock}  \: \xcd"{" SwitchBlockGroups\opt SwitchLabels\opt \xcd"}"  \\


\end{bbgrammarappendix}

\begin{bbgrammarappendix}{3.4in}

(\arabic{equation}) & SwitchBlockGroup \refstepcounter{equation}\label{prod:SwitchBlockGroup}  \: SwitchLabels BlockStmts  \\


\end{bbgrammarappendix}

\begin{bbgrammarappendix}{3.3in}

(\arabic{equation}) & SwitchBlockGroups \refstepcounter{equation}\label{prod:SwitchBlockGroups}  \: SwitchBlockGroup  \\

 &    \| SwitchBlockGroups SwitchBlockGroup \\

\end{bbgrammarappendix}

\begin{bbgrammarappendix}{3.9in}

(\arabic{equation}) & SwitchLabel \refstepcounter{equation}\label{prod:SwitchLabel}  \: \xcd"case" ConstantExp \xcd":"  \\

 &    \| \xcd"default" \xcd":" \\

\end{bbgrammarappendix}

\begin{bbgrammarappendix}{3.8in}

(\arabic{equation}) & SwitchLabels \refstepcounter{equation}\label{prod:SwitchLabels}  \: SwitchLabel  \\

 &    \| SwitchLabels SwitchLabel \\

\end{bbgrammarappendix}

\begin{bbgrammarappendix}{4.0in}

(\arabic{equation}) & SwitchStmt \refstepcounter{equation}\label{prod:SwitchStmt}  \: \xcd"switch" \xcd"(" Exp \xcd")" SwitchBlock  \\


\end{bbgrammarappendix}

\begin{bbgrammarappendix}{4.1in}

(\arabic{equation}) & ThrowStmt \refstepcounter{equation}\label{prod:ThrowStmt}  \: \xcd"throw" Exp \xcd";"  \\


\end{bbgrammarappendix}

\begin{bbgrammarappendix}{4.3in}

(\arabic{equation}) & TryStmt \refstepcounter{equation}\label{prod:TryStmt}  \: \xcd"try" Block Catches  \\

 &    \| \xcd"try" Block Catches\opt Finally \\

\end{bbgrammarappendix}

\begin{bbgrammarappendix}{4.6in}

(\arabic{equation}) & Type \refstepcounter{equation}\label{prod:Type}  \: FunctionType  \\

 &    \| ConstrainedType \\
 &    \| Void \\

\end{bbgrammarappendix}

\begin{bbgrammarappendix}{4.2in}

(\arabic{equation}) & TypeArgs \refstepcounter{equation}\label{prod:TypeArgs}  \: \xcd"[" TypeArgumentList \xcd"]"  \\


\end{bbgrammarappendix}

\begin{bbgrammarappendix}{3.4in}

(\arabic{equation}) & TypeArgumentList \refstepcounter{equation}\label{prod:TypeArgumentList}  \: Type  \\

 &    \| TypeArgumentList \xcd"," Type \\

\end{bbgrammarappendix}

\begin{bbgrammarappendix}{4.1in}

(\arabic{equation}) & TypeDecln \refstepcounter{equation}\label{prod:TypeDecln}  \: ClassDecln  \\

 &    \| StructDecln \\
 &    \| InterfaceDecln \\
 &    \| TypeDefDecln \\
 &    \| \xcd";" \\

\end{bbgrammarappendix}

\begin{bbgrammarappendix}{4.0in}

(\arabic{equation}) & TypeDeclns \refstepcounter{equation}\label{prod:TypeDeclns}  \: TypeDecln  \\

 &    \| TypeDeclns TypeDecln \\

\end{bbgrammarappendix}

\begin{bbgrammarappendix}{3.8in}

(\arabic{equation}) & TypeDefDecln \refstepcounter{equation}\label{prod:TypeDefDecln}  \: Mods\opt \xcd"type" Id TypeParams\opt Guard\opt \xcd"=" Type \xcd";"  \\

 &    \| Mods\opt \xcd"type" Id TypeParams\opt \xcd"(" FormalList \xcd")" Guard\opt \xcd"=" Type \xcd";" \\

\end{bbgrammarappendix}

\begin{bbgrammarappendix}{2.7in}

(\arabic{equation}) & TypeImportOnDemandDecln \refstepcounter{equation}\label{prod:TypeImportOnDemandDecln}  \: \xcd"import" PackageOrTypeName \xcd"." \xcd"*" \xcd";"  \\


\end{bbgrammarappendix}

\begin{bbgrammarappendix}{4.2in}

(\arabic{equation}) & TypeName \refstepcounter{equation}\label{prod:TypeName}  \: Id  \\

 &    \| TypeName \xcd"." Id \\

\end{bbgrammarappendix}

\begin{bbgrammarappendix}{4.1in}

(\arabic{equation}) & TypeParam \refstepcounter{equation}\label{prod:TypeParam}  \: Id  \\


\end{bbgrammarappendix}

\begin{bbgrammarappendix}{3.6in}

(\arabic{equation}) & TypeParamIList \refstepcounter{equation}\label{prod:TypeParamIList}  \: TypeParam  \\

 &    \| TypeParamIList \xcd"," TypeParam \\
 &    \| TypeParamIList \xcd"," \\

\end{bbgrammarappendix}

\begin{bbgrammarappendix}{3.7in}

(\arabic{equation}) & TypeParamList \refstepcounter{equation}\label{prod:TypeParamList}  \: TypeParam  \\

 &    \| TypeParamList \xcd"," TypeParam \\

\end{bbgrammarappendix}

\begin{bbgrammarappendix}{4.0in}

(\arabic{equation}) & TypeParams \refstepcounter{equation}\label{prod:TypeParams}  \: \xcd"[" TypeParamList \xcd"]"  \\


\end{bbgrammarappendix}

\begin{bbgrammarappendix}{3.9in}

(\arabic{equation}) & TypeParamsI \refstepcounter{equation}\label{prod:TypeParamsI}  \: \xcd"[" TypeParamIList \xcd"]"  \\


\end{bbgrammarappendix}

\begin{bbgrammarappendix}{3.1in}

(\arabic{equation}) & UnannotatedUnaryExp \refstepcounter{equation}\label{prod:UnannotatedUnaryExp}  \: PreIncrementExp  \\

 &    \| PreDecrementExp \\
 &    \| \xcd"+" UnaryExpNotPlusMinus \\
 &    \| \xcd"-" UnaryExpNotPlusMinus \\
 &    \| UnaryExpNotPlusMinus \\

\end{bbgrammarappendix}

\begin{bbgrammarappendix}{4.2in}

(\arabic{equation}) & UnaryExp \refstepcounter{equation}\label{prod:UnaryExp}  \: UnannotatedUnaryExp  \\

 &    \| Annotations UnannotatedUnaryExp \\

\end{bbgrammarappendix}

\begin{bbgrammarappendix}{3.0in}

(\arabic{equation}) & UnaryExpNotPlusMinus \refstepcounter{equation}\label{prod:UnaryExpNotPlusMinus}  \: PostfixExp  \\

 &    \| \xcd"~" UnaryExp \\
 &    \| \xcd"!" UnaryExp \\
 &    \| \xcd"^" UnaryExp \\
 &    \| \xcd"|" UnaryExp \\
 &    \| \xcd"&" UnaryExp \\
 &    \| \xcd"*" UnaryExp \\
 &    \| \xcd"/" UnaryExp \\
 &    \| \xcd"%" UnaryExp \\

\end{bbgrammarappendix}

\begin{bbgrammarappendix}{3.8in}

(\arabic{equation}) & VarDeclWType \refstepcounter{equation}\label{prod:VarDeclWType}  \: Id HasResultType \xcd"=" VariableInitializer  \\

 &    \| \xcd"[" IdList \xcd"]" HasResultType \xcd"=" VariableInitializer \\
 &    \| Id \xcd"[" IdList \xcd"]" HasResultType \xcd"=" VariableInitializer \\

\end{bbgrammarappendix}

\begin{bbgrammarappendix}{3.7in}

(\arabic{equation}) & VarDeclsWType \refstepcounter{equation}\label{prod:VarDeclsWType}  \: VarDeclWType  \\

 &    \| VarDeclsWType \xcd"," VarDeclWType \\

\end{bbgrammarappendix}

\begin{bbgrammarappendix}{4.0in}

(\arabic{equation}) & VarKeyword \refstepcounter{equation}\label{prod:VarKeyword}  \: \xcd"val"  \\

 &    \| \xcd"var" \\

\end{bbgrammarappendix}

\begin{bbgrammarappendix}{3.7in}

(\arabic{equation}) & VariableDeclr \refstepcounter{equation}\label{prod:VariableDeclr}  \: Id HasResultType\opt \xcd"=" VariableInitializer  \\

 &    \| \xcd"[" IdList \xcd"]" HasResultType\opt \xcd"=" VariableInitializer \\
 &    \| Id \xcd"[" IdList \xcd"]" HasResultType\opt \xcd"=" VariableInitializer \\

\end{bbgrammarappendix}

\begin{bbgrammarappendix}{3.6in}

(\arabic{equation}) & VariableDeclrs \refstepcounter{equation}\label{prod:VariableDeclrs}  \: VariableDeclr  \\

 &    \| VariableDeclrs \xcd"," VariableDeclr \\

\end{bbgrammarappendix}

\begin{bbgrammarappendix}{3.1in}

(\arabic{equation}) & VariableInitializer \refstepcounter{equation}\label{prod:VariableInitializer}  \: Exp  \\


\end{bbgrammarappendix}

\begin{bbgrammarappendix}{4.6in}

(\arabic{equation}) & Void \refstepcounter{equation}\label{prod:Void}  \: \xcd"void"  \\


\end{bbgrammarappendix}

\begin{bbgrammarappendix}{4.2in}

(\arabic{equation}) & WhenStmt \refstepcounter{equation}\label{prod:WhenStmt}  \: \xcd"when" \xcd"(" Exp \xcd")" Stmt  \\


\end{bbgrammarappendix}

\begin{bbgrammarappendix}{4.1in}

(\arabic{equation}) & WhileStmt \refstepcounter{equation}\label{prod:WhileStmt}  \: \xcd"while" \xcd"(" Exp \xcd")" Stmt  \\


\end{bbgrammarappendix}



\renewcommand{\bibname}{References}
\bibliographystyle{plain}
\bibliography{master}


\clearpage
%\documentclass[10pt,twoside,twocolumn,notitlepage]{report}
\documentclass[12pt,twoside,notitlepage]{report}
\usepackage{tex/x10}
\usepackage{tex/tenv}
\def\Hat{{\tt \char`\^}}
\usepackage{url}
\usepackage{times}
\usepackage{tex/txtt}
\usepackage{ifpdf}
\usepackage{tocloft}
\usepackage{tex/bcprules}
\usepackage{xspace}

\newif\ifdraft
%\drafttrue
\draftfalse

\pagestyle{headings}
\showboxdepth=0
\makeindex

\usepackage{tex/commands}

\usepackage[
pdfauthor={Vijay Saraswat, Bard Bloom, Igor Peshansky, Olivier Tardieu, and David Grove},
pdftitle={X10 Language Specification},
pdfcreator={pdftex},
pdfkeywords={X10},
linkcolor=blue,
citecolor=blue,
urlcolor=blue
]{hyperref}

\ifpdf
          \pdfinfo {
              /Author   (Vijay Saraswat, Bard Bloom, Igor Peshansky, Olivier Tardieu, and David Grove)
              /Title    (X10 Language Specification)
              /Keywords (X10)
              /Subject  ()
              /Creator  (TeX)
              /Producer (PDFLaTeX)
          }
\fi

\def\headertitle{The \XtenCurrVer{} Report (Draft) }
\def\integerversion{2.2}

% Sizes and dimensions

%\topmargin -.375in       %    Nominal distance from top of page to top of
                         %    box containing running head.
%\headsep 15pt            %    Space between running head and text.

%\textheight 9.0in        % Height of text (including footnotes and figures, 
                         % excluding running head and foot).

%\textwidth 5.5in         % Width of text line.
\columnsep 15pt          % Space between columns 
\columnseprule 0pt       % Width of rule between columns.

\parskip 5pt plus 2pt minus 2pt % Extra vertical space between paragraphs.
\parindent 0pt                  % Width of paragraph indentation.
%\topsep 0pt plus 2pt            % Extra vertical space, in addition to 
                                % \parskip, added above and below list and
                                % paragraphing environments.


\newif\iftwocolumn

\makeatletter
\twocolumnfalse
\if@twocolumn
\twocolumntrue
\fi
\makeatother

\iftwocolumn

\oddsidemargin  0in    % Left margin on odd-numbered pages.
\evensidemargin 0in    % Left margin on even-numbered pages.

\else

\oddsidemargin  .5in    % Left margin on odd-numbered pages.
\evensidemargin .5in    % Left margin on even-numbered pages.

\fi


\newtenv{example}{Example}[section]
\newtenv{planned}{Planned}[section]

\begin{document}

% \section{Work In Progress}
% \begin{itemize}
%     \item Rewrite first chapter
%     \item Describe library classes, including such fundamentals as Object and String
%     \item Examples for covariant/contravariant generics are wrong -- use Nate's examples
%     \item Describe local classes.
%     \item Reduce the use of \xcd`self` in constraints.
%     \item Copy sections of grammar to relevant sections of text.
%     \item Do something about 4.12.3
% \end{itemize}
% 
% {\bf Feedback:} 
% To help us the most, we would appreciate comments in one of these formats: 
% \begin{itemize}
% \item An annotated copy of the PDF document, if it's convenient.  Acrobat
%       Writer can produce helpful highlighting and sticky notes.  If you don't
%       use Acrobat Writer, don't fuss.
% \item Text comments.  Since the document is still being edited, page numbers
%       are going to be useless as pointers to the text.  If possible, we'd like
%       pointers to sections by number and title: {\em In 12.1, ``Empty
%       Statement'', please discuss side effects and performance implications
%       for this construct''}  If it's a long section, giving us a couple words
%       we can grep for would help too.
% \end{itemize}
% 
% Thank you very much!




% \parindent 0pt %!! 15pt                    % Width of paragraph indentation.

%\hfil {\bf 7 Feb 2005}
%\hfil \today{}

\input{first} 

\clearpage

{\parskip 0pt
\addtolength{\cftsecnumwidth}{0.5em}
\addtolength{\cftsubsecnumwidth}{0.5em}
%\addtolength{\cftsecindent}{0.5em}
\addtolength{\cftsubsecindent}{0.5em}
\tableofcontents
}


\input{Intro}
\input{Overview}
\input{Lex}
\input{Types}	
\input{Vars}
\input{Packages}
\input{Interfaces}
\input{Classes}
\input{Structs}
\input{Functions}
\input{Expressions}	
\input{Statements}	
\input{Places}	
\input{Activities}	
\input{Clocks}	
\input{Arrays}	
\input{Annotations}
\input{NativeCode}
\input{DefiniteAssignment}
\input{Grammar}


\renewcommand{\bibname}{References}
\bibliographystyle{plain}
\bibliography{master}


\clearpage
\input{x10.ind}

\appendix

\input{ChangeLog.tex}
\input{Options.tex}



{\em The \Xten{} language has been developed as part of the IBM PERCS
Project, which is supported in part by the Defense Advanced Research
Projects Agency (DARPA) under contract No. NBCH30390004.}

{\em Java and all Java-based trademarks are trademarks of Sun Microsystems,
Inc. in the United States, other countries, or both.}
\end{document}


\appendix

\chapter{Change Log}

\section{Changes from \Xten{} v2.2}

\begin{enumerate}

\item In previous versions of \Xten{} static fields were
  eagerly initialized in \xcd{Place 0} and the resulting values were
  serialized to all other places before execution of the user main
  function was started. Starting with \Xten{} v2.3, static fields are
  lazily initialized on a per-Place basis when the field is first read
  by an activity executing in a given Place.

\end{enumerate}

\section{Changes from \Xten{} v2.1}

\begin{enumerate}

\item Covariance and contravariance are gone.

\item Operator definitions are regularized.  A number of new operator symbols
      are available.

\item The operator \xcd`in` is gone.  \xcd`in` is now only a keyword.

\item Method functions and operator functions are gone.

\item \xcd`m..n` is now a type of struct called \xcd`IntRange`.  

\item \xcd`for(i in m..n)` now works.  The old forms, \xcd`for((i) in m..n)`
      and \xcd`for([i] in m..n)`, are no longer needed.

\item \xcd`(e as T)` now has type \xcd`T`.  (It used to have an identity
      constraint conjoined in.)

\item \xcd`var`s can no longer be assigned in their place of origin.  Use a
      \xcd`GlobalRef[Cell[T]]` instead.  We'll have a new idiom for this in 2.3.

\item The \xcd`-STATIC_CALLS` command-line flag is now \xcd`-STATIC_CHECKS`.

\item Any string may be written in backquotes to make an identifier: {\tt
      `while`}.

\item The \xcd`next` and \xcd`resume` keywords are gone; they have been
      replaced by static methods on \xcd`Clock`.

\item The typed array construction syntax \xcd`new Array[T][t1,t2]` is gone.
      Use \xcd`[t1 as T, t2]` (if just plain \xcd`[t1,t2]` doesn't work).

\end{enumerate}


\section{Changes from \Xten{} v2.0.6}

This document summarizes the main changes between X10 2.0.6 and X10 2.1.  The
descriptions are intended to be suggestive rather than definitive; see the
language specification for full details.

\subsection{Object Model}

\begin{enumerate}
\item Objects are now local rather than global.
   
    \begin{enumerate}
    \item The \Xcd{home} property is gone.
    \item \Xcd{at(P)S} produces deep copies of all objects reachable from
          lexically exposed variables in \xcd`S` 
          when it executes \Xcd{S}.  ({\bf Warning:} They are copied even in  
          \Xcd{at(here)S}.)
    \end{enumerate}

\item The \Xcd{GlobalRef[T]} struct is the only way to produce or manipulate
      cross-place references.
    \begin{enumerate}
    \item \Xcd{GlobalRef}'s have a \Xcd{home} property.
    \item Use \Xcd{GlobalRef[Foo](foo)} to make a new global reference.
    \item Use \Xcd{myGlobalRef()} to access the object referenced; this
          requires \Xcd{here == myGlobalRef.home}. 
    \end{enumerate}


\item  The \xcd`!` type modifier is no longer needed or present.

\item \Xcd{global} modifiers are now gone:
    
    \begin{enumerate}
    \item \Xcd{global} methods in {\em interfaces} are now the default. 
    \item \Xcd{global} {\em fields} are gone.  In some cases object copying
          will produce the same effect as global fields.  In other cases code
          must be rewritten.  It may be desirable to mark nonglobal fields
          \Xcd{transient} in many cases.
    \item \Xcd{global} {\em methods} are now marked \Xcd{@Global} instead.  
          Methods intended to be non-global may be marked \Xcd{@Pinned}.
    \end{enumerate}


\end{enumerate}

\subsection{Constructors}


\begin{enumerate}
\item \Xcd{proto} types are gone.
\item Constructors and the methods they call must satisfy a number of static
      checks.  
    
    \begin{enumerate}
    \item Constructors can only invoke \Xcd{private} or \Xcd{final} methods, 
          or methods annotated \xcd`@NonEscaping`.  
    \item Methods invoked by constructors cannot read fields before they are
          written. 
    \item The compiler ensures this with a detailed protocol. 
    \end{enumerate}

\item It is still impossible for X10 constructors to leak references to
      \Xcd{this} or observe uninitialized fields of an object.  Now, however,
      the mechanisms enforcing this are less obtrusive than in 2.0.6; the
      burden is largely on the compiler, not the programmer.
\end{enumerate}




%REF> \subsection{Call by Reference}
%REF> 
%REF> A very limited form of call-by-reference is now available.
%REF> 
%REF> 
%REF> \begin{enumerate}
%REF> 
%REF> \item Formal parameters to functions and methods may be \Xcd{ref} rather than
%REF>       \Xcd{var} or \Xcd{val}.  
%REF> \item Assignment to a \Xcd{ref} parameter \Xcd{x} changes the original
%REF>       location that the \Xcd{ref} refers to.  \eg, 
%REF>       \xcd`def inc(ref x:Int) { x ++; }`
%REF>       allows a call \Xcd{inc(n)} to increment a local \Xcd{var} \Xcd{n}.
%REF> \item Only local variables or \Xcd{ref} parameters can be passed as actual
%REF>       \Xcd{ref} parameters.  Fields, array elements, and other variable-like
%REF>       items cannot be. 
%REF> \item External \Xcd{ref} variables cannot be captured in closures. However,
%REF>       closures may have \Xcd{ref} parameters.
%REF> \item \Xcd{ref}s are {\em not} first-class objects in X10. They cannot be
%REF>       returned from functions, stored in data structures, etc.
%REF> \item These restrictions limit the possibilities of aliasing and the need for
%REF>       boxing of \Xcd{ref} parameters.  \Xcd{ref}s to stack locations cannot,
%REF>       with these restrictions, live past the death of the location's
%REF>       containing stack frame.      
%REF> \item This allows the implementation of many core constructs as syntactic
%REF>       sugar on library calls.   Programmers may use it, but mutability should
%REF>       generally be encapsulated inside objects rather than \Xcd{ref}
%REF>       parameters. 
%REF> \end{enumerate}
%REF> 

%ACC> \subsection{Accumulator Variables}
%ACC> 
%ACC> Accumulator variables generalize and make explicit collecting \Xcd{finish} in
%ACC> X10 2.0.6.  An \Xcd{acc} variable is declared: 
%ACC> \begin{xten}
%ACC> acc(r) A;
%ACC> \end{xten}
%ACC> where \Xcd{r} is a {\em reducer} (much as in 2.0.6): 
%ACC> \begin{xten}
%ACC> struct Reducer[T](zero:T, apply:global (T,T)=>T){}
%ACC> \end{xten}
%ACC> 
%ACC> Usage of \Xcd{A} is restricted in ways that make it determinate in the
%ACC> intended case of a pure, associative, commutative \Xcd{apply} with unit
%ACC> element \Xcd{zero}.  
%ACC> 
%ACC> \begin{enumerate}
%ACC> \item \Xcd{A} is initialized to \Xcd{r.zero}.  
%ACC> \item Multiple activities can {\em write} into \Xcd{A}.  In particular, the
%ACC>       ``assignment'' \Xcd{A = v} is approximately interpreted as 
%ACC>       \xcd`atomic{A = r.apply(A, v)}` --- that is, it accumulates \Xcd{v} into
%ACC>       \Xcd{A} using \Xcd{r.apply.}
%ACC> \item {\em Reading} of \Xcd{A} is restricted to situations where it makes
%ACC>       sense.  Specifically, only the activity in which \Xcd{A} is declared can
%ACC>       read from it, and it can only do so when all asyncs which it has spawned
%ACC>       have terminated -- \eg, outside of the scope of all \Xcd{async}s and
%ACC>       \Xcd{finish}es.  
%ACC> \item Formal parameters of functions may be marked \Xcd{acc x:T}.  The reducer
%ACC>       \Xcd{r} must not be specified; it is passed as an implicit parameter
%ACC>       going with the actual \Xcd{acc} variable.  
%ACC> \item X10 provides protocols for indexed collections of \Xcd{acc} variables,
%ACC>       presented as objects.
%ACC> \end{enumerate}
%ACC> 


\subsection{Implicit clocks for each finish}


Most clock operations can be accomplished using the new implicit clocks.

\begin{enumerate}
\item A \Xcd{finish} may be qualified with \Xcd{clocked}, which gives it a
      clock.
\item An \Xcd{async} in a \Xcd{clocked finish} may be marked \Xcd{clocked}.
      This registers it on the same clock as the enclosing \Xcd{finish}.  
\item \xcd`clocked async S` and \xcd`clocked finish S` may use \xcd`next` in
      the body of \Xcd{S} to advance the clock.
\item When the body of a \Xcd{clocked finish} completes, the \Xcd{clocked
      finish} is dropped form the clock.  It will still wait for spawned
      asyncs to terminate, but such asyncs need to wait for it.
\end{enumerate}


%CLOCAL>\subsection{Clocked local variables}
%CLOCAL>
%CLOCAL>Local \Xcd{val} and \Xcd{acc} variables may be \Xcd{clocked}.  They are
%CLOCAL>associated with the clock of the surrounding \Xcd{clocked finish}.  
%CLOCAL>Clocked variables have a {\em current} value and an {\em upcoming} value.  The
%CLOCAL>current value may be read at suitable times; the upcoming value may be
%CLOCAL>updated.  The \Xcd{next} phase makes the upcoming value current.

\subsection{Asynchronous initialization of val}

\Xcd{val}s can be initialized asynchronously.   As always with \Xcd{val}s,
they can only be read after it is guaranteed that they have been initialized.
For example, both of the \Xcd{print}s below are good.  However, the
commented-out \Xcd{print} in the \Xcd{async} is bad, since it is possible that
it will be executed before the initialization of \Xcd{a}. 
\begin{xten}
val a:Int;
finish {
  async {
     a = 1; 
     print("a=" + a);
  }
  // WRONG: print("a=" + a);
}
print("a=" + a);
\end{xten}



\subsection{Main Method}

The signature for the \Xcd{main} method is now: 
\begin{xten}
           def main(Array[String]) {..}
\end{xten}
or, if the arguments are actually used, 
\begin{xten}
           def main(argv: Array[String](1)) {..}
\end{xten}

\subsection{Assorted Changes}


\begin{enumerate}
\item The syntax for destructuring a point now uses brackets rather than
      braces: \Xcd{for( [i] in 1..10 )}, rather than the prior \Xcd{(i)}.  
\end{enumerate}

\subsection{Safety of atomic and when blocks}


\begin{enumerate}
\item Static effect annotations (\Xcd{safe}, \Xcd{sequential},
      \Xcd{nonblocking}, \Xcd{pinned}) are no longer used. They have been
      replaced by dynamic checks.
\item Using an inappropriate operation in the scope of an \Xcd{atomic} or
      \Xcd{when} construct will throw \Xcd{IllegalOperationException}.  
      The following are inappropriate:      
      \begin{itemize}
      \item \Xcd{when}
      \item \Xcd{resume()} or \Xcd{next} on clocks
      \item async
      \item \Xcd{Future.make()}, or \Xcd{Future.force()}.
      \item \Xcd{at}
      \end{itemize}

\end{enumerate}


\subsection{Removed Topics}

The following are gone: 

\begin{enumerate}
\item \Xcd{foreach} is gone.
\item All \Xcd{var}s are effectively \Xcd{shared}, so \Xcd{shared} is gone.
\item The place clause on \Xcd{async} is gone.  \Xcd{async (P) S} should be
      written \Xcd{at(P) async S}.
\item Checked exceptions are gone.
\item \Xcd{future} is gone.
\item \Xcd{await ... or ... } is gone.
\item \Xcd{const} is gone.

\end{enumerate}

\subsection{Deprecated}

The following constructs are still available, but are likely to be replaced in
a future version: 


\begin{enumerate}
\item \Xcd{ValRail}.
\item \Xcd{Rail}.
\item \xcd`ateach`
\item \xcd`offers`.  \index{offers}  The \xcd`offers` concept was experimental
      in 2.1, but was determined inadequate.  It has not been removed from the
      compiler yet, but it will be soon.  In the meantime, traces of it are
      still visible in the grammar.  They should not be used and can safely be ignored.
\end{enumerate}

\section{Changes from \Xten{} v2.0}
Some of these changes have been made obsolete in X10 2.2.

\begin{itemize}
\item \Xcd{Any} is now the top of the type hierarchy (every object,
  struct and function has a type that is a subtype of
  \Xcd{Any}). \Xcd{Any} defines \Xcd{home}, \Xcd{at}, \Xcd{toString},
  \Xcd{typeName}, \Xcd{equals} and \Xcd{hashCode}. \Xcd{Any} also defines the methods
  of \Xcd{Equals}, so \Xcd{Equals} is not needed any more.
\item Revised discussion of incomplete types.
\item The manual has been revised and brought into line with the current implementation. 
\end{itemize}
\section{Changes from \Xten{} v1.7}

The language has changed in the following ways.  
Some of these changes have been made obsolete in X10 2.2.

\begin{itemize}

\item {\bf Type system changes}: There are now three kinds of entities
  in an \Xten{} computation: objects, structs and functions. Their
  associated types are class types, struct types and function
  types. 

  Class and struct types are called {\em container types} in that they
  specify a collection of fields and methods. Container types have a
  name and a signature (the collection of members accessible on that
  type). Collection types support primitive equality \Xcd{==} and may
  support user-defined equality if they implement the {\tt
    x10.lang.Equals} interface. 

  Container types (and interface types) may be further qualified with
  constraints.

  A function type specifies a set of arguments and their type, the
  result type, and (optionally) a guard. A function application
  type-checks if the arguments are of the given type and the guard is
  satisfied, and the return value is of the given type.  A function
  type does not permit \Xcd{==} checks. Closure literals create
  instances of the corresponding function type.

  Container types may implement interfaces and zero or more function
  types.

  All types support a basic set of operations that return a string
  representation, a type name, and specify the home place of the entity.

  The type system is not unitary. However, any type may be used to
  instantiate a generic type. 

  There is no longer any notion of \Xcd{value} classes. \Xcd{value}
  classes must be re-written into structs or (reference) classes. 

\item {\bf Global object model}: Objects are instances of
  classes. Each object is associated with a globally unique
  identifier. Two objects are considered identical \Xcd{==} if their
  ids are identical. Classes may specify \Xcd{global} fields and
  methods. These can be accessed at any place. (\Xcd{global} fields
  must be immutable.)

\item{\bf Proto types.} For the decidability of dependent type
  checking it is necessary that the property graph is acyclic. This is
  ensured by enforcing rules on the leakage of \Xcd{this} in
  constructors. The rules are flexible enough to permit cycles to be
  created with normal fields, but not with properties.

\item{Place types.} Place types are now implemented. This means that
  non-global methods can be invoked on a variable, only if the
  variable's type is either a struct type or a function type, or a
  class type whose constraint specifies that the object is located in
  the current place.

  There is still no support for statically checking array access
  bounds, or performing place checks on array accesses.

\end{itemize}

\chapter{Options}

\subsection{Compiler Options}

The X10 compilers have many useful options. 

% -CHECK_INVARIANTS seems to check some internal compiler invariants.

\subsection{Optimization: {\tt -O} or {\tt -optimize}}

This flag causes the compiler to generate optimized code.


\subsection{Debugging: {\tt -DEBUG=boolean}}

This flag, if true, causes the compiler to generate debugging information.  It
is false by default.

\subsection{Call Style: {\tt -STATIC\_CHECKS, -VERBOSE\_CHECKS}}
\label{sect:Callstyle}
\index{STATIC\_CHECKS}
\index{VERBOSE\_CHECKS}
\index{dynamic checks}

By default, if a method call {\em could} be correct but is not {\em
necessarily} correct, the X10 compiler generates a dynamic check to ensure
that it is correct before it is performed.  For example, the following code: 
\begin{xten}
def use(n:Int{self == 0}) {}
def test(x:Int) { 
   use(x); // creates a dynamic cast
}
\end{xten}
compiles with \xcd`-STATIC_CHECKS`, even though it is possible that 
\xcd`x!=0` when \xcd`use(x)` is called.  In this case, the compiler inserts a
cast, which has the effect of checking that the call is correct before it
happens: 
\begin{xten}
def use(n:Int{self == 0}) {}
def test(x:Int) { 
   use(x as Int{self == 0}); 
}
\end{xten}
The compiler produces a warning that it inserted some dynamic casts.
If you then want to see what it did, use \xcd`-VERBOSE_CHECKS`.

You may also turn on static checking, with the \xcd`-STATIC_CHECKS` flag.  With
static checking, calls that cannot be proved correct statically will be
marked as errors.  The program without the dynamic cast fails to compile with
\xcd`-STATIC_CHECKS`.  





\subsection{Help: {\tt -help} and {\tt -- -help}}

These options cause the compiler to print a list of all command-line options.


\subsection{Source Path: {\tt -sourcepath {\em path}}}

This option tells the compiler where to look for X10 source code.  


\subsection{Class Path: {\tt -classpath {\em path}}}

This option tells the compiler where to look for compiled code in {\tt class}
files.

\subsection{Output Directory: {\tt -d {\em directory}}}

This option tells the compiler to produce its output files in the specified directory.

\subsection{Runtime {\tt -x10rt {\em impl}}}

This option tells which runtime implementation to use.  The choices are
\xcd`lapi`, \xcd`pgp`, \xcd`sockets`, \xcd`mpi`, and \xcd`standalone`.



\section{Execution Options: Java}

The Java execution command \xcd`x10` has a number of options as well. 

\subsection{{\tt -NUMBER\_OF\_LOCAL\_PLACES={\em number}}}

This option controls how many \xcd`Place`s the system will run on.  The
default is four, but you may control it as you wish.

\subsection{Heap Size: {\tt -mx {\em size}}}

Sets the maximum size of the heap. 

\subsection{Help: {\tt -h}}

Prints a listing of all execution options.



%\subsection{{\tt }}




{\em The \Xten{} language has been developed as part of the IBM PERCS
Project, which is supported in part by the Defense Advanced Research
Projects Agency (DARPA) under contract No. NBCH30390004.}

{\em Java and all Java-based trademarks are trademarks of Sun Microsystems,
Inc. in the United States, other countries, or both.}
\end{document}


\appendix

\chapter{Change Log}

\section{Changes from \Xten{} v2.2}

\begin{enumerate}

\item In previous versions of \Xten{} static fields were
  eagerly initialized in \xcd{Place 0} and the resulting values were
  serialized to all other places before execution of the user main
  function was started. Starting with \Xten{} v2.3, static fields are
  lazily initialized on a per-Place basis when the field is first read
  by an activity executing in a given Place.

\end{enumerate}

\section{Changes from \Xten{} v2.1}

\begin{enumerate}

\item Covariance and contravariance are gone.

\item Operator definitions are regularized.  A number of new operator symbols
      are available.

\item The operator \xcd`in` is gone.  \xcd`in` is now only a keyword.

\item Method functions and operator functions are gone.

\item \xcd`m..n` is now a type of struct called \xcd`IntRange`.  

\item \xcd`for(i in m..n)` now works.  The old forms, \xcd`for((i) in m..n)`
      and \xcd`for([i] in m..n)`, are no longer needed.

\item \xcd`(e as T)` now has type \xcd`T`.  (It used to have an identity
      constraint conjoined in.)

\item \xcd`var`s can no longer be assigned in their place of origin.  Use a
      \xcd`GlobalRef[Cell[T]]` instead.  We'll have a new idiom for this in 2.3.

\item The \xcd`-STATIC_CALLS` command-line flag is now \xcd`-STATIC_CHECKS`.

\item Any string may be written in backquotes to make an identifier: {\tt
      `while`}.

\item The \xcd`next` and \xcd`resume` keywords are gone; they have been
      replaced by static methods on \xcd`Clock`.

\item The typed array construction syntax \xcd`new Array[T][t1,t2]` is gone.
      Use \xcd`[t1 as T, t2]` (if just plain \xcd`[t1,t2]` doesn't work).

\end{enumerate}


\section{Changes from \Xten{} v2.0.6}

This document summarizes the main changes between X10 2.0.6 and X10 2.1.  The
descriptions are intended to be suggestive rather than definitive; see the
language specification for full details.

\subsection{Object Model}

\begin{enumerate}
\item Objects are now local rather than global.
   
    \begin{enumerate}
    \item The \Xcd{home} property is gone.
    \item \Xcd{at(P)S} produces deep copies of all objects reachable from
          lexically exposed variables in \xcd`S` 
          when it executes \Xcd{S}.  ({\bf Warning:} They are copied even in  
          \Xcd{at(here)S}.)
    \end{enumerate}

\item The \Xcd{GlobalRef[T]} struct is the only way to produce or manipulate
      cross-place references.
    \begin{enumerate}
    \item \Xcd{GlobalRef}'s have a \Xcd{home} property.
    \item Use \Xcd{GlobalRef[Foo](foo)} to make a new global reference.
    \item Use \Xcd{myGlobalRef()} to access the object referenced; this
          requires \Xcd{here == myGlobalRef.home}. 
    \end{enumerate}


\item  The \xcd`!` type modifier is no longer needed or present.

\item \Xcd{global} modifiers are now gone:
    
    \begin{enumerate}
    \item \Xcd{global} methods in {\em interfaces} are now the default. 
    \item \Xcd{global} {\em fields} are gone.  In some cases object copying
          will produce the same effect as global fields.  In other cases code
          must be rewritten.  It may be desirable to mark nonglobal fields
          \Xcd{transient} in many cases.
    \item \Xcd{global} {\em methods} are now marked \Xcd{@Global} instead.  
          Methods intended to be non-global may be marked \Xcd{@Pinned}.
    \end{enumerate}


\end{enumerate}

\subsection{Constructors}


\begin{enumerate}
\item \Xcd{proto} types are gone.
\item Constructors and the methods they call must satisfy a number of static
      checks.  
    
    \begin{enumerate}
    \item Constructors can only invoke \Xcd{private} or \Xcd{final} methods, 
          or methods annotated \xcd`@NonEscaping`.  
    \item Methods invoked by constructors cannot read fields before they are
          written. 
    \item The compiler ensures this with a detailed protocol. 
    \end{enumerate}

\item It is still impossible for X10 constructors to leak references to
      \Xcd{this} or observe uninitialized fields of an object.  Now, however,
      the mechanisms enforcing this are less obtrusive than in 2.0.6; the
      burden is largely on the compiler, not the programmer.
\end{enumerate}




%REF> \subsection{Call by Reference}
%REF> 
%REF> A very limited form of call-by-reference is now available.
%REF> 
%REF> 
%REF> \begin{enumerate}
%REF> 
%REF> \item Formal parameters to functions and methods may be \Xcd{ref} rather than
%REF>       \Xcd{var} or \Xcd{val}.  
%REF> \item Assignment to a \Xcd{ref} parameter \Xcd{x} changes the original
%REF>       location that the \Xcd{ref} refers to.  \eg, 
%REF>       \xcd`def inc(ref x:Int) { x ++; }`
%REF>       allows a call \Xcd{inc(n)} to increment a local \Xcd{var} \Xcd{n}.
%REF> \item Only local variables or \Xcd{ref} parameters can be passed as actual
%REF>       \Xcd{ref} parameters.  Fields, array elements, and other variable-like
%REF>       items cannot be. 
%REF> \item External \Xcd{ref} variables cannot be captured in closures. However,
%REF>       closures may have \Xcd{ref} parameters.
%REF> \item \Xcd{ref}s are {\em not} first-class objects in X10. They cannot be
%REF>       returned from functions, stored in data structures, etc.
%REF> \item These restrictions limit the possibilities of aliasing and the need for
%REF>       boxing of \Xcd{ref} parameters.  \Xcd{ref}s to stack locations cannot,
%REF>       with these restrictions, live past the death of the location's
%REF>       containing stack frame.      
%REF> \item This allows the implementation of many core constructs as syntactic
%REF>       sugar on library calls.   Programmers may use it, but mutability should
%REF>       generally be encapsulated inside objects rather than \Xcd{ref}
%REF>       parameters. 
%REF> \end{enumerate}
%REF> 

%ACC> \subsection{Accumulator Variables}
%ACC> 
%ACC> Accumulator variables generalize and make explicit collecting \Xcd{finish} in
%ACC> X10 2.0.6.  An \Xcd{acc} variable is declared: 
%ACC> \begin{xten}
%ACC> acc(r) A;
%ACC> \end{xten}
%ACC> where \Xcd{r} is a {\em reducer} (much as in 2.0.6): 
%ACC> \begin{xten}
%ACC> struct Reducer[T](zero:T, apply:global (T,T)=>T){}
%ACC> \end{xten}
%ACC> 
%ACC> Usage of \Xcd{A} is restricted in ways that make it determinate in the
%ACC> intended case of a pure, associative, commutative \Xcd{apply} with unit
%ACC> element \Xcd{zero}.  
%ACC> 
%ACC> \begin{enumerate}
%ACC> \item \Xcd{A} is initialized to \Xcd{r.zero}.  
%ACC> \item Multiple activities can {\em write} into \Xcd{A}.  In particular, the
%ACC>       ``assignment'' \Xcd{A = v} is approximately interpreted as 
%ACC>       \xcd`atomic{A = r.apply(A, v)}` --- that is, it accumulates \Xcd{v} into
%ACC>       \Xcd{A} using \Xcd{r.apply.}
%ACC> \item {\em Reading} of \Xcd{A} is restricted to situations where it makes
%ACC>       sense.  Specifically, only the activity in which \Xcd{A} is declared can
%ACC>       read from it, and it can only do so when all asyncs which it has spawned
%ACC>       have terminated -- \eg, outside of the scope of all \Xcd{async}s and
%ACC>       \Xcd{finish}es.  
%ACC> \item Formal parameters of functions may be marked \Xcd{acc x:T}.  The reducer
%ACC>       \Xcd{r} must not be specified; it is passed as an implicit parameter
%ACC>       going with the actual \Xcd{acc} variable.  
%ACC> \item X10 provides protocols for indexed collections of \Xcd{acc} variables,
%ACC>       presented as objects.
%ACC> \end{enumerate}
%ACC> 


\subsection{Implicit clocks for each finish}


Most clock operations can be accomplished using the new implicit clocks.

\begin{enumerate}
\item A \Xcd{finish} may be qualified with \Xcd{clocked}, which gives it a
      clock.
\item An \Xcd{async} in a \Xcd{clocked finish} may be marked \Xcd{clocked}.
      This registers it on the same clock as the enclosing \Xcd{finish}.  
\item \xcd`clocked async S` and \xcd`clocked finish S` may use \xcd`next` in
      the body of \Xcd{S} to advance the clock.
\item When the body of a \Xcd{clocked finish} completes, the \Xcd{clocked
      finish} is dropped form the clock.  It will still wait for spawned
      asyncs to terminate, but such asyncs need to wait for it.
\end{enumerate}


%CLOCAL>\subsection{Clocked local variables}
%CLOCAL>
%CLOCAL>Local \Xcd{val} and \Xcd{acc} variables may be \Xcd{clocked}.  They are
%CLOCAL>associated with the clock of the surrounding \Xcd{clocked finish}.  
%CLOCAL>Clocked variables have a {\em current} value and an {\em upcoming} value.  The
%CLOCAL>current value may be read at suitable times; the upcoming value may be
%CLOCAL>updated.  The \Xcd{next} phase makes the upcoming value current.

\subsection{Asynchronous initialization of val}

\Xcd{val}s can be initialized asynchronously.   As always with \Xcd{val}s,
they can only be read after it is guaranteed that they have been initialized.
For example, both of the \Xcd{print}s below are good.  However, the
commented-out \Xcd{print} in the \Xcd{async} is bad, since it is possible that
it will be executed before the initialization of \Xcd{a}. 
\begin{xten}
val a:Int;
finish {
  async {
     a = 1; 
     print("a=" + a);
  }
  // WRONG: print("a=" + a);
}
print("a=" + a);
\end{xten}



\subsection{Main Method}

The signature for the \Xcd{main} method is now: 
\begin{xten}
           def main(Array[String]) {..}
\end{xten}
or, if the arguments are actually used, 
\begin{xten}
           def main(argv: Array[String](1)) {..}
\end{xten}

\subsection{Assorted Changes}


\begin{enumerate}
\item The syntax for destructuring a point now uses brackets rather than
      braces: \Xcd{for( [i] in 1..10 )}, rather than the prior \Xcd{(i)}.  
\end{enumerate}

\subsection{Safety of atomic and when blocks}


\begin{enumerate}
\item Static effect annotations (\Xcd{safe}, \Xcd{sequential},
      \Xcd{nonblocking}, \Xcd{pinned}) are no longer used. They have been
      replaced by dynamic checks.
\item Using an inappropriate operation in the scope of an \Xcd{atomic} or
      \Xcd{when} construct will throw \Xcd{IllegalOperationException}.  
      The following are inappropriate:      
      \begin{itemize}
      \item \Xcd{when}
      \item \Xcd{resume()} or \Xcd{next} on clocks
      \item async
      \item \Xcd{Future.make()}, or \Xcd{Future.force()}.
      \item \Xcd{at}
      \end{itemize}

\end{enumerate}


\subsection{Removed Topics}

The following are gone: 

\begin{enumerate}
\item \Xcd{foreach} is gone.
\item All \Xcd{var}s are effectively \Xcd{shared}, so \Xcd{shared} is gone.
\item The place clause on \Xcd{async} is gone.  \Xcd{async (P) S} should be
      written \Xcd{at(P) async S}.
\item Checked exceptions are gone.
\item \Xcd{future} is gone.
\item \Xcd{await ... or ... } is gone.
\item \Xcd{const} is gone.

\end{enumerate}

\subsection{Deprecated}

The following constructs are still available, but are likely to be replaced in
a future version: 


\begin{enumerate}
\item \Xcd{ValRail}.
\item \Xcd{Rail}.
\item \xcd`ateach`
\item \xcd`offers`.  \index{offers}  The \xcd`offers` concept was experimental
      in 2.1, but was determined inadequate.  It has not been removed from the
      compiler yet, but it will be soon.  In the meantime, traces of it are
      still visible in the grammar.  They should not be used and can safely be ignored.
\end{enumerate}

\section{Changes from \Xten{} v2.0}
Some of these changes have been made obsolete in X10 2.2.

\begin{itemize}
\item \Xcd{Any} is now the top of the type hierarchy (every object,
  struct and function has a type that is a subtype of
  \Xcd{Any}). \Xcd{Any} defines \Xcd{home}, \Xcd{at}, \Xcd{toString},
  \Xcd{typeName}, \Xcd{equals} and \Xcd{hashCode}. \Xcd{Any} also defines the methods
  of \Xcd{Equals}, so \Xcd{Equals} is not needed any more.
\item Revised discussion of incomplete types.
\item The manual has been revised and brought into line with the current implementation. 
\end{itemize}
\section{Changes from \Xten{} v1.7}

The language has changed in the following ways.  
Some of these changes have been made obsolete in X10 2.2.

\begin{itemize}

\item {\bf Type system changes}: There are now three kinds of entities
  in an \Xten{} computation: objects, structs and functions. Their
  associated types are class types, struct types and function
  types. 

  Class and struct types are called {\em container types} in that they
  specify a collection of fields and methods. Container types have a
  name and a signature (the collection of members accessible on that
  type). Collection types support primitive equality \Xcd{==} and may
  support user-defined equality if they implement the {\tt
    x10.lang.Equals} interface. 

  Container types (and interface types) may be further qualified with
  constraints.

  A function type specifies a set of arguments and their type, the
  result type, and (optionally) a guard. A function application
  type-checks if the arguments are of the given type and the guard is
  satisfied, and the return value is of the given type.  A function
  type does not permit \Xcd{==} checks. Closure literals create
  instances of the corresponding function type.

  Container types may implement interfaces and zero or more function
  types.

  All types support a basic set of operations that return a string
  representation, a type name, and specify the home place of the entity.

  The type system is not unitary. However, any type may be used to
  instantiate a generic type. 

  There is no longer any notion of \Xcd{value} classes. \Xcd{value}
  classes must be re-written into structs or (reference) classes. 

\item {\bf Global object model}: Objects are instances of
  classes. Each object is associated with a globally unique
  identifier. Two objects are considered identical \Xcd{==} if their
  ids are identical. Classes may specify \Xcd{global} fields and
  methods. These can be accessed at any place. (\Xcd{global} fields
  must be immutable.)

\item{\bf Proto types.} For the decidability of dependent type
  checking it is necessary that the property graph is acyclic. This is
  ensured by enforcing rules on the leakage of \Xcd{this} in
  constructors. The rules are flexible enough to permit cycles to be
  created with normal fields, but not with properties.

\item{Place types.} Place types are now implemented. This means that
  non-global methods can be invoked on a variable, only if the
  variable's type is either a struct type or a function type, or a
  class type whose constraint specifies that the object is located in
  the current place.

  There is still no support for statically checking array access
  bounds, or performing place checks on array accesses.

\end{itemize}

\chapter{Options}

\subsection{Compiler Options}

The X10 compilers have many useful options. 

% -CHECK_INVARIANTS seems to check some internal compiler invariants.

\subsection{Optimization: {\tt -O} or {\tt -optimize}}

This flag causes the compiler to generate optimized code.


\subsection{Debugging: {\tt -DEBUG=boolean}}

This flag, if true, causes the compiler to generate debugging information.  It
is false by default.

\subsection{Call Style: {\tt -STATIC\_CHECKS, -VERBOSE\_CHECKS}}
\label{sect:Callstyle}
\index{STATIC\_CHECKS}
\index{VERBOSE\_CHECKS}
\index{dynamic checks}

By default, if a method call {\em could} be correct but is not {\em
necessarily} correct, the X10 compiler generates a dynamic check to ensure
that it is correct before it is performed.  For example, the following code: 
\begin{xten}
def use(n:Int{self == 0}) {}
def test(x:Int) { 
   use(x); // creates a dynamic cast
}
\end{xten}
compiles with \xcd`-STATIC_CHECKS`, even though it is possible that 
\xcd`x!=0` when \xcd`use(x)` is called.  In this case, the compiler inserts a
cast, which has the effect of checking that the call is correct before it
happens: 
\begin{xten}
def use(n:Int{self == 0}) {}
def test(x:Int) { 
   use(x as Int{self == 0}); 
}
\end{xten}
The compiler produces a warning that it inserted some dynamic casts.
If you then want to see what it did, use \xcd`-VERBOSE_CHECKS`.

You may also turn on static checking, with the \xcd`-STATIC_CHECKS` flag.  With
static checking, calls that cannot be proved correct statically will be
marked as errors.  The program without the dynamic cast fails to compile with
\xcd`-STATIC_CHECKS`.  





\subsection{Help: {\tt -help} and {\tt -- -help}}

These options cause the compiler to print a list of all command-line options.


\subsection{Source Path: {\tt -sourcepath {\em path}}}

This option tells the compiler where to look for X10 source code.  


\subsection{Class Path: {\tt -classpath {\em path}}}

This option tells the compiler where to look for compiled code in {\tt class}
files.

\subsection{Output Directory: {\tt -d {\em directory}}}

This option tells the compiler to produce its output files in the specified directory.

\subsection{Runtime {\tt -x10rt {\em impl}}}

This option tells which runtime implementation to use.  The choices are
\xcd`lapi`, \xcd`pgp`, \xcd`sockets`, \xcd`mpi`, and \xcd`standalone`.



\section{Execution Options: Java}

The Java execution command \xcd`x10` has a number of options as well. 

\subsection{{\tt -NUMBER\_OF\_LOCAL\_PLACES={\em number}}}

This option controls how many \xcd`Place`s the system will run on.  The
default is four, but you may control it as you wish.

\subsection{Heap Size: {\tt -mx {\em size}}}

Sets the maximum size of the heap. 

\subsection{Help: {\tt -h}}

Prints a listing of all execution options.



%\subsection{{\tt }}




{\em The \Xten{} language has been developed as part of the IBM PERCS
Project, which is supported in part by the Defense Advanced Research
Projects Agency (DARPA) under contract No. NBCH30390004.}

{\em Java and all Java-based trademarks are trademarks of Sun Microsystems,
Inc. in the United States, other countries, or both.}
\end{document}


\appendix

\chapter{Deprecations}

X10 version 2.2 has a few relics of previous versions, code that is being used
by libraries but is not intended for general programming.    They should be
ignored.

These are: 

\begin{itemize}

\item \xcd`acc` variables. \index{acc}

\item The \xcd`offers` clause, as seen in the {\it Offers} nonterminal in the
      grammar (\ref{prod:Offers}).\index{offers}\index{Offers}

\end{itemize}


\chapter{Change Log}

\section{Changes from \Xten{} v2.2}

\begin{enumerate}

\item In previous versions of \Xten{} static fields were
  eagerly initialized in \xcd{Place 0} and the resulting values were
  serialized to all other places before execution of the user main
  function was started. Starting with \Xten{} v2.3, static fields are
  lazily initialized on a per-Place basis when the field is first read
  by an activity executing in a given Place.

\end{enumerate}

\section{Changes from \Xten{} v2.1}

\begin{enumerate}

\item Covariance and contravariance are gone.

\item Operator definitions are regularized.  A number of new operator symbols
      are available.

\item The operator \xcd`in` is gone.  \xcd`in` is now only a keyword.

\item Method functions and operator functions are gone.

\item \xcd`m..n` is now a type of struct called \xcd`IntRange`.  

\item \xcd`for(i in m..n)` now works.  The old forms, \xcd`for((i) in m..n)`
      and \xcd`for([i] in m..n)`, are no longer needed.

\item \xcd`(e as T)` now has type \xcd`T`.  (It used to have an identity
      constraint conjoined in.)

\item \xcd`var`s can no longer be assigned in their place of origin.  Use a
      \xcd`GlobalRef[Cell[T]]` instead.  We'll have a new idiom for this in 2.3.

\item The \xcd`-STATIC_CALLS` command-line flag is now \xcd`-STATIC_CHECKS`.

\item Any string may be written in backquotes to make an identifier: {\tt
      `while`}.

\item The \xcd`next` and \xcd`resume` keywords are gone; they have been
      replaced by static methods on \xcd`Clock`.

\item The typed array construction syntax \xcd`new Array[T][t1,t2]` is gone.
      Use \xcd`[t1 as T, t2]` (if just plain \xcd`[t1,t2]` doesn't work).

\end{enumerate}


\section{Changes from \Xten{} v2.0.6}

This document summarizes the main changes between X10 2.0.6 and X10 2.1.  The
descriptions are intended to be suggestive rather than definitive; see the
language specification for full details.

\subsection{Object Model}

\begin{enumerate}
\item Objects are now local rather than global.
   
    \begin{enumerate}
    \item The \Xcd{home} property is gone.
    \item \Xcd{at(P)S} produces deep copies of all objects reachable from
          lexically exposed variables in \xcd`S` 
          when it executes \Xcd{S}.  ({\bf Warning:} They are copied even in  
          \Xcd{at(here)S}.)
    \end{enumerate}

\item The \Xcd{GlobalRef[T]} struct is the only way to produce or manipulate
      cross-place references.
    \begin{enumerate}
    \item \Xcd{GlobalRef}'s have a \Xcd{home} property.
    \item Use \Xcd{GlobalRef[Foo](foo)} to make a new global reference.
    \item Use \Xcd{myGlobalRef()} to access the object referenced; this
          requires \Xcd{here == myGlobalRef.home}. 
    \end{enumerate}


\item  The \xcd`!` type modifier is no longer needed or present.

\item \Xcd{global} modifiers are now gone:
    
    \begin{enumerate}
    \item \Xcd{global} methods in {\em interfaces} are now the default. 
    \item \Xcd{global} {\em fields} are gone.  In some cases object copying
          will produce the same effect as global fields.  In other cases code
          must be rewritten.  It may be desirable to mark nonglobal fields
          \Xcd{transient} in many cases.
    \item \Xcd{global} {\em methods} are now marked \Xcd{@Global} instead.  
          Methods intended to be non-global may be marked \Xcd{@Pinned}.
    \end{enumerate}


\end{enumerate}

\subsection{Constructors}


\begin{enumerate}
\item \Xcd{proto} types are gone.
\item Constructors and the methods they call must satisfy a number of static
      checks.  
    
    \begin{enumerate}
    \item Constructors can only invoke \Xcd{private} or \Xcd{final} methods, 
          or methods annotated \xcd`@NonEscaping`.  
    \item Methods invoked by constructors cannot read fields before they are
          written. 
    \item The compiler ensures this with a detailed protocol. 
    \end{enumerate}

\item It is still impossible for X10 constructors to leak references to
      \Xcd{this} or observe uninitialized fields of an object.  Now, however,
      the mechanisms enforcing this are less obtrusive than in 2.0.6; the
      burden is largely on the compiler, not the programmer.
\end{enumerate}




%REF> \subsection{Call by Reference}
%REF> 
%REF> A very limited form of call-by-reference is now available.
%REF> 
%REF> 
%REF> \begin{enumerate}
%REF> 
%REF> \item Formal parameters to functions and methods may be \Xcd{ref} rather than
%REF>       \Xcd{var} or \Xcd{val}.  
%REF> \item Assignment to a \Xcd{ref} parameter \Xcd{x} changes the original
%REF>       location that the \Xcd{ref} refers to.  \eg, 
%REF>       \xcd`def inc(ref x:Int) { x ++; }`
%REF>       allows a call \Xcd{inc(n)} to increment a local \Xcd{var} \Xcd{n}.
%REF> \item Only local variables or \Xcd{ref} parameters can be passed as actual
%REF>       \Xcd{ref} parameters.  Fields, array elements, and other variable-like
%REF>       items cannot be. 
%REF> \item External \Xcd{ref} variables cannot be captured in closures. However,
%REF>       closures may have \Xcd{ref} parameters.
%REF> \item \Xcd{ref}s are {\em not} first-class objects in X10. They cannot be
%REF>       returned from functions, stored in data structures, etc.
%REF> \item These restrictions limit the possibilities of aliasing and the need for
%REF>       boxing of \Xcd{ref} parameters.  \Xcd{ref}s to stack locations cannot,
%REF>       with these restrictions, live past the death of the location's
%REF>       containing stack frame.      
%REF> \item This allows the implementation of many core constructs as syntactic
%REF>       sugar on library calls.   Programmers may use it, but mutability should
%REF>       generally be encapsulated inside objects rather than \Xcd{ref}
%REF>       parameters. 
%REF> \end{enumerate}
%REF> 

%ACC> \subsection{Accumulator Variables}
%ACC> 
%ACC> Accumulator variables generalize and make explicit collecting \Xcd{finish} in
%ACC> X10 2.0.6.  An \Xcd{acc} variable is declared: 
%ACC> \begin{xten}
%ACC> acc(r) A;
%ACC> \end{xten}
%ACC> where \Xcd{r} is a {\em reducer} (much as in 2.0.6): 
%ACC> \begin{xten}
%ACC> struct Reducer[T](zero:T, apply:global (T,T)=>T){}
%ACC> \end{xten}
%ACC> 
%ACC> Usage of \Xcd{A} is restricted in ways that make it determinate in the
%ACC> intended case of a pure, associative, commutative \Xcd{apply} with unit
%ACC> element \Xcd{zero}.  
%ACC> 
%ACC> \begin{enumerate}
%ACC> \item \Xcd{A} is initialized to \Xcd{r.zero}.  
%ACC> \item Multiple activities can {\em write} into \Xcd{A}.  In particular, the
%ACC>       ``assignment'' \Xcd{A = v} is approximately interpreted as 
%ACC>       \xcd`atomic{A = r.apply(A, v)}` --- that is, it accumulates \Xcd{v} into
%ACC>       \Xcd{A} using \Xcd{r.apply.}
%ACC> \item {\em Reading} of \Xcd{A} is restricted to situations where it makes
%ACC>       sense.  Specifically, only the activity in which \Xcd{A} is declared can
%ACC>       read from it, and it can only do so when all asyncs which it has spawned
%ACC>       have terminated -- \eg, outside of the scope of all \Xcd{async}s and
%ACC>       \Xcd{finish}es.  
%ACC> \item Formal parameters of functions may be marked \Xcd{acc x:T}.  The reducer
%ACC>       \Xcd{r} must not be specified; it is passed as an implicit parameter
%ACC>       going with the actual \Xcd{acc} variable.  
%ACC> \item X10 provides protocols for indexed collections of \Xcd{acc} variables,
%ACC>       presented as objects.
%ACC> \end{enumerate}
%ACC> 


\subsection{Implicit clocks for each finish}


Most clock operations can be accomplished using the new implicit clocks.

\begin{enumerate}
\item A \Xcd{finish} may be qualified with \Xcd{clocked}, which gives it a
      clock.
\item An \Xcd{async} in a \Xcd{clocked finish} may be marked \Xcd{clocked}.
      This registers it on the same clock as the enclosing \Xcd{finish}.  
\item \xcd`clocked async S` and \xcd`clocked finish S` may use \xcd`next` in
      the body of \Xcd{S} to advance the clock.
\item When the body of a \Xcd{clocked finish} completes, the \Xcd{clocked
      finish} is dropped form the clock.  It will still wait for spawned
      asyncs to terminate, but such asyncs need to wait for it.
\end{enumerate}


%CLOCAL>\subsection{Clocked local variables}
%CLOCAL>
%CLOCAL>Local \Xcd{val} and \Xcd{acc} variables may be \Xcd{clocked}.  They are
%CLOCAL>associated with the clock of the surrounding \Xcd{clocked finish}.  
%CLOCAL>Clocked variables have a {\em current} value and an {\em upcoming} value.  The
%CLOCAL>current value may be read at suitable times; the upcoming value may be
%CLOCAL>updated.  The \Xcd{next} phase makes the upcoming value current.

\subsection{Asynchronous initialization of val}

\Xcd{val}s can be initialized asynchronously.   As always with \Xcd{val}s,
they can only be read after it is guaranteed that they have been initialized.
For example, both of the \Xcd{print}s below are good.  However, the
commented-out \Xcd{print} in the \Xcd{async} is bad, since it is possible that
it will be executed before the initialization of \Xcd{a}. 
\begin{xten}
val a:Int;
finish {
  async {
     a = 1; 
     print("a=" + a);
  }
  // WRONG: print("a=" + a);
}
print("a=" + a);
\end{xten}



\subsection{Main Method}

The signature for the \Xcd{main} method is now: 
\begin{xten}
           def main(Array[String]) {..}
\end{xten}
or, if the arguments are actually used, 
\begin{xten}
           def main(argv: Array[String](1)) {..}
\end{xten}

\subsection{Assorted Changes}


\begin{enumerate}
\item The syntax for destructuring a point now uses brackets rather than
      braces: \Xcd{for( [i] in 1..10 )}, rather than the prior \Xcd{(i)}.  
\end{enumerate}

\subsection{Safety of atomic and when blocks}


\begin{enumerate}
\item Static effect annotations (\Xcd{safe}, \Xcd{sequential},
      \Xcd{nonblocking}, \Xcd{pinned}) are no longer used. They have been
      replaced by dynamic checks.
\item Using an inappropriate operation in the scope of an \Xcd{atomic} or
      \Xcd{when} construct will throw \Xcd{IllegalOperationException}.  
      The following are inappropriate:      
      \begin{itemize}
      \item \Xcd{when}
      \item \Xcd{resume()} or \Xcd{next} on clocks
      \item async
      \item \Xcd{Future.make()}, or \Xcd{Future.force()}.
      \item \Xcd{at}
      \end{itemize}

\end{enumerate}


\subsection{Removed Topics}

The following are gone: 

\begin{enumerate}
\item \Xcd{foreach} is gone.
\item All \Xcd{var}s are effectively \Xcd{shared}, so \Xcd{shared} is gone.
\item The place clause on \Xcd{async} is gone.  \Xcd{async (P) S} should be
      written \Xcd{at(P) async S}.
\item Checked exceptions are gone.
\item \Xcd{future} is gone.
\item \Xcd{await ... or ... } is gone.
\item \Xcd{const} is gone.

\end{enumerate}

\subsection{Deprecated}

The following constructs are still available, but are likely to be replaced in
a future version: 


\begin{enumerate}
\item \Xcd{ValRail}.
\item \Xcd{Rail}.
\item \xcd`ateach`
\item \xcd`offers`.  \index{offers}  The \xcd`offers` concept was experimental
      in 2.1, but was determined inadequate.  It has not been removed from the
      compiler yet, but it will be soon.  In the meantime, traces of it are
      still visible in the grammar.  They should not be used and can safely be ignored.
\end{enumerate}

\section{Changes from \Xten{} v2.0}
Some of these changes have been made obsolete in X10 2.2.

\begin{itemize}
\item \Xcd{Any} is now the top of the type hierarchy (every object,
  struct and function has a type that is a subtype of
  \Xcd{Any}). \Xcd{Any} defines \Xcd{home}, \Xcd{at}, \Xcd{toString},
  \Xcd{typeName}, \Xcd{equals} and \Xcd{hashCode}. \Xcd{Any} also defines the methods
  of \Xcd{Equals}, so \Xcd{Equals} is not needed any more.
\item Revised discussion of incomplete types.
\item The manual has been revised and brought into line with the current implementation. 
\end{itemize}
\section{Changes from \Xten{} v1.7}

The language has changed in the following ways.  
Some of these changes have been made obsolete in X10 2.2.

\begin{itemize}

\item {\bf Type system changes}: There are now three kinds of entities
  in an \Xten{} computation: objects, structs and functions. Their
  associated types are class types, struct types and function
  types. 

  Class and struct types are called {\em container types} in that they
  specify a collection of fields and methods. Container types have a
  name and a signature (the collection of members accessible on that
  type). Collection types support primitive equality \Xcd{==} and may
  support user-defined equality if they implement the {\tt
    x10.lang.Equals} interface. 

  Container types (and interface types) may be further qualified with
  constraints.

  A function type specifies a set of arguments and their type, the
  result type, and (optionally) a guard. A function application
  type-checks if the arguments are of the given type and the guard is
  satisfied, and the return value is of the given type.  A function
  type does not permit \Xcd{==} checks. Closure literals create
  instances of the corresponding function type.

  Container types may implement interfaces and zero or more function
  types.

  All types support a basic set of operations that return a string
  representation, a type name, and specify the home place of the entity.

  The type system is not unitary. However, any type may be used to
  instantiate a generic type. 

  There is no longer any notion of \Xcd{value} classes. \Xcd{value}
  classes must be re-written into structs or (reference) classes. 

\item {\bf Global object model}: Objects are instances of
  classes. Each object is associated with a globally unique
  identifier. Two objects are considered identical \Xcd{==} if their
  ids are identical. Classes may specify \Xcd{global} fields and
  methods. These can be accessed at any place. (\Xcd{global} fields
  must be immutable.)

\item{\bf Proto types.} For the decidability of dependent type
  checking it is necessary that the property graph is acyclic. This is
  ensured by enforcing rules on the leakage of \Xcd{this} in
  constructors. The rules are flexible enough to permit cycles to be
  created with normal fields, but not with properties.

\item{Place types.} Place types are now implemented. This means that
  non-global methods can be invoked on a variable, only if the
  variable's type is either a struct type or a function type, or a
  class type whose constraint specifies that the object is located in
  the current place.

  There is still no support for statically checking array access
  bounds, or performing place checks on array accesses.

\end{itemize}

\chapter{Options}

\subsection{Compiler Options}

The X10 compilers have many useful options. 

% -CHECK_INVARIANTS seems to check some internal compiler invariants.

\subsection{Optimization: {\tt -O} or {\tt -optimize}}

This flag causes the compiler to generate optimized code.


\subsection{Debugging: {\tt -DEBUG=boolean}}

This flag, if true, causes the compiler to generate debugging information.  It
is false by default.

\subsection{Call Style: {\tt -STATIC\_CHECKS, -VERBOSE\_CHECKS}}
\label{sect:Callstyle}
\index{STATIC\_CHECKS}
\index{VERBOSE\_CHECKS}
\index{dynamic checks}

By default, if a method call {\em could} be correct but is not {\em
necessarily} correct, the X10 compiler generates a dynamic check to ensure
that it is correct before it is performed.  For example, the following code: 
\begin{xten}
def use(n:Int{self == 0}) {}
def test(x:Int) { 
   use(x); // creates a dynamic cast
}
\end{xten}
compiles with \xcd`-STATIC_CHECKS`, even though it is possible that 
\xcd`x!=0` when \xcd`use(x)` is called.  In this case, the compiler inserts a
cast, which has the effect of checking that the call is correct before it
happens: 
\begin{xten}
def use(n:Int{self == 0}) {}
def test(x:Int) { 
   use(x as Int{self == 0}); 
}
\end{xten}
The compiler produces a warning that it inserted some dynamic casts.
If you then want to see what it did, use \xcd`-VERBOSE_CHECKS`.

You may also turn on static checking, with the \xcd`-STATIC_CHECKS` flag.  With
static checking, calls that cannot be proved correct statically will be
marked as errors.  The program without the dynamic cast fails to compile with
\xcd`-STATIC_CHECKS`.  





\subsection{Help: {\tt -help} and {\tt -- -help}}

These options cause the compiler to print a list of all command-line options.


\subsection{Source Path: {\tt -sourcepath {\em path}}}

This option tells the compiler where to look for X10 source code.  


\subsection{Class Path: {\tt -classpath {\em path}}}

This option tells the compiler where to look for compiled code in {\tt class}
files.

\subsection{Output Directory: {\tt -d {\em directory}}}

This option tells the compiler to produce its output files in the specified directory.

\subsection{Runtime {\tt -x10rt {\em impl}}}

This option tells which runtime implementation to use.  The choices are
\xcd`lapi`, \xcd`pgp`, \xcd`sockets`, \xcd`mpi`, and \xcd`standalone`.



\section{Execution Options: Java}

The Java execution command \xcd`x10` has a number of options as well. 

\subsection{{\tt -NUMBER\_OF\_LOCAL\_PLACES={\em number}}}

This option controls how many \xcd`Place`s the system will run on.  The
default is four, but you may control it as you wish.

\subsection{Heap Size: {\tt -mx {\em size}}}

Sets the maximum size of the heap. 

\subsection{Help: {\tt -h}}

Prints a listing of all execution options.



%\subsection{{\tt }}


\chapter{Acknowledgments and Trademarks}

{\em The \Xten{} language has been developed as part of the IBM PERCS
Project, which is supported in part by the Defense Advanced Research
Projects Agency (DARPA) under contract No. NBCH30390004.}

{\em Java and all Java-based trademarks are trademarks of Sun Microsystems,
Inc. in the United States, other countries, or both.}
\end{document}
