%%\subsection{Variance of Type Parameters}
%%\index{covariant}
%%\index{contravariant}
%%\index{invariant}
%%\index{type parameter!covariant}
%%\index{type parameter!contravariant}
%%\index{type parameter!invariant}

%TODO - examples courtesy of Nate
% 
% class OutputStream[-A] {
%    def write(a: A) = /* implementation left as an exercise for the reader */
% }
% 
% Also:
% 
% interface Comparator[-A] {
%    def compare(A): Int;
% }
% 
% and:
% 
% class HashMap[-K,+V] { ... }
% 
% 

Type parameters of classes (though not of methods) can be {\em variant}.

Consider classes \xcd`Person :> Child`.  Every child is a person, but there
are people who are not children.  What is the relationship between
\xcd`Cell[Person]` and \xcd`Cell[Child]`?  

\subsubsection{Why Variance Is Necessary}

In this case, \xcd`Cell[Person]` and \xcd`Cell[Child]` should be unrelated.  
If we had \xcd`Cell[Person] :> Cell[Child]`, the following code would let us
assign a \xcd`old` (a \xcd`Person` but not a \xcd`Child`) to a
variable \xcd`young` of type \xcd`Child`, thereby breaking the type system: 
\begin{xten}
// INCORRECTLY assuming Cell[Person] :> Cell[Child]
val cc : Cell[Child] = new Cell[Child]();
val cp : Cell[Person] = cc; // legal upcast
cp.set(old);       // legal since old : Person
val young : Child = cc.get(); 
\end{xten}

Similarly, if \xcd`Cell[Person] <: Cell[Child]`: 
\begin{xten}
// INCORRECTLY assuming Cell[Person] <: Cell[Child]
val cp : Cell[Person] = new Cell[Person];
val cc : Cell[Child] = cp; // legal upcast
val cp.set(old); 
val young : Child = cc.get();
\end{xten}

So, there cannot be a subtyping relationship in either direction between the
two. And indeed, neither of these programs passes the X10 typechecker.


\subsubsection{Legitimate Variance}

The \xcd`Cell[Person]`-vs-\xcd`Cell[Child]` problems occur because it is
possible to both store and retrieve values from the same object. However,
entities with only one of the two capabilities {\em can} sensibly have some
subtyping relations. Furthermore, both sorts of entity are useful. An entity
which can store values but not retrieve them can nonetheless summarize them.
An object which can retrieve values but not store values can be constructed
with an initial value, providing a read-only cell.

So, X10 provides {\em variance} to support these options.  Type parameters
may be defined in one of three forms.  
\begin{enumerate}
\item {\em invariant}: Given a definition \xcd`class C[T]{...}`, \xcd`C[Person]` and
      \xcd`C[Child]` are unrelated classes; neither is a subclass of the
      other.
\item {\em covariant}: Given a definition \xcd`class C[+T]{...}` (the \xcd`+` indicates
      covariance), \xcd`C[Person] :> C[Child]`.  This is appropriate when
      \xcd`C` allows retrieving values but not setting them.
\item {\em contravariant}: Given a definition \xcd`class C[-T]{...}` (the \xcd`-` indicates
      contravariance), \xcd`C[Person] <: C[Child]`.  This is appropriate when
      \xcd`C` allows storing values but not retrieving them.
\end{enumerate}


The \xcd"T" parameter of \xcd"Cell" above is
invariant.  

A typical example of covariance is \xcd`Get`.  As the \xcd`example()` method
shows, a \xcd`Get[T]` must be constructed with its value, and will return that
value whenever desired.  \xcd`Get[T]` is only moderately useful as a class; it
is more useful as an interface for providing a limited (read) access to a more
powerful data structure.
%~~gen ^^^ Variance10
% package Variance_gone;
%~~vis
\begin{xten}
class Get[+T] {
  val x: T;
  def this(x: T) { this.x = x; }
  def get(): T = x;
  static def example() {
     val g : Get[Int] = new Get[Int](31);
     val n : Int = g.get();
     x10.io.Console.OUT.print("It's " + n);
     x10.io.Console.OUT.print("It's still " + g.get());
  }
}
\end{xten}
%~~siv
%~~neg

There are few if any {\em classes} with contravariant type parameters.
(Covariant type parameters are only moderately more common.)  
However, it is frequently useful to have {\em interfaces} with contravariant
type parameters.  For example: 
%~~gen ^^^ Variance20
% package Types_contravariance_a;
%~~vis
\begin{xten}
interface OutputStream[-T] {
   def write(T) : void;
}
interface ComparableTo[-T] {
   def compare(T) : Int;
}
\end{xten}
%~~siv
%
%~~neg
Clearly, \xcd`Int <: Any`. 
An \xcd`OutputStream[Int]` is only capable of writing \xcd`Int`s.  
An \xcd`OutputStream[Any]` is capable of writing anything.  In particular, it
can write \xcd`Int`s. Thus, an \xcd`OutputStream[Any]` can be used in place of
an \xcd`OutputStream[Int]`, and hence \xcd`OutputStream[Any] <: OutputStream[Int]`.
Similarly, a \xcd`ComparableTo[Int]` can be compared to an integer. A
\xcd`ComparableTo[Any]` can be compared to anything, and, in particular, to an
integer.  Thus \xcd`ComparableTo[Any] <: ComparableTo[Int]`.
So, both of these interfaces are contravariant.


Given types \xcd"S" and \xcd"T": 
\begin{itemize}
\item
If the parameter of \xcd"Get" is covariant, then
\xcd"Get[S]" is a subtype of \xcd"Get[T]" if
\xcd"S" is a {\em subtype} of \xcd"T".

\item
If the parameter of \xcd"Set" is contravariant, then
\xcd"OutputStream[S]" is a subtype of \xcd"OutputStream[T]" if
\xcd"S" is a {\em supertype} of \xcd"T".

\item
If the parameter of \xcd"Cell" is invariant, then
\xcd"Cell[S]" is a subtype of \xcd"Cell[T]" if
\xcd"S" is a {\em equal} to \xcd"T".
\end{itemize}


In order to make types marked as covariant and contravariant semantically
sound, X10 performs extra checks.  
A covariant type parameter is permitted to appear only in covariant type positions,
and a contravariant type parameter in contravariant positions. 
\begin{itemize}
\item The return type of a method is a covariant position.
\item The argument types of a method are contravariant positions.
\item Whether a type argument position of a generic class, interface or struct type \Xcd{C}
is covariant or contravariant is determined by the \Xcd{+} or \Xcd{-} annotation
at that position in the declaration of \Xcd{C}.
\end{itemize}


There are similar restrictions on use of covariant and contravariant variables.

\limitationx{} Full checking of covariance and contravariance is not yet
implemented.  Covariant and contravariant classes and structs should be used
with great caution.

%TODO: Yoav says ``There are other rules, not implemented or specified,
%involving fields, inheritance, etc.  There are several JIRAs on it. No idea
%what is the work around -- maybe just say ``limitation''?'''
