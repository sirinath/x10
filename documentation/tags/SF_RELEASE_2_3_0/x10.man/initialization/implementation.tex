This section discusses our implementation inside the X10 compiler of the hardhat initialization rules.
Our X10 code-base of more than 200K lines of code (loc) uses only 104 annotations.
We give some measurements such as compilation time and annotation overhead,
	and conclude with two examples for \code{@NonEscaping} and \code{@NoThisAccess}.
	
The X10 compiler is based on the Polyglot extensible compiler framework,
	which includes a dataflow framework that has 1309 loc. % DataFlow
X10 initialization rules extend this dataflow framework
	using two classes:
	one for checking definite-initialization for \emph{local variables} (805 loc), %InitChecker
	and another for \emph{fields} (951 loc). % CheckEscapingThis
(The rules of local variables are simpler than those for fields
	because local variables do not span multiple methods 
	and they must be assigned before use.
	The focus point of this paper has been object initialization,
		therefore these rules 
		were not described in this paper.)
The dataflow algorithm tracks for each field (or local)
	the flow of this information:
	(i)~whether the field was read (to find the set of fields each non-escaping method {reads}),
	(ii)~the minimal and maximal number of times it was sequential and asynchronously written %	minSeq, maxSeq, minAsync, maxAsync
	(to make sure a variable is assigned before read, and that final variables are assigned exactly once).
The number of times a variable is assigned is sufficient to range between 0, 1, 
	and \emph{more-than-one}, 
	because the error message is the same whether a final variable
	was assigned twice or more.
When flowing out of an \code{async}, sequential writes become asynchronous writes,
	and the opposite happens for a \code{finish}.

Our code-base consists of 5 major components:
	(i)~XRX: X10 runtime and libraries,
	(ii)~SPECjbb: SPECjbb from 2005 converted to X10,
	(iii)~M3R: map-reduce in X10,
	(iv)~UTS: global load balancing library,
	(v)~MISC: those include examples from our programmer guide, our test suite, jira issues, and samples.
SPECjbb and M3R are still under development and not publicly available,
	whereas the rest are open-source and available at \code{x10-lang.org}
	(see revision 23028 of \code{https://x10.svn.sf.net/svnroot/x10/trunk}).


\Ref{Table}{CompilationTime} shows the compilation times broken down according to the time spent 
	for checking fields and locals.
We can see that the initialization rules take only a small fraction (0-2\%) 
	from the total compilation time, and a maximum of 3.3 seconds for the entire M3R project.

\begin{table}[htpb!]
\begin{center}
\begin{tabular}{|l|r|r|r|r|r|}
\hline
&  XRX & SPECjbb  &   M3R & UTS & MISC \\
\hline
total   &        65,241 &  78,952  &     254,020 & 72,205  & 548,547 \\
fields  &        156 & 1,649   &     3,330  &  1,272  &  2,862 \\
locals  &        32 & 51    &  117 & 33 & 126 \\
\hline
\end{tabular}
\end{center}
\caption{Compilation times in milliseconds of our code-base broken down into the time spent by the initialization rules for fields and locals.
	We used a standard lenovo T500 laptop with 4GB of RAM and Intel Core 2 Duo processor.} %Intel Core 2 Duo T9600 (2.80GHz, 1066MHz FSB, 6MB Cache)
\label{Table:CompilationTime}
\end{table}


\Ref{Table}{AnnotationBurden} shows the annotation burden in our code-base.
X10 has only two possible method annotations: \code{@NonEscaping} and \code{@NoThisAccess}.
Recall that 
	all methods transitively called from a constructor are implicitly non-escaping,
	i.e., the user does not have to explicitly annotate them as \code{@NonEscaping},
	however 
	the compiler issues a warning recommending that they should be marked as such to show intent.
Obviously, the number of non-escaping methods is always greater or equal to the number of \code{@NonEscaping} annotations.
As can be seen, the annotations burden is minor: only 104 annotations in total.
	

\begin{table*}[htpb!]
\begin{center}
\begin{tabular}{|l|r|r|r|r|r|}
\hline
&  XRX & SPECjbb  &   M3R & UTS & MISC \\
\hline
\# of lines         		& 27,153 	&  14,603  &71,682 	&  2,765 &   155,345 \\
\# of files         		& 257 		& 63      &294 		&14  	&2,283 \\
\# of constructors   		&    276 	& 267     &401 		&23  	&1,297 \\
\# of methods        		&    2,216   & 2,475    &2,831    	&124 	&8,273 \\
\# of non-escaping methods 	&  8  		& 38      &34  		&3   	&83 \\
\# of \code{@NonEscaping}	&   7 		&  7      & 13  	&1   	&62 \\
\# of \code{@NoThisAccess} &  1 		&  0      & 1   	&0   	&12 \\
\hline
\end{tabular}
\end{center}
\caption{The annotation burden in our code-base.}
\label{Table:AnnotationBurden}
\end{table*}

Our applications only use \code{@NoThisAccess} twice:
	once in M3R to allow a subclass to determine the value of a final field of the superclass during initialization,
	and the second time in XRX in method \code{typeName()} of interface \code{Any}
	(this method may be overridden  and it is often called during construction for debugging purposes).

The following example shows a common pattern for using \code{@NonEscaping} 
	and a common refactoring that was done when converting Java code to X10.
%Should we mention converting the harmony classes to X10?
Class \code{HashMap} in Java calls \code{put} in two constructors:
	the deserialization constructor and the copy constructor (that gets a map argument and creates a copy of that map).
However, \code{put} is not a final method and it might be useful to override it in subclasses,
	and therefore it cannot be called during construction.
Thus, we refactored this code in X10 and called instead a non-escaping method called \code{putInternal}
	and method \code{put} delegates to that method:
\begin{lstlisting}
public def put(k: K, v: V) { putInternal(k,v); }
@NonEscaping protected final def putInternal(k:K, v:V) {
  ... }
\end{lstlisting}
A similar refactoring was also done in \code{HashMap} for method \code{rehash}.
%def this(x:SerialData) { ... putInternal(...) ... }

\emph{Asynchronous initialization} was not used in our big applications because they pre-date this feature.
(It is used in our smaller examples and tests more than 50 times.)
The pattern is especially useful for local variables,
	and more importantly, the analysis prevents bugs such as:
\begin{lstlisting}
val x:Int;
val y:Int;
finish {
  async {
    x = doCalculation1();
  }
  y = doCalculation2();
  // WRONG to use variable x here
}
// OK to use variable x here
\end{lstlisting}


