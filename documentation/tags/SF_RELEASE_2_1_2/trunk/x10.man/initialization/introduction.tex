Constructing an object in a safe way is not easy:
    it is well known that dynamic dispatching
    or leaking \this during object construction
    is error-prone~\cite{Dean:1996,Seo:2007:SBD:1522565.1522587,Gil:2009:WRS:1615184.1615216},
    and various type systems and verifiers have been proposed to
    handle safe object initialization~\cite{Hubert:2010:ESO:1888881.1888890,Zibin:2010:OIG:1869459.1869509,Fahndrich:2007:EOI:1297027.1297052,XinQi:2009}.
As languages become more and more complex,
    new pitfalls are created due to the interactions between
    language features.

X10 is an object oriented programming language with a sophisticated
    type system (constraints, class invariants, non-erased generics, closures)
    and concurrency constructs (asynchronous activities, multiple places, global references).
This paper shows that object initialization is a cross-cutting concern
    that interact with other features in the language.
We discuss several language designs that restrict these interactions,
    and explain why we chose the \emph{hardhat} design for X10.

{Hardhat} was termed in~\cite{Gil:2009:WRS:1615184.1615216}
    and it describes a design that prohibits dynamic dispatching or leaking \this during construction.
A hardhat design limits the user
    but also protects her from future bugs.
X10's hardhat design is even stricter due to additional language features
    such as concurrency, places, and closures.

On the other end of the spectrum,
    Java and C\# allow
    dynamic dispatching and leaking \this.
However, they still maintain type- and runtime- safety
    by relying on the fact that every type has a default zero value
    (whether that zero is 0, false, or \code{null}),
    and all fields are zero-initialized before the constructor begins.
As a consequence,
    a half-baked object can leak before all its fields are set. %\cite{Seo:2007:SBD:1522565.1522587} - reading uninitialized field references
Phrased differently,
    when reading a final field, one can read the default value initially and later read a different value.
Another source of subtle bugs is due to the synchronization barrier
    at the end of a constructor~\cite{JSR133}
    after which all assignments to final fields are guaranteed to be written.
The programmer is warned (in the documentation only!)
    that objects with final fields are thread-safe only if
    \this does not escape its constructor.
%Before JSR 133~\cite{JSR133},
%    immutable objects could have different values in different threads
%    if synchronization was not used properly.
Finally, if the type-system is augmented, for example, with non-null types, then
    a default value no longer exists,
    which leads to complicated type-systems for initialization~\cite{Fahndrich:2007:EOI:1297027.1297052,XinQi:2009}.

\mbox{C++}, as usual, gives you enough rope to hang yourself.
Fields are not zero-initialized, and therefore if \this leaks,
    one can read an uninitialized field.
Moreover, method calls are statically bound during construction,
    which may result in an exception at runtime
    if one tries to invoke a virtual method of an abstract class (see \Ref{Figure}{Dynamic-dispatch} below).
(Determining whether this happens is an intractable problem~\cite{Gil:1998:CTA:646155.679689}.)

The remainder of this paper is organized as follows.
\Ref{Section}{rules} presents the hardhat initialization rules of X10 version 2.1
    using examples,
    by slowly adding language features and describe their interaction with
    object initialization.
\Ref{Section}{designs} describes alternative designs for object initialization
    (one was implemented in X10 version 2.0 and another was under consideration for 2.1),
    weighing the pros and cons of each.
\Ref{Section}{implementation} outlines our implementation using the polyglot framework
    and the compilation time overhead of checking these initialization rules.
\Ref{Section}{case-study} investigates the consequences of having a hardhat design
    in a case study that converted a large collection of Java classes to X10.
\Ref{Section}{related-work} summarizes previous work in the field of object initialization.
Finally, \Ref{Section}{conclusion} concludes.
