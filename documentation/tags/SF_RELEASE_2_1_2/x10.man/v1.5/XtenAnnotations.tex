\chapter{Annotations and Compiler
Plugins}\label{XtenAnnotations}\index{annotations}


X10 provides an 
an annotation system and compiler plugin system for to allow the
compiler to be extended with new static analyses and new
transformations.

Annotations are interface types that decorate the abstract syntax tree
of an X10 program.  The X10 type-checker ensures that an annotation
is a legal interface type.
In X10, interfaces may declare
both methods and properties.  Therefore, like any interface type, an
annotation may instantiate
one or more of its interface's properties.
Unlike with Java
annotations,
property initializers need not be
compile-time constants;
however, a given compiler plugin
may do additional checks to constrain the allowable
initializer expressions.
The X10 type-checker does not check that
all properties of an annotation are initialized,
although this could be enforced by
a compiler plugin.

\section{Annotation syntax}

The annotation syntax consists of an ``\texttt{@}'' followed by an interface type.
\begin{x10}
532   Annotation ::= @ InterfaceType
533   Annotations ::= Annotation
534     | Annotations Annotation
535   Annotationsopt ::=
536     | Annotations
\end{x10}
Annotations can be applied to most syntactic constructs in the language
including class declarations, constructors, methods, field declarations,
local variable declarations and formal parameters, statements,
expressions, and types.
Multiple occurrences of the same annotation (i.e., multiple
annotations with the same interface type) on the same entity are permitted.

\begin{x10}
537   ClassModifier ::= Annotation
538   InterfaceModifier ::= Annotation
539   FieldModifier ::= Annotation
540   MethodModifier ::= Annotation
541   VariableModifier ::= Annotation
542   ConstructorModifier ::= Annotation
543   AbstractMethodModifier ::= Annotation
544   ConstantModifier ::= Annotation
545   Type ::= AnnotatedType
546   AnnotatedType ::= Type Annotations
547   Statement ::= AnnotatedStatement
548   AnnotatedStatement ::= Annotation Statement
549   Expression ::= AnnotatedExpression
550   AnnotatedExpression ::= ( Annotations ) Expression
\end{x10}
\noindent
Recall that interface types may have dependent parameters.

\noindent
The following examples illustrate the syntax:

\begin{itemize}
\item Declaration annotations:
\begin{x10}
  // class annotation
  @Value
  class Cons \{ ... \}

  // method annotation
  @PreCondition(0 <= i \&\& i < this.size)
  public Object get(int i) \{ ... \}

  // constructor annotation
  @Where(x != null)
  C(T x) \{ ... \}

  // constructor return type annotation
  C@Initialized(T x) \{ ... \}

  // variable annotation
  @Unique A x;
\end{x10}
\item Type annotations:
\begin{x10}
  List@Nonempty

  int@Range(1,4)

  double[][]@Size(n * n)
\end{x10}
\item Expression annotations:
\begin{x10}
  (@RemoteCall) m()

  3 == (@Bits(2)) 15
\end{x10}
\item Statement annotations:
\begin{x10}
  @Atomic \{ ... \}

  @MinIterations(1)
  @MaxIterations(n)
  for (int i = 0; i < n; i++) \{ ... \}

  // An annotated empty statement ;
  @Assert(x < y);
\end{x10}
\end{itemize}

\section{Annotation declarations}

Annotations are declared as interfaces.  They must be
subtypes of \texttt{x10.lang.annotation.Annotation}.
Annotations on types, expressions, statements, classes, fields,
methods, constructors, and local variable declarations (or
formal parameters)
must extend
\texttt{ExpressionAnnotation},
\texttt{StatementAnnotation},
\texttt{ClassAnnotation},
\texttt{FieldAnnotation},
\texttt{MethodAnnotation},
\texttt{ConstructorAnnotation}, and
\texttt{VariableAnnotation}, respectively.

\section{Compiler plugins}
\index{plugins}

After the base X10 semantic checking is completed, 
compiler plugins are loaded and run.  Plugins may perform
any number of compiler passes to implement
additional semantic checking and code transformations, including
transformations using the abstract syntax of the annotations
themselves.  Plugins should output valid X10 abstract
syntax trees.

Plugins are implemented in java as
Polyglot~\cite{ncm03} passes applied to the AST
after normal base X10 type checking.
Plugins to run are specified on the command-line.  The order of
execution is determined by the Polyglot pass scheduler.
\index{Polyglot}

To run compiler plugins, add the command-line option:
\begin{x10}
  -PLUGINS=P1,P2,...,Pn
\end{x10}
where \texttt{P1}, \texttt{P2}, \dots, \texttt{Pn} are classes that implement the
\texttt{CompilerPlugin} interface:
\index{CompilerPlugin}

\begin{x10}
  package polyglot.ext.x10.plugin;

  import polyglot.ext.x10.ExtensionInfo;
  import polyglot.frontend.Job;
  import polyglot.frontend.goals.Goal;

  public interface CompilerPlugin \{
      public Goal register(ExtensionInfo extInfo, Job job);
  \}
\end{x10}

\index{Goal}
The \texttt{Goal} object returned by the \texttt{register} method specifies dependencies on other passes.
Documentation for Polyglot can be found at:
\begin{x10}
  http://www.cs.cornell.edu/Projects/polyglot
\end{x10}
Most plugins should implement either \texttt{SimpleOnePassPlugin} or
\texttt{SimpleVisitorPlugin}.

The compiler loads plugin classes from the x10c classpath.

Plugins are given access to a Polyglot AST and type system.  Annotations are
represented in the AST as \texttt{Node}s with the following interface:
\index{Node}

\begin{x10}
  package polyglot.ext.x10.ast;

  public interface AnnotationNode extends Node \{
    X10ClassType annotation();
  \}
\end{x10}

Annotations for a \texttt{Node} object \texttt{n} can be accessed through the
node's extension object as follows:
\index{AnnotationNode}

\begin{x10}
  List<AnnotationNode> annotations =
    ((X10Ext) n.ext()).annotations();
  List<X10ClassType> annotationTypes =
    ((X10Ext) n.ext()).annotationInterfaces();
\end{x10}
In the type system, \texttt{X10TypeObject} has the following
method for accessing annotations:
\begin{x10}
  List<X10ClassType> annotations();
\end{x10}


%\balance
\bibliography{pm.bib,db.bib}

% \clearpage

\end{document}


