\chapter{Statements}\label{XtenStatements}\index{statements}

This chapter describes the statements in the sequential core of
\Xten{}.  Statements involving concurrency and distribution
are described in \Sref{XtenActivities}.

\section{Empty statement}

\begin{grammar}
Statement \: \xcd";" \\
\end{grammar}

The empty statement \xcd";" does nothing.  It is useful when a
loop header is evaluated for its side effects.  For example,
the following code sums the elements of a rail.

%~~gen
% package statements.should.be.called.commands.but.nobody.ever.does.that;
% class EmptyStatementExample {
% def summizmo (a:Rail[Int]!){
%~~vis
\begin{xten}
var sum: Int = 0;
for (var i: Int = 0; i < a.length; i++, sum += a(i))
    ;
\end{xten}
%~~siv
%}}
%~~neg

\section{Local variable declaration}

Short-lived variables are introduced by local variables declarations, as
described in \Sref{VariableDeclarations}. Local variables may be declared only
within a block statement (\Sref{Blocks}). The scope of a local variable
declaration is the statement itself and the subsequent statements in the
block.
%~~gen
% package statements.should.have.locals;
% class LocalExample {
% def example(a:Int) {
%~~vis
\begin{xten}
  if (a > 1) {
     val b = a/2;
     var c : Int = 0;
     // b and c are defined here
  }
  // b and c are not defined here.
\end{xten}
%~~siv
%} }
%~~neg


\section{Block statement}
\label{Blocks}

\begin{grammar}
Statement \: BlockStatement \\
BlockStatement \: \xcd"{" Statement\star \xcd"}" \\
\end{grammar}

A block statement consists of a sequence of statements delimited by
``\xcd"{"'' and ``\xcd"}"''. When a block is evaluated, the statements inside
of it are evaluated in order.  Blocks are useful for putting several
statements in a place where X10 asks for a single one, such as the consequent
of an \xcd`if`, and for limiting the scope of local variables.
%~~gen
% package statements.FOR.block.heads;
% class Example {
% def example(b:Boolean, S1:(Int)=>Void, S2:(Int)=>Void ) {
%~~vis
\begin{xten}
if (b) {
  // This is a block
  val v = 1;
  S1(v); 
  S2(v);
}
\end{xten}
%~~siv
%  } } 
%~~neg


\section{Expression statement}

\begin{grammar}
Statement \: ExpressionStatement \\
ExpressionStatement \: StatementExpression \xcd";" \\
StatementExpression \: Assignment \\
          \| NewExpression \\
          \| Call \\
\end{grammar}

The expression statement evaluates an expression.  The value of the expression
is not used.
Side effects of the expression occur.  
%~~gen
% package Sta.tem.ent.s.expressions;
% import x10.util.*;
%~~vis
\begin{xten}
class StmtEx {
  def this() { x10.io.Console.OUT.println("New StmtEx made");  }
  static def call() { x10.io.Console.OUT.println("call!");  }
  def example() {
     var a : Int = 0;
     a = 1; // assignment
     new StmtEx(); // allocation
     call(); // call
  }
}
\end{xten}
%~~siv
%
%~~neg
Note that only selected forms of expression can appear in expression
statements.  


\section{Labeled statement}

\begin{grammar}
Statement \: LabeledStatement \\
LabeledStatement \: Identifier \xcd":" LoopStatement \\
\end{grammar}

Loop statements (\xcd`for`, \xcd`while`, \xcd`do`, \xcd`ateach`,
\xcd`foreach`) may be labeled. The label may be used as the target of a break
or continue statement. The scope of a label is the statement labeled.
%~~gen
% package state.meant.labe.L;
% class Example {
% def example(a:(Int,Int) => Int, do_things_to:(Int)=>Int) {
%~~vis
\begin{xten}
lbl : for ((i) in 1..10) {
   for ((j) in i..10) {  
      if (a(i,j) == 0) break lbl;
      if (a(i,j) == 1) continue lbl;
      do_things_to(a(i,j)); 
   }
}
\end{xten}
%~~siv
%} } 
%~~neg


\section{Break statement}

\begin{grammar}
Statement \: BreakStatement \\
BreakStatement \: \xcd"break" Identifier\opt \\
\end{grammar}

An unlabeled break statement exits the currently enclosing loop or switch
statement. An labeled break statement exits the enclosing loop or switch
statement with the given label.
It is illegal to break out of a loop not defined in the current
method, constructor, initializer, or closure.  

The following code searches for an element of a two-dimensional
array and breaks out of the loop when it is found:

%~~gen
% package statements.come.from.banks.and.cranks;
% class LabelledBreakeyBreakyHeart {
% def findy(a:ValRail[ValRail[Int]], v:Int): Boolean {
%~~vis
\begin{xten}
var found: Boolean = false;
outer: for (var i: Int = 0; i < a.length; i++)
    for (var j: Int = 0; j < a(i).length; j++)
        if (a(i)(j) == v) {
            found = true;
            break outer;
        }
\end{xten}
%~~siv
% return found;
%}}
%~~neg

\section{Continue statement}

\begin{grammar}
Statement \: ContinueStatement \\
ContinueStatement \: \xcd"continue" Identifier\opt \\
\end{grammar}

An unlabeled \xcd`continue` skips the rest of the current iteration of the
innermost enclosing loop, and proceeds on to the next.  A labeled
\xcd`continue` does the same to the enclosing loop with that label.
It is illegal to continue a loop not defined in the current
method, constructor, initializer, or closure.

\section{If statement}

\begin{grammar}
Statement \: IfThenStatement \\
          \| IfThenElseStatement \\
IfThenStatement \: \xcd"if" \xcd"(" Expression \xcd")" Statement \\
IfThenElseStatement \: \xcd"if" \xcd"(" Expression \xcd")" Statement \xcd"else" Statement \\
\end{grammar}

An if statement comes in two forms: with and without an else
clause.

The if-then statement evaluates a condition expression, which must be of type
\xcd`Boolean`. If the condition is \xcd`true`, it evaluates the then-clause.
If the condition is \xcd"false", the if-then statement completes normally.

The if-then-else statement evaluates a condition expression and 
evaluates the then-clause if the condition is
\xcd"true"; otherwise, the \xcd`else`-clause is evaluated.

As is traditional in languages derived from Algol, the if-statement is syntactically
ambiguous.  That is, 
\begin{xten}
if (B1) if (B2) S1 else S2
\end{xten}
could be intended to mean either 
\begin{xten}
if (B1) { if (B2) S1 else S2 }
\end{xten} 
or 
\begin{xten}
if (B1) {if (B2) S1} else S2
\end{xten}
X10, as is traditional, attaches an \xcd`else` clause to the most recent
\xcd`if` that doesn't have one.
This example is interpreted as 
\xcd`if (B1) { if (B2) S1 else S2 }`. 



\section{Switch statement}

\begin{grammar}
Statement \: SwitchStatement \\
SwitchStatement \: \xcd"switch" \xcd"(" Expression \xcd")" \xcd"{" Case\plus \xcd"}" \\
Case \: \xcd"case" Expression \xcd":" Statement\star \\
     \| \xcd"default" \xcd":" Statement\star \\
\end{grammar}

A switch statement evaluates an index expression and then branches to
a case whose value equal to the value of the index expression.
If no such case exists, the switch branches to the 
\xcd"default" case, if any.

Statements in each case branch evaluated in sequence.  At the
end of the branch, normal control-flow falls through to the next case, if
any.  To prevent fall-through, a case branch may be exited using
a \xcd"break" statement.

The index expression must be of type \xcd"Int".
Case labels must be of type \xcd"Int", \xcd`Byte`, \xcd`Short`, or \xcd`Char`
and must be compile-time 
constants.  Case labels cannot be duplicated within the
\xcd"switch" statement.

In the following example, case \xcd`1` falls through to case3 \xcd`2`.  The
other cases are separated by \xcd`break`s.
%~~gen
% package Statement.Case;
% class Example {
% def example(i : Int, println: (String)=>Void) {
%~~vis
\begin{xten}
switch (i) {
  case 1: println("one, and ");
  case 2: println("two"); 
          break;
  case 3: println("three");
          break;
  default: println("Something else");
           break;
}
\end{xten}
%~~siv
% } } 
%~~neg


\section{While statement}
\index{\Xcd{while}}

\begin{grammar}
Statement \: WhileStatement \\
WhileStatement \: \xcd"while" \xcd"(" Expression \xcd")" Statement \\
\end{grammar}

A while statement evaluates a \xcd`Boolean`-valued condition and executes a
loop body if \xcd"true". If the loop body completes normally (either by
reaching the end or via a \xcd"continue" statement with the loop header as
target), the condition is reevaluated and the loop repeats if \xcd"true". If
the condition is \xcd"false", the loop exits.

%~~gen
% package Statements.AreFor.Flatements;
% class Example {
% def example(var n:Int) {
%~~vis
\begin{xten}
  while (n > 1) {
     n = (n % 2 == 1) ? 3*n+1 : n/2;
  }
\end{xten}
%~~siv
% } } 
%~~neg

\section{Do--while statement}
\index{\Xcd{do}}

\begin{grammar}
Statement \: DoWhileStatement \\
DoWhileStatement \: \xcd"do" Statement \xcd"while" \xcd"(" Expression \xcd")" \xcd";" \\
\end{grammar}


A do-while statement executes a loop body, and then evaluates a
\xcd`Boolean`-valued condition expression. If \xcd"true", the loop repeats.
Otherwise, the loop exits.



\section{For statement}
\index{\Xcd{for}}

\begin{grammar}
Statement \: ForStatement \\
          \| EnhancedForStatement \\
ForStatement \: \xcd"for" \xcd"("
        ForInit\opt \xcd";" Expression\opt \xcd";" ForUpdate\opt
        \xcd")" Statement \\
ForInit \:
        StatementExpression ( \xcd"," StatementExpression )\star
        \\
      \| LocalVariableDeclaration \\
ForUpdate \:
        StatementExpression ( \xcd"," StatementExpression )\star\\
EnhancedForStatement \: \xcd"for" \xcd"("
        Formal \xcd"in" Expression 
        \xcd")" Statement \\
\end{grammar}

\xcd`for` statements provide bounded iteration, such as looping over a list.
It has two forms: a basic form allowing near-arbitrary iteration, {\em a la}
C, and an enhanced form designed to iterate over a collection.

A basic \xcd`for` statement provides for arbitrary iteration in a somewhat
more organized fashion than a \xcd`while`.  \xcd`for(init; test; step)body` is
equivalent to: 
\begin{xten}
{
   init;
   while(test) {
      body;
      step;
   }
}
\end{xten}

\xcd`init` is performed before the loop, and is traditionally used to declare
and/or initialize the loop variables. It may be a single variable binding
statement, such as \xcd`var i:Int = 0` or \xcd`var i:Int=0, j:Int=100`. (Note
that a single variable binding statement may bind multiple variables.)
Variables introduced by \xcd`init` may appear anywhere in the \xcd`for`
statement, but not outside of it.  Or, it may be a sequence of expression
statements, such as \xcd`i=0, j=100`, operating on already-defined variables.
If omitted, \xcd`init` does nothing.

\xcd`test` is a Boolean-valued expression; an iteration of the loop will only
proceed if \xcd`test` is true at the beginning of the loop, after \xcd`init`
on the first iteration or or \xcd`step` on later ones. If omitted, \xcd`test`
defaults to \xcd`true`, giving a loop that will run until stopped by some
other means such as \xcd`break`, \xcd`return`, or \xcd`throw`.

\xcd`step` is performed after the loop body, between one iteration and the
next. It traditionally updates the loop variables from one iteration to the
next: \eg, \xcd`i++` and \xcd`i++,j--`.  If omitted, \xcd`step` does nothing.

\xcd`body` is a statement, often a code block, which is performed whenever
\xcd`test` is true.  If omitted, \xcd`body` does nothing.




\label{ForAllLoop}


An enhanced for statement is used to iterate over a collection, or other
structure designed to support iteration by implementing the interface
\xcd`Iterable[T]`.    The loop variable must be of type \xcd`T`, 
or destructurable from a value of type \xcd`T`
(\Sref{exploded-syntax}; in practice, this means that 
\xcd`for ((i) in 1..10)` iterates over numbers from 1 to 10, while 
\xcd`for (i in 1..10` iterates over \xcd`Point`s from 1 to 10).
Each iteration of the loop
binds the iteration variable to another element of the collection.

%~~gen
% package statements.for_for;
% class Example {
% def example(n:Int) {
%~~vis
\begin{xten}
  var sum : Int = 0;
  for ((i) in 1..n) sum += i;
\end{xten}
%~~siv
% } } 
%~~neg



Certain common variant cases are accepted.  If collection is of type
\xcd"Region", the iteration variable may be of type \xcd"Point". 
If the iteratable \xcd`e` is of type \xcd`Dist` or \xcd`Array`, it is treated
as if it were \xcd`e.region`.  





\section{Throw statement}
\index{throw}

\begin{grammar}
Statement \: ThrowStatement \\
ThrowStatement \: \xcd"throw" Expression \xcd";"
\end{grammar}

\index{Exception}
The \xcd"throw" statement throws an exception, which 
must be an instance of \xcd"x10.lang.Throwable". 

For example, the following code checks if an index is in range and
throws an exception if not.

%~~gen
% package statements.are.from.mars.expressions.are.from.venus;
% class ThrowStatementExample {
% def thingie(i:Int, x:ValRail[Boolean]) throws MyIndexOutOfBoundsException {
%~~vis
\begin{xten}
if (i < 0 || i >= x.length)
    throw new MyIndexOutOfBoundsException();
\end{xten}
%~~siv
%} }
% class MyIndexOutOfBoundsException extends Exception {}
%~~neg

\section{Try--catch statement}

\begin{grammar}
Statement \: TryStatement \\
TryStatement \: \xcd"try" BlockStatement Catch\plus  \\
             \| \xcd"try" BlockStatement Catch\star Finally \\
Catch \: \xcd"catch" \xcd"(" Formal \xcd")" BlockStatement \\
Finally \: \xcd"finally" BlockStatement \\
\end{grammar}

Exceptions are handled with a \xcd"try" statement.
A \xcd"try" statement consists of a \xcd"try" block, zero or more
\xcd"catch" blocks, and an optional \xcd"finally" block.

First, the \xcd"try" block is evaluated.  If the block throws an
exception, control transfers to the first matching \xcd"catch"
block, if any.  A \xcd"catch" matches if the value of the
exception thrown is a subclass of the \xcd"catch" block's formal
parameter type.

The \xcd"finally" block, if present, is evaluated on all normal
and exceptional control-flow paths from the \xcd"try" block.
If the \xcd"try" block completes normally
or via a \xcd"return", a \xcd"break", or a
\xcd"continue" statement, 
the \xcd"finally"
block is evaluated, and then control resumes at
the statement following the \xcd"try" statement, at the branch target, or at
the caller as appropriate.
If the \xcd"try" block completes
exceptionally, the \xcd"finally" block is evaluated after the
matching \xcd"catch" block, if any, and then the
exception is rethrown.

It is a static error to attempt to catch an exception type which is not
throwable by the block.  

\section{Return statement}
\label{ReturnStatement}
\index{ReturnStatement}
\begin{grammar}
Statement \: ReturnStatement \\
ReturnStatement \: \xcd"return" Expression \xcd";" \\
             \| \xcd"return" \xcd";" \\
\end{grammar}

Methods and closures may return values using a return statement.
If the method's return type is expliclty declared \xcd"Void",
the method must return without a value; otherwise, it must return
a value of the appropriate type.
