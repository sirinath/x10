With value arguments, type arguments, and constraints, the syntax for \Xten{}
types can often be verbose. 
For example, a non-null list of non-null strings is \\
%~~type~~`~~`~~ ~~import x10.util.*;
\xcd`List[String{self!=null}]{self!=null}`.
\Xten{} provides {\em type definitions}
to allow users to give short names to long types, and to commonly-used
combinations of types. 
We could name that type: 
%~~gen
% package TypeDefs.glip.first;
% import x10.util.*;
% class LnSn {
% static
%~~vis
\begin{xten}
type LnSn = List[String{self!=null}]{self!=null};
\end{xten}
%~~siv
%}
%~~neg
Or, we could abstract it somewhat, defining a type constructor
\xcd`Nonnull[T]` for the type of \xcd`T`'s which are not null:
%~~gen
% package TypeDefs.glip.second;
% import x10.util.*;
% class LnSn {
% def example() {
%~~vis
\begin{xten}
type Nonnull[T] = T{self!=null};
type LnSn = Nonnull[List[Nonnull[String]]];
var example : LnSn;
\end{xten}
%~~siv
%}}
%~~neg

Type definitions can also refer to values, in particular, inside of
constraints.  The type of \xcd`n`-element \xcd`Rails[Int]s` is 
%~~type~~`~~`~~n:Int ~~
\xcd`Rail[Int]{self.length == n}`
but it is often convenient to give a shorter name: 
%~~gen
% package TypeDefs.glip.third;
% class Xmpl {
% def example() {
%~~vis
\begin{xten}
type Rail(n:Int) = Rail[Int]{self.length == n};
var example : Rail(78); 
\end{xten}
%~~siv
%}}
%~~neg

Type definitions, like many other X10 abstractions, can have constraints on
their use. 
\bard{Have we seen this -- in Overview?}
\begin{xten}

\end{xten}


Type definitions have the following syntax:

\begin{grammar}
TypeDefinition \: 
                \xcd"type"~Identifier
                           ( \xcd"[" TypeParameters \xcd"]" )\opt \\
                        && ( \xcd"(" Formals \xcd")" )\opt
                            Constraint\opt \xcd"=" Type \\
\end{grammar}

\noindent
A type definition can be thought of as a type-valued function,
mapping type parameters and value parameters to a concrete type.
%
The following examples are legal type definitions, given \xcd`import x10.util.*`:
%~~gen
% import x10.util.*;
% class TypeDefBrow {
% def someTypeDefs () {
%~~vis
\begin{xten}
type StringSet = Set[String];
type MapToList[K,V] = Map[K,List[V]];
type Int(x: Int) = Int{self==x};
type Dist(r: Int) = Dist{self.rank==r};
type Dist(r: Region) = Dist{self.region==r};
type Redund(n:Int, r:Region){r.rank==n} = Dist{rank==n && region==r};
\end{xten}
%~~siv
% }}
%~~neg
\label{TypeDefGuard}
As the two definitions of \xcd"Dist" demonstrate, type definitions may 
be overloaded: two type definitions with different numbers of type
parameters or with different types of value
parameters, according to the method overloading rules
(\Sref{MethodOverload}), define distinct type constructors.

Type definitions may appear as (static) class or interface member or
in a block statement.

Type definitions are applicative, not generative; that is, they
define aliases for types but do not introduce new types.
Thus, the following code is legal:
%~~gen
% import x10.util.*;
% class TypeDefNonGenerative {
% def someTypeDefs () {
%~~vis
\begin{xten}
type A = Int;
type B = String;
type C = String;
a: A = 3;
b: B = new C("Hi");
c: C = b + ", Mom!";
\end{xten}
%~~siv
% }}
%~~neg
% An instance of a defined type with no type parameters and no
% value parameters may 
If a type definition has no type parameters and no value
parameters and is an alias for a class type, a \xcd"new"
expression may be used to create an instance of the class using
the type definition's name.
Given the following type definition:
\begin{xtenmath}
type A = C[T$_1$, $\dots$, T$_k$]{c};
\end{xtenmath}
where 
\xcdmath"C[T$_1$, $\dots$, T$_k$]" is a
class type, a constructor of \xcdmath"C" may be invoked with
\xcdmath"new A(e$_1$, $\dots$, e$_n$)", if the
invocation
\xcdmath"new C[T$_1$, $\dots$, T$_k$](e$_1$, $\dots$, e$_n$)" is
legal and if the constructor return type is a subtype of
\xcd"A".

The collection of type definitions in
\xcdmath"x10.lang.\_" is automatically imported in every compilation unit.
\index{import,type definitions}\label{X10LangUnderscore}

\if 0
All type definitions are members of their enclosing package or
class.  A compilation unit may have one or more type definitions
or class or interface declarations with the same name, as long
as the definitions have distinct parameters according to the
method overloading rules (\Sref{MethodOverload}).
\fi

