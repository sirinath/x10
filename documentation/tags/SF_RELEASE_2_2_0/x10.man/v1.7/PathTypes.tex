

Types now include /path-dependent types/ of the form base.f1.f2.f3.T.

(Paths) p ::= x         // x must be a final variable or property in scope
           |  this      // the current object
           |  super     // mostly an alias for this
           |  T.this    // a T object surrounding the current object
           |  C.f       // where f is a static property
           |  p.f       // where f is a property (or final field) in p
           |  p.C.f     // where f is a static property
(Types) T ::= p.T       // where T is a type property

Rationale: We allow super in paths because super is already allowed
widely in X10 expressions. Mostly this and super they are the same, but
they might be different if the inherited type of a field is different
from the type of the field in the current class.

Note that the compiler may create temporary final variables and use
their names to construct types. For instance:

class Foo<T>(T k) { 
  T k() {
    return k;
  }
  Foo<T>(k) clone() {
    return new Foo<T>(k);
  }
}
Foo foo = ...; // foo is not final. 
_ x = foo.k(); // what is the type?

The compiler generates a temporary variable for type checking:

Foo foo = ...; // foo is not final. 
final Foo _1 = foo;  grab current version of foo
_ x = foo.k();  // the type is _1.T

Note that types are only valid within certain scopes. They must be
rewritten when a field or method is selected to reflect the change in
scope. For example:

class Bar<T> {...}
class Foo(final Bar y) { 
  y.T z;  // the type of z is really this.y.T
};
Foo x = ...
x.y.T z = x.z   // The type of x.z is this.y.T[x/this] 

The rule is that the type of any field access x.f is the type of the
field f in the type of x with any occurrences of this replaced by x.
Likewise for method invocation.


In any context, the same type may have many different names.

class Bar<T> {...}
final Bar x = ...;
final Bar y = x;  
// now x.T is the same as y.T through the end of this block

Scala addresses name aliases by using canonicalization; in this example,
whenever the programmer writes y.T, the compiler would canonicalize it
to x.T. However, it is likely more synergistic with X10's dependent
types to use a constraints approach. In that case, the type checker
would track the constraint that x==y and would use it in type checking.
During type inference (see #Type Elision <#Type_Elision> above), type
constraint self==x should be added to y's type.

Paths must always have a base that is in scope.

[edit </mediawiki/index.php?title=X10_1.5_Design&action=edit&section=28>]


