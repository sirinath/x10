\documentclass[a4paper]{article}


\usepackage{fullpage}

\title{Initialization in X10 - Technical Report}

\author{Yoav Zibin \texttt{yoav.zibin@gmail.com}}

\date{}




\usepackage{xspace}

% Macros for R^nRS.

\def\@makechapterhead#1{%
  \vspace*{50\p@}%
  {\parindent \z@ \raggedright \normalfont
    \ifnum \c@secnumdepth >\m@ne
        \huge\bfseries \thechapter \space\space\space
        \nobreak
    \fi
    \interlinepenalty\@M
    \Huge \bfseries #1\par\nobreak
    \vskip 40\p@
  }}


\makeatletter

\newcommand{\topnewpage}{\@topnewpage}

% Chapters, sections, etc.

\newcommand{\vest}{}
\newcommand{\dotsfoo}{$\ldots\,$}

\newcommand{\sharpfoo}[1]{{\tt\##1}}
\newcommand{\schfalse}{\sharpfoo{f}}
\newcommand{\schtrue}{\sharpfoo{t}}

\newcommand{\singlequote}{{\tt'}}  %\char19
\newcommand{\doublequote}{{\tt"}}
\newcommand{\backquote}{{\tt\char18}}
\newcommand{\backwhack}{{\tt\char`\\}}
\newcommand{\atsign}{{\tt\char`\@}}
\newcommand{\sharpsign}{{\tt\#}}
\newcommand{\verticalbar}{{\tt|}}

\newcommand{\coerce}{\discretionary{->}{}{->}}

% Knuth's \in sucks big boulders
\def\elem{\hbox{\raise.13ex\hbox{$\scriptstyle\in$}}}

\newcommand{\meta}[1]{{\noindent\hbox{\rm$\langle$#1$\rangle$}}}
\let\hyper=\meta
\newcommand{\hyperi}[1]{\hyper{#1$_1$}}
\newcommand{\hyperii}[1]{\hyper{#1$_2$}}
\newcommand{\hyperj}[1]{\hyper{#1$_i$}}
\newcommand{\hypern}[1]{\hyper{#1$_n$}}
\newcommand{\var}[1]{\noindent\hbox{\it{}#1\/}}  % Careful, is \/ always the right thing?
\newcommand{\vari}[1]{\var{#1$_1$}}
\newcommand{\varii}[1]{\var{#1$_2$}}
\newcommand{\variii}[1]{\var{#1$_3$}}
\newcommand{\variv}[1]{\var{#1$_4$}}
\newcommand{\varj}[1]{\var{#1$_j$}}
\newcommand{\varn}[1]{\var{#1$_n$}}

\newcommand{\vr}[1]{{\noindent\hbox{$#1$\/}}}  % Careful, is \/ always the right thing?
\newcommand{\vri}[1]{\vr{#1_1}}
\newcommand{\vrii}[1]{\vr{#1_2}}
\newcommand{\vriii}[1]{\vr{#1_3}}
\newcommand{\vriv}[1]{\vr{#1_4}}
\newcommand{\vrv}[1]{\vr{#1_5}}
\newcommand{\vrj}[1]{\vr{#1_j}}
\newcommand{\vrn}[1]{\vr{#1_n}}


\newcommand{\defining}[1]{\mainindex{#1}{\em #1}}
\newcommand{\ide}[1]{{\schindex{#1}\frenchspacing\tt{#1}}}

\newcommand{\lambdaexp}{{\cf lambda} expression}
\newcommand{\Lambdaexp}{{\cf Lambda} expression}
\newcommand{\callcc}{{\tt call-with-current-continuation}}

% \reallyindex{SORTKEY}{HEADCS}{TYPE}
% writes (index-entry "SORTKEY" "HEADCS" TYPE PAGENUMBER)
% which becomes  \item \HEADCS{SORTKEY} mainpagenumber ; auxpagenumber ...

\global\def\reallyindex#1#2#3{%
\write\@indexfile{"#1" "#2" #3 \thepage}}

\newcommand{\mainschindex}[1]{\label{#1}\reallyindex{#1}{tt}{main}}
\newcommand{\mainindex}[1]{\reallyindex{#1@{\rm #1}{main}}}
\newcommand{\schindex}[1]{\reallyindex{#1}{tt}{aux}}
\newcommand{\sharpindex}[1]{\reallyindex{#1}{sharpfoo}{aux}}
%vj%\renewcommand{\index}[1]{\reallyindex{#1}{rm}{aux}}

\newcommand{\domain}[1]{#1}
\newcommand{\nodomain}[1]{}
%\newcommand{\todo}[1]{{\rm$[\![$!!~#1$]\!]$}}
\newcommand{\todo}[1]{}

% \frobq will make quote and backquote look nicer.
\def\frobqcats{%\catcode`\'=13 %\catcode`\{=13{}\catcode`\}=13{}
\catcode`\`=13{}}
{\frobqcats
\gdef\frobqdefs{%\def'{\singlequote}
\def`{\backquote}}}%\def\{{\char`\{}\def\}{\char`\}}
\def\frobq{\frobqcats\frobqdefs}

% \cf = code font
% Unfortunately, \cf \cf won't work at all, so don't even attempt to
% next constructions which use them...
\newcommand{\cf}{\frenchspacing\tt}

% Same as \obeycr, but doesn't do a \@gobblecr.
{\catcode`\^^M=13 \gdef\myobeycr{\catcode`\^^M=13 \def^^M{\\}}%
\gdef\restorecr{\catcode`\^^M=5 }}

{\catcode`\^^I=13 \gdef\obeytabs{\catcode`\^^I=13 \def^^I{\hbox{\hskip 4em}}}}

{\obeyspaces\gdef {\hbox{\hskip0.5em}}}

\gdef\gobblecr{\@gobblecr}

\def\setupcode{\@makeother\^}

% Scheme example environment
% At 11 points, one column, these are about 56 characters wide.
% That's 32 characters to the left of the => and about 20 to the right.

\newenvironment{x10noindent}{
  % Commands for scheme examples
  \newcommand{\ev}{\>\>\evalsto}
  \newcommand{\lev}{\\\>\evalsto}
  \newcommand{\unspecified}{{\em{}unspecified}}
  \newcommand{\scherror}{{\em{}error}}
  \setupcode
  \small \cf \obeytabs \obeyspaces \myobeycr
  \begin{tabbing}%
\qquad\=\hspace*{5em}\=\hspace*{9em}\=\kill%   was 16em
\gobblecr}{\unskip\end{tabbing}}

%\newenvironment{scheme}{\begin{schemenoindent}\+\kill}{\end{schemenoindent}}
\newenvironment{x10}{
  % Commands for scheme examples
  \newcommand{\ev}{\>\>\evalsto}
  \newcommand{\lev}{\\\>\evalsto}
  \renewcommand{\em}{\rmfamily\itshape}
  \newcommand{\unspecified}{{\em{}unspecified}}
  \newcommand{\scherror}{{\em{}error}}
  \setupcode
  \small \cf \obeyspaces \myobeycr
  \footnotesize
  \begin{tabbing}%
\qquad\=\hspace*{5em}\=\hspace*{9em}\=\+\kill%   was 16em
\gobblecr}{\unskip\end{tabbing}\normalsize}

\newcommand{\evalsto}{$\Longrightarrow$}

% Manual entries

\newenvironment{entry}[1]{
  \vspace{3.1ex plus .5ex minus .3ex}\noindent#1%
\unpenalty\nopagebreak}{\vspace{0ex plus 1ex minus 1ex}}

\newcommand{\exprtype}{syntax}

% Primitive prototype
\newcommand{\pproto}[2]{\unskip%
\hbox{\cf\spaceskip=0.5em#1}\hfill\penalty 0%
\hbox{ }\nobreak\hfill\hbox{\rm #2}\break}

% Parenthesized prototype
\newcommand{\proto}[3]{\pproto{(\mainschindex{#1}\hbox{#1}{\it#2\/})}{#3}}

% Variable prototype
\newcommand{\vproto}[2]{\mainschindex{#1}\pproto{#1}{#2}}

% Extending an existing definition (\proto without the index entry)
\newcommand{\rproto}[3]{\pproto{(\hbox{#1}{\it#2\/})}{#3}}

% Grammar environment

\newenvironment{grammar}{
  \def\:{& \goesto{} &}
  \def\|{& $\vert$& }
  \def\opt{$^?$\ }
  \def\star{$^*$\ }
  \def\plus{$^+$\ }
  \em
  \begin{tabular}{rcl}
  }{\unskip\end{tabular}}

%\newcommand{\unsection}{\unskip}
\newcommand{\unsection}{{\vskip -2ex}}

% Commands for grammars
\newcommand{\arbno}[1]{#1\hbox{\rm*}}  
\newcommand{\atleastone}[1]{#1\hbox{$^+$}}

\newcommand{\goesto}{{\normalfont{::=}}}

% mark modifications (for the grammar) From Igor Pechtchanski/Watson/IBM@IBMUS
\newlength{\modwidth}\setlength{\modwidth}{0.005in}
\newlength{\modskip}\setlength{\modskip}{.4em}
\newlength{\@modheight}
\newlength{\@modpos}
\providecommand{\markmod}[1]{%
  \setlength{\@modheight}{#1}%
  \addtolength{\@modheight}{-0.06in}%
  \setlength{\@modpos}{\linewidth}%
  \addtolength{\@modpos}{0.285in}%         Magic
  \addtolength{\@modpos}{\modwidth}%
  \addtolength{\@modpos}{\modskip}%
  \marginpar{\vspace{-\@modheight}%
             \hspace{-\@modpos}%
             \rule{\modwidth}{#1}}%
}

% The index

\def\theindex{%\@restonecoltrue\if@twocolumn\@restonecolfalse\fi
%\columnseprule \z@
%!! \columnsep 35pt
\clearpage
\@topnewpage[
    \centerline{\large\bf\uppercase{Alphabetic index of definitions of concepts,}}
    \centerline{\large\bf\uppercase{keywords, and procedures}}
    \vskip 1ex \bigskip]
    \markboth{Index}{Index}
    \addcontentsline{toc}{chapter}{Alphabetic index of 
 definitions of concepts, keywords, and procedures}
    \bgroup %\small
    \parindent\z@
    \parskip\z@ plus .1pt\relax\let\item\@idxitem}

\def\@idxitem{\par\hangindent 40pt}

\def\subitem{\par\hangindent 40pt \hspace*{20pt}}

\def\subsubitem{\par\hangindent 40pt \hspace*{30pt}}

\def\endtheindex{%\if@restonecol\onecolumn\else\clearpage\fi
\egroup}

\def\indexspace{\par \vskip 10pt plus 5pt minus 3pt\relax}

\makeatother

%\newcommand{\Xten}{{\sf X10}}
%\newcommand{\XtenCurrVer}{{\sf X10 v1.7}}
%\newcommand{\java}{{\sf Java}}
%\newcommand{\Java}{{\sf Java}}

\newcommand{\Xten}{X10}
\newcommand{\XtenCurrVer}{\Xten{} v1.7}
\newcommand{\Java}{Java}
\newcommand{\java}{\Java{}}

\newcommand{\futureext}[1]{{\em \paragraph{Future Extensions.}#1}}
\newcommand{\tbd}{} % marker for things to be done later.
\newcommand{\limitation}[1]{{\em Limitation: #1}} % marker for things to be done later.


\newcommand\grammarrule[1]{\emph{#1}}

% Rationale

\newenvironment{rationale}{%
\bgroup\noindent{\sc Rationale:}\space}{%
\egroup}

% Notes

\newenvironment{note}{%
\bgroup\noindent{\sc Note:}\space}{%
\egroup}

\newenvironment{staticrule*}{%
\bgroup\noindent{\textsc{Static Semantics Rule:}\space}}{%
\egroup}

\newenvironment{staticrule}[1]{%
\bgroup\noindent{\textsc{Static Semantics Rule} (#1):\space}}{%
\egroup}

\newcommand\Sref[1]{\S\ref{#1}}
\newcommand\figref[1]{Figure~\ref{#1}}
\newcommand\tabref[1]{Table~\ref{#1}}
\newcommand\exref[1]{Example~\ref{#1}}

\newcommand\eat[1]{}





\begin{document}


\maketitle


\lstset{language=java,basicstyle=\ttfamily\small}

\section{Introduction}
This technical report formalizes the hardhat initialization rules in X10
    using \emph{Featherweight X10} (FX10).
Read first the paper ``Object Initialization in X10" to understand the motivation behind the hardhat rules.
    %the terminology (e.g., raw and cooked objects),
%    the initialization rules of X10,
%    and their connection with X10's concurrent and distributed constructs (\finish, \async, and \code{at}).

\begin{proof}
Proof of the main theorem in the paper.
We will break the proof into several lemmas, each proving points (i)--(iv).
\end{proof}


\begin{Lemma}[closed]
  \textbf{(Closed is preserved)}
    For every closed expression~$\grave{\he} \neq \hl$, and a well-typed heap~$H$,
        if $H,\grave{\he} \reducesto H',\acute{\he}$,
        then $\acute{\he}$ is closed.
\end{Lemma}
\begin{proof}
Let's consider all the possible reduction rules in \Ref{Figure}{reduction}.
We will prove by induction on the size of $\he$.

Consider \RULE{R-Async}, where~$\grave{\he} = \async~\hl;\acute{\he}$:
\[\typerule{
}{
  \async~\hl;\acute{\he},H \reduce \acute{\he},H
}
\]
Then, because $\grave{\he}$ is closed, then $\acute{\he}$ is closed.

Consider \RULE{RC-Receiver}, where~$\grave{\he} = \he.\hm(\ol{\he})$:
\[\typerule{
    \he,H \reduce \he',H'
}{
  \he.\hm(\ol{\he}),H \reduce \he'.\hm(\ol{\he}),H'
}
\]
By induction (the size of~$\he$ is smaller than the size of~$\grave{\he}$), we have that~$\he'$ is closed.
Therefore,~$\acute{\he} = \he'.\hm(\ol{\he})$ is closed.

The only remaining two interesting cases are~\RULE{R-New} and~\RULE{R-Invoke}.
Consider~\RULE{R-Invoke} (\RULE{R-New} is handled similarly), where~$\grave{\he} = \hl'.\hm(\ol{\hl})$:
\[
\typerule{
    H(\hl')=\hC(\ldots)
        \gap
    \mbody{}(\hm,\hC)=\ol{\hx}.\he
}{
  \hl'.\hm(\ol{\hl}),H \reduce [\ol{\hl}/\ol{\hx},\hl'/\this]\he,H
}
\]
Note that \he might not be closed, i.e., it may contain method parameters or \this.
However, because the program is well-typed and~$\mbody{}(\hm,\hC)=\ol{\hx}.\he$,
    then \he may only contain either~$\hx_i$ or \this,
    and these are substituted by locations in~$[\ol{\hl}/\ol{\hx},\hl'/\this]\he$,
    thus~$\grave{\he}$ is closed.
\end{proof}

\begin{Lemma}[well-typed]
  \textbf{(Heap is well-typed)}
    For every closed expression~$\grave{\he} \neq \hl$, and a well-typed heap~$H$,
        if $H,\grave{\he} \reducesto H',\acute{\he}$,
        then $H'$ is well-typed.
\end{Lemma}
\begin{proof}
Let's consider all the possible reduction rules in \Ref{Figure}{reduction}.
We will prove by induction on the size of $\he$.

For the following rules we have that~$H=H'$, thus $H'$ is well-typed:
\RULE{R-Finish}
\RULE{R-Async}
\RULE{R-Field-Access}


\RULE{R-Field-Assign}

\end{proof}

%
\Section[val]{Final fields}
\hval and \hvar fields.

\subsection{Syntax}


\begin{figure}[htpb!]
\begin{center}
\begin{tabular}{|l|l|}
\hline

$\hF ::= \hFM ~ \hf:\hC$
& Field declaration. \\

$\hFM ::= \hval~|~\hvar$
& Field Modifier. \\

\hline
\end{tabular}
\end{center}
\caption{FX10 Syntax changes to support final fields (\hval).
    The new terminals are \hval and \hvar.
    }
\label{Figure:syntax-val}
\end{figure}


\subsection{Typing}
Fields also have a modifier (\hFM) so we define:
    $\fmodifier(\hf,\hC) = \hFM$ returns the field modifier of \hf in class \hC (either \hval or \hvar).

A method cannot assign to any \hval fields (no matter what the target is):
\beqs{methodVal}
    &\methodVal([\he_0,\ldots,\he_n])= \methodVal(\he_0) \hand \ldots \hand  \methodVal(\he_n)\\
    &\methodVal(\he")=
        \begin{cases}
        \htrue & \he" \equiv \hl \\
        \htrue & \he" \equiv \hx \\
        \methodVal(\he) & \he" \equiv \he.\hf \\
        \methodVal([\he,\he']) \hand \hFM=\hvar & \he" \equiv \he.\hf = \he' \gap \Gdash \he:\hC \gap \fmodifier{}(\hf,\hC)=\hFM \\
        \methodVal([\he',\ol{\he}]) & \he" \equiv \he'.\hm(\ol{\he}) \\
        \methodVal([\ol{\he}]) & \he" \equiv \hnew ~ \hC(\ol{\he}) \\
        \methodVal(\he) & \he" \equiv \hfinish~\he \\
        \methodVal([\he,\he']) & \he" \equiv \hasync~\he;\he' \\
        \end{cases}
\eeq

A constructor can assign to a \hval field at most once and only if the target is \this:
\beqs{ctorVal}
    &\ctorVal([\he_0,\ldots,\he_n],F)= \ctorVal(\he_0,F) \hand \ctorVal(\he_1,\AW([\he_0]) \cup F) \hand \ldots \hand  \ctorVal(\he_n,\AW([\he_0,\ldots,\he_{n-1}]) \cup F)\\
    &\ctorVal(\he",F)=
        \begin{cases}
        \htrue & \he" \equiv \hl \\
        \htrue & \he" \equiv \hx \\
        \ctorVal(\he,F) & \he" \equiv \he.\hf \\
        \ctorVal([\he,\he'],F \cup \{ \hf \}) \hand \\(\hFM=\hvar \hor (\he=\this \hand \hf \not \in F)) & \he" \equiv \he.\hf = \he' \gap \Gdash \he:\hC \gap \fmodifier{}(\hf,\hC)=\hFM \\
        \ctorVal([\he',\ol{\he}],F) & \he" \equiv \he'.\hm(\ol{\he}) \\
        \ctorVal([\ol{\he}],F) & \he" \equiv \hnew ~ \hC(\ol{\he}) \\
        \ctorVal(\he,F) & \he" \equiv \hfinish~\he \\
        \ctorVal([\he,\he'],F) & \he" \equiv \hasync~\he;\he' \\
        \end{cases}
\eeq

We need to check that we do not assign to a \hval field ($\methodVal(\he)$).


\beqst %{method-ok}
\typerule{
  \methodVal(\he)
}{
  \hMM ~ \hm(\ol{\hx}:\ol{\hD}):\hD=\he ~~\OK~\IN~\hC
}~\RULE{(T-Method-val)}
\eeq

Similarly, we need to check for constructors that \hval fields are treated correctly (assigned at most once).

\beqst %{ctor-ok}
\typerule{
  \hclass~\hC~\hextends~\hC'~\lb~\ldots~\rb
    \gap
  F_s = \fields{}(\hC')
    \gap
  \ctorVal([\ol{\he},\he'],F_s)
}{
  \ctor(\ol{\hx}:\ol{\hD}) \lb~\super(\ol{\he});\he'; \rb ~~\OK~\IN~\hC
}\gap \RULE{(T-Ctor)}
\eeq



Moreover, $\reductionVal$ is preserved during the reduction,
    i.e., a \hval is assigned at most once.

H, \he closed
$\reductionVal(\he,H)$  (we have everything in H - which val fields are already assigned)


\end{document}
