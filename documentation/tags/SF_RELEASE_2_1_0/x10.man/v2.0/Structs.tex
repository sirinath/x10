\chapter{Structs}
\label{XtenStructs}
\label{StructClasses}
\label{Structs}
\index{structs}

X10 objects are a powerful general-purpose programming tool. However, the
power must be paid for in space and time. In space, a typical object
implementation requires some extra memory for run-time class information, as
well as a pointer for each reference to the object.  In time, a typical object
requires an extra indirection to read or write data, and some computation to
figure out which method body to call.  

For high-performance computing, this overhead may not be acceptable for all
objects. X10 provides structs, which are stripped-down objects. They are less
powerful than objects; in particular they lack inheritance and mutable fields.
Without inheritance, method calls do not need to do any lookup; they can be
implemented directly. Accordingly, structs can be implemented and used more
cheaply than objects, potentially avoiding the space and time overhead.
(Currently, the C++ back end avoids the overhead, but the Java back end
implements structs as Java objects and does not avoid it.)

Structs and classes are interoperable. Both can implement interfaces (in
particular, like all X10 values they implement \xcd`Any`), and subprocedures
whose arguments are defined by interfaces can take both structs and classes.
(Some caution is necessary here: referring to a struct through an interface
requires overhead similar to that required for an object.)

They are also interconvertable, within the constraints of structs. If you
start off defining a struct and decide you need a class instead, the code
change required is simply changing the keyword \xcd`struct` to \xcd`class`. If
you have a class that does not use inheritance or mutable fields, it can be
converted to a struct by changing its keyword.



\section{Struct declaration}
A struct declaration has the structure: 
\begin{xtenmath}
$\mbox{\emph{StructModifiers}}^{\mbox{?}}$
struct C[X$_1$, $\ldots$, X$_n$](p$_1$:T$_1$, $\ldots$, p$_n$:T$_n$){c} 
   implements I$_1$, $\ldots$, I$_k$ {
$\mbox{\emph{StructBody}}$
}
\end{xtenmath}

Each field and method in a struct is implicitly \Xcd{global}, and each field
is \xcd`val`.

A struct \Xcd{S} cannot contain a field of type \Xcd{S}, or a field of struct
type \Xcd{T} which, recursively, contains a field of type \Xcd{S}.  This
restriction is necessary to permit \xcd`S` to be implemented as a contiguous
block of memory of size equal to the sum of the sizes of its fields.  


Values of a struct \Xcd{C} type can be created by invoking a constructor
defined in \Xcd{C}, but without prefixing it with \Xcd{new}: 
%~~gen
% package Structs.For.Muckts;
%~~vis
\begin{xten}
struct Polar(r:Double, theta:Double){
  def this(r:Double, theta:Double) {property(r,theta);}
  static val Origin = Polar(0,0);
  static val x0y1 = Polar(1, 3.14159/2);
}
\end{xten}
%~~siv
%
%~~neg

Structs support the same notions of generics, properties, and constrained
types that classes do.  For example, the \xcd`Pair` type below provides pairs
of values; the \xcd`diag()` method applies only when the two elements of the
pair are equal, and returns that common value: 
%~~gen
% package Structs.For.Muckts;
%~~vis
\begin{xten}
struct Pair[T,U](t:T, u:U) {
  def this(t:T, u:U) { property(t,u); }
  def diag(){T==U && t==u} = t;
}
\end{xten}
%~~siv
%
%~~neg


\section{Boxing of structs}
\index{auto-boxing!struct to interface}
\label{auto-boxing} 
If a struct \Xcd{S} implements an interface \Xcd{I} (\eg, \Xcd{Any}),
a value \Xcd{v} of type \Xcd{S} can be assigned to a variable of type
\Xcd{I}. The implementation creates an object \Xcd{o} that is an
instance of an anonymous class implementing \Xcd{I} and containing
\Xcd{v}.  The result of invoking a method of \Xcd{I} on \Xcd{o} is the
same as invoking it on \Xcd{v}. This operation is termed {\em auto-boxing}.
It allows full interoperability of structs and objects---at the cost of losing
the extra efficiency of the structs when they are boxed.

In a generic class or struct obtained by instantiating a type parameter
\Xcd{T} with a struct \Xcd{S}, variables declared at type \Xcd{T} in the body
of the class are not boxed. They are implemented as if they were declared at
type \Xcd{S}.

\section{Optional Implementation of \Xcd{Any} methods}
\label{StructAnyMethods}
\index{Any!structs}

Unlike objects, structs do not have global identity. Instead, two
structs are equal (\Xcd{==}) if and only if their corresponding fields
are equal (\Xcd{==}). 

All structs implement \Xcd{x10.lang.Any}. All 
structs have the following methods implicitly defined on them:
\begin{xten}
  property def home()=here;
  property def at(Place)=true;
  property def at(Object)=true;
\end{xten}
\noindent It is an error for a programmer to attempt to define them.
Note that the \xcd`home` of a struct evaluated in place \xcd`P` is equal to
\xcd`P`---a struct, unlike an object, can have different values of \xcd`home`. 

Structs are required to implement the following methods from \xcd`Any`.  
Programmers need not provide them; X10 will produce them automatically if 
the program does not include them. 
\begin{xten}
  public global safe def equals(Any):Boolean;
  public global safe def hashCode():Int;
  public global safe def typeName():String;
  public global safe def toString():String;  
\end{xten}


A programmer who provides an explicit implementation
of \Xcd{equals(Any)} for a struct \Xcd{S} should also consider
supplying a definition for \Xcd{equals(S):Boolean}. This will often
yield better performance since the cost of an upcast to \Xcd{Any} and
then a downcast to \Xcd{S} can be avoided.

\section{Primitive Types}

Certain types that might be built in to other languages are in fact
implemented as structs in package \xcd`x10.lang` in X10. Their methods and
operations are often provided with \xcd`@Native` (\Sref{NativeCode}) rather
than X10 code, however. These types are:
\begin{xten}
boolean, char, byte, short, int, long
float, double, ubyte, ushort, uint, ulong
\end{xten}
 
\section{Generic programming with structs}

An unconstrained type variable \Xcd{X} can be instantiated with \Xcd{Object} or
its subclasses or structs or functions.

Within a generic struct, all the operations of \Xcd{Any} are available
on a variable of type \Xcd{X}. Additionally, variables of type \Xcd{X} may
be used with \Xcd{==, !=}, in \Xcd{instanceof}, and casts.

\bard{The rest of this section is under discussion.  The example is wrong; it
ignores the fact that values can be functions.}
The programmer must be aware of the different interpretations of
equality for structs and classes and ensure that the code is correctly
written for both cases. If necessary the programmer can write code
that distinguishes between the two cases (a type parameter \Xcd{X} is
instantiated to a struct or not) as follows:

\begin{xten}
val x:X = ...;
if (x instanceof Object) { // x is a real object
   val x2 = x as Object; // this cast will always succeed.
   ...
} else { // x is a struct
   ...
}
\end{xten}
 
  
\section{Example structs}

\xcd`x10.lang.Complex` provides a detailed example of a practical struct,
suitable for use in a library.  For a shorter example, we define the
\xcd`Pair` struct---available in \xcd`x10.util.Pair`.  A \xcd`Pair` packages
two values of possibly unrelated type together in a single value, \eg, to
return two values from a function.

%~~gen
% package Structs.Pairs.Are.For.Squares;
%~~vis
\begin{xten}
struct Pair[T,U] {
    public val first:T;
    public val second:U;
    public def this(first:T, second:U):Pair[T,U] {
        this.first = first;
        this.second = second;
    }
    public global safe def toString():String {
        return "(" + first + ", " + second + ")";
    }
}
\end{xten}
%~~siv
%
%~~neg

