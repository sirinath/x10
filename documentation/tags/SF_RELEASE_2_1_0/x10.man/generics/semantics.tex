\newcommand\EX[3]{{\tt(}{#2}\;{\tt :}\;{#1}{\tt)}\;{#3}}
\newcommand\EXT[2]{{\tt[}{#1}{\tt]}\;{#2}}
%\newcommand\EX[3]{{\exists}\,{#2}\;{\tt :}\;{#1}{\tt .}~{#3}}
%\newcommand\EXT[2]{{\exists}\,{#1}{\tt .}~{#2}}
\newcommand\TY[2]{{#2}\;{\tt :}\;{#1}}
\newcommand\CAST[2]{{#2}~{\tt as}~{#1}}


\def\TConstr{\mbox{\sc T-Constr}}
\def\TInv{\mbox{\sc T-Inv}}
\def\TVar{\mbox{\sc T-Var}}
\def\TField{\mbox{\sc T-Field}}
\def\TInvk{\mbox{\sc T-Invk}}
\def\TNew{\mbox{\sc T-New}}
\def\TCast{\mbox{\sc T-Cast}}
\def\TUCast{\mbox{\sc T-UCast}}
\def\TDCast{\mbox{\sc T-DCast}}
\def\TSCast{\mbox{\sc T-SCast}}

\def\SRefl{\mbox{\sc S-Refl}}
\def\STrans{\mbox{\sc S-Trans}}
\def\SExtends{\mbox{\sc S-Extends}}
\def\SDep{\mbox{\sc S-Dep}}
\def\SC{\mbox{\sc S-Constraint}}

\def\RField{\mbox{\sc R-Field}}
\def\RCField{\mbox{\sc RC-Field}}
\def\RInvk{\mbox{\sc R-Invk}}
\def\RCInvkRecv{\mbox{\sc RC-Invk-Recv}}
\def\RCInvkArg{\mbox{\sc RC-Invk-Arg}}
\def\RCNewArg{\mbox{\sc RC-New-Arg}}
\def\RCast{\mbox{\sc R-Cast}}
\def\RCCast{\mbox{\sc RC-Cast}}

\newcommand\Xten{X10}
\newcommand\FJ{FJ}

To illustrate
the concepts behind constrained type-checking,
we formalize the type system in a small calculus CFGJ based on
Featherweight Java~\cite{FJ}.  The calculus is an extension of
Constrained Featherweight Java~\cite{constrained-types}.

\emph{This section has not been checked carefully and probably
contains many errors of omission and comission.}

The language is functional in that assignment is not
admitted. However, it is not difficult to introduce the notion of
mutable fields, and assignment to such fields. Since constrained types
may only refer to immutable state, the validity of these types is not
compromised by the introduction of state.
%
Further, we do not formalize overloading of methods. Rather, as
with \FJ{}, we simply require that the input program be such that the class
name {\tt C} and method name {\tt m} uniquely select the associated
method on the class. 
%
We do model properties, constrained clauses, class invariants, where
clauses in methods and constructors, and dependent type casts.

\subsection{Constraint system}

Constraints are assumed to be drawn from a fixed constraint system,
$\cal C$, with inference relation $\vdash_{\cal C}$ \cite{CCCC}.
All
constraint systems are required to support the trivial constraint
\true, conjunction, existential quantification and equality on
constraint terms. Constraint terms include {\tt final} variables
(including \this), the
special variable \self\ (which may occur only in constraints {\tt
c} and not in expressions {\tt e}), field selections {\tt t.f},
type variables, and type selections {\tt t.X}.

The syntax for the language is specified in Figure~\ref{CFGJ-syntax}.
In the syntax, ``,'' binds tighter than ``;''. We use the syntax
\EX{\tt T}{\tt x}{\tt c}
for the constraint obtained by existentially quantifying the
variable {\tt x} of type {\tt T} in {\tt c},
and the syntax
\EXT{\tt X}{\tt c}
for the constraint obtained by existentially quantifying the
type variable {\tt X} in {\tt c}.
{\tt p} ranges over
the collection of predicates supplied by the underlying constraint
system, and {\tt g} over the collection of functions.

A {\em class declaration}
$\tt\class \ C(\TY{\bar{T}}{\bar{f}})\ where\ c\ \extends\ D\{d\}\ \{\bar{M}\}$
declares a class {\tt C} with the
fields $\bar{\tt f}$ (of type $\bar{\tt T}$), a {\em declared class
invariant} {\tt c}, a {\em superclass invariant} {\tt d} and a
collection of methods $\bar{\tt M}$. The constraints {\tt c} and {\tt
d} are true for all instances of the class {\tt C} (this is verified
in the rule for type-checking constructors, \rn{T-New}).  In these
constraints, \this\ may be used to reference the current object;
\self{} does not have any meaning and must not be used.

A {\em method declaration}
${\tt def~m}[\bar{\tt X}](\TY{\bar{\tt T}}{\bar{\tt x}})\;:\;{\tt T}_0~{\tt where~c}~\{\ldots\}$
specifies the type of the arguments and the result, as
usual.  The formal type parameters $\bar{\tt X}$ and formal
value parameters $\bar{\tt x}$ may occur in the argument
types $\bar{\tt T}$ and the return type ${\tt T}_0$.  The constraint
{\tt c} specifies additional constraints on the parameters and
on
\this{} that must hold for a method invocation to be legal. Note that
\self{} does not make sense in {\tt c} (since no type is being
defined), and must not occur in {\tt c}.

A type is taken to be of the form {\tt C\{c\}} where {\tt C} is the
name of a class or interface and {\tt c} is a constraint; we say that
{\tt C} is the {\em base} of the type {\tt C\{c\}}.
We use the following shorthand for types: 
we write
{\tt \EX{S}{x}{C\{c\}}} for
{\tt C\{\EX{S}{x}{c}\}},
{\tt \EXT{X}{C\{c\}}} for
{\tt C\{\EXT{X}{c}\}},
and {\tt d,C\{c\}} for {\tt C\{d,c\}}.

We denote the application of the
substitution $\theta=[\bar{{\tt t}}/\bar{{\tt x}}]$ to a
constraint ${\tt c}$ by ${\tt c}[\bar{{\tt t}}/\bar{{\tt x}}]$. 
Application of substitutions is extended to
types by: ${\tt C\{c\}\theta}={\tt C\{c\theta\}}$.

We summarize here properties of constraint systems described in
\cite{CCCC} that are needed for the proofs: constraint systems may be
thought of as presented via an intuitionistic Gentzen proof system
supporting identity; affine and exchange on the left; existential
quantification and conjunction on the left and right; and closure
under substitution of terms.
%
All constraint systems are required to satisfy:
$\new\ {\tt C[\bar{\tt T}](\bar{\tt t})}.{\tt f}_i={\tt t}_i$
and
$\new\ {\tt C[\bar{\tt T}](\bar{\tt t})}.{\tt X}_i={\tt T}_i$
provided that
$\fields({\tt C})=\TY{\bar{\tt S}}{\bar{\tt f}}$ (for some sequence
of types $\bar{\tt S}$)
and
$\types({\tt C})=\bar{\tt X}$.

\begin{figure}[t]

\begin{center}
\footnotesize

\begin{tabular}{l@{\quad}rcl}
Constraint term & {\tt t} &::=&
      \self 
 \alt {\tt x}
 \alt {\tt t.f}
 \alt {\tt X}
 \alt {\tt t.X}
 \alt $\new~{\tt C}[\bar{\tt T}](\bar{\tt t})$
 \alt ${\tt g}(\bar{\tt t})$ \\
Constraint & {\tt c}, {\tt d} &{::=}&
      \true
 \alt ${\tt p}[\bar{\tt T}](\bar{\tt t})$
 \alt {\tt t} = {\tt t}
 \alt {\tt c,\,c}
 \alt $\EX{\tt T}{\tt x}{\tt c}$
 \alt $\EXT{\tt X}{\tt c}$ \\
Class & {\tt L} &{::=}&
     $\tt\class~C[\bar{X}](\TY{\bar{T}}{\bar{x}})~{\tt where}~c~\extends~T~\{\bar{M}\}$ \\
Method & {\tt M} &{::=}&
     $\tt def~m[\bar{X}](\TY{\bar{T}}{\bar{x}}):~T~{\tt where}~c = e$ \\
Expression & {\tt e} &{::=}&
     $\tt
     x
 \alt e.f
 \alt e.m[\bar{T}](\bar{e})
 \alt \new\ C\bar[\bar{T}](\bar{e})
 \alt \CAST{T}{e}$ \\
Type & {\tt S}, {\tt T}, {\tt U} &{::=}&
    $\tt C\{d\}$
 \alt $\tt X$
 \alt $\tt t.X$ \\
\end{tabular}
\end{center}

\caption{CFGJ Syntax}
\label{CFGJ-syntax}
\end{figure}

\subsection{Static semantics}

Typing judgments, shown in Figure~\ref{CFGJ-typing}
are of the form $\Gamma \vdash \TY{\tt T}{\tt e}$
where $\Gamma$ is a multiset of type assertions $\TY{\tt T}{\tt x}$,
kind assertions ${\tt X}$,
and constraints ${\tt c}$.
When $\Gamma$ is empty, it is
omitted. 

A type assertion \TY{\tt C\{c\}}{\tt x} constrains the variable {\tt x} to
contain references to only those objects {\tt o} that are instances of
(subclasses of) {\tt C} and for which the constraint {\tt c} is true
provided that occurrences of \self\ in {\tt c} are replaced by
{\tt o}. Thus in the constraint {\tt c} of a constrained type
{\tt C\{c\}}, \self\ may be used to reference the object whose type is
being specified. Note that \self\ is distinct from
\this: \this\ is permitted to occur in the clause of
a type {\tt T} only
if {\tt T} occurs in an instance field declaration or instance method
declaration of a class; as usual, \this{} is considered bound to the
instance of the class to which the field or method declaration
applies.

\renewcommand\andalso\quad

\begin{figure*}
\quad

\typicallabel{\TField}

\infrule[\TVar]{
\sigma(\Gamma, \TY{\tt C\{c\}}{\tt x}) \vdash_{\cal C} {\tt d}[{\tt x}/\self]
}{
\Gamma, \TY{\tt C\{c\}}{\tt x} \vdash \TY{\tt C\{d\}}{\tt x}
}

\infrule[\TCast]{
\Gamma \vdash \TY{\tt U}{\tt e}
}{
\Gamma \vdash \TY{\tt T}{\tt \CAST{T}{e}}
}

\infrule[\TField]{
\Gamma \vdash \TY{{\tt T}_0}{\tt e}
\andalso
\fields({\tt T}_0, {\tt z}_0) = \TY{\bar{\tt U}}{\bar{\tt f}}
\andalso
\mbox{(${\tt z}_0$ fresh)}
}{
\Gamma \vdash \TY{(\EX{{\tt T}_0}{{\tt z}_0}
                   {{\tt z}_0 = \this, {\tt z}_0.{\tt f}_i=\self})~{\tt U}_i}
                 {{\tt e.f}_i}
}

\infrule[\TInvk]{
\Gamma \vdash \TY{{\tt T}_0}{{\tt e}_0}
\andalso
\Gamma \vdash \TY{\bar{\tt T}}{\bar{\tt e}}
\andalso
\Gamma \vdash {\tt T}_0 \subtype {\tt C}
\\
\mtype({\tt C},{\tt m},{\tt x}_0) =
  [\bar{\tt X}]
  (\TY{\bar{\tt U}}{\bar{\tt x}})
  \rightarrow_{\tt c}
  {\tt U} \\
\Gamma' = \Gamma, \bar{\tt X},
        \bar{\tt X} = \bar{\tt S},
        \TY{{\tt T}_0}{{\tt x}_0},
        {\tt x}_0 = \this,
        \TY{\bar{\tt T}}{\bar{\tt x}}
\\
\Gamma' \vdash
  \bar{\tt T} \subtype \bar{\tt U}
\andalso
\sigma(\Gamma')
  \vdash_{\cal C} {\tt c}
\andalso
\mbox {(${\tt x}_0$, $\bar{\tt x}$, $\bar{\tt X}$ fresh)}
}{
\Gamma \vdash \TY{\EXT{\bar{\tt X}}{\EX{{\tt T}_0, \bar{\tt
T}}{{\tt x}_0, \bar{\tt x}}{\tt U}}}
                 {{\tt e}_0.{\tt m[\bar{\tt S}](\bar{\tt e})}}
}

\infrule[\TNew]{
\Gamma \vdash \TY{\bar{\tt T}}{\bar{\tt e}}
\\
\theta = [{\tt x}_0,\bar{\tt x}/\this,\this.\bar{\tt f}]
\andalso
\fields({\tt C},\theta) = \TY{\bar{\tt U}}{\bar{\tt f}}
\\
\Gamma, \bar{\tt X},
        \bar{\tt X} = \bar{\tt S},
        \TY{\tt C}{{\tt x}_0},
        \TY{\bar{\tt T}}{\bar{\tt x}}
  \vdash
        \bar{\tt T} \subtype \bar{\tt U}
\\
\sigma(\Gamma,
        \TY{\tt C}{{\tt x}_0},
        \TY{\bar{\tt T}}{\bar{\tt x}}
        )
        \vdash_{\cal C} \inv({\tt C},\theta)
}{
\Gamma \vdash \TY{\tt C\{\EX{\bar{T}}{\bar{\tt x}}{{\tt\self.\bar{f}}=\bar{\tt x}}\}}
{
\new\ {\tt C[\bar{\tt S}](\bar{\tt e})}}
}

\infrule{
\Gamma =
  \bar{\tt X},
  \TY{\bar{\tt T}}{\bar{\tt x}},
  \TY{\tt C}{\this},
  {\tt c}
\\
\Gamma
  \vdash \TY{\tt U}{\tt e}
\andalso
\Gamma
  \vdash {\tt U} \subtype {\tt T}
}{
{\tt def}~{\tt m}[\bar{\tt X}](\TY{\bar{\tt T}}{\bar{\tt x}}) : {\tt U}~{\tt where}~{\tt c} = {\tt e}~\mbox{OK in}\ {\tt C}}

\infrule{
\bar{\tt M}\ \mbox{OK in}\ {\tt C}
}{
\class\ {\tt C}[\bar{\tt X}](\TY{\bar{\tt T}}{\bar{\tt f}})\
{\tt where}~{\tt c}~\extends\ {\tt D}\{{\tt d}\}\ \{\bar{\tt M}\}\ \mbox{OK}
}

\infax[\SRefl]{
{\tt C} \subtype {\tt C}
}

\infrule[\SExtends]{
\class\ {\tt C}[\bar{\tt X}](\TY{\bar{\tt T}}{\bar{\tt f}})\
{\tt where}~{\tt c}~\extends\ {\tt D}\{{\tt d}\}\ \{\bar{\tt M}\}
}
{{\tt C} \subtype {\tt D}}

\infrule[\STrans]
{{\tt C} \subtype {\tt D} \ \ \ {\tt D} \subtype {\tt E}}
{{\tt C} \subtype {\tt E}}

\infrule[\SDep]{
{\tt C} \subtype {\tt D} \andalso
\sigma(\Gamma, \TY{\tt C\{c\}}{\tt x}) \vdash_{\cal C} {\tt
d}[{\tt x}/\self] \andalso \mbox{({\tt x} fresh)}
}
{\Gamma \vdash {\tt C\{c\} \subtype D\{d\}}}

\infrule[\SC]{
\Gamma \vdash_{\cal C} {\tt T} \subtype {\tt U}
}{
\Gamma \vdash {\tt T} \subtype {\tt U}
}

\caption{CFGJ semantics}
\label{CFGJ-typing}
\end{figure*}

\begin{figure*}

\quad

\typicallabel{RC-Field}

\infrule[\RField]{
\fields({\tt C})= \TY{\bar{\tt U}}{\bar{\tt f}}
}{
(\new\ {\tt C}[\bar{\tt T}](\bar{\tt e})).{\tt f}_i \derives {\tt e}_i
}

\infrule[\RCast]{
\vdash {\tt C} \subtype {\tt T}[\new\ {\tt C}[\bar{\tt T}](\bar{\tt e})/\self]
}{
{\CAST{\tt T}{\new\ {\tt C}[\bar{\tt T}](\bar{\tt e})} \derives
\new\ {\tt C}[\bar{\tt T}](\bar{\tt e})}
}

\typicallabel{R-Invk-Recv}

\infrule[\RInvk]{
\mathit{mbody}({\tt m},{\tt C})=\bar{x}. {\tt e}_0
}{
(\new\ {\tt C}[\bar{\tt T}](\bar{\tt e})).{\tt m}(\bar{\tt d}) \derives 
{\tt e}_0 [\bar{\tt d}/\bar{\tt x},\new\ {\tt C}[\bar{\tt T}](\bar{\tt e})/\this]
}

\infrule[\RCField]{
{\tt e} \derives {{\tt e}}'
}{
{\tt e}.{\tt f}_i \derives {{\tt e}}'.{\tt f}_i
}

\infrule[\RCCast]{
{\tt e} \derives {{\tt e}}'
}
{
\CAST{\tt T}{\tt e} \derives \CAST{\tt T}{\tt e'}
}

\infrule[\RCInvkRecv]{
{\tt e}_0 \derives {{\tt e}_0}'
}{
{\tt e}_0.{\tt m}(\bar{\tt e}) \derives {{\tt e}_0}'.{\tt m}(\bar{\tt e})
}

\infrule[\RCInvkArg]{
{\tt e}_i \derives {{\tt e}_i}'
}{
{\tt e}_0.{\tt m}(\ldots,{\tt e}_i,\ldots) \derives {{\tt e}_0}.{\tt m}(\ldots,{\tt e}_i',\ldots)
} 

\infrule[\RCNewArg]{
{\tt e}_i \derives {{\tt e}_i}'
}{
\new\ {\tt C}[\bar{\tt T}](\ldots,{\tt e}_i,\ldots) \derives
\new\ {\tt C}[\bar{\tt T}](\ldots,{\tt e}_i',\ldots)
}

\caption{CFGJ semantics}
\label{CFGJ-red-rules}
\end{figure*}


The definition of {\mtype({\tt C},{\tt m})} (the signature of a method
named {\tt m} in class {\tt C}), {\mbody({\tt C},{\tt m})}, (the body
associated with method {\tt m} in type {\tt C}) and \fields(C) (the
sequence of fields and their types inherited or defined at {\tt C}) is
essentially as specified in \FJ{}~\cite{FJ} with the difference that the
method of a signature is taken to be of the form
$[\bar{\tt X}](\TY{\bar{\tt S}}{\bar{\tt x}}) \to_{\tt c} {\tt T}$.
The variables {\tt x} and type variables {\tt X} are permitted
to occur in the types $\bar{\tt S}$ and ${\tt T}$, and in the
constraint ${\tt c}$, and are considered bound,
and subject to alpha-renaming.  The definitions of \mtype, \mbody,
\fields{} are extended to apply to constrained types by ignoring the
constraint.  For a substitution $\theta$ we define $\mtype({\tt
T},{\tt m},\theta)$ as the signature obtained by applying $\theta$ to
$\mtype({\tt T},{\tt m})$, renaming bound variables as necessary.
Similarly, for a substitution $\theta$ we define $\fields({\tt
T},\theta)$ to be $\TY{\bar{\tt S}\theta}{\bar{\tt f}}$, if the sequence of
inherited and defined fields of the class underlying the type {\tt T}
is $\TY{\bar{\tt S}}{\bar{\tt f}}$. We let $\fields({\tt T},{\tt x})$ stand for
$\fields({\tt T},[{\tt x}/\self]))$.

We define $\sigma(\Gamma)$ to be the set of
constraints obtained from $\Gamma$ by replacing each type assertion
$\TY{\tt C\{d\}}{\tt x}$
in $\Gamma$ with ${\tt d}[{\tt x}/\self],\inv({\tt C},{\tt x})$
and retaining any constraint in $\Gamma$.

\TVar{} extends the identity rule ($\Gamma, x:C \vdash x:C$) of
\FJ{} to take into account the constraint entailment relation.

\TCast{} encapsulates the three inference rules of \FJ{}:
\TUCast{}, \TDCast{} and \TSCast{} for upwards cast, downwards cast, and ``stupid'' cast respectively. 

%\TInv{} is a form of contraction that permits the class invariant {\tt c}
%of a class {\tt C} to enrich the type of any variable of type {\tt C}.

\textbf{This paragraph is out-of-date.}
In \TField, we postulate the existence of a receiver object {\tt o} of
the given static type (${\tt T}_0$). $\fields({\tt T}_0,{\tt o})$ is
the set of typed fields for ${\tt T}_0$ with all occurrences of 
\this{} replaced  by {\tt o}. We record in the resulting
constraint that ${\tt o.f}_i=\self$.\footnote{A new name {\tt o} is
necessary to name this object since {\tt e} cannot be used. Arbitrary
term expressions {\tt e} are not permitted in constraints; the
functions used in {\tt e} may not be known to the constraint system,
and {\tt e} may have side-effects.}  This permits transfer of
information that may have been recorded in ${\tt T}_0$ about the field
${\tt f}_i$. 

\textbf{This paragraph is out-of-date.}
Similarly, in \TInvk{} we postulate the existence of a receiver object
{\tt o} of the given static type. For any type $T$, object {\tt o} of
type $T$ and method name {\tt m}, let $\mtype({\tt T},{\tt m},{\tt
o})$ be a copy of the signature of the method with \this{} replaced by
{\tt o}. We establish (under the assumption that the formals
($\bar{\tt z}$) have the static type of the actuals)\footnote{This is
stronger than assuming $\bar{\tt Z}$.}  that actual types are subtypes
of the formal types, and the method constraint is satisfied. This
permits us to record the constraint {\tt d} on the return type, with
the formal variables $\bar{\tt z}$ existentially
quantified.\footnote{Recall that the $\bar{\tt z}$ may occur in {\tt
d} but must not occur in a type in the calling environment; hence they
must be existentially quantified in the resulting constraint.}

\textbf{This paragraph is out-of-date.}
In \TNew, similarly, we establish that the static types of the actual
arguments to the constructor are subtypes of the declared types of the
field, and contain enough information to satisfy the class invariant,
{\tt c}. The declared types (and {\tt c}) contain references to ${\tt
this.\bar{\tt f}}$; these must be replaced by the formals $\bar{\tt
f}$, which carry information about the static type of the
actuals. Note that the object {\tt o} we hypothesized in an analogous
situation in \TInvk{} does not exist; it will exist on successful
invocation of the constructor. The constrained clause of the \new{}
expression contains all the information that can be gleaned from the
static types of the actuals by assigning them to the corresponding
fields of the object being created.

Let {\tt C} be a class declared as
$${\tt \class\ C(\TY{\bar{\tt T}}{\bar{\tt f}})\ {\tt where}~c~extends\
D\{d\}\{\bar{\tt M}\}}.$$
Let
$\theta$ be a substitution and the type {\tt T} be based on {\tt C}.
We define $\inv(T,\theta)$
as the conjunction ${\tt c\theta,d\theta}$ and (recursively)
$\inv({\tt D},\theta)$.  We bottom out with $\inv({\tt
Object},\theta)=\true$. For a variable {\tt x}, we use the shorthand
$\inv({\tt C},{\tt x})$ to mean $\inv({\tt C},[{\tt x}/\self])$.

We add two subtyping rule to the rules of \FJ{}.
The first, \SDep, requires
that the subtype constraint entail the supertype constraint.
Whenever we state an assumption of the form ``{\tt x} is
fresh'' in a rule we mean it is not free in the consequent of the
rule.
The second rule, \SC, states that one type is a subtype of another if
the constraint system says so.

The operational semantics are essentially those of \FJ{}.

