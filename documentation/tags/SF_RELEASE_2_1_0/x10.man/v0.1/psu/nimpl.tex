% -*-LaTeX-*-

\chapter{Implementation of Numeric Datatypes}

(This appendix is not a part of P1178/D4, Draft Standard for the
Scheme Programming Language, but is included for information only.)

This appendix discusses several aspects of Scheme's numeric datatypes
that are of interest to implementors.  We assume that the reader is
familiar with the implementation of generic arithmetic in Lisp (or
other languages like Lisp in this respect), and focus on those
features that are both new and unusual (a good discussion of the
basics of generic arithmetic can be found in section 2.4
of~\cite{SICP}).  In this manner we hope to shed some light on common
problems that Scheme implementors may encounter.

\section{Exactness}

Exactness is the one fundamentally new idea that is introduced by
Scheme's arithmetic, and consequently it is the most important topic
for implementors to consider.

Exactness serves two basic purposes.  First, it can be used as an
``alarm'' to inform the programmer when the Scheme system has caused
precision to be lost; naively, if the result of a computation is
exact, the programmer can trust that it is correct.  Conversely, if
the result is inexact, the programmer is warned that the answer is
only approximate and that numerical analysis is required to determine
the error involved.

The second purpose of exactness is abstraction, isolating programmers
from the methods used to implement inexact arithmetic.  While it seems
likely that floating-point arithmetic will be the most common
representation, this abstraction permits implementors to experiment
with other numerical methods and to take advantage of unusual
hardware.  Provided that the implementor does a good job, many
programmers won't need to know what methods are being used (the
exception being programmers who need to guarantee tight bounds on the
error of their computations).

The property of exactness that serves these two purposes is a simple
one: an output of a computation is inexact whenever any of its inputs
is inexact.  This general rule holds in nearly all cases
(see~\ref{exactly}), but some of its consequences may be surprising;
we discuss three such consequences.

First, the ``rounding operations'' (\ide{floor}, \ide{ceiling},
\ide{truncate}, and \ide{round}) return inexact results when given
inexact arguments.  Many programmers familiar with other Lisp
implementations will expect these operations to return exact results,
and may be confused when their programs don't work correctly.
Implementors should be careful not to make the same mistake.  As a
concrete example, Common Lisp supports two datatypes, floating-point
numbers and integers, that respectively correspond to Scheme's inexact
reals and exact integers.  Unlike Scheme, Common Lisp provides two
sets of procedures for rounding (see~\cite{CLtL}, pp. 215-218).  The
procedures {\cf floor}, {\cf ceiling}, {\cf truncate}, and {\cf round}
convert floating-point numbers to integers, while the procedures {\cf
ffloor}, {\cf fceiling}, {\cf ftruncate}, and {\cf fround} convert
floating-point to floating-point.  Despite their names, it is the
second set of procedures that corresponds to Scheme's rounding
operations.

The second surprising consequence is that \ide{max} and \ide{min},
when given mixed exact and inexact arguments, return an inexact
result.  This is true even in the following case, which demonstrates
the surprising aspect:

\begin{scheme}
(max 3.9 4)            \ev  4.0%
\end{scheme}

Implementors must be careful to notice this, as the ``obvious'' way to
implement these procedures is to compare the arguments and return the
largest (smallest) one.  In Scheme, this ``obvious'' implementation is
incorrect; the procedures must notice when one or more of the
arguments is inexact and in that case guarantee that the result is
inexact.

\begin{note}
A problem arises in this situation when an exact argument is to be
coerced to an inexact result.  If there is no inexact number equal to
the argument, the implementation must choose one that is close to the
argument.  The implemention can choose the inexact number that is
closest to the argument, or it can choose the closest inexact number
that exceeds the argument in the appropriate direction.
\end{note}

Finally, and perhaps most surprising for implementors, is that
operations defined on integers (\ide{quotient}, \ide{remainder},
\ide{modulo}, \ide{gcd}, \ide{lcm}) and rationals (\ide{numerator},
\ide{denominator}) must accept inexact arguments (producing, of
course, inexact results).  This may not matter much to programmers,
but is notable for implementors.  As an example, if floating-point is
used to represent all inexact real numbers, this means that any
floating-point number satisfying the predicate \ide{integer?} can be
used as an argument to \ide{gcd}; furthermore, \ide{numerator} and
\ide{denominator} must accept all floating-point numbers as arguments.
Observe also, complex numbers with inexact zero imaginary part must
satisfy the \ide{real?} predicate, and thus might also satisfy
\ide{integer?} or \ide{rational?}.

Extending the integer and rational operations to cover these cases is
straightforward, especially if exact equivalents exist for every
inexact number; in that case merely map the inexact arguments to exact,
perform the operation, and apply \ide{exact->inexact} to the result
(this is one of several reasons why it is desirable to implement
unlimited-precision exact integers and rationals).  Otherwise, the
implementation is somewhat more involved; small implementations may
find it desirable to refuse such arguments as violations of an
implementation restriction.

\section{Transitivity of Order Predicates}

Unlike most other dialects of Lisp, Scheme requires the numerical
order predicates (\ide{=}, \ide{<}, \ide{<=}, \ide{>}, and \ide{>=})
to be transitive.  In most cases this is easily accomplished by
standard techniques; however, in the case of mixed exact and inexact
arithmetic the standard techniques may fail to satisfy this
requirement.  This problem arises specifically when floating-point is
used to represent inexact numbers, although it may occur with other
representations as well.

A particular technique that does not satisfy Scheme's requirements is
the \defining{floating-point contagion} rule of Common Lisp
(see~\cite{CLtL}, p.~194; interestingly, \cite{CLtL2}, pp.~289-291
explains that the rule was recently modified to be compatible with
Scheme's requirements).  This rule specifies that when operating on a
floating-point number and a rational number, the rational number is
first converted to floating-point and the result of the conversion is
operated on in place of the original rational.  In a Scheme system,
this rule can be interpreted as applying to inexact reals (implemented
as floating-point numbers) and exact rationals.  In the case of
number-valued operators, such as \ide{+}, this contagion rule always
generates a correct answer; not so for the order predicates.

The reason that this rule fails has to do with precision.
Floating-point numbers have a limited amount of precision, while exact
rational numbers have effectively unlimited precision.  Thus, the
conversion of an exact rational number to a floating-point number can
result in a loss of precision.  In mathematical terms, the mapping
from rational numbers to floating-point numbers is ``many-to-one'',
which means that coercion of two distinct rational numbers may result
in the same floating-point number.

The simplest way to avoid this problem is analogous to the
floating-point contagion rule: when comparing a floating-point number
to an exact rational, convert the floating-point number to an exact
rational and then compare the two rational numbers.  This works
because the mapping from floating-point to rational numbers, if
properly implemented, preserves all of the precision in the argument.
However, this can only work if unlimited-precision rational numbers
are available (another argument for implementing them).

When unlimited-precision rationals are not available (i.e. when all
exact real numbers are integers), a more complex method is required.
A detailed description of this method is beyond the scope of this
appendix, but we can sketch an outline.  Call the floating-point
argument $\vri{x}$ and the exact integer argument $\vri{n}$.  Convert
$\vri{n}$ to a floating-point number and call the result $\vrii{x}$.
Then convert $\vrii{x}$ back to an integer, and call the result
$\vrii{n}$.  Finally compare $\vri{x}$ to $\vrii{x}$, modifying the
result of the comparison to account for the fact that $\vrii{x}$
differs from $\vri{n}$ by $\vrii{n}-\vri{n}$.  (Note that this
description ignores some limit cases, such as when $\vri{n}$ is too
large to be converted to floating-point or $\vrii{x}$ is too large to
be converted to an integer, but these cases are easily handled.)

\section{External Representations}

The difficulties inherent in converting a binary floating-point number
to or from a decimal external representation have been known for some
time (for example, see~\cite{Matula68}, \cite{Matula70}, and
\cite{Heuristic}), but are still not widely appreciated.  This is of
interest to the Scheme implementor when floating-point representations
are used, and may also apply to other inexact representations.

The definition of \ide{number->string} places strong constraints on
that procedure when its argument is a floating-point number; it
simultaneously places strong constraints on \ide{string->number}.  Two
requirements from this definition deserve attention.  The first
requirement is that the result of \ide{number->string}, if passed to
\ide{string->number}, returns a number which is \ide{eqv?} to the
original argument of \ide{number->string} (in~\cite{Heuristic} this is
referred to as the ``internal identity requirement'').  The second
requirement is that the number of digits produced for decimal-point
notation is the minimum required to satisfy the first requirement.
The combination of these requirements is sufficiently severe that most
standard algorithms cannot be used (for example, the algorithms
described in~\cite{Knuth}, section 4.4).

Fortunately, when IEEE floating-point arithmetic is used, it is
possible to implement these procedures in a straightforward manner:
IEEE floating-point arithmetic specifies the implementation of
binary-to-decimal and decimal-to-binary conversions that satisfy the
internal identity requirement within certain ranges; it is desirable
to take advantage of these conversions as they are typically
implemented in hardware.  The binary-to-decimal conversion is not
required to minimize the number of output digits, but a suitable
conversion can be implemented by trimming its output and testing the
trimmed output for equivalence.

If IEEE floating-point arithmetic is unavailable these procedures will
have to be implemented in software.  To our knowledge, the only
correct algorithm for \ide{number->string} is that described
in~\cite{Heuristic}; we know of two correct algorithms for
\ide{string->number}, one of which is described in~\cite{readfloat}.

\section{Rationalize}

The procedure \ide{rationalize} is interesting because most
programming languages do not provide anything analogous to it.  For
simplicity, we present an algorithm which computes the correct result
for exact arguments (provided the implementation supports exact
rational numbers of unlimited precision), and produces a reasonable
answer for inexact arguments when inexact arithmetic is implemented
using floating-point.  Those interested in the theory of this
algorithm should refer to~\cite{CFractions} (we thank Alan Bawden for
contributing this algorithm).

\begin{scheme}
(define (rationalize x e)
  (simplest-rational (- x e) (+ x e)))

(define (simplest-rational x y)
  (define (simplest-rational-internal x y)
    ;; assumes 0 < X < Y
    (let ((fx (floor x))        
          (fy (floor y)))
      (cond ((not (< fx x))
             fx)
            ((= fx fy)
             (+ fx
                (/ (simplest-rational-internal
                    (/ (- y fy))
                    (/ (- x fx))))))

            (else
             (+ 1 fx)))))
  ;; do some juggling to satisfy preconditions
  ;; of simplest-rational-internal.
  (cond ((< y x)
         (simplest-rational y x))
        ((not (< x y))
         (if (rational? x) x (error)))
        ((positive? x)
         (simplest-rational-internal x y))
        ((negative? y)
         (- (simplest-rational-internal (- y)
                                        (- x))))
        (else
         (if (and (exact? x) (exact? y))
             0
             0.0))))%
\end{scheme}
