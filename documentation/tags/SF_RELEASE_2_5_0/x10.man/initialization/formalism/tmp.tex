\RULE{R-Finish}
\RULE{R-Async}
\RULE{R-Seq}
\RULE{R-Field-Access}
\RULE{R-Field-Assign}
\RULE{R-New-Object}
\RULE{R-New}
\RULE{R-Invoke}
\RULE{RC-Receiver}
\RULE{RC-Arguments}
\RULE{RC-Ctor}
\RULE{RC-Seq}
\RULE{RC-Field-Access}
\RULE{RC-Field-Assign1}
\RULE{RC-Field-Assign2}
\RULE{RC-Async1}
\RULE{RC-Async2}
\RULE{RC-Finish}
\RULE{RA-Receiver}
\RULE{RA-Arguments}
\RULE{RA-Ctor}
\RULE{RA-Seq}
\RULE{RA-Field-Access}
\RULE{RA-Field-Assign1}
\RULE{RA-Field-Assign2}

$\typerule{
}{
  \finish~\hl,H \reduce \hl,H
}$~\RULE{(R-Finish)}
\quad
$\typerule{
}{
  \async~\hl;\he,H \reduce \he,H
}$~\RULE{(R-Async)}
\\\\

$\typerule{
    H(\hl)=\hC(\ol{\hf}\mapsto\ol{\hl'})
}{
  \hl.\hf_i,H \reduce \hl_i',H
}$~\RULE{(R-Field-Access)}
\quad
$\typerule{
    H(\hl)=\hC(F)
        \gap
    F' = F[ \hf \mapsto \hl"]
}{
  \hl.\hf=\hl",H \reduce \hl",H[ \hl \mapsto \hC(F')]
}$~\RULE{(R-Field-Assign)}
\\\\

$\typerule{
    \hl" \not \in \dom(H)
}{
  \hnew ~ \hObject(),H \reduce \hl",H[ \hl" \mapsto \hObject()]
}$~\RULE{(R-New-Object)}
\\\\

$\typerule{
    \ctorbody{}(\hC) = \ol{\hx}.\hsuper(\ol{\he});\he";
        \gap
    \hclass~\hC~\hextends~\hD
        \gap
    \hnew ~ \hD([\ol{\hl}/\ol{x}]\ol{\he}),H \reduce \hl',H'
        \gap
    H'[\hl']=\hD(F)
}{
  \hnew ~ \hC(\ol{\hl}),H \reduce [\ol{\hl}/\ol{x},\hl'/\this]\he",H'[ \hl' \mapsto \hC(F)]
}$~\RULE{(R-New)}
\\\\
$\typerule{
    H(\hl')=\hC(\ldots)
        \gap
    \mbody{}(\hm,\hC)=\ol{x}.\he
}{
  \hl'.\hm(\ol{\hl}),H \reduce [\ol{\hl}/\ol{x},\hl'/\this]\he,H
}$~\RULE{(R-Invoke)}
\\\\

$\typerule{
    \he,H \reduce \he',H'
}{
  \he.\hm(\ol{\he}),H \reduce \he'.\hm(\ol{\he}),H'
}$~\RULE{(RC-Receiver)}
\\\\
$\typerule{
    \he,H \reduce \he',H'
}{
  \hl.\hm(\ol{\hl},\he,\ol{\he}),H \reduce \hl.\hm(\ol{\hl},\he',\ol{\he}),H'
}$~\RULE{(RC-Arguments)}
\\\\
$\typerule{
    \he,H \reduce \he',H'
}{
  \hnew ~ \hC(\ol{\hl},\he,\ol{\he}),H \reduce \hnew ~ \hC(\ol{\hl},\he',\ol{\he}),H'
}$~\RULE{(RC-Ctor)}
\\\\
$\typerule{
    \he,H \reduce \he',H'
}{
  \he.\hf,H \reduce \he'.\hf,H'
}$~\RULE{(RC-Field-Access)}
\\\\
$\typerule{
    \he,H \reduce \he',H'
}{
  \he.\hf=\he",H \reduce \he'.\hf=\he",H'
}$~\RULE{(RC-Field-Assign1)}
\qquad
$\typerule{
    \he,H \reduce \he',H'
}{
  \hl.\hf=\he,H \reduce \hl.\hf=\he',H'
}$~\RULE{(RC-Field-Assign2)}
\\\\
$\typerule{
    \he,H \reduce \he',H'
}{
  \async~\he;\he",H \reduce \async~\he';\he",H'
}$~\RULE{(RC-Async1)}
\qquad
$\typerule{
    \he,H \reduce \he',H'
}{
  \async~\he";\he,H \reduce \async~\he";\he',H'
}$~\RULE{(RC-Async2)}
\\\\
$\typerule{
    \he,H \reduce \he',H'
}{
  \finish~\he,H \reduce \finish~\he',H'
}$~\RULE{(RC-Finish)}
\\

$\typerule{
}{
  (\async~\he;\hl).\hm(\ol{\he}),H \reduce \async~\he; \hl.\hm(\ol{\he}),H
}$~\RULE{(RA-Receiver)}
\\
$\typerule{
}{
  \hl.\hm(\ol{\hl},(\async~\he;\hl'),\ol{\he}),H \reduce \async~\he; \hl.\hm(\ol{\hl},\hl',\ol{\he}),H
}$~\RULE{(RA-Arguments)}
\\
$\typerule{
}{
  \hnew ~ \hC(\ol{\hl},(\async~\he;\hl'),\ol{\he}),H \reduce \async~\he; \hnew ~ \hC(\ol{\hl},\hl',\ol{\he}),H
}$~\RULE{(RA-Ctor)}
\\
$\typerule{
}{
  (\async~\he;\hl).\hf,H \reduce \async~\he; \hl.\hf,H
}$~\RULE{(RA-Field-Access)}
\\
$\typerule{
}{
  (\async~\he;\hl).\hf=\he",H \reduce \async~\he; \hl.\hf=\he",H
}$~\RULE{(RA-Field-Assign1)}
\\
$\typerule{
}{
  \hl.\hf=(\async~\he;\hl'),H \reduce \async~\he; \hl.\hf=\hl',H
}$~\RULE{(RA-Field-Assign2)}
\\\\




























The following is an example program in this syntax:

\begin{lstlisting}
class E extends Object {}
class C2 extends Object {
  f1:E;
  f2:E;
  Read() SyncWrite(f1,f2) AsyncWrite(f1,f2) build(a1:E, a2:E) =
    finish
      async this.f2 = a2;
      m(a1);
  Read() SyncWrite() AsyncWrite(f1) m(a:E):E =
    async this.f1 = a;
    new E();
  escaping m2():C2 = this;
}
class C3 extends Object {
  f:E;
  Read() SyncWrite(f) AsyncWrite(f) build(a:E) =
    f=a;
    f; // ok
}
class C4 extends Object {
  fVar:E;
  ctor(a:E) =
    super();
    fVar;
    fVar=a; // ERR (read before write)
}
\end{lstlisting}




























\newcommand{\R}[2]{\code{R}(#1,#2,\hC)}
\newcommand{\SW}[1]{\code{SW}(#1,\hC)}
\newcommand{\AW}[1]{\code{AW}(#1,\hC)}

Function~$\R{\he}{F}$ returns whether
    \this does not escape from \he (i.e., \this can be used only as a field or method receiver),
    and whether
    \he only reads fields of \this that are in F or that have been previously written in e.
For example,
\beqs{R-example}
    \R{\this.\hf}{\{\hf\}}&=\htrue\\
    \R{\code{new Seq(this.f=new Object(), this.f)}}{ \emptyset}&=\htrue\\
\eeq

\beqs{R}
    &\R{[\he_0,\ldots,\he_n]}{F}= \R{\he_0}{F} \hand \R{\he_1}{\SW{[\he_0]} \cup F} \hand \ldots \hand  \R{\he_n}{\SW{[\he_0,\ldots,\he_{n-1}]} \cup F}\\
    &\R{\he"}{F}=
        \begin{cases}
        \htrue & \he" \equiv \hl \\
        \hfalse & \he" \equiv \this \\
        \htrue & \he" \equiv \hx \\
        \hf \in F & \he" \equiv \this.\hf \\
        \R{\he}{F} & \he" \equiv \he.\hf \\
        \R{\he'}{F} & \he" \equiv \this.\hf = \he' \\
        \R{[\he,\he']}{F} & \he" \equiv \he.\hf = \he' \\
        \R{[\he,\he']}{F} & \he" \equiv \he;\he' \\
        \hfalse & \he" \equiv \this.\hm(\ol{\he}) \gap \mmodifier{}(\hm,\hC)=\hescaping  \\
        R \subseteq F \hand \R{[\ol{\he}]}{F} & \he" \equiv \this.\hm(\ol{\he}) \gap \mmodifier{}(\hm,\hC)=\Read(R) \ldots  \\
        \R{[\he',\ol{\he}]}{F} & \he" \equiv \he'.\hm(\ol{\he}) \\
        \R{[\ol{\he}]}{F} & \he" \equiv \hnew{\hC}{\hr}{\ol{\he}} \\
        \R{\he}{F} & \he" \equiv \hfinish~\he \\
        \R{\he}{F} \hand \R{\he'}{F} & \he" \equiv \hasync~\he;\he' \\
        \end{cases}
\eeq


$\SW{\he}$ is the set of fields of \this that are definitely-(synchronously)-written in \he.
(The function is undefined if there is a call to an \hescaping method.)
\beqs{SW}
    &\SW{[\he_0,\ldots,\he_n]}= \SW{\he_0} \cup \ldots \cup  \SW{\he_n}\\
    &\SW{\he"}=
        \begin{cases}
        \emptyset & \he" \equiv \hl \\
        \emptyset & \he" \equiv \hx \\
        \SW{\he} & \he" \equiv \he.\hf \\
        \{ \hf \} \cup \SW{\he'} & \he" \equiv \this.\hf = \he' \\
        \SW{[\he,\he']} & \he" \equiv \he.\hf = \he' \\
        \SW{[\he,\he']} & \he" \equiv \he;\he' \\
        S \cup \SW{[\ol{\he}]} &
            \he" \equiv \this.\hm(\ol{\he}) \gap \mmodifier{}(\hm,\hC)=\ldots \SyncWrite(S) \ldots  \\
        \SW{[\he',\ol{\he}]} & \he" \equiv \he'.\hm(\ol{\he}) \\
        \SW{[\ol{\he}]} & \he" \equiv \hnew{\hC}{\hr}{\ol{\he}} \\
        \AW{\he} & \he" \equiv \hfinish~\he \\
        \SW{\he'} & \he" \equiv \hasync~\he;\he' \\
        \end{cases}
\eeq
$\AW{\he}$ is the set of fields of \this that are asynchronously written in \he.
Note that in $\AW{\hasync~\he;\he'}$ we collect writes to fields in both \he and $\he'$,
    whereas in $\SW{\hasync~\he;\he'}$ we only consider writes in $\he'$.
\beqs{AW}
    &\AW{[\he_0,\ldots,\he_n]}= \AW{\he_0} \cup \ldots \cup  \AW{\he_n}\\
    &\AW{\he"}=
        \begin{cases}
        \emptyset & \he" \equiv \hl \\
        \emptyset & \he" \equiv \hx \\
        \AW{\he} & \he" \equiv \he.\hf \\
        \{ \hf \} \cup \AW(\he') & \he" \equiv \this.\hf = \he' \\
        \AW{[\he,\he']} & \he" \equiv \he.\hf = \he' \\
        \AW{[\he,\he']} & \he" \equiv \he;\he' \\
        A \cup \AW{[\ol{\he}]} &
            \he" \equiv \this.\hm(\ol{\he}) \gap \mmodifier{}(\hm,\hC)=\ldots \AsyncWrite(A) \ldots  \\
        \AW{[\he',\ol{\he}]} & \he" \equiv \he'.\hm(\ol{\he}) \\
        \AW{[\ol{\he}]} & \he" \equiv \hnew{\hC}{\hr}{\ol{\he}} \\
        \AW{\he} & \he" \equiv \hfinish~\he \\
        \AW{\he} \cup \AW{\he'} & \he" \equiv \hasync~\he;\he' \\
        \end{cases}
\eeq
