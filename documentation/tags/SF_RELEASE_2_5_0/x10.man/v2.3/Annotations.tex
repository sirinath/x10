\chapter{Annotations}\label{XtenAnnotations}\index{annotations}


\Xten{} provides an 
an annotation system  system for to allow the
compiler to be extended with new static analyses and new
transformations.

Annotations are constraint-free interface types that decorate the abstract syntax tree
of an \Xten{} program.  The \Xten{} type-checker ensures that an annotation
is a legal interface type.
In \Xten{}, interfaces may declare
both methods and properties.  Therefore, like any interface type, an
annotation may instantiate
one or more of its interface's properties.
%%PLUGINNERY%%  Unlike with Java
%%PLUGINNERY%%  annotations,
%%PLUGINNERY%%  property initializers need not be
%%PLUGINNERY%%  compile-time constants;
%%PLUGINNERY%%  however, a given compiler plugin
%%PLUGINNERY%%  may do additional checks to constrain the allowable
%%PLUGINNERY%%  initializer expressions.
%%PLUGINNERY%%  The \Xten{} type-checker does not check that
%%PLUGINNERY%%  all properties of an annotation are initialized,
%%PLUGINNERY%%  although this could be enforced by
%%PLUGINNERY%%  a compiler plugin.

\section{Annotation syntax}

The annotation syntax consists of an ``\texttt{@}'' followed by an interface type.

%##(Annotations Annotation
\begin{bbgrammar}
%(FROM #(prod:Annotations)#)
         Annotations \: Annotation & (\ref{prod:Annotations}) \\
                     \| Annotations Annotation \\
%(FROM #(prod:Annotation)#)
          Annotation \: \xcd"@" NamedTypeNoConstraints & (\ref{prod:Annotation}) \\
\end{bbgrammar}
%##)

Annotations can be applied to most syntactic constructs in the language
including class declarations, constructors, methods, field declarations,
local variable declarations and formal parameters, statements,
expressions, and types.
Multiple occurrences of the same annotation (i.e., multiple
annotations with the same interface type) on the same entity are permitted.

%%OBSOLETE%% \begin{grammar}
%%OBSOLETE%% ClassModifier \: Annotation \\
%%OBSOLETE%% InterfaceModifier \: Annotation \\
%%OBSOLETE%% FieldModifier \: Annotation \\
%%OBSOLETE%% MethodModifier \: Annotation \\
%%OBSOLETE%% VariableModifier \: Annotation \\
%%OBSOLETE%% ConstructorModifier \: Annotation \\
%%OBSOLETE%% AbstractMethodModifier \: Annotation \\
%%OBSOLETE%% ConstantModifier \: Annotation \\
%%OBSOLETE%% Type \: AnnotatedType \\
%%OBSOLETE%% AnnotatedType \: Annotation\plus Type \\
%%OBSOLETE%% Statement \: AnnotatedStatement \\
%%OBSOLETE%% AnnotatedStatement \: Annotation\plus Statement \\
%%OBSOLETE%% Expression \: AnnotatedExpression \\
%%OBSOLETE%% AnnotatedExpression \: Annotation\plus Expression \\
%%OBSOLETE%% \end{grammar}

\noindent
Recall that interface types may have dependent parameters.

\noindent
The following examples illustrate the syntax:

\begin{itemize}
\item Declaration annotations:
\begin{xtennoindent}
  // class annotation
  @Value
  class Cons { ... }

  // method annotation
  @PreCondition(0 <= i && i < this.size)
  public def get(i: Int): T { ... }

  // constructor annotation
  @Where(x != null)
  def this(x: T) { ... }

  // constructor return type annotation
  def this(x: T): C@Initialized { ... }

  // variable annotation
  @Unique x: A;
\end{xtennoindent}
\item Type annotations:
\begin{xtennoindent}
  List@Nonempty

  Int@Range(1,4)

  Array[Array[Double]]@Size(n * n)
\end{xtennoindent}
\item Expression annotations:
\begin{xtennoindent}
  m()  @RemoteCall
\end{xtennoindent}
\item Statement annotations:
\begin{xtennoindent}
  @Atomic { ... }

  @MinIterations(0)
  @MaxIterations(n)
  for (var i: Int = 0; i < n; i++) { ... }

  // An annotated empty statement ;
  @Assert(x < y);
\end{xtennoindent}
\end{itemize}

\section{Annotation declarations}

Annotations are declared as interfaces.  They must be
subtypes of the interface \texttt{x10.lang.annotation.Annotation}.
Annotations on particular static entities must extend the corresponding
\xcd`Annotation` subclasses, as follows: 
\begin{itemize}
\item Expressions---\xcd`ExpressionAnnotation`
\item Statements---\xcd`StatementAnnotation`
\item Classes---\xcd`ClassAnnotation`
\item Fields---\xcd`FieldAnnotation`
\item Methods---\xcd`MethodAnnotation`
\item Imports---\xcd`ImportAnnotation`
\item Packages---\xcd`PackageAnnotation`
\end{itemize}
