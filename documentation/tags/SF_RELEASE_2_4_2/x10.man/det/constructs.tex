\subsection{X10}

The X10 construct \code{async S} spawns a new activity to execute
\code{S}. This activity can access variables in the lexical
environment. The construct \code{finish S} executes \code{S} and waits
for all asyncs spawned within \code{S} to terminate.

X10 2.2 introduces {\em clocked} versions of \code{async} and
\code{finish}. This design (which is the one we will consider in this
paper) is a simplification of the original design and has the nice
property that clocks are not present as data objects in the source
program, thus removing a potential source of error.

Clocks are motivated by the desire to support barrier computations in
which all threads in a given group must reach a point in the code (a
barrier) before all of them can progress past it. X10 permits many
clocks to exist at the same time, permits an activity to operate on
multiple clocks at the same time, and permits newly created activities
to operate on existing clocks.

X10 2.2 permits \code{async S} and \code{finish S} to be optionally
modified with a \code{clocked} qualifier.  When an activity executes
\code{clocked finish S} it creates a new clock \code{c}, registers
itself on it, and executes \code{finish S} with \code{c} as the {\em
  current clock}.  The clock object is implicit in that it cannot be
referred to in the code.  Within \code{S}, the \code{clocked async S}
construct can be used to spawn an async registered on the current
clock; \code{async S} should be used to spawn an async that is {\em
  not} registered on the current clock.

Note that \code{finish} blocks can be arbitrarily nested, so an
activity may at any given time be registered on a stack of clocks. The
current clock is always the one most recently pushed on to the stack.
No support is provided to register an activity on an ancestor
clock.

When a clocked async terminates it automatically de-registers from the
clock; when the parent activity that entered the \code{clocked finish}
block reaches the end of the block it also automatically de-registers
from the clock.

The special statement \code{advance} can be used to advance the
current clock.  Whenever an activity hits \code{advance} it blocks on
the current clock it is registered with. (It is illegal for an
un-clocked \code{async} to execute \code{advance}.) Once all
activities registered on the clock have reached an advance, all of
them can progress.

Additionally, X10 permits computations to be run on multiple places. A
place consists of data as well as activities that operate on
it. Typically a place is realized as  an operating system process in a
cluster or multi-node computer. The \code{at (p) S} construct is used
to cause the current activity to shift to place \code{p} and execute
\code{S}.

\code{at} blocks and (\code{clocked}) \code{async}, \code{finish}
blocks can be arbitrarily nested and arbitrarily combined with
recursion. That is, method invocations can return having spawned
\code{async}s that have not terminated. The body \code{S} of a
\code{clocked finish S} may spawn an \code{async} that is not
clocked, or may spawn a nested \code{clocked finish S}.
This flexibility makes X10 a very succinct and elegant
language for expressing many patterns of communication and
concurrency.

\subsection{Accumulators}
%% Compare with OpenMP reduction clause.
%% DOMP has arbitrary reduction types.

The class \code{Acc[T]} implements the notion of an accumulator. To
construct an instance of this class a binary associative and
commutative operator  \code{f} over \code{T} (the {\em reduction operator}) must be supplied,
together with the zero for the operator, \code{z} (satisfying \code{f(v,z)=z}
  for all \code{v}), and an initial value from \code{T}. The accumulator
  provides the following operations. For \code{a} such a value, the
  operator \code{a() <- v} accumulates the value \code{v} into
  \code{a}. The operator \code{a()} returns the current value. The
  operator \code{a() = v} sets the value of the accumulator to
  \code{v}.

There are no restrictions on storing accumulators into heap
data-structures, or aliasing them.  However, each accumulator is {\em
  registered} with the activity that created it. This registration is
automatically inherited by all activities spawned (transitively) by
this activity. It is an error for an activity to operate on an
accumulator it is not registered with.  Thus one can use the
flexibility of the heap to arrange for complex data-dependent
transmission pathways for the accumulator from point of creation to
point of use, e.g.{} arrays of accumulators, hash-maps, etc. In
particular, accumulators can be passed into arbitrary method
invocations, returned from methods etc, transmitted to other places
with no restrictions. The only restriction is that the accumulator can
only be used by the activity that created it or its children.

There are no restrictions on the use of the accumulate operation
\code{a()<- v}. The same activity can perform multiple accumulate
operations on one or more accumulators. Thus, the same set of spawned
activities can perform multiple Map Reduce patterns simultaneously.

However, the read operation \code{a()} and the update operation
\code{a()=v} are subject to a dynamic restriction. (1)~They may only
be performed by the activity that created \code{a}. (2)~Say that an
activity \code{A} is in {\em synchronous mode} if all asyncs spawned
by \code{A} have terminated or stopped at a next on a clock registered
on \code{A}.  In such a situation there is no concurrent mutator of
\code{a} (an activity not in the subtree of \code{A} that may have
acquired a reference to \code{a} through the heap is unable to perform
any operation on it). Hence it is safe to read \code{a}. Similarly for
update. Therefore the semantics of these operations require that the
operation {\em suspend} until such time as the synchronous condition
is true.

Intuitively, this does not introduce any deadlocks since the
spawned activities cannot be waiting for any operation by the parent
(other than an \code{advance}) in order to progress. It clearly
preserves serial semantics since a read will only reflect the results
of accumulations performed earlier in program order. A more detailed
treatment is provided in \Ref{Section}{semantics}.


In many cases the compiler can statically evaluate whether an activity
is registered on the clock and/or whether a synchronous call will
suspend. In such cases the checks can be eliminated.
\begin{example}
In the following example, the compiler can infer that \code{x()} wont suspend, due to
the \code{finish}. Hence it may eliminate the run-time suspension
check. Further it can establish that \code{x} is \code{@Sync}
registered with the current activity, hence it can eliminate the
access check in \code{x()}.
\begin{lstlisting}
val x:Acc[Int] = new Acc[Int](0, Int.+);
finish for (i in 0..100000) async
   x <- i;
Console.OUT.println("x is " + x());
\end{lstlisting}

\end{example}

\subsection{Clocked types}

The central idea behind clocked data-structures is that read/write
conflicts are avoided using ``double buffering.'' Two versions of the
data-structure are kept, the {\em current} and the {\em next}
versions. Reads can be performed simultaneously by multiple activities
-- they are performed on the current version of the
data-structure. Writes are performed on the next version of the
data-structure. On detection of termination of the current phase --
when all involved activities are quiescent -- the current and the next
versions are switched.

%%\code{Clocked[T]} and \code{ClockedAcc[T]} are distinguished in that
%%unlike the former the latter permits accumulation
%%operations. \code{ClockedAcc[T]} is not the same as
%%\code{Clocked[Acc[T]]} since the latter insists that only the creating
%%activity be able to read the accumulator
%%
Clocked objects are registered with activities, just like
accumulators.  This ensures that any other activity with access to the
clocked object (or its encapsulated objects) throught he heap is
unable to operate on them.  Only the activity that created a
\code{Clocked[T]} value \code{v} can enter a
\code{clocked finish S} block within which \code{v} can be operated on
by multiple activities. Such a block is said to be the {\em current
  block} for \code{v}. Only and precisely the \code{clocked asyncs}
spawned dynamically at the top-level in the body of the current block
can operate on \code{v}. In particular if a top-level clocked async
were to further spawn a \code{clocked finish S1}, then \code{v} would
not be considered accessible within \code{S1} (since the current block
and current clock have changed).  This is checked dynamically.

Similarly, \code{v} is not accessible within a \code{async S1} spawned
dynamically by \code{S} since \code{S1} operates asynchronously to the
clock and hence cannot access \code{v} determinately. This is also
checked dynamically. Note that a \code{clocked finish S} can advance
on its clock even if \code{S} dynamically spawns an un-clocked
\code{async} that is active. The only requirement to advance the clock
is that all clocked asyncs registered on the clock have reached an
\code{advance}. These rules imply that the activity that created
\code{v} can operate on \code{v} in multiple successive \code{clocked
  finish} statements, with arbitrary sequential code (accessing
\code{v}) in between.

\code{Clocked[T]} has a constructor that takes two \code{T} arguments,
these are used to initialize the \code{current} and \code{next} fields. These arguments
should be ``new'' (that is, no other data-structure should have a
reference to these arguments), and are considered registered with the
creating async and its descendants.

For \code{x:Clocked[T]} the following operations available:
\begin{itemize}
\item \code{x()}: this returns the value of the \code{current} field. Note
  that this operation can be called simultaneously by multiple
  activities  (unlike \code{Acc}). 

\item \code{x() = t}: This sets the value of the \code{next} field to \code{t}.
Note: write-write conflicts are possible since multiple activities may
try to set the value at the same time. So a lightweight static
technique should be used to establish that at most one activity is
writing to that location. In the case of arrays of clocked types this
can often be established using an ``owner computes'' rule.
\end{itemize}

Further the compiler treats invocations of methods defined on \code{T}
and invoked on \code{x()} specially. A method may be marked
\code{@Read} -- such a method must not mutate any state (this is
checkedby the compiler). Invocations of such methods are performed on
the \code{current} field. For instance, for arrays, the accessor
method is marked \code{@Read}.  A method may also be marked
\code{@Write} or \code{@Async} -- these methods may mutate state and
are invoked on the \code{next} field. The
first kind of method must be called by at most one activity in a phase
(and is subject at compile-time to the owner computes rule), the
second may be called by more than one activity. The first is permitted
to update state, the second may not, but may invoke reduction
operations. For arrays and accumulators, the updater method is marked
\code{@Write}. For accumulators, the accumulate method is marked
\code{@Async}.


%%\code{ClockedAcc[T]} has a constructor that takes two \code{T} values and a
%%\code{Reducer[T]} as argument. The two \code{T} values are used to initialize the
%%\code{current} and \code{next} fields. These arguments should be ``new'' (that is, no
%%other data-structure should have a reference to these arguments).
%%The reducer is used to perform
%%accumulate operations.
%%
%%Operations for \code{x:ClockedAcc[T]}:
%%\begin{itemize}
%%\item \code{x()}: this returns the value of the \code{current}
%%  field. This can be called simultaneously by multiple activities.
%%\item\code{x() <- t}:  this accumulates \code{t} into the next field. Note: No
%%     write-write conflicts are possible.
%%\item\code{x() = t} -- this resets the value of the next field to \code{t}. To avoid
%%     read/write and write/write conflicts, this operation should be
%%     invoked only by the closure argument of
%%     \code{Clock.advance(closure)}. (See below.)
%%\end{itemize}
%%

We add the following method on Clock:
\begin{lstlisting}
public static def advance(x:()=>void) {...}
\end{lstlisting}

If all activities registered on the clock invoke \code{advance(f)}
(for the same value \code{f}), then \code{f} is guaranteed to be
invoked by some activity \code{A} registered on the clock at a point in time
when all other activities have entered the \code{advance(f)} call
and the current/next swap has been performed for all registered
clocked values.  At this point -- also called the {\em clock quiescent
point} -- it is guaranteed that none of the other activities are
performing a read or write operation on user-accessible memory.
\footnote{
A possible implementation of \code{Clocked[T]}
is that a system-synthesized closure (that
performs the current/next swap) is run at the clock quiescent point
before the user specified closure is run.}
