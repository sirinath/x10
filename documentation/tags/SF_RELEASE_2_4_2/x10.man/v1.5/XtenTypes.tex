\chapter{Types}
\label{XtenTypes}\index{types}

{}\Xten{} is a {\em strongly typed} object language: every variable
and expression has a type that is known at compile-time. Further,
\Xten{} has a {\em unified} type system: all data items created at
runtime are {\em objects} (\S~\ref{XtenObjects}. Types limit the
values that variables can hold, and specify the places at which these
values lie.

{}\Xten{} supports two kinds of objects, {\em reference objects} and
{\em value objects}.  Reference objects are instances of {\em
reference classes} (\S~\ref{ReferenceClasses}). They may contain
mutable fields and must stay resident in the place in which they were
created. Value objects are instances of {\em value classes}
(\S~\ref{ValueClasses}). They are immutable and may be freely copied
from place to place. Either reference or value objects may be 
{\em scalar} (instances of a non-array class) or {\em aggregate} (instances
of arrays).

An \Xten{} type is either a {\em reference type} or a {\em value
type}.  Each type consists of a {\em data type}, which is a set of
values, and a {\em place type} which specifies the place at which the
value resides.  Types are constructed through the application of {\em
type constructors} (\S~\ref{TypeConstructors}).

Types are used in variable declarations, casts, object creation, array
creation, class literals and {\cf instanceof} expressions.\footnote{In
order to allow this version of the language to focus on the core new
ideas, \XtenCurrVer{} does not have user-definable classloaders,
though there is no technical reason why they could not have been
added.}

A variable is a storage location (\S~\ref{XtenVariables}). All
variables are initialized with a value and cannot be observed without
a value. 

Variables whose value may not be changed after initialization are
called {\em final variables} (or sometimes {\em constants}).  The
programmer indicates that a variable is final by using the annotation
{\tt final} in the variable declaration.  

%% Final variables play an 
%% important role in \Xten{}, as we shall discuss below. For this reason,
%% \Xten{} enforces the lexical restriction that all variables whose name
%% starts with an upper case letter are implicitly declare final. (It is
%% not an error to also explicitly declare such variables as
%% final.)\index{Upper-case Convention}

\section{Type constructors}\index{type constructors}\label{TypeConstructors}

An \Xten{} type is a pair specifying a {\em datatype} and a {\em
placetype}. Semantically, a datatype specifies a set of values and a
placetype specifies the set of places at which these values may
live. Thus taken together, a type specifies both the kind of value
permitted and its location. 

\begin{x10}
509   Type ::=  DataType  PlaceTypeSpecifieropt
510     | nullable  Type
511     | future <  Type > 
512   DataType ::=  PrimitiveType
513   DataType ::=  ClassOrInterfaceType
514     |  ArrayType
\end{x10}

For simplicity, this version of \Xten{} does not permit the
specification of generic classes or interfaces. This is expected to be
remedied in future versions of the language.

Every class and interface definition in \Xten{} defines a type with
the same name. Additionally, {}\Xten{} specifies three {\em type
constructors}: {\tt nullable}, the {\tt future}, and array type
constructors. We discuss these constructors and place types in detail
in the secions that follow; here we briefly discss interface and class
declarations.

\paragraph{Interface declarations.}\label{InterfaceTypes}
An interface declaration specifies a name, a list of extended
interfaces, and constants ({\tt public static final} fields) and
method signatures associated with the interface. Each interface
declaration introduces a type with the same name as the declaration.
Semantically, the data type is the set of all objects which are
instances of (value or reference) classes that implement the
interface. A class implements an interface if it says it does and if
it implements all the methods defined in the interface.


The {\em interface declaration} (\S~\ref{XtenInterfaces}) takes as
argument one or more interfaces (the {\em extended} interfaces), one
or more type parameters and the definition of constants and method
signatures and the name of the defined interface.  Each such
declaration introduces a data type.

\begin{x10}
426   DataType ::= ClassOrInterfaceType
433   ClassOrInterfaceType ::= TypeName 
13    ClassType ::= TypeName
15    TypeName ::= identifier
16     | TypeName . identifier
\end{x10}

\paragraph{Reference class declarations.}\label{ReferenceTypes}
The {\em reference class declaration} (\S~\ref{ReferenceClasses}) takes
as argument a reference class (the {\em extended class}), one or more
interfaces (the {\em implemented interfaces}), the definition of
fields, methods and inner types, and returns a class of the named type
(\S~\ref{ReferenceClasses}). Each such declaration introduces a data
type. Semantically, the data type is the set of all objects which are
instances of (subclasses of) the class.

\paragraph{Value class declarations.}
The {\em value class declaration} (\S~\ref{ValueClasses}) is
similar to the reference class declaration except that it must extend
either a value class or a reference class that has no mutable fields.
It may be used to construct a value type in the same way as a
reference class declaration can be used to construct a reference type.

\section{The \Xcd{nullable} type constructor}
\label{NullableTypeConstructor}\index{nullable@{\tt nullable}}

\Xten{} supports the type constructor, \xcd"nullable[T]".  For any
type \xcd"T", the type \xcd"nullable[T]" contains all the values of
type \xcd"T", and a special \xcd"null" value, unless \xcd"T" already
contains \xcd"null". This value is designated by the literal
\xcd"null", which is special in that it has the type
\xcd"nullable T" for all types \xcd"T".\index{null@{\tt null}}

The visibility of the type \xcd"nullable[T]" is the same as the
visibility of \xcd"T". The members of the type \xcd"nullable[T]" are
the same as those of type \xcd"T". Note that because of this
\xcd"nullable" may not be regarded as a generic class; rather it is a
special type constructor.  In fact, \xcd"nullable[T]" can be
considered a mixin; it is a subtype of \xcd"T".

%% TODO: Visibility of nullable^T.

This type constructor can be used in any type expression used to
declare variables (e.g., local variable{s}, method parameter{s},
class field{s}, iterator parameter{s}, try/catch parameter{s} etc).
It may be applied to value types, reference types or aggregate types.
It may not be used in an \xcd"extends" clause or an \xcd"implements"
clause in a class or interface declaration. It may not be used 
in a \xcd"new" expression---a \xcd"new" expression is used only to construct 
non-null values.

If \xcd"T" is a value
(respectively, reference) type, then \xcd"nullable[T]" is defined to be
a value (respectively, reference) type.

An immediate consequence of the definition of \xcd"nullable" is that
for any type \xcd"T", the type \xcd"nullable[nullable[T]]" is equal to
the type \xcd"nullable[T]".

Any attempt to access a field or invoke a method on the value
\xcd"null" results in a \xcd"NullPointerException".

An expression \xcd"e" of type \xcd"nullable[T]" may be checked for nullity
using the expression \xcd"e==null". (It is a compile-time error for
the static type of \xcd"e" to not be \xcd"nullable[T]", for some \xcd"T".)

\paragraph{Conversions}
\xcd"null" can be passed as an argument to a method call whose
corresponding formal parameter is of type \xcd"nullable[T]" for some type
\xcd"T". (This is a widening reference conversion, per \cite[Sec
5.1.4]{jls2}.) Similarly it may be returned from a method call of
return type \xcd"nullable T" for some type \xcd"T".

For any value \xcd"v" of type \xcd"T", the class cast expression
\xcd"(nullable[T]) v" succeeds and specifies a value of type \xcd"nullable[T]".
This value may be seen as the ``boxed'' version of \xcd"v".

\Xten{} permits the widening reference conversion from any type \xcd"T"
to the type \xcd"nullable[T1]" if \xcd"T" can be widened to the
type \xcd"T1". Thus, the type \xcd"T" is a subtype of the type \xcd"nullable[T]".
%in accordance with the LiskovSubstitutionPrinciple.

Correspondingly, a value \xcd"e" of type \xcd"nullable[T]" can be cast to the
type \xcd"T", resulting in a \xcd"NullPointerException" if \xcd"e" is
\xcd"null" and \xcd"nullable[T]" is not equal to \xcd"T", and in the
corresponding value of type \xcd"T" otherwise.  If \xcd"T" is a value
type this may be seen as the ``unboxing'' operator.

The expression \xcd"(T) null" throws a \xcd"ClassCastException"
if \xcd"T" is not equal to \xcd"nullable[T]"; otherwise it
returns \xcd"null" at type \xcd"T". Thus it may be used to check
whether \xcd"T" = \xcd"nullable[T]".

\paragraph{Arrays of nullary type}
The nullary type constructor may also be used in (aggregate) instance
creation expressions (e.g., \xcd"new nullable[T](R)"). In such a
case \xcd"T" must designate a class. Each member of the array is
initialized to \xcd"null", unless an explicit array initializer is
specified.

\paragraph{Implementation notes}
A value of type \xcd"nullable[T]" may be implemented by boxing a value of
type \xcd"T" unless the value is already boxed. The literal \xcd"null"
may be represented as the unique null reference.

\paragraph{\Java{} compatibility}

\java{} provides a somewhat different treatment of \xcd"null".  A
class definition extends a nullable type to produce a nullable type,
whereas primitive types such as \xcd"int" are not nullable---the
programmer has to explicitly use a boxed version of \xcd"int",
\xcd"Integer", to get the effect of \xcd"nullable int". Wherever \Java{} uses a
variable at reference type \xcd"T", and at runtime the variable may
carry the value \xcd"null", the \Xten{} programmer should declare the
variable at type \xcd"nullable[T]". However, there are many situations
in \java{} in which a variable at reference type \xcd"T" can be
statically determined to not carry null as a value. Such variables
should be declared at type \xcd"T" in \Xten{}.

\paragraph{Design rationale}

The need for \xcd"nullable" arose because \Xten{} has value types and
reference types, and arguably the ability to add a \xcd"null" value to
a type is orthogonal to whether the type is a value type or a
reference type. This argues for the notion of nullability as a type
constructor.

The key question that remains is whether it should be possible to
define ``towers'', that is, define the type constructor in such a way
that \xcd"nullable[nullable[T]]" is distinct from \xcd"nullable[T]". Here
one would think of nullable as a disjoint sum type constructor that
adds a value \xcd"null" to the interpretation of its argument type
even if it already has that value. Thus \xcd"nullable[nullable[T]]" is
distinct from \xcd"nullable T" because it has one more \xcd"null"
value. Explicit injection and projection functions of signature
\xcd"T => nullable[T]" and \xcd"nullable[T] => T", respectively,
would need to be provided.

The designers of \Xten{} felt that while such a definition might be
mathematically tenable, and programmatically interesting, it was
likely to be too confusing for programmers. More importantly, it would
be a deviation from current practice that is not forced by the core
focus of \Xten{} (concurrency and distribution). Hence the decision to
collapse the tower.  As discussed below, this results in no loss of
expressiveness because towers can be obtained through explicit
programming.

\paragraph{Examples}

Consider the following class:

\begin{xten}
final value Box { 
  public def this(v: nullable[Object]) { this.datum = v; }
  public var datum: nullable[Object];
}
\end{xten}

Now one may use a variable \xcd"x" at type \xcd"nullable[Box]" to
distinguish between the \xcd"null" at type \xcd"nullable[Box]" and at type
\xcd"nullable[Object]". In the first case the value
of \xcd"x" will be \xcd"null", in the second case the value of \xcd"x.datum"
will be \xcd"null".

Such a type may be used to define efficient code for memoization:

\begin{xten}
abstract class Memo {
  var values: Array[nullable[Box]];
  def this(n: int) {
    // initialized to all nulls
    values = new nullable[Box](n); 
  }
  abstract def compute(key: int): nullable[Object];
  def lookup(key: int) nullable[Object] = { 
   if (values(key) != null) 
     return values(key).datum;
   val v = compute(key);
   values(key) = new Box(v);
   return v;
  }
}
\end{xten}


% C#: http://blogs.msdn.com/ericgu/archive/2004/05/27/143221.aspx
% Nice: http://nice.sourceforge.net/cgi-bin/twiki/view/Doc/OptionTypes

 



\subsection{Future types}

The class \xcd"x10.lang.Future[T]"
is the type of all \xcd"future" expressions.
The type represents a value which when forced will return a value of type
\xcd"T". The class makes available the following methods:

\begin{xten}
package x10.lang;
public class Future[T] implements () => T {
  public def apply(): T = force();
  public def force(): T = ...;
  public def forced(): Boolean = ...;
}
\end{xten}
  

\section{Array Type Constructors}
\label{ArrayypeConstructors}\index{array types}

{}\XtenCurrVer{} does not have array class declarations
(\S~\ref{XtenArrays}). This means that user cannot define new array
class types. Instead arrays are created as instances of array types
constructed through the application of {\em array type constructors}
(\S~\ref{XtenArrays}).

The array type constructor takes as argument a type (the {\em base
type}), an optional distribution (\S~\ref{XtenDistributions}), and
optionally either the keyword {\cf reference} or {\cf value} (the
default is reference):
\begin{x10}
18    ArrayType ::= Type [ ]
438   ArrayType ::= X10ArrayType
439   X10ArrayType ::= Type [ . ]
440     | Type reference [ . ]
441     | Type value [ . ]
442     | Type [ DepParameterExpr ]
443     | Type reference [ DepParameterExpr ]
444     | Type value [ DepParameterExpr ]
\end{x10}

The array type {\cf Type[ ] } is the type of all arrays of base
type {\tt Type} defined over the distribution {\tt 0:N -> here}
for some positive integer {\tt N}.

The qualifier {\tt value} ({\tt reference}) specifies that the array
is a {\tt value}({\tt reference}) array. The array elements of a {\tt
value} array are all {\tt final}.\footnote{Note that the base type of a {\tt value} array can be a value class or a reference class, just as the 
type of a {\tt final} variable can be a value class or a reference class.
}If the qualifier is not specified,
the array is a {\tt reference} array.

The array type {\cf Type reference [.]} is the type of all (reference)
arrays of basetype {\tt Type}. Such an array can take on any
distribution, over any region. Similarly, {\cf Type value [.]} is the
type of all value arrays of basetype {\tt Type}.

\XtenCurrVer{} also allows a distribution to be specified between {\tt
[} and {\tt ]}. The distribution must be an expression of type
{\tt distribution} (e.g.{} a {\tt final} variable) whose
value does not depend on the value of any mutable variable.

Future extensions to \Xten{} will support a more general syntax for
arrays which allows for the specification of dependent types, 
e.g.{} {\tt double[:rank 3]}, the type of all arrays of 
{\tt double} of rank {\tt 3}.



\notfouro{\subsection{Place constraints}
\label{PlaceTypes}
\label{PlaceType}
\index{place types}
\label{DepType:PlaceType}\index{placetype}

An \Xten{} computation spans multiple places
(\Sref{XtenPlaces}). Each place constains data and activities that
operate on that data.  \XtenCurrVer{} does not permit the dynamic
creation of a place. Each \Xten{} computation is initiated with a
fixed number of places, as determined by a configuration parameter.
In this section we discuss how the programmer may supply place type
information, thereby allowing the compiler to check data locality,
i.e., that data items being accessed in an atomic section are local.

\begin{grammar}
PlaceConstraint     \: \xcd"!" Place\opt \\
Place              \:   Expression \\
\end{grammar}

Because of the importance of places in the \Xten{} design, special
syntactic support is provided for constrained types involving places.

All \Xten{} classes extend the class
\xcd"x10.lang.Object", which defines a property
\xcd"home" of type
\xcd"Place".

If a constrained reference type \xcd"T" has an \xcd"!p" suffix,
the constraint for \xcd"T" is implicitly assumed to contain the clause
\xcd"self.home==p"; that is,
\xcd"C{c}!p" is equivalent to \xcd"C{self.home==p && c}".

The place \xcd"p" may be ommitted. It defaults to \xcd"this" 
for types in field declarations, and to \xcd"here" elsewhere.


% The place specifier \xcd"any" specifies that the object can be
% located anywhere.  Thus, the location is unconstrained; that is,
% \xcd"C{c}!any" is equivalent to \xcd"C{c}".

% XXX ARRAY
%The place specifier \xcd"current" on an array base type
%specifies that an object with that type at point \xcd"p"
%in the array 
%is located at \xcd"dist(p)".  The \xcd"current" specifier can be
%used only with array types.


}

\section{Variables}\label{XtenVariables}\index{variables}

A variable of a reference data type {\tt reference R} where {\tt R} is
the name of an interface (possibly with type arguments) always holds a
reference to an instance of a class implementing the interface {\tt R}.

A variable of a reference data type {\tt R} where {\tt R} is the name
of a reference class (possibly with type arguments) always holds a
reference to an instance of the class {\tt R} or a class that is a
subclass of {\tt R}. 

A variable of a reference array data type {\tt R [D]} is always an
array which has as many variables as the size of the region underlying
the distribution {\cf D}. These variables are distributed across
places as specified by {\cf D} and have the type {\tt R}.

A variable of a nullary (reference or value) data type {\tt nullable
T} always holds either the value (named by) {\tt null} or a value of
type {\tt T} (these cases are not mutually exclusive).

A variable of a value data type {\tt value R} where {\tt R} is the
name of an interface (possibly with type arguments) always holds
either a reference to an instance of a class implementing {\tt R} or
an instance of a class implementing {\tt R}. No program can
distinguish between the two cases.

A variable of a value data type {\tt R} where {\tt R} is the name of a
value class always holds a reference to an instance of {\tt R} (or a
class that is a subclass of {\tt R}) or an instance of {\tt R} (or a
class that is a subclass of {\tt R}). No program can distinguish
between the two cases.

A variable of a value array data type {\tt V value [R]} is always an
array which has as many variables as the size of the region {\tt R}.
Each of these variables is immutable and has the type {\tt V}.

\Xten{} supports seven kinds of variables: final {\em class
variables} (static variables), {\em instance variables} (the instance
fields of a class), {\em array components}, {\em method parameters},
{\em constructor parameters}, {\em exception-handler parameters} and
{\em local variables}.

\subsection{Final variables}\label{FinalVariable}\index{variable!final}\index{final variable}
A final variable satisfies two conditions: 
\begin{itemize}
\item it can be assigned to at most once, 
\item it must be assigned to before use. 
\end{itemize}

\Xten{} follows \java{} language rules in this respect \cite[\S
4.5.4,8.3.1.2,16]{jls2}. Briefly, the compiler must undertake a
specific analysis to statically guarantee the two properties above.

\todo{Check if this analysis needs to be revisited.}

\subsection{Initial values of variables}
\label{NullaryConstructor}\index{nullary constructor}
\cbstart 
Every variable declared at a type must always contain a value of that type.

Every class variable must be initialized before it is read, through
the execution of an explicit initializer or a static block. Every
instance variable must be initialized before it is read, through the
execution of an explicit initializer or a constructor.  An instance
variable declared at a nullable type (and not declared to be {\tt
final}) is assumed to have an initializer which sets the value to {\tt
null}.

Each method and constructor parameter is initialized to the
corresponding argument value provided by the invoker of the method. An
exception-handling parameter is initialized to the object thrown by
the exception. A local variable must be explicitly given a value by
initialization or assignment, in a way that the compiler can verify
using the rules for definite assignment \cite[\S~16]{jls2}.

\cbend

\section{Objects}\label{XtenObjects}\index{Object}

An object is an instance of a scalar class or an array type.  It is
created by using a class instance creation expression
(\S~\ref{ClassCreation}) or an array creation
(\S~\ref{ArrayInitializer}) expression, such as an array
initializer. An object that is an instance of a reference (value) type
is called a {\em reference} ({\em value}) {\em object}.

All value and reference classes subclass from {\tt x10.lang.Object}.
This class has one {\tt const} field {\tt location} of type {\tt
x10.lang.place}. \index{place.location} Thus all objects in \Xten{}
are located (have a place). However, \Xten{} permits value objects to
be freely copied from place to place because they contain no mutable
state.  It is permissible for a read of the {\tt location} field of
such a value to always return {\tt here} (\S~\ref{Here});
therefore no space needs to be allocated in the object representation
for such a field.

In \XtenCurrVer{} a reference object stays resident at the place at
which it was created for its entire lifetime.

{}\Xten{} has no operation to dispose of a reference.  Instead the
collection of all objects across all places is globally garbage
collected.

{}\Xten{} objects do not have any synchronization information (e.g.{}
a lock) associated with them. Thus the methods on {\tt
java.lang.Object} for waiting/synchronizing/notification etc are not
available in \Xten. Instead the programmer should use atomic blocks
(\S~\ref{AtomicBlocks}) for mutual exclusion and clocks
(\S~\ref{XtenClocks}) for sequencing multiple parallel operations.

A reference object may have many references, stored in fields of
objects or components of arrays. A change to an object made through
one reference is visible through another reference. \Xten{} mandates
that all accesses to mutable objects shared between multiple
activities must occur in an atomic section (\S\ref{AtomicBlocks}).

\cbstart 
Note that the creation of a remote async activity
(\S~\ref{AsyncActivity}) {\cf A} at {\cf P} may cause the automatic creation of
references to remote objects at {\cf P}. (A reference to a remote
object is called a {\em remote object reference}, to a local object a
{\em local object reference}.)  For instance {\cf A} may be created
with a reference to an object at {\cf P} held in a variable referenced
by the statement in {\cf A}.  Similarly the return of a value by a
{\cf future} may cause the automatic creation of a remote object
reference, incurring some communication cost.  An {}\Xten{}
implementation should try to ensure that the creation of a second or
subsequent reference to the same remote object at a given place does
not incur any (additional) communication cost.

\cbend 

A reference to an object may carry with it the values of final fields
of the object. The implementation should try to ensure that the cost
of communicating the values of final fields of an object from the
place where it is hosted to any other place is not incurred more than
once for each target place.

{}\Xten{} does not have an operation (such as Pascal's ``dereference''
operation) which returns an object given a reference to the
object. Rather, most operations on object references are transparently
performed on the bound object, as indicated below. The operations on
objects and object references include:
\begin{itemize}

{}\item Field access (\S~\ref{FieldAccess}). An activity holding a
reference to a reference object may perform this operation only if the
object is local.  An activity holding a reference to a value object
may perform this operation regardless of the location of the object
(since value objects can be copied freely from place to place).  The
implementation should try to ensure that the cost of copying the field
from the place where the object was created to the referencing place
will be incurred at most once per referencing place, according to the
rule for final fields discussed above.

\item Method invocation (\S~\ref{MethodInvocation}).  An activity
holding a reference to a reference object may perform this operation
only if the object is local.  An activity holding a reference to a
value object may perform this operation regardless of the location of
the object (since value objects can be copied freely). The \Xten{}
implementation must attempt to ensure that the cost of copying enough
relevant state of the value object to enable this method invocation to
succeed is incurred at most once for each value object per place.

{}\item Casting (\S~\ref{ClassCast}).  An activity can perform this
operation on local or remote objects, and should not incur
communication costs (to bring over type information) more than once
per place.

{}\item {\cf instanceof} operator (\S~\ref{instanceOf}).  An activity
can perform this operation on local or remote objects, and should not
incur communication costs (to bring over type information) more than
once per place.

\item The stable equality operator {\cf ==} and {\cf !=}
(\S~\ref{StableEquality}). An activity can perform these operations on
local or remote objects, and should not incur communication costs
(to bring over relevant information) more than once per place.

% \item The ternary conditional operator {\cf ?:}
\end{itemize}

\section{Built-in types}
\cbstart 

The package {\tt x10.lang} provides a number of built-in class and
interface declarations that can be used to construct types.

\subsection{The class {\tt Object}}\label{Object}\index{Object}
The class {\cf x10.lang.Object} is a superclass of all other classes.
A variable of this type can hold a reference to an instance of any
scalar or array type.

\begin{x10}
package x10.lang;
public class Object \{
  public String toString() \{...\}
  public boolean equals(Object o) \{...\}
  public int hashCode() \{...\}
\}
\end{x10}

The method {\tt equals} and {\tt hashCode} are useful in hashtables,
and are defined as in \java. The default implementation of {\tt equals}
is stable equality, \S~\ref{StableEquality}. This method may be overridden
in a (value or reference) subclass.

\subsection{The class {\tt String}}
\Xten{} supports strings as in \java{}. A string object is immutable,
and has a concatenation operator ({\tt +})  available on it.

\subsection{Arithmetic classes}
Several value types are provided that encapsulate
abstractions (such as fixed point and floating point arithmetic)
commonly implemented in hardware by modern computers:

\begin{x10}
boolean byte short char 
int long 
double float 
\end{x10}

\XtenCurrVer{} defines these data types in the same way as the 
\java{} language. Specifically, a program may contain literals
that stand for specific instances of these classes. The syntax
for literals is the same as for \java{} (\S~ref{Literals}).

\futureext{
\Xten{} may provide mechansims in the future to permit the programmer
to specify how a specific value class is to be mapped to special
hardware operations (e.g.{} along the lines of
\cite{kava}). Similarly, mechanisms may be provided to permit the user
to specify new syntax for literals.
}
\subsection{Places, distributions, regions, clocks}
\Xten{} defines several other classes in the {\cf x10.lang}
package. Please consult the API documentation for more details.

\subsection{Java utility classes}
\XtenCurrVer{} programmers may import and use \java{} packages such as
{\tt java.util}, e.g.{} {\tt java.util.Set}, {\tt
java.lang.System}. \Xten{} programs should not invoke methods
that use the {\tt wait/notify/notifyAll} methods on such objects,
since this may interfere with \Xten{} synchronization. The
implementation does not make imported \java{} classes
automatically extend {\tt x10.lang.Object}. 

\futureext{
The above represents an {\em ad hoc} integration of \java{} libraries
into \Xten{}. It has the unfortunate consequence that not every run-time
value in an \Xten{} program execution is an instance of a subclass of
{\tt x10.lang.Object}. }

In the future a more principled and robust scheme will be worked
out. Such a scheme will need to attend to the integration of the
\java{} and \Xten{} type systems, and develop a notion of place for 
\java{} objects.

\cbend
\section{Conversions and Promotions}\label{XtenConversions}\label{XtenPromotions}\index{conversions}\index{promotions}

{}\XtenCurrVer{} supports \java's conversions and promotions
(identity, widening, narrowing, value set, assignment, method
invocation, string, casting conversions and numeric promotions)
appropriately modified to support \Xten's built-in numeric classes
rather than \java's primitive numeric types.

This decision may be revisited in future version of the language in
favor of a streamlined proposal for allowing user-defined
specification of conversions and promotions for value types, as part
of the syntax for user-defined operators.
