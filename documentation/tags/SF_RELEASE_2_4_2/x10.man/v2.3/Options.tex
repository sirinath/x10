\chapter{Options}

\subsection{Compiler Options}

The X10 compilers have many useful options. 

% -CHECK_INVARIANTS seems to check some internal compiler invariants.

\subsection{Optimization: {\tt -O} or {\tt -optimize}}

This flag causes the compiler to generate optimized code.


\subsection{Debugging: {\tt -DEBUG=boolean}}

This flag, if true, causes the compiler to generate debugging information.  It
is false by default.

\subsection{Call Style: {\tt -STATIC\_CHECKS, -VERBOSE\_CHECKS}}
\label{sect:Callstyle}
\index{STATIC\_CHECKS}
\index{VERBOSE\_CHECKS}
\index{dynamic checks}

By default, if a method call {\em could} be correct but is not {\em
necessarily} correct, the X10 compiler generates a dynamic check to ensure
that it is correct before it is performed.  For example, the following code: 
\begin{xten}
def use(n:Int{self == 0}) {}
def test(x:Int) { 
   use(x); // creates a dynamic cast
}
\end{xten}
compiles even though it is possible that 
\xcd`x!=0` when \xcd`use(x)` is called.  In this case, the compiler inserts a
cast, which has the effect of checking that the call is correct before it
happens: 
\begin{xten}
def use(n:Int{self == 0}) {}
def test(x:Int) { 
   use(x as Int{self == 0}); 
}
\end{xten}
The compiler produces a warning that it inserted some dynamic casts.
If you then want to see what it did, use \xcd`-VERBOSE_CHECKS`.

You may also turn on strict static checking, with the \xcd`-STATIC_CHECKS` flag.  With
static checking, calls that cannot be proved correct statically will be
marked as errors.  


\subsection{Help: {\tt -help} and {\tt -- -help}}

These options cause the compiler to print a list of all command-line options.


\subsection{Source Path: {\tt -sourcepath {\em path}}}

This option tells the compiler where to look for X10 source code.  


\subsection{(Deprecated) Class Path: {\tt -classpath {\em path}}}

This option is used in conjunction with the Java interoperability
feature to tell the compiler where to look for Java .class files that
may be used by the X10 code being compiled.

\subsection{Output Directory: {\tt -d {\em directory}}}

This option tells the compiler to produce its output files in the specified directory.

\subsection{Runtime {\tt -x10rt {\em impl}}}

This option tells which runtime implementation to use.  The choices are
\xcd`sockets`, \xcd`standalone`, \xcd`pami`, \xcd`mpi`, and \xcd`bgas_bgp`.

\subsection{Executable File {\tt -o {\em path}}}

This option tells the compiler what path to use for the executable file. 

\section{Execution Options: Java}

The Java execution command \xcd`x10` has a number of options as well. 

\subsection{Class Path: {\tt -classpath {\em path}}}

This option specifies the search path for class files. 

\subsection{Library Path: {\tt -libpath {\em path}}}

This option specifies the search path for native libraries.

\subsection{Heap Size: {\tt -mx {\em size}}}

Sets the maximum size of the heap. 

\subsection{Help: {\tt -h}}

Prints a listing of all execution options.



%\subsection{{\tt }}


\section{Running X10}

An X10 application is launched either by a direct invocation of the generated
executable or using a launcher command. The specification of the number of
places and the mapping from places to hosts is transport specific and
discussed in \Sref{sect:RunningManaged} for Managed X10 (Java back end) and
\Sref{sect:RunningNative} for Native X10 (C++ back end). For distributed runs,
the x10 distribution (libraries) and the compiled application code (binary or
bytecode) are expected to be available at the same paths on all the nodes.  

Detailed, up-to-date documentation may be found at
\url{http://x10-lang.org/documentation/practical-x10-programming/x10rt-implementations.html}


\section{Managed X10}
\label{sect:RunningManaged}


Managed X10 applications are launched using the x10 script followed by the qualified name of the main class.

\begin{xten}
x10c HelloWholeWorld.x10
x10 HelloWholeWorld
\end{xten}

The main purpose of the x10 script is to set the jvm classpath and the
\xcd`java.library.path` system property to ensure the x10 libraries are on the
path.  


\section{Native X10}
\label{sect:RunningNative}

On most platforms and for most transports, X10 applications can be launched by invoking the generated executable.

\begin{xten}
x10c++ -o HelloWholeWorld HelloWholeWorld.x10
./HelloWholeWorld
\end{xten}

On cygwin, X10 applications must be launched using the runx10 script followed by the name of the generated executable.

\begin{xten}
x10c++ -o HelloWholeWorld HelloWholeWorld.x10
runx10 HelloWholeWorld
\end{xten}
The purpose of the runx10 script is to ensure the x10 libraries are on the path. 
