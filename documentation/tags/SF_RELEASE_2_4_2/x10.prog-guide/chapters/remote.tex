
\chapter{There and Back Again: Computing with {\tt at}}

We've seen how an activity may shift execution to another {\tt Place}
temporarily by executing an ``{\tt at}'' statement or expression.  Such a shift
is potentially expensive, because it requires, at a minimum,
\begin{quote}
a message to the remote {\tt Place} to kick off the execution there;

as part of that kick-off, copying to the remote {\tt Place} all data used in the
operation;

a message from the remote {\tt Place} back home when the remote operation is
complete; and

if a value was being computed, it must be copied back.
\end{quote}

The moral is that {\tt at} must be used with care to minimize these costs,
particularly any unnecessary copying of data.  That is one of the two things
this chapter is about.  The other is the question of how
to pass to a remote {\tt Place} a reference to some object so that the
remote {\tt Place}'s computation can modify the object.

\section{Hidden Treasure: Unexpected Copies}\label{sec:hidden-treasure}

The first thing you need to do is to be aware of what {\em actually} gets
copied, as opposed to what you see named explicitly in the body of the {\tt at}.
Of course, there are simple situations where what you see is what you get:
\begin{verbatim}
    val a = 123;
    val b = 456;
    val c =  at(someOtherPlace) a + b;
\end{verbatim}
No mysteries here: {\tt a} and {\tt b} are {\tt Ints}, 4 bytes apiece, and those
8 bytes get copied.  Things get a little nastier if you have arrays:

\begin{verbatim}
   val bigArray = new Array[Int](HUGE_INT);
   // code here to initialize bigArray...
   for(p in Place.places()) at (p) doYourSlice(bigArray);
\end{verbatim}
Here every {\tt Place} gets to work with some slice of {\tt bigArray}. But what
gets copied?  All of {\tt bigArray} to each {\tt Place}: 4 bytes per {\tt Int}
times {\tt HUGE\_INT}.  Here's a simple example you can run to see that we're
telling you the truth:
%%START X10: src/remote/ArrayCopy.x10 arraycopy
\fromfile{ArrayCopy.x10}
\begin{xtennum}[]
public class ArrayCopy {
   public static def main(args: Array[String](1)) {
      val a = [1,2,3];
      Console.OUT.println("initial a is "+a);
      at(here.next()) {
         Console.OUT.println("in at, a is "+a);
         a(1) = 4;                              
         Console.OUT.println("after assignment in at, a is "+a); 
      }
      Console.OUT.println("back home, a is "+a); 
   }
}
\end{xtennum}
