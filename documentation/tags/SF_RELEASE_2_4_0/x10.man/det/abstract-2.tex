%\title{Sequential semantics for concurrent programs}

The safe parallelization of sequential programs is an important and
challenging task. A parallelization is safe if it ensures that no new
behaviors are introduced; the semantics of the parallel program is
identical to that of the sequential program. In particular, the
parallel program must not exhibit scheduler indeterminacy or deadlock.

This is a challenging problem.  Imperative memory locations are
famously indeterminate under concurrent read/write
operations. Further, most concurrent programming languages support
synchronization mechanisms that enable a thread to wait until some
appropriate condition is true, thus leading to the possibility of
deadlock.

In this paper we develop a very rich fragment of the concurrent,
imperative programming language \Xten{} in which {\em all} programs
have sequential semantics. The advantages of such a language are
striking: the programmer simply cannot introduce indeterminacy or
deadlock-related errors. All programs can be debugged as if they are
sequential programs, without having to reason about all possible
interleaving of threads.

The fragment includes all of \Xten{}'s most powerful constructs --
(clocked) \code{finish} and \code{async}, places and {}\code{at}. We
introduce two new abstractions -- {\em accumulators} and {\em clocked
  values} -- with lightweight compiler and run-time
support. Accumulators of type \code{T} permit multiple concurrent
writes, these are reduced into a single value by a user-specified
reduction operator. Accumulators are designed to be determinate and
deadlock-free at run-time, under all possible uses.  Clocked values of
type \code{T} operate on two values of \code{T} (the {\em current} and
the {\em next}). Read operations are performed on the current value,
write operations on the next. Such values are implicitly associated
with a clock and the current and next values are switched
(determinately) on quiescence of the clock. Clocked values capture the
common ``double buffering'' idiom. Clocked types are designed in such
a way that they are safe even in the presence of arbitrary aliasing
and patterns of concurrency. Non-accumulator clocked types require the
compiler to implement a lightweight ``owner computes'' rule to
establish disjointness of writes within a phase.

We show all programs written in this fragment of the language are
guaranteed to have sequential semantics. That is, programs in this
fragment may be developed and debugged as if they are sequential,
without running into the state explosion problem.

Further, many common patterns of concurrent execution (particularly
array-based computations) can be written in this language in an
``obvious'' way. These patterns include histograms, all-to-all
reductions, stencil computations, master/slave execution.  We show
that there are simple statically-checkable rules that can establish
for many common idioms that some run-time synchronization checks can
be avoided.

The key technical problem addressed is unrestricted aliasing in the
heap.  We develop the idea of an {\em implicit ownership domain}
(first introduced in the semantics of clocks in \Xten{} in
\cite{vj-clock}) that newly created objects may be registered with
the current activity so that they can be operated upon only by this
activity and its newly spawned children activity subtree.  This
permits safety to be established locally, independent of the context
in which the code is being run.

% todo: Vijay wanted to mention that all previous work only do separation of space (checking there is no interference in space),
% however, using clocks, we also consider separation of time.
