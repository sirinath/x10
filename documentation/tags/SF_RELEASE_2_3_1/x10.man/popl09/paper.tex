\documentclass[preprint,nocopyrightspace,9pt]{sigplanconf}
%\documentclass{llncs}

\newif\iflncs
\lncsfalse

\usepackage{times-lite}
\usepackage{mathptm}
\usepackage{txtt}
\usepackage{stmaryrd}
\usepackage{code}
\usepackage{bcprules}
%\usepackage{ttquot}
\usepackage{amsmath}
\usepackage{amssymb}
\usepackage{afterpage}
\usepackage{balance}
\usepackage{floatflt}
\usepackage{defs}
\usepackage{utils}
\usepackage{graphicx}
\usepackage{xspace}
\usepackage{ifpdf}
\usepackage{listings}
\usepackage{x10}
\usepackage{color}

\ifpreprint
\newcommand\todo[1]{\textcolor{red}{#1}}
\else
\newcommand\todo[1]{}
\fi
\renewcommand\todo[1]{}

  %\hyphenpenalty=1000


\newif\ifsemantics
%\semanticsfalse
\semanticstrue

\hfuzz=1pt

\pagestyle{plain}


\ifpdf
\setlength{\pdfpagewidth}{8.5in}
\setlength{\pdfpageheight}{11in}
\fi

\newcommand\val{\mbox{\tt val}}
\newcommand\klass{\mbox{\sf class}}
\newcommand\var{\mbox{\tt var}}
\newcommand\self{\mbox{\tt self}}
\newcommand\this{\mbox{\tt this}}
\newcommand\new{\mbox{\tt new}}
%\newcommand\extends{\unlhd}
%\newcommand\super{\unrhd}
\newcommand\extends{\mathrel{\mbox{\tt \textlt=}}}
\newcommand\super{\mathrel{\mbox{\tt \textgt=}}}
\newcommand\return{\mbox{\tt return}}
\newcommand\true{\mbox{\tt true}}
\newcommand\false{\mbox{\tt false}}
\newcommand\Object{\mbox{\tt Object}}
\newcommand\as{\mbox{\tt as}}
\newcommand\fields{\mbox{\sf fields}}
\newcommand\methods{\mbox{\sf methods}}
\newcommand\type{\mbox{\tt type}}
\newcommand\mtype{\mbox{\sf mtype}}
\newcommand\field{\mbox{\sf field}}
\newcommand\CFJ{{\sf CFJ}\xspace}
\newcommand\FJ{{\sf FJ}\xspace}
\newcommand\Java{Java\xspace}

\newcommand\Xten{{\sf X10}\xspace}
\newcommand\csharp{C$^{\sharp}$\xspace}
\newcommand\hmx{$\mathrm{HM}(X)$\xspace}
\newcommand\clpx{$\mathrm{CLP}(X)$\xspace}

\newcommand\xbar[1]{\ensuremath{\bar{\Xcd{#1}}}}
\newcommand\tbar[1]{\ensuremath{\bar{\tt {#1}}}}
\newcommand\exc[2]{\ensuremath{\exists}#1.~#2}
\newcommand\exty[3]{\ensuremath{\exists}#1\ty#2.~#3}
\newcommand\extyty[5]{\ensuremath{\exists}#1\ty#2,#3\ty#4.~#5}
\newcommand\extytyty[7]{\ensuremath{\exists}#1\ty#2,#3\ty#4,#5\ty#6.~#7}
\def\inv{\mathit{inv}}

\def\FGJ{{\sf FGJ}\xspace}

\newcommand\FXGL[1]{{\sf FXG(${\cal #1}$)}}
\def\FXGZ{\FXGL{\cdot}}
\def\FXGD{\FXGL{Q}}
\def\FXG{{\sf FXG}\xspace}

\def\has{\mbox{\tt has}}
\def\TConstr{\mbox{\sc T-Constr}}
\def\TInv{\mbox{\sc T-Inv}}
\def\TVar{\mbox{\sc T-Var}}
\def\TField{\mbox{\sc T-Field}}
\def\TInvk{\mbox{\sc T-Invk}}
\def\TNew{\mbox{\sc T-New}}
\def\TCast{\mbox{\sc T-Cast}}
\def\TUCast{\mbox{\sc T-UCast}}
\def\TDCast{\mbox{\sc T-DCast}}
\def\TSCast{\mbox{\sc T-SCast}}

\def\RField{\mbox{\sc R-Field}}
\def\RCField{\mbox{\sc RC-Field}}
\def\RInvk{\mbox{\sc R-Invk}}
\def\RCInvkRecv{\mbox{\sc RC-Invk-Recv}}
\def\RCInvkArg{\mbox{\sc RC-Invk-Arg}}
\def\RCNewArg{\mbox{\sc RC-New-Arg}}
\def\RCast{\mbox{\sc R-Cast}}
\def\RCCast{\mbox{\sc RC-Cast}}
\def\uhas{\underline{\has}}

\DeclareTextCommand{\textbraceleft}{OT1}{\texttt{\symbol{`\{}}}
\DeclareTextCommand{\textbraceright}{OT1}{\texttt{\symbol{`\}}}}

% \input{../../../../vj/res/pagesizes}
% \input{../../../../vj/res/vijay-macros}
\newcommand\alt{\bnf}

\newcommand\Implies{\Rightarrow}

\iflncs
\else
\newtheorem{example}{Example}[section]
\newtheorem{theorem}{Theorem}[section]
\newtheorem{lemma}[theorem]{Lemma}
\newtheorem{definition}[theorem]{Definition}
\newenvironment{proof}{
\trivlist
\item[\hskip \labelsep \textsc{Proof.}]
\selectfont
\ignorespaces}{$\Box$}

%\newtheorem{proof}[theorem]{Proof}
\fi

\begin{document}

\title{Constrained Kinds}

\authorinfo{Nathaniel Nystrom}{University of Texas, Arlington}
  {nystrom@uta.edu}

\eat{
\authorinfo{Olivier Tardieu}{IBM Research}
  {tardieu@us.ibm.com}
        
\authorinfo{Igor Peshansky}{IBM Research}
  {igorp@us.ibm.com}

\authorinfo{Vijay Saraswat}{IBM Research}
  {vsaraswa@us.ibm.com}
  }

\authorinfo{Olivier Tardieu \and Igor Peshansky \and Vijay
Saraswat}{IBM T.J.~Watson Research Center}
  {\{tardieu,igorp,vsaraswa\}@us.ibm.com}

% \conferenceinfo{POPL'08}{XXX}
% \copyrightyear{2008}
% \copyrightdata{[to be supplied]}

\maketitle

\begin{abstract}
Modern object-oriented languages such as \Xten require a rich framework
for types capable of expressing value-dependency, type-dependency
and supporting pluggable, application-specific extensions.

In earlier work, we presented the framework of \emph{constrained
types} for concurrent, object-oriented languages, parametrized by
an underlying constraint system $\cal X$. Constraint systems are a
very expressive framework for partial information. Types are viewed
as formulas \Xcd{C\{c\}} where \Xcd{C} is the name of a class
or an interface and \Xcd{c} is a constraint in $\cal X$ on the
immutable instance state of \Xcd{C} (the \emph{properties}).
Many (value-)dependent type systems for object-oriented languages
can be viewed as constrained types.

This paper extends the constrained types approach to handle
\emph{type-dependency} (``genericity''). The key idea is to extend
the constraint system to include predicates over types such as
\Xcd{X} is a subtype of \Xcd{T}.  Generic types are supported
by introducing type parameters and permitting programs to impose
constraints on such parameters.

To illustrate the underlying theory, we develop a formal family of
programming languages with a common set of sound type-checking rules
parameterized on a constraint system $\cal X$.  By varying $\cal X$
and extending the type system, we obtain languages with the power
of \FJ, \FGJ, and languages that provide dependent types, structural
subtyping, and constraints that relate values and types.  The core
of the \Xten language is a concrete instantiation of the framework.  We
describe the design and implementation of \Xten, which is available
for download at \texttt{x10-lang.org}.


\end{abstract}


\section{Introduction}
\label{sec:intro}
%
\section{Introduction}
\label{s:intr}

 Graph theoretic problems arise in several traditional and emerging scientific disciplines such as VLSI design, optimization, databases, and computational biology. There are plenty of theoretically fast parallel algorithms, for example, work-time optimal PRAM algorithms, for graph problems; however, in
 practice few parallel implementations beat the best sequential implementations for arbitrary, sparse
 graphs. The mismatch between theory and practice suggests a large gap between algorithmic model and the actual architecture. We observe that the gap is increasing as new diversified architectures emerge. Elegant solutions seem hard to come by from even combined efforts of algorithmic and architectural improvement. What is lacking is an effient way of mapping fine-grained parallelism expressed by the algorithm to target architectures with good performance. X10 is a new parallel programming language that provides expressive programming constructs and efficient runtime support that effectively helps reduce the gap between theory and practice in solving graph problems. In this paper we show that with X10 the fine-grained parallelism for a graph problem can be expressed much easier at a high algorithmic level, and the X10 program, compared with native C implementation, is much simpler and more elegant, and achieves comparable, and sometimes, even better performance. 

 The challenges of solving large-scale graph problems on current and emerging systems come from the irregular and combinatorial nature of the problem. Many of the important real world graphs, for example, internet topology, social interaction network, transfortation network, protein-protein interaction network, and etc., exhibit a ``small-world'' nature, and can be modeled as the so-called ``scale-free'' graph. There is no known efficient technique to partion such graph, which makes it hard to solve on distributed-memory systems. Also compared with the well-known sequential algorithms, for example, depth-first search (DFS) or breadth-first search (BFS) for the spanning tree problem, the parallel graph algorithms take exotic approaches such as ``graft-and-shortcut''. In the absence of efficient scheduling support of parallel activities, fine-grained parallelism incurs large overhead on current systems and oftentimes do not show practical parallel performance advantage. Lastly, graph algorithms tend to be load/store intensive compared with other scientific problems. For example,  They put great pressure on the memory subsystem. The problem obviously gets worse on distributed-memory architectures if necessary task management and memory affinity scheduling are not provided.  
 
 There are several features of X10 that make it extremely helpful in soving large-scale graph problems. X10 provides a shared virtual address space that obviates the need to partition a graph and issue message passing commands explicitly to access remote data. The irregular nature of the graph is also the reason why no SSCA benchmark has been implemented in MPI. X10 provides a wide range of constructs that are de. X10 has a lot of balancing.

 Our target architecure is a cluster of symmetric multiprocessor(SMP) nodes. Each SMP node may further comprise of chip multiprocessors (CMPs).  SMPs and CMPs are becoming very powerful and common place. Most of the high performance
computers are clusters of SMPs and/or CMPs. It is important to solve for them.
It is important to show flexibility but also good support of  PRAM algorithms for graph problems can be emulated much easier and more. 

The problem we consider if the spanning tree problem. It is notoriously hard to achieve good parallel performance.  Several good ones, we show X10 support that can do better. 

 The rest of the paper is organized as follows. Sections~\ref{s:design} describes algorithm design with the X10 language.
 Section~\ref{s:runtime} presents the workstealing runtime support for load-balancing in X10, and compare with other runtime systems, for example, CILK. 
 Section~\ref{s:results} provides our experimental results on current main-stream SMPs.
 In Section~\ref{s:concl} we conclude and give future work. 
 Throughout the paper, we
 use $n$ and $m$ to denote the number of vertices and the number of
 edges of an input graph $G=(V,E)$, respectively. 
  




%
%I. Introduction and overview (3 pages)

%Design of X10, concurrency, high productivity, high performance,
%practical language.
%
%Dependent types arise naturally. arrays, regions, distributions, place
%types.
%
%Indeed you can look around and recognize many OO type systems proposed
%in the last decade or so as specific kinds of applied dependent type
%systems.
%
%Our goal is to develop a general framework for dependent types for
%statically typed OO languages ("Java-like languages"). 

\Xten{} is a modern statically typed object-oriented
language designed for high productivity in the high performance
computing (HPC) domain~\cite{X10}. Built essentially on
sequential imperative object-oriented core
similar to
Scala or
$\mbox{Java}^{\mbox{\scriptsize\sc tm}}$,
\Xten{} introduces constructs for distribution and
fine-grained concurrency (asynchrony, atomicity, ordering).

The design of \Xten{} requires a rich type system to permit a
large variety of errors to be ruled out at compile time and to 
generate efficient code.  
\Xten{}, like most object-oriented languages supports classes;
however, it places
equal emphasis on {\em arrays}, a central data structure in high
performance computing.
In particular, \Xten{} supports dense,
distributed multi-dimensional arrays of value and reference types,
built over index sets known as {\em regions}.%, and mappings from index
%sets to places, known as {\em distributions}.  \Xten{} supports a rich
%algebra of operations over regions, distributions and arrays.

A key goal of \Xten{} is to rule out large classes of error
by design. For instance, the possibility of indexing a 2-d array with 3-d
points should simply be ruled out at compile-time. This means that one
must permit the programmer to express types such as \xcd{Region(2)},
the type of all two-dimensional regions;
\xcd{Array[int](5)}, the
type of all arrays of \xcd{int} of length \xcd{5};
\xcd{Array[int](Region(2))}, the type of all \xcd{int} arrays
over two-dimensional regions; and
\xcd{Tree\{loc==here\}}, the type of all \xcd{Tree} objects located on the
current node. For concurrent computations, one needs the ability to
statically check that a method is being invoked by an activity that is
registered with a given clock (i.e., dynamic barrier)~\cite{X10}.

For performance, it is necessary that array index accesses are
bounds-checked statically.  Further, certain regions (e.g.,
rectangular regions) may be represented particularly
efficiently.  Hence, if a variable is to range only over
rectangular regions, it is important that this information is
conveyed through the type system to the code generator.

In this paper we describe {\Xten}'s support for {\em
constrained types},
a form of {\em dependent
type}~\cite{dependent-types,xi99dependent,ocrz-ecoop03,aspinall-attapl,cayenne,epigram-matter,calc-constructions}---types parametrized by values---defined 
on predicates over the {\em immutable}
state of objects. Constrained types statically capture many common invariants
that naturally arise in code. For instance, typically the shape of an
array (the number of dimensions (the rank) and the size of each dimension)
is determined at
run time, but is fixed once the array is constructed. Thus, the shape of an
array is part of its immutable state.
Both mutable and immutable variables may have a constrained
type: the constraint specifies invariants on the immutable state
of the object stored in the variable. 

\Xten{} provides a framework for specifying and checking constrained types
that achieves certain desirable properties:
\begin{itemize}
\item 
{\bf Ease of use.}  
The syntax of constrained types is a simple and
natural extension of nominal class types.
% Constrained types in
% \Xten{} interoperate smoothly with Java libraries.

\item
{\bf Flexibility.}
The framework
permits the development of concrete,
specific type systems tailored to the application area at
hand.  \Xten{}'s compiler permits extension with different constraint systems
via compiler plugins, enabling a kind of pluggable type system~\cite{bracha04-pluggable}.
The framework is parametric in the kinds of
expressions used in the type system, permitting the installed constraint
system to interpret the constraints.

\item
{\bf Modularity.}
The rules for type-checking
are specified once in a way that is independent of the
particular vocabulary of operations used in the dependent type
system.
The type system supports separate compilation.

\item
{\bf Static checking.}  The framework permits mostly static
type-checking. The user is able to escape the confines of
static type-checking using dynamic casts.
%, as is common for Java-like
%languages.
\end{itemize}

\subsection{Constrained types}

\begin{figure}[t]
{
\footnotesize
\begin{xtenlines}
class List(n: int{self >= 0}) {
  var head: Object = null;
  var tail: List(n-1) = null;

  def this(): List(0) { property(0); }

  def this(head: Object, tail: List): List(tail.n+1) {
    property(tail.n+1);
    this.head = head;
    this.tail = tail;
  }

  def append(arg: List): List(n+arg.n) {
    return n==0
      ? arg : new List(head, tail.append(arg));
  }

  def reverse(): List(n) = rev(new List());
  def rev(acc: List): List(n+acc.n) {
    return n==0
      ? acc : tail.rev(new List(head, acc));
  }

  def filter(f: Predicate): List{self.n <= this.n} {
    if (n==0) return this;
    val l: List{self.n <= this.n-1} = tail.filter(f);
    return (f.isTrue(head)) ? new List(head,l) : l;
  }
}
\end{xtenlines}
}
\caption{
This program implements a mutable list of Objects. The size of a list
does not change through its lifetime, even though at different points
in time its head and tail might point to different structures.}
\label{fig:list-example}
\end{figure}

X10's sequential syntax is similar to Scala's.
We permit the definition of a class \xcd{C} to specify
a list of typed parameters or {\em properties},
${\tt f}_1: {\tt T}_1, \dots, {\tt f}_k: {\tt T}_k$,
similar in syntactic structure to a method formal parameter list.
%
Each property in this list is treated as a public final instance field.
%
We also permit the
specification of a {\em class invariant}
%, a {\em where clause}~\cite{where-clauses}
in the class definition. A class invariant
is a boolean expression on the properties of the class.
The compiler ensures that all
instances of the class created at run time satisfy the invariant.
%
Syntactically, the class invariant follows the property list.
%
For instance, we may specify a class \xcd{List} with an
\xcd{int} \xcd{length} property as follows:
\begin{xtennoindent}
  class List(length: int){length >= 0} {...}
\end{xtennoindent}
Given such a definition for a class \xcd{C}, types can be
constructed by {\em constraining} the properties of \xcd{C}.  In
principle, {\em any} boolean expression over the properties
specifies a type: the type of all instances of the class
satisfying the boolean expression. Thus, \xcd{List\{length == 3\}}
is a permissible type, as are \xcd{List\{length <= 42\}} and
even \xcd{List\{length * f() >= g()\}} where \xcd{f} and \xcd{g}
are functions on the immutable state of the \xcd{List} object
and the variables in scope where the type appears.
In practice, the constraint expression is restricted by the
particular constraint system in use.

Our basic approach to introducing constrained types into \Xten{}
is to follow the spirit of generic types, but to use values
instead of types.

In general, a {\em constrained type} is of the form \xcd{C\{e\}},
the name of a class or interface\footnote{In \Xten{}, primitive
types such as \xcd{int} and \xcd{double} are object types; thus,
for example, \xcd{int\{self==0\}} is a legal constrained type.}
\xcd{C}, called the {\em base class}, followed
by a {\em condition} \xcd{e},
a boolean expression on the properties of the
base class and the final variables in scope at the type.
Such a type represents a refinement of \xcd{C}: the set of all
instances of \xcd{C} whose immutable state satisfies the
condition \xcd{e}.
%
We write \xcd{C} for 
the vacuously constrained type \xcd{C\{true\}}, and
write
\tcd{C(${\tt e}_1,\ldots,{\tt e}_k$)} for
the type
\tcd{C\{${\tt f}_1$==${\tt e}_1,\ldots,{\tt f}_k$==${\tt e}_k$\}}
where \xcd{C} declares the $k$ properties
${\tt f}_1,\ldots,{\tt f}_k$.

Constrained types may occur wherever normal types occur. In
particular, they may be used to specify the types of properties,
(possibly mutable) local variables or fields,
arguments to methods, return types of methods; they may also be
used in casts, etc.

Using the definitions above, \xcd{List(n)}, shown in
Figure~\ref{fig:list-example}, is the type of all lists of
length \xcd{n}.
%
Intuitively, this definition states that a \xcd{List} has an \xcd{int}
property \xcd{n}, which must be non-negative.
The properties
of the
class are set through the invocation of \xcd{property}\tcd{(\ldots)}
(analogously to \xcd{super}\tcd{(\ldots)}) in the constructors
of the class.

In a constraint, the name \xcd{self} is bound and refers to the type being
constrained.  The name \xcd{this}, by contrast, is a free
variable in the
constraint and refers to the receiver parameter of the current
method or constructor.  Use of \xcd{this} is not permitted in static
methods.

The \xcd{List} class has two
fields that hold the head and tail of the list.  The fields are
declared with the \xcd{var} keyword, indicating that they are
not final.  Variables declared with the \xcd{val} keyword, or
without a keyword are final.

Constructors have ``return
types'' that can specify an invariant satisfied by the object being
constructed.  The compiler verifies that the
constructor return type and the class invariant are implied by the
\xcd{property} statement and any \xcd{super} calls in the constructor
body.
A constructor must either invoke another constructor of the same
class via a
\xcd{this} call
or must have a \xcd{property} statement on every
non-exceptional path
to ensure the properties are initialized.
The \xcd{List} class has two constructors: the first
constructor returns an empty list;
the second
returns a list of length \xcd{m+1}, where \xcd{m} is the length
of the second argument. 

In the second constructor (lines 7--11), as well as 
the \xcd{append} (line 13) and \xcd{rev} (line 20) methods,
the return type
depends on properties of the formal parameters. 
If an argument appears in a
return type then the parameter must be final,
ensuring the
argument points to the same object throughout the evaluation of
the method or constructor body.  A parameter may also depend on
another parameter in the argument list.

The use of constraints makes existential types very natural.
Consider the return type of \xcd{filter} (line 24): it specifies
that the list returned is of some unknown length. The only thing
known about it is that its size is bounded by \tcd{n}.
Thus,
constrained types naturally subsume existential dependent types.
Indeed, every base type \xcd{C} is an ``existential''
constrained type since it does not specify any constraint on its
properties. Thus, code written with constrained types can
interact seamlessly with legacy library code---using just base
types wherever appropriate.

The return type of \xcd{filter} also illustrates the difference
between \xcd{self} and \xcd{this}.  Here, \xcd{self} refers to
the \xcd{List} being returned by the method; \xcd{this} refers
to the method's receiver.

\subsection{Constraint system plugins}

The \Xten{} compiler allows  
programmers to extend the semantics of the language with
compiler plugins.  Plugins may be used to support different constraint
systems to be used in constrained types.
Constraint systems provide code for checking consistency and
entailment.

The condition of a constrained type is parsed and type-checked
as a normal boolean expression over properties and
the \xcd{final} variables in scope at the type.  Installed
constraint systems translate the expression into an internal
form, rejecting expressions that cannot be represented.
%
A given condition may be a conjunction of constraints from
multiple constraint systems.
A Nelson--Oppen procedure~\cite{nelson-oppen} is used to check
consistency of the constraints.

The \Xten{} compiler
implements a simple
equality-based constraint system.  Constraint solver plugins
have been implemented for inequality constraints, for
Presburger constraints using
the CVC3 theorem prover~\cite{cvc}, and for
set-based constraints also using CVC3.
These constraint systems are described in
Section~\ref{sec:examples} and the implementation is
discussed in Section~\ref{sec:impl}.

\subsection{Claims}

The paper presents constrained types in the \Xten{} programming
language.
We claim that the design is natural, easy to use, and useful. Many
example programs have been written using constrained types and are
available at {\tt x10.sf.net/\allowbreak applications/\allowbreak examples}.

As in staged languages~\cite{nielson-multistage,ts97-multistage}, the
design distinguishes between compile-time and run-time
evaluation. Constrained types are checked (mostly) at compile-time.
The compiler uses a constraint solver to perform universal reasoning
(e.g., ``for all possible values of method parameters'') for dependent
type-checking.  There is no run-time constraint-solving.  However,
run-time casts and \xcd{instanceof} checks involving dependent types
are permitted; these tests involve
arithmetic, not algebra---the values of all parameters are known.

The design supports separate compilation: a class needs to be
recompiled only when it is modified or when the method
and field signatures or invariants of classes on which it
depends are modified.

We claim that the design is flexible. The language design is
parametric on the constraint system being used.
%We are planning on extending the current
%implementation to support multiple user-defined constraint systems,
%thereby supporting pluggable types.
The compiler supports
integration of
different constraint solvers into the language.
Dependent clauses  also form
the basis of a general user-definable annotation framework we have
implemented separately~\cite{ns07-x10anno}. 

We claim the design is clean and modular. We present a simple core
language \CFJ, extending \FJ{}~\cite{FJ} with constrained types on top
of an arbitrary constraint system. We present rules for type-checking
\CFJ{} programs that are parametric in the constraint system
and establish subject reduction and progress theorems. 

%
% XXX contrast with hybrid type checking.

\paragraph{Rest of this paper.}

Section~\ref{sec:lang} describes the syntax and semantics of
constrained types.
Section~\ref{sec:examples} works through a number of
examples using a variety of constraint systems.
The compiler implementation, including support for constraint
system plugins, is described Section~\ref{sec:impl}.
A formal semantics for a core language with constrained types 
is presented in Section~\ref{sec:semantics}, and a soundness
proof is presented in the appendix.
Section~\ref{sec:related} reviews related work.
The paper concludes in Section~\ref{sec:future}
with a discussion of future work.




\section{Design considerations}
\label{sec:lang}
%
\section{Introduction}
\label{s:intr}

 Graph theoretic problems arise in several traditional and emerging scientific disciplines such as VLSI design, optimization, databases, and computational biology. There are plenty of theoretically fast parallel algorithms, for example, work-time optimal PRAM algorithms, for graph problems; however, in
 practice few parallel implementations beat the best sequential implementations for arbitrary, sparse
 graphs. The mismatch between theory and practice suggests a large gap between algorithmic model and the actual architecture. We observe that the gap is increasing as new diversified architectures emerge. Elegant solutions seem hard to come by from even combined efforts of algorithmic and architectural improvement. What is lacking is an effient way of mapping fine-grained parallelism expressed by the algorithm to target architectures with good performance. X10 is a new parallel programming language that provides expressive programming constructs and efficient runtime support that effectively helps reduce the gap between theory and practice in solving graph problems. In this paper we show that with X10 the fine-grained parallelism for a graph problem can be expressed much easier at a high algorithmic level, and the X10 program, compared with native C implementation, is much simpler and more elegant, and achieves comparable, and sometimes, even better performance. 

 The challenges of solving large-scale graph problems on current and emerging systems come from the irregular and combinatorial nature of the problem. Many of the important real world graphs, for example, internet topology, social interaction network, transfortation network, protein-protein interaction network, and etc., exhibit a ``small-world'' nature, and can be modeled as the so-called ``scale-free'' graph. There is no known efficient technique to partion such graph, which makes it hard to solve on distributed-memory systems. Also compared with the well-known sequential algorithms, for example, depth-first search (DFS) or breadth-first search (BFS) for the spanning tree problem, the parallel graph algorithms take exotic approaches such as ``graft-and-shortcut''. In the absence of efficient scheduling support of parallel activities, fine-grained parallelism incurs large overhead on current systems and oftentimes do not show practical parallel performance advantage. Lastly, graph algorithms tend to be load/store intensive compared with other scientific problems. For example,  They put great pressure on the memory subsystem. The problem obviously gets worse on distributed-memory architectures if necessary task management and memory affinity scheduling are not provided.  
 
 There are several features of X10 that make it extremely helpful in soving large-scale graph problems. X10 provides a shared virtual address space that obviates the need to partition a graph and issue message passing commands explicitly to access remote data. The irregular nature of the graph is also the reason why no SSCA benchmark has been implemented in MPI. X10 provides a wide range of constructs that are de. X10 has a lot of balancing.

 Our target architecure is a cluster of symmetric multiprocessor(SMP) nodes. Each SMP node may further comprise of chip multiprocessors (CMPs).  SMPs and CMPs are becoming very powerful and common place. Most of the high performance
computers are clusters of SMPs and/or CMPs. It is important to solve for them.
It is important to show flexibility but also good support of  PRAM algorithms for graph problems can be emulated much easier and more. 

The problem we consider if the spanning tree problem. It is notoriously hard to achieve good parallel performance.  Several good ones, we show X10 support that can do better. 

 The rest of the paper is organized as follows. Sections~\ref{s:design} describes algorithm design with the X10 language.
 Section~\ref{s:runtime} presents the workstealing runtime support for load-balancing in X10, and compare with other runtime systems, for example, CILK. 
 Section~\ref{s:results} provides our experimental results on current main-stream SMPs.
 In Section~\ref{s:concl} we conclude and give future work. 
 Throughout the paper, we
 use $n$ and $m$ to denote the number of vertices and the number of
 edges of an input graph $G=(V,E)$, respectively. 
  



\Xten{} is a class-based object-oriented language
that provides both dependent and generic types.
The language has a sequential core similar to Java or Scala, but
also
constructs for concurrency and distribution.

A key feature that interacts with generics is that the type
system provides both reference types and value types.
An instance of a reference type is an object on the
heap.  All reference types are subclasses of \xcd"Object".
Variables of reference type may be \xcd"null".
In contrast, an instance of a value type might be represented in
unboxed form on the call stack
and can never be \xcd"null".
Ideally, the design should support instantiation of generics on both 
reference and value types.

This section describes the design of generics for \Xten,
including several alternative designs.  These alternatives
demonstrate the expressiveness of constrained kinds.

%\footnote{We plan
%to support traits in a future version of the language.}
\todo{emph interactions with data constraints}

\subsection{Type constraints}

To permit genericity, variables \Xcd{X} must be admitted over types.
The choice of type variables is discussed below.  We assume here
that classes have a means of introducing new type variables
either as type parameters or as type members.
For instance,
the class \xcd"List" introduces a type variable \xcd"X"
representing the list's element type.

\Xten already supports constraints over values, so it is natural
to extend these to constraints over types.
Here, we ask: how should type variables be constrained?

Constraints occur in several places in the \Xten syntax.  They
are of course permitted in constrained types \xcd"C{c}".
Constraints may also be used as \emph{class invariants}, 
which are constraints on the class's properties and other
final variables in scope.
The class invariant must be established by the class's
constructor and subsequently holds for all instances of the
class.

Methods and constructors
may also have constraints, or \emph{guards}, on their parameters.
A guard must be satisfied by the caller of the method and
will hold throughout its body.  Type constraints used in the
method guard restrict the types of the arguments or of the method
receiver.  

\subsubsection{Nominal subtyping and equality constraints}

In class-based OO languages such as Java,
types are equipped with a partial
order (the \emph{subtyping} order) generated from the user program
through the ``\Xcd{extends}'' relationship.  
This motivates a very natural constraint system on types.  For a type
variable \Xcd{X} we should be able to assert the constraint
\Xcd{X}~$\extends$~\Xcd{T}: a valuation (mapping from variables to types) realizes
this constraint if it maps \Xcd{X} to a type that extends \Xcd{T}.
Constraints on types can specify either subtype (\xcd"<="),
supertype (\xcd">="), or equality bounds (\xcd"==").

Using subtyping constraints in the class invariant provides a
means to bound the type variables introduced by the class
declaration.  Constraints in constrained types
\xcd"C{c}", can bound
type-valued members of the base type \xcd"C".

These constraints also can be used in method guards.
This feature is similar to optional methods in CLU~\cite{clu} and to generalized type constraints in C$\sharp$~\cite{emir06}.
For instance, given a list of \xcd"T", one could define a
method \xcd"print" with a guard that requires that \xcd"T" be a
subtype of \xcd"Printable":
\begin{xtenmathnoindent}
  def print(){T <= Printable} {
    head.print();
    tail.print();
  }
\end{xtenmathnoindent}
This constraint ensures that the \xcd"head" field of type
\xcd"T" has a \xcd"print()" method.

\subsubsection{Structural constraints}

One should also
be able to require that a type have a
particular member---a field with a given name and type, or a method
with a given name and signature.
We introduce the constraints 
\Xcd{T} \Xcd{has} \Xcd{f:T} and \Xcd{T} \Xcd{has}
\Xcdmath{m($\tbar{x}\ty\tbar{S}$):T} to express this.
These
constraints allow one to define an alternative version of the
\xcd"print" methods above  as:
\begin{xtenmathnoindent}
  def print(){T has print(): void} {
    head.print();
    tail.print();
  }
\end{xtenmathnoindent}
Rather than restricting the actual receiver to lists whose
element type implements \xcd"Printable", with structural
constraints, any list whose element type has a \xcd"print"
method may be used.
\eat{
This feature makes it easier to integrate
third-party libraries, where interface names might not be
compatible.
}

Structural constraints on types are found in many languages.
For instance,
Haskell supports type
classes~\cite{haskell,haskell-type-classes}.
%
%ML's module system allows modules to be constrained by
%structural signatures~\cite{ml}.
In Modula-3, type equivalence is structural
rather than nominal as in object-oriented languages of the C
family (e.g., C++, Java, and \Xten{}).
Unity~\cite{malayeriIntegrating08}
is a Java-like language with both nominal and structural subtyping.

In the class invariant, a structural constraint can bound the
class's type variables, similarly to 
the language PolyJ~\cite{java-popl97}, which allows type
parameters to be
bounded using
structural \emph{where clauses}~\cite{where-clauses}.
For example, a sorted list class
could be written as follows in PolyJ:
{
\begin{xtennoindent}
class SortedList[T] where T {int compareTo(T)} {
  void add(T x) {... x.compareTo(y) ...}
  ...
}
\end{xtennoindent}}
The \xcd"where" clause states that the type parameter
\xcd"T" must have a
method \xcd"compareTo" with the given signature.
\xcd"SortedList" can be instantiated on any type
that provides the method.
With nominal bounds, \xcd"SortedList" could only require that
its parameters implement an interface such as \xcd"Comparable".

\subsubsection{Default values}

\Xten's type system provides both reference types and value
types.  An instance of a reference type is an object on the
heap.  Variables of reference type may be \xcd"null".
In contrast, an instance of a value type might live on the stack
and can never be \xcd"null".  In languages like Java with
primitive types, every type has a default value---\xcd"null" for
reference types and \xcd"false" or \xcd"0" for primitive
types---used to initialize arrays of that type.
In \Xten, some types do not have a default.  For example,
\xcd"int{self>0}" does not contain the value \xcd"0".
Consequently, a useful constraint is \xcd"T has default", which
holds if the type \xcd"T" has a default value.

\subsubsection{{\tt instanceof} constraints}

Lastly, we consider constraints of the form \xcd"x" \xcd"instanceof" \xcd"T".
By relating types and values in a single constraint, 
these constraints provide considerable expressive power.
For instance, 
consider the class declaration:
\begin{xtennoindent}
class C {
  def equals[T](x:T) {this instanceof T} = (this==x);
}
\end{xtennoindent}
The \xcd"equals" method can be called with any object
that is a supertype of \xcd"C".

\xcd"instanceof" constraints can be used to build intersection
types, e.g., \xcd"Object{self instanceof A,self instanceof B}"
should be a subtype of \xcd"A" and \xcd"B"..

\subsection{Type variables}
\label{sec:type-properties}
\label{sec:variance}

Languages such as Java~\cite{Java3} and
Scala~\cite{scala} introduce \emph{type parameters} on classes
and methods.
An alternative approach, used by BETA~\cite{beta}, 
Scala~\cite{scala}, and other languages is, to use type members.
In \Xten, one can
generalize properties to include type-valued properties:
A \emph{type property}
is a final object member initialized at construction time with a
concrete type.  

\subsubsection{Type properties}

\label{sec:usability}
\label{sec:parameters-vs-fields}

Like normal value properties, type properties
can be used in constrained types through the variable \xcd"self".
%
This immediately suggests use-site variance
constraints~\cite{unifying-genericity,variant-parametric-types}
on type properties.
The type of a list of integers, say, can be written as
\xcd"List{self.T==int}".  
Nominal subtyping constraints, then, may be used to
provide use-site variance constraints.
%
Consider the following subtypes of \xcd"List" with type property
\xcd"T":
\begin{itemize}
\item \xcd"List".  This type has no constraints on the type
property \xcd"T".
Any type that constrains \xcd"T"
is a subtype of \xcd"List".
\eat{
The type \xcd"List" is equivalent to \xcd"List{true}".
%
For a \xcd"List" \xcd"v", the return type of the \xcd"get" method
is \xcd"v.T".
Since the property \xcd"T" is unconstrained,
the caller can only assign the return value of \xcd"get"
to a variable of type \xcd"v.T".
}

\item \xcd"List{T==float}".
The type property \xcd"T" is bound to \xcd"float".
For a final expression \xcd"v" of this type,
\xcd"v.T" and \xcd"float" are equivalent types and can be used
interchangeably.

\item \xcdmath"List{T$\extends$Collection}".
This type constrains \xcd"T" to be a subtype of \xcd"Collection".
All instances of this type must bind \xcd"T" to a subtype of
\xcd"Collection"; for example \xcd"List[Set]" (i.e.,
\xcd"List{T==Set}") is a subtype of
\xcdmath"List{T$\extends$Collection}" because \xcd"T==Set" entails
\xcdmath"T$\extends$Collection".

\item \xcdmath"List{T$\super$String}".  This type bounds the type property
\xcd"T"
from below. 
\end{itemize}

While expressive,
type properties have a number of usability issues.
The key difference between type parameters and type properties
is that type properties are
instance \emph{members} bound during object construction.  Type
properties are thus accessible through expressions---\xcd"e.T" is
a legal type (if \xcd"e" is final)---and are inherited by subclasses.
These features give type properties more expressive power than
type parameters, as we shall describe below; however, because they 
provide similar functionality with often subtle distinctions,
type properties can be difficult to use, especially for novices,
and require more care in the design.
For instance,
since type properties are inherited,
the language design needs
to account for ambiguities introduced when the same name is
used for different type properties declared in or inherited into a class.

\eat{
Inheriting type properties may also lead to confusion
As an example, in the following hypothetical code extended with
type properties (declared as normal properties with the ``type''
\xcd"*"),
\xcd"HashMap"  inherits the properties \xcd"K" and \xcd"V" from
\xcd"AbstractMap".
\begin{xten}
class AbstractMap(K:*, V:*) {
  abstract def get(K): V;
  abstract def put(K, V): V;
}

class HashMap implements Map {
  def get(k: K): V = ...;
  def put(k: K, v: V): V = ...;
}
\end{xten}
A user more familiar with type parameters might declare
\xcd"HashMap" as follows:
\begin{xten}
class HashMap(K:*,V:*) implements Map(K,V) {
  def get(k: K): V = ...;
  def put(k: K, v: V): V = ...;
}
\end{xten}
This declaration would introduce a new pair of type properties
named \xcd"K" and
\xcd"V" that shadow the inherited properties.
A na{\"\i}ve implementation of type properties would store run-time
type information for all four properties in each instance
of \xcd"HashMap".
}

\paragraph{\normalfont\bf\em Virtual types.}

Type properties provide expressive power much like 
\emph{virtual
types}~\cite{beta,mp89-virtual-classes,ernst06-virtual};
moreover, they can also
be constrained at the use-site,
can be refined on a per-object basis without explicit subclassing,
and can be refined contravariantly as well as covariantly.

Thorup~\cite{thorup97}
proposed adding genericity to Java using virtual types.  For example,
a generic \xcd"List" class can be written as follows:
{
\begin{xten}
abstract class List {
  abstract typedef T;
  T get(int i) { ... }
}
\end{xten}}
\noindent
The virtual type \xcd"T" is unbound in \xcd"List", but 
can be refined by binding \xcd"T" in a subclass:
{
\begin{xten}
abstract class NumberList extends List {
  abstract typedef T as Number;
}
class IntList extends NumberList {
  final typedef T as Integer;
}
\end{xten}}
\noindent
Only classes where \xcd"T" is final bound, such as \xcd"IntList",
can be non-abstract.  Scala~\cite{scala} supports abstract types
and virtual types in a similar way.
%
The analogous definition of 
\xcd"List" in \Xten{} using type properties is as follows:
{
\begin{xten}
class List(T:*) {
  def get(i: int): T { ... }
}
\end{xten}}

\noindent
Unlike the virtual-type version,
the \Xten{} version of \xcd"List" is not abstract;
\xcd"T" need not be instantiated by a subclass because it can be
instantiated (constrained) on a per-object basis.
Rather than declaring subclasses of \xcd"List",
one uses the constrained subtypes
\xcdmath"List{T$\extends$Number}" and \xcd"List{T==Integer}".

Type properties can also be refined contravariantly.
For instance, one can write the type \xcdmath"List{T$\super$Integer}".

\paragraph{Self types.}

Type properties can also be used to support a form of self
type~\cite{bruce-binary,bsg95}.
%
Self types can be implemented by introducing a
type property \Xcd{type} to the root of the class hierarchy,
\Xcd{Object}:
\begin{xtenmath}
class Object(type:*){type <= Object} { $\dots$ }
\end{xtenmath}

\noindent
For any final path expression \Xcd{p}, the type
$\Xcd{p}.\Xcd{type}$ represents all instances of the fixed,
but statically unknown, run-time class referred to by \Xcd{p}.
Scala's path-dependent types~\cite{scala} and J\&'s
dependent classes~\cite{nqm06}
take a similar approach.

Self types are achieved by
constraining types so that if a path expression \Xcd{p}
has type \Xcd{C}, then
$\Xcd{p}.\Xcd{type} \subtype \Xcd{C}$.
In particular, one can add the class invariant
$\Xcd{this}.\Xcd{type} \subtype \Xcd{C}$ to every class \Xcd{C}.
This invariant ensures that
$\Xcd{this}.\Xcd{type}$ is a subtype
of the lexically enclosing class.

The property must be initialized to the given class, so, without
further language support, one must create an instance of
\xcd"Object" with \xcd"new" \xcd"Object(Object)" to initialize
the \xcd"type" property.

\subsubsection{Type parameters}

Most OO languages provide genericity through type parameters on
classes and methods.  The development of a nominal OO type
system with type parameters is now standard (cf.  FGJ~\cite{FJ}).

Scala~\cite{scala} supports definition-site variance
annotations:
a parameter may be declared in-, co-, or
contravariant.
If the parameter \xcd"X" of a class \xcd"C" is covariant,
then \xcd"S" a subtype of
\xcd"T" implies  \xcd"C[S]" is a subtype of \xcd"C[T]".
Similarly, if \xcd"X" is contravariant,
\xcd"C[T]" is a subtype of \xcd"C[S]".
Invariant parameters are the default; a covariant parameter is
declared by prepending ``\xcd"+"'' to the parameter name in the
class header; a contravariant parameter is declared by
prepending ``\xcd"-"''.  The usage of variant parameter types in
the body of their class must be
restricted to ensure the subtyping relation holds.

Java, by contrast, supports use-site variance through wildcards.
This has a number of usability problems~\cite{wildcards-are-evil},
which also occur with constrained type properties, above.

\subsection{Overloading and dispatch}

The next question to address is the overloading semantics for
methods with constraints on formal parameters and with method
guards.  This issue was considered in non-generic \Xten but was
revisited in light of type constraints.

One option is to ignore constraints when checking for
overloading.  This means that \xcd"m(int{self==0})" and
\xcd"m(int{self==1})", for instance, are considered to have the
same signature; if both occur within the same class, a
compile-time error occurs.

Another option is to allow the
overloading: methods are resolved at compile-time, based on the
constraints.  It is an error if a call could resolve to more
than one method.  One question is whether to rule out overlapping
methods (e.g., \xcd"m(int{self>=0})" and \xcd"m(int{self==1})"),
or to permit them and have the caller resolve any
ambiguities.

\todo{Use type constraints, not value constraints}

Allowing the overloading on constraints can also complicate method
overriding by introducing partial overrides.
Consider:
\begin{xtennoindent}
  class A {
    def m(x:int{self<=0}) = ...; // 1
    def m(x:int{self>=0}) = ...; // 2
  }
  class B {
    def m(x:int{self==0}) = ...; // 3
  }
\end{xtennoindent}
\noindent
\xcd"B.m"'s constraint on \xcd"x" partially overrides the
constraint on both \xcd"m" methods of \xcd"A".  Given a variable
\xcd"b" of type \xcd"B": \xcd"b.m(-1)" invokes \xcd"A.m" (method 1),
\xcd"b.m(0)" invokes \xcd"B.m" (method 3), and \xcd"b.m(1)"
invokes \xcd"A.m" (method 2).  Clients of \xcd"B" could get
confused about which method gets invoked.  One option is to
require that when a method with a given name is overridden, all
other methods with that name should be overridden as well.

Finally, one could support a form of predicate
dispatch~\cite{jpred}, selecting the method to invoke by
\emph{dynamically} evaluating the method guard.  Predicate
dispatch generalizes multi-method dispatch.  With type
constraints, multi-method dispatch can be simulated.
\todo{example}

\subsection{Implementation}

Finally, we turn to the implementation of generics.
To implement a generic class \xcd"C[X]" one can either generate a single 
class for \xcd"C" in the target language (homogeneous translation)
or generate one class per instantiation
\xcdmath"C[T$_1$]", \dots,
\xcdmath"C[T$_k$]" (heterogeneous translation).
The former approach reduces the amount of generated code; the
latter enables specialization based on the type arguments to
\xcd"C".  Hybrid approaches are possible as well.

Java's approach is to erase type parameters and to use the homogeneous
translation.  Erasure admits more dynamic errors because
it permits, for instance, a \xcd"C<A>" to be cast to \xcd"C<B>".
Retrieving a field of static type \xcd"B" could cause a run-time
type error when an \xcd"A" is returned instead.
The homogeneous translation is aided by a restriction that type
parameters cannot be instantiated on primitive types and by
using nominal subtyping bounds on types.
These restrictions ensure type parameters can be represented
with their type bound, or \xcd"Object" if unbounded.

PolyJ~\cite{java-popl97} supports structural bounds and uses a
homogeneous translation with adapter objects to allow generic
code to invoke methods on values of its type parameters.

Other languages, such as C++, use a heterogeneous
translation, specializing the generic class for each
instantiation.
C$\sharp$,
NextGen~\cite{nextgen}, and
Fortress~\cite{fortress} takes this approach as well, reducing
(static) code bloat by instantiating generic classes at run time.

A compromise approach is to specialize for only a few parameter
types, for example the primitive types, but to use a homogeneous
translation otherwise.

Representing type variables at run-time allows the language
to support run-time casts to generic types,
including possibly types instantiated on constrained types.

With
non-generic constrained types, casts like
\xcd"r"~\xcd"as"~\xcd"Region{rank==k}" can be implemented by
checking the run-time class of the value being
cast---\xcd"r"~\xcd"instanceof"~\xcd"Region"---and then
evaluating the constraint---\xcd"r.rank==k".
%
However, the issue is more subtle with generic casts.
For instance, to implement
\xcd"a"~\xcd"as"~\xcd"Array[int{self>=0}]"
one must check at run time that the concrete type used to instantiate
the \xcd"Array"'s type parameter is equivalent to
\xcd"int{self>=0}".  This check could involve a run-time
entailment check, 
breaking the phase distinction between
compile time and run time for constraint solving.

One approach is to restrict the language 
to rule out casts to type parameters 
and to generic types with subtyping constraints, ensuring that
entailment checks are not needed at run time.
Alternatively, 
the constraint solver could be embedded into the runtime system.
However, this
solution can result in inefficient run-time casts
if entailment checking for the given constraint system is expensive.
Finally, one can simply erase the constraints from the run-time
type information, preserving the base type.  As with Java's
erasure semantics, this approach is prone to run-time type
errors.

\subsection{X10 design decisions}

Given these considerations, the \Xten makes the following choices:
\begin{itemize}
\item \Xten supports subtyping and equality constraints on types
\item \Xten does not support structural bounds, but may do so in
the future.  \Xten has closures with structural subtyping, which
can be used in many of the cases structural type bounds would be
used.
\item Classes have type parameters with definition-site variance
rather than type properties with use-site variance annotations.
Properties are just too unfamiliar.
Usability outweighs expressive power. 
\item Run-time type information is preserved, but constraints
are not.  
\end{itemize}


\subsection{Related work}
\label{sec:related}
Constraint-based type systems, dependent types, and generic types
have been well-studied in the literature.

\paragraph{Constraint-based type systems.}

The use of constraints for type inference and subtyping has a history
going back to Mitchell~\cite{mitchell84} and by
Reynolds~\cite{reynolds85}.  These and subsequent systems are based on
constraints over types, but not over values.  Trifonov and
Smith~\cite{trifonov96} proposed a type system in which types are
refined using subtyping constraints.
Pottier~\cite{pottier96simplifying} presents a constraint-based type
system for an ML-like language with subtyping.  These developments
lead to \hmx~\cite{sulzmann97type}, a constraint-based framework for
Hindley--Milner-style type systems.  The framework is parametrized on
the specific constraint system $X$; instantiating $X$ yields
extensions of the HM type system.  Constraints in \hmx{} are over
types, not values. The \hmx{} approach is an important precursor to
our constrained types approach. The principal difference is that
\hmx{} applies to functional languages and does not integrate
dependent types.

%
Sulzmann and Stuckey~\cite{sulzmann-hmx-clpx} showed that the
type inference algorithm for \hmx can be encoded as a
constraint logic program parametrized by the constraint system
$X$. This is very much in spirit with our approach.
Constrained types open the door to {\em user-defined}
predicates and functions, effectively permitting the user to enrich
$\cal C$ (hence the power of the compile-time type-checker) by
developing application-specific constraints using a constraint
programming language such as CLP($\cal C$) \cite{clp} or the richer
RCC($\cal C$) \cite{DBLP:conf/fsttcs/JagadeesanNS05}.

\paragraph{Dependent types.}

Dependent type
systems~\cite{xi99dependent,calc-constructions,epigram,cayenne}
parametrize types on values.  Refinement type
systems~\cite{refinement-types,conditional-types,jones94,sized-types,flanagan-popl06,flanagan-fool06,liquid-types},
introduced by Freeman and Pfenning~\cite{refinement-types}, are dependent type
systems that extend a base type system through constraints on values.  These
systems do not treat value and type constraints uniformly.

Our work is closely related to DML, \cite{xi99dependent}, an
extension of ML with dependent types. DML is also built
parametrically on a constraint solver. Types are refinement types;
they do not affect the operational semantics and erasing the
constraints yields a legal DML program.  This differs from generic constrained
types, where erasure of subtyping constraints can prevent the program from
type-checking.
DML does not permit any run-time checking of constraints
(dynamic casts).

The most obvious distinction between DML and constrained types
lies in the target
domain: DML is designed for functional programming
whereas constrained types are designed for imperative, concurrent
object-oriented languages. 
But there are several other
crucial differences as well.

DML achieves its separation between compile-time and run-time processing
by not permitting program
variables to be used in types. Instead, a parallel set of (universally
or existentially quantified) ``index'' variables are
introduced.
%
Second, DML permits only variables of basic index sorts known to
the constraint solver (e.g., \Xcd{bool}, \Xcd{int}, \Xcd{nat}) to
occur in types. In contrast, constrained types permit program
variables at any type to occur in constrained types. As with DML
only operations specified by the constraint system are permitted in
types. However, these operations always include field selection and
equality on object references.  Note that DML-style constraints are easily
encoded in constrained types.

% {\em Conditional
% types}~\cite{conditional-types} extend refinement types to
% encode control-flow information in the types.
% %
% Jones introduced {\em qualified types}, which permit
% types to be constrained by a finite set of
% predicates~\cite{jones94}.
% %
% {\em Sized types}~\cite{sized-types}
% annotate types with the sizes of recursive data structures.
% Sizes are linear functions of size variables.
% Size inference is performed using a constraint solver for
% Presburger arithmetic~\cite{omega}.
% % constraints on types, support primitive recursion only

% Index objects must be pure.
% Singleton types int(n).
% ML$^{\Pi}_0$:
% Refinement of the ML type system: does not affect the
% operational semantics.  Can erase to ML$_0$.

% Jay and Sekanina 1996: array bounds checking based on shape
% types.

% Ada dependent types.
% Ada has constrained array definitions.  A constraint
% \cite{ada-ref-man}.  Not clear if they're dependent.  Are
% there other dependent types?  Generics are dependent?

        % Used for array bounds by Morrisett et al (I think--need
        % to find paper)

% Singleton types~\cite{aspinall-singletons}.

Logically qualified types, or liquid types~\cite{liquid-types},
permit types in a base Hindley--Milner-style type system to be refined with
conjunctions of logical qualifiers.  The subtyping relation is similar to
\Xten{}'s, that is, two liquid types are in the subtyping relation if their base
types are and if one type's qualifier implies the other's.
The Hindley--Milner type
inference algorithm is used to infer base types; these types are used as templates for inference of the liquid types.
The types of certain expressions are over-approximated to ensure inference
is decidable.
To improve precision of the inference algorithm, and hence
to reduce the annotation burden on the programmer, 
the type system is path sensitive.

Hybrid type-checking~\cite{flanagan-popl06,flanagan-fool06}
introduced another refinement type system.
While typing is undecidable, dynamic checks are inserted into
the program when necessary if the type-checker (which
includes a constraint solver) cannot determine
type safety statically.
In \FXG{}, dynamic type checks, including tests of dependent
constraints, are inserted only at explicit casts or
\Xcd{instanceof} expressions; constraint solving is performed at compile time.

% Where clauses for F-bounded polymorphism~\cite{where-clauses}
% Bounded quantification: Cardelli and Wegner.  Bound T with T'
% In F-bounded polymorphism~\cite{f-bounds}, type variables are bounded by a function of 
% the type variable. 
% Not dependent types.

Concoqtion~\cite{concoqtion} extends types in OCaml~\cite{ocaml}
with constraints written as Coq~\cite{coq} rules.
While the types are expressive, supporting the full generality
of the Coq language, proofs must be
provided to satisfy the type checker.
\Xten{} supports only constraints that can be checked by a
constraint solver during compilation.
Concoqtion encodes OCaml types and value to allow reasoning in
the Coq formulae; however, there is an impedance mismatch
caused by the differing syntax, representation, and behavior
of OCaml versus Coq.

\eat{
Cayenne~\cite{cayenne} is a Haskell-like language with fully dependent types.
There is no distinction between static and dynamic types.
Type-checking is undecidable.
There is no notion of datatype refinement as in DML.

Epigram~\cite{epigram,epigram-matter}
is a dependently typed functional programming language based on
a type theory with inductive families.
Epigram does not have a phase distinction between values and
types.
}

\eat{
$\lambda^{\sf Con}$ is a lambda calculus with assertions.
Findler, Felleisen, Contracts for higher-order functions (ICFP02)

  example: int[> 9]

contracts are either simple predicates or function contracts.
defined by (define/contract ...)

enforced at run-time.
}

% Jif~\cite{jif,jflow} is an extension of Java in which
% types are labeled with security policies enforced by the
% compiler.

\eat{
ESC/Java~\cite{esc-java}
allow programmers to write object invariants and pre- and
post-conditions that are enforced statically
by the compiler using an automated theorem prover.
Static checking is undecidable and, in the presence of loops,
is unsound (but still useful) unless the programmer supplies loop invariants.
ESC/Java can enforce invariants on mutable state.
}

% and Spec$\sharp$~\cite{specsharp}

\eat{
Pluggable and optional type systems were proposed by
Bracha~\cite{bracha04-pluggable} and provide another means of
specifying refinement types.
Type annotations, implemented in compiler plugins, serve only to
reject programs statically that might otherwise have dynamic
type errors.
CQual~\cite{foster-popl02} extends C with user-defined type
qualifiers.  These
qualifiers may be flow-sensitive and may be inferred. 
CQual supports only a fixed set of typing rules
for all qualifiers.
In contrast, the {\em semantic type qualifiers} of
Chin, Markstrum, and Millstein~\cite{chin05-qualifiers}
allow programmers to define typing rules for qualifiers
in a meta language that allows type-checking rules to be
specified declaratively.
JavaCOP~\cite{javacop-oopsla06} is a pluggable type system
framework for Java.  Annotations are defined in a meta language
that allows type-checking rules to be specified declaratively.
JSR 308~\cite{jsr308} is a proposal for adding user-defined type qualifiers
to Java.
}

% Holt, Cordy, the Turing programming language

% Ou, Tan, Mandelbaum, Walker, Dynamic typing with dependent types
% Separate dependent and simple parts of the program.
% Statically type the dependent parts.
% Dynamic checks when passing values into dependent part.

\paragraph{Genericity.}

Genericity in object-oriented languages is usually
supported through
type parametrization.

A number of proposals 
for adding genericity to Java quickly followed
the initial release of
the language~\cite{GJ,Pizza,java-popl97,thorup97,allen03}.
GJ~\cite{GJ} implements invariant type
parameters via type erasure.
PolyJ~\cite{java-popl97} supports run-time representation of types
via adapter objects, and also permits instantiation of
parameters on primitive types and structural parameter bounds.
Viroli and Natali~\cite{reflective-generics,type-passing-generics}
also support
a run-time representation of types, using Java's reflection API.
NextGen~\cite{nextgen,allen03} was implemented using run-time 
instantiation.
\Xten{}'s generics have a hybrid implementation, adopting PolyJ's
adapter object approach for dependent types and for 
type introspection and using NextGen's run-time
instantiation approach for greater efficiency.
% MixGen~\cite{allen04} extends NextGen with mixins.

\csharp also supports generic types via run-time instantiation in the
CLR~\cite{csharp-generics}.  Type parameters may be declared
with definition-site variance tags.
Generalized type constraints were proposed for
\csharp~\cite{emir06}.  Methods can be annotated with subtyping
constraints that must be satisfied to invoke the method.
Generic \Xten{} supports these constraints, as well as constraints
on values, with method and constructor where clauses.

\eat{
\FXG{} does not support \emph{bivariance}~\cite{variant-parametric-types}; a
class \xcd"C" is bivariant in a type property \xcd"X" if \xcd"C{self.X==S}" is
a subtype of \xcd"C{self.X==T}" for any \xcd"S" and \xcd"T".  Bivariance is
useful for writing code in which the property \xcd"X" is ignored.  One can
achieve  this effect in \FXG{} simply by leaving \xcd"X" unconstrained.
}

\eat{
Parametric types with use-site variance are related to existential types:
\xcd"C<+T>" corresponds to the bounded existential $\exists\tcd{X<:T}.C<X>$;
\xcd"C<-T>" corresponds to the bounded existential $\exists\tcd{X:>T}.C<X>$;
\xcd"C<*>" corresponds to the unbounded existential $\exists\tcd{X}.C<X>$.
\FXG{} has a similar correspondence:
\xcd"C{X<:T}" corresponds to the bounded existential \xcdmath"C\{\exists\tcd{self}:C.self.X<:T\}";
\xcd"C{X:>T}" corresponds to the bounded existential \xcdmath"C\{\exists\tcd{self}.C<X\}";
\xcd"C" corresponds to the unbounded existential \xcdmath"C\{\exists\tcd{self}.C<X\}".
}




\eat{\section{Constraint solver}
\label{sec:solver}

The goal of the constraint solver is 
to check an assertion $\xbar{c} \vdashC \Xcd{d}$.

\eat{
Inference

The first step is to normalize constraints
into a set of constraint judgments
$\xbar{c} \vdashC \Xcd{c}$ where $\Xcd{c}$ contains no conjunctions.


Once in normalized form, the inference proceeds as follows:
Select a constraint $\xbar{c} \vdashC \Xcd{c}$.
If not consistent, fail.
If valid, ok.
If not valid, generate assignment of variables that makes it
true, adding the assignment to the assumptions for all
constraints.

The inference algorithm must specify the criteria for:
\begin{itemize}
\item selecting the next constraint to solve
\item generating the variable assignment consistent with all
other constraints (to avoid backtracking)
\end{itemize}

Pick an unassigned variable, find weakest assignment that makes just
this clause true.  Does the weakest assignment exist?

Question: can we ensure each clause involves only one or two
unknowns?
}

We add the following rules to allow type arguments to calls to
be omitted.

\infrule[T-invk-inferred]{
\xbar{Y}~\mbox{fresh}
\\
\Gamma, \xbar{Y} \ty {\tt type}
\vdash
\Xcd{e}_0.\Xcd{m[}\xbar{Y}\Xcd{](}\xbar{e}\Xcd{)} \ty
\Xcd{T}
}{
\Gamma \vdash
\Xcd{e}_0.\Xcd{m(}\xbar{e}\Xcd{)} \ty
\Xcd{T}
}

\infrule[T-new-inferred]{
\xbar{Y}~\mbox{fresh}
\\
\Gamma, \xbar{Y} \ty {\tt type}
\vdash
\Xcd{new}~\Xcd{C[}\xbar{Y}\Xcd{](}\xbar{e}\Xcd{)} \ty
\Xcd{T}
}{
\Gamma \vdash
\Xcd{new}~\Xcd{C(}\xbar{e}\Xcd{)} \ty
\Xcd{T}
}

\subsection{Constraint representation}

%\newcommand\eqedge{\rightleftharpoons}
\newcommand\eqedge{\sim}
\newcommand\flowedge{\to}
\newcommand\treeedge[1]{\mapsto_{#1}}
\newcommand\typeedge{\mapsto_{\tt type}}

Represent a constraint as a graph $G$.
Each node represents a constraint term for a value or a type.
The node for a path $p$ is written $v_p$;
the node for a type $T$ is written $V_T$.
There are four kinds of edges:
\begin{enumerate}
\item undirected equivalence edges,
        $v_p \eqedge v_q$ and $V_S \eqedge V_T$,
\item type edges, $v_p \typeedge V_T$,
\item tree edges, $v_p \treeedge{f} v_{p.f}$
              and $v_p \treeedge{X} V_{p.X}$, and
\item flow edges, $V_S \flowedge V_T$.
\end{enumerate}

First, each constraint term is mapped to a node in the graph as
follows.
Associate each term $t$ with a node
$v_t$.  For each access path {\tt p.x}, add a tree edge
$v_{{\tt p}} \treeedge{{\tt x}} v_{{\tt p.x}}$.
For each path type {\tt p.X}, add a tree edge
$v_{{\tt p}} \treeedge{{\tt X}} V_{{\tt p.X}}$.
For each atomic formula ${\tt f}(\xbar{t})$, add the tree edge
$v_{{\tt f}(\xbar{t})} \treeedge{i} v_{t_i}$ for all $i$.
If term $t$ has type $T$, add $v_t \typeedge V_{t{\tt .type}}$
and
add $V_T \eqedge V_{t{\tt .type}}$ to $G$.

Type nodes are sets of classes.

Next, constraints are incorporated into the graph:

\begin{itemize}
\item
For constraint {\tt p==q}, add $v_{\tt p} \eqedge v_{\tt q}$ to $G$.

\item
For constraint {\tt S==T}, add $V_{\tt S} \eqedge V_{\tt T}$ to $G$.

\item
For constraint {\tt S<:T},
add $V_{\tt S} \flowedge V_{\tt T}$
to $G$.

\end{itemize}

\subsection{Solving}

A flow-path is a path that follows flow and equivalence edges
only.
A type-path is a path that follows type and equivalence edges
only.

Now, we saturate: 
If there is a type-path $v_t \typeedge^* V_{\tt C\{c\}}$,
add $c[t/\Xcd{self}]$ to the worklist.

        Can saturate lazily when doing a lookup.
        EXCEPT: a type may have an arbitrary constraint
                \xcd"C{self.x==3 && y > 7}", so affect is non-local
        EXCEPT: c is x.f==...
                with x: C{c}
                need to avoid infinite loop

To check:

\begin{itemize}
\item To check
constraint {\tt p==q}, check if $v_{\tt p} \eqedge^* v_{\tt q}$.
\item To check
constraint {\tt S<:T}, check if there is a flow-path from $V_{\tt S}$ to
$V_{\tt T}$.  This requires checking entailment of the type constraints and
adding more edges to the graph.  (XXX details!)
Add the flow edge to memoize.
\end{itemize}


}

%\newpage~\newpage

\section{Semantics}
\label{sec:semantics}
%\setlength{\afterruleskip}{\smallskipamount}
%\setlength{\afterruleskip}{\medskipamount}

\newcommand{\constraint}{{\tt constraint}}
\newcommand\cj[2]{{#1} \vdash {#2}~\constraint}
\newcommand\cjj[3]{{#1} \vdash {#2}~\constraint, {#3}~\constraint}
\newcommand\wj[2]{{#1} \vdash {#2}~\type}
\newcommand\tj[3]{{#1} \vdash {#2} \ty {#3}}
\newcommand\stj[3]{{#1} \vdash {#2} \subtype {#3}}

We now describe the semantics of languages in the \FX{} family.
We start with a core \FXZ{} language that support simple
\FJ-like types, then add value-dependent types, and type-dependent
types, separately, then finally add both. For uniformity we declare type-valued parameters and properties to be of ``type'' \type, instead of using square brackets to demarcate them.

\subsection{\FXZ}

The grammar for \FXZ{} is shown in Figure~\ref{fig:fx-grammar}.
The syntax is essentially that of \FJ{}.
Following the convention of \FJ{}, we use $\bar{x}$ to denote a
list $x_1, \dots, x_n$, and use $\bullet$ to denote the empty
list.

A program {\tt P} is a set of class declarations \tbar{L}.
Class names {\tt C} range over the declared classes in {\tt P} 
and {\tt Object}.
Classes have
properties (i.e., immutable fields) \tbar{f} and methods \tbar{M}.  We omit constructors
and require that the \new{} expression provide initializers
for all fields, including inherited fields. 
Methods are introduced with the {\tt def} keyword.

Both classes and methods may have constraint clauses
{\tt c}.  In the case of classes, {\tt c} is to be thought of as an
{\em invariant} satisfied by all instances of the class; in the case of
methods, {\tt c} is an additional condition, or {\em guard},
that must be satisfied by
the receiver 
and the actual arguments of the method in order for the method to
be applicable.

Expressions {\tt e} are either parameters {\tt x} (including the implicit
method parameter {\tt this}), field accesses, method invocations, \new{}
expressions, or casts (written {\tt e}~\as~{\tt T}).

The set of types includes classes {\tt C} and is closed under
constrained types ($\tt T\{c\}$) and existential
quantification ($\exists \tt x:T.~U$).
A value {\tt v} is of type {\tt C} if it is an instance of class {\tt C}; it is of type $\tt
T\{c\}$ if it is of type {\tt T} and it satisfies the constraint $\tt
c[v/self]$; it is of type $\exists \tt x:T.~U$
if there is some value {\tt w}
of type {\tt T} such that {\tt v} is of type
$\tt U[w/x]$.

The syntax for constraints in \FXZ{} is specified in
Figure~\ref{fig:fx-grammar}. As expected, constraints
relate property fields of objects. Neither casts
nor method invocations are permitted in constraints.

We distinguish a subset of these constraints as
{\em user constraints}---these are permitted to occur in
programs. For \FXZ{} the only user constraint permitted is the vacuous
{\tt true}. Thus the types occurring in user programs are isomorphic
to class types, and class and method definitions specialize to the
standard class and method definitions of \FJ{}. 

The constraints permitted by the syntax in
Figure~\ref{fig:fx-grammar} that
are not user constraints are used to define the static and
dynamic semantics of \FXZ{} (see, e.g., rule \TField{} in Figure~\ref{fig:FX}).
The use of this richer constraint set as well as constrained and existential types is
not necessary in \FXZ; it simply enables us to present the static and dynamic
semantics once for the entire family of \FX{} languages,
specifying the other members of the family as extensions
to these core semantics.

Existential constraints are introduced for convenience only:
${\tt T}\{\exists {\tt x}:{\tt U}.~{\tt c}\}$ is equivalent to $\exists {\tt y}:{\tt U}.~{\tt T}\{{\tt c}[{\tt y}/{\tt x}]\}$ choosing {\tt y} not free in {\tt T}.

\paragraph{Dynamic semantics.}
The operational semantics, shown in Figure~\ref{fig:sos},
is straightforward and essentially identical
to \FJ \cite{FJ}. It is described in terms of a non-deterministic
reduction relation on expressions.\eat{\footnote{
For simplicity, we enforce call-by-value-like semantics:
\RField, \RInvk, and \RCast{} require receivers of the form
``$\new~{\tt C}(\tbar{t})$'' instead of ``$\new~{\tt C}(\tbar{e})$''.
Otherwise, we would have to distinguish compile-time constraints over constraint terms from run-time constraints over expressions.}}

The only novelty is the use of the
subtyping relation to check that the cast is satisfied.
The typing rule for casts ({\sc T-New}) in Figure~\ref{fig:FX} specifies that if the arguments $\tbar{e}$ have type $\tbar{V}$ then $\new~{\tt C}(\tbar{e})$ has type $\exists\tbar{y}:\tbar{V}.~{\tt C}\{\self==\new~{\tt C}(\tbar{y})\}$, therefore {\RCast} requires this particular type to be a subtype of {\tt T}.
In \FXZ, this
test simply involves checking that the class of which the object is an
instance is a subclass of the class specified in the given type; in
languages with richer notions of type this operation may
involve run-time constraint solving using the properties of the object.

\paragraph{Static Semantics.}
Each language in the family is defined over a given input constraint system $\mathcal{X}$ that is required to support the trivial constraint \true{}, conjunction, existential quantification, and equality on constraint terms. Given a program {\tt P}, we now show how to derive from $\mathcal{X}$ a larger deduction system that captures the object-oriented structure of {\tt P} and lets us decide whether {\tt P} is well typed.

In the following, the context $\Gamma$ is always a
(finite, possibly empty) sequence of formulas $\tt x:T$ and constraints $\tt c$ satisfying:
\begin{enumerate}
  \item for any formula $\phi=\tt x:T$ or constraint $\phi=\tt c$ in the sequence all free variables $\tt y$
  occurring in $\tt T$ or $\tt c$ are declared by a formula $\tt
  y:U$ in the sequence to the left of $\phi$.

  \item for any variable $\tt x$, there is at most one
  formula $\tt x:T$ in $\Gamma$.
\end{enumerate}

\medskip

In the judgments that follow, 
for any formulas $\phi_1$ and
$\phi_2$, we adopt the convention that the rule $\Gamma \vdash \phi_1,~\phi_2$
is shorthand for the rules
$\Gamma \vdash \phi_1$
and
$\Gamma \vdash \phi_2$. Whenever
we state an assumption of the form ``{\tt x} is fresh'' in a rule we mean
it is not free in the consequent of the rule.

The judgments of interest are as follows:
\begin{enumerate}
	\item Well-formedness:\\
	  $\cj{\Gamma}{\tt c}$ \hfill constraint {\tt c} is well formed\\
	  $\wj{\Gamma}{\tt T}$ \hfill  type {\tt T} is well formed
	\item Member lookup:\\
	  $\fields({\tt C})=\tbar{f}:\tbar{T}$ \hfill class {\tt C} has fields \tbar{f} of type \tbar{T}\\
	  $\Gamma\vdash {\tt C}~\has~{\tt I}$ \hfill class {\tt C} has member {\tt I}\\
	  $\Gamma\vdash {\tt x}~\underline\has~{\tt I}$ \hfill variable {\tt x} has member {\tt I}\\
	  $~$ \hfill where ${\tt I}::= {\tt f}:{\tt T} \alt {\tt m}(\tbar{x}:\tbar{T})\{{\tt c}\}:{\tt U}={\tt e}$
	\item Constraints:\\
	  $\Gamma\vdash {\tt c}$ \hfill constraint {\tt c} holds
	\item Typing:\\
	  $\Gamma\vdash {\tt e}:{\tt T}$ \hfill expression {\tt e} has type {\tt T}\\
	  $\vdash {\tt def}~{\tt m}(\tbar{x}:\tbar{T})\{{\tt c}\}:{\tt U}={\tt e}~{\rm OK~in}~{\tt C}$ \\ $~$ \hfill method {\tt m} in class {\tt C} type checks\\
	  $\vdash {\tt class}~{\tt C}(\tbar{f}:\tbar{T})\{{\tt c}\}~{\tt extends}~{\tt D}~\{~\tbar{M}~\}~{\rm OK}$ \\ $~$ \hfill class {\tt C} type checks
	\item Subtyping:\\
	  $\Gamma \vdash {\tt S} \subtype {\tt T}$ \hfill type {\tt S} is a subtype of type {\tt T}
\end{enumerate}

\medskip

A program type checks iff all its classes do.  We now describe
in more detail each of these judgments, in turn.

\paragraph{1. Well-formedness.} A constraint {\tt c} is well formed in context $\Gamma$, written $\cj{\Gamma}{\tt c}$, iff its free variables are declared in $\Gamma$. The rules in Figure~\ref{fig:well} extend this requirement to types.

We say a program, context, or judgment is well formed iff all the constraints and types involved are well formed. By design, every judgment derived from a well-formed context is also well formed. As a consequence, if a program type checks, it is well formed.

\paragraph{2. Member lookup.} Figure~\ref{fig:lookup} specifies the fields and methods available on each class. We impose restrictions on inheritance as follows. Classes may only be extended by classes with stronger invariants. Fields cannot be overridden. Methods cannot be overloaded.
A method may only be overridden by a method with the same name,
arity, parameter types, a return type that is a subtype of the
overridden return type, and a weaker guard (i.e., the superclass
method guard must entail the subclass's).
\footnote{To avoid cluttering the inference rules, we
define overriding only informally; a formal definition is
straightforward.}

To prepare for the introduction of generic types later, we
distinguish members that are available on variables from members
available on classes.  For now,
${\tt x}:{\tt C}~\underline\has~{\tt I}[{\tt x}/\this]$ iff ${\tt C}~\has~{\tt I}$.

\begin{figure*}
\centering
\begin{tabular}{r@{\quad}rcl}
  (Program) & {\tt P} &{::=}& $\tbar{L}$ \\
  (Class declaration) & {\tt L} &{::=}& $\tt class~C(\tbar{f}:\tbar{T})\{c\}~extends~D~\{~\tbar{M}~\}$ \\
  (Method declaration)& {\tt M} &{::=}& $\tt def~m(\tbar{x}:\tbar{T})\{c\}:T=e;$ \\
  (Expression)& {\tt e} &{::=}& $\tt x$ \alt $\tt e.f$ \alt $\tt\new~C(\tbar{e})$ \alt $\tt e.m(\tbar{e})$ \alt $\tt e~\as~T$ \\
  (Value)& {\tt v} &{::=}& $\tt\new~C(\tbar{v})$ \\
  (Type)& {\tt T} &{::=}& $\tt C$ \alt $\tt T\{c\}$ \alt $\tt \exists x:T.~T$ \\
  (Constraint term) & {\tt t} &{::=}& $\tt x$ \alt $\tt t.f$ \alt $\tt\new~C(\tbar{t})$ \\
  (Constraint) & {\tt c} &{::=}& $\true$ \alt $\tt t==t$ \alt $\tt c,c$ \alt $\tt \exists x:T.~c$ \\
\end{tabular} 
\caption{\FX{} productions.
{\tt C} ranges over class names, {\tt f} over field names, {\tt m} over method names, {\tt x} over variable names.}
\label{fig:fx-grammar}
\end{figure*}


\begin{figure*}
\vspace{-\bigskipamount}
\begin{minipage}{.4\textwidth}
\quad\typicallabel{XXXXXX}
\infrule[\RField]
	{\fields({\tt C})=\tbar{f}:\tbar{T}}
	{\new~{\tt C}(\tbar{e}).{\tt f}_i \derives {\tt t}_i}

\infrule[\RCField]
	{{\tt e}\derives {\tt e}'}
	{{\tt e}.{\tt f}\derives {\tt e}'.{\tt f}}

\infrule[\RCInvkRecv]
	{{\tt e}\derives {\tt e}'}
	{{\tt e}.{\tt m}(\tbar{a})\derives {\tt e}'.{\tt m}(\tbar{a})}

\infrule[\RCCast]
	{{\tt e}\derives {\tt e}'}
	{{\tt e}~\as~{\tt T}\derives {\tt e}'~\as~{\tt T}}
\end{minipage}%
\begin{minipage}{.6\textwidth}
\quad\typicallabel{XXXXXX}
\infrule[\RCNewArg]
	{{\tt e}_i\derives {\tt e}'_i}
	{\new~{\tt C}(\ldots,{\tt e}_i,\ldots)\derives\new~{\tt C}(\ldots,{\tt e}'_i,\ldots)}

\infrule[\RInvk]
	{{\tt C}~\has~{\tt m}(\tbar{x}:\tbar{T})\{{\tt c}\}:{\tt U}={\tt b}}
	{\new~{\tt C}(\tbar{e}).{\tt m}(\tbar{a})\derives {\tt b}[\new~{\tt C}(\tbar{e}),\tbar{a}/\this,\tbar{x}]}

\infrule[\RCInvkArg]
	{{\tt a}_i\derives {\tt a}'_i}
	{{\tt e}.{\tt m}(\ldots,{\tt a}_i,\ldots)\derives {\tt e}.{\tt m}(\ldots,{\tt a}'_i,\ldots)}

\infrule[\RCast]
	{\vdash \tbar{e}:\tbar{V} \andalso \vdash\exists\tbar{y}:\tbar{V}.~{\tt C}\{\self==\new~{\tt C}(\tbar{y})\}\subtype {\tt T}}
	{\new~{\tt C}(\tbar{e})~\as~{\tt T}\derives\new~{\tt C}(\tbar{e})}
\end{minipage}
\caption{\FX{} operational semantics}
\label{fig:sos}
\end{figure*}


\begin{figure*}
\vspace{-\bigskipamount}
\begin{minipage}{.5\textwidth}
\quad\typicallabel{XXXXXX}
\infax[W-Object]
  {\wj{}{\tt Object}}

\infrule[W-Class]
  {{\tt class}~{\tt C}(\tbar{f}:\tbar{T})\{{\tt c}\}~{\tt extends}~{\tt D}~\{~\tbar{M}~\} \in {\tt P}}
  {\wj{}{\tt C}}
\end{minipage}%
\begin{minipage}{.5\textwidth}
\quad\typicallabel{XXXXXX}
\infrule[W-Dep]
        {\wj{\Gamma}{\tt T} \andalso \cj{\Gamma, \self \ty {\tt T}}{\tt c}}
	{\wj{\Gamma}{{\tt T}\{{\tt c}\}}}

\infrule[W-Exists]
        {\wj{\Gamma}{\tt T} \andalso \wj{\Gamma, {\tt x} \ty {\tt T}}{\tt U}}
        {\wj{\Gamma}{\exists {\tt x} \ty {\tt T}.~{\tt U}}}
\end{minipage}
\caption{\FX{} well-formed types}
\label{fig:well}
\end{figure*}

\begin{figure*}
\vspace{-\bigskipamount}
\begin{minipage}{.5\textwidth}
\quad\typicallabel{XXXXXX}
\infax[L-Fields-Object]
  {\vdash\fields({\tt Object})=\bullet}
\end{minipage}%
\begin{minipage}{.5\textwidth}
\quad\typicallabel{XXXXXX}
\infrule[L-Field-B]
  {\vdash\fields({\tt C})=\tbar{f}:\tbar{T}}
  {{\tt C}~\has~{\tt f}_i:{\tt T}_i}

\end{minipage}

\begin{minipage}{\textwidth}
\quad\typicallabel{XXXXXX}
\infrule[L-Fields-I]
  {{\tt class}~{\tt C}(\tbar{f}:\tbar{T})\{{\tt c}\}~{\tt extends}~{\tt D}~\{~\tbar{M}~\}\in {\tt P}
   \andalso
   \vdash\fields({\tt D})=\tbar{g}:\tbar{U}}
   {\vdash\fields({\tt C})=\tbar{g}: \tbar{U}, \tbar{f}: \tbar{T}}

\infrule[L-Method-B]
  {{\tt class}~{\tt C}(\tbar{f}:\tbar{T})\{{\tt c}\}~{\tt extends}~{\tt D}~\{~\tbar{M}~\}\in {\tt P}
   \andalso
   {\tt def}~{\tt m}(\tbar{x}\ty \tbar{U})\{{\tt d}\}\ty {\tt V}={\tt e}\in \tbar{M}}
  {\vdash {\tt C}~\has~{\tt m}(\tbar{x}:\tbar{U})\{{\tt d}\}:{\tt V}={\tt e}}

\infrule[L-Method-I]
  {{\tt class}~{\tt C}(\tbar{f}:\tbar{T})\{{\tt c}\}~{\tt extends}~{\tt D}~\{~\tbar{M}~\}\in {\tt P}
   \andalso
   \vdash {\tt D}~\has~{\tt m}(\tbar{x}:\tbar{U})\{{\tt d}\}:{\tt V}={\tt e}
   \andalso
   {\tt m}\not\in\tbar{M}}
  {\vdash {\tt C}~\has~{\tt m}(\tbar{x}:\tbar{U})\{{\tt d}\}:{\tt V}={\tt e}}
\end{minipage}

\begin{minipage}{.5\textwidth}
\quad\typicallabel{XXXXXX}
\infrule[L-Member-B]
  {{\tt C}~\has~{\tt I}}
  {{\tt x}:{\tt C}\vdash {\tt x}~\underline\has~{\tt I}[{\tt x}/\this]}
\end{minipage}%
\begin{minipage}{.5\textwidth}
\quad\typicallabel{XXXXXX}
\infrule[L-Member-C,E]
  {\Gamma,{\tt x}:{\tt T}\vdash {\tt x}~\underline\has~{\tt I} \andalso {\tt y}~\rm fresh}
  {\Gamma,{\tt x}:{\tt T}\{{\tt c}\}\vdash {\tt x}~\underline\has~{\tt I} \\
   \Gamma,{\tt x}:\exists {\tt y}:{\tt U}.~{\tt T}\vdash {\tt x}~\underline\has~{\tt I}}
\end{minipage}
\caption{\FX{} member lookup}
\label{fig:lookup}
\end{figure*}

\begin{figure*}
\begin{minipage}[t]{.2\textwidth}
\quad\typicallabel{XXXXXX}
\infrule[X-Proj]
  {\sigma(\Gamma)\vdashX{\tt c}}
  {\Gamma\vdash {\tt c}}
\end{minipage}
\begin{minipage}[t]{.8\textwidth}
\quad\typicallabel{XXXXXX}
\infrule[X-Sel]
  {\vdash\fields({\tt C})=\tbar{f}:\tbar{T}
  	\andalso
  	\Gamma\vdash\new~{\tt C}(\tbar{t}):{\tt U}
  	\andalso
  	\Gamma, \new~{\tt C}(\tbar{t}).{\tt f}_i=={\tt t}_i\vdash {\tt c}}
  {\Gamma\vdash {\tt c}}

\infrule[X-Inv]
  {{\tt class}~{\tt C}(\tbar{f}:\tbar{T})\{{\tt c}\}~{\tt extends}~{\tt D}~\{~\tbar{M}~\}\in {\tt P}
   \andalso
   \Gamma\vdash {\tt t}:{\tt U},~{\tt U}\subtype{\tt C}\andalso \Gamma,{\tt c}[{\tt t}/\this]\vdash {\tt d}}
  {\Gamma\vdash {\tt d}}
\end{minipage}%
\caption{\FX{} object constraint system}
\label{fig:object}
\end{figure*}

\begin{figure*}
\vspace{-\bigskipamount}
\begin{minipage}{.4\textwidth}
\quad\typicallabel{XXXXXX}
\infax[T-Var]
  {\Gamma,{\tt x}:{\tt T}\vdash {\tt x}:{\tt T}\{\self=={\tt x}\}}
\end{minipage}
\begin{minipage}{.5\textwidth}
\quad{}\typicallabel{XXXXXX}
\infrule[T-Cast]
	{\Gamma\vdash {\tt e}:{\tt U}\andalso\wj{\Gamma}{\tt T}}
	{\Gamma\vdash {\tt e}~\as~{\tt T}:{\tt T}}
\end{minipage}
\begin{minipage}{\textwidth}
\quad{}\typicallabel{XXXXXX}
\infrule[T-Field]
	{\Gamma\vdash {\tt e}:{\tt T} \andalso \Gamma,{\tt x}:{\tt T}\vdash {\tt x}~\underline\has~{\tt f}:{\tt U} \andalso {\tt x}~\rm fresh}
	{\Gamma\vdash {\tt e}.{\tt f}:\exists {\tt x}:{\tt T}.~{\tt U}\{\self=={\tt x}.{\tt f}\}}

\infrule[T-Invk]
	{\Gamma\vdash {\tt e}:{\tt T},~\tbar{a}:\tbar{U} \andalso 
	  \Gamma,{\tt x}:{\tt T},\tbar{y}:\tbar{U}\vdash {\tt
          x}~\underline\has~{\tt m}(\tbar{y}:\tbar{V})\{{\tt d}\}:{\tt W}={\tt b},~\tbar{U}\subtype\tbar{V},~{\tt d} \andalso {\tt x},\tbar{y}~\rm fresh}
	{\Gamma\vdash {\tt e}.{\tt m}(\tbar{a}):\exists {\tt x}:{\tt T}.~\exists\tbar{y}:\tbar{U}.~{\tt W}}

\infrule[T-New]
	{{\tt class}~{\tt C}(\tbar{f}:\tbar{T})\{{\tt c}\}~{\tt extends}~{\tt D}~\{~\tbar{M}~\}\in {\tt P}
		\andalso
		\fields({\tt C})=\tbar{g}:\tbar{U} \\
	  \Gamma\vdash\tbar{e}:\tbar{V} \andalso
    \Gamma,{\tt x}:{\tt C},\tbar{y}:\tbar{V},{\tt x}.\tbar{g}==\tbar{y}\vdash
    \tbar{V}\subtype\tbar{U}[{\tt x}/\this],~{\tt c}[{\tt x}/\this] \andalso {\tt x},\tbar{y}~\rm fresh}
	{\Gamma\vdash\new~{\tt C}(\tbar{e}):\exists\tbar{y}:\tbar{V}.~{\tt C}\{\self==\new~{\tt C}(\tbar{y})\}}
        
\infrule[OK-Method]
  {{\tt class}~{\tt C}(\tbar{f}:\tbar{T})\{{\tt c}\}~{\tt extends}~{\tt D}~\{~\tbar{M}~\}\in {\tt P} \andalso
    {\tt def}~{\tt m}(\tbar{x}:\tbar{U})\{{\tt d}\}:{\tt V}={\tt e}\in\tbar{M}\\
    \this:{\tt C},\tbar{x}:\tbar{U}\vdash\tbar{U}~\type,~{\tt d}~\constraint
    \andalso
    \this:{\tt C},\tbar{x}:\tbar{U},{\tt d}\vdash {\tt e}:{\tt W},~{\tt W}\subtype {\tt V}}
  {\vdash {\tt def}~{\tt m}(\tbar{x}:\tbar{U})\{{\tt d}\}:{\tt V}={\tt e}~{\rm OK~in}~{\tt C}}

\infrule[OK-Class]
  {\this:{\tt C},\tbar{f}:\tbar{T}\vdash\tbar{T}~\type,~{\tt c}~\constraint \andalso \wj{}{\tt D} \andalso \tbar{M}~{\rm OK~in}~{\tt C}}
  {\vdash {\tt class}~{\tt C}(\tbar{f}:\tbar{T})\{{\tt c}\}~{\tt extends}~{\tt D}~\{~\tbar{M}~\}~\rm OK}
\end{minipage}
\caption{\FX{} typing rules}\label{fig:FX}
\end{figure*}


\begin{figure*}
\vspace{-\bigskipamount}
\begin{minipage}{.25\textwidth}
\quad\typicallabel{XXX}
\infrule[S-Refl]
  {\wj{\Gamma}{\tt T}}
  {\Gamma\vdash {\tt T}\subtype {\tt T}}
\end{minipage}%
\begin{minipage}{.30\textwidth}
\quad\typicallabel{XXX}
\infrule[S-Trans]
	{\Gamma\vdash {\tt T}\subtype {\tt U}, {\tt U}\subtype {\tt V}}
	{\Gamma\vdash {\tt T}\subtype {\tt V}}
\end{minipage}%
\begin{minipage}{.45\textwidth}
\quad\typicallabel{XXXX}
\infrule[S-Const-L]
	{\Gamma\vdash{\tt T}\{{\tt c}\}~\type \andalso\Gamma,{\tt c}\vdash{\tt T}\subtype {\tt U}}
	{\Gamma\vdash {\tt T}\{{\tt c}\}\subtype {\tt U}}
\end{minipage}%

\begin{minipage}{.5\textwidth}
\quad\typicallabel{XXXXX}
\infrule[S-Class]
  {{\tt class}~{\tt C}(\tbar{f}:\tbar{T})\{{\tt c}\}~{\tt extends}~{\tt D}~\{~\tbar{M}~\}\in {\tt P}}
  {\Gamma\vdash {\tt C}\subtype {\tt D}}

\infrule[S-Exists-L]
  {\wj{\Gamma}{\tt T} \andalso \Gamma,{\tt x}:{\tt T}\vdash {\tt U}\subtype {\tt V} \andalso {\tt x}~\rm fresh}
  {\Gamma\vdash\exists {\tt x}:{\tt T}.~{\tt U}\subtype {\tt V}}
\end{minipage}%
\begin{minipage}{.5\textwidth}
\quad\typicallabel{XXXXX}
\infrule[S-Const-R]
	{\wj{\Gamma}{{\tt U}\{{\tt c}\}}\andalso\Gamma,\self:{\tt T}\vdash {\tt c},{\tt T}\subtype {\tt U} }
	{\Gamma\vdash {\tt T}\subtype {\tt U}\{{\tt c}\}}

\infrule[S-Exists-R]
  {\Gamma\vdash {\tt t}:{\tt T},~{\tt U}\subtype {\tt V}[{\tt t}/{\tt x}]}
  {\Gamma\vdash {\tt U}\subtype\exists {\tt x}:{\tt T}.~{\tt V}}
\end{minipage}        
\caption{\FX{} subtyping rules}\label{fig:subtyping}
\end{figure*}


\begin{figure*}
\vspace{-\bigskipamount}
\begin{minipage}{.5\textwidth}
\quad
\typicallabel{XXX}
\infax[W-Type]
	{\wj{}{\type}}

\infrule[S-Extends]
	{\Gamma\vdash {\tt T}\extends {\tt U} \andalso \Gamma\vdash{\tt T}:\type,~{\tt U}:\type}
	{\Gamma\vdash {\tt T}\subtype {\tt U}}

\infrule[L-Extends]
	{\Gamma\vdash {\tt T}\extends{\tt U}
         \andalso
         \Gamma,{\tt x}:{\tt U}\vdash {\tt x}~\underline\has~{\tt I}}
	{\Gamma,{\tt x}:{\tt T}\vdash {\tt x}~\underline\has~{\tt I}}
\end{minipage}%
\begin{minipage}{.5\textwidth}
\quad\typicallabel{XXX}
\infax[W-Var]
	{\wj{\Gamma,{\tt x}:\type}{\tt x}}

\infrule[W-Path]
	{\Gamma\vdash {\tt p}:{\tt T}
         \andalso
         \Gamma,{\tt x}:{\tt T}\vdash {\tt x}~\underline\has~{\tt f}:\type}
	{\wj{\Gamma}{{\tt p}.{\tt f}}}

\infrule[T-Type]
        {\wj{\Gamma}{\tt C\{{\tt c}\}}}
  {\Gamma\vdash{\tt C}\{{\tt c}\}:\type\{\self=={\tt C}\{{\tt c}\}\}}
\end{minipage}
\caption{\FXG}
\label{fig:FXG}
\end{figure*}

\paragraph{3. Constraints.}
In defining these judgments we will use \mbox{$\Gamma \vdashX {\tt c}$}, the judgment corresponding to the underlying constraint system. Formally, $\cal X$ permits judgments of the form $\Gamma \vdashX {\tt c}$ where $\Gamma::=\tbar{c}$ is a (finite, possibly empty) sequence of constraints. We define the {\em constraint
projection}, $\sigma(\Gamma)$ as follows.
%
\begin{center}
\begin{tabular}{l}
$\sigma(\epsilon)={\tt true}$\\
$\sigma({\tt x}:{\tt C}, \Gamma)=\sigma(\Gamma)$\\
$\sigma({\tt x}:{\tt T\{c\}}, \Gamma)={\tt c}[{\tt x}/\self], \sigma({\tt x}:{\tt T},\Gamma)$\\
$\sigma({\tt x}:\exists {\tt y}:{\tt T}.~{\tt U}, \Gamma)=\sigma({\tt z:T}, {\tt x}:{\tt U[{\tt z}/{\tt y}]},\Gamma)$\\
$\sigma({\tt c},\Gamma) = {\tt c}, \sigma(\Gamma)$
\end{tabular}
\end{center}
%
Above, in the fourth rule, 
we assume that alpha-equivalence is used to
choose a variable {\tt z} that does not
occur in the context under construction.

We define $\Gamma\vdash {\tt c}$ in Figure~\ref{fig:object}. {\sc X-Sel} permits the underlying constraint system $\mathcal{X}$ to interpret field accesses. {\sc X-Inv} handles class invariants. Note the use of subtyping here to prepare for the introduction of bounds on generics types later.

We say that a context $\Gamma$ is {\em consistent} if all (finite)
subsets of $\{{\tt c}\alt \sigma(\Gamma) \vdash {\tt c}\}$ are consistent.
In all type judgments presented below ({\sc T-Cast}, {\sc T-Field},
etc.), we make the implicit assumption that the context $\Gamma$ is
consistent; if it is inconsistent, the rule cannot be used and the
type of the given expression cannot be established (type-checking
fails).

\eat{
The following rules govern existential constraints:
\infrule[Exists-R]
  {\Gamma\vdash {\tt t}:{\tt T} \andalso \Gamma\vdash{\tt c}[{\tt t}/{\tt x}]}
  {\Gamma\vdash \exists {\tt x}:{\tt T}.~{\tt c}}

\infrule[Exists-L]
  {\Gamma,{\tt x}:{\tt T},{\tt c}\vdash {\tt d} \andalso {\tt x}~\rm fresh}
  {\Gamma,\exists {\tt x}:{\tt T}.~{\tt c}\vdash {\tt d}}
}

\paragraph{4. Typing.} The typing rules are specified in Figure~\ref{fig:FX}.

{\sc T-Var} is as expected, except that it asserts the constraint {\tt
self==x}, which records that any value of this type is known
statically to be equal to {\tt x}. This constraint is actually very
crucial---as we shall see in the other rules, once we establish that
an expression {\tt e} is of a given type {\tt T}, we ``transfer'' the
type to a freshly chosen variable {\tt z}.  If, in fact, {\tt e} has a
static ``name'' {\tt x} (i.e., {\tt e} is known statically to be
equal to {\tt x}; that is, it has type {\tt T\{self==x\}}), then
{\sc T-Var} lets us assert that {\tt z:T\{self==x\}}, i.e., that {\tt z}
equals {\tt x}.
Thus {\sc T-Var} provides an important base case for
reasoning statically about equality of values in the environment.

We do away with the three cast rules in \FJ{} in favor of a single
cast rule, requiring only that {\tt e} be of some type {\tt U}.  At run time,
{\tt e} will be checked to see if it is actually of type {\tt T} (see
{\sc R-Cast} in Figure~\ref{fig:sos}).

{\sc T-Field} may be understood through ``proxy'' reasoning as
follows:  Given the context $\Gamma$, assume the receiver {\tt e} can
be established to be of type {\tt T}. Now, we do not know the run-time
value of {\tt e}, so we shall assume that it is some fixed but unknown
``proxy'' value {\tt x} (of type {\tt T}) that is ``fresh'' in that it
is not known to be related to any known value (i.e., those recorded
in $\Gamma$).  If we can establish that {\tt x} has a field {\tt f} of
type {\tt U}\footnote{Note from the definition of
\fields{} in Figure~\ref{fig:lookup} that all occurrences of
\this{} in the declared type of the field {\tt f} will have been replaced
by {\tt x}.}, then we can assert that
{\tt e.f} has type {\tt U} and, further, that it equals {\tt x.f}.
Hence, we can assert that {\tt e.f} has type 
$\exists {\tt x}:{\tt T}.~{\tt U}\{\self={\tt x}.{\tt f}\}$.

{\sc T-Invk} has a similar structure to {\sc T-Field}: we use
proxy reasoning for the receiver and the arguments of the method
call. {\sc T-New} also uses the same proxy reasoning: however in this case
we can establish that the resulting value is equal to $\new~{\tt C}(\tbar{y})$
for some values $\bar{\tt y}$ of the given types.

{\sc OK-Method} and {\sc OK-Class} ensure that the types and constraints occurring in a program are well formed. Following from \FJ{}, these rules do not preclude the existence of cycles in the type declarations. We assume they are acyclic. {\sc OK-Method} checks that the actual type {\tt W} of the method body {\tt e} is a subtype of its declared return type {\tt V}. {\sc OK-Class} makes sure all methods of the class type check.


\paragraph{5. Subtyping.} The subtyping relation is defined in Figure~\ref{fig:subtyping}.
Unsurprisingly, it is reflexive ({\sc S-Refl}) and transitive ({\sc S-Trans}).
{\sc S-Exists-L} and {\sc S-Exists-R} handle existential types.
{\sc S-Const-L} and {\sc S-Const-R} handle constraints. The rules ensure all types are well formed.

\subsection{\FXD}

Turning \FXZ{} into a language with value-dependent types is straightforward
since the construction of the previous section is parametric in the underlying constraint system $\mathcal{X}$ and constraint propagation is already built into the typing rules.

First, we assume we are given a constraint system $\cal A$ with a vocabulary of primitive types ${\tt A}$,
predicates ${\tt p}$, and functions ${\tt q}$.  Zero arity
functions encode constants.

Second, we extend the productions of \FXZ{} as follows.
\begin{center}
\begin{tabular}{r@{\quad}rcl}
  (Type)& {\tt T} &{::=}& {\tt A} \\
  (Expression) & {\tt e} &{::=}& ${\tt q}(\tbar{e})$ \\
  (Values) & {\tt e} &{::=}& ${\tt q}()$~~~~for {\tt q} of arity zero\\
  (Constraint term) & {\tt t} &{::=}& ${\tt f}(\tbar{t})$ \\
  (Constraint) & {\tt c} &{::=}& ${\tt p}(\tbar{t})$ \\  
\end{tabular}
\end{center}

The obvious rule is needed to type function calls.

\infrule[T-Fun]
	{{\tt q}{\rm~has~type~\tbar{A} \rightarrow {\tt B}{\rm~in~}\mathcal{A}} \andalso \Gamma\vdash\tbar{e}:\tbar{A}} 
	{\Gamma\vdash{{\tt q}(\tbar{e}):\exists\tbar{x}:\tbar{A}.~{\tt B}\{\self=={\tt q}(\tbar{x})\}}}

Third, we authorize users to write constraints in $\cal A$ (except for existential constraints) in programs.

For instance, if $\mathcal{A}$ defines the type {\tt int}, integer literals, and the usual arithmetic operators we can declare:

\begin{xten}
class Count(n:int) extends Object {
  def inc():Count{self.n==this.n+1} =
  	new Count(this.n+1);
  
\end{xten}

In practice, it make sense to distinguish the functions of the constraint language from the function of the base language and define the {\sc T-Fun} method on a case-by-case basis to lift expressions of the base language to the constraint language.

\FXD corresponds to the \CFJ calculus presented
in prior work~\cite{constrained-types}.  As described there, \Xten
supports equality constraints and has been extended with constraint
systems for Presburger arithmetic and for set constraints over
\Xten's array sub-language.

\subsection{\FXG}
We now turn to showing how \FGJ{}-style generics can be supported in the \FX{} family.
\FXG{} is the language obtained by adding to \FXZ{} the
following productions:
\begin{center}
\begin{tabular}{r@{\quad}rcl}
  (Expression)& {\tt e} &{::=}& ${\tt C}\{{\tt c}\}$ \\
  (Value)& {\tt v} &{::=}& ${\tt C}\{{\tt c}\}$ \\
  (Path)& {\tt p} &{::=}& ${\tt x}$ \alt {\tt p}.{\tt f} \\
  (Type)& {\tt T} &{::=}& ${\tt p}$ \alt \type \\
  (Constraint term)& {\tt t} &{::=}& ${\tt T}$ \\
  (Constraint) & {\tt c} &{::=}& ${\tt T}\extends {\tt T}$
\end{tabular}
\end{center}
\noindent
and deduction rules of Figure~\ref{fig:FXG}.

First we introduce the ``type'' \type. \FGJ{} method type
parameters are modeled in \FXG{} as normal parameters of type
\type.\footnote{In concrete \Xten{} syntax type parameters are
distinguished from ordinary value parameters through the use of
``square'' brackets. This is particularly useful in implementing type
inference for generic parameters. We abstract these concerns away in
the abstract syntax presented in this section.}  Generic class
parameters are modeled as ordinary fields of type \type, with
parameter bound information recorded as a constraint in the class
invariant. This decision to use fields rather than parameters is
discussed further in Section~\ref{sec:parameters-vs-fields}. In brief,
it permits powerful idioms using fixed but unknown types without
requiring ``wildcards''.

The set of well-formed types is now enhanced to permit some fixed but unknown
types {\tt x} as well as \emph{path types} (cf. \cite{scala}), that is use type-valued fields of objects as types.\footnote{But we will not permit invocations of methods with return type \type\ to be 
used as types. This does indeed make sense, but developing
this theory further is beyond the scope of this paper.} We extend $\sigma$ in the obvious way:
%
\begin{center}
\begin{tabular}{l}
$	\sigma({\tt x}:\type, \Gamma)=\sigma(\Gamma)$\\
$\sigma({\tt x}:{\tt y}, \Gamma)=\sigma(\Gamma)$\\
$\sigma({\tt x}:{\tt p}.{\tt f}, \Gamma)=\sigma(\Gamma)$
\end{tabular}
\end{center}
%
Reciprocally, we permit class types ${\tt C}\{{\tt c}\}$ to be used as expressions. We type them accordingly ({\sc T-Type}). In contrast, the ``type'' \type{} is neither a valid expression nor a class type: it has no field, method, subclass, or superclass. It may however be constrained as usual as for instance in rule ({\sc T-Type}), that is to say we permit equality constraints over types.\footnote{Type equality is just equality over uninterpreted functions.}

The key idea is that information about type-valued expressions can
be accumulated through constraints. Specifically we introduce 
the ``extends'' constraint ${\tt T}\extends{\tt U}$. It may be used, for
instance, to specify upper bounds on type variables or fields (path
types). In \FXG{}, users are permitted to specify ``$==$'' and ``$\extends$'' constraints
about type variables, fields, and class types.

\begin{example}
The \FGJ{} parametric method

\begin{xten} 
<T> T id(T x) { return x; }
\end{xten}
\noindent can be represented as
\begin{xten} 
def id(T: type, x: T): T = x;
\end{xten}
\end{example}

\begin{example}
\noindent The \FGJ{} class 
\begin{xten} 
class Comparator<B> {
  int compare(B y) { ... } }
class SortedList<T extends Comparator<T>> { 
  int m(T x, T y) { return x.compare(y); } }
\end{xten}
\noindent can be represented as
\begin{xtenmath} 
class Comparator(B: type) {
  def compare(y:B):int = ...; }
class SortedList(T: type)
    {T $\extends$ Comparator{self.B==T}} { 
  def m(x:T, y:T):int = x.compare(y); }
\end{xtenmath}
\end{example}

We require the underlying constraint system $\mathcal{G}$ to treat ``$\extends$'' as a partial order relation (reflexive, antisymmetric, and transitive). It is possible for a program to specify constraints incompatible with the class hierarchy, e.g., ${\tt x}\extends{\tt C}$ and ${\tt x}\extends{\tt D}$ if both class {\tt C} and class {\tt D} extend {\tt Object}. We therefore require $\mathcal{G}$ to treat as inconsistent all sets of constraints on type-valued variables that admit no valuations where these variables take on classes as values.

The ``$\extends$'' constraint is used in two deduction rules. If type {\tt T} extends type {\tt U} then
\begin{itemize}
\item{\sc S-Extends}. {\tt T} is a subtype of {\tt U}. A method or constructor with argument type {\tt U} may be passed a parameter of type {\tt T}.
\item{\sc L-Extends}. If {\tt x} has type {\tt U} then {\tt x} has all the members of type {\tt T}. Note we only extend the ``$\underline\has$'' predicate that is used in typing judgments. On the other hand, the ``$\has$'' predicate used for method lookup in the operational semantics is not affected.
\end{itemize}

The modification of the lookup predicate is
necessary to permit typing method invocations with receivers of
generic types. It has the unfortunate side effect that we can no
longer ensure that type derivations---and hence types---are unique.
For instance, given the class definitions:
%
\begin{xten}
class A() extends Object { def m():A = new A(); }
class B() extends A { def m():B new B(); }
class C(f:type){this.f<=A} extends Object {}
class D(){this.f<=B} extends C { ..this.f.m().. }
\end{xten}
%
occurrences of $\this.{\tt f}$ in {\tt D} are bounded both by {\tt A} and {\tt B} hence 
$\this.{\tt f}.{\tt m}()$ may either be typed using the declaration of {\tt m} in {\tt A} or {\tt B}.

\paragraph{Type casts.} 
Consider the program:
\begin{xten}
class C() extends Object {}
class D(f:type, g:this.f) extends Object {}
\end{xten}
Class {\tt D} has a type parameter {\tt f} and a value field {\tt g} of type {\tt f}.
Thanks to constraints, if
${\tt e}=\new~{\tt D}({\tt C},\new~{\tt C}())$,
then expression ${\tt e}.{\tt g}$ can be shown to
have type {\tt C}.
In contrast $({\tt e}~\as~{\tt D}).{\tt g}$ has type
$\exists {\tt x}:{\tt D}.{\tt x}.{\tt f}\{\self=={\tt x.g}\}$.
The type of $({\tt e}~\as~{\tt D}).{\tt g}$ is essentially ``unknown''
because the cast erased all information about it. In \Xten, we choose to shield users from existential types and only permit casts of the form $({\tt e}~\as~{\tt D}\{\self.{\tt f}=={\tt t}\})$ where {\tt t} is a type in scope (class type, type parameter, or path type).


\subsection{\FXGD} 

No additional rules are needed beyond those of \FXG{} and \FXD{}. This
language permits type and value constraints, supporting \FGJ{} style
generics and value-dependent types.

\subsection{Results}
The following results hold for \FXGD supposing the program {\tt P} type checks.

\begin{theorem}[Subject Reduction] If $\Gamma$ is well formed and $\Gamma \vdash {\tt e:T}$ and ${\tt e} \derives {\tt e'}$, then
for some type {\tt S}, $\Gamma \vdash {\tt e':S},{S \subtype T}$.
\end{theorem}

Values are of the form $\tt v ::= \new\ C(\bar{\tt v}) \alt {\tt q}() \alt C\{c\}$.

\begin{theorem}[Progress]
If $\vdash {\tt e:T}$ then one of the following conditions holds:
\begin{enumerate}
\item {\tt e} is a value,
\item {\tt e} contains a cast sub-expression that is stuck,
\item there exists an $\tt e'$ s.t. $\tt e\derives e'$.
\end{enumerate}
\end{theorem}

\begin{theorem}[Type soundness]
If $\vdash {\tt e:T}$ and {\tt e}
reduces to a normal form ${\tt e'}$, then
either $\tt e'$ is a value {\tt v} and $\vdash {\tt v:S},{\tt S\subtype T}$ or
${\tt e'}$ contains  a stuck cast sub-expression.
\end{theorem}

\paragraph{Proof sketch.} The proof of the same results for a formal language essentially equivalent to \FXZ{} has been reported in \cite{constrained-types}. We discuss here the key insights that permit to revise this proof in order to encompass \FXGD{}.
\begin{itemize}
\item Subject reduction. Having potentially multiple types for an expression does not make the proof any harder as the subject reduction theorem let us choose {\tt S} among the possible types or ${\tt e}'$.

The main novelty of the \FXG{} type system is that it permits the $\underline\has$ predicate to look for methods in arbitrary superclasses or upper bounds of the type under scrutiny. This is not so much a concern for fields as they cannot be overridden. Methods can, we thus have to adapt the proof of subject reduction for the execution step corresponding to a method invocation (\RInvk).

First we observe that the operational semantics rule for method invocations (\RInvk) is required to employ the ``correct'' method for objects of class {\tt C} that is the first method {\tt m} found on the inheritance path from class {\tt C} to class {\tt Object} traversed from the bottom up. Second, thanks to overriding restrictions, we know that this method must have a return type that is a subtype of any other method {\tt m} defined in any superclass of {\tt C}. Finally, because constraint sets incompatible with the class hierarchy are rejected as inconsistent, we also know that the type of $\new~{\tt C}(\tbar{e})$ cannot be constrained to have any upper bound that is not {\tt C} itself or one of its superclasses. We therefore derive that any method instance one could use to type the expression $\new~{\tt C}(\tbar{e}).{\tt m}(\tbar{a})$ has a return type that is a supertype of the return type of the only method instance that can be used to make a step of execution. We assumed the program type checks, hence by {\sc OK-Method}, we know that the actual residue ${\tt b}[\new~{\tt C}(\tbar{e}),\tbar{a}/\this,\tbar{x}]$ is guaranteed to have a type that is a subtype of its declared type. Therefore, by transitivity of the subtyping relation, we can derive that if {\tt T} is a type of $\new~{\tt C}(\tbar{e}).{\tt m}(\tbar{a})$ then there exists a type of {\tt S} of ${\tt b}[\new~{\tt C}(\tbar{e}),\tbar{a}/\this,\tbar{x}]$ that is a subtype of {\tt T}.
\item Progress. \FXGD{} only differs from \FXZ{} in that it admits a new class of expressions: {\tt C}\{{\tt c}\}. But these are also values, therefore the proof of progress is essentially unchanged.
\item{Type soundness}. Direct consequence of the previous two theorems.
\end{itemize}



\eat{We proceed by induction on the last rule used in the proof of ${\tt e} \derives {\tt e'}$. The key case is {\sc R-Invk}. Because of rule 


\begin{itemize}
\item {\sc RC-Field}. If $\Gamma\vdash S<:T$ then the fields of $T$ are the first fields of $S$.
\item {\sc RC-Invk-Recv}. If $\Gamma\vdash S<:T$ and $\Gamma,x:T\vdash x~\underline\has~m(y:U)\{c\}:V=a$ then there exists $d$ and $W$ such that $\Gamma,x:S\vdash x~\underline\has~m(y:U)\{d\}:W=b$ and $d$ entails $c$ and $W<:V$.
\item {\sc RC-Cast}. Straightforward.
\item {\sc RC-New-Arg}. If $\Gamma\vdash S<:T$ then $\exists x:T.U<:\exists x:S.U$.
\item {\sc RC-Invk-Arg}. Same.
\item {\sc R-Invk}. If $\Gamma,x:C\vdash x~\has~m(y:U)\{c\}:V=a$ then for all $d$ and $W$ such that $\Gamma,x:C\vdash x~\underline\has~m(y:U)\{d\}:W=b$ it is the case that $c$ entails $d$ and $V<:W$.
\item {\sc R-Field}. We have $\Gamma\vdash t:V$ for some $V$, $\Gamma\vdash \new~C(t).f_i:W_i\{\exists x:(\exists y:V.C\{\self==\new~C(y)\}).\self==x.f_i\}$ where $\Gamma\vdash C~\has~f_i:W_i$. We prove $\Gamma\vdash V_i<:W_i\{\exists x:(\exists y:V.C\{\self==\new~C(y)\}).\self==x.f_i\}$.
\item {\sc R-Cast}. Straightforward.
\end{itemize}
}


\eat{
\section{Translation}
\label{sec:translation}
\label{sec:impl}
This section describes an implementation approach for generic types in
\Xten{} on a JVM, with bytecode rewriting.

The design is a hybrid design combining techniques of run-time
instantiation from
NextGen~\cite{nextgen,allen03,allen04} and type-passing from
PolyJ~\cite{java-popl97}.  Generic classes are translated
into ``template'' classes that are instantiated on demand at run time by
binding the type properties to concrete types.
%
Constraints on values are erased from type references.
Adapter objects are used to represent type
properties and constraints.  
Run-time type tests (e.g., casts) are translated
into code that checks those constraints at run time.
%
This design has been implemented in the \Xten{} compiler, built
on the Polyglot compiler framework~\cite{ncm03}.  The compiler
translates \Xten{} source to Java source, which is then compiled
to Java bytecode using an off-the-shelf Java compiler.\footnote{There is also
a translation from \Xten{} to C++ source, not described here.}
The \Xten{} runtime is augmented with a class loader
implementation that performs run-time instantiation.

\paragraph{Classes.}
Each class is translated into a \emph{template class}.
The template class is compiled by a Java compiler (e.g., javac)
to produce a class file.
At run time, when a constrained type \xcd"C{c}" is first referenced, a
class loader loads the template class for \xcd"C" and then
transforms its bytecode, specializing it to the constraint
\xcd"c".  The implementation specializes code based on type constraints,
not value constraints; we leave value-constraint specialization to
future work.
%
For example, consider the following classes.
{
\begin{xten}
class A[X] {
  var a: X;
}
\end{xten}}
{
\begin{xten}
class C {
  val x: A[int] = new A[int]();
  val y: int = x.a;
}
\end{xten}}

The compiler generates the following Java code:
{
\begin{xten}
@Parameters({"X"})
class A {
  @TypeProperty public static class X { }
  public x10.runtime.Type X;
  X a;
  @Synthetic public A(Class X) { this(); }
}
\end{xten}}
{
\begin{xten}
class C {
  final A x = new A(int.class);
  final int y = Runtime.to$int(x.a);
}
\end{xten}}

The member class \xcd"A.X" is used in place of the
type property \xcd"X".   The field \xcd"X" of type
\xcd"x10.runtime.Type" captures the actual constrained type on which \xcd"A"
is instantiated, and is used for run-time type tests.
The \xcd"@Parameters" annotation on \xcd"A" is used during
run-time instantiation to identify the type properties.
Synthetic constructors with added \xcd"Class" parameters are
used to pass instantiation arguments to the \xcd"new"
expression.
This code is compiled to Java bytecode.

When an expression (e.g., \xcd"new C()") is evaluated,
the class \xcd"C" is loaded.
The class loader transforms the bytecode as if it had
been written as follows:

{
\begin{xten}
class C {
  final A$$int x = new A$$int();
  final int y = x.a;
}
\end{xten}}

The class loader rewrites allocations of template classes
(e.g., \xcd"new A(int.class)") into allocations of the
instantiated classes (i.e., \xcd"new A$$int()").
The template class name and actual type arguments are mangled to
derive the name of the instantiated class.
This code cannot be generated directly because
class \xcd"A$$int" does not yet exist; the Java source compiler
would fail to compile \xcd"C".

Upon evaluation of the constructor,
the class \xcd"A$$int" is loaded.
The class loader intercepts
this, demangles the name, and loads the bytecode for the
template class \xcd"A".
The bytecode is transformed, replacing the type property \xcd"X"
with the concrete type \xcd"int".

Parameter types are coerced to and from the actual type
\xcd"T" (a Java primitive type or \xcd"Object") using
methodr \xcd"Runtime.to$T(Object)" and \xcd"Runtime.from(T)",
possibly with additional casts.
Both are eliminated from the transformed
bytecode, but are needed for the template class to type-check.

\eat{
Currently, the class loader instantiates the template for
every encountered combination of parameters.  If desired,
it is possible (and relatively easy) to optimize this scheme
to instantiate only for the Java primitive types and Object,
giving nine possible instantiations per parameter.
}

%Instantiations are used for representation.
%Adapter objects are used for run time type information.
%
%Could do instantiation eagerly, but quickly gets out of hand without
%whole-program analysis to limit the number of instantiations: 9
%instantiations for one type property, 81 for two type
%properties, 729 for three.

%Constructors are translated to static methods of their
%enclosing
%class.
%Constructor calls
%are translated to calls to static methods.

\eat{
Consider the code in Figure~\ref{fig:translation1}.  It contains most of the
features of generics that have to be translated.
\begin{figure*}[tp]
{\footnotesize
\begin{xtenmath}
class C[T] {
    var x: T;
    def this[T](x: T) { this.x = x; }
    def set(x: T) { this.x = x; }
    def get(): T { return this.x; }
    def map[S](f: F[T,S]): S { return f._(this.x); }
    def d() { return new D[T](); }
    def t() { return new T(); } // FIXME
    def isa(y: Object): boolean { return y instanceof C[T]; }
}
abstract class F[T,S] { S _(T x); }

val x : C = new C[String]();
val y : C[Int] = new C[Int]();
val z : C{T $\extends$ Array} = new C[Array[Int]]();
val f : F[String,Int] = ...;
x.map[Int](f);
new C[Int{self==3}]() instanceof C[Int{self<4}];
\end{xten}}
\caption{Code to translate}
\label{fig:translation1}
\end{figure*}

The translated version is shown in Figure~\ref{fig:translation2}.
\begin{figure*}[tp]
{\footnotesize
\begin{xten}
@Parameters({"T"})
class C {
    @TypeProperty public static class T { }
    T x;
    C(T x) { this.x = x; }
    @Synthetic C(Class T, T x) { this(x); }
    @Synthetic public static boolean instanceof\$(Object o, String constraint) { assert(false); return true; }
    public static boolean instanceof\$(Object o, String constraint, boolean b) { /*check constraint*/; return b; }
    public static Object cast\$(Object o, String constraint) { /*check constraint*/; return (C)o; }
    void set(T x) { this.x = x; }
    T get() { return this.x; }
    @Synthetic
    @Parameters("S")
    public static class map {
        public static class S { };
        public C c;
        public map(C c) { this.c = c; }
        @Synthetic
        public map(Class S, C c) { this(c); }
        public S apply(@InstantiateClass({"C\$T", "C\$map\$S"}) F f) { return f._(c.x); }
        @Synthetic
        public T apply(Class T, T x, T y) { return apply(x, y); } // We might only need one
    }
    @Synthetic
    @ParametricMethod("T")
    Object make\$map(Class T) { assert(false); return null; }
    @Synthetic
    Object make\$map(Class T, boolean ignored) {
        Object retval = null;
        try {
            X10RuntimeClassloader cl = (X10RuntimeClassloader)C.class.getClassLoader();
            Class<?> c = cl.instantiate(map.class, T); 
            retval = c.getDeclaredConstructor(new Class[] { C.class }).newInstance(this);
        }
        catch (IllegalAccessException e) { }
        catch (NoSuchMethodException e) { }
        catch (InstantiationException e) { }
        catch (InvocationTargetException e) { }
        return retval;
    }
    @InstantiateClass({"C\$T"}) D d() { return new D(T.class); }
    T t() { return new T(); } // FIXME
    boolean isa(Object y) { return Runtime.instanceof\$(C.instanceof\$(y, null), T.class); }
}
@Parameters({"T","S"})
abstract class F { ... }

C x = new C(String.class);
C y = new C(int.class);
C z = new C(((X10RuntimeClassloader)C.class.getClassLoader()).getClass("Array\$\$int"));
F f = ...;
((C.map)(Object)(C.map)x.make\$map(int.class)).apply(int.class, f);

Runtime.instanceof\$(C.instanceof\$(new C(int.class)(), "self<4"), int.class);
\end{xten}}
\caption{Translated code}
\label{fig:translation2}
\end{figure*}
}

\paragraph{Passing type arguments.}

For types visible at run time, annotations are used to
pass actual type arguments.  The annotation \xcd"@InstantiateClass"
is placed on
fields, methods,
method parameters, and classes to
indicate instantiation parameters for field
types, method return types, method parameters, and superclasses,
respectively.
Interface instantiations are similarly handled
by \xcd"@InstantiateInterfaces".
The annotation
\xcd"@Instantiation"
is used for parametrized exceptions.

\todo{IP: explain what the annotations are used for. --NN}

Type arguments are passed to allocation expressions as
synthetic constructor arguments.  Run-time type tests and casts
receive type parameters via the \xcd"Runtime.cast$" and
\xcd"Runtime.instanceof$" helper methods.

\paragraph{Eliminating method type parameters.}

For each parametrized method, a parametrized
adapter class with an \xcd"apply" method.  The adapter class
is annotated with \xcd"@ParametricMethod".
The parametrized method is invoked by instantiating the adapter
class through a generated factory method
and invoking its \xcd"apply()" method.

\paragraph{Parametrized exceptions.}

Parametrized exceptions are treated just like other classes.
Synthetic local classes, annotated with \xcd"@Instantiation",
are generated for each catch block with an instantiated
generic exception class.  Exception tables in the
bytecode are rewritten with the new exception types.

\paragraph{Run-time instantiation.}

The
\xcd"instanceof" and cast operations on
constrained types or type variables
are translated
to
similar operations on the instantiated type followed by calls
to
methods of the adapter object for the type
that evaluate the constraint.
% run-time constraint solving or other
% complex code that cannot be easily substituted in when rewriting
% the bytecode during instantiation.

}

\section{Conclusions}
\label{sec:conclusions}

We have presented a constraint-based framework \FXG{} for type-
and value-dependent types in an object-oriented language.
%
The use of constraints on type properties allows the design to
capture many features of generics in object-oriented languages
and then to extend these features with more
expressive power.  We have proved the type system sound.

We plan to extend the type system to account for mutable state.
We believe the extension is straightforward, although
cumbersome, because
constraints are only immutable state only and because the formalism
carefully controls occurrences of existential types.

The type system of \FXG formalizes the semantics of the \Xten{}
programming language.  The design admits an efficient
implementation for generics and dependent types in \Xten{},
available at \texttt{x10-lang.org}.
To improve the expressiveness of \Xten{}, we plan to implement
a type inference algorithm that infers constraints over types
and values, and to support user-defined constraints.

\section*{Acknowledgments} 
This material is based upon work supported in part by the Defense Advanced Research Projects Agency under its Agreement No HR0011-07-9-0002.


\eat{
\section*{Acknowledgments} 

The authors thank Bob Blainey,
Doug Lea, Jens Palsberg, and Lex Spoon
for valuable feedback on versions of the language.
We thank
Andrew Myers and
Michael Clarkson for providing us with their implementation of
PolyJ, on which our implementation was based, and for many
discussions over the years about parametrized types in Java.
}

\bibliographystyle{plain}
\bibliography{master}

% \appendix
% \onecolumn

% \section{An extended example}
% {\footnotesize
\begin{verbatim}
/**
   A distributed binary tree.
   @author Satish Chandra 4/6/2006
   @author vj
 */
//                             ____P0
//                            |     |
//                            |     |
//                          _P2  __P0
//                         |  | |   |
//                         |  | |   |
//                        P3 P2 P1 P0
//                         *  *  *  *
// Right child is always on the same place as its parent;
// left child is at a different place at the top few levels of the tree,
// but at the same place as its parent at the lower levels.

class Tree(localLeft: boolean,
           left: nullable Tree(& localLeft => loc=here),
           right: nullable Tree(& loc=here),
           next: nullable Tree) extends Object {
    def postOrder:Tree = {
        val result:Tree = this;
        if (right != null) {
            val result:Tree = right.postOrder();
            right.next = this;
            if (left != null) return left.postOrder(tt);
        } else if (left != null) return left.postOrder(tt);
        this
    }
    def postOrder(rest: Tree):Tree = {
        this.next = rest;
        postOrder
    }
    def sum:int = size + (right==null => 0 : right.sum()) + (left==null => 0 : left.sum)
}
value TreeMaker {
    // Create a binary tree on span places.
    def build(count:int, span:int): nullable Tree(& localLeft==(span/2==0)) = {
        if (count == 0) return null;
        {val ll:boolean = (span/2==0);
         new Tree(ll,  eval(ll => here : place.places(here.id+span/2)){build(count/2, span/2)},
           build(count/2, span/2),count)}
    }
}
\end{verbatim}}

\subsection{Places}
{\footnotesize
\begin{verbatim}
/**

 * This class implements the notion of places in X10. The maximum
 * number of places is determined by a configuration parameter
 * (MAX_PLACES). Each place is indexed by a nat, from 0 to MAX_PLACES;
 * thus there are MAX_PLACES+1 places. This ensures that there is
 * always at least 1 place, the 0'th place.

 * We use a dependent parameter to ensure that the compiler can track
 * indices for places.
 *
 * Note that place(i), for i <= MAX_PLACES, can now be used as a non-empty type.
 * Thus it is possible to run an async at another place, without using arays---
 * just use async(place(i)) {...} for an appropriate i.

 * @author Christoph von Praun
 * @author vj
 */

package x10.lang;

import x10.util.List;
import x10.util.Set;

public value class place (nat i : i <= MAX_PLACES){

    /** The number of places in this run of the system. Set on
     * initialization, through the command line/init parameters file.
     */
    config nat MAX_PLACES;

    // Create this array at the very beginning.
    private constant place value [] myPlaces = new place[MAX_PLACES+1] fun place (int i) {
	return new place( i )(); };

    /** The last place in this program execution.
     */
    public static final place LAST_PLACE = myPlaces[MAX_PLACES];

    /** The first place in this program execution.
     */
    public static final place FIRST_PLACE = myPlaces[0];
    public static final Set<place> places = makeSet( MAX_PLACES );

    /** Returns the set of places from first place to last place.
     */
    public static Set<place> makeSet( nat lastPlace ) {
	Set<place> result = new Set<place>();
	for ( int i : 0 .. lastPlace ) {
	    result.add( myPlaces[i] );
	}
	return result;
    }

    /**  Return the current place for this activity.
     */
    public static place here() {
	return activity.currentActivity().place();
    }

    /** Returns the next place, using modular arithmetic. Thus the
     * next place for the last place is the first place.
     */
    public place(i+1 % MAX_PLACES) next()  { return next( 1 ); }

    /** Returns the previous place, using modular arithmetic. Thus the
     * previous place for the first place is the last place.
     */
    public place(i-1 % MAX_PLACES) prev()  { return next( -1 ); }

    /** Returns the k'th next place, using modular arithmetic. k may
     * be negative.
     */
    public place(i+k % MAX_PLACES) next( int k ) {
	return places[ (i + k) % MAX_PLACES];
    }

    /**  Is this the first place?
     */
    public boolean isFirst() { return i==0; }

    /** Is this the last place?
     */
    public boolean isLast() { return i==MAX_PLACES; }
}
\end{verbatim}}
\subsection{$k$-dimensional regions}
{\footnotesize
\begin{verbatim}
package x10.lang;

/** A region represents a k-dimensional space of points. A region is a
 * dependent class, with the value parameter specifying the dimension
 * of the region.
 * @author vj
 * @date 12/24/2004
 */

public final value class region( int dimension : dimension >= 0 )  {

    /** Construct a 1-dimensional region, if low <= high. Otherwise
     * through a MalformedRegionException.
     */
    extern public region (: dimension==1) (int low, int high)
        throws MalformedRegionException;

    /** Construct a region, using the list of region(1)'s passed as
     * arguments to the constructor.
     */
    extern public region( List(dimension)<region(1)> regions );

    /** Throws IndexOutOfBoundException if i > dimension. Returns the
        region(1) associated with the i'th dimension of this otherwise.
     */
    extern public region(1) dimension( int i )
        throws IndexOutOfBoundException;


    /** Returns true iff the region contains every point between two
     * points in the region.
     */
    extern public boolean isConvex();

    /** Return the low bound for a 1-dimensional region.
     */
    extern public (:dimension=1) int low();

    /** Return the high bound for a 1-dimensional region.
     */
    extern public (:dimension=1) int high();

    /** Return the next element for a 1-dimensional region, if any.
     */
    extern public (:dimension=1) int next( int current )
        throws IndexOutOfBoundException;

    extern public region(dimension) union( region(dimension) r);
    extern public region(dimension) intersection( region(dimension) r);
    extern public region(dimension) difference( region(dimension) r);
    extern public region(dimension) convexHull();

    /**
       Returns true iff this is a superset of r.
     */
    extern public boolean contains( region(dimension) r);
    /**
       Returns true iff this is disjoint from r.
     */
    extern public boolean disjoint( region(dimension) r);

    /** Returns true iff the set of points in r and this are equal.
     */
    public boolean equal( region(dimension) r) {
        return this.contains(r) && r.contains(this);
    }

    // Static methods follow.

    public static region(2) upperTriangular(int size) {
        return upperTriangular(2)( size );
    }
    public static region(2) lowerTriangular(int size) {
        return lowerTriangular(2)( size );
    }
    public static region(2) banded(int size, int width) {
        return banded(2)( size );
    }

    /** Return an \code{upperTriangular} region for a dim-dimensional
     * space of size \code{size} in each dimension.
     */
    extern public static (int dim) region(dim) upperTriangular(int size);

    /** Return a lowerTriangular region for a dim-dimensional space of
     * size \code{size} in each dimension.
     */
    extern public static (int dim) region(dim) lowerTriangular(int size);

    /** Return a banded region of width {\code width} for a
     * dim-dimensional space of size {\code size} in each dimension.
     */
    extern public static (int dim) region(dim) banded(int size, int width);


}

\end{verbatim}}

\subsection{Point}
{\footnotesize
\begin{verbatim}
package x10.lang;

public final class point( region region ) {
    parameter int dimension = region.dimension;
    // an array of the given size.
    int[dimension] val;

    /** Create a point with the given values in each dimension.
     */
    public point( int[dimension] val ) {
        this.val = val;
    }

    /** Return the value of this point on the i'th dimension.
     */
    public int valAt( int i) throws IndexOutOfBoundException {
        if (i < 1 || i > dimension) throw new IndexOutOfBoundException();
        return val[i];
    }

    /** Return the next point in the given region on this given
     * dimension, if any.
     */
    public void inc( int i )
        throws IndexOutOfBoundException, MalformedRegionException {
        int val = valAt(i);
        val[i] = region.dimension(i).next( val );
    }

    /** Return true iff the point is on the upper boundary of the i'th
     * dimension.
     */
    public boolean onUpperBoundary(int i)
        throws IndexOutOfBoundException {
        int val = valAt(i);
        return val == region.dimension(i).high();
    }

    /** Return true iff the point is on the lower boundary of the i'th
     * dimension.
     */
    public boolean onLowerBoundary(int i)
        throws IndexOutOfBoundException {
        int val = valAt(i);
        return val == region.dimension(i).low();
    }
}
\end{verbatim}}

\subsection{Distribution}
{\footnotesize
\begin{verbatim}
package x10.lang;

/** A distribution is a mapping from a given region to a set of
 * places. It takes as parameter the region over which the mapping is
 * defined. The dimensionality of the distribution is the same as the
 * dimensionality of the underlying region.

   @author vj
   @date 12/24/2004
 */

public final value class distribution( region region ) {
    /** The parameter dimension may be used in constructing types derived
     * from the class distribution. For instance,
     * distribution(dimension=k) is the type of all k-dimensional
     * distributions.
     */
    parameter int dimension = region.dimension;

    /** places is the range of the distribution. Guranteed that if a
     * place P is in this set then for some point p in region,
     * this.valueAt(p)==P.
     */
    public final Set<place> places; // consider making this a parameter?

    /** Returns the place to which the point p in region is mapped.
     */
    extern public place valueAt(point(region) p);

    /** Returns the region mapped by this distribution to the place P.
        The value returned is a subset of this.region.
     */
    extern public region(dimension) restriction( place P );

    /** Returns the distribution obtained by range-restricting this to Ps.
        The region of the distribution returned is contained in this.region.
     */
    extern public distribution(:this.region.contains(region))
        restriction( Set<place> Ps );

    /** Returns a new distribution obtained by restricting this to the
     * domain region.intersection(R), where parameter R is a region
     * with the same dimension.
     */
    extern public (region(dimension) R) distribution(region.intersection(R))
        restriction();

    /** Returns the restriction of this to the domain region.difference(R),
        where parameter R is a region with the same dimension.
     */
    extern public (region(dimension) R) distribution(region.difference(R))
        difference();

    /** Takes as parameter a distribution D defined over a region
        disjoint from this. Returns a distribution defined over a
        region which is the union of this.region and D.region.
        This distribution must assume the value of D over D.region
        and this over this.region.

        @seealso distribution.asymmetricUnion.
     */
    extern public (distribution(:region.disjoint(this.region) &&
                                dimension=this.dimension) D)
        distribution(region.union(D.region)) union();

    /** Returns a distribution defined on region.union(R): it takes on
        this.valueAt(p) for all points p in region, and D.valueAt(p) for all
        points in R.difference(region).
     */
    extern public (region(dimension) R) distribution(region.union(R))
        asymmetricUnion( distribution(R) D);

    /** Return a distribution on region.setMinus(R) which takes on the
     * same value at each point in its domain as this. R is passed as
     * a parameter; this allows the type of the return value to be
     * parametric in R.
     */
    extern public (region(dimension) R) distribution(region.setMinus(R))
        setMinus();

    /** Return true iff the given distribution D, which must be over a
     * region of the same dimension as this, is defined over a subset
     * of this.region and agrees with it at each point.
     */
    extern public (region(dimension) r)
        boolean subDistribution( distribution(r) D);

    /** Returns true iff this and d map each point in their common
     * domain to the same place.
     */
    public boolean equal( distribution( region ) d ) {
        return this.subDistribution(region)(d)
            && d.subDistribution(region)(this);
    }

    /** Returns the unique 1-dimensional distribution U over the region 1..k,
     * (where k is the cardinality of Q) which maps the point [i] to the
     * i'th element in Q in canonical place-order.
     */
    extern public static distribution(:dimension=1) unique( Set<place> Q );

    /** Returns the constant distribution which maps every point in its
        region to the given place P.
    */
    extern public static (region R) distribution(R) constant( place P );

    /** Returns the block distribution over the given region, and over
     * place.MAX_PLACES places.
     */
    public static (region R) distribution(R) block() {
        return this.block(R)(place.places);
    }

    /** Returns the block distribution over the given region and the
     * given set of places. Chunks of the region are distributed over
     * s, in canonical order.
     */
    extern public static (region R) distribution(R) block( Set<place> s);


    /** Returns the cyclic distribution over the given region, and over
     * all places.
     */
    public static (region R) distribution(R) cyclic() {
        return this.cyclic(R)(place.places);
    }

    extern public static (region R) distribution(R) cyclic( Set<place> s);

    /** Returns the block-cyclic distribution over the given region, and over
     * place.MAX_PLACES places. Exception thrown if blockSize < 1.
     */
    extern public static (region R)
        distribution(R) blockCyclic( int blockSize)
        throws MalformedRegionException;

    /** Returns a distribution which assigns a random place in the
     * given set of places to each point in the region.
     */
    extern public static (region R) distribution(R) random();

    /** Returns a distribution which assigns some arbitrary place in
     * the given set of places to each point in the region. There are
     * no guarantees on this assignment, e.g. all points may be
     * assigned to the same place.
     */
    extern public static (region R) distribution(R) arbitrary();

}
\end{verbatim}}

\subsection{Arrays}
Finally we can now define arrays. An array is built over a
distribution and a base type.

{\footnotesize
\begin{verbatim}
package x10.lang;

/** The class of all  multidimensional, distributed arrays in X10.

    <p> I dont yet know how to handle B@current base type for the
    array.

 * @author vj 12/24/2004
 */

public final value class array ( distribution dist )<B@P> {
    parameter int dimension = dist.dimension;
    parameter region(dimension) region = dist.region;

    /** Return an array initialized with the given function which
        maps each point in region to a value in B.
     */
    extern public array( Fun<point(region),B@P> init);

    /** Return the value of the array at the given point in the
     * region.
     */
    extern public B@P valueAt(point(region) p);

    /** Return the value obtained by reducing the given array with the
        function fun, which is assumed to be associative and
        commutative. unit should satisfy fun(unit,x)=x=fun(x,unit).
     */
    extern public B reduce(Fun<B@?,Fun<B@?,B@?>> fun, B@? unit);


    /** Return an array of B with the same distribution as this, by
        scanning this with the function fun, and unit unit.
     */
    extern public array(dist)<B> scan(Fun<B@?,Fun<B@?,B@?>> fun, B@? unit);

    /** Return an array of B@P defined on the intersection of the
        region underlying the array and the parameter region R.
     */
    extern public (region(dimension) R)
        array(dist.restriction(R)())<B@P>  restriction();

    /** Return an array of B@P defined on the intersection of
        the region underlying this and the parametric distribution.
     */
    public  (distribution(:dimension=this.dimension) D)
        array(dist.restriction(D.region)())<B@P> restriction();

    /** Take as parameter a distribution D of the same dimension as *
     * this, and defined over a disjoint region. Take as argument an *
     * array other over D. Return an array whose distribution is the
     * union of this and D and which takes on the value
     * this.atValue(p) for p in this.region and other.atValue(p) for p
     * in other.region.
     */
    extern public (distribution(:region.disjoint(this.region) &&
                                dimension=this.dimension) D)
        array(dist.union(D))<B@P> compose( array(D)<B@P> other);

    /** Return the array obtained by overlaying this array on top of
        other. The method takes as parameter a distribution D over the
        same dimension. It returns an array over the distribution
        dist.asymmetricUnion(D).
     */
    extern public (distribution(:dimension=this.dimension) D)
        array(dist.asymmetricUnion(D))<B@P> overlay( array(D)<B@P> other);

    extern public array<B> overlay(array<B> other);

    /** Assume given an array a over distribution dist, but with
     * basetype C@P. Assume given a function f: B@P -> C@P -> D@P.
     * Return an array with distribution dist over the type D@P
     * containing fun(this.atValue(p),a.atValue(p)) for each p in
     * dist.region.
     */
    extern public <C@P, D>
        array(dist)<D@P> lift(Fun<B@P, Fun<C@P, D@P>> fun, array(dist)<C@P> a);

    /**  Return an array of B with distribution d initialized
         with the value b at every point in d.
     */
    extern public static (distribution D) <B@P> array(D)<B@P> constant(B@? b);

}
\end{verbatim}}


\begin{example}
 The code for {\tt List} translates as given in Table~\ref{List-translation}.
\end{example}

\begin{figure*}
{\footnotesize
\begin{verbatim}
  public value class List <Node> {
    public final nat n;   // is a parameter
    nullable Node node = null;
    nullable List<Node> rest = null;  // All assignments must check n = this.n-1.

    /** Returns the empty list. Defined only when the parameter n
        has the value 0. Invocation: new List(0)<Node>().
     */
    public List ( final nat n ) {
      assume n==0;
      this.n = n;
    }

    /** Returns a list of length 1 containing the given node.
        Invocation: new List(1)<Node>( node ).
     */
    public List ( final nat n, Node node ) {
      assume n==1;                         // From the constructor precondition.
      assert 0==0 : "DependentTypeError"; // For the constructor call.
      assert n>=1 : "DependentTypeError"; // For the this call.
      this(n, node, new List<Node>(0));
    }

    public List ( final nat n, Node node, List<Node> rest ) {
      assume n>=1;                               // From the constructor precondition
      assume rest.n==n-1 : "DependentTypeError"; // From the argument type.
      this.n = n;
      this.node = node;
      assert rest.n==n-1 : "DependentTypeError"; // For the field assignment.
      this.rest = rest;
    }

    public  List<Node> append( List<Node> arg ) {
      if (n == 0) {
          final List<Node> result = arg;
          assert n+arg.n == result.n : "DependentTypeError"; // For the return value
          return result;
      } else {
          assume rest.n == n-1;
          final List<Node> argval = rest.append(arg);
          assume argval.n == rest.n+arg.n;
          assert n+arg.n-1== argval.n : "DependentTypeError"; // For the constructor call.
          final List<Node> result = new List<Node>(n+arg.n, node, argval);
          assume result.n == n+arg.n;
          assert n+arg.n == result.n : "DependentTypeError"; // For the return value
          return result;
      }
    }

\end{verbatim}}
\caption{Translation of {\tt List} (contd in Table~\ref{List-translation-2}).}\label{List-translation}
\end{figure*}
\begin{figure*}
{\footnotesize
\begin{verbatim}
    public  List<Node> rev() {
      final List<Node> arg = new List<Node>(0);
      assume arg.n = 0;                           // From the constructor call.
      final List<Node> result = rev( arg );
      assume result.n == n+arg.n;                  // From the method signature
      assert n == result.n : "DependentTypeError"; // For the return value.
      return result;
    }

    public  List(n+arg.n)<Node> rev( final List<Node> arg) {
      if (n==0) {
         assert n+arg.n == arg.n : "DependentTypeError"; // For the return value.
         return arg;
      } else {
        assert 1+arg.n-1=arg.n : "DependentTypeError"; // For the argument to the constructor
        final List<Node> arg2 = new List<Node>(1+arg.n,node, arg));
        assume arg2.n==1+arg.n;                      // From the constructor invocation
        final List<Node> restval = rest;             // Read from a mutable field of parametric type
        assume restval.n == n-1;                     // From the field read.
        final List(restval.n+arg2.n)<Node> result = restval.rev( arg2 );
        assume result.n=restval.n+arg2.n
        assert n+arg.n == result.n                   // For the return value
        return result;
    }

    /** Return a list of compile-time unknown length, obtained by filtering
        this with f. */
    public List<Node> filter(fun<Node, boolean> f) {
         if (n==0) return this;
         if (f(node)) {
           final List<Node> l = rest.filter(f);
           assert l.n+1-1==l.n : "DependentTypeError"; // For the constructor call
           return new List<Node>(l.n+1,node, l);
         } else {
           return rest.filter(f);
         }
    }

    /** Return a list of m numbers from o..m-1. */
    public static  List<nat> gen( final nat m ) {
         assert 0 <= m : "DependentTypeError";        // Precondition for method call.
         final List<nat> result = gen(0,m);
         assume result.n=m-0 : "DependentTypeError";  // From the method signature
         assert m == result.n : "DependentTypeError"; // For the return value
         return result;
    }

    /** Return a list of (m-i) elements, from i to m-1. */
    public static List<nat> gen(final nat i, final nat m) {
      assume i <= m;                                   // Method precondition.
      if (i==m) {
        assert m-i == 0 : "DependentTypeError";        // For the constructor call
        final List result = new List<nat>(m-i);
        assume result.n == 0;                          // From the constructor call.
        assert m-i == result.n : "DependentTypeError"; // For the return value.
        return result;
      } else {
        assert i+1 <= m : "DependentTypeError";        // For the method call.
        final List<nat> arg = gen(i+1,m);
        assume arg.n = m-(i+1);                        // From the method call.
        assert m-i-1 = arg.n;                          // For the constructor invocation.
        final List result = new List<nat>(m-i, i, arg);
        assume result.n = m-i;                         // From the constructor invocation.
        assert m-i == result.n : "DependentTypeError"; // For the return value
        return result;
    }
  }
\end{verbatim}}
\caption{Translation of {\tt List} (continued).}\label{List-translation-2}
\end{figure*}

\section{Type-checking dependent classes}

Each programming language---such as \Xten{}---will specify the base
underlying classes (and the operations on them) which can occur as
types in parameter lists. For instance, in the code for {\tt List}
above, the only type that appears in parameter lists is {\tt int}, and
the only operations on {\tt int} are addition, subtraction, {\tt >=},
{\tt ==}, and the only constants are {\tt 0} and {\tt 1}.  (This
language falls within Presburger arithmetic, a decidable fragment of
arithmetic.)  The compiler must come equipped with a constraint solver
(decision procedure) that can answer questions of the form: does one
constraint entail another?  Constraints are atomic formulas built up
from these operations, using variables. For instance, the compiler
must answer each one of:
{\footnotesize
\begin{verbatim}
  n >= 2 |- n-1 >= 0
  n >= 0, m >= 0 |- m+n >= 0
\end{verbatim}}

Ultimately, the only variables that will occur in constraints are
those that correspond to {\tt config} parameters and those that are
defined by implicit parameter definitions. We need to establish that
the verification of any class will generate only a finite number of
constraints, hence only a finite constraint problem for the constraint
solver.

Second, it should be possible for instances of user-defined classes
(and operations on them) to occur as type parameters. For the compiler
to check conditions involving such values, it is necessary that the
underlying constraint solver be extended.

There are two general ways in which the constraint solver may be
extended.  Both require that the programmer single out some classes
and methods on those classes as {\em pure}. (We shall think of
constants as corresponding to zero-ary methods.) Only instances of
pure classes and expressions involving pure methods on these instances
are allowed in parameter expressions.

How shall constraints be generated for such pure methods? First, the
programmer may explicitly supply with each pure method {\tt T m(T1 x1,
..., Tn xn)} a constraint on {\tt n+2} variables in the constraint
system of the underlying solver that is entailed by {\tt y =
o.m(x1,..., xn)}. Whenever the compiler has to perform reasoning on an
expression involving this method invocation, it uses the constraint
supplied by the programmer. A second more ambitious possibility is
that a symbolic evaluator of the language may be run on the body of
the method to automatically generate the corresponding constraint.

Finally an additional possibility is that the constraint solver itself
be made extensible. In this case, when a user writes a class which is
intended to be used in specifying parameters, he also supplies an
additional program which is used to extend the underlying constraint
solver used by the compiler. This program adds more primitive
constraints and knows how to perform reasoning using these
constraints. This is how I expect we will initially implement the
\Xten{} language. As language designers and implementers we will
provide constraint solvers for finite functions and {\tt Herbrand}
terms on top of arithmetic.





\end{document}
