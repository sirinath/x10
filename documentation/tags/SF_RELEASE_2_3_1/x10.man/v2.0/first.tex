% First page

\thispagestyle{empty}

% \todo{"another" report?}

\title{Report on the Programming Language \Xten \\
\large Version \integerversion}
%\ifdraft
\author{\textsc{Draft} --- \today \\
\\
Vijay Saraswat, Bard Bloom, Igor Peshansky, Olivier Tardieu, and David Grove\\
\\
Please send comments to 
\texttt{bardb@us.ibm.com}}
%\else
%\author{
%Vijay Saraswat \\
%Please send comments to \\
%\texttt{vsaraswa@us.ibm.com}}
%\fi
%\date{\today}

\maketitle

\newcommand\authorsc[1]{#1}
%\newcommand\authorsc[1]{\textsc{#1}}

This report provides a description of the programming
language \Xten. \Xten{} is a class-based object-oriented
programming language designed for high-performance, high-productivity
computing on high-end computers supporting $\approx 10^5$ hardware threads
and $\approx 10^{15}$ operations per second. 

\Xten{} is based on state-of-the-art object-oriented programming
languages and deviates from them only as necessary to support its
design goals. The language is intended to have a simple and clear
semantics and be readily accessible to mainstream OO programmers. It
is intended to support a wide variety of concurrent programming
idioms.
%, incuding data parallelism, task parallelism, pipelining.
%producer/consumer and divide and conquer.

%We expect to revise this document in the light of experience gained in implementing
%and using this language.

The \Xten{} design team consists of
\authorsc{Bard Bloom}, 
\authorsc{Ganesh Bikshandi}, 
\authorsc{David Cunningham},
\authorsc{Robert Fuhrer},
\authorsc{David Grove},
\authorsc{Sreedhar Kodali}, 
\authorsc{Nathaniel Nystrom},
\authorsc{Igor Peshansky}, 
\authorsc{Vijay Saraswat},
\authorsc{Olivier Tardieu}.

Past members include
\authorsc{Shivali Agarwal}, 
\authorsc{David Bacon}, 
\authorsc{Raj Barik}, 
\authorsc{Bob Blainey}, 
\authorsc{Philippe Charles}, 
\authorsc{Perry Cheng}, 
\authorsc{Christopher Donawa}, 
\authorsc{Julian Dolby}, 
\authorsc{Kemal Ebcio\u{g}lu},
\authorsc{Patrick Gallop}, 
\authorsc{Christian Grothoff}, 
\authorsc{Allan Kielstra}, 
\authorsc{Sriram Krishnamoorthy}, 
\authorsc{Bruce Lucas},
\authorsc{Vivek Sarkar},
\authorsc{Armando Solar-Lezama},  
\authorsc{S. Alexander Spoon}, 
\authorsc{Sayantan Sur}, 
\authorsc{Christoph von Praun},
\authorsc{Pradeep Varma}, 
\authorsc{Krishna Nandivada Venkata},
\authorsc{Jan Vitek}, and
\authorsc{Tong Wen}.

For extended discussions and support we would like to thank: 
Gheorghe Almasi,
Robert Blackmore,
Robert Callahan, 
Calin Cascaval, 
Norman Cohen, 
Elmootaz Elnozahy, 
John Field,
Kevin Gildea,
Chulho Kim,
Orren Krieger, 
Doug Lea, 
John McCalpin, 
Paul McKenney, 
Andrew Myers,
Filip Pizlo, 
Ram Rajamony,
R.~K. Shyamasundar, 
V.~T. Rajan, 
Frank Tip,
Mandana Vaziri,
and
Hanhong Xue.


We thank Jonathan Rhees and William Clinger with help in obtaining the
\LaTeX{} style file and macros used in producing the Scheme report,
on which this document is based. We acknowledge the influence of
the $\mbox{\Java}^{\mbox{\textsc{tm}}}$ Language
Specification \cite{jls2}.
%document, as evidenced by the numerous citations in the text.

This document revises Version 1.7 of the Report, released in
September 2008.  It documents the language corresponding to Version
2.0 of the implementation. Version 1.7 of the report was co-authored by
Nathaniel Nystrom. The design of structs in \Xten{} was led by Olivier Tardieu
and Nathaniel Nystrom.

Earlier implementations benefited from significant contributions by
Raj Barik, 
Philippe Charles, 
David Cunningham,
Christopher Donawa, 
Robert Fuhrer,
Christian Grothoff,
Nathaniel Nystrom,  
Igor Peshansky,  
Vijay Saraswat,
Vivek Sarkar, 
Olivier Tardieu,  
Pradeep Varma, 
Krishna Nandivada Venkata, and
Christoph von Praun.
Tong Wen has written many application programs
in \Xten{}. Guojing Cong has helped in the
development of many applications.
The implementation of generics in \Xten{} was influenced by the
implementation of PolyJ~\cite{polyj} by Andrew Myers and Michael Clarkson.

