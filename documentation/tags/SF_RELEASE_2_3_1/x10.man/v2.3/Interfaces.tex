\chapter{Interfaces}




\label{XtenInterfaces}\index{interface}

An interface specifies signatures for zero or more public methods, property
methods,
\xcd`static val`s, 
classes, structs, interfaces, types
and an invariant. 

The following puny example illustrates all these features: 
% TODO Well, it would if there weren't a compiler bug in the way.
%~~gen ^^^Interfaces_static_val
% package Interfaces_static_val;
% 
%~~vis
\begin{xten}
interface Pushable{prio() != 0} {
  def push(): void;
  static val MAX_PRIO = 100;
  abstract class Pushedness{}
  struct Pushy{}
  interface Pushing{}
  static type Shove = Int;
  property text():String;
  property prio():Int;
}
class MessageButton(text:String)
  implements Pushable{self.prio()==Pushable.MAX_PRIO} {
  public def push() { 
    x10.io.Console.OUT.println(text + " pushed");
  }
  public property text() = text;
  public property prio() = Pushable.MAX_PRIO;
}
\end{xten}
%~~siv
%
%~~neg
\noindent
\xcd`Pushable` defines two property methods, one normal method, and a static
value.  It also 
establishes an invariant, that \xcd`prio() != 0`. 
\xcd`MessageButton` implements a constrained version of \xcd`Pushable`,
\viz\ one with maximum priority.  It
defines the \xcd`push()` method given in the interface, as a \xcd`public`
method---interface methods are implicitly \xcd`public`.

\limitation{X10 may not always detect that type invariants of interfaces are
satisfied, even when they obviously are.}
%% TODO - is this a JIRA?  

A container---a class or struct---can {\em implement} an interface,
typically by having all the methods and property methods that the interface
requires, and by providing a suitable \xcd`implements` clause in its definition.

A variable may be declared to be of interface type.  Such a variable has all
the property and normal methods declared (directly or indirectly) by the
interface; 
nothing else is statically available.  Values of any concrete type which
implement the interface may be stored in the variable.  

\begin{ex}
The following code puts two quite different objects into the variable
\xcd`star`, both of which satisfy the interface \xcd`Star`.
%~~gen ^^^ Interfaces6l3f
% package Interfaces6l3f;
%~~vis
\begin{xten}
interface Star { def rise():void; }
class AlphaCentauri implements Star {
   public def rise() {}
}
class ElvisPresley implements Star {
   public def rise() {}
}
class Example {
   static def example() {
      var star : Star;
      star = new AlphaCentauri();
      star.rise();
      star = new ElvisPresley();
      star.rise();
   }
}
\end{xten}
%~~siv
%
%~~neg
\end{ex}
An interface may extend several interfaces, giving
X10 a large fraction of the power of multiple inheritance at a tiny fraction
of the cost.

\begin{ex}
%~~gen ^^^ Interfaces6g4u
% package Interfaces6g4u;
%~~vis
\begin{xten}
interface Star{}
interface Dog{}
class Sirius implements Dog, Star{}
class Lassie implements Dog, Star{}
\end{xten}
%~~siv
%
%~~neg
\end{ex}


\section{Interface Syntax}

\label{DepType:Interface}

%##(InterfaceDeclaration TypeParamsI Guard ExtendsInterfaces InterfaceBody InterfaceMemberDeclaration
\begin{bbgrammar}
%(FROM #(prod:InterfaceDecln)#)
      InterfaceDecln \: Mods\opt \xcd"interface" Id TypeParamsI\opt Properties\opt Guard\opt ExtendsInterfaces\opt InterfaceBody & (\ref{prod:InterfaceDecln}) \\
%(FROM #(prod:TypeParamsI)#)
         TypeParamsI \: \xcd"[" TypeParamIList \xcd"]" & (\ref{prod:TypeParamsI}) \\
%(FROM #(prod:Guard)#)
               Guard \: DepParams & (\ref{prod:Guard}) \\
%(FROM #(prod:ExtendsInterfaces)#)
   ExtendsInterfaces \: \xcd"extends" Type & (\ref{prod:ExtendsInterfaces}) \\
                     \| ExtendsInterfaces \xcd"," Type \\
%(FROM #(prod:InterfaceBody)#)
       InterfaceBody \: \xcd"{" InterfaceMemberDeclns\opt \xcd"}" & (\ref{prod:InterfaceBody}) \\
%(FROM #(prod:InterfaceMemberDecln)#)
InterfaceMemberDecln \: MethodDecln & (\ref{prod:InterfaceMemberDecln}) \\
                     \| PropMethodDecln \\
                     \| FieldDecln \\
                     \| TypeDecln \\
\end{bbgrammar}
%##)


\noindent
The invariant associated with an interface is the conjunction of the
invariants associated with its superinterfaces and the invariant
defined at the interface. 



A class \xcd"C"  implements an interface \xcd"I" if \xcd`I`, or a subtype of \xcd`I`, appears in the \xcd`implements` list
of \xcd`C`.  
In this case,
 \xcd`C` implicitly gets all the methods and property methods of \xcd`I`,
      as \xcd`abstract` \xcd`public` methods.  If \xcd`C` does not declare
      them explicitly, then they are \xcd`abstract`, and \xcd`C` must be
      \xcd`abstract` as well.   If \xcd`C` does declare them all, \xcd`C` may
      be concrete.



If \xcd`C` implements \xcd`I`, then the class invariant
(\Sref{DepType:ClassGuardDef}) for \xcd`C`,   $\mathit{inv}($\xcd"C"$)$, implies
the class invariant for \xcd`I`, $\mathit{inv}($\xcd"I"$)$.  That is, if the
interface \xcd`I` specifies some requirement, then every class \xcd`C` that
implements it satisfies that requirement.

\section{Access to Members}

All interface members are \xcd`public`, whether or not they are declared
public.  There is little purpose to non-public methods of an interface; they
would specify that implementing classes and structs have methods that cannot
be seen.

\section{Member Specification}


An interface can specify that all containers implementing it must have certain
instance methods.  It cannot require constructors or static methods, though.


\begin{ex}
The \xcd`Stat` interface requires that its implementers provide 
an \xcd`ick` method.  
It can't insist that implementations provide a static method 
like \xcd`meth`, or a nullary constructor.
%~~gen ^^^ Interfaces2y3i
% 
% package Interfaces2y3i;
% // NOTEST-stupid-packaging-issue
%~~vis
\begin{xten}
interface Stat {
  def ick():void; 
  // ERROR: static def meth():Int;
  // ERROR: static def this();
}
class Example implements Stat {
  public def ick() {}
  def example() {
     this.ick();
  }
}
\end{xten}
%~~siv
%
%~~neg

\end{ex}

\section{Property Methods}

An interface may declare \xcd`property` methods.  All non-\xcd`abstract`
containers implementing such an interface must provide all the \xcd`property`
methods specified.  

\section{Field Definitions}
\index{interface!field definition in}

An interface may declare a \xcd`val` field, with a value.  This field is implicitly
\xcd`public static val`.  In particular, it is {\em not} an instance field. 
%~~gen ^^^ Interfaces10
% package Interface.Field;
%~~vis
\begin{xten}
interface KnowsPi {
  PI = 3.14159265358;
}
\end{xten}
%~~siv
%
%~~neg

Classes and structs implementing such an interface get the interface's fields as
\xcd`public static` fields.  Unlike  methods, there is no need
for the implementing class to declare them. 
%~~gen ^^^ Interfaces20
% package Interface.Field.Two;
% interface KnowsPi {PI = 3.14159265358;}
%~~vis
\begin{xten}
class Circle implements KnowsPi {
  static def area(r:Double) = PI * r * r;
}
class UsesPi {
  def circumf(r:Double) = 2 * r * KnowsPi.PI;
}
\end{xten}
%~~siv
%
%~~neg

\subsection{Fine Points of Fields}

If two parent interfaces give different static fields of the same name, 
those fields must be referred to by qualified names.
%~~gen ^^^ Interface_field_name_collision
% 
%~~vis
\begin{xten}
interface E1 {static val a = 1;}
interface E2 {static val a = 2;}
interface E3 extends E1, E2{}
class Example implements E3 {
  def example() = E1.a + E2.a;
}
\end{xten}
%~~siv
%
%~~neg

If the {\em same} field \xcd`a` is inherited through many paths, there is no need to
disambiguate it:
%~~gen ^^^ Interfaces_multi
% package Interfaces.Mult.Inher.Field;
%~~vis
\begin{xten}
interface I1 { static val a = 1;} 
interface I2 extends I1 {}
interface I3 extends I1 {}
interface I4 extends I2,I3 {}
class Example implements I4 {
  def example() = a;
}
\end{xten}
%~~siv
%
%~~neg

The initializer of a field in an interface may be any expression.  It is
evaluated under the same rules as a \xcd`static` field of a class. 

\begin{ex}
In this example, a class \xcd`TheOne` is defined,
with an inner interface \xcd`WelshOrFrench`, whose field \xcd`UN` (named in
either Welsh or French) has value 1.  Note that \xcd`WelshOrFrench` does not
define any methods, so it can be trivially added to the \xcd`implements`
clause of any class, as it is for \xcd`Onesome`. 
This allows the body of \xcd`Onesome` to use \xcd`UN` through an unqualified
name, as is done in \xcd`example()`.

%~~gen ^^^ Interfaces3l4a
% package Interfaces3l4a;
%~~vis
\begin{xten}
class TheOne {
  static val ONE = 1;
  interface WelshOrFrench {
    val UN = 1;
  }
  static class Onesome implements WelshOrFrench {
    static def example() {
      assert UN == ONE;
    }
  }
}
\end{xten}
%~~siv
% class Hook{ def run() {TheOne.Onesome.example(); return true;}}
%~~neg
\end{ex}

\section{Generic Interfaces}

Interfaces, like classes and structs, can have type parameters.  
The discussion of generics in \Sref{TypeParameters} applies to interfaces,
without modification.

\begin{ex}
%~~gen ^^^ Interfaces7n1z
% package Interfaces7n1z;
%~~vis
\begin{xten}
interface ListOfFuns[T,U] extends x10.util.List[(T)=>U] {}
\end{xten}
%~~siv
%
%~~neg

\end{ex}

\section{Interface Inheritance}

The {\em direct superinterfaces} of a non-generic interface \xcd`I` are the interfaces
(if any) mentioned in the \xcd`extends` clause of \xcd`I`'s definition.
If \xcd`I`  is generic, the direct superinterfaces are of an instantiation of
\xcd`I` are the corresponding instantiations of those interfaces.
A {\em superinterface} of \xcd`I` is either \xcd`I` itself, or a direct
superinterface of a superinterface of \xcd`I`, and similarly for generic
interfaces.    

\xcd`I` inherits the members of all of its superinterfaces. Any class or
struct that has \xcd`I` in its \xcd`implements` clause also implements all of
\xcd`I`'s superinterfaces. 


Classes and structs may be declared to implement multiple interfaces. Semantically, the
interface type is the set of all objects that are instances of classes
or structs that
implement the interface. A class or struct implements an interface if it is declared to
and if it concretely or abstractly implements all the methods and properties
defined in the interface. For example, \xcd`Kim` implements
\xcd`Person`, and hence \xcd`Named` and \xcd`Mobile`. It would be a static
error if \xcd`Kim` had no \xcd`name` method, unless \xcd`Kim` were also
declared \xcd`abstract`.

%~~gen ^^^ Types140
%interface Named {
%   def name():String;
% }
% interface Mobile {
%   def move(howFar:Int):void;
% }
% interface Person extends Named, Mobile {}
% interface NamedPoint extends Named, Mobile{} 
%~~vis
\begin{xten}
class Kim implements Person {
   var pos : Int = 0;
   public def name() = "Kim (" + pos + ")";
   public def move(dPos:Int) { pos += dPos; }
}
\end{xten}
%~~siv
%
%~~neg






\section{Members of an Interface}

The members of an interface \xcd`I` are the union of the following sets: 
\begin{enumerate}
\item All of the members appearing in \xcd`I`'s declaration;
\item All the members of its direct super-interfaces, except those which are
      hidden (\Sref{sect:Hiding}) by \xcd`I`
\item The members of \xcd`Any`.
\end{enumerate}

Overriding for instances is defined as for classes, \Sref{MethodOverload}
