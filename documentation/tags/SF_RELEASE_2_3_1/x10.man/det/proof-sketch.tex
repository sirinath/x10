The formalism with clocked types is too complicated, so I'll start by focusing just on accumulators.

Examples why accumulators need these two dynamic checks:
1) that a read is done in the same block the accumulator was defined,
2) that a write is done only in children blocks.

1) This check is done in rule R-Acc-R,
when reading from "a()" we check that "a" was created in the same block.
For example, the following is disallowed:
\begin{lstlisting}
val a = new Acc(0);
async { val z = a(); } // allowing this read will lead to indeterminacy (returns either 0 or 1), because the second async might or might not have executed.
async { a<-1; }
\end{lstlisting}

2) This check is done in rule R-Acc-W,
when writing to "a" we check that "a" is in $\pi$ (the permissions of the current block).
If we assume that we do allow field assignment,
then the reason for this check is that non-children activities might gain access to the accumulator,
and the wait done at the read won't wait for those activities thus leading to indeterminacy.
For example:
\begin{lstlisting}
val acc1 = new Acc(0);
val cell = new Cell(acc1);
async { 
	val a = cell.value; // might be an alias to acc1 or acc2
	val x = a(); // allowing this read will lead to indeterminacy (returns 0, 1, or 3)
}
async {
	val acc2 = new Acc(1);
	cell.value = acc2; // data race!
	acc2 <- 2;
}
\end{lstlisting}
However, we disallowed field assignment to avoid data races (and have determinacy).
Without field assignment, all objects can be stack allocated (an object cannot outlive its block, because an object can only point to objects created before it in scope).
Therefore, the second check vacaously passes (it always succeeds because an accumulator cannot leak to a non-children activity).
Phrased differently, our formalism has no need to keep track of $\pi$ (because we do not have field assignment).

The formalism without clocked types and without $\pi$ looks as follows:
\begin{figure}[t]
\begin{center}
\begin{tabular}{|l|l|}
\hline

$\hP ::= \ol{\hL},\hS$ & Program. \\

$\hL ::= \hclass ~ \hC~\hextends~\hD~\lb~\ol{\hF};~\ol{\hM}~\rb$
& cLass declaration. \\

$\hF ::= \hvar\,\hf:\hC$
& Field declaration. \\

$\hM ::= \hdef\ \hm(\ol{\hx}:\ol{\hC}):\hC\{\hS\}$
& Method declaration. \\

$\hp ::= \hl ~~|~~ \hx$
& Path. \\ %(location or parameter)

$\he ::=  \hp.\hf  ~|~ \hnew{\hC}(\ol{\hp}) ~|~ \hp()$
& Expressions. \\ %: locations, parameters, field access,  %invocation, \code{new}
$\hS ::=  \epsilon ~|~  \hp.\hm(\ol{\hp}); ~|~$ &  \\
$~~~~~ \valt{\hx}{\he} ~|~ \hp \leftarrow \hp ~|~ \hS~\hS ~|~  $ &\\
$~~~~~ \finishasync{\blockt{\ol{\hl}}{\hS}} $ & Statements. \\ %: locations, parameters, field access, invocation, \code{new}

\hline
\end{tabular}
\end{center}
\caption{FX10 Syntax without clocked types.}
\label{Figure:syntax2}
\end{figure}


\begin{figure*}[t]

\begin{center}
\begin{tabular}{|c|}
\hline
$\typerule{
}{
 \epsilon,\cH \reduce \cH
}$~\RULE{(R-Epsilon)}
~
$\typerule{
 \hS,\cH \reduce \hS',\cH' ~|~ \cH'
}{
  \begin{array}{rcl}
    \finishasync{\blockt{\ol{\hl}}{\hS}},\cH&\reduce &\finishasync{\blockt{\ol{\hl}}{\hS'}},\cH'
    ~|~ \cH' \\
    \hS~\hS_1, \cH &\reduce& \hS'~\hS_1,\cH' ~|~ \hS_1,\cH' \\
  \end{array}
}$~\RULE{(R-Trans)} 
\\\\

$\typerule{
    \cH(\hl)=\hC(\ol{\hl'})
}{
  \hval{\hx}{\hl.\hf_i}{\hS},\cH \reduce \hS[\hl_i'/\hx],\cH
}$~\RULE{(R-Access)} 
\gap
$\typerule{
    \cH(\hl')=\hC(\ldots)
        \gap
    \mbody{}(\hm,\hC)=\ol{\hx}.\hS
}{
  \hl'.\hm(\ol{\hl}),\cH \reduce \hS[\ol{\hl}/\ol{\hx},\hl'/\hthis],\cH
}$~\RULE{(R-Invoke)} 
\\\\
$\typerule{
\cH(\ha)=\hAcc(\hr,\hv) \gap  \gap \ha\in \ol{\hl}
}{
\blockt{\ol{\hl}}{\hval{\hx}{\ha()}~\hS},\cH \reduce
\blockt{\ol{\hl}}{\hS[\hv/\hx]}, \cH'
}$~\RULE{(R-Acc-R)} 
~
$\typerule{
  \cH(\ha)=\hAcc(\hr,\hv)\gap \hw=\hr(\hv,\hl) 
}{
  \ha \leftarrow \hl,\cH \reduce \cH[\ha \mapsto \hAcc(\hr,\hw)]
}$~\RULE{(R-Acc-W)} 
\\\\
$\typerule{
    \hl' \not \in \dom(\cH)
}{
\blockt{\ol{\ha}}{\valt{\hx}{\hnew{\hC(\ol{\hl})}}~\hS},\cH \reduce
   \blockt{\ol{\ha},\hl}{\hS[\hl'/\hx]},\cH[ \hl' \mapsto \hC(\ol{\hl})] \\
}$~\RULE{(R-New)} 
\\\\
\hline

$\typerule{
  \hS,\cH \reduce \hS', \cH' ~|~ \cH'
}{
  \async{\blockt{\ol{\hl}}{\hS"}}~\hS, \cH \reduce \async{\blockt{\ol{\hl}}{\hS"}}~\hS', \cH' ~|~ \async{\blockt{\ol{\hl}}{\hS"}}, \cH'\\
}$~\RULE{(R-Async)-}
\\
\hline
\end{tabular}
\end{center}


\caption{FX10 Reduction Rules without clocked types}
\label{Figure:reduction2}
\end{figure*}

We now prove deadlock freedom, determinacy, and that in the sequential scheduler a configuration has a single successor.

A read from an accumulator waits until all its children activities terminate.
We now prove that this wait cannot lead to deadlock.
\begin{Lemma}
\textbf{Deadlock Freedom~}
There cannot be a circle in the accumulators wait-for graph.
\end{Lemma}
\begin{proof}
Assume we have a circle in the wait-for graph, i.e., 
	there is a serie $a_1, \ldots, a_n$ of accumulators where $a_i$ waits for its children activities that include a read of $a_{i+1}$
	(and $a_n$ for $a_1$).
If any of $a_i$ are in a \emph{proper} child activity, then we have a contradiction that $a_1$ is in a child activity of $a_n$
	(because the proper-child relationship is acyclic).
Otherwise, all reads are done by the same activity, thus $n=1$ (an activity is executing one statement at a time).
\end{proof}

\begin{Lemma}
\textbf{Sequential Scheduler~}
Given a configuration~$\hS,\cH$ and well-typed program $\ol{\hL},\hS$,
    then at most one of the reduction rules (excluding \RULE{(R-Async)-})
    is applicable on~$\hS,\cH$.
\label{Lemma:Sequential}
\end{Lemma}
\begin{proof}
By structural induction on~\hS.
According to the syntax \hS has one of the following forms:
	\beqst
	\epsilon
	\gap 
	\hp.\hm(\ol{\hp});
	\gap  
	\hS' \hS" 
	\gap 
	\valt{\hx}{\he} 
	\\
	\hp \leftarrow \hp
	\gap 
	\finishasync{\blockt{\ol{\hl}}{\hS}}
	\eeq
	where $\he$ is either
	$\hp.\hf  \gap  \hnew{\hC}(\ol{\hp}) \gap \hp()$.
Each form can match at most one reduction rule:
	(i)~$\epsilon$ can only match \RULE{(R-Epsilon)},
	(ii)~$\hp.\hm(\ol{\hp});$ can only match \RULE{(R-Invoke)},
	(iii)~$\hS' \hS"$ can only match the second outcome of \RULE{(R-Trans)},
	(iv)~$\valt{\hx}{\hp.\hf}$ can only match \RULE{(R-Access)},
	(v)~$\valt{\hx}{\hnew{\hC}(\ol{\hp})}$ can only match \RULE{(R-New)},
	(vi)~$\valt{\hx}{\hp()}$ can only match \RULE{(R-Acc-R)},
	(vii)~$\hp \leftarrow \hp$ can only match \RULE{(R-Acc-W)},
	(viii)~$\finishasync{\blockt{\ol{\hl}}{\hS}}$ can only match the first outcome of \RULE{(R-Trans)}.
\end{proof}

\begin{Theorem}
\textbf{Determinacy~}
Given a well-typed program $\ol{\hL},\hS$ in heap $\cH$,
    if $\hS,\cH \stackrel{*}\reduce \cH'$ and $\hS,\cH \stackrel{*}\reduce \cH"$
    then $\cH'$ and $\cH"$ are isomorphic.
\end{Theorem}
\begin{proof}
From \Ref{Lemma}{Sequential}, if we exclude rule \RULE{(R-Async)-} then 
	we get a sequential scheduler.
Therefore we just need to prove that if both these two rules apply on a statement:
	\RULE{(R-Async)-} and \RULE{(R-Trans)},
	then the outcome is the same.
Consider a statement that might match both rules:
	$\async{\blockt{\ol{\hl}}{\hS_0}}~\hS_1$.
We need to show that we can execute $\hS_0$ and $\hS_1$ in parallel and get the same outcome as if we would have executed
	them sequentially (first $\hS_0$ and then $\hS_1$).
	
If one excludes accumulators, then the resulting language has no side-effects (because we do not support field assignment to avoid data races).
Without side-effects we can trivially execute $\hS_0$ and $\hS_1$ in parallel
	(there might be a problem only if they have a side-effect on the same object).
With accumulators, we do have side-effects from rule \RULE{(R-Acc-W)} that shows that an accumulator-write mutates the heap.
Consider the set of all accumulators~$A$ that currently exists in the heap before the execution of $\hS_0$ and $\hS_1$.
Then the side-effects of $\hS_0$ and $\hS_1$ are only reads and writes from $A$ 
	(accumulators created during execution of $\hS_i$
	cannot leak to $\hS_{1-i}$ because we do not have field assignment.
	In the general X10 this might happen, and therefore we use $\pi$ to track the write-permissions of different activities,
	to ensure that even if leaking is possible then a write is still illegal at runtime).
Consider the intersection of side-effects of $\hS_0$ and $\hS_1$.
The problematic cases are a write-read or write-write conflicts.
Write-write conflicts are determinate with accumulators because the reduction operator is associative-commutative.
Write-read conflicts are possible, but only $\hS_1$ might read from an accumulator 
	(a proper child activity cannot read from an accumulator due to \RULE{(R-Acc-R)}),
	and $\hS_1$ can read only after $\hS_0$ was terminated (\RULE{(R-Acc-R)}),
	therefore it will always read the same value (after all the writes to the accumulator finished).
\end{proof}

