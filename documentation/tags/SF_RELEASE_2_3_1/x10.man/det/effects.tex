% 8/17/2011 5:03:31 AM

We represent effects in X10 as follows. 

We use types to classify objects, i.e.{} to specify sets of
objects. A {\em location set} is of the form \code{T.f} where \code{T}
is a type and \code{f} is a mutable field of type \code{T}. It stands
for the set of locations \code{x.f} where \code{x} is of type
\code{T}. We do not need to introduce any distinct notion of regions
into the language.

The key insight is that two memory locations are distinct if they are
fields in two objects \code{m} and\code{n} such that for some property
\code{P}, \code{m.p != n.p}. Therefore we introduce enough properties
into classes to ensure that we can distinguish between objects that
are simultaneously being operated on. Specifically, the location sets
\code{S\{c\}.f} and \code{S\{d\}.f} are disjoint if
\code{c[x/self],d[x/self]} is not satisfiable for any \code{x:S}. 

Methods are decorated with \code{@Read(L)} and \code{@Write(M)}
annotations, where \code{L},\code{M} are location sets. The annotation
is {\em valid} if every read (write) of a mutable location \code{o.f} in the
body of the method it is the case that \code{o.f} lies in (the set
described by) \code{L} (\code{M}).

We permit location sets to be named:
\begin{lstlisting}
  static locs Cargos = Tree.cargo;
  static locs LeftCargo(up:Tree) = Tree{self.up==up,self.left==true}.cargo;
\end{lstlisting}

Unlike the region system of \cite{DPJ} we do not need to introduce
explicitly named, intensional regions, rather we can work with
extensional representations of location sets (the set of all locations
satisfying a certain condition).  There is no need for a separate
space of region names with constructors. Two location sets are
disjoint if their constraints are mutually unsatisfiable, not because
they are named differently. In particular, we do not need to assume
that the heap is partitioned into a tree of regions.

Two key properties of the X10 type system are worth recalling. The run-time
heap cannot contain cycles involving only properties. This in turn
depends on the fact that the X10 type system prevents \code{this} from
escaping during object construction\cite{X10-object-initialization}.

We mark a property as \code{@ersatz} to indicate that space is not
allocated at run-time for this property. Therefore the value of this
property is not accessible at run-time -- it cannot be read and stored
into variables. It may only be accessed in constraints that are
statically checked. Dynamic casts cannot refer to \code{@ersatz}
properties.

We enrich the vocabulary of constraints. First, we permit
existentially quantified constraints \code{x:T\^c} -- this represents
the constraint \code{c} in which the variable \code{x} of type
\code{T} is existentially quantified. Second, we permit the {\em
  extended field selector} \code{e.f\$i} where \code{e} is an
expression, \code{f} a field name and \code{i} is a \code{UInt}. 

%% Alternatively we just permit regular expressions as field
%% selectors?
%% So take T{self.r==x} where r is a regular expression to mean
%% T{p:??^(p in r, self.p==x)}. So we have to have a notion of a
%% type-consistent path, and then be able to select a path on a given
%% receiver, maybe use some other operator for this, and not .?


%% Hmm dont want to be able to count and do arithmetic, that will
%% take us out of reg exp land. 

%% Do we need MONA?
%% May need to permit full-fledged regular expressions. 

% Where do we need the flexibility of saying two region parameters are disjoint?

\begin{example}
  \begin{lstlisting}
type Tree(up:Tree,l:Boolean)=Tree{self.up==up,self.left==l};
class Tree (@ersatz up:Tree, @ersatz left:Boolean) {
    var left:Tree(this,true);
    var right:Tree(this,false);
    var cargo:Int;
    static def make(up:Tree, left:Boolean):Tree(up,left) {
        val x = new Tree(up, left);
        x.left = new Tree(x, true);
        x.right = new Tree(x, false);
        return x;
    }
    def makeConstant(x:Int)
        @Write(Tree{i:UInt^self.up$i=this}.cargo)
        @Read(Tree{i:UInt^self.up$i=this}.left)              ]
            @Read(Tree{i:UInt^self.up$i=this}.right)
    {
        finish {
            @Write(Tree{self==this}.cargo)
            this.cargo = x;
            if (left != null)
                async left.makeConstant(x);
            if (right != null)
                async right.makeConstant(x);
        }
    }
}
  \end{lstlisting}
In this example the fields \code{left},\code{right} and \code{cargo}
are mutable. It is possible to mutate a tree \code{p} -- replace
\code{p}'s \code{left} child with another tree \code{q}. However, \code{q} cannot be in
\code{p}'s \code{right} subtree, because then its \code{up} field or
\code{left} field would not have the right value. That is, once a
\code{Tree} object is created it can only belong to a specific tree in
a specific position.

Note that a \code{Tree} can be created with \code{null} parent and
children. This is how the root is created:
\code{new Tree(null,true)} or \code{new tree(null, false)}.

In checking the validity of the effect annotation on
\code{makeConstant} the compiler must check subtyping relations. That
is, for the \code{@Write} annotation it must check:
\begin{lstlisting}
Tree{i:UInt^(self.up$i.up==this,self.up$i.left==true} <: Tree{i:UInt^(self.up$i==this}
Tree{i:UInt^(self.up$i.up==this,self.up$i.left==false} <: Tree{i:UInt^(self.up$i==this}
Tree{self==this} <: Tree{i:UInt^(self.up$i==this)}
\end{lstlisting}

This is easily verified.
The notion of ``distinctions from the left'' and ``distinctions from
the right'' of \cite{DPJ} are not needed. These arise naturally
through the use of constraints (distinct access paths, vs distinct fields).
\end{example}

The machinery of ``region path lists'' of \code{DPJ} is not needed, it
is provided for by the constraint language in X10. 

\begin{example}
\begin{lstlisting}
class Tree (@ersatz up:Tree, @ersatz left:Boolean) {
    var left:Tree(this, true);
    var right:Tree(this,false);
    var mass:Double;
    var force:Double;
    static def make(up:Tree, left:Boolean):Tree{up,left} {
        val x = new Tree(up, left);
        x.left = new Tree(x, true);
        x.right = new Tree(x, false);
        return x;
    }
    def makeConstant(mass:Double, force:Double)
        @Write(Tree{i:UInt^self.up$i=this}.(mass,force))
        @Read(Tree{i:UInt^self.up$i=this}.left)              ]
            @Read(Tree{i:UInt^self.up$i=this}.right)
    {
        finish {
            @Write(Tree{self==this}.mass)
            async this.mass = mass;

            @Write(Tree{self==this}.force)
            async this.force = force;            

            if (left != null)
                async left.makeConstant(mass,force);
            if (right != null)
                async right.makeConstant(mass, force);
        }
    }
}
\end{lstlisting}
\end{example}


\begin{example}
\begin{lstlisting}
class Tree (@ersatz up:Tree, @ersatz left:Boolean) {
    var left:Tree{self.up==this, self.left==true};
    var right:Tree{self.up==this, self.left==false};
    var mass:Double;
    var force:Double;
    var link:Tree;
    static def make(up:Tree, left:Boolean):Tree{self.up==up,self.left==left} {
        val x = new Tree(up, left);
        x.left = new Tree(x, true);
        x.right = new Tree(x, false);
        return x;
    }
    def computeForce()
        @Write(Tree{i:UInt^self.up$i=this}.force)
        @Read(Tree{i:UInt^self.up$i=this}.left)              ]
        @Read(Tree{i:UInt^self.up$i=this}.right)
        @Read(Tree.mass)            
    {
        finish {
            @Write(Tree{self==this}.mass)
                async this.force = (this.mass*link.mass)*G;

            if (left != null)
                async left.computeForce();
            if (right != null)
                async right.computeForce();
        }
    }
}
\end{lstlisting}
\end{example}




