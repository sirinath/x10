

\chapter{Functions}
\label{Functions}
\label{functions}
\index{function}
\index{closure}
\index{function!literal}
\label{Closures}

\section{Overview}
Functions, the last of the three kinds of values in X10, encapsulate pieces of
code which can be applied to a vector of arguments to produce a value.
Functions, when applied, can do nearly anything that any other code could do:
fail to terminate, throw an exception, modify variables, spawn activities,
execute in several places, and so on. X10 functions are not mathematical
functions: the \xcd`f(1)` may return \xcd`true` on one call and \xcd`false` on
an immediately following call.



A \emph{function literal} \xcd"(x1:T1,..,xn:Tn){c}:T=>e" creates a function of
type\\ \xcd"(x1:T1,...,xn:Tn){c}=>T" (\Sref{FunctionType}).  For example, 
\xcd`(x:Int):Int => x*x` is a function literal describing the squaring function on
integers.   
\xcd`null` is also a function value.

\limitation{X10 functions cannot have type arguments or constraints.}

Function application is written \xcd`f(a,b,c)`, following common mathematical
usage. 
\index{Exception!unchecked}


The function body may be a block.  To compute integer squares by repeated
addition (inefficiently), one may write: 
%~~gen ^^^ Functions10
% package Functions10;
% class Examplllll {
% static 
%~~vis
\begin{xten}
val sq: (Int) => Int 
      = (n:Int) => {
           var s : Int = 0;
           val abs_n = n < 0 ? -n : n;
           for (i in 1..abs_n) s += abs_n;
           s
        };
\end{xten}
%~~siv
%}
% class Hook{ def run() { return Examplllll.sq(5) == 25; } }
%~~neg




A function literal evaluates to a function entity $f$. When {$f$} is
applied to a suitable list of actual parameters \xcd`a1` through \xcd`an`, it
evaluates \xcd`e` with the formal parameters bound to the actual parameters.
So, the following are equivalent, where \xcd`e` is an expression involving
\xcd`x1` and \xcd`x2`\footnote{Strictly, there are a few other requirements;
  \eg, \xcd`result` must be a \xcd`var` of type \xcd`T` defined outside the
  outer block, the variables \xcd`a1` and \xcd`a2` had better not appear in
  \xcd`e`, and everything in sight had better typecheck properly.}

%~~gen ^^^ Functions20
% package Functions.function_literal_a;
% abstract class FunctionsTooManyFlippingFunctions[T, T1, T2]{
% abstract def arg1():T1;
% abstract def arg2():T2;
% def thing1(e:T) {var result:T;
%~~vis
\begin{xten}
{
  val f = (x1:T1,x2:T2){true}:T => e;
  val a1 : T1 = arg1();
  val a2 : T2 = arg2();
  result = f(a1,a2);
}
\end{xten}
%~~siv
%}}
%~~neg
and 
%~~gen ^^^ Functions30
% package functions2_b;
% abstract class FunctionsTooManyFlippingFunctions[T, T1, T2]{
% abstract def arg1():T1;
% abstract def arg2():T2;
% def thing1(e:T) {var result:T;
%~~vis
\begin{xten}
{
  val a1 : T1 = arg1();
  val a2 : T2 = arg2();
  {
     val x1 : T1 = a1;
     val x2 : T2 = a2;
     result = e;
  }  
}
\end{xten}
%~~siv
%}}
%~~neg
\noindent
This equivalence does not hold if the body is a statement rather than an expression.
A few language features are forbidden (\xcd`break` or \xcd`continue` of a loop
that surrounds the function literal) or mean something different (\xcd`return`
inside a function returns from the function, not the surrounding block). 



Function types may be used in \Xcd{implements} clauses of class
definitions. Suitable operator definitions must be supplied, with 
\xcdmath"public operator this(x1:T1, $\ldots$, xn:Tn)" declarations.
Instances of such classes may be used as functions of the
given type.  Indeed, an object may behave like any (fixed) number of
functions, since the class it is an instance of may implement any
(fixed) number of function types.
\begin{eg}
Instances of the \xcd`Funny` class behave like two functions: 
a constant function on Booleans, and a linear function on 
pairs of \xcd`Int`s.  
%~~gen ^^^ Functions6w6w
% package Functions6w6w;
%~~vis
\begin{xten}
class Funny implements (Boolean) => Int, 
                       (Int, Int) => Int
{
  public operator this(Boolean) = 1;
  public operator this(x:Int, y:Int) = 10*x+y;
  static def example() {
    val f <: Funny  = new Funny();
    assert f(true) == 1; // (Boolean)=>Int behavior
    assert f(1,2) == 12; // (Int,Int)=>Int behavior
  }
}
\end{xten}
%~~siv
% class Hook{ def run() { Funny.example(); return true; }}
%~~neg

\end{eg}


\section{Function Application}
\index{function!application}
\label{sect:FunctionApplication}

The basic operation on functions is function application.  
(Since, \eg, array lookup has the same type as function application, these
rules are used for array lookup as well, and so on.) 

A function with type 
\xcdmath"(x$_1$:T$_1$, $\ldots$, x$_n$:T$_n$){c} => T"
can be applied to a sequence of expressions 
\xcdmath"e$_1$, $\ldots$, e$_n$"
if: 
\begin{itemize}
\item \xcdmath"e$_1$" is of type \xcdmath"T$_1$[e$_1$/x$_1$]",
\item $\ldots$,
\item \xcdmath"e$_n$" is of type 
      \xcdmath"T$_n$[e$_1$/x$_1$, $\ldots$, e$_n$/x$_n$]",
\item X10 can prove that \xcdmath"c[e$_1$/x$_1$, $\ldots$, e$_n$/x$_n$]" holds.
\end{itemize}
\noindent
In this case, if the application terminates normally, it returns a
value of type 
\xcdmath"T[e$_1$/x$_1$, $\ldots$, e$_n$/x$_n$]".

\begin{ex}
Consider
%~~gen ^^^ Functions6p3q
% package Functions6p3q;
% class Whatever {
% var 
%~~vis
\begin{xten}
 f : (a:Int{a!=0}, b:Int{b!=a}){b!=0} => Int{self != a}
\end{xten}
%~~siv
% = null;
%}
%~~neg

Then the call \xcd`f(3,4)` is allowed, because: 
\begin{itemize}
\item \xcd`3` is of type \xcd`Int{a!=0}` with \xcd`a` replaced by \xcd`3`, 
\viz{}  \xcd`Int{3 != 0}`; 
\item \xcd`4` is of type \xcd`Int{b!=a}` 
with \xcd`a` replaced by \xcd`3` and 
\xcd`b` replaced by \xcd`4`, 
\viz{} \xcd`Int{3 != 4}`. 
\item The guard \xcd`b!=0`, 
with \xcd`a` replaced by \xcd`3` and 
\xcd`b` replaced by \xcd`4`, 
is \xcd`4!=0`, which is true.  
\end{itemize}
So, \xcd`f(3,4)` will return a value of type 
\xcd`Int{self != a}` with 
 \xcd`a` replaced by \xcd`3` and 
\xcd`b` replaced by \xcd`4`, 
which is to say, \xcd`Int{self!=3}`.
\end{ex}

%\section{Implementation Notes}
%\begin{itemize}
%
%\item Note that e.m.(T1,...,Tn) will evaluate e to create a
%  function. This function will be applied later to given
%  arguments. Thus this syntax can be used to evaluate the receiver of
%  a method call ahead of the actual invocation. The resulting function
%  can be used multiple times, of course.
%\end{itemize}


\section{Function Literals}
\index{literal!function}
\label{FunctionLiteral}

\Xten{} provides first-class, typed functions, often called {\em closures}.

%##(ClosureExp Formals Guard DepParams  HasResultType ClosureBody
\begin{bbgrammar}
%(FROM #(prod:ClosureExp)#)
          ClosureExp \: Formals Guard\opt HasResultType\opt \xcd"=>" ClosureBody & (\ref{prod:ClosureExp}) \\
%(FROM #(prod:Formals)#)
             Formals \: \xcd"(" FormalList\opt \xcd")" & (\ref{prod:Formals}) \\
%(FROM #(prod:Guard)#)
               Guard \: DepParams & (\ref{prod:Guard}) \\
%(FROM #(prod:HasResultType)#)
       HasResultType \: ResultType & (\ref{prod:HasResultType}) \\
                     \| \xcd"<:" Type \\
%(FROM #(prod:ClosureBody)#)
         ClosureBody \: Exp & (\ref{prod:ClosureBody}) \\
                     \| Annotations\opt \xcd"{" BlockStmts\opt LastExp \xcd"}" \\
                     \| Annotations\opt Block \\
\end{bbgrammar}
%##)

Functions have zero or more formal parameters and an optional return type.
The body has the 
same syntax as a method body; it may be either an expression, a block
of statements, or a block terminated by an expression to return. In
particular, a value may be returned from the body of the function
using a return statement (\Sref{ReturnStatement}). 

The type of a function is a function type as described in \Sref{FunctionType}.  In some
cases the 
return type \Xcd{T} is also optional and defaults to the type of the
body. If a formal \Xcd{xi} does not occur in any
\Xcd{Tj}, \Xcd{c}, \Xcd{T} or \Xcd{e}, the declaration \Xcd{xi:Ti} may
be replaced by just \Xcd{Ti}. \Eg,  \xcd`(Int)=>7` is the integer function returning
7 for all inputs.

\label{ClosureGuard}


As with methods, a function may declare a guard to
constrain the actual parameters with which it may be invoked.
The guard may refer to the type parameters, formal parameters,
and any \xcd`val`s in scope at the function expression.

\begin{ex}
%~~gen ^^^ Functions5g5m
% package Functions5g5m;
% class clogua {
%  public static def main(argv:Array[String](1)) {
%~~vis
\begin{xten}
    val n = 3;
    val f : (x:Int){x != n} => Int  
          = (x:Int){x != n} => (12/(n-x));
    Console.OUT.println("f(5)=" + f(5));    
\end{xten}
%~~siv
%}} 
%~~neg

\end{ex}

The body of the function is evaluated when the function is
invoked by a call expression (\Sref{Call}), not at the function's
place in the program text.

As with methods, a function with return type \xcd"void" cannot
have a terminating expression. 
If the return type is omitted, it is inferred, as described in
\Sref{TypeInference}.
It is a static error if the return type cannot be inferred.  \Eg,
\xcd`(Int)=>null` is not well-specified; X10 does not know which type of
\xcd`null` is intended.  
%~~exp~~`~~`~~ ~~ ^^^ Functions40
But \xcd`(Int):Array[Double](1) => null` is legal.


\begin{ex}
The following method takes a function parameter and uses it to
test each element of the list, returning the first matching
element.  It returns \xcd`no` if no element matches.

%~~gen ^^^ Functions50
% package functions2.oh.no;
% import x10.util.*;
% class Finder {
% static 
%~~vis
\begin{xten}
def find[T](f: (T) => Boolean, xs: List[T], no:T): T = {
  for (x: T in xs)
    if (f(x)) return x;
  no
  }
\end{xten}
%~~siv
% }
%~~neg

The method may be invoked thus, to find a positive element of \xcd`xs`, or
return \xcd`0` if there is no positive element.
%~~gen ^^^ Functions60
% package functions2.oh.no.my.ears;
% import x10.util.*;
% class Finderator {
% static def find[T](f: (T) => Boolean, xs: x10.util.List[T], absent:T): T = {
%  for (x: T in xs)
%    if (f(x)) return x;
%  absent
%}
% static def checkery() {
%~~vis
\begin{xten}
xs: List[Int] = new ArrayList[Int]();
x: Int = find((x: Int) => x>0, xs, 0);
\end{xten}
%~~siv
%}}
%~~neg

\end{ex}



\subsection{Outer variable access}
\index{function!outer variables in}

In a function
\xcdmath"(x$_1$: T$_1$, $\dots$, x$_n$: T$_n$){c} => { s }"
the types \xcdmath"T$_i$", the guard \xcd"c" and the body \xcd"s"
may access many, though not all, sorts of variables from outer scopes.  
Specifically, they can access: 
\begin{itemize}
\item All fields of the enclosing object(s) and class(es);
\item All type parameters;
\item All \xcd`val` variables;
\end{itemize}
\noindent
\xcd`var` variables cannot be accessed.


The function body may refer to instances of enclosing classes using
the syntax \xcd"C.this", where \xcd"C" is the name of the
enclosing class.  \xcd`this` refers to the instance of the immediately
enclosing class, as usual.


\begin{eg}
 The following is legal.  
Note that \xcd`a` is not a local \xcd`var` variable. It is a field of
\xcd`this`. A reference to \xcd`a` is simply short for \xcd`this.a`, which is
a use of a \xcd`val` variable (\xcd`this`).  
%~~gen ^^^ Functions70
% package Functions.scoping_rules;
%~~vis
\begin{xten}
class Lambda {
   var a : Int = 0;
   val b = 0;
   def m(var c : Int, val d : Int) {
      var e : Int = 0;
      val f : Int = 0;
      val closure = (var i: Int, val j: Int) => {
    	  return a + b + d + f + i 
               + j + this.a + Lambda.this.a;
          // c and e are not usable here
      };
      return closure;
   }
}
\end{xten}
%~~siv
%
%~~neg


%~~gen ^^^ Functions9u3u
% // OK, we want to do the negative tests, but they don't work properly.
% class Lambda {
%    var a : Int = 0;
%    val b = 0;
%    def m(var c : Int, val d : Int) {
%       var e : Int = 0;
%       val f : Int = 0;
%       val closure = (var i: Int, val j: Int) => {
%     	  // return a + b + d + f + j + this.a + Lambda.this.a;
%           // ERROR: return c;
%           // ERROR: return e;
%       };
%       return closure;
%    }
% }
% 
% package Functions9u3u;
%~~vis
\begin{xten}
\end{xten}
%~~siv
%
%~~neg



\end{eg}

%%METHOD-SELECTION%% \section{Method selectors}
%%METHOD-SELECTION%% \label{MethodSelectors}
%%METHOD-SELECTION%% \index{function!method selector}
%%METHOD-SELECTION%% \index{method!underlying function}
%%METHOD-SELECTION%% 
%%METHOD-SELECTION%% A method selector expression allows a method to be used as a
%%METHOD-SELECTION%% first-class function, without writing a function expression for it.
%%METHOD-SELECTION%% 
%%METHOD-SELECTION%% \begin{ex}
%%METHOD-SELECTION%% Consider a class \xcd`Span` defining ranges of integers.  
%%METHOD-SELECTION%% 
%%METHOD-SELECTION%% %~todo~gen ^^^ Functions80
%%METHOD-SELECTION%% % package Functions2.Span;
%%METHOD-SELECTION%% %~todo~vis
%%METHOD-SELECTION%% \begin{xten}
%%METHOD-SELECTION%% class Span(low:Int, high:Int) {
%%METHOD-SELECTION%%   def this(low:Int, high:Int) {property(low,high);}
%%METHOD-SELECTION%%   def between(n:Int) = low <= n && n <= high;
%%METHOD-SELECTION%%   static def example() {
%%METHOD-SELECTION%%     val digit = new Span(0,9);
%%METHOD-SELECTION%%     val isDigit : (Int) => Boolean = digit.between.(Int);
%%METHOD-SELECTION%%     assert isDigit(8);
%%METHOD-SELECTION%%   }
%%METHOD-SELECTION%% }
%%METHOD-SELECTION%% \end{xten}
%%METHOD-SELECTION%% %~todo~siv
%%METHOD-SELECTION%% % class Hook{ def run() {Span.example(); return true;}}
%%METHOD-SELECTION%% %~todo~neg
%%METHOD-SELECTION%% \noindent
%%METHOD-SELECTION%% 
%%METHOD-SELECTION%% 
%%METHOD-SELECTION%% In \xcd`example()`, 
%%METHOD-SELECTION%% %~todo~exp~~`~~`~~ digit:Span~~class Span(low:Int, high:Int) {def this(low:Int, high:Int) {property(low,high);} def between(n:Int) = low <= n && n <= high;} ^^^ Functions90
%%METHOD-SELECTION%% \xcd`digit.between.(Int)` 
%%METHOD-SELECTION%% is a unary function testing whether its argument is between zero
%%METHOD-SELECTION%% and nine.  It could also be written 
%%METHOD-SELECTION%% %~todo~exp~~`~~`~~ digit:Span~~class Span(low:Int, high:Int) {def this(low:Int, high:Int) {property(low,high);} def between(n:Int) = low <= n && n <= high;} ^^^ Functions100
%%METHOD-SELECTION%% \xcd`(n:Int) => digit.between(n)`.
%%METHOD-SELECTION%% \end{ex}
%%METHOD-SELECTION%% 
%%METHOD-SELECTION%% 
%%METHOD-SELECTION%% %##(MethodSelection
%%METHOD-SELECTION%% \begin{bbgrammar}
%%METHOD-SELECTION%% %(FROM #(prod:MethodSelection)#)
%%METHOD-SELECTION%%      MethodSelection \: MethodName \xcd"." \xcd"(" FormalParamList\opt \xcd")" & (\ref{prod:MethodSelection}) \\
%%METHOD-SELECTION%%                     \| Primary \xcd"." Id \xcd"." \xcd"(" FormalParamList\opt \xcd")" \\
%%METHOD-SELECTION%%                     \| \xcd"super" \xcd"." Id \xcd"." \xcd"(" FormalParamList\opt \xcd")" \\
%%METHOD-SELECTION%%                     \| ClassName \xcd"." \xcd"super"  \xcd"." Id \xcd"." \xcd"(" FormalParamList\opt \xcd")" \\
%%METHOD-SELECTION%% \end{bbgrammar}
%%METHOD-SELECTION%% %##)
%%METHOD-SELECTION%% 
%%METHOD-SELECTION%% The \emph{method selector expression} \Xcd{e.m.(T1,...,Tn)} is type
%%METHOD-SELECTION%% correct only if  the static type of \Xcd{e} is a
%%METHOD-SELECTION%% class or struct or interface \xcd`V` with a method
%%METHOD-SELECTION%% \Xcd{m(x1:T1,...xn:Tn)\{c\}:T} defined on it (for some
%%METHOD-SELECTION%% \Xcd{x1,...,xn,c,T)}. At runtime the evaluation of this expression
%%METHOD-SELECTION%% evaluates \Xcd{e} to a value \Xcd{v} and creates a function \Xcd{f}
%%METHOD-SELECTION%% which, when applied to an argument list \Xcd{(a1,...,an)} (of the right
%%METHOD-SELECTION%% type) yields the value obtained by evaluating \Xcd{v.m(a1,...,an)}.
%%METHOD-SELECTION%% 
%%METHOD-SELECTION%% Thus, the method selector
%%METHOD-SELECTION%% \begin{xtenmath}
%%METHOD-SELECTION%% e.m.(T$_1$, $\dots$, T$_n$)
%%METHOD-SELECTION%% \end{xtenmath}
%%METHOD-SELECTION%% \noindent behaves as if it were the function
%%METHOD-SELECTION%% \begin{xtenmath}
%%METHOD-SELECTION%% ((v:V)=>
%%METHOD-SELECTION%%   (x$_1$: T$_1$, $\dots$, x$_n$: T$_n$){c} 
%%METHOD-SELECTION%%   => v.m(x$_1$, $\dots$, x$_n$))
%%METHOD-SELECTION%% (e)
%%METHOD-SELECTION%% \end{xtenmath}
%%METHOD-SELECTION%% 
%%METHOD-SELECTION%% 
%%METHOD-SELECTION%% 
%%METHOD-SELECTION%% Because of overloading, a method name is not sufficient to
%%METHOD-SELECTION%% uniquely identify a function for a given class.
%%METHOD-SELECTION%% One needs the argument type information as well.
%%METHOD-SELECTION%% The selector syntax (dot) is used to distinguish \xcd"e.m()" (a
%%METHOD-SELECTION%% method invocation on \xcd"e" of method named \xcd"m" with no arguments)
%%METHOD-SELECTION%% from \xcd"e.m.()"
%%METHOD-SELECTION%% (the function bound to the method). 
%%METHOD-SELECTION%% 
%%METHOD-SELECTION%% A static method provides a binding from a name to a function that is
%%METHOD-SELECTION%% independent of any instance of a class; rather it is associated with the
%%METHOD-SELECTION%% class itself. The static function selector
%%METHOD-SELECTION%% \xcdmath"T.m.(T$_1$, $\dots$, T$_n$)" denotes the
%%METHOD-SELECTION%% function bound to the static method named \xcd"m", with argument types
%%METHOD-SELECTION%% \xcdmath"(T$_1$, $\dots$, T$_n$)" for the type \xcd"T". The return type
%%METHOD-SELECTION%% of the function is specified by the declaration of \xcd"T.m".
%%METHOD-SELECTION%% 
%%METHOD-SELECTION%% There is no difference between using a function defined directly 
%%METHOD-SELECTION%% directly using the function syntax, or obtained via static or
%%METHOD-SELECTION%% instance function selectors.
%%METHOD-SELECTION%% 
%%OP-FUN%% o
%%OP-FUN%% \section{Operator functions}
%%OP-FUN%% \label{OperatorFunction}
%%OP-FUN%% \index{function!operator}
%%OP-FUN%% Every binary operator (e.g.,
%%OP-FUN%% \xcd"+",
%%OP-FUN%% \xcd"-",
%%OP-FUN%% \xcd"*",
%%OP-FUN%% \xcd"/",
%%OP-FUN%% \dots) has a family of functions, one for
%%OP-FUN%% each type on which the operator is defined. The function can be
%%OP-FUN%% selected using the ``\xcd`.`'' syntax:
%%OP-FUN%% 
%%OP-FUN%% 
%%OP-FUN%% \begin{xtenmath}
%%OP-FUN%% String.+    $\equiv$ (x: String, y: String): String => x + y
%%OP-FUN%% Long.-      $\equiv$ (x: Long, y: Long): Long => x - y
%%OP-FUN%% Float.-     $\equiv$ (x: Float, y: Float): Float => x - y
%%OP-FUN%% Boolean.&   $\equiv$ (x: Boolean, y: Boolean): Boolean => x & y
%%OP-FUN%% Int.<       $\equiv$ (x: Int, y: Int): Boolean => x < y
%%OP-FUN%% \end{xtenmath}
%%OP-FUN%% 
%%OP-FUN%% %~x~gen ^^^ Functions110
%%OP-FUN%% % package Functions.Operatorfunctionsgracklegrackle;
%%OP-FUN%% % class JustATest {
%%OP-FUN%% % val dummy = [String.+,
%%OP-FUN%% %  Long.-,
%%OP-FUN%% %  Float.-,
%%OP-FUN%% %  Boolean.&,
%%OP-FUN%% %  Int.<
%%OP-FUN%% %  ];
%%OP-FUN%% % }
%%OP-FUN%% %~x~vis
%%OP-FUN%% \begin{xten}
%%OP-FUN%% \end{xten}
%%OP-FUN%% %~x~siv
%%OP-FUN%% %
%%OP-FUN%% %~x~neg
%%OP-FUN%% 
%%OP-FUN%% 
%%OP-FUN%% Unary and binary promotion (\Sref{XtenPromotions}) is not performed
%%OP-FUN%% when invoking these
%%OP-FUN%% operations; instead, the operands are coerced individually via implicit
%%OP-FUN%% coercions (\Sref{XtenConversions}), as appropriate.
%%OP-FUN%% 
%%OP-FUN%% 

\section{Functions as objects of type {\tt Any}}
\label{FunctionAnyMethods}

\label{FunctionEquality}
\index{function!equality} \index{equality!function} Two functions \Xcd{f} and
\Xcd{g} are equal if both were obtained by the same evaluation of a function
literal.\footnote{A literal may occur in program text within a loop, and hence
  may be evaluated multiple times.} Further, it is guaranteed that if two
functions are equal then they refer to the same locations in the environment
and represent the same code, so their executions in an identical situation are
indistinguishable. (Specifically, if \xcd`f == g`, then \xcd`f(1)` can be
substituted for \xcd`g(1)` and the result will be identical. However, there is
no guarantee that \xcd`f(1)==g(1)` will evaluate to true. Indeed, there is no
guarantee that \xcd`f(1)==f(1)` will evaluate to true either, as \xcd`f` might
be a function which returns {$n$} on its {$n^{th}$} invocation. However,
\xcd`f(1)==f(1)` and \xcd`f(1)==g(1)` are interchangeable.)
\index{function!==}


Every function type implements all the methods of \Xcd{Any}.
\xcd`f.equals(g)` is equivalent to \xcd`f==g`.  The behavior of   \xcd`hashCode`, 
\xcd`toString`, and \xcd`typeName` is up to the implementation, but 
respect \xcd`equals` and the basic contracts of \xcd`Any`. 

\index{function!equals}
\index{function!hashCode}
\index{function!toString}
\index{function!typeName}
\index{function!home}
\index{function!at(Place)}
\index{function!at(Object)}
