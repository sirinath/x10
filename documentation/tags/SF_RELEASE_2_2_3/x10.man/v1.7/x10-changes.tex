\extrapart{Changes from v0.32}

This is the first reference manual that corresponds to a working
implementation. As such a number of details missing from v0.32 have
been spelled out. A number of mistakes have been corrected, and
clarifications added.

The semantics of exception handling across asynchronous activities has
been clarified.

Exploded syntax has been introduced to make it convenient to
destructure points. 

Arrays are implemented at all built-in types as well as at all
user-defined types.

\section{Limitations}

\begin{itemize}
\item All the static semantics rules are not yet implemented. 
Thus if your program is already correct, it will execute correctly. If
it is not correct, it may still execute and give a
result. Specifically, the implementation does not yet check {\tt
local} annotations.

\item The implementation does not support a way to specify just the rank of
an array in its type.

\item Arrays of arrays (``jagged'' arrays) are not yet implemented.

\item Local, 1-d, 0-based arrays do not currently support all the
functionality of more general arrays. (They are optimized through to
\Java{} arrays.)

\end{itemize}

\XtenCurrVer{} does not formally define a model for interoperability with
Java. The following guidelines may be used temporarily:

\begin{itemize}
  \item \Java{} classes may be imported.
  \item Confine the use of \Java{} to system classes, e.g.,
{\tt  System.out} for I/O.
  \item Do not use synchronized keyword, or create new {\tt java.lang.Threads}. 
  \item \Java{} objects may be created. These are ``placeless''. Do not use
     such an object as an argument to async, or wherever a placeful
     value is expectd.
\end{itemize}

\section{Future work}

Language needs to be extended to support generic types, with
type and value parameters.

Language needs to be extended to support type inference.

Language needs to be extended to support implicit syntax.
