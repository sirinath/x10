A closed statement is reduced to a closed statement.
\begin{lemma}
\label{Lemma:closed} \textbf{(Closed statement)}
If $\vdash \hS$ and $\hS,H \reducesto \hS',H'$ then $\vdash \hS'$.
\end{lemma}
\begin{proof}
The only rules that introduce free variables are \RULE{R-Val}
    and \RULE{R-Invoke}.

In rule \RULE{R-Val} we have \[
    \hval{\hx}{\he}{\hS},H \reduce \hS[\hl/\hx], H'
\]
Therefore, \hS might have a free variable (\hx), but~$\hS[\hl/\hx]$ is closed.

In rule \RULE{R-Invoke} we have \[
    \hl'.\hm(\ol{\hl}),H \reduce \hS[\ol{\hl}/\ol{\hx},\hl'/\hthis],H
\]
The method body can only refer to the free variables~$\ol{\hx}$ and~$\this$,
    therefore~$\hS[\ol{\hl}/\ol{\hx},\hl'/\hthis]$ is closed.
\end{proof}



Monotonicity of cooked heaps: during the process of reduction, the heap stays correctly-cooked and it becomes more cooked.
\begin{lemma}
\label{Lemma:correctly-cooked} \textbf{(Correctly-cooked heaps)}
If $\vdash H$ and (i)~$\hS,H \reducesto \hS',H'$ or~(ii)~$\hS,H \reducesto H'$, then $\vdash H'$ and~$H' \vdash H$.
\end{lemma}
\begin{proof}
By definition:
    (i)~$\vdash H$ if for every object $o=C(F)$ in the range of $H$ and
    every field $f \in \dom(F)$ it is the case that every object~$\hl=H(F(f))$ is cooked ($\cooked_H(\hl)$).
    (ii)~$H' \vdash H$
    if for every~$\hl \in \dom(H)$, we have~$H(\hl)=\hC(F)$,~$H'(\hl)=\hC(F')$, and~$\dom(F)\subseteq\dom(F')$.

By structural induction on~\hS.

For \RULE{R-Term} and \RULE{R-Step}, by induction $\vdash H'$ and~$H' \vdash H$.

For \RULE{R-Val}, $\he,H \reduce \hl',H'$ in two possible ways: \RULE{R-New} or \RULE{R-Access}.
In \RULE{R-New}, we created a new location~$\hl'$ that is not connected to any other location, so $\vdash H'$ and~$H' \vdash H$.
In \RULE{R-Access}, $H'=H$.

In \RULE{R-Invoke}, $H'=H$.

Only \RULE{R-Assign} ($\hl.\hf=\hl'$) changes the heap, and it requires that~$\cooked_H(\hl')$ so $\vdash H'$.
Let~$H(\hl)=\hC(F)$ and~$H'(\hl)=\hC(F')$, where~$F'=F[ \hf \mapsto \hl']$, so~$\dom(F)\subseteq\dom(F')$.
Thus~$H' \vdash H$.
\end{proof}




\begin{lemma}  \textbf{(Properties of plus and minus)}
\label{Lemma:plus}
For all $\phi$,
    (i)~$+\pm\phi=+\phi$,
    (ii)~$-\pm\phi=-\phi$.
\end{lemma}
\begin{proof}
Routine.
\end{proof}



\begin{lemma}   \textbf{(Properties of the logic system)}
\label{Lemma:logic}
Let $\phi \vdash \psi$.
Then
    (i)~$+\phi \vdash +\psi$,
    (ii)~$-\phi \vdash -\psi$,
    (iii)~for all~$\phi'$, we have~$\phi',\phi \vdash \phi',\psi$.
\end{lemma}
\begin{proof}
Routine.
\end{proof}



\begin{lemma}   \textbf{(Properties of heaps and formulas)}
\label{Lemma:heaps}
For all heaps~$H$,
(i)~$H \vdash -\phi$,
(ii)~if $H \vdash +\phi$ then $H \vdash \phi$,
(iii)~if $H \vdash \phi$ and $H \vdash \psi$ then $H \vdash \phi,\psi$,
(iv)~if $H \vdash \phi$ and~$H' \vdash H$ then $H' \vdash \phi$.
\end{lemma}
\begin{proof}
Routine.
\end{proof}


The meaning of substitution for statement is to replace all the \emph{free} occurrences of a variable.
For example, for the following statement~\hS \[
\hx.\hf=\hl;\hval{\hx}{\hnew{\hC}}{\hx.\hf=\hl}
\]
if we substitute \hx with $\hl'$, written as~$\hS[\hl'/\hx]$, the result is \[
\hl'.\hf=\hl;\hval{\hx}{\hnew{\hC}}{\hx.\hf=\hl}
\]


\begin{lemma}  \textbf{(Substitution lemma)}
\label{Lemma:substitution}
Given a heap $H$, a location~\hl whose type is $\hC$, a variable \hx whose type is $\hD$, a statement \hS,
    and \[
    \vdash H \gap \hC \st \hD \gap \phi~\hS~\psi
    \] then~$\phi[\hl/\hx]~\hS[\hl/\hx]~\psi[\hl/\hx]$.
\end{lemma}
\begin{proof}
Routine (by induction on the derivation sequence of~$\phi~\hS~\psi$, considering each rule in \Ref{Figure}{effects}).
The only sensitive point is that the type of~\hl is a subtype \hx, and therefore in \RULE{T-Invoke} we use the fact that when method~$\hm$ is overridden by~$\hm'$
    then their inferred decorations~$(\phi, \psi)$ and~$(\phi', \psi')$ satisfy $\phi \vdash \phi'$ and $\psi' \vdash \psi$.
\end{proof}



\begin{lemma}  \textbf{(Preservation for terminal configurations)}
\label{Lemma:Preservation1}
Let $\acute{\phi}~\acute{\hS}~\acute{\psi}$, $\vdash \acute{\hS}$, $\vdash H$, $H \vdash \acute{\phi}$, and $\acute{\hS},H \reducesto H'$.
Then $H' \vdash +\acute{\psi}$.
\end{lemma}
\begin{proof}
By structural induction on~$\acute{\hS}$.
Only two rules result with a terminal configuration:
    \RULE{R-Term} (for \hfinish and \hasync)
    and \RULE{R-Assign}.


$\bullet$
In \RULE{R-Term} for \hfinish we have:
\beq{pre1}
    \acute{\hS}=\finish{\hS} \gap \hS,H \reduce H'
\eeq
From~$\acute{\phi}~\acute{\hS}~\acute{\psi}$ and \RULE{T-Finish}
\beq{pre2}
    \acute{\psi}=+\psi \gap \phi~\hS~\psi \gap \acute{\phi}=\phi \gap \phi~\finish{\hS}~ +\psi
\eeq
Because $\hS,H \reduce H'$ and~$\phi~\hS~\psi$ (from \eq{pre2}), by induction we have
\beq{pre3}
    H' \vdash +\psi
\eeq
Because~$\acute{\psi}=+\psi$ and $++\psi=+\psi$ (\Ref{Lemma}{plus}), from~\eq{pre3}, we conclude~$H' \vdash +\acute{\psi}$.

$\bullet$
In \RULE{R-Term} for \hasync we do the same proof but use the fact that~$+-\psi=+\psi$ (\Ref{Lemma}{plus}).

$\bullet$
In \RULE{R-Assign} we have:
\beqs{pre4}
    &\acute{\hS}=\hl.\hf=\hl' \gap H(\hl)=\hC(F) \gap \cooked_H(\hl')
    \\
    &H'=H[ \hl \mapsto \hC(F[ \hf \mapsto \hl'])]
\eeq
Thus,~$H' \vdash +\hl.\hf$.
From~$\acute{\phi}~\acute{\hS}~\acute{\psi}$ and \RULE{T-Assign}
\beq{pre5}
    \acute{\phi}=\phi
    \gap
    \acute{\psi}=+\hl.\hf
    \gap
    \phi ~ \hl.\hf=\hl' ~ + \hl.\hf
    \gap
    \phi \vdash +\hl'
\eeq
From~\eq{pre4} and~\eq{pre5}, we conclude~$H' \vdash \acute{\psi}$.
Because~$+\acute{\psi}=\acute{\psi}$ we also have~$H' \vdash +\acute{\psi}$.
\end{proof}


\begin{lemma}  \textbf{(Preservation for non-terminal configurations)}
\label{Lemma:Preservation2}
Let
\beq{np}
    \acute{\phi}~\acute{\hS}~\acute{\psi}
    \gap
    \vdash \acute{\hS}
    \gap
    \vdash H
    \gap
    H \vdash \acute{\phi}
    \gap
    \acute{\hS},H \reducesto \grave{\hS},H'
\eeq
Then there exists $\grave{\phi},\grave{\psi}$ such
    that
\beq{np-res}
H' \vdash \grave{\phi}
\gap
\grave{\phi}~\grave{\hS}~\grave{\psi}
\gap
\grave{\phi} \vdash \acute{\phi}
\gap
\grave{\psi} \vdash \acute{\psi}
\eeq
\end{lemma}
\begin{proof}
By structural induction on~$\acute{\hS}$.

$\bullet$
For \RULE{R-Term} for sequences we have
\beq{np1}
 \acute{\hS} = \hS~\hS' \gap
 \grave{\hS}=\hS' \gap
 \hS,H \reduce H'
\eeq
From \RULE{T-Seq} and~\eq{np} ($\acute{\phi}~\acute{\hS}~\acute{\psi}$),
\beq{np2}
  \acute{\phi}=\phi
    \gap
  \acute{\psi}=\psi_1,\psi_2
    \gap
  \phi~\hS~\psi_1
    \gap
  \phi,\psi_1~\hS'~\psi_2
\eeq
From~\eq{np1} and~\eq{np2}, by \Ref{Lemma}{Preservation1}
\beq{np3}
    H'\vdash +\psi_1.
\eeq
Let~$\grave{\phi}=\phi,\psi_1$ and~$\grave{\psi}=\acute{\psi}$.
Then from~\eq{np2}
\beq{np4}
    \grave{\phi}~\grave{\hS}~\psi_2
\eeq
Because~$\grave{\phi}=\phi,\psi_1$ then~$\grave{\phi} \vdash \psi_1$, and because~$\grave{\psi}=\acute{\psi}=\psi_1,\psi_2$ then
\beq{np4b}
    \grave{\phi}~\grave{\hS}~\grave{\psi}
\eeq
Trivially,
\beqs{np5}
    &\phi,\psi_1 \vdash \phi
    \gap
    \grave{\psi} = \acute{\psi} \\
    &\grave{\phi} \vdash \acute{\phi}
    \gap
    \grave{\psi} \vdash \acute{\psi}
\eeq
From~\eq{np3} and \Ref{Lemma}{heaps},
\beq{np5a}
    H'\vdash \psi_1
\eeq
Because~$H' \vdash H$ (\Ref{Lemma}{correctly-cooked}) and~$H \vdash\phi$, then from \Ref{Lemma}{heaps}~$H' \vdash\phi$.
Together,~$H' \vdash \phi,\psi_1$.
Because~$\grave{\phi}=\phi,\psi_1$, we got~$H' \vdash \grave{\phi}$.


$\bullet$
For \RULE{R-Step} for \hfinish, we have
\beq{np6}
 \acute{\hS} = \finish{\hS} \gap
 \grave{\hS}=\finish{\hS'} \gap
 \hS,H \reduce \hS', H'
\eeq
Because~$\acute{\phi}~\acute{\hS}~\acute{\psi}$, from \RULE{T-Finish},
\beq{np7}
  \acute{\phi}=\phi
    \gap
  \acute{\psi}=+\psi
    \gap
  \phi~\hS~\psi
\eeq
From~\eq{np6} and~\eq{np7}, by induction, exists~$\phi'$ and~$\psi'$,
\beq{np8}
  \phi'~\hS'~\psi'
    \gap
  \phi' \vdash \phi
    \gap
  \psi' \vdash \psi
    \gap
  H' \vdash \phi'
\eeq
From \RULE{T-Finish} and \eq{np8},
\beq{np9}
  \phi'~\finish{\hS'}~+\psi'
\eeq
Let~$\grave{\phi}=\phi'$ and~$\grave{\psi}=+\psi'$, then from~\eq{np8} and
    from \Ref{Lemma}{logic},
\beq{np10}
    \phi' \vdash \phi
    \gap
    +\psi' \vdash +\psi
    \gap
    \grave{\phi}~\grave{\hS}~\grave{\psi}
    \gap
    H' \vdash \phi'
\eeq

$\bullet$
The proof for the first \RULE{R-Step} for \hasync is like that of \hfinish above (but instead we use the fact that
    if~$\psi' \vdash \psi$ then~$-\psi' \vdash -\psi$ in \Ref{Lemma}{logic}).

$\bullet$
For the second \hasync \RULE{R-Step}, we have
\beq{np11}
 \acute{\hS} = \async{\hS_1}~\hS \gap
 \grave{\hS}=\async{\hS_1}~\hS' \gap
 \hS,H \reduce \hS', H'
\eeq
Because~$\acute{\phi}~\acute{\hS}~\acute{\psi}$, from \RULE{T-Seq},
\beq{np12}
  \acute{\phi}=\phi
    \gap
  \acute{\psi}=\psi_1,\psi_2
    \gap
  \phi~\async{\hS_1}~\psi_1
        \gap
  \phi,\psi_1~\hS~\psi_2
\eeq
From \RULE{T-Async} we know that there exists~$\psi'_1$ such that~$\psi_1=-\psi'_1$,
    therefore from \Ref{Lemma}{heaps}, we have that for any~$H$, it holds~$H \vdash \psi_1$.
Thus,~$H \vdash \phi,\psi_1$.
From that and~$\phi,\psi_1~\hS~\psi_2$ we can apply the induction and get that there exists~$\phi'$ and~$\psi'$ such that
\beq{np13}
    \phi'~\hS'~\psi'
    \gap
    \phi' \vdash \phi,\psi_1
    \gap
    \psi' \vdash \psi_2
    \gap
    H' \vdash \phi'
\eeq
Then from~\eq{np12},~\eq{np13},
\beq{np14}
    \phi'~\async{\hS_1}~\psi_1
    \gap
    \phi',\psi_1~\hS'~\psi'
\eeq
From~\eq{np14} and \RULE{T-Seq}
\beq{np15}
    \phi'~\async{\hS_1}~\hS'~\psi_1,\psi'
\eeq
Let~$\grave{\phi}=\phi'$ and~$\grave{\psi}=\psi_1,\psi'$, then
\beq{np16}
    \grave{\phi}~\async{\hS_1}~\hS'~\grave{\psi}
    \gap
    \phi' \vdash \phi
    \gap
    \psi_1,\psi' \vdash \psi_1,\psi_2
    \gap
    H' \vdash \grave{\phi}
\eeq



$\bullet$
For \RULE{R-Step} for sequences, we have
\beq{np26}
 \acute{\hS} =\hS~\hS_1 \gap
 \grave{\hS}=\hS'~\hS_1 \gap
 \hS,H \reduce \hS', H'
\eeq
Because~$\acute{\phi}~\acute{\hS}~\acute{\psi}$, from \RULE{T-Seq},
\beq{np27}
  \acute{\phi}=\phi
    \gap
  \acute{\psi}=\psi_1,\psi_2
    \gap
  \phi~\hS~\psi_1
    \gap
  \phi,\psi_1~\hS_1~\psi_2
\eeq
From~\eq{np26} and~\eq{np27}, by induction ($\hS,H \reduce \hS', H'$ and~$H \vdash \phi$ and~$\phi~\hS~\psi_1$),
    exists~$\phi'$ and~$\psi'_1$,
\beq{np28}
  \phi'~\hS'~\psi'_1
    \gap
  \phi' \vdash \phi
    \gap
  \psi'_1 \vdash \psi_1
    \gap
  H' \vdash \phi'
\eeq
Thus, from~\eq{np28}, we have
\beq{np28a}
\phi',\psi'_1 \vdash \phi,\psi_1
\eeq
From~\eq{np27} and~\eq{np28a},
\beq{np28b}
  \phi',\psi'_1~\hS_1~\psi_2
\eeq
\eeq
From \RULE{T-Seq},~\eq{np28} ($\phi'~\hS'~\psi'_1$) and \eq{np28b},
\beq{np29}
  \phi'~\hS'~\hS_1~\psi'_1,\psi_2
\eeq
Let~$\grave{\phi}=\phi'$ and~$\grave{\psi}=\psi'_1,\psi_2$, then from~\eq{np28} and~\eq{np29} and
    from \Ref{Lemma}{logic},
\beq{np30}
    \phi' \vdash \phi
    \gap
    \psi'_1,\psi_2 \vdash \psi_1,\psi_2
    \gap
    \grave{\phi}~\grave{\hS}~\grave{\psi}
    \gap
    H' \vdash \phi'
\eeq

$\bullet$
For \RULE{R-Val},
\beq{np36}
 \acute{\hS} =\hval{\hx}{\he}{\hS} \gap
 \grave{\hS}= \hS[\hl/\hx] \gap
 \he,H \reduce \hl,H'
\eeq
We have two possible ways to reduce \he:
    either with \RULE{R-New} or \RULE{R-Access}.
Consider first \RULE{R-New}, so
\beq{np36b}
 \he = \hnew{\hC} \gap
 \hl \not \in \dom(H) \gap
 H'=H[ \hl \mapsto \hC()]
\eeq
Because~$\acute{\phi}~\hval{\hx}{\hnew{\hC}}{\hS}~\acute{\psi}$, from \RULE{T-New},
\beq{np37}
  \acute{\phi} = \phi \gap
  \acute{\psi} = \psi \gap
  \phi~\hS~\psi
\eeq
Because~$H$ contains only locations~\hl (not variables~\hx), and~$H \vdash \phi$,
    then~$\phi$ and~$\psi$ cannot contain variables~\hx, thus
    $\phi[\hl/\hx]=\phi$ and~$\psi[\hl/\hx]=\psi$.
From \Ref{Lemma}{substitution} (because $H$ and \hS are well-typed, so the type of~\hl is a subtype of~\hx) and~\eq{np37}, we have
\beq{np38}
  \phi~\hS[\hl/\hx]~\psi
\eeq
From \Ref{Lemma}{heaps}, because~$H' \vdash H$ (\Ref{Lemma}{correctly-cooked}), then
\beq{np39}
  H' \vdash \phi
\eeq
We choose~$\grave{\phi}=\phi$ and~$\grave{\psi}=\psi$, thus from~\eq{np38} and~\eq{np39},
\beq{np40}
  H' \vdash \grave{\phi} \gap
  \grave{\phi} \vdash \phi  \gap
  \grave{\psi} \vdash \psi  \gap
  \grave{\phi}~\grave{\hS}~\grave{\psi}
\eeq
Now consider \RULE{R-Access}, so
\beq{np41}
 \he = \hl'.\hf_i \gap
 H(\hl')=\hC(\ol{\hf}\mapsto\ol{\hl"}) \gap
 \hl = \hl"_i \gap
 H'=H
\eeq
Because~$\vdash H$ then~$\hl$ is cooked, i.e., \[H \vdash +\hl \]
Because~$\acute{\phi}~\hval{\hx}{\hl'.\hf_i}{\hS}~\acute{\psi}$, from \RULE{T-Access},
\beq{np42}
  \acute{\phi} = \phi \gap
  \acute{\psi} = \psi \gap
    \phi \vdash +\hl'.\hf_i \gap
    \phi, +\hx~\hS~\psi
\eeq
Like for \RULE{R-New}, we have that~$\phi[\hl/\hx]=\phi$ and~$\psi[\hl/\hx]=\psi$.
From \Ref{Lemma}{substitution} and~\eq{np42}, we have
\beq{np43}
  \phi,+\hl~\hS[\hl/\hx]~\psi
\eeq
Because~$H'=H$ and~$H \vdash \phi$ and~$H \vdash +\hl$,
    then
\beq{np44}
  H' \vdash \phi,+\hl
\eeq
We choose~$\grave{\phi}=\phi,+\hl$ and~$\grave{\psi}=\psi$, thus from~\eq{np43} and~\eq{np44},
\beq{np45}
  H' \vdash \grave{\phi} \gap
  \grave{\phi} \vdash \phi  \gap
  \grave{\psi} \vdash \psi  \gap
  \grave{\phi}~\grave{\hS}~\grave{\psi}
\eeq

$\bullet$
For \RULE{R-Invoke},
\beqs{np56}
 &\acute{\hS} =\hl'.\hm(\ol{\hl}) \gap
 \grave{\hS}=\hS[\ol{\hl}/\ol{\hx},\hl'/\hthis] \\
 &H(\hl')=\hC(\ldots) \gap
 \mbody{}(\hm,\hC)=\ol{\hx}.\hS \gap
 H'=H
\eeq
Because~$\acute{\phi}~\acute{\hS}~\acute{\psi}$, from \RULE{T-Invoke},
\beqs{np57}
  & \acute{\phi}=\phi
    \gap
  \acute{\psi}=\psi'[\hl'/\hthis,\ol{\hl}/\ol{\hx}]
    \\
  & \hm(\ol{\hx}):: \phi' \Rightarrow \psi'
    \gap
  \phi \vdash \phi'[\hl'/\hthis,\ol{\hl}/\ol{\hx}]
\eeq
From the definition of~$\hm(\ol{\hx}):: \phi' \Rightarrow \psi'$ we have that
\beq{np58}
    \phi'~\hS~\psi'
\eeq
From \Ref{Lemma}{substitution} (because $H$ and $\acute{\hS}$ are well-typed,
    so the type of~$\ol{\hl}$ is a subtype of~$\ol{\hx}$,
    and the type of $\hl'$ is  a subtype of~\this)
    and~\eq{np58}, we have
\beq{np59}
    \phi'[\hl'/\hthis,\ol{\hl}/\ol{\hx}]~~\hS[\hl'/\hthis,\ol{\hl}/\ol{\hx}]~~\psi'[\hl'/\hthis,\ol{\hl}/\ol{\hx}]
\eeq
We choose~$\grave{\phi}=\acute{\phi}=\phi$ and~$\grave{\psi}=\acute{\psi}=\psi'[\hl'/\hthis,\ol{\hl}/\ol{\hx}]$,
    thus from~\eq{np56},~\eq{np57} and~\eq{np59},
\beq{np60}
  H' \vdash \grave{\phi} \gap
  \grave{\phi} \vdash \acute{\phi}  \gap
  \grave{\psi} \vdash \acute{\psi}  \gap
  \grave{\phi}~\grave{\hS}~\grave{\psi}
\eeq

$\bullet$
(Note that \RULE{R-Assign} was handled in \Ref{Lemma}{Preservation1}.)
\end{proof}



\begin{Theorem}{ \textbf{(Preservation)}}
\label{Theorem:Preservation}
Let \[
\phi~\hS~\psi \gap \vdash \hS \gap \vdash H \gap H \vdash \phi
\]
(a) If $\hS,H \reducesto H'$ then \[
\vdash H' \gap H' \vdash H \gap H' \vdash +\psi
\]
(b) If $\hS,H \reducesto \hS',H'$ then \[
\vdash \hS' \gap \vdash H' \gap H' \vdash H
\] and
there exists $\phi',\psi'$ such
that \[
H' \vdash \phi' \gap \phi'~\hS'~\psi' \gap \phi' \vdash \phi \gap \psi' \vdash \psi
\]
\end{Theorem}
\begin{proof}
Follows from \Ref{Lemma}{closed},  \Ref{Lemma}{correctly-cooked}, \Ref{Lemma}{Preservation1},
    and \Ref{Lemma}{Preservation2}.
\end{proof}

Progress means that a proper non-terminal configuration can always be reduced.
\begin{Theorem}{ \textbf{(Progress)}}
Let \[
\acute{\phi}~\acute{\hS}~\acute{\psi} \gap \vdash \acute{\hS} \gap \vdash H \gap H \vdash \acute{\phi}
\]
Then there exists $\grave{\hS},H'$ such
    that either $\acute{\hS},H \reducesto H'$ or $\acute{\hS},H \reducesto \grave{\hS},H'$.
\end{Theorem}
\begin{proof}
By structural induction on~$\acute{\hS}$.
According to the syntax,~$\acute{\hS}$ is one of the following:
\beqst
\hS &::=  \hp.\hf = \hp; ~|~ \hp.\hm(\ol{\hp});  ~|~ \hval{\hx}{\he}{\hS} ~|~ \finish{\hS}~|~ \async{\hS} ~|~ \hS~\hS\\
\he &::=  \hp.\hf  ~|~ \hnew{\hC}\\
\eeq
Because~$\vdash \acute{\hS}$ we know that \hp cannot be a variable (only a location in~$H$), so
\beqst
\hS &::=  \hl.\hf = \hl; ~|~ \hl.\hm(\ol{\hl});  ~|~ \hval{\hx}{\he}{\hS} ~|~ \finish{\hS}~|~ \async{\hS} ~|~ \hS~\hS\\
\he &::=  \hl.\hf  ~|~ \hnew{\hC}\\
\eeq
We will show that for each of these possibilities, there is a matching reduction rule that can be applied.

$\bullet$
If~$\acute{\hS} = \finish{\hS}$, then because~$\acute{\phi}~\acute{\hS}~\acute{\psi}$, we have from \RULE{T-Finish},
\beq{p1}
\phi~\hS~\psi \gap
\acute{\phi}=\phi \gap
\acute{\psi}=+\psi
\eeq
Therefore, by induction, there exists $\hS',H'$ such
    that either ${\hS},H \reducesto H'$ or ${\hS},H \reducesto {\hS'},H'$.
Therefore, we can either apply \RULE{R-Term} or \RULE{R-Step} on $\acute{\hS}$,
    i.e., either $\acute{\hS},H \reducesto H'$ or $\acute{\hS},H \reducesto \finish{\hS'},H'$.

$\bullet$
If~$\acute{\hS} = \async{\hS}$, then the same proof as above holds (except changing $+$ to $-$).

$\bullet$
If~$\acute{\hS} = \hS_1 \hS_2$, then because~$\acute{\phi}~\acute{\hS}~\acute{\psi}$, we have from \RULE{T-Seq},
\beq{p2}
  \phi~\hS_1~\psi_1
        \gap
    \phi,\psi_1~\hS_2~\psi_2 \gap
    \acute{\phi}=\phi \gap
    \acute{\psi}=\psi_1,\psi_2
\eeq
By induction, there exists $\hS'_1,H'$ such
    that either ${\hS_1},H \reducesto H'$ or ${\hS_1},H \reducesto {\hS'_1},H'$.
Therefore, we can either apply \RULE{R-Term} or \RULE{R-Step} on $\acute{\hS}$,
    i.e., either $\acute{\hS},H \reducesto \hS_2,H'$ or $\acute{\hS},H \reducesto \hS'_1~\hS_2,H'$.

$\bullet$
If~$\acute{\hS} = \hl'.\hm(\ol{\hl})$, then we can simply apply \RULE{R-Invoke}.

$\bullet$
If~$\acute{\hS} = \hl.\hf=\hl'$, then because~$\acute{\phi}~\acute{\hS}~\acute{\psi}$, we have from \RULE{T-Assign}, that \[
\acute{\phi} \vdash +\hl'
\]
And because~$H \vdash \acute{\phi}$ we have that~$\cooked_H(\hl')$.
Thus we can apply \RULE{R-Assign}, i.e., \[
    H(\hl)=\hC(F) \gap
    \acute{\hS},H \reducesto H[ \hl \mapsto \hC(F[ \hf \mapsto \hl'])]
\]

$\bullet$
Finally,~$\acute{\hS} = \hval{\hx}{\he}{\hS}$.
If $\he = \hnew{\hC}$ then by \RULE{R-New}, we have \[
    \he,H \reduce \hl',H[ \hl' \mapsto \hC()]
\]
If $\he = \hl.\hf$, then because~$\acute{\phi}~\acute{\hS}~\acute{\psi}$, we have from \RULE{T-Access}, \[
    \acute{\phi} \vdash +\hl.\hf
\]
And because~$H \vdash \acute{\phi}$, we have that \hl has the field~\hf in the heap, i.e., \[
    H(\hl)=\hC(F) \gap \hf \in \dom(F)
\]
Thus, we can apply \RULE{R-Access} and get~$\hl.\hf,H \reduce \hl',H$.
In both cases, we have that~\he was reduced to~$\hl'$, thus we can apply \RULE{R-Val} and get \[
    \acute{\hS},H \reduce \hS[\hl'/\hx], H'
\]
\end{proof}
