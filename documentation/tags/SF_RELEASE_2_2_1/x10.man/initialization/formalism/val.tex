
\Section[val]{Final fields}
\hval and \hvar fields.

\subsection{Syntax}


\begin{figure}[htpb!]
\begin{center}
\begin{tabular}{|l|l|}
\hline

$\hF ::= \hFM ~ \hf:\hC$
& Field declaration. \\

$\hFM ::= \hval~|~\hvar$
& Field Modifier. \\

\hline
\end{tabular}
\end{center}
\caption{FX10 Syntax changes to support final fields (\hval).
    The new terminals are \hval and \hvar.
    }
\label{Figure:syntax-val}
\end{figure}


\subsection{Typing}
Fields also have a modifier (\hFM) so we define:
    $\fmodifier(\hf,\hC) = \hFM$ returns the field modifier of \hf in class \hC (either \hval or \hvar).

A method cannot assign to any \hval fields (no matter what the target is):
\beqs{methodVal}
    &\methodVal([\he_0,\ldots,\he_n])= \methodVal(\he_0) \hand \ldots \hand  \methodVal(\he_n)\\
    &\methodVal(\he")=
        \begin{cases}
        \htrue & \he" \equiv \hl \\
        \htrue & \he" \equiv \hx \\
        \methodVal(\he) & \he" \equiv \he.\hf \\
        \methodVal([\he,\he']) \hand \hFM=\hvar & \he" \equiv \he.\hf = \he' \gap \Gdash \he:\hC \gap \fmodifier{}(\hf,\hC)=\hFM \\
        \methodVal([\he',\ol{\he}]) & \he" \equiv \he'.\hm(\ol{\he}) \\
        \methodVal([\ol{\he}]) & \he" \equiv \hnew ~ \hC(\ol{\he}) \\
        \methodVal(\he) & \he" \equiv \hfinish~\he \\
        \methodVal([\he,\he']) & \he" \equiv \hasync~\he;\he' \\
        \end{cases}
\eeq

A constructor can assign to a \hval field at most once and only if the target is \this:
\beqs{ctorVal}
    &\ctorVal([\he_0,\ldots,\he_n],F)= \ctorVal(\he_0,F) \hand \ctorVal(\he_1,\AW{[\he_0]} \cup F) \hand \ldots \hand  \ctorVal(\he_n,\AW{[\he_0,\ldots,\he_{n-1}]} \cup F)\\
    &\ctorVal(\he",F)=
        \begin{cases}
        \htrue & \he" \equiv \hl \\
        \htrue & \he" \equiv \hx \\
        \ctorVal(\he,F) & \he" \equiv \he.\hf \\
        \ctorVal([\he,\he'],F \cup \{ \hf \}) \hand \\(\hFM=\hvar \hor (\he=\this \hand \hf \not \in F)) & \he" \equiv \he.\hf = \he' \gap \Gdash \he:\hC \gap \fmodifier{}(\hf,\hC)=\hFM \\
        \ctorVal([\he',\ol{\he}],F) & \he" \equiv \he'.\hm(\ol{\he}) \\
        \ctorVal([\ol{\he}],F) & \he" \equiv \hnew ~ \hC(\ol{\he}) \\
        \ctorVal(\he,F) & \he" \equiv \hfinish~\he \\
        \ctorVal([\he,\he'],F) & \he" \equiv \hasync~\he;\he' \\
        \end{cases}
\eeq

We need to check that we do not assign to a \hval field ($\methodVal(\he)$).


\beqst %{method-ok}
\typerule{
  \methodVal(\he)
}{
  \hMM ~ \hm(\ol{\hx}:\ol{\hD}):\hD=\he ~~\OK~\IN~\hC
}~\RULE{(T-Method)}
\eeq

Similarly, we need to check for constructors that \hval fields are treated correctly (assigned at most once).

\beqst %{ctor-ok}
\typerule{
  \hclass~\hC~\hextends~\hC'~\lb~\ldots~\rb
    \gap
  F_s = \fields{}(\hC')
    \gap
  \ctorVal([\ol{\he},\he'],F_s)
}{
  \ctor(\ol{\hx}:\ol{\hD}) \lb~\super(\ol{\he});\he'; \rb ~~\OK~\IN~\hC
}\gap \RULE{(T-Ctor)}
\eeq



Moreover, $\reductionVal$ is preserved during the reduction,
    i.e., a \hval is assigned at most once.

H, \he closed
$\reductionVal(\he,H)$  (we have everything in H - which val fields are already assigned)
